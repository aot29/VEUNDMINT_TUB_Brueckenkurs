% MINTMOD Version P0.1.0, needs to be consistent with preprocesser object in tex2x and MPragma-Version at the end of this file

% Parameter aus Konvertierungsprozess (PDF und HTML-Erzeugung wenn vom Konverter aus gestartet) werden hier eingefuegt, Preambleincludes werden am Schluss angehaengt

\newif\ifttm                % gesetzt falls Uebersetzung in HTML stattfindet, sonst uebersetzung in PDF

% Wahl der Notationsvariante ist im PDF immer std, in der HTML-Uebersetzung wird vom Konverter die Auswahl modifiziert
\newif\ifvariantstd
\newif\ifvariantunotation
\variantstdtrue % Diese Zeile wird vom Konverter erkannt und ggf. modifiziert, daher nicht veraendern!


\def\MOutputDVI{1}
\def\MOutputPDF{2}
\def\MOutputHTML{3}
\newcounter{MOutput}

\ifttm
\usepackage{german}
\usepackage{array}
\usepackage{amsmath}
\usepackage{amssymb}
\usepackage{amsthm}
\else
\documentclass[ngerman,oneside]{scrbook}
\usepackage{etex}
\usepackage[latin1]{inputenc}
\usepackage{textcomp}
\usepackage[ngerman]{babel}
\usepackage[pdftex]{color}
\usepackage{xcolor}
\usepackage{graphicx}
\usepackage[all]{xy}
\usepackage{fancyhdr}
\usepackage{verbatim}
\usepackage{array}
\usepackage{float}
\usepackage{makeidx}
\usepackage{amsmath}
\usepackage{amstext}
\usepackage{amssymb}
\usepackage{amsthm}
\usepackage[ngerman]{varioref}
\usepackage{framed}
\usepackage{supertabular}
\usepackage{longtable}
\usepackage{maxpage}
\usepackage{tikz}
\usepackage{tikzscale}
\usepackage{tikz-3dplot}
\usepackage{bibgerm}
\usepackage{chemarrow}
\usepackage{polynom}
%\usepackage{draftwatermark}
\usepackage{pdflscape}
\usetikzlibrary{calc}
\usetikzlibrary{through}
\usetikzlibrary{shapes.geometric}
\usetikzlibrary{arrows}
\usetikzlibrary{intersections}
\usetikzlibrary{decorations.pathmorphing}
\usetikzlibrary{external}
\usetikzlibrary{patterns}
\usetikzlibrary{fadings}
\usepackage[colorlinks=true,linkcolor=blue]{hyperref} 
\usepackage[all]{hypcap}
%\usepackage[colorlinks=true,linkcolor=blue,bookmarksopen=true]{hyperref} 
\usepackage{ifpdf}

\usepackage{movie15}

\setcounter{tocdepth}{2} % In Inhaltsverzeichnis bis subsection
\setcounter{secnumdepth}{3} % Nummeriert bis subsubsection

\setlength{\LTpost}{0pt} % Fuer longtable
\setlength{\parindent}{0pt}
\setlength{\parskip}{8pt}
%\setlength{\parskip}{9pt plus 2pt minus 1pt}
\setlength{\abovecaptionskip}{-0.25ex}
\setlength{\belowcaptionskip}{-0.25ex}
\fi

\ifttm
\newcommand{\MDebugMessage}[1]{\special{html:<!-- debugprint;;}#1\special{html:; //-->}}
\else
%\newcommand{\MDebugMessage}[1]{\immediate\write\mintlog{#1}}
\newcommand{\MDebugMessage}[1]{}
\fi

\def\MPageHeaderDef{%
\pagestyle{fancy}%
\fancyhead[r]{(C) VE\&MINT-Projekt}
\fancyfoot[c]{\thepage\\--- CCL BY-SA 3.0 ---}
}


\ifttm%
\def\MRelax{}%
\else%
\def\MRelax{\relax}%
\fi%

%--------------------------- Uebernahme von speziellen XML-Versionen einiger LaTeX-Kommandos aus xmlbefehle.tex vom alten Kasseler Konverter ---------------

\newcommand{\MSep}{\left\|{\phantom{\frac1g}}\right.}

\newcommand{\ML}{L}

\newcommand{\MGGT}{\mathrm{ggT}}


\ifttm
% Verhindert dass die subsection-nummer doppelt in der toccaption auftaucht (sollte ggf. in toccaption gefixt werden so dass diese Ueberschreibung nicht notwendig ist)
\renewcommand{\thesubsection}{}
% Kommandos die ttm nicht kennt
\newcommand{\binomial}[2]{{#1 \choose #2}} %  Binomialkoeffizienten
\newcommand{\eur}{\begin{html}&euro;\end{html}}
\newcommand{\square}{\begin{html}&square;\end{html}}
\newcommand{\glqq}{"'}  \newcommand{\grqq}{"'}
\newcommand{\nRightarrow}{\special{html: &nrArr; }}
\newcommand{\nmid}{\special{html: &nmid; }}
\newcommand{\nparallel}{\begin{html}&nparallel;\end{html}}
\newcommand{\mapstoo}{\begin{html}<mo>&map;</mo>\end{html}}

% Schnitt und Vereinigungssymbole von Mengen haben zu kleine Abstaende; korrigiert:
\newcommand{\ccup}{\,\!\cup\,\!}
\newcommand{\ccap}{\,\!\cap\,\!}


% Umsetzung von mathbb im HTML
\renewcommand{\mathbb}[1]{\begin{html}<mo>&#1opf;</mo>\end{html}}
\fi

%---------------------- Strukturierung ----------------------------------------------------------------------------------------------------------------------

%---------------------- Kapselung des sectioning findet auf drei Ebenen statt:
% 1. Die LateX-Befehl
% 2. Die D-Versionen der Befehle, die nur die Grade der Abschnitte umhaengen falls notwendig
% 3. Die M-Versionen der Befehle, die zusaetzliche Formatierungen vornehmen, Skripten starten und das HTML codieren
% Im Modultext duerfen nur die M-Befehle verwendet werden!

\ifttm

  \def\Dsubsubsubsection#1{\subsubsubsection{#1}}
  \def\Dsubsubsection#1{\subsubsection{#1}\addtocounter{subsubsection}{1}} % ttm-Fehler korrigieren
  \def\Dsubsection#1{\subsection{#1}}
  \def\Dsection#1{\section{#1}} % Im HTML wird nur der Sektionstitel gegeben
  \def\Dchapter#1{\chapter{#1}}
  \def\Dsubsubsubsectionx#1{\subsubsubsection*{#1}}
  \def\Dsubsubsectionx#1{\subsubsection*{#1}}
  \def\Dsubsectionx#1{\subsection*{#1}}
  \def\Dsectionx#1{\section*{#1}}
  \def\Dchapterx#1{\chapter*{#1}}

\else

  \def\Dsubsubsubsection#1{\subsubsection{#1}}
  \def\Dsubsubsection#1{\subsection{#1}}
  \def\Dsubsection#1{\section{#1}}
  \def\Dsection#1{\chapter{#1}}
  \def\Dchapter#1{\title{#1}}
  \def\Dsubsubsubsectionx#1{\subsubsection*{#1}}
  \def\Dsubsubsectionx#1{\subsection*{#1}}
  \def\Dsubsectionx#1{\section*{#1}}
  \def\Dsectionx#1{\chapter*{#1}}

\fi

\newcommand{\MStdPoints}{4}
\newcommand{\MSetPoints}[1]{\renewcommand{\MStdPoints}{#1}}

% Befehl zum Abbruch der Erstellung (nur PDF)
\newcommand{\MAbort}[1]{\err{#1}}

% Prefix vor Dateieinbindungen, wird in der Baumdatei mit \renewcommand modifiziert
% und auf das Verzeichnisprefix gesetzt, in dem das gerade bearbeitete tex-Dokument liegt.
% Im HTML wird es auf das Verzeichnis der HTML-Datei gesetzt.
% Das Prefix muss mit / enden !
\newcommand{\MDPrefix}{.}

% MRegisterFile notiert eine Datei zur Einbindung in den HTML-Baum. Grafiken mit MGraphics werden automatisch eingebunden.
% Mit MLastFile erhaelt man eine Markierung fuer die zuletzt registrierte Datei.
% Diese Markierung wird im postprocessing durch den physikalischen Dateinamen ersetzt, aber nur den Namen (d.h. \MMaterial gehoert noch davor, vgl Definition von MGraphics)
% Parameter: Pfad/Name der Datei bzw. des Ordners, relativ zur Position des Modul-Tex-Dokuments.
\ifttm
\newcommand{\MRegisterFile}[1]{\addtocounter{MFileNumber}{1}\special{html:<!-- registerfile;;}#1\special{html:;;}\MDPrefix\special{html:;;}\arabic{MFileNumber}\special{html:; //-->}}
\else
\newcommand{\MRegisterFile}[1]{\addtocounter{MFileNumber}{1}}
\fi

% Testen welcher Uebersetzer hier am Werk ist

\ifttm
\setcounter{MOutput}{3}
\else
\ifx\pdfoutput\undefined
  \pdffalse
  \setcounter{MOutput}{\MOutputDVI}
  \message{Verarbeitung mit latex, Ausgabe in dvi.}
\else
  \setcounter{MOutput}{\MOutputPDF}
  \message{Verarbeitung mit pdflatex, Ausgabe in pdf.}
  \ifnum \pdfoutput=0
    \pdffalse
  \setcounter{MOutput}{\MOutputDVI}
  \message{Verarbeitung mit pdflatex, Ausgabe in dvi.}
  \else
    \ifnum\pdfoutput=1
    \pdftrue
  \setcounter{MOutput}{\MOutputPDF}
  \message{Verarbeitung mit pdflatex, Ausgabe in pdf.}
    \fi
  \fi
\fi
\fi

\ifnum\value{MOutput}=\MOutputPDF
\DeclareGraphicsExtensions{.pdf,.png,.jpg}
\fi

\ifnum\value{MOutput}=\MOutputDVI
\DeclareGraphicsExtensions{.eps,.png,.jpg}
\fi

\ifnum\value{MOutput}=\MOutputHTML
% Wird vom Konverter leider nicht erkannt und daher in split.pm hardcodiert!
\DeclareGraphicsExtensions{.png,.jpg,.gif}
\fi

% Umdefinition der hyperref-Nummerierung im PDF-Modus
\ifttm
\else
\renewcommand{\theHfigure}{\arabic{chapter}.\arabic{section}.\arabic{figure}}
\fi

% Makro, um in der HTML-Ausgabe die zuerst zu oeffnende Datei zu kennzeichnen
\ifttm
\newcommand{\MGlobalStart}{\special{html:<!-- mglobalstarttag -->}}
\else
\newcommand{\MGlobalStart}{}
\fi

% Makro, um bei scormlogin ein pullen des Benutzers bei Aufruf der Seite zu erzwingen (typischerweise auf der Einstiegsseite)
\ifttm
\newcommand{\MPullSite}{\special{html:<!-- pullsite //-->}}
\else
\newcommand{\MPullSite}{}
\fi

% Makro, um in der HTML-Ausgabe die Kapiteluebersicht zu kennzeichnen
\ifttm
\newcommand{\MGlobalChapterTag}{\special{html:<!-- mglobalchaptertag -->}}
\else
\newcommand{\MGlobalChapterTag}{}
\fi

% Makro, um in der HTML-Ausgabe die Konfiguration zu kennzeichnen
\ifttm
\newcommand{\MGlobalConfTag}{\special{html:<!-- mglobalconfigtag -->}}
\else
\newcommand{\MGlobalConfTag}{}
\fi

% Makro, um in der HTML-Ausgabe die Standortbeschreibung zu kennzeichnen
\ifttm
\newcommand{\MGlobalLocationTag}{\special{html:<!-- mgloballocationtag -->}}
\else
\newcommand{\MGlobalLocationTag}{}
\fi

% Makro, um in der HTML-Ausgabe die persoenlichen Daten zu kennzeichnen
\ifttm
\newcommand{\MGlobalDataTag}{\special{html:<!-- mglobaldatatag -->}}
\else
\newcommand{\MGlobalDataTag}{}
\fi

% Makro, um in der HTML-Ausgabe die Suchseite zu kennzeichnen
\ifttm
\newcommand{\MGlobalSearchTag}{\special{html:<!-- mglobalsearchtag -->}}
\else
\newcommand{\MGlobalSearchTag}{}
\fi

% Makro, um in der HTML-Ausgabe die Favoritenseite zu kennzeichnen
\ifttm
\newcommand{\MGlobalFavoTag}{\special{html:<!-- mglobalfavoritestag -->}}
\else
\newcommand{\MGlobalFavoTag}{}
\fi

% Makro, um in der HTML-Ausgabe die Eingangstestseite zu kennzeichnen
\ifttm
\newcommand{\MGlobalSTestTag}{\special{html:<!-- mglobalstesttag -->}}
\else
\newcommand{\MGlobalSTestTag}{}
\fi

% Makro, um in der PDF-Ausgabe ein Wasserzeichen zu definieren
\ifttm
\newcommand{\MWatermarkSettings}{\relax}
\else
\newcommand{\MWatermarkSettings}{%
% \SetWatermarkText{(c) MINT-Kolleg Baden-W�rttemberg 2014}
% \SetWatermarkLightness{0.85}
% \SetWatermarkScale{1.5}
}
\fi

\ifttm
\newcommand{\MBinom}[2]{\left({\begin{array}{c} #1 \\ #2 \end{array}}\right)}
\else
\newcommand{\MBinom}[2]{\binom{#1}{#2}}
\fi

\ifttm
\newcommand{\DeclareMathOperator}[2]{\def#1{\mathrm{#2}}}
\newcommand{\operatorname}[1]{\mathrm{#1}}
\fi

%----------------- Makros fuer die gemischte HTML/PDF-Konvertierung ------------------------------

\newcommand{\MTestName}{\relax} % wird durch Test-Umgebung gesetzt

% Fuer experimentelle Kursinhalte, die im Release-Umsetzungsvorgang eine Fehlermeldung
% produzieren sollen aber sonst normal umgesetzt werden
\newenvironment{MExperimental}{%
}{%
}

% Wird von ttm nicht richtig umgesetzt!!
\newenvironment{MExerciseItems}{%
\renewcommand\theenumi{\alph{enumi}}%
\begin{enumerate}%
}{%
\end{enumerate}%
}


\definecolor{infoshadecolor}{rgb}{0.75,0.75,0.75}
\definecolor{exmpshadecolor}{rgb}{0.875,0.875,0.875}
\definecolor{expeshadecolor}{rgb}{0.95,0.95,0.95}
\definecolor{framecolor}{rgb}{0.2,0.2,0.2}

% Bei PDF-Uebersetzung wird hinter den Start jeder Satz/Info-aehnlichen Umgebung eine leere mbox gesetzt, damit
% fuehrende Listen oder enums nicht den Zeilenumbruch kaputtmachen
%\ifttm
\def\MTB{}
%\else
%\def\MTB{\mbox{}}
%\fi


\ifttm
\newcommand{\MRelates}{\special{html:<mi>&wedgeq;</mi>}}
\else
\def\MRelates{\stackrel{\scriptscriptstyle\wedge}{=}}
\fi

\def\MInch{\text{''}}
\def\Mdd{\textit{''}}

\ifttm
\def\MNL{ \newline }
\newenvironment{MArray}[1]{\begin{array}{#1}}{\end{array}}
\else
\def\MNL{ \\ }
\newenvironment{MArray}[1]{\begin{array}{#1}}{\end{array}}
\fi

\newcommand{\MBox}[1]{$\mathrm{#1}$}
\newcommand{\MMBox}[1]{\mathrm{#1}}


\ifttm%
\newcommand{\Mtfrac}[2]{{\textstyle \frac{#1}{#2}}}
\newcommand{\Mdfrac}[2]{{\displaystyle \frac{#1}{#2}}}
\newcommand{\Mmeasuredangle}{\special{html:<mi>&angmsd;</mi>}}
\else%
\newcommand{\Mtfrac}[2]{\tfrac{#1}{#2}}
\newcommand{\Mdfrac}[2]{\dfrac{#1}{#2}}
\newcommand{\Mmeasuredangle}{\measuredangle}
\relax
\fi

% Matrizen und Vektoren

% Inhalt wird in der Form a & b \\ c & d erwartet
% Vorsicht: MVector = Komponentenspalte, MVec = Variablensymbol
\ifttm%
\newcommand{\MVector}[1]{\left({\begin{array}{c}#1\end{array}}\right)}
\else%
\newcommand{\MVector}[1]{\begin{pmatrix}#1\end{pmatrix}}
\fi



\newcommand{\MVec}[1]{\vec{#1}}
\newcommand{\MDVec}[1]{\overrightarrow{#1}}

%----------------- Umgebungen fuer Definitionen und Saetze ----------------------------------------

% Fuegt einen Tabellen-Zeilenumbruch ein im PDF, aber nicht im HTML
\newcommand{\TSkip}{\ifttm \else&\ \\\fi}

\newenvironment{infoshaded}{%
\def\FrameCommand{\fboxsep=\FrameSep \fcolorbox{framecolor}{infoshadecolor}}%
\MakeFramed {\advance\hsize-\width \FrameRestore}}%
{\endMakeFramed}

\newenvironment{expeshaded}{%
\def\FrameCommand{\fboxsep=\FrameSep \fcolorbox{framecolor}{expeshadecolor}}%
\MakeFramed {\advance\hsize-\width \FrameRestore}}%
{\endMakeFramed}

\newenvironment{exmpshaded}{%
\def\FrameCommand{\fboxsep=\FrameSep \fcolorbox{framecolor}{exmpshadecolor}}%
\MakeFramed {\advance\hsize-\width \FrameRestore}}%
{\endMakeFramed}

\def\STDCOLOR{black}

\ifttm%
\else%
\newtheoremstyle{MSatzStyle}
  {1cm}                   %Space above
  {1cm}                   %Space below
  {\normalfont\itshape}   %Body font
  {}                      %Indent amount (empty = no indent,
                          %\parindent = para indent)
  {\normalfont\bfseries}  %Thm head font
  {}                      %Punctuation after thm head
  {\newline}              %Space after thm head: " " = normal interword
                          %space; \newline = linebreak
  {\thmname{#1}\thmnumber{ #2}\thmnote{ (#3)}}
                          %Thm head spec (can be left empty, meaning
                          %`normal')
                          %
\newtheoremstyle{MDefStyle}
  {1cm}                   %Space above
  {1cm}                   %Space below
  {\normalfont}           %Body font
  {}                      %Indent amount (empty = no indent,
                          %\parindent = para indent)
  {\normalfont\bfseries}  %Thm head font
  {}                      %Punctuation after thm head
  {\newline}              %Space after thm head: " " = normal interword
                          %space; \newline = linebreak
  {\thmname{#1}\thmnumber{ #2}\thmnote{ (#3)}}
                          %Thm head spec (can be left empty, meaning
                          %`normal')
\fi%

\newcommand{\MInfoText}{Info}

\newcounter{MHintCounter}
\newcounter{MCodeEditCounter}

\newcounter{MLastIndex}  % Enthaelt die dritte Stelle (Indexnummer) des letzten angelegten Objekts
\newcounter{MLastType}   % Enthaelt den Typ des letzten angelegten Objekts (mithilfe der unten definierten Konstanten). Die Entscheidung, wie der Typ dargstellt wird, wird in split.pm beim Postprocessing getroffen.
\newcounter{MLastTypeEq} % =1 falls das Label in einer Matheumgebung (equation, eqnarray usw.) steht, =2 falls das Label in einer table-Umgebung steht

% Da ttm keine Zahlmakros verarbeiten kann, werden diese Nummern in den Zuweisungen hardcodiert!
\def\MTypeSection{1}          %# Zaehler ist section
\def\MTypeSubsection{2}       %# Zaehler ist subsection
\def\MTypeSubsubsection{3}    %# Zaehler ist subsubsection
\def\MTypeInfo{4}             %# Eine Infobox, Separatzaehler fuer die Chemie (auch wenn es dort nicht nummeriert wird) ist MInfoCounter
\def\MTypeExercise{5}         %# Eine Aufgabe, Separatzaehler fuer die Chemie ist MExerciseCounter
\def\MTypeExample{6}          %# Eine Beispielbox, Separatzaehler fuer die Chemie ist MExampleCounter
\def\MTypeExperiment{7}       %# Eine Versuchsbox, Separatzaehler fuer die Chemie ist MExperimentCounter
\def\MTypeGraphics{8}         %# Eine Graphik, Separatzaehler fuer alle FB ist MGraphicsCounter
\def\MTypeTable{9}            %# Eine Tabellennummer, hat keinen Zaehler da durch table gezaehlt wird
\def\MTypeEquation{10}        %# Eine Gleichungsnummer, hat keinen Zaehler da durch equation/eqnarray gezaehlt wird
\def\MTypeTheorem{11}         % Ein theorem oder xtheorem, Separatzaehler fuer die Chemie ist MTheoremCounter
\def\MTypeVideo{12}           %# Ein Video,Separatzaehler fuer alle FB ist MVideoCounter
\def\MTypeEntry{13}           %# Ein Eintrag fuer die Stichwortliste, wird nicht gezaehlt sondern erhaelt im preparsing ein unique-label 

% Zaehler fuer das Labelsystem sind prefixcounter, jeder Zaehler wird VOR dem gezaehlten Objekt inkrementiert und zaehlt daher das aktuelle Objekt
\newcounter{MInfoCounter}
\newcounter{MExerciseCounter}
\newcounter{MExampleCounter}
\newcounter{MExperimentCounter}
\newcounter{MGraphicsCounter}
\newcounter{MTableCounter}
\newcounter{MEquationCounter}  % Nur im HTML, sonst durch "equation"-counter von latex realisiert
\newcounter{MTheoremCounter}
\newcounter{MObjectCounter}   % Gemeinsamer Zaehler fuer Objekte (ausser Grafiken/Tabellen) in Mathe/Info/Physik
\newcounter{MVideoCounter}
\newcounter{MEntryCounter}

\newcounter{MTestSite} % 1 = Subsubsection ist eine Pruefungsseite, 0 = ist eine normale Seite (inkl. Hilfeseite)

\def\MCell{$\phantom{a}$}

\newenvironment{MExportExercise}{\begin{MExercise}}{\end{MExercise}} % wird von mconvert abgefangen

\def\MGenerateExNumber{%
\ifnum\value{MSepNumbers}=0%
\arabic{section}.\arabic{subsection}.\arabic{MObjectCounter}\setcounter{MLastIndex}{\value{MObjectCounter}}%
\else%
\arabic{section}.\arabic{subsection}.\arabic{MExerciseCounter}\setcounter{MLastIndex}{\value{MExerciseCounter}}%
\fi%
}%

\def\MGenerateExmpNumber{%
\ifnum\value{MSepNumbers}=0%
\arabic{section}.\arabic{subsection}.\arabic{MObjectCounter}\setcounter{MLastIndex}{\value{MObjectCounter}}%
\else%
\arabic{section}.\arabic{subsection}.\arabic{MExerciseCounter}\setcounter{MLastIndex}{\value{MExampleCounter}}%
\fi%
}%

\def\MGenerateInfoNumber{%
\ifnum\value{MSepNumbers}=0%
\arabic{section}.\arabic{subsection}.\arabic{MObjectCounter}\setcounter{MLastIndex}{\value{MObjectCounter}}%
\else%
\arabic{section}.\arabic{subsection}.\arabic{MExerciseCounter}\setcounter{MLastIndex}{\value{MInfoCounter}}%
\fi%
}%

\def\MGenerateSiteNumber{%
\arabic{section}.\arabic{subsection}.\arabic{subsubsection}%
}%

% Funktionalitaet fuer Auswahlaufgaben

\newcounter{MExerciseCollectionCounter} % = 0 falls nicht in collection-Umgebung, ansonsten Schachtelungstiefe
\newcounter{MExerciseCollectionTextCounter} % wird von MExercise-Umgebung inkrementiert und von MExerciseCollection-Umgebung auf Null gesetzt

\ifttm
% MExerciseCollection gruppiert Aufgaben, die dynamisch aus der Datenbank gezogen werden und nicht direkt in der HTML-Seite stehen
% Parameter: #1 = ID der Collection, muss eindeutig fuer alle IN DER DB VORHANDENEN collections sein unabhaengig vom Kurs
%            #2 = Optionsargument (im Moment: 1 = Iterative Auswahl, 2 = Zufallsbasierte Auswahl)
\newenvironment{MExerciseCollection}[2]{%
\addtocounter{MExerciseCollectionCounter}{1}
\setcounter{MExerciseCollectionTextCounter}{0}
\special{html:<!-- mexercisecollectionstart;;}#1\special{html:;;}#2\special{html:;; //-->}%
}{%
\special{html:<!-- mexercisecollectionstop //-->}%
\addtocounter{MExerciseCollectionCounter}{-1}
}
\else
\newenvironment{MExerciseCollection}[2]{%
\addtocounter{MExerciseCollectionCounter}{1}
\setcounter{MExerciseCollectionTextCounter}{0}
}{%
\addtocounter{MExerciseCollectionCounter}{-1}
}
\fi

% Bei Uebersetzung nach PDF werden die theorem-Umgebungen verwendet, bei Uebersetzung in HTML ein manuelles Makro
\ifttm%

  \newenvironment{MHint}[1]{  \special{html:<button name="Name_MHint}\arabic{MHintCounter}\special{html:" class="hintbutton_closed" id="MHint}\arabic{MHintCounter}\special{html:_button" %
  type="button" onclick="toggle_hint('MHint}\arabic{MHintCounter}\special{html:');">}#1\special{html:</button>}
  \special{html:<div class="hint" style="display:none" id="MHint}\arabic{MHintCounter}\special{html:"> }}{\begin{html}</div>\end{html}\addtocounter{MHintCounter}{1}}

  \newenvironment{MCOSHZusatz}{  \special{html:<button name="Name_MHint}\arabic{MHintCounter}\special{html:" class="chintbutton_closed" id="MHint}\arabic{MHintCounter}\special{html:_button" %
  type="button" onclick="toggle_hint('MHint}\arabic{MHintCounter}\special{html:');">}Weiterf�hrende Inhalte\special{html:</button>}
  \special{html:<div class="hintc" style="display:none" id="MHint}\arabic{MHintCounter}\special{html:">
  <div class="coshwarn">Diese Inhalte gehen �ber das Kursniveau hinaus und werden in den Aufgaben und Tests nicht abgefragt.</div><br />}
  \addtocounter{MHintCounter}{1}}{\begin{html}</div>\end{html}}

  
  \newenvironment{MDefinition}{\begin{definition}\setcounter{MLastIndex}{\value{definition}}\ \\}{\end{definition}}

  
  \newenvironment{MExercise}{
  \renewcommand{\MStdPoints}{4}
  \addtocounter{MExerciseCounter}{1}
  \addtocounter{MObjectCounter}{1}
  \setcounter{MLastType}{5}

  \ifnum\value{MExerciseCollectionCounter}=0\else\addtocounter{MExerciseCollectionTextCounter}{1}\special{html:<!-- mexercisetextstart;;}\arabic{MExerciseCollectionTextCounter}\special{html:;; //-->}\fi
  \special{html:<div class="aufgabe" id="ADIV_}\MGenerateExNumber\special{html:">}%
  \textbf{Aufgabe \MGenerateExNumber
  } \ \\}{
  \special{html:</div><!-- mfeedbackbutton;Aufgabe;}\arabic{MTestSite}\special{html:;}\MGenerateExNumber\special{html:; //-->}
  \ifnum\value{MExerciseCollectionCounter}=0\else\special{html:<!-- mexercisetextstop //-->}\fi
  }

  % Stellt eine Kombination aus Aufgabe, Loesungstext und Eingabefeld bereit,
  % bei der Aufgabentext und Musterloesung sowie die zugehoerigen Feldelemente
  % extern bezogen und div-aktualisiert werden, das Eingabefeld aber immer das gleiche ist.
  \newenvironment{MFetchExercise}{
  \addtocounter{MExerciseCounter}{1}
  \addtocounter{MObjectCounter}{1}
  \setcounter{MLastType}{5}

  \special{html:<div class="aufgabe" id="ADIV_}\MGenerateExNumber\special{html:">}%
  \textbf{Aufgabe \MGenerateExNumber
  } \ \\%
  \special{html:</div><div class="exfetch_text" id="ADIVTEXT_}\MGenerateExNumber\special{html:">}%
  \special{html:</div><div class="exfetch_sol" id="ADIVSOL_}\MGenerateExNumber\special{html:">}%
  \special{html:</div><div class="exfetch_input" id="ADIVINPUT_}\MGenerateExNumber\special{html:">}%
  }{
  \special{html:</div>}
  }

  \newenvironment{MExample}{
  \addtocounter{MExampleCounter}{1}
  \addtocounter{MObjectCounter}{1}
  \setcounter{MLastType}{6}
  \begin{html}
  <div class="exmp">
  <div class="exmprahmen">
  \end{html}\textbf{Beispiel
  \ifnum\value{MSepNumbers}=0
  \arabic{section}.\arabic{subsection}.\arabic{MObjectCounter}\setcounter{MLastIndex}{\value{MObjectCounter}}
  \else
  \arabic{section}.\arabic{subsection}.\arabic{MExampleCounter}\setcounter{MLastIndex}{\value{MExampleCounter}}
  \fi
  } \ \\}{\begin{html}</div>
  </div>
  \end{html}
  \special{html:<!-- mfeedbackbutton;Beispiel;}\arabic{MTestSite}\special{html:;}\MGenerateExmpNumber\special{html:; //-->}
  }

  \newenvironment{MExperiment}{
  \addtocounter{MExperimentCounter}{1}
  \addtocounter{MObjectCounter}{1}
  \setcounter{MLastType}{7}
  \begin{html}
  <div class="expe">
  <div class="experahmen">
  \end{html}\textbf{Versuch
  \ifnum\value{MSepNumbers}=0
  \arabic{section}.\arabic{subsection}.\arabic{MObjectCounter}\setcounter{MLastIndex}{\value{MObjectCounter}}
  \else
%  \arabic{MExperimentCounter}\setcounter{MLastIndex}{\value{MExperimentCounter}}
  \arabic{section}.\arabic{subsection}.\arabic{MExperimentCounter}\setcounter{MLastIndex}{\value{MExperimentCounter}}
  \fi
  } \ \\}{\begin{html}</div>
  </div>
  \end{html}}

  \newenvironment{MChemInfo}{
  \setcounter{MLastType}{4}
  \begin{html}
  <div class="info">
  <div class="inforahmen">
  \end{html}}{\begin{html}</div>
  </div>
  \end{html}}

  \newenvironment{MXInfo}[1]{
  \addtocounter{MInfoCounter}{1}
  \addtocounter{MObjectCounter}{1}
  \setcounter{MLastType}{4}
  \begin{html}
  <div class="info">
  <div class="inforahmen">
  \end{html}\textbf{#1
  \ifnum\value{MInfoNumbers}=0
  \else
    \ifnum\value{MSepNumbers}=0
    \arabic{section}.\arabic{subsection}.\arabic{MObjectCounter}\setcounter{MLastIndex}{\value{MObjectCounter}}
    \else
    \arabic{MInfoCounter}\setcounter{MLastIndex}{\value{MInfoCounter}}
    \fi
  \fi
  } \ \\}{\begin{html}</div>
  </div>
  \end{html}
  \special{html:<!-- mfeedbackbutton;Info;}\arabic{MTestSite}\special{html:;}\MGenerateInfoNumber\special{html:; //-->}
  }

  \newenvironment{MInfo}{\ifnum\value{MInfoNumbers}=0\begin{MChemInfo}\else\begin{MXInfo}{Info}\ \\ \fi}{\ifnum\value{MInfoNumbers}=0\end{MChemInfo}\else\end{MXInfo}\fi}

\else%

  \theoremstyle{MSatzStyle}
  \newtheorem{thm}{Satz}[section]
  \newtheorem{thmc}{Satz}
  \theoremstyle{MDefStyle}
  \newtheorem{defn}[thm]{Definition}
  \newtheorem{exmp}[thm]{Beispiel}
  \newtheorem{info}[thm]{\MInfoText}
  \theoremstyle{MDefStyle}
  \newtheorem{defnc}{Definition}
  \theoremstyle{MDefStyle}
  \newtheorem{exmpc}{Beispiel}[section]
  \theoremstyle{MDefStyle}
  \newtheorem{infoc}{\MInfoText}
  \theoremstyle{MDefStyle}
  \newtheorem{exrc}{Aufgabe}[section]
  \theoremstyle{MDefStyle}
  \newtheorem{verc}{Versuch}[section]
  
  \newenvironment{MFetchExercise}{}{} % kann im PDF nicht dargestellt werden
  
  \newenvironment{MExercise}{\begin{exrc}\renewcommand{\MStdPoints}{1}\MTB}{\end{exrc}}
  \newenvironment{MHint}[1]{\ \\ \underline{#1:}\\}{}
  \newenvironment{MCOSHZusatz}{\ \\ \underline{Weiterf�hrende Inhalte:}\\}{}
  \newenvironment{MDefinition}{\ifnum\value{MInfoNumbers}=0\begin{defnc}\else\begin{defn}\fi\MTB}{\ifnum\value{MInfoNumbers}=0\end{defnc}\else\end{defn}\fi}
%  \newenvironment{MExample}{\begin{exmp}}{\ \linebreak[1] \ \ \ \ $\phantom{a}$ \ \hfill $\blacklozenge$\end{exmp}}
  \newenvironment{MExample}{
    \ifnum\value{MInfoNumbers}=0\begin{exmpc}\else\begin{exmp}\fi
    \MTB
    \begin{exmpshaded}
    \ \newline
}{
    \end{exmpshaded}
    \ifnum\value{MInfoNumbers}=0\end{exmpc}\else\end{exmp}\fi
}
  \newenvironment{MChemInfo}{\begin{infoshaded}}{\end{infoshaded}}

  \newenvironment{MInfo}{\ifnum\value{MInfoNumbers}=0\begin{MChemInfo}\else\renewcommand{\MInfoText}{Info}\begin{info}\begin{infoshaded}
  \MTB
   \ \newline
    \fi
  }{\ifnum\value{MInfoNumbers}=0\end{MChemInfo}\else\end{infoshaded}\end{info}\fi}

  \newenvironment{MXInfo}[1]{
    \renewcommand{\MInfoText}{#1}
    \ifnum\value{MInfoNumbers}=0\begin{infoc}\else\begin{info}\fi%
    \MTB
    \begin{infoshaded}
    \ \newline
  }{\end{infoshaded}\ifnum\value{MInfoNumbers}=0\end{infoc}\else\end{info}\fi}

  \newenvironment{MExperiment}{
    \renewcommand{\MInfoText}{Versuch}
    \ifnum\value{MInfoNumbers}=0\begin{verc}\else\begin{info}\fi
    \MTB
    \begin{expeshaded}
    \ \newline
  }{
    \end{expeshaded}
    \ifnum\value{MInfoNumbers}=0\end{verc}\else\end{info}\fi
  }
\fi%

% MHint sollte nicht direkt fuer Loesungen benutzt werden wegen solutionselect
\newenvironment{MSolution}{\begin{MHint}{L"osung}}{\end{MHint}}

\newcounter{MCodeCounter}

\ifttm
\newenvironment{MCode}{\special{html:<!-- mcodestart -->}\ttfamily\color{blue}}{\special{html:<!-- mcodestop -->}}
\else
\newenvironment{MCode}{\begin{flushleft}\ttfamily\addtocounter{MCodeCounter}{1}}{\addtocounter{MCodeCounter}{-1}\end{flushleft}}
% Ohne color-Statement da inkompatible mit framed/shaded-Boxen aus dem framed-package
\fi

%----------------- Sonderdefinitionen fuer Symbole, die der Konverter nicht kann ----------------------------------------------

\ifttm%
\newcommand{\MUnderset}[2]{\underbrace{#2}_{#1}}%
\else%
\newcommand{\MUnderset}[2]{\underset{#1}{#2}}%
\fi%

\ifttm
\newcommand{\MThinspace}{\special{html:<mi>&#x2009;</mi>}}
\else
\newcommand{\MThinspace}{\,}
\fi

\ifttm
\newcommand{\glq}{\begin{html}&sbquo;\end{html}}
\newcommand{\grq}{\begin{html}&lsquo;\end{html}}
\newcommand{\glqq}{\begin{html}&bdquo;\end{html}}
\newcommand{\grqq}{\begin{html}&ldquo;\end{html}}
\fi

\ifttm
\newcommand{\MNdash}{\begin{html}&ndash;\end{html}}
\else
\newcommand{\MNdash}{--}
\fi

%\ifttm\def\MIU{\special{html:<mi>&#8520;</mi>}}\else\def\MIU{\mathrm{i}}\fi
\def\MIU{\mathrm{i}}
\def\MEU{e} % TU9-Onlinekurs: italic-e
%\def\MEU{\mathrm{e}} % Alte Onlinemodule: roman-e
\def\MD{d} % Kursives d in Integralen im TU9-Onlinekurs
%\def\MD{\mathrm{d}} % roman-d in den alten Onlinemodulen
\def\MDB{\|}

%zusaetzlicher Leerraum vor "\MD"
\ifttm%
\def\MDSpace{\special{html:<mi>&#x2009;</mi>}}
\else%
\def\MDSpace{\,}
\fi%
\newcommand{\MDwSp}{\MDSpace\MD}%

\ifttm
\def\Mdq{\dq}
\else
\def\Mdq{\dq}
\fi

\def\MSpan#1{\left<{#1}\right>}
\def\MSetminus{\setminus}
\def\MIM{I}

\ifttm
\newcommand{\ld}{\text{ld}}
\newcommand{\lg}{\text{lg}}
\else
\DeclareMathOperator{\ld}{ld}
%\newcommand{\lg}{\text{lg}} % in latex schon definiert
\fi


\def\Mmapsto{\ifttm\special{html:<mi>&mapsto;</mi>}\else\mapsto\fi} 
\def\Mvarphi{\ifttm\phi\else\varphi\fi}
\def\Mphi{\ifttm\varphi\else\phi\fi}
\ifttm%
\newcommand{\MEumu}{\special{html:<mi>&#x3BC;</mi>}}%
\else%
\newcommand{\MEumu}{\textrm{\textmu}}%
\fi
\def\Mvarepsilon{\ifttm\epsilon\else\varepsilon\fi}
\def\Mepsilon{\ifttm\varepsilon\else\epsilon\fi}
\def\Mvarkappa{\ifttm\kappa\else\varkappa\fi}
\def\Mkappa{\ifttm\varkappa\else\kappa\fi}
\def\Mcomplement{\ifttm\special{html:<mi>&comp;</mi>}\else\complement\fi} 
\def\MWW{\mathrm{WW}}
\def\Mmod{\ifttm\special{html:<mi>&nbsp;mod&nbsp;</mi>}\else\mod\fi} 

\ifttm%
\def\mod{\text{\;mod\;}}%
\def\MNEquiv{\special{html:<mi>&NotCongruent;</mi>}}% 
\def\MNSubseteq{\special{html:<mi>&NotSubsetEqual;</mi>}}%
\def\MEmptyset{\special{html:<mi>&empty;</mi>}}%
\def\MVDots{\special{html:<mi>&#x22EE;</mi>}}%
\def\MHDots{\special{html:<mi>&#x2026;</mi>}}%
\def\Mddag{\special{html:<mi>&#x1202;</mi>}}%
\def\sphericalangle{\special{html:<mi>&measuredangle;</mi>}}%
\def\nparallel{\special{html:<mi>&nparallel;</mi>}}%
\def\MProofEnd{\special{html:<mi>&#x25FB;</mi>}}%
\newenvironment{MProof}[1]{\underline{#1}:\MCR\MCR}{\hfill $\MProofEnd$}%
\else%
\def\MNEquiv{\not\equiv}%
\def\MNSubseteq{\not\subseteq}%
\def\MEmptyset{\emptyset}%
\def\MVDots{\vdots}%
\def\MHDots{\hdots}%
\def\Mddag{\ddag}%
\newenvironment{MProof}[1]{\begin{proof}[#1]}{\end{proof}}%
\fi%



% Spaces zum Auffuellen von Tabellenbreiten, die nur im HTML wirken
\ifttm%
\def\MTSP{\:}%
\else%
\def\MTSP{}%
\fi%

\DeclareMathOperator{\arsinh}{arsinh}
\DeclareMathOperator{\arcosh}{arcosh}
\DeclareMathOperator{\artanh}{artanh}
\DeclareMathOperator{\arcoth}{arcoth}


\newcommand{\MMathSet}[1]{\mathbb{#1}}
\def\N{\MMathSet{N}}
\def\Z{\MMathSet{Z}}
\def\Q{\MMathSet{Q}}
\def\R{\MMathSet{R}}
\def\C{\MMathSet{C}}

\newcounter{MForLoopCounter}
\newcommand{\MForLoop}[2]{\setcounter{MForLoopCounter}{#1}\ifnum\value{MForLoopCounter}=0{}\else{{#2}\addtocounter{MForLoopCounter}{-1}\MForLoop{\value{MForLoopCounter}}{#2}}\fi}

\newcounter{MSiteCounter}
\newcounter{MFieldCounter} % Kombination section.subsection.site.field ist eindeutig in allen Modulen, field alleine nicht

\newcounter{MiniMarkerCounter}

\ifttm
\newenvironment{MMiniPageP}[1]{\begin{minipage}{#1\linewidth}\special{html:<!-- minimarker;;}\arabic{MiniMarkerCounter}\special{html:;;#1; //-->}}{\end{minipage}\addtocounter{MiniMarkerCounter}{1}}
\else
\newenvironment{MMiniPageP}[1]{\begin{minipage}{#1\linewidth}}{\end{minipage}\addtocounter{MiniMarkerCounter}{1}}
\fi

\newcounter{AlignCounter}

\newcommand{\MStartJustify}{\ifttm\special{html:<!-- startalign;;}\arabic{AlignCounter}\special{html:;;justify; //-->}\fi}
\newcommand{\MStopJustify}{\ifttm\special{html:<!-- stopalign;;}\arabic{AlignCounter}\special{html:; //-->}\fi\addtocounter{AlignCounter}{1}}

\newenvironment{MJTabular}[1]{
\MStartJustify
\begin{tabular}{#1}
}{
\end{tabular}
\MStopJustify
}

\newcommand{\MImageLeft}[2]{
\begin{center}
\begin{tabular}{lc}
\MStartJustify
\begin{MMiniPageP}{0.65}
#1
\end{MMiniPageP}
\MStopJustify
&
\begin{MMiniPageP}{0.3}
#2  
\end{MMiniPageP}
\end{tabular}
\end{center}
}

\newcommand{\MImageHalf}[2]{
\begin{center}
\begin{tabular}{lc}
\MStartJustify
\begin{MMiniPageP}{0.45}
#1
\end{MMiniPageP}
\MStopJustify
&
\begin{MMiniPageP}{0.45}
#2  
\end{MMiniPageP}
\end{tabular}
\end{center}
}

\newcommand{\MBigImageLeft}[2]{
\begin{center}
\begin{tabular}{lc}
\MStartJustify
\begin{MMiniPageP}{0.25}
#1
\end{MMiniPageP}
\MStopJustify
&
\begin{MMiniPageP}{0.7}
#2  
\end{MMiniPageP}
\end{tabular}
\end{center}
}

\ifttm
\def\No{\mathbb{N}_0}
\else
\def\No{\ensuremath{\N_0}}
\fi
\def\MT{\textrm{\tiny T}}
\newcommand{\MTranspose}[1]{{#1}^{\MT}}
\ifttm
\newcommand{\MRe}{\mathsf{Re}}
\newcommand{\MIm}{\mathsf{Im}}
\else
\DeclareMathOperator{\MRe}{Re}
\DeclareMathOperator{\MIm}{Im}
\fi

\newcommand{\Mid}{\mathrm{id}}
\newcommand{\MFeinheit}{\mathrm{feinh}}

\ifttm
\newcommand{\Msubstack}[1]{\begin{array}{c}{#1}\end{array}}
\else
\newcommand{\Msubstack}[1]{\substack{#1}}
\fi

% Typen von Fragefeldern:
% 1 = Alphanumerisch, case-sensitive-Vergleich
% 2 = Ja/Nein-Checkbox, Loesung ist 0 oder 1   (OPTION = Image-id fuer Rueckmeldung)
% 3 = Reelle Zahlen Geparset
% 4 = Funktionen Geparset (mit Stuetzstellen zur ueberpruefung)

% Dieser Befehl erstellt ein interaktives Aufgabenfeld. Parameter:
% - #1 Laenge in Zeichen
% - #2 Loesungstext (alphanumerisch, case sensitive)
% - #3 AufgabenID (alphanumerisch, case sensitive)
% - #4 Typ (Kennnummer)
% - #5 String fuer Optionen (ggf. mit Semikolon getrennte Einzelstrings)
% - #6 Anzahl Punkte
% - #7 uxid (kann z.B. Loesungsstring sein)
% ACHTUNG: Die langen Zeilen bitte so lassen, Zeilenumbrueche im tex werden in div's umgesetzt
\newcommand{\MQuestionID}[7]{
\ifttm
\special{html:<!-- mdeclareuxid;;}UX#7\special{html:;;}\arabic{section}\special{html:;;}#3\special{html:;; //-->}%
\special{html:<!-- mdeclarepoints;;}\arabic{section}\special{html:;;}#3\special{html:;;}#6\special{html:;;}\arabic{MTestSite}\special{html:;;}\arabic{chapter}%
\special{html:;; //--><!-- onloadstart //-->CreateQuestionObj("}#7\special{html:",}\arabic{MFieldCounter}\special{html:,"}#2%
\special{html:","}#3\special{html:",}#4\special{html:,"}#5\special{html:",}#6\special{html:,}\arabic{MTestSite}\special{html:,}\arabic{section}%
\special{html:);<!-- onloadstop //-->}%
\special{html:<input mfieldtype="}#4\special{html:" name="Name_}#3\special{html:" id="}#3\special{html:" type="text" size="}#1\special{html:" maxlength="}#1%
\special{html:" }\ifnum\value{MGroupActive}=0\special{html:onfocus="handlerFocus(}\arabic{MFieldCounter}%
\special{html:);" onblur="handlerBlur(}\arabic{MFieldCounter}\special{html:);" onkeyup="handlerChange(}\arabic{MFieldCounter}\special{html:,0);" onpaste="handlerChange(}\arabic{MFieldCounter}\special{html:,0);" oninput="handlerChange(}\arabic{MFieldCounter}\special{html:,0);" onpropertychange="handlerChange(}\arabic{MFieldCounter}\special{html:,0);"/>}%
\special{html:<img src="images/questionmark.gif" width="20" height="20" border="0" align="absmiddle" id="}QM#3\special{html:"/>}
\else%
\special{html:onblur="handlerBlur(}\arabic{MFieldCounter}%
\special{html:);" onfocus="handlerFocus(}\arabic{MFieldCounter}\special{html:);" onkeyup="handlerChange(}\arabic{MFieldCounter}\special{html:,1);" onpaste="handlerChange(}\arabic{MFieldCounter}\special{html:,1);" oninput="handlerChange(}\arabic{MFieldCounter}\special{html:,1);" onpropertychange="handlerChange(}\arabic{MFieldCounter}\special{html:,1);"/>}%
\special{html:<img src="images/questionmark.gif" width="20" height="20" border="0" align="absmiddle" id="}QM#3\special{html:"/>}\fi%
\else%
\ifnum\value{QBoxFlag}=1\fbox{$\phantom{\MForLoop{#1}{b}}$}\else$\phantom{\MForLoop{#1}{b}}$\fi%
\fi%
}

% ACHTUNG: Die langen Zeilen bitte so lassen, Zeilenumbrueche im tex werden in div's umgesetzt
% QuestionCheckbox macht ausserhalb einer QuestionGroup keinen Sinn!
% #1 = solution (1 oder 0), ggf. mit ::smc abgetrennt auszuschliessende single-choice-boxen (UXIDs durch , getrennt), #2 = id, #3 = points, #4 = uxid
\newcommand{\MQuestionCheckbox}[4]{
\ifttm
\special{html:<!-- mdeclareuxid;;}UX#4\special{html:;;}\arabic{section}\special{html:;;}#2\special{html:;; //-->}%
\ifnum\value{MGroupActive}=0\MDebugMessage{ERROR: Checkbox Nr. \arabic{MFieldCounter}\ ist nicht in einer Kontrollgruppe, es wird niemals eine Loesung angezeigt!}\fi
\special{html: %
<!-- mdeclarepoints;;}\arabic{section}\special{html:;;}#2\special{html:;;}#3\special{html:;;}\arabic{MTestSite}\special{html:;;}\arabic{chapter}%
\special{html:;; //--><!-- onloadstart //-->CreateQuestionObj("}#4\special{html:",}\arabic{MFieldCounter}\special{html:,"}#1\special{html:","}#2\special{html:",2,"IMG}#2%
\special{html:",}#3\special{html:,}\arabic{MTestSite}\special{html:,}\arabic{section}\special{html:);<!-- onloadstop //-->}%
\special{html:<input mfieldtype="2" type="checkbox" name="Name_}#2\special{html:" id="}#2\special{html:" onchange="handlerChange(}\arabic{MFieldCounter}\special{html:,1);"/><img src="images/questionmark.gif" name="}Name_IMG#2%
\special{html:" width="20" height="20" border="0" align="absmiddle" id="}IMG#2\special{html:"/> }%
\else%
\ifnum\value{QBoxFlag}=1\fbox{$\phantom{X}$}\else$\phantom{X}$\fi%
\fi%
}

\def\MGenerateID{QFELD_\arabic{section}.\arabic{subsection}.\arabic{MSiteCounter}.QF\arabic{MFieldCounter}}

% #1 = 0/1 ggf. mit ::smc abgetrennt auszuschliessende single-choice-boxen (UXIDs durch , getrennt ohne UX), #2 = uxid ohne UX
\newcommand{\MCheckbox}[2]{
\MQuestionCheckbox{#1}{\MGenerateID}{\MStdPoints}{#2}
\addtocounter{MFieldCounter}{1}
}

% Erster Parameter: Zeichenlaenge der Eingabebox, zweiter Parameter: Loesungstext
\newcommand{\MQuestion}[2]{
\MQuestionID{#1}{#2}{\MGenerateID}{1}{0}{\MStdPoints}{#2}
\addtocounter{MFieldCounter}{1}
}

% Erster Parameter: Zeichenlaenge der Eingabebox, zweiter Parameter: Loesungstext
\newcommand{\MLQuestion}[3]{
\MQuestionID{#1}{#2}{\MGenerateID}{1}{0}{\MStdPoints}{#3}
\addtocounter{MFieldCounter}{1}
}

% Parameter: Laenge des Feldes, Loesung (wird auch geparsed), Stellen Genauigkeit hinter dem Komma, weitere Stellen werden mathematisch gerundet vor Vergleich
\newcommand{\MParsedQuestion}[3]{
\MQuestionID{#1}{#2}{\MGenerateID}{3}{#3}{\MStdPoints}{#2}
\addtocounter{MFieldCounter}{1}
}

% Parameter: Laenge des Feldes, Loesung (wird auch geparsed), Stellen Genauigkeit hinter dem Komma, weitere Stellen werden mathematisch gerundet vor Vergleich
\newcommand{\MLParsedQuestion}[4]{
\MQuestionID{#1}{#2}{\MGenerateID}{3}{#3}{\MStdPoints}{#4}
\addtocounter{MFieldCounter}{1}
}

% Parameter: Laenge des Feldes, Loesungsfunktion, Anzahl Stuetzstellen, Funktionsvariablen durch Kommata getrennt (nicht case-sensitive), Anzahl Nachkommastellen im Vergleich
\newcommand{\MFunctionQuestion}[5]{
\MQuestionID{#1}{#2}{\MGenerateID}{4}{#3;#4;#5;0}{\MStdPoints}{#2}
\addtocounter{MFieldCounter}{1}
}

% Parameter: Laenge des Feldes, Loesungsfunktion, Anzahl Stuetzstellen, Funktionsvariablen durch Kommata getrennt (nicht case-sensitive), Anzahl Nachkommastellen im Vergleich, UXID
\newcommand{\MLFunctionQuestion}[6]{
\MQuestionID{#1}{#2}{\MGenerateID}{4}{#3;#4;#5;0}{\MStdPoints}{#6}
\addtocounter{MFieldCounter}{1}
}

% Parameter: Laenge des Feldes, Loesungsintervall, Genauigkeit der Zahlenwertpruefung
\newcommand{\MIntervalQuestion}[3]{
\MQuestionID{#1}{#2}{\MGenerateID}{6}{#3}{\MStdPoints}{#2}
\addtocounter{MFieldCounter}{1}
}

% Parameter: Laenge des Feldes, Loesungsintervall, Genauigkeit der Zahlenwertpruefung, UXID
\newcommand{\MLIntervalQuestion}[4]{
\MQuestionID{#1}{#2}{\MGenerateID}{6}{#3}{\MStdPoints}{#4}
\addtocounter{MFieldCounter}{1}
}

% Parameter: Laenge des Feldes, Loesungsfunktion, Anzahl Stuetzstellen, Funktionsvariable (nicht case-sensitive), Anzahl Nachkommastellen im Vergleich, Vereinfachungsbedingung
% Vereinfachungsbedingung ist eine der Folgenden:
% 0 = Keine Vereinfachungsbedingung
% 1 = Keine Klammern (runde oder eckige) mehr im vereinfachten Ausdruck
% 2 = Faktordarstellung (Term hat Produkte als letzte Operation, Summen als vorgeschaltete Operation)
% 3 = Summendarstellung (Term hat Summen als letzte Operation, Produkte als vorgeschaltete Operation)
% Flag 512: Besondere Stuetzstellen (nur >1 und nur schwach rational), sonst symmetrisch um Nullpunkt und ganze Zahlen inkl. Null werden getroffen
\newcommand{\MSimplifyQuestion}[6]{
\MQuestionID{#1}{#2}{\MGenerateID}{4}{#3;#4;#5;#6}{\MStdPoints}{#2}
\addtocounter{MFieldCounter}{1}
}

\newcommand{\MLSimplifyQuestion}[7]{
\MQuestionID{#1}{#2}{\MGenerateID}{4}{#3;#4;#5;#6}{\MStdPoints}{#7}
\addtocounter{MFieldCounter}{1}
}

% Parameter: Laenge des Feldes, Loesung (optionaler Ausdruck), Anzahl Stuetzstellen, Funktionsvariable (nicht case-sensitive), Anzahl Nachkommastellen im Vergleich, Spezialtyp (string-id)
\newcommand{\MLSpecialQuestion}[7]{
\MQuestionID{#1}{#2}{\MGenerateID}{7}{#3;#4;#5;#6}{\MStdPoints}{#7}
\addtocounter{MFieldCounter}{1}
}

\newcounter{MGroupStart}
\newcounter{MGroupEnd}
\newcounter{MGroupActive}

\newenvironment{MQuestionGroup}{
\setcounter{MGroupStart}{\value{MFieldCounter}}
\setcounter{MGroupActive}{1}
}{
\setcounter{MGroupActive}{0}
\setcounter{MGroupEnd}{\value{MFieldCounter}}
\addtocounter{MGroupEnd}{-1}
}

\newcommand{\MGroupButton}[1]{
\ifttm
\special{html:<button name="Name_Group}\arabic{MGroupStart}\special{html:to}\arabic{MGroupEnd}\special{html:" id="Group}\arabic{MGroupStart}\special{html:to}\arabic{MGroupEnd}\special{html:" %
type="button" onclick="group_button(}\arabic{MGroupStart}\special{html:,}\arabic{MGroupEnd}\special{html:);">}#1\special{html:</button>}
\else
\phantom{#1}
\fi
}

%----------------- Makros fuer die modularisierte Darstellung ------------------------------------

\def\MyText#1{#1}

% is used internally by the conversion package, should not be used by original tex documents
\def\MOrgLabel#1{\relax}

\ifttm

% Ein MLabel wird im html codiert durch das tag <!-- mmlabel;;Labelbezeichner;;SubjectArea;;chapter;;section;;subsection;;Index;;Objekttyp; //-->
\def\MLabel#1{%
\ifnum\value{MLastType}=8%
\ifnum\value{MCaptionOn}=0%
\MDebugMessage{ERROR: Grafik \arabic{MGraphicsCounter} hat separates label: #1 (Grafiklabels sollten nur in der Caption stehen)}%
\fi
\fi
\ifnum\value{MLastType}=12%
\ifnum\value{MCaptionOn}=0%
\MDebugMessage{ERROR: Video \arabic{MVideoCounter} hat separates label: #1 (Videolabels sollten nur in der Caption stehen}%
\fi
\fi
\ifnum\value{MLastType}=10\setcounter{MLastIndex}{\value{equation}}\fi
\label{#1}\begin{html}<!-- mmlabel;;#1;;\end{html}\arabic{MSubjectArea}\special{html:;;}\arabic{chapter}\special{html:;;}\arabic{section}\special{html:;;}\arabic{subsection}\special{html:;;}\arabic{MLastIndex}\special{html:;;}\arabic{MLastType}\special{html:; //-->}}%

\else

% Sonderbehandlung im PDF fuer Abbildungen in separater aux-Datei, da MGraphics die figure-Umgebung nicht verwendet
\def\MLabel#1{%
\ifnum\value{MLastType}=8%
\ifnum\value{MCaptionOn}=0%
\MDebugMessage{ERROR: Grafik \arabic{MGraphicsCounter} hat separates label: #1 (Grafiklabels sollten nur in der Caption stehen}%
\fi
\fi
\ifnum\value{MLastType}=12%
\ifnum\value{MCaptionOn}=0%
\MDebugMessage{ERROR: Video \arabic{MVideoCounter} hat separates label: #1 (Videolabels sollten nur in der Caption stehen}%
\fi
\fi
\label{#1}%
}%

\fi

% Gibt Begriff des referenzierten Objekts mit aus, aber nur im HTML, daher nur in Ausnahmefaellen (z.B. Copyrightliste) sinnvoll
\def\MCRef#1{\ifttm\special{html:<!-- mmref;;}#1\special{html:;;1; //-->}\else\vref{#1}\fi}


\def\MRef#1{\ifttm\special{html:<!-- mmref;;}#1\special{html:;;0; //-->}\else\vref{#1}\fi}
\def\MERef#1{\ifttm\special{html:<!-- mmref;;}#1\special{html:;;0; //-->}\else\eqref{#1}\fi}
\def\MNRef#1{\ifttm\special{html:<!-- mmref;;}#1\special{html:;;0; //-->}\else\ref{#1}\fi}
\def\MSRef#1#2{\ifttm\special{html:<!-- msref;;}#1\special{html:;;}#2\special{html:; //-->}\else \if#2\empty \ref{#1} \else \hyperref[#1]{#2}\fi\fi} 

\def\MRefRange#1#2{\ifttm\MRef{#1} bis 
\MRef{#2}\else\vrefrange[\unskip]{#1}{#2}\fi}

\def\MRefTwo#1#2{\ifttm\MRef{#1} und \MRef{#2}\else%
\let\vRefTLRsav=\reftextlabelrange\let\vRefTPRsav=\reftextpagerange%
\def\reftextlabelrange##1##2{\ref{##1} und~\ref{##2}}%
\def\reftextpagerange##1##2{auf den Seiten~\pageref{#1} und~\pageref{#2}}%
\vrefrange[\unskip]{#1}{#2}%
\let\reftextlabelrange=\vRefTLRsav\let\reftextpagerange=\vRefTPRsav\fi}

% MSectionChapter definiert falls notwendig das Kapitel vor der section. Das ist notwendig, wenn nur ein Einzelmodul uebersetzt wird.
% MChaptersGiven ist ein Counter, der von mconvert.pl vordefiniert wird.
\ifttm
\newcommand{\MSectionChapter}{\ifnum\value{MChaptersGiven}=0{\Dchapter{Modul}}\else{}\fi}
\else
\newcommand{\MSectionChapter}{\ifnum\value{chapter}=0{\Dchapter{Modul}}\else{}\fi}
\fi


\def\MChapter#1{\ifnum\value{MSSEnd}>0{\MSubsectionEndMacros}\addtocounter{MSSEnd}{-1}\fi\Dchapter{#1}}
\def\MSubject#1{\MChapter{#1}} % Schluesselwort HELPSECTION ist reserviert fuer Hilfesektion

\newcommand{\MSectionID}{UNKNOWNID}

\ifttm
\newcommand{\MSetSectionID}[1]{\renewcommand{\MSectionID}{#1}}
\else
\newcommand{\MSetSectionID}[1]{\renewcommand{\MSectionID}{#1}\tikzsetexternalprefix{#1}}
\fi


\newcommand{\MSection}[1]{\MSetSectionID{MODULID}\ifnum\value{MSSEnd}>0{\MSubsectionEndMacros}\addtocounter{MSSEnd}{-1}\fi\MSectionChapter\Dsection{#1}\MSectionStartMacros{#1}\setcounter{MLastIndex}{-1}\setcounter{MLastType}{1}} % Sections werden ueber das section-Feld im mmlabel-Tag identifiziert, nicht ueber das Indexfeld

\def\MSubsection#1{\ifnum\value{MSSEnd}>0{\MSubsectionEndMacros}\addtocounter{MSSEnd}{-1}\fi\ifttm\else\clearpage\fi\Dsubsection{#1}\MSubsectionStartMacros\setcounter{MLastIndex}{-1}\setcounter{MLastType}{2}\addtocounter{MSSEnd}{1}}% Subsections werden ueber das subsection-Feld im mmlabel-Tag identifiziert, nicht ueber das Indexfeld
\def\MSubsectionx#1{\Dsubsectionx{#1}} % Nur zur Verwendung in MSectionStart gedacht
\def\MSubsubsection#1{\Dsubsubsection{#1}\setcounter{MLastIndex}{\value{subsubsection}}\setcounter{MLastType}{3}\ifttm\special{html:<!-- sectioninfo;;}\arabic{section}\special{html:;;}\arabic{subsection}\special{html:;;}\arabic{subsubsection}\special{html:;;1;;}\arabic{MTestSite}\special{html:; //-->}\fi}
\def\MSubsubsectionx#1{\Dsubsubsectionx{#1}\ifttm\special{html:<!-- sectioninfo;;}\arabic{section}\special{html:;;}\arabic{subsection}\special{html:;;}\arabic{subsubsection}\special{html:;;0;;}\arabic{MTestSite}\special{html:; //-->}\else\addcontentsline{toc}{subsection}{#1}\fi}

\ifttm
\def\MSubsubsubsectionx#1{\ \newline\textbf{#1}\special{html:<br />}}
\else
\def\MSubsubsubsectionx#1{\ \newline
\textbf{#1}\ \\
}
\fi


% Dieses Skript wird zu Beginn jedes Modulabschnitts (=Webseite) ausgefuehrt und initialisiert den Aufgabenfeldzaehler
\newcommand{\MPageScripts}{
\setcounter{MFieldCounter}{1}
\addtocounter{MSiteCounter}{1}
\setcounter{MHintCounter}{1}
\setcounter{MCodeEditCounter}{1}
\setcounter{MGroupActive}{0}
\DoQBoxes
% Feldvariablen werden im HTML-Header in conv.pl eingestellt
}

% Dieses Skript wird zum Ende jedes Modulabschnitts (=Webseite) ausgefuehrt
\ifttm
\newcommand{\MEndScripts}{\special{html:<br /><!-- mfeedbackbutton;Seite;}\arabic{MTestSite}\special{html:;}\MGenerateSiteNumber\special{html:; //-->}
}
\else
\newcommand{\MEndScripts}{\relax}
\fi


\newcounter{QBoxFlag}
\newcommand{\DoQBoxes}{\setcounter{QBoxFlag}{1}}
\newcommand{\NoQBoxes}{\setcounter{QBoxFlag}{0}}

\newcounter{MXCTest}
\newcounter{MXCounter}
\newcounter{MSCounter}



\ifttm

% Struktur des sectioninfo-Tags: <!-- sectioninfo;;section;;subsection;;subsubsection;;nr_ausgeben;;testpage; //-->

%Fuegt eine zusaetzliche html-Seite an hinter ALLEN bisherigen und zukuenftigen content-Seiten ausserhalb der vor-zurueck-Schleife (d.h. nur durch Button oder MIntLink erreichbar!)
% #1 = Titel des Modulabschnitts, #2 = Kurztitel fuer die Buttons, #3 = Buttonkennung (STD = default nehmen, NONE = Ohne Button in der Navigation)
\newenvironment{MSContent}[3]{\special{html:<div class="xcontent}\arabic{MSCounter}\special{html:"><!-- scontent;-;}\arabic{MSCounter};-;#1;-;#2;-;#3\special{html: //-->}\MPageScripts\MSubsubsectionx{#1}}{\MEndScripts\special{html:<!-- endscontent;;}\arabic{MSCounter}\special{html: //--></div>}\addtocounter{MSCounter}{1}}

% Fuegt eine zusaetzliche html-Seite ein hinter den bereits vorhandenen content-Seiten (oder als erste Seite) innerhalb der vor-zurueck-Schleife der Navigation
% #1 = Titel des Modulabschnitts, #2 = Kurztitel fuer die Buttons, #3 = Buttonkennung (STD = Defaultbutton, NONE = Ohne Button in der Navigation)
\newenvironment{MXContent}[3]{\special{html:<div class="xcontent}\arabic{MXCounter}\special{html:"><!-- xcontent;-;}\arabic{MXCounter};-;#1;-;#2;-;#3\special{html: //-->}\MPageScripts\MSubsubsection{#1}}{\MEndScripts\special{html:<!-- endxcontent;;}\arabic{MXCounter}\special{html: //--></div>}\addtocounter{MXCounter}{1}}

% Fuegt eine zusaetzliche html-Seite ein die keine subsubsection-Nummer bekommt, nur zur internen Verwendung in mintmod.tex gedacht!
% #1 = Titel des Modulabschnitts, #2 = Kurztitel fuer die Buttons, #3 = Buttonkennung (STD = Defaultbutton, NONE = Ohne Button in der Navigation)
% \newenvironment{MUContent}[3]{\special{html:<div class="xcontent}\arabic{MXCounter}\special{html:"><!-- xcontent;-;}\arabic{MXCounter};-;#1;-;#2;-;#3\special{html: //-->}\MPageScripts\MSubsubsectionx{#1}}{\MEndScripts\special{html:<!-- endxcontent;;}\arabic{MXCounter}\special{html: //--></div>}\addtocounter{MXCounter}{1}}

\newcommand{\MDeclareSiteUXID}[1]{\special{html:<!-- mdeclaresiteuxid;;}#1\special{html:;;}\arabic{chapter}\special{html:;;}\arabic{section}\special{html:;; //-->}}

\else

%\newcommand{\MSubsubsection}[1]{\refstepcounter{subsubsection} \addcontentsline{toc}{subsubsection}{\thesubsubsection. #1}}


% Fuegt eine zusaetzliche html-Seite an hinter den bereits vorhandenen content-Seiten
% #1 = Titel des Modulabschnitts, #2 = Kurztitel fuer die Buttons, #3 = Iconkennung (im PDF wirkungslos)
%\newenvironment{MUContent}[3]{\ifnum\value{MXCTest}>0{\MDebugMessage{ERROR: Geschachtelter SContent}}\fi\MPageScripts\MSubsubsectionx{#1}\addtocounter{MXCTest}{1}}{\addtocounter{MXCounter}{1}\addtocounter{MXCTest}{-1}}
\newenvironment{MXContent}[3]{\ifnum\value{MXCTest}>0{\MDebugMessage{ERROR: Geschachtelter SContent}}\fi\MPageScripts\MSubsubsection{#1}\addtocounter{MXCTest}{1}}{\addtocounter{MXCounter}{1}\addtocounter{MXCTest}{-1}}
\newenvironment{MSContent}[3]{\ifnum\value{MXCTest}>0{\MDebugMessage{ERROR: Geschachtelter XContent}}\fi\MPageScripts\MSubsubsectionx{#1}\addtocounter{MXCTest}{1}}{\addtocounter{MSCounter}{1}\addtocounter{MXCTest}{-1}}

\newcommand{\MDeclareSiteUXID}[1]{\relax}

\fi 

% GHEADER und GFOOTER werden von split.pm gefunden, aber nur, wenn nicht HELPSITE oder TESTSITE
\ifttm
\newenvironment{MSectionStart}{\special{html:<div class="xcontent0">}\MSubsubsectionx{Modul\"ubersicht}}{\setcounter{MSSEnd}{0}\special{html:</div>}}
% Darf nicht als XContent nummeriert werden, darf nicht als XContent gelabelt werden, wird aber in eine xcontent-div gesetzt fuer Python-parsing
\else
\newenvironment{MSectionStart}{\MSubsectionx{Modul\"ubersicht}}{\setcounter{MSSEnd}{0}}
\fi

\newenvironment{MIntro}{\begin{MXContent}{Einf\"uhrung}{Einf\"uhrung}{genetisch}}{\end{MXContent}}
\newenvironment{MContent}{\begin{MXContent}{Inhalt}{Inhalt}{beweis}}{\end{MXContent}}
\newenvironment{MExercises}{\ifttm\else\clearpage\fi\begin{MXContent}{Aufgaben}{Aufgaben}{aufgb}\special{html:<!-- declareexcsymb //-->}}{\end{MXContent}}

% #1 = Lesbare Testbezeichnung
\newenvironment{MTest}[1]{%
\renewcommand{\MTestName}{#1}
\ifttm\else\clearpage\fi%
\addtocounter{MTestSite}{1}%
\begin{MXContent}{#1}{#1}{STD} % {aufgb}%
\special{html:<!-- declaretestsymb //-->}
\begin{MQuestionGroup}%
\MInTestHeader
}%
{%
\end{MQuestionGroup}%
\ \\ \ \\%
\MInTestFooter
\end{MXContent}\addtocounter{MTestSite}{-1}%
}

\newenvironment{MExtra}{\ifttm\else\clearpage\fi\begin{MXContent}{Zus\"atzliche Inhalte}{Zusatz}{weiterfhrg}}{\end{MXContent}}

\makeindex

\ifttm
\def\MPrintIndex{
\ifnum\value{MSSEnd}>0{\MSubsectionEndMacros}\addtocounter{MSSEnd}{-1}\fi
\renewcommand{\indexname}{Stichwortverzeichnis}
\special{html:<p><!-- printindex //--></p>}
}
\else
\def\MPrintIndex{
\ifnum\value{MSSEnd}>0{\MSubsectionEndMacros}\addtocounter{MSSEnd}{-1}\fi
\renewcommand{\indexname}{Stichwortverzeichnis}
\addcontentsline{toc}{section}{Stichwortverzeichnis}
\printindex
}
\fi


% Konstanten fuer die Modulfaecher

\def\MINTMathematics{1}
\def\MINTInformatics{2}
\def\MINTChemistry{3}
\def\MINTPhysics{4}
\def\MINTEngineering{5}

\newcounter{MSubjectArea}
\newcounter{MInfoNumbers} % Gibt an, ob die Infoboxen nummeriert werden sollen
\newcounter{MSepNumbers} % Gibt an, ob Beispiele und Experimente separat nummeriert werden sollen
\newcommand{\MSetSubject}[1]{
 % ttm kapiert setcounter mit Parametern nicht, also per if abragen und einsetzen
\ifnum#1=1\setcounter{MSubjectArea}{1}\setcounter{MInfoNumbers}{1}\setcounter{MSepNumbers}{0}\fi
\ifnum#1=2\setcounter{MSubjectArea}{2}\setcounter{MInfoNumbers}{1}\setcounter{MSepNumbers}{0}\fi
\ifnum#1=3\setcounter{MSubjectArea}{3}\setcounter{MInfoNumbers}{0}\setcounter{MSepNumbers}{1}\fi
\ifnum#1=4\setcounter{MSubjectArea}{4}\setcounter{MInfoNumbers}{0}\setcounter{MSepNumbers}{0}\fi
\ifnum#1=5\setcounter{MSubjectArea}{5}\setcounter{MInfoNumbers}{1}\setcounter{MSepNumbers}{0}\fi
% Separate Nummerntechnik fuer unsere Chemiker: alles dreistellig
\ifnum#1=3
  \ifttm
  \renewcommand{\theequation}{\arabic{section}.\arabic{subsection}.\arabic{equation}}
  \renewcommand{\thetable}{\arabic{section}.\arabic{subsection}.\arabic{table}} 
  \renewcommand{\thefigure}{\arabic{section}.\arabic{subsection}.\arabic{figure}} 
  \else
  \renewcommand{\theequation}{\arabic{chapter}.\arabic{section}.\arabic{equation}}
  \renewcommand{\thetable}{\arabic{chapter}.\arabic{section}.\arabic{table}}
  \renewcommand{\thefigure}{\arabic{chapter}.\arabic{section}.\arabic{figure}}
  \fi
\else
  \ifttm
  \renewcommand{\theequation}{\arabic{section}.\arabic{subsection}.\arabic{equation}}
  \renewcommand{\thetable}{\arabic{table}}
  \renewcommand{\thefigure}{\arabic{figure}}
  \else
  \renewcommand{\theequation}{\arabic{chapter}.\arabic{section}.\arabic{equation}}
  \renewcommand{\thetable}{\arabic{table}}
  \renewcommand{\thefigure}{\arabic{figure}}
  \fi
\fi
}

% Fuer tikz Autogenerierung
\newcounter{MTIKZAutofilenumber}

% Spezielle Counter fuer die Bentz-Module
\newcounter{mycounter}
\newcounter{chemapplet}
\newcounter{physapplet}

\newcounter{MSSEnd} % Ist 1 falls ein MSubsection aktiv ist, der einen MSubsectionEndMacro-Aufruf verursacht
\newcounter{MFileNumber}
\def\MLastFile{\special{html:[[!-- mfileref;;}\arabic{MFileNumber}\special{html:; //--]]}}

% Vollstaendiger Pfad ist \MMaterial / \MLastFilePath / \MLastFileName    ==   \MMaterial / \MLastFile

% Wird nur bei kompletter Baum-Erstellung ausgefuehrt!
% #1 = Lesbare Modulbezeichnung
\newcommand{\MSectionStartMacros}[1]{
\setcounter{MTestSite}{0}
\setcounter{MCaptionOn}{0}
\setcounter{MLastTypeEq}{0}
\setcounter{MSSEnd}{0}
\setcounter{MFileNumber}{0} % Preinkrekement-Counter
\setcounter{MTIKZAutofilenumber}{0}
\setcounter{mycounter}{1}
\setcounter{physapplet}{1}
\setcounter{chemapplet}{0}
\ifttm
\special{html:<!-- mdeclaresection;;}\arabic{chapter}\special{html:;;}\arabic{section}\special{html:;;}#1\special{html:;; //-->}%
\else
\setcounter{thmc}{0}
\setcounter{exmpc}{0}
\setcounter{verc}{0}
\setcounter{infoc}{0}
\fi
\setcounter{MiniMarkerCounter}{1}
\setcounter{AlignCounter}{1}
\setcounter{MXCTest}{0}
\setcounter{MCodeCounter}{0}
\setcounter{MEntryCounter}{0}
}

% Wird immer ausgefuehrt
\newcommand{\MSubsectionStartMacros}{
\ifttm\else\MPageHeaderDef\fi
\MWatermarkSettings
\setcounter{MXCounter}{0}
\setcounter{MSCounter}{0}
\setcounter{MSiteCounter}{1}
\setcounter{MExerciseCollectionCounter}{0}
% Zaehler fuer das Labelsystem zuruecksetzen (prefix-Zaehler)
\setcounter{MInfoCounter}{0}
\setcounter{MExerciseCounter}{0}
\setcounter{MExampleCounter}{0}
\setcounter{MExperimentCounter}{0}
\setcounter{MGraphicsCounter}{0}
\setcounter{MTableCounter}{0}
\setcounter{MTheoremCounter}{0}
\setcounter{MObjectCounter}{0}
\setcounter{MEquationCounter}{0}
\setcounter{MVideoCounter}{0}
\setcounter{equation}{0}
\setcounter{figure}{0}
}

\newcommand{\MSubsectionEndMacros}{
% Bei Chemiemodulen das PSE einhaengen, es soll als SContent am Ende erscheinen
\special{html:<!-- subsectionend //-->}
\ifnum\value{MSubjectArea}=3{\MIncludePSE}\fi
}


\ifttm
%\newcommand{\MEmbed}[1]{\MRegisterFile{#1}\begin{html}<embed src="\end{html}\MMaterial/\MLastFile\begin{html}" width="192" height="189"></embed>\end{html}}
\newcommand{\MEmbed}[1]{\MRegisterFile{#1}\begin{html}<embed src="\end{html}\MMaterial/\MLastFile\begin{html}"></embed>\end{html}}
\fi

%----------------- Makros fuer die Textdarstellung -----------------------------------------------

\ifttm
% MUGraphics bindet eine Grafik ein:
% Parameter 1: Dateiname der Grafik, relativ zur Position des Modul-Tex-Dokuments
% Parameter 2: Skalierungsoptionen fuer PDF (fuer includegraphics)
% Parameter 3: Titel fuer die Grafik, wird unter die Grafik mit der Grafiknummer gesetzt und kann MLabel bzw. MCopyrightLabel enthalten
% Parameter 4: Skalierungsoptionen fuer HTML (css-styles)

% ERSATZ: <img alt="My Image" src="data:image/png;base64,iVBORwA<MoreBase64SringHere>" />


\newcommand{\MUGraphics}[4]{\MRegisterFile{#1}\begin{html}
<div class="imagecenter">
<center>
<div>
<img src="\end{html}\MMaterial/\MLastFile\begin{html}" style="#4" alt="\end{html}\MMaterial/\MLastFile\begin{html}"/>
</div>
<div class="bildtext">
\end{html}
\addtocounter{MGraphicsCounter}{1}
\setcounter{MLastIndex}{\value{MGraphicsCounter}}
\setcounter{MLastType}{8}
\addtocounter{MCaptionOn}{1}
\ifnum\value{MSepNumbers}=0
\textbf{Abbildung \arabic{MGraphicsCounter}:} #3
\else
\textbf{Abbildung \arabic{section}.\arabic{subsection}.\arabic{MGraphicsCounter}:} #3
\fi
\addtocounter{MCaptionOn}{-1}
\begin{html}
</div>
</center>
</div>
<br />
\end{html}%
\special{html:<!-- mfeedbackbutton;Abbildung;}\arabic{MGraphicsCounter}\special{html:;}\arabic{section}.\arabic{subsection}.\arabic{MGraphicsCounter}\special{html:; //-->}%
}

% MVideo bindet ein Video als Einzeldatei ein:
% Parameter 1: Dateiname des Videos, relativ zur Position des Modul-Tex-Dokuments, ohne die Endung ".mp4"
% Parameter 2: Titel fuer das Video (kann MLabel oder MCopyrightLabel enthalten), wird unter das Video mit der Videonummer gesetzt
\newcommand{\MVideo}[2]{\MRegisterFile{#1.mp4}\begin{html}
<div class="imagecenter">
<center>
<div>
<video width="95\%" controls="controls"><source src="\end{html}\MMaterial/#1.mp4\begin{html}" type="video/mp4">Ihr Browser kann keine MP4-Videos abspielen!</video>
</div>
<div class="bildtext">
\end{html}
\addtocounter{MVideoCounter}{1}
\setcounter{MLastIndex}{\value{MVideoCounter}}
\setcounter{MLastType}{12}
\addtocounter{MCaptionOn}{1}
\ifnum\value{MSepNumbers}=0
\textbf{Video \arabic{MVideoCounter}:} #2
\else
\textbf{Video \arabic{section}.\arabic{subsection}.\arabic{MVideoCounter}:} #2
\fi
\addtocounter{MCaptionOn}{-1}
\begin{html}
</div>
</center>
</div>
<br />
\end{html}}

\newcommand{\MDVideo}[2]{\MRegisterFile{#1.mp4}\MRegisterFile{#1.ogv}\begin{html}
<div class="imagecenter">
<center>
<div>
<video width="70\%" controls><source src="\end{html}\MMaterial/#1.mp4\begin{html}" type="video/mp4"><source src="\end{html}\MMaterial/#1.ogv\begin{html}" type="video/ogg">Ihr Browser kann keine MP4-Videos abspielen!</video>
</div>
<br />
#2
</center>
</div>
<br />
\end{html}
}

\newcommand{\MGraphics}[3]{\MUGraphics{#1}{#2}{#3}{}}

\else

\newcommand{\MVideo}[2]{%
% Kein Video im PDF darstellbar, trotzdem so tun als ob da eines waere
\begin{center}
(Video nicht darstellbar)
\end{center}
\addtocounter{MVideoCounter}{1}
\setcounter{MLastIndex}{\value{MVideoCounter}}
\setcounter{MLastType}{12}
\addtocounter{MCaptionOn}{1}
\ifnum\value{MSepNumbers}=0
\textbf{Video \arabic{MVideoCounter}:} #2
\else
\textbf{Video \arabic{section}.\arabic{subsection}.\arabic{MVideoCounter}:} #2
\fi
\addtocounter{MCaptionOn}{-1}
}


% MGraphics bindet eine Grafik ein:
% Parameter 1: Dateiname der Grafik, relativ zur Position des Modul-Tex-Dokuments
% Parameter 2: Skalierungsoptionen fuer PDF (fuer includegraphics)
% Parameter 3: Titel fuer die Grafik, wird unter die Grafik mit der Grafiknummer gesetzt
\newcommand{\MGraphics}[3]{%
\MRegisterFile{#1}%
\ %
\begin{figure}[H]%
\centering{%
\includegraphics[#2]{\MDPrefix/#1}%
\addtocounter{MCaptionOn}{1}%
\caption{#3}%
\addtocounter{MCaptionOn}{-1}%
}%
\end{figure}%
\addtocounter{MGraphicsCounter}{1}\setcounter{MLastIndex}{\value{MGraphicsCounter}}\setcounter{MLastType}{8}\ %
%\ \\Abbildung \ifnum\value{MSepNumbers}=0\else\arabic{chapter}.\arabic{section}.\fi\arabic{MGraphicsCounter}: #3%
}

\newcommand{\MUGraphics}[4]{\MGraphics{#1}{#2}{#3}}


\fi

\newcounter{MCaptionOn} % = 1 falls eine Grafikcaption aktiv ist, = 0 sonst


% MGraphicsSolo bindet eine Grafik pur ein ohne Titel
% Parameter 1: Dateiname der Grafik, relativ zur Position des Modul-Tex-Dokuments
% Parameter 2: Skalierungsoptionen (wirken nur im PDF)
\newcommand{\MGraphicsSolo}[2]{\MUGraphicsSolo{#1}{#2}{}}

% MUGraphicsSolo bindet eine Grafik pur ein ohne Titel, aber mit HTML-Skalierung
% Parameter 1: Dateiname der Grafik, relativ zur Position des Modul-Tex-Dokuments
% Parameter 2: Skalierungsoptionen (wirken nur im PDF)
% Parameter 3: Skalierungsoptionen (wirken nur im HTML), als style-format: "width=???, height=???"
\ifttm
\newcommand{\MUGraphicsSolo}[3]{\MRegisterFile{#1}\begin{html}
<img src="\end{html}\MMaterial/\MLastFile\begin{html}" style="\end{html}#3\begin{html}" alt="\end{html}\MMaterial/\MLastFile\begin{html}"/>
\end{html}%
\special{html:<!-- mfeedbackbutton;Abbildung;}#1\special{html:;}\MMaterial/\MLastFile\special{html:; //-->}%
}
\else
\newcommand{\MUGraphicsSolo}[3]{\MRegisterFile{#1}\includegraphics[#2]{\MDPrefix/#1}}
\fi

% Externer Link mit URL
% Erster Parameter: Vollstaendige(!) URL des Links
% Zweiter Parameter: Text fuer den Link
\newcommand{\MExtLink}[2]{\ifttm\special{html:<a target="_new" href="}#1\special{html:">}#2\special{html:</a>}\else\href{#1}{#2}\fi} % ohne MINTERLINK!


% Interner Link, die verlinkte Datei muss im gleichen Verzeichnis liegen wie die Modul-Texdatei
% Erster Parameter: Dateiname
% Zweiter Parameter: Text fuer den Link
\newcommand{\MIntLink}[2]{\ifttm\MRegisterFile{#1}\special{html:<a class="MINTERLINK" target="_new" href="}\MMaterial/\MLastFile\special{html:">}#2\special{html:</a>}\else{\href{#1}{#2}}\fi}


\ifttm
\def\MMaterial{:localmaterial:}
\else
\def\MMaterial{\MDPrefix}
\fi

\ifttm
\def\MNoFile#1{:directmaterial:#1}
\else
\def\MNoFile#1{#1}
\fi

\newcommand{\MChem}[1]{$\mathrm{#1}$}

\newcommand{\MApplet}[3]{
% Bindet ein Java-Applet ein, die Parameter sind:
% (wird nur im HTML, aber nicht im PDF erstellt)
% #1 Dateiname des Applets (muss mit ".class" enden)
% #2 = Breite in Pixeln
% #3 = Hoehe in Pixeln
\ifttm
\MRegisterFile{#1}
\begin{html}
<applet code="\end{html}\MMaterial/\MLastFile\begin{html}" width="#2" height="#3" alt="[Java-Applet kann nicht gestartet werden]"></applet>
\end{html}
\fi
}

\newcommand{\MScriptPage}[2]{
% Bindet eine JavaScript-Datei ein, die eine eigene Seite bekommt
% (wird nur im HTML, aber nicht im PDF erstellt)
% #1 Dateiname des Programms (sollte mit ".js" enden)
% #2 = Kurztitel der Seite
\ifttm
\begin{MSContent}{#2}{#2}{puzzle}
\MRegisterFile{#1}
\begin{html}
<script src="\MMaterial/\MLastFile" type="text/javascript"></script>
\end{html}
\end{MSContent}
\fi
}

\newcommand{\MIncludePSE}{
% Bindet bei Chemie-Modulen das PSE ein
% (wird nur im HTML, aber nicht im PDF erstellt)
\ifttm
\special{html:<!-- includepse //-->}
\begin{MSContent}{Periodensystem der Elemente}{PSE}{table}
\MRegisterFile{../files/pse.js}
\MRegisterFile{../files/radio.png}
\begin{html}
<script src="\MMaterial/../files/pse.js" type="text/javascript"></script>
<p id="divid"><br /><br />
<script language="javascript" type="text/javascript">
    startpse("divid","\MMaterial/../files"); 
</script>
</p>
<br />
<br />
<br />
<p>Die Farben der Elementsymbole geben an: <font style="color:Red">gasf&ouml;rmig </font> <font style="color:Blue">fl&uuml;ssig </font> fest</p>
<p>Die Elemente der Gruppe 1 A, 2 A, 3 A usw. geh&ouml;ren zu den Hauptgruppenelementen.</p>
<p>Die Elemente der Gruppe 1 B, 2 B, 3 B usw. geh&ouml;ren zu den Nebengruppenelementen.</p>
<p>() kennzeichnet die Masse des stabilsten Isotops</p>
\end{html}
\end{MSContent}
\fi
}

\newcommand{\MAppletArchive}[4]{
% Bindet ein Java-Applet ein, die Parameter sind:
% (wird nur im HTML, aber nicht im PDF erstellt)
% #1 Dateiname der Klasse mit Appletaufruf (muss mit ".class" enden)
% #2 Dateiname des Archivs (muss mit ".jar" enden)
% #3 = Breite in Pixeln
% #4 = Hoehe in Pixeln
\ifttm
\MRegisterFile{#2}
\begin{html}
<applet code="#1" archive="\end{html}\MMaterial/\MLastFile\begin{html}" codebase="." width="#3" height="#4" alt="[Java-Archiv kann nicht gestartet werden]"></applet>
\end{html}
\fi
}

% Bindet in der Haupttexdatei ein MINT-Modul ein. Parameter 1 ist das Verzeichnis (relativ zur Haupttexdatei), Parameter 2 ist der Dateinahme ohne Pfad.
\newcommand{\IncludeModule}[2]{
\renewcommand{\MDPrefix}{#1}
\input{#1/#2}
\ifnum\value{MSSEnd}>0{\MSubsectionEndMacros}\addtocounter{MSSEnd}{-1}\fi
}

% Der ttm-Konverter setzt keine Makros im \input um, also muss hier getrickst werden:
% Das MDPrefix muss in den einzelnen Modulen manuell eingesetzt werden
\newcommand{\MInputFile}[1]{
\ifttm
\input{#1}
\else
\input{#1}
\fi
}


\newcommand{\MCases}[1]{\left\lbrace{\begin{array}{rl} #1 \end{array}}\right.}

\ifttm
\newenvironment{MCaseEnv}{\left\lbrace\begin{array}{rl}}{\end{array}\right.}
\else
\newenvironment{MCaseEnv}{\left\lbrace\begin{array}{rl}}{\end{array}\right.}
\fi

\def\MSkip{\ifttm\MCR\fi}

\ifttm
\def\MCR{\special{html:<br />}}
\else
\def\MCR{\ \\}
\fi


% Pragmas - Sind Schluesselwoerter, die dem Preprocessing sowie dem Konverter uebergeben werden und bestimmte
%           Aktionen ausloesen. Im Output (PDF und HTML) tauchen sie nicht auf.
\newcommand{\MPragma}[1]{%
\ifttm%
\special{html:<!-- mpragma;-;}#1\special{html:;; -->}%
\else%
% MPragmas werden vom Preprozessor direkt im LaTeX gefunden
\fi%
}

% Ersatz der Befehle textsubscript und textsuperscript, die ttm nicht kennt
\ifttm%
\newcommand{\MTextsubscript}[1]{\special{html:<sub>}#1\special{html:</sub>}}%
\newcommand{\MTextsuperscript}[1]{\special{html:<sup>}#1\special{html:</sup>}}%
\else%
\newcommand{\MTextsubscript}[1]{\textsubscript{#1}}%
\newcommand{\MTextsuperscript}[1]{\textsuperscript{#1}}%
\fi

%------------------ Einbindung von dia-Diagrammen ----------------------------------------------
% Beim preprocessing wird aus jeder dia-Datei eine tex-Datei und eine pdf-Datei erzeugt,
% diese werden hier jeweils im PDF und HTML eingebunden
% Parameter: Dateiname der mit dia erstellten Datei (OHNE die Endung .dia)
\ifttm%
\newcommand{\MDia}[1]{%
\MGraphicsSolo{#1minthtml.png}{}%
}
\else%
\newcommand{\MDia}[1]{%
\MGraphicsSolo{#1mintpdf.png}{scale=0.1667}%
}
\fi%

% subsup funktioniert im Ausdruck $D={\R}^+_0$, also \R geklammert und sup zuerst
% \ifttm
% \def\MSubsup#1#2#3{\special{html:<msubsup>} #1 #2 #3\special{html:</msubsup>}}
% \else
% \def\MSubsup#1#2#3{{#1}^{#3}_{#2}}
% \fi

%\input{local.tex}

% \ifttm
% \else
% \newwrite\mintlog
% \immediate\openout\mintlog=mintlog.txt
% \fi

% ----------------------- tikz autogenerator -------------------------------------------------------------------

\newcommand{\Mtikzexternalize}{\tikzexternalize}% wird bei Konvertierung ueber mconvert ggf. ausgehebelt!

\ifttm
\else
\tikzset%
{
  % Defines a custom style which generates pdf and converts to (low and hi-res quality) png and svg, then deletes the pdf
  % Important: DO NOT directly convert from pdf to hires-png or from svg to png with GraphViz convert as it has some problems and memory leaks
  png export/.style=%
  {
    external/system call/.add={}{; 
      pdf2svg "\image.pdf" "\image.svg" ; 
      convert -density 112.5 -transparent white "\image.pdf" "\image.png"; 
      inkscape --export-png="\image.4x.png" --export-dpi=450 --export-background-opacity=0 --without-gui "\image.svg"; 
      rm "\image.pdf"; rm "\image.log"; rm "\image.dpth"; rm "\image.idx"
    },
    external/force remake,
  }
}
\tikzset{png export}
\tikzsetexternalprefix{}
% PNGs bei externer Erzeugung in "richtiger" Groesse einbinden
\pgfkeys{/pgf/images/include external/.code={\includegraphics[scale=0.64]{#1}}}
\fi

% Spezielle Umgebung fuer Autogenerierung, Bildernamen sind nur innerhalb eines Moduls (einer MSection) eindeutig)

\newcommand{\MTIKZautofilename}{tikzautofile}

\ifttm
% HTML-Version: Vom Autogenerator erzeugte png-Datei einbinden, tikz selbst nicht ausfuehren (sprich: #1 schlucken)
\newcommand{\MTikzAuto}[1]{%
\addtocounter{MTIKZAutofilenumber}{1}
\renewcommand{\MTIKZautofilename}{mtikzauto_\arabic{MTIKZAutofilenumber}}
\MUGraphicsSolo{\MSectionID\MTIKZautofilename.4x.png}{scale=1}{\special{html:[[!-- svgstyle;}\MSectionID\MTIKZautofilename\special{html: //--]]}} % Styleinfos werden aus original-png, nicht 4x-png geholt!
%\MRegisterFile{\MSectionID\MTIKZautofilename.png} % not used right now
%\MRegisterFile{\MSectionID\MTIKZautofilename.svg}
}
\else%
% PDF-Version: Falls Autogenerator aktiv wird Datei automatisch benannt und exportiert
\newcommand{\MTikzAuto}[1]{%
\addtocounter{MTIKZAutofilenumber}{1}%
\renewcommand{\MTIKZautofilename}{mtikzauto_\arabic{MTIKZAutofilenumber}}
\tikzsetnextfilename{\MTIKZautofilename}%
#1%
}
\fi

% In einer reinen LaTeX-Uebersetzung kapselt der Preambelinclude-Befehl nur input,
% in einer konvertergesteuerten PDF/HTML-Uebersetzung wird er dagegen entfernt und
% die Preambeln an mintmod angehaengt, die Ersetzung wird von mconvert.pl vorgenommen.

\newcommand{\MPreambleInclude}[1]{\input{#1}}

% Globale Watermarksettings (werden auch nochmal zu Beginn jedes subsection gesetzt,
% muessen hier aber auch global ausgefuehrt wegen Einfuehrungsseiten und Inhaltsverzeichnis

\MWatermarkSettings
% ---------------------------------- Parametrisierte Aufgaben ----------------------------------------

\ifttm
\newenvironment{MPExercise}{%
\begin{MExercise}%
}{%
\special{html:<button name="Name_MPEX}\arabic{MExerciseCounter}\special{html:" id="MPEX}\arabic{MExerciseCounter}%
\special{html:" type="button" onclick="reroll('}\arabic{MExerciseCounter}\special{html:');">Neue Aufgabe erzeugen</button>}%
\end{MExercise}%
}
\else
\newenvironment{MPExercise}{%
\begin{MExercise}%
}{%
\end{MExercise}%
}
\fi

% Parameter: Name, Min, Max, PDF-Standard. Name in Deklaration OHNE backslash, im Code MIT Backslash
\ifttm
\newcommand{\MGlobalInteger}[4]{\special{html:%
<!-- onloadstart //-->%
MVAR.push(createGlobalInteger("}#1\special{html:",}#2\special{html:,}#3\special{html:,}#4\special{html:)); %
<!-- onloadstop //-->%
<!-- viewmodelstart //-->%
ob}#1\special{html:: ko.observable(rerollMVar("}#1\special{html:")),%
<!-- viewmodelstop //-->%
}%
}%
\else%
\newcommand{\MGlobalInteger}[4]{\newcounter{mvc_#1}\setcounter{mvc_#1}{#4}}
\fi

% Parameter: Name, Min, Max, PDF-Standard. Name in Deklaration OHNE backslash, im Code MIT Backslash, Wert ist Wurzel von value
\ifttm
\newcommand{\MGlobalSqrt}[4]{\special{html:%
<!-- onloadstart //-->%
MVAR.push(createGlobalSqrt("}#1\special{html:",}#2\special{html:,}#3\special{html:,}#4\special{html:)); %
<!-- onloadstop //-->%
<!-- viewmodelstart //-->%
ob}#1\special{html:: ko.observable(rerollMVar("}#1\special{html:")),%
<!-- viewmodelstop //-->%
}%
}%
\else%
\newcommand{\MGlobalSqrt}[4]{\newcounter{mvc_#1}\setcounter{mvc_#1}{#4}}% Funktioniert nicht als Wurzel !!!
\fi

% Parameter: Name, Min, Max, PDF-Standard zaehler, PDF-Standard nenner. Name in Deklaration OHNE backslash, im Code MIT Backslash
\ifttm
\newcommand{\MGlobalFraction}[5]{\special{html:%
<!-- onloadstart //-->%
MVAR.push(createGlobalFraction("}#1\special{html:",}#2\special{html:,}#3\special{html:,}#4\special{html:,}#5\special{html:)); %
<!-- onloadstop //-->%
<!-- viewmodelstart //-->%
ob}#1\special{html:: ko.observable(rerollMVar("}#1\special{html:")),%
<!-- viewmodelstop //-->%
}%
}%
\else%
\newcommand{\MGlobalFraction}[5]{\newcounter{mvc_#1}\setcounter{mvc_#1}{#4}} % Funktioniert nicht als Bruch !!!
\fi

% MVar darf im HTML nur in MEvalMathDisplay-Umgebungen genutzt werden oder in Strings die an den Parser uebergeben werden
\ifttm%
\newcommand{\MVar}[1]{\special{html:[var_}#1\special{html:]}}%
\else%
\newcommand{\MVar}[1]{\arabic{mvc_#1}}%
\fi

\ifttm%
\newcommand{\MRerollButton}[2]{\special{html:<button type="button" onclick="rerollMVar('}#1\special{html:');">}#2\special{html:</button>}}%
\else%
\newcommand{\MRerollButton}[2]{\relax}% Keine sinnvolle Entsprechung im PDF
\fi

% MEvalMathDisplay fuer HTML wird in mconvert.pl im preprocessing realisiert
% PDF: eine equation*-Umgebung (ueber amsmath)
% HTML: Eine Mathjax-Tex-Umgebung, deren Auswertung mit knockout-obervablen gekoppelt ist
% PDF-Version hier nur fuer pdflatex-only-Uebersetzung gegeben

\ifttm\else\newenvironment{MEvalMathDisplay}{\begin{equation*}}{\end{equation*}}\fi

% ---------------------------------- Spezialbefehle fuer AD ------------------------------------------

%Abk�rzung f�r \longrightarrow:
\newcommand{\lto}{\ensuremath{\longrightarrow}}

%Makro f�r Funktionen:
\newcommand{\exfunction}[5]
{\begin{array}{rrcl}
 #1 \colon  & #2 &\lto & #3 \\[.05cm]  
  & #4 &\longmapsto  & #5 
\end{array}}

\newcommand{\function}[5]{%
#1:\;\left\lbrace{\begin{array}{rcl}
 #2 &\lto & #3 \\
 #4 &\longmapsto  & #5 \end{array}}\right.}


%Die Identit�t:
\DeclareMathOperator{\Id}{Id}

%Die Signumfunktion:
\DeclareMathOperator{\sgn}{sgn}

%Zwei Betonungskommandos (k�nnen angepasst werden):
\newcommand{\highlight}[1]{#1}
\newcommand{\modstextbf}[1]{#1}
\newcommand{\modsemph}[1]{#1}


% ---------------------------------- Spezialbefehle fuer JL ------------------------------------------


\def\jccolorfkt{green!50!black} %Farbe des Funktionsgraphen
\def\jccolorfktarea{green!25!white} %Farbe der Fl"ache unter dem Graphen
\def\jccolorfktareahell{green!12!white} %helle Einf"arbung der Fl"ache unter dem Graphen
\def\jccolorfktwert{green!50!black} %Farbe einzelner Punkte des Graphen

\newcommand{\MPfadBilder}{Bilder}

\ifttm%
\newcommand{\jMD}{\,\MD}%
\else%
\newcommand{\jMD}{\;\MD}%
\fi%

\def\jHTMLHinweisBedienung{\MInputHint{%
Mit Hilfe der Symbole am oberen Rand des Fensters
k"onnen Sie durch die einzelnen Abschnitte navigieren.}}

\def\jHTMLHinweisEingabeText{\MInputHint{%
Geben Sie jeweils ein Wort oder Zeichen als Antwort ein.}}

\def\jHTMLHinweisEingabeTerm{\MInputHint{%
Klammern Sie Ihre Terme, um eine eindeutige Eingabe zu erhalten. 
Beispiel: Der Term $\frac{3x+1}{x-2}$ soll in der Form
\texttt{(3*x+1)/((x+2)^2}$ eingegeben werden (wobei auch Leerzeichen 
eingegeben werden k"onnen, damit eine Formel besser lesbar ist).}}

\def\jHTMLHinweisEingabeIntervalle{\MInputHint{%
Intervalle werden links mit einer "offnenden Klammer und rechts mit einer 
schlie"senden Klammer angegeben. Eine runde Klammer wird verwendet, wenn der 
Rand nicht dazu geh"ort, eine eckige, wenn er dazu geh"ort. 
Als Trennzeichen wird ein Komma oder ein Semikolon akzeptiert.
Beispiele: $(a, b)$ offenes Intervall,
$[a; b)$ links abgeschlossenes, rechts offenes Intervall von $a$ bis $b$. 
Die Eingabe $]a;b[$ f"ur ein offenes Intervall wird nicht akzeptiert.
F"ur $\infty$ kann \texttt{infty} oder \texttt{unendlich} geschrieben werden.}}

\def\jHTMLHinweisEingabeFunktionen{\MInputHint{%
Schreiben Sie Malpunkte (geschrieben als \texttt{*}) aus und setzen Sie Klammern um Argumente f�r Funktionen.
Beispiele: Polynom: \texttt{3*x + 0.1}, Sinusfunktion: \texttt{sin(x)}, 
Verkettung von cos und Wurzel: \texttt{cos(sqrt(3*x))}.}}

\def\jHTMLHinweisEingabeFunktionenSinCos{\MInputHint{%
Die Sinusfunktion $\sin x$ wird in der Form \texttt{sin(x)} angegeben, %
$\cos\left(\sqrt{3 x}\right)$ durch \texttt{cos(sqrt(3*x))}.}}

\def\jHTMLHinweisEingabeFunktionenExp{\MInputHint{%
Die Exponentialfunktion $\MEU^{3x^4 + 5}$ wird als
\texttt{exp(3 * x^4 + 5)} angegeben, %
$\ln\left(\sqrt{x} + 3.2\right)$ durch \texttt{ln(sqrt(x) + 3.2)}.}}

% ---------------------------------- Spezialbefehle fuer Fachbereich Physik --------------------------

\newcommand{\E}{{e}}
\newcommand{\ME}[1]{\cdot 10^{#1}}
\newcommand{\MU}[1]{\;\mathrm{#1}}
\newcommand{\MPG}[3]{%
  \ifnum#2=0%
    #1\ \mathrm{#3}%
  \else%
    #1\cdot 10^{#2}\ \mathrm{#3}%
  \fi}%
%

\newcommand{\MMul}{\MExponentensymbXYZl} % Nur eine Abkuerzung


% ---------------------------------- Stichwortfunktionialitaet ---------------------------------------

% mpreindexentry wird durch Auswahlroutine in conv.pl durch mindexentry substitutiert
\ifttm%
\def\MIndex#1{\index{#1}\special{html:<!-- mpreindexentry;;}#1\special{html:;;}\arabic{MSubjectArea}\special{html:;;}%
\arabic{chapter}\special{html:;;}\arabic{section}\special{html:;;}\arabic{subsection}\special{html:;;}\arabic{MEntryCounter}\special{html:; //-->}%
\setcounter{MLastIndex}{\value{MEntryCounter}}%
\addtocounter{MEntryCounter}{1}%
}%
% Copyrightliste wird als tex-Datei im preprocessing von conv.pl erzeugt und unter converter/tex/entrycollection.tex abgelegt
% Der input-Befehl funktioniert nur, wenn die aufrufende tex-Datei auf der obersten Ebene liegt (d.h. selbst kein input/include ist, insbesondere keine Moduldatei)
\def\MEntryList{} % \input funktioniert nicht, weil ttm (und damit das \input) ausgefuehrt wird, bevor Datei da ist
\else%
\def\MIndex#1{\index{#1}}
\def\MEntryList{\MAbort{Stichwortliste nur im HTML realisierbar}}%
\fi%

\def\MEntry#1#2{\textbf{#1}\MIndex{#2}} % Idee: MLastType auf neuen Entry-Typ und dann ein MLabel vergeben mit autogen-Nummer

% ---------------------------------- Befehle fuer Tests ----------------------------------------------

% MEquationItem stellt eine Eingabezeile der Form Vorgabe = Antwortfeld her, der zweite Parameter kann z.B. MSimplifyQuestion-Befehl sein
\ifttm
\newcommand{\MEquationItem}[2]{{#1}$\,=\,${#2}}%
\else%
\newcommand{\MEquationItem}[2]{{#1}$\;\;=\,${#2}}%
\fi

\ifttm
\newcommand{\MInputHint}[1]{%
\ifnum%
\if\value{MTestSite}>0%
\else%
{\color{blue}#1}%
\fi%
\fi%
}
\else
\newcommand{\MInputHint}[1]{\relax}
\fi

\ifttm
\newcommand{\MInTestHeader}{%
Dies ist ein einreichbarer Test:
\begin{itemize}
\item{Im Gegensatz zu den offenen Aufgaben werden beim Eingeben keine Hinweise zur Formulierung der mathematischen Ausdr�cke gegeben.}
\item{Der Test kann jederzeit neu gestartet oder verlassen werden.}
\item{Der Test kann durch die Buttons am Ende der Seite beendet und abgeschickt, oder zur�ckgesetzt werden.}
\item{Der Test kann mehrfach probiert werden. F�r die Statistik z�hlt die zuletzt abgeschickte Version.}
\end{itemize}
}
\else
\newcommand{\MInTestHeader}{%
\relax
}
\fi

\ifttm
\newcommand{\MInTestFooter}{%
\special{html:<button name="Name_TESTFINISH" id="TESTFINISH" type="button" onclick="finish_button('}\MTestName\special{html:');">Test auswerten</button>}%
\begin{html}
&nbsp;&nbsp;&nbsp;&nbsp;&nbsp;&nbsp;&nbsp;&nbsp;
<button name="Name_TESTRESET" id="TESTRESET" type="button" onclick="reset_button();">Test zur�cksetzen</button>
<br />
<br />
<div class="xreply">
<p name="Name_TESTEVAL" id="TESTEVAL">
Hier erscheint die Testauswertung!
<br />
</p>
</div>
\end{html}
}
\else
\newcommand{\MInTestFooter}{%
\relax
}
\fi


% ---------------------------------- Notationsmakros -------------------------------------------------------------

% Notationsmakros die nicht von der Kursvariante abhaengig sind

\newcommand{\MZahltrennzeichen}[1]{\renewcommand{\MZXYZhltrennzeichen}{#1}}

\ifttm
\newcommand{\MZahl}[3][\MZXYZhltrennzeichen]{\edef\MZXYZtemp{\noexpand\special{html:<mn>#2#1#3</mn>}}\MZXYZtemp}
\else
\newcommand{\MZahl}[3][\MZXYZhltrennzeichen]{{}#2{#1}#3}
\fi

\newcommand{\MEinheitenabstand}[1]{\renewcommand{\MEinheitenabstXYZnd}{#1}}
\ifttm
\newcommand{\MEinheit}[2][\MEinheitenabstXYZnd]{{}#1\edef\MEINHtemp{\noexpand\special{html:<mi mathvariant="normal">#2</mi>}}\MEINHtemp} 
\else
\newcommand{\MEinheit}[2][\MEinheitenabstXYZnd]{{}#1 \mathrm{#2}} 
\fi

\newcommand{\MExponentensymbol}[1]{\renewcommand{\MExponentensymbXYZl}{#1}}
\newcommand{\MExponent}[2][\MExponentensymbXYZl]{{}#1{} 10^{#2}} 

%Punkte in 2 und 3 Dimensionen
\newcommand{\MPointTwo}[3][]{#1(#2\MCoordPointSep #3{}#1)}
\newcommand{\MPointThree}[4][]{#1(#2\MCoordPointSep #3\MCoordPointSep #4{}#1)}
\newcommand{\MPointTwoAS}[2]{\left(#1\MCoordPointSep #2\right)}
\newcommand{\MPointThreeAS}[3]{\left(#1\MCoordPointSep #2\MCoordPointSep #3\right)}

% Masseinheit, Standardabstand: \,
\newcommand{\MEinheitenabstXYZnd}{\MThinspace} 

% Horizontaler Leerraum zwischen herausgestellter Formel und Interpunktion
\ifttm
\newcommand{\MDFPSpace}{\,}
\newcommand{\MDFPaSpace}{\,\,}
\newcommand{\MBlank}{\ }
\else
\newcommand{\MDFPSpace}{\;}
\newcommand{\MDFPaSpace}{\;\;}
\newcommand{\MBlank}{\ }
\fi

% Satzende in herausgestellter Formel mit horizontalem Leerraum
\newcommand{\MDFPeriod}{\MDFPSpace .}

% Separation von Aufzaehlung und Bedingung in Menge
\newcommand{\MCondSetSep}{\,:\,} %oder '\mid'

% Konverter kennt mathopen nicht
\ifttm
\def\mathopen#1{}
\fi

% -----------------------------------START Rouletteaufgaben ------------------------------------------------------------

\ifttm
% #1 = Dateiname, #2 = eindeutige ID fuer das Roulette im Kurs
\newcommand{\MDirectRouletteExercises}[2]{
\begin{MExercise}
\texttt{Im HTML erscheinen hier Aufgaben aus einer Aufgabenliste...}
\end{MExercise}
}
\else
\newcommand{\MDirectRouletteExercises}[2]{\relax} % wird durch mconvert.pl gefunden und ersetzt
\fi


% ---------------------------------- START Makros, die von der Kursvariante abhaengen ----------------------------------

\ifvariantunotation
  % unotation = An Universitaeten uebliche Notation
  \def\MVariant{unotation}

  % Trennzeichen fuer Dezimalzahlen
  \newcommand{\MZXYZhltrennzeichen}{.}

  % Exponent zur Basis 10 in der Exponentialschreibweise, 
  % Standardmalzeichen: \times
  \newcommand{\MExponentensymbXYZl}{\times} 

  % Begrenzungszeichen fuer offene Intervalle
  \newcommand{\MoIl}[1][]{\mbox{}#1(\mathopen{}} % bzw. ']'
  \newcommand{\MoIr}[1][]{#1)\mbox{}} % bzw. '['

  % Zahlen-Separation im IntervaLL
  \newcommand{\MIntvlSep}{,} %oder ';'

  % Separation von Elementen in Mengen
  \newcommand{\MElSetSep}{,} %oder ';'

  % Separation von Koordinaten in Punkten
  \newcommand{\MCoordPointSep}{,} %oder ';' oder '|', '\MThinspace|\MThinspace'

\else
  % An dieser Stelle wird angenommen, dass std-Variante aktiv ist
  % std = beschlossene Notation im TU9-Onlinekurs 
  \def\MVariant{std}

  % Trennzeichen fuer Dezimalzahlen
  \newcommand{\MZXYZhltrennzeichen}{,}

  % Exponent zur Basis 10 in der Exponentialschreibweise, 
  % Standardmalzeichen: \times
  \newcommand{\MExponentensymbXYZl}{\times} 

  % Begrenzungszeichen fuer offene Intervalle
  \newcommand{\MoIl}[1][]{\mbox{}#1]\mathopen{}} % bzw. '('
  \newcommand{\MoIr}[1][]{#1[\mbox{}} % bzw. ')'

  % Zahlen-Separation im IntervaLL
  \newcommand{\MIntvlSep}{;} %oder ','
  
  % Separation von Elementen in Mengen
  \newcommand{\MElSetSep}{;} %oder ','

  % Separation von Koordinaten in Punkten
  \newcommand{\MCoordPointSep}{;} %oder '|', '\MThinspace|\MThinspace'

\fi



% ---------------------------------- ENDE Makros, die von der Kursvariante abhaengen ----------------------------------


% diese Kommandos setzen Mathemodus vorraus
\newcommand{\MGeoAbstand}[2]{[\overline{{#1}{#2}}]}
\newcommand{\MGeoGerade}[2]{{#1}{#2}}
\newcommand{\MGeoStrecke}[2]{\overline{{#1}{#2}}}
\newcommand{\MGeoDreieck}[3]{{#1}{#2}{#3}}

%
\ifttm
\newcommand{\MOhm}{\special{html:<mn>&#x3A9;</mn>}}
\else
\newcommand{\MOhm}{\Omega} %\varOmega
\fi


\def\PERCTAG{\MAbort{PERCTAG ist zur internen verwendung in mconvert.pl reserviert, dieses Makro darf sonst nicht benutzt werden.}}

% Im Gegensatz zu einfachen html-Umgebungen werden MDirectHTML-Umgebungen von mconvert.pl am ganzen ttm-Prozess vorbeigeschleust und aus dem PDF komplett ausgeschnitten
\ifttm%
\newenvironment{MDirectHTML}{\begin{html}}{\end{html}}%
\else%
\newenvironment{MDirectHTML}{\begin{html}}{\end{html}}%
\fi

% Im Gegensatz zu einfachen Mathe-Umgebungen werden MDirectMath-Umgebungen von mconvert.pl am ganzen ttm-Prozess vorbeigeschleust, ueber MathJax realisiert, und im PDF als $$ ... $$ gesetzt
\ifttm%
\newenvironment{MDirectMath}{\begin{html}}{\end{html}}%
\else%
\newenvironment{MDirectMath}{\begin{equation*}}{\end{equation*}}% Vorsicht, auch \[ und \] werden in amsmath durch equation* redefiniert
\fi

% ---------------------------------- Location Management ---------------------------------------------

% #1 = buttonname (muss in files/images liegen und Format 48x48 haben), #2 = Vollstaendiger Einrichtungsname, #3 = Kuerzel der Einrichtung,  #4 = Name der include-texdatei
\ifttm
\newcommand{\MLocationSite}[3]{\special{html:<!-- mlocation;;}#1\special{html:;;}#2\special{html:;;}#3\special{html:;; //-->}}
\else
\newcommand{\MLocationSite}[3]{\relax}
\fi

% ---------------------------------- Copyright Management --------------------------------------------

\newcommand{\MCCLicense}{%
{\color{green}\textbf{CC BY-SA 3.0}}
}

\newcommand{\MCopyrightLabel}[1]{ (\MSRef{L_COPYRIGHTCOLLECTION}{Lizenz})\MLabel{#1}}

% Copyrightliste wird als tex-Datei im preprocessing erzeugt und unter converter/tex/copyrightcollection.tex abgelegt
% Der input-Befehl funktioniert nur, wenn die aufrufende tex-Datei auf der obersten Ebene liegt (d.h. selbst kein input/include ist, insbesondere keine Moduldatei)
\newcommand{\MCopyrightCollection}{\input{copyrightcollection.tex}}

% MCopyrightNotice fuegt eine Copyrightnotiz ein, der parser ersetzt diese durch CopyrightNoticePOST im preparsing, diese Definition wird nur fuer reine pdflatex-Uebersetzungen gebraucht
% Parameter: #1: Kurze Lizenzbeschreibung (typischerweise \MCCLicense)
%            #2: Link zum Original (http://...) oder NONE falls das Bild selbst ein Original ist, oder TIKZ falls das Bild aus einer tikz-Umgebung stammt
%            #3: Link zum Autor (http://...) oder MINT falls Original im MINT-Kolleg erstellt oder NONE falls Autor unbekannt
%            #4: Bemerkung (z.B. dass Datei mit Maple exportiert wurde)
%            #5: Labelstring fuer existierendes Label auf das copyrighted Objekt, mit MCopyrightLabel erzeugt
%            Keines der Felder darf leer sein!
\newcommand{\MCopyrightNotice}[5]{\MCopyrightNoticePOST{#1}{#2}{#3}{#4}{#5}}

\ifttm%
\newcommand{\MCopyrightNoticePOST}[5]{\relax}%
\else%
\newcommand{\MCopyrightNoticePOST}[5]{\relax}%
\fi%

% ---------------------------------- Meldungen fuer den Benutzer des Konverters ----------------------
\MPragma{mintmodversion;P0.1.0}
\MPragma{usercomment;This is file mintmod.tex version P0.1.0}


% ----------------------------------- Spezialelemente fuer Konfigurationsseite, werden nicht von mintscripts.js verwaltet --

% #1 = DOM-id der Box
\ifttm\newcommand{\MConfigbox}[1]{\special{html:<input cfieldtype="2" type="checkbox" name="Name_}#1\special{html:" id="}#1\special{html:" onchange="confHandlerChange('}#1\special{html:');"/>}}\fi % darf im PDF nicht aufgerufen werden!


\MPragma{MathSkip}

%\Mtikzexternalize

\begin{document}
%% MINTMOD Version P0.1.0, needs to be consistent with preprocesser object in tex2x and MPragma-Version at the end of this file

% Parameter aus Konvertierungsprozess (PDF und HTML-Erzeugung wenn vom Konverter aus gestartet) werden hier eingefuegt, Preambleincludes werden am Schluss angehaengt

\newif\ifttm                % gesetzt falls Uebersetzung in HTML stattfindet, sonst uebersetzung in PDF

% Wahl der Notationsvariante ist im PDF immer std, in der HTML-Uebersetzung wird vom Konverter die Auswahl modifiziert
\newif\ifvariantstd
\newif\ifvariantunotation
\variantstdtrue % Diese Zeile wird vom Konverter erkannt und ggf. modifiziert, daher nicht veraendern!


\def\MOutputDVI{1}
\def\MOutputPDF{2}
\def\MOutputHTML{3}
\newcounter{MOutput}

\ifttm
\usepackage{german}
\usepackage{array}
\usepackage{amsmath}
\usepackage{amssymb}
\usepackage{amsthm}
\else
\documentclass[ngerman,oneside]{scrbook}
\usepackage{etex}
\usepackage[latin1]{inputenc}
\usepackage{textcomp}
\usepackage[ngerman]{babel}
\usepackage[pdftex]{color}
\usepackage{xcolor}
\usepackage{graphicx}
\usepackage[all]{xy}
\usepackage{fancyhdr}
\usepackage{verbatim}
\usepackage{array}
\usepackage{float}
\usepackage{makeidx}
\usepackage{amsmath}
\usepackage{amstext}
\usepackage{amssymb}
\usepackage{amsthm}
\usepackage[ngerman]{varioref}
\usepackage{framed}
\usepackage{supertabular}
\usepackage{longtable}
\usepackage{maxpage}
\usepackage{tikz}
\usepackage{tikzscale}
\usepackage{tikz-3dplot}
\usepackage{bibgerm}
\usepackage{chemarrow}
\usepackage{polynom}
%\usepackage{draftwatermark}
\usepackage{pdflscape}
\usetikzlibrary{calc}
\usetikzlibrary{through}
\usetikzlibrary{shapes.geometric}
\usetikzlibrary{arrows}
\usetikzlibrary{intersections}
\usetikzlibrary{decorations.pathmorphing}
\usetikzlibrary{external}
\usetikzlibrary{patterns}
\usetikzlibrary{fadings}
\usepackage[colorlinks=true,linkcolor=blue]{hyperref} 
\usepackage[all]{hypcap}
%\usepackage[colorlinks=true,linkcolor=blue,bookmarksopen=true]{hyperref} 
\usepackage{ifpdf}

\usepackage{movie15}

\setcounter{tocdepth}{2} % In Inhaltsverzeichnis bis subsection
\setcounter{secnumdepth}{3} % Nummeriert bis subsubsection

\setlength{\LTpost}{0pt} % Fuer longtable
\setlength{\parindent}{0pt}
\setlength{\parskip}{8pt}
%\setlength{\parskip}{9pt plus 2pt minus 1pt}
\setlength{\abovecaptionskip}{-0.25ex}
\setlength{\belowcaptionskip}{-0.25ex}
\fi

\ifttm
\newcommand{\MDebugMessage}[1]{\special{html:<!-- debugprint;;}#1\special{html:; //-->}}
\else
%\newcommand{\MDebugMessage}[1]{\immediate\write\mintlog{#1}}
\newcommand{\MDebugMessage}[1]{}
\fi

\def\MPageHeaderDef{%
\pagestyle{fancy}%
\fancyhead[r]{(C) VE\&MINT-Projekt}
\fancyfoot[c]{\thepage\\--- CCL BY-SA 3.0 ---}
}


\ifttm%
\def\MRelax{}%
\else%
\def\MRelax{\relax}%
\fi%

%--------------------------- Uebernahme von speziellen XML-Versionen einiger LaTeX-Kommandos aus xmlbefehle.tex vom alten Kasseler Konverter ---------------

\newcommand{\MSep}{\left\|{\phantom{\frac1g}}\right.}

\newcommand{\ML}{L}

\newcommand{\MGGT}{\mathrm{ggT}}


\ifttm
% Verhindert dass die subsection-nummer doppelt in der toccaption auftaucht (sollte ggf. in toccaption gefixt werden so dass diese Ueberschreibung nicht notwendig ist)
\renewcommand{\thesubsection}{}
% Kommandos die ttm nicht kennt
\newcommand{\binomial}[2]{{#1 \choose #2}} %  Binomialkoeffizienten
\newcommand{\eur}{\begin{html}&euro;\end{html}}
\newcommand{\square}{\begin{html}&square;\end{html}}
\newcommand{\glqq}{"'}  \newcommand{\grqq}{"'}
\newcommand{\nRightarrow}{\special{html: &nrArr; }}
\newcommand{\nmid}{\special{html: &nmid; }}
\newcommand{\nparallel}{\begin{html}&nparallel;\end{html}}
\newcommand{\mapstoo}{\begin{html}<mo>&map;</mo>\end{html}}

% Schnitt und Vereinigungssymbole von Mengen haben zu kleine Abstaende; korrigiert:
\newcommand{\ccup}{\,\!\cup\,\!}
\newcommand{\ccap}{\,\!\cap\,\!}


% Umsetzung von mathbb im HTML
\renewcommand{\mathbb}[1]{\begin{html}<mo>&#1opf;</mo>\end{html}}
\fi

%---------------------- Strukturierung ----------------------------------------------------------------------------------------------------------------------

%---------------------- Kapselung des sectioning findet auf drei Ebenen statt:
% 1. Die LateX-Befehl
% 2. Die D-Versionen der Befehle, die nur die Grade der Abschnitte umhaengen falls notwendig
% 3. Die M-Versionen der Befehle, die zusaetzliche Formatierungen vornehmen, Skripten starten und das HTML codieren
% Im Modultext duerfen nur die M-Befehle verwendet werden!

\ifttm

  \def\Dsubsubsubsection#1{\subsubsubsection{#1}}
  \def\Dsubsubsection#1{\subsubsection{#1}\addtocounter{subsubsection}{1}} % ttm-Fehler korrigieren
  \def\Dsubsection#1{\subsection{#1}}
  \def\Dsection#1{\section{#1}} % Im HTML wird nur der Sektionstitel gegeben
  \def\Dchapter#1{\chapter{#1}}
  \def\Dsubsubsubsectionx#1{\subsubsubsection*{#1}}
  \def\Dsubsubsectionx#1{\subsubsection*{#1}}
  \def\Dsubsectionx#1{\subsection*{#1}}
  \def\Dsectionx#1{\section*{#1}}
  \def\Dchapterx#1{\chapter*{#1}}

\else

  \def\Dsubsubsubsection#1{\subsubsection{#1}}
  \def\Dsubsubsection#1{\subsection{#1}}
  \def\Dsubsection#1{\section{#1}}
  \def\Dsection#1{\chapter{#1}}
  \def\Dchapter#1{\title{#1}}
  \def\Dsubsubsubsectionx#1{\subsubsection*{#1}}
  \def\Dsubsubsectionx#1{\subsection*{#1}}
  \def\Dsubsectionx#1{\section*{#1}}
  \def\Dsectionx#1{\chapter*{#1}}

\fi

\newcommand{\MStdPoints}{4}
\newcommand{\MSetPoints}[1]{\renewcommand{\MStdPoints}{#1}}

% Befehl zum Abbruch der Erstellung (nur PDF)
\newcommand{\MAbort}[1]{\err{#1}}

% Prefix vor Dateieinbindungen, wird in der Baumdatei mit \renewcommand modifiziert
% und auf das Verzeichnisprefix gesetzt, in dem das gerade bearbeitete tex-Dokument liegt.
% Im HTML wird es auf das Verzeichnis der HTML-Datei gesetzt.
% Das Prefix muss mit / enden !
\newcommand{\MDPrefix}{.}

% MRegisterFile notiert eine Datei zur Einbindung in den HTML-Baum. Grafiken mit MGraphics werden automatisch eingebunden.
% Mit MLastFile erhaelt man eine Markierung fuer die zuletzt registrierte Datei.
% Diese Markierung wird im postprocessing durch den physikalischen Dateinamen ersetzt, aber nur den Namen (d.h. \MMaterial gehoert noch davor, vgl Definition von MGraphics)
% Parameter: Pfad/Name der Datei bzw. des Ordners, relativ zur Position des Modul-Tex-Dokuments.
\ifttm
\newcommand{\MRegisterFile}[1]{\addtocounter{MFileNumber}{1}\special{html:<!-- registerfile;;}#1\special{html:;;}\MDPrefix\special{html:;;}\arabic{MFileNumber}\special{html:; //-->}}
\else
\newcommand{\MRegisterFile}[1]{\addtocounter{MFileNumber}{1}}
\fi

% Testen welcher Uebersetzer hier am Werk ist

\ifttm
\setcounter{MOutput}{3}
\else
\ifx\pdfoutput\undefined
  \pdffalse
  \setcounter{MOutput}{\MOutputDVI}
  \message{Verarbeitung mit latex, Ausgabe in dvi.}
\else
  \setcounter{MOutput}{\MOutputPDF}
  \message{Verarbeitung mit pdflatex, Ausgabe in pdf.}
  \ifnum \pdfoutput=0
    \pdffalse
  \setcounter{MOutput}{\MOutputDVI}
  \message{Verarbeitung mit pdflatex, Ausgabe in dvi.}
  \else
    \ifnum\pdfoutput=1
    \pdftrue
  \setcounter{MOutput}{\MOutputPDF}
  \message{Verarbeitung mit pdflatex, Ausgabe in pdf.}
    \fi
  \fi
\fi
\fi

\ifnum\value{MOutput}=\MOutputPDF
\DeclareGraphicsExtensions{.pdf,.png,.jpg}
\fi

\ifnum\value{MOutput}=\MOutputDVI
\DeclareGraphicsExtensions{.eps,.png,.jpg}
\fi

\ifnum\value{MOutput}=\MOutputHTML
% Wird vom Konverter leider nicht erkannt und daher in split.pm hardcodiert!
\DeclareGraphicsExtensions{.png,.jpg,.gif}
\fi

% Umdefinition der hyperref-Nummerierung im PDF-Modus
\ifttm
\else
\renewcommand{\theHfigure}{\arabic{chapter}.\arabic{section}.\arabic{figure}}
\fi

% Makro, um in der HTML-Ausgabe die zuerst zu oeffnende Datei zu kennzeichnen
\ifttm
\newcommand{\MGlobalStart}{\special{html:<!-- mglobalstarttag -->}}
\else
\newcommand{\MGlobalStart}{}
\fi

% Makro, um bei scormlogin ein pullen des Benutzers bei Aufruf der Seite zu erzwingen (typischerweise auf der Einstiegsseite)
\ifttm
\newcommand{\MPullSite}{\special{html:<!-- pullsite //-->}}
\else
\newcommand{\MPullSite}{}
\fi

% Makro, um in der HTML-Ausgabe die Kapiteluebersicht zu kennzeichnen
\ifttm
\newcommand{\MGlobalChapterTag}{\special{html:<!-- mglobalchaptertag -->}}
\else
\newcommand{\MGlobalChapterTag}{}
\fi

% Makro, um in der HTML-Ausgabe die Konfiguration zu kennzeichnen
\ifttm
\newcommand{\MGlobalConfTag}{\special{html:<!-- mglobalconfigtag -->}}
\else
\newcommand{\MGlobalConfTag}{}
\fi

% Makro, um in der HTML-Ausgabe die Standortbeschreibung zu kennzeichnen
\ifttm
\newcommand{\MGlobalLocationTag}{\special{html:<!-- mgloballocationtag -->}}
\else
\newcommand{\MGlobalLocationTag}{}
\fi

% Makro, um in der HTML-Ausgabe die persoenlichen Daten zu kennzeichnen
\ifttm
\newcommand{\MGlobalDataTag}{\special{html:<!-- mglobaldatatag -->}}
\else
\newcommand{\MGlobalDataTag}{}
\fi

% Makro, um in der HTML-Ausgabe die Suchseite zu kennzeichnen
\ifttm
\newcommand{\MGlobalSearchTag}{\special{html:<!-- mglobalsearchtag -->}}
\else
\newcommand{\MGlobalSearchTag}{}
\fi

% Makro, um in der HTML-Ausgabe die Favoritenseite zu kennzeichnen
\ifttm
\newcommand{\MGlobalFavoTag}{\special{html:<!-- mglobalfavoritestag -->}}
\else
\newcommand{\MGlobalFavoTag}{}
\fi

% Makro, um in der HTML-Ausgabe die Eingangstestseite zu kennzeichnen
\ifttm
\newcommand{\MGlobalSTestTag}{\special{html:<!-- mglobalstesttag -->}}
\else
\newcommand{\MGlobalSTestTag}{}
\fi

% Makro, um in der PDF-Ausgabe ein Wasserzeichen zu definieren
\ifttm
\newcommand{\MWatermarkSettings}{\relax}
\else
\newcommand{\MWatermarkSettings}{%
% \SetWatermarkText{(c) MINT-Kolleg Baden-W�rttemberg 2014}
% \SetWatermarkLightness{0.85}
% \SetWatermarkScale{1.5}
}
\fi

\ifttm
\newcommand{\MBinom}[2]{\left({\begin{array}{c} #1 \\ #2 \end{array}}\right)}
\else
\newcommand{\MBinom}[2]{\binom{#1}{#2}}
\fi

\ifttm
\newcommand{\DeclareMathOperator}[2]{\def#1{\mathrm{#2}}}
\newcommand{\operatorname}[1]{\mathrm{#1}}
\fi

%----------------- Makros fuer die gemischte HTML/PDF-Konvertierung ------------------------------

\newcommand{\MTestName}{\relax} % wird durch Test-Umgebung gesetzt

% Fuer experimentelle Kursinhalte, die im Release-Umsetzungsvorgang eine Fehlermeldung
% produzieren sollen aber sonst normal umgesetzt werden
\newenvironment{MExperimental}{%
}{%
}

% Wird von ttm nicht richtig umgesetzt!!
\newenvironment{MExerciseItems}{%
\renewcommand\theenumi{\alph{enumi}}%
\begin{enumerate}%
}{%
\end{enumerate}%
}


\definecolor{infoshadecolor}{rgb}{0.75,0.75,0.75}
\definecolor{exmpshadecolor}{rgb}{0.875,0.875,0.875}
\definecolor{expeshadecolor}{rgb}{0.95,0.95,0.95}
\definecolor{framecolor}{rgb}{0.2,0.2,0.2}

% Bei PDF-Uebersetzung wird hinter den Start jeder Satz/Info-aehnlichen Umgebung eine leere mbox gesetzt, damit
% fuehrende Listen oder enums nicht den Zeilenumbruch kaputtmachen
%\ifttm
\def\MTB{}
%\else
%\def\MTB{\mbox{}}
%\fi


\ifttm
\newcommand{\MRelates}{\special{html:<mi>&wedgeq;</mi>}}
\else
\def\MRelates{\stackrel{\scriptscriptstyle\wedge}{=}}
\fi

\def\MInch{\text{''}}
\def\Mdd{\textit{''}}

\ifttm
\def\MNL{ \newline }
\newenvironment{MArray}[1]{\begin{array}{#1}}{\end{array}}
\else
\def\MNL{ \\ }
\newenvironment{MArray}[1]{\begin{array}{#1}}{\end{array}}
\fi

\newcommand{\MBox}[1]{$\mathrm{#1}$}
\newcommand{\MMBox}[1]{\mathrm{#1}}


\ifttm%
\newcommand{\Mtfrac}[2]{{\textstyle \frac{#1}{#2}}}
\newcommand{\Mdfrac}[2]{{\displaystyle \frac{#1}{#2}}}
\newcommand{\Mmeasuredangle}{\special{html:<mi>&angmsd;</mi>}}
\else%
\newcommand{\Mtfrac}[2]{\tfrac{#1}{#2}}
\newcommand{\Mdfrac}[2]{\dfrac{#1}{#2}}
\newcommand{\Mmeasuredangle}{\measuredangle}
\relax
\fi

% Matrizen und Vektoren

% Inhalt wird in der Form a & b \\ c & d erwartet
% Vorsicht: MVector = Komponentenspalte, MVec = Variablensymbol
\ifttm%
\newcommand{\MVector}[1]{\left({\begin{array}{c}#1\end{array}}\right)}
\else%
\newcommand{\MVector}[1]{\begin{pmatrix}#1\end{pmatrix}}
\fi



\newcommand{\MVec}[1]{\vec{#1}}
\newcommand{\MDVec}[1]{\overrightarrow{#1}}

%----------------- Umgebungen fuer Definitionen und Saetze ----------------------------------------

% Fuegt einen Tabellen-Zeilenumbruch ein im PDF, aber nicht im HTML
\newcommand{\TSkip}{\ifttm \else&\ \\\fi}

\newenvironment{infoshaded}{%
\def\FrameCommand{\fboxsep=\FrameSep \fcolorbox{framecolor}{infoshadecolor}}%
\MakeFramed {\advance\hsize-\width \FrameRestore}}%
{\endMakeFramed}

\newenvironment{expeshaded}{%
\def\FrameCommand{\fboxsep=\FrameSep \fcolorbox{framecolor}{expeshadecolor}}%
\MakeFramed {\advance\hsize-\width \FrameRestore}}%
{\endMakeFramed}

\newenvironment{exmpshaded}{%
\def\FrameCommand{\fboxsep=\FrameSep \fcolorbox{framecolor}{exmpshadecolor}}%
\MakeFramed {\advance\hsize-\width \FrameRestore}}%
{\endMakeFramed}

\def\STDCOLOR{black}

\ifttm%
\else%
\newtheoremstyle{MSatzStyle}
  {1cm}                   %Space above
  {1cm}                   %Space below
  {\normalfont\itshape}   %Body font
  {}                      %Indent amount (empty = no indent,
                          %\parindent = para indent)
  {\normalfont\bfseries}  %Thm head font
  {}                      %Punctuation after thm head
  {\newline}              %Space after thm head: " " = normal interword
                          %space; \newline = linebreak
  {\thmname{#1}\thmnumber{ #2}\thmnote{ (#3)}}
                          %Thm head spec (can be left empty, meaning
                          %`normal')
                          %
\newtheoremstyle{MDefStyle}
  {1cm}                   %Space above
  {1cm}                   %Space below
  {\normalfont}           %Body font
  {}                      %Indent amount (empty = no indent,
                          %\parindent = para indent)
  {\normalfont\bfseries}  %Thm head font
  {}                      %Punctuation after thm head
  {\newline}              %Space after thm head: " " = normal interword
                          %space; \newline = linebreak
  {\thmname{#1}\thmnumber{ #2}\thmnote{ (#3)}}
                          %Thm head spec (can be left empty, meaning
                          %`normal')
\fi%

\newcommand{\MInfoText}{Info}

\newcounter{MHintCounter}
\newcounter{MCodeEditCounter}

\newcounter{MLastIndex}  % Enthaelt die dritte Stelle (Indexnummer) des letzten angelegten Objekts
\newcounter{MLastType}   % Enthaelt den Typ des letzten angelegten Objekts (mithilfe der unten definierten Konstanten). Die Entscheidung, wie der Typ dargstellt wird, wird in split.pm beim Postprocessing getroffen.
\newcounter{MLastTypeEq} % =1 falls das Label in einer Matheumgebung (equation, eqnarray usw.) steht, =2 falls das Label in einer table-Umgebung steht

% Da ttm keine Zahlmakros verarbeiten kann, werden diese Nummern in den Zuweisungen hardcodiert!
\def\MTypeSection{1}          %# Zaehler ist section
\def\MTypeSubsection{2}       %# Zaehler ist subsection
\def\MTypeSubsubsection{3}    %# Zaehler ist subsubsection
\def\MTypeInfo{4}             %# Eine Infobox, Separatzaehler fuer die Chemie (auch wenn es dort nicht nummeriert wird) ist MInfoCounter
\def\MTypeExercise{5}         %# Eine Aufgabe, Separatzaehler fuer die Chemie ist MExerciseCounter
\def\MTypeExample{6}          %# Eine Beispielbox, Separatzaehler fuer die Chemie ist MExampleCounter
\def\MTypeExperiment{7}       %# Eine Versuchsbox, Separatzaehler fuer die Chemie ist MExperimentCounter
\def\MTypeGraphics{8}         %# Eine Graphik, Separatzaehler fuer alle FB ist MGraphicsCounter
\def\MTypeTable{9}            %# Eine Tabellennummer, hat keinen Zaehler da durch table gezaehlt wird
\def\MTypeEquation{10}        %# Eine Gleichungsnummer, hat keinen Zaehler da durch equation/eqnarray gezaehlt wird
\def\MTypeTheorem{11}         % Ein theorem oder xtheorem, Separatzaehler fuer die Chemie ist MTheoremCounter
\def\MTypeVideo{12}           %# Ein Video,Separatzaehler fuer alle FB ist MVideoCounter
\def\MTypeEntry{13}           %# Ein Eintrag fuer die Stichwortliste, wird nicht gezaehlt sondern erhaelt im preparsing ein unique-label 

% Zaehler fuer das Labelsystem sind prefixcounter, jeder Zaehler wird VOR dem gezaehlten Objekt inkrementiert und zaehlt daher das aktuelle Objekt
\newcounter{MInfoCounter}
\newcounter{MExerciseCounter}
\newcounter{MExampleCounter}
\newcounter{MExperimentCounter}
\newcounter{MGraphicsCounter}
\newcounter{MTableCounter}
\newcounter{MEquationCounter}  % Nur im HTML, sonst durch "equation"-counter von latex realisiert
\newcounter{MTheoremCounter}
\newcounter{MObjectCounter}   % Gemeinsamer Zaehler fuer Objekte (ausser Grafiken/Tabellen) in Mathe/Info/Physik
\newcounter{MVideoCounter}
\newcounter{MEntryCounter}

\newcounter{MTestSite} % 1 = Subsubsection ist eine Pruefungsseite, 0 = ist eine normale Seite (inkl. Hilfeseite)

\def\MCell{$\phantom{a}$}

\newenvironment{MExportExercise}{\begin{MExercise}}{\end{MExercise}} % wird von mconvert abgefangen

\def\MGenerateExNumber{%
\ifnum\value{MSepNumbers}=0%
\arabic{section}.\arabic{subsection}.\arabic{MObjectCounter}\setcounter{MLastIndex}{\value{MObjectCounter}}%
\else%
\arabic{section}.\arabic{subsection}.\arabic{MExerciseCounter}\setcounter{MLastIndex}{\value{MExerciseCounter}}%
\fi%
}%

\def\MGenerateExmpNumber{%
\ifnum\value{MSepNumbers}=0%
\arabic{section}.\arabic{subsection}.\arabic{MObjectCounter}\setcounter{MLastIndex}{\value{MObjectCounter}}%
\else%
\arabic{section}.\arabic{subsection}.\arabic{MExerciseCounter}\setcounter{MLastIndex}{\value{MExampleCounter}}%
\fi%
}%

\def\MGenerateInfoNumber{%
\ifnum\value{MSepNumbers}=0%
\arabic{section}.\arabic{subsection}.\arabic{MObjectCounter}\setcounter{MLastIndex}{\value{MObjectCounter}}%
\else%
\arabic{section}.\arabic{subsection}.\arabic{MExerciseCounter}\setcounter{MLastIndex}{\value{MInfoCounter}}%
\fi%
}%

\def\MGenerateSiteNumber{%
\arabic{section}.\arabic{subsection}.\arabic{subsubsection}%
}%

% Funktionalitaet fuer Auswahlaufgaben

\newcounter{MExerciseCollectionCounter} % = 0 falls nicht in collection-Umgebung, ansonsten Schachtelungstiefe
\newcounter{MExerciseCollectionTextCounter} % wird von MExercise-Umgebung inkrementiert und von MExerciseCollection-Umgebung auf Null gesetzt

\ifttm
% MExerciseCollection gruppiert Aufgaben, die dynamisch aus der Datenbank gezogen werden und nicht direkt in der HTML-Seite stehen
% Parameter: #1 = ID der Collection, muss eindeutig fuer alle IN DER DB VORHANDENEN collections sein unabhaengig vom Kurs
%            #2 = Optionsargument (im Moment: 1 = Iterative Auswahl, 2 = Zufallsbasierte Auswahl)
\newenvironment{MExerciseCollection}[2]{%
\addtocounter{MExerciseCollectionCounter}{1}
\setcounter{MExerciseCollectionTextCounter}{0}
\special{html:<!-- mexercisecollectionstart;;}#1\special{html:;;}#2\special{html:;; //-->}%
}{%
\special{html:<!-- mexercisecollectionstop //-->}%
\addtocounter{MExerciseCollectionCounter}{-1}
}
\else
\newenvironment{MExerciseCollection}[2]{%
\addtocounter{MExerciseCollectionCounter}{1}
\setcounter{MExerciseCollectionTextCounter}{0}
}{%
\addtocounter{MExerciseCollectionCounter}{-1}
}
\fi

% Bei Uebersetzung nach PDF werden die theorem-Umgebungen verwendet, bei Uebersetzung in HTML ein manuelles Makro
\ifttm%

  \newenvironment{MHint}[1]{  \special{html:<button name="Name_MHint}\arabic{MHintCounter}\special{html:" class="hintbutton_closed" id="MHint}\arabic{MHintCounter}\special{html:_button" %
  type="button" onclick="toggle_hint('MHint}\arabic{MHintCounter}\special{html:');">}#1\special{html:</button>}
  \special{html:<div class="hint" style="display:none" id="MHint}\arabic{MHintCounter}\special{html:"> }}{\begin{html}</div>\end{html}\addtocounter{MHintCounter}{1}}

  \newenvironment{MCOSHZusatz}{  \special{html:<button name="Name_MHint}\arabic{MHintCounter}\special{html:" class="chintbutton_closed" id="MHint}\arabic{MHintCounter}\special{html:_button" %
  type="button" onclick="toggle_hint('MHint}\arabic{MHintCounter}\special{html:');">}Weiterf�hrende Inhalte\special{html:</button>}
  \special{html:<div class="hintc" style="display:none" id="MHint}\arabic{MHintCounter}\special{html:">
  <div class="coshwarn">Diese Inhalte gehen �ber das Kursniveau hinaus und werden in den Aufgaben und Tests nicht abgefragt.</div><br />}
  \addtocounter{MHintCounter}{1}}{\begin{html}</div>\end{html}}

  
  \newenvironment{MDefinition}{\begin{definition}\setcounter{MLastIndex}{\value{definition}}\ \\}{\end{definition}}

  
  \newenvironment{MExercise}{
  \renewcommand{\MStdPoints}{4}
  \addtocounter{MExerciseCounter}{1}
  \addtocounter{MObjectCounter}{1}
  \setcounter{MLastType}{5}

  \ifnum\value{MExerciseCollectionCounter}=0\else\addtocounter{MExerciseCollectionTextCounter}{1}\special{html:<!-- mexercisetextstart;;}\arabic{MExerciseCollectionTextCounter}\special{html:;; //-->}\fi
  \special{html:<div class="aufgabe" id="ADIV_}\MGenerateExNumber\special{html:">}%
  \textbf{Aufgabe \MGenerateExNumber
  } \ \\}{
  \special{html:</div><!-- mfeedbackbutton;Aufgabe;}\arabic{MTestSite}\special{html:;}\MGenerateExNumber\special{html:; //-->}
  \ifnum\value{MExerciseCollectionCounter}=0\else\special{html:<!-- mexercisetextstop //-->}\fi
  }

  % Stellt eine Kombination aus Aufgabe, Loesungstext und Eingabefeld bereit,
  % bei der Aufgabentext und Musterloesung sowie die zugehoerigen Feldelemente
  % extern bezogen und div-aktualisiert werden, das Eingabefeld aber immer das gleiche ist.
  \newenvironment{MFetchExercise}{
  \addtocounter{MExerciseCounter}{1}
  \addtocounter{MObjectCounter}{1}
  \setcounter{MLastType}{5}

  \special{html:<div class="aufgabe" id="ADIV_}\MGenerateExNumber\special{html:">}%
  \textbf{Aufgabe \MGenerateExNumber
  } \ \\%
  \special{html:</div><div class="exfetch_text" id="ADIVTEXT_}\MGenerateExNumber\special{html:">}%
  \special{html:</div><div class="exfetch_sol" id="ADIVSOL_}\MGenerateExNumber\special{html:">}%
  \special{html:</div><div class="exfetch_input" id="ADIVINPUT_}\MGenerateExNumber\special{html:">}%
  }{
  \special{html:</div>}
  }

  \newenvironment{MExample}{
  \addtocounter{MExampleCounter}{1}
  \addtocounter{MObjectCounter}{1}
  \setcounter{MLastType}{6}
  \begin{html}
  <div class="exmp">
  <div class="exmprahmen">
  \end{html}\textbf{Beispiel
  \ifnum\value{MSepNumbers}=0
  \arabic{section}.\arabic{subsection}.\arabic{MObjectCounter}\setcounter{MLastIndex}{\value{MObjectCounter}}
  \else
  \arabic{section}.\arabic{subsection}.\arabic{MExampleCounter}\setcounter{MLastIndex}{\value{MExampleCounter}}
  \fi
  } \ \\}{\begin{html}</div>
  </div>
  \end{html}
  \special{html:<!-- mfeedbackbutton;Beispiel;}\arabic{MTestSite}\special{html:;}\MGenerateExmpNumber\special{html:; //-->}
  }

  \newenvironment{MExperiment}{
  \addtocounter{MExperimentCounter}{1}
  \addtocounter{MObjectCounter}{1}
  \setcounter{MLastType}{7}
  \begin{html}
  <div class="expe">
  <div class="experahmen">
  \end{html}\textbf{Versuch
  \ifnum\value{MSepNumbers}=0
  \arabic{section}.\arabic{subsection}.\arabic{MObjectCounter}\setcounter{MLastIndex}{\value{MObjectCounter}}
  \else
%  \arabic{MExperimentCounter}\setcounter{MLastIndex}{\value{MExperimentCounter}}
  \arabic{section}.\arabic{subsection}.\arabic{MExperimentCounter}\setcounter{MLastIndex}{\value{MExperimentCounter}}
  \fi
  } \ \\}{\begin{html}</div>
  </div>
  \end{html}}

  \newenvironment{MChemInfo}{
  \setcounter{MLastType}{4}
  \begin{html}
  <div class="info">
  <div class="inforahmen">
  \end{html}}{\begin{html}</div>
  </div>
  \end{html}}

  \newenvironment{MXInfo}[1]{
  \addtocounter{MInfoCounter}{1}
  \addtocounter{MObjectCounter}{1}
  \setcounter{MLastType}{4}
  \begin{html}
  <div class="info">
  <div class="inforahmen">
  \end{html}\textbf{#1
  \ifnum\value{MInfoNumbers}=0
  \else
    \ifnum\value{MSepNumbers}=0
    \arabic{section}.\arabic{subsection}.\arabic{MObjectCounter}\setcounter{MLastIndex}{\value{MObjectCounter}}
    \else
    \arabic{MInfoCounter}\setcounter{MLastIndex}{\value{MInfoCounter}}
    \fi
  \fi
  } \ \\}{\begin{html}</div>
  </div>
  \end{html}
  \special{html:<!-- mfeedbackbutton;Info;}\arabic{MTestSite}\special{html:;}\MGenerateInfoNumber\special{html:; //-->}
  }

  \newenvironment{MInfo}{\ifnum\value{MInfoNumbers}=0\begin{MChemInfo}\else\begin{MXInfo}{Info}\ \\ \fi}{\ifnum\value{MInfoNumbers}=0\end{MChemInfo}\else\end{MXInfo}\fi}

\else%

  \theoremstyle{MSatzStyle}
  \newtheorem{thm}{Satz}[section]
  \newtheorem{thmc}{Satz}
  \theoremstyle{MDefStyle}
  \newtheorem{defn}[thm]{Definition}
  \newtheorem{exmp}[thm]{Beispiel}
  \newtheorem{info}[thm]{\MInfoText}
  \theoremstyle{MDefStyle}
  \newtheorem{defnc}{Definition}
  \theoremstyle{MDefStyle}
  \newtheorem{exmpc}{Beispiel}[section]
  \theoremstyle{MDefStyle}
  \newtheorem{infoc}{\MInfoText}
  \theoremstyle{MDefStyle}
  \newtheorem{exrc}{Aufgabe}[section]
  \theoremstyle{MDefStyle}
  \newtheorem{verc}{Versuch}[section]
  
  \newenvironment{MFetchExercise}{}{} % kann im PDF nicht dargestellt werden
  
  \newenvironment{MExercise}{\begin{exrc}\renewcommand{\MStdPoints}{1}\MTB}{\end{exrc}}
  \newenvironment{MHint}[1]{\ \\ \underline{#1:}\\}{}
  \newenvironment{MCOSHZusatz}{\ \\ \underline{Weiterf�hrende Inhalte:}\\}{}
  \newenvironment{MDefinition}{\ifnum\value{MInfoNumbers}=0\begin{defnc}\else\begin{defn}\fi\MTB}{\ifnum\value{MInfoNumbers}=0\end{defnc}\else\end{defn}\fi}
%  \newenvironment{MExample}{\begin{exmp}}{\ \linebreak[1] \ \ \ \ $\phantom{a}$ \ \hfill $\blacklozenge$\end{exmp}}
  \newenvironment{MExample}{
    \ifnum\value{MInfoNumbers}=0\begin{exmpc}\else\begin{exmp}\fi
    \MTB
    \begin{exmpshaded}
    \ \newline
}{
    \end{exmpshaded}
    \ifnum\value{MInfoNumbers}=0\end{exmpc}\else\end{exmp}\fi
}
  \newenvironment{MChemInfo}{\begin{infoshaded}}{\end{infoshaded}}

  \newenvironment{MInfo}{\ifnum\value{MInfoNumbers}=0\begin{MChemInfo}\else\renewcommand{\MInfoText}{Info}\begin{info}\begin{infoshaded}
  \MTB
   \ \newline
    \fi
  }{\ifnum\value{MInfoNumbers}=0\end{MChemInfo}\else\end{infoshaded}\end{info}\fi}

  \newenvironment{MXInfo}[1]{
    \renewcommand{\MInfoText}{#1}
    \ifnum\value{MInfoNumbers}=0\begin{infoc}\else\begin{info}\fi%
    \MTB
    \begin{infoshaded}
    \ \newline
  }{\end{infoshaded}\ifnum\value{MInfoNumbers}=0\end{infoc}\else\end{info}\fi}

  \newenvironment{MExperiment}{
    \renewcommand{\MInfoText}{Versuch}
    \ifnum\value{MInfoNumbers}=0\begin{verc}\else\begin{info}\fi
    \MTB
    \begin{expeshaded}
    \ \newline
  }{
    \end{expeshaded}
    \ifnum\value{MInfoNumbers}=0\end{verc}\else\end{info}\fi
  }
\fi%

% MHint sollte nicht direkt fuer Loesungen benutzt werden wegen solutionselect
\newenvironment{MSolution}{\begin{MHint}{L"osung}}{\end{MHint}}

\newcounter{MCodeCounter}

\ifttm
\newenvironment{MCode}{\special{html:<!-- mcodestart -->}\ttfamily\color{blue}}{\special{html:<!-- mcodestop -->}}
\else
\newenvironment{MCode}{\begin{flushleft}\ttfamily\addtocounter{MCodeCounter}{1}}{\addtocounter{MCodeCounter}{-1}\end{flushleft}}
% Ohne color-Statement da inkompatible mit framed/shaded-Boxen aus dem framed-package
\fi

%----------------- Sonderdefinitionen fuer Symbole, die der Konverter nicht kann ----------------------------------------------

\ifttm%
\newcommand{\MUnderset}[2]{\underbrace{#2}_{#1}}%
\else%
\newcommand{\MUnderset}[2]{\underset{#1}{#2}}%
\fi%

\ifttm
\newcommand{\MThinspace}{\special{html:<mi>&#x2009;</mi>}}
\else
\newcommand{\MThinspace}{\,}
\fi

\ifttm
\newcommand{\glq}{\begin{html}&sbquo;\end{html}}
\newcommand{\grq}{\begin{html}&lsquo;\end{html}}
\newcommand{\glqq}{\begin{html}&bdquo;\end{html}}
\newcommand{\grqq}{\begin{html}&ldquo;\end{html}}
\fi

\ifttm
\newcommand{\MNdash}{\begin{html}&ndash;\end{html}}
\else
\newcommand{\MNdash}{--}
\fi

%\ifttm\def\MIU{\special{html:<mi>&#8520;</mi>}}\else\def\MIU{\mathrm{i}}\fi
\def\MIU{\mathrm{i}}
\def\MEU{e} % TU9-Onlinekurs: italic-e
%\def\MEU{\mathrm{e}} % Alte Onlinemodule: roman-e
\def\MD{d} % Kursives d in Integralen im TU9-Onlinekurs
%\def\MD{\mathrm{d}} % roman-d in den alten Onlinemodulen
\def\MDB{\|}

%zusaetzlicher Leerraum vor "\MD"
\ifttm%
\def\MDSpace{\special{html:<mi>&#x2009;</mi>}}
\else%
\def\MDSpace{\,}
\fi%
\newcommand{\MDwSp}{\MDSpace\MD}%

\ifttm
\def\Mdq{\dq}
\else
\def\Mdq{\dq}
\fi

\def\MSpan#1{\left<{#1}\right>}
\def\MSetminus{\setminus}
\def\MIM{I}

\ifttm
\newcommand{\ld}{\text{ld}}
\newcommand{\lg}{\text{lg}}
\else
\DeclareMathOperator{\ld}{ld}
%\newcommand{\lg}{\text{lg}} % in latex schon definiert
\fi


\def\Mmapsto{\ifttm\special{html:<mi>&mapsto;</mi>}\else\mapsto\fi} 
\def\Mvarphi{\ifttm\phi\else\varphi\fi}
\def\Mphi{\ifttm\varphi\else\phi\fi}
\ifttm%
\newcommand{\MEumu}{\special{html:<mi>&#x3BC;</mi>}}%
\else%
\newcommand{\MEumu}{\textrm{\textmu}}%
\fi
\def\Mvarepsilon{\ifttm\epsilon\else\varepsilon\fi}
\def\Mepsilon{\ifttm\varepsilon\else\epsilon\fi}
\def\Mvarkappa{\ifttm\kappa\else\varkappa\fi}
\def\Mkappa{\ifttm\varkappa\else\kappa\fi}
\def\Mcomplement{\ifttm\special{html:<mi>&comp;</mi>}\else\complement\fi} 
\def\MWW{\mathrm{WW}}
\def\Mmod{\ifttm\special{html:<mi>&nbsp;mod&nbsp;</mi>}\else\mod\fi} 

\ifttm%
\def\mod{\text{\;mod\;}}%
\def\MNEquiv{\special{html:<mi>&NotCongruent;</mi>}}% 
\def\MNSubseteq{\special{html:<mi>&NotSubsetEqual;</mi>}}%
\def\MEmptyset{\special{html:<mi>&empty;</mi>}}%
\def\MVDots{\special{html:<mi>&#x22EE;</mi>}}%
\def\MHDots{\special{html:<mi>&#x2026;</mi>}}%
\def\Mddag{\special{html:<mi>&#x1202;</mi>}}%
\def\sphericalangle{\special{html:<mi>&measuredangle;</mi>}}%
\def\nparallel{\special{html:<mi>&nparallel;</mi>}}%
\def\MProofEnd{\special{html:<mi>&#x25FB;</mi>}}%
\newenvironment{MProof}[1]{\underline{#1}:\MCR\MCR}{\hfill $\MProofEnd$}%
\else%
\def\MNEquiv{\not\equiv}%
\def\MNSubseteq{\not\subseteq}%
\def\MEmptyset{\emptyset}%
\def\MVDots{\vdots}%
\def\MHDots{\hdots}%
\def\Mddag{\ddag}%
\newenvironment{MProof}[1]{\begin{proof}[#1]}{\end{proof}}%
\fi%



% Spaces zum Auffuellen von Tabellenbreiten, die nur im HTML wirken
\ifttm%
\def\MTSP{\:}%
\else%
\def\MTSP{}%
\fi%

\DeclareMathOperator{\arsinh}{arsinh}
\DeclareMathOperator{\arcosh}{arcosh}
\DeclareMathOperator{\artanh}{artanh}
\DeclareMathOperator{\arcoth}{arcoth}


\newcommand{\MMathSet}[1]{\mathbb{#1}}
\def\N{\MMathSet{N}}
\def\Z{\MMathSet{Z}}
\def\Q{\MMathSet{Q}}
\def\R{\MMathSet{R}}
\def\C{\MMathSet{C}}

\newcounter{MForLoopCounter}
\newcommand{\MForLoop}[2]{\setcounter{MForLoopCounter}{#1}\ifnum\value{MForLoopCounter}=0{}\else{{#2}\addtocounter{MForLoopCounter}{-1}\MForLoop{\value{MForLoopCounter}}{#2}}\fi}

\newcounter{MSiteCounter}
\newcounter{MFieldCounter} % Kombination section.subsection.site.field ist eindeutig in allen Modulen, field alleine nicht

\newcounter{MiniMarkerCounter}

\ifttm
\newenvironment{MMiniPageP}[1]{\begin{minipage}{#1\linewidth}\special{html:<!-- minimarker;;}\arabic{MiniMarkerCounter}\special{html:;;#1; //-->}}{\end{minipage}\addtocounter{MiniMarkerCounter}{1}}
\else
\newenvironment{MMiniPageP}[1]{\begin{minipage}{#1\linewidth}}{\end{minipage}\addtocounter{MiniMarkerCounter}{1}}
\fi

\newcounter{AlignCounter}

\newcommand{\MStartJustify}{\ifttm\special{html:<!-- startalign;;}\arabic{AlignCounter}\special{html:;;justify; //-->}\fi}
\newcommand{\MStopJustify}{\ifttm\special{html:<!-- stopalign;;}\arabic{AlignCounter}\special{html:; //-->}\fi\addtocounter{AlignCounter}{1}}

\newenvironment{MJTabular}[1]{
\MStartJustify
\begin{tabular}{#1}
}{
\end{tabular}
\MStopJustify
}

\newcommand{\MImageLeft}[2]{
\begin{center}
\begin{tabular}{lc}
\MStartJustify
\begin{MMiniPageP}{0.65}
#1
\end{MMiniPageP}
\MStopJustify
&
\begin{MMiniPageP}{0.3}
#2  
\end{MMiniPageP}
\end{tabular}
\end{center}
}

\newcommand{\MImageHalf}[2]{
\begin{center}
\begin{tabular}{lc}
\MStartJustify
\begin{MMiniPageP}{0.45}
#1
\end{MMiniPageP}
\MStopJustify
&
\begin{MMiniPageP}{0.45}
#2  
\end{MMiniPageP}
\end{tabular}
\end{center}
}

\newcommand{\MBigImageLeft}[2]{
\begin{center}
\begin{tabular}{lc}
\MStartJustify
\begin{MMiniPageP}{0.25}
#1
\end{MMiniPageP}
\MStopJustify
&
\begin{MMiniPageP}{0.7}
#2  
\end{MMiniPageP}
\end{tabular}
\end{center}
}

\ifttm
\def\No{\mathbb{N}_0}
\else
\def\No{\ensuremath{\N_0}}
\fi
\def\MT{\textrm{\tiny T}}
\newcommand{\MTranspose}[1]{{#1}^{\MT}}
\ifttm
\newcommand{\MRe}{\mathsf{Re}}
\newcommand{\MIm}{\mathsf{Im}}
\else
\DeclareMathOperator{\MRe}{Re}
\DeclareMathOperator{\MIm}{Im}
\fi

\newcommand{\Mid}{\mathrm{id}}
\newcommand{\MFeinheit}{\mathrm{feinh}}

\ifttm
\newcommand{\Msubstack}[1]{\begin{array}{c}{#1}\end{array}}
\else
\newcommand{\Msubstack}[1]{\substack{#1}}
\fi

% Typen von Fragefeldern:
% 1 = Alphanumerisch, case-sensitive-Vergleich
% 2 = Ja/Nein-Checkbox, Loesung ist 0 oder 1   (OPTION = Image-id fuer Rueckmeldung)
% 3 = Reelle Zahlen Geparset
% 4 = Funktionen Geparset (mit Stuetzstellen zur ueberpruefung)

% Dieser Befehl erstellt ein interaktives Aufgabenfeld. Parameter:
% - #1 Laenge in Zeichen
% - #2 Loesungstext (alphanumerisch, case sensitive)
% - #3 AufgabenID (alphanumerisch, case sensitive)
% - #4 Typ (Kennnummer)
% - #5 String fuer Optionen (ggf. mit Semikolon getrennte Einzelstrings)
% - #6 Anzahl Punkte
% - #7 uxid (kann z.B. Loesungsstring sein)
% ACHTUNG: Die langen Zeilen bitte so lassen, Zeilenumbrueche im tex werden in div's umgesetzt
\newcommand{\MQuestionID}[7]{
\ifttm
\special{html:<!-- mdeclareuxid;;}UX#7\special{html:;;}\arabic{section}\special{html:;;}#3\special{html:;; //-->}%
\special{html:<!-- mdeclarepoints;;}\arabic{section}\special{html:;;}#3\special{html:;;}#6\special{html:;;}\arabic{MTestSite}\special{html:;;}\arabic{chapter}%
\special{html:;; //--><!-- onloadstart //-->CreateQuestionObj("}#7\special{html:",}\arabic{MFieldCounter}\special{html:,"}#2%
\special{html:","}#3\special{html:",}#4\special{html:,"}#5\special{html:",}#6\special{html:,}\arabic{MTestSite}\special{html:,}\arabic{section}%
\special{html:);<!-- onloadstop //-->}%
\special{html:<input mfieldtype="}#4\special{html:" name="Name_}#3\special{html:" id="}#3\special{html:" type="text" size="}#1\special{html:" maxlength="}#1%
\special{html:" }\ifnum\value{MGroupActive}=0\special{html:onfocus="handlerFocus(}\arabic{MFieldCounter}%
\special{html:);" onblur="handlerBlur(}\arabic{MFieldCounter}\special{html:);" onkeyup="handlerChange(}\arabic{MFieldCounter}\special{html:,0);" onpaste="handlerChange(}\arabic{MFieldCounter}\special{html:,0);" oninput="handlerChange(}\arabic{MFieldCounter}\special{html:,0);" onpropertychange="handlerChange(}\arabic{MFieldCounter}\special{html:,0);"/>}%
\special{html:<img src="images/questionmark.gif" width="20" height="20" border="0" align="absmiddle" id="}QM#3\special{html:"/>}
\else%
\special{html:onblur="handlerBlur(}\arabic{MFieldCounter}%
\special{html:);" onfocus="handlerFocus(}\arabic{MFieldCounter}\special{html:);" onkeyup="handlerChange(}\arabic{MFieldCounter}\special{html:,1);" onpaste="handlerChange(}\arabic{MFieldCounter}\special{html:,1);" oninput="handlerChange(}\arabic{MFieldCounter}\special{html:,1);" onpropertychange="handlerChange(}\arabic{MFieldCounter}\special{html:,1);"/>}%
\special{html:<img src="images/questionmark.gif" width="20" height="20" border="0" align="absmiddle" id="}QM#3\special{html:"/>}\fi%
\else%
\ifnum\value{QBoxFlag}=1\fbox{$\phantom{\MForLoop{#1}{b}}$}\else$\phantom{\MForLoop{#1}{b}}$\fi%
\fi%
}

% ACHTUNG: Die langen Zeilen bitte so lassen, Zeilenumbrueche im tex werden in div's umgesetzt
% QuestionCheckbox macht ausserhalb einer QuestionGroup keinen Sinn!
% #1 = solution (1 oder 0), ggf. mit ::smc abgetrennt auszuschliessende single-choice-boxen (UXIDs durch , getrennt), #2 = id, #3 = points, #4 = uxid
\newcommand{\MQuestionCheckbox}[4]{
\ifttm
\special{html:<!-- mdeclareuxid;;}UX#4\special{html:;;}\arabic{section}\special{html:;;}#2\special{html:;; //-->}%
\ifnum\value{MGroupActive}=0\MDebugMessage{ERROR: Checkbox Nr. \arabic{MFieldCounter}\ ist nicht in einer Kontrollgruppe, es wird niemals eine Loesung angezeigt!}\fi
\special{html: %
<!-- mdeclarepoints;;}\arabic{section}\special{html:;;}#2\special{html:;;}#3\special{html:;;}\arabic{MTestSite}\special{html:;;}\arabic{chapter}%
\special{html:;; //--><!-- onloadstart //-->CreateQuestionObj("}#4\special{html:",}\arabic{MFieldCounter}\special{html:,"}#1\special{html:","}#2\special{html:",2,"IMG}#2%
\special{html:",}#3\special{html:,}\arabic{MTestSite}\special{html:,}\arabic{section}\special{html:);<!-- onloadstop //-->}%
\special{html:<input mfieldtype="2" type="checkbox" name="Name_}#2\special{html:" id="}#2\special{html:" onchange="handlerChange(}\arabic{MFieldCounter}\special{html:,1);"/><img src="images/questionmark.gif" name="}Name_IMG#2%
\special{html:" width="20" height="20" border="0" align="absmiddle" id="}IMG#2\special{html:"/> }%
\else%
\ifnum\value{QBoxFlag}=1\fbox{$\phantom{X}$}\else$\phantom{X}$\fi%
\fi%
}

\def\MGenerateID{QFELD_\arabic{section}.\arabic{subsection}.\arabic{MSiteCounter}.QF\arabic{MFieldCounter}}

% #1 = 0/1 ggf. mit ::smc abgetrennt auszuschliessende single-choice-boxen (UXIDs durch , getrennt ohne UX), #2 = uxid ohne UX
\newcommand{\MCheckbox}[2]{
\MQuestionCheckbox{#1}{\MGenerateID}{\MStdPoints}{#2}
\addtocounter{MFieldCounter}{1}
}

% Erster Parameter: Zeichenlaenge der Eingabebox, zweiter Parameter: Loesungstext
\newcommand{\MQuestion}[2]{
\MQuestionID{#1}{#2}{\MGenerateID}{1}{0}{\MStdPoints}{#2}
\addtocounter{MFieldCounter}{1}
}

% Erster Parameter: Zeichenlaenge der Eingabebox, zweiter Parameter: Loesungstext
\newcommand{\MLQuestion}[3]{
\MQuestionID{#1}{#2}{\MGenerateID}{1}{0}{\MStdPoints}{#3}
\addtocounter{MFieldCounter}{1}
}

% Parameter: Laenge des Feldes, Loesung (wird auch geparsed), Stellen Genauigkeit hinter dem Komma, weitere Stellen werden mathematisch gerundet vor Vergleich
\newcommand{\MParsedQuestion}[3]{
\MQuestionID{#1}{#2}{\MGenerateID}{3}{#3}{\MStdPoints}{#2}
\addtocounter{MFieldCounter}{1}
}

% Parameter: Laenge des Feldes, Loesung (wird auch geparsed), Stellen Genauigkeit hinter dem Komma, weitere Stellen werden mathematisch gerundet vor Vergleich
\newcommand{\MLParsedQuestion}[4]{
\MQuestionID{#1}{#2}{\MGenerateID}{3}{#3}{\MStdPoints}{#4}
\addtocounter{MFieldCounter}{1}
}

% Parameter: Laenge des Feldes, Loesungsfunktion, Anzahl Stuetzstellen, Funktionsvariablen durch Kommata getrennt (nicht case-sensitive), Anzahl Nachkommastellen im Vergleich
\newcommand{\MFunctionQuestion}[5]{
\MQuestionID{#1}{#2}{\MGenerateID}{4}{#3;#4;#5;0}{\MStdPoints}{#2}
\addtocounter{MFieldCounter}{1}
}

% Parameter: Laenge des Feldes, Loesungsfunktion, Anzahl Stuetzstellen, Funktionsvariablen durch Kommata getrennt (nicht case-sensitive), Anzahl Nachkommastellen im Vergleich, UXID
\newcommand{\MLFunctionQuestion}[6]{
\MQuestionID{#1}{#2}{\MGenerateID}{4}{#3;#4;#5;0}{\MStdPoints}{#6}
\addtocounter{MFieldCounter}{1}
}

% Parameter: Laenge des Feldes, Loesungsintervall, Genauigkeit der Zahlenwertpruefung
\newcommand{\MIntervalQuestion}[3]{
\MQuestionID{#1}{#2}{\MGenerateID}{6}{#3}{\MStdPoints}{#2}
\addtocounter{MFieldCounter}{1}
}

% Parameter: Laenge des Feldes, Loesungsintervall, Genauigkeit der Zahlenwertpruefung, UXID
\newcommand{\MLIntervalQuestion}[4]{
\MQuestionID{#1}{#2}{\MGenerateID}{6}{#3}{\MStdPoints}{#4}
\addtocounter{MFieldCounter}{1}
}

% Parameter: Laenge des Feldes, Loesungsfunktion, Anzahl Stuetzstellen, Funktionsvariable (nicht case-sensitive), Anzahl Nachkommastellen im Vergleich, Vereinfachungsbedingung
% Vereinfachungsbedingung ist eine der Folgenden:
% 0 = Keine Vereinfachungsbedingung
% 1 = Keine Klammern (runde oder eckige) mehr im vereinfachten Ausdruck
% 2 = Faktordarstellung (Term hat Produkte als letzte Operation, Summen als vorgeschaltete Operation)
% 3 = Summendarstellung (Term hat Summen als letzte Operation, Produkte als vorgeschaltete Operation)
% Flag 512: Besondere Stuetzstellen (nur >1 und nur schwach rational), sonst symmetrisch um Nullpunkt und ganze Zahlen inkl. Null werden getroffen
\newcommand{\MSimplifyQuestion}[6]{
\MQuestionID{#1}{#2}{\MGenerateID}{4}{#3;#4;#5;#6}{\MStdPoints}{#2}
\addtocounter{MFieldCounter}{1}
}

\newcommand{\MLSimplifyQuestion}[7]{
\MQuestionID{#1}{#2}{\MGenerateID}{4}{#3;#4;#5;#6}{\MStdPoints}{#7}
\addtocounter{MFieldCounter}{1}
}

% Parameter: Laenge des Feldes, Loesung (optionaler Ausdruck), Anzahl Stuetzstellen, Funktionsvariable (nicht case-sensitive), Anzahl Nachkommastellen im Vergleich, Spezialtyp (string-id)
\newcommand{\MLSpecialQuestion}[7]{
\MQuestionID{#1}{#2}{\MGenerateID}{7}{#3;#4;#5;#6}{\MStdPoints}{#7}
\addtocounter{MFieldCounter}{1}
}

\newcounter{MGroupStart}
\newcounter{MGroupEnd}
\newcounter{MGroupActive}

\newenvironment{MQuestionGroup}{
\setcounter{MGroupStart}{\value{MFieldCounter}}
\setcounter{MGroupActive}{1}
}{
\setcounter{MGroupActive}{0}
\setcounter{MGroupEnd}{\value{MFieldCounter}}
\addtocounter{MGroupEnd}{-1}
}

\newcommand{\MGroupButton}[1]{
\ifttm
\special{html:<button name="Name_Group}\arabic{MGroupStart}\special{html:to}\arabic{MGroupEnd}\special{html:" id="Group}\arabic{MGroupStart}\special{html:to}\arabic{MGroupEnd}\special{html:" %
type="button" onclick="group_button(}\arabic{MGroupStart}\special{html:,}\arabic{MGroupEnd}\special{html:);">}#1\special{html:</button>}
\else
\phantom{#1}
\fi
}

%----------------- Makros fuer die modularisierte Darstellung ------------------------------------

\def\MyText#1{#1}

% is used internally by the conversion package, should not be used by original tex documents
\def\MOrgLabel#1{\relax}

\ifttm

% Ein MLabel wird im html codiert durch das tag <!-- mmlabel;;Labelbezeichner;;SubjectArea;;chapter;;section;;subsection;;Index;;Objekttyp; //-->
\def\MLabel#1{%
\ifnum\value{MLastType}=8%
\ifnum\value{MCaptionOn}=0%
\MDebugMessage{ERROR: Grafik \arabic{MGraphicsCounter} hat separates label: #1 (Grafiklabels sollten nur in der Caption stehen)}%
\fi
\fi
\ifnum\value{MLastType}=12%
\ifnum\value{MCaptionOn}=0%
\MDebugMessage{ERROR: Video \arabic{MVideoCounter} hat separates label: #1 (Videolabels sollten nur in der Caption stehen}%
\fi
\fi
\ifnum\value{MLastType}=10\setcounter{MLastIndex}{\value{equation}}\fi
\label{#1}\begin{html}<!-- mmlabel;;#1;;\end{html}\arabic{MSubjectArea}\special{html:;;}\arabic{chapter}\special{html:;;}\arabic{section}\special{html:;;}\arabic{subsection}\special{html:;;}\arabic{MLastIndex}\special{html:;;}\arabic{MLastType}\special{html:; //-->}}%

\else

% Sonderbehandlung im PDF fuer Abbildungen in separater aux-Datei, da MGraphics die figure-Umgebung nicht verwendet
\def\MLabel#1{%
\ifnum\value{MLastType}=8%
\ifnum\value{MCaptionOn}=0%
\MDebugMessage{ERROR: Grafik \arabic{MGraphicsCounter} hat separates label: #1 (Grafiklabels sollten nur in der Caption stehen}%
\fi
\fi
\ifnum\value{MLastType}=12%
\ifnum\value{MCaptionOn}=0%
\MDebugMessage{ERROR: Video \arabic{MVideoCounter} hat separates label: #1 (Videolabels sollten nur in der Caption stehen}%
\fi
\fi
\label{#1}%
}%

\fi

% Gibt Begriff des referenzierten Objekts mit aus, aber nur im HTML, daher nur in Ausnahmefaellen (z.B. Copyrightliste) sinnvoll
\def\MCRef#1{\ifttm\special{html:<!-- mmref;;}#1\special{html:;;1; //-->}\else\vref{#1}\fi}


\def\MRef#1{\ifttm\special{html:<!-- mmref;;}#1\special{html:;;0; //-->}\else\vref{#1}\fi}
\def\MERef#1{\ifttm\special{html:<!-- mmref;;}#1\special{html:;;0; //-->}\else\eqref{#1}\fi}
\def\MNRef#1{\ifttm\special{html:<!-- mmref;;}#1\special{html:;;0; //-->}\else\ref{#1}\fi}
\def\MSRef#1#2{\ifttm\special{html:<!-- msref;;}#1\special{html:;;}#2\special{html:; //-->}\else \if#2\empty \ref{#1} \else \hyperref[#1]{#2}\fi\fi} 

\def\MRefRange#1#2{\ifttm\MRef{#1} bis 
\MRef{#2}\else\vrefrange[\unskip]{#1}{#2}\fi}

\def\MRefTwo#1#2{\ifttm\MRef{#1} und \MRef{#2}\else%
\let\vRefTLRsav=\reftextlabelrange\let\vRefTPRsav=\reftextpagerange%
\def\reftextlabelrange##1##2{\ref{##1} und~\ref{##2}}%
\def\reftextpagerange##1##2{auf den Seiten~\pageref{#1} und~\pageref{#2}}%
\vrefrange[\unskip]{#1}{#2}%
\let\reftextlabelrange=\vRefTLRsav\let\reftextpagerange=\vRefTPRsav\fi}

% MSectionChapter definiert falls notwendig das Kapitel vor der section. Das ist notwendig, wenn nur ein Einzelmodul uebersetzt wird.
% MChaptersGiven ist ein Counter, der von mconvert.pl vordefiniert wird.
\ifttm
\newcommand{\MSectionChapter}{\ifnum\value{MChaptersGiven}=0{\Dchapter{Modul}}\else{}\fi}
\else
\newcommand{\MSectionChapter}{\ifnum\value{chapter}=0{\Dchapter{Modul}}\else{}\fi}
\fi


\def\MChapter#1{\ifnum\value{MSSEnd}>0{\MSubsectionEndMacros}\addtocounter{MSSEnd}{-1}\fi\Dchapter{#1}}
\def\MSubject#1{\MChapter{#1}} % Schluesselwort HELPSECTION ist reserviert fuer Hilfesektion

\newcommand{\MSectionID}{UNKNOWNID}

\ifttm
\newcommand{\MSetSectionID}[1]{\renewcommand{\MSectionID}{#1}}
\else
\newcommand{\MSetSectionID}[1]{\renewcommand{\MSectionID}{#1}\tikzsetexternalprefix{#1}}
\fi


\newcommand{\MSection}[1]{\MSetSectionID{MODULID}\ifnum\value{MSSEnd}>0{\MSubsectionEndMacros}\addtocounter{MSSEnd}{-1}\fi\MSectionChapter\Dsection{#1}\MSectionStartMacros{#1}\setcounter{MLastIndex}{-1}\setcounter{MLastType}{1}} % Sections werden ueber das section-Feld im mmlabel-Tag identifiziert, nicht ueber das Indexfeld

\def\MSubsection#1{\ifnum\value{MSSEnd}>0{\MSubsectionEndMacros}\addtocounter{MSSEnd}{-1}\fi\ifttm\else\clearpage\fi\Dsubsection{#1}\MSubsectionStartMacros\setcounter{MLastIndex}{-1}\setcounter{MLastType}{2}\addtocounter{MSSEnd}{1}}% Subsections werden ueber das subsection-Feld im mmlabel-Tag identifiziert, nicht ueber das Indexfeld
\def\MSubsectionx#1{\Dsubsectionx{#1}} % Nur zur Verwendung in MSectionStart gedacht
\def\MSubsubsection#1{\Dsubsubsection{#1}\setcounter{MLastIndex}{\value{subsubsection}}\setcounter{MLastType}{3}\ifttm\special{html:<!-- sectioninfo;;}\arabic{section}\special{html:;;}\arabic{subsection}\special{html:;;}\arabic{subsubsection}\special{html:;;1;;}\arabic{MTestSite}\special{html:; //-->}\fi}
\def\MSubsubsectionx#1{\Dsubsubsectionx{#1}\ifttm\special{html:<!-- sectioninfo;;}\arabic{section}\special{html:;;}\arabic{subsection}\special{html:;;}\arabic{subsubsection}\special{html:;;0;;}\arabic{MTestSite}\special{html:; //-->}\else\addcontentsline{toc}{subsection}{#1}\fi}

\ifttm
\def\MSubsubsubsectionx#1{\ \newline\textbf{#1}\special{html:<br />}}
\else
\def\MSubsubsubsectionx#1{\ \newline
\textbf{#1}\ \\
}
\fi


% Dieses Skript wird zu Beginn jedes Modulabschnitts (=Webseite) ausgefuehrt und initialisiert den Aufgabenfeldzaehler
\newcommand{\MPageScripts}{
\setcounter{MFieldCounter}{1}
\addtocounter{MSiteCounter}{1}
\setcounter{MHintCounter}{1}
\setcounter{MCodeEditCounter}{1}
\setcounter{MGroupActive}{0}
\DoQBoxes
% Feldvariablen werden im HTML-Header in conv.pl eingestellt
}

% Dieses Skript wird zum Ende jedes Modulabschnitts (=Webseite) ausgefuehrt
\ifttm
\newcommand{\MEndScripts}{\special{html:<br /><!-- mfeedbackbutton;Seite;}\arabic{MTestSite}\special{html:;}\MGenerateSiteNumber\special{html:; //-->}
}
\else
\newcommand{\MEndScripts}{\relax}
\fi


\newcounter{QBoxFlag}
\newcommand{\DoQBoxes}{\setcounter{QBoxFlag}{1}}
\newcommand{\NoQBoxes}{\setcounter{QBoxFlag}{0}}

\newcounter{MXCTest}
\newcounter{MXCounter}
\newcounter{MSCounter}



\ifttm

% Struktur des sectioninfo-Tags: <!-- sectioninfo;;section;;subsection;;subsubsection;;nr_ausgeben;;testpage; //-->

%Fuegt eine zusaetzliche html-Seite an hinter ALLEN bisherigen und zukuenftigen content-Seiten ausserhalb der vor-zurueck-Schleife (d.h. nur durch Button oder MIntLink erreichbar!)
% #1 = Titel des Modulabschnitts, #2 = Kurztitel fuer die Buttons, #3 = Buttonkennung (STD = default nehmen, NONE = Ohne Button in der Navigation)
\newenvironment{MSContent}[3]{\special{html:<div class="xcontent}\arabic{MSCounter}\special{html:"><!-- scontent;-;}\arabic{MSCounter};-;#1;-;#2;-;#3\special{html: //-->}\MPageScripts\MSubsubsectionx{#1}}{\MEndScripts\special{html:<!-- endscontent;;}\arabic{MSCounter}\special{html: //--></div>}\addtocounter{MSCounter}{1}}

% Fuegt eine zusaetzliche html-Seite ein hinter den bereits vorhandenen content-Seiten (oder als erste Seite) innerhalb der vor-zurueck-Schleife der Navigation
% #1 = Titel des Modulabschnitts, #2 = Kurztitel fuer die Buttons, #3 = Buttonkennung (STD = Defaultbutton, NONE = Ohne Button in der Navigation)
\newenvironment{MXContent}[3]{\special{html:<div class="xcontent}\arabic{MXCounter}\special{html:"><!-- xcontent;-;}\arabic{MXCounter};-;#1;-;#2;-;#3\special{html: //-->}\MPageScripts\MSubsubsection{#1}}{\MEndScripts\special{html:<!-- endxcontent;;}\arabic{MXCounter}\special{html: //--></div>}\addtocounter{MXCounter}{1}}

% Fuegt eine zusaetzliche html-Seite ein die keine subsubsection-Nummer bekommt, nur zur internen Verwendung in mintmod.tex gedacht!
% #1 = Titel des Modulabschnitts, #2 = Kurztitel fuer die Buttons, #3 = Buttonkennung (STD = Defaultbutton, NONE = Ohne Button in der Navigation)
% \newenvironment{MUContent}[3]{\special{html:<div class="xcontent}\arabic{MXCounter}\special{html:"><!-- xcontent;-;}\arabic{MXCounter};-;#1;-;#2;-;#3\special{html: //-->}\MPageScripts\MSubsubsectionx{#1}}{\MEndScripts\special{html:<!-- endxcontent;;}\arabic{MXCounter}\special{html: //--></div>}\addtocounter{MXCounter}{1}}

\newcommand{\MDeclareSiteUXID}[1]{\special{html:<!-- mdeclaresiteuxid;;}#1\special{html:;;}\arabic{chapter}\special{html:;;}\arabic{section}\special{html:;; //-->}}

\else

%\newcommand{\MSubsubsection}[1]{\refstepcounter{subsubsection} \addcontentsline{toc}{subsubsection}{\thesubsubsection. #1}}


% Fuegt eine zusaetzliche html-Seite an hinter den bereits vorhandenen content-Seiten
% #1 = Titel des Modulabschnitts, #2 = Kurztitel fuer die Buttons, #3 = Iconkennung (im PDF wirkungslos)
%\newenvironment{MUContent}[3]{\ifnum\value{MXCTest}>0{\MDebugMessage{ERROR: Geschachtelter SContent}}\fi\MPageScripts\MSubsubsectionx{#1}\addtocounter{MXCTest}{1}}{\addtocounter{MXCounter}{1}\addtocounter{MXCTest}{-1}}
\newenvironment{MXContent}[3]{\ifnum\value{MXCTest}>0{\MDebugMessage{ERROR: Geschachtelter SContent}}\fi\MPageScripts\MSubsubsection{#1}\addtocounter{MXCTest}{1}}{\addtocounter{MXCounter}{1}\addtocounter{MXCTest}{-1}}
\newenvironment{MSContent}[3]{\ifnum\value{MXCTest}>0{\MDebugMessage{ERROR: Geschachtelter XContent}}\fi\MPageScripts\MSubsubsectionx{#1}\addtocounter{MXCTest}{1}}{\addtocounter{MSCounter}{1}\addtocounter{MXCTest}{-1}}

\newcommand{\MDeclareSiteUXID}[1]{\relax}

\fi 

% GHEADER und GFOOTER werden von split.pm gefunden, aber nur, wenn nicht HELPSITE oder TESTSITE
\ifttm
\newenvironment{MSectionStart}{\special{html:<div class="xcontent0">}\MSubsubsectionx{Modul\"ubersicht}}{\setcounter{MSSEnd}{0}\special{html:</div>}}
% Darf nicht als XContent nummeriert werden, darf nicht als XContent gelabelt werden, wird aber in eine xcontent-div gesetzt fuer Python-parsing
\else
\newenvironment{MSectionStart}{\MSubsectionx{Modul\"ubersicht}}{\setcounter{MSSEnd}{0}}
\fi

\newenvironment{MIntro}{\begin{MXContent}{Einf\"uhrung}{Einf\"uhrung}{genetisch}}{\end{MXContent}}
\newenvironment{MContent}{\begin{MXContent}{Inhalt}{Inhalt}{beweis}}{\end{MXContent}}
\newenvironment{MExercises}{\ifttm\else\clearpage\fi\begin{MXContent}{Aufgaben}{Aufgaben}{aufgb}\special{html:<!-- declareexcsymb //-->}}{\end{MXContent}}

% #1 = Lesbare Testbezeichnung
\newenvironment{MTest}[1]{%
\renewcommand{\MTestName}{#1}
\ifttm\else\clearpage\fi%
\addtocounter{MTestSite}{1}%
\begin{MXContent}{#1}{#1}{STD} % {aufgb}%
\special{html:<!-- declaretestsymb //-->}
\begin{MQuestionGroup}%
\MInTestHeader
}%
{%
\end{MQuestionGroup}%
\ \\ \ \\%
\MInTestFooter
\end{MXContent}\addtocounter{MTestSite}{-1}%
}

\newenvironment{MExtra}{\ifttm\else\clearpage\fi\begin{MXContent}{Zus\"atzliche Inhalte}{Zusatz}{weiterfhrg}}{\end{MXContent}}

\makeindex

\ifttm
\def\MPrintIndex{
\ifnum\value{MSSEnd}>0{\MSubsectionEndMacros}\addtocounter{MSSEnd}{-1}\fi
\renewcommand{\indexname}{Stichwortverzeichnis}
\special{html:<p><!-- printindex //--></p>}
}
\else
\def\MPrintIndex{
\ifnum\value{MSSEnd}>0{\MSubsectionEndMacros}\addtocounter{MSSEnd}{-1}\fi
\renewcommand{\indexname}{Stichwortverzeichnis}
\addcontentsline{toc}{section}{Stichwortverzeichnis}
\printindex
}
\fi


% Konstanten fuer die Modulfaecher

\def\MINTMathematics{1}
\def\MINTInformatics{2}
\def\MINTChemistry{3}
\def\MINTPhysics{4}
\def\MINTEngineering{5}

\newcounter{MSubjectArea}
\newcounter{MInfoNumbers} % Gibt an, ob die Infoboxen nummeriert werden sollen
\newcounter{MSepNumbers} % Gibt an, ob Beispiele und Experimente separat nummeriert werden sollen
\newcommand{\MSetSubject}[1]{
 % ttm kapiert setcounter mit Parametern nicht, also per if abragen und einsetzen
\ifnum#1=1\setcounter{MSubjectArea}{1}\setcounter{MInfoNumbers}{1}\setcounter{MSepNumbers}{0}\fi
\ifnum#1=2\setcounter{MSubjectArea}{2}\setcounter{MInfoNumbers}{1}\setcounter{MSepNumbers}{0}\fi
\ifnum#1=3\setcounter{MSubjectArea}{3}\setcounter{MInfoNumbers}{0}\setcounter{MSepNumbers}{1}\fi
\ifnum#1=4\setcounter{MSubjectArea}{4}\setcounter{MInfoNumbers}{0}\setcounter{MSepNumbers}{0}\fi
\ifnum#1=5\setcounter{MSubjectArea}{5}\setcounter{MInfoNumbers}{1}\setcounter{MSepNumbers}{0}\fi
% Separate Nummerntechnik fuer unsere Chemiker: alles dreistellig
\ifnum#1=3
  \ifttm
  \renewcommand{\theequation}{\arabic{section}.\arabic{subsection}.\arabic{equation}}
  \renewcommand{\thetable}{\arabic{section}.\arabic{subsection}.\arabic{table}} 
  \renewcommand{\thefigure}{\arabic{section}.\arabic{subsection}.\arabic{figure}} 
  \else
  \renewcommand{\theequation}{\arabic{chapter}.\arabic{section}.\arabic{equation}}
  \renewcommand{\thetable}{\arabic{chapter}.\arabic{section}.\arabic{table}}
  \renewcommand{\thefigure}{\arabic{chapter}.\arabic{section}.\arabic{figure}}
  \fi
\else
  \ifttm
  \renewcommand{\theequation}{\arabic{section}.\arabic{subsection}.\arabic{equation}}
  \renewcommand{\thetable}{\arabic{table}}
  \renewcommand{\thefigure}{\arabic{figure}}
  \else
  \renewcommand{\theequation}{\arabic{chapter}.\arabic{section}.\arabic{equation}}
  \renewcommand{\thetable}{\arabic{table}}
  \renewcommand{\thefigure}{\arabic{figure}}
  \fi
\fi
}

% Fuer tikz Autogenerierung
\newcounter{MTIKZAutofilenumber}

% Spezielle Counter fuer die Bentz-Module
\newcounter{mycounter}
\newcounter{chemapplet}
\newcounter{physapplet}

\newcounter{MSSEnd} % Ist 1 falls ein MSubsection aktiv ist, der einen MSubsectionEndMacro-Aufruf verursacht
\newcounter{MFileNumber}
\def\MLastFile{\special{html:[[!-- mfileref;;}\arabic{MFileNumber}\special{html:; //--]]}}

% Vollstaendiger Pfad ist \MMaterial / \MLastFilePath / \MLastFileName    ==   \MMaterial / \MLastFile

% Wird nur bei kompletter Baum-Erstellung ausgefuehrt!
% #1 = Lesbare Modulbezeichnung
\newcommand{\MSectionStartMacros}[1]{
\setcounter{MTestSite}{0}
\setcounter{MCaptionOn}{0}
\setcounter{MLastTypeEq}{0}
\setcounter{MSSEnd}{0}
\setcounter{MFileNumber}{0} % Preinkrekement-Counter
\setcounter{MTIKZAutofilenumber}{0}
\setcounter{mycounter}{1}
\setcounter{physapplet}{1}
\setcounter{chemapplet}{0}
\ifttm
\special{html:<!-- mdeclaresection;;}\arabic{chapter}\special{html:;;}\arabic{section}\special{html:;;}#1\special{html:;; //-->}%
\else
\setcounter{thmc}{0}
\setcounter{exmpc}{0}
\setcounter{verc}{0}
\setcounter{infoc}{0}
\fi
\setcounter{MiniMarkerCounter}{1}
\setcounter{AlignCounter}{1}
\setcounter{MXCTest}{0}
\setcounter{MCodeCounter}{0}
\setcounter{MEntryCounter}{0}
}

% Wird immer ausgefuehrt
\newcommand{\MSubsectionStartMacros}{
\ifttm\else\MPageHeaderDef\fi
\MWatermarkSettings
\setcounter{MXCounter}{0}
\setcounter{MSCounter}{0}
\setcounter{MSiteCounter}{1}
\setcounter{MExerciseCollectionCounter}{0}
% Zaehler fuer das Labelsystem zuruecksetzen (prefix-Zaehler)
\setcounter{MInfoCounter}{0}
\setcounter{MExerciseCounter}{0}
\setcounter{MExampleCounter}{0}
\setcounter{MExperimentCounter}{0}
\setcounter{MGraphicsCounter}{0}
\setcounter{MTableCounter}{0}
\setcounter{MTheoremCounter}{0}
\setcounter{MObjectCounter}{0}
\setcounter{MEquationCounter}{0}
\setcounter{MVideoCounter}{0}
\setcounter{equation}{0}
\setcounter{figure}{0}
}

\newcommand{\MSubsectionEndMacros}{
% Bei Chemiemodulen das PSE einhaengen, es soll als SContent am Ende erscheinen
\special{html:<!-- subsectionend //-->}
\ifnum\value{MSubjectArea}=3{\MIncludePSE}\fi
}


\ifttm
%\newcommand{\MEmbed}[1]{\MRegisterFile{#1}\begin{html}<embed src="\end{html}\MMaterial/\MLastFile\begin{html}" width="192" height="189"></embed>\end{html}}
\newcommand{\MEmbed}[1]{\MRegisterFile{#1}\begin{html}<embed src="\end{html}\MMaterial/\MLastFile\begin{html}"></embed>\end{html}}
\fi

%----------------- Makros fuer die Textdarstellung -----------------------------------------------

\ifttm
% MUGraphics bindet eine Grafik ein:
% Parameter 1: Dateiname der Grafik, relativ zur Position des Modul-Tex-Dokuments
% Parameter 2: Skalierungsoptionen fuer PDF (fuer includegraphics)
% Parameter 3: Titel fuer die Grafik, wird unter die Grafik mit der Grafiknummer gesetzt und kann MLabel bzw. MCopyrightLabel enthalten
% Parameter 4: Skalierungsoptionen fuer HTML (css-styles)

% ERSATZ: <img alt="My Image" src="data:image/png;base64,iVBORwA<MoreBase64SringHere>" />


\newcommand{\MUGraphics}[4]{\MRegisterFile{#1}\begin{html}
<div class="imagecenter">
<center>
<div>
<img src="\end{html}\MMaterial/\MLastFile\begin{html}" style="#4" alt="\end{html}\MMaterial/\MLastFile\begin{html}"/>
</div>
<div class="bildtext">
\end{html}
\addtocounter{MGraphicsCounter}{1}
\setcounter{MLastIndex}{\value{MGraphicsCounter}}
\setcounter{MLastType}{8}
\addtocounter{MCaptionOn}{1}
\ifnum\value{MSepNumbers}=0
\textbf{Abbildung \arabic{MGraphicsCounter}:} #3
\else
\textbf{Abbildung \arabic{section}.\arabic{subsection}.\arabic{MGraphicsCounter}:} #3
\fi
\addtocounter{MCaptionOn}{-1}
\begin{html}
</div>
</center>
</div>
<br />
\end{html}%
\special{html:<!-- mfeedbackbutton;Abbildung;}\arabic{MGraphicsCounter}\special{html:;}\arabic{section}.\arabic{subsection}.\arabic{MGraphicsCounter}\special{html:; //-->}%
}

% MVideo bindet ein Video als Einzeldatei ein:
% Parameter 1: Dateiname des Videos, relativ zur Position des Modul-Tex-Dokuments, ohne die Endung ".mp4"
% Parameter 2: Titel fuer das Video (kann MLabel oder MCopyrightLabel enthalten), wird unter das Video mit der Videonummer gesetzt
\newcommand{\MVideo}[2]{\MRegisterFile{#1.mp4}\begin{html}
<div class="imagecenter">
<center>
<div>
<video width="95\%" controls="controls"><source src="\end{html}\MMaterial/#1.mp4\begin{html}" type="video/mp4">Ihr Browser kann keine MP4-Videos abspielen!</video>
</div>
<div class="bildtext">
\end{html}
\addtocounter{MVideoCounter}{1}
\setcounter{MLastIndex}{\value{MVideoCounter}}
\setcounter{MLastType}{12}
\addtocounter{MCaptionOn}{1}
\ifnum\value{MSepNumbers}=0
\textbf{Video \arabic{MVideoCounter}:} #2
\else
\textbf{Video \arabic{section}.\arabic{subsection}.\arabic{MVideoCounter}:} #2
\fi
\addtocounter{MCaptionOn}{-1}
\begin{html}
</div>
</center>
</div>
<br />
\end{html}}

\newcommand{\MDVideo}[2]{\MRegisterFile{#1.mp4}\MRegisterFile{#1.ogv}\begin{html}
<div class="imagecenter">
<center>
<div>
<video width="70\%" controls><source src="\end{html}\MMaterial/#1.mp4\begin{html}" type="video/mp4"><source src="\end{html}\MMaterial/#1.ogv\begin{html}" type="video/ogg">Ihr Browser kann keine MP4-Videos abspielen!</video>
</div>
<br />
#2
</center>
</div>
<br />
\end{html}
}

\newcommand{\MGraphics}[3]{\MUGraphics{#1}{#2}{#3}{}}

\else

\newcommand{\MVideo}[2]{%
% Kein Video im PDF darstellbar, trotzdem so tun als ob da eines waere
\begin{center}
(Video nicht darstellbar)
\end{center}
\addtocounter{MVideoCounter}{1}
\setcounter{MLastIndex}{\value{MVideoCounter}}
\setcounter{MLastType}{12}
\addtocounter{MCaptionOn}{1}
\ifnum\value{MSepNumbers}=0
\textbf{Video \arabic{MVideoCounter}:} #2
\else
\textbf{Video \arabic{section}.\arabic{subsection}.\arabic{MVideoCounter}:} #2
\fi
\addtocounter{MCaptionOn}{-1}
}


% MGraphics bindet eine Grafik ein:
% Parameter 1: Dateiname der Grafik, relativ zur Position des Modul-Tex-Dokuments
% Parameter 2: Skalierungsoptionen fuer PDF (fuer includegraphics)
% Parameter 3: Titel fuer die Grafik, wird unter die Grafik mit der Grafiknummer gesetzt
\newcommand{\MGraphics}[3]{%
\MRegisterFile{#1}%
\ %
\begin{figure}[H]%
\centering{%
\includegraphics[#2]{\MDPrefix/#1}%
\addtocounter{MCaptionOn}{1}%
\caption{#3}%
\addtocounter{MCaptionOn}{-1}%
}%
\end{figure}%
\addtocounter{MGraphicsCounter}{1}\setcounter{MLastIndex}{\value{MGraphicsCounter}}\setcounter{MLastType}{8}\ %
%\ \\Abbildung \ifnum\value{MSepNumbers}=0\else\arabic{chapter}.\arabic{section}.\fi\arabic{MGraphicsCounter}: #3%
}

\newcommand{\MUGraphics}[4]{\MGraphics{#1}{#2}{#3}}


\fi

\newcounter{MCaptionOn} % = 1 falls eine Grafikcaption aktiv ist, = 0 sonst


% MGraphicsSolo bindet eine Grafik pur ein ohne Titel
% Parameter 1: Dateiname der Grafik, relativ zur Position des Modul-Tex-Dokuments
% Parameter 2: Skalierungsoptionen (wirken nur im PDF)
\newcommand{\MGraphicsSolo}[2]{\MUGraphicsSolo{#1}{#2}{}}

% MUGraphicsSolo bindet eine Grafik pur ein ohne Titel, aber mit HTML-Skalierung
% Parameter 1: Dateiname der Grafik, relativ zur Position des Modul-Tex-Dokuments
% Parameter 2: Skalierungsoptionen (wirken nur im PDF)
% Parameter 3: Skalierungsoptionen (wirken nur im HTML), als style-format: "width=???, height=???"
\ifttm
\newcommand{\MUGraphicsSolo}[3]{\MRegisterFile{#1}\begin{html}
<img src="\end{html}\MMaterial/\MLastFile\begin{html}" style="\end{html}#3\begin{html}" alt="\end{html}\MMaterial/\MLastFile\begin{html}"/>
\end{html}%
\special{html:<!-- mfeedbackbutton;Abbildung;}#1\special{html:;}\MMaterial/\MLastFile\special{html:; //-->}%
}
\else
\newcommand{\MUGraphicsSolo}[3]{\MRegisterFile{#1}\includegraphics[#2]{\MDPrefix/#1}}
\fi

% Externer Link mit URL
% Erster Parameter: Vollstaendige(!) URL des Links
% Zweiter Parameter: Text fuer den Link
\newcommand{\MExtLink}[2]{\ifttm\special{html:<a target="_new" href="}#1\special{html:">}#2\special{html:</a>}\else\href{#1}{#2}\fi} % ohne MINTERLINK!


% Interner Link, die verlinkte Datei muss im gleichen Verzeichnis liegen wie die Modul-Texdatei
% Erster Parameter: Dateiname
% Zweiter Parameter: Text fuer den Link
\newcommand{\MIntLink}[2]{\ifttm\MRegisterFile{#1}\special{html:<a class="MINTERLINK" target="_new" href="}\MMaterial/\MLastFile\special{html:">}#2\special{html:</a>}\else{\href{#1}{#2}}\fi}


\ifttm
\def\MMaterial{:localmaterial:}
\else
\def\MMaterial{\MDPrefix}
\fi

\ifttm
\def\MNoFile#1{:directmaterial:#1}
\else
\def\MNoFile#1{#1}
\fi

\newcommand{\MChem}[1]{$\mathrm{#1}$}

\newcommand{\MApplet}[3]{
% Bindet ein Java-Applet ein, die Parameter sind:
% (wird nur im HTML, aber nicht im PDF erstellt)
% #1 Dateiname des Applets (muss mit ".class" enden)
% #2 = Breite in Pixeln
% #3 = Hoehe in Pixeln
\ifttm
\MRegisterFile{#1}
\begin{html}
<applet code="\end{html}\MMaterial/\MLastFile\begin{html}" width="#2" height="#3" alt="[Java-Applet kann nicht gestartet werden]"></applet>
\end{html}
\fi
}

\newcommand{\MScriptPage}[2]{
% Bindet eine JavaScript-Datei ein, die eine eigene Seite bekommt
% (wird nur im HTML, aber nicht im PDF erstellt)
% #1 Dateiname des Programms (sollte mit ".js" enden)
% #2 = Kurztitel der Seite
\ifttm
\begin{MSContent}{#2}{#2}{puzzle}
\MRegisterFile{#1}
\begin{html}
<script src="\MMaterial/\MLastFile" type="text/javascript"></script>
\end{html}
\end{MSContent}
\fi
}

\newcommand{\MIncludePSE}{
% Bindet bei Chemie-Modulen das PSE ein
% (wird nur im HTML, aber nicht im PDF erstellt)
\ifttm
\special{html:<!-- includepse //-->}
\begin{MSContent}{Periodensystem der Elemente}{PSE}{table}
\MRegisterFile{../files/pse.js}
\MRegisterFile{../files/radio.png}
\begin{html}
<script src="\MMaterial/../files/pse.js" type="text/javascript"></script>
<p id="divid"><br /><br />
<script language="javascript" type="text/javascript">
    startpse("divid","\MMaterial/../files"); 
</script>
</p>
<br />
<br />
<br />
<p>Die Farben der Elementsymbole geben an: <font style="color:Red">gasf&ouml;rmig </font> <font style="color:Blue">fl&uuml;ssig </font> fest</p>
<p>Die Elemente der Gruppe 1 A, 2 A, 3 A usw. geh&ouml;ren zu den Hauptgruppenelementen.</p>
<p>Die Elemente der Gruppe 1 B, 2 B, 3 B usw. geh&ouml;ren zu den Nebengruppenelementen.</p>
<p>() kennzeichnet die Masse des stabilsten Isotops</p>
\end{html}
\end{MSContent}
\fi
}

\newcommand{\MAppletArchive}[4]{
% Bindet ein Java-Applet ein, die Parameter sind:
% (wird nur im HTML, aber nicht im PDF erstellt)
% #1 Dateiname der Klasse mit Appletaufruf (muss mit ".class" enden)
% #2 Dateiname des Archivs (muss mit ".jar" enden)
% #3 = Breite in Pixeln
% #4 = Hoehe in Pixeln
\ifttm
\MRegisterFile{#2}
\begin{html}
<applet code="#1" archive="\end{html}\MMaterial/\MLastFile\begin{html}" codebase="." width="#3" height="#4" alt="[Java-Archiv kann nicht gestartet werden]"></applet>
\end{html}
\fi
}

% Bindet in der Haupttexdatei ein MINT-Modul ein. Parameter 1 ist das Verzeichnis (relativ zur Haupttexdatei), Parameter 2 ist der Dateinahme ohne Pfad.
\newcommand{\IncludeModule}[2]{
\renewcommand{\MDPrefix}{#1}
\input{#1/#2}
\ifnum\value{MSSEnd}>0{\MSubsectionEndMacros}\addtocounter{MSSEnd}{-1}\fi
}

% Der ttm-Konverter setzt keine Makros im \input um, also muss hier getrickst werden:
% Das MDPrefix muss in den einzelnen Modulen manuell eingesetzt werden
\newcommand{\MInputFile}[1]{
\ifttm
\input{#1}
\else
\input{#1}
\fi
}


\newcommand{\MCases}[1]{\left\lbrace{\begin{array}{rl} #1 \end{array}}\right.}

\ifttm
\newenvironment{MCaseEnv}{\left\lbrace\begin{array}{rl}}{\end{array}\right.}
\else
\newenvironment{MCaseEnv}{\left\lbrace\begin{array}{rl}}{\end{array}\right.}
\fi

\def\MSkip{\ifttm\MCR\fi}

\ifttm
\def\MCR{\special{html:<br />}}
\else
\def\MCR{\ \\}
\fi


% Pragmas - Sind Schluesselwoerter, die dem Preprocessing sowie dem Konverter uebergeben werden und bestimmte
%           Aktionen ausloesen. Im Output (PDF und HTML) tauchen sie nicht auf.
\newcommand{\MPragma}[1]{%
\ifttm%
\special{html:<!-- mpragma;-;}#1\special{html:;; -->}%
\else%
% MPragmas werden vom Preprozessor direkt im LaTeX gefunden
\fi%
}

% Ersatz der Befehle textsubscript und textsuperscript, die ttm nicht kennt
\ifttm%
\newcommand{\MTextsubscript}[1]{\special{html:<sub>}#1\special{html:</sub>}}%
\newcommand{\MTextsuperscript}[1]{\special{html:<sup>}#1\special{html:</sup>}}%
\else%
\newcommand{\MTextsubscript}[1]{\textsubscript{#1}}%
\newcommand{\MTextsuperscript}[1]{\textsuperscript{#1}}%
\fi

%------------------ Einbindung von dia-Diagrammen ----------------------------------------------
% Beim preprocessing wird aus jeder dia-Datei eine tex-Datei und eine pdf-Datei erzeugt,
% diese werden hier jeweils im PDF und HTML eingebunden
% Parameter: Dateiname der mit dia erstellten Datei (OHNE die Endung .dia)
\ifttm%
\newcommand{\MDia}[1]{%
\MGraphicsSolo{#1minthtml.png}{}%
}
\else%
\newcommand{\MDia}[1]{%
\MGraphicsSolo{#1mintpdf.png}{scale=0.1667}%
}
\fi%

% subsup funktioniert im Ausdruck $D={\R}^+_0$, also \R geklammert und sup zuerst
% \ifttm
% \def\MSubsup#1#2#3{\special{html:<msubsup>} #1 #2 #3\special{html:</msubsup>}}
% \else
% \def\MSubsup#1#2#3{{#1}^{#3}_{#2}}
% \fi

%\input{local.tex}

% \ifttm
% \else
% \newwrite\mintlog
% \immediate\openout\mintlog=mintlog.txt
% \fi

% ----------------------- tikz autogenerator -------------------------------------------------------------------

\newcommand{\Mtikzexternalize}{\tikzexternalize}% wird bei Konvertierung ueber mconvert ggf. ausgehebelt!

\ifttm
\else
\tikzset%
{
  % Defines a custom style which generates pdf and converts to (low and hi-res quality) png and svg, then deletes the pdf
  % Important: DO NOT directly convert from pdf to hires-png or from svg to png with GraphViz convert as it has some problems and memory leaks
  png export/.style=%
  {
    external/system call/.add={}{; 
      pdf2svg "\image.pdf" "\image.svg" ; 
      convert -density 112.5 -transparent white "\image.pdf" "\image.png"; 
      inkscape --export-png="\image.4x.png" --export-dpi=450 --export-background-opacity=0 --without-gui "\image.svg"; 
      rm "\image.pdf"; rm "\image.log"; rm "\image.dpth"; rm "\image.idx"
    },
    external/force remake,
  }
}
\tikzset{png export}
\tikzsetexternalprefix{}
% PNGs bei externer Erzeugung in "richtiger" Groesse einbinden
\pgfkeys{/pgf/images/include external/.code={\includegraphics[scale=0.64]{#1}}}
\fi

% Spezielle Umgebung fuer Autogenerierung, Bildernamen sind nur innerhalb eines Moduls (einer MSection) eindeutig)

\newcommand{\MTIKZautofilename}{tikzautofile}

\ifttm
% HTML-Version: Vom Autogenerator erzeugte png-Datei einbinden, tikz selbst nicht ausfuehren (sprich: #1 schlucken)
\newcommand{\MTikzAuto}[1]{%
\addtocounter{MTIKZAutofilenumber}{1}
\renewcommand{\MTIKZautofilename}{mtikzauto_\arabic{MTIKZAutofilenumber}}
\MUGraphicsSolo{\MSectionID\MTIKZautofilename.4x.png}{scale=1}{\special{html:[[!-- svgstyle;}\MSectionID\MTIKZautofilename\special{html: //--]]}} % Styleinfos werden aus original-png, nicht 4x-png geholt!
%\MRegisterFile{\MSectionID\MTIKZautofilename.png} % not used right now
%\MRegisterFile{\MSectionID\MTIKZautofilename.svg}
}
\else%
% PDF-Version: Falls Autogenerator aktiv wird Datei automatisch benannt und exportiert
\newcommand{\MTikzAuto}[1]{%
\addtocounter{MTIKZAutofilenumber}{1}%
\renewcommand{\MTIKZautofilename}{mtikzauto_\arabic{MTIKZAutofilenumber}}
\tikzsetnextfilename{\MTIKZautofilename}%
#1%
}
\fi

% In einer reinen LaTeX-Uebersetzung kapselt der Preambelinclude-Befehl nur input,
% in einer konvertergesteuerten PDF/HTML-Uebersetzung wird er dagegen entfernt und
% die Preambeln an mintmod angehaengt, die Ersetzung wird von mconvert.pl vorgenommen.

\newcommand{\MPreambleInclude}[1]{\input{#1}}

% Globale Watermarksettings (werden auch nochmal zu Beginn jedes subsection gesetzt,
% muessen hier aber auch global ausgefuehrt wegen Einfuehrungsseiten und Inhaltsverzeichnis

\MWatermarkSettings
% ---------------------------------- Parametrisierte Aufgaben ----------------------------------------

\ifttm
\newenvironment{MPExercise}{%
\begin{MExercise}%
}{%
\special{html:<button name="Name_MPEX}\arabic{MExerciseCounter}\special{html:" id="MPEX}\arabic{MExerciseCounter}%
\special{html:" type="button" onclick="reroll('}\arabic{MExerciseCounter}\special{html:');">Neue Aufgabe erzeugen</button>}%
\end{MExercise}%
}
\else
\newenvironment{MPExercise}{%
\begin{MExercise}%
}{%
\end{MExercise}%
}
\fi

% Parameter: Name, Min, Max, PDF-Standard. Name in Deklaration OHNE backslash, im Code MIT Backslash
\ifttm
\newcommand{\MGlobalInteger}[4]{\special{html:%
<!-- onloadstart //-->%
MVAR.push(createGlobalInteger("}#1\special{html:",}#2\special{html:,}#3\special{html:,}#4\special{html:)); %
<!-- onloadstop //-->%
<!-- viewmodelstart //-->%
ob}#1\special{html:: ko.observable(rerollMVar("}#1\special{html:")),%
<!-- viewmodelstop //-->%
}%
}%
\else%
\newcommand{\MGlobalInteger}[4]{\newcounter{mvc_#1}\setcounter{mvc_#1}{#4}}
\fi

% Parameter: Name, Min, Max, PDF-Standard. Name in Deklaration OHNE backslash, im Code MIT Backslash, Wert ist Wurzel von value
\ifttm
\newcommand{\MGlobalSqrt}[4]{\special{html:%
<!-- onloadstart //-->%
MVAR.push(createGlobalSqrt("}#1\special{html:",}#2\special{html:,}#3\special{html:,}#4\special{html:)); %
<!-- onloadstop //-->%
<!-- viewmodelstart //-->%
ob}#1\special{html:: ko.observable(rerollMVar("}#1\special{html:")),%
<!-- viewmodelstop //-->%
}%
}%
\else%
\newcommand{\MGlobalSqrt}[4]{\newcounter{mvc_#1}\setcounter{mvc_#1}{#4}}% Funktioniert nicht als Wurzel !!!
\fi

% Parameter: Name, Min, Max, PDF-Standard zaehler, PDF-Standard nenner. Name in Deklaration OHNE backslash, im Code MIT Backslash
\ifttm
\newcommand{\MGlobalFraction}[5]{\special{html:%
<!-- onloadstart //-->%
MVAR.push(createGlobalFraction("}#1\special{html:",}#2\special{html:,}#3\special{html:,}#4\special{html:,}#5\special{html:)); %
<!-- onloadstop //-->%
<!-- viewmodelstart //-->%
ob}#1\special{html:: ko.observable(rerollMVar("}#1\special{html:")),%
<!-- viewmodelstop //-->%
}%
}%
\else%
\newcommand{\MGlobalFraction}[5]{\newcounter{mvc_#1}\setcounter{mvc_#1}{#4}} % Funktioniert nicht als Bruch !!!
\fi

% MVar darf im HTML nur in MEvalMathDisplay-Umgebungen genutzt werden oder in Strings die an den Parser uebergeben werden
\ifttm%
\newcommand{\MVar}[1]{\special{html:[var_}#1\special{html:]}}%
\else%
\newcommand{\MVar}[1]{\arabic{mvc_#1}}%
\fi

\ifttm%
\newcommand{\MRerollButton}[2]{\special{html:<button type="button" onclick="rerollMVar('}#1\special{html:');">}#2\special{html:</button>}}%
\else%
\newcommand{\MRerollButton}[2]{\relax}% Keine sinnvolle Entsprechung im PDF
\fi

% MEvalMathDisplay fuer HTML wird in mconvert.pl im preprocessing realisiert
% PDF: eine equation*-Umgebung (ueber amsmath)
% HTML: Eine Mathjax-Tex-Umgebung, deren Auswertung mit knockout-obervablen gekoppelt ist
% PDF-Version hier nur fuer pdflatex-only-Uebersetzung gegeben

\ifttm\else\newenvironment{MEvalMathDisplay}{\begin{equation*}}{\end{equation*}}\fi

% ---------------------------------- Spezialbefehle fuer AD ------------------------------------------

%Abk�rzung f�r \longrightarrow:
\newcommand{\lto}{\ensuremath{\longrightarrow}}

%Makro f�r Funktionen:
\newcommand{\exfunction}[5]
{\begin{array}{rrcl}
 #1 \colon  & #2 &\lto & #3 \\[.05cm]  
  & #4 &\longmapsto  & #5 
\end{array}}

\newcommand{\function}[5]{%
#1:\;\left\lbrace{\begin{array}{rcl}
 #2 &\lto & #3 \\
 #4 &\longmapsto  & #5 \end{array}}\right.}


%Die Identit�t:
\DeclareMathOperator{\Id}{Id}

%Die Signumfunktion:
\DeclareMathOperator{\sgn}{sgn}

%Zwei Betonungskommandos (k�nnen angepasst werden):
\newcommand{\highlight}[1]{#1}
\newcommand{\modstextbf}[1]{#1}
\newcommand{\modsemph}[1]{#1}


% ---------------------------------- Spezialbefehle fuer JL ------------------------------------------


\def\jccolorfkt{green!50!black} %Farbe des Funktionsgraphen
\def\jccolorfktarea{green!25!white} %Farbe der Fl"ache unter dem Graphen
\def\jccolorfktareahell{green!12!white} %helle Einf"arbung der Fl"ache unter dem Graphen
\def\jccolorfktwert{green!50!black} %Farbe einzelner Punkte des Graphen

\newcommand{\MPfadBilder}{Bilder}

\ifttm%
\newcommand{\jMD}{\,\MD}%
\else%
\newcommand{\jMD}{\;\MD}%
\fi%

\def\jHTMLHinweisBedienung{\MInputHint{%
Mit Hilfe der Symbole am oberen Rand des Fensters
k"onnen Sie durch die einzelnen Abschnitte navigieren.}}

\def\jHTMLHinweisEingabeText{\MInputHint{%
Geben Sie jeweils ein Wort oder Zeichen als Antwort ein.}}

\def\jHTMLHinweisEingabeTerm{\MInputHint{%
Klammern Sie Ihre Terme, um eine eindeutige Eingabe zu erhalten. 
Beispiel: Der Term $\frac{3x+1}{x-2}$ soll in der Form
\texttt{(3*x+1)/((x+2)^2}$ eingegeben werden (wobei auch Leerzeichen 
eingegeben werden k"onnen, damit eine Formel besser lesbar ist).}}

\def\jHTMLHinweisEingabeIntervalle{\MInputHint{%
Intervalle werden links mit einer "offnenden Klammer und rechts mit einer 
schlie"senden Klammer angegeben. Eine runde Klammer wird verwendet, wenn der 
Rand nicht dazu geh"ort, eine eckige, wenn er dazu geh"ort. 
Als Trennzeichen wird ein Komma oder ein Semikolon akzeptiert.
Beispiele: $(a, b)$ offenes Intervall,
$[a; b)$ links abgeschlossenes, rechts offenes Intervall von $a$ bis $b$. 
Die Eingabe $]a;b[$ f"ur ein offenes Intervall wird nicht akzeptiert.
F"ur $\infty$ kann \texttt{infty} oder \texttt{unendlich} geschrieben werden.}}

\def\jHTMLHinweisEingabeFunktionen{\MInputHint{%
Schreiben Sie Malpunkte (geschrieben als \texttt{*}) aus und setzen Sie Klammern um Argumente f�r Funktionen.
Beispiele: Polynom: \texttt{3*x + 0.1}, Sinusfunktion: \texttt{sin(x)}, 
Verkettung von cos und Wurzel: \texttt{cos(sqrt(3*x))}.}}

\def\jHTMLHinweisEingabeFunktionenSinCos{\MInputHint{%
Die Sinusfunktion $\sin x$ wird in der Form \texttt{sin(x)} angegeben, %
$\cos\left(\sqrt{3 x}\right)$ durch \texttt{cos(sqrt(3*x))}.}}

\def\jHTMLHinweisEingabeFunktionenExp{\MInputHint{%
Die Exponentialfunktion $\MEU^{3x^4 + 5}$ wird als
\texttt{exp(3 * x^4 + 5)} angegeben, %
$\ln\left(\sqrt{x} + 3.2\right)$ durch \texttt{ln(sqrt(x) + 3.2)}.}}

% ---------------------------------- Spezialbefehle fuer Fachbereich Physik --------------------------

\newcommand{\E}{{e}}
\newcommand{\ME}[1]{\cdot 10^{#1}}
\newcommand{\MU}[1]{\;\mathrm{#1}}
\newcommand{\MPG}[3]{%
  \ifnum#2=0%
    #1\ \mathrm{#3}%
  \else%
    #1\cdot 10^{#2}\ \mathrm{#3}%
  \fi}%
%

\newcommand{\MMul}{\MExponentensymbXYZl} % Nur eine Abkuerzung


% ---------------------------------- Stichwortfunktionialitaet ---------------------------------------

% mpreindexentry wird durch Auswahlroutine in conv.pl durch mindexentry substitutiert
\ifttm%
\def\MIndex#1{\index{#1}\special{html:<!-- mpreindexentry;;}#1\special{html:;;}\arabic{MSubjectArea}\special{html:;;}%
\arabic{chapter}\special{html:;;}\arabic{section}\special{html:;;}\arabic{subsection}\special{html:;;}\arabic{MEntryCounter}\special{html:; //-->}%
\setcounter{MLastIndex}{\value{MEntryCounter}}%
\addtocounter{MEntryCounter}{1}%
}%
% Copyrightliste wird als tex-Datei im preprocessing von conv.pl erzeugt und unter converter/tex/entrycollection.tex abgelegt
% Der input-Befehl funktioniert nur, wenn die aufrufende tex-Datei auf der obersten Ebene liegt (d.h. selbst kein input/include ist, insbesondere keine Moduldatei)
\def\MEntryList{} % \input funktioniert nicht, weil ttm (und damit das \input) ausgefuehrt wird, bevor Datei da ist
\else%
\def\MIndex#1{\index{#1}}
\def\MEntryList{\MAbort{Stichwortliste nur im HTML realisierbar}}%
\fi%

\def\MEntry#1#2{\textbf{#1}\MIndex{#2}} % Idee: MLastType auf neuen Entry-Typ und dann ein MLabel vergeben mit autogen-Nummer

% ---------------------------------- Befehle fuer Tests ----------------------------------------------

% MEquationItem stellt eine Eingabezeile der Form Vorgabe = Antwortfeld her, der zweite Parameter kann z.B. MSimplifyQuestion-Befehl sein
\ifttm
\newcommand{\MEquationItem}[2]{{#1}$\,=\,${#2}}%
\else%
\newcommand{\MEquationItem}[2]{{#1}$\;\;=\,${#2}}%
\fi

\ifttm
\newcommand{\MInputHint}[1]{%
\ifnum%
\if\value{MTestSite}>0%
\else%
{\color{blue}#1}%
\fi%
\fi%
}
\else
\newcommand{\MInputHint}[1]{\relax}
\fi

\ifttm
\newcommand{\MInTestHeader}{%
Dies ist ein einreichbarer Test:
\begin{itemize}
\item{Im Gegensatz zu den offenen Aufgaben werden beim Eingeben keine Hinweise zur Formulierung der mathematischen Ausdr�cke gegeben.}
\item{Der Test kann jederzeit neu gestartet oder verlassen werden.}
\item{Der Test kann durch die Buttons am Ende der Seite beendet und abgeschickt, oder zur�ckgesetzt werden.}
\item{Der Test kann mehrfach probiert werden. F�r die Statistik z�hlt die zuletzt abgeschickte Version.}
\end{itemize}
}
\else
\newcommand{\MInTestHeader}{%
\relax
}
\fi

\ifttm
\newcommand{\MInTestFooter}{%
\special{html:<button name="Name_TESTFINISH" id="TESTFINISH" type="button" onclick="finish_button('}\MTestName\special{html:');">Test auswerten</button>}%
\begin{html}
&nbsp;&nbsp;&nbsp;&nbsp;&nbsp;&nbsp;&nbsp;&nbsp;
<button name="Name_TESTRESET" id="TESTRESET" type="button" onclick="reset_button();">Test zur�cksetzen</button>
<br />
<br />
<div class="xreply">
<p name="Name_TESTEVAL" id="TESTEVAL">
Hier erscheint die Testauswertung!
<br />
</p>
</div>
\end{html}
}
\else
\newcommand{\MInTestFooter}{%
\relax
}
\fi


% ---------------------------------- Notationsmakros -------------------------------------------------------------

% Notationsmakros die nicht von der Kursvariante abhaengig sind

\newcommand{\MZahltrennzeichen}[1]{\renewcommand{\MZXYZhltrennzeichen}{#1}}

\ifttm
\newcommand{\MZahl}[3][\MZXYZhltrennzeichen]{\edef\MZXYZtemp{\noexpand\special{html:<mn>#2#1#3</mn>}}\MZXYZtemp}
\else
\newcommand{\MZahl}[3][\MZXYZhltrennzeichen]{{}#2{#1}#3}
\fi

\newcommand{\MEinheitenabstand}[1]{\renewcommand{\MEinheitenabstXYZnd}{#1}}
\ifttm
\newcommand{\MEinheit}[2][\MEinheitenabstXYZnd]{{}#1\edef\MEINHtemp{\noexpand\special{html:<mi mathvariant="normal">#2</mi>}}\MEINHtemp} 
\else
\newcommand{\MEinheit}[2][\MEinheitenabstXYZnd]{{}#1 \mathrm{#2}} 
\fi

\newcommand{\MExponentensymbol}[1]{\renewcommand{\MExponentensymbXYZl}{#1}}
\newcommand{\MExponent}[2][\MExponentensymbXYZl]{{}#1{} 10^{#2}} 

%Punkte in 2 und 3 Dimensionen
\newcommand{\MPointTwo}[3][]{#1(#2\MCoordPointSep #3{}#1)}
\newcommand{\MPointThree}[4][]{#1(#2\MCoordPointSep #3\MCoordPointSep #4{}#1)}
\newcommand{\MPointTwoAS}[2]{\left(#1\MCoordPointSep #2\right)}
\newcommand{\MPointThreeAS}[3]{\left(#1\MCoordPointSep #2\MCoordPointSep #3\right)}

% Masseinheit, Standardabstand: \,
\newcommand{\MEinheitenabstXYZnd}{\MThinspace} 

% Horizontaler Leerraum zwischen herausgestellter Formel und Interpunktion
\ifttm
\newcommand{\MDFPSpace}{\,}
\newcommand{\MDFPaSpace}{\,\,}
\newcommand{\MBlank}{\ }
\else
\newcommand{\MDFPSpace}{\;}
\newcommand{\MDFPaSpace}{\;\;}
\newcommand{\MBlank}{\ }
\fi

% Satzende in herausgestellter Formel mit horizontalem Leerraum
\newcommand{\MDFPeriod}{\MDFPSpace .}

% Separation von Aufzaehlung und Bedingung in Menge
\newcommand{\MCondSetSep}{\,:\,} %oder '\mid'

% Konverter kennt mathopen nicht
\ifttm
\def\mathopen#1{}
\fi

% -----------------------------------START Rouletteaufgaben ------------------------------------------------------------

\ifttm
% #1 = Dateiname, #2 = eindeutige ID fuer das Roulette im Kurs
\newcommand{\MDirectRouletteExercises}[2]{
\begin{MExercise}
\texttt{Im HTML erscheinen hier Aufgaben aus einer Aufgabenliste...}
\end{MExercise}
}
\else
\newcommand{\MDirectRouletteExercises}[2]{\relax} % wird durch mconvert.pl gefunden und ersetzt
\fi


% ---------------------------------- START Makros, die von der Kursvariante abhaengen ----------------------------------

\ifvariantunotation
  % unotation = An Universitaeten uebliche Notation
  \def\MVariant{unotation}

  % Trennzeichen fuer Dezimalzahlen
  \newcommand{\MZXYZhltrennzeichen}{.}

  % Exponent zur Basis 10 in der Exponentialschreibweise, 
  % Standardmalzeichen: \times
  \newcommand{\MExponentensymbXYZl}{\times} 

  % Begrenzungszeichen fuer offene Intervalle
  \newcommand{\MoIl}[1][]{\mbox{}#1(\mathopen{}} % bzw. ']'
  \newcommand{\MoIr}[1][]{#1)\mbox{}} % bzw. '['

  % Zahlen-Separation im IntervaLL
  \newcommand{\MIntvlSep}{,} %oder ';'

  % Separation von Elementen in Mengen
  \newcommand{\MElSetSep}{,} %oder ';'

  % Separation von Koordinaten in Punkten
  \newcommand{\MCoordPointSep}{,} %oder ';' oder '|', '\MThinspace|\MThinspace'

\else
  % An dieser Stelle wird angenommen, dass std-Variante aktiv ist
  % std = beschlossene Notation im TU9-Onlinekurs 
  \def\MVariant{std}

  % Trennzeichen fuer Dezimalzahlen
  \newcommand{\MZXYZhltrennzeichen}{,}

  % Exponent zur Basis 10 in der Exponentialschreibweise, 
  % Standardmalzeichen: \times
  \newcommand{\MExponentensymbXYZl}{\times} 

  % Begrenzungszeichen fuer offene Intervalle
  \newcommand{\MoIl}[1][]{\mbox{}#1]\mathopen{}} % bzw. '('
  \newcommand{\MoIr}[1][]{#1[\mbox{}} % bzw. ')'

  % Zahlen-Separation im IntervaLL
  \newcommand{\MIntvlSep}{;} %oder ','
  
  % Separation von Elementen in Mengen
  \newcommand{\MElSetSep}{;} %oder ','

  % Separation von Koordinaten in Punkten
  \newcommand{\MCoordPointSep}{;} %oder '|', '\MThinspace|\MThinspace'

\fi



% ---------------------------------- ENDE Makros, die von der Kursvariante abhaengen ----------------------------------


% diese Kommandos setzen Mathemodus vorraus
\newcommand{\MGeoAbstand}[2]{[\overline{{#1}{#2}}]}
\newcommand{\MGeoGerade}[2]{{#1}{#2}}
\newcommand{\MGeoStrecke}[2]{\overline{{#1}{#2}}}
\newcommand{\MGeoDreieck}[3]{{#1}{#2}{#3}}

%
\ifttm
\newcommand{\MOhm}{\special{html:<mn>&#x3A9;</mn>}}
\else
\newcommand{\MOhm}{\Omega} %\varOmega
\fi


\def\PERCTAG{\MAbort{PERCTAG ist zur internen verwendung in mconvert.pl reserviert, dieses Makro darf sonst nicht benutzt werden.}}

% Im Gegensatz zu einfachen html-Umgebungen werden MDirectHTML-Umgebungen von mconvert.pl am ganzen ttm-Prozess vorbeigeschleust und aus dem PDF komplett ausgeschnitten
\ifttm%
\newenvironment{MDirectHTML}{\begin{html}}{\end{html}}%
\else%
\newenvironment{MDirectHTML}{\begin{html}}{\end{html}}%
\fi

% Im Gegensatz zu einfachen Mathe-Umgebungen werden MDirectMath-Umgebungen von mconvert.pl am ganzen ttm-Prozess vorbeigeschleust, ueber MathJax realisiert, und im PDF als $$ ... $$ gesetzt
\ifttm%
\newenvironment{MDirectMath}{\begin{html}}{\end{html}}%
\else%
\newenvironment{MDirectMath}{\begin{equation*}}{\end{equation*}}% Vorsicht, auch \[ und \] werden in amsmath durch equation* redefiniert
\fi

% ---------------------------------- Location Management ---------------------------------------------

% #1 = buttonname (muss in files/images liegen und Format 48x48 haben), #2 = Vollstaendiger Einrichtungsname, #3 = Kuerzel der Einrichtung,  #4 = Name der include-texdatei
\ifttm
\newcommand{\MLocationSite}[3]{\special{html:<!-- mlocation;;}#1\special{html:;;}#2\special{html:;;}#3\special{html:;; //-->}}
\else
\newcommand{\MLocationSite}[3]{\relax}
\fi

% ---------------------------------- Copyright Management --------------------------------------------

\newcommand{\MCCLicense}{%
{\color{green}\textbf{CC BY-SA 3.0}}
}

\newcommand{\MCopyrightLabel}[1]{ (\MSRef{L_COPYRIGHTCOLLECTION}{Lizenz})\MLabel{#1}}

% Copyrightliste wird als tex-Datei im preprocessing erzeugt und unter converter/tex/copyrightcollection.tex abgelegt
% Der input-Befehl funktioniert nur, wenn die aufrufende tex-Datei auf der obersten Ebene liegt (d.h. selbst kein input/include ist, insbesondere keine Moduldatei)
\newcommand{\MCopyrightCollection}{\input{copyrightcollection.tex}}

% MCopyrightNotice fuegt eine Copyrightnotiz ein, der parser ersetzt diese durch CopyrightNoticePOST im preparsing, diese Definition wird nur fuer reine pdflatex-Uebersetzungen gebraucht
% Parameter: #1: Kurze Lizenzbeschreibung (typischerweise \MCCLicense)
%            #2: Link zum Original (http://...) oder NONE falls das Bild selbst ein Original ist, oder TIKZ falls das Bild aus einer tikz-Umgebung stammt
%            #3: Link zum Autor (http://...) oder MINT falls Original im MINT-Kolleg erstellt oder NONE falls Autor unbekannt
%            #4: Bemerkung (z.B. dass Datei mit Maple exportiert wurde)
%            #5: Labelstring fuer existierendes Label auf das copyrighted Objekt, mit MCopyrightLabel erzeugt
%            Keines der Felder darf leer sein!
\newcommand{\MCopyrightNotice}[5]{\MCopyrightNoticePOST{#1}{#2}{#3}{#4}{#5}}

\ifttm%
\newcommand{\MCopyrightNoticePOST}[5]{\relax}%
\else%
\newcommand{\MCopyrightNoticePOST}[5]{\relax}%
\fi%

% ---------------------------------- Meldungen fuer den Benutzer des Konverters ----------------------
\MPragma{mintmodversion;P0.1.0}
\MPragma{usercomment;This is file mintmod.tex version P0.1.0}


% ----------------------------------- Spezialelemente fuer Konfigurationsseite, werden nicht von mintscripts.js verwaltet --

% #1 = DOM-id der Box
\ifttm\newcommand{\MConfigbox}[1]{\special{html:<input cfieldtype="2" type="checkbox" name="Name_}#1\special{html:" id="}#1\special{html:" onchange="confHandlerChange('}#1\special{html:');"/>}}\fi % darf im PDF nicht aufgerufen werden!


\MPragma{MathSkip}

%\Mtikzexternalize

\begin{document}
%% MINTMOD Version P0.1.0, needs to be consistent with preprocesser object in tex2x and MPragma-Version at the end of this file

% Parameter aus Konvertierungsprozess (PDF und HTML-Erzeugung wenn vom Konverter aus gestartet) werden hier eingefuegt, Preambleincludes werden am Schluss angehaengt

\newif\ifttm                % gesetzt falls Uebersetzung in HTML stattfindet, sonst uebersetzung in PDF

% Wahl der Notationsvariante ist im PDF immer std, in der HTML-Uebersetzung wird vom Konverter die Auswahl modifiziert
\newif\ifvariantstd
\newif\ifvariantunotation
\variantstdtrue % Diese Zeile wird vom Konverter erkannt und ggf. modifiziert, daher nicht veraendern!


\def\MOutputDVI{1}
\def\MOutputPDF{2}
\def\MOutputHTML{3}
\newcounter{MOutput}

\ifttm
\usepackage{german}
\usepackage{array}
\usepackage{amsmath}
\usepackage{amssymb}
\usepackage{amsthm}
\else
\documentclass[ngerman,oneside]{scrbook}
\usepackage{etex}
\usepackage[latin1]{inputenc}
\usepackage{textcomp}
\usepackage[ngerman]{babel}
\usepackage[pdftex]{color}
\usepackage{xcolor}
\usepackage{graphicx}
\usepackage[all]{xy}
\usepackage{fancyhdr}
\usepackage{verbatim}
\usepackage{array}
\usepackage{float}
\usepackage{makeidx}
\usepackage{amsmath}
\usepackage{amstext}
\usepackage{amssymb}
\usepackage{amsthm}
\usepackage[ngerman]{varioref}
\usepackage{framed}
\usepackage{supertabular}
\usepackage{longtable}
\usepackage{maxpage}
\usepackage{tikz}
\usepackage{tikzscale}
\usepackage{tikz-3dplot}
\usepackage{bibgerm}
\usepackage{chemarrow}
\usepackage{polynom}
%\usepackage{draftwatermark}
\usepackage{pdflscape}
\usetikzlibrary{calc}
\usetikzlibrary{through}
\usetikzlibrary{shapes.geometric}
\usetikzlibrary{arrows}
\usetikzlibrary{intersections}
\usetikzlibrary{decorations.pathmorphing}
\usetikzlibrary{external}
\usetikzlibrary{patterns}
\usetikzlibrary{fadings}
\usepackage[colorlinks=true,linkcolor=blue]{hyperref} 
\usepackage[all]{hypcap}
%\usepackage[colorlinks=true,linkcolor=blue,bookmarksopen=true]{hyperref} 
\usepackage{ifpdf}

\usepackage{movie15}

\setcounter{tocdepth}{2} % In Inhaltsverzeichnis bis subsection
\setcounter{secnumdepth}{3} % Nummeriert bis subsubsection

\setlength{\LTpost}{0pt} % Fuer longtable
\setlength{\parindent}{0pt}
\setlength{\parskip}{8pt}
%\setlength{\parskip}{9pt plus 2pt minus 1pt}
\setlength{\abovecaptionskip}{-0.25ex}
\setlength{\belowcaptionskip}{-0.25ex}
\fi

\ifttm
\newcommand{\MDebugMessage}[1]{\special{html:<!-- debugprint;;}#1\special{html:; //-->}}
\else
%\newcommand{\MDebugMessage}[1]{\immediate\write\mintlog{#1}}
\newcommand{\MDebugMessage}[1]{}
\fi

\def\MPageHeaderDef{%
\pagestyle{fancy}%
\fancyhead[r]{(C) VE\&MINT-Projekt}
\fancyfoot[c]{\thepage\\--- CCL BY-SA 3.0 ---}
}


\ifttm%
\def\MRelax{}%
\else%
\def\MRelax{\relax}%
\fi%

%--------------------------- Uebernahme von speziellen XML-Versionen einiger LaTeX-Kommandos aus xmlbefehle.tex vom alten Kasseler Konverter ---------------

\newcommand{\MSep}{\left\|{\phantom{\frac1g}}\right.}

\newcommand{\ML}{L}

\newcommand{\MGGT}{\mathrm{ggT}}


\ifttm
% Verhindert dass die subsection-nummer doppelt in der toccaption auftaucht (sollte ggf. in toccaption gefixt werden so dass diese Ueberschreibung nicht notwendig ist)
\renewcommand{\thesubsection}{}
% Kommandos die ttm nicht kennt
\newcommand{\binomial}[2]{{#1 \choose #2}} %  Binomialkoeffizienten
\newcommand{\eur}{\begin{html}&euro;\end{html}}
\newcommand{\square}{\begin{html}&square;\end{html}}
\newcommand{\glqq}{"'}  \newcommand{\grqq}{"'}
\newcommand{\nRightarrow}{\special{html: &nrArr; }}
\newcommand{\nmid}{\special{html: &nmid; }}
\newcommand{\nparallel}{\begin{html}&nparallel;\end{html}}
\newcommand{\mapstoo}{\begin{html}<mo>&map;</mo>\end{html}}

% Schnitt und Vereinigungssymbole von Mengen haben zu kleine Abstaende; korrigiert:
\newcommand{\ccup}{\,\!\cup\,\!}
\newcommand{\ccap}{\,\!\cap\,\!}


% Umsetzung von mathbb im HTML
\renewcommand{\mathbb}[1]{\begin{html}<mo>&#1opf;</mo>\end{html}}
\fi

%---------------------- Strukturierung ----------------------------------------------------------------------------------------------------------------------

%---------------------- Kapselung des sectioning findet auf drei Ebenen statt:
% 1. Die LateX-Befehl
% 2. Die D-Versionen der Befehle, die nur die Grade der Abschnitte umhaengen falls notwendig
% 3. Die M-Versionen der Befehle, die zusaetzliche Formatierungen vornehmen, Skripten starten und das HTML codieren
% Im Modultext duerfen nur die M-Befehle verwendet werden!

\ifttm

  \def\Dsubsubsubsection#1{\subsubsubsection{#1}}
  \def\Dsubsubsection#1{\subsubsection{#1}\addtocounter{subsubsection}{1}} % ttm-Fehler korrigieren
  \def\Dsubsection#1{\subsection{#1}}
  \def\Dsection#1{\section{#1}} % Im HTML wird nur der Sektionstitel gegeben
  \def\Dchapter#1{\chapter{#1}}
  \def\Dsubsubsubsectionx#1{\subsubsubsection*{#1}}
  \def\Dsubsubsectionx#1{\subsubsection*{#1}}
  \def\Dsubsectionx#1{\subsection*{#1}}
  \def\Dsectionx#1{\section*{#1}}
  \def\Dchapterx#1{\chapter*{#1}}

\else

  \def\Dsubsubsubsection#1{\subsubsection{#1}}
  \def\Dsubsubsection#1{\subsection{#1}}
  \def\Dsubsection#1{\section{#1}}
  \def\Dsection#1{\chapter{#1}}
  \def\Dchapter#1{\title{#1}}
  \def\Dsubsubsubsectionx#1{\subsubsection*{#1}}
  \def\Dsubsubsectionx#1{\subsection*{#1}}
  \def\Dsubsectionx#1{\section*{#1}}
  \def\Dsectionx#1{\chapter*{#1}}

\fi

\newcommand{\MStdPoints}{4}
\newcommand{\MSetPoints}[1]{\renewcommand{\MStdPoints}{#1}}

% Befehl zum Abbruch der Erstellung (nur PDF)
\newcommand{\MAbort}[1]{\err{#1}}

% Prefix vor Dateieinbindungen, wird in der Baumdatei mit \renewcommand modifiziert
% und auf das Verzeichnisprefix gesetzt, in dem das gerade bearbeitete tex-Dokument liegt.
% Im HTML wird es auf das Verzeichnis der HTML-Datei gesetzt.
% Das Prefix muss mit / enden !
\newcommand{\MDPrefix}{.}

% MRegisterFile notiert eine Datei zur Einbindung in den HTML-Baum. Grafiken mit MGraphics werden automatisch eingebunden.
% Mit MLastFile erhaelt man eine Markierung fuer die zuletzt registrierte Datei.
% Diese Markierung wird im postprocessing durch den physikalischen Dateinamen ersetzt, aber nur den Namen (d.h. \MMaterial gehoert noch davor, vgl Definition von MGraphics)
% Parameter: Pfad/Name der Datei bzw. des Ordners, relativ zur Position des Modul-Tex-Dokuments.
\ifttm
\newcommand{\MRegisterFile}[1]{\addtocounter{MFileNumber}{1}\special{html:<!-- registerfile;;}#1\special{html:;;}\MDPrefix\special{html:;;}\arabic{MFileNumber}\special{html:; //-->}}
\else
\newcommand{\MRegisterFile}[1]{\addtocounter{MFileNumber}{1}}
\fi

% Testen welcher Uebersetzer hier am Werk ist

\ifttm
\setcounter{MOutput}{3}
\else
\ifx\pdfoutput\undefined
  \pdffalse
  \setcounter{MOutput}{\MOutputDVI}
  \message{Verarbeitung mit latex, Ausgabe in dvi.}
\else
  \setcounter{MOutput}{\MOutputPDF}
  \message{Verarbeitung mit pdflatex, Ausgabe in pdf.}
  \ifnum \pdfoutput=0
    \pdffalse
  \setcounter{MOutput}{\MOutputDVI}
  \message{Verarbeitung mit pdflatex, Ausgabe in dvi.}
  \else
    \ifnum\pdfoutput=1
    \pdftrue
  \setcounter{MOutput}{\MOutputPDF}
  \message{Verarbeitung mit pdflatex, Ausgabe in pdf.}
    \fi
  \fi
\fi
\fi

\ifnum\value{MOutput}=\MOutputPDF
\DeclareGraphicsExtensions{.pdf,.png,.jpg}
\fi

\ifnum\value{MOutput}=\MOutputDVI
\DeclareGraphicsExtensions{.eps,.png,.jpg}
\fi

\ifnum\value{MOutput}=\MOutputHTML
% Wird vom Konverter leider nicht erkannt und daher in split.pm hardcodiert!
\DeclareGraphicsExtensions{.png,.jpg,.gif}
\fi

% Umdefinition der hyperref-Nummerierung im PDF-Modus
\ifttm
\else
\renewcommand{\theHfigure}{\arabic{chapter}.\arabic{section}.\arabic{figure}}
\fi

% Makro, um in der HTML-Ausgabe die zuerst zu oeffnende Datei zu kennzeichnen
\ifttm
\newcommand{\MGlobalStart}{\special{html:<!-- mglobalstarttag -->}}
\else
\newcommand{\MGlobalStart}{}
\fi

% Makro, um bei scormlogin ein pullen des Benutzers bei Aufruf der Seite zu erzwingen (typischerweise auf der Einstiegsseite)
\ifttm
\newcommand{\MPullSite}{\special{html:<!-- pullsite //-->}}
\else
\newcommand{\MPullSite}{}
\fi

% Makro, um in der HTML-Ausgabe die Kapiteluebersicht zu kennzeichnen
\ifttm
\newcommand{\MGlobalChapterTag}{\special{html:<!-- mglobalchaptertag -->}}
\else
\newcommand{\MGlobalChapterTag}{}
\fi

% Makro, um in der HTML-Ausgabe die Konfiguration zu kennzeichnen
\ifttm
\newcommand{\MGlobalConfTag}{\special{html:<!-- mglobalconfigtag -->}}
\else
\newcommand{\MGlobalConfTag}{}
\fi

% Makro, um in der HTML-Ausgabe die Standortbeschreibung zu kennzeichnen
\ifttm
\newcommand{\MGlobalLocationTag}{\special{html:<!-- mgloballocationtag -->}}
\else
\newcommand{\MGlobalLocationTag}{}
\fi

% Makro, um in der HTML-Ausgabe die persoenlichen Daten zu kennzeichnen
\ifttm
\newcommand{\MGlobalDataTag}{\special{html:<!-- mglobaldatatag -->}}
\else
\newcommand{\MGlobalDataTag}{}
\fi

% Makro, um in der HTML-Ausgabe die Suchseite zu kennzeichnen
\ifttm
\newcommand{\MGlobalSearchTag}{\special{html:<!-- mglobalsearchtag -->}}
\else
\newcommand{\MGlobalSearchTag}{}
\fi

% Makro, um in der HTML-Ausgabe die Favoritenseite zu kennzeichnen
\ifttm
\newcommand{\MGlobalFavoTag}{\special{html:<!-- mglobalfavoritestag -->}}
\else
\newcommand{\MGlobalFavoTag}{}
\fi

% Makro, um in der HTML-Ausgabe die Eingangstestseite zu kennzeichnen
\ifttm
\newcommand{\MGlobalSTestTag}{\special{html:<!-- mglobalstesttag -->}}
\else
\newcommand{\MGlobalSTestTag}{}
\fi

% Makro, um in der PDF-Ausgabe ein Wasserzeichen zu definieren
\ifttm
\newcommand{\MWatermarkSettings}{\relax}
\else
\newcommand{\MWatermarkSettings}{%
% \SetWatermarkText{(c) MINT-Kolleg Baden-W�rttemberg 2014}
% \SetWatermarkLightness{0.85}
% \SetWatermarkScale{1.5}
}
\fi

\ifttm
\newcommand{\MBinom}[2]{\left({\begin{array}{c} #1 \\ #2 \end{array}}\right)}
\else
\newcommand{\MBinom}[2]{\binom{#1}{#2}}
\fi

\ifttm
\newcommand{\DeclareMathOperator}[2]{\def#1{\mathrm{#2}}}
\newcommand{\operatorname}[1]{\mathrm{#1}}
\fi

%----------------- Makros fuer die gemischte HTML/PDF-Konvertierung ------------------------------

\newcommand{\MTestName}{\relax} % wird durch Test-Umgebung gesetzt

% Fuer experimentelle Kursinhalte, die im Release-Umsetzungsvorgang eine Fehlermeldung
% produzieren sollen aber sonst normal umgesetzt werden
\newenvironment{MExperimental}{%
}{%
}

% Wird von ttm nicht richtig umgesetzt!!
\newenvironment{MExerciseItems}{%
\renewcommand\theenumi{\alph{enumi}}%
\begin{enumerate}%
}{%
\end{enumerate}%
}


\definecolor{infoshadecolor}{rgb}{0.75,0.75,0.75}
\definecolor{exmpshadecolor}{rgb}{0.875,0.875,0.875}
\definecolor{expeshadecolor}{rgb}{0.95,0.95,0.95}
\definecolor{framecolor}{rgb}{0.2,0.2,0.2}

% Bei PDF-Uebersetzung wird hinter den Start jeder Satz/Info-aehnlichen Umgebung eine leere mbox gesetzt, damit
% fuehrende Listen oder enums nicht den Zeilenumbruch kaputtmachen
%\ifttm
\def\MTB{}
%\else
%\def\MTB{\mbox{}}
%\fi


\ifttm
\newcommand{\MRelates}{\special{html:<mi>&wedgeq;</mi>}}
\else
\def\MRelates{\stackrel{\scriptscriptstyle\wedge}{=}}
\fi

\def\MInch{\text{''}}
\def\Mdd{\textit{''}}

\ifttm
\def\MNL{ \newline }
\newenvironment{MArray}[1]{\begin{array}{#1}}{\end{array}}
\else
\def\MNL{ \\ }
\newenvironment{MArray}[1]{\begin{array}{#1}}{\end{array}}
\fi

\newcommand{\MBox}[1]{$\mathrm{#1}$}
\newcommand{\MMBox}[1]{\mathrm{#1}}


\ifttm%
\newcommand{\Mtfrac}[2]{{\textstyle \frac{#1}{#2}}}
\newcommand{\Mdfrac}[2]{{\displaystyle \frac{#1}{#2}}}
\newcommand{\Mmeasuredangle}{\special{html:<mi>&angmsd;</mi>}}
\else%
\newcommand{\Mtfrac}[2]{\tfrac{#1}{#2}}
\newcommand{\Mdfrac}[2]{\dfrac{#1}{#2}}
\newcommand{\Mmeasuredangle}{\measuredangle}
\relax
\fi

% Matrizen und Vektoren

% Inhalt wird in der Form a & b \\ c & d erwartet
% Vorsicht: MVector = Komponentenspalte, MVec = Variablensymbol
\ifttm%
\newcommand{\MVector}[1]{\left({\begin{array}{c}#1\end{array}}\right)}
\else%
\newcommand{\MVector}[1]{\begin{pmatrix}#1\end{pmatrix}}
\fi



\newcommand{\MVec}[1]{\vec{#1}}
\newcommand{\MDVec}[1]{\overrightarrow{#1}}

%----------------- Umgebungen fuer Definitionen und Saetze ----------------------------------------

% Fuegt einen Tabellen-Zeilenumbruch ein im PDF, aber nicht im HTML
\newcommand{\TSkip}{\ifttm \else&\ \\\fi}

\newenvironment{infoshaded}{%
\def\FrameCommand{\fboxsep=\FrameSep \fcolorbox{framecolor}{infoshadecolor}}%
\MakeFramed {\advance\hsize-\width \FrameRestore}}%
{\endMakeFramed}

\newenvironment{expeshaded}{%
\def\FrameCommand{\fboxsep=\FrameSep \fcolorbox{framecolor}{expeshadecolor}}%
\MakeFramed {\advance\hsize-\width \FrameRestore}}%
{\endMakeFramed}

\newenvironment{exmpshaded}{%
\def\FrameCommand{\fboxsep=\FrameSep \fcolorbox{framecolor}{exmpshadecolor}}%
\MakeFramed {\advance\hsize-\width \FrameRestore}}%
{\endMakeFramed}

\def\STDCOLOR{black}

\ifttm%
\else%
\newtheoremstyle{MSatzStyle}
  {1cm}                   %Space above
  {1cm}                   %Space below
  {\normalfont\itshape}   %Body font
  {}                      %Indent amount (empty = no indent,
                          %\parindent = para indent)
  {\normalfont\bfseries}  %Thm head font
  {}                      %Punctuation after thm head
  {\newline}              %Space after thm head: " " = normal interword
                          %space; \newline = linebreak
  {\thmname{#1}\thmnumber{ #2}\thmnote{ (#3)}}
                          %Thm head spec (can be left empty, meaning
                          %`normal')
                          %
\newtheoremstyle{MDefStyle}
  {1cm}                   %Space above
  {1cm}                   %Space below
  {\normalfont}           %Body font
  {}                      %Indent amount (empty = no indent,
                          %\parindent = para indent)
  {\normalfont\bfseries}  %Thm head font
  {}                      %Punctuation after thm head
  {\newline}              %Space after thm head: " " = normal interword
                          %space; \newline = linebreak
  {\thmname{#1}\thmnumber{ #2}\thmnote{ (#3)}}
                          %Thm head spec (can be left empty, meaning
                          %`normal')
\fi%

\newcommand{\MInfoText}{Info}

\newcounter{MHintCounter}
\newcounter{MCodeEditCounter}

\newcounter{MLastIndex}  % Enthaelt die dritte Stelle (Indexnummer) des letzten angelegten Objekts
\newcounter{MLastType}   % Enthaelt den Typ des letzten angelegten Objekts (mithilfe der unten definierten Konstanten). Die Entscheidung, wie der Typ dargstellt wird, wird in split.pm beim Postprocessing getroffen.
\newcounter{MLastTypeEq} % =1 falls das Label in einer Matheumgebung (equation, eqnarray usw.) steht, =2 falls das Label in einer table-Umgebung steht

% Da ttm keine Zahlmakros verarbeiten kann, werden diese Nummern in den Zuweisungen hardcodiert!
\def\MTypeSection{1}          %# Zaehler ist section
\def\MTypeSubsection{2}       %# Zaehler ist subsection
\def\MTypeSubsubsection{3}    %# Zaehler ist subsubsection
\def\MTypeInfo{4}             %# Eine Infobox, Separatzaehler fuer die Chemie (auch wenn es dort nicht nummeriert wird) ist MInfoCounter
\def\MTypeExercise{5}         %# Eine Aufgabe, Separatzaehler fuer die Chemie ist MExerciseCounter
\def\MTypeExample{6}          %# Eine Beispielbox, Separatzaehler fuer die Chemie ist MExampleCounter
\def\MTypeExperiment{7}       %# Eine Versuchsbox, Separatzaehler fuer die Chemie ist MExperimentCounter
\def\MTypeGraphics{8}         %# Eine Graphik, Separatzaehler fuer alle FB ist MGraphicsCounter
\def\MTypeTable{9}            %# Eine Tabellennummer, hat keinen Zaehler da durch table gezaehlt wird
\def\MTypeEquation{10}        %# Eine Gleichungsnummer, hat keinen Zaehler da durch equation/eqnarray gezaehlt wird
\def\MTypeTheorem{11}         % Ein theorem oder xtheorem, Separatzaehler fuer die Chemie ist MTheoremCounter
\def\MTypeVideo{12}           %# Ein Video,Separatzaehler fuer alle FB ist MVideoCounter
\def\MTypeEntry{13}           %# Ein Eintrag fuer die Stichwortliste, wird nicht gezaehlt sondern erhaelt im preparsing ein unique-label 

% Zaehler fuer das Labelsystem sind prefixcounter, jeder Zaehler wird VOR dem gezaehlten Objekt inkrementiert und zaehlt daher das aktuelle Objekt
\newcounter{MInfoCounter}
\newcounter{MExerciseCounter}
\newcounter{MExampleCounter}
\newcounter{MExperimentCounter}
\newcounter{MGraphicsCounter}
\newcounter{MTableCounter}
\newcounter{MEquationCounter}  % Nur im HTML, sonst durch "equation"-counter von latex realisiert
\newcounter{MTheoremCounter}
\newcounter{MObjectCounter}   % Gemeinsamer Zaehler fuer Objekte (ausser Grafiken/Tabellen) in Mathe/Info/Physik
\newcounter{MVideoCounter}
\newcounter{MEntryCounter}

\newcounter{MTestSite} % 1 = Subsubsection ist eine Pruefungsseite, 0 = ist eine normale Seite (inkl. Hilfeseite)

\def\MCell{$\phantom{a}$}

\newenvironment{MExportExercise}{\begin{MExercise}}{\end{MExercise}} % wird von mconvert abgefangen

\def\MGenerateExNumber{%
\ifnum\value{MSepNumbers}=0%
\arabic{section}.\arabic{subsection}.\arabic{MObjectCounter}\setcounter{MLastIndex}{\value{MObjectCounter}}%
\else%
\arabic{section}.\arabic{subsection}.\arabic{MExerciseCounter}\setcounter{MLastIndex}{\value{MExerciseCounter}}%
\fi%
}%

\def\MGenerateExmpNumber{%
\ifnum\value{MSepNumbers}=0%
\arabic{section}.\arabic{subsection}.\arabic{MObjectCounter}\setcounter{MLastIndex}{\value{MObjectCounter}}%
\else%
\arabic{section}.\arabic{subsection}.\arabic{MExerciseCounter}\setcounter{MLastIndex}{\value{MExampleCounter}}%
\fi%
}%

\def\MGenerateInfoNumber{%
\ifnum\value{MSepNumbers}=0%
\arabic{section}.\arabic{subsection}.\arabic{MObjectCounter}\setcounter{MLastIndex}{\value{MObjectCounter}}%
\else%
\arabic{section}.\arabic{subsection}.\arabic{MExerciseCounter}\setcounter{MLastIndex}{\value{MInfoCounter}}%
\fi%
}%

\def\MGenerateSiteNumber{%
\arabic{section}.\arabic{subsection}.\arabic{subsubsection}%
}%

% Funktionalitaet fuer Auswahlaufgaben

\newcounter{MExerciseCollectionCounter} % = 0 falls nicht in collection-Umgebung, ansonsten Schachtelungstiefe
\newcounter{MExerciseCollectionTextCounter} % wird von MExercise-Umgebung inkrementiert und von MExerciseCollection-Umgebung auf Null gesetzt

\ifttm
% MExerciseCollection gruppiert Aufgaben, die dynamisch aus der Datenbank gezogen werden und nicht direkt in der HTML-Seite stehen
% Parameter: #1 = ID der Collection, muss eindeutig fuer alle IN DER DB VORHANDENEN collections sein unabhaengig vom Kurs
%            #2 = Optionsargument (im Moment: 1 = Iterative Auswahl, 2 = Zufallsbasierte Auswahl)
\newenvironment{MExerciseCollection}[2]{%
\addtocounter{MExerciseCollectionCounter}{1}
\setcounter{MExerciseCollectionTextCounter}{0}
\special{html:<!-- mexercisecollectionstart;;}#1\special{html:;;}#2\special{html:;; //-->}%
}{%
\special{html:<!-- mexercisecollectionstop //-->}%
\addtocounter{MExerciseCollectionCounter}{-1}
}
\else
\newenvironment{MExerciseCollection}[2]{%
\addtocounter{MExerciseCollectionCounter}{1}
\setcounter{MExerciseCollectionTextCounter}{0}
}{%
\addtocounter{MExerciseCollectionCounter}{-1}
}
\fi

% Bei Uebersetzung nach PDF werden die theorem-Umgebungen verwendet, bei Uebersetzung in HTML ein manuelles Makro
\ifttm%

  \newenvironment{MHint}[1]{  \special{html:<button name="Name_MHint}\arabic{MHintCounter}\special{html:" class="hintbutton_closed" id="MHint}\arabic{MHintCounter}\special{html:_button" %
  type="button" onclick="toggle_hint('MHint}\arabic{MHintCounter}\special{html:');">}#1\special{html:</button>}
  \special{html:<div class="hint" style="display:none" id="MHint}\arabic{MHintCounter}\special{html:"> }}{\begin{html}</div>\end{html}\addtocounter{MHintCounter}{1}}

  \newenvironment{MCOSHZusatz}{  \special{html:<button name="Name_MHint}\arabic{MHintCounter}\special{html:" class="chintbutton_closed" id="MHint}\arabic{MHintCounter}\special{html:_button" %
  type="button" onclick="toggle_hint('MHint}\arabic{MHintCounter}\special{html:');">}Weiterf�hrende Inhalte\special{html:</button>}
  \special{html:<div class="hintc" style="display:none" id="MHint}\arabic{MHintCounter}\special{html:">
  <div class="coshwarn">Diese Inhalte gehen �ber das Kursniveau hinaus und werden in den Aufgaben und Tests nicht abgefragt.</div><br />}
  \addtocounter{MHintCounter}{1}}{\begin{html}</div>\end{html}}

  
  \newenvironment{MDefinition}{\begin{definition}\setcounter{MLastIndex}{\value{definition}}\ \\}{\end{definition}}

  
  \newenvironment{MExercise}{
  \renewcommand{\MStdPoints}{4}
  \addtocounter{MExerciseCounter}{1}
  \addtocounter{MObjectCounter}{1}
  \setcounter{MLastType}{5}

  \ifnum\value{MExerciseCollectionCounter}=0\else\addtocounter{MExerciseCollectionTextCounter}{1}\special{html:<!-- mexercisetextstart;;}\arabic{MExerciseCollectionTextCounter}\special{html:;; //-->}\fi
  \special{html:<div class="aufgabe" id="ADIV_}\MGenerateExNumber\special{html:">}%
  \textbf{Aufgabe \MGenerateExNumber
  } \ \\}{
  \special{html:</div><!-- mfeedbackbutton;Aufgabe;}\arabic{MTestSite}\special{html:;}\MGenerateExNumber\special{html:; //-->}
  \ifnum\value{MExerciseCollectionCounter}=0\else\special{html:<!-- mexercisetextstop //-->}\fi
  }

  % Stellt eine Kombination aus Aufgabe, Loesungstext und Eingabefeld bereit,
  % bei der Aufgabentext und Musterloesung sowie die zugehoerigen Feldelemente
  % extern bezogen und div-aktualisiert werden, das Eingabefeld aber immer das gleiche ist.
  \newenvironment{MFetchExercise}{
  \addtocounter{MExerciseCounter}{1}
  \addtocounter{MObjectCounter}{1}
  \setcounter{MLastType}{5}

  \special{html:<div class="aufgabe" id="ADIV_}\MGenerateExNumber\special{html:">}%
  \textbf{Aufgabe \MGenerateExNumber
  } \ \\%
  \special{html:</div><div class="exfetch_text" id="ADIVTEXT_}\MGenerateExNumber\special{html:">}%
  \special{html:</div><div class="exfetch_sol" id="ADIVSOL_}\MGenerateExNumber\special{html:">}%
  \special{html:</div><div class="exfetch_input" id="ADIVINPUT_}\MGenerateExNumber\special{html:">}%
  }{
  \special{html:</div>}
  }

  \newenvironment{MExample}{
  \addtocounter{MExampleCounter}{1}
  \addtocounter{MObjectCounter}{1}
  \setcounter{MLastType}{6}
  \begin{html}
  <div class="exmp">
  <div class="exmprahmen">
  \end{html}\textbf{Beispiel
  \ifnum\value{MSepNumbers}=0
  \arabic{section}.\arabic{subsection}.\arabic{MObjectCounter}\setcounter{MLastIndex}{\value{MObjectCounter}}
  \else
  \arabic{section}.\arabic{subsection}.\arabic{MExampleCounter}\setcounter{MLastIndex}{\value{MExampleCounter}}
  \fi
  } \ \\}{\begin{html}</div>
  </div>
  \end{html}
  \special{html:<!-- mfeedbackbutton;Beispiel;}\arabic{MTestSite}\special{html:;}\MGenerateExmpNumber\special{html:; //-->}
  }

  \newenvironment{MExperiment}{
  \addtocounter{MExperimentCounter}{1}
  \addtocounter{MObjectCounter}{1}
  \setcounter{MLastType}{7}
  \begin{html}
  <div class="expe">
  <div class="experahmen">
  \end{html}\textbf{Versuch
  \ifnum\value{MSepNumbers}=0
  \arabic{section}.\arabic{subsection}.\arabic{MObjectCounter}\setcounter{MLastIndex}{\value{MObjectCounter}}
  \else
%  \arabic{MExperimentCounter}\setcounter{MLastIndex}{\value{MExperimentCounter}}
  \arabic{section}.\arabic{subsection}.\arabic{MExperimentCounter}\setcounter{MLastIndex}{\value{MExperimentCounter}}
  \fi
  } \ \\}{\begin{html}</div>
  </div>
  \end{html}}

  \newenvironment{MChemInfo}{
  \setcounter{MLastType}{4}
  \begin{html}
  <div class="info">
  <div class="inforahmen">
  \end{html}}{\begin{html}</div>
  </div>
  \end{html}}

  \newenvironment{MXInfo}[1]{
  \addtocounter{MInfoCounter}{1}
  \addtocounter{MObjectCounter}{1}
  \setcounter{MLastType}{4}
  \begin{html}
  <div class="info">
  <div class="inforahmen">
  \end{html}\textbf{#1
  \ifnum\value{MInfoNumbers}=0
  \else
    \ifnum\value{MSepNumbers}=0
    \arabic{section}.\arabic{subsection}.\arabic{MObjectCounter}\setcounter{MLastIndex}{\value{MObjectCounter}}
    \else
    \arabic{MInfoCounter}\setcounter{MLastIndex}{\value{MInfoCounter}}
    \fi
  \fi
  } \ \\}{\begin{html}</div>
  </div>
  \end{html}
  \special{html:<!-- mfeedbackbutton;Info;}\arabic{MTestSite}\special{html:;}\MGenerateInfoNumber\special{html:; //-->}
  }

  \newenvironment{MInfo}{\ifnum\value{MInfoNumbers}=0\begin{MChemInfo}\else\begin{MXInfo}{Info}\ \\ \fi}{\ifnum\value{MInfoNumbers}=0\end{MChemInfo}\else\end{MXInfo}\fi}

\else%

  \theoremstyle{MSatzStyle}
  \newtheorem{thm}{Satz}[section]
  \newtheorem{thmc}{Satz}
  \theoremstyle{MDefStyle}
  \newtheorem{defn}[thm]{Definition}
  \newtheorem{exmp}[thm]{Beispiel}
  \newtheorem{info}[thm]{\MInfoText}
  \theoremstyle{MDefStyle}
  \newtheorem{defnc}{Definition}
  \theoremstyle{MDefStyle}
  \newtheorem{exmpc}{Beispiel}[section]
  \theoremstyle{MDefStyle}
  \newtheorem{infoc}{\MInfoText}
  \theoremstyle{MDefStyle}
  \newtheorem{exrc}{Aufgabe}[section]
  \theoremstyle{MDefStyle}
  \newtheorem{verc}{Versuch}[section]
  
  \newenvironment{MFetchExercise}{}{} % kann im PDF nicht dargestellt werden
  
  \newenvironment{MExercise}{\begin{exrc}\renewcommand{\MStdPoints}{1}\MTB}{\end{exrc}}
  \newenvironment{MHint}[1]{\ \\ \underline{#1:}\\}{}
  \newenvironment{MCOSHZusatz}{\ \\ \underline{Weiterf�hrende Inhalte:}\\}{}
  \newenvironment{MDefinition}{\ifnum\value{MInfoNumbers}=0\begin{defnc}\else\begin{defn}\fi\MTB}{\ifnum\value{MInfoNumbers}=0\end{defnc}\else\end{defn}\fi}
%  \newenvironment{MExample}{\begin{exmp}}{\ \linebreak[1] \ \ \ \ $\phantom{a}$ \ \hfill $\blacklozenge$\end{exmp}}
  \newenvironment{MExample}{
    \ifnum\value{MInfoNumbers}=0\begin{exmpc}\else\begin{exmp}\fi
    \MTB
    \begin{exmpshaded}
    \ \newline
}{
    \end{exmpshaded}
    \ifnum\value{MInfoNumbers}=0\end{exmpc}\else\end{exmp}\fi
}
  \newenvironment{MChemInfo}{\begin{infoshaded}}{\end{infoshaded}}

  \newenvironment{MInfo}{\ifnum\value{MInfoNumbers}=0\begin{MChemInfo}\else\renewcommand{\MInfoText}{Info}\begin{info}\begin{infoshaded}
  \MTB
   \ \newline
    \fi
  }{\ifnum\value{MInfoNumbers}=0\end{MChemInfo}\else\end{infoshaded}\end{info}\fi}

  \newenvironment{MXInfo}[1]{
    \renewcommand{\MInfoText}{#1}
    \ifnum\value{MInfoNumbers}=0\begin{infoc}\else\begin{info}\fi%
    \MTB
    \begin{infoshaded}
    \ \newline
  }{\end{infoshaded}\ifnum\value{MInfoNumbers}=0\end{infoc}\else\end{info}\fi}

  \newenvironment{MExperiment}{
    \renewcommand{\MInfoText}{Versuch}
    \ifnum\value{MInfoNumbers}=0\begin{verc}\else\begin{info}\fi
    \MTB
    \begin{expeshaded}
    \ \newline
  }{
    \end{expeshaded}
    \ifnum\value{MInfoNumbers}=0\end{verc}\else\end{info}\fi
  }
\fi%

% MHint sollte nicht direkt fuer Loesungen benutzt werden wegen solutionselect
\newenvironment{MSolution}{\begin{MHint}{L"osung}}{\end{MHint}}

\newcounter{MCodeCounter}

\ifttm
\newenvironment{MCode}{\special{html:<!-- mcodestart -->}\ttfamily\color{blue}}{\special{html:<!-- mcodestop -->}}
\else
\newenvironment{MCode}{\begin{flushleft}\ttfamily\addtocounter{MCodeCounter}{1}}{\addtocounter{MCodeCounter}{-1}\end{flushleft}}
% Ohne color-Statement da inkompatible mit framed/shaded-Boxen aus dem framed-package
\fi

%----------------- Sonderdefinitionen fuer Symbole, die der Konverter nicht kann ----------------------------------------------

\ifttm%
\newcommand{\MUnderset}[2]{\underbrace{#2}_{#1}}%
\else%
\newcommand{\MUnderset}[2]{\underset{#1}{#2}}%
\fi%

\ifttm
\newcommand{\MThinspace}{\special{html:<mi>&#x2009;</mi>}}
\else
\newcommand{\MThinspace}{\,}
\fi

\ifttm
\newcommand{\glq}{\begin{html}&sbquo;\end{html}}
\newcommand{\grq}{\begin{html}&lsquo;\end{html}}
\newcommand{\glqq}{\begin{html}&bdquo;\end{html}}
\newcommand{\grqq}{\begin{html}&ldquo;\end{html}}
\fi

\ifttm
\newcommand{\MNdash}{\begin{html}&ndash;\end{html}}
\else
\newcommand{\MNdash}{--}
\fi

%\ifttm\def\MIU{\special{html:<mi>&#8520;</mi>}}\else\def\MIU{\mathrm{i}}\fi
\def\MIU{\mathrm{i}}
\def\MEU{e} % TU9-Onlinekurs: italic-e
%\def\MEU{\mathrm{e}} % Alte Onlinemodule: roman-e
\def\MD{d} % Kursives d in Integralen im TU9-Onlinekurs
%\def\MD{\mathrm{d}} % roman-d in den alten Onlinemodulen
\def\MDB{\|}

%zusaetzlicher Leerraum vor "\MD"
\ifttm%
\def\MDSpace{\special{html:<mi>&#x2009;</mi>}}
\else%
\def\MDSpace{\,}
\fi%
\newcommand{\MDwSp}{\MDSpace\MD}%

\ifttm
\def\Mdq{\dq}
\else
\def\Mdq{\dq}
\fi

\def\MSpan#1{\left<{#1}\right>}
\def\MSetminus{\setminus}
\def\MIM{I}

\ifttm
\newcommand{\ld}{\text{ld}}
\newcommand{\lg}{\text{lg}}
\else
\DeclareMathOperator{\ld}{ld}
%\newcommand{\lg}{\text{lg}} % in latex schon definiert
\fi


\def\Mmapsto{\ifttm\special{html:<mi>&mapsto;</mi>}\else\mapsto\fi} 
\def\Mvarphi{\ifttm\phi\else\varphi\fi}
\def\Mphi{\ifttm\varphi\else\phi\fi}
\ifttm%
\newcommand{\MEumu}{\special{html:<mi>&#x3BC;</mi>}}%
\else%
\newcommand{\MEumu}{\textrm{\textmu}}%
\fi
\def\Mvarepsilon{\ifttm\epsilon\else\varepsilon\fi}
\def\Mepsilon{\ifttm\varepsilon\else\epsilon\fi}
\def\Mvarkappa{\ifttm\kappa\else\varkappa\fi}
\def\Mkappa{\ifttm\varkappa\else\kappa\fi}
\def\Mcomplement{\ifttm\special{html:<mi>&comp;</mi>}\else\complement\fi} 
\def\MWW{\mathrm{WW}}
\def\Mmod{\ifttm\special{html:<mi>&nbsp;mod&nbsp;</mi>}\else\mod\fi} 

\ifttm%
\def\mod{\text{\;mod\;}}%
\def\MNEquiv{\special{html:<mi>&NotCongruent;</mi>}}% 
\def\MNSubseteq{\special{html:<mi>&NotSubsetEqual;</mi>}}%
\def\MEmptyset{\special{html:<mi>&empty;</mi>}}%
\def\MVDots{\special{html:<mi>&#x22EE;</mi>}}%
\def\MHDots{\special{html:<mi>&#x2026;</mi>}}%
\def\Mddag{\special{html:<mi>&#x1202;</mi>}}%
\def\sphericalangle{\special{html:<mi>&measuredangle;</mi>}}%
\def\nparallel{\special{html:<mi>&nparallel;</mi>}}%
\def\MProofEnd{\special{html:<mi>&#x25FB;</mi>}}%
\newenvironment{MProof}[1]{\underline{#1}:\MCR\MCR}{\hfill $\MProofEnd$}%
\else%
\def\MNEquiv{\not\equiv}%
\def\MNSubseteq{\not\subseteq}%
\def\MEmptyset{\emptyset}%
\def\MVDots{\vdots}%
\def\MHDots{\hdots}%
\def\Mddag{\ddag}%
\newenvironment{MProof}[1]{\begin{proof}[#1]}{\end{proof}}%
\fi%



% Spaces zum Auffuellen von Tabellenbreiten, die nur im HTML wirken
\ifttm%
\def\MTSP{\:}%
\else%
\def\MTSP{}%
\fi%

\DeclareMathOperator{\arsinh}{arsinh}
\DeclareMathOperator{\arcosh}{arcosh}
\DeclareMathOperator{\artanh}{artanh}
\DeclareMathOperator{\arcoth}{arcoth}


\newcommand{\MMathSet}[1]{\mathbb{#1}}
\def\N{\MMathSet{N}}
\def\Z{\MMathSet{Z}}
\def\Q{\MMathSet{Q}}
\def\R{\MMathSet{R}}
\def\C{\MMathSet{C}}

\newcounter{MForLoopCounter}
\newcommand{\MForLoop}[2]{\setcounter{MForLoopCounter}{#1}\ifnum\value{MForLoopCounter}=0{}\else{{#2}\addtocounter{MForLoopCounter}{-1}\MForLoop{\value{MForLoopCounter}}{#2}}\fi}

\newcounter{MSiteCounter}
\newcounter{MFieldCounter} % Kombination section.subsection.site.field ist eindeutig in allen Modulen, field alleine nicht

\newcounter{MiniMarkerCounter}

\ifttm
\newenvironment{MMiniPageP}[1]{\begin{minipage}{#1\linewidth}\special{html:<!-- minimarker;;}\arabic{MiniMarkerCounter}\special{html:;;#1; //-->}}{\end{minipage}\addtocounter{MiniMarkerCounter}{1}}
\else
\newenvironment{MMiniPageP}[1]{\begin{minipage}{#1\linewidth}}{\end{minipage}\addtocounter{MiniMarkerCounter}{1}}
\fi

\newcounter{AlignCounter}

\newcommand{\MStartJustify}{\ifttm\special{html:<!-- startalign;;}\arabic{AlignCounter}\special{html:;;justify; //-->}\fi}
\newcommand{\MStopJustify}{\ifttm\special{html:<!-- stopalign;;}\arabic{AlignCounter}\special{html:; //-->}\fi\addtocounter{AlignCounter}{1}}

\newenvironment{MJTabular}[1]{
\MStartJustify
\begin{tabular}{#1}
}{
\end{tabular}
\MStopJustify
}

\newcommand{\MImageLeft}[2]{
\begin{center}
\begin{tabular}{lc}
\MStartJustify
\begin{MMiniPageP}{0.65}
#1
\end{MMiniPageP}
\MStopJustify
&
\begin{MMiniPageP}{0.3}
#2  
\end{MMiniPageP}
\end{tabular}
\end{center}
}

\newcommand{\MImageHalf}[2]{
\begin{center}
\begin{tabular}{lc}
\MStartJustify
\begin{MMiniPageP}{0.45}
#1
\end{MMiniPageP}
\MStopJustify
&
\begin{MMiniPageP}{0.45}
#2  
\end{MMiniPageP}
\end{tabular}
\end{center}
}

\newcommand{\MBigImageLeft}[2]{
\begin{center}
\begin{tabular}{lc}
\MStartJustify
\begin{MMiniPageP}{0.25}
#1
\end{MMiniPageP}
\MStopJustify
&
\begin{MMiniPageP}{0.7}
#2  
\end{MMiniPageP}
\end{tabular}
\end{center}
}

\ifttm
\def\No{\mathbb{N}_0}
\else
\def\No{\ensuremath{\N_0}}
\fi
\def\MT{\textrm{\tiny T}}
\newcommand{\MTranspose}[1]{{#1}^{\MT}}
\ifttm
\newcommand{\MRe}{\mathsf{Re}}
\newcommand{\MIm}{\mathsf{Im}}
\else
\DeclareMathOperator{\MRe}{Re}
\DeclareMathOperator{\MIm}{Im}
\fi

\newcommand{\Mid}{\mathrm{id}}
\newcommand{\MFeinheit}{\mathrm{feinh}}

\ifttm
\newcommand{\Msubstack}[1]{\begin{array}{c}{#1}\end{array}}
\else
\newcommand{\Msubstack}[1]{\substack{#1}}
\fi

% Typen von Fragefeldern:
% 1 = Alphanumerisch, case-sensitive-Vergleich
% 2 = Ja/Nein-Checkbox, Loesung ist 0 oder 1   (OPTION = Image-id fuer Rueckmeldung)
% 3 = Reelle Zahlen Geparset
% 4 = Funktionen Geparset (mit Stuetzstellen zur ueberpruefung)

% Dieser Befehl erstellt ein interaktives Aufgabenfeld. Parameter:
% - #1 Laenge in Zeichen
% - #2 Loesungstext (alphanumerisch, case sensitive)
% - #3 AufgabenID (alphanumerisch, case sensitive)
% - #4 Typ (Kennnummer)
% - #5 String fuer Optionen (ggf. mit Semikolon getrennte Einzelstrings)
% - #6 Anzahl Punkte
% - #7 uxid (kann z.B. Loesungsstring sein)
% ACHTUNG: Die langen Zeilen bitte so lassen, Zeilenumbrueche im tex werden in div's umgesetzt
\newcommand{\MQuestionID}[7]{
\ifttm
\special{html:<!-- mdeclareuxid;;}UX#7\special{html:;;}\arabic{section}\special{html:;;}#3\special{html:;; //-->}%
\special{html:<!-- mdeclarepoints;;}\arabic{section}\special{html:;;}#3\special{html:;;}#6\special{html:;;}\arabic{MTestSite}\special{html:;;}\arabic{chapter}%
\special{html:;; //--><!-- onloadstart //-->CreateQuestionObj("}#7\special{html:",}\arabic{MFieldCounter}\special{html:,"}#2%
\special{html:","}#3\special{html:",}#4\special{html:,"}#5\special{html:",}#6\special{html:,}\arabic{MTestSite}\special{html:,}\arabic{section}%
\special{html:);<!-- onloadstop //-->}%
\special{html:<input mfieldtype="}#4\special{html:" name="Name_}#3\special{html:" id="}#3\special{html:" type="text" size="}#1\special{html:" maxlength="}#1%
\special{html:" }\ifnum\value{MGroupActive}=0\special{html:onfocus="handlerFocus(}\arabic{MFieldCounter}%
\special{html:);" onblur="handlerBlur(}\arabic{MFieldCounter}\special{html:);" onkeyup="handlerChange(}\arabic{MFieldCounter}\special{html:,0);" onpaste="handlerChange(}\arabic{MFieldCounter}\special{html:,0);" oninput="handlerChange(}\arabic{MFieldCounter}\special{html:,0);" onpropertychange="handlerChange(}\arabic{MFieldCounter}\special{html:,0);"/>}%
\special{html:<img src="images/questionmark.gif" width="20" height="20" border="0" align="absmiddle" id="}QM#3\special{html:"/>}
\else%
\special{html:onblur="handlerBlur(}\arabic{MFieldCounter}%
\special{html:);" onfocus="handlerFocus(}\arabic{MFieldCounter}\special{html:);" onkeyup="handlerChange(}\arabic{MFieldCounter}\special{html:,1);" onpaste="handlerChange(}\arabic{MFieldCounter}\special{html:,1);" oninput="handlerChange(}\arabic{MFieldCounter}\special{html:,1);" onpropertychange="handlerChange(}\arabic{MFieldCounter}\special{html:,1);"/>}%
\special{html:<img src="images/questionmark.gif" width="20" height="20" border="0" align="absmiddle" id="}QM#3\special{html:"/>}\fi%
\else%
\ifnum\value{QBoxFlag}=1\fbox{$\phantom{\MForLoop{#1}{b}}$}\else$\phantom{\MForLoop{#1}{b}}$\fi%
\fi%
}

% ACHTUNG: Die langen Zeilen bitte so lassen, Zeilenumbrueche im tex werden in div's umgesetzt
% QuestionCheckbox macht ausserhalb einer QuestionGroup keinen Sinn!
% #1 = solution (1 oder 0), ggf. mit ::smc abgetrennt auszuschliessende single-choice-boxen (UXIDs durch , getrennt), #2 = id, #3 = points, #4 = uxid
\newcommand{\MQuestionCheckbox}[4]{
\ifttm
\special{html:<!-- mdeclareuxid;;}UX#4\special{html:;;}\arabic{section}\special{html:;;}#2\special{html:;; //-->}%
\ifnum\value{MGroupActive}=0\MDebugMessage{ERROR: Checkbox Nr. \arabic{MFieldCounter}\ ist nicht in einer Kontrollgruppe, es wird niemals eine Loesung angezeigt!}\fi
\special{html: %
<!-- mdeclarepoints;;}\arabic{section}\special{html:;;}#2\special{html:;;}#3\special{html:;;}\arabic{MTestSite}\special{html:;;}\arabic{chapter}%
\special{html:;; //--><!-- onloadstart //-->CreateQuestionObj("}#4\special{html:",}\arabic{MFieldCounter}\special{html:,"}#1\special{html:","}#2\special{html:",2,"IMG}#2%
\special{html:",}#3\special{html:,}\arabic{MTestSite}\special{html:,}\arabic{section}\special{html:);<!-- onloadstop //-->}%
\special{html:<input mfieldtype="2" type="checkbox" name="Name_}#2\special{html:" id="}#2\special{html:" onchange="handlerChange(}\arabic{MFieldCounter}\special{html:,1);"/><img src="images/questionmark.gif" name="}Name_IMG#2%
\special{html:" width="20" height="20" border="0" align="absmiddle" id="}IMG#2\special{html:"/> }%
\else%
\ifnum\value{QBoxFlag}=1\fbox{$\phantom{X}$}\else$\phantom{X}$\fi%
\fi%
}

\def\MGenerateID{QFELD_\arabic{section}.\arabic{subsection}.\arabic{MSiteCounter}.QF\arabic{MFieldCounter}}

% #1 = 0/1 ggf. mit ::smc abgetrennt auszuschliessende single-choice-boxen (UXIDs durch , getrennt ohne UX), #2 = uxid ohne UX
\newcommand{\MCheckbox}[2]{
\MQuestionCheckbox{#1}{\MGenerateID}{\MStdPoints}{#2}
\addtocounter{MFieldCounter}{1}
}

% Erster Parameter: Zeichenlaenge der Eingabebox, zweiter Parameter: Loesungstext
\newcommand{\MQuestion}[2]{
\MQuestionID{#1}{#2}{\MGenerateID}{1}{0}{\MStdPoints}{#2}
\addtocounter{MFieldCounter}{1}
}

% Erster Parameter: Zeichenlaenge der Eingabebox, zweiter Parameter: Loesungstext
\newcommand{\MLQuestion}[3]{
\MQuestionID{#1}{#2}{\MGenerateID}{1}{0}{\MStdPoints}{#3}
\addtocounter{MFieldCounter}{1}
}

% Parameter: Laenge des Feldes, Loesung (wird auch geparsed), Stellen Genauigkeit hinter dem Komma, weitere Stellen werden mathematisch gerundet vor Vergleich
\newcommand{\MParsedQuestion}[3]{
\MQuestionID{#1}{#2}{\MGenerateID}{3}{#3}{\MStdPoints}{#2}
\addtocounter{MFieldCounter}{1}
}

% Parameter: Laenge des Feldes, Loesung (wird auch geparsed), Stellen Genauigkeit hinter dem Komma, weitere Stellen werden mathematisch gerundet vor Vergleich
\newcommand{\MLParsedQuestion}[4]{
\MQuestionID{#1}{#2}{\MGenerateID}{3}{#3}{\MStdPoints}{#4}
\addtocounter{MFieldCounter}{1}
}

% Parameter: Laenge des Feldes, Loesungsfunktion, Anzahl Stuetzstellen, Funktionsvariablen durch Kommata getrennt (nicht case-sensitive), Anzahl Nachkommastellen im Vergleich
\newcommand{\MFunctionQuestion}[5]{
\MQuestionID{#1}{#2}{\MGenerateID}{4}{#3;#4;#5;0}{\MStdPoints}{#2}
\addtocounter{MFieldCounter}{1}
}

% Parameter: Laenge des Feldes, Loesungsfunktion, Anzahl Stuetzstellen, Funktionsvariablen durch Kommata getrennt (nicht case-sensitive), Anzahl Nachkommastellen im Vergleich, UXID
\newcommand{\MLFunctionQuestion}[6]{
\MQuestionID{#1}{#2}{\MGenerateID}{4}{#3;#4;#5;0}{\MStdPoints}{#6}
\addtocounter{MFieldCounter}{1}
}

% Parameter: Laenge des Feldes, Loesungsintervall, Genauigkeit der Zahlenwertpruefung
\newcommand{\MIntervalQuestion}[3]{
\MQuestionID{#1}{#2}{\MGenerateID}{6}{#3}{\MStdPoints}{#2}
\addtocounter{MFieldCounter}{1}
}

% Parameter: Laenge des Feldes, Loesungsintervall, Genauigkeit der Zahlenwertpruefung, UXID
\newcommand{\MLIntervalQuestion}[4]{
\MQuestionID{#1}{#2}{\MGenerateID}{6}{#3}{\MStdPoints}{#4}
\addtocounter{MFieldCounter}{1}
}

% Parameter: Laenge des Feldes, Loesungsfunktion, Anzahl Stuetzstellen, Funktionsvariable (nicht case-sensitive), Anzahl Nachkommastellen im Vergleich, Vereinfachungsbedingung
% Vereinfachungsbedingung ist eine der Folgenden:
% 0 = Keine Vereinfachungsbedingung
% 1 = Keine Klammern (runde oder eckige) mehr im vereinfachten Ausdruck
% 2 = Faktordarstellung (Term hat Produkte als letzte Operation, Summen als vorgeschaltete Operation)
% 3 = Summendarstellung (Term hat Summen als letzte Operation, Produkte als vorgeschaltete Operation)
% Flag 512: Besondere Stuetzstellen (nur >1 und nur schwach rational), sonst symmetrisch um Nullpunkt und ganze Zahlen inkl. Null werden getroffen
\newcommand{\MSimplifyQuestion}[6]{
\MQuestionID{#1}{#2}{\MGenerateID}{4}{#3;#4;#5;#6}{\MStdPoints}{#2}
\addtocounter{MFieldCounter}{1}
}

\newcommand{\MLSimplifyQuestion}[7]{
\MQuestionID{#1}{#2}{\MGenerateID}{4}{#3;#4;#5;#6}{\MStdPoints}{#7}
\addtocounter{MFieldCounter}{1}
}

% Parameter: Laenge des Feldes, Loesung (optionaler Ausdruck), Anzahl Stuetzstellen, Funktionsvariable (nicht case-sensitive), Anzahl Nachkommastellen im Vergleich, Spezialtyp (string-id)
\newcommand{\MLSpecialQuestion}[7]{
\MQuestionID{#1}{#2}{\MGenerateID}{7}{#3;#4;#5;#6}{\MStdPoints}{#7}
\addtocounter{MFieldCounter}{1}
}

\newcounter{MGroupStart}
\newcounter{MGroupEnd}
\newcounter{MGroupActive}

\newenvironment{MQuestionGroup}{
\setcounter{MGroupStart}{\value{MFieldCounter}}
\setcounter{MGroupActive}{1}
}{
\setcounter{MGroupActive}{0}
\setcounter{MGroupEnd}{\value{MFieldCounter}}
\addtocounter{MGroupEnd}{-1}
}

\newcommand{\MGroupButton}[1]{
\ifttm
\special{html:<button name="Name_Group}\arabic{MGroupStart}\special{html:to}\arabic{MGroupEnd}\special{html:" id="Group}\arabic{MGroupStart}\special{html:to}\arabic{MGroupEnd}\special{html:" %
type="button" onclick="group_button(}\arabic{MGroupStart}\special{html:,}\arabic{MGroupEnd}\special{html:);">}#1\special{html:</button>}
\else
\phantom{#1}
\fi
}

%----------------- Makros fuer die modularisierte Darstellung ------------------------------------

\def\MyText#1{#1}

% is used internally by the conversion package, should not be used by original tex documents
\def\MOrgLabel#1{\relax}

\ifttm

% Ein MLabel wird im html codiert durch das tag <!-- mmlabel;;Labelbezeichner;;SubjectArea;;chapter;;section;;subsection;;Index;;Objekttyp; //-->
\def\MLabel#1{%
\ifnum\value{MLastType}=8%
\ifnum\value{MCaptionOn}=0%
\MDebugMessage{ERROR: Grafik \arabic{MGraphicsCounter} hat separates label: #1 (Grafiklabels sollten nur in der Caption stehen)}%
\fi
\fi
\ifnum\value{MLastType}=12%
\ifnum\value{MCaptionOn}=0%
\MDebugMessage{ERROR: Video \arabic{MVideoCounter} hat separates label: #1 (Videolabels sollten nur in der Caption stehen}%
\fi
\fi
\ifnum\value{MLastType}=10\setcounter{MLastIndex}{\value{equation}}\fi
\label{#1}\begin{html}<!-- mmlabel;;#1;;\end{html}\arabic{MSubjectArea}\special{html:;;}\arabic{chapter}\special{html:;;}\arabic{section}\special{html:;;}\arabic{subsection}\special{html:;;}\arabic{MLastIndex}\special{html:;;}\arabic{MLastType}\special{html:; //-->}}%

\else

% Sonderbehandlung im PDF fuer Abbildungen in separater aux-Datei, da MGraphics die figure-Umgebung nicht verwendet
\def\MLabel#1{%
\ifnum\value{MLastType}=8%
\ifnum\value{MCaptionOn}=0%
\MDebugMessage{ERROR: Grafik \arabic{MGraphicsCounter} hat separates label: #1 (Grafiklabels sollten nur in der Caption stehen}%
\fi
\fi
\ifnum\value{MLastType}=12%
\ifnum\value{MCaptionOn}=0%
\MDebugMessage{ERROR: Video \arabic{MVideoCounter} hat separates label: #1 (Videolabels sollten nur in der Caption stehen}%
\fi
\fi
\label{#1}%
}%

\fi

% Gibt Begriff des referenzierten Objekts mit aus, aber nur im HTML, daher nur in Ausnahmefaellen (z.B. Copyrightliste) sinnvoll
\def\MCRef#1{\ifttm\special{html:<!-- mmref;;}#1\special{html:;;1; //-->}\else\vref{#1}\fi}


\def\MRef#1{\ifttm\special{html:<!-- mmref;;}#1\special{html:;;0; //-->}\else\vref{#1}\fi}
\def\MERef#1{\ifttm\special{html:<!-- mmref;;}#1\special{html:;;0; //-->}\else\eqref{#1}\fi}
\def\MNRef#1{\ifttm\special{html:<!-- mmref;;}#1\special{html:;;0; //-->}\else\ref{#1}\fi}
\def\MSRef#1#2{\ifttm\special{html:<!-- msref;;}#1\special{html:;;}#2\special{html:; //-->}\else \if#2\empty \ref{#1} \else \hyperref[#1]{#2}\fi\fi} 

\def\MRefRange#1#2{\ifttm\MRef{#1} bis 
\MRef{#2}\else\vrefrange[\unskip]{#1}{#2}\fi}

\def\MRefTwo#1#2{\ifttm\MRef{#1} und \MRef{#2}\else%
\let\vRefTLRsav=\reftextlabelrange\let\vRefTPRsav=\reftextpagerange%
\def\reftextlabelrange##1##2{\ref{##1} und~\ref{##2}}%
\def\reftextpagerange##1##2{auf den Seiten~\pageref{#1} und~\pageref{#2}}%
\vrefrange[\unskip]{#1}{#2}%
\let\reftextlabelrange=\vRefTLRsav\let\reftextpagerange=\vRefTPRsav\fi}

% MSectionChapter definiert falls notwendig das Kapitel vor der section. Das ist notwendig, wenn nur ein Einzelmodul uebersetzt wird.
% MChaptersGiven ist ein Counter, der von mconvert.pl vordefiniert wird.
\ifttm
\newcommand{\MSectionChapter}{\ifnum\value{MChaptersGiven}=0{\Dchapter{Modul}}\else{}\fi}
\else
\newcommand{\MSectionChapter}{\ifnum\value{chapter}=0{\Dchapter{Modul}}\else{}\fi}
\fi


\def\MChapter#1{\ifnum\value{MSSEnd}>0{\MSubsectionEndMacros}\addtocounter{MSSEnd}{-1}\fi\Dchapter{#1}}
\def\MSubject#1{\MChapter{#1}} % Schluesselwort HELPSECTION ist reserviert fuer Hilfesektion

\newcommand{\MSectionID}{UNKNOWNID}

\ifttm
\newcommand{\MSetSectionID}[1]{\renewcommand{\MSectionID}{#1}}
\else
\newcommand{\MSetSectionID}[1]{\renewcommand{\MSectionID}{#1}\tikzsetexternalprefix{#1}}
\fi


\newcommand{\MSection}[1]{\MSetSectionID{MODULID}\ifnum\value{MSSEnd}>0{\MSubsectionEndMacros}\addtocounter{MSSEnd}{-1}\fi\MSectionChapter\Dsection{#1}\MSectionStartMacros{#1}\setcounter{MLastIndex}{-1}\setcounter{MLastType}{1}} % Sections werden ueber das section-Feld im mmlabel-Tag identifiziert, nicht ueber das Indexfeld

\def\MSubsection#1{\ifnum\value{MSSEnd}>0{\MSubsectionEndMacros}\addtocounter{MSSEnd}{-1}\fi\ifttm\else\clearpage\fi\Dsubsection{#1}\MSubsectionStartMacros\setcounter{MLastIndex}{-1}\setcounter{MLastType}{2}\addtocounter{MSSEnd}{1}}% Subsections werden ueber das subsection-Feld im mmlabel-Tag identifiziert, nicht ueber das Indexfeld
\def\MSubsectionx#1{\Dsubsectionx{#1}} % Nur zur Verwendung in MSectionStart gedacht
\def\MSubsubsection#1{\Dsubsubsection{#1}\setcounter{MLastIndex}{\value{subsubsection}}\setcounter{MLastType}{3}\ifttm\special{html:<!-- sectioninfo;;}\arabic{section}\special{html:;;}\arabic{subsection}\special{html:;;}\arabic{subsubsection}\special{html:;;1;;}\arabic{MTestSite}\special{html:; //-->}\fi}
\def\MSubsubsectionx#1{\Dsubsubsectionx{#1}\ifttm\special{html:<!-- sectioninfo;;}\arabic{section}\special{html:;;}\arabic{subsection}\special{html:;;}\arabic{subsubsection}\special{html:;;0;;}\arabic{MTestSite}\special{html:; //-->}\else\addcontentsline{toc}{subsection}{#1}\fi}

\ifttm
\def\MSubsubsubsectionx#1{\ \newline\textbf{#1}\special{html:<br />}}
\else
\def\MSubsubsubsectionx#1{\ \newline
\textbf{#1}\ \\
}
\fi


% Dieses Skript wird zu Beginn jedes Modulabschnitts (=Webseite) ausgefuehrt und initialisiert den Aufgabenfeldzaehler
\newcommand{\MPageScripts}{
\setcounter{MFieldCounter}{1}
\addtocounter{MSiteCounter}{1}
\setcounter{MHintCounter}{1}
\setcounter{MCodeEditCounter}{1}
\setcounter{MGroupActive}{0}
\DoQBoxes
% Feldvariablen werden im HTML-Header in conv.pl eingestellt
}

% Dieses Skript wird zum Ende jedes Modulabschnitts (=Webseite) ausgefuehrt
\ifttm
\newcommand{\MEndScripts}{\special{html:<br /><!-- mfeedbackbutton;Seite;}\arabic{MTestSite}\special{html:;}\MGenerateSiteNumber\special{html:; //-->}
}
\else
\newcommand{\MEndScripts}{\relax}
\fi


\newcounter{QBoxFlag}
\newcommand{\DoQBoxes}{\setcounter{QBoxFlag}{1}}
\newcommand{\NoQBoxes}{\setcounter{QBoxFlag}{0}}

\newcounter{MXCTest}
\newcounter{MXCounter}
\newcounter{MSCounter}



\ifttm

% Struktur des sectioninfo-Tags: <!-- sectioninfo;;section;;subsection;;subsubsection;;nr_ausgeben;;testpage; //-->

%Fuegt eine zusaetzliche html-Seite an hinter ALLEN bisherigen und zukuenftigen content-Seiten ausserhalb der vor-zurueck-Schleife (d.h. nur durch Button oder MIntLink erreichbar!)
% #1 = Titel des Modulabschnitts, #2 = Kurztitel fuer die Buttons, #3 = Buttonkennung (STD = default nehmen, NONE = Ohne Button in der Navigation)
\newenvironment{MSContent}[3]{\special{html:<div class="xcontent}\arabic{MSCounter}\special{html:"><!-- scontent;-;}\arabic{MSCounter};-;#1;-;#2;-;#3\special{html: //-->}\MPageScripts\MSubsubsectionx{#1}}{\MEndScripts\special{html:<!-- endscontent;;}\arabic{MSCounter}\special{html: //--></div>}\addtocounter{MSCounter}{1}}

% Fuegt eine zusaetzliche html-Seite ein hinter den bereits vorhandenen content-Seiten (oder als erste Seite) innerhalb der vor-zurueck-Schleife der Navigation
% #1 = Titel des Modulabschnitts, #2 = Kurztitel fuer die Buttons, #3 = Buttonkennung (STD = Defaultbutton, NONE = Ohne Button in der Navigation)
\newenvironment{MXContent}[3]{\special{html:<div class="xcontent}\arabic{MXCounter}\special{html:"><!-- xcontent;-;}\arabic{MXCounter};-;#1;-;#2;-;#3\special{html: //-->}\MPageScripts\MSubsubsection{#1}}{\MEndScripts\special{html:<!-- endxcontent;;}\arabic{MXCounter}\special{html: //--></div>}\addtocounter{MXCounter}{1}}

% Fuegt eine zusaetzliche html-Seite ein die keine subsubsection-Nummer bekommt, nur zur internen Verwendung in mintmod.tex gedacht!
% #1 = Titel des Modulabschnitts, #2 = Kurztitel fuer die Buttons, #3 = Buttonkennung (STD = Defaultbutton, NONE = Ohne Button in der Navigation)
% \newenvironment{MUContent}[3]{\special{html:<div class="xcontent}\arabic{MXCounter}\special{html:"><!-- xcontent;-;}\arabic{MXCounter};-;#1;-;#2;-;#3\special{html: //-->}\MPageScripts\MSubsubsectionx{#1}}{\MEndScripts\special{html:<!-- endxcontent;;}\arabic{MXCounter}\special{html: //--></div>}\addtocounter{MXCounter}{1}}

\newcommand{\MDeclareSiteUXID}[1]{\special{html:<!-- mdeclaresiteuxid;;}#1\special{html:;;}\arabic{chapter}\special{html:;;}\arabic{section}\special{html:;; //-->}}

\else

%\newcommand{\MSubsubsection}[1]{\refstepcounter{subsubsection} \addcontentsline{toc}{subsubsection}{\thesubsubsection. #1}}


% Fuegt eine zusaetzliche html-Seite an hinter den bereits vorhandenen content-Seiten
% #1 = Titel des Modulabschnitts, #2 = Kurztitel fuer die Buttons, #3 = Iconkennung (im PDF wirkungslos)
%\newenvironment{MUContent}[3]{\ifnum\value{MXCTest}>0{\MDebugMessage{ERROR: Geschachtelter SContent}}\fi\MPageScripts\MSubsubsectionx{#1}\addtocounter{MXCTest}{1}}{\addtocounter{MXCounter}{1}\addtocounter{MXCTest}{-1}}
\newenvironment{MXContent}[3]{\ifnum\value{MXCTest}>0{\MDebugMessage{ERROR: Geschachtelter SContent}}\fi\MPageScripts\MSubsubsection{#1}\addtocounter{MXCTest}{1}}{\addtocounter{MXCounter}{1}\addtocounter{MXCTest}{-1}}
\newenvironment{MSContent}[3]{\ifnum\value{MXCTest}>0{\MDebugMessage{ERROR: Geschachtelter XContent}}\fi\MPageScripts\MSubsubsectionx{#1}\addtocounter{MXCTest}{1}}{\addtocounter{MSCounter}{1}\addtocounter{MXCTest}{-1}}

\newcommand{\MDeclareSiteUXID}[1]{\relax}

\fi 

% GHEADER und GFOOTER werden von split.pm gefunden, aber nur, wenn nicht HELPSITE oder TESTSITE
\ifttm
\newenvironment{MSectionStart}{\special{html:<div class="xcontent0">}\MSubsubsectionx{Modul\"ubersicht}}{\setcounter{MSSEnd}{0}\special{html:</div>}}
% Darf nicht als XContent nummeriert werden, darf nicht als XContent gelabelt werden, wird aber in eine xcontent-div gesetzt fuer Python-parsing
\else
\newenvironment{MSectionStart}{\MSubsectionx{Modul\"ubersicht}}{\setcounter{MSSEnd}{0}}
\fi

\newenvironment{MIntro}{\begin{MXContent}{Einf\"uhrung}{Einf\"uhrung}{genetisch}}{\end{MXContent}}
\newenvironment{MContent}{\begin{MXContent}{Inhalt}{Inhalt}{beweis}}{\end{MXContent}}
\newenvironment{MExercises}{\ifttm\else\clearpage\fi\begin{MXContent}{Aufgaben}{Aufgaben}{aufgb}\special{html:<!-- declareexcsymb //-->}}{\end{MXContent}}

% #1 = Lesbare Testbezeichnung
\newenvironment{MTest}[1]{%
\renewcommand{\MTestName}{#1}
\ifttm\else\clearpage\fi%
\addtocounter{MTestSite}{1}%
\begin{MXContent}{#1}{#1}{STD} % {aufgb}%
\special{html:<!-- declaretestsymb //-->}
\begin{MQuestionGroup}%
\MInTestHeader
}%
{%
\end{MQuestionGroup}%
\ \\ \ \\%
\MInTestFooter
\end{MXContent}\addtocounter{MTestSite}{-1}%
}

\newenvironment{MExtra}{\ifttm\else\clearpage\fi\begin{MXContent}{Zus\"atzliche Inhalte}{Zusatz}{weiterfhrg}}{\end{MXContent}}

\makeindex

\ifttm
\def\MPrintIndex{
\ifnum\value{MSSEnd}>0{\MSubsectionEndMacros}\addtocounter{MSSEnd}{-1}\fi
\renewcommand{\indexname}{Stichwortverzeichnis}
\special{html:<p><!-- printindex //--></p>}
}
\else
\def\MPrintIndex{
\ifnum\value{MSSEnd}>0{\MSubsectionEndMacros}\addtocounter{MSSEnd}{-1}\fi
\renewcommand{\indexname}{Stichwortverzeichnis}
\addcontentsline{toc}{section}{Stichwortverzeichnis}
\printindex
}
\fi


% Konstanten fuer die Modulfaecher

\def\MINTMathematics{1}
\def\MINTInformatics{2}
\def\MINTChemistry{3}
\def\MINTPhysics{4}
\def\MINTEngineering{5}

\newcounter{MSubjectArea}
\newcounter{MInfoNumbers} % Gibt an, ob die Infoboxen nummeriert werden sollen
\newcounter{MSepNumbers} % Gibt an, ob Beispiele und Experimente separat nummeriert werden sollen
\newcommand{\MSetSubject}[1]{
 % ttm kapiert setcounter mit Parametern nicht, also per if abragen und einsetzen
\ifnum#1=1\setcounter{MSubjectArea}{1}\setcounter{MInfoNumbers}{1}\setcounter{MSepNumbers}{0}\fi
\ifnum#1=2\setcounter{MSubjectArea}{2}\setcounter{MInfoNumbers}{1}\setcounter{MSepNumbers}{0}\fi
\ifnum#1=3\setcounter{MSubjectArea}{3}\setcounter{MInfoNumbers}{0}\setcounter{MSepNumbers}{1}\fi
\ifnum#1=4\setcounter{MSubjectArea}{4}\setcounter{MInfoNumbers}{0}\setcounter{MSepNumbers}{0}\fi
\ifnum#1=5\setcounter{MSubjectArea}{5}\setcounter{MInfoNumbers}{1}\setcounter{MSepNumbers}{0}\fi
% Separate Nummerntechnik fuer unsere Chemiker: alles dreistellig
\ifnum#1=3
  \ifttm
  \renewcommand{\theequation}{\arabic{section}.\arabic{subsection}.\arabic{equation}}
  \renewcommand{\thetable}{\arabic{section}.\arabic{subsection}.\arabic{table}} 
  \renewcommand{\thefigure}{\arabic{section}.\arabic{subsection}.\arabic{figure}} 
  \else
  \renewcommand{\theequation}{\arabic{chapter}.\arabic{section}.\arabic{equation}}
  \renewcommand{\thetable}{\arabic{chapter}.\arabic{section}.\arabic{table}}
  \renewcommand{\thefigure}{\arabic{chapter}.\arabic{section}.\arabic{figure}}
  \fi
\else
  \ifttm
  \renewcommand{\theequation}{\arabic{section}.\arabic{subsection}.\arabic{equation}}
  \renewcommand{\thetable}{\arabic{table}}
  \renewcommand{\thefigure}{\arabic{figure}}
  \else
  \renewcommand{\theequation}{\arabic{chapter}.\arabic{section}.\arabic{equation}}
  \renewcommand{\thetable}{\arabic{table}}
  \renewcommand{\thefigure}{\arabic{figure}}
  \fi
\fi
}

% Fuer tikz Autogenerierung
\newcounter{MTIKZAutofilenumber}

% Spezielle Counter fuer die Bentz-Module
\newcounter{mycounter}
\newcounter{chemapplet}
\newcounter{physapplet}

\newcounter{MSSEnd} % Ist 1 falls ein MSubsection aktiv ist, der einen MSubsectionEndMacro-Aufruf verursacht
\newcounter{MFileNumber}
\def\MLastFile{\special{html:[[!-- mfileref;;}\arabic{MFileNumber}\special{html:; //--]]}}

% Vollstaendiger Pfad ist \MMaterial / \MLastFilePath / \MLastFileName    ==   \MMaterial / \MLastFile

% Wird nur bei kompletter Baum-Erstellung ausgefuehrt!
% #1 = Lesbare Modulbezeichnung
\newcommand{\MSectionStartMacros}[1]{
\setcounter{MTestSite}{0}
\setcounter{MCaptionOn}{0}
\setcounter{MLastTypeEq}{0}
\setcounter{MSSEnd}{0}
\setcounter{MFileNumber}{0} % Preinkrekement-Counter
\setcounter{MTIKZAutofilenumber}{0}
\setcounter{mycounter}{1}
\setcounter{physapplet}{1}
\setcounter{chemapplet}{0}
\ifttm
\special{html:<!-- mdeclaresection;;}\arabic{chapter}\special{html:;;}\arabic{section}\special{html:;;}#1\special{html:;; //-->}%
\else
\setcounter{thmc}{0}
\setcounter{exmpc}{0}
\setcounter{verc}{0}
\setcounter{infoc}{0}
\fi
\setcounter{MiniMarkerCounter}{1}
\setcounter{AlignCounter}{1}
\setcounter{MXCTest}{0}
\setcounter{MCodeCounter}{0}
\setcounter{MEntryCounter}{0}
}

% Wird immer ausgefuehrt
\newcommand{\MSubsectionStartMacros}{
\ifttm\else\MPageHeaderDef\fi
\MWatermarkSettings
\setcounter{MXCounter}{0}
\setcounter{MSCounter}{0}
\setcounter{MSiteCounter}{1}
\setcounter{MExerciseCollectionCounter}{0}
% Zaehler fuer das Labelsystem zuruecksetzen (prefix-Zaehler)
\setcounter{MInfoCounter}{0}
\setcounter{MExerciseCounter}{0}
\setcounter{MExampleCounter}{0}
\setcounter{MExperimentCounter}{0}
\setcounter{MGraphicsCounter}{0}
\setcounter{MTableCounter}{0}
\setcounter{MTheoremCounter}{0}
\setcounter{MObjectCounter}{0}
\setcounter{MEquationCounter}{0}
\setcounter{MVideoCounter}{0}
\setcounter{equation}{0}
\setcounter{figure}{0}
}

\newcommand{\MSubsectionEndMacros}{
% Bei Chemiemodulen das PSE einhaengen, es soll als SContent am Ende erscheinen
\special{html:<!-- subsectionend //-->}
\ifnum\value{MSubjectArea}=3{\MIncludePSE}\fi
}


\ifttm
%\newcommand{\MEmbed}[1]{\MRegisterFile{#1}\begin{html}<embed src="\end{html}\MMaterial/\MLastFile\begin{html}" width="192" height="189"></embed>\end{html}}
\newcommand{\MEmbed}[1]{\MRegisterFile{#1}\begin{html}<embed src="\end{html}\MMaterial/\MLastFile\begin{html}"></embed>\end{html}}
\fi

%----------------- Makros fuer die Textdarstellung -----------------------------------------------

\ifttm
% MUGraphics bindet eine Grafik ein:
% Parameter 1: Dateiname der Grafik, relativ zur Position des Modul-Tex-Dokuments
% Parameter 2: Skalierungsoptionen fuer PDF (fuer includegraphics)
% Parameter 3: Titel fuer die Grafik, wird unter die Grafik mit der Grafiknummer gesetzt und kann MLabel bzw. MCopyrightLabel enthalten
% Parameter 4: Skalierungsoptionen fuer HTML (css-styles)

% ERSATZ: <img alt="My Image" src="data:image/png;base64,iVBORwA<MoreBase64SringHere>" />


\newcommand{\MUGraphics}[4]{\MRegisterFile{#1}\begin{html}
<div class="imagecenter">
<center>
<div>
<img src="\end{html}\MMaterial/\MLastFile\begin{html}" style="#4" alt="\end{html}\MMaterial/\MLastFile\begin{html}"/>
</div>
<div class="bildtext">
\end{html}
\addtocounter{MGraphicsCounter}{1}
\setcounter{MLastIndex}{\value{MGraphicsCounter}}
\setcounter{MLastType}{8}
\addtocounter{MCaptionOn}{1}
\ifnum\value{MSepNumbers}=0
\textbf{Abbildung \arabic{MGraphicsCounter}:} #3
\else
\textbf{Abbildung \arabic{section}.\arabic{subsection}.\arabic{MGraphicsCounter}:} #3
\fi
\addtocounter{MCaptionOn}{-1}
\begin{html}
</div>
</center>
</div>
<br />
\end{html}%
\special{html:<!-- mfeedbackbutton;Abbildung;}\arabic{MGraphicsCounter}\special{html:;}\arabic{section}.\arabic{subsection}.\arabic{MGraphicsCounter}\special{html:; //-->}%
}

% MVideo bindet ein Video als Einzeldatei ein:
% Parameter 1: Dateiname des Videos, relativ zur Position des Modul-Tex-Dokuments, ohne die Endung ".mp4"
% Parameter 2: Titel fuer das Video (kann MLabel oder MCopyrightLabel enthalten), wird unter das Video mit der Videonummer gesetzt
\newcommand{\MVideo}[2]{\MRegisterFile{#1.mp4}\begin{html}
<div class="imagecenter">
<center>
<div>
<video width="95\%" controls="controls"><source src="\end{html}\MMaterial/#1.mp4\begin{html}" type="video/mp4">Ihr Browser kann keine MP4-Videos abspielen!</video>
</div>
<div class="bildtext">
\end{html}
\addtocounter{MVideoCounter}{1}
\setcounter{MLastIndex}{\value{MVideoCounter}}
\setcounter{MLastType}{12}
\addtocounter{MCaptionOn}{1}
\ifnum\value{MSepNumbers}=0
\textbf{Video \arabic{MVideoCounter}:} #2
\else
\textbf{Video \arabic{section}.\arabic{subsection}.\arabic{MVideoCounter}:} #2
\fi
\addtocounter{MCaptionOn}{-1}
\begin{html}
</div>
</center>
</div>
<br />
\end{html}}

\newcommand{\MDVideo}[2]{\MRegisterFile{#1.mp4}\MRegisterFile{#1.ogv}\begin{html}
<div class="imagecenter">
<center>
<div>
<video width="70\%" controls><source src="\end{html}\MMaterial/#1.mp4\begin{html}" type="video/mp4"><source src="\end{html}\MMaterial/#1.ogv\begin{html}" type="video/ogg">Ihr Browser kann keine MP4-Videos abspielen!</video>
</div>
<br />
#2
</center>
</div>
<br />
\end{html}
}

\newcommand{\MGraphics}[3]{\MUGraphics{#1}{#2}{#3}{}}

\else

\newcommand{\MVideo}[2]{%
% Kein Video im PDF darstellbar, trotzdem so tun als ob da eines waere
\begin{center}
(Video nicht darstellbar)
\end{center}
\addtocounter{MVideoCounter}{1}
\setcounter{MLastIndex}{\value{MVideoCounter}}
\setcounter{MLastType}{12}
\addtocounter{MCaptionOn}{1}
\ifnum\value{MSepNumbers}=0
\textbf{Video \arabic{MVideoCounter}:} #2
\else
\textbf{Video \arabic{section}.\arabic{subsection}.\arabic{MVideoCounter}:} #2
\fi
\addtocounter{MCaptionOn}{-1}
}


% MGraphics bindet eine Grafik ein:
% Parameter 1: Dateiname der Grafik, relativ zur Position des Modul-Tex-Dokuments
% Parameter 2: Skalierungsoptionen fuer PDF (fuer includegraphics)
% Parameter 3: Titel fuer die Grafik, wird unter die Grafik mit der Grafiknummer gesetzt
\newcommand{\MGraphics}[3]{%
\MRegisterFile{#1}%
\ %
\begin{figure}[H]%
\centering{%
\includegraphics[#2]{\MDPrefix/#1}%
\addtocounter{MCaptionOn}{1}%
\caption{#3}%
\addtocounter{MCaptionOn}{-1}%
}%
\end{figure}%
\addtocounter{MGraphicsCounter}{1}\setcounter{MLastIndex}{\value{MGraphicsCounter}}\setcounter{MLastType}{8}\ %
%\ \\Abbildung \ifnum\value{MSepNumbers}=0\else\arabic{chapter}.\arabic{section}.\fi\arabic{MGraphicsCounter}: #3%
}

\newcommand{\MUGraphics}[4]{\MGraphics{#1}{#2}{#3}}


\fi

\newcounter{MCaptionOn} % = 1 falls eine Grafikcaption aktiv ist, = 0 sonst


% MGraphicsSolo bindet eine Grafik pur ein ohne Titel
% Parameter 1: Dateiname der Grafik, relativ zur Position des Modul-Tex-Dokuments
% Parameter 2: Skalierungsoptionen (wirken nur im PDF)
\newcommand{\MGraphicsSolo}[2]{\MUGraphicsSolo{#1}{#2}{}}

% MUGraphicsSolo bindet eine Grafik pur ein ohne Titel, aber mit HTML-Skalierung
% Parameter 1: Dateiname der Grafik, relativ zur Position des Modul-Tex-Dokuments
% Parameter 2: Skalierungsoptionen (wirken nur im PDF)
% Parameter 3: Skalierungsoptionen (wirken nur im HTML), als style-format: "width=???, height=???"
\ifttm
\newcommand{\MUGraphicsSolo}[3]{\MRegisterFile{#1}\begin{html}
<img src="\end{html}\MMaterial/\MLastFile\begin{html}" style="\end{html}#3\begin{html}" alt="\end{html}\MMaterial/\MLastFile\begin{html}"/>
\end{html}%
\special{html:<!-- mfeedbackbutton;Abbildung;}#1\special{html:;}\MMaterial/\MLastFile\special{html:; //-->}%
}
\else
\newcommand{\MUGraphicsSolo}[3]{\MRegisterFile{#1}\includegraphics[#2]{\MDPrefix/#1}}
\fi

% Externer Link mit URL
% Erster Parameter: Vollstaendige(!) URL des Links
% Zweiter Parameter: Text fuer den Link
\newcommand{\MExtLink}[2]{\ifttm\special{html:<a target="_new" href="}#1\special{html:">}#2\special{html:</a>}\else\href{#1}{#2}\fi} % ohne MINTERLINK!


% Interner Link, die verlinkte Datei muss im gleichen Verzeichnis liegen wie die Modul-Texdatei
% Erster Parameter: Dateiname
% Zweiter Parameter: Text fuer den Link
\newcommand{\MIntLink}[2]{\ifttm\MRegisterFile{#1}\special{html:<a class="MINTERLINK" target="_new" href="}\MMaterial/\MLastFile\special{html:">}#2\special{html:</a>}\else{\href{#1}{#2}}\fi}


\ifttm
\def\MMaterial{:localmaterial:}
\else
\def\MMaterial{\MDPrefix}
\fi

\ifttm
\def\MNoFile#1{:directmaterial:#1}
\else
\def\MNoFile#1{#1}
\fi

\newcommand{\MChem}[1]{$\mathrm{#1}$}

\newcommand{\MApplet}[3]{
% Bindet ein Java-Applet ein, die Parameter sind:
% (wird nur im HTML, aber nicht im PDF erstellt)
% #1 Dateiname des Applets (muss mit ".class" enden)
% #2 = Breite in Pixeln
% #3 = Hoehe in Pixeln
\ifttm
\MRegisterFile{#1}
\begin{html}
<applet code="\end{html}\MMaterial/\MLastFile\begin{html}" width="#2" height="#3" alt="[Java-Applet kann nicht gestartet werden]"></applet>
\end{html}
\fi
}

\newcommand{\MScriptPage}[2]{
% Bindet eine JavaScript-Datei ein, die eine eigene Seite bekommt
% (wird nur im HTML, aber nicht im PDF erstellt)
% #1 Dateiname des Programms (sollte mit ".js" enden)
% #2 = Kurztitel der Seite
\ifttm
\begin{MSContent}{#2}{#2}{puzzle}
\MRegisterFile{#1}
\begin{html}
<script src="\MMaterial/\MLastFile" type="text/javascript"></script>
\end{html}
\end{MSContent}
\fi
}

\newcommand{\MIncludePSE}{
% Bindet bei Chemie-Modulen das PSE ein
% (wird nur im HTML, aber nicht im PDF erstellt)
\ifttm
\special{html:<!-- includepse //-->}
\begin{MSContent}{Periodensystem der Elemente}{PSE}{table}
\MRegisterFile{../files/pse.js}
\MRegisterFile{../files/radio.png}
\begin{html}
<script src="\MMaterial/../files/pse.js" type="text/javascript"></script>
<p id="divid"><br /><br />
<script language="javascript" type="text/javascript">
    startpse("divid","\MMaterial/../files"); 
</script>
</p>
<br />
<br />
<br />
<p>Die Farben der Elementsymbole geben an: <font style="color:Red">gasf&ouml;rmig </font> <font style="color:Blue">fl&uuml;ssig </font> fest</p>
<p>Die Elemente der Gruppe 1 A, 2 A, 3 A usw. geh&ouml;ren zu den Hauptgruppenelementen.</p>
<p>Die Elemente der Gruppe 1 B, 2 B, 3 B usw. geh&ouml;ren zu den Nebengruppenelementen.</p>
<p>() kennzeichnet die Masse des stabilsten Isotops</p>
\end{html}
\end{MSContent}
\fi
}

\newcommand{\MAppletArchive}[4]{
% Bindet ein Java-Applet ein, die Parameter sind:
% (wird nur im HTML, aber nicht im PDF erstellt)
% #1 Dateiname der Klasse mit Appletaufruf (muss mit ".class" enden)
% #2 Dateiname des Archivs (muss mit ".jar" enden)
% #3 = Breite in Pixeln
% #4 = Hoehe in Pixeln
\ifttm
\MRegisterFile{#2}
\begin{html}
<applet code="#1" archive="\end{html}\MMaterial/\MLastFile\begin{html}" codebase="." width="#3" height="#4" alt="[Java-Archiv kann nicht gestartet werden]"></applet>
\end{html}
\fi
}

% Bindet in der Haupttexdatei ein MINT-Modul ein. Parameter 1 ist das Verzeichnis (relativ zur Haupttexdatei), Parameter 2 ist der Dateinahme ohne Pfad.
\newcommand{\IncludeModule}[2]{
\renewcommand{\MDPrefix}{#1}
\input{#1/#2}
\ifnum\value{MSSEnd}>0{\MSubsectionEndMacros}\addtocounter{MSSEnd}{-1}\fi
}

% Der ttm-Konverter setzt keine Makros im \input um, also muss hier getrickst werden:
% Das MDPrefix muss in den einzelnen Modulen manuell eingesetzt werden
\newcommand{\MInputFile}[1]{
\ifttm
\input{#1}
\else
\input{#1}
\fi
}


\newcommand{\MCases}[1]{\left\lbrace{\begin{array}{rl} #1 \end{array}}\right.}

\ifttm
\newenvironment{MCaseEnv}{\left\lbrace\begin{array}{rl}}{\end{array}\right.}
\else
\newenvironment{MCaseEnv}{\left\lbrace\begin{array}{rl}}{\end{array}\right.}
\fi

\def\MSkip{\ifttm\MCR\fi}

\ifttm
\def\MCR{\special{html:<br />}}
\else
\def\MCR{\ \\}
\fi


% Pragmas - Sind Schluesselwoerter, die dem Preprocessing sowie dem Konverter uebergeben werden und bestimmte
%           Aktionen ausloesen. Im Output (PDF und HTML) tauchen sie nicht auf.
\newcommand{\MPragma}[1]{%
\ifttm%
\special{html:<!-- mpragma;-;}#1\special{html:;; -->}%
\else%
% MPragmas werden vom Preprozessor direkt im LaTeX gefunden
\fi%
}

% Ersatz der Befehle textsubscript und textsuperscript, die ttm nicht kennt
\ifttm%
\newcommand{\MTextsubscript}[1]{\special{html:<sub>}#1\special{html:</sub>}}%
\newcommand{\MTextsuperscript}[1]{\special{html:<sup>}#1\special{html:</sup>}}%
\else%
\newcommand{\MTextsubscript}[1]{\textsubscript{#1}}%
\newcommand{\MTextsuperscript}[1]{\textsuperscript{#1}}%
\fi

%------------------ Einbindung von dia-Diagrammen ----------------------------------------------
% Beim preprocessing wird aus jeder dia-Datei eine tex-Datei und eine pdf-Datei erzeugt,
% diese werden hier jeweils im PDF und HTML eingebunden
% Parameter: Dateiname der mit dia erstellten Datei (OHNE die Endung .dia)
\ifttm%
\newcommand{\MDia}[1]{%
\MGraphicsSolo{#1minthtml.png}{}%
}
\else%
\newcommand{\MDia}[1]{%
\MGraphicsSolo{#1mintpdf.png}{scale=0.1667}%
}
\fi%

% subsup funktioniert im Ausdruck $D={\R}^+_0$, also \R geklammert und sup zuerst
% \ifttm
% \def\MSubsup#1#2#3{\special{html:<msubsup>} #1 #2 #3\special{html:</msubsup>}}
% \else
% \def\MSubsup#1#2#3{{#1}^{#3}_{#2}}
% \fi

%\input{local.tex}

% \ifttm
% \else
% \newwrite\mintlog
% \immediate\openout\mintlog=mintlog.txt
% \fi

% ----------------------- tikz autogenerator -------------------------------------------------------------------

\newcommand{\Mtikzexternalize}{\tikzexternalize}% wird bei Konvertierung ueber mconvert ggf. ausgehebelt!

\ifttm
\else
\tikzset%
{
  % Defines a custom style which generates pdf and converts to (low and hi-res quality) png and svg, then deletes the pdf
  % Important: DO NOT directly convert from pdf to hires-png or from svg to png with GraphViz convert as it has some problems and memory leaks
  png export/.style=%
  {
    external/system call/.add={}{; 
      pdf2svg "\image.pdf" "\image.svg" ; 
      convert -density 112.5 -transparent white "\image.pdf" "\image.png"; 
      inkscape --export-png="\image.4x.png" --export-dpi=450 --export-background-opacity=0 --without-gui "\image.svg"; 
      rm "\image.pdf"; rm "\image.log"; rm "\image.dpth"; rm "\image.idx"
    },
    external/force remake,
  }
}
\tikzset{png export}
\tikzsetexternalprefix{}
% PNGs bei externer Erzeugung in "richtiger" Groesse einbinden
\pgfkeys{/pgf/images/include external/.code={\includegraphics[scale=0.64]{#1}}}
\fi

% Spezielle Umgebung fuer Autogenerierung, Bildernamen sind nur innerhalb eines Moduls (einer MSection) eindeutig)

\newcommand{\MTIKZautofilename}{tikzautofile}

\ifttm
% HTML-Version: Vom Autogenerator erzeugte png-Datei einbinden, tikz selbst nicht ausfuehren (sprich: #1 schlucken)
\newcommand{\MTikzAuto}[1]{%
\addtocounter{MTIKZAutofilenumber}{1}
\renewcommand{\MTIKZautofilename}{mtikzauto_\arabic{MTIKZAutofilenumber}}
\MUGraphicsSolo{\MSectionID\MTIKZautofilename.4x.png}{scale=1}{\special{html:[[!-- svgstyle;}\MSectionID\MTIKZautofilename\special{html: //--]]}} % Styleinfos werden aus original-png, nicht 4x-png geholt!
%\MRegisterFile{\MSectionID\MTIKZautofilename.png} % not used right now
%\MRegisterFile{\MSectionID\MTIKZautofilename.svg}
}
\else%
% PDF-Version: Falls Autogenerator aktiv wird Datei automatisch benannt und exportiert
\newcommand{\MTikzAuto}[1]{%
\addtocounter{MTIKZAutofilenumber}{1}%
\renewcommand{\MTIKZautofilename}{mtikzauto_\arabic{MTIKZAutofilenumber}}
\tikzsetnextfilename{\MTIKZautofilename}%
#1%
}
\fi

% In einer reinen LaTeX-Uebersetzung kapselt der Preambelinclude-Befehl nur input,
% in einer konvertergesteuerten PDF/HTML-Uebersetzung wird er dagegen entfernt und
% die Preambeln an mintmod angehaengt, die Ersetzung wird von mconvert.pl vorgenommen.

\newcommand{\MPreambleInclude}[1]{\input{#1}}

% Globale Watermarksettings (werden auch nochmal zu Beginn jedes subsection gesetzt,
% muessen hier aber auch global ausgefuehrt wegen Einfuehrungsseiten und Inhaltsverzeichnis

\MWatermarkSettings
% ---------------------------------- Parametrisierte Aufgaben ----------------------------------------

\ifttm
\newenvironment{MPExercise}{%
\begin{MExercise}%
}{%
\special{html:<button name="Name_MPEX}\arabic{MExerciseCounter}\special{html:" id="MPEX}\arabic{MExerciseCounter}%
\special{html:" type="button" onclick="reroll('}\arabic{MExerciseCounter}\special{html:');">Neue Aufgabe erzeugen</button>}%
\end{MExercise}%
}
\else
\newenvironment{MPExercise}{%
\begin{MExercise}%
}{%
\end{MExercise}%
}
\fi

% Parameter: Name, Min, Max, PDF-Standard. Name in Deklaration OHNE backslash, im Code MIT Backslash
\ifttm
\newcommand{\MGlobalInteger}[4]{\special{html:%
<!-- onloadstart //-->%
MVAR.push(createGlobalInteger("}#1\special{html:",}#2\special{html:,}#3\special{html:,}#4\special{html:)); %
<!-- onloadstop //-->%
<!-- viewmodelstart //-->%
ob}#1\special{html:: ko.observable(rerollMVar("}#1\special{html:")),%
<!-- viewmodelstop //-->%
}%
}%
\else%
\newcommand{\MGlobalInteger}[4]{\newcounter{mvc_#1}\setcounter{mvc_#1}{#4}}
\fi

% Parameter: Name, Min, Max, PDF-Standard. Name in Deklaration OHNE backslash, im Code MIT Backslash, Wert ist Wurzel von value
\ifttm
\newcommand{\MGlobalSqrt}[4]{\special{html:%
<!-- onloadstart //-->%
MVAR.push(createGlobalSqrt("}#1\special{html:",}#2\special{html:,}#3\special{html:,}#4\special{html:)); %
<!-- onloadstop //-->%
<!-- viewmodelstart //-->%
ob}#1\special{html:: ko.observable(rerollMVar("}#1\special{html:")),%
<!-- viewmodelstop //-->%
}%
}%
\else%
\newcommand{\MGlobalSqrt}[4]{\newcounter{mvc_#1}\setcounter{mvc_#1}{#4}}% Funktioniert nicht als Wurzel !!!
\fi

% Parameter: Name, Min, Max, PDF-Standard zaehler, PDF-Standard nenner. Name in Deklaration OHNE backslash, im Code MIT Backslash
\ifttm
\newcommand{\MGlobalFraction}[5]{\special{html:%
<!-- onloadstart //-->%
MVAR.push(createGlobalFraction("}#1\special{html:",}#2\special{html:,}#3\special{html:,}#4\special{html:,}#5\special{html:)); %
<!-- onloadstop //-->%
<!-- viewmodelstart //-->%
ob}#1\special{html:: ko.observable(rerollMVar("}#1\special{html:")),%
<!-- viewmodelstop //-->%
}%
}%
\else%
\newcommand{\MGlobalFraction}[5]{\newcounter{mvc_#1}\setcounter{mvc_#1}{#4}} % Funktioniert nicht als Bruch !!!
\fi

% MVar darf im HTML nur in MEvalMathDisplay-Umgebungen genutzt werden oder in Strings die an den Parser uebergeben werden
\ifttm%
\newcommand{\MVar}[1]{\special{html:[var_}#1\special{html:]}}%
\else%
\newcommand{\MVar}[1]{\arabic{mvc_#1}}%
\fi

\ifttm%
\newcommand{\MRerollButton}[2]{\special{html:<button type="button" onclick="rerollMVar('}#1\special{html:');">}#2\special{html:</button>}}%
\else%
\newcommand{\MRerollButton}[2]{\relax}% Keine sinnvolle Entsprechung im PDF
\fi

% MEvalMathDisplay fuer HTML wird in mconvert.pl im preprocessing realisiert
% PDF: eine equation*-Umgebung (ueber amsmath)
% HTML: Eine Mathjax-Tex-Umgebung, deren Auswertung mit knockout-obervablen gekoppelt ist
% PDF-Version hier nur fuer pdflatex-only-Uebersetzung gegeben

\ifttm\else\newenvironment{MEvalMathDisplay}{\begin{equation*}}{\end{equation*}}\fi

% ---------------------------------- Spezialbefehle fuer AD ------------------------------------------

%Abk�rzung f�r \longrightarrow:
\newcommand{\lto}{\ensuremath{\longrightarrow}}

%Makro f�r Funktionen:
\newcommand{\exfunction}[5]
{\begin{array}{rrcl}
 #1 \colon  & #2 &\lto & #3 \\[.05cm]  
  & #4 &\longmapsto  & #5 
\end{array}}

\newcommand{\function}[5]{%
#1:\;\left\lbrace{\begin{array}{rcl}
 #2 &\lto & #3 \\
 #4 &\longmapsto  & #5 \end{array}}\right.}


%Die Identit�t:
\DeclareMathOperator{\Id}{Id}

%Die Signumfunktion:
\DeclareMathOperator{\sgn}{sgn}

%Zwei Betonungskommandos (k�nnen angepasst werden):
\newcommand{\highlight}[1]{#1}
\newcommand{\modstextbf}[1]{#1}
\newcommand{\modsemph}[1]{#1}


% ---------------------------------- Spezialbefehle fuer JL ------------------------------------------


\def\jccolorfkt{green!50!black} %Farbe des Funktionsgraphen
\def\jccolorfktarea{green!25!white} %Farbe der Fl"ache unter dem Graphen
\def\jccolorfktareahell{green!12!white} %helle Einf"arbung der Fl"ache unter dem Graphen
\def\jccolorfktwert{green!50!black} %Farbe einzelner Punkte des Graphen

\newcommand{\MPfadBilder}{Bilder}

\ifttm%
\newcommand{\jMD}{\,\MD}%
\else%
\newcommand{\jMD}{\;\MD}%
\fi%

\def\jHTMLHinweisBedienung{\MInputHint{%
Mit Hilfe der Symbole am oberen Rand des Fensters
k"onnen Sie durch die einzelnen Abschnitte navigieren.}}

\def\jHTMLHinweisEingabeText{\MInputHint{%
Geben Sie jeweils ein Wort oder Zeichen als Antwort ein.}}

\def\jHTMLHinweisEingabeTerm{\MInputHint{%
Klammern Sie Ihre Terme, um eine eindeutige Eingabe zu erhalten. 
Beispiel: Der Term $\frac{3x+1}{x-2}$ soll in der Form
\texttt{(3*x+1)/((x+2)^2}$ eingegeben werden (wobei auch Leerzeichen 
eingegeben werden k"onnen, damit eine Formel besser lesbar ist).}}

\def\jHTMLHinweisEingabeIntervalle{\MInputHint{%
Intervalle werden links mit einer "offnenden Klammer und rechts mit einer 
schlie"senden Klammer angegeben. Eine runde Klammer wird verwendet, wenn der 
Rand nicht dazu geh"ort, eine eckige, wenn er dazu geh"ort. 
Als Trennzeichen wird ein Komma oder ein Semikolon akzeptiert.
Beispiele: $(a, b)$ offenes Intervall,
$[a; b)$ links abgeschlossenes, rechts offenes Intervall von $a$ bis $b$. 
Die Eingabe $]a;b[$ f"ur ein offenes Intervall wird nicht akzeptiert.
F"ur $\infty$ kann \texttt{infty} oder \texttt{unendlich} geschrieben werden.}}

\def\jHTMLHinweisEingabeFunktionen{\MInputHint{%
Schreiben Sie Malpunkte (geschrieben als \texttt{*}) aus und setzen Sie Klammern um Argumente f�r Funktionen.
Beispiele: Polynom: \texttt{3*x + 0.1}, Sinusfunktion: \texttt{sin(x)}, 
Verkettung von cos und Wurzel: \texttt{cos(sqrt(3*x))}.}}

\def\jHTMLHinweisEingabeFunktionenSinCos{\MInputHint{%
Die Sinusfunktion $\sin x$ wird in der Form \texttt{sin(x)} angegeben, %
$\cos\left(\sqrt{3 x}\right)$ durch \texttt{cos(sqrt(3*x))}.}}

\def\jHTMLHinweisEingabeFunktionenExp{\MInputHint{%
Die Exponentialfunktion $\MEU^{3x^4 + 5}$ wird als
\texttt{exp(3 * x^4 + 5)} angegeben, %
$\ln\left(\sqrt{x} + 3.2\right)$ durch \texttt{ln(sqrt(x) + 3.2)}.}}

% ---------------------------------- Spezialbefehle fuer Fachbereich Physik --------------------------

\newcommand{\E}{{e}}
\newcommand{\ME}[1]{\cdot 10^{#1}}
\newcommand{\MU}[1]{\;\mathrm{#1}}
\newcommand{\MPG}[3]{%
  \ifnum#2=0%
    #1\ \mathrm{#3}%
  \else%
    #1\cdot 10^{#2}\ \mathrm{#3}%
  \fi}%
%

\newcommand{\MMul}{\MExponentensymbXYZl} % Nur eine Abkuerzung


% ---------------------------------- Stichwortfunktionialitaet ---------------------------------------

% mpreindexentry wird durch Auswahlroutine in conv.pl durch mindexentry substitutiert
\ifttm%
\def\MIndex#1{\index{#1}\special{html:<!-- mpreindexentry;;}#1\special{html:;;}\arabic{MSubjectArea}\special{html:;;}%
\arabic{chapter}\special{html:;;}\arabic{section}\special{html:;;}\arabic{subsection}\special{html:;;}\arabic{MEntryCounter}\special{html:; //-->}%
\setcounter{MLastIndex}{\value{MEntryCounter}}%
\addtocounter{MEntryCounter}{1}%
}%
% Copyrightliste wird als tex-Datei im preprocessing von conv.pl erzeugt und unter converter/tex/entrycollection.tex abgelegt
% Der input-Befehl funktioniert nur, wenn die aufrufende tex-Datei auf der obersten Ebene liegt (d.h. selbst kein input/include ist, insbesondere keine Moduldatei)
\def\MEntryList{} % \input funktioniert nicht, weil ttm (und damit das \input) ausgefuehrt wird, bevor Datei da ist
\else%
\def\MIndex#1{\index{#1}}
\def\MEntryList{\MAbort{Stichwortliste nur im HTML realisierbar}}%
\fi%

\def\MEntry#1#2{\textbf{#1}\MIndex{#2}} % Idee: MLastType auf neuen Entry-Typ und dann ein MLabel vergeben mit autogen-Nummer

% ---------------------------------- Befehle fuer Tests ----------------------------------------------

% MEquationItem stellt eine Eingabezeile der Form Vorgabe = Antwortfeld her, der zweite Parameter kann z.B. MSimplifyQuestion-Befehl sein
\ifttm
\newcommand{\MEquationItem}[2]{{#1}$\,=\,${#2}}%
\else%
\newcommand{\MEquationItem}[2]{{#1}$\;\;=\,${#2}}%
\fi

\ifttm
\newcommand{\MInputHint}[1]{%
\ifnum%
\if\value{MTestSite}>0%
\else%
{\color{blue}#1}%
\fi%
\fi%
}
\else
\newcommand{\MInputHint}[1]{\relax}
\fi

\ifttm
\newcommand{\MInTestHeader}{%
Dies ist ein einreichbarer Test:
\begin{itemize}
\item{Im Gegensatz zu den offenen Aufgaben werden beim Eingeben keine Hinweise zur Formulierung der mathematischen Ausdr�cke gegeben.}
\item{Der Test kann jederzeit neu gestartet oder verlassen werden.}
\item{Der Test kann durch die Buttons am Ende der Seite beendet und abgeschickt, oder zur�ckgesetzt werden.}
\item{Der Test kann mehrfach probiert werden. F�r die Statistik z�hlt die zuletzt abgeschickte Version.}
\end{itemize}
}
\else
\newcommand{\MInTestHeader}{%
\relax
}
\fi

\ifttm
\newcommand{\MInTestFooter}{%
\special{html:<button name="Name_TESTFINISH" id="TESTFINISH" type="button" onclick="finish_button('}\MTestName\special{html:');">Test auswerten</button>}%
\begin{html}
&nbsp;&nbsp;&nbsp;&nbsp;&nbsp;&nbsp;&nbsp;&nbsp;
<button name="Name_TESTRESET" id="TESTRESET" type="button" onclick="reset_button();">Test zur�cksetzen</button>
<br />
<br />
<div class="xreply">
<p name="Name_TESTEVAL" id="TESTEVAL">
Hier erscheint die Testauswertung!
<br />
</p>
</div>
\end{html}
}
\else
\newcommand{\MInTestFooter}{%
\relax
}
\fi


% ---------------------------------- Notationsmakros -------------------------------------------------------------

% Notationsmakros die nicht von der Kursvariante abhaengig sind

\newcommand{\MZahltrennzeichen}[1]{\renewcommand{\MZXYZhltrennzeichen}{#1}}

\ifttm
\newcommand{\MZahl}[3][\MZXYZhltrennzeichen]{\edef\MZXYZtemp{\noexpand\special{html:<mn>#2#1#3</mn>}}\MZXYZtemp}
\else
\newcommand{\MZahl}[3][\MZXYZhltrennzeichen]{{}#2{#1}#3}
\fi

\newcommand{\MEinheitenabstand}[1]{\renewcommand{\MEinheitenabstXYZnd}{#1}}
\ifttm
\newcommand{\MEinheit}[2][\MEinheitenabstXYZnd]{{}#1\edef\MEINHtemp{\noexpand\special{html:<mi mathvariant="normal">#2</mi>}}\MEINHtemp} 
\else
\newcommand{\MEinheit}[2][\MEinheitenabstXYZnd]{{}#1 \mathrm{#2}} 
\fi

\newcommand{\MExponentensymbol}[1]{\renewcommand{\MExponentensymbXYZl}{#1}}
\newcommand{\MExponent}[2][\MExponentensymbXYZl]{{}#1{} 10^{#2}} 

%Punkte in 2 und 3 Dimensionen
\newcommand{\MPointTwo}[3][]{#1(#2\MCoordPointSep #3{}#1)}
\newcommand{\MPointThree}[4][]{#1(#2\MCoordPointSep #3\MCoordPointSep #4{}#1)}
\newcommand{\MPointTwoAS}[2]{\left(#1\MCoordPointSep #2\right)}
\newcommand{\MPointThreeAS}[3]{\left(#1\MCoordPointSep #2\MCoordPointSep #3\right)}

% Masseinheit, Standardabstand: \,
\newcommand{\MEinheitenabstXYZnd}{\MThinspace} 

% Horizontaler Leerraum zwischen herausgestellter Formel und Interpunktion
\ifttm
\newcommand{\MDFPSpace}{\,}
\newcommand{\MDFPaSpace}{\,\,}
\newcommand{\MBlank}{\ }
\else
\newcommand{\MDFPSpace}{\;}
\newcommand{\MDFPaSpace}{\;\;}
\newcommand{\MBlank}{\ }
\fi

% Satzende in herausgestellter Formel mit horizontalem Leerraum
\newcommand{\MDFPeriod}{\MDFPSpace .}

% Separation von Aufzaehlung und Bedingung in Menge
\newcommand{\MCondSetSep}{\,:\,} %oder '\mid'

% Konverter kennt mathopen nicht
\ifttm
\def\mathopen#1{}
\fi

% -----------------------------------START Rouletteaufgaben ------------------------------------------------------------

\ifttm
% #1 = Dateiname, #2 = eindeutige ID fuer das Roulette im Kurs
\newcommand{\MDirectRouletteExercises}[2]{
\begin{MExercise}
\texttt{Im HTML erscheinen hier Aufgaben aus einer Aufgabenliste...}
\end{MExercise}
}
\else
\newcommand{\MDirectRouletteExercises}[2]{\relax} % wird durch mconvert.pl gefunden und ersetzt
\fi


% ---------------------------------- START Makros, die von der Kursvariante abhaengen ----------------------------------

\ifvariantunotation
  % unotation = An Universitaeten uebliche Notation
  \def\MVariant{unotation}

  % Trennzeichen fuer Dezimalzahlen
  \newcommand{\MZXYZhltrennzeichen}{.}

  % Exponent zur Basis 10 in der Exponentialschreibweise, 
  % Standardmalzeichen: \times
  \newcommand{\MExponentensymbXYZl}{\times} 

  % Begrenzungszeichen fuer offene Intervalle
  \newcommand{\MoIl}[1][]{\mbox{}#1(\mathopen{}} % bzw. ']'
  \newcommand{\MoIr}[1][]{#1)\mbox{}} % bzw. '['

  % Zahlen-Separation im IntervaLL
  \newcommand{\MIntvlSep}{,} %oder ';'

  % Separation von Elementen in Mengen
  \newcommand{\MElSetSep}{,} %oder ';'

  % Separation von Koordinaten in Punkten
  \newcommand{\MCoordPointSep}{,} %oder ';' oder '|', '\MThinspace|\MThinspace'

\else
  % An dieser Stelle wird angenommen, dass std-Variante aktiv ist
  % std = beschlossene Notation im TU9-Onlinekurs 
  \def\MVariant{std}

  % Trennzeichen fuer Dezimalzahlen
  \newcommand{\MZXYZhltrennzeichen}{,}

  % Exponent zur Basis 10 in der Exponentialschreibweise, 
  % Standardmalzeichen: \times
  \newcommand{\MExponentensymbXYZl}{\times} 

  % Begrenzungszeichen fuer offene Intervalle
  \newcommand{\MoIl}[1][]{\mbox{}#1]\mathopen{}} % bzw. '('
  \newcommand{\MoIr}[1][]{#1[\mbox{}} % bzw. ')'

  % Zahlen-Separation im IntervaLL
  \newcommand{\MIntvlSep}{;} %oder ','
  
  % Separation von Elementen in Mengen
  \newcommand{\MElSetSep}{;} %oder ','

  % Separation von Koordinaten in Punkten
  \newcommand{\MCoordPointSep}{;} %oder '|', '\MThinspace|\MThinspace'

\fi



% ---------------------------------- ENDE Makros, die von der Kursvariante abhaengen ----------------------------------


% diese Kommandos setzen Mathemodus vorraus
\newcommand{\MGeoAbstand}[2]{[\overline{{#1}{#2}}]}
\newcommand{\MGeoGerade}[2]{{#1}{#2}}
\newcommand{\MGeoStrecke}[2]{\overline{{#1}{#2}}}
\newcommand{\MGeoDreieck}[3]{{#1}{#2}{#3}}

%
\ifttm
\newcommand{\MOhm}{\special{html:<mn>&#x3A9;</mn>}}
\else
\newcommand{\MOhm}{\Omega} %\varOmega
\fi


\def\PERCTAG{\MAbort{PERCTAG ist zur internen verwendung in mconvert.pl reserviert, dieses Makro darf sonst nicht benutzt werden.}}

% Im Gegensatz zu einfachen html-Umgebungen werden MDirectHTML-Umgebungen von mconvert.pl am ganzen ttm-Prozess vorbeigeschleust und aus dem PDF komplett ausgeschnitten
\ifttm%
\newenvironment{MDirectHTML}{\begin{html}}{\end{html}}%
\else%
\newenvironment{MDirectHTML}{\begin{html}}{\end{html}}%
\fi

% Im Gegensatz zu einfachen Mathe-Umgebungen werden MDirectMath-Umgebungen von mconvert.pl am ganzen ttm-Prozess vorbeigeschleust, ueber MathJax realisiert, und im PDF als $$ ... $$ gesetzt
\ifttm%
\newenvironment{MDirectMath}{\begin{html}}{\end{html}}%
\else%
\newenvironment{MDirectMath}{\begin{equation*}}{\end{equation*}}% Vorsicht, auch \[ und \] werden in amsmath durch equation* redefiniert
\fi

% ---------------------------------- Location Management ---------------------------------------------

% #1 = buttonname (muss in files/images liegen und Format 48x48 haben), #2 = Vollstaendiger Einrichtungsname, #3 = Kuerzel der Einrichtung,  #4 = Name der include-texdatei
\ifttm
\newcommand{\MLocationSite}[3]{\special{html:<!-- mlocation;;}#1\special{html:;;}#2\special{html:;;}#3\special{html:;; //-->}}
\else
\newcommand{\MLocationSite}[3]{\relax}
\fi

% ---------------------------------- Copyright Management --------------------------------------------

\newcommand{\MCCLicense}{%
{\color{green}\textbf{CC BY-SA 3.0}}
}

\newcommand{\MCopyrightLabel}[1]{ (\MSRef{L_COPYRIGHTCOLLECTION}{Lizenz})\MLabel{#1}}

% Copyrightliste wird als tex-Datei im preprocessing erzeugt und unter converter/tex/copyrightcollection.tex abgelegt
% Der input-Befehl funktioniert nur, wenn die aufrufende tex-Datei auf der obersten Ebene liegt (d.h. selbst kein input/include ist, insbesondere keine Moduldatei)
\newcommand{\MCopyrightCollection}{\input{copyrightcollection.tex}}

% MCopyrightNotice fuegt eine Copyrightnotiz ein, der parser ersetzt diese durch CopyrightNoticePOST im preparsing, diese Definition wird nur fuer reine pdflatex-Uebersetzungen gebraucht
% Parameter: #1: Kurze Lizenzbeschreibung (typischerweise \MCCLicense)
%            #2: Link zum Original (http://...) oder NONE falls das Bild selbst ein Original ist, oder TIKZ falls das Bild aus einer tikz-Umgebung stammt
%            #3: Link zum Autor (http://...) oder MINT falls Original im MINT-Kolleg erstellt oder NONE falls Autor unbekannt
%            #4: Bemerkung (z.B. dass Datei mit Maple exportiert wurde)
%            #5: Labelstring fuer existierendes Label auf das copyrighted Objekt, mit MCopyrightLabel erzeugt
%            Keines der Felder darf leer sein!
\newcommand{\MCopyrightNotice}[5]{\MCopyrightNoticePOST{#1}{#2}{#3}{#4}{#5}}

\ifttm%
\newcommand{\MCopyrightNoticePOST}[5]{\relax}%
\else%
\newcommand{\MCopyrightNoticePOST}[5]{\relax}%
\fi%

% ---------------------------------- Meldungen fuer den Benutzer des Konverters ----------------------
\MPragma{mintmodversion;P0.1.0}
\MPragma{usercomment;This is file mintmod.tex version P0.1.0}


% ----------------------------------- Spezialelemente fuer Konfigurationsseite, werden nicht von mintscripts.js verwaltet --

% #1 = DOM-id der Box
\ifttm\newcommand{\MConfigbox}[1]{\special{html:<input cfieldtype="2" type="checkbox" name="Name_}#1\special{html:" id="}#1\special{html:" onchange="confHandlerChange('}#1\special{html:');"/>}}\fi % darf im PDF nicht aufgerufen werden!


\MPragma{MathSkip}

%\Mtikzexternalize

\begin{document}
%\input{mintmod.tex}
\MPragma{MathSkip}

%\Mtikzexternalize

\begin{document}
%\include{vbkm03_eng}
%\include{vbkm02_eng}
\MSection{Inequalities in one Variable}
\MLabel{VBKM03}
\MSetSectionID{ungl}

\begin{MSectionStart}
\MDeclareSiteUXID{VBKM03_START}

Inequalities arise by relating terms using one of the comparing symbols $\leq$, $<$, $\geq$, or $>$. Simple 
inequalities usually have intervals as their solution sets. But the solution of inequalities is often
more difficult than the solution of equations. Hence, specific types of inequalities will be explained
in more detail.

This module consists of:

\begin{itemize}
\item{Section~\MNRef{M03_Ungleichungen}: \MSRef{M03_Ungleichungen}{Inequalities and their Solution Sets},}
\item{Section~\MNRef{M03_Umformen}: \MSRef{M03_Umformen}{Transformation of Inequalities},}
\item{Section~\MNRef{M03_Betragsungleichungen}: \MSRef{M03_Betragsungleichungen}{Absolute Value Inequalities and 
Quadratic Inequalities},}
\item{and Section~\MNRef{M03_Abschlusstest}: \MSRef{M03_Abschlusstest}{Final Test}.}
\end{itemize}

\end{MSectionStart}

\MSubsection{Inequalities and their Solution Sets}
\MLabel{M03_Ungleichungen}

\begin{MIntro}

\begin{MInfo}
\MDeclareSiteUXID{VBKM03_UngleichungenIntro}
If two numbers are related by one of the \MEntry{comparing symbols}{comparing symbols} 
$\leq$, $<$, $\geq$, or $>$, a statement is generated that can be true or false depending on 
the numbers:
\begin{itemize}
\item{$a<b$ (reads: ``$a$ is strictly less than $b$'' or simply ``$a$ is less than $b$'') is true if the number $a$ is less than and not equal to $b$.}
\item{$a \leq b$ (reads: ``$a$ is less than $b$'') is true if $a$ is less than or equal to $b$.}
\item{$a>b$ (reads: ``$a$ is strictly greater than $b$'' or simply ``$a$ is greater than $b$'') is true if the number $a$ is greater and not equal to $b$.}
\item{$a \geq b$ (reads: ``$a$ is greater than $b$'') is true if the number $a$ is greater than or equal to $b$.}
\end{itemize}
\end{MInfo}

The relating symbols describe how the given values are related to each other on the number line: 
$a<b$ means that $a$ is to the left of $b$ on the number line.

\begin{MExample}
The statements $2<4$, $-12\leq 2$, $4>1$, and $3\geq 3$ are true,
but the statements $2<\sqrt2$ and $3>3$ are false.

\begin{center}
\MTikzAuto{%
\begin{tikzpicture}
% reelle Achse
\draw[->,color=black] (-1,0.0) -- (5,0.0);
\foreach \x in {-1, 0, 1, 2, 3, 4}
\draw[shift={(\x,0)},color=black] (0pt,2pt) -- (0pt,-2pt) node[below] {\footnotesize $\x$};
\draw (4.9,-0.3) node[] {$\mathbb{R}$};
% Enden:
\draw [fill = blue] (2,0) circle (1.5pt);
\draw [fill = blue] (4,0) circle (1.5pt);
\end{tikzpicture}
}%

On the number line, the number $2$ is to the left of the number $4$, thus $2<4$.
\end{center}

\end{MExample}

Here, $a<b$ means the same as $b>a$, likewise $a\leq b$ means the same as $b\geq a$. But it
should be noted that the opposite of the statement $a<b$ is the statement $a\geq b$ and not
$a>b$. If terms with a variable occur in an inequality, the problem is to find the number range
of the variable such that the inequality is true. 
\end{MIntro}

\begin{MXContent}{Solving simple Inequalities}{Solving}{STD}
\MDeclareSiteUXID{VBKM03_EinfacheUngleichungen}

If the variable occurs isolated in the inequality, the solution set is an interval, see also info
box \MRef{VBKM01_Intervalle}: 


\begin{MInfo}
\MLabel{M03_Aufloesungen}
The \MEntry{solved inequalities}{inequalities (solved)} 
have the following \MEntry{intervals}{intervals} as their solution sets:
\begin{itemize}
\item{$x< a$ has the solution set $\MoIl[\left] -\infty\MIntvlSep a\MoIr[\right]$, i.e.\ all $x$ less than $a$.}
\item{$x\leq a$ has the solution set $\MoIl[\left] -\infty\MIntvlSep a\right]$, 
i.e. all $x$ less than or equal to $a$.}
\item{$x> a$ has the solution set $\MoIl[\left] a\MIntvlSep \infty\MoIr[\right]$, i.e.\ all
 $x$ greater than $a$.}
\item{$x\geq a$ has the solution set $\left[a\MIntvlSep \infty\MoIr[\right]$, i.e.\ all $x$
greater than or equal to $a$.}
\end{itemize}
Here, $x$ is the variable and $a$ is a specific value. 
If the variable does not occur in the inequality anymore, the solution set is either
$\R=\MoIl[\left] -\infty\MIntvlSep \infty\MoIr[\right]$ if the inequality is satisfied, 
or the empty set $\lbrace \rbrace$ if the inequality is not satisfied.
\end{MInfo}

The symbol $\infty$ means \MEntry{infinity}{infinity}. A finite interval has the form 
$\MoIl a\MIntvlSep b\MoIr$ which reads ``all numbers between $a$ and $b$''. If the interval
is to be finite only on one side, the other interval boundary can be replaced by the symbol 
$\infty$ (right-hand side) or $-\infty$ (left-hand side).

As for equations one tries to find a solved inequality by applying transformations that do
not change the solution set. From the solved inequality the solution set can be easily seen.


\begin{MInfo}
\MLabel{VBKM03_AequivalenzumformungenUngleichungen}
To obtain a solved inequality from an unsolved inequality the following 
\MEntry{equivalent transformations}{equivalent transformations (inequality)} are allowed:
\begin{itemize}
\item{adding a constant to both sides of the inequality: $a<b$ is equivalent to $a+c<b+c$.}
\item{multiplying both sides of the inequality by a positive constant: $a<b$ 
is equivalent to $a\cdot c<b\cdot c$ if $c>0$.}
\item{multiplying both sides of the inequality by a negative constant and inverting the 
comparing symbol: $a<b$ is equivalent to $a\cdot c>b\cdot c$ if $c<0$.}
\end{itemize}
\end{MInfo}

\begin{MExample}
The inequality $-\frac34x-\frac12<2$ is solved stepwise by the above transformations:
\begin{eqnarray*}
&&-\frac34x-\frac12 < 2 \;\; \MSep +\frac12\ \\
&\Leftrightarrow&-\frac34x < 2+\frac12 \;\; \MSep \cdot\left({-\frac43}\right)\ \\
&\Leftrightarrow&x > -\frac43\left({2+\frac12}\right) \;\; \MSep \;\text{simplifying}\\
&\Leftrightarrow&x >  -\frac{20}{6} \;=\; -\frac{10}{3} \MDFPeriod
\end{eqnarray*}
So, the initial inequality has the solution set 
 $\MoIl[\left] -\frac{10}{3}\MIntvlSep \infty\MoIr[\right]$. 
Importantly, multiplying the inequality by the negative number $-\frac43$ inverts the 
comparing symbol.
\end{MExample}

\begin{MExercise}
Are the following inequalities true or false?

\begin{MQuestionGroup}
\begin{tabular}{lll}
\MCheckbox{0}{UG1} & \ \ &  $\frac12>1-\frac13$\\
\MCheckbox{1}{UG2} & \ \ & $a^2\geq 2a b-b^2$ (where $a$ and $b$ are unknown numbers)\\
\MCheckbox{1}{UG3} & \ \ & $\frac12<\frac23<\frac34$\\
\MCheckbox{0}{UG4} & \ \ & Let $a<b$, then also $a^2<b^2$.
\end{tabular}
\end{MQuestionGroup}
\MGroupButton{Check input}

\begin{MHint}{Solution}
The first inequality can be simplified to $\frac12>\frac23$, which, after multiplying by $6$, 
is equivalent to $3>4$. This statement is false. The second inequality can be simplified by 
collecting all numbers on the left-hand side: $a^2-2a b+b^2\geq 0$. Since $a^2-2a b+b^2=(a-b)^2$,
this statement is true for all $a$ and $b$. Multiplying the third chain of inequalities by the
least common denominator $12$ results in the chain of inequalities $6<8<9$. This statement is true.
In contrast, the last statement is false, since for example, for $a=-1$ and $b=1$, the term
$a^2=1$ is not less than $b^2=1$. Taking the square of terms is not an equivalent transformation.
\end{MHint}
\end{MExercise}


\begin{MExercise}
Find the solution sets of the following inequalities.
\begin{MExerciseItems}
\item{$2x+1> 3x-1$ has the solution interval \MEquationItem{$\ML$}{\MLIntervalQuestion{30}{(-infty,2)}{5}{TXH1}}.}
\item{$-3x-\frac12\leq x+\frac12$ has the solution interval \MEquationItem{$\ML$}{\MLIntervalQuestion{30}{[-1/4,infty)}{5}{TXH2}}.}
\item{$x-\frac12\leq x+\frac12$ has the solution interval \MEquationItem{$\ML$}{\MLIntervalQuestion{30}{(-infty,infty)}{5}{TXH3}}.}
\end{MExerciseItems}
\MInputHint{Enter the intervals in the form \texttt{(a;b)}, for the interval boundaries also fractions and
\texttt{infinity} or \texttt{-infinity} can be used. Take care whether the interval boundaries are included 
or excluded.}

\begin{MHint}{Solution}
Transformation of the first inequality results in
\begin{eqnarray*}
&& 2x+1 > 3x-1\;\; \MSep+1\ \\
&\Leftrightarrow& 2x+2 > 3x\;\; \MSep-2x\ \\
&\Leftrightarrow&2 > x
\end{eqnarray*}
and hence the solution interval is $\ML=\MoIl[\left] -\infty\MIntvlSep 2\MoIr[\right]$. 
Transformation of the second inequality results in
\begin{eqnarray*}
&&-3x-\frac12\leq x+\frac12 \;\; \MSep +3x-\frac12\ \\
&\Leftrightarrow&-1\leq 4x \;\; \MSep \cdot \frac14\ \\
&\Leftrightarrow&-\frac14\leq  x
\end{eqnarray*}
and hence $\ML=\left[-\frac14\MIntvlSep \infty\MoIr[\right]$. 
Transformation of the third inequality results in
\begin{eqnarray*}
&&x-\frac12\leq x+\frac12\;\; \MSep-x\ \\
&\Leftrightarrow&-\frac12\leq \frac12 \MDFPeriod
\end{eqnarray*}
This statement does not depend on $x\in\R$ and is always true, 
thus the solution set is $\ML=\R=\MoIl[\left] -\infty\MIntvlSep \infty\MoIr[\right]$.
\end{MHint}
\end{MExercise}

\begin{MInfo}
An inequality in one variable $x$ is \MEntry{linear}{inequality (linear)} if on both sides of the 
inequality only multiples of $x$ and constants occur. Each linear inequality can be transformed 
into a solved inequality by one of the equivalent transformations described in the info box
\MRef{M03_Aufloesungen}.
\end{MInfo}

\end{MXContent}

\begin{MXContent}{Specific Transformations}{Specific Transformations}{STD}
\MDeclareSiteUXID{VBKM03_SpezielleUmformungen}
The following equivalent transformations are useful if the variable occurs 
in the denominator of an expression. But they can only be applied under certain
restrictions:

\begin{MInfo}
Under the restriction that none of the occurring denominators is zero (the corresponding variable values are
 never solutions) and the fractions on both sides have the same sign, the reciprocal can be taken
on both sides of the inequality while inverting the comparing symbol.
\end{MInfo}

\begin{MExample}
For example, the inequality $\frac1{2x}\leq \frac1{3x}$ is equivalent to $2x\geq 3x$
(comparing symbol inverted) as long as $x\not=0$. The new inequality has the solution
set $\MoIl[\left] -\infty\MIntvlSep 0\right]$. However, since the value $x=0$ was excluded (and 
does not belong to the domain of the initial inequality either) the solution set of 
$\frac1{2x}\leq \frac1{3x}$ is $\ML=\MoIl[\left] -\infty\MIntvlSep 0\MoIr[\right]$.
\end{MExample}

\begin{MExercise}
Find the solution sets of the following inequalities.
\begin{MExerciseItems}
\item{$\frac1x\geq\frac13$ has the solution set \MEquationItem{$\ML$}{\MLIntervalQuestion{20}{(0;3]}{3}{KKL1}}.}
\item{$\frac1x<\frac1{\sqrt{x}}$ has the solution set \MEquationItem{$\ML$}{\MLIntervalQuestion{20}{(1;infty)}{3}{KKL2}}.}
\end{MExerciseItems}

\begin{MHint}{Solution}
For the first inequality, the value $x=0$ is not in the domain, hence this value is excluded. For 
$x>0$, taking the reciprocal while inverting the comparing symbol is allowed and results in
$x\leq 3$. Together with the condition above the solution interval is $\ML=\MoIl 0\MIntvlSep 3]$.
For $x<0$ the reciprocal rule cannot be applied. However, it can be seen, even without any rule, that
none of the values $x<0$ can be a solution, since then $\frac1x$ is negative as well and not greater than
or equal to $\frac13$.

The domain of the second inequality is $\MoIl 0\MIntvlSep \infty\MoIr$, since 
only for these values of $x$ taking the square root is
defined and only for $x\neq 0$ the denominators are non-zero. On the domain, 
taking the reciprocal while inverting the comparing symbol is 
allowed and results in $x>\sqrt{x}$. Since $\sqrt{x}>0$, the inequality can be 
divided by $\sqrt{x}$ resulting in $\sqrt{x}>1$. This inequality has the solution
set $\ML=\MoIl 1\MIntvlSep \infty\MoIr$ which occurs also in the domain.
\end{MHint}

\end{MExercise}

Please note for the last part of the exercise:

\begin{MInfo}
Taking the square on both sides of an inequality is not an equivalent transformation and 
possibly does change the solution set.
\end{MInfo}

For example, $x=-2$ is no solution of $x>\sqrt{x}$, but indeed a solution of $x^2>x$. However,
this transformation can be applied if the case analysis for the transformation
is carried out correctly and the domain of the initial inequality is taken into account. This method is
described in more detail in the next section.
\end{MXContent}

\MSubsection{Transformation of Inequalities}
\MLabel{M03_Umformen}

\begin{MXContent}{Transformation with Case Analysis}{Case Analysis}{STD}
\MDeclareSiteUXID{VBKM03_UmformungenFallunterscheidungen}
The simple linear transformations described in the previous section are equivalent transformations.
They do not change the solution set of the corresponding inequality. For nonlinear
inequalities advanced solution methods are required. Usually, these methods need 
a case analysis depending on the sign, since, in contrast to the situation for
equations described in Modul~\MNRef{VBKM02}, now also the inequality can be inverted during 
transformation.

 
\begin{MInfo}
If an inequality is multiplied by a term in which the variable $x$ occurs, a case analysis 
is required and for each case the transformation has to be considered separately: 

\begin{itemize}
\item{For those values of $x$, for which the multiplied term is positive, the comparing symbol
of the inequality is unchanged.}
\item{For those values of $x$, for which the multiplied term is negative, the comparing symbol
of the inequality is inverted.}
\item{The case that the multiplied term is zero has to be excluded during the transformation and
has to be considered separately, if necessary.}
\end{itemize}
\ \\ \ \\
The solution sets found in the individual cases have to be checked with respect to the case conditions
as described for the solution of \MSRef{VBKM02_FallBetrag}{absolute value equations}.
\end{MInfo}

In contrast, adding terms in which the variable occurs, does not require a case analysis. Usually, 
transformations involving case analyses are mandatory if the variable occurs in the denominator or 
in a composite term.


\begin{MExample}
The inequality $\frac1{2x}\leq 1$ can be simplified by multiplying both sides of the inequality
by the term $2x$:

\begin{itemize}
\item{Under the condition $x>0$ this results in the new inequality $1\leq 2x$. It has the solution set
$\ML_1=\left[\frac12\MIntvlSep \infty\MoIr[\right]$. The condition $x>0$ is satisfied by all elements of the solution 
set.}
\item{Under the condition $x<0$ this results in the new inequality $1\geq 2x$. It has the solution set
 $\MoIl[\left] -\infty\MIntvlSep \frac12\right]$. Because of the additional
condition $x<0$ only the elements of the set 
$\ML_2=\MoIl[\left] -\infty\MIntvlSep 0\MoIr[\right]$ are solutions.}
\item{The single case $x=0$ is no solution since this value is not in the domain of the inequality. 
In this case multiplying the inequality by $x$ is not allowed.}
\end{itemize}
So, altogether one obtains the union set 
$\ML=\ML_1\cup \ML_2=\R\MSetminus\left[0\MIntvlSep \frac12\MoIr[\right]$ as solution set:
\ \\ \ \\
\begin{center}
\MTikzAuto{%
\begin{tikzpicture}
% reelle Achse
\draw[->,color=black] (-1,0.0) -- (5,0.0);
\foreach \x in {-1, 0, 1, 2, 3, 4}
\draw[shift={(\x,0)},color=black] (0pt,2pt) -- (0pt,-2pt) node[below] {\footnotesize $\x$};
\draw (4.9,-0.3) node[] {$\mathbb{R}$};
% Enden:
\draw [line width=2.0pt,color=blue] (-1,0.0)-- (0,0.0);
\draw [line width=2.0pt,color=blue] (0.5,0.0)-- (5,0.0);
\draw [fill = blue] (0.5,0) circle (1.5pt);
\draw [fill = white] (0,0) circle (1.5pt);
\end{tikzpicture}
}
\end{center}
\end{MExample}

As in Modul~\MNRef{VBKM02} the following statement holds for the solution set.

\begin{MInfo}
The cases have to be chosen such that all elements of the domain of the inequality are covered. 
For the solution set in an individual case, it has to be checked that the solution set satisfies the 
corresponding case condition. For any case, the resulting solution set has to be reduced to
the solution subset satisfying the case condition. The union of the solution sets for the individual cases
is the solution set of the initial inequality.
\end{MInfo}

\end{MXContent}

\begin{MExercises}
\MDeclareSiteUXID{VBKM03_Fallunterscheidungen_Exercises}
If the inequality is multiplied by a composite term, it has to be investigated precisely for which values
of $x$ the case analysis has to be done:

\begin{MExercise}
Find the solution set of the inequality $\frac1{4-2x}<3$. 
The domain of the inequality is $D=\R\MSetminus \lbrace 2\rbrace$ since only for these
values of $x$ the denominator is non-zero. If the inequality is multiplied by the term
$4-2x$, three cases have to be distinguished. Fill in the blanks in the following text 
accordingly:

\begin{MExerciseItems}
\item{On the interval \MLIntervalQuestion{15}{(-infty,2)}{4}{GOM1} the term is positive, the comparing symbol 
remains unchanged, and the new inequality reads $1\:<\:$\MLSimplifyQuestion{15}{3*(4-2*x)}{5}{x}{5}{0}{SIMPLE2}.
Linear transformations result in the solution set 
\MEquationItem{$\ML_1$}{\MLIntervalQuestion{20}{(-infty,11/6)}{3}{MIXGOM}}. 
The elements of this set satisfy the case condition.}
\item{On the interval \MLIntervalQuestion{15}{(2,infty)}{4}{GOM2} the term is negative, 
the comparing symbol is inverted. Initially, the new inequality has the solution set 
\MLIntervalQuestion{20}{(11/6,infty)}{4}{INT1}, because of the case condition only the 
subset \MEquationItem{$\ML_2$}{\MLIntervalQuestion{20}{(2,infty)}{4}{GOM3}} is allowed.}
\item{The single value $x=2$ is no solution of the initial inequality since 
it is not in \MLQuestion{25}{domain}{UGX}.}
\end{MExerciseItems}

Sketch the solution set of the inequality and indicate the boundary points.

\begin{MHint}{Solution}
On the interval $\MoIl[\left] -\infty\MIntvlSep 2\MoIr[\right]$ the term is positive. 
The corresponding solution set is $\MoIl[\left] -\infty\MIntvlSep \frac{11}{6}\MoIr[\right]$.
In contrast, on the interval $\MoIl 2\MIntvlSep \infty\MoIr$ the term is negative, 
the comparing symbol is inverted. Initially, the new inequality has the solution set
 $\MoIl[\left] \frac{11}{6}\MIntvlSep \infty\MoIr[\right]$, because of the case condition
$x>2$ only the subset $\ML_2=\MoIl 2\MIntvlSep \infty\MoIr$ is allowed.
So, altogether the union set $\ML=\ML_1\cup\ML_2=\R\MSetminus \left[\frac{11}6\MIntvlSep 2\right]$
is the solution set of the initial inequality excluding the boundary points:

\begin{center}
\MTikzAuto{%
\begin{tikzpicture}
% reelle Achse
\draw[->,color=black] (-1,0.0) -- (5,0.0);
\foreach \x in {-1, 0, 1, 2, 3, 4}
\draw[shift={(\x,0)},color=black] (0pt,2pt) -- (0pt,-2pt) node[below] {\footnotesize $\x$};
\draw (4.9,-0.3) node[] {$\mathbb{R}$};
% Enden:
\draw [line width=2.0pt,color=blue] (-1,0.0)-- (1.83,0.0);
\draw [line width=2.0pt,color=blue] (2,0.0)-- (5,0.0);
\draw [fill = white] (1.83,0) circle (1.5pt);
\draw [fill = white] (2,0) circle (1.5pt);
\end{tikzpicture}
}
\end{center}
\end{MHint}
\end{MExercise}

\begin{MExercise}
The solution set of the inequality $\frac{x-1}{x-2}\leq 1$ is 
\MEquationItem{$\ML$}{\MLIntervalQuestion{20}{(-infty,2)}{4}{IGU1}}.

\begin{MHint}{Solution}
The domain of the inequality is $D=\R\MSetminus\lbrace 2\rbrace$.

\begin{itemize}
\item{For $x>2$, multiplying the inequality by the term $x-2$ results in $x-1\leq x-2$, 
which is equivalent to the false statement $-1\leq -2$. Thus, this case does not contribute a solution to the solution set.}
\item{For $x<2$, multiplying the inequality by the term $x-2$ results in $x-1\geq x-2$, 
which is equivalent to the true statement $-1\geq -2$. 
Because of the case condition the solution interval for this case is only 
$\ML_2=\MoIl[\left] -\infty\MIntvlSep 2\MoIr[\right]$.}
\item{The single value $x=2$ is no solution.}
\end{itemize}
So, altogether the solution set is 
 $\ML=\MoIl[\left] -\infty\MIntvlSep 2\MoIr[\right]$ 
excluding the boundary points (even though the comparing symbol $\leq$ occurred in the initial
inequality).
\end{MHint}
\end{MExercise}


\begin{MExercise}
The solution set of the inequality $\frac1{1-\sqrt{x}}<1+\sqrt{x}$ is \MEquationItem{$\ML$}{\MLIntervalQuestion{20}{(1,infty)}{4}{IGU2}}.
\ \\ \ \\
\begin{MHint}{Solution}
The domain of the inequality is $D=[0\MIntvlSep \infty\MoIr\MSetminus \lbrace 1\rbrace$
since only for these values of $x$ the square root is defined and the denominator is non-zero.
\begin{itemize}
\item{For $0\leq x<1$, multiplying the inequality by the term $1-\sqrt{x}$ results in
  $1<(1+\sqrt{x})(1-\sqrt{x})$, which is equivalent to $1<1-x$. This 
inequality is satisfied for $x<0$, but these values of $x$ violate the case condition and 
thus, they are not in the solution set.}
\item{For $x>1$, multiplying the inequality by the term $1-\sqrt{x}$ results in $1>1-x$, 
which is equivalent to $x>0$. But only the values of $x$ in the interval
  $\MoIl 1\MIntvlSep \infty\MoIr$ satisfy the case condition, hence $\ML=\MoIl 1\MIntvlSep \infty\MoIr$ 
is the only solution interval of the initial inequality.}
\item{The single value $x=1$ is no solution.}
\end{itemize}
\end{MHint}
\end{MExercise}


\end{MExercises}

\MSubsection{Absolute Value Inequalities and Quadratic Inequalities}
\MLabel{M03_Betragsungleichungen}

\begin{MIntro}
\MDeclareSiteUXID{VBKM03_Betragsungleichungen_Intro}
As in the approach in Modul~\MNRef{VBKM02} and in the previous section 
\MEntry{absolute values}{inequalities (absolute values)} in inequalities are solved 
by a case analysis:

\begin{MInfo}
To solve an \MEntry{absolute value inequality}{absolute value inequality} two cases are distinguished:

\begin{itemize}
\item{For those values of $x$, for which the absolute value term is non-negative the absolute value can be omitted or
replaced by simple brackets, respectively.}
\item{For those values of $x$, for which the absolute value term is negative the term is bracketed and negated.}
\end{itemize}
\ \\
Then, the solution sets arising from the case analysis will be restricted as described in the 
\MSRef{VBKM02_FallBetrag}{previous module} and merged to the solution set of the initial inequality. 
\end{MInfo}

\begin{MExample}
To solve the absolute value inequality $|4x-2|<1$ two cases are distinguished:
\begin{itemize}
\item{For $x\geq \frac12$, the absolute value term is non-negative: 
In this case the inequality is equivalent to $(4x-2)<1$ or $x<\frac34$, respectively. 
Because of the case condition the solution set is only 
$\ML_1=\left[\frac12\MIntvlSep \frac34\MoIr[\right]$ in this case.}
\item{For $x<\frac12$, the absolute value term is negative: 
In this case the inequality is equivalent to $-(4x-2)<1$ or $x>\frac14$, respectively. 
Only the subset $\ML_2=\MoIl[\left] \frac14\MIntvlSep \frac12\MoIr[\right]$ 
satisfies the case condition and is the solution set.}
\end{itemize}
The union of the two solution intervals results in the solution set
$\ML=\MoIl[\left] \frac14\MIntvlSep \frac34\MoIr[\right]$ for the initial absolute value inequality:

\begin{center}
\MTikzAuto{%
\begin{tikzpicture}
% reelle Achse
\draw[->,color=black] (-1,0.0) -- (5,0.0);
\foreach \x in {-1, 0, 1, 2, 3, 4}
\draw[shift={(\x,0)},color=black] (0pt,2pt) -- (0pt,-2pt) node[below] {\footnotesize $\x$};
\draw (4.9,-0.3) node[] {$\mathbb{R}$};
% Enden:
\draw [line width=2.0pt,color=blue] (0.25,0.0)-- (0.75,0.0);
\draw [fill = white] (0.25,0) circle (1.5pt);
\draw [fill = white] (0.75,0) circle (1.5pt);
\end{tikzpicture}
}
\end{center}
\end{MExample}

\begin{MExercise}
To solve the absolute value inequality $|x-1|<2|x-1|+x$ two cases are distinguished:
\begin{MExerciseItems}
\item{On the interval \MLIntervalQuestion{20}{[1,infty)}{3}{UGL1}, both
terms in the absolute value terms are non-negative. 
The solution set of the inequality is in this case 
\MEquationItem{$\ML_1$}{\MLIntervalQuestion{20}{[1,infty)}{3}{UGL2}}.}
\item{On the interval \MLIntervalQuestion{20}{(-infty,1)}{3}{UGL3}, both
terms in the absolute value terms are negative. 
The solution set of the inequality is in this case
\MEquationItem{$\ML_2$}{\MLIntervalQuestion{20}{(-infty,1)}{3}{UGL4}}.}
\end{MExerciseItems}
The union of the two intervals results in the solution interval 
\MEquationItem{$\ML$}{\MLIntervalQuestion{25}{(-infty,infty)}{4}{UGL5}}.
\ \\ \ \\
\begin{MHint}{Solution}
For $x\in [1\MIntvlSep \infty\MoIr$, both terms in the absolute value terms are non-negative, 
one obtains the inequality $x-1<2(x-1)+x$, which is equivalent to $x>\frac12$. 
Because of the case condition one obtains $\ML_1=[1\MIntvlSep \infty\MoIr$ as solution set.
For $x\in\MoIl[\left] -\infty\MIntvlSep 1\MoIr[\right]$, 
both terms in the absolute value terms are negative and
one obtains $-(x-1)<-2(x-1)+x$. 
This inequality is equivalent to the inequality $x-1<x$ which is
always true. Thus,
the solution set for the second case is
$\ML_2=\MoIl[\left] -\infty\MIntvlSep 1\MoIr[\right]$.
\ \\ \ \\
Since $\ML=\ML_1\cup \ML_2=\R=\MoIl[\left] -\infty\MIntvlSep \infty\MoIr[\right]$ 
the inequality is always satisfied.
\end{MHint}
\end{MExercise}

\end{MIntro}

\begin{MXContent}{Quadratic Absolute Value Inequalities}{Quadratic Inequalities}{STD}
\MDeclareSiteUXID{VBKM03_QuadratischeUngleichungen}
\begin{MInfo}
An inequality is called \MEntry{quadratic}{inequality (quadratic)} in $x$ 
if it can be transformed into $x^2 + p x + q < 0$ (other comparing symbols are allowed).
\end{MInfo}
\ \\ \ \\
Hence, quadratic inequalities can be solved in two ways: by investigating the roots 
and the opening behaviour of the polynomial and by completing the square. Often completing
the square is simpler:
 

\begin{MInfo}
To solve an inequality by \MEntry{completing the square}{completing the square (inequalities)} 
one tries to transform it into the form $(x+a)^2<b$. Taking the square root then results
in the absolute value inequality $|x+a|<\sqrt{b}$ with the solution set 
$\MoIl[\left] -a-\sqrt{b}\MIntvlSep -a+\sqrt{b}\MoIr[\right]$ if $b\geq 0$. Otherwise 
the inequality is unsolvable.

The inverted inequality $|x+a|>\sqrt{b}$ has the solution set
$\MoIl[\left] -\infty\MIntvlSep -a-\sqrt{b}\MoIr[\right]\cup \MoIl[\left] -a+\sqrt{b}\MIntvlSep \infty\MoIr[\right]$. 
For $\leq$ and $\geq$ the corresponding boundary points have to be included.
\end{MInfo}

Always note the calculation rule $\sqrt{x^2}=|x|$ described in Modul~\MNRef{VBKM01}.

\begin{MExample}
Find the solution of the inequality $2x^2\geq 4x+2$. Collecting the terms on the left-hand side and dividing 
the inequality by $2$ results in $x^2-2x-1\geq0$. Completing the square on the 
left-hand side to the second binomial formula results in the equivalent inequality $x^2-2x+1\geq 2$
or $(x-1)^2\geq 2$, respectively. Taking the square root results in the absolute value 
inequality $|x-1|\geq\sqrt{2}$ with the solution set 
$\ML=\MoIl[\left] -\infty\MIntvlSep 1-\sqrt{2}\right]\cup \left[1+\sqrt{2}\MIntvlSep \infty\MoIr[\right]$.
\end{MExample}

On the other hand, the inequality $x^2-2x-1\geq0$ can be investigated as follows:
The left-hand side describes a parabola opened upwards. The roots $x_{1,2}=1\pm \sqrt2$ 
can be found using the $pq$ formula:

\begin{center}
%%\MUGraphicsSolo{parabelu.png}{width=0.4\linewidth}{width:400px}
\MTikzAuto{%
\begin{tikzpicture}[x=1.0cm, y=1.0cm,scale=1.50] 
\draw[black] (-1,0) -- (3,0) (0,-2) -- (0,2);
\foreach \x in {-1, 1, 2, 3}
\draw[shift={(\x,0)},color=black] (0pt,0pt) -- (0pt,-2.0pt) node[below=1.0pt] {\scriptsize $\x$};
\foreach \x in {-0.5, 0.5, ..., 3.0}
\draw[shift={(\x,0)},color=black] (0pt,0pt) -- (0pt,-1.0pt);
\foreach \y in {-2, -1, 1, 2}
\draw[shift={(0,\y)},color=black] (0pt,0pt) -- (-2.0pt,0pt) node[left=1.0pt] {\scriptsize $\y$};
\foreach \y in {-1.5, -0.5, 0.5, 1.5}
\draw[shift={(0,\y)},color=black] (0pt,0pt) -- (-1.0pt,0pt);
\draw[black] (-0.0pt,-0.0pt) node[anchor=north east] {\scriptsize $0$};
\clip(-1.0,-3.0) rectangle (3.0,2.0);
\draw[smooth,samples=21,domain=-1:3, line width=1.0pt,color=red!50!black] plot(\x,{\x*\x-2*\x-1});
\end{tikzpicture}
}
\end{center}
Since the parabola opens upwards, the inequality $x^2-2x-1\geq0$ is satisfied by the
values of $x$ in the parabola branches left and right to the roots, i.e. by the set 
$\ML=\MoIl[\left] -\infty\MIntvlSep 1-\sqrt{2}\right]\cup \left[1+\sqrt{2}\MIntvlSep \infty\MoIr[\right]$.

\begin{MInfo}
\MLabel{M03_InfoFormen}
Depending on the roots of $x^2+ p x + q$, the opening of the parabola and the 
comparing symbol, the quadratic inequality $x^2 +p x +q <0$ (including other comparing symbols) 
has one of the following solution sets:

\begin{itemize}
\item{the set of real numbers $\R$,}
\item{two branches $\MoIl[\left] -\infty\MIntvlSep x_1\MoIr[\right]\cup \MoIl[\left] x_2\MIntvlSep \infty\MoIr[\right]$ (including the boundary points for $\leq$ and $\geq$),}
\item{an interval $\MoIl x_1\MIntvlSep x_2\MoIr$ (including the boundary points for $\leq$ and $\geq$ if applicable),}
\item{a single point $x_1$,}
\item{the pointed set $\R\MSetminus\lbrace x_1\rbrace$,}
\item{the empty set $\lbrace\rbrace$.}
\end{itemize}
\end{MInfo}

Fill in the blanks in the following text describing the solution of a quadratic
inequality by investigating the behaviour of the parabola:

\begin{MExercise}
Find the solution set of the inequality $x^2+6x< -5$. 
Transformation results in the inequality \MLSimplifyQuestion{15}{x^2+6*x+5}{5}{x}{5}{1}{OBXP1}$<0$.
Using the $p q$ formula one obtains the set of roots
\MLParsedQuestion{9}{-1,-5}{3}{PXL}. The left-hand side
describes a parabola opening \MLQuestion{10}{upwards}{ObenX}.
It belongs to an inequality involving the comparing symbol $<$, hence 
the solution set is \MEquationItem{$\ML$}{\MLIntervalQuestion{15}{(-5,-1)}{5}{INVX}}.
\ \\ \ \\
\begin{MHint}{Solution}
Transformation results in $x^2+6x+5<0$. Using the $p q$ formula
one obtains the roots $x_{1,2}=-3\pm\sqrt{9-5}$, i.e.\ $x_1=-1$ and $x_2=-5$.
The left-hand side describes a parabola opening upwards. It satisfies the inequality 
involving $<$ only on the interval $\MoIl[\left] -5\MIntvlSep -1\MoIr[\right]$ excluding
the boundary points.
\end{MHint}
\end{MExercise}

\end{MXContent}

\begin{MXContent}{Further Types of Inequalities}{Further Types of Inequalities}{STD}
\MDeclareSiteUXID{VBKM03_WeitereUngleichungstypen}
Many other types of inequalities can be transformed into quadratic inequalities. Sometimes, 
case analyses have to be done or excluded values in the domain have to be observed:

\begin{MInfo}
An inequality containing \MEntry{fractions}{inequality (fractions)}, where the 
variable $x$ occurs in the denominator of composite terms, can be transformed into a 
form without fractions by multiplying the inequality by the least common denominator. 
However, in doing so, the roots of the denominators have to be excluded from the domain
of the new inequality. 

Additionally, if the inequality is multiplied by a term, different cases have to be distinguished 
depending on the sign of the term.
\end{MInfo}

\begin{MExample}
The inequality $2-\frac1x\leq x$ can be transformed by multiplying the inequality by $x$. Here, three 
cases have to be distinguished:
\begin{itemize}
\item{For $x>0$, the comparing symbol in the inequality is unchanged. The new inequality
reads $2x-1\leq x^2$ and is equivalent to $x^2-2x+1\geq 0$ or $(x-1)^2\geq 0$, respectively.
This inequality is always satisfied. Because of the case condition one obtains 
the solution set $\ML_1=\MoIl 0\MIntvlSep \infty\MoIr$.}
\item{For $x<0$, the comparing symbol in the inequality is inverted. The new inequality
reads $2x-1\geq x^2$ and is equivalent to $x^2-2x+1\leq 0$ or $(x-1)^2\leq 0$, respectively.
This inequality is only satisfied for $x=1$. But this value is excluded by the case condition, 
i.e.\ $\ML_2=\{\}$.}
\item{The single value $x=0$ is not in the domain of the initial inequality and hence it is
no solution.}
\end{itemize}

So, altogether one obtains the union set 
$\ML=\MoIl 0\MIntvlSep \infty\MoIr$ as solution set of the initial inequality.
\end{MExample}

Inequalities involving composite fraction and root terms often do not have solution
sets of the types described in info box~\MRef{M03_InfoFormen}:

\begin{MExample}
Find the solution set of the inequality $\sqrt{x}+\frac1{\sqrt{x}}>2$. 
The domain of the inequality is $\MoIl 0\MIntvlSep \infty\MoIr$.
Multiplying by $\sqrt{x}$ results in the inequality $x+1>2\sqrt x$. 
Here, no case analysis is required since $\sqrt{x}>0$ is in the domain.
Transformation results in $x-2\sqrt{x}+1>0$ or $(\sqrt{x}-1)^2>0$, respectively, 
which is satisfied for all $x\not=1$ in the domain.
Hence, the solution set of the initial inequality is
$\ML=\MoIl 0\MIntvlSep \infty\MoIr\MSetminus\lbrace 1\rbrace$:
\ \\ \ \\
\begin{center}
\MTikzAuto{%
\begin{tikzpicture}
% reelle Achse
\draw[->,color=black] (-1,0.0) -- (5,0.0);
\foreach \x in {-1, 0, 1, 2, 3, 4}
\draw[shift={(\x,0)},color=black] (0pt,2pt) -- (0pt,-2pt) node[below] {\footnotesize $\x$};
\draw (4.9,-0.3) node[] {$\mathbb{R}$};
% Enden:
\draw [line width=2.0pt,color=blue] (0,0.0)-- (1,0.0);
\draw [line width=2.0pt,color=blue] (1,0.0)-- (5,0.0);
\draw [fill = white] (0,0) circle (1.5pt);
\draw [fill = white] (1,0) circle (1.5pt);
\end{tikzpicture}
}
\end{center}

\end{MExample}

\end{MXContent}


\MSubsection{Final Test}
\MLabel{M03_Abschlusstest}

\begin{MTest}{Final Test Modul 3}
\MDeclareSiteUXID{VBKM03_Abschlusstest}

\begin{MExercise}
Find the value of the parameter $\alpha$ such that the inequality $2x^2\leq x-\alpha$ 
has exactly one solution:
\begin{MExerciseItems}
\item{The parameter value is \MEquationItem{$\alpha$}{\MLParsedQuestion{10}{1/8}{3}{PMA1}}.}
\item{In this case \MEquationItem{$x$}{\MLParsedQuestion{10}{1/4}{3}{PMA2}} is the only solution
of the inequality.}
\end{MExerciseItems}
\end{MExercise}


\begin{MExercise}
Find an absolute value function $g(x)$ describing the following graph as easy as possible.

%%\MUGraphics{abs3.png}{width=0.5\linewidth}{Funktionsgraph von $g(x)$.}{width:300px}
\begin{center}
\MTikzAuto{%
\begin{tikzpicture}[x=1.4cm, y=1.9cm] 
\draw[black] (-3,0) -- (3,0) (0,-3) -- (0,3);
\foreach \x in {-3, -2, -1, 1, 2, 3}
\draw[shift={(\x,0)},color=black] (0pt,0pt) -- (0pt,-3.0pt) node[below=1.0pt] {\normalsize $\x$};
\foreach \x in {-3.0, -2.8, ..., 3.0}
\draw[shift={(\x,0)},color=black] (0pt,0pt) -- (0pt,-1.5pt);
\foreach \y in {-3, -2, -1, 1, 2, 3}
\draw[shift={(0,\y)},color=black] (0pt,0pt) -- (-3.0pt,0pt) node[left=1.0pt] {\normalsize $\y$};
\foreach \y in {-3.0, -2.8, ..., 3.0}
\draw[shift={(0,\y)},color=black] (0pt,0pt) -- (-1.5pt,0pt);
%%\draw[black] (-0.0pt,-0.5pt) node[anchor=north east] {\small $0$};
\clip(-3.0,-3.0) rectangle (3.0,3.0);
\draw[black, line width=1.0pt,color=black] (-3,-3) -- (1,1) -- (2,4);
\end{tikzpicture}
}
\par
Graph of the function $g(x)$.
\end{center}
Try to find a representation of the form $g(x)=|x+a|+b x+c$. 
The kink in the graph indicates how the absolute value term looks like.

\begin{MExerciseItems}
\item{Find the solution set of the inequality $g(x)\leq x$ by means of the graph.\\
The solution set is \MEquationItem{$\ML$}{\MLIntervalQuestion{20}{(-infty;1]}{5}{AUX1}}.}
\item{\MEquationItem{$g(x)$}{\MLSimplifyQuestion{20}{abs(x-1)+2*x-1}{10}{x}{10}{0}{SIMPLE3}}. 
\\\MInputHint{Absolute values can be entered in the form \texttt{betrag(x-a)} or \texttt{abs(x-a)}.}}
\end{MExerciseItems}
\end{MExercise}

\begin{MExercise}
Which positive real numbers $x$ satisfy the following inequalities?
\begin{MExerciseItems}
\item{$|3x-6|\leq x+2$ has the solution set
\MEquationItem{$\ML$}{\MLIntervalQuestion{16}{[1,4]}{4}{COSH1}} (written as an interval).}
\item{$\frac{x+1}{x-1}\geq 2$ has the solution set 
\MEquationItem{$\ML$}{\MLIntervalQuestion{16}{(1,3]}{4}{COSH2}} (written as an interval).}
\end{MExerciseItems}
\MInputHint{Enter open intervals in the form $(3;5)$, closed intervals in the form 
$[3;5]$. Infinity can be entered a a word or shortly a \texttt{infty}. Do not use 
notation $]a;b[$ for open intervals. Sets can be entered by listing the elements
 $\lbrace 1;2;3\rbrace$. For the set brackets enter AltGr+7 or AltGr+0, respectively.}
\end{MExercise}

\end{MTest}

\newpage
\MPrintIndex

\end{document}

%\input{mintmod.tex}
\MPragma{MathSkip}

%\Mtikzexternalize

\begin{document}

\MSection{Equations in one Variable}
\MLabel{VBKM02}
\MSetSectionID{VBKM02} % wird fuer tikz-Dateien verwendet

\begin{MSectionStart}
\MDeclareSiteUXID{VBKM02_START}

Equations arise by equating two terms in which variables occur. Simple equations can be solved by 
applying transformations and solution formulas. For more sophisticated equations case analyses are 
required. This module consists of

\begin{itemize}
\item{Section \MNRef{M02_EinfacheGleichungen}: \MSRef{M02_EinfacheGleichungen}{Simple Equations},}
%\item{dem Abschnitt \MNRef{M02_Wurzelgleichungen}, \MSRef{M02_Wurzelgleichungen}{Gleichungen mit Wurzeln},}
\item{Section \MNRef{M02_Betragsgleichungen}: \MSRef{M02_Betragsgleichungen}{Absolute Value Equations},}
\item{and Section \MNRef{M02_Abschlusstest}: \MSRef{M02_Abschlusstest}{Final Test}.}
\end{itemize}
\end{MSectionStart}

\MSubsection{Simple Equations}
\MLabel{M02_EinfacheGleichungen}

\begin{MIntro}
\MDeclareSiteUXID{VBKM02_EinfacheGleichungenIntro}
\begin{MInfo}
An \MEntry{equation}{equation} is an expression of the form
$$
\text{left-hand side} \;=\; \text{right-hand side} \MDFPSpace
$$
with mathematical expressions on both sides of the equation. These expressions generally contain variables or unknowns (e.q.\ $x$). 
Depending on the variable values an equation is satisfied if both sides of the equation evaluate to the same value. An equation is not 
satisfied if the sides of the equation evaluate to different values.
\end{MInfo}

Equations describe relations between expressions or model a problem to be solved. An equation itself cannot
be true or false but some variables satisfy the equation and others do not. To test whether the equation is 
true or false for a single variable value this value has to be inserted into the equation. Then, both sides of the 
equation are evaluated to certain values. The equation is satisfied by an inserted variable value if the evaluated 
values coincide:

\begin{MExample}
The equation $2x-1=x^2$ has the right-hand side $x^2$ and the left-hand side $2x-1$. Inserting $x=1$ 
results in the value $1$ on both sides of the equals sign, hence $x=1$ is a solution of this equation. 
However, $x=2$ is no solution since the left-hand side of the equation is evaluated to the value $4$ while the 
right-hand side is evaluated to the value $3$.
\end{MExample}

\begin{MInfo}
The \MEntry{solution set}{solution set} $\ML$ of an equation is the set of all numbers satisfying the relation 
$$
\text{left-hand side} \;=\; \text{right-hand side} \MDFPSpace
$$
if inserted into the the equation instead of the variable (e.q. $x$).
\end{MInfo}

Typical problems concerning equations are:
\begin{itemize}
 \item{specify the solution set of an equation, i.e.\ find all variable values satisfying the equation,}
 \item{transform the equation, in particular, solve an equation for the variables, and}
 \item{find an equation modelling a problem described textually.}
\end{itemize}


\begin{MExample}
  A savings deposit is to be designed such that it gives a fixed return per year. The bank intends to achieve
  that the saver returns for a five years deposit exactly 600~Euro more than for a deposit of only two years. 

  First, the textual problem is translated into an equation with the variable $r$ denoting the return per year. Then, 
  the equation is $5\cdot r=2\cdot r+600$. It says that five payments (left-hand side of the equation) equal 
  two payments plus 600 (the unit Euro is then omitted during calculation).

  This equation can be solved for $r$ very easily by subtracting the term $2r$ to both sides of the equation. Then, the 
  equation reads $3r=600$ and dividing by $3$ results in the solution $r=200$.

  Thus, the bank has to offer a return of 200~Euro per year to reach the required savings target. 
\end{MExample}

% \begin{MExercise}
% �bersetzen Sie die folgenden Aussagen in eine Gleichung (ohne sie zu l�sen):
% \begin{MExerciseItems}
% \item{In elf Jahren ist Max doppelt so alt wie jetzt ($t$ = jetziges Alter in Jahren):  ???}
% \item{Ein Kunde hat eine Rechnung �ber 1000 Euro zu bezahlen. Die monatlich zu zahlende Rate bleibt f�r ihn gleich unabh�ngig davon ob er
% \begin{itemize}
%  \item{die Rechnung mit $m$ gleichgro�en Raten abbezahlt,}
%  \item{vier Raten mehr als $m$ einteilt und zus�tzlich 10 Euro Verzugsgeb�hren pro Monat an die Bank zahlt (die nicht zur Rechnungsbegleichung dienen).}
% \end{itemize}
% Ist $m$ die Anzahl der Monate, so gilt ???
% }
% \end{MExerciseItems}
% \end{MExercise}


\begin{MInfo}
Two equations are said to be \MEntry{equivalent}{equivalence} if they have the same solution set.

An \MEntry{equivalent transformation}{equivalent transformation} is a special transformation that changes
the equation but not its solution set. Important equivalent transformations are

\begin{itemize}
 \item{adding/subtracting terms to both sides of the equation,}
 \item{exchanging both sides of the equation,}
 \item{transformation of terms on one side of the equation, and}
 \item{inserting known relations.}
 \end{itemize}

The following transformations are considered as equivalent transformations only if the used term is non-zero 
(which can depend on the variables):

\begin{itemize}
 \item{multiplying/dividing by a term (this term has to be non-zero),}
 \item{taking the reciprocal on both sides of the equation (both sides have to be non-zero).}
\end{itemize}
\end{MInfo}

Here, the following \MEntry{notation}{equivalent transformation (notation)} is used:

\begin{itemize}
 \item{equivalent equations are indicated by the symbol $\Leftrightarrow$ (which reads: if and only if, i.e. one
  equation is satisfied if and only if the other equation is satisfied).}

\item{the symbol is underset by the transforming operation (or, for solutions with more than one line, the 
  transforming operation is written next to the transformation).}
\end{itemize}

Importantly, the reader should be able to understand which transformation was carried out.

\begin{MExample}
This example illustrates two simple single-lined equivalent transformations. Even though the symbol $\Leftrightarrow$ 
is two-sided the notation is interpreted in such a way that the transformation is applied from 
left to right:
$$
3x-x^2 \;=\; 2x-x^2+1\;\;\MUnderset{+x^2}\Leftrightarrow\;\;  3x\;=\; 2x+1 \;\;\MUnderset{-2x}\Leftrightarrow\;\; x \;=\; 1 \MDFPeriod
$$
The left equation and the right equation are equivalent. On the left we have the initial equation
(corresponding to a certain textual problem) and on the right we have an equivalent equation 
showing the solution immediately.
\end{MExample}

\begin{MExample}
\MLabel{BSP_Umformungen1}
For several complicated transformations the transformation steps should be written below each other.
In this case the respective transformations are separated from the equation by vertical bars. 

\begin{eqnarray*}
& \text{Start:} & 12+t \;=\;\Mdfrac{2t}{2t^2}+t \ \ \ \ \MSep \ -t\ \\ \ \\
& \Leftrightarrow & 12 \;=\; \Mdfrac{2t}{2t^2}  \ \ \ \ \MSep \ \text{sides exchanged}\ \\ \ \\
& \Leftrightarrow & \Mdfrac{2t}{2t^2} \;=\; 12  \ \ \ \ \MSep \ \text{left-hand side transformed}\ \\ \ \\
& \Leftrightarrow & \Mdfrac1t \;=\; 12  \ \ \ \ \MSep \ \text{reciprocals taken}\ \\ \ \\
& \Leftrightarrow & t \;=\; \Mdfrac{1}{12} \MDFPeriod
\end{eqnarray*}

Here, after the vertical bar both short symbols as, e.g. $-t$, and textual descriptions are allowed. 
Importantly, the reader should be able to understand which transformations were carried out and should be able to decide whether 
they are correct.

\end{MExample}
\end{MIntro}

\begin{MXContent}{Conditions in Transformations}{Conditions}{STD}
\MDeclareSiteUXID{VBKM02_Bedingungen}
Multiplication, division, and taking reciprocals are only equivalent transformations if the factors or terms are 
non-zero. In example~\MRef{BSP_Umformungen1}, the reader understands that both sides of the equation are non-zero 
such that the transformation is allowed. If the variables themselves are used in the transformation it has to be 
noted that the respective term has to be non-zero. The solution at the end of the transformations is then 
only valid for variable values satisfying the transformation conditions. All other values have to be checked 
\textit{separately}, typically by inserting the value into the equation:

\begin{MExample}
In this example, the necessary transformation conditions are not problematic:
\begin{eqnarray*}
& \text{Start:} & 9x \;=\;81x^2  \ \ \ \ \MSep \ :x\text{, transformation allowed if }x\not=0\ \\ \ \\
& \Leftrightarrow & 9 \;=\; 81x  \ \ \ \ \MSep \ :81\text{ and exchange sides}\ \\ \ \\
& \Leftrightarrow & x \;=\; \Mdfrac19 \ \ \ \ \text{and this value satisfies the condition }x\not=0 \MDFPeriod
\end{eqnarray*}
The value $x=0$, initially rejected by the transformation condition, has to be checked separately. The equation $9x=81x^2$ is 
also satisfied for $x=0$, hence $x=0$ is also a solution of the equation. In set notation, the 
solution set is $\ML=\lbrace 0\MElSetSep \Mtfrac19\rbrace$.
\end{MExample}

Anyway, values violating a condition have to be checked separately, in particular, they can be finally 
part of the solution.

\begin{MExample}
\begin{eqnarray*}
& \text{Start:} & x^2-2x \;=\; 2x-4 \ \ \ \ \MSep \ \text{factor out on both sides} \ \\ \ \\
& \Leftrightarrow & x\cdot (x-2) \;=\; 2\cdot (x-2) \ \ \ \ | \ :(x-2)\text{, transformation only allowed if }x\not=2\ \\ \ \\
& \Leftrightarrow & x\;=\; 2  \MDFPeriod
\end{eqnarray*}
This value of $x$ violates the condition $x\not=2$. Hence, this is possibly no solution. Inserting $x=2$
into the initial equation gives $2^2-2\cdot 2=0$ on the left-hand side and also $2\cdot 2-4=0$ on the 
right-hand side. Hence, $x=2$ is indeed a solution, even though it violated the transformation condition.
\end{MExample}

\begin{MExercise}
Find the solution of the equation $(x-2)(x-3)=x^2-9$ by transforming the right-hand side using the third 
binomial formula and then dividing by a common factor. 

The solution is $x$ = \MLParsedQuestion{5}{3}{5}{EASY1}.

\begin{MHint}{Solution}
The correct transformation steps including conditions are
\begin{eqnarray*}
& \text{Start:} & (x-2)(x-3)\;=\; x^2-9 \ \ \ \ \MSep \ \text{transformation of the right-hand side} \ \\ \ \\
& \Leftrightarrow & (x-2)(x-3) \;=\; (x+3)(x-3) \ \ \ \ \MSep \ :(x-3)\text{, transformation allowed if }x\not=3\ \\ \ \\
& \Leftrightarrow & x-2\;=\; x+3 \ \ \ \ \MSep \ -x \ \\ \ \\
& \Leftrightarrow & -2 \;=\; 3 \; \text{is a wrong equation.}
\end{eqnarray*}
Importantly, this equation is only wrong for $x\not=3$. We have to check $x=3$ separately, and indeed 
$x=3$ satisfies the initial equation.
\end{MHint}
\end{MExercise}
\end{MXContent}

\begin{MXContent}{Proportionality and Rule of Three}{Proportionality}{STD}
\MLabel{VBKM02_Dreisatz}
\MDeclareSiteUXID{VBKM02_Dreisatz}
In practise, a relation between two quantities that occurs frequently is the 
\MEntry{proportionality}{proportionality} between two quantities, e.g. between 
mass and volume, time and travelled distance or weight (quantity) of a product and its price. 
The relation can be exemplary for certain fixed quantities. Then, a first aim is to formulate the 
resulting relation for another application example. 
The procedure shall be illustrated by an example. 


\begin{MExample}
$5\MEinheit{kg}$ of apples cost $3$ Euro. How much do $11\MEinheit{kg}$
of apples cost?
\par
The initial relation can be formulated as follows:
$$
5\MEinheit{kg} \MDFPSpace \MRelates \MDFPSpace 3\MBlank \text{Euro}
\MDFPeriod
$$
It is assumed that these quantities are proportional to each other. In the next
step the relation between the quantities is reduced to a unit of one of the quantities, 
namely to the unit of the given quantity. Here, both quantities 
are multiplied by $1/5$ -- i.e. the unit is $1\MEinheit{kg}$ --:
$$
1\MEinheit{kg} \MDFPSpace \MRelates \MDFPSpace 
\frac{1}{5}\cdot 3\MBlank \text{Euro} = \MZahl{0}{6}\MBlank \text{Euro}
\MDFPeriod
$$
Finally, both sides of the equation are multiplied by the multiple of the 
respective unit of the specified quantity, in this case by the factor $11$: 
$$
11\MEinheit{kg} \MDFPSpace \MRelates \MDFPSpace 
11\cdot\MZahl{0}{6}\MBlank \text{Euro} = \MZahl{6}{6}\MBlank \text{Euro}
\MDFPeriod
$$
The required price for $11\MEinheit{kg}$ of apples is therefore
$\MZahl{6}{60}\MBlank \text{Euro}$.
\end{MExample}

We have derived the required relation by deriving a relation for one unit of a quantity from 
the initial relation. This procedure demonstrated here as an example is called 
\MEntry{rule of three}{rule of three}.

The posed problem can also be solved by introducing a proportionality factor. 
Again, we consider the example above.

\begin{MExample}
The price $P$ is proportional to the mass $m$. Hence, it exists a constant 
$k$ with
$$
P=k m \MDFPeriod
$$
Since this relation also holds for the given values $m_0=5\MEinheit{kg}$
and $P_0=3\MBlank\text{Euro}$ it follows 
\begin{eqnarray*}
  P_0=k m_0 & \MTSP\MSep\MTSP & \text{multiplying by}\MBlank \frac{1}{m_0} \\
  \Longleftrightarrow\MDFPSpace\frac{P_0}{m_0}=k &;& 
\end{eqnarray*}
hence in this case
$$
k=\frac{3}{5} = \MZahl{0}{6} \MDFPSpace,
$$
taken in the unit of Euro per kg. (As a scientist you would correctly write 
 $k=\MZahl{0}{6} \MBlank \text{Euro}/\MEinheit[]{kg}$, since proportionality factors 
generally carry a dimensional unit.) Using $m_1=11\MEinheit{kg}$, one obtains 
finally
$$
P_1=k m_1 = \MZahl{0}{6}\cdot 11 =\MZahl{6}{6} \MBlank \text{(Euro)}
\MDFPSpace
$$
which is the same result as for using the rule of three (see previous example).
\end{MExample}

\begin{MExercise}
A car takes $9$~minutes to travel a distance of $6\MEinheit{km}$.
\begin{MExerciseItems}
\item{%
Which distance $s$ the car travels within $15$~minutes?
\medskip\par
The solution is $s_{15}$ = \MLParsedQuestion{5}{10}{5}{EASY2}$\MEinheit{km}$.
}
\item{%
The proportionality factor between travelled distance $s$ and travelling 
time $t$ is the velocity $v$ of the car.
\medskip\par
The velocity is $v$ = \MLParsedQuestion{5}{40}{5}{EASY3}$\MEinheit{km}/\MEinheit{h}$.
}
\end{MExerciseItems}
\begin{MHint}{Solution}
>From the given values we know that the car travels $\frac{6}{9}\MEinheit{km}
=\frac{2}{3}\MEinheit{km}$  within one minute and therefore 
$15\cdot\frac{2}{3}\MEinheit{km}=10\MEinheit{km}$ within $15$~minutes.
\par
So, the velocity is
$$v=\frac{10\MEinheit{km}}{15\MEinheit{min}} = 
\frac{10\MEinheit{km}}{(1/4)\MEinheit{h}} =
40\MEinheit{}\frac{\MEinheit[]{km}}{\MEinheit[]{h}}
\MDFPeriod
$$
\end{MHint}
\end{MExercise}
\end{MXContent}

\begin{MXContent}{Solving linear Equations}{Solving}{STD}
\MLabel{VBKM02_LineareGleichungenLoesen}
\MDeclareSiteUXID{VBKM02_Aufloesen}
\begin{MInfo}
A \MEntry{linear equation}{equation (linear)} is an equation in which only multiples of 
variables and constants occur.

For a linear equation in one variable (here the variable $x$) one of the following 
three statements holds:
\begin{itemize}
 \item{The equation has no solution.}
 \item{The equation has a single solution.}
 \item{Every value of $x$ is a solution of the equation.}
\end{itemize}
\end{MInfo}

These three cases are distinguished by means of the transformation steps:
\begin{itemize}
\item{If the transformation ends up in a statement that is wrong for all $x$ (e.g. $1=0$) 
then the equation is unsolvable.}
\item{If the transformation ends up in a statement that is true for all $x$ (e.g. $1=1$)
then the equation is solvable for all values of $x$.}
\item{Otherwise, the equation can be solved, i.e. it can be transformed into 
the equation $x=\text{value}$ which is the solution.}
\end{itemize}

\begin{MXInfo}{Set notation}
Using the set notation (with the solution set $\ML$) these cases can be expressed as follows:
\begin{itemize}
 \item{$\ML=\lbrace\rbrace$ or $L=\MEmptyset$ if there is no solution,}
 \item{$\ML=\lbrace \text{value}\rbrace$ if there is a single solution,}
 \item{$\ML=\R$ if all real numbers $x$ are a solution.}
\end{itemize}
\end{MXInfo}


\begin{MExample}
The linear equation $3x+2=2x-1$ has one solution. 
This solution is obtained by equivalent transformations:
$$
3x+2 \;=\; 2x-1 \;\;\MUnderset{-2x}\Leftrightarrow\;\; x+2\;=\;-1\;\;\MUnderset{-2}\Leftrightarrow\;\; x\;=\; -3 \MDFPeriod
$$
Hence, $x=-3$ is the only solution.
\end{MExample}

\begin{MExample}
The linear equation $3x+3=9x+9$ has the solution:
$$
3x+3 \;=\; 9x+9 \;\;\MUnderset{:(x+1)}\Leftrightarrow\;\; 3\;=\;9 \MDFPeriod
$$
This statement is wrong. Hence, for all $x\not=-1$ (transformation condition) the equation is wrong.
Inserting $x=-1$ satisfies the equation, and so the only solution is $x=-1$.

Alternatively, the equation could have been transformed as follows:
$$
3x+3 \;=\; 9x+9 \;\;\MUnderset{-3x-9}\Leftrightarrow\;\; -6 \;=\; 6x \;\; \Leftrightarrow\;\; x \;=\; -1 \MDFPeriod
$$
\end{MExample}

\begin{MExercise}
Transform the following linear equations and specify their solution sets:
\MInputHint{Enter simply \texttt{$\lbrace a\rbrace$} for a unit set and $\lbrace\rbrace$ for an empty set.}\ \\
\begin{MExerciseItems}
\item{The equation $x-1=1-x$ has the solution set \MEquationItem{$\ML$}{\MLParsedQuestion{4}{1,1}{4}{LUA1}},}
\item{The equation $4x-2=2x+2$ has the solution set \MEquationItem{$\ML$}{\MLParsedQuestion{4}{2,2}{4}{LUA2}},}
\item{The equation $2x-6=2x-10$ has the solution set \MEquationItem{$\ML$}{\MLParsedQuestion{4}{}{4}{LUA3}}.}
\end{MExerciseItems}

\begin{MHint}{Solution}
The first equation can be transformed into $2x=2$ or $x=1$, respectively, so 
the solution set is $\ML=\lbrace 1\rbrace$. The second equation can be transformed into $2x=4$ and the solution set is 
$\ML=\lbrace 2\rbrace$. The third equation can be transformed into $-6=-10$ which is a false statement, 
hence $\ML=\lbrace\rbrace$.
\end{MHint}
\end{MExercise}

\begin{MExercise}
Find the solution of the general linear equation $a x=b$ with $a$ and $b$ being real numbers.
Specify the values of $a$ and $b$ for which the following three cases occur:
\begin{itemize}
 \item{Every value of $x$ is a solution ($\ML=\R$) if 
\MEquationItem{$a$}{\MLParsedQuestion{4}{0}{4}{ALG1}} and $b=0$.}
 \item{There is no solution ($\ML=\MEmptyset$) if \MEquationItem{$a$}{\MLParsedQuestion{4}{0}{4}{ALG2}} 
and $b\not=$\MLParsedQuestion{4}{0}{4}{ALG3}.}
 \item{Otherwise, there is a single solution, namely 
\MEquationItem{$x$}{\MLSimplifyQuestion{5}{b/a}{10}{a,b}{10}{513}{VBKM02ALTFALL3}}.}
\end{itemize}

\begin{MHint}{Solution}
Every value of $x$ is a solution ($\ML=\R$) if $a=0$ and $b=0$.
There is no solution ($\ML=\MEmptyset$) if $a=0$ and $b\not=0$.
Otherwise, there is only one solution, namely $x=\Mtfrac{b}{a}$.
\end{MHint}

\end{MExercise}


\end{MXContent}

\begin{MXContent}{Solving quadratic Equations}{Quadratic Equations}{STD}
\MLabel{VBKM02_QuadratischeGleichungen}
\MDeclareSiteUXID{VBKM02_Quadratische Gleichungen}
\begin{MInfo}
A \MEntry{quadratic equation}{equation (quadratic)} is an equation of the form
  $a x^2 + b x + c = 0$ with $a\not=0$, or, in reduced form, $x^2+ p x + q=0$. 
This form is obtained by dividing the equation by $a$.

For a quadratic equation in one variable (here the variable $x$) one of the following three statements 
holds:
\begin{itemize}
 \item{The quadratic equation has no solution: $\ML=\lbrace\rbrace$.}
 \item{The quadratic equation has a single solution $\ML=\lbrace x_1\rbrace$.}
 \item{The quadratic equation has two different solutions $\ML=\lbrace x_1\MElSetSep x_2\rbrace$.}
\end{itemize}
\end{MInfo}

The solutions are obtained by applying \MEntry{quadratic solution formulas}{solution formulas}.

\begin{MInfo}
\MLabel{VBKM02_pqFormel}
The \MEntry{$p q$ formula}{pq formula} for solving the equation $x^2+p x + q = 0$ reads
$$
x_{1,2} \;=\; -\Mdfrac{p}{2}\pm \sqrt{\Mdfrac14p^2-q} \MDFPeriod
$$
Here, the equation has
\begin{itemize}
\item{no (real) solution if $\Mtfrac14p^2-q<0$ (taking the square root is not allowed),}
\item{a single solution $x_1=-\Mtfrac{p}{2}$ if $\Mtfrac14p^2=q$ and the square root is zero,}
\item{two different solutions if the square root is a positive number.}
\end{itemize}

The expression $D:=\Mtfrac14p^2-q$ underneath the square root considered above is called 
\textbf{discriminant}.
\end{MInfo}
The solution of a quadratic equation is often described by an alternative formula:
\begin{MInfo}
\MLabel{VBKM02_abcFormel}
For the equation $a x^2+b x + c = 0$ with $a\ne 0$ the
\MEntry{$a b c$ formula}{abc formula} reads
$$
x_{1,2} \;=\; \frac{-b\pm\sqrt{b^2-4 a c}}{2a} \MDFPeriod
$$
Here, the equation has
\begin{itemize}
\item{no (real) solution if $b^2-4 a c<0$ (the square root of a negative number is undefined within the
range of real numbers),}
\item{a single solution $x_1=-\Mtfrac{b}{2a}$ if $b^2=4 a c$ and the square root is zero,}
\item{two different solutions if the square root is a positive number.}
\end{itemize}
Again, the expression $D:=b^2-4 a c$ underneath the square root considered above is called 
\textbf{discriminant}.
\end{MInfo}
Certainly, both formulas result in the same solutions. (For applying the $pq$ formula, the quadratic 
equation is to be divided by the factor $a$ of the quadratic term.)
\medskip\par
This three different cases correspond to three different numbers of intersection points between 
the graph of a parabola opening upwards $f(x)=x^2+p x+ q$ and the $x$ axis (applying the $p q$ formula).

\begin{center}
%%\MUGraphicsSolo{para1b.png}{width=0.2\linewidth}{width:230px}\ \ 
%%\MUGraphicsSolo{para2b.png}{width=0.2\linewidth}{width:230px}\ \ 
%%\MUGraphicsSolo{para3b.png}{width=0.2\linewidth}{width:230px}
\MTikzAuto{%
\begin{tikzpicture}[x=1.0cm, y=1.0cm,scale=0.60] 
\foreach \sx/\fsy in {-7.0cm/0.5,0.0cm/0.0,7.0cm/-0.5} {
\begin{scope}[xshift=\sx,yshift=0]
\draw[black] (-3,0) -- (3,0) (0,-3) -- (0,3);
\foreach \x in {-3, -2, -1, 1, 2, 3}
\draw[shift={(\x,0)},color=black] (0pt,0pt) -- (0pt,-2.0pt) node[below=1.0pt] {\tiny $\x$};
\foreach \x in {-2.5, -1.5, ..., 2.5}
\draw[shift={(\x,0)},color=black] (0pt,0pt) -- (0pt,-1.0pt);
\foreach \y in {-3, -2, -1, 1, 2, 3}
\draw[shift={(0,\y)},color=black] (0pt,0pt) -- (-2.0pt,0pt) node[left=1.0pt] {\tiny $\y$};
\foreach \y in {-2.5, -1.5, ..., 2.5}
\draw[shift={(0,\y)},color=black] (0pt,0pt) -- (-1.0pt,0pt);
\draw[black] (-0.0pt,-0.0pt) node[anchor=north east] {\tiny $0$};
\clip(-3.0,-3.0) rectangle (3.0,3.0);
\draw[smooth,samples=27,domain=-3:3, line width=1.0pt,color=red!50!black] plot(\x,{\x*\x+\fsy});
\end{scope}
}
\end{tikzpicture}
}
\end{center}

Three cases: no intersection point, one intersection point and two intersection points with the
$x$ axis.


\begin{MExample}
The quadratic equation $x^2-x+1=0$ has no solution since the discriminant $\Mtfrac14p^2-q=-\Mtfrac34$
within the $p q$ formula is negative.
In contrast, the equation $x^2-x-1=0$ has two solutions
\begin{eqnarray*}
x_1 &=& \Mdfrac12+\sqrt{\Mdfrac14+1} \;=\; \Mdfrac12(1+\sqrt5) \MDFPSpace ,\ \\
x_2 &=& \Mdfrac12-\sqrt{\Mdfrac14+1} \;=\; \Mdfrac12(1-\sqrt5) \MDFPeriod
\end{eqnarray*}
\end{MExample}

\begin{MInfo}
\MLabel{VBKM02_Scheitelpunktform}
%%Eine quadratische Gleichung ist in \MEntry{Scheitelpunktform}{Scheitelpunktform}, wenn sie die Form $a\cdot (x-s)^2=d$ besitzt mit $a\not=0$ und $d\geq 0$.
%%F�r den Funktionsausdruck der zugeh�rigen Parabel liest sich diese Form $f(x)=a\cdot (x-s)^2-d$.
%%In dieser Situation ist $(s\,|\,-d)$ der \MEntry{Scheitelpunkt}{Scheitelpunkt (Parabel)} der Parabel.
%%
%%Falls $a>0$ ist gibt es zwei L�sungen
%%$$
%%x_1 \;=\; s-\sqrt{\Mdfrac{d}{a}} \MDFPSpace,\MDFPaSpace x_2 \;=\; s+\sqrt{\Mdfrac{d}{a}}
%%$$
%%der Gleichung, diese liegen symmetrisch um die $x$-Koordinate des Scheitelpunkts. F�r $d=0$ gibt es nur eine L�sung.
%%
%%Das Vorzeichen von $a$ bestimmt, ob die Gleichung eine nach oben oder unten ge�ffnete Parabel beschreibt.
The function expression of a parabola has \MEntry{vertex form}{vertex form} if the function
has the form $f(x)=a\cdot (x-s)^2-d$ with $a\ne 0$. In this case, $\MPointTwo{s}{-d}$
is the \MEntry{vertex}{vertex (parabola)} of the parabola. The corresponding 
quadratic equation for $f(x)=0$ then reads $a\cdot (x-s)^2=d$.

Dividing this equation by $a$ one obtains the equivalent quadratic equation 
$(x-s)^2=\frac{d}{a}$. Since the left-hand side is a square of a real number, only 
solutions exist if and only if the right-hand side is non-negative as well, i.e.
$\frac{d}{a}\ge 0$. By taking the square root, taking the two possible signs into
account, one obtains $x-s=\pm\sqrt{\frac{d}{a}}$.

So, for $\frac{d}{a}>0$ two solutions of the equation exist:
$$
x_1 \;=\; s-\sqrt{\Mdfrac{d}{a}} \MDFPSpace,\MDFPaSpace x_2 \;=\; s+\sqrt{\Mdfrac{d}{a}}\,;
$$
they are symmetric to the $x$ coordinate $s$ of the vertex. For $d=0$, only one
solution exists.

The sign of $a$ determines whether the function expression describes a parabola 
opening upwards or downwards. 
\end{MInfo}

The quadratic equation has only one single solution $s$ if it can be transformed into the form
$(x-s)^2=0$.

\begin{MInfo}
\MLabel{VBKM02_QuadratischErgaenzung}

Any quadratic equation can be transformed (after collecting terms on the left-hand side
and normalisation, if necessary) to vertex form by 
\MEntry{completing the square}{completing the square}. For this, a constant is
added to both sides of the equation such that on the left-hand side we have a term of the 
form $x^2\pm 2s x+s^2$ to which the first or second binomial formula can be applied.
\end{MInfo}

\begin{MExample}
Adding the constant $2$ transforms the equation $x^2-4x+2=0$ into the
form $x^2-4x+4=2$ or into the form $(x-2)^2=2$, respectively. From this,
the two solutions $x_1=2-\sqrt{2}$ and $2+\sqrt{2}$ can be seen immediately.
In contrast, the quadratic equation $x^2+x=-2$ has no solution since completing
the square results in $x^2+x+\Mtfrac14=-\Mtfrac74$ or $(x+\Mtfrac12)^2=-\Mtfrac74$, respectively, 
where the right-hand side is negative for $a=1$.
\end{MExample}

\begin{MExercise}
Find the solutions of the following quadratic equations by completing the square after 
collecting terms on the left-hand side and normalisation (i.e.\ selecting $a=1$):

\begin{MExerciseItems}
\item{$x^2=8x-1$ has the vertex form \MEquationItem{\MLSimplifyQuestion{20}{x^2-8*x+16}{5}{x}{5}{0}{MSPF1}}{\MLParsedQuestion{5}{15}{5}{GLD1}}.\\The solution set is \MEquationItem{$\ML$}{\MLParsedQuestion{30}{4-sqrt(15),4+sqrt(15)}{5}{SQRT1}}.}
\item{$x^2=2x+2+2x^2$ has the vertex form \MEquationItem{\MLSimplifyQuestion{20}{x^2+2*x+1}{5}{x}{5}{0}{MSPF2}}{\MLParsedQuestion{5}{-1}{5}{GLD2}}.\\The solution set is \MEquationItem{$\ML$}{\MLParsedQuestion{30}{}{5}{SQRT2}.}}
\item{$x^2-6x+18=-x^2+6x$ has the vertex form \MEquationItem{\MLSimplifyQuestion{20}{x^2-6*x+9}{5}{x}{5}{0}{MSPF3}}{\MLParsedQuestion{5}{0}{5}{GLD3}}.\\The solution set is \MEquationItem{$\ML$}{\MLParsedQuestion{30}{3,3}{5}{SQRT3}.}}
\end{MExerciseItems}
\MInputHint{Enter sets in the form \texttt{$\lbrace$ a;b;c;$\ldots\rbrace$}. Enter the empty set as $\lbrace\rbrace$.}

\begin{MHint}{Solution}
The transformations are for the first equation
\begin{eqnarray*}
&&x^2=8x-1 \ \\
&\Leftrightarrow&x^2-8x+1=0 \ \\
&\Leftrightarrow&x^2-8x+16=15 \ \\
&\Leftrightarrow&(x-4)^2=15 \ \\
&& \ML=\lbrace 4-\sqrt{15}\MElSetSep 4+\sqrt{15}\rbrace
\end{eqnarray*}
and for the second equation
\begin{eqnarray*}
&&x^2=2x+2+2x^2 \ \\
&\Leftrightarrow&x^2+2x+2=0 \ \\
&\Leftrightarrow&x^2+2x+1=-1 \ \\
&\Leftrightarrow&(x+1)^2=-1\ \\
&& \ML=\lbrace\rbrace
\end{eqnarray*}
and for the third equation
\begin{eqnarray*}
&&x^2-6x+18=-x^2+6x \ \\
&\Leftrightarrow&2x^2-12x+18=0 \ \\
&\Leftrightarrow&x^2-6x+9=0 \ \\
&\Leftrightarrow&(x-3)^2=0\ \\
&& \ML=\lbrace3\rbrace \MDFPeriod
\end{eqnarray*}
\end{MHint}

\end{MExercise}

\end{MXContent}


\MSubsection{Absolute Value Equations}
\MLabel{M02_Betragsgleichungen}

\begin{MIntro}
\MDeclareSiteUXID{VBKM02_Betragsgleichungen_Intro}
\MLabel{VBKM02_Betrag}
The absolute value $|x|$ assigns a variable $x\in\R$ its value without sign: If $x\geq 0$, then $|x|=x$, 
otherwise $|x|=-x$ (see figure).

\begin{center}
%%\MUGraphicsSolo{betrag1.png}{width=0.4\linewidth}{width:350px}\\
\MTikzAuto{%
\begin{tikzpicture}[x=1.0cm, y=1.0cm] 
\draw[black] (-3,0) -- (3,0) (0,-3) -- (0,3);
\foreach \x in {-3, -2, -1, 1, 2, 3}
\draw[shift={(\x,0)},color=black] (0pt,0pt) -- (0pt,-2.0pt) node[below=1.0pt] {\scriptsize $\x$};
\foreach \x in {-2.5, -1.5, ..., 2.5}
\draw[shift={(\x,0)},color=black] (0pt,0pt) -- (0pt,-1.0pt);
\foreach \y in {-3, -2, -1, 1, 2, 3}
\draw[shift={(0,\y)},color=black] (0pt,0pt) -- (-2.0pt,0pt) node[left=1.0pt] {\scriptsize $\y$};
\foreach \y in {-2.5, -1.5, ..., 2.5}
\draw[shift={(0,\y)},color=black] (0pt,0pt) -- (-1.0pt,0pt);
\draw[black] (-0.0pt,-0.5pt) node[anchor=north east] {\scriptsize $0$};
\clip(-3.0,-3.0) rectangle (3.0,3.0);
\draw[black, line width=1.0pt,color=red!50!black] (-3,3) -- (0,0) -- (3,3);
\end{tikzpicture}
}
\par
The absolute value $|x|$ as a function of $x$.
\end{center}

Absolute value equations are equations in which one absolute value or several absolute values occur. 
Problems arise since the absolute value is calculated by distinguishing the two cases  
$$
\left|{\,\text{term}\,}\right| \;\; =\;\; \MCases{\text{term} & \text{for}\;\text{term}\geq 0\\ -\text{term} & \text{for}\;\text{Term}<0}\,.
$$
For solving absolute value equations, these cases have to be solved step by step 
and analysed to find the solutions.

\begin{MExample}
Obviously, the absolute value equation $|x|=2$ has the solution set $\ML=\lbrace 2\MElSetSep -2\rbrace$.
Just as easy, it can be seen that $|x-1|=3$ has the solution set $\ML=\lbrace -2\MElSetSep 4\rbrace$.
\end{MExample}

As soon as beside the absolute value several other terms occur a case analysis is required.
In the following section we will explain in detail how this analysis is done and how it is 
written correctly since the case analysis will play an important role in the next modules.
\end{MIntro}

\begin{MXContent}{Carry out a Case Analysis}{Case Analysis}{STD}
\MDeclareSiteUXID{VBKM02_Fallunterscheidungen}

\begin{MInfo}
\MLabel{VBKM02_FallBetrag}
To solve an \MEntry{absolute value equation}{absolute value equation} two cases are distinguished:
\begin{itemize}
\item{For all values of $x$ for which the absolute value term is non-negative the absolute value can be omitted or
replaced by simple brackets, respectively.}
\item{For all values of $x$ for which the absolute value term is negative the term is bracketed and negated.}
\end{itemize}
Then, the solution sets from the case analyses will be restricted to satisfy the case conditions. 
Only if this procedure is finished for all cases, the solution subsets will be merged to the 
solution set of the initial equation. 
\end{MInfo}

For solving absolute value equations it is important to write down the solution steps correctly
and to distinguish the cases clearly.

The following video demonstrates a detailed written solution of the absolute value
equation $|2x-4|=6$ by case analysis.


\MVideo{vidbsp1}{Carry out a case analysis.\MCopyrightLabel{VBKM06_Video_Beispiel1}}
\ \\
\MCopyrightNotice{\MCCLicense}{FSZ}{MINT}{Dozentin: Dipl.-ing. Heike Herold}{VBKM06_Video_Beispiel1}

The case analysis presented in the video reads shortly
$$
|2x-4| \;=\; \MCases{2x-4 &\text{for}\;x\geq 2 \\ -2x+4 & \text{for}\;x<2} \;=\;\MCases{2x-4 &\text{for}\;x\geq 2 \\ -2x+4 & \text{otherwise}} \MDFPeriod
$$
\MInputHint{In einem Eingabefeld w�rde man das als \texttt{falls(x>=2;2*x-4;-2*x+4)} eintippen. Auch \texttt{falls(x<2;-2*x+4;2*x-4)} w�re richtig.}

\begin{MExercise}
Describe the values of the expression $2\cdot |x-4|$ by a case analysis:
\begin{center}
$\displaystyle 2\cdot |x-4|$ = \MLSimplifyQuestion{30}{falls(x>=4,2*(x-4),-2*(x-4))}{15}{x}{4}{128}{VBKM02ALTFALL1}.
\end{center}
\MInputHint{Enter the case analysis in the form \texttt{falls(BEDINGUNG;W1;W2)}, where \texttt{W1} is the value of the expression if the corresponding condition is satisfied. Do not use the absolute value function.}
\begin{MHint}{Solution}
\begin{eqnarray*}
2\cdot |x-4| \;=\; \MCases{2x-8 &\text{for}\;x\geq 4 \\ -2x+8 & \text{for}\;x<4}
\end{eqnarray*}
\end{MHint}
\end{MExercise}

\begin{MExercise}
Reproduce the steps shown in the video \MRef{VBKM06_Video_Beispiel1} to solve the absolute 
value equation $|6+3x|=12$.

The case analysis reads shortly $|6+3x|$ = \MLSimplifyQuestion{40}{for(x>=-2,6+3*x,-6-3*x)}{15}{x}{3}{128}{VBKM02ALTFALL2}.
\begin{MHint}{Solution}
\begin{eqnarray*}
|6+3x| \;=\; \MCases{6+3x &\text{for}\;x\geq -2 \\ -6-3x & \text{for}\;x<-2}
\end{eqnarray*}
\end{MHint}\\
\MInputHint{Enter the case analysis in the form \texttt{falls(BEDINGUNG;W1;W2)}, where \texttt{W1} is the value of the expression if the corresponding condition is satisfied.
You can copy one of the input examples into the input field and adapt it to the new equation.}
\ \\ \ \\
Finding the solution for each case and checking the case conditions leads to the solution set
\MEquationItem{$\ML$}{\MLParsedQuestion{10}{2,-6}{3}{LVI}} for the equation $|6+3x|=12$.

\begin{MHint}{Solution}
\begin{eqnarray*}
\ML \;=\; \{-6\MElSetSep 2\}
\end{eqnarray*}
\end{MHint}\\
\MInputHint{Mengen k�nnen in der Form \texttt{$\lbrace$a;b;c;\ldots$\rbrace$} eingegeben werden.}
\end{MExercise}

You can practise the stepwise solution of absolute value equations within the following exercise. 


\MDirectRouletteExercises{abs_equations.rtex}{VBKM02_ABSEQTRAINING}


\end{MXContent}

\begin{MXContent}{Mixed Equations}{Mixed Equations}{STD}
\MDeclareSiteUXID{VBKM02_GemischteGleichungen}

\begin{MInfo}
If an equation contains both absolute values and other expressions, the case analysis 
has to be done according to the absolute value terms and applied only to these.
\end{MInfo}

Finally, keep in mind to crosscheck the found solution sets with the case conditions.

\begin{MExample}
Solve the equation $|x-1|+x^2=1$. Here, the case analysis is as follows:
\begin{itemize}
\item{For $x\geq 1$, the absolute value bars can be replaced by normal brackets which results
in the quadratic equation $(x-1)+x^2=1$ that is transformed into the equation $x^2+x-2=0$.
Using the $p q$ formula we get the two solutions
\begin{eqnarray*}
x_1 & =& -\Mdfrac12-\sqrt{\Mdfrac94}\;=\; -2 \MDFPSpace, \\
x_2 & =& -\Mdfrac12+\sqrt{\Mdfrac94}\;=\; 1 \MDFPSpace
\end{eqnarray*}
of which only $x_2$ satisfies the case condition.
}
\item{For $x<1$, one obtains the quadratic equation $-(x-1)+x^2=1$ that is 
transformed into the equation $x^2-x=0$ or $x\cdot (x-1)=0$, respectively. The product representation 
indicates the two solutions $x_3=0$ and $x_4=1$. Because of the case condition only 
$x_3=0$ is a solution of the initial equation.
}
\end{itemize}
So, altogether $\ML=\lbrace 0\MElSetSep 1\rbrace$ is the solution set of the 
initial equation.
\end{MExample}

%TODO: Intervalle sind hier noch garnicht erklaert
%TODO: Kodierung von offenen Intervallgrenzen in "\MLIntervalQuestion" (?!)
\begin{MExercise}
Find the solution set of the mixed equation $|x-3|\cdot x=9$.
\begin{MExerciseItems}
\item{If $x$ is in the interval \MLIntervalQuestion{14}{[3,infty)}{5}{GIM1} the  
absolute value term is non-negative.\\ One obtains the quadratic equation 
\MEquationItem{\MLFunctionQuestion{15}{x^2-3*x-9}{5}{x}{5}{GIM2}}{$0$}.\\
The solution set is \MLParsedQuestion{35}{3/2-sqrt(45/4)\MElSetSep 3/2+sqrt(45/4)}{5}{GIM3}.\\
Only the solution \MLParsedQuestion{15}{3/2+sqrt(45/4)}{5}{GIM4} satisfies the case condition.}

\item{If $x$ is in the interval \MLIntervalQuestion{14}{(-infty,3)}{5}{GIM5} the absolute 
value term is negative.\\  One obtains the normalised quadratic equation 
\MEquationItem{\MLFunctionQuestion{15}{x^2-3*x+9}{5}{x}{5}{GIM6}}{$0$}.\\
The solution set is \MLParsedQuestion{30}{}{5}{GIM7}.}
\end{MExerciseItems}
%TODO: Text und Kodierung anpassen
\MInputHint{Enter open intervals in the form $(3;5)$ and closed intervals in the form $[3;5]$. Enter
``infinity'' as text or simply as \texttt{infty}. Do not use the notation $]a;b[$ fir open intervals. 
Sets can be entered by listing the elements in the form $\lbrace 1\MElSetSep 2\MElSetSep 3\rbrace$. For 
the set brackets enter AltGr+7 or AltGr+0, respectively.} 

So, altogether the solution set is \MEquationItem{$\ML$}{\MLParsedQuestion{20}{3/2+sqrt(45/4),3/2+sqrt(45/4)}{5}{GIM8}}.

\begin{MHint}{Solution}
If $x$ is in the interval $\left[3\MIntvlSep \infty\MoIr[\right]$ the absolute value term is non-negative
and one obtains the quadratic equation $x^2-3x-9=0$ with the solution set 
$\ML=\lbrace \Mtfrac32-\sqrt{\Mtfrac{45}{4}};\Mtfrac32+\sqrt{\Mtfrac{45}{4}}\rbrace$. Only the 
larger solution $\Mtfrac32+\sqrt{\Mtfrac{45}{4}}$ satisfies the condition $x\geq 3$. This can 
also be seen without any calculator by estimating $\sqrt{\Mtfrac{45}{4}}\geq\sqrt{\Mtfrac{36}{4}}=3$.
In contrast, if $x$ is in the interval $\MoIl[\left]-\infty\MIntvlSep 3\MoIr[\right]$
the absolute value term is negative. One obtains the normalised quadratic equation $x^2-3x+9=0$.
Because of $\Mtfrac14p^2-q<0$ in the $pq$ formula this equation is unsolvable. Hence, the initial
equation has only one solution $\Mtfrac32+\sqrt{\Mtfrac{45}{4}}$.
\end{MHint}
\end{MExercise}

\begin{MExercise}
Find the solutions of the mixed absolute value equation $3|2x+1|=|x-5|$ by visualising 
the different cases on the number line and finally solving the equation by case analysis. 
First, visualise the case analysis for each absolute value. 

The solution set is \MLParsedQuestion{10}{-8/5,2/7}{5}{PARSEDQUEST1}.

\begin{MHint}{Solution}
Visualising the different cases for the expressions $|2x+1|$ and $|x-5|$ above each other 
indicates all cases to be distinguished:

%%\MGraphics{fallunterscheidung3.png}{scale=1}{Graphische Darstellung der drei F�lle.\MCopyrightLabel{VBKM02_Grafik_Fallunterscheidung3}}
%%\MCopyrightNotice{\MCCLicense}{NONE}{VEMINT}{Im Rahmen des VE\&MINT-Projekts}{VBKM02_Grafik_Fallunterscheidung3}
\begin{center}
\MTikzAuto{%
\begin{tikzpicture}[x=0.35cm, y=1.0cm] 
\fill[color=yellow!20!white] (-7,-1.2) rectangle (15,6.2);
\foreach \sy in {0.0cm,2.5cm,5.0cm} {
\begin{scope}[xshift=0,yshift=\sy]
\draw[-stealth',black] (-6,0) -- (13.5,0);
\foreach \x in {-4, 0, 4, 8, 12}
\draw[shift={(\x,0)},color=black] (0pt,0pt) -- (0pt,3.0pt) (0pt,0pt) node[below=1.0pt] {\small $\x$};
\foreach \x in {-5, -3, -2, -1, 1, 2, 3, 5, 6, 7, 9, 10, 11}
\draw[shift={(\x,0)},color=black] (0pt,0pt) -- (0pt,1.0pt);
\end{scope}
\def\rdst{0.2}
\def\tdst{0.6}
\def\vdst{0.4}
\draw[color=red,line width=2.2pt] (-6.0,\rdst) -- (-0.5,\rdst);
\draw[color=green!60!black,line width=2.2pt] (-0.5,\rdst) -- (5.0,\rdst);
\draw[color=orange,line width=2.2pt] (5.0,\rdst) -- (12.0,\rdst);
\draw[color=red,line width=2.2pt] (-3.25,\tdst) node {Case (1)};
\draw[color=green!60!black,line width=2.2pt] (2.25,\tdst) node {Case (2)};
\draw[color=orange,line width=2.2pt] (8.5,\tdst) node {Case (3)};
\draw[color=blue,line width=0.8pt] (-0.5,\vdst) -- (-0.5,-0.5) node[below=0pt] {$-\frac{1}{2}$} ;
\draw[color=blue,line width=0.8pt] (5.0,\vdst) -- (5.0,-0.5) node[below=0pt] {$5$} ;
\begin{scope}[xshift=0,yshift=2.5cm]
\draw[color=red,line width=2.2pt] (-6.0,\rdst) -- (5.0,\rdst);
\draw[color=orange,line width=2.2pt] (5.0,\rdst) -- (12.0,\rdst);
\draw[color=red,line width=2.2pt] (-0.50,\tdst) node {$x-5<0$};
\draw[color=orange,line width=2.2pt] (8.5,\tdst) node {$x-5>0$};
\draw[color=blue,line width=0.8pt] (5.0,\vdst) -- (5.0,-0.5) node[below=0pt] {$5$} ;
\end{scope}
\begin{scope}[xshift=0,yshift=5cm]
\draw[color=red,line width=2.2pt] (-6.0,\rdst) -- (-0.5,\rdst);
\draw[color=orange,line width=2.2pt] (-0.5,\rdst) -- (12.0,\rdst);
\draw[color=red,line width=2.2pt] (-3.25,\tdst) node {$2x+1<0$};
\draw[color=orange,line width=2.2pt] (5.75,\tdst) node {$2x+1>0$};
\draw[color=blue,line width=0.8pt] (-0.5,\vdst) -- (-0.5,-0.5) node[below=0pt] {$-\frac{1}{2}$} ;
\end{scope}
}
\end{tikzpicture}
}
\par
Illustration of the three cases
\end{center}

According to the figure above the following three cases have to be distinguished:
\begin{itemize}
\item{Case (1): For $x<-\Mtfrac12$ both terms in the absolute value terms are negative.}
\item{Case (2): For $-\Mtfrac12\leq x<5$ the term in the second absolute value term is negative but the 
term in the first one is not.}
\item{Case (3): For $5\leq x$ both terms in the absolute value terms are non-negative.}
\item{Obviously, there is no $x$ for which the first term is negative and the second 
term is non-negative.}
\end{itemize}

So, the solutions can be summarised:
\begin{itemize}
\item{In case (1), both absolute values reverse the sign of the terms:\\ $3|2x+1|=|x-5|\;\Leftrightarrow\;3(-(2x+1)) = -(x-5)$.\\
This equation has the solution $x=-\Mtfrac85$ satisfying the case condition.}
\item{In case (2), only the second absolute value reverses the sign of the term: \\$3|2x+1|=|x-5|\;\Leftrightarrow\;3(2x+1) = -(x-5)$.\\
This equation has the solution $x=\Mtfrac27$ satisfying the case condition.}
\item{In case (3), the absolute value bars in both terms can be omitted (replaced by normal brackets): \\
 $3|2x+1|=|x-5|\;\Leftrightarrow\;3(2x+1) = (x-5)$.\\
This equation has the solution $x=-\Mtfrac85$, but this solution does \textit{not} 
satisfy the case condition. Thus, it will be discarded within its case analysis.}
\end{itemize}
Therefore, the solution set is $\lbrace -\Mtfrac85\MElSetSep \Mtfrac27\rbrace$.
\end{MHint}
\end{MExercise}


\end{MXContent}

\MSubsection{Final Test}
\MLabel{M02_Abschlusstest}

\begin{MTest}{Final Test Modul 2}
\MDeclareSiteUXID{VBKM02_Abschlusstest}

\begin{MExercise}
Find an absolute value term describing the following graph of a function as easy as possible:

%%\MUGraphics{abs2.png}{width=0.5\linewidth}{Funktionsgraph von $f(x)$.}{width:300px}
\begin{center}
\MTikzAuto{%
\begin{tikzpicture}[x=0.8cm, y=1.12cm] 
\draw[black] (-5,0) -- (5,0) (0,-5) -- (0,5);
\foreach \x in {-4, -2, 2, 4}
\draw[shift={(\x,0)},color=black] (0pt,0pt) -- (0pt,-3.0pt) node[below=1.0pt] {\normalsize $\x$};
\foreach \x in {-4.8, -4.4, ..., 5.0}
\draw[shift={(\x,0)},color=black] (0pt,0pt) -- (0pt,-1.5pt);
\foreach \y in {-4, -2, 2, 4}
\draw[shift={(0,\y)},color=black] (0pt,0pt) -- (-3.0pt,0pt) node[left=1.0pt] {\normalsize $\y$};
\foreach \y in {-4.8, -4.4, ..., 5.0}
\draw[shift={(0,\y)},color=black] (0pt,0pt) -- (-1.5pt,0pt);
%%\draw[black] (-0.0pt,-0.5pt) node[anchor=north east] {\small $0$};
\clip(-5.0,-5.0) rectangle (5.0,5.0);
\draw[black, line width=1.0pt,color=black] (-1,6) -- (3,-2) -- (5,2);
\end{tikzpicture}
}
\par
Graph of the function $f(x)$.
\end{center}

Answer: $f(x)=$ \MLSimplifyQuestion{20}{2*abs(x-3)-2}{5}{x}{5}{0}{KA0}\: .
\end{MExercise}

\begin{MExercise}
Solve the following equations:
\begin{MExerciseItems}
\item{$|2x-3|=8$ has the solution set \MLParsedQuestion{16}{11/2,-5/2}{5}{KA1}.}
\item{$|x-2|\cdot x=0$ has the solution set \MLParsedQuestion{16}{0,2}{5}{KA2}.} 
\end{MExerciseItems}
\MInputHint{Enter sets in the form \texttt{$\lbrace$ a;b;c;$\ldots\rbrace$}. 
Enter the empty set as $\lbrace\rbrace$.}
\end{MExercise}

\begin{MExercise}
A camera has a resolution of $6$ megapixels, i.e. -- for convenience -- 
6 million pixels and produces images in format $2:3$. Which size has a 
quadratic pixel on a print-out of format $(60$ cm$) \times (40$ cm$)$? 
Specify the side length of a pixel in millimetre. 

Answer: \MLParsedQuestion{10}{0.2}{5}{VPIXQ}\ \ \ \ (without the unit mm).
\end{MExercise}

\begin{MExercise}
Find the solution set of the mixed equation $|x-1|\cdot (x+1)=3$.\\
Answer: \MEquationItem{$\ML$}{\MLParsedQuestion{15}{2,2}{5}{KAX4}}.
\end{MExercise}


\end{MTest}


\newpage
\MPrintIndex

\end{document}

\MSection{Inequalities in one Variable}
\MLabel{VBKM03}
\MSetSectionID{ungl}

\begin{MSectionStart}
\MDeclareSiteUXID{VBKM03_START}

Inequalities arise by relating terms using one of the comparing symbols $\leq$, $<$, $\geq$, or $>$. Simple 
inequalities usually have intervals as their solution sets. But the solution of inequalities is often
more difficult than the solution of equations. Hence, specific types of inequalities will be explained
in more detail.

This module consists of:

\begin{itemize}
\item{Section~\MNRef{M03_Ungleichungen}: \MSRef{M03_Ungleichungen}{Inequalities and their Solution Sets},}
\item{Section~\MNRef{M03_Umformen}: \MSRef{M03_Umformen}{Transformation of Inequalities},}
\item{Section~\MNRef{M03_Betragsungleichungen}: \MSRef{M03_Betragsungleichungen}{Absolute Value Inequalities and 
Quadratic Inequalities},}
\item{and Section~\MNRef{M03_Abschlusstest}: \MSRef{M03_Abschlusstest}{Final Test}.}
\end{itemize}

\end{MSectionStart}

\MSubsection{Inequalities and their Solution Sets}
\MLabel{M03_Ungleichungen}

\begin{MIntro}

\begin{MInfo}
\MDeclareSiteUXID{VBKM03_UngleichungenIntro}
If two numbers are related by one of the \MEntry{comparing symbols}{comparing symbols} 
$\leq$, $<$, $\geq$, or $>$, a statement is generated that can be true or false depending on 
the numbers:
\begin{itemize}
\item{$a<b$ (reads: ``$a$ is strictly less than $b$'' or simply ``$a$ is less than $b$'') is true if the number $a$ is less than and not equal to $b$.}
\item{$a \leq b$ (reads: ``$a$ is less than $b$'') is true if $a$ is less than or equal to $b$.}
\item{$a>b$ (reads: ``$a$ is strictly greater than $b$'' or simply ``$a$ is greater than $b$'') is true if the number $a$ is greater and not equal to $b$.}
\item{$a \geq b$ (reads: ``$a$ is greater than $b$'') is true if the number $a$ is greater than or equal to $b$.}
\end{itemize}
\end{MInfo}

The relating symbols describe how the given values are related to each other on the number line: 
$a<b$ means that $a$ is to the left of $b$ on the number line.

\begin{MExample}
The statements $2<4$, $-12\leq 2$, $4>1$, and $3\geq 3$ are true,
but the statements $2<\sqrt2$ and $3>3$ are false.

\begin{center}
\MTikzAuto{%
\begin{tikzpicture}
% reelle Achse
\draw[->,color=black] (-1,0.0) -- (5,0.0);
\foreach \x in {-1, 0, 1, 2, 3, 4}
\draw[shift={(\x,0)},color=black] (0pt,2pt) -- (0pt,-2pt) node[below] {\footnotesize $\x$};
\draw (4.9,-0.3) node[] {$\mathbb{R}$};
% Enden:
\draw [fill = blue] (2,0) circle (1.5pt);
\draw [fill = blue] (4,0) circle (1.5pt);
\end{tikzpicture}
}%

On the number line, the number $2$ is to the left of the number $4$, thus $2<4$.
\end{center}

\end{MExample}

Here, $a<b$ means the same as $b>a$, likewise $a\leq b$ means the same as $b\geq a$. But it
should be noted that the opposite of the statement $a<b$ is the statement $a\geq b$ and not
$a>b$. If terms with a variable occur in an inequality, the problem is to find the number range
of the variable such that the inequality is true. 
\end{MIntro}

\begin{MXContent}{Solving simple Inequalities}{Solving}{STD}
\MDeclareSiteUXID{VBKM03_EinfacheUngleichungen}

If the variable occurs isolated in the inequality, the solution set is an interval, see also info
box \MRef{VBKM01_Intervalle}: 


\begin{MInfo}
\MLabel{M03_Aufloesungen}
The \MEntry{solved inequalities}{inequalities (solved)} 
have the following \MEntry{intervals}{intervals} as their solution sets:
\begin{itemize}
\item{$x< a$ has the solution set $\MoIl[\left] -\infty\MIntvlSep a\MoIr[\right]$, i.e.\ all $x$ less than $a$.}
\item{$x\leq a$ has the solution set $\MoIl[\left] -\infty\MIntvlSep a\right]$, 
i.e. all $x$ less than or equal to $a$.}
\item{$x> a$ has the solution set $\MoIl[\left] a\MIntvlSep \infty\MoIr[\right]$, i.e.\ all
 $x$ greater than $a$.}
\item{$x\geq a$ has the solution set $\left[a\MIntvlSep \infty\MoIr[\right]$, i.e.\ all $x$
greater than or equal to $a$.}
\end{itemize}
Here, $x$ is the variable and $a$ is a specific value. 
If the variable does not occur in the inequality anymore, the solution set is either
$\R=\MoIl[\left] -\infty\MIntvlSep \infty\MoIr[\right]$ if the inequality is satisfied, 
or the empty set $\lbrace \rbrace$ if the inequality is not satisfied.
\end{MInfo}

The symbol $\infty$ means \MEntry{infinity}{infinity}. A finite interval has the form 
$\MoIl a\MIntvlSep b\MoIr$ which reads ``all numbers between $a$ and $b$''. If the interval
is to be finite only on one side, the other interval boundary can be replaced by the symbol 
$\infty$ (right-hand side) or $-\infty$ (left-hand side).

As for equations one tries to find a solved inequality by applying transformations that do
not change the solution set. From the solved inequality the solution set can be easily seen.


\begin{MInfo}
\MLabel{VBKM03_AequivalenzumformungenUngleichungen}
To obtain a solved inequality from an unsolved inequality the following 
\MEntry{equivalent transformations}{equivalent transformations (inequality)} are allowed:
\begin{itemize}
\item{adding a constant to both sides of the inequality: $a<b$ is equivalent to $a+c<b+c$.}
\item{multiplying both sides of the inequality by a positive constant: $a<b$ 
is equivalent to $a\cdot c<b\cdot c$ if $c>0$.}
\item{multiplying both sides of the inequality by a negative constant and inverting the 
comparing symbol: $a<b$ is equivalent to $a\cdot c>b\cdot c$ if $c<0$.}
\end{itemize}
\end{MInfo}

\begin{MExample}
The inequality $-\frac34x-\frac12<2$ is solved stepwise by the above transformations:
\begin{eqnarray*}
&&-\frac34x-\frac12 < 2 \;\; \MSep +\frac12\ \\
&\Leftrightarrow&-\frac34x < 2+\frac12 \;\; \MSep \cdot\left({-\frac43}\right)\ \\
&\Leftrightarrow&x > -\frac43\left({2+\frac12}\right) \;\; \MSep \;\text{simplifying}\\
&\Leftrightarrow&x >  -\frac{20}{6} \;=\; -\frac{10}{3} \MDFPeriod
\end{eqnarray*}
So, the initial inequality has the solution set 
 $\MoIl[\left] -\frac{10}{3}\MIntvlSep \infty\MoIr[\right]$. 
Importantly, multiplying the inequality by the negative number $-\frac43$ inverts the 
comparing symbol.
\end{MExample}

\begin{MExercise}
Are the following inequalities true or false?

\begin{MQuestionGroup}
\begin{tabular}{lll}
\MCheckbox{0}{UG1} & \ \ &  $\frac12>1-\frac13$\\
\MCheckbox{1}{UG2} & \ \ & $a^2\geq 2a b-b^2$ (where $a$ and $b$ are unknown numbers)\\
\MCheckbox{1}{UG3} & \ \ & $\frac12<\frac23<\frac34$\\
\MCheckbox{0}{UG4} & \ \ & Let $a<b$, then also $a^2<b^2$.
\end{tabular}
\end{MQuestionGroup}
\MGroupButton{Check input}

\begin{MHint}{Solution}
The first inequality can be simplified to $\frac12>\frac23$, which, after multiplying by $6$, 
is equivalent to $3>4$. This statement is false. The second inequality can be simplified by 
collecting all numbers on the left-hand side: $a^2-2a b+b^2\geq 0$. Since $a^2-2a b+b^2=(a-b)^2$,
this statement is true for all $a$ and $b$. Multiplying the third chain of inequalities by the
least common denominator $12$ results in the chain of inequalities $6<8<9$. This statement is true.
In contrast, the last statement is false, since for example, for $a=-1$ and $b=1$, the term
$a^2=1$ is not less than $b^2=1$. Taking the square of terms is not an equivalent transformation.
\end{MHint}
\end{MExercise}


\begin{MExercise}
Find the solution sets of the following inequalities.
\begin{MExerciseItems}
\item{$2x+1> 3x-1$ has the solution interval \MEquationItem{$\ML$}{\MLIntervalQuestion{30}{(-infty,2)}{5}{TXH1}}.}
\item{$-3x-\frac12\leq x+\frac12$ has the solution interval \MEquationItem{$\ML$}{\MLIntervalQuestion{30}{[-1/4,infty)}{5}{TXH2}}.}
\item{$x-\frac12\leq x+\frac12$ has the solution interval \MEquationItem{$\ML$}{\MLIntervalQuestion{30}{(-infty,infty)}{5}{TXH3}}.}
\end{MExerciseItems}
\MInputHint{Enter the intervals in the form \texttt{(a;b)}, for the interval boundaries also fractions and
\texttt{infinity} or \texttt{-infinity} can be used. Take care whether the interval boundaries are included 
or excluded.}

\begin{MHint}{Solution}
Transformation of the first inequality results in
\begin{eqnarray*}
&& 2x+1 > 3x-1\;\; \MSep+1\ \\
&\Leftrightarrow& 2x+2 > 3x\;\; \MSep-2x\ \\
&\Leftrightarrow&2 > x
\end{eqnarray*}
and hence the solution interval is $\ML=\MoIl[\left] -\infty\MIntvlSep 2\MoIr[\right]$. 
Transformation of the second inequality results in
\begin{eqnarray*}
&&-3x-\frac12\leq x+\frac12 \;\; \MSep +3x-\frac12\ \\
&\Leftrightarrow&-1\leq 4x \;\; \MSep \cdot \frac14\ \\
&\Leftrightarrow&-\frac14\leq  x
\end{eqnarray*}
and hence $\ML=\left[-\frac14\MIntvlSep \infty\MoIr[\right]$. 
Transformation of the third inequality results in
\begin{eqnarray*}
&&x-\frac12\leq x+\frac12\;\; \MSep-x\ \\
&\Leftrightarrow&-\frac12\leq \frac12 \MDFPeriod
\end{eqnarray*}
This statement does not depend on $x\in\R$ and is always true, 
thus the solution set is $\ML=\R=\MoIl[\left] -\infty\MIntvlSep \infty\MoIr[\right]$.
\end{MHint}
\end{MExercise}

\begin{MInfo}
An inequality in one variable $x$ is \MEntry{linear}{inequality (linear)} if on both sides of the 
inequality only multiples of $x$ and constants occur. Each linear inequality can be transformed 
into a solved inequality by one of the equivalent transformations described in the info box
\MRef{M03_Aufloesungen}.
\end{MInfo}

\end{MXContent}

\begin{MXContent}{Specific Transformations}{Specific Transformations}{STD}
\MDeclareSiteUXID{VBKM03_SpezielleUmformungen}
The following equivalent transformations are useful if the variable occurs 
in the denominator of an expression. But they can only be applied under certain
restrictions:

\begin{MInfo}
Under the restriction that none of the occurring denominators is zero (the corresponding variable values are
 never solutions) and the fractions on both sides have the same sign, the reciprocal can be taken
on both sides of the inequality while inverting the comparing symbol.
\end{MInfo}

\begin{MExample}
For example, the inequality $\frac1{2x}\leq \frac1{3x}$ is equivalent to $2x\geq 3x$
(comparing symbol inverted) as long as $x\not=0$. The new inequality has the solution
set $\MoIl[\left] -\infty\MIntvlSep 0\right]$. However, since the value $x=0$ was excluded (and 
does not belong to the domain of the initial inequality either) the solution set of 
$\frac1{2x}\leq \frac1{3x}$ is $\ML=\MoIl[\left] -\infty\MIntvlSep 0\MoIr[\right]$.
\end{MExample}

\begin{MExercise}
Find the solution sets of the following inequalities.
\begin{MExerciseItems}
\item{$\frac1x\geq\frac13$ has the solution set \MEquationItem{$\ML$}{\MLIntervalQuestion{20}{(0;3]}{3}{KKL1}}.}
\item{$\frac1x<\frac1{\sqrt{x}}$ has the solution set \MEquationItem{$\ML$}{\MLIntervalQuestion{20}{(1;infty)}{3}{KKL2}}.}
\end{MExerciseItems}

\begin{MHint}{Solution}
For the first inequality, the value $x=0$ is not in the domain, hence this value is excluded. For 
$x>0$, taking the reciprocal while inverting the comparing symbol is allowed and results in
$x\leq 3$. Together with the condition above the solution interval is $\ML=\MoIl 0\MIntvlSep 3]$.
For $x<0$ the reciprocal rule cannot be applied. However, it can be seen, even without any rule, that
none of the values $x<0$ can be a solution, since then $\frac1x$ is negative as well and not greater than
or equal to $\frac13$.

The domain of the second inequality is $\MoIl 0\MIntvlSep \infty\MoIr$, since 
only for these values of $x$ taking the square root is
defined and only for $x\neq 0$ the denominators are non-zero. On the domain, 
taking the reciprocal while inverting the comparing symbol is 
allowed and results in $x>\sqrt{x}$. Since $\sqrt{x}>0$, the inequality can be 
divided by $\sqrt{x}$ resulting in $\sqrt{x}>1$. This inequality has the solution
set $\ML=\MoIl 1\MIntvlSep \infty\MoIr$ which occurs also in the domain.
\end{MHint}

\end{MExercise}

Please note for the last part of the exercise:

\begin{MInfo}
Taking the square on both sides of an inequality is not an equivalent transformation and 
possibly does change the solution set.
\end{MInfo}

For example, $x=-2$ is no solution of $x>\sqrt{x}$, but indeed a solution of $x^2>x$. However,
this transformation can be applied if the case analysis for the transformation
is carried out correctly and the domain of the initial inequality is taken into account. This method is
described in more detail in the next section.
\end{MXContent}

\MSubsection{Transformation of Inequalities}
\MLabel{M03_Umformen}

\begin{MXContent}{Transformation with Case Analysis}{Case Analysis}{STD}
\MDeclareSiteUXID{VBKM03_UmformungenFallunterscheidungen}
The simple linear transformations described in the previous section are equivalent transformations.
They do not change the solution set of the corresponding inequality. For nonlinear
inequalities advanced solution methods are required. Usually, these methods need 
a case analysis depending on the sign, since, in contrast to the situation for
equations described in Modul~\MNRef{VBKM02}, now also the inequality can be inverted during 
transformation.

 
\begin{MInfo}
If an inequality is multiplied by a term in which the variable $x$ occurs, a case analysis 
is required and for each case the transformation has to be considered separately: 

\begin{itemize}
\item{For those values of $x$, for which the multiplied term is positive, the comparing symbol
of the inequality is unchanged.}
\item{For those values of $x$, for which the multiplied term is negative, the comparing symbol
of the inequality is inverted.}
\item{The case that the multiplied term is zero has to be excluded during the transformation and
has to be considered separately, if necessary.}
\end{itemize}
\ \\ \ \\
The solution sets found in the individual cases have to be checked with respect to the case conditions
as described for the solution of \MSRef{VBKM02_FallBetrag}{absolute value equations}.
\end{MInfo}

In contrast, adding terms in which the variable occurs, does not require a case analysis. Usually, 
transformations involving case analyses are mandatory if the variable occurs in the denominator or 
in a composite term.


\begin{MExample}
The inequality $\frac1{2x}\leq 1$ can be simplified by multiplying both sides of the inequality
by the term $2x$:

\begin{itemize}
\item{Under the condition $x>0$ this results in the new inequality $1\leq 2x$. It has the solution set
$\ML_1=\left[\frac12\MIntvlSep \infty\MoIr[\right]$. The condition $x>0$ is satisfied by all elements of the solution 
set.}
\item{Under the condition $x<0$ this results in the new inequality $1\geq 2x$. It has the solution set
 $\MoIl[\left] -\infty\MIntvlSep \frac12\right]$. Because of the additional
condition $x<0$ only the elements of the set 
$\ML_2=\MoIl[\left] -\infty\MIntvlSep 0\MoIr[\right]$ are solutions.}
\item{The single case $x=0$ is no solution since this value is not in the domain of the inequality. 
In this case multiplying the inequality by $x$ is not allowed.}
\end{itemize}
So, altogether one obtains the union set 
$\ML=\ML_1\cup \ML_2=\R\MSetminus\left[0\MIntvlSep \frac12\MoIr[\right]$ as solution set:
\ \\ \ \\
\begin{center}
\MTikzAuto{%
\begin{tikzpicture}
% reelle Achse
\draw[->,color=black] (-1,0.0) -- (5,0.0);
\foreach \x in {-1, 0, 1, 2, 3, 4}
\draw[shift={(\x,0)},color=black] (0pt,2pt) -- (0pt,-2pt) node[below] {\footnotesize $\x$};
\draw (4.9,-0.3) node[] {$\mathbb{R}$};
% Enden:
\draw [line width=2.0pt,color=blue] (-1,0.0)-- (0,0.0);
\draw [line width=2.0pt,color=blue] (0.5,0.0)-- (5,0.0);
\draw [fill = blue] (0.5,0) circle (1.5pt);
\draw [fill = white] (0,0) circle (1.5pt);
\end{tikzpicture}
}
\end{center}
\end{MExample}

As in Modul~\MNRef{VBKM02} the following statement holds for the solution set.

\begin{MInfo}
The cases have to be chosen such that all elements of the domain of the inequality are covered. 
For the solution set in an individual case, it has to be checked that the solution set satisfies the 
corresponding case condition. For any case, the resulting solution set has to be reduced to
the solution subset satisfying the case condition. The union of the solution sets for the individual cases
is the solution set of the initial inequality.
\end{MInfo}

\end{MXContent}

\begin{MExercises}
\MDeclareSiteUXID{VBKM03_Fallunterscheidungen_Exercises}
If the inequality is multiplied by a composite term, it has to be investigated precisely for which values
of $x$ the case analysis has to be done:

\begin{MExercise}
Find the solution set of the inequality $\frac1{4-2x}<3$. 
The domain of the inequality is $D=\R\MSetminus \lbrace 2\rbrace$ since only for these
values of $x$ the denominator is non-zero. If the inequality is multiplied by the term
$4-2x$, three cases have to be distinguished. Fill in the blanks in the following text 
accordingly:

\begin{MExerciseItems}
\item{On the interval \MLIntervalQuestion{15}{(-infty,2)}{4}{GOM1} the term is positive, the comparing symbol 
remains unchanged, and the new inequality reads $1\:<\:$\MLSimplifyQuestion{15}{3*(4-2*x)}{5}{x}{5}{0}{SIMPLE2}.
Linear transformations result in the solution set 
\MEquationItem{$\ML_1$}{\MLIntervalQuestion{20}{(-infty,11/6)}{3}{MIXGOM}}. 
The elements of this set satisfy the case condition.}
\item{On the interval \MLIntervalQuestion{15}{(2,infty)}{4}{GOM2} the term is negative, 
the comparing symbol is inverted. Initially, the new inequality has the solution set 
\MLIntervalQuestion{20}{(11/6,infty)}{4}{INT1}, because of the case condition only the 
subset \MEquationItem{$\ML_2$}{\MLIntervalQuestion{20}{(2,infty)}{4}{GOM3}} is allowed.}
\item{The single value $x=2$ is no solution of the initial inequality since 
it is not in \MLQuestion{25}{domain}{UGX}.}
\end{MExerciseItems}

Sketch the solution set of the inequality and indicate the boundary points.

\begin{MHint}{Solution}
On the interval $\MoIl[\left] -\infty\MIntvlSep 2\MoIr[\right]$ the term is positive. 
The corresponding solution set is $\MoIl[\left] -\infty\MIntvlSep \frac{11}{6}\MoIr[\right]$.
In contrast, on the interval $\MoIl 2\MIntvlSep \infty\MoIr$ the term is negative, 
the comparing symbol is inverted. Initially, the new inequality has the solution set
 $\MoIl[\left] \frac{11}{6}\MIntvlSep \infty\MoIr[\right]$, because of the case condition
$x>2$ only the subset $\ML_2=\MoIl 2\MIntvlSep \infty\MoIr$ is allowed.
So, altogether the union set $\ML=\ML_1\cup\ML_2=\R\MSetminus \left[\frac{11}6\MIntvlSep 2\right]$
is the solution set of the initial inequality excluding the boundary points:

\begin{center}
\MTikzAuto{%
\begin{tikzpicture}
% reelle Achse
\draw[->,color=black] (-1,0.0) -- (5,0.0);
\foreach \x in {-1, 0, 1, 2, 3, 4}
\draw[shift={(\x,0)},color=black] (0pt,2pt) -- (0pt,-2pt) node[below] {\footnotesize $\x$};
\draw (4.9,-0.3) node[] {$\mathbb{R}$};
% Enden:
\draw [line width=2.0pt,color=blue] (-1,0.0)-- (1.83,0.0);
\draw [line width=2.0pt,color=blue] (2,0.0)-- (5,0.0);
\draw [fill = white] (1.83,0) circle (1.5pt);
\draw [fill = white] (2,0) circle (1.5pt);
\end{tikzpicture}
}
\end{center}
\end{MHint}
\end{MExercise}

\begin{MExercise}
The solution set of the inequality $\frac{x-1}{x-2}\leq 1$ is 
\MEquationItem{$\ML$}{\MLIntervalQuestion{20}{(-infty,2)}{4}{IGU1}}.

\begin{MHint}{Solution}
The domain of the inequality is $D=\R\MSetminus\lbrace 2\rbrace$.

\begin{itemize}
\item{For $x>2$, multiplying the inequality by the term $x-2$ results in $x-1\leq x-2$, 
which is equivalent to the false statement $-1\leq -2$. Thus, this case does not contribute a solution to the solution set.}
\item{For $x<2$, multiplying the inequality by the term $x-2$ results in $x-1\geq x-2$, 
which is equivalent to the true statement $-1\geq -2$. 
Because of the case condition the solution interval for this case is only 
$\ML_2=\MoIl[\left] -\infty\MIntvlSep 2\MoIr[\right]$.}
\item{The single value $x=2$ is no solution.}
\end{itemize}
So, altogether the solution set is 
 $\ML=\MoIl[\left] -\infty\MIntvlSep 2\MoIr[\right]$ 
excluding the boundary points (even though the comparing symbol $\leq$ occurred in the initial
inequality).
\end{MHint}
\end{MExercise}


\begin{MExercise}
The solution set of the inequality $\frac1{1-\sqrt{x}}<1+\sqrt{x}$ is \MEquationItem{$\ML$}{\MLIntervalQuestion{20}{(1,infty)}{4}{IGU2}}.
\ \\ \ \\
\begin{MHint}{Solution}
The domain of the inequality is $D=[0\MIntvlSep \infty\MoIr\MSetminus \lbrace 1\rbrace$
since only for these values of $x$ the square root is defined and the denominator is non-zero.
\begin{itemize}
\item{For $0\leq x<1$, multiplying the inequality by the term $1-\sqrt{x}$ results in
  $1<(1+\sqrt{x})(1-\sqrt{x})$, which is equivalent to $1<1-x$. This 
inequality is satisfied for $x<0$, but these values of $x$ violate the case condition and 
thus, they are not in the solution set.}
\item{For $x>1$, multiplying the inequality by the term $1-\sqrt{x}$ results in $1>1-x$, 
which is equivalent to $x>0$. But only the values of $x$ in the interval
  $\MoIl 1\MIntvlSep \infty\MoIr$ satisfy the case condition, hence $\ML=\MoIl 1\MIntvlSep \infty\MoIr$ 
is the only solution interval of the initial inequality.}
\item{The single value $x=1$ is no solution.}
\end{itemize}
\end{MHint}
\end{MExercise}


\end{MExercises}

\MSubsection{Absolute Value Inequalities and Quadratic Inequalities}
\MLabel{M03_Betragsungleichungen}

\begin{MIntro}
\MDeclareSiteUXID{VBKM03_Betragsungleichungen_Intro}
As in the approach in Modul~\MNRef{VBKM02} and in the previous section 
\MEntry{absolute values}{inequalities (absolute values)} in inequalities are solved 
by a case analysis:

\begin{MInfo}
To solve an \MEntry{absolute value inequality}{absolute value inequality} two cases are distinguished:

\begin{itemize}
\item{For those values of $x$, for which the absolute value term is non-negative the absolute value can be omitted or
replaced by simple brackets, respectively.}
\item{For those values of $x$, for which the absolute value term is negative the term is bracketed and negated.}
\end{itemize}
\ \\
Then, the solution sets arising from the case analysis will be restricted as described in the 
\MSRef{VBKM02_FallBetrag}{previous module} and merged to the solution set of the initial inequality. 
\end{MInfo}

\begin{MExample}
To solve the absolute value inequality $|4x-2|<1$ two cases are distinguished:
\begin{itemize}
\item{For $x\geq \frac12$, the absolute value term is non-negative: 
In this case the inequality is equivalent to $(4x-2)<1$ or $x<\frac34$, respectively. 
Because of the case condition the solution set is only 
$\ML_1=\left[\frac12\MIntvlSep \frac34\MoIr[\right]$ in this case.}
\item{For $x<\frac12$, the absolute value term is negative: 
In this case the inequality is equivalent to $-(4x-2)<1$ or $x>\frac14$, respectively. 
Only the subset $\ML_2=\MoIl[\left] \frac14\MIntvlSep \frac12\MoIr[\right]$ 
satisfies the case condition and is the solution set.}
\end{itemize}
The union of the two solution intervals results in the solution set
$\ML=\MoIl[\left] \frac14\MIntvlSep \frac34\MoIr[\right]$ for the initial absolute value inequality:

\begin{center}
\MTikzAuto{%
\begin{tikzpicture}
% reelle Achse
\draw[->,color=black] (-1,0.0) -- (5,0.0);
\foreach \x in {-1, 0, 1, 2, 3, 4}
\draw[shift={(\x,0)},color=black] (0pt,2pt) -- (0pt,-2pt) node[below] {\footnotesize $\x$};
\draw (4.9,-0.3) node[] {$\mathbb{R}$};
% Enden:
\draw [line width=2.0pt,color=blue] (0.25,0.0)-- (0.75,0.0);
\draw [fill = white] (0.25,0) circle (1.5pt);
\draw [fill = white] (0.75,0) circle (1.5pt);
\end{tikzpicture}
}
\end{center}
\end{MExample}

\begin{MExercise}
To solve the absolute value inequality $|x-1|<2|x-1|+x$ two cases are distinguished:
\begin{MExerciseItems}
\item{On the interval \MLIntervalQuestion{20}{[1,infty)}{3}{UGL1}, both
terms in the absolute value terms are non-negative. 
The solution set of the inequality is in this case 
\MEquationItem{$\ML_1$}{\MLIntervalQuestion{20}{[1,infty)}{3}{UGL2}}.}
\item{On the interval \MLIntervalQuestion{20}{(-infty,1)}{3}{UGL3}, both
terms in the absolute value terms are negative. 
The solution set of the inequality is in this case
\MEquationItem{$\ML_2$}{\MLIntervalQuestion{20}{(-infty,1)}{3}{UGL4}}.}
\end{MExerciseItems}
The union of the two intervals results in the solution interval 
\MEquationItem{$\ML$}{\MLIntervalQuestion{25}{(-infty,infty)}{4}{UGL5}}.
\ \\ \ \\
\begin{MHint}{Solution}
For $x\in [1\MIntvlSep \infty\MoIr$, both terms in the absolute value terms are non-negative, 
one obtains the inequality $x-1<2(x-1)+x$, which is equivalent to $x>\frac12$. 
Because of the case condition one obtains $\ML_1=[1\MIntvlSep \infty\MoIr$ as solution set.
For $x\in\MoIl[\left] -\infty\MIntvlSep 1\MoIr[\right]$, 
both terms in the absolute value terms are negative and
one obtains $-(x-1)<-2(x-1)+x$. 
This inequality is equivalent to the inequality $x-1<x$ which is
always true. Thus,
the solution set for the second case is
$\ML_2=\MoIl[\left] -\infty\MIntvlSep 1\MoIr[\right]$.
\ \\ \ \\
Since $\ML=\ML_1\cup \ML_2=\R=\MoIl[\left] -\infty\MIntvlSep \infty\MoIr[\right]$ 
the inequality is always satisfied.
\end{MHint}
\end{MExercise}

\end{MIntro}

\begin{MXContent}{Quadratic Absolute Value Inequalities}{Quadratic Inequalities}{STD}
\MDeclareSiteUXID{VBKM03_QuadratischeUngleichungen}
\begin{MInfo}
An inequality is called \MEntry{quadratic}{inequality (quadratic)} in $x$ 
if it can be transformed into $x^2 + p x + q < 0$ (other comparing symbols are allowed).
\end{MInfo}
\ \\ \ \\
Hence, quadratic inequalities can be solved in two ways: by investigating the roots 
and the opening behaviour of the polynomial and by completing the square. Often completing
the square is simpler:
 

\begin{MInfo}
To solve an inequality by \MEntry{completing the square}{completing the square (inequalities)} 
one tries to transform it into the form $(x+a)^2<b$. Taking the square root then results
in the absolute value inequality $|x+a|<\sqrt{b}$ with the solution set 
$\MoIl[\left] -a-\sqrt{b}\MIntvlSep -a+\sqrt{b}\MoIr[\right]$ if $b\geq 0$. Otherwise 
the inequality is unsolvable.

The inverted inequality $|x+a|>\sqrt{b}$ has the solution set
$\MoIl[\left] -\infty\MIntvlSep -a-\sqrt{b}\MoIr[\right]\cup \MoIl[\left] -a+\sqrt{b}\MIntvlSep \infty\MoIr[\right]$. 
For $\leq$ and $\geq$ the corresponding boundary points have to be included.
\end{MInfo}

Always note the calculation rule $\sqrt{x^2}=|x|$ described in Modul~\MNRef{VBKM01}.

\begin{MExample}
Find the solution of the inequality $2x^2\geq 4x+2$. Collecting the terms on the left-hand side and dividing 
the inequality by $2$ results in $x^2-2x-1\geq0$. Completing the square on the 
left-hand side to the second binomial formula results in the equivalent inequality $x^2-2x+1\geq 2$
or $(x-1)^2\geq 2$, respectively. Taking the square root results in the absolute value 
inequality $|x-1|\geq\sqrt{2}$ with the solution set 
$\ML=\MoIl[\left] -\infty\MIntvlSep 1-\sqrt{2}\right]\cup \left[1+\sqrt{2}\MIntvlSep \infty\MoIr[\right]$.
\end{MExample}

On the other hand, the inequality $x^2-2x-1\geq0$ can be investigated as follows:
The left-hand side describes a parabola opened upwards. The roots $x_{1,2}=1\pm \sqrt2$ 
can be found using the $pq$ formula:

\begin{center}
%%\MUGraphicsSolo{parabelu.png}{width=0.4\linewidth}{width:400px}
\MTikzAuto{%
\begin{tikzpicture}[x=1.0cm, y=1.0cm,scale=1.50] 
\draw[black] (-1,0) -- (3,0) (0,-2) -- (0,2);
\foreach \x in {-1, 1, 2, 3}
\draw[shift={(\x,0)},color=black] (0pt,0pt) -- (0pt,-2.0pt) node[below=1.0pt] {\scriptsize $\x$};
\foreach \x in {-0.5, 0.5, ..., 3.0}
\draw[shift={(\x,0)},color=black] (0pt,0pt) -- (0pt,-1.0pt);
\foreach \y in {-2, -1, 1, 2}
\draw[shift={(0,\y)},color=black] (0pt,0pt) -- (-2.0pt,0pt) node[left=1.0pt] {\scriptsize $\y$};
\foreach \y in {-1.5, -0.5, 0.5, 1.5}
\draw[shift={(0,\y)},color=black] (0pt,0pt) -- (-1.0pt,0pt);
\draw[black] (-0.0pt,-0.0pt) node[anchor=north east] {\scriptsize $0$};
\clip(-1.0,-3.0) rectangle (3.0,2.0);
\draw[smooth,samples=21,domain=-1:3, line width=1.0pt,color=red!50!black] plot(\x,{\x*\x-2*\x-1});
\end{tikzpicture}
}
\end{center}
Since the parabola opens upwards, the inequality $x^2-2x-1\geq0$ is satisfied by the
values of $x$ in the parabola branches left and right to the roots, i.e. by the set 
$\ML=\MoIl[\left] -\infty\MIntvlSep 1-\sqrt{2}\right]\cup \left[1+\sqrt{2}\MIntvlSep \infty\MoIr[\right]$.

\begin{MInfo}
\MLabel{M03_InfoFormen}
Depending on the roots of $x^2+ p x + q$, the opening of the parabola and the 
comparing symbol, the quadratic inequality $x^2 +p x +q <0$ (including other comparing symbols) 
has one of the following solution sets:

\begin{itemize}
\item{the set of real numbers $\R$,}
\item{two branches $\MoIl[\left] -\infty\MIntvlSep x_1\MoIr[\right]\cup \MoIl[\left] x_2\MIntvlSep \infty\MoIr[\right]$ (including the boundary points for $\leq$ and $\geq$),}
\item{an interval $\MoIl x_1\MIntvlSep x_2\MoIr$ (including the boundary points for $\leq$ and $\geq$ if applicable),}
\item{a single point $x_1$,}
\item{the pointed set $\R\MSetminus\lbrace x_1\rbrace$,}
\item{the empty set $\lbrace\rbrace$.}
\end{itemize}
\end{MInfo}

Fill in the blanks in the following text describing the solution of a quadratic
inequality by investigating the behaviour of the parabola:

\begin{MExercise}
Find the solution set of the inequality $x^2+6x< -5$. 
Transformation results in the inequality \MLSimplifyQuestion{15}{x^2+6*x+5}{5}{x}{5}{1}{OBXP1}$<0$.
Using the $p q$ formula one obtains the set of roots
\MLParsedQuestion{9}{-1,-5}{3}{PXL}. The left-hand side
describes a parabola opening \MLQuestion{10}{upwards}{ObenX}.
It belongs to an inequality involving the comparing symbol $<$, hence 
the solution set is \MEquationItem{$\ML$}{\MLIntervalQuestion{15}{(-5,-1)}{5}{INVX}}.
\ \\ \ \\
\begin{MHint}{Solution}
Transformation results in $x^2+6x+5<0$. Using the $p q$ formula
one obtains the roots $x_{1,2}=-3\pm\sqrt{9-5}$, i.e.\ $x_1=-1$ and $x_2=-5$.
The left-hand side describes a parabola opening upwards. It satisfies the inequality 
involving $<$ only on the interval $\MoIl[\left] -5\MIntvlSep -1\MoIr[\right]$ excluding
the boundary points.
\end{MHint}
\end{MExercise}

\end{MXContent}

\begin{MXContent}{Further Types of Inequalities}{Further Types of Inequalities}{STD}
\MDeclareSiteUXID{VBKM03_WeitereUngleichungstypen}
Many other types of inequalities can be transformed into quadratic inequalities. Sometimes, 
case analyses have to be done or excluded values in the domain have to be observed:

\begin{MInfo}
An inequality containing \MEntry{fractions}{inequality (fractions)}, where the 
variable $x$ occurs in the denominator of composite terms, can be transformed into a 
form without fractions by multiplying the inequality by the least common denominator. 
However, in doing so, the roots of the denominators have to be excluded from the domain
of the new inequality. 

Additionally, if the inequality is multiplied by a term, different cases have to be distinguished 
depending on the sign of the term.
\end{MInfo}

\begin{MExample}
The inequality $2-\frac1x\leq x$ can be transformed by multiplying the inequality by $x$. Here, three 
cases have to be distinguished:
\begin{itemize}
\item{For $x>0$, the comparing symbol in the inequality is unchanged. The new inequality
reads $2x-1\leq x^2$ and is equivalent to $x^2-2x+1\geq 0$ or $(x-1)^2\geq 0$, respectively.
This inequality is always satisfied. Because of the case condition one obtains 
the solution set $\ML_1=\MoIl 0\MIntvlSep \infty\MoIr$.}
\item{For $x<0$, the comparing symbol in the inequality is inverted. The new inequality
reads $2x-1\geq x^2$ and is equivalent to $x^2-2x+1\leq 0$ or $(x-1)^2\leq 0$, respectively.
This inequality is only satisfied for $x=1$. But this value is excluded by the case condition, 
i.e.\ $\ML_2=\{\}$.}
\item{The single value $x=0$ is not in the domain of the initial inequality and hence it is
no solution.}
\end{itemize}

So, altogether one obtains the union set 
$\ML=\MoIl 0\MIntvlSep \infty\MoIr$ as solution set of the initial inequality.
\end{MExample}

Inequalities involving composite fraction and root terms often do not have solution
sets of the types described in info box~\MRef{M03_InfoFormen}:

\begin{MExample}
Find the solution set of the inequality $\sqrt{x}+\frac1{\sqrt{x}}>2$. 
The domain of the inequality is $\MoIl 0\MIntvlSep \infty\MoIr$.
Multiplying by $\sqrt{x}$ results in the inequality $x+1>2\sqrt x$. 
Here, no case analysis is required since $\sqrt{x}>0$ is in the domain.
Transformation results in $x-2\sqrt{x}+1>0$ or $(\sqrt{x}-1)^2>0$, respectively, 
which is satisfied for all $x\not=1$ in the domain.
Hence, the solution set of the initial inequality is
$\ML=\MoIl 0\MIntvlSep \infty\MoIr\MSetminus\lbrace 1\rbrace$:
\ \\ \ \\
\begin{center}
\MTikzAuto{%
\begin{tikzpicture}
% reelle Achse
\draw[->,color=black] (-1,0.0) -- (5,0.0);
\foreach \x in {-1, 0, 1, 2, 3, 4}
\draw[shift={(\x,0)},color=black] (0pt,2pt) -- (0pt,-2pt) node[below] {\footnotesize $\x$};
\draw (4.9,-0.3) node[] {$\mathbb{R}$};
% Enden:
\draw [line width=2.0pt,color=blue] (0,0.0)-- (1,0.0);
\draw [line width=2.0pt,color=blue] (1,0.0)-- (5,0.0);
\draw [fill = white] (0,0) circle (1.5pt);
\draw [fill = white] (1,0) circle (1.5pt);
\end{tikzpicture}
}
\end{center}

\end{MExample}

\end{MXContent}


\MSubsection{Final Test}
\MLabel{M03_Abschlusstest}

\begin{MTest}{Final Test Modul 3}
\MDeclareSiteUXID{VBKM03_Abschlusstest}

\begin{MExercise}
Find the value of the parameter $\alpha$ such that the inequality $2x^2\leq x-\alpha$ 
has exactly one solution:
\begin{MExerciseItems}
\item{The parameter value is \MEquationItem{$\alpha$}{\MLParsedQuestion{10}{1/8}{3}{PMA1}}.}
\item{In this case \MEquationItem{$x$}{\MLParsedQuestion{10}{1/4}{3}{PMA2}} is the only solution
of the inequality.}
\end{MExerciseItems}
\end{MExercise}


\begin{MExercise}
Find an absolute value function $g(x)$ describing the following graph as easy as possible.

%%\MUGraphics{abs3.png}{width=0.5\linewidth}{Funktionsgraph von $g(x)$.}{width:300px}
\begin{center}
\MTikzAuto{%
\begin{tikzpicture}[x=1.4cm, y=1.9cm] 
\draw[black] (-3,0) -- (3,0) (0,-3) -- (0,3);
\foreach \x in {-3, -2, -1, 1, 2, 3}
\draw[shift={(\x,0)},color=black] (0pt,0pt) -- (0pt,-3.0pt) node[below=1.0pt] {\normalsize $\x$};
\foreach \x in {-3.0, -2.8, ..., 3.0}
\draw[shift={(\x,0)},color=black] (0pt,0pt) -- (0pt,-1.5pt);
\foreach \y in {-3, -2, -1, 1, 2, 3}
\draw[shift={(0,\y)},color=black] (0pt,0pt) -- (-3.0pt,0pt) node[left=1.0pt] {\normalsize $\y$};
\foreach \y in {-3.0, -2.8, ..., 3.0}
\draw[shift={(0,\y)},color=black] (0pt,0pt) -- (-1.5pt,0pt);
%%\draw[black] (-0.0pt,-0.5pt) node[anchor=north east] {\small $0$};
\clip(-3.0,-3.0) rectangle (3.0,3.0);
\draw[black, line width=1.0pt,color=black] (-3,-3) -- (1,1) -- (2,4);
\end{tikzpicture}
}
\par
Graph of the function $g(x)$.
\end{center}
Try to find a representation of the form $g(x)=|x+a|+b x+c$. 
The kink in the graph indicates how the absolute value term looks like.

\begin{MExerciseItems}
\item{Find the solution set of the inequality $g(x)\leq x$ by means of the graph.\\
The solution set is \MEquationItem{$\ML$}{\MLIntervalQuestion{20}{(-infty;1]}{5}{AUX1}}.}
\item{\MEquationItem{$g(x)$}{\MLSimplifyQuestion{20}{abs(x-1)+2*x-1}{10}{x}{10}{0}{SIMPLE3}}. 
\\\MInputHint{Absolute values can be entered in the form \texttt{betrag(x-a)} or \texttt{abs(x-a)}.}}
\end{MExerciseItems}
\end{MExercise}

\begin{MExercise}
Which positive real numbers $x$ satisfy the following inequalities?
\begin{MExerciseItems}
\item{$|3x-6|\leq x+2$ has the solution set
\MEquationItem{$\ML$}{\MLIntervalQuestion{16}{[1,4]}{4}{COSH1}} (written as an interval).}
\item{$\frac{x+1}{x-1}\geq 2$ has the solution set 
\MEquationItem{$\ML$}{\MLIntervalQuestion{16}{(1,3]}{4}{COSH2}} (written as an interval).}
\end{MExerciseItems}
\MInputHint{Enter open intervals in the form $(3;5)$, closed intervals in the form 
$[3;5]$. Infinity can be entered a a word or shortly a \texttt{infty}. Do not use 
notation $]a;b[$ for open intervals. Sets can be entered by listing the elements
 $\lbrace 1;2;3\rbrace$. For the set brackets enter AltGr+7 or AltGr+0, respectively.}
\end{MExercise}

\end{MTest}

\newpage
\MPrintIndex

\end{document}

%% MINTMOD Version P0.1.0, needs to be consistent with preprocesser object in tex2x and MPragma-Version at the end of this file

% Parameter aus Konvertierungsprozess (PDF und HTML-Erzeugung wenn vom Konverter aus gestartet) werden hier eingefuegt, Preambleincludes werden am Schluss angehaengt

\newif\ifttm                % gesetzt falls Uebersetzung in HTML stattfindet, sonst uebersetzung in PDF

% Wahl der Notationsvariante ist im PDF immer std, in der HTML-Uebersetzung wird vom Konverter die Auswahl modifiziert
\newif\ifvariantstd
\newif\ifvariantunotation
\variantstdtrue % Diese Zeile wird vom Konverter erkannt und ggf. modifiziert, daher nicht veraendern!


\def\MOutputDVI{1}
\def\MOutputPDF{2}
\def\MOutputHTML{3}
\newcounter{MOutput}

\ifttm
\usepackage{german}
\usepackage{array}
\usepackage{amsmath}
\usepackage{amssymb}
\usepackage{amsthm}
\else
\documentclass[ngerman,oneside]{scrbook}
\usepackage{etex}
\usepackage[latin1]{inputenc}
\usepackage{textcomp}
\usepackage[ngerman]{babel}
\usepackage[pdftex]{color}
\usepackage{xcolor}
\usepackage{graphicx}
\usepackage[all]{xy}
\usepackage{fancyhdr}
\usepackage{verbatim}
\usepackage{array}
\usepackage{float}
\usepackage{makeidx}
\usepackage{amsmath}
\usepackage{amstext}
\usepackage{amssymb}
\usepackage{amsthm}
\usepackage[ngerman]{varioref}
\usepackage{framed}
\usepackage{supertabular}
\usepackage{longtable}
\usepackage{maxpage}
\usepackage{tikz}
\usepackage{tikzscale}
\usepackage{tikz-3dplot}
\usepackage{bibgerm}
\usepackage{chemarrow}
\usepackage{polynom}
%\usepackage{draftwatermark}
\usepackage{pdflscape}
\usetikzlibrary{calc}
\usetikzlibrary{through}
\usetikzlibrary{shapes.geometric}
\usetikzlibrary{arrows}
\usetikzlibrary{intersections}
\usetikzlibrary{decorations.pathmorphing}
\usetikzlibrary{external}
\usetikzlibrary{patterns}
\usetikzlibrary{fadings}
\usepackage[colorlinks=true,linkcolor=blue]{hyperref} 
\usepackage[all]{hypcap}
%\usepackage[colorlinks=true,linkcolor=blue,bookmarksopen=true]{hyperref} 
\usepackage{ifpdf}

\usepackage{movie15}

\setcounter{tocdepth}{2} % In Inhaltsverzeichnis bis subsection
\setcounter{secnumdepth}{3} % Nummeriert bis subsubsection

\setlength{\LTpost}{0pt} % Fuer longtable
\setlength{\parindent}{0pt}
\setlength{\parskip}{8pt}
%\setlength{\parskip}{9pt plus 2pt minus 1pt}
\setlength{\abovecaptionskip}{-0.25ex}
\setlength{\belowcaptionskip}{-0.25ex}
\fi

\ifttm
\newcommand{\MDebugMessage}[1]{\special{html:<!-- debugprint;;}#1\special{html:; //-->}}
\else
%\newcommand{\MDebugMessage}[1]{\immediate\write\mintlog{#1}}
\newcommand{\MDebugMessage}[1]{}
\fi

\def\MPageHeaderDef{%
\pagestyle{fancy}%
\fancyhead[r]{(C) VE\&MINT-Projekt}
\fancyfoot[c]{\thepage\\--- CCL BY-SA 3.0 ---}
}


\ifttm%
\def\MRelax{}%
\else%
\def\MRelax{\relax}%
\fi%

%--------------------------- Uebernahme von speziellen XML-Versionen einiger LaTeX-Kommandos aus xmlbefehle.tex vom alten Kasseler Konverter ---------------

\newcommand{\MSep}{\left\|{\phantom{\frac1g}}\right.}

\newcommand{\ML}{L}

\newcommand{\MGGT}{\mathrm{ggT}}


\ifttm
% Verhindert dass die subsection-nummer doppelt in der toccaption auftaucht (sollte ggf. in toccaption gefixt werden so dass diese Ueberschreibung nicht notwendig ist)
\renewcommand{\thesubsection}{}
% Kommandos die ttm nicht kennt
\newcommand{\binomial}[2]{{#1 \choose #2}} %  Binomialkoeffizienten
\newcommand{\eur}{\begin{html}&euro;\end{html}}
\newcommand{\square}{\begin{html}&square;\end{html}}
\newcommand{\glqq}{"'}  \newcommand{\grqq}{"'}
\newcommand{\nRightarrow}{\special{html: &nrArr; }}
\newcommand{\nmid}{\special{html: &nmid; }}
\newcommand{\nparallel}{\begin{html}&nparallel;\end{html}}
\newcommand{\mapstoo}{\begin{html}<mo>&map;</mo>\end{html}}

% Schnitt und Vereinigungssymbole von Mengen haben zu kleine Abstaende; korrigiert:
\newcommand{\ccup}{\,\!\cup\,\!}
\newcommand{\ccap}{\,\!\cap\,\!}


% Umsetzung von mathbb im HTML
\renewcommand{\mathbb}[1]{\begin{html}<mo>&#1opf;</mo>\end{html}}
\fi

%---------------------- Strukturierung ----------------------------------------------------------------------------------------------------------------------

%---------------------- Kapselung des sectioning findet auf drei Ebenen statt:
% 1. Die LateX-Befehl
% 2. Die D-Versionen der Befehle, die nur die Grade der Abschnitte umhaengen falls notwendig
% 3. Die M-Versionen der Befehle, die zusaetzliche Formatierungen vornehmen, Skripten starten und das HTML codieren
% Im Modultext duerfen nur die M-Befehle verwendet werden!

\ifttm

  \def\Dsubsubsubsection#1{\subsubsubsection{#1}}
  \def\Dsubsubsection#1{\subsubsection{#1}\addtocounter{subsubsection}{1}} % ttm-Fehler korrigieren
  \def\Dsubsection#1{\subsection{#1}}
  \def\Dsection#1{\section{#1}} % Im HTML wird nur der Sektionstitel gegeben
  \def\Dchapter#1{\chapter{#1}}
  \def\Dsubsubsubsectionx#1{\subsubsubsection*{#1}}
  \def\Dsubsubsectionx#1{\subsubsection*{#1}}
  \def\Dsubsectionx#1{\subsection*{#1}}
  \def\Dsectionx#1{\section*{#1}}
  \def\Dchapterx#1{\chapter*{#1}}

\else

  \def\Dsubsubsubsection#1{\subsubsection{#1}}
  \def\Dsubsubsection#1{\subsection{#1}}
  \def\Dsubsection#1{\section{#1}}
  \def\Dsection#1{\chapter{#1}}
  \def\Dchapter#1{\title{#1}}
  \def\Dsubsubsubsectionx#1{\subsubsection*{#1}}
  \def\Dsubsubsectionx#1{\subsection*{#1}}
  \def\Dsubsectionx#1{\section*{#1}}
  \def\Dsectionx#1{\chapter*{#1}}

\fi

\newcommand{\MStdPoints}{4}
\newcommand{\MSetPoints}[1]{\renewcommand{\MStdPoints}{#1}}

% Befehl zum Abbruch der Erstellung (nur PDF)
\newcommand{\MAbort}[1]{\err{#1}}

% Prefix vor Dateieinbindungen, wird in der Baumdatei mit \renewcommand modifiziert
% und auf das Verzeichnisprefix gesetzt, in dem das gerade bearbeitete tex-Dokument liegt.
% Im HTML wird es auf das Verzeichnis der HTML-Datei gesetzt.
% Das Prefix muss mit / enden !
\newcommand{\MDPrefix}{.}

% MRegisterFile notiert eine Datei zur Einbindung in den HTML-Baum. Grafiken mit MGraphics werden automatisch eingebunden.
% Mit MLastFile erhaelt man eine Markierung fuer die zuletzt registrierte Datei.
% Diese Markierung wird im postprocessing durch den physikalischen Dateinamen ersetzt, aber nur den Namen (d.h. \MMaterial gehoert noch davor, vgl Definition von MGraphics)
% Parameter: Pfad/Name der Datei bzw. des Ordners, relativ zur Position des Modul-Tex-Dokuments.
\ifttm
\newcommand{\MRegisterFile}[1]{\addtocounter{MFileNumber}{1}\special{html:<!-- registerfile;;}#1\special{html:;;}\MDPrefix\special{html:;;}\arabic{MFileNumber}\special{html:; //-->}}
\else
\newcommand{\MRegisterFile}[1]{\addtocounter{MFileNumber}{1}}
\fi

% Testen welcher Uebersetzer hier am Werk ist

\ifttm
\setcounter{MOutput}{3}
\else
\ifx\pdfoutput\undefined
  \pdffalse
  \setcounter{MOutput}{\MOutputDVI}
  \message{Verarbeitung mit latex, Ausgabe in dvi.}
\else
  \setcounter{MOutput}{\MOutputPDF}
  \message{Verarbeitung mit pdflatex, Ausgabe in pdf.}
  \ifnum \pdfoutput=0
    \pdffalse
  \setcounter{MOutput}{\MOutputDVI}
  \message{Verarbeitung mit pdflatex, Ausgabe in dvi.}
  \else
    \ifnum\pdfoutput=1
    \pdftrue
  \setcounter{MOutput}{\MOutputPDF}
  \message{Verarbeitung mit pdflatex, Ausgabe in pdf.}
    \fi
  \fi
\fi
\fi

\ifnum\value{MOutput}=\MOutputPDF
\DeclareGraphicsExtensions{.pdf,.png,.jpg}
\fi

\ifnum\value{MOutput}=\MOutputDVI
\DeclareGraphicsExtensions{.eps,.png,.jpg}
\fi

\ifnum\value{MOutput}=\MOutputHTML
% Wird vom Konverter leider nicht erkannt und daher in split.pm hardcodiert!
\DeclareGraphicsExtensions{.png,.jpg,.gif}
\fi

% Umdefinition der hyperref-Nummerierung im PDF-Modus
\ifttm
\else
\renewcommand{\theHfigure}{\arabic{chapter}.\arabic{section}.\arabic{figure}}
\fi

% Makro, um in der HTML-Ausgabe die zuerst zu oeffnende Datei zu kennzeichnen
\ifttm
\newcommand{\MGlobalStart}{\special{html:<!-- mglobalstarttag -->}}
\else
\newcommand{\MGlobalStart}{}
\fi

% Makro, um bei scormlogin ein pullen des Benutzers bei Aufruf der Seite zu erzwingen (typischerweise auf der Einstiegsseite)
\ifttm
\newcommand{\MPullSite}{\special{html:<!-- pullsite //-->}}
\else
\newcommand{\MPullSite}{}
\fi

% Makro, um in der HTML-Ausgabe die Kapiteluebersicht zu kennzeichnen
\ifttm
\newcommand{\MGlobalChapterTag}{\special{html:<!-- mglobalchaptertag -->}}
\else
\newcommand{\MGlobalChapterTag}{}
\fi

% Makro, um in der HTML-Ausgabe die Konfiguration zu kennzeichnen
\ifttm
\newcommand{\MGlobalConfTag}{\special{html:<!-- mglobalconfigtag -->}}
\else
\newcommand{\MGlobalConfTag}{}
\fi

% Makro, um in der HTML-Ausgabe die Standortbeschreibung zu kennzeichnen
\ifttm
\newcommand{\MGlobalLocationTag}{\special{html:<!-- mgloballocationtag -->}}
\else
\newcommand{\MGlobalLocationTag}{}
\fi

% Makro, um in der HTML-Ausgabe die persoenlichen Daten zu kennzeichnen
\ifttm
\newcommand{\MGlobalDataTag}{\special{html:<!-- mglobaldatatag -->}}
\else
\newcommand{\MGlobalDataTag}{}
\fi

% Makro, um in der HTML-Ausgabe die Suchseite zu kennzeichnen
\ifttm
\newcommand{\MGlobalSearchTag}{\special{html:<!-- mglobalsearchtag -->}}
\else
\newcommand{\MGlobalSearchTag}{}
\fi

% Makro, um in der HTML-Ausgabe die Favoritenseite zu kennzeichnen
\ifttm
\newcommand{\MGlobalFavoTag}{\special{html:<!-- mglobalfavoritestag -->}}
\else
\newcommand{\MGlobalFavoTag}{}
\fi

% Makro, um in der HTML-Ausgabe die Eingangstestseite zu kennzeichnen
\ifttm
\newcommand{\MGlobalSTestTag}{\special{html:<!-- mglobalstesttag -->}}
\else
\newcommand{\MGlobalSTestTag}{}
\fi

% Makro, um in der PDF-Ausgabe ein Wasserzeichen zu definieren
\ifttm
\newcommand{\MWatermarkSettings}{\relax}
\else
\newcommand{\MWatermarkSettings}{%
% \SetWatermarkText{(c) MINT-Kolleg Baden-W�rttemberg 2014}
% \SetWatermarkLightness{0.85}
% \SetWatermarkScale{1.5}
}
\fi

\ifttm
\newcommand{\MBinom}[2]{\left({\begin{array}{c} #1 \\ #2 \end{array}}\right)}
\else
\newcommand{\MBinom}[2]{\binom{#1}{#2}}
\fi

\ifttm
\newcommand{\DeclareMathOperator}[2]{\def#1{\mathrm{#2}}}
\newcommand{\operatorname}[1]{\mathrm{#1}}
\fi

%----------------- Makros fuer die gemischte HTML/PDF-Konvertierung ------------------------------

\newcommand{\MTestName}{\relax} % wird durch Test-Umgebung gesetzt

% Fuer experimentelle Kursinhalte, die im Release-Umsetzungsvorgang eine Fehlermeldung
% produzieren sollen aber sonst normal umgesetzt werden
\newenvironment{MExperimental}{%
}{%
}

% Wird von ttm nicht richtig umgesetzt!!
\newenvironment{MExerciseItems}{%
\renewcommand\theenumi{\alph{enumi}}%
\begin{enumerate}%
}{%
\end{enumerate}%
}


\definecolor{infoshadecolor}{rgb}{0.75,0.75,0.75}
\definecolor{exmpshadecolor}{rgb}{0.875,0.875,0.875}
\definecolor{expeshadecolor}{rgb}{0.95,0.95,0.95}
\definecolor{framecolor}{rgb}{0.2,0.2,0.2}

% Bei PDF-Uebersetzung wird hinter den Start jeder Satz/Info-aehnlichen Umgebung eine leere mbox gesetzt, damit
% fuehrende Listen oder enums nicht den Zeilenumbruch kaputtmachen
%\ifttm
\def\MTB{}
%\else
%\def\MTB{\mbox{}}
%\fi


\ifttm
\newcommand{\MRelates}{\special{html:<mi>&wedgeq;</mi>}}
\else
\def\MRelates{\stackrel{\scriptscriptstyle\wedge}{=}}
\fi

\def\MInch{\text{''}}
\def\Mdd{\textit{''}}

\ifttm
\def\MNL{ \newline }
\newenvironment{MArray}[1]{\begin{array}{#1}}{\end{array}}
\else
\def\MNL{ \\ }
\newenvironment{MArray}[1]{\begin{array}{#1}}{\end{array}}
\fi

\newcommand{\MBox}[1]{$\mathrm{#1}$}
\newcommand{\MMBox}[1]{\mathrm{#1}}


\ifttm%
\newcommand{\Mtfrac}[2]{{\textstyle \frac{#1}{#2}}}
\newcommand{\Mdfrac}[2]{{\displaystyle \frac{#1}{#2}}}
\newcommand{\Mmeasuredangle}{\special{html:<mi>&angmsd;</mi>}}
\else%
\newcommand{\Mtfrac}[2]{\tfrac{#1}{#2}}
\newcommand{\Mdfrac}[2]{\dfrac{#1}{#2}}
\newcommand{\Mmeasuredangle}{\measuredangle}
\relax
\fi

% Matrizen und Vektoren

% Inhalt wird in der Form a & b \\ c & d erwartet
% Vorsicht: MVector = Komponentenspalte, MVec = Variablensymbol
\ifttm%
\newcommand{\MVector}[1]{\left({\begin{array}{c}#1\end{array}}\right)}
\else%
\newcommand{\MVector}[1]{\begin{pmatrix}#1\end{pmatrix}}
\fi



\newcommand{\MVec}[1]{\vec{#1}}
\newcommand{\MDVec}[1]{\overrightarrow{#1}}

%----------------- Umgebungen fuer Definitionen und Saetze ----------------------------------------

% Fuegt einen Tabellen-Zeilenumbruch ein im PDF, aber nicht im HTML
\newcommand{\TSkip}{\ifttm \else&\ \\\fi}

\newenvironment{infoshaded}{%
\def\FrameCommand{\fboxsep=\FrameSep \fcolorbox{framecolor}{infoshadecolor}}%
\MakeFramed {\advance\hsize-\width \FrameRestore}}%
{\endMakeFramed}

\newenvironment{expeshaded}{%
\def\FrameCommand{\fboxsep=\FrameSep \fcolorbox{framecolor}{expeshadecolor}}%
\MakeFramed {\advance\hsize-\width \FrameRestore}}%
{\endMakeFramed}

\newenvironment{exmpshaded}{%
\def\FrameCommand{\fboxsep=\FrameSep \fcolorbox{framecolor}{exmpshadecolor}}%
\MakeFramed {\advance\hsize-\width \FrameRestore}}%
{\endMakeFramed}

\def\STDCOLOR{black}

\ifttm%
\else%
\newtheoremstyle{MSatzStyle}
  {1cm}                   %Space above
  {1cm}                   %Space below
  {\normalfont\itshape}   %Body font
  {}                      %Indent amount (empty = no indent,
                          %\parindent = para indent)
  {\normalfont\bfseries}  %Thm head font
  {}                      %Punctuation after thm head
  {\newline}              %Space after thm head: " " = normal interword
                          %space; \newline = linebreak
  {\thmname{#1}\thmnumber{ #2}\thmnote{ (#3)}}
                          %Thm head spec (can be left empty, meaning
                          %`normal')
                          %
\newtheoremstyle{MDefStyle}
  {1cm}                   %Space above
  {1cm}                   %Space below
  {\normalfont}           %Body font
  {}                      %Indent amount (empty = no indent,
                          %\parindent = para indent)
  {\normalfont\bfseries}  %Thm head font
  {}                      %Punctuation after thm head
  {\newline}              %Space after thm head: " " = normal interword
                          %space; \newline = linebreak
  {\thmname{#1}\thmnumber{ #2}\thmnote{ (#3)}}
                          %Thm head spec (can be left empty, meaning
                          %`normal')
\fi%

\newcommand{\MInfoText}{Info}

\newcounter{MHintCounter}
\newcounter{MCodeEditCounter}

\newcounter{MLastIndex}  % Enthaelt die dritte Stelle (Indexnummer) des letzten angelegten Objekts
\newcounter{MLastType}   % Enthaelt den Typ des letzten angelegten Objekts (mithilfe der unten definierten Konstanten). Die Entscheidung, wie der Typ dargstellt wird, wird in split.pm beim Postprocessing getroffen.
\newcounter{MLastTypeEq} % =1 falls das Label in einer Matheumgebung (equation, eqnarray usw.) steht, =2 falls das Label in einer table-Umgebung steht

% Da ttm keine Zahlmakros verarbeiten kann, werden diese Nummern in den Zuweisungen hardcodiert!
\def\MTypeSection{1}          %# Zaehler ist section
\def\MTypeSubsection{2}       %# Zaehler ist subsection
\def\MTypeSubsubsection{3}    %# Zaehler ist subsubsection
\def\MTypeInfo{4}             %# Eine Infobox, Separatzaehler fuer die Chemie (auch wenn es dort nicht nummeriert wird) ist MInfoCounter
\def\MTypeExercise{5}         %# Eine Aufgabe, Separatzaehler fuer die Chemie ist MExerciseCounter
\def\MTypeExample{6}          %# Eine Beispielbox, Separatzaehler fuer die Chemie ist MExampleCounter
\def\MTypeExperiment{7}       %# Eine Versuchsbox, Separatzaehler fuer die Chemie ist MExperimentCounter
\def\MTypeGraphics{8}         %# Eine Graphik, Separatzaehler fuer alle FB ist MGraphicsCounter
\def\MTypeTable{9}            %# Eine Tabellennummer, hat keinen Zaehler da durch table gezaehlt wird
\def\MTypeEquation{10}        %# Eine Gleichungsnummer, hat keinen Zaehler da durch equation/eqnarray gezaehlt wird
\def\MTypeTheorem{11}         % Ein theorem oder xtheorem, Separatzaehler fuer die Chemie ist MTheoremCounter
\def\MTypeVideo{12}           %# Ein Video,Separatzaehler fuer alle FB ist MVideoCounter
\def\MTypeEntry{13}           %# Ein Eintrag fuer die Stichwortliste, wird nicht gezaehlt sondern erhaelt im preparsing ein unique-label 

% Zaehler fuer das Labelsystem sind prefixcounter, jeder Zaehler wird VOR dem gezaehlten Objekt inkrementiert und zaehlt daher das aktuelle Objekt
\newcounter{MInfoCounter}
\newcounter{MExerciseCounter}
\newcounter{MExampleCounter}
\newcounter{MExperimentCounter}
\newcounter{MGraphicsCounter}
\newcounter{MTableCounter}
\newcounter{MEquationCounter}  % Nur im HTML, sonst durch "equation"-counter von latex realisiert
\newcounter{MTheoremCounter}
\newcounter{MObjectCounter}   % Gemeinsamer Zaehler fuer Objekte (ausser Grafiken/Tabellen) in Mathe/Info/Physik
\newcounter{MVideoCounter}
\newcounter{MEntryCounter}

\newcounter{MTestSite} % 1 = Subsubsection ist eine Pruefungsseite, 0 = ist eine normale Seite (inkl. Hilfeseite)

\def\MCell{$\phantom{a}$}

\newenvironment{MExportExercise}{\begin{MExercise}}{\end{MExercise}} % wird von mconvert abgefangen

\def\MGenerateExNumber{%
\ifnum\value{MSepNumbers}=0%
\arabic{section}.\arabic{subsection}.\arabic{MObjectCounter}\setcounter{MLastIndex}{\value{MObjectCounter}}%
\else%
\arabic{section}.\arabic{subsection}.\arabic{MExerciseCounter}\setcounter{MLastIndex}{\value{MExerciseCounter}}%
\fi%
}%

\def\MGenerateExmpNumber{%
\ifnum\value{MSepNumbers}=0%
\arabic{section}.\arabic{subsection}.\arabic{MObjectCounter}\setcounter{MLastIndex}{\value{MObjectCounter}}%
\else%
\arabic{section}.\arabic{subsection}.\arabic{MExerciseCounter}\setcounter{MLastIndex}{\value{MExampleCounter}}%
\fi%
}%

\def\MGenerateInfoNumber{%
\ifnum\value{MSepNumbers}=0%
\arabic{section}.\arabic{subsection}.\arabic{MObjectCounter}\setcounter{MLastIndex}{\value{MObjectCounter}}%
\else%
\arabic{section}.\arabic{subsection}.\arabic{MExerciseCounter}\setcounter{MLastIndex}{\value{MInfoCounter}}%
\fi%
}%

\def\MGenerateSiteNumber{%
\arabic{section}.\arabic{subsection}.\arabic{subsubsection}%
}%

% Funktionalitaet fuer Auswahlaufgaben

\newcounter{MExerciseCollectionCounter} % = 0 falls nicht in collection-Umgebung, ansonsten Schachtelungstiefe
\newcounter{MExerciseCollectionTextCounter} % wird von MExercise-Umgebung inkrementiert und von MExerciseCollection-Umgebung auf Null gesetzt

\ifttm
% MExerciseCollection gruppiert Aufgaben, die dynamisch aus der Datenbank gezogen werden und nicht direkt in der HTML-Seite stehen
% Parameter: #1 = ID der Collection, muss eindeutig fuer alle IN DER DB VORHANDENEN collections sein unabhaengig vom Kurs
%            #2 = Optionsargument (im Moment: 1 = Iterative Auswahl, 2 = Zufallsbasierte Auswahl)
\newenvironment{MExerciseCollection}[2]{%
\addtocounter{MExerciseCollectionCounter}{1}
\setcounter{MExerciseCollectionTextCounter}{0}
\special{html:<!-- mexercisecollectionstart;;}#1\special{html:;;}#2\special{html:;; //-->}%
}{%
\special{html:<!-- mexercisecollectionstop //-->}%
\addtocounter{MExerciseCollectionCounter}{-1}
}
\else
\newenvironment{MExerciseCollection}[2]{%
\addtocounter{MExerciseCollectionCounter}{1}
\setcounter{MExerciseCollectionTextCounter}{0}
}{%
\addtocounter{MExerciseCollectionCounter}{-1}
}
\fi

% Bei Uebersetzung nach PDF werden die theorem-Umgebungen verwendet, bei Uebersetzung in HTML ein manuelles Makro
\ifttm%

  \newenvironment{MHint}[1]{  \special{html:<button name="Name_MHint}\arabic{MHintCounter}\special{html:" class="hintbutton_closed" id="MHint}\arabic{MHintCounter}\special{html:_button" %
  type="button" onclick="toggle_hint('MHint}\arabic{MHintCounter}\special{html:');">}#1\special{html:</button>}
  \special{html:<div class="hint" style="display:none" id="MHint}\arabic{MHintCounter}\special{html:"> }}{\begin{html}</div>\end{html}\addtocounter{MHintCounter}{1}}

  \newenvironment{MCOSHZusatz}{  \special{html:<button name="Name_MHint}\arabic{MHintCounter}\special{html:" class="chintbutton_closed" id="MHint}\arabic{MHintCounter}\special{html:_button" %
  type="button" onclick="toggle_hint('MHint}\arabic{MHintCounter}\special{html:');">}Weiterf�hrende Inhalte\special{html:</button>}
  \special{html:<div class="hintc" style="display:none" id="MHint}\arabic{MHintCounter}\special{html:">
  <div class="coshwarn">Diese Inhalte gehen �ber das Kursniveau hinaus und werden in den Aufgaben und Tests nicht abgefragt.</div><br />}
  \addtocounter{MHintCounter}{1}}{\begin{html}</div>\end{html}}

  
  \newenvironment{MDefinition}{\begin{definition}\setcounter{MLastIndex}{\value{definition}}\ \\}{\end{definition}}

  
  \newenvironment{MExercise}{
  \renewcommand{\MStdPoints}{4}
  \addtocounter{MExerciseCounter}{1}
  \addtocounter{MObjectCounter}{1}
  \setcounter{MLastType}{5}

  \ifnum\value{MExerciseCollectionCounter}=0\else\addtocounter{MExerciseCollectionTextCounter}{1}\special{html:<!-- mexercisetextstart;;}\arabic{MExerciseCollectionTextCounter}\special{html:;; //-->}\fi
  \special{html:<div class="aufgabe" id="ADIV_}\MGenerateExNumber\special{html:">}%
  \textbf{Aufgabe \MGenerateExNumber
  } \ \\}{
  \special{html:</div><!-- mfeedbackbutton;Aufgabe;}\arabic{MTestSite}\special{html:;}\MGenerateExNumber\special{html:; //-->}
  \ifnum\value{MExerciseCollectionCounter}=0\else\special{html:<!-- mexercisetextstop //-->}\fi
  }

  % Stellt eine Kombination aus Aufgabe, Loesungstext und Eingabefeld bereit,
  % bei der Aufgabentext und Musterloesung sowie die zugehoerigen Feldelemente
  % extern bezogen und div-aktualisiert werden, das Eingabefeld aber immer das gleiche ist.
  \newenvironment{MFetchExercise}{
  \addtocounter{MExerciseCounter}{1}
  \addtocounter{MObjectCounter}{1}
  \setcounter{MLastType}{5}

  \special{html:<div class="aufgabe" id="ADIV_}\MGenerateExNumber\special{html:">}%
  \textbf{Aufgabe \MGenerateExNumber
  } \ \\%
  \special{html:</div><div class="exfetch_text" id="ADIVTEXT_}\MGenerateExNumber\special{html:">}%
  \special{html:</div><div class="exfetch_sol" id="ADIVSOL_}\MGenerateExNumber\special{html:">}%
  \special{html:</div><div class="exfetch_input" id="ADIVINPUT_}\MGenerateExNumber\special{html:">}%
  }{
  \special{html:</div>}
  }

  \newenvironment{MExample}{
  \addtocounter{MExampleCounter}{1}
  \addtocounter{MObjectCounter}{1}
  \setcounter{MLastType}{6}
  \begin{html}
  <div class="exmp">
  <div class="exmprahmen">
  \end{html}\textbf{Beispiel
  \ifnum\value{MSepNumbers}=0
  \arabic{section}.\arabic{subsection}.\arabic{MObjectCounter}\setcounter{MLastIndex}{\value{MObjectCounter}}
  \else
  \arabic{section}.\arabic{subsection}.\arabic{MExampleCounter}\setcounter{MLastIndex}{\value{MExampleCounter}}
  \fi
  } \ \\}{\begin{html}</div>
  </div>
  \end{html}
  \special{html:<!-- mfeedbackbutton;Beispiel;}\arabic{MTestSite}\special{html:;}\MGenerateExmpNumber\special{html:; //-->}
  }

  \newenvironment{MExperiment}{
  \addtocounter{MExperimentCounter}{1}
  \addtocounter{MObjectCounter}{1}
  \setcounter{MLastType}{7}
  \begin{html}
  <div class="expe">
  <div class="experahmen">
  \end{html}\textbf{Versuch
  \ifnum\value{MSepNumbers}=0
  \arabic{section}.\arabic{subsection}.\arabic{MObjectCounter}\setcounter{MLastIndex}{\value{MObjectCounter}}
  \else
%  \arabic{MExperimentCounter}\setcounter{MLastIndex}{\value{MExperimentCounter}}
  \arabic{section}.\arabic{subsection}.\arabic{MExperimentCounter}\setcounter{MLastIndex}{\value{MExperimentCounter}}
  \fi
  } \ \\}{\begin{html}</div>
  </div>
  \end{html}}

  \newenvironment{MChemInfo}{
  \setcounter{MLastType}{4}
  \begin{html}
  <div class="info">
  <div class="inforahmen">
  \end{html}}{\begin{html}</div>
  </div>
  \end{html}}

  \newenvironment{MXInfo}[1]{
  \addtocounter{MInfoCounter}{1}
  \addtocounter{MObjectCounter}{1}
  \setcounter{MLastType}{4}
  \begin{html}
  <div class="info">
  <div class="inforahmen">
  \end{html}\textbf{#1
  \ifnum\value{MInfoNumbers}=0
  \else
    \ifnum\value{MSepNumbers}=0
    \arabic{section}.\arabic{subsection}.\arabic{MObjectCounter}\setcounter{MLastIndex}{\value{MObjectCounter}}
    \else
    \arabic{MInfoCounter}\setcounter{MLastIndex}{\value{MInfoCounter}}
    \fi
  \fi
  } \ \\}{\begin{html}</div>
  </div>
  \end{html}
  \special{html:<!-- mfeedbackbutton;Info;}\arabic{MTestSite}\special{html:;}\MGenerateInfoNumber\special{html:; //-->}
  }

  \newenvironment{MInfo}{\ifnum\value{MInfoNumbers}=0\begin{MChemInfo}\else\begin{MXInfo}{Info}\ \\ \fi}{\ifnum\value{MInfoNumbers}=0\end{MChemInfo}\else\end{MXInfo}\fi}

\else%

  \theoremstyle{MSatzStyle}
  \newtheorem{thm}{Satz}[section]
  \newtheorem{thmc}{Satz}
  \theoremstyle{MDefStyle}
  \newtheorem{defn}[thm]{Definition}
  \newtheorem{exmp}[thm]{Beispiel}
  \newtheorem{info}[thm]{\MInfoText}
  \theoremstyle{MDefStyle}
  \newtheorem{defnc}{Definition}
  \theoremstyle{MDefStyle}
  \newtheorem{exmpc}{Beispiel}[section]
  \theoremstyle{MDefStyle}
  \newtheorem{infoc}{\MInfoText}
  \theoremstyle{MDefStyle}
  \newtheorem{exrc}{Aufgabe}[section]
  \theoremstyle{MDefStyle}
  \newtheorem{verc}{Versuch}[section]
  
  \newenvironment{MFetchExercise}{}{} % kann im PDF nicht dargestellt werden
  
  \newenvironment{MExercise}{\begin{exrc}\renewcommand{\MStdPoints}{1}\MTB}{\end{exrc}}
  \newenvironment{MHint}[1]{\ \\ \underline{#1:}\\}{}
  \newenvironment{MCOSHZusatz}{\ \\ \underline{Weiterf�hrende Inhalte:}\\}{}
  \newenvironment{MDefinition}{\ifnum\value{MInfoNumbers}=0\begin{defnc}\else\begin{defn}\fi\MTB}{\ifnum\value{MInfoNumbers}=0\end{defnc}\else\end{defn}\fi}
%  \newenvironment{MExample}{\begin{exmp}}{\ \linebreak[1] \ \ \ \ $\phantom{a}$ \ \hfill $\blacklozenge$\end{exmp}}
  \newenvironment{MExample}{
    \ifnum\value{MInfoNumbers}=0\begin{exmpc}\else\begin{exmp}\fi
    \MTB
    \begin{exmpshaded}
    \ \newline
}{
    \end{exmpshaded}
    \ifnum\value{MInfoNumbers}=0\end{exmpc}\else\end{exmp}\fi
}
  \newenvironment{MChemInfo}{\begin{infoshaded}}{\end{infoshaded}}

  \newenvironment{MInfo}{\ifnum\value{MInfoNumbers}=0\begin{MChemInfo}\else\renewcommand{\MInfoText}{Info}\begin{info}\begin{infoshaded}
  \MTB
   \ \newline
    \fi
  }{\ifnum\value{MInfoNumbers}=0\end{MChemInfo}\else\end{infoshaded}\end{info}\fi}

  \newenvironment{MXInfo}[1]{
    \renewcommand{\MInfoText}{#1}
    \ifnum\value{MInfoNumbers}=0\begin{infoc}\else\begin{info}\fi%
    \MTB
    \begin{infoshaded}
    \ \newline
  }{\end{infoshaded}\ifnum\value{MInfoNumbers}=0\end{infoc}\else\end{info}\fi}

  \newenvironment{MExperiment}{
    \renewcommand{\MInfoText}{Versuch}
    \ifnum\value{MInfoNumbers}=0\begin{verc}\else\begin{info}\fi
    \MTB
    \begin{expeshaded}
    \ \newline
  }{
    \end{expeshaded}
    \ifnum\value{MInfoNumbers}=0\end{verc}\else\end{info}\fi
  }
\fi%

% MHint sollte nicht direkt fuer Loesungen benutzt werden wegen solutionselect
\newenvironment{MSolution}{\begin{MHint}{L"osung}}{\end{MHint}}

\newcounter{MCodeCounter}

\ifttm
\newenvironment{MCode}{\special{html:<!-- mcodestart -->}\ttfamily\color{blue}}{\special{html:<!-- mcodestop -->}}
\else
\newenvironment{MCode}{\begin{flushleft}\ttfamily\addtocounter{MCodeCounter}{1}}{\addtocounter{MCodeCounter}{-1}\end{flushleft}}
% Ohne color-Statement da inkompatible mit framed/shaded-Boxen aus dem framed-package
\fi

%----------------- Sonderdefinitionen fuer Symbole, die der Konverter nicht kann ----------------------------------------------

\ifttm%
\newcommand{\MUnderset}[2]{\underbrace{#2}_{#1}}%
\else%
\newcommand{\MUnderset}[2]{\underset{#1}{#2}}%
\fi%

\ifttm
\newcommand{\MThinspace}{\special{html:<mi>&#x2009;</mi>}}
\else
\newcommand{\MThinspace}{\,}
\fi

\ifttm
\newcommand{\glq}{\begin{html}&sbquo;\end{html}}
\newcommand{\grq}{\begin{html}&lsquo;\end{html}}
\newcommand{\glqq}{\begin{html}&bdquo;\end{html}}
\newcommand{\grqq}{\begin{html}&ldquo;\end{html}}
\fi

\ifttm
\newcommand{\MNdash}{\begin{html}&ndash;\end{html}}
\else
\newcommand{\MNdash}{--}
\fi

%\ifttm\def\MIU{\special{html:<mi>&#8520;</mi>}}\else\def\MIU{\mathrm{i}}\fi
\def\MIU{\mathrm{i}}
\def\MEU{e} % TU9-Onlinekurs: italic-e
%\def\MEU{\mathrm{e}} % Alte Onlinemodule: roman-e
\def\MD{d} % Kursives d in Integralen im TU9-Onlinekurs
%\def\MD{\mathrm{d}} % roman-d in den alten Onlinemodulen
\def\MDB{\|}

%zusaetzlicher Leerraum vor "\MD"
\ifttm%
\def\MDSpace{\special{html:<mi>&#x2009;</mi>}}
\else%
\def\MDSpace{\,}
\fi%
\newcommand{\MDwSp}{\MDSpace\MD}%

\ifttm
\def\Mdq{\dq}
\else
\def\Mdq{\dq}
\fi

\def\MSpan#1{\left<{#1}\right>}
\def\MSetminus{\setminus}
\def\MIM{I}

\ifttm
\newcommand{\ld}{\text{ld}}
\newcommand{\lg}{\text{lg}}
\else
\DeclareMathOperator{\ld}{ld}
%\newcommand{\lg}{\text{lg}} % in latex schon definiert
\fi


\def\Mmapsto{\ifttm\special{html:<mi>&mapsto;</mi>}\else\mapsto\fi} 
\def\Mvarphi{\ifttm\phi\else\varphi\fi}
\def\Mphi{\ifttm\varphi\else\phi\fi}
\ifttm%
\newcommand{\MEumu}{\special{html:<mi>&#x3BC;</mi>}}%
\else%
\newcommand{\MEumu}{\textrm{\textmu}}%
\fi
\def\Mvarepsilon{\ifttm\epsilon\else\varepsilon\fi}
\def\Mepsilon{\ifttm\varepsilon\else\epsilon\fi}
\def\Mvarkappa{\ifttm\kappa\else\varkappa\fi}
\def\Mkappa{\ifttm\varkappa\else\kappa\fi}
\def\Mcomplement{\ifttm\special{html:<mi>&comp;</mi>}\else\complement\fi} 
\def\MWW{\mathrm{WW}}
\def\Mmod{\ifttm\special{html:<mi>&nbsp;mod&nbsp;</mi>}\else\mod\fi} 

\ifttm%
\def\mod{\text{\;mod\;}}%
\def\MNEquiv{\special{html:<mi>&NotCongruent;</mi>}}% 
\def\MNSubseteq{\special{html:<mi>&NotSubsetEqual;</mi>}}%
\def\MEmptyset{\special{html:<mi>&empty;</mi>}}%
\def\MVDots{\special{html:<mi>&#x22EE;</mi>}}%
\def\MHDots{\special{html:<mi>&#x2026;</mi>}}%
\def\Mddag{\special{html:<mi>&#x1202;</mi>}}%
\def\sphericalangle{\special{html:<mi>&measuredangle;</mi>}}%
\def\nparallel{\special{html:<mi>&nparallel;</mi>}}%
\def\MProofEnd{\special{html:<mi>&#x25FB;</mi>}}%
\newenvironment{MProof}[1]{\underline{#1}:\MCR\MCR}{\hfill $\MProofEnd$}%
\else%
\def\MNEquiv{\not\equiv}%
\def\MNSubseteq{\not\subseteq}%
\def\MEmptyset{\emptyset}%
\def\MVDots{\vdots}%
\def\MHDots{\hdots}%
\def\Mddag{\ddag}%
\newenvironment{MProof}[1]{\begin{proof}[#1]}{\end{proof}}%
\fi%



% Spaces zum Auffuellen von Tabellenbreiten, die nur im HTML wirken
\ifttm%
\def\MTSP{\:}%
\else%
\def\MTSP{}%
\fi%

\DeclareMathOperator{\arsinh}{arsinh}
\DeclareMathOperator{\arcosh}{arcosh}
\DeclareMathOperator{\artanh}{artanh}
\DeclareMathOperator{\arcoth}{arcoth}


\newcommand{\MMathSet}[1]{\mathbb{#1}}
\def\N{\MMathSet{N}}
\def\Z{\MMathSet{Z}}
\def\Q{\MMathSet{Q}}
\def\R{\MMathSet{R}}
\def\C{\MMathSet{C}}

\newcounter{MForLoopCounter}
\newcommand{\MForLoop}[2]{\setcounter{MForLoopCounter}{#1}\ifnum\value{MForLoopCounter}=0{}\else{{#2}\addtocounter{MForLoopCounter}{-1}\MForLoop{\value{MForLoopCounter}}{#2}}\fi}

\newcounter{MSiteCounter}
\newcounter{MFieldCounter} % Kombination section.subsection.site.field ist eindeutig in allen Modulen, field alleine nicht

\newcounter{MiniMarkerCounter}

\ifttm
\newenvironment{MMiniPageP}[1]{\begin{minipage}{#1\linewidth}\special{html:<!-- minimarker;;}\arabic{MiniMarkerCounter}\special{html:;;#1; //-->}}{\end{minipage}\addtocounter{MiniMarkerCounter}{1}}
\else
\newenvironment{MMiniPageP}[1]{\begin{minipage}{#1\linewidth}}{\end{minipage}\addtocounter{MiniMarkerCounter}{1}}
\fi

\newcounter{AlignCounter}

\newcommand{\MStartJustify}{\ifttm\special{html:<!-- startalign;;}\arabic{AlignCounter}\special{html:;;justify; //-->}\fi}
\newcommand{\MStopJustify}{\ifttm\special{html:<!-- stopalign;;}\arabic{AlignCounter}\special{html:; //-->}\fi\addtocounter{AlignCounter}{1}}

\newenvironment{MJTabular}[1]{
\MStartJustify
\begin{tabular}{#1}
}{
\end{tabular}
\MStopJustify
}

\newcommand{\MImageLeft}[2]{
\begin{center}
\begin{tabular}{lc}
\MStartJustify
\begin{MMiniPageP}{0.65}
#1
\end{MMiniPageP}
\MStopJustify
&
\begin{MMiniPageP}{0.3}
#2  
\end{MMiniPageP}
\end{tabular}
\end{center}
}

\newcommand{\MImageHalf}[2]{
\begin{center}
\begin{tabular}{lc}
\MStartJustify
\begin{MMiniPageP}{0.45}
#1
\end{MMiniPageP}
\MStopJustify
&
\begin{MMiniPageP}{0.45}
#2  
\end{MMiniPageP}
\end{tabular}
\end{center}
}

\newcommand{\MBigImageLeft}[2]{
\begin{center}
\begin{tabular}{lc}
\MStartJustify
\begin{MMiniPageP}{0.25}
#1
\end{MMiniPageP}
\MStopJustify
&
\begin{MMiniPageP}{0.7}
#2  
\end{MMiniPageP}
\end{tabular}
\end{center}
}

\ifttm
\def\No{\mathbb{N}_0}
\else
\def\No{\ensuremath{\N_0}}
\fi
\def\MT{\textrm{\tiny T}}
\newcommand{\MTranspose}[1]{{#1}^{\MT}}
\ifttm
\newcommand{\MRe}{\mathsf{Re}}
\newcommand{\MIm}{\mathsf{Im}}
\else
\DeclareMathOperator{\MRe}{Re}
\DeclareMathOperator{\MIm}{Im}
\fi

\newcommand{\Mid}{\mathrm{id}}
\newcommand{\MFeinheit}{\mathrm{feinh}}

\ifttm
\newcommand{\Msubstack}[1]{\begin{array}{c}{#1}\end{array}}
\else
\newcommand{\Msubstack}[1]{\substack{#1}}
\fi

% Typen von Fragefeldern:
% 1 = Alphanumerisch, case-sensitive-Vergleich
% 2 = Ja/Nein-Checkbox, Loesung ist 0 oder 1   (OPTION = Image-id fuer Rueckmeldung)
% 3 = Reelle Zahlen Geparset
% 4 = Funktionen Geparset (mit Stuetzstellen zur ueberpruefung)

% Dieser Befehl erstellt ein interaktives Aufgabenfeld. Parameter:
% - #1 Laenge in Zeichen
% - #2 Loesungstext (alphanumerisch, case sensitive)
% - #3 AufgabenID (alphanumerisch, case sensitive)
% - #4 Typ (Kennnummer)
% - #5 String fuer Optionen (ggf. mit Semikolon getrennte Einzelstrings)
% - #6 Anzahl Punkte
% - #7 uxid (kann z.B. Loesungsstring sein)
% ACHTUNG: Die langen Zeilen bitte so lassen, Zeilenumbrueche im tex werden in div's umgesetzt
\newcommand{\MQuestionID}[7]{
\ifttm
\special{html:<!-- mdeclareuxid;;}UX#7\special{html:;;}\arabic{section}\special{html:;;}#3\special{html:;; //-->}%
\special{html:<!-- mdeclarepoints;;}\arabic{section}\special{html:;;}#3\special{html:;;}#6\special{html:;;}\arabic{MTestSite}\special{html:;;}\arabic{chapter}%
\special{html:;; //--><!-- onloadstart //-->CreateQuestionObj("}#7\special{html:",}\arabic{MFieldCounter}\special{html:,"}#2%
\special{html:","}#3\special{html:",}#4\special{html:,"}#5\special{html:",}#6\special{html:,}\arabic{MTestSite}\special{html:,}\arabic{section}%
\special{html:);<!-- onloadstop //-->}%
\special{html:<input mfieldtype="}#4\special{html:" name="Name_}#3\special{html:" id="}#3\special{html:" type="text" size="}#1\special{html:" maxlength="}#1%
\special{html:" }\ifnum\value{MGroupActive}=0\special{html:onfocus="handlerFocus(}\arabic{MFieldCounter}%
\special{html:);" onblur="handlerBlur(}\arabic{MFieldCounter}\special{html:);" onkeyup="handlerChange(}\arabic{MFieldCounter}\special{html:,0);" onpaste="handlerChange(}\arabic{MFieldCounter}\special{html:,0);" oninput="handlerChange(}\arabic{MFieldCounter}\special{html:,0);" onpropertychange="handlerChange(}\arabic{MFieldCounter}\special{html:,0);"/>}%
\special{html:<img src="images/questionmark.gif" width="20" height="20" border="0" align="absmiddle" id="}QM#3\special{html:"/>}
\else%
\special{html:onblur="handlerBlur(}\arabic{MFieldCounter}%
\special{html:);" onfocus="handlerFocus(}\arabic{MFieldCounter}\special{html:);" onkeyup="handlerChange(}\arabic{MFieldCounter}\special{html:,1);" onpaste="handlerChange(}\arabic{MFieldCounter}\special{html:,1);" oninput="handlerChange(}\arabic{MFieldCounter}\special{html:,1);" onpropertychange="handlerChange(}\arabic{MFieldCounter}\special{html:,1);"/>}%
\special{html:<img src="images/questionmark.gif" width="20" height="20" border="0" align="absmiddle" id="}QM#3\special{html:"/>}\fi%
\else%
\ifnum\value{QBoxFlag}=1\fbox{$\phantom{\MForLoop{#1}{b}}$}\else$\phantom{\MForLoop{#1}{b}}$\fi%
\fi%
}

% ACHTUNG: Die langen Zeilen bitte so lassen, Zeilenumbrueche im tex werden in div's umgesetzt
% QuestionCheckbox macht ausserhalb einer QuestionGroup keinen Sinn!
% #1 = solution (1 oder 0), ggf. mit ::smc abgetrennt auszuschliessende single-choice-boxen (UXIDs durch , getrennt), #2 = id, #3 = points, #4 = uxid
\newcommand{\MQuestionCheckbox}[4]{
\ifttm
\special{html:<!-- mdeclareuxid;;}UX#4\special{html:;;}\arabic{section}\special{html:;;}#2\special{html:;; //-->}%
\ifnum\value{MGroupActive}=0\MDebugMessage{ERROR: Checkbox Nr. \arabic{MFieldCounter}\ ist nicht in einer Kontrollgruppe, es wird niemals eine Loesung angezeigt!}\fi
\special{html: %
<!-- mdeclarepoints;;}\arabic{section}\special{html:;;}#2\special{html:;;}#3\special{html:;;}\arabic{MTestSite}\special{html:;;}\arabic{chapter}%
\special{html:;; //--><!-- onloadstart //-->CreateQuestionObj("}#4\special{html:",}\arabic{MFieldCounter}\special{html:,"}#1\special{html:","}#2\special{html:",2,"IMG}#2%
\special{html:",}#3\special{html:,}\arabic{MTestSite}\special{html:,}\arabic{section}\special{html:);<!-- onloadstop //-->}%
\special{html:<input mfieldtype="2" type="checkbox" name="Name_}#2\special{html:" id="}#2\special{html:" onchange="handlerChange(}\arabic{MFieldCounter}\special{html:,1);"/><img src="images/questionmark.gif" name="}Name_IMG#2%
\special{html:" width="20" height="20" border="0" align="absmiddle" id="}IMG#2\special{html:"/> }%
\else%
\ifnum\value{QBoxFlag}=1\fbox{$\phantom{X}$}\else$\phantom{X}$\fi%
\fi%
}

\def\MGenerateID{QFELD_\arabic{section}.\arabic{subsection}.\arabic{MSiteCounter}.QF\arabic{MFieldCounter}}

% #1 = 0/1 ggf. mit ::smc abgetrennt auszuschliessende single-choice-boxen (UXIDs durch , getrennt ohne UX), #2 = uxid ohne UX
\newcommand{\MCheckbox}[2]{
\MQuestionCheckbox{#1}{\MGenerateID}{\MStdPoints}{#2}
\addtocounter{MFieldCounter}{1}
}

% Erster Parameter: Zeichenlaenge der Eingabebox, zweiter Parameter: Loesungstext
\newcommand{\MQuestion}[2]{
\MQuestionID{#1}{#2}{\MGenerateID}{1}{0}{\MStdPoints}{#2}
\addtocounter{MFieldCounter}{1}
}

% Erster Parameter: Zeichenlaenge der Eingabebox, zweiter Parameter: Loesungstext
\newcommand{\MLQuestion}[3]{
\MQuestionID{#1}{#2}{\MGenerateID}{1}{0}{\MStdPoints}{#3}
\addtocounter{MFieldCounter}{1}
}

% Parameter: Laenge des Feldes, Loesung (wird auch geparsed), Stellen Genauigkeit hinter dem Komma, weitere Stellen werden mathematisch gerundet vor Vergleich
\newcommand{\MParsedQuestion}[3]{
\MQuestionID{#1}{#2}{\MGenerateID}{3}{#3}{\MStdPoints}{#2}
\addtocounter{MFieldCounter}{1}
}

% Parameter: Laenge des Feldes, Loesung (wird auch geparsed), Stellen Genauigkeit hinter dem Komma, weitere Stellen werden mathematisch gerundet vor Vergleich
\newcommand{\MLParsedQuestion}[4]{
\MQuestionID{#1}{#2}{\MGenerateID}{3}{#3}{\MStdPoints}{#4}
\addtocounter{MFieldCounter}{1}
}

% Parameter: Laenge des Feldes, Loesungsfunktion, Anzahl Stuetzstellen, Funktionsvariablen durch Kommata getrennt (nicht case-sensitive), Anzahl Nachkommastellen im Vergleich
\newcommand{\MFunctionQuestion}[5]{
\MQuestionID{#1}{#2}{\MGenerateID}{4}{#3;#4;#5;0}{\MStdPoints}{#2}
\addtocounter{MFieldCounter}{1}
}

% Parameter: Laenge des Feldes, Loesungsfunktion, Anzahl Stuetzstellen, Funktionsvariablen durch Kommata getrennt (nicht case-sensitive), Anzahl Nachkommastellen im Vergleich, UXID
\newcommand{\MLFunctionQuestion}[6]{
\MQuestionID{#1}{#2}{\MGenerateID}{4}{#3;#4;#5;0}{\MStdPoints}{#6}
\addtocounter{MFieldCounter}{1}
}

% Parameter: Laenge des Feldes, Loesungsintervall, Genauigkeit der Zahlenwertpruefung
\newcommand{\MIntervalQuestion}[3]{
\MQuestionID{#1}{#2}{\MGenerateID}{6}{#3}{\MStdPoints}{#2}
\addtocounter{MFieldCounter}{1}
}

% Parameter: Laenge des Feldes, Loesungsintervall, Genauigkeit der Zahlenwertpruefung, UXID
\newcommand{\MLIntervalQuestion}[4]{
\MQuestionID{#1}{#2}{\MGenerateID}{6}{#3}{\MStdPoints}{#4}
\addtocounter{MFieldCounter}{1}
}

% Parameter: Laenge des Feldes, Loesungsfunktion, Anzahl Stuetzstellen, Funktionsvariable (nicht case-sensitive), Anzahl Nachkommastellen im Vergleich, Vereinfachungsbedingung
% Vereinfachungsbedingung ist eine der Folgenden:
% 0 = Keine Vereinfachungsbedingung
% 1 = Keine Klammern (runde oder eckige) mehr im vereinfachten Ausdruck
% 2 = Faktordarstellung (Term hat Produkte als letzte Operation, Summen als vorgeschaltete Operation)
% 3 = Summendarstellung (Term hat Summen als letzte Operation, Produkte als vorgeschaltete Operation)
% Flag 512: Besondere Stuetzstellen (nur >1 und nur schwach rational), sonst symmetrisch um Nullpunkt und ganze Zahlen inkl. Null werden getroffen
\newcommand{\MSimplifyQuestion}[6]{
\MQuestionID{#1}{#2}{\MGenerateID}{4}{#3;#4;#5;#6}{\MStdPoints}{#2}
\addtocounter{MFieldCounter}{1}
}

\newcommand{\MLSimplifyQuestion}[7]{
\MQuestionID{#1}{#2}{\MGenerateID}{4}{#3;#4;#5;#6}{\MStdPoints}{#7}
\addtocounter{MFieldCounter}{1}
}

% Parameter: Laenge des Feldes, Loesung (optionaler Ausdruck), Anzahl Stuetzstellen, Funktionsvariable (nicht case-sensitive), Anzahl Nachkommastellen im Vergleich, Spezialtyp (string-id)
\newcommand{\MLSpecialQuestion}[7]{
\MQuestionID{#1}{#2}{\MGenerateID}{7}{#3;#4;#5;#6}{\MStdPoints}{#7}
\addtocounter{MFieldCounter}{1}
}

\newcounter{MGroupStart}
\newcounter{MGroupEnd}
\newcounter{MGroupActive}

\newenvironment{MQuestionGroup}{
\setcounter{MGroupStart}{\value{MFieldCounter}}
\setcounter{MGroupActive}{1}
}{
\setcounter{MGroupActive}{0}
\setcounter{MGroupEnd}{\value{MFieldCounter}}
\addtocounter{MGroupEnd}{-1}
}

\newcommand{\MGroupButton}[1]{
\ifttm
\special{html:<button name="Name_Group}\arabic{MGroupStart}\special{html:to}\arabic{MGroupEnd}\special{html:" id="Group}\arabic{MGroupStart}\special{html:to}\arabic{MGroupEnd}\special{html:" %
type="button" onclick="group_button(}\arabic{MGroupStart}\special{html:,}\arabic{MGroupEnd}\special{html:);">}#1\special{html:</button>}
\else
\phantom{#1}
\fi
}

%----------------- Makros fuer die modularisierte Darstellung ------------------------------------

\def\MyText#1{#1}

% is used internally by the conversion package, should not be used by original tex documents
\def\MOrgLabel#1{\relax}

\ifttm

% Ein MLabel wird im html codiert durch das tag <!-- mmlabel;;Labelbezeichner;;SubjectArea;;chapter;;section;;subsection;;Index;;Objekttyp; //-->
\def\MLabel#1{%
\ifnum\value{MLastType}=8%
\ifnum\value{MCaptionOn}=0%
\MDebugMessage{ERROR: Grafik \arabic{MGraphicsCounter} hat separates label: #1 (Grafiklabels sollten nur in der Caption stehen)}%
\fi
\fi
\ifnum\value{MLastType}=12%
\ifnum\value{MCaptionOn}=0%
\MDebugMessage{ERROR: Video \arabic{MVideoCounter} hat separates label: #1 (Videolabels sollten nur in der Caption stehen}%
\fi
\fi
\ifnum\value{MLastType}=10\setcounter{MLastIndex}{\value{equation}}\fi
\label{#1}\begin{html}<!-- mmlabel;;#1;;\end{html}\arabic{MSubjectArea}\special{html:;;}\arabic{chapter}\special{html:;;}\arabic{section}\special{html:;;}\arabic{subsection}\special{html:;;}\arabic{MLastIndex}\special{html:;;}\arabic{MLastType}\special{html:; //-->}}%

\else

% Sonderbehandlung im PDF fuer Abbildungen in separater aux-Datei, da MGraphics die figure-Umgebung nicht verwendet
\def\MLabel#1{%
\ifnum\value{MLastType}=8%
\ifnum\value{MCaptionOn}=0%
\MDebugMessage{ERROR: Grafik \arabic{MGraphicsCounter} hat separates label: #1 (Grafiklabels sollten nur in der Caption stehen}%
\fi
\fi
\ifnum\value{MLastType}=12%
\ifnum\value{MCaptionOn}=0%
\MDebugMessage{ERROR: Video \arabic{MVideoCounter} hat separates label: #1 (Videolabels sollten nur in der Caption stehen}%
\fi
\fi
\label{#1}%
}%

\fi

% Gibt Begriff des referenzierten Objekts mit aus, aber nur im HTML, daher nur in Ausnahmefaellen (z.B. Copyrightliste) sinnvoll
\def\MCRef#1{\ifttm\special{html:<!-- mmref;;}#1\special{html:;;1; //-->}\else\vref{#1}\fi}


\def\MRef#1{\ifttm\special{html:<!-- mmref;;}#1\special{html:;;0; //-->}\else\vref{#1}\fi}
\def\MERef#1{\ifttm\special{html:<!-- mmref;;}#1\special{html:;;0; //-->}\else\eqref{#1}\fi}
\def\MNRef#1{\ifttm\special{html:<!-- mmref;;}#1\special{html:;;0; //-->}\else\ref{#1}\fi}
\def\MSRef#1#2{\ifttm\special{html:<!-- msref;;}#1\special{html:;;}#2\special{html:; //-->}\else \if#2\empty \ref{#1} \else \hyperref[#1]{#2}\fi\fi} 

\def\MRefRange#1#2{\ifttm\MRef{#1} bis 
\MRef{#2}\else\vrefrange[\unskip]{#1}{#2}\fi}

\def\MRefTwo#1#2{\ifttm\MRef{#1} und \MRef{#2}\else%
\let\vRefTLRsav=\reftextlabelrange\let\vRefTPRsav=\reftextpagerange%
\def\reftextlabelrange##1##2{\ref{##1} und~\ref{##2}}%
\def\reftextpagerange##1##2{auf den Seiten~\pageref{#1} und~\pageref{#2}}%
\vrefrange[\unskip]{#1}{#2}%
\let\reftextlabelrange=\vRefTLRsav\let\reftextpagerange=\vRefTPRsav\fi}

% MSectionChapter definiert falls notwendig das Kapitel vor der section. Das ist notwendig, wenn nur ein Einzelmodul uebersetzt wird.
% MChaptersGiven ist ein Counter, der von mconvert.pl vordefiniert wird.
\ifttm
\newcommand{\MSectionChapter}{\ifnum\value{MChaptersGiven}=0{\Dchapter{Modul}}\else{}\fi}
\else
\newcommand{\MSectionChapter}{\ifnum\value{chapter}=0{\Dchapter{Modul}}\else{}\fi}
\fi


\def\MChapter#1{\ifnum\value{MSSEnd}>0{\MSubsectionEndMacros}\addtocounter{MSSEnd}{-1}\fi\Dchapter{#1}}
\def\MSubject#1{\MChapter{#1}} % Schluesselwort HELPSECTION ist reserviert fuer Hilfesektion

\newcommand{\MSectionID}{UNKNOWNID}

\ifttm
\newcommand{\MSetSectionID}[1]{\renewcommand{\MSectionID}{#1}}
\else
\newcommand{\MSetSectionID}[1]{\renewcommand{\MSectionID}{#1}\tikzsetexternalprefix{#1}}
\fi


\newcommand{\MSection}[1]{\MSetSectionID{MODULID}\ifnum\value{MSSEnd}>0{\MSubsectionEndMacros}\addtocounter{MSSEnd}{-1}\fi\MSectionChapter\Dsection{#1}\MSectionStartMacros{#1}\setcounter{MLastIndex}{-1}\setcounter{MLastType}{1}} % Sections werden ueber das section-Feld im mmlabel-Tag identifiziert, nicht ueber das Indexfeld

\def\MSubsection#1{\ifnum\value{MSSEnd}>0{\MSubsectionEndMacros}\addtocounter{MSSEnd}{-1}\fi\ifttm\else\clearpage\fi\Dsubsection{#1}\MSubsectionStartMacros\setcounter{MLastIndex}{-1}\setcounter{MLastType}{2}\addtocounter{MSSEnd}{1}}% Subsections werden ueber das subsection-Feld im mmlabel-Tag identifiziert, nicht ueber das Indexfeld
\def\MSubsectionx#1{\Dsubsectionx{#1}} % Nur zur Verwendung in MSectionStart gedacht
\def\MSubsubsection#1{\Dsubsubsection{#1}\setcounter{MLastIndex}{\value{subsubsection}}\setcounter{MLastType}{3}\ifttm\special{html:<!-- sectioninfo;;}\arabic{section}\special{html:;;}\arabic{subsection}\special{html:;;}\arabic{subsubsection}\special{html:;;1;;}\arabic{MTestSite}\special{html:; //-->}\fi}
\def\MSubsubsectionx#1{\Dsubsubsectionx{#1}\ifttm\special{html:<!-- sectioninfo;;}\arabic{section}\special{html:;;}\arabic{subsection}\special{html:;;}\arabic{subsubsection}\special{html:;;0;;}\arabic{MTestSite}\special{html:; //-->}\else\addcontentsline{toc}{subsection}{#1}\fi}

\ifttm
\def\MSubsubsubsectionx#1{\ \newline\textbf{#1}\special{html:<br />}}
\else
\def\MSubsubsubsectionx#1{\ \newline
\textbf{#1}\ \\
}
\fi


% Dieses Skript wird zu Beginn jedes Modulabschnitts (=Webseite) ausgefuehrt und initialisiert den Aufgabenfeldzaehler
\newcommand{\MPageScripts}{
\setcounter{MFieldCounter}{1}
\addtocounter{MSiteCounter}{1}
\setcounter{MHintCounter}{1}
\setcounter{MCodeEditCounter}{1}
\setcounter{MGroupActive}{0}
\DoQBoxes
% Feldvariablen werden im HTML-Header in conv.pl eingestellt
}

% Dieses Skript wird zum Ende jedes Modulabschnitts (=Webseite) ausgefuehrt
\ifttm
\newcommand{\MEndScripts}{\special{html:<br /><!-- mfeedbackbutton;Seite;}\arabic{MTestSite}\special{html:;}\MGenerateSiteNumber\special{html:; //-->}
}
\else
\newcommand{\MEndScripts}{\relax}
\fi


\newcounter{QBoxFlag}
\newcommand{\DoQBoxes}{\setcounter{QBoxFlag}{1}}
\newcommand{\NoQBoxes}{\setcounter{QBoxFlag}{0}}

\newcounter{MXCTest}
\newcounter{MXCounter}
\newcounter{MSCounter}



\ifttm

% Struktur des sectioninfo-Tags: <!-- sectioninfo;;section;;subsection;;subsubsection;;nr_ausgeben;;testpage; //-->

%Fuegt eine zusaetzliche html-Seite an hinter ALLEN bisherigen und zukuenftigen content-Seiten ausserhalb der vor-zurueck-Schleife (d.h. nur durch Button oder MIntLink erreichbar!)
% #1 = Titel des Modulabschnitts, #2 = Kurztitel fuer die Buttons, #3 = Buttonkennung (STD = default nehmen, NONE = Ohne Button in der Navigation)
\newenvironment{MSContent}[3]{\special{html:<div class="xcontent}\arabic{MSCounter}\special{html:"><!-- scontent;-;}\arabic{MSCounter};-;#1;-;#2;-;#3\special{html: //-->}\MPageScripts\MSubsubsectionx{#1}}{\MEndScripts\special{html:<!-- endscontent;;}\arabic{MSCounter}\special{html: //--></div>}\addtocounter{MSCounter}{1}}

% Fuegt eine zusaetzliche html-Seite ein hinter den bereits vorhandenen content-Seiten (oder als erste Seite) innerhalb der vor-zurueck-Schleife der Navigation
% #1 = Titel des Modulabschnitts, #2 = Kurztitel fuer die Buttons, #3 = Buttonkennung (STD = Defaultbutton, NONE = Ohne Button in der Navigation)
\newenvironment{MXContent}[3]{\special{html:<div class="xcontent}\arabic{MXCounter}\special{html:"><!-- xcontent;-;}\arabic{MXCounter};-;#1;-;#2;-;#3\special{html: //-->}\MPageScripts\MSubsubsection{#1}}{\MEndScripts\special{html:<!-- endxcontent;;}\arabic{MXCounter}\special{html: //--></div>}\addtocounter{MXCounter}{1}}

% Fuegt eine zusaetzliche html-Seite ein die keine subsubsection-Nummer bekommt, nur zur internen Verwendung in mintmod.tex gedacht!
% #1 = Titel des Modulabschnitts, #2 = Kurztitel fuer die Buttons, #3 = Buttonkennung (STD = Defaultbutton, NONE = Ohne Button in der Navigation)
% \newenvironment{MUContent}[3]{\special{html:<div class="xcontent}\arabic{MXCounter}\special{html:"><!-- xcontent;-;}\arabic{MXCounter};-;#1;-;#2;-;#3\special{html: //-->}\MPageScripts\MSubsubsectionx{#1}}{\MEndScripts\special{html:<!-- endxcontent;;}\arabic{MXCounter}\special{html: //--></div>}\addtocounter{MXCounter}{1}}

\newcommand{\MDeclareSiteUXID}[1]{\special{html:<!-- mdeclaresiteuxid;;}#1\special{html:;;}\arabic{chapter}\special{html:;;}\arabic{section}\special{html:;; //-->}}

\else

%\newcommand{\MSubsubsection}[1]{\refstepcounter{subsubsection} \addcontentsline{toc}{subsubsection}{\thesubsubsection. #1}}


% Fuegt eine zusaetzliche html-Seite an hinter den bereits vorhandenen content-Seiten
% #1 = Titel des Modulabschnitts, #2 = Kurztitel fuer die Buttons, #3 = Iconkennung (im PDF wirkungslos)
%\newenvironment{MUContent}[3]{\ifnum\value{MXCTest}>0{\MDebugMessage{ERROR: Geschachtelter SContent}}\fi\MPageScripts\MSubsubsectionx{#1}\addtocounter{MXCTest}{1}}{\addtocounter{MXCounter}{1}\addtocounter{MXCTest}{-1}}
\newenvironment{MXContent}[3]{\ifnum\value{MXCTest}>0{\MDebugMessage{ERROR: Geschachtelter SContent}}\fi\MPageScripts\MSubsubsection{#1}\addtocounter{MXCTest}{1}}{\addtocounter{MXCounter}{1}\addtocounter{MXCTest}{-1}}
\newenvironment{MSContent}[3]{\ifnum\value{MXCTest}>0{\MDebugMessage{ERROR: Geschachtelter XContent}}\fi\MPageScripts\MSubsubsectionx{#1}\addtocounter{MXCTest}{1}}{\addtocounter{MSCounter}{1}\addtocounter{MXCTest}{-1}}

\newcommand{\MDeclareSiteUXID}[1]{\relax}

\fi 

% GHEADER und GFOOTER werden von split.pm gefunden, aber nur, wenn nicht HELPSITE oder TESTSITE
\ifttm
\newenvironment{MSectionStart}{\special{html:<div class="xcontent0">}\MSubsubsectionx{Modul\"ubersicht}}{\setcounter{MSSEnd}{0}\special{html:</div>}}
% Darf nicht als XContent nummeriert werden, darf nicht als XContent gelabelt werden, wird aber in eine xcontent-div gesetzt fuer Python-parsing
\else
\newenvironment{MSectionStart}{\MSubsectionx{Modul\"ubersicht}}{\setcounter{MSSEnd}{0}}
\fi

\newenvironment{MIntro}{\begin{MXContent}{Einf\"uhrung}{Einf\"uhrung}{genetisch}}{\end{MXContent}}
\newenvironment{MContent}{\begin{MXContent}{Inhalt}{Inhalt}{beweis}}{\end{MXContent}}
\newenvironment{MExercises}{\ifttm\else\clearpage\fi\begin{MXContent}{Aufgaben}{Aufgaben}{aufgb}\special{html:<!-- declareexcsymb //-->}}{\end{MXContent}}

% #1 = Lesbare Testbezeichnung
\newenvironment{MTest}[1]{%
\renewcommand{\MTestName}{#1}
\ifttm\else\clearpage\fi%
\addtocounter{MTestSite}{1}%
\begin{MXContent}{#1}{#1}{STD} % {aufgb}%
\special{html:<!-- declaretestsymb //-->}
\begin{MQuestionGroup}%
\MInTestHeader
}%
{%
\end{MQuestionGroup}%
\ \\ \ \\%
\MInTestFooter
\end{MXContent}\addtocounter{MTestSite}{-1}%
}

\newenvironment{MExtra}{\ifttm\else\clearpage\fi\begin{MXContent}{Zus\"atzliche Inhalte}{Zusatz}{weiterfhrg}}{\end{MXContent}}

\makeindex

\ifttm
\def\MPrintIndex{
\ifnum\value{MSSEnd}>0{\MSubsectionEndMacros}\addtocounter{MSSEnd}{-1}\fi
\renewcommand{\indexname}{Stichwortverzeichnis}
\special{html:<p><!-- printindex //--></p>}
}
\else
\def\MPrintIndex{
\ifnum\value{MSSEnd}>0{\MSubsectionEndMacros}\addtocounter{MSSEnd}{-1}\fi
\renewcommand{\indexname}{Stichwortverzeichnis}
\addcontentsline{toc}{section}{Stichwortverzeichnis}
\printindex
}
\fi


% Konstanten fuer die Modulfaecher

\def\MINTMathematics{1}
\def\MINTInformatics{2}
\def\MINTChemistry{3}
\def\MINTPhysics{4}
\def\MINTEngineering{5}

\newcounter{MSubjectArea}
\newcounter{MInfoNumbers} % Gibt an, ob die Infoboxen nummeriert werden sollen
\newcounter{MSepNumbers} % Gibt an, ob Beispiele und Experimente separat nummeriert werden sollen
\newcommand{\MSetSubject}[1]{
 % ttm kapiert setcounter mit Parametern nicht, also per if abragen und einsetzen
\ifnum#1=1\setcounter{MSubjectArea}{1}\setcounter{MInfoNumbers}{1}\setcounter{MSepNumbers}{0}\fi
\ifnum#1=2\setcounter{MSubjectArea}{2}\setcounter{MInfoNumbers}{1}\setcounter{MSepNumbers}{0}\fi
\ifnum#1=3\setcounter{MSubjectArea}{3}\setcounter{MInfoNumbers}{0}\setcounter{MSepNumbers}{1}\fi
\ifnum#1=4\setcounter{MSubjectArea}{4}\setcounter{MInfoNumbers}{0}\setcounter{MSepNumbers}{0}\fi
\ifnum#1=5\setcounter{MSubjectArea}{5}\setcounter{MInfoNumbers}{1}\setcounter{MSepNumbers}{0}\fi
% Separate Nummerntechnik fuer unsere Chemiker: alles dreistellig
\ifnum#1=3
  \ifttm
  \renewcommand{\theequation}{\arabic{section}.\arabic{subsection}.\arabic{equation}}
  \renewcommand{\thetable}{\arabic{section}.\arabic{subsection}.\arabic{table}} 
  \renewcommand{\thefigure}{\arabic{section}.\arabic{subsection}.\arabic{figure}} 
  \else
  \renewcommand{\theequation}{\arabic{chapter}.\arabic{section}.\arabic{equation}}
  \renewcommand{\thetable}{\arabic{chapter}.\arabic{section}.\arabic{table}}
  \renewcommand{\thefigure}{\arabic{chapter}.\arabic{section}.\arabic{figure}}
  \fi
\else
  \ifttm
  \renewcommand{\theequation}{\arabic{section}.\arabic{subsection}.\arabic{equation}}
  \renewcommand{\thetable}{\arabic{table}}
  \renewcommand{\thefigure}{\arabic{figure}}
  \else
  \renewcommand{\theequation}{\arabic{chapter}.\arabic{section}.\arabic{equation}}
  \renewcommand{\thetable}{\arabic{table}}
  \renewcommand{\thefigure}{\arabic{figure}}
  \fi
\fi
}

% Fuer tikz Autogenerierung
\newcounter{MTIKZAutofilenumber}

% Spezielle Counter fuer die Bentz-Module
\newcounter{mycounter}
\newcounter{chemapplet}
\newcounter{physapplet}

\newcounter{MSSEnd} % Ist 1 falls ein MSubsection aktiv ist, der einen MSubsectionEndMacro-Aufruf verursacht
\newcounter{MFileNumber}
\def\MLastFile{\special{html:[[!-- mfileref;;}\arabic{MFileNumber}\special{html:; //--]]}}

% Vollstaendiger Pfad ist \MMaterial / \MLastFilePath / \MLastFileName    ==   \MMaterial / \MLastFile

% Wird nur bei kompletter Baum-Erstellung ausgefuehrt!
% #1 = Lesbare Modulbezeichnung
\newcommand{\MSectionStartMacros}[1]{
\setcounter{MTestSite}{0}
\setcounter{MCaptionOn}{0}
\setcounter{MLastTypeEq}{0}
\setcounter{MSSEnd}{0}
\setcounter{MFileNumber}{0} % Preinkrekement-Counter
\setcounter{MTIKZAutofilenumber}{0}
\setcounter{mycounter}{1}
\setcounter{physapplet}{1}
\setcounter{chemapplet}{0}
\ifttm
\special{html:<!-- mdeclaresection;;}\arabic{chapter}\special{html:;;}\arabic{section}\special{html:;;}#1\special{html:;; //-->}%
\else
\setcounter{thmc}{0}
\setcounter{exmpc}{0}
\setcounter{verc}{0}
\setcounter{infoc}{0}
\fi
\setcounter{MiniMarkerCounter}{1}
\setcounter{AlignCounter}{1}
\setcounter{MXCTest}{0}
\setcounter{MCodeCounter}{0}
\setcounter{MEntryCounter}{0}
}

% Wird immer ausgefuehrt
\newcommand{\MSubsectionStartMacros}{
\ifttm\else\MPageHeaderDef\fi
\MWatermarkSettings
\setcounter{MXCounter}{0}
\setcounter{MSCounter}{0}
\setcounter{MSiteCounter}{1}
\setcounter{MExerciseCollectionCounter}{0}
% Zaehler fuer das Labelsystem zuruecksetzen (prefix-Zaehler)
\setcounter{MInfoCounter}{0}
\setcounter{MExerciseCounter}{0}
\setcounter{MExampleCounter}{0}
\setcounter{MExperimentCounter}{0}
\setcounter{MGraphicsCounter}{0}
\setcounter{MTableCounter}{0}
\setcounter{MTheoremCounter}{0}
\setcounter{MObjectCounter}{0}
\setcounter{MEquationCounter}{0}
\setcounter{MVideoCounter}{0}
\setcounter{equation}{0}
\setcounter{figure}{0}
}

\newcommand{\MSubsectionEndMacros}{
% Bei Chemiemodulen das PSE einhaengen, es soll als SContent am Ende erscheinen
\special{html:<!-- subsectionend //-->}
\ifnum\value{MSubjectArea}=3{\MIncludePSE}\fi
}


\ifttm
%\newcommand{\MEmbed}[1]{\MRegisterFile{#1}\begin{html}<embed src="\end{html}\MMaterial/\MLastFile\begin{html}" width="192" height="189"></embed>\end{html}}
\newcommand{\MEmbed}[1]{\MRegisterFile{#1}\begin{html}<embed src="\end{html}\MMaterial/\MLastFile\begin{html}"></embed>\end{html}}
\fi

%----------------- Makros fuer die Textdarstellung -----------------------------------------------

\ifttm
% MUGraphics bindet eine Grafik ein:
% Parameter 1: Dateiname der Grafik, relativ zur Position des Modul-Tex-Dokuments
% Parameter 2: Skalierungsoptionen fuer PDF (fuer includegraphics)
% Parameter 3: Titel fuer die Grafik, wird unter die Grafik mit der Grafiknummer gesetzt und kann MLabel bzw. MCopyrightLabel enthalten
% Parameter 4: Skalierungsoptionen fuer HTML (css-styles)

% ERSATZ: <img alt="My Image" src="data:image/png;base64,iVBORwA<MoreBase64SringHere>" />


\newcommand{\MUGraphics}[4]{\MRegisterFile{#1}\begin{html}
<div class="imagecenter">
<center>
<div>
<img src="\end{html}\MMaterial/\MLastFile\begin{html}" style="#4" alt="\end{html}\MMaterial/\MLastFile\begin{html}"/>
</div>
<div class="bildtext">
\end{html}
\addtocounter{MGraphicsCounter}{1}
\setcounter{MLastIndex}{\value{MGraphicsCounter}}
\setcounter{MLastType}{8}
\addtocounter{MCaptionOn}{1}
\ifnum\value{MSepNumbers}=0
\textbf{Abbildung \arabic{MGraphicsCounter}:} #3
\else
\textbf{Abbildung \arabic{section}.\arabic{subsection}.\arabic{MGraphicsCounter}:} #3
\fi
\addtocounter{MCaptionOn}{-1}
\begin{html}
</div>
</center>
</div>
<br />
\end{html}%
\special{html:<!-- mfeedbackbutton;Abbildung;}\arabic{MGraphicsCounter}\special{html:;}\arabic{section}.\arabic{subsection}.\arabic{MGraphicsCounter}\special{html:; //-->}%
}

% MVideo bindet ein Video als Einzeldatei ein:
% Parameter 1: Dateiname des Videos, relativ zur Position des Modul-Tex-Dokuments, ohne die Endung ".mp4"
% Parameter 2: Titel fuer das Video (kann MLabel oder MCopyrightLabel enthalten), wird unter das Video mit der Videonummer gesetzt
\newcommand{\MVideo}[2]{\MRegisterFile{#1.mp4}\begin{html}
<div class="imagecenter">
<center>
<div>
<video width="95\%" controls="controls"><source src="\end{html}\MMaterial/#1.mp4\begin{html}" type="video/mp4">Ihr Browser kann keine MP4-Videos abspielen!</video>
</div>
<div class="bildtext">
\end{html}
\addtocounter{MVideoCounter}{1}
\setcounter{MLastIndex}{\value{MVideoCounter}}
\setcounter{MLastType}{12}
\addtocounter{MCaptionOn}{1}
\ifnum\value{MSepNumbers}=0
\textbf{Video \arabic{MVideoCounter}:} #2
\else
\textbf{Video \arabic{section}.\arabic{subsection}.\arabic{MVideoCounter}:} #2
\fi
\addtocounter{MCaptionOn}{-1}
\begin{html}
</div>
</center>
</div>
<br />
\end{html}}

\newcommand{\MDVideo}[2]{\MRegisterFile{#1.mp4}\MRegisterFile{#1.ogv}\begin{html}
<div class="imagecenter">
<center>
<div>
<video width="70\%" controls><source src="\end{html}\MMaterial/#1.mp4\begin{html}" type="video/mp4"><source src="\end{html}\MMaterial/#1.ogv\begin{html}" type="video/ogg">Ihr Browser kann keine MP4-Videos abspielen!</video>
</div>
<br />
#2
</center>
</div>
<br />
\end{html}
}

\newcommand{\MGraphics}[3]{\MUGraphics{#1}{#2}{#3}{}}

\else

\newcommand{\MVideo}[2]{%
% Kein Video im PDF darstellbar, trotzdem so tun als ob da eines waere
\begin{center}
(Video nicht darstellbar)
\end{center}
\addtocounter{MVideoCounter}{1}
\setcounter{MLastIndex}{\value{MVideoCounter}}
\setcounter{MLastType}{12}
\addtocounter{MCaptionOn}{1}
\ifnum\value{MSepNumbers}=0
\textbf{Video \arabic{MVideoCounter}:} #2
\else
\textbf{Video \arabic{section}.\arabic{subsection}.\arabic{MVideoCounter}:} #2
\fi
\addtocounter{MCaptionOn}{-1}
}


% MGraphics bindet eine Grafik ein:
% Parameter 1: Dateiname der Grafik, relativ zur Position des Modul-Tex-Dokuments
% Parameter 2: Skalierungsoptionen fuer PDF (fuer includegraphics)
% Parameter 3: Titel fuer die Grafik, wird unter die Grafik mit der Grafiknummer gesetzt
\newcommand{\MGraphics}[3]{%
\MRegisterFile{#1}%
\ %
\begin{figure}[H]%
\centering{%
\includegraphics[#2]{\MDPrefix/#1}%
\addtocounter{MCaptionOn}{1}%
\caption{#3}%
\addtocounter{MCaptionOn}{-1}%
}%
\end{figure}%
\addtocounter{MGraphicsCounter}{1}\setcounter{MLastIndex}{\value{MGraphicsCounter}}\setcounter{MLastType}{8}\ %
%\ \\Abbildung \ifnum\value{MSepNumbers}=0\else\arabic{chapter}.\arabic{section}.\fi\arabic{MGraphicsCounter}: #3%
}

\newcommand{\MUGraphics}[4]{\MGraphics{#1}{#2}{#3}}


\fi

\newcounter{MCaptionOn} % = 1 falls eine Grafikcaption aktiv ist, = 0 sonst


% MGraphicsSolo bindet eine Grafik pur ein ohne Titel
% Parameter 1: Dateiname der Grafik, relativ zur Position des Modul-Tex-Dokuments
% Parameter 2: Skalierungsoptionen (wirken nur im PDF)
\newcommand{\MGraphicsSolo}[2]{\MUGraphicsSolo{#1}{#2}{}}

% MUGraphicsSolo bindet eine Grafik pur ein ohne Titel, aber mit HTML-Skalierung
% Parameter 1: Dateiname der Grafik, relativ zur Position des Modul-Tex-Dokuments
% Parameter 2: Skalierungsoptionen (wirken nur im PDF)
% Parameter 3: Skalierungsoptionen (wirken nur im HTML), als style-format: "width=???, height=???"
\ifttm
\newcommand{\MUGraphicsSolo}[3]{\MRegisterFile{#1}\begin{html}
<img src="\end{html}\MMaterial/\MLastFile\begin{html}" style="\end{html}#3\begin{html}" alt="\end{html}\MMaterial/\MLastFile\begin{html}"/>
\end{html}%
\special{html:<!-- mfeedbackbutton;Abbildung;}#1\special{html:;}\MMaterial/\MLastFile\special{html:; //-->}%
}
\else
\newcommand{\MUGraphicsSolo}[3]{\MRegisterFile{#1}\includegraphics[#2]{\MDPrefix/#1}}
\fi

% Externer Link mit URL
% Erster Parameter: Vollstaendige(!) URL des Links
% Zweiter Parameter: Text fuer den Link
\newcommand{\MExtLink}[2]{\ifttm\special{html:<a target="_new" href="}#1\special{html:">}#2\special{html:</a>}\else\href{#1}{#2}\fi} % ohne MINTERLINK!


% Interner Link, die verlinkte Datei muss im gleichen Verzeichnis liegen wie die Modul-Texdatei
% Erster Parameter: Dateiname
% Zweiter Parameter: Text fuer den Link
\newcommand{\MIntLink}[2]{\ifttm\MRegisterFile{#1}\special{html:<a class="MINTERLINK" target="_new" href="}\MMaterial/\MLastFile\special{html:">}#2\special{html:</a>}\else{\href{#1}{#2}}\fi}


\ifttm
\def\MMaterial{:localmaterial:}
\else
\def\MMaterial{\MDPrefix}
\fi

\ifttm
\def\MNoFile#1{:directmaterial:#1}
\else
\def\MNoFile#1{#1}
\fi

\newcommand{\MChem}[1]{$\mathrm{#1}$}

\newcommand{\MApplet}[3]{
% Bindet ein Java-Applet ein, die Parameter sind:
% (wird nur im HTML, aber nicht im PDF erstellt)
% #1 Dateiname des Applets (muss mit ".class" enden)
% #2 = Breite in Pixeln
% #3 = Hoehe in Pixeln
\ifttm
\MRegisterFile{#1}
\begin{html}
<applet code="\end{html}\MMaterial/\MLastFile\begin{html}" width="#2" height="#3" alt="[Java-Applet kann nicht gestartet werden]"></applet>
\end{html}
\fi
}

\newcommand{\MScriptPage}[2]{
% Bindet eine JavaScript-Datei ein, die eine eigene Seite bekommt
% (wird nur im HTML, aber nicht im PDF erstellt)
% #1 Dateiname des Programms (sollte mit ".js" enden)
% #2 = Kurztitel der Seite
\ifttm
\begin{MSContent}{#2}{#2}{puzzle}
\MRegisterFile{#1}
\begin{html}
<script src="\MMaterial/\MLastFile" type="text/javascript"></script>
\end{html}
\end{MSContent}
\fi
}

\newcommand{\MIncludePSE}{
% Bindet bei Chemie-Modulen das PSE ein
% (wird nur im HTML, aber nicht im PDF erstellt)
\ifttm
\special{html:<!-- includepse //-->}
\begin{MSContent}{Periodensystem der Elemente}{PSE}{table}
\MRegisterFile{../files/pse.js}
\MRegisterFile{../files/radio.png}
\begin{html}
<script src="\MMaterial/../files/pse.js" type="text/javascript"></script>
<p id="divid"><br /><br />
<script language="javascript" type="text/javascript">
    startpse("divid","\MMaterial/../files"); 
</script>
</p>
<br />
<br />
<br />
<p>Die Farben der Elementsymbole geben an: <font style="color:Red">gasf&ouml;rmig </font> <font style="color:Blue">fl&uuml;ssig </font> fest</p>
<p>Die Elemente der Gruppe 1 A, 2 A, 3 A usw. geh&ouml;ren zu den Hauptgruppenelementen.</p>
<p>Die Elemente der Gruppe 1 B, 2 B, 3 B usw. geh&ouml;ren zu den Nebengruppenelementen.</p>
<p>() kennzeichnet die Masse des stabilsten Isotops</p>
\end{html}
\end{MSContent}
\fi
}

\newcommand{\MAppletArchive}[4]{
% Bindet ein Java-Applet ein, die Parameter sind:
% (wird nur im HTML, aber nicht im PDF erstellt)
% #1 Dateiname der Klasse mit Appletaufruf (muss mit ".class" enden)
% #2 Dateiname des Archivs (muss mit ".jar" enden)
% #3 = Breite in Pixeln
% #4 = Hoehe in Pixeln
\ifttm
\MRegisterFile{#2}
\begin{html}
<applet code="#1" archive="\end{html}\MMaterial/\MLastFile\begin{html}" codebase="." width="#3" height="#4" alt="[Java-Archiv kann nicht gestartet werden]"></applet>
\end{html}
\fi
}

% Bindet in der Haupttexdatei ein MINT-Modul ein. Parameter 1 ist das Verzeichnis (relativ zur Haupttexdatei), Parameter 2 ist der Dateinahme ohne Pfad.
\newcommand{\IncludeModule}[2]{
\renewcommand{\MDPrefix}{#1}
\input{#1/#2}
\ifnum\value{MSSEnd}>0{\MSubsectionEndMacros}\addtocounter{MSSEnd}{-1}\fi
}

% Der ttm-Konverter setzt keine Makros im \input um, also muss hier getrickst werden:
% Das MDPrefix muss in den einzelnen Modulen manuell eingesetzt werden
\newcommand{\MInputFile}[1]{
\ifttm
\input{#1}
\else
\input{#1}
\fi
}


\newcommand{\MCases}[1]{\left\lbrace{\begin{array}{rl} #1 \end{array}}\right.}

\ifttm
\newenvironment{MCaseEnv}{\left\lbrace\begin{array}{rl}}{\end{array}\right.}
\else
\newenvironment{MCaseEnv}{\left\lbrace\begin{array}{rl}}{\end{array}\right.}
\fi

\def\MSkip{\ifttm\MCR\fi}

\ifttm
\def\MCR{\special{html:<br />}}
\else
\def\MCR{\ \\}
\fi


% Pragmas - Sind Schluesselwoerter, die dem Preprocessing sowie dem Konverter uebergeben werden und bestimmte
%           Aktionen ausloesen. Im Output (PDF und HTML) tauchen sie nicht auf.
\newcommand{\MPragma}[1]{%
\ifttm%
\special{html:<!-- mpragma;-;}#1\special{html:;; -->}%
\else%
% MPragmas werden vom Preprozessor direkt im LaTeX gefunden
\fi%
}

% Ersatz der Befehle textsubscript und textsuperscript, die ttm nicht kennt
\ifttm%
\newcommand{\MTextsubscript}[1]{\special{html:<sub>}#1\special{html:</sub>}}%
\newcommand{\MTextsuperscript}[1]{\special{html:<sup>}#1\special{html:</sup>}}%
\else%
\newcommand{\MTextsubscript}[1]{\textsubscript{#1}}%
\newcommand{\MTextsuperscript}[1]{\textsuperscript{#1}}%
\fi

%------------------ Einbindung von dia-Diagrammen ----------------------------------------------
% Beim preprocessing wird aus jeder dia-Datei eine tex-Datei und eine pdf-Datei erzeugt,
% diese werden hier jeweils im PDF und HTML eingebunden
% Parameter: Dateiname der mit dia erstellten Datei (OHNE die Endung .dia)
\ifttm%
\newcommand{\MDia}[1]{%
\MGraphicsSolo{#1minthtml.png}{}%
}
\else%
\newcommand{\MDia}[1]{%
\MGraphicsSolo{#1mintpdf.png}{scale=0.1667}%
}
\fi%

% subsup funktioniert im Ausdruck $D={\R}^+_0$, also \R geklammert und sup zuerst
% \ifttm
% \def\MSubsup#1#2#3{\special{html:<msubsup>} #1 #2 #3\special{html:</msubsup>}}
% \else
% \def\MSubsup#1#2#3{{#1}^{#3}_{#2}}
% \fi

%\input{local.tex}

% \ifttm
% \else
% \newwrite\mintlog
% \immediate\openout\mintlog=mintlog.txt
% \fi

% ----------------------- tikz autogenerator -------------------------------------------------------------------

\newcommand{\Mtikzexternalize}{\tikzexternalize}% wird bei Konvertierung ueber mconvert ggf. ausgehebelt!

\ifttm
\else
\tikzset%
{
  % Defines a custom style which generates pdf and converts to (low and hi-res quality) png and svg, then deletes the pdf
  % Important: DO NOT directly convert from pdf to hires-png or from svg to png with GraphViz convert as it has some problems and memory leaks
  png export/.style=%
  {
    external/system call/.add={}{; 
      pdf2svg "\image.pdf" "\image.svg" ; 
      convert -density 112.5 -transparent white "\image.pdf" "\image.png"; 
      inkscape --export-png="\image.4x.png" --export-dpi=450 --export-background-opacity=0 --without-gui "\image.svg"; 
      rm "\image.pdf"; rm "\image.log"; rm "\image.dpth"; rm "\image.idx"
    },
    external/force remake,
  }
}
\tikzset{png export}
\tikzsetexternalprefix{}
% PNGs bei externer Erzeugung in "richtiger" Groesse einbinden
\pgfkeys{/pgf/images/include external/.code={\includegraphics[scale=0.64]{#1}}}
\fi

% Spezielle Umgebung fuer Autogenerierung, Bildernamen sind nur innerhalb eines Moduls (einer MSection) eindeutig)

\newcommand{\MTIKZautofilename}{tikzautofile}

\ifttm
% HTML-Version: Vom Autogenerator erzeugte png-Datei einbinden, tikz selbst nicht ausfuehren (sprich: #1 schlucken)
\newcommand{\MTikzAuto}[1]{%
\addtocounter{MTIKZAutofilenumber}{1}
\renewcommand{\MTIKZautofilename}{mtikzauto_\arabic{MTIKZAutofilenumber}}
\MUGraphicsSolo{\MSectionID\MTIKZautofilename.4x.png}{scale=1}{\special{html:[[!-- svgstyle;}\MSectionID\MTIKZautofilename\special{html: //--]]}} % Styleinfos werden aus original-png, nicht 4x-png geholt!
%\MRegisterFile{\MSectionID\MTIKZautofilename.png} % not used right now
%\MRegisterFile{\MSectionID\MTIKZautofilename.svg}
}
\else%
% PDF-Version: Falls Autogenerator aktiv wird Datei automatisch benannt und exportiert
\newcommand{\MTikzAuto}[1]{%
\addtocounter{MTIKZAutofilenumber}{1}%
\renewcommand{\MTIKZautofilename}{mtikzauto_\arabic{MTIKZAutofilenumber}}
\tikzsetnextfilename{\MTIKZautofilename}%
#1%
}
\fi

% In einer reinen LaTeX-Uebersetzung kapselt der Preambelinclude-Befehl nur input,
% in einer konvertergesteuerten PDF/HTML-Uebersetzung wird er dagegen entfernt und
% die Preambeln an mintmod angehaengt, die Ersetzung wird von mconvert.pl vorgenommen.

\newcommand{\MPreambleInclude}[1]{\input{#1}}

% Globale Watermarksettings (werden auch nochmal zu Beginn jedes subsection gesetzt,
% muessen hier aber auch global ausgefuehrt wegen Einfuehrungsseiten und Inhaltsverzeichnis

\MWatermarkSettings
% ---------------------------------- Parametrisierte Aufgaben ----------------------------------------

\ifttm
\newenvironment{MPExercise}{%
\begin{MExercise}%
}{%
\special{html:<button name="Name_MPEX}\arabic{MExerciseCounter}\special{html:" id="MPEX}\arabic{MExerciseCounter}%
\special{html:" type="button" onclick="reroll('}\arabic{MExerciseCounter}\special{html:');">Neue Aufgabe erzeugen</button>}%
\end{MExercise}%
}
\else
\newenvironment{MPExercise}{%
\begin{MExercise}%
}{%
\end{MExercise}%
}
\fi

% Parameter: Name, Min, Max, PDF-Standard. Name in Deklaration OHNE backslash, im Code MIT Backslash
\ifttm
\newcommand{\MGlobalInteger}[4]{\special{html:%
<!-- onloadstart //-->%
MVAR.push(createGlobalInteger("}#1\special{html:",}#2\special{html:,}#3\special{html:,}#4\special{html:)); %
<!-- onloadstop //-->%
<!-- viewmodelstart //-->%
ob}#1\special{html:: ko.observable(rerollMVar("}#1\special{html:")),%
<!-- viewmodelstop //-->%
}%
}%
\else%
\newcommand{\MGlobalInteger}[4]{\newcounter{mvc_#1}\setcounter{mvc_#1}{#4}}
\fi

% Parameter: Name, Min, Max, PDF-Standard. Name in Deklaration OHNE backslash, im Code MIT Backslash, Wert ist Wurzel von value
\ifttm
\newcommand{\MGlobalSqrt}[4]{\special{html:%
<!-- onloadstart //-->%
MVAR.push(createGlobalSqrt("}#1\special{html:",}#2\special{html:,}#3\special{html:,}#4\special{html:)); %
<!-- onloadstop //-->%
<!-- viewmodelstart //-->%
ob}#1\special{html:: ko.observable(rerollMVar("}#1\special{html:")),%
<!-- viewmodelstop //-->%
}%
}%
\else%
\newcommand{\MGlobalSqrt}[4]{\newcounter{mvc_#1}\setcounter{mvc_#1}{#4}}% Funktioniert nicht als Wurzel !!!
\fi

% Parameter: Name, Min, Max, PDF-Standard zaehler, PDF-Standard nenner. Name in Deklaration OHNE backslash, im Code MIT Backslash
\ifttm
\newcommand{\MGlobalFraction}[5]{\special{html:%
<!-- onloadstart //-->%
MVAR.push(createGlobalFraction("}#1\special{html:",}#2\special{html:,}#3\special{html:,}#4\special{html:,}#5\special{html:)); %
<!-- onloadstop //-->%
<!-- viewmodelstart //-->%
ob}#1\special{html:: ko.observable(rerollMVar("}#1\special{html:")),%
<!-- viewmodelstop //-->%
}%
}%
\else%
\newcommand{\MGlobalFraction}[5]{\newcounter{mvc_#1}\setcounter{mvc_#1}{#4}} % Funktioniert nicht als Bruch !!!
\fi

% MVar darf im HTML nur in MEvalMathDisplay-Umgebungen genutzt werden oder in Strings die an den Parser uebergeben werden
\ifttm%
\newcommand{\MVar}[1]{\special{html:[var_}#1\special{html:]}}%
\else%
\newcommand{\MVar}[1]{\arabic{mvc_#1}}%
\fi

\ifttm%
\newcommand{\MRerollButton}[2]{\special{html:<button type="button" onclick="rerollMVar('}#1\special{html:');">}#2\special{html:</button>}}%
\else%
\newcommand{\MRerollButton}[2]{\relax}% Keine sinnvolle Entsprechung im PDF
\fi

% MEvalMathDisplay fuer HTML wird in mconvert.pl im preprocessing realisiert
% PDF: eine equation*-Umgebung (ueber amsmath)
% HTML: Eine Mathjax-Tex-Umgebung, deren Auswertung mit knockout-obervablen gekoppelt ist
% PDF-Version hier nur fuer pdflatex-only-Uebersetzung gegeben

\ifttm\else\newenvironment{MEvalMathDisplay}{\begin{equation*}}{\end{equation*}}\fi

% ---------------------------------- Spezialbefehle fuer AD ------------------------------------------

%Abk�rzung f�r \longrightarrow:
\newcommand{\lto}{\ensuremath{\longrightarrow}}

%Makro f�r Funktionen:
\newcommand{\exfunction}[5]
{\begin{array}{rrcl}
 #1 \colon  & #2 &\lto & #3 \\[.05cm]  
  & #4 &\longmapsto  & #5 
\end{array}}

\newcommand{\function}[5]{%
#1:\;\left\lbrace{\begin{array}{rcl}
 #2 &\lto & #3 \\
 #4 &\longmapsto  & #5 \end{array}}\right.}


%Die Identit�t:
\DeclareMathOperator{\Id}{Id}

%Die Signumfunktion:
\DeclareMathOperator{\sgn}{sgn}

%Zwei Betonungskommandos (k�nnen angepasst werden):
\newcommand{\highlight}[1]{#1}
\newcommand{\modstextbf}[1]{#1}
\newcommand{\modsemph}[1]{#1}


% ---------------------------------- Spezialbefehle fuer JL ------------------------------------------


\def\jccolorfkt{green!50!black} %Farbe des Funktionsgraphen
\def\jccolorfktarea{green!25!white} %Farbe der Fl"ache unter dem Graphen
\def\jccolorfktareahell{green!12!white} %helle Einf"arbung der Fl"ache unter dem Graphen
\def\jccolorfktwert{green!50!black} %Farbe einzelner Punkte des Graphen

\newcommand{\MPfadBilder}{Bilder}

\ifttm%
\newcommand{\jMD}{\,\MD}%
\else%
\newcommand{\jMD}{\;\MD}%
\fi%

\def\jHTMLHinweisBedienung{\MInputHint{%
Mit Hilfe der Symbole am oberen Rand des Fensters
k"onnen Sie durch die einzelnen Abschnitte navigieren.}}

\def\jHTMLHinweisEingabeText{\MInputHint{%
Geben Sie jeweils ein Wort oder Zeichen als Antwort ein.}}

\def\jHTMLHinweisEingabeTerm{\MInputHint{%
Klammern Sie Ihre Terme, um eine eindeutige Eingabe zu erhalten. 
Beispiel: Der Term $\frac{3x+1}{x-2}$ soll in der Form
\texttt{(3*x+1)/((x+2)^2}$ eingegeben werden (wobei auch Leerzeichen 
eingegeben werden k"onnen, damit eine Formel besser lesbar ist).}}

\def\jHTMLHinweisEingabeIntervalle{\MInputHint{%
Intervalle werden links mit einer "offnenden Klammer und rechts mit einer 
schlie"senden Klammer angegeben. Eine runde Klammer wird verwendet, wenn der 
Rand nicht dazu geh"ort, eine eckige, wenn er dazu geh"ort. 
Als Trennzeichen wird ein Komma oder ein Semikolon akzeptiert.
Beispiele: $(a, b)$ offenes Intervall,
$[a; b)$ links abgeschlossenes, rechts offenes Intervall von $a$ bis $b$. 
Die Eingabe $]a;b[$ f"ur ein offenes Intervall wird nicht akzeptiert.
F"ur $\infty$ kann \texttt{infty} oder \texttt{unendlich} geschrieben werden.}}

\def\jHTMLHinweisEingabeFunktionen{\MInputHint{%
Schreiben Sie Malpunkte (geschrieben als \texttt{*}) aus und setzen Sie Klammern um Argumente f�r Funktionen.
Beispiele: Polynom: \texttt{3*x + 0.1}, Sinusfunktion: \texttt{sin(x)}, 
Verkettung von cos und Wurzel: \texttt{cos(sqrt(3*x))}.}}

\def\jHTMLHinweisEingabeFunktionenSinCos{\MInputHint{%
Die Sinusfunktion $\sin x$ wird in der Form \texttt{sin(x)} angegeben, %
$\cos\left(\sqrt{3 x}\right)$ durch \texttt{cos(sqrt(3*x))}.}}

\def\jHTMLHinweisEingabeFunktionenExp{\MInputHint{%
Die Exponentialfunktion $\MEU^{3x^4 + 5}$ wird als
\texttt{exp(3 * x^4 + 5)} angegeben, %
$\ln\left(\sqrt{x} + 3.2\right)$ durch \texttt{ln(sqrt(x) + 3.2)}.}}

% ---------------------------------- Spezialbefehle fuer Fachbereich Physik --------------------------

\newcommand{\E}{{e}}
\newcommand{\ME}[1]{\cdot 10^{#1}}
\newcommand{\MU}[1]{\;\mathrm{#1}}
\newcommand{\MPG}[3]{%
  \ifnum#2=0%
    #1\ \mathrm{#3}%
  \else%
    #1\cdot 10^{#2}\ \mathrm{#3}%
  \fi}%
%

\newcommand{\MMul}{\MExponentensymbXYZl} % Nur eine Abkuerzung


% ---------------------------------- Stichwortfunktionialitaet ---------------------------------------

% mpreindexentry wird durch Auswahlroutine in conv.pl durch mindexentry substitutiert
\ifttm%
\def\MIndex#1{\index{#1}\special{html:<!-- mpreindexentry;;}#1\special{html:;;}\arabic{MSubjectArea}\special{html:;;}%
\arabic{chapter}\special{html:;;}\arabic{section}\special{html:;;}\arabic{subsection}\special{html:;;}\arabic{MEntryCounter}\special{html:; //-->}%
\setcounter{MLastIndex}{\value{MEntryCounter}}%
\addtocounter{MEntryCounter}{1}%
}%
% Copyrightliste wird als tex-Datei im preprocessing von conv.pl erzeugt und unter converter/tex/entrycollection.tex abgelegt
% Der input-Befehl funktioniert nur, wenn die aufrufende tex-Datei auf der obersten Ebene liegt (d.h. selbst kein input/include ist, insbesondere keine Moduldatei)
\def\MEntryList{} % \input funktioniert nicht, weil ttm (und damit das \input) ausgefuehrt wird, bevor Datei da ist
\else%
\def\MIndex#1{\index{#1}}
\def\MEntryList{\MAbort{Stichwortliste nur im HTML realisierbar}}%
\fi%

\def\MEntry#1#2{\textbf{#1}\MIndex{#2}} % Idee: MLastType auf neuen Entry-Typ und dann ein MLabel vergeben mit autogen-Nummer

% ---------------------------------- Befehle fuer Tests ----------------------------------------------

% MEquationItem stellt eine Eingabezeile der Form Vorgabe = Antwortfeld her, der zweite Parameter kann z.B. MSimplifyQuestion-Befehl sein
\ifttm
\newcommand{\MEquationItem}[2]{{#1}$\,=\,${#2}}%
\else%
\newcommand{\MEquationItem}[2]{{#1}$\;\;=\,${#2}}%
\fi

\ifttm
\newcommand{\MInputHint}[1]{%
\ifnum%
\if\value{MTestSite}>0%
\else%
{\color{blue}#1}%
\fi%
\fi%
}
\else
\newcommand{\MInputHint}[1]{\relax}
\fi

\ifttm
\newcommand{\MInTestHeader}{%
Dies ist ein einreichbarer Test:
\begin{itemize}
\item{Im Gegensatz zu den offenen Aufgaben werden beim Eingeben keine Hinweise zur Formulierung der mathematischen Ausdr�cke gegeben.}
\item{Der Test kann jederzeit neu gestartet oder verlassen werden.}
\item{Der Test kann durch die Buttons am Ende der Seite beendet und abgeschickt, oder zur�ckgesetzt werden.}
\item{Der Test kann mehrfach probiert werden. F�r die Statistik z�hlt die zuletzt abgeschickte Version.}
\end{itemize}
}
\else
\newcommand{\MInTestHeader}{%
\relax
}
\fi

\ifttm
\newcommand{\MInTestFooter}{%
\special{html:<button name="Name_TESTFINISH" id="TESTFINISH" type="button" onclick="finish_button('}\MTestName\special{html:');">Test auswerten</button>}%
\begin{html}
&nbsp;&nbsp;&nbsp;&nbsp;&nbsp;&nbsp;&nbsp;&nbsp;
<button name="Name_TESTRESET" id="TESTRESET" type="button" onclick="reset_button();">Test zur�cksetzen</button>
<br />
<br />
<div class="xreply">
<p name="Name_TESTEVAL" id="TESTEVAL">
Hier erscheint die Testauswertung!
<br />
</p>
</div>
\end{html}
}
\else
\newcommand{\MInTestFooter}{%
\relax
}
\fi


% ---------------------------------- Notationsmakros -------------------------------------------------------------

% Notationsmakros die nicht von der Kursvariante abhaengig sind

\newcommand{\MZahltrennzeichen}[1]{\renewcommand{\MZXYZhltrennzeichen}{#1}}

\ifttm
\newcommand{\MZahl}[3][\MZXYZhltrennzeichen]{\edef\MZXYZtemp{\noexpand\special{html:<mn>#2#1#3</mn>}}\MZXYZtemp}
\else
\newcommand{\MZahl}[3][\MZXYZhltrennzeichen]{{}#2{#1}#3}
\fi

\newcommand{\MEinheitenabstand}[1]{\renewcommand{\MEinheitenabstXYZnd}{#1}}
\ifttm
\newcommand{\MEinheit}[2][\MEinheitenabstXYZnd]{{}#1\edef\MEINHtemp{\noexpand\special{html:<mi mathvariant="normal">#2</mi>}}\MEINHtemp} 
\else
\newcommand{\MEinheit}[2][\MEinheitenabstXYZnd]{{}#1 \mathrm{#2}} 
\fi

\newcommand{\MExponentensymbol}[1]{\renewcommand{\MExponentensymbXYZl}{#1}}
\newcommand{\MExponent}[2][\MExponentensymbXYZl]{{}#1{} 10^{#2}} 

%Punkte in 2 und 3 Dimensionen
\newcommand{\MPointTwo}[3][]{#1(#2\MCoordPointSep #3{}#1)}
\newcommand{\MPointThree}[4][]{#1(#2\MCoordPointSep #3\MCoordPointSep #4{}#1)}
\newcommand{\MPointTwoAS}[2]{\left(#1\MCoordPointSep #2\right)}
\newcommand{\MPointThreeAS}[3]{\left(#1\MCoordPointSep #2\MCoordPointSep #3\right)}

% Masseinheit, Standardabstand: \,
\newcommand{\MEinheitenabstXYZnd}{\MThinspace} 

% Horizontaler Leerraum zwischen herausgestellter Formel und Interpunktion
\ifttm
\newcommand{\MDFPSpace}{\,}
\newcommand{\MDFPaSpace}{\,\,}
\newcommand{\MBlank}{\ }
\else
\newcommand{\MDFPSpace}{\;}
\newcommand{\MDFPaSpace}{\;\;}
\newcommand{\MBlank}{\ }
\fi

% Satzende in herausgestellter Formel mit horizontalem Leerraum
\newcommand{\MDFPeriod}{\MDFPSpace .}

% Separation von Aufzaehlung und Bedingung in Menge
\newcommand{\MCondSetSep}{\,:\,} %oder '\mid'

% Konverter kennt mathopen nicht
\ifttm
\def\mathopen#1{}
\fi

% -----------------------------------START Rouletteaufgaben ------------------------------------------------------------

\ifttm
% #1 = Dateiname, #2 = eindeutige ID fuer das Roulette im Kurs
\newcommand{\MDirectRouletteExercises}[2]{
\begin{MExercise}
\texttt{Im HTML erscheinen hier Aufgaben aus einer Aufgabenliste...}
\end{MExercise}
}
\else
\newcommand{\MDirectRouletteExercises}[2]{\relax} % wird durch mconvert.pl gefunden und ersetzt
\fi


% ---------------------------------- START Makros, die von der Kursvariante abhaengen ----------------------------------

\ifvariantunotation
  % unotation = An Universitaeten uebliche Notation
  \def\MVariant{unotation}

  % Trennzeichen fuer Dezimalzahlen
  \newcommand{\MZXYZhltrennzeichen}{.}

  % Exponent zur Basis 10 in der Exponentialschreibweise, 
  % Standardmalzeichen: \times
  \newcommand{\MExponentensymbXYZl}{\times} 

  % Begrenzungszeichen fuer offene Intervalle
  \newcommand{\MoIl}[1][]{\mbox{}#1(\mathopen{}} % bzw. ']'
  \newcommand{\MoIr}[1][]{#1)\mbox{}} % bzw. '['

  % Zahlen-Separation im IntervaLL
  \newcommand{\MIntvlSep}{,} %oder ';'

  % Separation von Elementen in Mengen
  \newcommand{\MElSetSep}{,} %oder ';'

  % Separation von Koordinaten in Punkten
  \newcommand{\MCoordPointSep}{,} %oder ';' oder '|', '\MThinspace|\MThinspace'

\else
  % An dieser Stelle wird angenommen, dass std-Variante aktiv ist
  % std = beschlossene Notation im TU9-Onlinekurs 
  \def\MVariant{std}

  % Trennzeichen fuer Dezimalzahlen
  \newcommand{\MZXYZhltrennzeichen}{,}

  % Exponent zur Basis 10 in der Exponentialschreibweise, 
  % Standardmalzeichen: \times
  \newcommand{\MExponentensymbXYZl}{\times} 

  % Begrenzungszeichen fuer offene Intervalle
  \newcommand{\MoIl}[1][]{\mbox{}#1]\mathopen{}} % bzw. '('
  \newcommand{\MoIr}[1][]{#1[\mbox{}} % bzw. ')'

  % Zahlen-Separation im IntervaLL
  \newcommand{\MIntvlSep}{;} %oder ','
  
  % Separation von Elementen in Mengen
  \newcommand{\MElSetSep}{;} %oder ','

  % Separation von Koordinaten in Punkten
  \newcommand{\MCoordPointSep}{;} %oder '|', '\MThinspace|\MThinspace'

\fi



% ---------------------------------- ENDE Makros, die von der Kursvariante abhaengen ----------------------------------


% diese Kommandos setzen Mathemodus vorraus
\newcommand{\MGeoAbstand}[2]{[\overline{{#1}{#2}}]}
\newcommand{\MGeoGerade}[2]{{#1}{#2}}
\newcommand{\MGeoStrecke}[2]{\overline{{#1}{#2}}}
\newcommand{\MGeoDreieck}[3]{{#1}{#2}{#3}}

%
\ifttm
\newcommand{\MOhm}{\special{html:<mn>&#x3A9;</mn>}}
\else
\newcommand{\MOhm}{\Omega} %\varOmega
\fi


\def\PERCTAG{\MAbort{PERCTAG ist zur internen verwendung in mconvert.pl reserviert, dieses Makro darf sonst nicht benutzt werden.}}

% Im Gegensatz zu einfachen html-Umgebungen werden MDirectHTML-Umgebungen von mconvert.pl am ganzen ttm-Prozess vorbeigeschleust und aus dem PDF komplett ausgeschnitten
\ifttm%
\newenvironment{MDirectHTML}{\begin{html}}{\end{html}}%
\else%
\newenvironment{MDirectHTML}{\begin{html}}{\end{html}}%
\fi

% Im Gegensatz zu einfachen Mathe-Umgebungen werden MDirectMath-Umgebungen von mconvert.pl am ganzen ttm-Prozess vorbeigeschleust, ueber MathJax realisiert, und im PDF als $$ ... $$ gesetzt
\ifttm%
\newenvironment{MDirectMath}{\begin{html}}{\end{html}}%
\else%
\newenvironment{MDirectMath}{\begin{equation*}}{\end{equation*}}% Vorsicht, auch \[ und \] werden in amsmath durch equation* redefiniert
\fi

% ---------------------------------- Location Management ---------------------------------------------

% #1 = buttonname (muss in files/images liegen und Format 48x48 haben), #2 = Vollstaendiger Einrichtungsname, #3 = Kuerzel der Einrichtung,  #4 = Name der include-texdatei
\ifttm
\newcommand{\MLocationSite}[3]{\special{html:<!-- mlocation;;}#1\special{html:;;}#2\special{html:;;}#3\special{html:;; //-->}}
\else
\newcommand{\MLocationSite}[3]{\relax}
\fi

% ---------------------------------- Copyright Management --------------------------------------------

\newcommand{\MCCLicense}{%
{\color{green}\textbf{CC BY-SA 3.0}}
}

\newcommand{\MCopyrightLabel}[1]{ (\MSRef{L_COPYRIGHTCOLLECTION}{Lizenz})\MLabel{#1}}

% Copyrightliste wird als tex-Datei im preprocessing erzeugt und unter converter/tex/copyrightcollection.tex abgelegt
% Der input-Befehl funktioniert nur, wenn die aufrufende tex-Datei auf der obersten Ebene liegt (d.h. selbst kein input/include ist, insbesondere keine Moduldatei)
\newcommand{\MCopyrightCollection}{\input{copyrightcollection.tex}}

% MCopyrightNotice fuegt eine Copyrightnotiz ein, der parser ersetzt diese durch CopyrightNoticePOST im preparsing, diese Definition wird nur fuer reine pdflatex-Uebersetzungen gebraucht
% Parameter: #1: Kurze Lizenzbeschreibung (typischerweise \MCCLicense)
%            #2: Link zum Original (http://...) oder NONE falls das Bild selbst ein Original ist, oder TIKZ falls das Bild aus einer tikz-Umgebung stammt
%            #3: Link zum Autor (http://...) oder MINT falls Original im MINT-Kolleg erstellt oder NONE falls Autor unbekannt
%            #4: Bemerkung (z.B. dass Datei mit Maple exportiert wurde)
%            #5: Labelstring fuer existierendes Label auf das copyrighted Objekt, mit MCopyrightLabel erzeugt
%            Keines der Felder darf leer sein!
\newcommand{\MCopyrightNotice}[5]{\MCopyrightNoticePOST{#1}{#2}{#3}{#4}{#5}}

\ifttm%
\newcommand{\MCopyrightNoticePOST}[5]{\relax}%
\else%
\newcommand{\MCopyrightNoticePOST}[5]{\relax}%
\fi%

% ---------------------------------- Meldungen fuer den Benutzer des Konverters ----------------------
\MPragma{mintmodversion;P0.1.0}
\MPragma{usercomment;This is file mintmod.tex version P0.1.0}


% ----------------------------------- Spezialelemente fuer Konfigurationsseite, werden nicht von mintscripts.js verwaltet --

% #1 = DOM-id der Box
\ifttm\newcommand{\MConfigbox}[1]{\special{html:<input cfieldtype="2" type="checkbox" name="Name_}#1\special{html:" id="}#1\special{html:" onchange="confHandlerChange('}#1\special{html:');"/>}}\fi % darf im PDF nicht aufgerufen werden!


\MPragma{MathSkip}

%\Mtikzexternalize

\begin{document}

\MSection{Equations in one Variable}
\MLabel{VBKM02}
\MSetSectionID{VBKM02} % wird fuer tikz-Dateien verwendet

\begin{MSectionStart}
\MDeclareSiteUXID{VBKM02_START}

Equations arise by equating two terms in which variables occur. Simple equations can be solved by 
applying transformations and solution formulas. For more sophisticated equations case analyses are 
required. This module consists of

\begin{itemize}
\item{Section \MNRef{M02_EinfacheGleichungen}: \MSRef{M02_EinfacheGleichungen}{Simple Equations},}
%\item{dem Abschnitt \MNRef{M02_Wurzelgleichungen}, \MSRef{M02_Wurzelgleichungen}{Gleichungen mit Wurzeln},}
\item{Section \MNRef{M02_Betragsgleichungen}: \MSRef{M02_Betragsgleichungen}{Absolute Value Equations},}
\item{and Section \MNRef{M02_Abschlusstest}: \MSRef{M02_Abschlusstest}{Final Test}.}
\end{itemize}
\end{MSectionStart}

\MSubsection{Simple Equations}
\MLabel{M02_EinfacheGleichungen}

\begin{MIntro}
\MDeclareSiteUXID{VBKM02_EinfacheGleichungenIntro}
\begin{MInfo}
An \MEntry{equation}{equation} is an expression of the form
$$
\text{left-hand side} \;=\; \text{right-hand side} \MDFPSpace
$$
with mathematical expressions on both sides of the equation. These expressions generally contain variables or unknowns (e.q.\ $x$). 
Depending on the variable values an equation is satisfied if both sides of the equation evaluate to the same value. An equation is not 
satisfied if the sides of the equation evaluate to different values.
\end{MInfo}

Equations describe relations between expressions or model a problem to be solved. An equation itself cannot
be true or false but some variables satisfy the equation and others do not. To test whether the equation is 
true or false for a single variable value this value has to be inserted into the equation. Then, both sides of the 
equation are evaluated to certain values. The equation is satisfied by an inserted variable value if the evaluated 
values coincide:

\begin{MExample}
The equation $2x-1=x^2$ has the right-hand side $x^2$ and the left-hand side $2x-1$. Inserting $x=1$ 
results in the value $1$ on both sides of the equals sign, hence $x=1$ is a solution of this equation. 
However, $x=2$ is no solution since the left-hand side of the equation is evaluated to the value $4$ while the 
right-hand side is evaluated to the value $3$.
\end{MExample}

\begin{MInfo}
The \MEntry{solution set}{solution set} $\ML$ of an equation is the set of all numbers satisfying the relation 
$$
\text{left-hand side} \;=\; \text{right-hand side} \MDFPSpace
$$
if inserted into the the equation instead of the variable (e.q. $x$).
\end{MInfo}

Typical problems concerning equations are:
\begin{itemize}
 \item{specify the solution set of an equation, i.e.\ find all variable values satisfying the equation,}
 \item{transform the equation, in particular, solve an equation for the variables, and}
 \item{find an equation modelling a problem described textually.}
\end{itemize}


\begin{MExample}
  A savings deposit is to be designed such that it gives a fixed return per year. The bank intends to achieve
  that the saver returns for a five years deposit exactly 600~Euro more than for a deposit of only two years. 

  First, the textual problem is translated into an equation with the variable $r$ denoting the return per year. Then, 
  the equation is $5\cdot r=2\cdot r+600$. It says that five payments (left-hand side of the equation) equal 
  two payments plus 600 (the unit Euro is then omitted during calculation).

  This equation can be solved for $r$ very easily by subtracting the term $2r$ to both sides of the equation. Then, the 
  equation reads $3r=600$ and dividing by $3$ results in the solution $r=200$.

  Thus, the bank has to offer a return of 200~Euro per year to reach the required savings target. 
\end{MExample}

% \begin{MExercise}
% �bersetzen Sie die folgenden Aussagen in eine Gleichung (ohne sie zu l�sen):
% \begin{MExerciseItems}
% \item{In elf Jahren ist Max doppelt so alt wie jetzt ($t$ = jetziges Alter in Jahren):  ???}
% \item{Ein Kunde hat eine Rechnung �ber 1000 Euro zu bezahlen. Die monatlich zu zahlende Rate bleibt f�r ihn gleich unabh�ngig davon ob er
% \begin{itemize}
%  \item{die Rechnung mit $m$ gleichgro�en Raten abbezahlt,}
%  \item{vier Raten mehr als $m$ einteilt und zus�tzlich 10 Euro Verzugsgeb�hren pro Monat an die Bank zahlt (die nicht zur Rechnungsbegleichung dienen).}
% \end{itemize}
% Ist $m$ die Anzahl der Monate, so gilt ???
% }
% \end{MExerciseItems}
% \end{MExercise}


\begin{MInfo}
Two equations are said to be \MEntry{equivalent}{equivalence} if they have the same solution set.

An \MEntry{equivalent transformation}{equivalent transformation} is a special transformation that changes
the equation but not its solution set. Important equivalent transformations are

\begin{itemize}
 \item{adding/subtracting terms to both sides of the equation,}
 \item{exchanging both sides of the equation,}
 \item{transformation of terms on one side of the equation, and}
 \item{inserting known relations.}
 \end{itemize}

The following transformations are considered as equivalent transformations only if the used term is non-zero 
(which can depend on the variables):

\begin{itemize}
 \item{multiplying/dividing by a term (this term has to be non-zero),}
 \item{taking the reciprocal on both sides of the equation (both sides have to be non-zero).}
\end{itemize}
\end{MInfo}

Here, the following \MEntry{notation}{equivalent transformation (notation)} is used:

\begin{itemize}
 \item{equivalent equations are indicated by the symbol $\Leftrightarrow$ (which reads: if and only if, i.e. one
  equation is satisfied if and only if the other equation is satisfied).}

\item{the symbol is underset by the transforming operation (or, for solutions with more than one line, the 
  transforming operation is written next to the transformation).}
\end{itemize}

Importantly, the reader should be able to understand which transformation was carried out.

\begin{MExample}
This example illustrates two simple single-lined equivalent transformations. Even though the symbol $\Leftrightarrow$ 
is two-sided the notation is interpreted in such a way that the transformation is applied from 
left to right:
$$
3x-x^2 \;=\; 2x-x^2+1\;\;\MUnderset{+x^2}\Leftrightarrow\;\;  3x\;=\; 2x+1 \;\;\MUnderset{-2x}\Leftrightarrow\;\; x \;=\; 1 \MDFPeriod
$$
The left equation and the right equation are equivalent. On the left we have the initial equation
(corresponding to a certain textual problem) and on the right we have an equivalent equation 
showing the solution immediately.
\end{MExample}

\begin{MExample}
\MLabel{BSP_Umformungen1}
For several complicated transformations the transformation steps should be written below each other.
In this case the respective transformations are separated from the equation by vertical bars. 

\begin{eqnarray*}
& \text{Start:} & 12+t \;=\;\Mdfrac{2t}{2t^2}+t \ \ \ \ \MSep \ -t\ \\ \ \\
& \Leftrightarrow & 12 \;=\; \Mdfrac{2t}{2t^2}  \ \ \ \ \MSep \ \text{sides exchanged}\ \\ \ \\
& \Leftrightarrow & \Mdfrac{2t}{2t^2} \;=\; 12  \ \ \ \ \MSep \ \text{left-hand side transformed}\ \\ \ \\
& \Leftrightarrow & \Mdfrac1t \;=\; 12  \ \ \ \ \MSep \ \text{reciprocals taken}\ \\ \ \\
& \Leftrightarrow & t \;=\; \Mdfrac{1}{12} \MDFPeriod
\end{eqnarray*}

Here, after the vertical bar both short symbols as, e.g. $-t$, and textual descriptions are allowed. 
Importantly, the reader should be able to understand which transformations were carried out and should be able to decide whether 
they are correct.

\end{MExample}
\end{MIntro}

\begin{MXContent}{Conditions in Transformations}{Conditions}{STD}
\MDeclareSiteUXID{VBKM02_Bedingungen}
Multiplication, division, and taking reciprocals are only equivalent transformations if the factors or terms are 
non-zero. In example~\MRef{BSP_Umformungen1}, the reader understands that both sides of the equation are non-zero 
such that the transformation is allowed. If the variables themselves are used in the transformation it has to be 
noted that the respective term has to be non-zero. The solution at the end of the transformations is then 
only valid for variable values satisfying the transformation conditions. All other values have to be checked 
\textit{separately}, typically by inserting the value into the equation:

\begin{MExample}
In this example, the necessary transformation conditions are not problematic:
\begin{eqnarray*}
& \text{Start:} & 9x \;=\;81x^2  \ \ \ \ \MSep \ :x\text{, transformation allowed if }x\not=0\ \\ \ \\
& \Leftrightarrow & 9 \;=\; 81x  \ \ \ \ \MSep \ :81\text{ and exchange sides}\ \\ \ \\
& \Leftrightarrow & x \;=\; \Mdfrac19 \ \ \ \ \text{and this value satisfies the condition }x\not=0 \MDFPeriod
\end{eqnarray*}
The value $x=0$, initially rejected by the transformation condition, has to be checked separately. The equation $9x=81x^2$ is 
also satisfied for $x=0$, hence $x=0$ is also a solution of the equation. In set notation, the 
solution set is $\ML=\lbrace 0\MElSetSep \Mtfrac19\rbrace$.
\end{MExample}

Anyway, values violating a condition have to be checked separately, in particular, they can be finally 
part of the solution.

\begin{MExample}
\begin{eqnarray*}
& \text{Start:} & x^2-2x \;=\; 2x-4 \ \ \ \ \MSep \ \text{factor out on both sides} \ \\ \ \\
& \Leftrightarrow & x\cdot (x-2) \;=\; 2\cdot (x-2) \ \ \ \ | \ :(x-2)\text{, transformation only allowed if }x\not=2\ \\ \ \\
& \Leftrightarrow & x\;=\; 2  \MDFPeriod
\end{eqnarray*}
This value of $x$ violates the condition $x\not=2$. Hence, this is possibly no solution. Inserting $x=2$
into the initial equation gives $2^2-2\cdot 2=0$ on the left-hand side and also $2\cdot 2-4=0$ on the 
right-hand side. Hence, $x=2$ is indeed a solution, even though it violated the transformation condition.
\end{MExample}

\begin{MExercise}
Find the solution of the equation $(x-2)(x-3)=x^2-9$ by transforming the right-hand side using the third 
binomial formula and then dividing by a common factor. 

The solution is $x$ = \MLParsedQuestion{5}{3}{5}{EASY1}.

\begin{MHint}{Solution}
The correct transformation steps including conditions are
\begin{eqnarray*}
& \text{Start:} & (x-2)(x-3)\;=\; x^2-9 \ \ \ \ \MSep \ \text{transformation of the right-hand side} \ \\ \ \\
& \Leftrightarrow & (x-2)(x-3) \;=\; (x+3)(x-3) \ \ \ \ \MSep \ :(x-3)\text{, transformation allowed if }x\not=3\ \\ \ \\
& \Leftrightarrow & x-2\;=\; x+3 \ \ \ \ \MSep \ -x \ \\ \ \\
& \Leftrightarrow & -2 \;=\; 3 \; \text{is a wrong equation.}
\end{eqnarray*}
Importantly, this equation is only wrong for $x\not=3$. We have to check $x=3$ separately, and indeed 
$x=3$ satisfies the initial equation.
\end{MHint}
\end{MExercise}
\end{MXContent}

\begin{MXContent}{Proportionality and Rule of Three}{Proportionality}{STD}
\MLabel{VBKM02_Dreisatz}
\MDeclareSiteUXID{VBKM02_Dreisatz}
In practise, a relation between two quantities that occurs frequently is the 
\MEntry{proportionality}{proportionality} between two quantities, e.g. between 
mass and volume, time and travelled distance or weight (quantity) of a product and its price. 
The relation can be exemplary for certain fixed quantities. Then, a first aim is to formulate the 
resulting relation for another application example. 
The procedure shall be illustrated by an example. 


\begin{MExample}
$5\MEinheit{kg}$ of apples cost $3$ Euro. How much do $11\MEinheit{kg}$
of apples cost?
\par
The initial relation can be formulated as follows:
$$
5\MEinheit{kg} \MDFPSpace \MRelates \MDFPSpace 3\MBlank \text{Euro}
\MDFPeriod
$$
It is assumed that these quantities are proportional to each other. In the next
step the relation between the quantities is reduced to a unit of one of the quantities, 
namely to the unit of the given quantity. Here, both quantities 
are multiplied by $1/5$ -- i.e. the unit is $1\MEinheit{kg}$ --:
$$
1\MEinheit{kg} \MDFPSpace \MRelates \MDFPSpace 
\frac{1}{5}\cdot 3\MBlank \text{Euro} = \MZahl{0}{6}\MBlank \text{Euro}
\MDFPeriod
$$
Finally, both sides of the equation are multiplied by the multiple of the 
respective unit of the specified quantity, in this case by the factor $11$: 
$$
11\MEinheit{kg} \MDFPSpace \MRelates \MDFPSpace 
11\cdot\MZahl{0}{6}\MBlank \text{Euro} = \MZahl{6}{6}\MBlank \text{Euro}
\MDFPeriod
$$
The required price for $11\MEinheit{kg}$ of apples is therefore
$\MZahl{6}{60}\MBlank \text{Euro}$.
\end{MExample}

We have derived the required relation by deriving a relation for one unit of a quantity from 
the initial relation. This procedure demonstrated here as an example is called 
\MEntry{rule of three}{rule of three}.

The posed problem can also be solved by introducing a proportionality factor. 
Again, we consider the example above.

\begin{MExample}
The price $P$ is proportional to the mass $m$. Hence, it exists a constant 
$k$ with
$$
P=k m \MDFPeriod
$$
Since this relation also holds for the given values $m_0=5\MEinheit{kg}$
and $P_0=3\MBlank\text{Euro}$ it follows 
\begin{eqnarray*}
  P_0=k m_0 & \MTSP\MSep\MTSP & \text{multiplying by}\MBlank \frac{1}{m_0} \\
  \Longleftrightarrow\MDFPSpace\frac{P_0}{m_0}=k &;& 
\end{eqnarray*}
hence in this case
$$
k=\frac{3}{5} = \MZahl{0}{6} \MDFPSpace,
$$
taken in the unit of Euro per kg. (As a scientist you would correctly write 
 $k=\MZahl{0}{6} \MBlank \text{Euro}/\MEinheit[]{kg}$, since proportionality factors 
generally carry a dimensional unit.) Using $m_1=11\MEinheit{kg}$, one obtains 
finally
$$
P_1=k m_1 = \MZahl{0}{6}\cdot 11 =\MZahl{6}{6} \MBlank \text{(Euro)}
\MDFPSpace
$$
which is the same result as for using the rule of three (see previous example).
\end{MExample}

\begin{MExercise}
A car takes $9$~minutes to travel a distance of $6\MEinheit{km}$.
\begin{MExerciseItems}
\item{%
Which distance $s$ the car travels within $15$~minutes?
\medskip\par
The solution is $s_{15}$ = \MLParsedQuestion{5}{10}{5}{EASY2}$\MEinheit{km}$.
}
\item{%
The proportionality factor between travelled distance $s$ and travelling 
time $t$ is the velocity $v$ of the car.
\medskip\par
The velocity is $v$ = \MLParsedQuestion{5}{40}{5}{EASY3}$\MEinheit{km}/\MEinheit{h}$.
}
\end{MExerciseItems}
\begin{MHint}{Solution}
>From the given values we know that the car travels $\frac{6}{9}\MEinheit{km}
=\frac{2}{3}\MEinheit{km}$  within one minute and therefore 
$15\cdot\frac{2}{3}\MEinheit{km}=10\MEinheit{km}$ within $15$~minutes.
\par
So, the velocity is
$$v=\frac{10\MEinheit{km}}{15\MEinheit{min}} = 
\frac{10\MEinheit{km}}{(1/4)\MEinheit{h}} =
40\MEinheit{}\frac{\MEinheit[]{km}}{\MEinheit[]{h}}
\MDFPeriod
$$
\end{MHint}
\end{MExercise}
\end{MXContent}

\begin{MXContent}{Solving linear Equations}{Solving}{STD}
\MLabel{VBKM02_LineareGleichungenLoesen}
\MDeclareSiteUXID{VBKM02_Aufloesen}
\begin{MInfo}
A \MEntry{linear equation}{equation (linear)} is an equation in which only multiples of 
variables and constants occur.

For a linear equation in one variable (here the variable $x$) one of the following 
three statements holds:
\begin{itemize}
 \item{The equation has no solution.}
 \item{The equation has a single solution.}
 \item{Every value of $x$ is a solution of the equation.}
\end{itemize}
\end{MInfo}

These three cases are distinguished by means of the transformation steps:
\begin{itemize}
\item{If the transformation ends up in a statement that is wrong for all $x$ (e.g. $1=0$) 
then the equation is unsolvable.}
\item{If the transformation ends up in a statement that is true for all $x$ (e.g. $1=1$)
then the equation is solvable for all values of $x$.}
\item{Otherwise, the equation can be solved, i.e. it can be transformed into 
the equation $x=\text{value}$ which is the solution.}
\end{itemize}

\begin{MXInfo}{Set notation}
Using the set notation (with the solution set $\ML$) these cases can be expressed as follows:
\begin{itemize}
 \item{$\ML=\lbrace\rbrace$ or $L=\MEmptyset$ if there is no solution,}
 \item{$\ML=\lbrace \text{value}\rbrace$ if there is a single solution,}
 \item{$\ML=\R$ if all real numbers $x$ are a solution.}
\end{itemize}
\end{MXInfo}


\begin{MExample}
The linear equation $3x+2=2x-1$ has one solution. 
This solution is obtained by equivalent transformations:
$$
3x+2 \;=\; 2x-1 \;\;\MUnderset{-2x}\Leftrightarrow\;\; x+2\;=\;-1\;\;\MUnderset{-2}\Leftrightarrow\;\; x\;=\; -3 \MDFPeriod
$$
Hence, $x=-3$ is the only solution.
\end{MExample}

\begin{MExample}
The linear equation $3x+3=9x+9$ has the solution:
$$
3x+3 \;=\; 9x+9 \;\;\MUnderset{:(x+1)}\Leftrightarrow\;\; 3\;=\;9 \MDFPeriod
$$
This statement is wrong. Hence, for all $x\not=-1$ (transformation condition) the equation is wrong.
Inserting $x=-1$ satisfies the equation, and so the only solution is $x=-1$.

Alternatively, the equation could have been transformed as follows:
$$
3x+3 \;=\; 9x+9 \;\;\MUnderset{-3x-9}\Leftrightarrow\;\; -6 \;=\; 6x \;\; \Leftrightarrow\;\; x \;=\; -1 \MDFPeriod
$$
\end{MExample}

\begin{MExercise}
Transform the following linear equations and specify their solution sets:
\MInputHint{Enter simply \texttt{$\lbrace a\rbrace$} for a unit set and $\lbrace\rbrace$ for an empty set.}\ \\
\begin{MExerciseItems}
\item{The equation $x-1=1-x$ has the solution set \MEquationItem{$\ML$}{\MLParsedQuestion{4}{1,1}{4}{LUA1}},}
\item{The equation $4x-2=2x+2$ has the solution set \MEquationItem{$\ML$}{\MLParsedQuestion{4}{2,2}{4}{LUA2}},}
\item{The equation $2x-6=2x-10$ has the solution set \MEquationItem{$\ML$}{\MLParsedQuestion{4}{}{4}{LUA3}}.}
\end{MExerciseItems}

\begin{MHint}{Solution}
The first equation can be transformed into $2x=2$ or $x=1$, respectively, so 
the solution set is $\ML=\lbrace 1\rbrace$. The second equation can be transformed into $2x=4$ and the solution set is 
$\ML=\lbrace 2\rbrace$. The third equation can be transformed into $-6=-10$ which is a false statement, 
hence $\ML=\lbrace\rbrace$.
\end{MHint}
\end{MExercise}

\begin{MExercise}
Find the solution of the general linear equation $a x=b$ with $a$ and $b$ being real numbers.
Specify the values of $a$ and $b$ for which the following three cases occur:
\begin{itemize}
 \item{Every value of $x$ is a solution ($\ML=\R$) if 
\MEquationItem{$a$}{\MLParsedQuestion{4}{0}{4}{ALG1}} and $b=0$.}
 \item{There is no solution ($\ML=\MEmptyset$) if \MEquationItem{$a$}{\MLParsedQuestion{4}{0}{4}{ALG2}} 
and $b\not=$\MLParsedQuestion{4}{0}{4}{ALG3}.}
 \item{Otherwise, there is a single solution, namely 
\MEquationItem{$x$}{\MLSimplifyQuestion{5}{b/a}{10}{a,b}{10}{513}{VBKM02ALTFALL3}}.}
\end{itemize}

\begin{MHint}{Solution}
Every value of $x$ is a solution ($\ML=\R$) if $a=0$ and $b=0$.
There is no solution ($\ML=\MEmptyset$) if $a=0$ and $b\not=0$.
Otherwise, there is only one solution, namely $x=\Mtfrac{b}{a}$.
\end{MHint}

\end{MExercise}


\end{MXContent}

\begin{MXContent}{Solving quadratic Equations}{Quadratic Equations}{STD}
\MLabel{VBKM02_QuadratischeGleichungen}
\MDeclareSiteUXID{VBKM02_Quadratische Gleichungen}
\begin{MInfo}
A \MEntry{quadratic equation}{equation (quadratic)} is an equation of the form
  $a x^2 + b x + c = 0$ with $a\not=0$, or, in reduced form, $x^2+ p x + q=0$. 
This form is obtained by dividing the equation by $a$.

For a quadratic equation in one variable (here the variable $x$) one of the following three statements 
holds:
\begin{itemize}
 \item{The quadratic equation has no solution: $\ML=\lbrace\rbrace$.}
 \item{The quadratic equation has a single solution $\ML=\lbrace x_1\rbrace$.}
 \item{The quadratic equation has two different solutions $\ML=\lbrace x_1\MElSetSep x_2\rbrace$.}
\end{itemize}
\end{MInfo}

The solutions are obtained by applying \MEntry{quadratic solution formulas}{solution formulas}.

\begin{MInfo}
\MLabel{VBKM02_pqFormel}
The \MEntry{$p q$ formula}{pq formula} for solving the equation $x^2+p x + q = 0$ reads
$$
x_{1,2} \;=\; -\Mdfrac{p}{2}\pm \sqrt{\Mdfrac14p^2-q} \MDFPeriod
$$
Here, the equation has
\begin{itemize}
\item{no (real) solution if $\Mtfrac14p^2-q<0$ (taking the square root is not allowed),}
\item{a single solution $x_1=-\Mtfrac{p}{2}$ if $\Mtfrac14p^2=q$ and the square root is zero,}
\item{two different solutions if the square root is a positive number.}
\end{itemize}

The expression $D:=\Mtfrac14p^2-q$ underneath the square root considered above is called 
\textbf{discriminant}.
\end{MInfo}
The solution of a quadratic equation is often described by an alternative formula:
\begin{MInfo}
\MLabel{VBKM02_abcFormel}
For the equation $a x^2+b x + c = 0$ with $a\ne 0$ the
\MEntry{$a b c$ formula}{abc formula} reads
$$
x_{1,2} \;=\; \frac{-b\pm\sqrt{b^2-4 a c}}{2a} \MDFPeriod
$$
Here, the equation has
\begin{itemize}
\item{no (real) solution if $b^2-4 a c<0$ (the square root of a negative number is undefined within the
range of real numbers),}
\item{a single solution $x_1=-\Mtfrac{b}{2a}$ if $b^2=4 a c$ and the square root is zero,}
\item{two different solutions if the square root is a positive number.}
\end{itemize}
Again, the expression $D:=b^2-4 a c$ underneath the square root considered above is called 
\textbf{discriminant}.
\end{MInfo}
Certainly, both formulas result in the same solutions. (For applying the $pq$ formula, the quadratic 
equation is to be divided by the factor $a$ of the quadratic term.)
\medskip\par
This three different cases correspond to three different numbers of intersection points between 
the graph of a parabola opening upwards $f(x)=x^2+p x+ q$ and the $x$ axis (applying the $p q$ formula).

\begin{center}
%%\MUGraphicsSolo{para1b.png}{width=0.2\linewidth}{width:230px}\ \ 
%%\MUGraphicsSolo{para2b.png}{width=0.2\linewidth}{width:230px}\ \ 
%%\MUGraphicsSolo{para3b.png}{width=0.2\linewidth}{width:230px}
\MTikzAuto{%
\begin{tikzpicture}[x=1.0cm, y=1.0cm,scale=0.60] 
\foreach \sx/\fsy in {-7.0cm/0.5,0.0cm/0.0,7.0cm/-0.5} {
\begin{scope}[xshift=\sx,yshift=0]
\draw[black] (-3,0) -- (3,0) (0,-3) -- (0,3);
\foreach \x in {-3, -2, -1, 1, 2, 3}
\draw[shift={(\x,0)},color=black] (0pt,0pt) -- (0pt,-2.0pt) node[below=1.0pt] {\tiny $\x$};
\foreach \x in {-2.5, -1.5, ..., 2.5}
\draw[shift={(\x,0)},color=black] (0pt,0pt) -- (0pt,-1.0pt);
\foreach \y in {-3, -2, -1, 1, 2, 3}
\draw[shift={(0,\y)},color=black] (0pt,0pt) -- (-2.0pt,0pt) node[left=1.0pt] {\tiny $\y$};
\foreach \y in {-2.5, -1.5, ..., 2.5}
\draw[shift={(0,\y)},color=black] (0pt,0pt) -- (-1.0pt,0pt);
\draw[black] (-0.0pt,-0.0pt) node[anchor=north east] {\tiny $0$};
\clip(-3.0,-3.0) rectangle (3.0,3.0);
\draw[smooth,samples=27,domain=-3:3, line width=1.0pt,color=red!50!black] plot(\x,{\x*\x+\fsy});
\end{scope}
}
\end{tikzpicture}
}
\end{center}

Three cases: no intersection point, one intersection point and two intersection points with the
$x$ axis.


\begin{MExample}
The quadratic equation $x^2-x+1=0$ has no solution since the discriminant $\Mtfrac14p^2-q=-\Mtfrac34$
within the $p q$ formula is negative.
In contrast, the equation $x^2-x-1=0$ has two solutions
\begin{eqnarray*}
x_1 &=& \Mdfrac12+\sqrt{\Mdfrac14+1} \;=\; \Mdfrac12(1+\sqrt5) \MDFPSpace ,\ \\
x_2 &=& \Mdfrac12-\sqrt{\Mdfrac14+1} \;=\; \Mdfrac12(1-\sqrt5) \MDFPeriod
\end{eqnarray*}
\end{MExample}

\begin{MInfo}
\MLabel{VBKM02_Scheitelpunktform}
%%Eine quadratische Gleichung ist in \MEntry{Scheitelpunktform}{Scheitelpunktform}, wenn sie die Form $a\cdot (x-s)^2=d$ besitzt mit $a\not=0$ und $d\geq 0$.
%%F�r den Funktionsausdruck der zugeh�rigen Parabel liest sich diese Form $f(x)=a\cdot (x-s)^2-d$.
%%In dieser Situation ist $(s\,|\,-d)$ der \MEntry{Scheitelpunkt}{Scheitelpunkt (Parabel)} der Parabel.
%%
%%Falls $a>0$ ist gibt es zwei L�sungen
%%$$
%%x_1 \;=\; s-\sqrt{\Mdfrac{d}{a}} \MDFPSpace,\MDFPaSpace x_2 \;=\; s+\sqrt{\Mdfrac{d}{a}}
%%$$
%%der Gleichung, diese liegen symmetrisch um die $x$-Koordinate des Scheitelpunkts. F�r $d=0$ gibt es nur eine L�sung.
%%
%%Das Vorzeichen von $a$ bestimmt, ob die Gleichung eine nach oben oder unten ge�ffnete Parabel beschreibt.
The function expression of a parabola has \MEntry{vertex form}{vertex form} if the function
has the form $f(x)=a\cdot (x-s)^2-d$ with $a\ne 0$. In this case, $\MPointTwo{s}{-d}$
is the \MEntry{vertex}{vertex (parabola)} of the parabola. The corresponding 
quadratic equation for $f(x)=0$ then reads $a\cdot (x-s)^2=d$.

Dividing this equation by $a$ one obtains the equivalent quadratic equation 
$(x-s)^2=\frac{d}{a}$. Since the left-hand side is a square of a real number, only 
solutions exist if and only if the right-hand side is non-negative as well, i.e.
$\frac{d}{a}\ge 0$. By taking the square root, taking the two possible signs into
account, one obtains $x-s=\pm\sqrt{\frac{d}{a}}$.

So, for $\frac{d}{a}>0$ two solutions of the equation exist:
$$
x_1 \;=\; s-\sqrt{\Mdfrac{d}{a}} \MDFPSpace,\MDFPaSpace x_2 \;=\; s+\sqrt{\Mdfrac{d}{a}}\,;
$$
they are symmetric to the $x$ coordinate $s$ of the vertex. For $d=0$, only one
solution exists.

The sign of $a$ determines whether the function expression describes a parabola 
opening upwards or downwards. 
\end{MInfo}

The quadratic equation has only one single solution $s$ if it can be transformed into the form
$(x-s)^2=0$.

\begin{MInfo}
\MLabel{VBKM02_QuadratischErgaenzung}

Any quadratic equation can be transformed (after collecting terms on the left-hand side
and normalisation, if necessary) to vertex form by 
\MEntry{completing the square}{completing the square}. For this, a constant is
added to both sides of the equation such that on the left-hand side we have a term of the 
form $x^2\pm 2s x+s^2$ to which the first or second binomial formula can be applied.
\end{MInfo}

\begin{MExample}
Adding the constant $2$ transforms the equation $x^2-4x+2=0$ into the
form $x^2-4x+4=2$ or into the form $(x-2)^2=2$, respectively. From this,
the two solutions $x_1=2-\sqrt{2}$ and $2+\sqrt{2}$ can be seen immediately.
In contrast, the quadratic equation $x^2+x=-2$ has no solution since completing
the square results in $x^2+x+\Mtfrac14=-\Mtfrac74$ or $(x+\Mtfrac12)^2=-\Mtfrac74$, respectively, 
where the right-hand side is negative for $a=1$.
\end{MExample}

\begin{MExercise}
Find the solutions of the following quadratic equations by completing the square after 
collecting terms on the left-hand side and normalisation (i.e.\ selecting $a=1$):

\begin{MExerciseItems}
\item{$x^2=8x-1$ has the vertex form \MEquationItem{\MLSimplifyQuestion{20}{x^2-8*x+16}{5}{x}{5}{0}{MSPF1}}{\MLParsedQuestion{5}{15}{5}{GLD1}}.\\The solution set is \MEquationItem{$\ML$}{\MLParsedQuestion{30}{4-sqrt(15),4+sqrt(15)}{5}{SQRT1}}.}
\item{$x^2=2x+2+2x^2$ has the vertex form \MEquationItem{\MLSimplifyQuestion{20}{x^2+2*x+1}{5}{x}{5}{0}{MSPF2}}{\MLParsedQuestion{5}{-1}{5}{GLD2}}.\\The solution set is \MEquationItem{$\ML$}{\MLParsedQuestion{30}{}{5}{SQRT2}.}}
\item{$x^2-6x+18=-x^2+6x$ has the vertex form \MEquationItem{\MLSimplifyQuestion{20}{x^2-6*x+9}{5}{x}{5}{0}{MSPF3}}{\MLParsedQuestion{5}{0}{5}{GLD3}}.\\The solution set is \MEquationItem{$\ML$}{\MLParsedQuestion{30}{3,3}{5}{SQRT3}.}}
\end{MExerciseItems}
\MInputHint{Enter sets in the form \texttt{$\lbrace$ a;b;c;$\ldots\rbrace$}. Enter the empty set as $\lbrace\rbrace$.}

\begin{MHint}{Solution}
The transformations are for the first equation
\begin{eqnarray*}
&&x^2=8x-1 \ \\
&\Leftrightarrow&x^2-8x+1=0 \ \\
&\Leftrightarrow&x^2-8x+16=15 \ \\
&\Leftrightarrow&(x-4)^2=15 \ \\
&& \ML=\lbrace 4-\sqrt{15}\MElSetSep 4+\sqrt{15}\rbrace
\end{eqnarray*}
and for the second equation
\begin{eqnarray*}
&&x^2=2x+2+2x^2 \ \\
&\Leftrightarrow&x^2+2x+2=0 \ \\
&\Leftrightarrow&x^2+2x+1=-1 \ \\
&\Leftrightarrow&(x+1)^2=-1\ \\
&& \ML=\lbrace\rbrace
\end{eqnarray*}
and for the third equation
\begin{eqnarray*}
&&x^2-6x+18=-x^2+6x \ \\
&\Leftrightarrow&2x^2-12x+18=0 \ \\
&\Leftrightarrow&x^2-6x+9=0 \ \\
&\Leftrightarrow&(x-3)^2=0\ \\
&& \ML=\lbrace3\rbrace \MDFPeriod
\end{eqnarray*}
\end{MHint}

\end{MExercise}

\end{MXContent}


\MSubsection{Absolute Value Equations}
\MLabel{M02_Betragsgleichungen}

\begin{MIntro}
\MDeclareSiteUXID{VBKM02_Betragsgleichungen_Intro}
\MLabel{VBKM02_Betrag}
The absolute value $|x|$ assigns a variable $x\in\R$ its value without sign: If $x\geq 0$, then $|x|=x$, 
otherwise $|x|=-x$ (see figure).

\begin{center}
%%\MUGraphicsSolo{betrag1.png}{width=0.4\linewidth}{width:350px}\\
\MTikzAuto{%
\begin{tikzpicture}[x=1.0cm, y=1.0cm] 
\draw[black] (-3,0) -- (3,0) (0,-3) -- (0,3);
\foreach \x in {-3, -2, -1, 1, 2, 3}
\draw[shift={(\x,0)},color=black] (0pt,0pt) -- (0pt,-2.0pt) node[below=1.0pt] {\scriptsize $\x$};
\foreach \x in {-2.5, -1.5, ..., 2.5}
\draw[shift={(\x,0)},color=black] (0pt,0pt) -- (0pt,-1.0pt);
\foreach \y in {-3, -2, -1, 1, 2, 3}
\draw[shift={(0,\y)},color=black] (0pt,0pt) -- (-2.0pt,0pt) node[left=1.0pt] {\scriptsize $\y$};
\foreach \y in {-2.5, -1.5, ..., 2.5}
\draw[shift={(0,\y)},color=black] (0pt,0pt) -- (-1.0pt,0pt);
\draw[black] (-0.0pt,-0.5pt) node[anchor=north east] {\scriptsize $0$};
\clip(-3.0,-3.0) rectangle (3.0,3.0);
\draw[black, line width=1.0pt,color=red!50!black] (-3,3) -- (0,0) -- (3,3);
\end{tikzpicture}
}
\par
The absolute value $|x|$ as a function of $x$.
\end{center}

Absolute value equations are equations in which one absolute value or several absolute values occur. 
Problems arise since the absolute value is calculated by distinguishing the two cases  
$$
\left|{\,\text{term}\,}\right| \;\; =\;\; \MCases{\text{term} & \text{for}\;\text{term}\geq 0\\ -\text{term} & \text{for}\;\text{Term}<0}\,.
$$
For solving absolute value equations, these cases have to be solved step by step 
and analysed to find the solutions.

\begin{MExample}
Obviously, the absolute value equation $|x|=2$ has the solution set $\ML=\lbrace 2\MElSetSep -2\rbrace$.
Just as easy, it can be seen that $|x-1|=3$ has the solution set $\ML=\lbrace -2\MElSetSep 4\rbrace$.
\end{MExample}

As soon as beside the absolute value several other terms occur a case analysis is required.
In the following section we will explain in detail how this analysis is done and how it is 
written correctly since the case analysis will play an important role in the next modules.
\end{MIntro}

\begin{MXContent}{Carry out a Case Analysis}{Case Analysis}{STD}
\MDeclareSiteUXID{VBKM02_Fallunterscheidungen}

\begin{MInfo}
\MLabel{VBKM02_FallBetrag}
To solve an \MEntry{absolute value equation}{absolute value equation} two cases are distinguished:
\begin{itemize}
\item{For all values of $x$ for which the absolute value term is non-negative the absolute value can be omitted or
replaced by simple brackets, respectively.}
\item{For all values of $x$ for which the absolute value term is negative the term is bracketed and negated.}
\end{itemize}
Then, the solution sets from the case analyses will be restricted to satisfy the case conditions. 
Only if this procedure is finished for all cases, the solution subsets will be merged to the 
solution set of the initial equation. 
\end{MInfo}

For solving absolute value equations it is important to write down the solution steps correctly
and to distinguish the cases clearly.

The following video demonstrates a detailed written solution of the absolute value
equation $|2x-4|=6$ by case analysis.


\MVideo{vidbsp1}{Carry out a case analysis.\MCopyrightLabel{VBKM06_Video_Beispiel1}}
\ \\
\MCopyrightNotice{\MCCLicense}{FSZ}{MINT}{Dozentin: Dipl.-ing. Heike Herold}{VBKM06_Video_Beispiel1}

The case analysis presented in the video reads shortly
$$
|2x-4| \;=\; \MCases{2x-4 &\text{for}\;x\geq 2 \\ -2x+4 & \text{for}\;x<2} \;=\;\MCases{2x-4 &\text{for}\;x\geq 2 \\ -2x+4 & \text{otherwise}} \MDFPeriod
$$
\MInputHint{In einem Eingabefeld w�rde man das als \texttt{falls(x>=2;2*x-4;-2*x+4)} eintippen. Auch \texttt{falls(x<2;-2*x+4;2*x-4)} w�re richtig.}

\begin{MExercise}
Describe the values of the expression $2\cdot |x-4|$ by a case analysis:
\begin{center}
$\displaystyle 2\cdot |x-4|$ = \MLSimplifyQuestion{30}{falls(x>=4,2*(x-4),-2*(x-4))}{15}{x}{4}{128}{VBKM02ALTFALL1}.
\end{center}
\MInputHint{Enter the case analysis in the form \texttt{falls(BEDINGUNG;W1;W2)}, where \texttt{W1} is the value of the expression if the corresponding condition is satisfied. Do not use the absolute value function.}
\begin{MHint}{Solution}
\begin{eqnarray*}
2\cdot |x-4| \;=\; \MCases{2x-8 &\text{for}\;x\geq 4 \\ -2x+8 & \text{for}\;x<4}
\end{eqnarray*}
\end{MHint}
\end{MExercise}

\begin{MExercise}
Reproduce the steps shown in the video \MRef{VBKM06_Video_Beispiel1} to solve the absolute 
value equation $|6+3x|=12$.

The case analysis reads shortly $|6+3x|$ = \MLSimplifyQuestion{40}{for(x>=-2,6+3*x,-6-3*x)}{15}{x}{3}{128}{VBKM02ALTFALL2}.
\begin{MHint}{Solution}
\begin{eqnarray*}
|6+3x| \;=\; \MCases{6+3x &\text{for}\;x\geq -2 \\ -6-3x & \text{for}\;x<-2}
\end{eqnarray*}
\end{MHint}\\
\MInputHint{Enter the case analysis in the form \texttt{falls(BEDINGUNG;W1;W2)}, where \texttt{W1} is the value of the expression if the corresponding condition is satisfied.
You can copy one of the input examples into the input field and adapt it to the new equation.}
\ \\ \ \\
Finding the solution for each case and checking the case conditions leads to the solution set
\MEquationItem{$\ML$}{\MLParsedQuestion{10}{2,-6}{3}{LVI}} for the equation $|6+3x|=12$.

\begin{MHint}{Solution}
\begin{eqnarray*}
\ML \;=\; \{-6\MElSetSep 2\}
\end{eqnarray*}
\end{MHint}\\
\MInputHint{Mengen k�nnen in der Form \texttt{$\lbrace$a;b;c;\ldots$\rbrace$} eingegeben werden.}
\end{MExercise}

You can practise the stepwise solution of absolute value equations within the following exercise. 


\MDirectRouletteExercises{abs_equations.rtex}{VBKM02_ABSEQTRAINING}


\end{MXContent}

\begin{MXContent}{Mixed Equations}{Mixed Equations}{STD}
\MDeclareSiteUXID{VBKM02_GemischteGleichungen}

\begin{MInfo}
If an equation contains both absolute values and other expressions, the case analysis 
has to be done according to the absolute value terms and applied only to these.
\end{MInfo}

Finally, keep in mind to crosscheck the found solution sets with the case conditions.

\begin{MExample}
Solve the equation $|x-1|+x^2=1$. Here, the case analysis is as follows:
\begin{itemize}
\item{For $x\geq 1$, the absolute value bars can be replaced by normal brackets which results
in the quadratic equation $(x-1)+x^2=1$ that is transformed into the equation $x^2+x-2=0$.
Using the $p q$ formula we get the two solutions
\begin{eqnarray*}
x_1 & =& -\Mdfrac12-\sqrt{\Mdfrac94}\;=\; -2 \MDFPSpace, \\
x_2 & =& -\Mdfrac12+\sqrt{\Mdfrac94}\;=\; 1 \MDFPSpace
\end{eqnarray*}
of which only $x_2$ satisfies the case condition.
}
\item{For $x<1$, one obtains the quadratic equation $-(x-1)+x^2=1$ that is 
transformed into the equation $x^2-x=0$ or $x\cdot (x-1)=0$, respectively. The product representation 
indicates the two solutions $x_3=0$ and $x_4=1$. Because of the case condition only 
$x_3=0$ is a solution of the initial equation.
}
\end{itemize}
So, altogether $\ML=\lbrace 0\MElSetSep 1\rbrace$ is the solution set of the 
initial equation.
\end{MExample}

%TODO: Intervalle sind hier noch garnicht erklaert
%TODO: Kodierung von offenen Intervallgrenzen in "\MLIntervalQuestion" (?!)
\begin{MExercise}
Find the solution set of the mixed equation $|x-3|\cdot x=9$.
\begin{MExerciseItems}
\item{If $x$ is in the interval \MLIntervalQuestion{14}{[3,infty)}{5}{GIM1} the  
absolute value term is non-negative.\\ One obtains the quadratic equation 
\MEquationItem{\MLFunctionQuestion{15}{x^2-3*x-9}{5}{x}{5}{GIM2}}{$0$}.\\
The solution set is \MLParsedQuestion{35}{3/2-sqrt(45/4)\MElSetSep 3/2+sqrt(45/4)}{5}{GIM3}.\\
Only the solution \MLParsedQuestion{15}{3/2+sqrt(45/4)}{5}{GIM4} satisfies the case condition.}

\item{If $x$ is in the interval \MLIntervalQuestion{14}{(-infty,3)}{5}{GIM5} the absolute 
value term is negative.\\  One obtains the normalised quadratic equation 
\MEquationItem{\MLFunctionQuestion{15}{x^2-3*x+9}{5}{x}{5}{GIM6}}{$0$}.\\
The solution set is \MLParsedQuestion{30}{}{5}{GIM7}.}
\end{MExerciseItems}
%TODO: Text und Kodierung anpassen
\MInputHint{Enter open intervals in the form $(3;5)$ and closed intervals in the form $[3;5]$. Enter
``infinity'' as text or simply as \texttt{infty}. Do not use the notation $]a;b[$ fir open intervals. 
Sets can be entered by listing the elements in the form $\lbrace 1\MElSetSep 2\MElSetSep 3\rbrace$. For 
the set brackets enter AltGr+7 or AltGr+0, respectively.} 

So, altogether the solution set is \MEquationItem{$\ML$}{\MLParsedQuestion{20}{3/2+sqrt(45/4),3/2+sqrt(45/4)}{5}{GIM8}}.

\begin{MHint}{Solution}
If $x$ is in the interval $\left[3\MIntvlSep \infty\MoIr[\right]$ the absolute value term is non-negative
and one obtains the quadratic equation $x^2-3x-9=0$ with the solution set 
$\ML=\lbrace \Mtfrac32-\sqrt{\Mtfrac{45}{4}};\Mtfrac32+\sqrt{\Mtfrac{45}{4}}\rbrace$. Only the 
larger solution $\Mtfrac32+\sqrt{\Mtfrac{45}{4}}$ satisfies the condition $x\geq 3$. This can 
also be seen without any calculator by estimating $\sqrt{\Mtfrac{45}{4}}\geq\sqrt{\Mtfrac{36}{4}}=3$.
In contrast, if $x$ is in the interval $\MoIl[\left]-\infty\MIntvlSep 3\MoIr[\right]$
the absolute value term is negative. One obtains the normalised quadratic equation $x^2-3x+9=0$.
Because of $\Mtfrac14p^2-q<0$ in the $pq$ formula this equation is unsolvable. Hence, the initial
equation has only one solution $\Mtfrac32+\sqrt{\Mtfrac{45}{4}}$.
\end{MHint}
\end{MExercise}

\begin{MExercise}
Find the solutions of the mixed absolute value equation $3|2x+1|=|x-5|$ by visualising 
the different cases on the number line and finally solving the equation by case analysis. 
First, visualise the case analysis for each absolute value. 

The solution set is \MLParsedQuestion{10}{-8/5,2/7}{5}{PARSEDQUEST1}.

\begin{MHint}{Solution}
Visualising the different cases for the expressions $|2x+1|$ and $|x-5|$ above each other 
indicates all cases to be distinguished:

%%\MGraphics{fallunterscheidung3.png}{scale=1}{Graphische Darstellung der drei F�lle.\MCopyrightLabel{VBKM02_Grafik_Fallunterscheidung3}}
%%\MCopyrightNotice{\MCCLicense}{NONE}{VEMINT}{Im Rahmen des VE\&MINT-Projekts}{VBKM02_Grafik_Fallunterscheidung3}
\begin{center}
\MTikzAuto{%
\begin{tikzpicture}[x=0.35cm, y=1.0cm] 
\fill[color=yellow!20!white] (-7,-1.2) rectangle (15,6.2);
\foreach \sy in {0.0cm,2.5cm,5.0cm} {
\begin{scope}[xshift=0,yshift=\sy]
\draw[-stealth',black] (-6,0) -- (13.5,0);
\foreach \x in {-4, 0, 4, 8, 12}
\draw[shift={(\x,0)},color=black] (0pt,0pt) -- (0pt,3.0pt) (0pt,0pt) node[below=1.0pt] {\small $\x$};
\foreach \x in {-5, -3, -2, -1, 1, 2, 3, 5, 6, 7, 9, 10, 11}
\draw[shift={(\x,0)},color=black] (0pt,0pt) -- (0pt,1.0pt);
\end{scope}
\def\rdst{0.2}
\def\tdst{0.6}
\def\vdst{0.4}
\draw[color=red,line width=2.2pt] (-6.0,\rdst) -- (-0.5,\rdst);
\draw[color=green!60!black,line width=2.2pt] (-0.5,\rdst) -- (5.0,\rdst);
\draw[color=orange,line width=2.2pt] (5.0,\rdst) -- (12.0,\rdst);
\draw[color=red,line width=2.2pt] (-3.25,\tdst) node {Case (1)};
\draw[color=green!60!black,line width=2.2pt] (2.25,\tdst) node {Case (2)};
\draw[color=orange,line width=2.2pt] (8.5,\tdst) node {Case (3)};
\draw[color=blue,line width=0.8pt] (-0.5,\vdst) -- (-0.5,-0.5) node[below=0pt] {$-\frac{1}{2}$} ;
\draw[color=blue,line width=0.8pt] (5.0,\vdst) -- (5.0,-0.5) node[below=0pt] {$5$} ;
\begin{scope}[xshift=0,yshift=2.5cm]
\draw[color=red,line width=2.2pt] (-6.0,\rdst) -- (5.0,\rdst);
\draw[color=orange,line width=2.2pt] (5.0,\rdst) -- (12.0,\rdst);
\draw[color=red,line width=2.2pt] (-0.50,\tdst) node {$x-5<0$};
\draw[color=orange,line width=2.2pt] (8.5,\tdst) node {$x-5>0$};
\draw[color=blue,line width=0.8pt] (5.0,\vdst) -- (5.0,-0.5) node[below=0pt] {$5$} ;
\end{scope}
\begin{scope}[xshift=0,yshift=5cm]
\draw[color=red,line width=2.2pt] (-6.0,\rdst) -- (-0.5,\rdst);
\draw[color=orange,line width=2.2pt] (-0.5,\rdst) -- (12.0,\rdst);
\draw[color=red,line width=2.2pt] (-3.25,\tdst) node {$2x+1<0$};
\draw[color=orange,line width=2.2pt] (5.75,\tdst) node {$2x+1>0$};
\draw[color=blue,line width=0.8pt] (-0.5,\vdst) -- (-0.5,-0.5) node[below=0pt] {$-\frac{1}{2}$} ;
\end{scope}
}
\end{tikzpicture}
}
\par
Illustration of the three cases
\end{center}

According to the figure above the following three cases have to be distinguished:
\begin{itemize}
\item{Case (1): For $x<-\Mtfrac12$ both terms in the absolute value terms are negative.}
\item{Case (2): For $-\Mtfrac12\leq x<5$ the term in the second absolute value term is negative but the 
term in the first one is not.}
\item{Case (3): For $5\leq x$ both terms in the absolute value terms are non-negative.}
\item{Obviously, there is no $x$ for which the first term is negative and the second 
term is non-negative.}
\end{itemize}

So, the solutions can be summarised:
\begin{itemize}
\item{In case (1), both absolute values reverse the sign of the terms:\\ $3|2x+1|=|x-5|\;\Leftrightarrow\;3(-(2x+1)) = -(x-5)$.\\
This equation has the solution $x=-\Mtfrac85$ satisfying the case condition.}
\item{In case (2), only the second absolute value reverses the sign of the term: \\$3|2x+1|=|x-5|\;\Leftrightarrow\;3(2x+1) = -(x-5)$.\\
This equation has the solution $x=\Mtfrac27$ satisfying the case condition.}
\item{In case (3), the absolute value bars in both terms can be omitted (replaced by normal brackets): \\
 $3|2x+1|=|x-5|\;\Leftrightarrow\;3(2x+1) = (x-5)$.\\
This equation has the solution $x=-\Mtfrac85$, but this solution does \textit{not} 
satisfy the case condition. Thus, it will be discarded within its case analysis.}
\end{itemize}
Therefore, the solution set is $\lbrace -\Mtfrac85\MElSetSep \Mtfrac27\rbrace$.
\end{MHint}
\end{MExercise}


\end{MXContent}

\MSubsection{Final Test}
\MLabel{M02_Abschlusstest}

\begin{MTest}{Final Test Modul 2}
\MDeclareSiteUXID{VBKM02_Abschlusstest}

\begin{MExercise}
Find an absolute value term describing the following graph of a function as easy as possible:

%%\MUGraphics{abs2.png}{width=0.5\linewidth}{Funktionsgraph von $f(x)$.}{width:300px}
\begin{center}
\MTikzAuto{%
\begin{tikzpicture}[x=0.8cm, y=1.12cm] 
\draw[black] (-5,0) -- (5,0) (0,-5) -- (0,5);
\foreach \x in {-4, -2, 2, 4}
\draw[shift={(\x,0)},color=black] (0pt,0pt) -- (0pt,-3.0pt) node[below=1.0pt] {\normalsize $\x$};
\foreach \x in {-4.8, -4.4, ..., 5.0}
\draw[shift={(\x,0)},color=black] (0pt,0pt) -- (0pt,-1.5pt);
\foreach \y in {-4, -2, 2, 4}
\draw[shift={(0,\y)},color=black] (0pt,0pt) -- (-3.0pt,0pt) node[left=1.0pt] {\normalsize $\y$};
\foreach \y in {-4.8, -4.4, ..., 5.0}
\draw[shift={(0,\y)},color=black] (0pt,0pt) -- (-1.5pt,0pt);
%%\draw[black] (-0.0pt,-0.5pt) node[anchor=north east] {\small $0$};
\clip(-5.0,-5.0) rectangle (5.0,5.0);
\draw[black, line width=1.0pt,color=black] (-1,6) -- (3,-2) -- (5,2);
\end{tikzpicture}
}
\par
Graph of the function $f(x)$.
\end{center}

Answer: $f(x)=$ \MLSimplifyQuestion{20}{2*abs(x-3)-2}{5}{x}{5}{0}{KA0}\: .
\end{MExercise}

\begin{MExercise}
Solve the following equations:
\begin{MExerciseItems}
\item{$|2x-3|=8$ has the solution set \MLParsedQuestion{16}{11/2,-5/2}{5}{KA1}.}
\item{$|x-2|\cdot x=0$ has the solution set \MLParsedQuestion{16}{0,2}{5}{KA2}.} 
\end{MExerciseItems}
\MInputHint{Enter sets in the form \texttt{$\lbrace$ a;b;c;$\ldots\rbrace$}. 
Enter the empty set as $\lbrace\rbrace$.}
\end{MExercise}

\begin{MExercise}
A camera has a resolution of $6$ megapixels, i.e. -- for convenience -- 
6 million pixels and produces images in format $2:3$. Which size has a 
quadratic pixel on a print-out of format $(60$ cm$) \times (40$ cm$)$? 
Specify the side length of a pixel in millimetre. 

Answer: \MLParsedQuestion{10}{0.2}{5}{VPIXQ}\ \ \ \ (without the unit mm).
\end{MExercise}

\begin{MExercise}
Find the solution set of the mixed equation $|x-1|\cdot (x+1)=3$.\\
Answer: \MEquationItem{$\ML$}{\MLParsedQuestion{15}{2,2}{5}{KAX4}}.
\end{MExercise}


\end{MTest}


\newpage
\MPrintIndex

\end{document}

\MSection{Inequalities in one Variable}
\MLabel{VBKM03}
\MSetSectionID{ungl}

\begin{MSectionStart}
\MDeclareSiteUXID{VBKM03_START}

Inequalities arise by relating terms using one of the comparing symbols $\leq$, $<$, $\geq$, or $>$. Simple 
inequalities usually have intervals as their solution sets. But the solution of inequalities is often
more difficult than the solution of equations. Hence, specific types of inequalities will be explained
in more detail.

This module consists of:

\begin{itemize}
\item{Section~\MNRef{M03_Ungleichungen}: \MSRef{M03_Ungleichungen}{Inequalities and their Solution Sets},}
\item{Section~\MNRef{M03_Umformen}: \MSRef{M03_Umformen}{Transformation of Inequalities},}
\item{Section~\MNRef{M03_Betragsungleichungen}: \MSRef{M03_Betragsungleichungen}{Absolute Value Inequalities and 
Quadratic Inequalities},}
\item{and Section~\MNRef{M03_Abschlusstest}: \MSRef{M03_Abschlusstest}{Final Test}.}
\end{itemize}

\end{MSectionStart}

\MSubsection{Inequalities and their Solution Sets}
\MLabel{M03_Ungleichungen}

\begin{MIntro}

\begin{MInfo}
\MDeclareSiteUXID{VBKM03_UngleichungenIntro}
If two numbers are related by one of the \MEntry{comparing symbols}{comparing symbols} 
$\leq$, $<$, $\geq$, or $>$, a statement is generated that can be true or false depending on 
the numbers:
\begin{itemize}
\item{$a<b$ (reads: ``$a$ is strictly less than $b$'' or simply ``$a$ is less than $b$'') is true if the number $a$ is less than and not equal to $b$.}
\item{$a \leq b$ (reads: ``$a$ is less than $b$'') is true if $a$ is less than or equal to $b$.}
\item{$a>b$ (reads: ``$a$ is strictly greater than $b$'' or simply ``$a$ is greater than $b$'') is true if the number $a$ is greater and not equal to $b$.}
\item{$a \geq b$ (reads: ``$a$ is greater than $b$'') is true if the number $a$ is greater than or equal to $b$.}
\end{itemize}
\end{MInfo}

The relating symbols describe how the given values are related to each other on the number line: 
$a<b$ means that $a$ is to the left of $b$ on the number line.

\begin{MExample}
The statements $2<4$, $-12\leq 2$, $4>1$, and $3\geq 3$ are true,
but the statements $2<\sqrt2$ and $3>3$ are false.

\begin{center}
\MTikzAuto{%
\begin{tikzpicture}
% reelle Achse
\draw[->,color=black] (-1,0.0) -- (5,0.0);
\foreach \x in {-1, 0, 1, 2, 3, 4}
\draw[shift={(\x,0)},color=black] (0pt,2pt) -- (0pt,-2pt) node[below] {\footnotesize $\x$};
\draw (4.9,-0.3) node[] {$\mathbb{R}$};
% Enden:
\draw [fill = blue] (2,0) circle (1.5pt);
\draw [fill = blue] (4,0) circle (1.5pt);
\end{tikzpicture}
}%

On the number line, the number $2$ is to the left of the number $4$, thus $2<4$.
\end{center}

\end{MExample}

Here, $a<b$ means the same as $b>a$, likewise $a\leq b$ means the same as $b\geq a$. But it
should be noted that the opposite of the statement $a<b$ is the statement $a\geq b$ and not
$a>b$. If terms with a variable occur in an inequality, the problem is to find the number range
of the variable such that the inequality is true. 
\end{MIntro}

\begin{MXContent}{Solving simple Inequalities}{Solving}{STD}
\MDeclareSiteUXID{VBKM03_EinfacheUngleichungen}

If the variable occurs isolated in the inequality, the solution set is an interval, see also info
box \MRef{VBKM01_Intervalle}: 


\begin{MInfo}
\MLabel{M03_Aufloesungen}
The \MEntry{solved inequalities}{inequalities (solved)} 
have the following \MEntry{intervals}{intervals} as their solution sets:
\begin{itemize}
\item{$x< a$ has the solution set $\MoIl[\left] -\infty\MIntvlSep a\MoIr[\right]$, i.e.\ all $x$ less than $a$.}
\item{$x\leq a$ has the solution set $\MoIl[\left] -\infty\MIntvlSep a\right]$, 
i.e. all $x$ less than or equal to $a$.}
\item{$x> a$ has the solution set $\MoIl[\left] a\MIntvlSep \infty\MoIr[\right]$, i.e.\ all
 $x$ greater than $a$.}
\item{$x\geq a$ has the solution set $\left[a\MIntvlSep \infty\MoIr[\right]$, i.e.\ all $x$
greater than or equal to $a$.}
\end{itemize}
Here, $x$ is the variable and $a$ is a specific value. 
If the variable does not occur in the inequality anymore, the solution set is either
$\R=\MoIl[\left] -\infty\MIntvlSep \infty\MoIr[\right]$ if the inequality is satisfied, 
or the empty set $\lbrace \rbrace$ if the inequality is not satisfied.
\end{MInfo}

The symbol $\infty$ means \MEntry{infinity}{infinity}. A finite interval has the form 
$\MoIl a\MIntvlSep b\MoIr$ which reads ``all numbers between $a$ and $b$''. If the interval
is to be finite only on one side, the other interval boundary can be replaced by the symbol 
$\infty$ (right-hand side) or $-\infty$ (left-hand side).

As for equations one tries to find a solved inequality by applying transformations that do
not change the solution set. From the solved inequality the solution set can be easily seen.


\begin{MInfo}
\MLabel{VBKM03_AequivalenzumformungenUngleichungen}
To obtain a solved inequality from an unsolved inequality the following 
\MEntry{equivalent transformations}{equivalent transformations (inequality)} are allowed:
\begin{itemize}
\item{adding a constant to both sides of the inequality: $a<b$ is equivalent to $a+c<b+c$.}
\item{multiplying both sides of the inequality by a positive constant: $a<b$ 
is equivalent to $a\cdot c<b\cdot c$ if $c>0$.}
\item{multiplying both sides of the inequality by a negative constant and inverting the 
comparing symbol: $a<b$ is equivalent to $a\cdot c>b\cdot c$ if $c<0$.}
\end{itemize}
\end{MInfo}

\begin{MExample}
The inequality $-\frac34x-\frac12<2$ is solved stepwise by the above transformations:
\begin{eqnarray*}
&&-\frac34x-\frac12 < 2 \;\; \MSep +\frac12\ \\
&\Leftrightarrow&-\frac34x < 2+\frac12 \;\; \MSep \cdot\left({-\frac43}\right)\ \\
&\Leftrightarrow&x > -\frac43\left({2+\frac12}\right) \;\; \MSep \;\text{simplifying}\\
&\Leftrightarrow&x >  -\frac{20}{6} \;=\; -\frac{10}{3} \MDFPeriod
\end{eqnarray*}
So, the initial inequality has the solution set 
 $\MoIl[\left] -\frac{10}{3}\MIntvlSep \infty\MoIr[\right]$. 
Importantly, multiplying the inequality by the negative number $-\frac43$ inverts the 
comparing symbol.
\end{MExample}

\begin{MExercise}
Are the following inequalities true or false?

\begin{MQuestionGroup}
\begin{tabular}{lll}
\MCheckbox{0}{UG1} & \ \ &  $\frac12>1-\frac13$\\
\MCheckbox{1}{UG2} & \ \ & $a^2\geq 2a b-b^2$ (where $a$ and $b$ are unknown numbers)\\
\MCheckbox{1}{UG3} & \ \ & $\frac12<\frac23<\frac34$\\
\MCheckbox{0}{UG4} & \ \ & Let $a<b$, then also $a^2<b^2$.
\end{tabular}
\end{MQuestionGroup}
\MGroupButton{Check input}

\begin{MHint}{Solution}
The first inequality can be simplified to $\frac12>\frac23$, which, after multiplying by $6$, 
is equivalent to $3>4$. This statement is false. The second inequality can be simplified by 
collecting all numbers on the left-hand side: $a^2-2a b+b^2\geq 0$. Since $a^2-2a b+b^2=(a-b)^2$,
this statement is true for all $a$ and $b$. Multiplying the third chain of inequalities by the
least common denominator $12$ results in the chain of inequalities $6<8<9$. This statement is true.
In contrast, the last statement is false, since for example, for $a=-1$ and $b=1$, the term
$a^2=1$ is not less than $b^2=1$. Taking the square of terms is not an equivalent transformation.
\end{MHint}
\end{MExercise}


\begin{MExercise}
Find the solution sets of the following inequalities.
\begin{MExerciseItems}
\item{$2x+1> 3x-1$ has the solution interval \MEquationItem{$\ML$}{\MLIntervalQuestion{30}{(-infty,2)}{5}{TXH1}}.}
\item{$-3x-\frac12\leq x+\frac12$ has the solution interval \MEquationItem{$\ML$}{\MLIntervalQuestion{30}{[-1/4,infty)}{5}{TXH2}}.}
\item{$x-\frac12\leq x+\frac12$ has the solution interval \MEquationItem{$\ML$}{\MLIntervalQuestion{30}{(-infty,infty)}{5}{TXH3}}.}
\end{MExerciseItems}
\MInputHint{Enter the intervals in the form \texttt{(a;b)}, for the interval boundaries also fractions and
\texttt{infinity} or \texttt{-infinity} can be used. Take care whether the interval boundaries are included 
or excluded.}

\begin{MHint}{Solution}
Transformation of the first inequality results in
\begin{eqnarray*}
&& 2x+1 > 3x-1\;\; \MSep+1\ \\
&\Leftrightarrow& 2x+2 > 3x\;\; \MSep-2x\ \\
&\Leftrightarrow&2 > x
\end{eqnarray*}
and hence the solution interval is $\ML=\MoIl[\left] -\infty\MIntvlSep 2\MoIr[\right]$. 
Transformation of the second inequality results in
\begin{eqnarray*}
&&-3x-\frac12\leq x+\frac12 \;\; \MSep +3x-\frac12\ \\
&\Leftrightarrow&-1\leq 4x \;\; \MSep \cdot \frac14\ \\
&\Leftrightarrow&-\frac14\leq  x
\end{eqnarray*}
and hence $\ML=\left[-\frac14\MIntvlSep \infty\MoIr[\right]$. 
Transformation of the third inequality results in
\begin{eqnarray*}
&&x-\frac12\leq x+\frac12\;\; \MSep-x\ \\
&\Leftrightarrow&-\frac12\leq \frac12 \MDFPeriod
\end{eqnarray*}
This statement does not depend on $x\in\R$ and is always true, 
thus the solution set is $\ML=\R=\MoIl[\left] -\infty\MIntvlSep \infty\MoIr[\right]$.
\end{MHint}
\end{MExercise}

\begin{MInfo}
An inequality in one variable $x$ is \MEntry{linear}{inequality (linear)} if on both sides of the 
inequality only multiples of $x$ and constants occur. Each linear inequality can be transformed 
into a solved inequality by one of the equivalent transformations described in the info box
\MRef{M03_Aufloesungen}.
\end{MInfo}

\end{MXContent}

\begin{MXContent}{Specific Transformations}{Specific Transformations}{STD}
\MDeclareSiteUXID{VBKM03_SpezielleUmformungen}
The following equivalent transformations are useful if the variable occurs 
in the denominator of an expression. But they can only be applied under certain
restrictions:

\begin{MInfo}
Under the restriction that none of the occurring denominators is zero (the corresponding variable values are
 never solutions) and the fractions on both sides have the same sign, the reciprocal can be taken
on both sides of the inequality while inverting the comparing symbol.
\end{MInfo}

\begin{MExample}
For example, the inequality $\frac1{2x}\leq \frac1{3x}$ is equivalent to $2x\geq 3x$
(comparing symbol inverted) as long as $x\not=0$. The new inequality has the solution
set $\MoIl[\left] -\infty\MIntvlSep 0\right]$. However, since the value $x=0$ was excluded (and 
does not belong to the domain of the initial inequality either) the solution set of 
$\frac1{2x}\leq \frac1{3x}$ is $\ML=\MoIl[\left] -\infty\MIntvlSep 0\MoIr[\right]$.
\end{MExample}

\begin{MExercise}
Find the solution sets of the following inequalities.
\begin{MExerciseItems}
\item{$\frac1x\geq\frac13$ has the solution set \MEquationItem{$\ML$}{\MLIntervalQuestion{20}{(0;3]}{3}{KKL1}}.}
\item{$\frac1x<\frac1{\sqrt{x}}$ has the solution set \MEquationItem{$\ML$}{\MLIntervalQuestion{20}{(1;infty)}{3}{KKL2}}.}
\end{MExerciseItems}

\begin{MHint}{Solution}
For the first inequality, the value $x=0$ is not in the domain, hence this value is excluded. For 
$x>0$, taking the reciprocal while inverting the comparing symbol is allowed and results in
$x\leq 3$. Together with the condition above the solution interval is $\ML=\MoIl 0\MIntvlSep 3]$.
For $x<0$ the reciprocal rule cannot be applied. However, it can be seen, even without any rule, that
none of the values $x<0$ can be a solution, since then $\frac1x$ is negative as well and not greater than
or equal to $\frac13$.

The domain of the second inequality is $\MoIl 0\MIntvlSep \infty\MoIr$, since 
only for these values of $x$ taking the square root is
defined and only for $x\neq 0$ the denominators are non-zero. On the domain, 
taking the reciprocal while inverting the comparing symbol is 
allowed and results in $x>\sqrt{x}$. Since $\sqrt{x}>0$, the inequality can be 
divided by $\sqrt{x}$ resulting in $\sqrt{x}>1$. This inequality has the solution
set $\ML=\MoIl 1\MIntvlSep \infty\MoIr$ which occurs also in the domain.
\end{MHint}

\end{MExercise}

Please note for the last part of the exercise:

\begin{MInfo}
Taking the square on both sides of an inequality is not an equivalent transformation and 
possibly does change the solution set.
\end{MInfo}

For example, $x=-2$ is no solution of $x>\sqrt{x}$, but indeed a solution of $x^2>x$. However,
this transformation can be applied if the case analysis for the transformation
is carried out correctly and the domain of the initial inequality is taken into account. This method is
described in more detail in the next section.
\end{MXContent}

\MSubsection{Transformation of Inequalities}
\MLabel{M03_Umformen}

\begin{MXContent}{Transformation with Case Analysis}{Case Analysis}{STD}
\MDeclareSiteUXID{VBKM03_UmformungenFallunterscheidungen}
The simple linear transformations described in the previous section are equivalent transformations.
They do not change the solution set of the corresponding inequality. For nonlinear
inequalities advanced solution methods are required. Usually, these methods need 
a case analysis depending on the sign, since, in contrast to the situation for
equations described in Modul~\MNRef{VBKM02}, now also the inequality can be inverted during 
transformation.

 
\begin{MInfo}
If an inequality is multiplied by a term in which the variable $x$ occurs, a case analysis 
is required and for each case the transformation has to be considered separately: 

\begin{itemize}
\item{For those values of $x$, for which the multiplied term is positive, the comparing symbol
of the inequality is unchanged.}
\item{For those values of $x$, for which the multiplied term is negative, the comparing symbol
of the inequality is inverted.}
\item{The case that the multiplied term is zero has to be excluded during the transformation and
has to be considered separately, if necessary.}
\end{itemize}
\ \\ \ \\
The solution sets found in the individual cases have to be checked with respect to the case conditions
as described for the solution of \MSRef{VBKM02_FallBetrag}{absolute value equations}.
\end{MInfo}

In contrast, adding terms in which the variable occurs, does not require a case analysis. Usually, 
transformations involving case analyses are mandatory if the variable occurs in the denominator or 
in a composite term.


\begin{MExample}
The inequality $\frac1{2x}\leq 1$ can be simplified by multiplying both sides of the inequality
by the term $2x$:

\begin{itemize}
\item{Under the condition $x>0$ this results in the new inequality $1\leq 2x$. It has the solution set
$\ML_1=\left[\frac12\MIntvlSep \infty\MoIr[\right]$. The condition $x>0$ is satisfied by all elements of the solution 
set.}
\item{Under the condition $x<0$ this results in the new inequality $1\geq 2x$. It has the solution set
 $\MoIl[\left] -\infty\MIntvlSep \frac12\right]$. Because of the additional
condition $x<0$ only the elements of the set 
$\ML_2=\MoIl[\left] -\infty\MIntvlSep 0\MoIr[\right]$ are solutions.}
\item{The single case $x=0$ is no solution since this value is not in the domain of the inequality. 
In this case multiplying the inequality by $x$ is not allowed.}
\end{itemize}
So, altogether one obtains the union set 
$\ML=\ML_1\cup \ML_2=\R\MSetminus\left[0\MIntvlSep \frac12\MoIr[\right]$ as solution set:
\ \\ \ \\
\begin{center}
\MTikzAuto{%
\begin{tikzpicture}
% reelle Achse
\draw[->,color=black] (-1,0.0) -- (5,0.0);
\foreach \x in {-1, 0, 1, 2, 3, 4}
\draw[shift={(\x,0)},color=black] (0pt,2pt) -- (0pt,-2pt) node[below] {\footnotesize $\x$};
\draw (4.9,-0.3) node[] {$\mathbb{R}$};
% Enden:
\draw [line width=2.0pt,color=blue] (-1,0.0)-- (0,0.0);
\draw [line width=2.0pt,color=blue] (0.5,0.0)-- (5,0.0);
\draw [fill = blue] (0.5,0) circle (1.5pt);
\draw [fill = white] (0,0) circle (1.5pt);
\end{tikzpicture}
}
\end{center}
\end{MExample}

As in Modul~\MNRef{VBKM02} the following statement holds for the solution set.

\begin{MInfo}
The cases have to be chosen such that all elements of the domain of the inequality are covered. 
For the solution set in an individual case, it has to be checked that the solution set satisfies the 
corresponding case condition. For any case, the resulting solution set has to be reduced to
the solution subset satisfying the case condition. The union of the solution sets for the individual cases
is the solution set of the initial inequality.
\end{MInfo}

\end{MXContent}

\begin{MExercises}
\MDeclareSiteUXID{VBKM03_Fallunterscheidungen_Exercises}
If the inequality is multiplied by a composite term, it has to be investigated precisely for which values
of $x$ the case analysis has to be done:

\begin{MExercise}
Find the solution set of the inequality $\frac1{4-2x}<3$. 
The domain of the inequality is $D=\R\MSetminus \lbrace 2\rbrace$ since only for these
values of $x$ the denominator is non-zero. If the inequality is multiplied by the term
$4-2x$, three cases have to be distinguished. Fill in the blanks in the following text 
accordingly:

\begin{MExerciseItems}
\item{On the interval \MLIntervalQuestion{15}{(-infty,2)}{4}{GOM1} the term is positive, the comparing symbol 
remains unchanged, and the new inequality reads $1\:<\:$\MLSimplifyQuestion{15}{3*(4-2*x)}{5}{x}{5}{0}{SIMPLE2}.
Linear transformations result in the solution set 
\MEquationItem{$\ML_1$}{\MLIntervalQuestion{20}{(-infty,11/6)}{3}{MIXGOM}}. 
The elements of this set satisfy the case condition.}
\item{On the interval \MLIntervalQuestion{15}{(2,infty)}{4}{GOM2} the term is negative, 
the comparing symbol is inverted. Initially, the new inequality has the solution set 
\MLIntervalQuestion{20}{(11/6,infty)}{4}{INT1}, because of the case condition only the 
subset \MEquationItem{$\ML_2$}{\MLIntervalQuestion{20}{(2,infty)}{4}{GOM3}} is allowed.}
\item{The single value $x=2$ is no solution of the initial inequality since 
it is not in \MLQuestion{25}{domain}{UGX}.}
\end{MExerciseItems}

Sketch the solution set of the inequality and indicate the boundary points.

\begin{MHint}{Solution}
On the interval $\MoIl[\left] -\infty\MIntvlSep 2\MoIr[\right]$ the term is positive. 
The corresponding solution set is $\MoIl[\left] -\infty\MIntvlSep \frac{11}{6}\MoIr[\right]$.
In contrast, on the interval $\MoIl 2\MIntvlSep \infty\MoIr$ the term is negative, 
the comparing symbol is inverted. Initially, the new inequality has the solution set
 $\MoIl[\left] \frac{11}{6}\MIntvlSep \infty\MoIr[\right]$, because of the case condition
$x>2$ only the subset $\ML_2=\MoIl 2\MIntvlSep \infty\MoIr$ is allowed.
So, altogether the union set $\ML=\ML_1\cup\ML_2=\R\MSetminus \left[\frac{11}6\MIntvlSep 2\right]$
is the solution set of the initial inequality excluding the boundary points:

\begin{center}
\MTikzAuto{%
\begin{tikzpicture}
% reelle Achse
\draw[->,color=black] (-1,0.0) -- (5,0.0);
\foreach \x in {-1, 0, 1, 2, 3, 4}
\draw[shift={(\x,0)},color=black] (0pt,2pt) -- (0pt,-2pt) node[below] {\footnotesize $\x$};
\draw (4.9,-0.3) node[] {$\mathbb{R}$};
% Enden:
\draw [line width=2.0pt,color=blue] (-1,0.0)-- (1.83,0.0);
\draw [line width=2.0pt,color=blue] (2,0.0)-- (5,0.0);
\draw [fill = white] (1.83,0) circle (1.5pt);
\draw [fill = white] (2,0) circle (1.5pt);
\end{tikzpicture}
}
\end{center}
\end{MHint}
\end{MExercise}

\begin{MExercise}
The solution set of the inequality $\frac{x-1}{x-2}\leq 1$ is 
\MEquationItem{$\ML$}{\MLIntervalQuestion{20}{(-infty,2)}{4}{IGU1}}.

\begin{MHint}{Solution}
The domain of the inequality is $D=\R\MSetminus\lbrace 2\rbrace$.

\begin{itemize}
\item{For $x>2$, multiplying the inequality by the term $x-2$ results in $x-1\leq x-2$, 
which is equivalent to the false statement $-1\leq -2$. Thus, this case does not contribute a solution to the solution set.}
\item{For $x<2$, multiplying the inequality by the term $x-2$ results in $x-1\geq x-2$, 
which is equivalent to the true statement $-1\geq -2$. 
Because of the case condition the solution interval for this case is only 
$\ML_2=\MoIl[\left] -\infty\MIntvlSep 2\MoIr[\right]$.}
\item{The single value $x=2$ is no solution.}
\end{itemize}
So, altogether the solution set is 
 $\ML=\MoIl[\left] -\infty\MIntvlSep 2\MoIr[\right]$ 
excluding the boundary points (even though the comparing symbol $\leq$ occurred in the initial
inequality).
\end{MHint}
\end{MExercise}


\begin{MExercise}
The solution set of the inequality $\frac1{1-\sqrt{x}}<1+\sqrt{x}$ is \MEquationItem{$\ML$}{\MLIntervalQuestion{20}{(1,infty)}{4}{IGU2}}.
\ \\ \ \\
\begin{MHint}{Solution}
The domain of the inequality is $D=[0\MIntvlSep \infty\MoIr\MSetminus \lbrace 1\rbrace$
since only for these values of $x$ the square root is defined and the denominator is non-zero.
\begin{itemize}
\item{For $0\leq x<1$, multiplying the inequality by the term $1-\sqrt{x}$ results in
  $1<(1+\sqrt{x})(1-\sqrt{x})$, which is equivalent to $1<1-x$. This 
inequality is satisfied for $x<0$, but these values of $x$ violate the case condition and 
thus, they are not in the solution set.}
\item{For $x>1$, multiplying the inequality by the term $1-\sqrt{x}$ results in $1>1-x$, 
which is equivalent to $x>0$. But only the values of $x$ in the interval
  $\MoIl 1\MIntvlSep \infty\MoIr$ satisfy the case condition, hence $\ML=\MoIl 1\MIntvlSep \infty\MoIr$ 
is the only solution interval of the initial inequality.}
\item{The single value $x=1$ is no solution.}
\end{itemize}
\end{MHint}
\end{MExercise}


\end{MExercises}

\MSubsection{Absolute Value Inequalities and Quadratic Inequalities}
\MLabel{M03_Betragsungleichungen}

\begin{MIntro}
\MDeclareSiteUXID{VBKM03_Betragsungleichungen_Intro}
As in the approach in Modul~\MNRef{VBKM02} and in the previous section 
\MEntry{absolute values}{inequalities (absolute values)} in inequalities are solved 
by a case analysis:

\begin{MInfo}
To solve an \MEntry{absolute value inequality}{absolute value inequality} two cases are distinguished:

\begin{itemize}
\item{For those values of $x$, for which the absolute value term is non-negative the absolute value can be omitted or
replaced by simple brackets, respectively.}
\item{For those values of $x$, for which the absolute value term is negative the term is bracketed and negated.}
\end{itemize}
\ \\
Then, the solution sets arising from the case analysis will be restricted as described in the 
\MSRef{VBKM02_FallBetrag}{previous module} and merged to the solution set of the initial inequality. 
\end{MInfo}

\begin{MExample}
To solve the absolute value inequality $|4x-2|<1$ two cases are distinguished:
\begin{itemize}
\item{For $x\geq \frac12$, the absolute value term is non-negative: 
In this case the inequality is equivalent to $(4x-2)<1$ or $x<\frac34$, respectively. 
Because of the case condition the solution set is only 
$\ML_1=\left[\frac12\MIntvlSep \frac34\MoIr[\right]$ in this case.}
\item{For $x<\frac12$, the absolute value term is negative: 
In this case the inequality is equivalent to $-(4x-2)<1$ or $x>\frac14$, respectively. 
Only the subset $\ML_2=\MoIl[\left] \frac14\MIntvlSep \frac12\MoIr[\right]$ 
satisfies the case condition and is the solution set.}
\end{itemize}
The union of the two solution intervals results in the solution set
$\ML=\MoIl[\left] \frac14\MIntvlSep \frac34\MoIr[\right]$ for the initial absolute value inequality:

\begin{center}
\MTikzAuto{%
\begin{tikzpicture}
% reelle Achse
\draw[->,color=black] (-1,0.0) -- (5,0.0);
\foreach \x in {-1, 0, 1, 2, 3, 4}
\draw[shift={(\x,0)},color=black] (0pt,2pt) -- (0pt,-2pt) node[below] {\footnotesize $\x$};
\draw (4.9,-0.3) node[] {$\mathbb{R}$};
% Enden:
\draw [line width=2.0pt,color=blue] (0.25,0.0)-- (0.75,0.0);
\draw [fill = white] (0.25,0) circle (1.5pt);
\draw [fill = white] (0.75,0) circle (1.5pt);
\end{tikzpicture}
}
\end{center}
\end{MExample}

\begin{MExercise}
To solve the absolute value inequality $|x-1|<2|x-1|+x$ two cases are distinguished:
\begin{MExerciseItems}
\item{On the interval \MLIntervalQuestion{20}{[1,infty)}{3}{UGL1}, both
terms in the absolute value terms are non-negative. 
The solution set of the inequality is in this case 
\MEquationItem{$\ML_1$}{\MLIntervalQuestion{20}{[1,infty)}{3}{UGL2}}.}
\item{On the interval \MLIntervalQuestion{20}{(-infty,1)}{3}{UGL3}, both
terms in the absolute value terms are negative. 
The solution set of the inequality is in this case
\MEquationItem{$\ML_2$}{\MLIntervalQuestion{20}{(-infty,1)}{3}{UGL4}}.}
\end{MExerciseItems}
The union of the two intervals results in the solution interval 
\MEquationItem{$\ML$}{\MLIntervalQuestion{25}{(-infty,infty)}{4}{UGL5}}.
\ \\ \ \\
\begin{MHint}{Solution}
For $x\in [1\MIntvlSep \infty\MoIr$, both terms in the absolute value terms are non-negative, 
one obtains the inequality $x-1<2(x-1)+x$, which is equivalent to $x>\frac12$. 
Because of the case condition one obtains $\ML_1=[1\MIntvlSep \infty\MoIr$ as solution set.
For $x\in\MoIl[\left] -\infty\MIntvlSep 1\MoIr[\right]$, 
both terms in the absolute value terms are negative and
one obtains $-(x-1)<-2(x-1)+x$. 
This inequality is equivalent to the inequality $x-1<x$ which is
always true. Thus,
the solution set for the second case is
$\ML_2=\MoIl[\left] -\infty\MIntvlSep 1\MoIr[\right]$.
\ \\ \ \\
Since $\ML=\ML_1\cup \ML_2=\R=\MoIl[\left] -\infty\MIntvlSep \infty\MoIr[\right]$ 
the inequality is always satisfied.
\end{MHint}
\end{MExercise}

\end{MIntro}

\begin{MXContent}{Quadratic Absolute Value Inequalities}{Quadratic Inequalities}{STD}
\MDeclareSiteUXID{VBKM03_QuadratischeUngleichungen}
\begin{MInfo}
An inequality is called \MEntry{quadratic}{inequality (quadratic)} in $x$ 
if it can be transformed into $x^2 + p x + q < 0$ (other comparing symbols are allowed).
\end{MInfo}
\ \\ \ \\
Hence, quadratic inequalities can be solved in two ways: by investigating the roots 
and the opening behaviour of the polynomial and by completing the square. Often completing
the square is simpler:
 

\begin{MInfo}
To solve an inequality by \MEntry{completing the square}{completing the square (inequalities)} 
one tries to transform it into the form $(x+a)^2<b$. Taking the square root then results
in the absolute value inequality $|x+a|<\sqrt{b}$ with the solution set 
$\MoIl[\left] -a-\sqrt{b}\MIntvlSep -a+\sqrt{b}\MoIr[\right]$ if $b\geq 0$. Otherwise 
the inequality is unsolvable.

The inverted inequality $|x+a|>\sqrt{b}$ has the solution set
$\MoIl[\left] -\infty\MIntvlSep -a-\sqrt{b}\MoIr[\right]\cup \MoIl[\left] -a+\sqrt{b}\MIntvlSep \infty\MoIr[\right]$. 
For $\leq$ and $\geq$ the corresponding boundary points have to be included.
\end{MInfo}

Always note the calculation rule $\sqrt{x^2}=|x|$ described in Modul~\MNRef{VBKM01}.

\begin{MExample}
Find the solution of the inequality $2x^2\geq 4x+2$. Collecting the terms on the left-hand side and dividing 
the inequality by $2$ results in $x^2-2x-1\geq0$. Completing the square on the 
left-hand side to the second binomial formula results in the equivalent inequality $x^2-2x+1\geq 2$
or $(x-1)^2\geq 2$, respectively. Taking the square root results in the absolute value 
inequality $|x-1|\geq\sqrt{2}$ with the solution set 
$\ML=\MoIl[\left] -\infty\MIntvlSep 1-\sqrt{2}\right]\cup \left[1+\sqrt{2}\MIntvlSep \infty\MoIr[\right]$.
\end{MExample}

On the other hand, the inequality $x^2-2x-1\geq0$ can be investigated as follows:
The left-hand side describes a parabola opened upwards. The roots $x_{1,2}=1\pm \sqrt2$ 
can be found using the $pq$ formula:

\begin{center}
%%\MUGraphicsSolo{parabelu.png}{width=0.4\linewidth}{width:400px}
\MTikzAuto{%
\begin{tikzpicture}[x=1.0cm, y=1.0cm,scale=1.50] 
\draw[black] (-1,0) -- (3,0) (0,-2) -- (0,2);
\foreach \x in {-1, 1, 2, 3}
\draw[shift={(\x,0)},color=black] (0pt,0pt) -- (0pt,-2.0pt) node[below=1.0pt] {\scriptsize $\x$};
\foreach \x in {-0.5, 0.5, ..., 3.0}
\draw[shift={(\x,0)},color=black] (0pt,0pt) -- (0pt,-1.0pt);
\foreach \y in {-2, -1, 1, 2}
\draw[shift={(0,\y)},color=black] (0pt,0pt) -- (-2.0pt,0pt) node[left=1.0pt] {\scriptsize $\y$};
\foreach \y in {-1.5, -0.5, 0.5, 1.5}
\draw[shift={(0,\y)},color=black] (0pt,0pt) -- (-1.0pt,0pt);
\draw[black] (-0.0pt,-0.0pt) node[anchor=north east] {\scriptsize $0$};
\clip(-1.0,-3.0) rectangle (3.0,2.0);
\draw[smooth,samples=21,domain=-1:3, line width=1.0pt,color=red!50!black] plot(\x,{\x*\x-2*\x-1});
\end{tikzpicture}
}
\end{center}
Since the parabola opens upwards, the inequality $x^2-2x-1\geq0$ is satisfied by the
values of $x$ in the parabola branches left and right to the roots, i.e. by the set 
$\ML=\MoIl[\left] -\infty\MIntvlSep 1-\sqrt{2}\right]\cup \left[1+\sqrt{2}\MIntvlSep \infty\MoIr[\right]$.

\begin{MInfo}
\MLabel{M03_InfoFormen}
Depending on the roots of $x^2+ p x + q$, the opening of the parabola and the 
comparing symbol, the quadratic inequality $x^2 +p x +q <0$ (including other comparing symbols) 
has one of the following solution sets:

\begin{itemize}
\item{the set of real numbers $\R$,}
\item{two branches $\MoIl[\left] -\infty\MIntvlSep x_1\MoIr[\right]\cup \MoIl[\left] x_2\MIntvlSep \infty\MoIr[\right]$ (including the boundary points for $\leq$ and $\geq$),}
\item{an interval $\MoIl x_1\MIntvlSep x_2\MoIr$ (including the boundary points for $\leq$ and $\geq$ if applicable),}
\item{a single point $x_1$,}
\item{the pointed set $\R\MSetminus\lbrace x_1\rbrace$,}
\item{the empty set $\lbrace\rbrace$.}
\end{itemize}
\end{MInfo}

Fill in the blanks in the following text describing the solution of a quadratic
inequality by investigating the behaviour of the parabola:

\begin{MExercise}
Find the solution set of the inequality $x^2+6x< -5$. 
Transformation results in the inequality \MLSimplifyQuestion{15}{x^2+6*x+5}{5}{x}{5}{1}{OBXP1}$<0$.
Using the $p q$ formula one obtains the set of roots
\MLParsedQuestion{9}{-1,-5}{3}{PXL}. The left-hand side
describes a parabola opening \MLQuestion{10}{upwards}{ObenX}.
It belongs to an inequality involving the comparing symbol $<$, hence 
the solution set is \MEquationItem{$\ML$}{\MLIntervalQuestion{15}{(-5,-1)}{5}{INVX}}.
\ \\ \ \\
\begin{MHint}{Solution}
Transformation results in $x^2+6x+5<0$. Using the $p q$ formula
one obtains the roots $x_{1,2}=-3\pm\sqrt{9-5}$, i.e.\ $x_1=-1$ and $x_2=-5$.
The left-hand side describes a parabola opening upwards. It satisfies the inequality 
involving $<$ only on the interval $\MoIl[\left] -5\MIntvlSep -1\MoIr[\right]$ excluding
the boundary points.
\end{MHint}
\end{MExercise}

\end{MXContent}

\begin{MXContent}{Further Types of Inequalities}{Further Types of Inequalities}{STD}
\MDeclareSiteUXID{VBKM03_WeitereUngleichungstypen}
Many other types of inequalities can be transformed into quadratic inequalities. Sometimes, 
case analyses have to be done or excluded values in the domain have to be observed:

\begin{MInfo}
An inequality containing \MEntry{fractions}{inequality (fractions)}, where the 
variable $x$ occurs in the denominator of composite terms, can be transformed into a 
form without fractions by multiplying the inequality by the least common denominator. 
However, in doing so, the roots of the denominators have to be excluded from the domain
of the new inequality. 

Additionally, if the inequality is multiplied by a term, different cases have to be distinguished 
depending on the sign of the term.
\end{MInfo}

\begin{MExample}
The inequality $2-\frac1x\leq x$ can be transformed by multiplying the inequality by $x$. Here, three 
cases have to be distinguished:
\begin{itemize}
\item{For $x>0$, the comparing symbol in the inequality is unchanged. The new inequality
reads $2x-1\leq x^2$ and is equivalent to $x^2-2x+1\geq 0$ or $(x-1)^2\geq 0$, respectively.
This inequality is always satisfied. Because of the case condition one obtains 
the solution set $\ML_1=\MoIl 0\MIntvlSep \infty\MoIr$.}
\item{For $x<0$, the comparing symbol in the inequality is inverted. The new inequality
reads $2x-1\geq x^2$ and is equivalent to $x^2-2x+1\leq 0$ or $(x-1)^2\leq 0$, respectively.
This inequality is only satisfied for $x=1$. But this value is excluded by the case condition, 
i.e.\ $\ML_2=\{\}$.}
\item{The single value $x=0$ is not in the domain of the initial inequality and hence it is
no solution.}
\end{itemize}

So, altogether one obtains the union set 
$\ML=\MoIl 0\MIntvlSep \infty\MoIr$ as solution set of the initial inequality.
\end{MExample}

Inequalities involving composite fraction and root terms often do not have solution
sets of the types described in info box~\MRef{M03_InfoFormen}:

\begin{MExample}
Find the solution set of the inequality $\sqrt{x}+\frac1{\sqrt{x}}>2$. 
The domain of the inequality is $\MoIl 0\MIntvlSep \infty\MoIr$.
Multiplying by $\sqrt{x}$ results in the inequality $x+1>2\sqrt x$. 
Here, no case analysis is required since $\sqrt{x}>0$ is in the domain.
Transformation results in $x-2\sqrt{x}+1>0$ or $(\sqrt{x}-1)^2>0$, respectively, 
which is satisfied for all $x\not=1$ in the domain.
Hence, the solution set of the initial inequality is
$\ML=\MoIl 0\MIntvlSep \infty\MoIr\MSetminus\lbrace 1\rbrace$:
\ \\ \ \\
\begin{center}
\MTikzAuto{%
\begin{tikzpicture}
% reelle Achse
\draw[->,color=black] (-1,0.0) -- (5,0.0);
\foreach \x in {-1, 0, 1, 2, 3, 4}
\draw[shift={(\x,0)},color=black] (0pt,2pt) -- (0pt,-2pt) node[below] {\footnotesize $\x$};
\draw (4.9,-0.3) node[] {$\mathbb{R}$};
% Enden:
\draw [line width=2.0pt,color=blue] (0,0.0)-- (1,0.0);
\draw [line width=2.0pt,color=blue] (1,0.0)-- (5,0.0);
\draw [fill = white] (0,0) circle (1.5pt);
\draw [fill = white] (1,0) circle (1.5pt);
\end{tikzpicture}
}
\end{center}

\end{MExample}

\end{MXContent}


\MSubsection{Final Test}
\MLabel{M03_Abschlusstest}

\begin{MTest}{Final Test Modul 3}
\MDeclareSiteUXID{VBKM03_Abschlusstest}

\begin{MExercise}
Find the value of the parameter $\alpha$ such that the inequality $2x^2\leq x-\alpha$ 
has exactly one solution:
\begin{MExerciseItems}
\item{The parameter value is \MEquationItem{$\alpha$}{\MLParsedQuestion{10}{1/8}{3}{PMA1}}.}
\item{In this case \MEquationItem{$x$}{\MLParsedQuestion{10}{1/4}{3}{PMA2}} is the only solution
of the inequality.}
\end{MExerciseItems}
\end{MExercise}


\begin{MExercise}
Find an absolute value function $g(x)$ describing the following graph as easy as possible.

%%\MUGraphics{abs3.png}{width=0.5\linewidth}{Funktionsgraph von $g(x)$.}{width:300px}
\begin{center}
\MTikzAuto{%
\begin{tikzpicture}[x=1.4cm, y=1.9cm] 
\draw[black] (-3,0) -- (3,0) (0,-3) -- (0,3);
\foreach \x in {-3, -2, -1, 1, 2, 3}
\draw[shift={(\x,0)},color=black] (0pt,0pt) -- (0pt,-3.0pt) node[below=1.0pt] {\normalsize $\x$};
\foreach \x in {-3.0, -2.8, ..., 3.0}
\draw[shift={(\x,0)},color=black] (0pt,0pt) -- (0pt,-1.5pt);
\foreach \y in {-3, -2, -1, 1, 2, 3}
\draw[shift={(0,\y)},color=black] (0pt,0pt) -- (-3.0pt,0pt) node[left=1.0pt] {\normalsize $\y$};
\foreach \y in {-3.0, -2.8, ..., 3.0}
\draw[shift={(0,\y)},color=black] (0pt,0pt) -- (-1.5pt,0pt);
%%\draw[black] (-0.0pt,-0.5pt) node[anchor=north east] {\small $0$};
\clip(-3.0,-3.0) rectangle (3.0,3.0);
\draw[black, line width=1.0pt,color=black] (-3,-3) -- (1,1) -- (2,4);
\end{tikzpicture}
}
\par
Graph of the function $g(x)$.
\end{center}
Try to find a representation of the form $g(x)=|x+a|+b x+c$. 
The kink in the graph indicates how the absolute value term looks like.

\begin{MExerciseItems}
\item{Find the solution set of the inequality $g(x)\leq x$ by means of the graph.\\
The solution set is \MEquationItem{$\ML$}{\MLIntervalQuestion{20}{(-infty;1]}{5}{AUX1}}.}
\item{\MEquationItem{$g(x)$}{\MLSimplifyQuestion{20}{abs(x-1)+2*x-1}{10}{x}{10}{0}{SIMPLE3}}. 
\\\MInputHint{Absolute values can be entered in the form \texttt{betrag(x-a)} or \texttt{abs(x-a)}.}}
\end{MExerciseItems}
\end{MExercise}

\begin{MExercise}
Which positive real numbers $x$ satisfy the following inequalities?
\begin{MExerciseItems}
\item{$|3x-6|\leq x+2$ has the solution set
\MEquationItem{$\ML$}{\MLIntervalQuestion{16}{[1,4]}{4}{COSH1}} (written as an interval).}
\item{$\frac{x+1}{x-1}\geq 2$ has the solution set 
\MEquationItem{$\ML$}{\MLIntervalQuestion{16}{(1,3]}{4}{COSH2}} (written as an interval).}
\end{MExerciseItems}
\MInputHint{Enter open intervals in the form $(3;5)$, closed intervals in the form 
$[3;5]$. Infinity can be entered a a word or shortly a \texttt{infty}. Do not use 
notation $]a;b[$ for open intervals. Sets can be entered by listing the elements
 $\lbrace 1;2;3\rbrace$. For the set brackets enter AltGr+7 or AltGr+0, respectively.}
\end{MExercise}

\end{MTest}

\newpage
\MPrintIndex

\end{document}

%% MINTMOD Version P0.1.0, needs to be consistent with preprocesser object in tex2x and MPragma-Version at the end of this file

% Parameter aus Konvertierungsprozess (PDF und HTML-Erzeugung wenn vom Konverter aus gestartet) werden hier eingefuegt, Preambleincludes werden am Schluss angehaengt

\newif\ifttm                % gesetzt falls Uebersetzung in HTML stattfindet, sonst uebersetzung in PDF

% Wahl der Notationsvariante ist im PDF immer std, in der HTML-Uebersetzung wird vom Konverter die Auswahl modifiziert
\newif\ifvariantstd
\newif\ifvariantunotation
\variantstdtrue % Diese Zeile wird vom Konverter erkannt und ggf. modifiziert, daher nicht veraendern!


\def\MOutputDVI{1}
\def\MOutputPDF{2}
\def\MOutputHTML{3}
\newcounter{MOutput}

\ifttm
\usepackage{german}
\usepackage{array}
\usepackage{amsmath}
\usepackage{amssymb}
\usepackage{amsthm}
\else
\documentclass[ngerman,oneside]{scrbook}
\usepackage{etex}
\usepackage[latin1]{inputenc}
\usepackage{textcomp}
\usepackage[ngerman]{babel}
\usepackage[pdftex]{color}
\usepackage{xcolor}
\usepackage{graphicx}
\usepackage[all]{xy}
\usepackage{fancyhdr}
\usepackage{verbatim}
\usepackage{array}
\usepackage{float}
\usepackage{makeidx}
\usepackage{amsmath}
\usepackage{amstext}
\usepackage{amssymb}
\usepackage{amsthm}
\usepackage[ngerman]{varioref}
\usepackage{framed}
\usepackage{supertabular}
\usepackage{longtable}
\usepackage{maxpage}
\usepackage{tikz}
\usepackage{tikzscale}
\usepackage{tikz-3dplot}
\usepackage{bibgerm}
\usepackage{chemarrow}
\usepackage{polynom}
%\usepackage{draftwatermark}
\usepackage{pdflscape}
\usetikzlibrary{calc}
\usetikzlibrary{through}
\usetikzlibrary{shapes.geometric}
\usetikzlibrary{arrows}
\usetikzlibrary{intersections}
\usetikzlibrary{decorations.pathmorphing}
\usetikzlibrary{external}
\usetikzlibrary{patterns}
\usetikzlibrary{fadings}
\usepackage[colorlinks=true,linkcolor=blue]{hyperref} 
\usepackage[all]{hypcap}
%\usepackage[colorlinks=true,linkcolor=blue,bookmarksopen=true]{hyperref} 
\usepackage{ifpdf}

\usepackage{movie15}

\setcounter{tocdepth}{2} % In Inhaltsverzeichnis bis subsection
\setcounter{secnumdepth}{3} % Nummeriert bis subsubsection

\setlength{\LTpost}{0pt} % Fuer longtable
\setlength{\parindent}{0pt}
\setlength{\parskip}{8pt}
%\setlength{\parskip}{9pt plus 2pt minus 1pt}
\setlength{\abovecaptionskip}{-0.25ex}
\setlength{\belowcaptionskip}{-0.25ex}
\fi

\ifttm
\newcommand{\MDebugMessage}[1]{\special{html:<!-- debugprint;;}#1\special{html:; //-->}}
\else
%\newcommand{\MDebugMessage}[1]{\immediate\write\mintlog{#1}}
\newcommand{\MDebugMessage}[1]{}
\fi

\def\MPageHeaderDef{%
\pagestyle{fancy}%
\fancyhead[r]{(C) VE\&MINT-Projekt}
\fancyfoot[c]{\thepage\\--- CCL BY-SA 3.0 ---}
}


\ifttm%
\def\MRelax{}%
\else%
\def\MRelax{\relax}%
\fi%

%--------------------------- Uebernahme von speziellen XML-Versionen einiger LaTeX-Kommandos aus xmlbefehle.tex vom alten Kasseler Konverter ---------------

\newcommand{\MSep}{\left\|{\phantom{\frac1g}}\right.}

\newcommand{\ML}{L}

\newcommand{\MGGT}{\mathrm{ggT}}


\ifttm
% Verhindert dass die subsection-nummer doppelt in der toccaption auftaucht (sollte ggf. in toccaption gefixt werden so dass diese Ueberschreibung nicht notwendig ist)
\renewcommand{\thesubsection}{}
% Kommandos die ttm nicht kennt
\newcommand{\binomial}[2]{{#1 \choose #2}} %  Binomialkoeffizienten
\newcommand{\eur}{\begin{html}&euro;\end{html}}
\newcommand{\square}{\begin{html}&square;\end{html}}
\newcommand{\glqq}{"'}  \newcommand{\grqq}{"'}
\newcommand{\nRightarrow}{\special{html: &nrArr; }}
\newcommand{\nmid}{\special{html: &nmid; }}
\newcommand{\nparallel}{\begin{html}&nparallel;\end{html}}
\newcommand{\mapstoo}{\begin{html}<mo>&map;</mo>\end{html}}

% Schnitt und Vereinigungssymbole von Mengen haben zu kleine Abstaende; korrigiert:
\newcommand{\ccup}{\,\!\cup\,\!}
\newcommand{\ccap}{\,\!\cap\,\!}


% Umsetzung von mathbb im HTML
\renewcommand{\mathbb}[1]{\begin{html}<mo>&#1opf;</mo>\end{html}}
\fi

%---------------------- Strukturierung ----------------------------------------------------------------------------------------------------------------------

%---------------------- Kapselung des sectioning findet auf drei Ebenen statt:
% 1. Die LateX-Befehl
% 2. Die D-Versionen der Befehle, die nur die Grade der Abschnitte umhaengen falls notwendig
% 3. Die M-Versionen der Befehle, die zusaetzliche Formatierungen vornehmen, Skripten starten und das HTML codieren
% Im Modultext duerfen nur die M-Befehle verwendet werden!

\ifttm

  \def\Dsubsubsubsection#1{\subsubsubsection{#1}}
  \def\Dsubsubsection#1{\subsubsection{#1}\addtocounter{subsubsection}{1}} % ttm-Fehler korrigieren
  \def\Dsubsection#1{\subsection{#1}}
  \def\Dsection#1{\section{#1}} % Im HTML wird nur der Sektionstitel gegeben
  \def\Dchapter#1{\chapter{#1}}
  \def\Dsubsubsubsectionx#1{\subsubsubsection*{#1}}
  \def\Dsubsubsectionx#1{\subsubsection*{#1}}
  \def\Dsubsectionx#1{\subsection*{#1}}
  \def\Dsectionx#1{\section*{#1}}
  \def\Dchapterx#1{\chapter*{#1}}

\else

  \def\Dsubsubsubsection#1{\subsubsection{#1}}
  \def\Dsubsubsection#1{\subsection{#1}}
  \def\Dsubsection#1{\section{#1}}
  \def\Dsection#1{\chapter{#1}}
  \def\Dchapter#1{\title{#1}}
  \def\Dsubsubsubsectionx#1{\subsubsection*{#1}}
  \def\Dsubsubsectionx#1{\subsection*{#1}}
  \def\Dsubsectionx#1{\section*{#1}}
  \def\Dsectionx#1{\chapter*{#1}}

\fi

\newcommand{\MStdPoints}{4}
\newcommand{\MSetPoints}[1]{\renewcommand{\MStdPoints}{#1}}

% Befehl zum Abbruch der Erstellung (nur PDF)
\newcommand{\MAbort}[1]{\err{#1}}

% Prefix vor Dateieinbindungen, wird in der Baumdatei mit \renewcommand modifiziert
% und auf das Verzeichnisprefix gesetzt, in dem das gerade bearbeitete tex-Dokument liegt.
% Im HTML wird es auf das Verzeichnis der HTML-Datei gesetzt.
% Das Prefix muss mit / enden !
\newcommand{\MDPrefix}{.}

% MRegisterFile notiert eine Datei zur Einbindung in den HTML-Baum. Grafiken mit MGraphics werden automatisch eingebunden.
% Mit MLastFile erhaelt man eine Markierung fuer die zuletzt registrierte Datei.
% Diese Markierung wird im postprocessing durch den physikalischen Dateinamen ersetzt, aber nur den Namen (d.h. \MMaterial gehoert noch davor, vgl Definition von MGraphics)
% Parameter: Pfad/Name der Datei bzw. des Ordners, relativ zur Position des Modul-Tex-Dokuments.
\ifttm
\newcommand{\MRegisterFile}[1]{\addtocounter{MFileNumber}{1}\special{html:<!-- registerfile;;}#1\special{html:;;}\MDPrefix\special{html:;;}\arabic{MFileNumber}\special{html:; //-->}}
\else
\newcommand{\MRegisterFile}[1]{\addtocounter{MFileNumber}{1}}
\fi

% Testen welcher Uebersetzer hier am Werk ist

\ifttm
\setcounter{MOutput}{3}
\else
\ifx\pdfoutput\undefined
  \pdffalse
  \setcounter{MOutput}{\MOutputDVI}
  \message{Verarbeitung mit latex, Ausgabe in dvi.}
\else
  \setcounter{MOutput}{\MOutputPDF}
  \message{Verarbeitung mit pdflatex, Ausgabe in pdf.}
  \ifnum \pdfoutput=0
    \pdffalse
  \setcounter{MOutput}{\MOutputDVI}
  \message{Verarbeitung mit pdflatex, Ausgabe in dvi.}
  \else
    \ifnum\pdfoutput=1
    \pdftrue
  \setcounter{MOutput}{\MOutputPDF}
  \message{Verarbeitung mit pdflatex, Ausgabe in pdf.}
    \fi
  \fi
\fi
\fi

\ifnum\value{MOutput}=\MOutputPDF
\DeclareGraphicsExtensions{.pdf,.png,.jpg}
\fi

\ifnum\value{MOutput}=\MOutputDVI
\DeclareGraphicsExtensions{.eps,.png,.jpg}
\fi

\ifnum\value{MOutput}=\MOutputHTML
% Wird vom Konverter leider nicht erkannt und daher in split.pm hardcodiert!
\DeclareGraphicsExtensions{.png,.jpg,.gif}
\fi

% Umdefinition der hyperref-Nummerierung im PDF-Modus
\ifttm
\else
\renewcommand{\theHfigure}{\arabic{chapter}.\arabic{section}.\arabic{figure}}
\fi

% Makro, um in der HTML-Ausgabe die zuerst zu oeffnende Datei zu kennzeichnen
\ifttm
\newcommand{\MGlobalStart}{\special{html:<!-- mglobalstarttag -->}}
\else
\newcommand{\MGlobalStart}{}
\fi

% Makro, um bei scormlogin ein pullen des Benutzers bei Aufruf der Seite zu erzwingen (typischerweise auf der Einstiegsseite)
\ifttm
\newcommand{\MPullSite}{\special{html:<!-- pullsite //-->}}
\else
\newcommand{\MPullSite}{}
\fi

% Makro, um in der HTML-Ausgabe die Kapiteluebersicht zu kennzeichnen
\ifttm
\newcommand{\MGlobalChapterTag}{\special{html:<!-- mglobalchaptertag -->}}
\else
\newcommand{\MGlobalChapterTag}{}
\fi

% Makro, um in der HTML-Ausgabe die Konfiguration zu kennzeichnen
\ifttm
\newcommand{\MGlobalConfTag}{\special{html:<!-- mglobalconfigtag -->}}
\else
\newcommand{\MGlobalConfTag}{}
\fi

% Makro, um in der HTML-Ausgabe die Standortbeschreibung zu kennzeichnen
\ifttm
\newcommand{\MGlobalLocationTag}{\special{html:<!-- mgloballocationtag -->}}
\else
\newcommand{\MGlobalLocationTag}{}
\fi

% Makro, um in der HTML-Ausgabe die persoenlichen Daten zu kennzeichnen
\ifttm
\newcommand{\MGlobalDataTag}{\special{html:<!-- mglobaldatatag -->}}
\else
\newcommand{\MGlobalDataTag}{}
\fi

% Makro, um in der HTML-Ausgabe die Suchseite zu kennzeichnen
\ifttm
\newcommand{\MGlobalSearchTag}{\special{html:<!-- mglobalsearchtag -->}}
\else
\newcommand{\MGlobalSearchTag}{}
\fi

% Makro, um in der HTML-Ausgabe die Favoritenseite zu kennzeichnen
\ifttm
\newcommand{\MGlobalFavoTag}{\special{html:<!-- mglobalfavoritestag -->}}
\else
\newcommand{\MGlobalFavoTag}{}
\fi

% Makro, um in der HTML-Ausgabe die Eingangstestseite zu kennzeichnen
\ifttm
\newcommand{\MGlobalSTestTag}{\special{html:<!-- mglobalstesttag -->}}
\else
\newcommand{\MGlobalSTestTag}{}
\fi

% Makro, um in der PDF-Ausgabe ein Wasserzeichen zu definieren
\ifttm
\newcommand{\MWatermarkSettings}{\relax}
\else
\newcommand{\MWatermarkSettings}{%
% \SetWatermarkText{(c) MINT-Kolleg Baden-W�rttemberg 2014}
% \SetWatermarkLightness{0.85}
% \SetWatermarkScale{1.5}
}
\fi

\ifttm
\newcommand{\MBinom}[2]{\left({\begin{array}{c} #1 \\ #2 \end{array}}\right)}
\else
\newcommand{\MBinom}[2]{\binom{#1}{#2}}
\fi

\ifttm
\newcommand{\DeclareMathOperator}[2]{\def#1{\mathrm{#2}}}
\newcommand{\operatorname}[1]{\mathrm{#1}}
\fi

%----------------- Makros fuer die gemischte HTML/PDF-Konvertierung ------------------------------

\newcommand{\MTestName}{\relax} % wird durch Test-Umgebung gesetzt

% Fuer experimentelle Kursinhalte, die im Release-Umsetzungsvorgang eine Fehlermeldung
% produzieren sollen aber sonst normal umgesetzt werden
\newenvironment{MExperimental}{%
}{%
}

% Wird von ttm nicht richtig umgesetzt!!
\newenvironment{MExerciseItems}{%
\renewcommand\theenumi{\alph{enumi}}%
\begin{enumerate}%
}{%
\end{enumerate}%
}


\definecolor{infoshadecolor}{rgb}{0.75,0.75,0.75}
\definecolor{exmpshadecolor}{rgb}{0.875,0.875,0.875}
\definecolor{expeshadecolor}{rgb}{0.95,0.95,0.95}
\definecolor{framecolor}{rgb}{0.2,0.2,0.2}

% Bei PDF-Uebersetzung wird hinter den Start jeder Satz/Info-aehnlichen Umgebung eine leere mbox gesetzt, damit
% fuehrende Listen oder enums nicht den Zeilenumbruch kaputtmachen
%\ifttm
\def\MTB{}
%\else
%\def\MTB{\mbox{}}
%\fi


\ifttm
\newcommand{\MRelates}{\special{html:<mi>&wedgeq;</mi>}}
\else
\def\MRelates{\stackrel{\scriptscriptstyle\wedge}{=}}
\fi

\def\MInch{\text{''}}
\def\Mdd{\textit{''}}

\ifttm
\def\MNL{ \newline }
\newenvironment{MArray}[1]{\begin{array}{#1}}{\end{array}}
\else
\def\MNL{ \\ }
\newenvironment{MArray}[1]{\begin{array}{#1}}{\end{array}}
\fi

\newcommand{\MBox}[1]{$\mathrm{#1}$}
\newcommand{\MMBox}[1]{\mathrm{#1}}


\ifttm%
\newcommand{\Mtfrac}[2]{{\textstyle \frac{#1}{#2}}}
\newcommand{\Mdfrac}[2]{{\displaystyle \frac{#1}{#2}}}
\newcommand{\Mmeasuredangle}{\special{html:<mi>&angmsd;</mi>}}
\else%
\newcommand{\Mtfrac}[2]{\tfrac{#1}{#2}}
\newcommand{\Mdfrac}[2]{\dfrac{#1}{#2}}
\newcommand{\Mmeasuredangle}{\measuredangle}
\relax
\fi

% Matrizen und Vektoren

% Inhalt wird in der Form a & b \\ c & d erwartet
% Vorsicht: MVector = Komponentenspalte, MVec = Variablensymbol
\ifttm%
\newcommand{\MVector}[1]{\left({\begin{array}{c}#1\end{array}}\right)}
\else%
\newcommand{\MVector}[1]{\begin{pmatrix}#1\end{pmatrix}}
\fi



\newcommand{\MVec}[1]{\vec{#1}}
\newcommand{\MDVec}[1]{\overrightarrow{#1}}

%----------------- Umgebungen fuer Definitionen und Saetze ----------------------------------------

% Fuegt einen Tabellen-Zeilenumbruch ein im PDF, aber nicht im HTML
\newcommand{\TSkip}{\ifttm \else&\ \\\fi}

\newenvironment{infoshaded}{%
\def\FrameCommand{\fboxsep=\FrameSep \fcolorbox{framecolor}{infoshadecolor}}%
\MakeFramed {\advance\hsize-\width \FrameRestore}}%
{\endMakeFramed}

\newenvironment{expeshaded}{%
\def\FrameCommand{\fboxsep=\FrameSep \fcolorbox{framecolor}{expeshadecolor}}%
\MakeFramed {\advance\hsize-\width \FrameRestore}}%
{\endMakeFramed}

\newenvironment{exmpshaded}{%
\def\FrameCommand{\fboxsep=\FrameSep \fcolorbox{framecolor}{exmpshadecolor}}%
\MakeFramed {\advance\hsize-\width \FrameRestore}}%
{\endMakeFramed}

\def\STDCOLOR{black}

\ifttm%
\else%
\newtheoremstyle{MSatzStyle}
  {1cm}                   %Space above
  {1cm}                   %Space below
  {\normalfont\itshape}   %Body font
  {}                      %Indent amount (empty = no indent,
                          %\parindent = para indent)
  {\normalfont\bfseries}  %Thm head font
  {}                      %Punctuation after thm head
  {\newline}              %Space after thm head: " " = normal interword
                          %space; \newline = linebreak
  {\thmname{#1}\thmnumber{ #2}\thmnote{ (#3)}}
                          %Thm head spec (can be left empty, meaning
                          %`normal')
                          %
\newtheoremstyle{MDefStyle}
  {1cm}                   %Space above
  {1cm}                   %Space below
  {\normalfont}           %Body font
  {}                      %Indent amount (empty = no indent,
                          %\parindent = para indent)
  {\normalfont\bfseries}  %Thm head font
  {}                      %Punctuation after thm head
  {\newline}              %Space after thm head: " " = normal interword
                          %space; \newline = linebreak
  {\thmname{#1}\thmnumber{ #2}\thmnote{ (#3)}}
                          %Thm head spec (can be left empty, meaning
                          %`normal')
\fi%

\newcommand{\MInfoText}{Info}

\newcounter{MHintCounter}
\newcounter{MCodeEditCounter}

\newcounter{MLastIndex}  % Enthaelt die dritte Stelle (Indexnummer) des letzten angelegten Objekts
\newcounter{MLastType}   % Enthaelt den Typ des letzten angelegten Objekts (mithilfe der unten definierten Konstanten). Die Entscheidung, wie der Typ dargstellt wird, wird in split.pm beim Postprocessing getroffen.
\newcounter{MLastTypeEq} % =1 falls das Label in einer Matheumgebung (equation, eqnarray usw.) steht, =2 falls das Label in einer table-Umgebung steht

% Da ttm keine Zahlmakros verarbeiten kann, werden diese Nummern in den Zuweisungen hardcodiert!
\def\MTypeSection{1}          %# Zaehler ist section
\def\MTypeSubsection{2}       %# Zaehler ist subsection
\def\MTypeSubsubsection{3}    %# Zaehler ist subsubsection
\def\MTypeInfo{4}             %# Eine Infobox, Separatzaehler fuer die Chemie (auch wenn es dort nicht nummeriert wird) ist MInfoCounter
\def\MTypeExercise{5}         %# Eine Aufgabe, Separatzaehler fuer die Chemie ist MExerciseCounter
\def\MTypeExample{6}          %# Eine Beispielbox, Separatzaehler fuer die Chemie ist MExampleCounter
\def\MTypeExperiment{7}       %# Eine Versuchsbox, Separatzaehler fuer die Chemie ist MExperimentCounter
\def\MTypeGraphics{8}         %# Eine Graphik, Separatzaehler fuer alle FB ist MGraphicsCounter
\def\MTypeTable{9}            %# Eine Tabellennummer, hat keinen Zaehler da durch table gezaehlt wird
\def\MTypeEquation{10}        %# Eine Gleichungsnummer, hat keinen Zaehler da durch equation/eqnarray gezaehlt wird
\def\MTypeTheorem{11}         % Ein theorem oder xtheorem, Separatzaehler fuer die Chemie ist MTheoremCounter
\def\MTypeVideo{12}           %# Ein Video,Separatzaehler fuer alle FB ist MVideoCounter
\def\MTypeEntry{13}           %# Ein Eintrag fuer die Stichwortliste, wird nicht gezaehlt sondern erhaelt im preparsing ein unique-label 

% Zaehler fuer das Labelsystem sind prefixcounter, jeder Zaehler wird VOR dem gezaehlten Objekt inkrementiert und zaehlt daher das aktuelle Objekt
\newcounter{MInfoCounter}
\newcounter{MExerciseCounter}
\newcounter{MExampleCounter}
\newcounter{MExperimentCounter}
\newcounter{MGraphicsCounter}
\newcounter{MTableCounter}
\newcounter{MEquationCounter}  % Nur im HTML, sonst durch "equation"-counter von latex realisiert
\newcounter{MTheoremCounter}
\newcounter{MObjectCounter}   % Gemeinsamer Zaehler fuer Objekte (ausser Grafiken/Tabellen) in Mathe/Info/Physik
\newcounter{MVideoCounter}
\newcounter{MEntryCounter}

\newcounter{MTestSite} % 1 = Subsubsection ist eine Pruefungsseite, 0 = ist eine normale Seite (inkl. Hilfeseite)

\def\MCell{$\phantom{a}$}

\newenvironment{MExportExercise}{\begin{MExercise}}{\end{MExercise}} % wird von mconvert abgefangen

\def\MGenerateExNumber{%
\ifnum\value{MSepNumbers}=0%
\arabic{section}.\arabic{subsection}.\arabic{MObjectCounter}\setcounter{MLastIndex}{\value{MObjectCounter}}%
\else%
\arabic{section}.\arabic{subsection}.\arabic{MExerciseCounter}\setcounter{MLastIndex}{\value{MExerciseCounter}}%
\fi%
}%

\def\MGenerateExmpNumber{%
\ifnum\value{MSepNumbers}=0%
\arabic{section}.\arabic{subsection}.\arabic{MObjectCounter}\setcounter{MLastIndex}{\value{MObjectCounter}}%
\else%
\arabic{section}.\arabic{subsection}.\arabic{MExerciseCounter}\setcounter{MLastIndex}{\value{MExampleCounter}}%
\fi%
}%

\def\MGenerateInfoNumber{%
\ifnum\value{MSepNumbers}=0%
\arabic{section}.\arabic{subsection}.\arabic{MObjectCounter}\setcounter{MLastIndex}{\value{MObjectCounter}}%
\else%
\arabic{section}.\arabic{subsection}.\arabic{MExerciseCounter}\setcounter{MLastIndex}{\value{MInfoCounter}}%
\fi%
}%

\def\MGenerateSiteNumber{%
\arabic{section}.\arabic{subsection}.\arabic{subsubsection}%
}%

% Funktionalitaet fuer Auswahlaufgaben

\newcounter{MExerciseCollectionCounter} % = 0 falls nicht in collection-Umgebung, ansonsten Schachtelungstiefe
\newcounter{MExerciseCollectionTextCounter} % wird von MExercise-Umgebung inkrementiert und von MExerciseCollection-Umgebung auf Null gesetzt

\ifttm
% MExerciseCollection gruppiert Aufgaben, die dynamisch aus der Datenbank gezogen werden und nicht direkt in der HTML-Seite stehen
% Parameter: #1 = ID der Collection, muss eindeutig fuer alle IN DER DB VORHANDENEN collections sein unabhaengig vom Kurs
%            #2 = Optionsargument (im Moment: 1 = Iterative Auswahl, 2 = Zufallsbasierte Auswahl)
\newenvironment{MExerciseCollection}[2]{%
\addtocounter{MExerciseCollectionCounter}{1}
\setcounter{MExerciseCollectionTextCounter}{0}
\special{html:<!-- mexercisecollectionstart;;}#1\special{html:;;}#2\special{html:;; //-->}%
}{%
\special{html:<!-- mexercisecollectionstop //-->}%
\addtocounter{MExerciseCollectionCounter}{-1}
}
\else
\newenvironment{MExerciseCollection}[2]{%
\addtocounter{MExerciseCollectionCounter}{1}
\setcounter{MExerciseCollectionTextCounter}{0}
}{%
\addtocounter{MExerciseCollectionCounter}{-1}
}
\fi

% Bei Uebersetzung nach PDF werden die theorem-Umgebungen verwendet, bei Uebersetzung in HTML ein manuelles Makro
\ifttm%

  \newenvironment{MHint}[1]{  \special{html:<button name="Name_MHint}\arabic{MHintCounter}\special{html:" class="hintbutton_closed" id="MHint}\arabic{MHintCounter}\special{html:_button" %
  type="button" onclick="toggle_hint('MHint}\arabic{MHintCounter}\special{html:');">}#1\special{html:</button>}
  \special{html:<div class="hint" style="display:none" id="MHint}\arabic{MHintCounter}\special{html:"> }}{\begin{html}</div>\end{html}\addtocounter{MHintCounter}{1}}

  \newenvironment{MCOSHZusatz}{  \special{html:<button name="Name_MHint}\arabic{MHintCounter}\special{html:" class="chintbutton_closed" id="MHint}\arabic{MHintCounter}\special{html:_button" %
  type="button" onclick="toggle_hint('MHint}\arabic{MHintCounter}\special{html:');">}Weiterf�hrende Inhalte\special{html:</button>}
  \special{html:<div class="hintc" style="display:none" id="MHint}\arabic{MHintCounter}\special{html:">
  <div class="coshwarn">Diese Inhalte gehen �ber das Kursniveau hinaus und werden in den Aufgaben und Tests nicht abgefragt.</div><br />}
  \addtocounter{MHintCounter}{1}}{\begin{html}</div>\end{html}}

  
  \newenvironment{MDefinition}{\begin{definition}\setcounter{MLastIndex}{\value{definition}}\ \\}{\end{definition}}

  
  \newenvironment{MExercise}{
  \renewcommand{\MStdPoints}{4}
  \addtocounter{MExerciseCounter}{1}
  \addtocounter{MObjectCounter}{1}
  \setcounter{MLastType}{5}

  \ifnum\value{MExerciseCollectionCounter}=0\else\addtocounter{MExerciseCollectionTextCounter}{1}\special{html:<!-- mexercisetextstart;;}\arabic{MExerciseCollectionTextCounter}\special{html:;; //-->}\fi
  \special{html:<div class="aufgabe" id="ADIV_}\MGenerateExNumber\special{html:">}%
  \textbf{Aufgabe \MGenerateExNumber
  } \ \\}{
  \special{html:</div><!-- mfeedbackbutton;Aufgabe;}\arabic{MTestSite}\special{html:;}\MGenerateExNumber\special{html:; //-->}
  \ifnum\value{MExerciseCollectionCounter}=0\else\special{html:<!-- mexercisetextstop //-->}\fi
  }

  % Stellt eine Kombination aus Aufgabe, Loesungstext und Eingabefeld bereit,
  % bei der Aufgabentext und Musterloesung sowie die zugehoerigen Feldelemente
  % extern bezogen und div-aktualisiert werden, das Eingabefeld aber immer das gleiche ist.
  \newenvironment{MFetchExercise}{
  \addtocounter{MExerciseCounter}{1}
  \addtocounter{MObjectCounter}{1}
  \setcounter{MLastType}{5}

  \special{html:<div class="aufgabe" id="ADIV_}\MGenerateExNumber\special{html:">}%
  \textbf{Aufgabe \MGenerateExNumber
  } \ \\%
  \special{html:</div><div class="exfetch_text" id="ADIVTEXT_}\MGenerateExNumber\special{html:">}%
  \special{html:</div><div class="exfetch_sol" id="ADIVSOL_}\MGenerateExNumber\special{html:">}%
  \special{html:</div><div class="exfetch_input" id="ADIVINPUT_}\MGenerateExNumber\special{html:">}%
  }{
  \special{html:</div>}
  }

  \newenvironment{MExample}{
  \addtocounter{MExampleCounter}{1}
  \addtocounter{MObjectCounter}{1}
  \setcounter{MLastType}{6}
  \begin{html}
  <div class="exmp">
  <div class="exmprahmen">
  \end{html}\textbf{Beispiel
  \ifnum\value{MSepNumbers}=0
  \arabic{section}.\arabic{subsection}.\arabic{MObjectCounter}\setcounter{MLastIndex}{\value{MObjectCounter}}
  \else
  \arabic{section}.\arabic{subsection}.\arabic{MExampleCounter}\setcounter{MLastIndex}{\value{MExampleCounter}}
  \fi
  } \ \\}{\begin{html}</div>
  </div>
  \end{html}
  \special{html:<!-- mfeedbackbutton;Beispiel;}\arabic{MTestSite}\special{html:;}\MGenerateExmpNumber\special{html:; //-->}
  }

  \newenvironment{MExperiment}{
  \addtocounter{MExperimentCounter}{1}
  \addtocounter{MObjectCounter}{1}
  \setcounter{MLastType}{7}
  \begin{html}
  <div class="expe">
  <div class="experahmen">
  \end{html}\textbf{Versuch
  \ifnum\value{MSepNumbers}=0
  \arabic{section}.\arabic{subsection}.\arabic{MObjectCounter}\setcounter{MLastIndex}{\value{MObjectCounter}}
  \else
%  \arabic{MExperimentCounter}\setcounter{MLastIndex}{\value{MExperimentCounter}}
  \arabic{section}.\arabic{subsection}.\arabic{MExperimentCounter}\setcounter{MLastIndex}{\value{MExperimentCounter}}
  \fi
  } \ \\}{\begin{html}</div>
  </div>
  \end{html}}

  \newenvironment{MChemInfo}{
  \setcounter{MLastType}{4}
  \begin{html}
  <div class="info">
  <div class="inforahmen">
  \end{html}}{\begin{html}</div>
  </div>
  \end{html}}

  \newenvironment{MXInfo}[1]{
  \addtocounter{MInfoCounter}{1}
  \addtocounter{MObjectCounter}{1}
  \setcounter{MLastType}{4}
  \begin{html}
  <div class="info">
  <div class="inforahmen">
  \end{html}\textbf{#1
  \ifnum\value{MInfoNumbers}=0
  \else
    \ifnum\value{MSepNumbers}=0
    \arabic{section}.\arabic{subsection}.\arabic{MObjectCounter}\setcounter{MLastIndex}{\value{MObjectCounter}}
    \else
    \arabic{MInfoCounter}\setcounter{MLastIndex}{\value{MInfoCounter}}
    \fi
  \fi
  } \ \\}{\begin{html}</div>
  </div>
  \end{html}
  \special{html:<!-- mfeedbackbutton;Info;}\arabic{MTestSite}\special{html:;}\MGenerateInfoNumber\special{html:; //-->}
  }

  \newenvironment{MInfo}{\ifnum\value{MInfoNumbers}=0\begin{MChemInfo}\else\begin{MXInfo}{Info}\ \\ \fi}{\ifnum\value{MInfoNumbers}=0\end{MChemInfo}\else\end{MXInfo}\fi}

\else%

  \theoremstyle{MSatzStyle}
  \newtheorem{thm}{Satz}[section]
  \newtheorem{thmc}{Satz}
  \theoremstyle{MDefStyle}
  \newtheorem{defn}[thm]{Definition}
  \newtheorem{exmp}[thm]{Beispiel}
  \newtheorem{info}[thm]{\MInfoText}
  \theoremstyle{MDefStyle}
  \newtheorem{defnc}{Definition}
  \theoremstyle{MDefStyle}
  \newtheorem{exmpc}{Beispiel}[section]
  \theoremstyle{MDefStyle}
  \newtheorem{infoc}{\MInfoText}
  \theoremstyle{MDefStyle}
  \newtheorem{exrc}{Aufgabe}[section]
  \theoremstyle{MDefStyle}
  \newtheorem{verc}{Versuch}[section]
  
  \newenvironment{MFetchExercise}{}{} % kann im PDF nicht dargestellt werden
  
  \newenvironment{MExercise}{\begin{exrc}\renewcommand{\MStdPoints}{1}\MTB}{\end{exrc}}
  \newenvironment{MHint}[1]{\ \\ \underline{#1:}\\}{}
  \newenvironment{MCOSHZusatz}{\ \\ \underline{Weiterf�hrende Inhalte:}\\}{}
  \newenvironment{MDefinition}{\ifnum\value{MInfoNumbers}=0\begin{defnc}\else\begin{defn}\fi\MTB}{\ifnum\value{MInfoNumbers}=0\end{defnc}\else\end{defn}\fi}
%  \newenvironment{MExample}{\begin{exmp}}{\ \linebreak[1] \ \ \ \ $\phantom{a}$ \ \hfill $\blacklozenge$\end{exmp}}
  \newenvironment{MExample}{
    \ifnum\value{MInfoNumbers}=0\begin{exmpc}\else\begin{exmp}\fi
    \MTB
    \begin{exmpshaded}
    \ \newline
}{
    \end{exmpshaded}
    \ifnum\value{MInfoNumbers}=0\end{exmpc}\else\end{exmp}\fi
}
  \newenvironment{MChemInfo}{\begin{infoshaded}}{\end{infoshaded}}

  \newenvironment{MInfo}{\ifnum\value{MInfoNumbers}=0\begin{MChemInfo}\else\renewcommand{\MInfoText}{Info}\begin{info}\begin{infoshaded}
  \MTB
   \ \newline
    \fi
  }{\ifnum\value{MInfoNumbers}=0\end{MChemInfo}\else\end{infoshaded}\end{info}\fi}

  \newenvironment{MXInfo}[1]{
    \renewcommand{\MInfoText}{#1}
    \ifnum\value{MInfoNumbers}=0\begin{infoc}\else\begin{info}\fi%
    \MTB
    \begin{infoshaded}
    \ \newline
  }{\end{infoshaded}\ifnum\value{MInfoNumbers}=0\end{infoc}\else\end{info}\fi}

  \newenvironment{MExperiment}{
    \renewcommand{\MInfoText}{Versuch}
    \ifnum\value{MInfoNumbers}=0\begin{verc}\else\begin{info}\fi
    \MTB
    \begin{expeshaded}
    \ \newline
  }{
    \end{expeshaded}
    \ifnum\value{MInfoNumbers}=0\end{verc}\else\end{info}\fi
  }
\fi%

% MHint sollte nicht direkt fuer Loesungen benutzt werden wegen solutionselect
\newenvironment{MSolution}{\begin{MHint}{L"osung}}{\end{MHint}}

\newcounter{MCodeCounter}

\ifttm
\newenvironment{MCode}{\special{html:<!-- mcodestart -->}\ttfamily\color{blue}}{\special{html:<!-- mcodestop -->}}
\else
\newenvironment{MCode}{\begin{flushleft}\ttfamily\addtocounter{MCodeCounter}{1}}{\addtocounter{MCodeCounter}{-1}\end{flushleft}}
% Ohne color-Statement da inkompatible mit framed/shaded-Boxen aus dem framed-package
\fi

%----------------- Sonderdefinitionen fuer Symbole, die der Konverter nicht kann ----------------------------------------------

\ifttm%
\newcommand{\MUnderset}[2]{\underbrace{#2}_{#1}}%
\else%
\newcommand{\MUnderset}[2]{\underset{#1}{#2}}%
\fi%

\ifttm
\newcommand{\MThinspace}{\special{html:<mi>&#x2009;</mi>}}
\else
\newcommand{\MThinspace}{\,}
\fi

\ifttm
\newcommand{\glq}{\begin{html}&sbquo;\end{html}}
\newcommand{\grq}{\begin{html}&lsquo;\end{html}}
\newcommand{\glqq}{\begin{html}&bdquo;\end{html}}
\newcommand{\grqq}{\begin{html}&ldquo;\end{html}}
\fi

\ifttm
\newcommand{\MNdash}{\begin{html}&ndash;\end{html}}
\else
\newcommand{\MNdash}{--}
\fi

%\ifttm\def\MIU{\special{html:<mi>&#8520;</mi>}}\else\def\MIU{\mathrm{i}}\fi
\def\MIU{\mathrm{i}}
\def\MEU{e} % TU9-Onlinekurs: italic-e
%\def\MEU{\mathrm{e}} % Alte Onlinemodule: roman-e
\def\MD{d} % Kursives d in Integralen im TU9-Onlinekurs
%\def\MD{\mathrm{d}} % roman-d in den alten Onlinemodulen
\def\MDB{\|}

%zusaetzlicher Leerraum vor "\MD"
\ifttm%
\def\MDSpace{\special{html:<mi>&#x2009;</mi>}}
\else%
\def\MDSpace{\,}
\fi%
\newcommand{\MDwSp}{\MDSpace\MD}%

\ifttm
\def\Mdq{\dq}
\else
\def\Mdq{\dq}
\fi

\def\MSpan#1{\left<{#1}\right>}
\def\MSetminus{\setminus}
\def\MIM{I}

\ifttm
\newcommand{\ld}{\text{ld}}
\newcommand{\lg}{\text{lg}}
\else
\DeclareMathOperator{\ld}{ld}
%\newcommand{\lg}{\text{lg}} % in latex schon definiert
\fi


\def\Mmapsto{\ifttm\special{html:<mi>&mapsto;</mi>}\else\mapsto\fi} 
\def\Mvarphi{\ifttm\phi\else\varphi\fi}
\def\Mphi{\ifttm\varphi\else\phi\fi}
\ifttm%
\newcommand{\MEumu}{\special{html:<mi>&#x3BC;</mi>}}%
\else%
\newcommand{\MEumu}{\textrm{\textmu}}%
\fi
\def\Mvarepsilon{\ifttm\epsilon\else\varepsilon\fi}
\def\Mepsilon{\ifttm\varepsilon\else\epsilon\fi}
\def\Mvarkappa{\ifttm\kappa\else\varkappa\fi}
\def\Mkappa{\ifttm\varkappa\else\kappa\fi}
\def\Mcomplement{\ifttm\special{html:<mi>&comp;</mi>}\else\complement\fi} 
\def\MWW{\mathrm{WW}}
\def\Mmod{\ifttm\special{html:<mi>&nbsp;mod&nbsp;</mi>}\else\mod\fi} 

\ifttm%
\def\mod{\text{\;mod\;}}%
\def\MNEquiv{\special{html:<mi>&NotCongruent;</mi>}}% 
\def\MNSubseteq{\special{html:<mi>&NotSubsetEqual;</mi>}}%
\def\MEmptyset{\special{html:<mi>&empty;</mi>}}%
\def\MVDots{\special{html:<mi>&#x22EE;</mi>}}%
\def\MHDots{\special{html:<mi>&#x2026;</mi>}}%
\def\Mddag{\special{html:<mi>&#x1202;</mi>}}%
\def\sphericalangle{\special{html:<mi>&measuredangle;</mi>}}%
\def\nparallel{\special{html:<mi>&nparallel;</mi>}}%
\def\MProofEnd{\special{html:<mi>&#x25FB;</mi>}}%
\newenvironment{MProof}[1]{\underline{#1}:\MCR\MCR}{\hfill $\MProofEnd$}%
\else%
\def\MNEquiv{\not\equiv}%
\def\MNSubseteq{\not\subseteq}%
\def\MEmptyset{\emptyset}%
\def\MVDots{\vdots}%
\def\MHDots{\hdots}%
\def\Mddag{\ddag}%
\newenvironment{MProof}[1]{\begin{proof}[#1]}{\end{proof}}%
\fi%



% Spaces zum Auffuellen von Tabellenbreiten, die nur im HTML wirken
\ifttm%
\def\MTSP{\:}%
\else%
\def\MTSP{}%
\fi%

\DeclareMathOperator{\arsinh}{arsinh}
\DeclareMathOperator{\arcosh}{arcosh}
\DeclareMathOperator{\artanh}{artanh}
\DeclareMathOperator{\arcoth}{arcoth}


\newcommand{\MMathSet}[1]{\mathbb{#1}}
\def\N{\MMathSet{N}}
\def\Z{\MMathSet{Z}}
\def\Q{\MMathSet{Q}}
\def\R{\MMathSet{R}}
\def\C{\MMathSet{C}}

\newcounter{MForLoopCounter}
\newcommand{\MForLoop}[2]{\setcounter{MForLoopCounter}{#1}\ifnum\value{MForLoopCounter}=0{}\else{{#2}\addtocounter{MForLoopCounter}{-1}\MForLoop{\value{MForLoopCounter}}{#2}}\fi}

\newcounter{MSiteCounter}
\newcounter{MFieldCounter} % Kombination section.subsection.site.field ist eindeutig in allen Modulen, field alleine nicht

\newcounter{MiniMarkerCounter}

\ifttm
\newenvironment{MMiniPageP}[1]{\begin{minipage}{#1\linewidth}\special{html:<!-- minimarker;;}\arabic{MiniMarkerCounter}\special{html:;;#1; //-->}}{\end{minipage}\addtocounter{MiniMarkerCounter}{1}}
\else
\newenvironment{MMiniPageP}[1]{\begin{minipage}{#1\linewidth}}{\end{minipage}\addtocounter{MiniMarkerCounter}{1}}
\fi

\newcounter{AlignCounter}

\newcommand{\MStartJustify}{\ifttm\special{html:<!-- startalign;;}\arabic{AlignCounter}\special{html:;;justify; //-->}\fi}
\newcommand{\MStopJustify}{\ifttm\special{html:<!-- stopalign;;}\arabic{AlignCounter}\special{html:; //-->}\fi\addtocounter{AlignCounter}{1}}

\newenvironment{MJTabular}[1]{
\MStartJustify
\begin{tabular}{#1}
}{
\end{tabular}
\MStopJustify
}

\newcommand{\MImageLeft}[2]{
\begin{center}
\begin{tabular}{lc}
\MStartJustify
\begin{MMiniPageP}{0.65}
#1
\end{MMiniPageP}
\MStopJustify
&
\begin{MMiniPageP}{0.3}
#2  
\end{MMiniPageP}
\end{tabular}
\end{center}
}

\newcommand{\MImageHalf}[2]{
\begin{center}
\begin{tabular}{lc}
\MStartJustify
\begin{MMiniPageP}{0.45}
#1
\end{MMiniPageP}
\MStopJustify
&
\begin{MMiniPageP}{0.45}
#2  
\end{MMiniPageP}
\end{tabular}
\end{center}
}

\newcommand{\MBigImageLeft}[2]{
\begin{center}
\begin{tabular}{lc}
\MStartJustify
\begin{MMiniPageP}{0.25}
#1
\end{MMiniPageP}
\MStopJustify
&
\begin{MMiniPageP}{0.7}
#2  
\end{MMiniPageP}
\end{tabular}
\end{center}
}

\ifttm
\def\No{\mathbb{N}_0}
\else
\def\No{\ensuremath{\N_0}}
\fi
\def\MT{\textrm{\tiny T}}
\newcommand{\MTranspose}[1]{{#1}^{\MT}}
\ifttm
\newcommand{\MRe}{\mathsf{Re}}
\newcommand{\MIm}{\mathsf{Im}}
\else
\DeclareMathOperator{\MRe}{Re}
\DeclareMathOperator{\MIm}{Im}
\fi

\newcommand{\Mid}{\mathrm{id}}
\newcommand{\MFeinheit}{\mathrm{feinh}}

\ifttm
\newcommand{\Msubstack}[1]{\begin{array}{c}{#1}\end{array}}
\else
\newcommand{\Msubstack}[1]{\substack{#1}}
\fi

% Typen von Fragefeldern:
% 1 = Alphanumerisch, case-sensitive-Vergleich
% 2 = Ja/Nein-Checkbox, Loesung ist 0 oder 1   (OPTION = Image-id fuer Rueckmeldung)
% 3 = Reelle Zahlen Geparset
% 4 = Funktionen Geparset (mit Stuetzstellen zur ueberpruefung)

% Dieser Befehl erstellt ein interaktives Aufgabenfeld. Parameter:
% - #1 Laenge in Zeichen
% - #2 Loesungstext (alphanumerisch, case sensitive)
% - #3 AufgabenID (alphanumerisch, case sensitive)
% - #4 Typ (Kennnummer)
% - #5 String fuer Optionen (ggf. mit Semikolon getrennte Einzelstrings)
% - #6 Anzahl Punkte
% - #7 uxid (kann z.B. Loesungsstring sein)
% ACHTUNG: Die langen Zeilen bitte so lassen, Zeilenumbrueche im tex werden in div's umgesetzt
\newcommand{\MQuestionID}[7]{
\ifttm
\special{html:<!-- mdeclareuxid;;}UX#7\special{html:;;}\arabic{section}\special{html:;;}#3\special{html:;; //-->}%
\special{html:<!-- mdeclarepoints;;}\arabic{section}\special{html:;;}#3\special{html:;;}#6\special{html:;;}\arabic{MTestSite}\special{html:;;}\arabic{chapter}%
\special{html:;; //--><!-- onloadstart //-->CreateQuestionObj("}#7\special{html:",}\arabic{MFieldCounter}\special{html:,"}#2%
\special{html:","}#3\special{html:",}#4\special{html:,"}#5\special{html:",}#6\special{html:,}\arabic{MTestSite}\special{html:,}\arabic{section}%
\special{html:);<!-- onloadstop //-->}%
\special{html:<input mfieldtype="}#4\special{html:" name="Name_}#3\special{html:" id="}#3\special{html:" type="text" size="}#1\special{html:" maxlength="}#1%
\special{html:" }\ifnum\value{MGroupActive}=0\special{html:onfocus="handlerFocus(}\arabic{MFieldCounter}%
\special{html:);" onblur="handlerBlur(}\arabic{MFieldCounter}\special{html:);" onkeyup="handlerChange(}\arabic{MFieldCounter}\special{html:,0);" onpaste="handlerChange(}\arabic{MFieldCounter}\special{html:,0);" oninput="handlerChange(}\arabic{MFieldCounter}\special{html:,0);" onpropertychange="handlerChange(}\arabic{MFieldCounter}\special{html:,0);"/>}%
\special{html:<img src="images/questionmark.gif" width="20" height="20" border="0" align="absmiddle" id="}QM#3\special{html:"/>}
\else%
\special{html:onblur="handlerBlur(}\arabic{MFieldCounter}%
\special{html:);" onfocus="handlerFocus(}\arabic{MFieldCounter}\special{html:);" onkeyup="handlerChange(}\arabic{MFieldCounter}\special{html:,1);" onpaste="handlerChange(}\arabic{MFieldCounter}\special{html:,1);" oninput="handlerChange(}\arabic{MFieldCounter}\special{html:,1);" onpropertychange="handlerChange(}\arabic{MFieldCounter}\special{html:,1);"/>}%
\special{html:<img src="images/questionmark.gif" width="20" height="20" border="0" align="absmiddle" id="}QM#3\special{html:"/>}\fi%
\else%
\ifnum\value{QBoxFlag}=1\fbox{$\phantom{\MForLoop{#1}{b}}$}\else$\phantom{\MForLoop{#1}{b}}$\fi%
\fi%
}

% ACHTUNG: Die langen Zeilen bitte so lassen, Zeilenumbrueche im tex werden in div's umgesetzt
% QuestionCheckbox macht ausserhalb einer QuestionGroup keinen Sinn!
% #1 = solution (1 oder 0), ggf. mit ::smc abgetrennt auszuschliessende single-choice-boxen (UXIDs durch , getrennt), #2 = id, #3 = points, #4 = uxid
\newcommand{\MQuestionCheckbox}[4]{
\ifttm
\special{html:<!-- mdeclareuxid;;}UX#4\special{html:;;}\arabic{section}\special{html:;;}#2\special{html:;; //-->}%
\ifnum\value{MGroupActive}=0\MDebugMessage{ERROR: Checkbox Nr. \arabic{MFieldCounter}\ ist nicht in einer Kontrollgruppe, es wird niemals eine Loesung angezeigt!}\fi
\special{html: %
<!-- mdeclarepoints;;}\arabic{section}\special{html:;;}#2\special{html:;;}#3\special{html:;;}\arabic{MTestSite}\special{html:;;}\arabic{chapter}%
\special{html:;; //--><!-- onloadstart //-->CreateQuestionObj("}#4\special{html:",}\arabic{MFieldCounter}\special{html:,"}#1\special{html:","}#2\special{html:",2,"IMG}#2%
\special{html:",}#3\special{html:,}\arabic{MTestSite}\special{html:,}\arabic{section}\special{html:);<!-- onloadstop //-->}%
\special{html:<input mfieldtype="2" type="checkbox" name="Name_}#2\special{html:" id="}#2\special{html:" onchange="handlerChange(}\arabic{MFieldCounter}\special{html:,1);"/><img src="images/questionmark.gif" name="}Name_IMG#2%
\special{html:" width="20" height="20" border="0" align="absmiddle" id="}IMG#2\special{html:"/> }%
\else%
\ifnum\value{QBoxFlag}=1\fbox{$\phantom{X}$}\else$\phantom{X}$\fi%
\fi%
}

\def\MGenerateID{QFELD_\arabic{section}.\arabic{subsection}.\arabic{MSiteCounter}.QF\arabic{MFieldCounter}}

% #1 = 0/1 ggf. mit ::smc abgetrennt auszuschliessende single-choice-boxen (UXIDs durch , getrennt ohne UX), #2 = uxid ohne UX
\newcommand{\MCheckbox}[2]{
\MQuestionCheckbox{#1}{\MGenerateID}{\MStdPoints}{#2}
\addtocounter{MFieldCounter}{1}
}

% Erster Parameter: Zeichenlaenge der Eingabebox, zweiter Parameter: Loesungstext
\newcommand{\MQuestion}[2]{
\MQuestionID{#1}{#2}{\MGenerateID}{1}{0}{\MStdPoints}{#2}
\addtocounter{MFieldCounter}{1}
}

% Erster Parameter: Zeichenlaenge der Eingabebox, zweiter Parameter: Loesungstext
\newcommand{\MLQuestion}[3]{
\MQuestionID{#1}{#2}{\MGenerateID}{1}{0}{\MStdPoints}{#3}
\addtocounter{MFieldCounter}{1}
}

% Parameter: Laenge des Feldes, Loesung (wird auch geparsed), Stellen Genauigkeit hinter dem Komma, weitere Stellen werden mathematisch gerundet vor Vergleich
\newcommand{\MParsedQuestion}[3]{
\MQuestionID{#1}{#2}{\MGenerateID}{3}{#3}{\MStdPoints}{#2}
\addtocounter{MFieldCounter}{1}
}

% Parameter: Laenge des Feldes, Loesung (wird auch geparsed), Stellen Genauigkeit hinter dem Komma, weitere Stellen werden mathematisch gerundet vor Vergleich
\newcommand{\MLParsedQuestion}[4]{
\MQuestionID{#1}{#2}{\MGenerateID}{3}{#3}{\MStdPoints}{#4}
\addtocounter{MFieldCounter}{1}
}

% Parameter: Laenge des Feldes, Loesungsfunktion, Anzahl Stuetzstellen, Funktionsvariablen durch Kommata getrennt (nicht case-sensitive), Anzahl Nachkommastellen im Vergleich
\newcommand{\MFunctionQuestion}[5]{
\MQuestionID{#1}{#2}{\MGenerateID}{4}{#3;#4;#5;0}{\MStdPoints}{#2}
\addtocounter{MFieldCounter}{1}
}

% Parameter: Laenge des Feldes, Loesungsfunktion, Anzahl Stuetzstellen, Funktionsvariablen durch Kommata getrennt (nicht case-sensitive), Anzahl Nachkommastellen im Vergleich, UXID
\newcommand{\MLFunctionQuestion}[6]{
\MQuestionID{#1}{#2}{\MGenerateID}{4}{#3;#4;#5;0}{\MStdPoints}{#6}
\addtocounter{MFieldCounter}{1}
}

% Parameter: Laenge des Feldes, Loesungsintervall, Genauigkeit der Zahlenwertpruefung
\newcommand{\MIntervalQuestion}[3]{
\MQuestionID{#1}{#2}{\MGenerateID}{6}{#3}{\MStdPoints}{#2}
\addtocounter{MFieldCounter}{1}
}

% Parameter: Laenge des Feldes, Loesungsintervall, Genauigkeit der Zahlenwertpruefung, UXID
\newcommand{\MLIntervalQuestion}[4]{
\MQuestionID{#1}{#2}{\MGenerateID}{6}{#3}{\MStdPoints}{#4}
\addtocounter{MFieldCounter}{1}
}

% Parameter: Laenge des Feldes, Loesungsfunktion, Anzahl Stuetzstellen, Funktionsvariable (nicht case-sensitive), Anzahl Nachkommastellen im Vergleich, Vereinfachungsbedingung
% Vereinfachungsbedingung ist eine der Folgenden:
% 0 = Keine Vereinfachungsbedingung
% 1 = Keine Klammern (runde oder eckige) mehr im vereinfachten Ausdruck
% 2 = Faktordarstellung (Term hat Produkte als letzte Operation, Summen als vorgeschaltete Operation)
% 3 = Summendarstellung (Term hat Summen als letzte Operation, Produkte als vorgeschaltete Operation)
% Flag 512: Besondere Stuetzstellen (nur >1 und nur schwach rational), sonst symmetrisch um Nullpunkt und ganze Zahlen inkl. Null werden getroffen
\newcommand{\MSimplifyQuestion}[6]{
\MQuestionID{#1}{#2}{\MGenerateID}{4}{#3;#4;#5;#6}{\MStdPoints}{#2}
\addtocounter{MFieldCounter}{1}
}

\newcommand{\MLSimplifyQuestion}[7]{
\MQuestionID{#1}{#2}{\MGenerateID}{4}{#3;#4;#5;#6}{\MStdPoints}{#7}
\addtocounter{MFieldCounter}{1}
}

% Parameter: Laenge des Feldes, Loesung (optionaler Ausdruck), Anzahl Stuetzstellen, Funktionsvariable (nicht case-sensitive), Anzahl Nachkommastellen im Vergleich, Spezialtyp (string-id)
\newcommand{\MLSpecialQuestion}[7]{
\MQuestionID{#1}{#2}{\MGenerateID}{7}{#3;#4;#5;#6}{\MStdPoints}{#7}
\addtocounter{MFieldCounter}{1}
}

\newcounter{MGroupStart}
\newcounter{MGroupEnd}
\newcounter{MGroupActive}

\newenvironment{MQuestionGroup}{
\setcounter{MGroupStart}{\value{MFieldCounter}}
\setcounter{MGroupActive}{1}
}{
\setcounter{MGroupActive}{0}
\setcounter{MGroupEnd}{\value{MFieldCounter}}
\addtocounter{MGroupEnd}{-1}
}

\newcommand{\MGroupButton}[1]{
\ifttm
\special{html:<button name="Name_Group}\arabic{MGroupStart}\special{html:to}\arabic{MGroupEnd}\special{html:" id="Group}\arabic{MGroupStart}\special{html:to}\arabic{MGroupEnd}\special{html:" %
type="button" onclick="group_button(}\arabic{MGroupStart}\special{html:,}\arabic{MGroupEnd}\special{html:);">}#1\special{html:</button>}
\else
\phantom{#1}
\fi
}

%----------------- Makros fuer die modularisierte Darstellung ------------------------------------

\def\MyText#1{#1}

% is used internally by the conversion package, should not be used by original tex documents
\def\MOrgLabel#1{\relax}

\ifttm

% Ein MLabel wird im html codiert durch das tag <!-- mmlabel;;Labelbezeichner;;SubjectArea;;chapter;;section;;subsection;;Index;;Objekttyp; //-->
\def\MLabel#1{%
\ifnum\value{MLastType}=8%
\ifnum\value{MCaptionOn}=0%
\MDebugMessage{ERROR: Grafik \arabic{MGraphicsCounter} hat separates label: #1 (Grafiklabels sollten nur in der Caption stehen)}%
\fi
\fi
\ifnum\value{MLastType}=12%
\ifnum\value{MCaptionOn}=0%
\MDebugMessage{ERROR: Video \arabic{MVideoCounter} hat separates label: #1 (Videolabels sollten nur in der Caption stehen}%
\fi
\fi
\ifnum\value{MLastType}=10\setcounter{MLastIndex}{\value{equation}}\fi
\label{#1}\begin{html}<!-- mmlabel;;#1;;\end{html}\arabic{MSubjectArea}\special{html:;;}\arabic{chapter}\special{html:;;}\arabic{section}\special{html:;;}\arabic{subsection}\special{html:;;}\arabic{MLastIndex}\special{html:;;}\arabic{MLastType}\special{html:; //-->}}%

\else

% Sonderbehandlung im PDF fuer Abbildungen in separater aux-Datei, da MGraphics die figure-Umgebung nicht verwendet
\def\MLabel#1{%
\ifnum\value{MLastType}=8%
\ifnum\value{MCaptionOn}=0%
\MDebugMessage{ERROR: Grafik \arabic{MGraphicsCounter} hat separates label: #1 (Grafiklabels sollten nur in der Caption stehen}%
\fi
\fi
\ifnum\value{MLastType}=12%
\ifnum\value{MCaptionOn}=0%
\MDebugMessage{ERROR: Video \arabic{MVideoCounter} hat separates label: #1 (Videolabels sollten nur in der Caption stehen}%
\fi
\fi
\label{#1}%
}%

\fi

% Gibt Begriff des referenzierten Objekts mit aus, aber nur im HTML, daher nur in Ausnahmefaellen (z.B. Copyrightliste) sinnvoll
\def\MCRef#1{\ifttm\special{html:<!-- mmref;;}#1\special{html:;;1; //-->}\else\vref{#1}\fi}


\def\MRef#1{\ifttm\special{html:<!-- mmref;;}#1\special{html:;;0; //-->}\else\vref{#1}\fi}
\def\MERef#1{\ifttm\special{html:<!-- mmref;;}#1\special{html:;;0; //-->}\else\eqref{#1}\fi}
\def\MNRef#1{\ifttm\special{html:<!-- mmref;;}#1\special{html:;;0; //-->}\else\ref{#1}\fi}
\def\MSRef#1#2{\ifttm\special{html:<!-- msref;;}#1\special{html:;;}#2\special{html:; //-->}\else \if#2\empty \ref{#1} \else \hyperref[#1]{#2}\fi\fi} 

\def\MRefRange#1#2{\ifttm\MRef{#1} bis 
\MRef{#2}\else\vrefrange[\unskip]{#1}{#2}\fi}

\def\MRefTwo#1#2{\ifttm\MRef{#1} und \MRef{#2}\else%
\let\vRefTLRsav=\reftextlabelrange\let\vRefTPRsav=\reftextpagerange%
\def\reftextlabelrange##1##2{\ref{##1} und~\ref{##2}}%
\def\reftextpagerange##1##2{auf den Seiten~\pageref{#1} und~\pageref{#2}}%
\vrefrange[\unskip]{#1}{#2}%
\let\reftextlabelrange=\vRefTLRsav\let\reftextpagerange=\vRefTPRsav\fi}

% MSectionChapter definiert falls notwendig das Kapitel vor der section. Das ist notwendig, wenn nur ein Einzelmodul uebersetzt wird.
% MChaptersGiven ist ein Counter, der von mconvert.pl vordefiniert wird.
\ifttm
\newcommand{\MSectionChapter}{\ifnum\value{MChaptersGiven}=0{\Dchapter{Modul}}\else{}\fi}
\else
\newcommand{\MSectionChapter}{\ifnum\value{chapter}=0{\Dchapter{Modul}}\else{}\fi}
\fi


\def\MChapter#1{\ifnum\value{MSSEnd}>0{\MSubsectionEndMacros}\addtocounter{MSSEnd}{-1}\fi\Dchapter{#1}}
\def\MSubject#1{\MChapter{#1}} % Schluesselwort HELPSECTION ist reserviert fuer Hilfesektion

\newcommand{\MSectionID}{UNKNOWNID}

\ifttm
\newcommand{\MSetSectionID}[1]{\renewcommand{\MSectionID}{#1}}
\else
\newcommand{\MSetSectionID}[1]{\renewcommand{\MSectionID}{#1}\tikzsetexternalprefix{#1}}
\fi


\newcommand{\MSection}[1]{\MSetSectionID{MODULID}\ifnum\value{MSSEnd}>0{\MSubsectionEndMacros}\addtocounter{MSSEnd}{-1}\fi\MSectionChapter\Dsection{#1}\MSectionStartMacros{#1}\setcounter{MLastIndex}{-1}\setcounter{MLastType}{1}} % Sections werden ueber das section-Feld im mmlabel-Tag identifiziert, nicht ueber das Indexfeld

\def\MSubsection#1{\ifnum\value{MSSEnd}>0{\MSubsectionEndMacros}\addtocounter{MSSEnd}{-1}\fi\ifttm\else\clearpage\fi\Dsubsection{#1}\MSubsectionStartMacros\setcounter{MLastIndex}{-1}\setcounter{MLastType}{2}\addtocounter{MSSEnd}{1}}% Subsections werden ueber das subsection-Feld im mmlabel-Tag identifiziert, nicht ueber das Indexfeld
\def\MSubsectionx#1{\Dsubsectionx{#1}} % Nur zur Verwendung in MSectionStart gedacht
\def\MSubsubsection#1{\Dsubsubsection{#1}\setcounter{MLastIndex}{\value{subsubsection}}\setcounter{MLastType}{3}\ifttm\special{html:<!-- sectioninfo;;}\arabic{section}\special{html:;;}\arabic{subsection}\special{html:;;}\arabic{subsubsection}\special{html:;;1;;}\arabic{MTestSite}\special{html:; //-->}\fi}
\def\MSubsubsectionx#1{\Dsubsubsectionx{#1}\ifttm\special{html:<!-- sectioninfo;;}\arabic{section}\special{html:;;}\arabic{subsection}\special{html:;;}\arabic{subsubsection}\special{html:;;0;;}\arabic{MTestSite}\special{html:; //-->}\else\addcontentsline{toc}{subsection}{#1}\fi}

\ifttm
\def\MSubsubsubsectionx#1{\ \newline\textbf{#1}\special{html:<br />}}
\else
\def\MSubsubsubsectionx#1{\ \newline
\textbf{#1}\ \\
}
\fi


% Dieses Skript wird zu Beginn jedes Modulabschnitts (=Webseite) ausgefuehrt und initialisiert den Aufgabenfeldzaehler
\newcommand{\MPageScripts}{
\setcounter{MFieldCounter}{1}
\addtocounter{MSiteCounter}{1}
\setcounter{MHintCounter}{1}
\setcounter{MCodeEditCounter}{1}
\setcounter{MGroupActive}{0}
\DoQBoxes
% Feldvariablen werden im HTML-Header in conv.pl eingestellt
}

% Dieses Skript wird zum Ende jedes Modulabschnitts (=Webseite) ausgefuehrt
\ifttm
\newcommand{\MEndScripts}{\special{html:<br /><!-- mfeedbackbutton;Seite;}\arabic{MTestSite}\special{html:;}\MGenerateSiteNumber\special{html:; //-->}
}
\else
\newcommand{\MEndScripts}{\relax}
\fi


\newcounter{QBoxFlag}
\newcommand{\DoQBoxes}{\setcounter{QBoxFlag}{1}}
\newcommand{\NoQBoxes}{\setcounter{QBoxFlag}{0}}

\newcounter{MXCTest}
\newcounter{MXCounter}
\newcounter{MSCounter}



\ifttm

% Struktur des sectioninfo-Tags: <!-- sectioninfo;;section;;subsection;;subsubsection;;nr_ausgeben;;testpage; //-->

%Fuegt eine zusaetzliche html-Seite an hinter ALLEN bisherigen und zukuenftigen content-Seiten ausserhalb der vor-zurueck-Schleife (d.h. nur durch Button oder MIntLink erreichbar!)
% #1 = Titel des Modulabschnitts, #2 = Kurztitel fuer die Buttons, #3 = Buttonkennung (STD = default nehmen, NONE = Ohne Button in der Navigation)
\newenvironment{MSContent}[3]{\special{html:<div class="xcontent}\arabic{MSCounter}\special{html:"><!-- scontent;-;}\arabic{MSCounter};-;#1;-;#2;-;#3\special{html: //-->}\MPageScripts\MSubsubsectionx{#1}}{\MEndScripts\special{html:<!-- endscontent;;}\arabic{MSCounter}\special{html: //--></div>}\addtocounter{MSCounter}{1}}

% Fuegt eine zusaetzliche html-Seite ein hinter den bereits vorhandenen content-Seiten (oder als erste Seite) innerhalb der vor-zurueck-Schleife der Navigation
% #1 = Titel des Modulabschnitts, #2 = Kurztitel fuer die Buttons, #3 = Buttonkennung (STD = Defaultbutton, NONE = Ohne Button in der Navigation)
\newenvironment{MXContent}[3]{\special{html:<div class="xcontent}\arabic{MXCounter}\special{html:"><!-- xcontent;-;}\arabic{MXCounter};-;#1;-;#2;-;#3\special{html: //-->}\MPageScripts\MSubsubsection{#1}}{\MEndScripts\special{html:<!-- endxcontent;;}\arabic{MXCounter}\special{html: //--></div>}\addtocounter{MXCounter}{1}}

% Fuegt eine zusaetzliche html-Seite ein die keine subsubsection-Nummer bekommt, nur zur internen Verwendung in mintmod.tex gedacht!
% #1 = Titel des Modulabschnitts, #2 = Kurztitel fuer die Buttons, #3 = Buttonkennung (STD = Defaultbutton, NONE = Ohne Button in der Navigation)
% \newenvironment{MUContent}[3]{\special{html:<div class="xcontent}\arabic{MXCounter}\special{html:"><!-- xcontent;-;}\arabic{MXCounter};-;#1;-;#2;-;#3\special{html: //-->}\MPageScripts\MSubsubsectionx{#1}}{\MEndScripts\special{html:<!-- endxcontent;;}\arabic{MXCounter}\special{html: //--></div>}\addtocounter{MXCounter}{1}}

\newcommand{\MDeclareSiteUXID}[1]{\special{html:<!-- mdeclaresiteuxid;;}#1\special{html:;;}\arabic{chapter}\special{html:;;}\arabic{section}\special{html:;; //-->}}

\else

%\newcommand{\MSubsubsection}[1]{\refstepcounter{subsubsection} \addcontentsline{toc}{subsubsection}{\thesubsubsection. #1}}


% Fuegt eine zusaetzliche html-Seite an hinter den bereits vorhandenen content-Seiten
% #1 = Titel des Modulabschnitts, #2 = Kurztitel fuer die Buttons, #3 = Iconkennung (im PDF wirkungslos)
%\newenvironment{MUContent}[3]{\ifnum\value{MXCTest}>0{\MDebugMessage{ERROR: Geschachtelter SContent}}\fi\MPageScripts\MSubsubsectionx{#1}\addtocounter{MXCTest}{1}}{\addtocounter{MXCounter}{1}\addtocounter{MXCTest}{-1}}
\newenvironment{MXContent}[3]{\ifnum\value{MXCTest}>0{\MDebugMessage{ERROR: Geschachtelter SContent}}\fi\MPageScripts\MSubsubsection{#1}\addtocounter{MXCTest}{1}}{\addtocounter{MXCounter}{1}\addtocounter{MXCTest}{-1}}
\newenvironment{MSContent}[3]{\ifnum\value{MXCTest}>0{\MDebugMessage{ERROR: Geschachtelter XContent}}\fi\MPageScripts\MSubsubsectionx{#1}\addtocounter{MXCTest}{1}}{\addtocounter{MSCounter}{1}\addtocounter{MXCTest}{-1}}

\newcommand{\MDeclareSiteUXID}[1]{\relax}

\fi 

% GHEADER und GFOOTER werden von split.pm gefunden, aber nur, wenn nicht HELPSITE oder TESTSITE
\ifttm
\newenvironment{MSectionStart}{\special{html:<div class="xcontent0">}\MSubsubsectionx{Modul\"ubersicht}}{\setcounter{MSSEnd}{0}\special{html:</div>}}
% Darf nicht als XContent nummeriert werden, darf nicht als XContent gelabelt werden, wird aber in eine xcontent-div gesetzt fuer Python-parsing
\else
\newenvironment{MSectionStart}{\MSubsectionx{Modul\"ubersicht}}{\setcounter{MSSEnd}{0}}
\fi

\newenvironment{MIntro}{\begin{MXContent}{Einf\"uhrung}{Einf\"uhrung}{genetisch}}{\end{MXContent}}
\newenvironment{MContent}{\begin{MXContent}{Inhalt}{Inhalt}{beweis}}{\end{MXContent}}
\newenvironment{MExercises}{\ifttm\else\clearpage\fi\begin{MXContent}{Aufgaben}{Aufgaben}{aufgb}\special{html:<!-- declareexcsymb //-->}}{\end{MXContent}}

% #1 = Lesbare Testbezeichnung
\newenvironment{MTest}[1]{%
\renewcommand{\MTestName}{#1}
\ifttm\else\clearpage\fi%
\addtocounter{MTestSite}{1}%
\begin{MXContent}{#1}{#1}{STD} % {aufgb}%
\special{html:<!-- declaretestsymb //-->}
\begin{MQuestionGroup}%
\MInTestHeader
}%
{%
\end{MQuestionGroup}%
\ \\ \ \\%
\MInTestFooter
\end{MXContent}\addtocounter{MTestSite}{-1}%
}

\newenvironment{MExtra}{\ifttm\else\clearpage\fi\begin{MXContent}{Zus\"atzliche Inhalte}{Zusatz}{weiterfhrg}}{\end{MXContent}}

\makeindex

\ifttm
\def\MPrintIndex{
\ifnum\value{MSSEnd}>0{\MSubsectionEndMacros}\addtocounter{MSSEnd}{-1}\fi
\renewcommand{\indexname}{Stichwortverzeichnis}
\special{html:<p><!-- printindex //--></p>}
}
\else
\def\MPrintIndex{
\ifnum\value{MSSEnd}>0{\MSubsectionEndMacros}\addtocounter{MSSEnd}{-1}\fi
\renewcommand{\indexname}{Stichwortverzeichnis}
\addcontentsline{toc}{section}{Stichwortverzeichnis}
\printindex
}
\fi


% Konstanten fuer die Modulfaecher

\def\MINTMathematics{1}
\def\MINTInformatics{2}
\def\MINTChemistry{3}
\def\MINTPhysics{4}
\def\MINTEngineering{5}

\newcounter{MSubjectArea}
\newcounter{MInfoNumbers} % Gibt an, ob die Infoboxen nummeriert werden sollen
\newcounter{MSepNumbers} % Gibt an, ob Beispiele und Experimente separat nummeriert werden sollen
\newcommand{\MSetSubject}[1]{
 % ttm kapiert setcounter mit Parametern nicht, also per if abragen und einsetzen
\ifnum#1=1\setcounter{MSubjectArea}{1}\setcounter{MInfoNumbers}{1}\setcounter{MSepNumbers}{0}\fi
\ifnum#1=2\setcounter{MSubjectArea}{2}\setcounter{MInfoNumbers}{1}\setcounter{MSepNumbers}{0}\fi
\ifnum#1=3\setcounter{MSubjectArea}{3}\setcounter{MInfoNumbers}{0}\setcounter{MSepNumbers}{1}\fi
\ifnum#1=4\setcounter{MSubjectArea}{4}\setcounter{MInfoNumbers}{0}\setcounter{MSepNumbers}{0}\fi
\ifnum#1=5\setcounter{MSubjectArea}{5}\setcounter{MInfoNumbers}{1}\setcounter{MSepNumbers}{0}\fi
% Separate Nummerntechnik fuer unsere Chemiker: alles dreistellig
\ifnum#1=3
  \ifttm
  \renewcommand{\theequation}{\arabic{section}.\arabic{subsection}.\arabic{equation}}
  \renewcommand{\thetable}{\arabic{section}.\arabic{subsection}.\arabic{table}} 
  \renewcommand{\thefigure}{\arabic{section}.\arabic{subsection}.\arabic{figure}} 
  \else
  \renewcommand{\theequation}{\arabic{chapter}.\arabic{section}.\arabic{equation}}
  \renewcommand{\thetable}{\arabic{chapter}.\arabic{section}.\arabic{table}}
  \renewcommand{\thefigure}{\arabic{chapter}.\arabic{section}.\arabic{figure}}
  \fi
\else
  \ifttm
  \renewcommand{\theequation}{\arabic{section}.\arabic{subsection}.\arabic{equation}}
  \renewcommand{\thetable}{\arabic{table}}
  \renewcommand{\thefigure}{\arabic{figure}}
  \else
  \renewcommand{\theequation}{\arabic{chapter}.\arabic{section}.\arabic{equation}}
  \renewcommand{\thetable}{\arabic{table}}
  \renewcommand{\thefigure}{\arabic{figure}}
  \fi
\fi
}

% Fuer tikz Autogenerierung
\newcounter{MTIKZAutofilenumber}

% Spezielle Counter fuer die Bentz-Module
\newcounter{mycounter}
\newcounter{chemapplet}
\newcounter{physapplet}

\newcounter{MSSEnd} % Ist 1 falls ein MSubsection aktiv ist, der einen MSubsectionEndMacro-Aufruf verursacht
\newcounter{MFileNumber}
\def\MLastFile{\special{html:[[!-- mfileref;;}\arabic{MFileNumber}\special{html:; //--]]}}

% Vollstaendiger Pfad ist \MMaterial / \MLastFilePath / \MLastFileName    ==   \MMaterial / \MLastFile

% Wird nur bei kompletter Baum-Erstellung ausgefuehrt!
% #1 = Lesbare Modulbezeichnung
\newcommand{\MSectionStartMacros}[1]{
\setcounter{MTestSite}{0}
\setcounter{MCaptionOn}{0}
\setcounter{MLastTypeEq}{0}
\setcounter{MSSEnd}{0}
\setcounter{MFileNumber}{0} % Preinkrekement-Counter
\setcounter{MTIKZAutofilenumber}{0}
\setcounter{mycounter}{1}
\setcounter{physapplet}{1}
\setcounter{chemapplet}{0}
\ifttm
\special{html:<!-- mdeclaresection;;}\arabic{chapter}\special{html:;;}\arabic{section}\special{html:;;}#1\special{html:;; //-->}%
\else
\setcounter{thmc}{0}
\setcounter{exmpc}{0}
\setcounter{verc}{0}
\setcounter{infoc}{0}
\fi
\setcounter{MiniMarkerCounter}{1}
\setcounter{AlignCounter}{1}
\setcounter{MXCTest}{0}
\setcounter{MCodeCounter}{0}
\setcounter{MEntryCounter}{0}
}

% Wird immer ausgefuehrt
\newcommand{\MSubsectionStartMacros}{
\ifttm\else\MPageHeaderDef\fi
\MWatermarkSettings
\setcounter{MXCounter}{0}
\setcounter{MSCounter}{0}
\setcounter{MSiteCounter}{1}
\setcounter{MExerciseCollectionCounter}{0}
% Zaehler fuer das Labelsystem zuruecksetzen (prefix-Zaehler)
\setcounter{MInfoCounter}{0}
\setcounter{MExerciseCounter}{0}
\setcounter{MExampleCounter}{0}
\setcounter{MExperimentCounter}{0}
\setcounter{MGraphicsCounter}{0}
\setcounter{MTableCounter}{0}
\setcounter{MTheoremCounter}{0}
\setcounter{MObjectCounter}{0}
\setcounter{MEquationCounter}{0}
\setcounter{MVideoCounter}{0}
\setcounter{equation}{0}
\setcounter{figure}{0}
}

\newcommand{\MSubsectionEndMacros}{
% Bei Chemiemodulen das PSE einhaengen, es soll als SContent am Ende erscheinen
\special{html:<!-- subsectionend //-->}
\ifnum\value{MSubjectArea}=3{\MIncludePSE}\fi
}


\ifttm
%\newcommand{\MEmbed}[1]{\MRegisterFile{#1}\begin{html}<embed src="\end{html}\MMaterial/\MLastFile\begin{html}" width="192" height="189"></embed>\end{html}}
\newcommand{\MEmbed}[1]{\MRegisterFile{#1}\begin{html}<embed src="\end{html}\MMaterial/\MLastFile\begin{html}"></embed>\end{html}}
\fi

%----------------- Makros fuer die Textdarstellung -----------------------------------------------

\ifttm
% MUGraphics bindet eine Grafik ein:
% Parameter 1: Dateiname der Grafik, relativ zur Position des Modul-Tex-Dokuments
% Parameter 2: Skalierungsoptionen fuer PDF (fuer includegraphics)
% Parameter 3: Titel fuer die Grafik, wird unter die Grafik mit der Grafiknummer gesetzt und kann MLabel bzw. MCopyrightLabel enthalten
% Parameter 4: Skalierungsoptionen fuer HTML (css-styles)

% ERSATZ: <img alt="My Image" src="data:image/png;base64,iVBORwA<MoreBase64SringHere>" />


\newcommand{\MUGraphics}[4]{\MRegisterFile{#1}\begin{html}
<div class="imagecenter">
<center>
<div>
<img src="\end{html}\MMaterial/\MLastFile\begin{html}" style="#4" alt="\end{html}\MMaterial/\MLastFile\begin{html}"/>
</div>
<div class="bildtext">
\end{html}
\addtocounter{MGraphicsCounter}{1}
\setcounter{MLastIndex}{\value{MGraphicsCounter}}
\setcounter{MLastType}{8}
\addtocounter{MCaptionOn}{1}
\ifnum\value{MSepNumbers}=0
\textbf{Abbildung \arabic{MGraphicsCounter}:} #3
\else
\textbf{Abbildung \arabic{section}.\arabic{subsection}.\arabic{MGraphicsCounter}:} #3
\fi
\addtocounter{MCaptionOn}{-1}
\begin{html}
</div>
</center>
</div>
<br />
\end{html}%
\special{html:<!-- mfeedbackbutton;Abbildung;}\arabic{MGraphicsCounter}\special{html:;}\arabic{section}.\arabic{subsection}.\arabic{MGraphicsCounter}\special{html:; //-->}%
}

% MVideo bindet ein Video als Einzeldatei ein:
% Parameter 1: Dateiname des Videos, relativ zur Position des Modul-Tex-Dokuments, ohne die Endung ".mp4"
% Parameter 2: Titel fuer das Video (kann MLabel oder MCopyrightLabel enthalten), wird unter das Video mit der Videonummer gesetzt
\newcommand{\MVideo}[2]{\MRegisterFile{#1.mp4}\begin{html}
<div class="imagecenter">
<center>
<div>
<video width="95\%" controls="controls"><source src="\end{html}\MMaterial/#1.mp4\begin{html}" type="video/mp4">Ihr Browser kann keine MP4-Videos abspielen!</video>
</div>
<div class="bildtext">
\end{html}
\addtocounter{MVideoCounter}{1}
\setcounter{MLastIndex}{\value{MVideoCounter}}
\setcounter{MLastType}{12}
\addtocounter{MCaptionOn}{1}
\ifnum\value{MSepNumbers}=0
\textbf{Video \arabic{MVideoCounter}:} #2
\else
\textbf{Video \arabic{section}.\arabic{subsection}.\arabic{MVideoCounter}:} #2
\fi
\addtocounter{MCaptionOn}{-1}
\begin{html}
</div>
</center>
</div>
<br />
\end{html}}

\newcommand{\MDVideo}[2]{\MRegisterFile{#1.mp4}\MRegisterFile{#1.ogv}\begin{html}
<div class="imagecenter">
<center>
<div>
<video width="70\%" controls><source src="\end{html}\MMaterial/#1.mp4\begin{html}" type="video/mp4"><source src="\end{html}\MMaterial/#1.ogv\begin{html}" type="video/ogg">Ihr Browser kann keine MP4-Videos abspielen!</video>
</div>
<br />
#2
</center>
</div>
<br />
\end{html}
}

\newcommand{\MGraphics}[3]{\MUGraphics{#1}{#2}{#3}{}}

\else

\newcommand{\MVideo}[2]{%
% Kein Video im PDF darstellbar, trotzdem so tun als ob da eines waere
\begin{center}
(Video nicht darstellbar)
\end{center}
\addtocounter{MVideoCounter}{1}
\setcounter{MLastIndex}{\value{MVideoCounter}}
\setcounter{MLastType}{12}
\addtocounter{MCaptionOn}{1}
\ifnum\value{MSepNumbers}=0
\textbf{Video \arabic{MVideoCounter}:} #2
\else
\textbf{Video \arabic{section}.\arabic{subsection}.\arabic{MVideoCounter}:} #2
\fi
\addtocounter{MCaptionOn}{-1}
}


% MGraphics bindet eine Grafik ein:
% Parameter 1: Dateiname der Grafik, relativ zur Position des Modul-Tex-Dokuments
% Parameter 2: Skalierungsoptionen fuer PDF (fuer includegraphics)
% Parameter 3: Titel fuer die Grafik, wird unter die Grafik mit der Grafiknummer gesetzt
\newcommand{\MGraphics}[3]{%
\MRegisterFile{#1}%
\ %
\begin{figure}[H]%
\centering{%
\includegraphics[#2]{\MDPrefix/#1}%
\addtocounter{MCaptionOn}{1}%
\caption{#3}%
\addtocounter{MCaptionOn}{-1}%
}%
\end{figure}%
\addtocounter{MGraphicsCounter}{1}\setcounter{MLastIndex}{\value{MGraphicsCounter}}\setcounter{MLastType}{8}\ %
%\ \\Abbildung \ifnum\value{MSepNumbers}=0\else\arabic{chapter}.\arabic{section}.\fi\arabic{MGraphicsCounter}: #3%
}

\newcommand{\MUGraphics}[4]{\MGraphics{#1}{#2}{#3}}


\fi

\newcounter{MCaptionOn} % = 1 falls eine Grafikcaption aktiv ist, = 0 sonst


% MGraphicsSolo bindet eine Grafik pur ein ohne Titel
% Parameter 1: Dateiname der Grafik, relativ zur Position des Modul-Tex-Dokuments
% Parameter 2: Skalierungsoptionen (wirken nur im PDF)
\newcommand{\MGraphicsSolo}[2]{\MUGraphicsSolo{#1}{#2}{}}

% MUGraphicsSolo bindet eine Grafik pur ein ohne Titel, aber mit HTML-Skalierung
% Parameter 1: Dateiname der Grafik, relativ zur Position des Modul-Tex-Dokuments
% Parameter 2: Skalierungsoptionen (wirken nur im PDF)
% Parameter 3: Skalierungsoptionen (wirken nur im HTML), als style-format: "width=???, height=???"
\ifttm
\newcommand{\MUGraphicsSolo}[3]{\MRegisterFile{#1}\begin{html}
<img src="\end{html}\MMaterial/\MLastFile\begin{html}" style="\end{html}#3\begin{html}" alt="\end{html}\MMaterial/\MLastFile\begin{html}"/>
\end{html}%
\special{html:<!-- mfeedbackbutton;Abbildung;}#1\special{html:;}\MMaterial/\MLastFile\special{html:; //-->}%
}
\else
\newcommand{\MUGraphicsSolo}[3]{\MRegisterFile{#1}\includegraphics[#2]{\MDPrefix/#1}}
\fi

% Externer Link mit URL
% Erster Parameter: Vollstaendige(!) URL des Links
% Zweiter Parameter: Text fuer den Link
\newcommand{\MExtLink}[2]{\ifttm\special{html:<a target="_new" href="}#1\special{html:">}#2\special{html:</a>}\else\href{#1}{#2}\fi} % ohne MINTERLINK!


% Interner Link, die verlinkte Datei muss im gleichen Verzeichnis liegen wie die Modul-Texdatei
% Erster Parameter: Dateiname
% Zweiter Parameter: Text fuer den Link
\newcommand{\MIntLink}[2]{\ifttm\MRegisterFile{#1}\special{html:<a class="MINTERLINK" target="_new" href="}\MMaterial/\MLastFile\special{html:">}#2\special{html:</a>}\else{\href{#1}{#2}}\fi}


\ifttm
\def\MMaterial{:localmaterial:}
\else
\def\MMaterial{\MDPrefix}
\fi

\ifttm
\def\MNoFile#1{:directmaterial:#1}
\else
\def\MNoFile#1{#1}
\fi

\newcommand{\MChem}[1]{$\mathrm{#1}$}

\newcommand{\MApplet}[3]{
% Bindet ein Java-Applet ein, die Parameter sind:
% (wird nur im HTML, aber nicht im PDF erstellt)
% #1 Dateiname des Applets (muss mit ".class" enden)
% #2 = Breite in Pixeln
% #3 = Hoehe in Pixeln
\ifttm
\MRegisterFile{#1}
\begin{html}
<applet code="\end{html}\MMaterial/\MLastFile\begin{html}" width="#2" height="#3" alt="[Java-Applet kann nicht gestartet werden]"></applet>
\end{html}
\fi
}

\newcommand{\MScriptPage}[2]{
% Bindet eine JavaScript-Datei ein, die eine eigene Seite bekommt
% (wird nur im HTML, aber nicht im PDF erstellt)
% #1 Dateiname des Programms (sollte mit ".js" enden)
% #2 = Kurztitel der Seite
\ifttm
\begin{MSContent}{#2}{#2}{puzzle}
\MRegisterFile{#1}
\begin{html}
<script src="\MMaterial/\MLastFile" type="text/javascript"></script>
\end{html}
\end{MSContent}
\fi
}

\newcommand{\MIncludePSE}{
% Bindet bei Chemie-Modulen das PSE ein
% (wird nur im HTML, aber nicht im PDF erstellt)
\ifttm
\special{html:<!-- includepse //-->}
\begin{MSContent}{Periodensystem der Elemente}{PSE}{table}
\MRegisterFile{../files/pse.js}
\MRegisterFile{../files/radio.png}
\begin{html}
<script src="\MMaterial/../files/pse.js" type="text/javascript"></script>
<p id="divid"><br /><br />
<script language="javascript" type="text/javascript">
    startpse("divid","\MMaterial/../files"); 
</script>
</p>
<br />
<br />
<br />
<p>Die Farben der Elementsymbole geben an: <font style="color:Red">gasf&ouml;rmig </font> <font style="color:Blue">fl&uuml;ssig </font> fest</p>
<p>Die Elemente der Gruppe 1 A, 2 A, 3 A usw. geh&ouml;ren zu den Hauptgruppenelementen.</p>
<p>Die Elemente der Gruppe 1 B, 2 B, 3 B usw. geh&ouml;ren zu den Nebengruppenelementen.</p>
<p>() kennzeichnet die Masse des stabilsten Isotops</p>
\end{html}
\end{MSContent}
\fi
}

\newcommand{\MAppletArchive}[4]{
% Bindet ein Java-Applet ein, die Parameter sind:
% (wird nur im HTML, aber nicht im PDF erstellt)
% #1 Dateiname der Klasse mit Appletaufruf (muss mit ".class" enden)
% #2 Dateiname des Archivs (muss mit ".jar" enden)
% #3 = Breite in Pixeln
% #4 = Hoehe in Pixeln
\ifttm
\MRegisterFile{#2}
\begin{html}
<applet code="#1" archive="\end{html}\MMaterial/\MLastFile\begin{html}" codebase="." width="#3" height="#4" alt="[Java-Archiv kann nicht gestartet werden]"></applet>
\end{html}
\fi
}

% Bindet in der Haupttexdatei ein MINT-Modul ein. Parameter 1 ist das Verzeichnis (relativ zur Haupttexdatei), Parameter 2 ist der Dateinahme ohne Pfad.
\newcommand{\IncludeModule}[2]{
\renewcommand{\MDPrefix}{#1}
\input{#1/#2}
\ifnum\value{MSSEnd}>0{\MSubsectionEndMacros}\addtocounter{MSSEnd}{-1}\fi
}

% Der ttm-Konverter setzt keine Makros im \input um, also muss hier getrickst werden:
% Das MDPrefix muss in den einzelnen Modulen manuell eingesetzt werden
\newcommand{\MInputFile}[1]{
\ifttm
\input{#1}
\else
\input{#1}
\fi
}


\newcommand{\MCases}[1]{\left\lbrace{\begin{array}{rl} #1 \end{array}}\right.}

\ifttm
\newenvironment{MCaseEnv}{\left\lbrace\begin{array}{rl}}{\end{array}\right.}
\else
\newenvironment{MCaseEnv}{\left\lbrace\begin{array}{rl}}{\end{array}\right.}
\fi

\def\MSkip{\ifttm\MCR\fi}

\ifttm
\def\MCR{\special{html:<br />}}
\else
\def\MCR{\ \\}
\fi


% Pragmas - Sind Schluesselwoerter, die dem Preprocessing sowie dem Konverter uebergeben werden und bestimmte
%           Aktionen ausloesen. Im Output (PDF und HTML) tauchen sie nicht auf.
\newcommand{\MPragma}[1]{%
\ifttm%
\special{html:<!-- mpragma;-;}#1\special{html:;; -->}%
\else%
% MPragmas werden vom Preprozessor direkt im LaTeX gefunden
\fi%
}

% Ersatz der Befehle textsubscript und textsuperscript, die ttm nicht kennt
\ifttm%
\newcommand{\MTextsubscript}[1]{\special{html:<sub>}#1\special{html:</sub>}}%
\newcommand{\MTextsuperscript}[1]{\special{html:<sup>}#1\special{html:</sup>}}%
\else%
\newcommand{\MTextsubscript}[1]{\textsubscript{#1}}%
\newcommand{\MTextsuperscript}[1]{\textsuperscript{#1}}%
\fi

%------------------ Einbindung von dia-Diagrammen ----------------------------------------------
% Beim preprocessing wird aus jeder dia-Datei eine tex-Datei und eine pdf-Datei erzeugt,
% diese werden hier jeweils im PDF und HTML eingebunden
% Parameter: Dateiname der mit dia erstellten Datei (OHNE die Endung .dia)
\ifttm%
\newcommand{\MDia}[1]{%
\MGraphicsSolo{#1minthtml.png}{}%
}
\else%
\newcommand{\MDia}[1]{%
\MGraphicsSolo{#1mintpdf.png}{scale=0.1667}%
}
\fi%

% subsup funktioniert im Ausdruck $D={\R}^+_0$, also \R geklammert und sup zuerst
% \ifttm
% \def\MSubsup#1#2#3{\special{html:<msubsup>} #1 #2 #3\special{html:</msubsup>}}
% \else
% \def\MSubsup#1#2#3{{#1}^{#3}_{#2}}
% \fi

%\input{local.tex}

% \ifttm
% \else
% \newwrite\mintlog
% \immediate\openout\mintlog=mintlog.txt
% \fi

% ----------------------- tikz autogenerator -------------------------------------------------------------------

\newcommand{\Mtikzexternalize}{\tikzexternalize}% wird bei Konvertierung ueber mconvert ggf. ausgehebelt!

\ifttm
\else
\tikzset%
{
  % Defines a custom style which generates pdf and converts to (low and hi-res quality) png and svg, then deletes the pdf
  % Important: DO NOT directly convert from pdf to hires-png or from svg to png with GraphViz convert as it has some problems and memory leaks
  png export/.style=%
  {
    external/system call/.add={}{; 
      pdf2svg "\image.pdf" "\image.svg" ; 
      convert -density 112.5 -transparent white "\image.pdf" "\image.png"; 
      inkscape --export-png="\image.4x.png" --export-dpi=450 --export-background-opacity=0 --without-gui "\image.svg"; 
      rm "\image.pdf"; rm "\image.log"; rm "\image.dpth"; rm "\image.idx"
    },
    external/force remake,
  }
}
\tikzset{png export}
\tikzsetexternalprefix{}
% PNGs bei externer Erzeugung in "richtiger" Groesse einbinden
\pgfkeys{/pgf/images/include external/.code={\includegraphics[scale=0.64]{#1}}}
\fi

% Spezielle Umgebung fuer Autogenerierung, Bildernamen sind nur innerhalb eines Moduls (einer MSection) eindeutig)

\newcommand{\MTIKZautofilename}{tikzautofile}

\ifttm
% HTML-Version: Vom Autogenerator erzeugte png-Datei einbinden, tikz selbst nicht ausfuehren (sprich: #1 schlucken)
\newcommand{\MTikzAuto}[1]{%
\addtocounter{MTIKZAutofilenumber}{1}
\renewcommand{\MTIKZautofilename}{mtikzauto_\arabic{MTIKZAutofilenumber}}
\MUGraphicsSolo{\MSectionID\MTIKZautofilename.4x.png}{scale=1}{\special{html:[[!-- svgstyle;}\MSectionID\MTIKZautofilename\special{html: //--]]}} % Styleinfos werden aus original-png, nicht 4x-png geholt!
%\MRegisterFile{\MSectionID\MTIKZautofilename.png} % not used right now
%\MRegisterFile{\MSectionID\MTIKZautofilename.svg}
}
\else%
% PDF-Version: Falls Autogenerator aktiv wird Datei automatisch benannt und exportiert
\newcommand{\MTikzAuto}[1]{%
\addtocounter{MTIKZAutofilenumber}{1}%
\renewcommand{\MTIKZautofilename}{mtikzauto_\arabic{MTIKZAutofilenumber}}
\tikzsetnextfilename{\MTIKZautofilename}%
#1%
}
\fi

% In einer reinen LaTeX-Uebersetzung kapselt der Preambelinclude-Befehl nur input,
% in einer konvertergesteuerten PDF/HTML-Uebersetzung wird er dagegen entfernt und
% die Preambeln an mintmod angehaengt, die Ersetzung wird von mconvert.pl vorgenommen.

\newcommand{\MPreambleInclude}[1]{\input{#1}}

% Globale Watermarksettings (werden auch nochmal zu Beginn jedes subsection gesetzt,
% muessen hier aber auch global ausgefuehrt wegen Einfuehrungsseiten und Inhaltsverzeichnis

\MWatermarkSettings
% ---------------------------------- Parametrisierte Aufgaben ----------------------------------------

\ifttm
\newenvironment{MPExercise}{%
\begin{MExercise}%
}{%
\special{html:<button name="Name_MPEX}\arabic{MExerciseCounter}\special{html:" id="MPEX}\arabic{MExerciseCounter}%
\special{html:" type="button" onclick="reroll('}\arabic{MExerciseCounter}\special{html:');">Neue Aufgabe erzeugen</button>}%
\end{MExercise}%
}
\else
\newenvironment{MPExercise}{%
\begin{MExercise}%
}{%
\end{MExercise}%
}
\fi

% Parameter: Name, Min, Max, PDF-Standard. Name in Deklaration OHNE backslash, im Code MIT Backslash
\ifttm
\newcommand{\MGlobalInteger}[4]{\special{html:%
<!-- onloadstart //-->%
MVAR.push(createGlobalInteger("}#1\special{html:",}#2\special{html:,}#3\special{html:,}#4\special{html:)); %
<!-- onloadstop //-->%
<!-- viewmodelstart //-->%
ob}#1\special{html:: ko.observable(rerollMVar("}#1\special{html:")),%
<!-- viewmodelstop //-->%
}%
}%
\else%
\newcommand{\MGlobalInteger}[4]{\newcounter{mvc_#1}\setcounter{mvc_#1}{#4}}
\fi

% Parameter: Name, Min, Max, PDF-Standard. Name in Deklaration OHNE backslash, im Code MIT Backslash, Wert ist Wurzel von value
\ifttm
\newcommand{\MGlobalSqrt}[4]{\special{html:%
<!-- onloadstart //-->%
MVAR.push(createGlobalSqrt("}#1\special{html:",}#2\special{html:,}#3\special{html:,}#4\special{html:)); %
<!-- onloadstop //-->%
<!-- viewmodelstart //-->%
ob}#1\special{html:: ko.observable(rerollMVar("}#1\special{html:")),%
<!-- viewmodelstop //-->%
}%
}%
\else%
\newcommand{\MGlobalSqrt}[4]{\newcounter{mvc_#1}\setcounter{mvc_#1}{#4}}% Funktioniert nicht als Wurzel !!!
\fi

% Parameter: Name, Min, Max, PDF-Standard zaehler, PDF-Standard nenner. Name in Deklaration OHNE backslash, im Code MIT Backslash
\ifttm
\newcommand{\MGlobalFraction}[5]{\special{html:%
<!-- onloadstart //-->%
MVAR.push(createGlobalFraction("}#1\special{html:",}#2\special{html:,}#3\special{html:,}#4\special{html:,}#5\special{html:)); %
<!-- onloadstop //-->%
<!-- viewmodelstart //-->%
ob}#1\special{html:: ko.observable(rerollMVar("}#1\special{html:")),%
<!-- viewmodelstop //-->%
}%
}%
\else%
\newcommand{\MGlobalFraction}[5]{\newcounter{mvc_#1}\setcounter{mvc_#1}{#4}} % Funktioniert nicht als Bruch !!!
\fi

% MVar darf im HTML nur in MEvalMathDisplay-Umgebungen genutzt werden oder in Strings die an den Parser uebergeben werden
\ifttm%
\newcommand{\MVar}[1]{\special{html:[var_}#1\special{html:]}}%
\else%
\newcommand{\MVar}[1]{\arabic{mvc_#1}}%
\fi

\ifttm%
\newcommand{\MRerollButton}[2]{\special{html:<button type="button" onclick="rerollMVar('}#1\special{html:');">}#2\special{html:</button>}}%
\else%
\newcommand{\MRerollButton}[2]{\relax}% Keine sinnvolle Entsprechung im PDF
\fi

% MEvalMathDisplay fuer HTML wird in mconvert.pl im preprocessing realisiert
% PDF: eine equation*-Umgebung (ueber amsmath)
% HTML: Eine Mathjax-Tex-Umgebung, deren Auswertung mit knockout-obervablen gekoppelt ist
% PDF-Version hier nur fuer pdflatex-only-Uebersetzung gegeben

\ifttm\else\newenvironment{MEvalMathDisplay}{\begin{equation*}}{\end{equation*}}\fi

% ---------------------------------- Spezialbefehle fuer AD ------------------------------------------

%Abk�rzung f�r \longrightarrow:
\newcommand{\lto}{\ensuremath{\longrightarrow}}

%Makro f�r Funktionen:
\newcommand{\exfunction}[5]
{\begin{array}{rrcl}
 #1 \colon  & #2 &\lto & #3 \\[.05cm]  
  & #4 &\longmapsto  & #5 
\end{array}}

\newcommand{\function}[5]{%
#1:\;\left\lbrace{\begin{array}{rcl}
 #2 &\lto & #3 \\
 #4 &\longmapsto  & #5 \end{array}}\right.}


%Die Identit�t:
\DeclareMathOperator{\Id}{Id}

%Die Signumfunktion:
\DeclareMathOperator{\sgn}{sgn}

%Zwei Betonungskommandos (k�nnen angepasst werden):
\newcommand{\highlight}[1]{#1}
\newcommand{\modstextbf}[1]{#1}
\newcommand{\modsemph}[1]{#1}


% ---------------------------------- Spezialbefehle fuer JL ------------------------------------------


\def\jccolorfkt{green!50!black} %Farbe des Funktionsgraphen
\def\jccolorfktarea{green!25!white} %Farbe der Fl"ache unter dem Graphen
\def\jccolorfktareahell{green!12!white} %helle Einf"arbung der Fl"ache unter dem Graphen
\def\jccolorfktwert{green!50!black} %Farbe einzelner Punkte des Graphen

\newcommand{\MPfadBilder}{Bilder}

\ifttm%
\newcommand{\jMD}{\,\MD}%
\else%
\newcommand{\jMD}{\;\MD}%
\fi%

\def\jHTMLHinweisBedienung{\MInputHint{%
Mit Hilfe der Symbole am oberen Rand des Fensters
k"onnen Sie durch die einzelnen Abschnitte navigieren.}}

\def\jHTMLHinweisEingabeText{\MInputHint{%
Geben Sie jeweils ein Wort oder Zeichen als Antwort ein.}}

\def\jHTMLHinweisEingabeTerm{\MInputHint{%
Klammern Sie Ihre Terme, um eine eindeutige Eingabe zu erhalten. 
Beispiel: Der Term $\frac{3x+1}{x-2}$ soll in der Form
\texttt{(3*x+1)/((x+2)^2}$ eingegeben werden (wobei auch Leerzeichen 
eingegeben werden k"onnen, damit eine Formel besser lesbar ist).}}

\def\jHTMLHinweisEingabeIntervalle{\MInputHint{%
Intervalle werden links mit einer "offnenden Klammer und rechts mit einer 
schlie"senden Klammer angegeben. Eine runde Klammer wird verwendet, wenn der 
Rand nicht dazu geh"ort, eine eckige, wenn er dazu geh"ort. 
Als Trennzeichen wird ein Komma oder ein Semikolon akzeptiert.
Beispiele: $(a, b)$ offenes Intervall,
$[a; b)$ links abgeschlossenes, rechts offenes Intervall von $a$ bis $b$. 
Die Eingabe $]a;b[$ f"ur ein offenes Intervall wird nicht akzeptiert.
F"ur $\infty$ kann \texttt{infty} oder \texttt{unendlich} geschrieben werden.}}

\def\jHTMLHinweisEingabeFunktionen{\MInputHint{%
Schreiben Sie Malpunkte (geschrieben als \texttt{*}) aus und setzen Sie Klammern um Argumente f�r Funktionen.
Beispiele: Polynom: \texttt{3*x + 0.1}, Sinusfunktion: \texttt{sin(x)}, 
Verkettung von cos und Wurzel: \texttt{cos(sqrt(3*x))}.}}

\def\jHTMLHinweisEingabeFunktionenSinCos{\MInputHint{%
Die Sinusfunktion $\sin x$ wird in der Form \texttt{sin(x)} angegeben, %
$\cos\left(\sqrt{3 x}\right)$ durch \texttt{cos(sqrt(3*x))}.}}

\def\jHTMLHinweisEingabeFunktionenExp{\MInputHint{%
Die Exponentialfunktion $\MEU^{3x^4 + 5}$ wird als
\texttt{exp(3 * x^4 + 5)} angegeben, %
$\ln\left(\sqrt{x} + 3.2\right)$ durch \texttt{ln(sqrt(x) + 3.2)}.}}

% ---------------------------------- Spezialbefehle fuer Fachbereich Physik --------------------------

\newcommand{\E}{{e}}
\newcommand{\ME}[1]{\cdot 10^{#1}}
\newcommand{\MU}[1]{\;\mathrm{#1}}
\newcommand{\MPG}[3]{%
  \ifnum#2=0%
    #1\ \mathrm{#3}%
  \else%
    #1\cdot 10^{#2}\ \mathrm{#3}%
  \fi}%
%

\newcommand{\MMul}{\MExponentensymbXYZl} % Nur eine Abkuerzung


% ---------------------------------- Stichwortfunktionialitaet ---------------------------------------

% mpreindexentry wird durch Auswahlroutine in conv.pl durch mindexentry substitutiert
\ifttm%
\def\MIndex#1{\index{#1}\special{html:<!-- mpreindexentry;;}#1\special{html:;;}\arabic{MSubjectArea}\special{html:;;}%
\arabic{chapter}\special{html:;;}\arabic{section}\special{html:;;}\arabic{subsection}\special{html:;;}\arabic{MEntryCounter}\special{html:; //-->}%
\setcounter{MLastIndex}{\value{MEntryCounter}}%
\addtocounter{MEntryCounter}{1}%
}%
% Copyrightliste wird als tex-Datei im preprocessing von conv.pl erzeugt und unter converter/tex/entrycollection.tex abgelegt
% Der input-Befehl funktioniert nur, wenn die aufrufende tex-Datei auf der obersten Ebene liegt (d.h. selbst kein input/include ist, insbesondere keine Moduldatei)
\def\MEntryList{} % \input funktioniert nicht, weil ttm (und damit das \input) ausgefuehrt wird, bevor Datei da ist
\else%
\def\MIndex#1{\index{#1}}
\def\MEntryList{\MAbort{Stichwortliste nur im HTML realisierbar}}%
\fi%

\def\MEntry#1#2{\textbf{#1}\MIndex{#2}} % Idee: MLastType auf neuen Entry-Typ und dann ein MLabel vergeben mit autogen-Nummer

% ---------------------------------- Befehle fuer Tests ----------------------------------------------

% MEquationItem stellt eine Eingabezeile der Form Vorgabe = Antwortfeld her, der zweite Parameter kann z.B. MSimplifyQuestion-Befehl sein
\ifttm
\newcommand{\MEquationItem}[2]{{#1}$\,=\,${#2}}%
\else%
\newcommand{\MEquationItem}[2]{{#1}$\;\;=\,${#2}}%
\fi

\ifttm
\newcommand{\MInputHint}[1]{%
\ifnum%
\if\value{MTestSite}>0%
\else%
{\color{blue}#1}%
\fi%
\fi%
}
\else
\newcommand{\MInputHint}[1]{\relax}
\fi

\ifttm
\newcommand{\MInTestHeader}{%
Dies ist ein einreichbarer Test:
\begin{itemize}
\item{Im Gegensatz zu den offenen Aufgaben werden beim Eingeben keine Hinweise zur Formulierung der mathematischen Ausdr�cke gegeben.}
\item{Der Test kann jederzeit neu gestartet oder verlassen werden.}
\item{Der Test kann durch die Buttons am Ende der Seite beendet und abgeschickt, oder zur�ckgesetzt werden.}
\item{Der Test kann mehrfach probiert werden. F�r die Statistik z�hlt die zuletzt abgeschickte Version.}
\end{itemize}
}
\else
\newcommand{\MInTestHeader}{%
\relax
}
\fi

\ifttm
\newcommand{\MInTestFooter}{%
\special{html:<button name="Name_TESTFINISH" id="TESTFINISH" type="button" onclick="finish_button('}\MTestName\special{html:');">Test auswerten</button>}%
\begin{html}
&nbsp;&nbsp;&nbsp;&nbsp;&nbsp;&nbsp;&nbsp;&nbsp;
<button name="Name_TESTRESET" id="TESTRESET" type="button" onclick="reset_button();">Test zur�cksetzen</button>
<br />
<br />
<div class="xreply">
<p name="Name_TESTEVAL" id="TESTEVAL">
Hier erscheint die Testauswertung!
<br />
</p>
</div>
\end{html}
}
\else
\newcommand{\MInTestFooter}{%
\relax
}
\fi


% ---------------------------------- Notationsmakros -------------------------------------------------------------

% Notationsmakros die nicht von der Kursvariante abhaengig sind

\newcommand{\MZahltrennzeichen}[1]{\renewcommand{\MZXYZhltrennzeichen}{#1}}

\ifttm
\newcommand{\MZahl}[3][\MZXYZhltrennzeichen]{\edef\MZXYZtemp{\noexpand\special{html:<mn>#2#1#3</mn>}}\MZXYZtemp}
\else
\newcommand{\MZahl}[3][\MZXYZhltrennzeichen]{{}#2{#1}#3}
\fi

\newcommand{\MEinheitenabstand}[1]{\renewcommand{\MEinheitenabstXYZnd}{#1}}
\ifttm
\newcommand{\MEinheit}[2][\MEinheitenabstXYZnd]{{}#1\edef\MEINHtemp{\noexpand\special{html:<mi mathvariant="normal">#2</mi>}}\MEINHtemp} 
\else
\newcommand{\MEinheit}[2][\MEinheitenabstXYZnd]{{}#1 \mathrm{#2}} 
\fi

\newcommand{\MExponentensymbol}[1]{\renewcommand{\MExponentensymbXYZl}{#1}}
\newcommand{\MExponent}[2][\MExponentensymbXYZl]{{}#1{} 10^{#2}} 

%Punkte in 2 und 3 Dimensionen
\newcommand{\MPointTwo}[3][]{#1(#2\MCoordPointSep #3{}#1)}
\newcommand{\MPointThree}[4][]{#1(#2\MCoordPointSep #3\MCoordPointSep #4{}#1)}
\newcommand{\MPointTwoAS}[2]{\left(#1\MCoordPointSep #2\right)}
\newcommand{\MPointThreeAS}[3]{\left(#1\MCoordPointSep #2\MCoordPointSep #3\right)}

% Masseinheit, Standardabstand: \,
\newcommand{\MEinheitenabstXYZnd}{\MThinspace} 

% Horizontaler Leerraum zwischen herausgestellter Formel und Interpunktion
\ifttm
\newcommand{\MDFPSpace}{\,}
\newcommand{\MDFPaSpace}{\,\,}
\newcommand{\MBlank}{\ }
\else
\newcommand{\MDFPSpace}{\;}
\newcommand{\MDFPaSpace}{\;\;}
\newcommand{\MBlank}{\ }
\fi

% Satzende in herausgestellter Formel mit horizontalem Leerraum
\newcommand{\MDFPeriod}{\MDFPSpace .}

% Separation von Aufzaehlung und Bedingung in Menge
\newcommand{\MCondSetSep}{\,:\,} %oder '\mid'

% Konverter kennt mathopen nicht
\ifttm
\def\mathopen#1{}
\fi

% -----------------------------------START Rouletteaufgaben ------------------------------------------------------------

\ifttm
% #1 = Dateiname, #2 = eindeutige ID fuer das Roulette im Kurs
\newcommand{\MDirectRouletteExercises}[2]{
\begin{MExercise}
\texttt{Im HTML erscheinen hier Aufgaben aus einer Aufgabenliste...}
\end{MExercise}
}
\else
\newcommand{\MDirectRouletteExercises}[2]{\relax} % wird durch mconvert.pl gefunden und ersetzt
\fi


% ---------------------------------- START Makros, die von der Kursvariante abhaengen ----------------------------------

\ifvariantunotation
  % unotation = An Universitaeten uebliche Notation
  \def\MVariant{unotation}

  % Trennzeichen fuer Dezimalzahlen
  \newcommand{\MZXYZhltrennzeichen}{.}

  % Exponent zur Basis 10 in der Exponentialschreibweise, 
  % Standardmalzeichen: \times
  \newcommand{\MExponentensymbXYZl}{\times} 

  % Begrenzungszeichen fuer offene Intervalle
  \newcommand{\MoIl}[1][]{\mbox{}#1(\mathopen{}} % bzw. ']'
  \newcommand{\MoIr}[1][]{#1)\mbox{}} % bzw. '['

  % Zahlen-Separation im IntervaLL
  \newcommand{\MIntvlSep}{,} %oder ';'

  % Separation von Elementen in Mengen
  \newcommand{\MElSetSep}{,} %oder ';'

  % Separation von Koordinaten in Punkten
  \newcommand{\MCoordPointSep}{,} %oder ';' oder '|', '\MThinspace|\MThinspace'

\else
  % An dieser Stelle wird angenommen, dass std-Variante aktiv ist
  % std = beschlossene Notation im TU9-Onlinekurs 
  \def\MVariant{std}

  % Trennzeichen fuer Dezimalzahlen
  \newcommand{\MZXYZhltrennzeichen}{,}

  % Exponent zur Basis 10 in der Exponentialschreibweise, 
  % Standardmalzeichen: \times
  \newcommand{\MExponentensymbXYZl}{\times} 

  % Begrenzungszeichen fuer offene Intervalle
  \newcommand{\MoIl}[1][]{\mbox{}#1]\mathopen{}} % bzw. '('
  \newcommand{\MoIr}[1][]{#1[\mbox{}} % bzw. ')'

  % Zahlen-Separation im IntervaLL
  \newcommand{\MIntvlSep}{;} %oder ','
  
  % Separation von Elementen in Mengen
  \newcommand{\MElSetSep}{;} %oder ','

  % Separation von Koordinaten in Punkten
  \newcommand{\MCoordPointSep}{;} %oder '|', '\MThinspace|\MThinspace'

\fi



% ---------------------------------- ENDE Makros, die von der Kursvariante abhaengen ----------------------------------


% diese Kommandos setzen Mathemodus vorraus
\newcommand{\MGeoAbstand}[2]{[\overline{{#1}{#2}}]}
\newcommand{\MGeoGerade}[2]{{#1}{#2}}
\newcommand{\MGeoStrecke}[2]{\overline{{#1}{#2}}}
\newcommand{\MGeoDreieck}[3]{{#1}{#2}{#3}}

%
\ifttm
\newcommand{\MOhm}{\special{html:<mn>&#x3A9;</mn>}}
\else
\newcommand{\MOhm}{\Omega} %\varOmega
\fi


\def\PERCTAG{\MAbort{PERCTAG ist zur internen verwendung in mconvert.pl reserviert, dieses Makro darf sonst nicht benutzt werden.}}

% Im Gegensatz zu einfachen html-Umgebungen werden MDirectHTML-Umgebungen von mconvert.pl am ganzen ttm-Prozess vorbeigeschleust und aus dem PDF komplett ausgeschnitten
\ifttm%
\newenvironment{MDirectHTML}{\begin{html}}{\end{html}}%
\else%
\newenvironment{MDirectHTML}{\begin{html}}{\end{html}}%
\fi

% Im Gegensatz zu einfachen Mathe-Umgebungen werden MDirectMath-Umgebungen von mconvert.pl am ganzen ttm-Prozess vorbeigeschleust, ueber MathJax realisiert, und im PDF als $$ ... $$ gesetzt
\ifttm%
\newenvironment{MDirectMath}{\begin{html}}{\end{html}}%
\else%
\newenvironment{MDirectMath}{\begin{equation*}}{\end{equation*}}% Vorsicht, auch \[ und \] werden in amsmath durch equation* redefiniert
\fi

% ---------------------------------- Location Management ---------------------------------------------

% #1 = buttonname (muss in files/images liegen und Format 48x48 haben), #2 = Vollstaendiger Einrichtungsname, #3 = Kuerzel der Einrichtung,  #4 = Name der include-texdatei
\ifttm
\newcommand{\MLocationSite}[3]{\special{html:<!-- mlocation;;}#1\special{html:;;}#2\special{html:;;}#3\special{html:;; //-->}}
\else
\newcommand{\MLocationSite}[3]{\relax}
\fi

% ---------------------------------- Copyright Management --------------------------------------------

\newcommand{\MCCLicense}{%
{\color{green}\textbf{CC BY-SA 3.0}}
}

\newcommand{\MCopyrightLabel}[1]{ (\MSRef{L_COPYRIGHTCOLLECTION}{Lizenz})\MLabel{#1}}

% Copyrightliste wird als tex-Datei im preprocessing erzeugt und unter converter/tex/copyrightcollection.tex abgelegt
% Der input-Befehl funktioniert nur, wenn die aufrufende tex-Datei auf der obersten Ebene liegt (d.h. selbst kein input/include ist, insbesondere keine Moduldatei)
\newcommand{\MCopyrightCollection}{\input{copyrightcollection.tex}}

% MCopyrightNotice fuegt eine Copyrightnotiz ein, der parser ersetzt diese durch CopyrightNoticePOST im preparsing, diese Definition wird nur fuer reine pdflatex-Uebersetzungen gebraucht
% Parameter: #1: Kurze Lizenzbeschreibung (typischerweise \MCCLicense)
%            #2: Link zum Original (http://...) oder NONE falls das Bild selbst ein Original ist, oder TIKZ falls das Bild aus einer tikz-Umgebung stammt
%            #3: Link zum Autor (http://...) oder MINT falls Original im MINT-Kolleg erstellt oder NONE falls Autor unbekannt
%            #4: Bemerkung (z.B. dass Datei mit Maple exportiert wurde)
%            #5: Labelstring fuer existierendes Label auf das copyrighted Objekt, mit MCopyrightLabel erzeugt
%            Keines der Felder darf leer sein!
\newcommand{\MCopyrightNotice}[5]{\MCopyrightNoticePOST{#1}{#2}{#3}{#4}{#5}}

\ifttm%
\newcommand{\MCopyrightNoticePOST}[5]{\relax}%
\else%
\newcommand{\MCopyrightNoticePOST}[5]{\relax}%
\fi%

% ---------------------------------- Meldungen fuer den Benutzer des Konverters ----------------------
\MPragma{mintmodversion;P0.1.0}
\MPragma{usercomment;This is file mintmod.tex version P0.1.0}


% ----------------------------------- Spezialelemente fuer Konfigurationsseite, werden nicht von mintscripts.js verwaltet --

% #1 = DOM-id der Box
\ifttm\newcommand{\MConfigbox}[1]{\special{html:<input cfieldtype="2" type="checkbox" name="Name_}#1\special{html:" id="}#1\special{html:" onchange="confHandlerChange('}#1\special{html:');"/>}}\fi % darf im PDF nicht aufgerufen werden!


\MPragma{MathSkip}

%\Mtikzexternalize

\begin{document}

\MSection{Equations in one Variable}
\MLabel{VBKM02}
\MSetSectionID{VBKM02} % wird fuer tikz-Dateien verwendet

\begin{MSectionStart}
\MDeclareSiteUXID{VBKM02_START}

Equations arise by equating two terms in which variables occur. Simple equations can be solved by 
applying transformations and solution formulas. For more sophisticated equations case analyses are 
required. This module consists of

\begin{itemize}
\item{Section \MNRef{M02_EinfacheGleichungen}: \MSRef{M02_EinfacheGleichungen}{Simple Equations},}
%\item{dem Abschnitt \MNRef{M02_Wurzelgleichungen}, \MSRef{M02_Wurzelgleichungen}{Gleichungen mit Wurzeln},}
\item{Section \MNRef{M02_Betragsgleichungen}: \MSRef{M02_Betragsgleichungen}{Absolute Value Equations},}
\item{and Section \MNRef{M02_Abschlusstest}: \MSRef{M02_Abschlusstest}{Final Test}.}
\end{itemize}
\end{MSectionStart}

\MSubsection{Simple Equations}
\MLabel{M02_EinfacheGleichungen}

\begin{MIntro}
\MDeclareSiteUXID{VBKM02_EinfacheGleichungenIntro}
\begin{MInfo}
An \MEntry{equation}{equation} is an expression of the form
$$
\text{left-hand side} \;=\; \text{right-hand side} \MDFPSpace
$$
with mathematical expressions on both sides of the equation. These expressions generally contain variables or unknowns (e.q.\ $x$). 
Depending on the variable values an equation is satisfied if both sides of the equation evaluate to the same value. An equation is not 
satisfied if the sides of the equation evaluate to different values.
\end{MInfo}

Equations describe relations between expressions or model a problem to be solved. An equation itself cannot
be true or false but some variables satisfy the equation and others do not. To test whether the equation is 
true or false for a single variable value this value has to be inserted into the equation. Then, both sides of the 
equation are evaluated to certain values. The equation is satisfied by an inserted variable value if the evaluated 
values coincide:

\begin{MExample}
The equation $2x-1=x^2$ has the right-hand side $x^2$ and the left-hand side $2x-1$. Inserting $x=1$ 
results in the value $1$ on both sides of the equals sign, hence $x=1$ is a solution of this equation. 
However, $x=2$ is no solution since the left-hand side of the equation is evaluated to the value $4$ while the 
right-hand side is evaluated to the value $3$.
\end{MExample}

\begin{MInfo}
The \MEntry{solution set}{solution set} $\ML$ of an equation is the set of all numbers satisfying the relation 
$$
\text{left-hand side} \;=\; \text{right-hand side} \MDFPSpace
$$
if inserted into the the equation instead of the variable (e.q. $x$).
\end{MInfo}

Typical problems concerning equations are:
\begin{itemize}
 \item{specify the solution set of an equation, i.e.\ find all variable values satisfying the equation,}
 \item{transform the equation, in particular, solve an equation for the variables, and}
 \item{find an equation modelling a problem described textually.}
\end{itemize}


\begin{MExample}
  A savings deposit is to be designed such that it gives a fixed return per year. The bank intends to achieve
  that the saver returns for a five years deposit exactly 600~Euro more than for a deposit of only two years. 

  First, the textual problem is translated into an equation with the variable $r$ denoting the return per year. Then, 
  the equation is $5\cdot r=2\cdot r+600$. It says that five payments (left-hand side of the equation) equal 
  two payments plus 600 (the unit Euro is then omitted during calculation).

  This equation can be solved for $r$ very easily by subtracting the term $2r$ to both sides of the equation. Then, the 
  equation reads $3r=600$ and dividing by $3$ results in the solution $r=200$.

  Thus, the bank has to offer a return of 200~Euro per year to reach the required savings target. 
\end{MExample}

% \begin{MExercise}
% �bersetzen Sie die folgenden Aussagen in eine Gleichung (ohne sie zu l�sen):
% \begin{MExerciseItems}
% \item{In elf Jahren ist Max doppelt so alt wie jetzt ($t$ = jetziges Alter in Jahren):  ???}
% \item{Ein Kunde hat eine Rechnung �ber 1000 Euro zu bezahlen. Die monatlich zu zahlende Rate bleibt f�r ihn gleich unabh�ngig davon ob er
% \begin{itemize}
%  \item{die Rechnung mit $m$ gleichgro�en Raten abbezahlt,}
%  \item{vier Raten mehr als $m$ einteilt und zus�tzlich 10 Euro Verzugsgeb�hren pro Monat an die Bank zahlt (die nicht zur Rechnungsbegleichung dienen).}
% \end{itemize}
% Ist $m$ die Anzahl der Monate, so gilt ???
% }
% \end{MExerciseItems}
% \end{MExercise}


\begin{MInfo}
Two equations are said to be \MEntry{equivalent}{equivalence} if they have the same solution set.

An \MEntry{equivalent transformation}{equivalent transformation} is a special transformation that changes
the equation but not its solution set. Important equivalent transformations are

\begin{itemize}
 \item{adding/subtracting terms to both sides of the equation,}
 \item{exchanging both sides of the equation,}
 \item{transformation of terms on one side of the equation, and}
 \item{inserting known relations.}
 \end{itemize}

The following transformations are considered as equivalent transformations only if the used term is non-zero 
(which can depend on the variables):

\begin{itemize}
 \item{multiplying/dividing by a term (this term has to be non-zero),}
 \item{taking the reciprocal on both sides of the equation (both sides have to be non-zero).}
\end{itemize}
\end{MInfo}

Here, the following \MEntry{notation}{equivalent transformation (notation)} is used:

\begin{itemize}
 \item{equivalent equations are indicated by the symbol $\Leftrightarrow$ (which reads: if and only if, i.e. one
  equation is satisfied if and only if the other equation is satisfied).}

\item{the symbol is underset by the transforming operation (or, for solutions with more than one line, the 
  transforming operation is written next to the transformation).}
\end{itemize}

Importantly, the reader should be able to understand which transformation was carried out.

\begin{MExample}
This example illustrates two simple single-lined equivalent transformations. Even though the symbol $\Leftrightarrow$ 
is two-sided the notation is interpreted in such a way that the transformation is applied from 
left to right:
$$
3x-x^2 \;=\; 2x-x^2+1\;\;\MUnderset{+x^2}\Leftrightarrow\;\;  3x\;=\; 2x+1 \;\;\MUnderset{-2x}\Leftrightarrow\;\; x \;=\; 1 \MDFPeriod
$$
The left equation and the right equation are equivalent. On the left we have the initial equation
(corresponding to a certain textual problem) and on the right we have an equivalent equation 
showing the solution immediately.
\end{MExample}

\begin{MExample}
\MLabel{BSP_Umformungen1}
For several complicated transformations the transformation steps should be written below each other.
In this case the respective transformations are separated from the equation by vertical bars. 

\begin{eqnarray*}
& \text{Start:} & 12+t \;=\;\Mdfrac{2t}{2t^2}+t \ \ \ \ \MSep \ -t\ \\ \ \\
& \Leftrightarrow & 12 \;=\; \Mdfrac{2t}{2t^2}  \ \ \ \ \MSep \ \text{sides exchanged}\ \\ \ \\
& \Leftrightarrow & \Mdfrac{2t}{2t^2} \;=\; 12  \ \ \ \ \MSep \ \text{left-hand side transformed}\ \\ \ \\
& \Leftrightarrow & \Mdfrac1t \;=\; 12  \ \ \ \ \MSep \ \text{reciprocals taken}\ \\ \ \\
& \Leftrightarrow & t \;=\; \Mdfrac{1}{12} \MDFPeriod
\end{eqnarray*}

Here, after the vertical bar both short symbols as, e.g. $-t$, and textual descriptions are allowed. 
Importantly, the reader should be able to understand which transformations were carried out and should be able to decide whether 
they are correct.

\end{MExample}
\end{MIntro}

\begin{MXContent}{Conditions in Transformations}{Conditions}{STD}
\MDeclareSiteUXID{VBKM02_Bedingungen}
Multiplication, division, and taking reciprocals are only equivalent transformations if the factors or terms are 
non-zero. In example~\MRef{BSP_Umformungen1}, the reader understands that both sides of the equation are non-zero 
such that the transformation is allowed. If the variables themselves are used in the transformation it has to be 
noted that the respective term has to be non-zero. The solution at the end of the transformations is then 
only valid for variable values satisfying the transformation conditions. All other values have to be checked 
\textit{separately}, typically by inserting the value into the equation:

\begin{MExample}
In this example, the necessary transformation conditions are not problematic:
\begin{eqnarray*}
& \text{Start:} & 9x \;=\;81x^2  \ \ \ \ \MSep \ :x\text{, transformation allowed if }x\not=0\ \\ \ \\
& \Leftrightarrow & 9 \;=\; 81x  \ \ \ \ \MSep \ :81\text{ and exchange sides}\ \\ \ \\
& \Leftrightarrow & x \;=\; \Mdfrac19 \ \ \ \ \text{and this value satisfies the condition }x\not=0 \MDFPeriod
\end{eqnarray*}
The value $x=0$, initially rejected by the transformation condition, has to be checked separately. The equation $9x=81x^2$ is 
also satisfied for $x=0$, hence $x=0$ is also a solution of the equation. In set notation, the 
solution set is $\ML=\lbrace 0\MElSetSep \Mtfrac19\rbrace$.
\end{MExample}

Anyway, values violating a condition have to be checked separately, in particular, they can be finally 
part of the solution.

\begin{MExample}
\begin{eqnarray*}
& \text{Start:} & x^2-2x \;=\; 2x-4 \ \ \ \ \MSep \ \text{factor out on both sides} \ \\ \ \\
& \Leftrightarrow & x\cdot (x-2) \;=\; 2\cdot (x-2) \ \ \ \ | \ :(x-2)\text{, transformation only allowed if }x\not=2\ \\ \ \\
& \Leftrightarrow & x\;=\; 2  \MDFPeriod
\end{eqnarray*}
This value of $x$ violates the condition $x\not=2$. Hence, this is possibly no solution. Inserting $x=2$
into the initial equation gives $2^2-2\cdot 2=0$ on the left-hand side and also $2\cdot 2-4=0$ on the 
right-hand side. Hence, $x=2$ is indeed a solution, even though it violated the transformation condition.
\end{MExample}

\begin{MExercise}
Find the solution of the equation $(x-2)(x-3)=x^2-9$ by transforming the right-hand side using the third 
binomial formula and then dividing by a common factor. 

The solution is $x$ = \MLParsedQuestion{5}{3}{5}{EASY1}.

\begin{MHint}{Solution}
The correct transformation steps including conditions are
\begin{eqnarray*}
& \text{Start:} & (x-2)(x-3)\;=\; x^2-9 \ \ \ \ \MSep \ \text{transformation of the right-hand side} \ \\ \ \\
& \Leftrightarrow & (x-2)(x-3) \;=\; (x+3)(x-3) \ \ \ \ \MSep \ :(x-3)\text{, transformation allowed if }x\not=3\ \\ \ \\
& \Leftrightarrow & x-2\;=\; x+3 \ \ \ \ \MSep \ -x \ \\ \ \\
& \Leftrightarrow & -2 \;=\; 3 \; \text{is a wrong equation.}
\end{eqnarray*}
Importantly, this equation is only wrong for $x\not=3$. We have to check $x=3$ separately, and indeed 
$x=3$ satisfies the initial equation.
\end{MHint}
\end{MExercise}
\end{MXContent}

\begin{MXContent}{Proportionality and Rule of Three}{Proportionality}{STD}
\MLabel{VBKM02_Dreisatz}
\MDeclareSiteUXID{VBKM02_Dreisatz}
In practise, a relation between two quantities that occurs frequently is the 
\MEntry{proportionality}{proportionality} between two quantities, e.g. between 
mass and volume, time and travelled distance or weight (quantity) of a product and its price. 
The relation can be exemplary for certain fixed quantities. Then, a first aim is to formulate the 
resulting relation for another application example. 
The procedure shall be illustrated by an example. 


\begin{MExample}
$5\MEinheit{kg}$ of apples cost $3$ Euro. How much do $11\MEinheit{kg}$
of apples cost?
\par
The initial relation can be formulated as follows:
$$
5\MEinheit{kg} \MDFPSpace \MRelates \MDFPSpace 3\MBlank \text{Euro}
\MDFPeriod
$$
It is assumed that these quantities are proportional to each other. In the next
step the relation between the quantities is reduced to a unit of one of the quantities, 
namely to the unit of the given quantity. Here, both quantities 
are multiplied by $1/5$ -- i.e. the unit is $1\MEinheit{kg}$ --:
$$
1\MEinheit{kg} \MDFPSpace \MRelates \MDFPSpace 
\frac{1}{5}\cdot 3\MBlank \text{Euro} = \MZahl{0}{6}\MBlank \text{Euro}
\MDFPeriod
$$
Finally, both sides of the equation are multiplied by the multiple of the 
respective unit of the specified quantity, in this case by the factor $11$: 
$$
11\MEinheit{kg} \MDFPSpace \MRelates \MDFPSpace 
11\cdot\MZahl{0}{6}\MBlank \text{Euro} = \MZahl{6}{6}\MBlank \text{Euro}
\MDFPeriod
$$
The required price for $11\MEinheit{kg}$ of apples is therefore
$\MZahl{6}{60}\MBlank \text{Euro}$.
\end{MExample}

We have derived the required relation by deriving a relation for one unit of a quantity from 
the initial relation. This procedure demonstrated here as an example is called 
\MEntry{rule of three}{rule of three}.

The posed problem can also be solved by introducing a proportionality factor. 
Again, we consider the example above.

\begin{MExample}
The price $P$ is proportional to the mass $m$. Hence, it exists a constant 
$k$ with
$$
P=k m \MDFPeriod
$$
Since this relation also holds for the given values $m_0=5\MEinheit{kg}$
and $P_0=3\MBlank\text{Euro}$ it follows 
\begin{eqnarray*}
  P_0=k m_0 & \MTSP\MSep\MTSP & \text{multiplying by}\MBlank \frac{1}{m_0} \\
  \Longleftrightarrow\MDFPSpace\frac{P_0}{m_0}=k &;& 
\end{eqnarray*}
hence in this case
$$
k=\frac{3}{5} = \MZahl{0}{6} \MDFPSpace,
$$
taken in the unit of Euro per kg. (As a scientist you would correctly write 
 $k=\MZahl{0}{6} \MBlank \text{Euro}/\MEinheit[]{kg}$, since proportionality factors 
generally carry a dimensional unit.) Using $m_1=11\MEinheit{kg}$, one obtains 
finally
$$
P_1=k m_1 = \MZahl{0}{6}\cdot 11 =\MZahl{6}{6} \MBlank \text{(Euro)}
\MDFPSpace
$$
which is the same result as for using the rule of three (see previous example).
\end{MExample}

\begin{MExercise}
A car takes $9$~minutes to travel a distance of $6\MEinheit{km}$.
\begin{MExerciseItems}
\item{%
Which distance $s$ the car travels within $15$~minutes?
\medskip\par
The solution is $s_{15}$ = \MLParsedQuestion{5}{10}{5}{EASY2}$\MEinheit{km}$.
}
\item{%
The proportionality factor between travelled distance $s$ and travelling 
time $t$ is the velocity $v$ of the car.
\medskip\par
The velocity is $v$ = \MLParsedQuestion{5}{40}{5}{EASY3}$\MEinheit{km}/\MEinheit{h}$.
}
\end{MExerciseItems}
\begin{MHint}{Solution}
>From the given values we know that the car travels $\frac{6}{9}\MEinheit{km}
=\frac{2}{3}\MEinheit{km}$  within one minute and therefore 
$15\cdot\frac{2}{3}\MEinheit{km}=10\MEinheit{km}$ within $15$~minutes.
\par
So, the velocity is
$$v=\frac{10\MEinheit{km}}{15\MEinheit{min}} = 
\frac{10\MEinheit{km}}{(1/4)\MEinheit{h}} =
40\MEinheit{}\frac{\MEinheit[]{km}}{\MEinheit[]{h}}
\MDFPeriod
$$
\end{MHint}
\end{MExercise}
\end{MXContent}

\begin{MXContent}{Solving linear Equations}{Solving}{STD}
\MLabel{VBKM02_LineareGleichungenLoesen}
\MDeclareSiteUXID{VBKM02_Aufloesen}
\begin{MInfo}
A \MEntry{linear equation}{equation (linear)} is an equation in which only multiples of 
variables and constants occur.

For a linear equation in one variable (here the variable $x$) one of the following 
three statements holds:
\begin{itemize}
 \item{The equation has no solution.}
 \item{The equation has a single solution.}
 \item{Every value of $x$ is a solution of the equation.}
\end{itemize}
\end{MInfo}

These three cases are distinguished by means of the transformation steps:
\begin{itemize}
\item{If the transformation ends up in a statement that is wrong for all $x$ (e.g. $1=0$) 
then the equation is unsolvable.}
\item{If the transformation ends up in a statement that is true for all $x$ (e.g. $1=1$)
then the equation is solvable for all values of $x$.}
\item{Otherwise, the equation can be solved, i.e. it can be transformed into 
the equation $x=\text{value}$ which is the solution.}
\end{itemize}

\begin{MXInfo}{Set notation}
Using the set notation (with the solution set $\ML$) these cases can be expressed as follows:
\begin{itemize}
 \item{$\ML=\lbrace\rbrace$ or $L=\MEmptyset$ if there is no solution,}
 \item{$\ML=\lbrace \text{value}\rbrace$ if there is a single solution,}
 \item{$\ML=\R$ if all real numbers $x$ are a solution.}
\end{itemize}
\end{MXInfo}


\begin{MExample}
The linear equation $3x+2=2x-1$ has one solution. 
This solution is obtained by equivalent transformations:
$$
3x+2 \;=\; 2x-1 \;\;\MUnderset{-2x}\Leftrightarrow\;\; x+2\;=\;-1\;\;\MUnderset{-2}\Leftrightarrow\;\; x\;=\; -3 \MDFPeriod
$$
Hence, $x=-3$ is the only solution.
\end{MExample}

\begin{MExample}
The linear equation $3x+3=9x+9$ has the solution:
$$
3x+3 \;=\; 9x+9 \;\;\MUnderset{:(x+1)}\Leftrightarrow\;\; 3\;=\;9 \MDFPeriod
$$
This statement is wrong. Hence, for all $x\not=-1$ (transformation condition) the equation is wrong.
Inserting $x=-1$ satisfies the equation, and so the only solution is $x=-1$.

Alternatively, the equation could have been transformed as follows:
$$
3x+3 \;=\; 9x+9 \;\;\MUnderset{-3x-9}\Leftrightarrow\;\; -6 \;=\; 6x \;\; \Leftrightarrow\;\; x \;=\; -1 \MDFPeriod
$$
\end{MExample}

\begin{MExercise}
Transform the following linear equations and specify their solution sets:
\MInputHint{Enter simply \texttt{$\lbrace a\rbrace$} for a unit set and $\lbrace\rbrace$ for an empty set.}\ \\
\begin{MExerciseItems}
\item{The equation $x-1=1-x$ has the solution set \MEquationItem{$\ML$}{\MLParsedQuestion{4}{1,1}{4}{LUA1}},}
\item{The equation $4x-2=2x+2$ has the solution set \MEquationItem{$\ML$}{\MLParsedQuestion{4}{2,2}{4}{LUA2}},}
\item{The equation $2x-6=2x-10$ has the solution set \MEquationItem{$\ML$}{\MLParsedQuestion{4}{}{4}{LUA3}}.}
\end{MExerciseItems}

\begin{MHint}{Solution}
The first equation can be transformed into $2x=2$ or $x=1$, respectively, so 
the solution set is $\ML=\lbrace 1\rbrace$. The second equation can be transformed into $2x=4$ and the solution set is 
$\ML=\lbrace 2\rbrace$. The third equation can be transformed into $-6=-10$ which is a false statement, 
hence $\ML=\lbrace\rbrace$.
\end{MHint}
\end{MExercise}

\begin{MExercise}
Find the solution of the general linear equation $a x=b$ with $a$ and $b$ being real numbers.
Specify the values of $a$ and $b$ for which the following three cases occur:
\begin{itemize}
 \item{Every value of $x$ is a solution ($\ML=\R$) if 
\MEquationItem{$a$}{\MLParsedQuestion{4}{0}{4}{ALG1}} and $b=0$.}
 \item{There is no solution ($\ML=\MEmptyset$) if \MEquationItem{$a$}{\MLParsedQuestion{4}{0}{4}{ALG2}} 
and $b\not=$\MLParsedQuestion{4}{0}{4}{ALG3}.}
 \item{Otherwise, there is a single solution, namely 
\MEquationItem{$x$}{\MLSimplifyQuestion{5}{b/a}{10}{a,b}{10}{513}{VBKM02ALTFALL3}}.}
\end{itemize}

\begin{MHint}{Solution}
Every value of $x$ is a solution ($\ML=\R$) if $a=0$ and $b=0$.
There is no solution ($\ML=\MEmptyset$) if $a=0$ and $b\not=0$.
Otherwise, there is only one solution, namely $x=\Mtfrac{b}{a}$.
\end{MHint}

\end{MExercise}


\end{MXContent}

\begin{MXContent}{Solving quadratic Equations}{Quadratic Equations}{STD}
\MLabel{VBKM02_QuadratischeGleichungen}
\MDeclareSiteUXID{VBKM02_Quadratische Gleichungen}
\begin{MInfo}
A \MEntry{quadratic equation}{equation (quadratic)} is an equation of the form
  $a x^2 + b x + c = 0$ with $a\not=0$, or, in reduced form, $x^2+ p x + q=0$. 
This form is obtained by dividing the equation by $a$.

For a quadratic equation in one variable (here the variable $x$) one of the following three statements 
holds:
\begin{itemize}
 \item{The quadratic equation has no solution: $\ML=\lbrace\rbrace$.}
 \item{The quadratic equation has a single solution $\ML=\lbrace x_1\rbrace$.}
 \item{The quadratic equation has two different solutions $\ML=\lbrace x_1\MElSetSep x_2\rbrace$.}
\end{itemize}
\end{MInfo}

The solutions are obtained by applying \MEntry{quadratic solution formulas}{solution formulas}.

\begin{MInfo}
\MLabel{VBKM02_pqFormel}
The \MEntry{$p q$ formula}{pq formula} for solving the equation $x^2+p x + q = 0$ reads
$$
x_{1,2} \;=\; -\Mdfrac{p}{2}\pm \sqrt{\Mdfrac14p^2-q} \MDFPeriod
$$
Here, the equation has
\begin{itemize}
\item{no (real) solution if $\Mtfrac14p^2-q<0$ (taking the square root is not allowed),}
\item{a single solution $x_1=-\Mtfrac{p}{2}$ if $\Mtfrac14p^2=q$ and the square root is zero,}
\item{two different solutions if the square root is a positive number.}
\end{itemize}

The expression $D:=\Mtfrac14p^2-q$ underneath the square root considered above is called 
\textbf{discriminant}.
\end{MInfo}
The solution of a quadratic equation is often described by an alternative formula:
\begin{MInfo}
\MLabel{VBKM02_abcFormel}
For the equation $a x^2+b x + c = 0$ with $a\ne 0$ the
\MEntry{$a b c$ formula}{abc formula} reads
$$
x_{1,2} \;=\; \frac{-b\pm\sqrt{b^2-4 a c}}{2a} \MDFPeriod
$$
Here, the equation has
\begin{itemize}
\item{no (real) solution if $b^2-4 a c<0$ (the square root of a negative number is undefined within the
range of real numbers),}
\item{a single solution $x_1=-\Mtfrac{b}{2a}$ if $b^2=4 a c$ and the square root is zero,}
\item{two different solutions if the square root is a positive number.}
\end{itemize}
Again, the expression $D:=b^2-4 a c$ underneath the square root considered above is called 
\textbf{discriminant}.
\end{MInfo}
Certainly, both formulas result in the same solutions. (For applying the $pq$ formula, the quadratic 
equation is to be divided by the factor $a$ of the quadratic term.)
\medskip\par
This three different cases correspond to three different numbers of intersection points between 
the graph of a parabola opening upwards $f(x)=x^2+p x+ q$ and the $x$ axis (applying the $p q$ formula).

\begin{center}
%%\MUGraphicsSolo{para1b.png}{width=0.2\linewidth}{width:230px}\ \ 
%%\MUGraphicsSolo{para2b.png}{width=0.2\linewidth}{width:230px}\ \ 
%%\MUGraphicsSolo{para3b.png}{width=0.2\linewidth}{width:230px}
\MTikzAuto{%
\begin{tikzpicture}[x=1.0cm, y=1.0cm,scale=0.60] 
\foreach \sx/\fsy in {-7.0cm/0.5,0.0cm/0.0,7.0cm/-0.5} {
\begin{scope}[xshift=\sx,yshift=0]
\draw[black] (-3,0) -- (3,0) (0,-3) -- (0,3);
\foreach \x in {-3, -2, -1, 1, 2, 3}
\draw[shift={(\x,0)},color=black] (0pt,0pt) -- (0pt,-2.0pt) node[below=1.0pt] {\tiny $\x$};
\foreach \x in {-2.5, -1.5, ..., 2.5}
\draw[shift={(\x,0)},color=black] (0pt,0pt) -- (0pt,-1.0pt);
\foreach \y in {-3, -2, -1, 1, 2, 3}
\draw[shift={(0,\y)},color=black] (0pt,0pt) -- (-2.0pt,0pt) node[left=1.0pt] {\tiny $\y$};
\foreach \y in {-2.5, -1.5, ..., 2.5}
\draw[shift={(0,\y)},color=black] (0pt,0pt) -- (-1.0pt,0pt);
\draw[black] (-0.0pt,-0.0pt) node[anchor=north east] {\tiny $0$};
\clip(-3.0,-3.0) rectangle (3.0,3.0);
\draw[smooth,samples=27,domain=-3:3, line width=1.0pt,color=red!50!black] plot(\x,{\x*\x+\fsy});
\end{scope}
}
\end{tikzpicture}
}
\end{center}

Three cases: no intersection point, one intersection point and two intersection points with the
$x$ axis.


\begin{MExample}
The quadratic equation $x^2-x+1=0$ has no solution since the discriminant $\Mtfrac14p^2-q=-\Mtfrac34$
within the $p q$ formula is negative.
In contrast, the equation $x^2-x-1=0$ has two solutions
\begin{eqnarray*}
x_1 &=& \Mdfrac12+\sqrt{\Mdfrac14+1} \;=\; \Mdfrac12(1+\sqrt5) \MDFPSpace ,\ \\
x_2 &=& \Mdfrac12-\sqrt{\Mdfrac14+1} \;=\; \Mdfrac12(1-\sqrt5) \MDFPeriod
\end{eqnarray*}
\end{MExample}

\begin{MInfo}
\MLabel{VBKM02_Scheitelpunktform}
%%Eine quadratische Gleichung ist in \MEntry{Scheitelpunktform}{Scheitelpunktform}, wenn sie die Form $a\cdot (x-s)^2=d$ besitzt mit $a\not=0$ und $d\geq 0$.
%%F�r den Funktionsausdruck der zugeh�rigen Parabel liest sich diese Form $f(x)=a\cdot (x-s)^2-d$.
%%In dieser Situation ist $(s\,|\,-d)$ der \MEntry{Scheitelpunkt}{Scheitelpunkt (Parabel)} der Parabel.
%%
%%Falls $a>0$ ist gibt es zwei L�sungen
%%$$
%%x_1 \;=\; s-\sqrt{\Mdfrac{d}{a}} \MDFPSpace,\MDFPaSpace x_2 \;=\; s+\sqrt{\Mdfrac{d}{a}}
%%$$
%%der Gleichung, diese liegen symmetrisch um die $x$-Koordinate des Scheitelpunkts. F�r $d=0$ gibt es nur eine L�sung.
%%
%%Das Vorzeichen von $a$ bestimmt, ob die Gleichung eine nach oben oder unten ge�ffnete Parabel beschreibt.
The function expression of a parabola has \MEntry{vertex form}{vertex form} if the function
has the form $f(x)=a\cdot (x-s)^2-d$ with $a\ne 0$. In this case, $\MPointTwo{s}{-d}$
is the \MEntry{vertex}{vertex (parabola)} of the parabola. The corresponding 
quadratic equation for $f(x)=0$ then reads $a\cdot (x-s)^2=d$.

Dividing this equation by $a$ one obtains the equivalent quadratic equation 
$(x-s)^2=\frac{d}{a}$. Since the left-hand side is a square of a real number, only 
solutions exist if and only if the right-hand side is non-negative as well, i.e.
$\frac{d}{a}\ge 0$. By taking the square root, taking the two possible signs into
account, one obtains $x-s=\pm\sqrt{\frac{d}{a}}$.

So, for $\frac{d}{a}>0$ two solutions of the equation exist:
$$
x_1 \;=\; s-\sqrt{\Mdfrac{d}{a}} \MDFPSpace,\MDFPaSpace x_2 \;=\; s+\sqrt{\Mdfrac{d}{a}}\,;
$$
they are symmetric to the $x$ coordinate $s$ of the vertex. For $d=0$, only one
solution exists.

The sign of $a$ determines whether the function expression describes a parabola 
opening upwards or downwards. 
\end{MInfo}

The quadratic equation has only one single solution $s$ if it can be transformed into the form
$(x-s)^2=0$.

\begin{MInfo}
\MLabel{VBKM02_QuadratischErgaenzung}

Any quadratic equation can be transformed (after collecting terms on the left-hand side
and normalisation, if necessary) to vertex form by 
\MEntry{completing the square}{completing the square}. For this, a constant is
added to both sides of the equation such that on the left-hand side we have a term of the 
form $x^2\pm 2s x+s^2$ to which the first or second binomial formula can be applied.
\end{MInfo}

\begin{MExample}
Adding the constant $2$ transforms the equation $x^2-4x+2=0$ into the
form $x^2-4x+4=2$ or into the form $(x-2)^2=2$, respectively. From this,
the two solutions $x_1=2-\sqrt{2}$ and $2+\sqrt{2}$ can be seen immediately.
In contrast, the quadratic equation $x^2+x=-2$ has no solution since completing
the square results in $x^2+x+\Mtfrac14=-\Mtfrac74$ or $(x+\Mtfrac12)^2=-\Mtfrac74$, respectively, 
where the right-hand side is negative for $a=1$.
\end{MExample}

\begin{MExercise}
Find the solutions of the following quadratic equations by completing the square after 
collecting terms on the left-hand side and normalisation (i.e.\ selecting $a=1$):

\begin{MExerciseItems}
\item{$x^2=8x-1$ has the vertex form \MEquationItem{\MLSimplifyQuestion{20}{x^2-8*x+16}{5}{x}{5}{0}{MSPF1}}{\MLParsedQuestion{5}{15}{5}{GLD1}}.\\The solution set is \MEquationItem{$\ML$}{\MLParsedQuestion{30}{4-sqrt(15),4+sqrt(15)}{5}{SQRT1}}.}
\item{$x^2=2x+2+2x^2$ has the vertex form \MEquationItem{\MLSimplifyQuestion{20}{x^2+2*x+1}{5}{x}{5}{0}{MSPF2}}{\MLParsedQuestion{5}{-1}{5}{GLD2}}.\\The solution set is \MEquationItem{$\ML$}{\MLParsedQuestion{30}{}{5}{SQRT2}.}}
\item{$x^2-6x+18=-x^2+6x$ has the vertex form \MEquationItem{\MLSimplifyQuestion{20}{x^2-6*x+9}{5}{x}{5}{0}{MSPF3}}{\MLParsedQuestion{5}{0}{5}{GLD3}}.\\The solution set is \MEquationItem{$\ML$}{\MLParsedQuestion{30}{3,3}{5}{SQRT3}.}}
\end{MExerciseItems}
\MInputHint{Enter sets in the form \texttt{$\lbrace$ a;b;c;$\ldots\rbrace$}. Enter the empty set as $\lbrace\rbrace$.}

\begin{MHint}{Solution}
The transformations are for the first equation
\begin{eqnarray*}
&&x^2=8x-1 \ \\
&\Leftrightarrow&x^2-8x+1=0 \ \\
&\Leftrightarrow&x^2-8x+16=15 \ \\
&\Leftrightarrow&(x-4)^2=15 \ \\
&& \ML=\lbrace 4-\sqrt{15}\MElSetSep 4+\sqrt{15}\rbrace
\end{eqnarray*}
and for the second equation
\begin{eqnarray*}
&&x^2=2x+2+2x^2 \ \\
&\Leftrightarrow&x^2+2x+2=0 \ \\
&\Leftrightarrow&x^2+2x+1=-1 \ \\
&\Leftrightarrow&(x+1)^2=-1\ \\
&& \ML=\lbrace\rbrace
\end{eqnarray*}
and for the third equation
\begin{eqnarray*}
&&x^2-6x+18=-x^2+6x \ \\
&\Leftrightarrow&2x^2-12x+18=0 \ \\
&\Leftrightarrow&x^2-6x+9=0 \ \\
&\Leftrightarrow&(x-3)^2=0\ \\
&& \ML=\lbrace3\rbrace \MDFPeriod
\end{eqnarray*}
\end{MHint}

\end{MExercise}

\end{MXContent}


\MSubsection{Absolute Value Equations}
\MLabel{M02_Betragsgleichungen}

\begin{MIntro}
\MDeclareSiteUXID{VBKM02_Betragsgleichungen_Intro}
\MLabel{VBKM02_Betrag}
The absolute value $|x|$ assigns a variable $x\in\R$ its value without sign: If $x\geq 0$, then $|x|=x$, 
otherwise $|x|=-x$ (see figure).

\begin{center}
%%\MUGraphicsSolo{betrag1.png}{width=0.4\linewidth}{width:350px}\\
\MTikzAuto{%
\begin{tikzpicture}[x=1.0cm, y=1.0cm] 
\draw[black] (-3,0) -- (3,0) (0,-3) -- (0,3);
\foreach \x in {-3, -2, -1, 1, 2, 3}
\draw[shift={(\x,0)},color=black] (0pt,0pt) -- (0pt,-2.0pt) node[below=1.0pt] {\scriptsize $\x$};
\foreach \x in {-2.5, -1.5, ..., 2.5}
\draw[shift={(\x,0)},color=black] (0pt,0pt) -- (0pt,-1.0pt);
\foreach \y in {-3, -2, -1, 1, 2, 3}
\draw[shift={(0,\y)},color=black] (0pt,0pt) -- (-2.0pt,0pt) node[left=1.0pt] {\scriptsize $\y$};
\foreach \y in {-2.5, -1.5, ..., 2.5}
\draw[shift={(0,\y)},color=black] (0pt,0pt) -- (-1.0pt,0pt);
\draw[black] (-0.0pt,-0.5pt) node[anchor=north east] {\scriptsize $0$};
\clip(-3.0,-3.0) rectangle (3.0,3.0);
\draw[black, line width=1.0pt,color=red!50!black] (-3,3) -- (0,0) -- (3,3);
\end{tikzpicture}
}
\par
The absolute value $|x|$ as a function of $x$.
\end{center}

Absolute value equations are equations in which one absolute value or several absolute values occur. 
Problems arise since the absolute value is calculated by distinguishing the two cases  
$$
\left|{\,\text{term}\,}\right| \;\; =\;\; \MCases{\text{term} & \text{for}\;\text{term}\geq 0\\ -\text{term} & \text{for}\;\text{Term}<0}\,.
$$
For solving absolute value equations, these cases have to be solved step by step 
and analysed to find the solutions.

\begin{MExample}
Obviously, the absolute value equation $|x|=2$ has the solution set $\ML=\lbrace 2\MElSetSep -2\rbrace$.
Just as easy, it can be seen that $|x-1|=3$ has the solution set $\ML=\lbrace -2\MElSetSep 4\rbrace$.
\end{MExample}

As soon as beside the absolute value several other terms occur a case analysis is required.
In the following section we will explain in detail how this analysis is done and how it is 
written correctly since the case analysis will play an important role in the next modules.
\end{MIntro}

\begin{MXContent}{Carry out a Case Analysis}{Case Analysis}{STD}
\MDeclareSiteUXID{VBKM02_Fallunterscheidungen}

\begin{MInfo}
\MLabel{VBKM02_FallBetrag}
To solve an \MEntry{absolute value equation}{absolute value equation} two cases are distinguished:
\begin{itemize}
\item{For all values of $x$ for which the absolute value term is non-negative the absolute value can be omitted or
replaced by simple brackets, respectively.}
\item{For all values of $x$ for which the absolute value term is negative the term is bracketed and negated.}
\end{itemize}
Then, the solution sets from the case analyses will be restricted to satisfy the case conditions. 
Only if this procedure is finished for all cases, the solution subsets will be merged to the 
solution set of the initial equation. 
\end{MInfo}

For solving absolute value equations it is important to write down the solution steps correctly
and to distinguish the cases clearly.

The following video demonstrates a detailed written solution of the absolute value
equation $|2x-4|=6$ by case analysis.


\MVideo{vidbsp1}{Carry out a case analysis.\MCopyrightLabel{VBKM06_Video_Beispiel1}}
\ \\
\MCopyrightNotice{\MCCLicense}{FSZ}{MINT}{Dozentin: Dipl.-ing. Heike Herold}{VBKM06_Video_Beispiel1}

The case analysis presented in the video reads shortly
$$
|2x-4| \;=\; \MCases{2x-4 &\text{for}\;x\geq 2 \\ -2x+4 & \text{for}\;x<2} \;=\;\MCases{2x-4 &\text{for}\;x\geq 2 \\ -2x+4 & \text{otherwise}} \MDFPeriod
$$
\MInputHint{In einem Eingabefeld w�rde man das als \texttt{falls(x>=2;2*x-4;-2*x+4)} eintippen. Auch \texttt{falls(x<2;-2*x+4;2*x-4)} w�re richtig.}

\begin{MExercise}
Describe the values of the expression $2\cdot |x-4|$ by a case analysis:
\begin{center}
$\displaystyle 2\cdot |x-4|$ = \MLSimplifyQuestion{30}{falls(x>=4,2*(x-4),-2*(x-4))}{15}{x}{4}{128}{VBKM02ALTFALL1}.
\end{center}
\MInputHint{Enter the case analysis in the form \texttt{falls(BEDINGUNG;W1;W2)}, where \texttt{W1} is the value of the expression if the corresponding condition is satisfied. Do not use the absolute value function.}
\begin{MHint}{Solution}
\begin{eqnarray*}
2\cdot |x-4| \;=\; \MCases{2x-8 &\text{for}\;x\geq 4 \\ -2x+8 & \text{for}\;x<4}
\end{eqnarray*}
\end{MHint}
\end{MExercise}

\begin{MExercise}
Reproduce the steps shown in the video \MRef{VBKM06_Video_Beispiel1} to solve the absolute 
value equation $|6+3x|=12$.

The case analysis reads shortly $|6+3x|$ = \MLSimplifyQuestion{40}{for(x>=-2,6+3*x,-6-3*x)}{15}{x}{3}{128}{VBKM02ALTFALL2}.
\begin{MHint}{Solution}
\begin{eqnarray*}
|6+3x| \;=\; \MCases{6+3x &\text{for}\;x\geq -2 \\ -6-3x & \text{for}\;x<-2}
\end{eqnarray*}
\end{MHint}\\
\MInputHint{Enter the case analysis in the form \texttt{falls(BEDINGUNG;W1;W2)}, where \texttt{W1} is the value of the expression if the corresponding condition is satisfied.
You can copy one of the input examples into the input field and adapt it to the new equation.}
\ \\ \ \\
Finding the solution for each case and checking the case conditions leads to the solution set
\MEquationItem{$\ML$}{\MLParsedQuestion{10}{2,-6}{3}{LVI}} for the equation $|6+3x|=12$.

\begin{MHint}{Solution}
\begin{eqnarray*}
\ML \;=\; \{-6\MElSetSep 2\}
\end{eqnarray*}
\end{MHint}\\
\MInputHint{Mengen k�nnen in der Form \texttt{$\lbrace$a;b;c;\ldots$\rbrace$} eingegeben werden.}
\end{MExercise}

You can practise the stepwise solution of absolute value equations within the following exercise. 


\MDirectRouletteExercises{abs_equations.rtex}{VBKM02_ABSEQTRAINING}


\end{MXContent}

\begin{MXContent}{Mixed Equations}{Mixed Equations}{STD}
\MDeclareSiteUXID{VBKM02_GemischteGleichungen}

\begin{MInfo}
If an equation contains both absolute values and other expressions, the case analysis 
has to be done according to the absolute value terms and applied only to these.
\end{MInfo}

Finally, keep in mind to crosscheck the found solution sets with the case conditions.

\begin{MExample}
Solve the equation $|x-1|+x^2=1$. Here, the case analysis is as follows:
\begin{itemize}
\item{For $x\geq 1$, the absolute value bars can be replaced by normal brackets which results
in the quadratic equation $(x-1)+x^2=1$ that is transformed into the equation $x^2+x-2=0$.
Using the $p q$ formula we get the two solutions
\begin{eqnarray*}
x_1 & =& -\Mdfrac12-\sqrt{\Mdfrac94}\;=\; -2 \MDFPSpace, \\
x_2 & =& -\Mdfrac12+\sqrt{\Mdfrac94}\;=\; 1 \MDFPSpace
\end{eqnarray*}
of which only $x_2$ satisfies the case condition.
}
\item{For $x<1$, one obtains the quadratic equation $-(x-1)+x^2=1$ that is 
transformed into the equation $x^2-x=0$ or $x\cdot (x-1)=0$, respectively. The product representation 
indicates the two solutions $x_3=0$ and $x_4=1$. Because of the case condition only 
$x_3=0$ is a solution of the initial equation.
}
\end{itemize}
So, altogether $\ML=\lbrace 0\MElSetSep 1\rbrace$ is the solution set of the 
initial equation.
\end{MExample}

%TODO: Intervalle sind hier noch garnicht erklaert
%TODO: Kodierung von offenen Intervallgrenzen in "\MLIntervalQuestion" (?!)
\begin{MExercise}
Find the solution set of the mixed equation $|x-3|\cdot x=9$.
\begin{MExerciseItems}
\item{If $x$ is in the interval \MLIntervalQuestion{14}{[3,infty)}{5}{GIM1} the  
absolute value term is non-negative.\\ One obtains the quadratic equation 
\MEquationItem{\MLFunctionQuestion{15}{x^2-3*x-9}{5}{x}{5}{GIM2}}{$0$}.\\
The solution set is \MLParsedQuestion{35}{3/2-sqrt(45/4)\MElSetSep 3/2+sqrt(45/4)}{5}{GIM3}.\\
Only the solution \MLParsedQuestion{15}{3/2+sqrt(45/4)}{5}{GIM4} satisfies the case condition.}

\item{If $x$ is in the interval \MLIntervalQuestion{14}{(-infty,3)}{5}{GIM5} the absolute 
value term is negative.\\  One obtains the normalised quadratic equation 
\MEquationItem{\MLFunctionQuestion{15}{x^2-3*x+9}{5}{x}{5}{GIM6}}{$0$}.\\
The solution set is \MLParsedQuestion{30}{}{5}{GIM7}.}
\end{MExerciseItems}
%TODO: Text und Kodierung anpassen
\MInputHint{Enter open intervals in the form $(3;5)$ and closed intervals in the form $[3;5]$. Enter
``infinity'' as text or simply as \texttt{infty}. Do not use the notation $]a;b[$ fir open intervals. 
Sets can be entered by listing the elements in the form $\lbrace 1\MElSetSep 2\MElSetSep 3\rbrace$. For 
the set brackets enter AltGr+7 or AltGr+0, respectively.} 

So, altogether the solution set is \MEquationItem{$\ML$}{\MLParsedQuestion{20}{3/2+sqrt(45/4),3/2+sqrt(45/4)}{5}{GIM8}}.

\begin{MHint}{Solution}
If $x$ is in the interval $\left[3\MIntvlSep \infty\MoIr[\right]$ the absolute value term is non-negative
and one obtains the quadratic equation $x^2-3x-9=0$ with the solution set 
$\ML=\lbrace \Mtfrac32-\sqrt{\Mtfrac{45}{4}};\Mtfrac32+\sqrt{\Mtfrac{45}{4}}\rbrace$. Only the 
larger solution $\Mtfrac32+\sqrt{\Mtfrac{45}{4}}$ satisfies the condition $x\geq 3$. This can 
also be seen without any calculator by estimating $\sqrt{\Mtfrac{45}{4}}\geq\sqrt{\Mtfrac{36}{4}}=3$.
In contrast, if $x$ is in the interval $\MoIl[\left]-\infty\MIntvlSep 3\MoIr[\right]$
the absolute value term is negative. One obtains the normalised quadratic equation $x^2-3x+9=0$.
Because of $\Mtfrac14p^2-q<0$ in the $pq$ formula this equation is unsolvable. Hence, the initial
equation has only one solution $\Mtfrac32+\sqrt{\Mtfrac{45}{4}}$.
\end{MHint}
\end{MExercise}

\begin{MExercise}
Find the solutions of the mixed absolute value equation $3|2x+1|=|x-5|$ by visualising 
the different cases on the number line and finally solving the equation by case analysis. 
First, visualise the case analysis for each absolute value. 

The solution set is \MLParsedQuestion{10}{-8/5,2/7}{5}{PARSEDQUEST1}.

\begin{MHint}{Solution}
Visualising the different cases for the expressions $|2x+1|$ and $|x-5|$ above each other 
indicates all cases to be distinguished:

%%\MGraphics{fallunterscheidung3.png}{scale=1}{Graphische Darstellung der drei F�lle.\MCopyrightLabel{VBKM02_Grafik_Fallunterscheidung3}}
%%\MCopyrightNotice{\MCCLicense}{NONE}{VEMINT}{Im Rahmen des VE\&MINT-Projekts}{VBKM02_Grafik_Fallunterscheidung3}
\begin{center}
\MTikzAuto{%
\begin{tikzpicture}[x=0.35cm, y=1.0cm] 
\fill[color=yellow!20!white] (-7,-1.2) rectangle (15,6.2);
\foreach \sy in {0.0cm,2.5cm,5.0cm} {
\begin{scope}[xshift=0,yshift=\sy]
\draw[-stealth',black] (-6,0) -- (13.5,0);
\foreach \x in {-4, 0, 4, 8, 12}
\draw[shift={(\x,0)},color=black] (0pt,0pt) -- (0pt,3.0pt) (0pt,0pt) node[below=1.0pt] {\small $\x$};
\foreach \x in {-5, -3, -2, -1, 1, 2, 3, 5, 6, 7, 9, 10, 11}
\draw[shift={(\x,0)},color=black] (0pt,0pt) -- (0pt,1.0pt);
\end{scope}
\def\rdst{0.2}
\def\tdst{0.6}
\def\vdst{0.4}
\draw[color=red,line width=2.2pt] (-6.0,\rdst) -- (-0.5,\rdst);
\draw[color=green!60!black,line width=2.2pt] (-0.5,\rdst) -- (5.0,\rdst);
\draw[color=orange,line width=2.2pt] (5.0,\rdst) -- (12.0,\rdst);
\draw[color=red,line width=2.2pt] (-3.25,\tdst) node {Case (1)};
\draw[color=green!60!black,line width=2.2pt] (2.25,\tdst) node {Case (2)};
\draw[color=orange,line width=2.2pt] (8.5,\tdst) node {Case (3)};
\draw[color=blue,line width=0.8pt] (-0.5,\vdst) -- (-0.5,-0.5) node[below=0pt] {$-\frac{1}{2}$} ;
\draw[color=blue,line width=0.8pt] (5.0,\vdst) -- (5.0,-0.5) node[below=0pt] {$5$} ;
\begin{scope}[xshift=0,yshift=2.5cm]
\draw[color=red,line width=2.2pt] (-6.0,\rdst) -- (5.0,\rdst);
\draw[color=orange,line width=2.2pt] (5.0,\rdst) -- (12.0,\rdst);
\draw[color=red,line width=2.2pt] (-0.50,\tdst) node {$x-5<0$};
\draw[color=orange,line width=2.2pt] (8.5,\tdst) node {$x-5>0$};
\draw[color=blue,line width=0.8pt] (5.0,\vdst) -- (5.0,-0.5) node[below=0pt] {$5$} ;
\end{scope}
\begin{scope}[xshift=0,yshift=5cm]
\draw[color=red,line width=2.2pt] (-6.0,\rdst) -- (-0.5,\rdst);
\draw[color=orange,line width=2.2pt] (-0.5,\rdst) -- (12.0,\rdst);
\draw[color=red,line width=2.2pt] (-3.25,\tdst) node {$2x+1<0$};
\draw[color=orange,line width=2.2pt] (5.75,\tdst) node {$2x+1>0$};
\draw[color=blue,line width=0.8pt] (-0.5,\vdst) -- (-0.5,-0.5) node[below=0pt] {$-\frac{1}{2}$} ;
\end{scope}
}
\end{tikzpicture}
}
\par
Illustration of the three cases
\end{center}

According to the figure above the following three cases have to be distinguished:
\begin{itemize}
\item{Case (1): For $x<-\Mtfrac12$ both terms in the absolute value terms are negative.}
\item{Case (2): For $-\Mtfrac12\leq x<5$ the term in the second absolute value term is negative but the 
term in the first one is not.}
\item{Case (3): For $5\leq x$ both terms in the absolute value terms are non-negative.}
\item{Obviously, there is no $x$ for which the first term is negative and the second 
term is non-negative.}
\end{itemize}

So, the solutions can be summarised:
\begin{itemize}
\item{In case (1), both absolute values reverse the sign of the terms:\\ $3|2x+1|=|x-5|\;\Leftrightarrow\;3(-(2x+1)) = -(x-5)$.\\
This equation has the solution $x=-\Mtfrac85$ satisfying the case condition.}
\item{In case (2), only the second absolute value reverses the sign of the term: \\$3|2x+1|=|x-5|\;\Leftrightarrow\;3(2x+1) = -(x-5)$.\\
This equation has the solution $x=\Mtfrac27$ satisfying the case condition.}
\item{In case (3), the absolute value bars in both terms can be omitted (replaced by normal brackets): \\
 $3|2x+1|=|x-5|\;\Leftrightarrow\;3(2x+1) = (x-5)$.\\
This equation has the solution $x=-\Mtfrac85$, but this solution does \textit{not} 
satisfy the case condition. Thus, it will be discarded within its case analysis.}
\end{itemize}
Therefore, the solution set is $\lbrace -\Mtfrac85\MElSetSep \Mtfrac27\rbrace$.
\end{MHint}
\end{MExercise}


\end{MXContent}

\MSubsection{Final Test}
\MLabel{M02_Abschlusstest}

\begin{MTest}{Final Test Modul 2}
\MDeclareSiteUXID{VBKM02_Abschlusstest}

\begin{MExercise}
Find an absolute value term describing the following graph of a function as easy as possible:

%%\MUGraphics{abs2.png}{width=0.5\linewidth}{Funktionsgraph von $f(x)$.}{width:300px}
\begin{center}
\MTikzAuto{%
\begin{tikzpicture}[x=0.8cm, y=1.12cm] 
\draw[black] (-5,0) -- (5,0) (0,-5) -- (0,5);
\foreach \x in {-4, -2, 2, 4}
\draw[shift={(\x,0)},color=black] (0pt,0pt) -- (0pt,-3.0pt) node[below=1.0pt] {\normalsize $\x$};
\foreach \x in {-4.8, -4.4, ..., 5.0}
\draw[shift={(\x,0)},color=black] (0pt,0pt) -- (0pt,-1.5pt);
\foreach \y in {-4, -2, 2, 4}
\draw[shift={(0,\y)},color=black] (0pt,0pt) -- (-3.0pt,0pt) node[left=1.0pt] {\normalsize $\y$};
\foreach \y in {-4.8, -4.4, ..., 5.0}
\draw[shift={(0,\y)},color=black] (0pt,0pt) -- (-1.5pt,0pt);
%%\draw[black] (-0.0pt,-0.5pt) node[anchor=north east] {\small $0$};
\clip(-5.0,-5.0) rectangle (5.0,5.0);
\draw[black, line width=1.0pt,color=black] (-1,6) -- (3,-2) -- (5,2);
\end{tikzpicture}
}
\par
Graph of the function $f(x)$.
\end{center}

Answer: $f(x)=$ \MLSimplifyQuestion{20}{2*abs(x-3)-2}{5}{x}{5}{0}{KA0}\: .
\end{MExercise}

\begin{MExercise}
Solve the following equations:
\begin{MExerciseItems}
\item{$|2x-3|=8$ has the solution set \MLParsedQuestion{16}{11/2,-5/2}{5}{KA1}.}
\item{$|x-2|\cdot x=0$ has the solution set \MLParsedQuestion{16}{0,2}{5}{KA2}.} 
\end{MExerciseItems}
\MInputHint{Enter sets in the form \texttt{$\lbrace$ a;b;c;$\ldots\rbrace$}. 
Enter the empty set as $\lbrace\rbrace$.}
\end{MExercise}

\begin{MExercise}
A camera has a resolution of $6$ megapixels, i.e. -- for convenience -- 
6 million pixels and produces images in format $2:3$. Which size has a 
quadratic pixel on a print-out of format $(60$ cm$) \times (40$ cm$)$? 
Specify the side length of a pixel in millimetre. 

Answer: \MLParsedQuestion{10}{0.2}{5}{VPIXQ}\ \ \ \ (without the unit mm).
\end{MExercise}

\begin{MExercise}
Find the solution set of the mixed equation $|x-1|\cdot (x+1)=3$.\\
Answer: \MEquationItem{$\ML$}{\MLParsedQuestion{15}{2,2}{5}{KAX4}}.
\end{MExercise}


\end{MTest}


\newpage
\MPrintIndex

\end{document}

\MSection{Inequalities in one Variable}
\MLabel{VBKM03}
\MSetSectionID{ungl}

\begin{MSectionStart}
\MDeclareSiteUXID{VBKM03_START}

Inequalities arise by relating terms using one of the comparing symbols $\leq$, $<$, $\geq$, or $>$. Simple 
inequalities usually have intervals as their solution sets. But solving inequalities is often
more difficult than solving equations. Hence, specific types of inequalities will be explained
in more detail.

%This module consists of:

%\begin{itemize}
%\item{Section~\MNRef{M03_Ungleichungen}: \MSRef{M03_Ungleichungen}{Inequalities and their Solution Sets},}
%\item{Section~\MNRef{M03_Umformen}: \MSRef{M03_Umformen}{Transformation of Inequalities},}
%\item{Section~\MNRef{M03_Betragsungleichungen}: \MSRef{M03_Betragsungleichungen}{Absolute Value Inequalities and 
%Quadratic Inequalities},}
%\item{and Section~\MNRef{M03_Abschlusstest}: \MSRef{M03_Abschlusstest}{Final Test}.}
%\end{itemize}
\MModstartBox
\end{MSectionStart}

\MSubsection{Inequalities and their Solution Sets}
\MLabel{M03_Ungleichungen}

\begin{MIntro}

\begin{MInfo}
\MDeclareSiteUXID{VBKM03_UngleichungenIntro}
If two numbers are related by one of the \MEntry{comparing symbols}{comparing symbols} 
$\leq$, $<$, $\geq$, or $>$, a statement is generated that can be true or false depending on 
the numbers:
\begin{itemize}
\item{$a<b$ (reads: ``$a$ is strictly less than $b$'' or simply ``$a$ is less than $b$'') is true if the number $a$ is less than and not equal to $b$.}
\item{$a \leq b$ (reads: ``$a$ is less than $b$'') is true if $a$ is less than or equal to $b$.}
\item{$a>b$ (reads: ``$a$ is strictly greater than $b$'' or simply ``$a$ is greater than $b$'') is true if the number $a$ is greater and not equal to $b$.}
\item{$a \geq b$ (reads: ``$a$ is greater than $b$'') is true if the number $a$ is greater than or equal to $b$.}
\end{itemize}
\end{MInfo}

The relating symbols describe how the given values are related to each other on the number line: 
$a<b$ means that $a$ is to the left of $b$ on the number line.

\begin{MExample}
The statements $2<4$, $-12\leq 2$, $4>1$, and $3\geq 3$ are true,
but the statements $2<\sqrt2$ and $3>3$ are false.

\begin{center}
\MTikzAuto{%
\begin{tikzpicture}
% reelle Achse
\draw[->,color=black] (-1,0.0) -- (5,0.0);
\foreach \x in {-1, 0, 1, 2, 3, 4}
\draw[shift={(\x,0)},color=black] (0pt,2pt) -- (0pt,-2pt) node[below] {\footnotesize $\x$};
\draw (4.9,-0.3) node[] {$\mathbb{R}$};
% Enden:
\draw [fill = blue] (2,0) circle (1.5pt);
\draw [fill = blue] (4,0) circle (1.5pt);
\end{tikzpicture}
}%

On the number line, the number $2$ is to the left of the number $4$, thus $2<4$.
\end{center}

\end{MExample}

Here, $a<b$ means the same as $b>a$, likewise $a\leq b$ means the same as $b\geq a$. But it
should be noted that the opposite of the statement $a<b$ is the statement $a\geq b$ and not
$a>b$. If terms with a variable occur in an inequality, the problem is to find the number range
of the variable such that the inequality is true. 
\end{MIntro}

\begin{MXContent}{Solving simple Inequalities}{Solving}{STD}
\MDeclareSiteUXID{VBKM03_EinfacheUngleichungen}

If the variable occurs isolated in the inequality, the solution set is an interval, see also info
box \MRef{VBKM01_Intervalle}: 


\begin{MInfo}
\MLabel{M03_Aufloesungen}
The \MEntry{solved inequalities}{inequalities (solved)} 
have the following \MEntry{intervals}{intervals} as their solution sets:
\begin{itemize}
\item{$x< a$ has the solution set $\MoIl[\left] -\infty\MIntvlSep a\MoIr[\right]$, i.e.\ all $x$ less than $a$.}
\item{$x\leq a$ has the solution set $\MoIl[\left] -\infty\MIntvlSep \right]$, 
i.e. all $x$ less than or equal to $a$.}
\item{$x> a$ has the solution set $\MoIl[\left] a\MIntvlSep \infty\MoIr[\right]$, i.e.\ all
 $x$ greater than $a$.}
\item{$x\geq a$ has the solution set $\left[a\MIntvlSep \infty\MoIr[\right]$, i.e.\ all $x$
greater than or equal to $a$.}
\end{itemize}
Here, $x$ is the variable and $a$ is a specific value. 
If the variable does not occur in the inequality anymore, the solution set is either
$\R=\MoIl[\left] -\infty\MIntvlSep \infty\MoIr[\right]$ if the inequality is satisfied, 
or the empty set $\lbrace \rbrace$ if the inequality is not satisfied.
\end{MInfo}

The symbol $\infty$ means \MEntry{infinity}{infinity}. A finite interval has the form 
$\MoIl a\MIntvlSep b\MoIr$ which reads ``all numbers between $a$ and $b$''. If the interval
is to be finite only on one side, the other interval boundary can be replaced by the symbol 
$\infty$ (right-hand side) or $-\infty$ (left-hand side).

As for equations one tries to find a solved inequality by applying transformations that do
not change the solution set. From the solved inequality the solution set can be easily seen.


\begin{MInfo}
\MLabel{VBKM03_AequivalenzumformungenUngleichungen}
To obtain a solved inequality from an unsolved inequality the following 
\MEntry{equivalent transformations}{equivalent transformations (inequality)} are allowed:
\begin{itemize}
\item{adding a constant to both sides of the inequality: $a<b$ is equivalent to $a+c<b+c$.}
\item{multiplying both sides of the inequality by a positive constant: $a<b$ 
is equivalent to $a\cdot c<b\cdot c$ if $c>0$.}
\item{multiplying both sides of the inequality by a negative constant and inverting the 
comparing symbol: $a<b$ is equivalent to $a\cdot c>b\cdot c$ if $c<0$.}
\end{itemize}
\end{MInfo}

\begin{MExample}
The inequality $-\frac34x-\frac12<2$ is solved stepwise by the above transformations:
\begin{eqnarray*}
&&-\frac34x-\frac12 < 2 \;\; \MSep +\frac12\ \\
&\Leftrightarrow&-\frac34x < 2+\frac12 \;\; \MSep \cdot\left({-\frac43}\right)\ \\
&\Leftrightarrow&x > -\frac43\left({2+\frac12}\right) \;\; \MSep \;\text{simplifying}\\
&\Leftrightarrow&x >  -\frac{20}{6} \;=\; -\frac{10}{3} \MDFPeriod
\end{eqnarray*}
So, the initial inequality has the solution set 
 $\MoIl[\left] -\frac{10}{3}\MIntvlSep \infty\MoIr[\right]$. 
Importantly, multiplying the inequality by the negative number $-\frac43$ inverts the 
comparing symbol.
\end{MExample}

\begin{MExercise}
Are the following inequalities true or false?

\begin{MQuestionGroup}
\begin{tabular}{lll}
\MCheckbox{0}{UG1} & \ \ &  $\frac12>1-\frac13$\\
\MCheckbox{1}{UG2} & \ \ & $a^2\geq 2a b-b^2$ (where $a$ and $b$ are unknown numbers)\\
\MCheckbox{1}{UG3} & \ \ & $\frac12<\frac23<\frac34$\\
\MCheckbox{0}{UG4} & \ \ & Let $a<b$, then also $a^2<b^2$.
\end{tabular}
\end{MQuestionGroup}
\MGroupButton{Check input}

\begin{MHint}{Solution}
The first inequality can be simplified to $\frac12>\frac23$, which, after multiplying by $6$, 
is equivalent to $3>4$. This statement is false. The second inequality can be simplified by 
collecting all numbers on the left-hand side: $a^2-2a b+b^2\geq 0$. Since $a^2-2a b+b^2=(a-b)^2$,
this statement is true for all $a$ and $b$. Multiplying the third chain of inequalities by the
least common denominator $12$ results in the chain of inequalities $6<8<9$. This statement is true.
In contrast, the last statement is false, since for example, for $a=-1$ and $b=1$, the term
$a^2=1$ is not less than $b^2=1$. Taking the square of terms is not an equivalent transformation.
\end{MHint}
\end{MExercise}


\begin{MExercise}
Find the solution sets of the following inequalities.
\begin{MExerciseItems}
\item{$2x+1> 3x-1$ has the solution interval \MEquationItem{$\ML$}{\MLIntervalQuestion{30}{(-infty,2)}{5}{TXH1}}.}
\item{$-3x-\frac12\leq x+\frac12$ has the solution interval \MEquationItem{$\ML$}{\MLIntervalQuestion{30}{[-1/4,infty)}{5}{TXH2}}.}
\item{$x-\frac12\leq x+\frac12$ has the solution interval \MEquationItem{$\ML$}{\MLIntervalQuestion{30}{(-infty,infty)}{5}{TXH3}}.}
\end{MExerciseItems}
\MInputHint{Enter the intervals in the form $[a;b]$, $\MoIl a; b]$, etc., for the interval boundaries also fractions and
\texttt{infinity} or \texttt{-infinity} can be used. Take care whether the interval boundaries are included 
or excluded.}

\begin{MHint}{Solution}
Transformation of the first inequality results in
\begin{eqnarray*}
&& 2x+1 > 3x-1\;\; \MSep+1\ \\
&\Leftrightarrow& 2x+2 > 3x\;\; \MSep-2x\ \\
&\Leftrightarrow&2 > x
\end{eqnarray*}
and hence the solution interval is $\ML=\MoIl[\left] -\infty\MIntvlSep 2\MoIr[\right]$. 
Transformation of the second inequality results in
\begin{eqnarray*}
&&-3x-\frac12\leq x+\frac12 \;\; \MSep +3x-\frac12\ \\
&\Leftrightarrow&-1\leq 4x \;\; \MSep \cdot \frac14\ \\
&\Leftrightarrow&-\frac14\leq  x
\end{eqnarray*}
and hence $\ML=\left[-\frac14\MIntvlSep \infty\MoIr[\right]$. 
Transformation of the third inequality results in
\begin{eqnarray*}
&&x-\frac12\leq x+\frac12\;\; \MSep-x\ \\
&\Leftrightarrow&-\frac12\leq \frac12 \MDFPeriod
\end{eqnarray*}
This statement does not depend on $x\in\R$ and is always true, 
thus the solution set is $\ML=\R=\MoIl[\left] -\infty\MIntvlSep \infty\MoIr[\right]$.
\end{MHint}
\end{MExercise}

\begin{MInfo}
An inequality in one variable $x$ is \MEntry{linear}{inequality (linear)} if on both sides of the 
inequality only multiples of $x$ and constants occur. Each linear inequality can be transformed 
into a solved inequality by one of the equivalent transformations described in the info box
\MRef{M03_Aufloesungen}.
\end{MInfo}

\end{MXContent}

\begin{MXContent}{Specific Transformations}{Specific Transformations}{STD}
\MDeclareSiteUXID{VBKM03_SpezielleUmformungen}
The following equivalent transformations are useful if the variable occurs 
in the denominator of an expression. But they can only be applied under certain
restrictions:

\begin{MInfo}
Under the restriction that none of the occurring denominators is zero (the corresponding variable values are
 never solutions) and the fractions on both sides have the same sign, the reciprocal can be taken
on both sides of the inequality while inverting the comparing symbol.
\end{MInfo}

\begin{MExample}
For example, the inequality $\frac1{2x}\leq \frac1{3x}$ is equivalent to $2x\geq 3x$
(comparing symbol inverted) as long as $x\not=0$. The new inequality has the solution
set $\MoIl[\left] -\infty\MIntvlSep 0\right]$. However, since the value $x=0$ was excluded (and 
does not belong to the domain of the initial inequality either) the solution set of 
$\frac1{2x}\leq \frac1{3x}$ is $\ML=\MoIl[\left] -\infty\MIntvlSep 0\MoIr[\right]$.
\end{MExample}

\begin{MExercise}
Find the solution sets of the following inequalities.
\begin{MExerciseItems}
\item{$\frac1x\geq\frac13$ has the solution set \MEquationItem{$\ML$}{\MLIntervalQuestion{20}{(0;3]}{3}{KKL1}}.}
\item{$\frac1x<\frac1{\sqrt{x}}$ has the solution set \MEquationItem{$\ML$}{\MLIntervalQuestion{20}{(1;infty)}{3}{KKL2}}.}
\end{MExerciseItems}

\begin{MHint}{Solution}
For the first inequality, the value $x=0$ is not in the domain, hence this value is excluded. For 
$x>0$, taking the reciprocal while inverting the comparing symbol is allowed and results in
$x\leq 3$. Together with the condition above the solution interval is $\ML=\MoIl 0\MIntvlSep 3]$.
For $x<0$ the reciprocal rule cannot be applied. However, it can be seen, even without any rule, that
none of the values $x<0$ can be a solution, since then $\frac1x$ is negative as well and not greater than
or equal to $\frac13$.

The domain of the second inequality is $\MoIl 0\MIntvlSep \infty\MoIr$, since 
only for these values of $x$ taking the square root is
defined and only for $x\neq 0$ the denominators are non-zero. On the domain, 
taking the reciprocal while inverting the comparing symbol is 
allowed and results in $x>\sqrt{x}$. Since $\sqrt{x}>0$, the inequality can be 
divided by $\sqrt{x}$ resulting in $\sqrt{x}>1$. This inequality has the solution
set $\ML=\MoIl 1\MIntvlSep \infty\MoIr$ which occurs also in the domain.
\end{MHint}

\end{MExercise}

Please note for the last part of the exercise:

\begin{MInfo}
Taking the square on both sides of an inequality is not an equivalent transformation and 
possibly does change the solution set.
\end{MInfo}

For example, $x=-2$ is no solution of $x>\sqrt{x}$, but indeed a solution of $x^2>x$. However,
this transformation can be applied if the case analysis for the transformation
is carried out correctly and the domain of the initial inequality is taken into account. This method is
described in more detail in the next section.
\end{MXContent}

\MSubsection{Transformation of Inequalities}
\MLabel{M03_Umformen}

\begin{MXContent}{Transformation with Case Analysis}{Case Analysis}{STD}
\MDeclareSiteUXID{VBKM03_UmformungenFallunterscheidungen}
The simple linear transformations described in the previous section are equivalent transformations.
They do not change the solution set of the corresponding inequality. For nonlinear
inequalities advanced solution methods are required. Usually, these methods need 
a case analysis depending on the sign, since, in contrast to the situation for
equations described in Modul~\MNRef{VBKM02}, now also the inequality can be inverted during 
transformation.

 
\begin{MInfo}
If an inequality is multiplied by a term in which the variable $x$ occurs, a case analysis 
is required and for each case the transformation has to be considered separately: 

\begin{itemize}
\item{For those values of $x$, for which the multiplied term is positive, the comparing symbol
of the inequality is unchanged.}
\item{For those values of $x$, for which the multiplied term is negative, the comparing symbol
of the inequality is inverted.}
\item{The case that the multiplied term is zero has to be excluded during the transformation and
has to be considered separately, if necessary.}
\end{itemize}
\ \\ \ \\
The solution sets found in the individual cases have to be checked with respect to the case conditions
as described for the solution of \MSRef{VBKM02_FallBetrag}{absolute value equations}.
\end{MInfo}

In contrast, adding terms in which the variable occurs, does not require a case analysis. Usually, 
transformations involving case analyses are mandatory if the variable occurs in the denominator or 
in a composite term.


\begin{MExample}
The inequality $\frac1{2x}\leq 1$ can be simplified by multiplying both sides of the inequality
by the term $2x$:

\begin{itemize}
\item{Under the condition $x>0$ this results in the new inequality $1\leq 2x$. It has the solution set
$\ML_1=\left[\frac12\MIntvlSep \infty\MoIr[\right]$. The condition $x>0$ is satisfied by all elements of the solution 
set.}
\item{Under the condition $x<0$ this results in the new inequality $1\geq 2x$. It has the solution set
 $\MoIl[\left] -\infty\MIntvlSep \frac12\right]$. Because of the additional
condition $x<0$ only the elements of the set 
$\ML_2=\MoIl[\left] -\infty\MIntvlSep 0\MoIr[\right]$ are solutions.}
\item{The single case $x=0$ is no solution since this value is not in the domain of the inequality. 
In this case multiplying the inequality by $x$ is not allowed.}
\end{itemize}
So, altogether one obtains the union set 
$\ML=\ML_1\cup \ML_2=\R\MSetminus\left[0\MIntvlSep \frac12\MoIr[\right]$ as solution set:
\ \\ \ \\
\begin{center}
\MTikzAuto{%
\begin{tikzpicture}
% reelle Achse
\draw[->,color=black] (-1,0.0) -- (5,0.0);
\foreach \x in {-1, 0, 1, 2, 3, 4}
\draw[shift={(\x,0)},color=black] (0pt,2pt) -- (0pt,-2pt) node[below] {\footnotesize $\x$};
\draw (4.9,-0.3) node[] {$\mathbb{R}$};
% Enden:
\draw [line width=2.0pt,color=blue] (-1,0.0)-- (0,0.0);
\draw [line width=2.0pt,color=blue] (0.5,0.0)-- (5,0.0);
\draw [fill = blue] (0.5,0) circle (1.5pt);
\draw [fill = white] (0,0) circle (1.5pt);
\end{tikzpicture}
}
\end{center}
\end{MExample}

As in Modul~\MNRef{VBKM02} the following statement holds for the solution set.

\begin{MInfo}
The cases have to be chosen such that all elements of the domain of the inequality are covered. 
For the solution set in an individual case, it has to be checked that the solution set satisfies the 
corresponding case condition. For any case, the resulting solution set has to be reduced to
the solution subset satisfying the case condition. The union of the solution sets for the individual cases
is the solution set of the initial inequality.
\end{MInfo}

\end{MXContent}

\begin{MExercises}
\MDeclareSiteUXID{VBKM03_Fallunterscheidungen_Exercises}
If the inequality is multiplied by a composite term, it has to be investigated precisely for which values
of $x$ the case analysis has to be done:

\begin{MExercise}
Find the solution set of the inequality $\frac1{4-2x}<3$. 
The domain of the inequality is $D=\R\MSetminus \lbrace 2\rbrace$ since only for these
values of $x$ the denominator is non-zero. If the inequality is multiplied by the term
$4-2x$, three cases have to be distinguished. Fill in the blanks in the following text 
accordingly:

\begin{MExerciseItems}
\item{On the interval \MLIntervalQuestion{15}{(-infty,2)}{4}{GOM1} the term is positive, the comparing symbol 
remains unchanged, and the new inequality reads $1\:<\:$\MLSimplifyQuestion{15}{3*(4-2*x)}{5}{x}{5}{0}{SIMPLE2}.
Linear transformations result in the solution set 
\MEquationItem{$\ML_1$}{\MLIntervalQuestion{20}{(-infty,11/6)}{3}{MIXGOM}}. 
The elements of this set satisfy the case condition.}
\item{On the interval \MLIntervalQuestion{15}{(2,infty)}{4}{GOM2} the term is negative, 
the comparing symbol is inverted. Initially, the new inequality has the solution set 
\MLIntervalQuestion{20}{(11/6,infty)}{4}{INT1}, because of the case condition only the 
subset \MEquationItem{$\ML_2$}{\MLIntervalQuestion{20}{(2,infty)}{4}{GOM3}} is allowed.}
\item{The single value $x=2$ is no solution of the initial inequality since 
it is not in \MLQuestion{25}{domain}{UGX}.}
\end{MExerciseItems}

Sketch the solution set of the inequality and indicate the boundary points.

\begin{MHint}{Solution}
On the interval $\MoIl[\left] -\infty\MIntvlSep 2\MoIr[\right]$ the term is positive. 
The corresponding solution set is $\MoIl[\left] -\infty\MIntvlSep \frac{11}{6}\MoIr[\right]$.
In contrast, on the interval $\MoIl 2\MIntvlSep \infty\MoIr$ the term is negative, 
the comparing symbol is inverted. Initially, the new inequality has the solution set
 $\MoIl[\left] \frac{11}{6}\MIntvlSep \infty\MoIr[\right]$, because of the case condition
$x>2$ only the subset $\ML_2=\MoIl 2\MIntvlSep \infty\MoIr$ is allowed.
So, altogether the union set $\ML=\ML_1\cup\ML_2=\R\MSetminus \left[\frac{11}6\MIntvlSep 2\right]$
is the solution set of the initial inequality excluding the boundary points:

\begin{center}
\MTikzAuto{%
\begin{tikzpicture}
% reelle Achse
\draw[->,color=black] (-1,0.0) -- (5,0.0);
\foreach \x in {-1, 0, 1, 2, 3, 4}
\draw[shift={(\x,0)},color=black] (0pt,2pt) -- (0pt,-2pt) node[below] {\footnotesize $\x$};
\draw (4.9,-0.3) node[] {$\mathbb{R}$};
% Enden:
\draw [line width=2.0pt,color=blue] (-1,0.0)-- (1.83,0.0);
\draw [line width=2.0pt,color=blue] (2,0.0)-- (5,0.0);
\draw [fill = white] (1.83,0) circle (1.5pt);
\draw [fill = white] (2,0) circle (1.5pt);
\end{tikzpicture}
}
\end{center}
\end{MHint}
\end{MExercise}

\begin{MExercise}
The solution set of the inequality $\frac{x-1}{x-2}\leq 1$ is 
\MEquationItem{$\ML$}{\MLIntervalQuestion{20}{(-infty,2)}{4}{IGU1}}.

\begin{MHint}{Solution}
The domain of the inequality is $D=\R\MSetminus\lbrace 2\rbrace$.

\begin{itemize}
\item{For $x>2$, multiplying the inequality by the term $x-2$ results in $x-1\leq x-2$, 
which is equivalent to the false statement $-1\leq -2$. Thus, this case does not contribute a solution to the solution set.}
\item{For $x<2$, multiplying the inequality by the term $x-2$ results in $x-1\geq x-2$, 
which is equivalent to the true statement $-1\geq -2$. 
Because of the case condition the solution interval for this case is only 
$\ML_2=\MoIl[\left] -\infty\MIntvlSep 2\MoIr[\right]$.}
\item{The single value $x=2$ is no solution.}
\end{itemize}
So, altogether the solution set is 
 $\ML=\MoIl[\left] -\infty\MIntvlSep 2\MoIr[\right]$ 
excluding the boundary points (even though the comparing symbol $\leq$ occurred in the initial
inequality).
\end{MHint}
\end{MExercise}


\begin{MExercise}
The solution set of the inequality $\frac1{1-\sqrt{x}}<1+\sqrt{x}$ is \MEquationItem{$\ML$}{\MLIntervalQuestion{20}{(1,infty)}{4}{IGU2}}.
\ \\ \ \\
\begin{MHint}{Solution}
The domain of the inequality is $D=[0\MIntvlSep \infty\MoIr\MSetminus \lbrace 1\rbrace$
since only for these values of $x$ the square root is defined and the denominator is non-zero.
\begin{itemize}
\item{For $0\leq x<1$, multiplying the inequality by the term $1-\sqrt{x}$ results in
  $1<(1+\sqrt{x})(1-\sqrt{x})$, which is equivalent to $1<1-x$. This 
inequality is satisfied for $x<0$, but these values of $x$ violate the case condition and 
thus, they are not in the solution set.}
\item{For $x>1$, multiplying the inequality by the term $1-\sqrt{x}$ results in $1>1-x$, 
which is equivalent to $x>0$. But only the values of $x$ in the interval
  $\MoIl 1\MIntvlSep \infty\MoIr$ satisfy the case condition, hence $\ML=\MoIl 1\MIntvlSep \infty\MoIr$ 
is the only solution interval of the initial inequality.}
\item{The single value $x=1$ is no solution.}
\end{itemize}
\end{MHint}
\end{MExercise}


\end{MExercises}

\MSubsection{Absolute Value Inequalities and Quadratic Inequalities}
\MLabel{M03_Betragsungleichungen}

\begin{MIntro}
\MDeclareSiteUXID{VBKM03_Betragsungleichungen_Intro}
As in the approach in Modul~\MNRef{VBKM02} and in the previous section 
\MEntry{absolute values}{inequalities (absolute values)} in inequalities are solved 
by a case analysis:

\begin{MInfo}
To solve an \MEntry{absolute value inequality}{absolute value inequality} two cases are distinguished:

\begin{itemize}
\item{For those values of $x$, for which the absolute value term is non-negative the absolute value can be omitted or
replaced by simple brackets, respectively.}
\item{For those values of $x$, for which the absolute value term is negative the term is bracketed and negated.}
\end{itemize}
\ \\
Then, the solution sets arising from the case analysis will be restricted as described in the 
\MSRef{VBKM02_FallBetrag}{previous module} and merged to the solution set of the initial inequality. 
\end{MInfo}

\begin{MExample}
To solve the absolute value inequality $|4x-2|<1$ two cases are distinguished:
\begin{itemize}
\item{For $x\geq \frac12$, the absolute value term is non-negative: 
In this case the inequality is equivalent to $(4x-2)<1$ or $x<\frac34$, respectively. 
Because of the case condition the solution set is only 
$\ML_1=\left[\frac12\MIntvlSep \frac34\MoIr[\right]$ in this case.}
\item{For $x<\frac12$, the absolute value term is negative: 
In this case the inequality is equivalent to $-(4x-2)<1$ or $x>\frac14$, respectively. 
Only the subset $\ML_2=\MoIl[\left] \frac14\MIntvlSep \frac12\MoIr[\right]$ 
satisfies the case condition and is the solution set.}
\end{itemize}
The union of the two solution intervals results in the solution set
$\ML=\MoIl[\left] \frac14\MIntvlSep \frac34\MoIr[\right]$ for the initial absolute value inequality:

\begin{center}
\MTikzAuto{%
\begin{tikzpicture}
% reelle Achse
\draw[->,color=black] (-1,0.0) -- (5,0.0);
\foreach \x in {-1, 0, 1, 2, 3, 4}
\draw[shift={(\x,0)},color=black] (0pt,2pt) -- (0pt,-2pt) node[below] {\footnotesize $\x$};
\draw (4.9,-0.3) node[] {$\mathbb{R}$};
% Enden:
\draw [line width=2.0pt,color=blue] (0.25,0.0)-- (0.75,0.0);
\draw [fill = white] (0.25,0) circle (1.5pt);
\draw [fill = white] (0.75,0) circle (1.5pt);
\end{tikzpicture}
}
\end{center}
\end{MExample}

\begin{MExercise}
To solve the absolute value inequality $|x-1|<2|x-1|+x$ two cases are distinguished:
\begin{MExerciseItems}
\item{On the interval \MLIntervalQuestion{20}{[1,infty)}{3}{UGL1}, both
terms in the absolute value terms are non-negative. 
The solution set of the inequality is in this case 
\MEquationItem{$\ML_1$}{\MLIntervalQuestion{20}{[1,infty)}{3}{UGL2}}.}
\item{On the interval \MLIntervalQuestion{20}{(-infty,1)}{3}{UGL3}, both
terms in the absolute value terms are negative. 
The solution set of the inequality is in this case
\MEquationItem{$\ML_2$}{\MLIntervalQuestion{20}{(-infty,1)}{3}{UGL4}}.}
\end{MExerciseItems}
The union of the two intervals results in the solution interval 
\MEquationItem{$\ML$}{\MLIntervalQuestion{25}{(-infty,infty)}{4}{UGL5}}.
\ \\ \ \\
\begin{MHint}{Solution}
For $x\in [1\MIntvlSep \infty\MoIr$, both terms in the absolute value terms are non-negative, 
one obtains the inequality $x-1<2(x-1)+x$, which is equivalent to $x>\frac12$. 
Because of the case condition one obtains $\ML_1=[1\MIntvlSep \infty\MoIr$ as solution set.
For $x\in\MoIl[\left] -\infty\MIntvlSep 1\MoIr[\right]$, 
both terms in the absolute value terms are negative and
one obtains $-(x-1)<-2(x-1)+x$. 
This inequality is equivalent to the inequality $x-1<x$ which is
always true. Thus,
the solution set for the second case is
$\ML_2=\MoIl[\left] -\infty\MIntvlSep 1\MoIr[\right]$.
\ \\ \ \\
Since $\ML=\ML_1\cup \ML_2=\R=\MoIl[\left] -\infty\MIntvlSep \infty\MoIr[\right]$ 
the inequality is always satisfied.
\end{MHint}
\end{MExercise}

\end{MIntro}

\begin{MXContent}{Quadratic Absolute Value Inequalities}{Quadratic Inequalities}{STD}
\MDeclareSiteUXID{VBKM03_QuadratischeUngleichungen}
\begin{MInfo}
An inequality is called \MEntry{quadratic}{inequality (quadratic)} in $x$ 
if it can be transformed into $x^2 + p x + q < 0$ (other comparing symbols are allowed).
\end{MInfo}
\ \\ \ \\
Hence, quadratic inequalities can be solved in two ways: by investigating the roots 
and the opening behaviour of the polynomial and by completing the square. Often completing
the square is simpler:
 

\begin{MInfo}
To solve an inequality by \MEntry{completing the square}{completing the square (inequalities)} 
one tries to transform it into the form $(x+a)^2<b$. Taking the square root then results
in the absolute value inequality $|x+a|<\sqrt{b}$ with the solution set 
$\MoIl[\left] -a-\sqrt{b}\MIntvlSep -a+\sqrt{b}\MoIr[\right]$ if $b\geq 0$. Otherwise 
the inequality is unsolvable.

The inverted inequality $|x+a|>\sqrt{b}$ has the solution set
$\MoIl[\left] -\infty\MIntvlSep -a-\sqrt{b}\MoIr[\right]\cup \MoIl[\left] -a+\sqrt{b}\MIntvlSep \infty\MoIr[\right]$. 
For $\leq$ and $\geq$ the corresponding boundary points have to be included.
\end{MInfo}

Always note the calculation rule $\sqrt{x^2}=|x|$ described in Modul~\MNRef{VBKM01}.

\begin{MExample}
Find the solution of the inequality $2x^2\geq 4x+2$. Collecting the terms on the left-hand side and dividing 
the inequality by $2$ results in $x^2-2x-1\geq0$. Completing the square on the 
left-hand side to the second binomial formula results in the equivalent inequality $x^2-2x+1\geq 2$
or $(x-1)^2\geq 2$, respectively. Taking the square root results in the absolute value 
inequality $|x-1|\geq\sqrt{2}$ with the solution set 
$\ML=\MoIl[\left] -\infty\MIntvlSep 1-\sqrt{2}\right]\cup \left[1+\sqrt{2}\MIntvlSep \infty\MoIr[\right]$.
\end{MExample}

On the other hand, the inequality $x^2-2x-1\geq0$ can be investigated as follows:
The left-hand side describes a parabola opened upwards. The roots $x_{1,2}=1\pm \sqrt2$ 
can be found using the $pq$ formula:

\begin{center}
%%\MUGraphicsSolo{parabelu.png}{width=0.4\linewidth}{width:400px}
\MTikzAuto{%
\begin{tikzpicture}[x=1.0cm, y=1.0cm,scale=1.50] 
\draw[black] (-1,0) -- (3,0) (0,-2) -- (0,2);
\foreach \x in {-1, 1, 2, 3}
\draw[shift={(\x,0)},color=black] (0pt,0pt) -- (0pt,-2.0pt) node[below=1.0pt] {\scriptsize $\x$};
\foreach \x in {-0.5, 0.5, ..., 3.0}
\draw[shift={(\x,0)},color=black] (0pt,0pt) -- (0pt,-1.0pt);
\foreach \y in {-2, -1, 1, 2}
\draw[shift={(0,\y)},color=black] (0pt,0pt) -- (-2.0pt,0pt) node[left=1.0pt] {\scriptsize $\y$};
\foreach \y in {-1.5, -0.5, 0.5, 1.5}
\draw[shift={(0,\y)},color=black] (0pt,0pt) -- (-1.0pt,0pt);
\draw[black] (-0.0pt,-0.0pt) node[anchor=north east] {\scriptsize $0$};
\clip(-1.0,-3.0) rectangle (3.0,2.0);
\draw[smooth,samples=21,domain=-1:3, line width=1.0pt,color=red!50!black] plot(\x,{\x*\x-2*\x-1});
\end{tikzpicture}
}
\end{center}
Since the parabola opens upwards, the inequality $x^2-2x-1\geq0$ is satisfied by the
values of $x$ in the parabola branches left and right to the roots, i.e. by the set 
$\ML=\MoIl[\left] -\infty\MIntvlSep 1-\sqrt{2}\right]\cup \left[1+\sqrt{2}\MIntvlSep \infty\MoIr[\right]$.

\begin{MInfo}
\MLabel{M03_InfoFormen}
Depending on the roots of $x^2+ p x + q$, the opening of the parabola and the 
comparing symbol, the quadratic inequality $x^2 +p x +q <0$ (including other comparing symbols) 
has one of the following solution sets:

\begin{itemize}
\item{the set of real numbers $\R$,}
\item{two branches $\MoIl[\left] -\infty\MIntvlSep x_1\MoIr[\right]\cup \MoIl[\left] x_2\MIntvlSep \infty\MoIr[\right]$ (including the boundary points for $\leq$ and $\geq$),}
\item{an interval $\MoIl x_1\MIntvlSep x_2\MoIr$ (including the boundary points for $\leq$ and $\geq$ if applicable),}
\item{a single point $x_1$,}
\item{the pointed set $\R\MSetminus\lbrace x_1\rbrace$,}
\item{the empty set $\lbrace\rbrace$.}
\end{itemize}
\end{MInfo}

Fill in the blanks in the following text describing the solution of a quadratic
inequality by investigating the behaviour of the parabola:

\begin{MExercise}
Find the solution set of the inequality $x^2+6x< -5$. 
Transformation results in the inequality \MLSimplifyQuestion{15}{x^2+6*x+5}{5}{x}{5}{1}{OBXP1}$<0$.
Using the $p q$ formula one obtains the set of roots
\MLParsedQuestion{9}{-1,-5}{3}{PXL}. The left-hand side
describes a parabola opening \MLQuestion{10}{upwards}{ObenX}.
It belongs to an inequality involving the comparing symbol $<$, hence 
the solution set is \MEquationItem{$\ML$}{\MLIntervalQuestion{15}{(-5,-1)}{5}{INVX}}.
\ \\ \ \\
\begin{MHint}{Solution}
Transformation results in $x^2+6x+5<0$. Using the $p q$ formula
one obtains the roots $x_{1,2}=-3\pm\sqrt{9-5}$, i.e.\ $x_1=-1$ and $x_2=-5$.
The left-hand side describes a parabola opening upwards. It satisfies the inequality 
involving $<$ only on the interval $\MoIl[\left] -5\MIntvlSep -1\MoIr[\right]$ excluding
the boundary points.
\end{MHint}
\end{MExercise}

\end{MXContent}

\begin{MXContent}{Further Types of Inequalities}{Further Types of Inequalities}{STD}
\MDeclareSiteUXID{VBKM03_WeitereUngleichungstypen}
Many other types of inequalities can be transformed into quadratic inequalities. Sometimes, 
case analyses have to be done or excluded values in the domain have to be observed:

\begin{MInfo}
An inequality containing \MEntry{fractions}{inequality (fractions)}, where the 
variable $x$ occurs in the denominator of composite terms, can be transformed into a 
form without fractions by multiplying the inequality by the least common denominator. 
However, in doing so, the roots of the denominators have to be excluded from the domain
of the new inequality. 

Additionally, if the inequality is multiplied by a term, different cases have to be distinguished 
depending on the sign of the term.
\end{MInfo}

\begin{MExample}
The inequality $2-\frac1x\leq x$ can be transformed by multiplying the inequality by $x$. Here, three 
cases have to be distinguished:
\begin{itemize}
\item{For $x>0$, the comparing symbol in the inequality is unchanged. The new inequality
reads $2x-1\leq x^2$ and is equivalent to $x^2-2x+1\geq 0$ or $(x-1)^2\geq 0$, respectively.
This inequality is always satisfied. Because of the case condition one obtains 
the solution set $\ML_1=\MoIl 0\MIntvlSep \infty\MoIr$.}
\item{For $x<0$, the comparing symbol in the inequality is inverted. The new inequality
reads $2x-1\geq x^2$ and is equivalent to $x^2-2x+1\leq 0$ or $(x-1)^2\leq 0$, respectively.
This inequality is only satisfied for $x=1$. But this value is excluded by the case condition, 
i.e.\ $\ML_2=\{\}$.}
\item{The single value $x=0$ is not in the domain of the initial inequality and hence it is
no solution.}
\end{itemize}

So, altogether one obtains the union set 
$\ML=\MoIl 0\MIntvlSep \infty\MoIr$ as solution set of the initial inequality.
\end{MExample}

Inequalities involving composite fraction and root terms often do not have solution
sets of the types described in info box~\MRef{M03_InfoFormen}:

\begin{MExample}
Find the solution set of the inequality $\sqrt{x}+\frac1{\sqrt{x}}>2$. 
The domain of the inequality is $\MoIl 0\MIntvlSep \infty\MoIr$.
Multiplying by $\sqrt{x}$ results in the inequality $x+1>2\sqrt x$. 
Here, no case analysis is required since $\sqrt{x}>0$ is in the domain.
Transformation results in $x-2\sqrt{x}+1>0$ or $(\sqrt{x}-1)^2>0$, respectively, 
which is satisfied for all $x\not=1$ in the domain.
Hence, the solution set of the initial inequality is
$\ML=\MoIl 0\MIntvlSep \infty\MoIr\MSetminus\lbrace 1\rbrace$:
\ \\ \ \\
\begin{center}
\MTikzAuto{%
\begin{tikzpicture}
% reelle Achse
\draw[->,color=black] (-1,0.0) -- (5,0.0);
\foreach \x in {-1, 0, 1, 2, 3, 4}
\draw[shift={(\x,0)},color=black] (0pt,2pt) -- (0pt,-2pt) node[below] {\footnotesize $\x$};
\draw (4.9,-0.3) node[] {$\mathbb{R}$};
% Enden:
\draw [line width=2.0pt,color=blue] (0,0.0)-- (1,0.0);
\draw [line width=2.0pt,color=blue] (1,0.0)-- (5,0.0);
\draw [fill = white] (0,0) circle (1.5pt);
\draw [fill = white] (1,0) circle (1.5pt);
\end{tikzpicture}
}
\end{center}

\end{MExample}

\end{MXContent}


\MSubsection{Final Test}
\MLabel{M03_Abschlusstest}

\begin{MTest}{Final Test Modul 3}
\MDeclareSiteUXID{VBKM03_Abschlusstest}

\begin{MExercise}
Find the value of the parameter $\alpha$ such that the inequality $2x^2\leq x-\alpha$ 
has exactly one solution:
\begin{MExerciseItems}
\item{The parameter value is \MEquationItem{$\alpha$}{\MLParsedQuestion{10}{1/8}{3}{PMA1}}.}
\item{In this case \MEquationItem{$x$}{\MLParsedQuestion{10}{1/4}{3}{PMA2}} is the only solution
of the inequality.}
\end{MExerciseItems}
\end{MExercise}


\begin{MExercise}
Find an absolute value function $g(x)$ describing the following graph as easy as possible.

%%\MUGraphics{abs3.png}{width=0.5\linewidth}{Funktionsgraph von $g(x)$.}{width:300px}
\begin{center}
\MTikzAuto{%
\begin{tikzpicture}[x=1.4cm, y=1.9cm] 
\draw[black] (-3,0) -- (3,0) (0,-3) -- (0,3);
\foreach \x in {-3, -2, -1, 1, 2, 3}
\draw[shift={(\x,0)},color=black] (0pt,0pt) -- (0pt,-3.0pt) node[below=1.0pt] {\normalsize $\x$};
\foreach \x in {-3.0, -2.8, ..., 3.0}
\draw[shift={(\x,0)},color=black] (0pt,0pt) -- (0pt,-1.5pt);
\foreach \y in {-3, -2, -1, 1, 2, 3}
\draw[shift={(0,\y)},color=black] (0pt,0pt) -- (-3.0pt,0pt) node[left=1.0pt] {\normalsize $\y$};
\foreach \y in {-3.0, -2.8, ..., 3.0}
\draw[shift={(0,\y)},color=black] (0pt,0pt) -- (-1.5pt,0pt);
%%\draw[black] (-0.0pt,-0.5pt) node[anchor=north east] {\small $0$};
\clip(-3.0,-3.0) rectangle (3.0,3.0);
\draw[black, line width=1.0pt,color=black] (-3,-3) -- (1,1) -- (2,4);
\end{tikzpicture}
}
\par
Graph of the function $g(x)$.
\end{center}
Try to find a representation of the form $g(x)=|x+a|+b x+c$. 
The kink in the graph indicates how the absolute value term looks like.

\begin{MExerciseItems}
\item{Find the solution set of the inequality $g(x)\leq x$ by means of the graph.\\
The solution set is \MEquationItem{$\ML$}{\MLIntervalQuestion{20}{(-infty;1]}{5}{AUX1}}.}
\item{\MEquationItem{$g(x)$}{\MLSimplifyQuestion{20}{abs(x-1)+2*x-1}{10}{x}{10}{0}{SIMPLE3}}. 
\\\MInputHint{Absolute values can be entered in the form \texttt{betrag(x-a)} or \texttt{abs(x-a)}.}}
\end{MExerciseItems}
\end{MExercise}

\begin{MExercise}
Which positive real numbers $x$ satisfy the following inequalities?
\begin{MExerciseItems}
\item{$|3x-6|\leq x+2$ has the solution set
\MEquationItem{$\ML$}{\MLIntervalQuestion{16}{[1,4]}{4}{COSH1}} (written as an interval).}
\item{$\frac{x+1}{x-1}\geq 2$ has the solution set 
\MEquationItem{$\ML$}{\MLIntervalQuestion{16}{(1,3]}{4}{COSH2}} (written as an interval).}
\end{MExerciseItems}
\MInputHint{Enter open intervals in the form $(3;5)$, closed intervals in the form 
$[3;5]$. Infinity can be entered a a word or shortly a \texttt{infty}. Do not use 
notation $]a;b[$ for open intervals. Sets can be entered by listing the elements
 $\lbrace 1;2;3\rbrace$. For the set brackets enter AltGr+7 or AltGr+0, respectively.}
\end{MExercise}

\end{MTest}

\newpage
\MPrintIndex

\end{document}
