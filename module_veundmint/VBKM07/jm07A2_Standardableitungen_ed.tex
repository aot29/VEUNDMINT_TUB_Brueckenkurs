%LaTeX-2e-File, Liedtke, 20140731.
%Inhalt: Einf"uhrung in die Differentialrechnung, Abschnitt 2.
%zuletzt bearbeitet: 20140922.

%Abschnitt 2: Standardableitungen
%\MSubsection{Standardableitungen}
%\begin{MContent}

%\MSubsubsection{Ableitung von Polynomen}
\begin{MXContent}{Ableitung von Polynomen}{Ableitung von Polynomen}{STD}
 
Aus der Einf"uhrung der Ableitung ergibt sich f"ur eine Gerade
$f: \R \to \R, x \mapsto a_1 x + a_0$, wobei $a_1$ und $a_0$ gegebene Zahlen 
sind, dass die Ableitung von $f$ an der Stelle $x_0$ gleich $f'(x_0) = a_1$ ist.
Wenn beispielsweise $a_1 = 2$ und $a_0 = 1$, dann ist $f(x) = 2 x + 1$, und die
Ableitung ist $f'(x) = 2$.
Denn mit $f(x_0) = 2 x_0 + 1$ ist
\[
f(x) = 2 x + 1 = 2 x_0 - 2 x_0 + 2 x + 1 %
 = f(x_0) + 2 (x - x_0) + 0 \cdot (x - x_0).
\]
Wird die Restfunktion $r(x) := 0$ eingef"uhrt, sind die Bedingungen zur 
Differenzierbarkeit erf"ullt, und es ist $f'(x_0) = 2$ die Ableitung von $f$ 
in $x_0$.

F"ur Monome $x^n$ mit $n > 1$ ist es am einfachsten, die Ableitung "uber den 
Differenzenquotienten zu bestimmen. Damit ergibt sich folgende Aussage:

\begin{MXInfo}{Ableitung von $x^n$}
Gegeben sind eine nat"urliche Zahl $n$ und eine reelle Zahlen $r$.

Die konstante Funktion $f(x) := r = r \cdot x^0$ hat die Ableitung
$f'(x) = 0$.

Die Funktion $f(x) := r \cdot x^n$ hat die Ableitung 
\[
f'(x) = r \cdot n \cdot x^{n-1} %%
\]
\end{MXInfo}

\begin{MExample}
Es wird die Funktion $f: \R \to \R$ mit $f(x) = 5 x^3$ betrachtet.
Mit obigen Bezeichnungen ist $r = 5$ und $n = 3$. Damit erh"alt man die 
Ableitung 
\[
f(x) = 5 \cdot 3 x^{3 - 1} = 15 x^2. %%
\]
\end{MExample}


F"ur Wurzelfunktionen ergibt sich eine entsprechende Aussage. Allerdings 
ist zu beachten, dass Wurzelfunktionen nur f"ur $x > 0$ differenzierbar sind.
Denn die Tangente an den Funktionsgraphen durch den Punkt $(0;0)$ verl"auft 
parallel zur $y$-Achse und beschreibt somit keine Funktion. 

\begin{MXInfo}{Ableitung von $x^{\frac{1}{n}}$}
%F"ur $n \in \Z$ und $n \notin \{0, 1, -1 \}$ ist die Wurzelfunktion
F"ur $n \in \Z$ mit $n \neq 0$ ist die Funktion
$f(x) := x^{\frac{1}{n}}$ f"ur $x \geq 0$ differenzierbar, und es gilt
\[
f'(x) = \frac{1}{n} \cdot x^{\frac{1}{n}-1} \text{ f"ur } x > 0 %%
\]%
\end{MXInfo}
F"ur $n \geq 2$ haben wir es mit Wurzelfunktionen zu tun.
Und nat"urlich sind die Sonderf"alle, die identische Abbildung $g(x) = x$ 
und $h(x) = x^{-1}$, auf ihrem gesamten Definitionsbereich differenzierbar.
%differenzierbar, aber eben keine Wurzelfunktionen, weshalb sie oben nicht 
%aufgef"uhrt sind.

%Beispielsweise ist die Ableitung von $f(x) := \sqrt{x} = x^{\frac{1}{2}}$
%f"ur $x > 0$ durch
%$f'(x) = \frac{1}{2} x^{\frac{1}{2}-1} %
%= \frac{1}{2} x^{-\frac{1}{2}} = \frac{1}{2 \cdot \sqrt{x}}$ 
%gegeben.

\begin{MExample}
Die Wurzelfunktion $f: [0, \infty) \to \R$ mit
 $f(x) := \sqrt{x} = x^{\frac{1}{2}}$
ist f"ur $x > 0$ differenzierbar. Die Ableitung ist durch
\[
f'(x) = \frac{1}{2} \cdot x^{\frac{1}{2}-1} %
= \frac{1}{2} \cdot x^{-\frac{1}{2}} = \frac{1}{2 \cdot \sqrt{x}} %%
\] 
gegeben.

%Bild:
\ifttm
\MUGraphics{\MPfadBilder/jb07A2_Wurzelfunktion.png}{scale=0.5}%
{Graph von $\sqrt{x}$ mit Tangente in $x_0 = 1$}{}
\else
\begin{center}
%Datei: {\MPfadBilder/jb07A1_Wurzelfunktion.tex}
%\input{\MPfadBilder/jb07A1_Wurzelfunktion.tex}
%LaTeX-File, Liedtke, 20140916.
%VBKM-Modul 7 Differentialrechnung: Bild zur Exponentialfunktion
%Bildname: jb07A2_Wurzelfunktion
%Erstellt: 20140922, Liedtke.
\begin{small}
\renewcommand{\jTikZScale}{0.8}
\tikzsetnextfilename{jb07A2_Wurzelfunktion}
\begin{tikzpicture}[line width=1.5pt,scale=\jTikZScale]
%\begin{small}
%\begin{tikzpicture}[line width=1.5pt,scale=\jTikZScale, %
%declare function={
%  fkt(\x) = sqrt(\x);
%}
%] %[every node/.style={fill=white}] 
%Koordinatenachsen:
\draw[->] (-0.6, 0) -- (4.8, 0) node[below left]{$x$}; %x-Achse
\draw[->] (0, -0.6) -- (0, 3) node[below left]{$y$}; %y-Achse
%Achsenbeschriftung:
\foreach \x in {1, 2, 3, 4} \draw (\x, 0) -- ++(0, -0.1) %
 node[below] {$\x$};
\foreach \y in {1, 2} \draw (0, \y) -- ++(-0.1, 0) node[left] {$\y$};
%\node[below left] at (0, 0) {$0$};
%Funktion:
\draw[domain=0:4,samples=120,color=\jccolorfkt] %
 plot (\x, {sqrt(\x)});
%Tangenten y = f(x_0) + f'(x_0) * (x - x_0) im Punkt $(x_0, f(x_0)$ des Graphen:
%$x_0 := 1$:
\draw[samples=120,color=blue!50!black] %
 (0.5, {1 - 1/2 * 1/2}) -- ++(1, {1/2});
%Punkt:
\filldraw[color=black, fill=black] (1, 0) circle (2pt);
%end of file
\end{tikzpicture}
\end{small}
\end{center}
\fi
%Bildende.
Die Tangente in $x_0 = 1$ an den Graphen der Wurzelfunktion $f(x) = \sqrt{x}$ hat 
die Steigung $\frac{1}{2 \sqrt{1}} = \frac{1}{2}$.
\end{MExample}

Eine entsprechende Aussage gilt auch f"ur allgemeine Exponenten $p \in \R$ mit 
$p \neq 0$ f"ur $x > 0$:
Die Ableitung von $f(x) = x^p$ f"ur $x > 0$ ist
\[
f'(x) = p \cdot x^{p-1} %%
\]
\end{MXContent}


%\MSubsubsection{Ableitung trigonometrischer Funktionen}
\begin{MXContent}{Ableitung trigonometrischer Funktionen}{Ableitung trigonometrischer Funktionen}{STD}

Die Sinusfunktion ist periodisch mit Periode $2 \pi$. Somit gen"ugt es, die
Funktion auf einem Intervall der L"ange $2 \pi$ zu betrachten. Einen Ausschnitt 
des Graphen f"ur $-\pi \leq x \leq \pi$ zeigt die folgende Abbildung:

%Bild:
\ifttm
\begin{center}
\MUGraphicsSolo{\MPfadBilder/jb07A2_SinusFktUndAbl.png}{scale=0.8}{}
%\MUGraphicsSolo{\MPfadBilder/jb07A2_SinusFktGraph.png}{scale=0.8}{}
%\MUGraphicsSolo{\MPfadBilder/jb07A2_SinusAbl.png}{scale=0.8}{}
%\MUGraphicsSolo{jb07A2_SinusFktGraph.png}{scale=1}{}
%\MUGraphicsSolo{jb07A2_SinusAbl.png}{scale=1}{}
\end{center}
\else
\begin{center}
%Bild: {\MPfadBilder/jb07A2_SinusFktGraph.tex}%
%LaTeX-File, Liedtke, 20140826.
%VBKM-Modul 7 Differentialrechnung: Bild zur Sinusfunktion.
%Bildname: jb07A2_SinusFktGraph.tex.
%Erstellt: 20140827, Liedtke.
%Bearbeitet: 20140829, Liedtke (Dateiname angepasst).
%Bearbeitet: 20140901, Liedtke (Dateiname ohne Endung erfasst).
\begin{small}
\renewcommand{\jTikZScale}{1.0}
\tikzsetnextfilename{jb07A2_SinusFktUndAbl}
%\tikzsetnextfilename{jb07A2_SinusFktGraph}
\begin{tikzpicture}[line width=1.5pt,scale=\jTikZScale, %
declare function={
  fkt(\x) = sin(\x r);
  fktabl(\x) = cos(\x r);
}
] %[every node/.style={fill=white}] 
%
%Graph der Sinusfunktion:
\node[right] at (-6,1) {Sinusfunktion};
\begin{scope}%[xshift=-6]
%Koordinatenachsen:
\draw[->] (-3.6, 0) -- (4, 0) node[below left]{$x$}; %x-Achse
\draw[->] (0, -1.6) -- (0, 1.6) node[below left]{$y$}; %y-Achse
%Achsenbeschriftung:
\foreach \x in {{-pi}, {-pi/2}} \draw (\x, 0) -- ++(0, -0.1);
\node[below] at ({-pi}, 0) {$-\pi$};
\node[below] at ({-pi/2}, 0) {$-\frac{\pi}{2}$};
\foreach \x in {{pi/2}, pi} \draw (\x, 0) -- ++(0, -0.1);
\node[below] at ({pi/2}, 0) {$-\frac{\pi}{2}$};
\node[below] at ({pi}, 0) {$-\pi$};
\foreach \y in {-1} \draw (0, \y) -- ++(-0.1, 0) %
 node[left] {$\y$};
\foreach \y in {1} \draw (0, \y) -- ++(-0.1, 0) %
 node[left] {$\y$};
%\node[below left] at (0, 0) {$0$};
%Funktion:
\draw[domain=-3.14:3.14,samples=120,color=\jccolorfkt] %
 plot (\x, {fkt(\x)});
%Tangenten in verschiedenen Punkten:
\draw[samples=120,color=blue!50!black] %
 plot (-0.5,-0.5) -- (0.5,0.5);
\draw[samples=120,color=blue!50!black] %
 plot ({pi/3},1) -- ({2*pi/3},1);
\node[above] at ({pi/2}, 1) {$\sin'(\pi/2) = 0$};
\draw[samples=120,color=blue!50!black] %
 plot ({-2*pi/3},-1) -- ({-pi/3},-1);
\node[below] at ({-pi/2}, -1) {$\sin'(-\pi/2) = 0$};
\end{scope}
%end of file
%
%
%Graph der Ableitung der Sinusfunktion:
%Bild: {\MPfadBilder/jb07A2_SinusAbl.tex}
%LaTeX-File, Liedtke, 20140826.
%VBKM-Modul 7 Differentialrechnung: Bild zur Ableitung der Sinusfunktion.
%Bildname: jb07A2_SinusAbl.tex.
%Erstellt: 20140827, Liedtke.
%Bearbeitet: 20140829, Liedtke (Dateiname angepasst).
%Bearbeitet: 20140901, Liedtke (Dateiname ohne Endung erfasst).
%
%\tikzsetnextfilename{jb07A2_SinusAbl}
%\begin{tikzpicture}[line width=1.5pt,scale=\jTikZScale, %
%declare function={
%  fktabl(\x) = cos(\x r);
%}
%] %[every node/.style={fill=white}] 
\node[right] at (-6,-2.8) {Ableitung};
\begin{scope}[yshift=-3.8cm]
%Koordinatenachsen:
\draw[->] (-3.6, 0) -- (4, 0) node[below left]{$x$}; %x-Achse
\draw[->] (0, -1.6) -- (0, 1.6) node[below left]{$y$}; %y-Achse
%Achsenbeschriftung:
\foreach \x in {{-pi}, {-pi/2}} \draw (\x, 0) -- ++(0, 0.1); 
\node[below] at ({-pi}, 0) {$-\pi$}; 
\node[below] at ({-pi/2}, 0) {$-\frac{\pi}{2}$}; 
\foreach \x in {{pi/2}, {pi}} \draw (\x, 0) -- ++(0, -0.1);
\node[below] at ({pi/2}, 0) {$\frac{\pi}{2}$}; 
\node[below] at ({pi}, 0) {$\pi$}; 
\foreach \y in {-1} \draw (0, \y) -- ++(0.1, 0);
\foreach \y in {1} \draw (0, \y) -- ++(-0.1, 0) %
 node[below left] {$\y$};
%\node[below left] at (0, 0) {$0$};
%Funktion:
\draw[domain=-3.14:3.14,samples=120,color=blue!50!white] %
 plot (\x, {fktabl(\x)});
%Punkte
\filldraw[color=black,fill=black] ({-pi}, -1) circle (2pt);
\filldraw[color=black,fill=black] ({-pi/2}, 0) circle (2pt);
\filldraw[color=black,fill=black] (0, 1) circle (2pt);
\filldraw[color=black,fill=black] ({pi/2}, 0) circle (2pt);
\filldraw[color=black,fill=black] ({pi}, -1) circle (2pt);
\end{scope}
\end{tikzpicture}
\end{small}
\end{center}
%end of file
\fi
%Bildende.

Der Sinus ist als Quotient der Gegenkathede durch die Hypotenuse definiert.
F"ur kleine Winkel ist die Gegenkathede n"aherungsweise gleich lang wie der 
Winkel im Bogenma"s, wie aus der Definition am Einheitskreis zu 
sehen ist. Damit ist der Differenzenquotient ungef"ahr $1$. 
Anschaulich verst"andlich ist, dass die Steigung der 
Tangente im Nullpunkt somit $1$ ist, wie dies in der zweiten Abbildung 
eingetragen ist. An den Stellen $\frac{\pi}{2}$ und $-\frac{\pi}{2}$ 
ist die Tangentensteigung gleich $0$. Eine genaue Betrachtung best"atigt 
die Werte und zeigt, dass sich insgesamt als Ableitungsfunktion die 
Kosinusfunktion ergibt.

\begin{MXInfo}{Ableitung trigonometrischer Funktionen}
F"ur die Sinusfunktion $f(x) := \sin(x)$ gilt 
\[
f'(x) = \cos(x) %%
\]
F"ur die Kosinusfunktion $g(x) := \cos(x)$ gilt 
\[
g'(x) = -\sin(x) %%
\]
F"ur die Tangensfunktion $h(x) := \tan(x)$ f"ur $x \neq \frac{\pi}{2} + k \pi$ 
mit $k \in \Z$ gilt
\[
h'(x) = 1 + (\tan(x))^2 = \frac{1}{\cos^2(x)} %%
\]
\end{MXInfo}
Letzteres ergibt sich auch aus den nachfolgend erl"auterten Rechenregeln und der 
Definition des Tangens als Quotient von Sinus und Kosinus.
\end{MXContent}


%\MSubsubsection{Ableitung der Exponentialfunktion}
\begin{MXContent}{Ableitung der Exponentialfunktion}{Ableitung der Exponentialfunktion}{STD}

Die Exponentialfunktion $f(x) := \exp(x)$ hat die besondere Eigenschaft, 
dass ihre Ableitung wiederum $f'(x) = \exp(x)$ ist.

F"ur die Stelle $0$ ist dies anschaulich nachvollziehbar, wenn der 
Graph m"oglichst genau gezeichnet wird, dass die Steigung der Tangente
gleich $f'(0) = 1$ und somit gleich $\MEU^0$ ist.

Wenn der Diffenzenquotient mit der Rechenregel 
$\MEU^{x_0+h} = \MEU^{x_0} \cdot \MEU^{h}$ f"ur die Exponentialfunktion 
umgeformt wird, ergibt sich daraus obiges Ergebnis 
$f'(x) = \MEU^x \cdot f'(0) = \MEU^x$.

%Bild:
\ifttm
\begin{center}
\MUGraphicsSolo{\MPfadBilder/jb07A2_Exponentialfunktion.png}{scale=0.5}{}
\end{center}
\else
\begin{center}
%Datei: {\MPfadBilder/jb07A1_BspExponentialfunktion.tex}
%\input{\MPfadBilder/jb07A1_BspExponentialfunktion.tex}
%LaTeX-File, Liedtke, 20140916.
%VBKM-Modul 7 Differentialrechnung: Bild zur Exponentialfunktion
%Bildname: jb07A2_BspExponentialfunktion
%Erstellt: 20140916, Liedtke.
\begin{small}
\renewcommand{\jTikZScale}{0.8}
\tikzsetnextfilename{jb07A2_Exponentialfunktion}
\begin{tikzpicture}[line width=1.5pt,scale=\jTikZScale]
%\begin{small}
%\begin{tikzpicture}[line width=1.5pt,scale=\jTikZScale, %
%declare function={
%  fkt(\x) = sin(\x r);
%}
%] %[every node/.style={fill=white}] 
%Koordinatenachsen:
\draw[->] (-3.6, 0) -- (4, 0) node[below left]{$x$}; %x-Achse
\draw[->] (0, -0.6) -- (0, 4) node[below left]{$y$}; %y-Achse
%Achsenbeschriftung:
\foreach \x in {-3, -2, -1, 1, 2, 3} \draw (\x, 0) -- ++(0, -0.1) %
 node[below] {$\x$};
\foreach \y in {1, 2, 3} \draw (0, \y) -- ++(-0.1, 0) node[left] {$\y$};
%\node[below left] at (0, 0) {$0$};
%Funktion:
\draw[domain=-2.4:1.3,samples=120,color=\jccolorfkt] %
 plot (\x, {exp(\x)});
%Tangenten in verschiedenen Punkten:
\draw[samples=120,color=blue!50!black] %
 (-0.45, 0.45) -- (0.55, 1.55);
%end of file
\end{tikzpicture}
\end{small}
\end{center}
\fi
%Bildende.
\end{MXContent}


%\MSubsubsection{Ableitung der Logarithmusfunktion}
\begin{MXContent}{Ableitung der Logarithmusfunktion}{Ableitung der Logarithmusfunktion}{STD}

Die Logarithmusfunktion $f(y) := \ln(y)$ f"ur $y > 0$ ist die Umkehrfunktion
der Exponentialfunktion $y = \MEU^x$.
Die Ableitung und damit die Steigung $m$ der Tangente an den 
Graphen der Exponentialfunktion ist $m = \MEU^x$. 

Der Graph der Umkehrfunktion $\ln(y)$ ergibt sich aus der Spiegelung an der 
ersten Winkelhalbierenden. Die Steigung der gespiegelten Tangente an die 
Exponentialfunktion ist dann der Kehrwert $f'(y) = \frac{1}{\MEU^x}$ an der 
Stelle $y = \MEU^x$, also $f'(y) = \frac{1}{y}$ (vergleiche auch Aufgabe
\MNRef{jm07A1Aufgabe:AblUmkehrFktGerade}).

Indem wieder die gewohnte Bezeichnung f"ur die unabh"angige Variable verwendet
wird, ergibt sich $f'(x) = \frac{1}{x}$ als Ableitung der 
Logarithmusfunktion $f(x) = \ln(x)$.

%Bild:
\ifttm
\MUGraphics{\MPfadBilder/jb07A2_Logarithmusfunktion.png}{scale=0.5}%
{Tangenten an $\exp$ und an die Umkehrfunktion $\ln$}{}
\else
\begin{center}
%Datei: {\MPfadBilder/jb07A1_BspExponentialfunktion.tex}
%\input{\MPfadBilder/jb07A1_BspExponentialfunktion.tex}
%LaTeX-File, Liedtke, 20140916.
%VBKM-Modul 7 Differentialrechnung: Bild zur Exponentialfunktion
%Bildname: jb07A2_BspExponentialfunktion
%Erstellt: 20140916, Liedtke.
\begin{small}
\renewcommand{\jTikZScale}{0.8}
\tikzsetnextfilename{jb07A2_Logarithmusfunktion}
\begin{tikzpicture}[line width=1.5pt,scale=\jTikZScale]
%\begin{small}
%\begin{tikzpicture}[line width=1.5pt,scale=\jTikZScale, %
%declare function={
%  fkt(\x) = sin(\x r);
%}
%] %[every node/.style={fill=white}] 
%Koordinatenachsen:
\draw[->] (-3.2, 0) -- (4.6, 0) node[below left]{$x$}; %x-Achse
\draw[->] (0, -3.2) -- (0, 4) node[below left]{$y$}; %y-Achse
%Achsenbeschriftung:
\foreach \x in {-3, -2, -1} \draw (\x, 0) -- ++(0, -0.1); % node[below] {$\x$};
\foreach \x in {1, 2, 3} \draw (\x, 0) -- ++(0, -0.1) node[below] {$\x$};
\foreach \y in {-3, -2, -1} \draw (0, \y) -- ++(-0.1, 0); % node[left] {$\y$};
\foreach \y in {1, 2, 3} \draw (0, \y) -- ++(-0.1, 0) node[left] {$\y$};
%\node[below left] at (0, 0) {$0$};
%
%Funktion exp:
\draw[domain=-2.4:1.3,samples=120,color=white!50!black] %\jccolorfkt] %
 plot (\x, {exp(\x)});
%Tangenten in verschiedenen Punkten:
%x_0 = 0:
%\draw[samples=120,color=blue!50!black] %
% (-0.45, 0.45) -- (0.55, 1.55);
%
%x_0 = 1:
% ({3/4}, {exp(1) - 1/4 * exp(1)}) -- ++({2/4}, {2/4*exp(1)});
%
%x_0 = ln(2): y = exp(ln(2)) + 2 * (x - ln(2)) = 2 + 2 * (x - ln(2)):
\draw[style=dashed,samples=120,color=blue!50!black] %
 ({ln(2) - 3/8}, {5/4}) -- ++({6/8}, {6/8*2});
%
%Punkte:
%\filldraw[color=black] (0, 1) circle (2pt);
\filldraw[color=black] (0, 2) circle (2pt);
\filldraw[color=black] ({ln(2)}, 2) circle (2pt);
%
%Spiegelachse (erste Winkelhalbierende):
\draw[samples=120,style=dotted,color=white!50!black] %
 (-2.1, -2.1) -- (3.1, 3.1);
%
%Umkehrfunktion ln:
\draw[domain=0.1:{exp(1.3},samples=120,color=\jccolorfkt] %
 plot (\x, {ln(\x)});
%Tangenten in verschiedenen Punkten:
%\draw[samples=120,color=blue!50!black] %
% (0.45, -0.45) -- (1.55, 0.55);
%
%\draw[style=dotted,samples=120,color=blue!50!black] %
% ({exp(1) - 1/2}, {1 - 1/2 * 1/exp(1)}) -- ++(1, {1/exp(1)});
%\draw[style=dotted,samples=120,color=blue!50!black] %
% ({exp(1) - 1/4*exp(1)}, {3/4}) -- ++({2/4*exp(1)}, {2/4});
%x_0 = 2: y = ln(2) + 1/2 * (x - 2):
\draw[samples=120,color=blue!50!black] %
 ({2 - 3/4}, {ln(2) - 1/2 * 3/4}) -- ++({2*3/4}, {2*3/4*1/2});
%Punkte:
%\filldraw[color=black] (1, 0) circle (2pt);
\filldraw[color=black] (2, 0) circle (2pt);
\filldraw[color=black] (2, {ln(2)}) circle (2pt);
\end{tikzpicture}
\end{small}
\end{center}
\fi
%Bildende.
In der obigen Abbildung ist die Tangente an den Graphen von $\exp$ im
Punkt $(x_0, \MEU^{x_0})$ f"ur $x_0 = \ln(2)$ angedeutet, sodass die 
Tangentensteigung $m = 2$ ist. Die Umkehrabbildung der Tangente hat dann 
die Steigung $\frac{1}{m} = \frac{1}{2}$. 
Geometrisch ist es die Steigung der gespiegelten Tangente, also die Tangente
an die Umkehrfunktion $\ln$.
%Dann hat die Tangente an den gespiegelten 
%Graphen, also den Graphen von $\ln$, die Steigung $\frac{1}{m} = \frac{1}{2}$.

\end{MXContent}

%\end{MContent}

%end of file.

