%LaTeX-2e-File, Liedtke, 20140731.
%Inhalt: Einf"uhrung in die Differentialrechnung, Abschnitt 2.
%zuletzt bearbeitet: 20140922.

%Abschnitt 2: Standardableitungen
%\MSubsection{Standardableitungen}
%\begin{MContent}

%\MSubsubsection{Ableitung von Polynomen}
\begin{MXContent}{Ableitung von Polynomen}{Ableitung von Polynomen}{STD}
 
Aus der Einf"uhrung der Ableitung mithilfe des Differenzenquotienten ergibt sich f"ur eine linear-affine Funktion (siehe Modul \MRef{VBKM06}, Abschnitt \MRef{VBKM06_sec:linear-affin}) $f\left(x\right) = m x + b$, wobei $m$ und $b$ gegebene Zahlen sind, dass die Ableitung von $f$ an der Stelle $x_0$ gleich $f'(x_0) = m$ ist. (�berpr�fen Sie dies gerne selbst!) F"ur Monome $x^n$ mit $n \geq 1$ ist es am einfachsten, die Ableitung "uber den Differenzenquotienten zu bestimmen. Ohne detaillierte Rechnung oder Beweis ergeben sich die folgenden Aussagen:

\begin{MXInfo}{Ableitung von $x^n$}
Gegeben sind eine nat"urliche Zahl $n$ und eine reelle Zahl $r$.

Die konstante Funktion $f(x) := r = r \cdot x^0$ hat die Ableitung
$f'(x) = 0$.

Die Funktion $f(x) := r \cdot x^n$ hat die Ableitung 
\[
f'(x) = r \cdot n \cdot x^{n-1} %%
\]
(Diese Ableitungsregel gilt im �brigen auch, f�r $n\in\mathbb{Z}\setminus\{0\}$)
\end{MXInfo}

�berpr�fen Sie auch diese Aussagen selbst!

\begin{MExample}
Wir untersuchen die Funktion $f: \R \rightarrow \R$ mit $f(x) = 5 x^3$.
Vergleichen wir die gegebene Funktion mit den oben verwendeten Bezeichnungen, dann sind $r = 5$ und $n = 3$. Damit erh"alt man die 
Ableitung 
\[
f'(x) = 5 \cdot 3 x^{3 - 1} = 15 x^2. %%
\]
\end{MExample}


F"ur Wurzelfunktionen ergibt sich eine entsprechende Aussage. Allerdings 
ist zu beachten, dass Wurzelfunktionen nur f"ur $x > 0$ differenzierbar sind.
Denn die Tangente an den Funktionsgraphen durch den Punkt $(0;0)$ verl"auft 
parallel zur $y$-Achse und beschreibt somit keine Funktion. 

\begin{MXInfo}{Ableitung von $x^{\frac{1}{n}}$}
F"ur $n \in \Z$ mit $n \neq 0$ ist die Funktion
$f(x) := x^{\frac{1}{n}}$ f"ur $x \geq 0$ differenzierbar, und es gilt
\[
f'(x) = \frac{1}{n} \cdot x^{\frac{1}{n}-1} \text{ f"ur } x > 0 %%
\]%
\end{MXInfo}

F"ur $n\in\mathbb{N}$ werden durch $f(x) = x^{\frac{1}{n}}$ Wurzelfunktionen beschrieben. Die hier beschriebene Ableitungsregel gilt nat�rlich auch f�r $n = 1$ oder $n = -1$.
\begin{MExample}
Die Wurzelfunktion $f: [0, \infty) \rightarrow \R$ mit
 $f(x) := \sqrt{x} = x^{\frac{1}{2}}$
ist f"ur $x > 0$ differenzierbar. Die Ableitung ist durch
\[
f'(x) = \frac{1}{2} \cdot x^{\frac{1}{2}-1} %
= \frac{1}{2} \cdot x^{-\frac{1}{2}} = \frac{1}{2 \cdot \sqrt{x}} %%
\] 
gegeben. In diesem Beispiel ist leicht zu sehen, dass die Ableitung in $x_0 = 0$ nicht existiert.

%Bild:
\ifttm
\MUGraphics{\MPfadBilder/jb07A2_Wurzelfunktion.png}{scale=0.5}%
{Graph von $\sqrt{x}$ mit Tangente in $x_0 = 1$}{}
\else
\begin{center}
%Datei: {\MPfadBilder/jb07A1_Wurzelfunktion.tex}
%\input{\MPfadBilder/jb07A1_Wurzelfunktion.tex}
%LaTeX-File, Liedtke, 20140916.
%VBKM-Modul 7 Differentialrechnung: Bild zur Exponentialfunktion
%Bildname: jb07A2_Wurzelfunktion
%Erstellt: 20140922, Liedtke.
\begin{small}
\renewcommand{\jTikZScale}{0.8}
\tikzsetnextfilename{jb07A2_Wurzelfunktion}
\begin{tikzpicture}[line width=1.5pt,scale=\jTikZScale]
%\begin{small}
%\begin{tikzpicture}[line width=1.5pt,scale=\jTikZScale, %
%declare function={
%  fkt(\x) = sqrt(\x);
%}
%] %[every node/.style={fill=white}] 
%Koordinatenachsen:
\draw[->] (-0.6, 0) -- (4.8, 0) node[below left]{$x$}; %x-Achse
\draw[->] (0, -0.6) -- (0, 3) node[below left]{$y$}; %y-Achse
%Achsenbeschriftung:
\foreach \x in {1, 2, 3, 4} \draw (\x, 0) -- ++(0, -0.1) %
 node[below] {$\x$};
\foreach \y in {1, 2} \draw (0, \y) -- ++(-0.1, 0) node[left] {$\y$};
%\node[below left] at (0, 0) {$0$};
%Funktion:
\draw[domain=0:4,samples=120,color=\jccolorfkt] %
 plot (\x, {sqrt(\x)});
%Tangenten y = f(x_0) + f'(x_0) * (x - x_0) im Punkt $(x_0, f(x_0)$ des Graphen:
%$x_0 := 1$:
\draw[samples=120,color=blue!50!black] %
 (0.5, {1 - 1/2 * 1/2}) -- ++(1, {1/2});
%Punkt:
\filldraw[color=black, fill=black] (1, 0) circle (2pt);
%end of file
\end{tikzpicture}
\end{small}
\end{center}
\fi
%Bildende.
Die Tangente in $x_0 = 1$ an den Graphen der Wurzelfunktion $f(x) = \sqrt{x}$ hat 
die Steigung $\frac{1}{2 \sqrt{1}} = \frac{1}{2}$.
\end{MExample}

Die bisher getroffenen Aussagen k�nnen f"ur Exponenten $p \in \R$ mit 
$p \neq 0$ f"ur $x > 0$ verallgemeinert werden:
Die Ableitung von $f(x) = x^p$ f"ur $x > 0$ ist
\[
f'(x) = p \cdot x^{p-1} %%
\]
\end{MXContent}



\begin{MXContent}{Ableitung spezieller Funktionen}{Ableitung spezieller Funktionen}{STD}

\MSubsubsectionx{Ableitung trigonometrischer Funktionen}

Die Sinusfunktion ist periodisch mit Periode $2 \pi$. Somit gen"ugt es, die
Funktion auf einem Intervall der L"ange $2 \pi$ zu betrachten. Einen Ausschnitt 
des Graphen f"ur $-\pi \leq x \leq \pi$ zeigt die folgende Abbildung:

%Bild:
\ifttm
\begin{center}
\MUGraphicsSolo{\MPfadBilder/jb07A2_SinusFktUndAbl.png}{scale=0.8}{}
%\MUGraphicsSolo{\MPfadBilder/jb07A2_SinusFktGraph.png}{scale=0.8}{}
%\MUGraphicsSolo{\MPfadBilder/jb07A2_SinusAbl.png}{scale=0.8}{}
%\MUGraphicsSolo{jb07A2_SinusFktGraph.png}{scale=1}{}
%\MUGraphicsSolo{jb07A2_SinusAbl.png}{scale=1}{}
\end{center}
\else
\begin{center}
%Bild: {\MPfadBilder/jb07A2_SinusFktGraph.tex}%
%LaTeX-File, Liedtke, 20140826.
%VBKM-Modul 7 Differentialrechnung: Bild zur Sinusfunktion.
%Bildname: jb07A2_SinusFktGraph.tex.
%Erstellt: 20140827, Liedtke.
%Bearbeitet: 20140829, Liedtke (Dateiname angepasst).
%Bearbeitet: 20140901, Liedtke (Dateiname ohne Endung erfasst).
\begin{small}
\renewcommand{\jTikZScale}{1.0}
\tikzsetnextfilename{jb07A2_SinusFktUndAbl}
%\tikzsetnextfilename{jb07A2_SinusFktGraph}
\begin{tikzpicture}[line width=1.5pt,scale=\jTikZScale, %
declare function={
  fkt(\x) = sin(\x r);
  fktabl(\x) = cos(\x r);
}
] %[every node/.style={fill=white}] 
%
%Graph der Sinusfunktion:
\node[right] at (-6,1) {Sinusfunktion};
\begin{scope}%[xshift=-6]
%Koordinatenachsen:
\draw[->] (-3.6, 0) -- (4, 0) node[below left]{$x$}; %x-Achse
\draw[->] (0, -1.6) -- (0, 1.6) node[below left]{$y$}; %y-Achse
%Achsenbeschriftung:
\foreach \x in {{-pi}, {-pi/2}} \draw (\x, 0) -- ++(0, -0.1);
\node[below] at ({-pi}, 0) {$-\pi$};
\node[below] at ({-pi/2}, 0) {$-\frac{\pi}{2}$};
\foreach \x in {{pi/2}, pi} \draw (\x, 0) -- ++(0, -0.1);
\node[below] at ({pi/2}, 0) {$-\frac{\pi}{2}$};
\node[below] at ({pi}, 0) {$-\pi$};
\foreach \y in {-1} \draw (0, \y) -- ++(-0.1, 0) %
 node[left] {$\y$};
\foreach \y in {1} \draw (0, \y) -- ++(-0.1, 0) %
 node[left] {$\y$};
%\node[below left] at (0, 0) {$0$};
%Funktion:
\draw[domain=-3.14:3.14,samples=120,color=\jccolorfkt] %
 plot (\x, {fkt(\x)});
%Tangenten in verschiedenen Punkten:
\draw[samples=120,color=blue!50!black] %
 plot (-0.5,-0.5) -- (0.5,0.5);
\draw[samples=120,color=blue!50!black] %
 plot ({pi/3},1) -- ({2*pi/3},1);
\node[above] at ({pi/2}, 1) {$\sin'(\pi/2) = 0$};
\draw[samples=120,color=blue!50!black] %
 plot ({-2*pi/3},-1) -- ({-pi/3},-1);
\node[below] at ({-pi/2}, -1) {$\sin'(-\pi/2) = 0$};
\end{scope}
%end of file
%
%
%Graph der Ableitung der Sinusfunktion:
%Bild: {\MPfadBilder/jb07A2_SinusAbl.tex}
%LaTeX-File, Liedtke, 20140826.
%VBKM-Modul 7 Differentialrechnung: Bild zur Ableitung der Sinusfunktion.
%Bildname: jb07A2_SinusAbl.tex.
%Erstellt: 20140827, Liedtke.
%Bearbeitet: 20140829, Liedtke (Dateiname angepasst).
%Bearbeitet: 20140901, Liedtke (Dateiname ohne Endung erfasst).
%
%\tikzsetnextfilename{jb07A2_SinusAbl}
%\begin{tikzpicture}[line width=1.5pt,scale=\jTikZScale, %
%declare function={
%  fktabl(\x) = cos(\x r);
%}
%] %[every node/.style={fill=white}] 
\node[right] at (-6,-2.8) {Ableitung};
\begin{scope}[yshift=-3.8cm]
%Koordinatenachsen:
\draw[->] (-3.6, 0) -- (4, 0) node[below left]{$x$}; %x-Achse
\draw[->] (0, -1.6) -- (0, 1.6) node[below left]{$y$}; %y-Achse
%Achsenbeschriftung:
\foreach \x in {{-pi}, {-pi/2}} \draw (\x, 0) -- ++(0, 0.1); 
\node[below] at ({-pi}, 0) {$-\pi$}; 
\node[below] at ({-pi/2}, 0) {$-\frac{\pi}{2}$}; 
\foreach \x in {{pi/2}, {pi}} \draw (\x, 0) -- ++(0, -0.1);
\node[below] at ({pi/2}, 0) {$\frac{\pi}{2}$}; 
\node[below] at ({pi}, 0) {$\pi$}; 
\foreach \y in {-1} \draw (0, \y) -- ++(0.1, 0);
\foreach \y in {1} \draw (0, \y) -- ++(-0.1, 0) %
 node[below left] {$\y$};
%\node[below left] at (0, 0) {$0$};
%Funktion:
\draw[domain=-3.14:3.14,samples=120,color=blue!50!white] %
 plot (\x, {fktabl(\x)});
%Punkte
\filldraw[color=black,fill=black] ({-pi}, -1) circle (2pt);
\filldraw[color=black,fill=black] ({-pi/2}, 0) circle (2pt);
\filldraw[color=black,fill=black] (0, 1) circle (2pt);
\filldraw[color=black,fill=black] ({pi/2}, 0) circle (2pt);
\filldraw[color=black,fill=black] ({pi}, -1) circle (2pt);
\end{scope}
\end{tikzpicture}
\end{small}
\end{center}
%end of file
\fi
%Bildende.

Wie in der Abbildung zu sehen ist, ist die Steigung des Sinus bei $x_0 = \pm\frac{\pi}{2}$ gerade $f'(\pm\frac{\pi}{2}) = 0$. Legt man eine Tangente an der Stelle $x_0 = 0$ an den Sinus, wird man die Steigung $f'(0) = 1$ finden. Untersucht man die Stellen $x_0 = \pm\pi$, ist sofort einsichtig, dass dort dieselbe Steigung herrschen muss wie bei $x_0 = 0$, aber mit umgedrehtem Vorzeichen. Die Steigung ist dort also $f'(\pm\pi) = -1$. Als Ableitung des Sinus suchen wir also eine Funktion, die genau diese Eigenschaften erf�llt. Eine genaue Untersuchung der Bereiche zwischen diesen speziell ausgesuchten Stellen ergibt, dass der Kosinus die Ableitung des Sinus beschreibt:

\begin{MXInfo}{Ableitung trigonometrischer Funktionen}
F"ur die Sinusfunktion $f(x) := \sin(x)$ gilt 
\[
f'(x) = \cos(x) %%
\]
F"ur die Kosinusfunktion $g(x) := \cos(x)$ gilt 
\[
g'(x) = -\sin(x) %%
\]
F"ur die Tangensfunktion $h(x) := \tan(x)$ f"ur $x \neq \frac{\pi}{2} + k \pi$ 
mit $k \in \Z$ gilt
\[
h'(x) = 1 + (\tan(x))^2 = \frac{1}{\cos^2(x)} %%
\]
\end{MXInfo}
Letzteres ergibt sich auch aus den nachfolgend erl"auterten Rechenregeln und der 
Definition des Tangens als Quotient von Sinus und Kosinus.

\MSubsubsectionx{Ableitung der Exponentialfunktion}

Die Exponentialfunktion $f(x) := e^{x} = \exp(x)$ hat die besondere Eigenschaft, 
dass ihre Ableitung wiederum $f'(x) = \exp(x)$ ist.

\MSubsubsectionx{Ableitung der Logarithmusfunktion}

%Die Logarithmusfunktion $f(y) := \ln(y)$ f"ur $y > 0$ ist die Umkehrfunktion
%der Exponentialfunktion $y = \MEU^x$. Die Ableitung und damit die Steigung $m$ einer Tangente an den 
%Graphen der Exponentialfunktion ist $m = \MEU^x$. 

%In der folgenden Abbildung ist die Tangente an den Graphen der Exponentialfunktion im
%Punkt $(x_0, \MEU^{x_0})$ f"ur $x_0 = \ln(2)$ angedeutet, sodass die 
%Tangentensteigung $m = 2$ ist. Die Umkehrabbildung der Tangente hat dann 
%die Steigung $\frac{1}{m} = \frac{1}{2}$. 
%Geometrisch ist es die Steigung der gespiegelten Tangente, also die Tangente
%an die Umkehrfunktion $\ln$.

Verwenden wir wieder die gewohnten Bezeichnungen f�r die unabh�ngige Variable, k�nnen wir ohne Beweis die Ableitung der Logarithmusfunktion angeben. Es gilt: $f'(x) = \frac{1}{x}$ ist die Ableitung der 
Logarithmusfunktion $f(x) = \ln(x)$.

%Bild:
\ifttm
\MUGraphics{\MPfadBilder/jb07A2_Logarithmusfunktion.png}{scale=0.5}%
{Tangenten an $\exp$ und an die Umkehrfunktion $\ln$}{}
\else
\begin{center}
%Datei: {\MPfadBilder/jb07A1_BspExponentialfunktion.tex}
%\input{\MPfadBilder/jb07A1_BspExponentialfunktion.tex}
%LaTeX-File, Liedtke, 20140916.
%VBKM-Modul 7 Differentialrechnung: Bild zur Exponentialfunktion
%Bildname: jb07A2_BspExponentialfunktion
%Erstellt: 20140916, Liedtke.
\begin{small}
\renewcommand{\jTikZScale}{0.8}
\tikzsetnextfilename{jb07A2_Logarithmusfunktion}
\begin{tikzpicture}[line width=1.5pt,scale=\jTikZScale]
%\begin{small}
%\begin{tikzpicture}[line width=1.5pt,scale=\jTikZScale, %
%declare function={
%  fkt(\x) = sin(\x r);
%}
%] %[every node/.style={fill=white}] 
%Koordinatenachsen:
\draw[->] (-3.2, 0) -- (4.6, 0) node[below left]{$x$}; %x-Achse
\draw[->] (0, -3.2) -- (0, 4) node[below left]{$y$}; %y-Achse
%Achsenbeschriftung:
\foreach \x in {-3, -2, -1} \draw (\x, 0) -- ++(0, -0.1); % node[below] {$\x$};
\foreach \x in {1, 2, 3} \draw (\x, 0) -- ++(0, -0.1) node[below] {$\x$};
\foreach \y in {-3, -2, -1} \draw (0, \y) -- ++(-0.1, 0); % node[left] {$\y$};
\foreach \y in {1, 2, 3} \draw (0, \y) -- ++(-0.1, 0) node[left] {$\y$};
%\node[below left] at (0, 0) {$0$};
%
%Funktion exp:
\draw[domain=-2.4:1.3,samples=120,color=white!50!black] %\jccolorfkt] %
 plot (\x, {exp(\x)});
%Tangenten in verschiedenen Punkten:
%x_0 = 0:
%\draw[samples=120,color=blue!50!black] %
% (-0.45, 0.45) -- (0.55, 1.55);
%
%x_0 = 1:
% ({3/4}, {exp(1) - 1/4 * exp(1)}) -- ++({2/4}, {2/4*exp(1)});
%
%x_0 = ln(2): y = exp(ln(2)) + 2 * (x - ln(2)) = 2 + 2 * (x - ln(2)):
\draw[style=dashed,samples=120,color=blue!50!black] %
 ({ln(2) - 3/8}, {5/4}) -- ++({6/8}, {6/8*2});
%
%Punkte:
%\filldraw[color=black] (0, 1) circle (2pt);
\filldraw[color=black] (0, 2) circle (2pt);
\filldraw[color=black] ({ln(2)}, 2) circle (2pt);
%
%Spiegelachse (erste Winkelhalbierende):
\draw[samples=120,style=dotted,color=white!50!black] %
 (-2.1, -2.1) -- (3.1, 3.1);
%
%Umkehrfunktion ln:
\draw[domain=0.1:{exp(1.3},samples=120,color=\jccolorfkt] %
 plot (\x, {ln(\x)});
%Tangenten in verschiedenen Punkten:
%\draw[samples=120,color=blue!50!black] %
% (0.45, -0.45) -- (1.55, 0.55);
%
%\draw[style=dotted,samples=120,color=blue!50!black] %
% ({exp(1) - 1/2}, {1 - 1/2 * 1/exp(1)}) -- ++(1, {1/exp(1)});
%\draw[style=dotted,samples=120,color=blue!50!black] %
% ({exp(1) - 1/4*exp(1)}, {3/4}) -- ++({2/4*exp(1)}, {2/4});
%x_0 = 2: y = ln(2) + 1/2 * (x - 2):
\draw[samples=120,color=blue!50!black] %
 ({2 - 3/4}, {ln(2) - 1/2 * 3/4}) -- ++({2*3/4}, {2*3/4*1/2});
%Punkte:
%\filldraw[color=black] (1, 0) circle (2pt);
\filldraw[color=black] (2, 0) circle (2pt);
\filldraw[color=black] (2, {ln(2)}) circle (2pt);
\end{tikzpicture}
\end{small}
\end{center}
\fi
%Bildende.

\end{MXContent}

%\end{MContent}

%end of file.

