%LaTeX-2e-File, Liedtke, 20140731.
%Inhalt: Einf"uhrung der Ableitung, Abschnitt 6: Zusammenfassung.
%zuletzt bearbeitet: 20140930.

%\MSubsection{Zusammenfassung}

%\MSubsubsection{Ableitung}
\begin{MXContent}{Ableitung}{Ableitung}{STD}
Es ist $f: [a, b] \rightarrow \R$ in $x_0 \in (a, b)$ differenzierbar, wenn
\[
m = \lim_{h \rightarrow 0} \frac{f(x_0 + h) - f(x_0)}{h} %%
\]
existiert. Dann hei"st $m$ die Ableitung von $f$ an der Stelle $x_0$,
und wird mit $\frac{\MD f(x_0)}{\MD x} := f'(x) := m$ bezeichnet.

Im Folgenden werden die wichtigsten Rechenregeln und Aussagen zusammengefasst.
\end{MXContent}
 
%\MSubsubsection{Standardableitungen}
\begin{MXContent}{Standardableitungen}{Standardableitungen}{STD}
\begin{tabular}{lp{55mm}l}
Konstante Funktion & $f(x) := a$ & 
 $f'(x) = 0$ \\
Monom & $f(x) := x^n$ f"ur $n \in \Z$ und $n \neq 0$ & 
 $f'(x) = n \cdot x^{n-1}$ \\
Wurzel & $f(x) := x^{\frac{1}{n}}$ f"ur $x \geq 0$ und 
 f"ur $n \in \Z$ und $n \neq 0$ & 
 $f'(x) = n \cdot x^{n-1}$ f"ur $x > 0$ \\
Sinusfunktion & $f(x) := \sin(x)$ & $f'(x) = \cos(x)$ \\
Kosinusfunktion & $f(x) := \cos(x)$ & $f'(x) = -\sin(x)$ \\
Tangensfunktion & $f(x) := \tan(x)$ f"ur $x \neq \frac{\pi}{2} + k \pi$ %
 & $f'(x) = 1 + (\tan(x))^2 = \frac{1}{\cos^2(x)}$ \\
Exponentialfunktion & $f(x) := \exp(x)$ & $f'(x) = \exp(x)$ \\
Logarithmusfunktion & $f(x) := \ln(x)$ f"ur $x > 0$ & $f'(x) = \frac{1}{x}$ %%
\end{tabular}
\end{MXContent}

%\MSubsubsection{Rechenregeln}
\begin{MXContent}{Rechenregeln}{Rechenregeln}{STD}
Die Funktionen $g$, $h$, $u$ und $v$ seien differenzierbar. Dann ist 
auch die nachstehend definierte Funktion $f$ differenzierbar, und 
die Ableitung von $f$ kann wie angegeben berechnet werden:
\begin{description}
\item[Vielfache] F"ur $f(x) := r \cdot g(x)$ ist
 $f'(x) = r \cdot g'(x)$
\item[Summe] F"ur $f(x) := g(x) + h(x)$ ist
  $f'(x) = g'(x) + h'(x)$
\item[Produktregel] F"ur $f(x) := u(x) \cdot v(x)$ ist 
 $f'(x) = u'(x) \cdot v(x) + u(x) \cdot v'(x)$
\item[Quotientenregel] F"ur $f(x) := \frac{u(x)}{v(x)}$ ist 
 $f'(x) = \frac{u'(x) \cdot v(x) - u(x) \cdot v'(x)}{(v(x))^2}$
\item[Kettenregel] F"ur $f(x) := (g \circ u)(x) = g(u(x))$ ist
 $f'(x) = g'(u(x)) \cdot u'(x)$
\end{description}
\end{MXContent}

%\MSubsubsection{Eigenschaften}
\begin{MXContent}{Eigenschaften}{Eigenschaften}{STD}
Eine differenzierbare Funktion $f: [a, b] \rightarrow \R$ ist stetig, und es gilt:
\begin{itemize}
\item Wenn $f'(x) \geq 0$ f"ur alle $x \in (a, b)$ gilt, dann ist $f$ monoton 
wachsend.
\item Wenn $f'(x) \leq 0$ f"ur alle $x \in (a, b)$ gilt, dann ist $f$ monoton 
fallend.
\end{itemize}
Wenn $f$ zweimal differenzierbar ist, gilt zudem (wobei die zweite Ableitung
mit $f^{(2)}$ bezeichnet wird):
\begin{itemize}
\item Wenn $f^{(2)}(x) \geq 0$ f"ur alle $x \in (a, b)$ gilt, dann ist $f$ konvex 
(linksgekr"ummt). 
\item Wenn $f^{(2)}(x) \leq 0$ f"ur alle $x \in (a, b)$ gilt, dann ist $f$ konkav 
(rechtsgekr"ummt). 
\end{itemize}

\begin{MXInfo}{Extremstellen}
Wenn $x_0$ eine Extremstellen in $(a, b)$ ist, dann gilt $f'(x_0) = 0$. Somit 
k"onnen nur solche Stellen der differenzierbaren Funktion $f$ auf $(a, b)$ 
Extremstellen sein.

Wenn $f'(x_0) = 0$ ist und zudem ein Vorzeichenwechsel der Ableitung in $x_0$ 
vorliegt oder die zweite Ableitung ungleich null ist, dann ist $x_0$ eine 
Extremstellen. Au"serdem kann dann festgesetllt werden, ob es sich um eine 
Minimalstelle oder eine Maximalstelle von $f$ handelt:
Wenn $f'(x_0) = 0$ ist, und
\begin{itemize}
\item $f'(x) < 0$ f"ur $x < x_0$ und $f'(x) > 0$ f"ur $x > x_0$ gilt, 
 dann ist $x_0$ eine Minimalstelle von $f$.
\item $f'(x) > 0$ f"ur $x < x_0$ und $f'(x) < 0$ f"ur $x > <_0$ gilt,
 dann ist $x_0$ eine Maximalstelle von $f$.
\end{itemize}

Oder wenn $f'(x_0) = 0$ ist, die zweite Ableitung in $x_0$ existiert und
\begin{itemize}
\item $f^{(2)}(x_0) > 0$ gilt, dann ist $x_0$ eine Minimalstelle von $f$.
\item $f^{(2)}(x_0) < 0$ gilt, dann ist $x_0$ eine Maximalstelle von $f$.
\end{itemize}
\end{MXInfo}

\begin{MXInfo}{Wendestellen}
Wenn $w_0$ eine Wendestellen in $(a, b)$ ist und $f$ zweimal differenzierbar 
ist, muss $f^{(2)}(w_0) = 0$ gelten. Somit gen"ugt es, diese Stellen weiter
zu untersuchen.

Es ist $w_0$ eine Wendestelle in $(a, b)$, wenn  $f^{(2)}(w_0) = 0$ ist und 
eine der folgenden weiteren Eigenschaften gelten:
\begin{itemize}
\item In $w_0$ liegt ein Vorzeichenwechsel der zweiten Ableitung vor (das hei"st,
f"ur kleinere Werte $x < w_0$ ist $f^{(2)}(x) < 0$ und f"ur gr"o"sere Werte 
$x > w_0$ ist $f^{(2)}(x) > 0$ oder umgekehrt).
%
\item Oder es gilt $f^{(3)}(w_0) \neq 0$.
\end{itemize}
Das letzte Kriterium erfordert neben der Bedingung $f^{(2)}(w_0)$ nat"urlich, 
dass die dritte Ableitung existiert und in der Praxis dann erst mal berechnet 
werden muss.
%Hinreichend f"ur eine Wendestelle $w_0$ ist, dass $f^{(2)}(w_0) = 0$ und 
% $f^{(3)}(w_0) \neq 0$ gilt.
\end{MXInfo}
\end{MXContent}

%\MSubsubsection{Anwendungen}
\begin{MXContent}{Anwendungen}{Anwendungen}{STD}
F"ur differenzierbare Funktionen wurden folgende Anwendungen vorgestellt:
\begin{itemize}
%\item Regel von de l'Hospital
\item Kurvendiskussion
\item Optimierungsaufgabe
\end{itemize}
\end{MXContent}

%end of file.

