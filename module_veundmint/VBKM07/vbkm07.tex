%LaTeX-2e-Datei, Liedtke, 20140729.
%Inhalt: Modul 7 VEMINT-Basiskurs Mathematik.
%Thema:  Differentialrechnung.
%zuletzt bearbeitet: 20140922.

% MINTMOD Version P0.1.0, needs to be consistent with preprocesser object in tex2x and MPragma-Version at the end of this file

% Parameter aus Konvertierungsprozess (PDF und HTML-Erzeugung wenn vom Konverter aus gestartet) werden hier eingefuegt, Preambleincludes werden am Schluss angehaengt

\newif\ifttm                % gesetzt falls Uebersetzung in HTML stattfindet, sonst uebersetzung in PDF

% Wahl der Notationsvariante ist im PDF immer std, in der HTML-Uebersetzung wird vom Konverter die Auswahl modifiziert
\newif\ifvariantstd
\newif\ifvariantunotation
\variantstdtrue % Diese Zeile wird vom Konverter erkannt und ggf. modifiziert, daher nicht veraendern!


\def\MOutputDVI{1}
\def\MOutputPDF{2}
\def\MOutputHTML{3}
\newcounter{MOutput}

\ifttm
\usepackage{german}
\usepackage{array}
\usepackage{amsmath}
\usepackage{amssymb}
\usepackage{amsthm}
\else
\documentclass[ngerman,oneside]{scrbook}
\usepackage{etex}
\usepackage[latin1]{inputenc}
\usepackage{textcomp}
\usepackage[ngerman]{babel}
\usepackage[pdftex]{color}
\usepackage{xcolor}
\usepackage{graphicx}
\usepackage[all]{xy}
\usepackage{fancyhdr}
\usepackage{verbatim}
\usepackage{array}
\usepackage{float}
\usepackage{makeidx}
\usepackage{amsmath}
\usepackage{amstext}
\usepackage{amssymb}
\usepackage{amsthm}
\usepackage[ngerman]{varioref}
\usepackage{framed}
\usepackage{supertabular}
\usepackage{longtable}
\usepackage{maxpage}
\usepackage{tikz}
\usepackage{tikzscale}
\usepackage{tikz-3dplot}
\usepackage{bibgerm}
\usepackage{chemarrow}
\usepackage{polynom}
%\usepackage{draftwatermark}
\usepackage{pdflscape}
\usetikzlibrary{calc}
\usetikzlibrary{through}
\usetikzlibrary{shapes.geometric}
\usetikzlibrary{arrows}
\usetikzlibrary{intersections}
\usetikzlibrary{decorations.pathmorphing}
\usetikzlibrary{external}
\usetikzlibrary{patterns}
\usetikzlibrary{fadings}
\usepackage[colorlinks=true,linkcolor=blue]{hyperref} 
\usepackage[all]{hypcap}
%\usepackage[colorlinks=true,linkcolor=blue,bookmarksopen=true]{hyperref} 
\usepackage{ifpdf}

\usepackage{movie15}

\setcounter{tocdepth}{2} % In Inhaltsverzeichnis bis subsection
\setcounter{secnumdepth}{3} % Nummeriert bis subsubsection

\setlength{\LTpost}{0pt} % Fuer longtable
\setlength{\parindent}{0pt}
\setlength{\parskip}{8pt}
%\setlength{\parskip}{9pt plus 2pt minus 1pt}
\setlength{\abovecaptionskip}{-0.25ex}
\setlength{\belowcaptionskip}{-0.25ex}
\fi

\ifttm
\newcommand{\MDebugMessage}[1]{\special{html:<!-- debugprint;;}#1\special{html:; //-->}}
\else
%\newcommand{\MDebugMessage}[1]{\immediate\write\mintlog{#1}}
\newcommand{\MDebugMessage}[1]{}
\fi

\def\MPageHeaderDef{%
\pagestyle{fancy}%
\fancyhead[r]{(C) VE\&MINT-Projekt}
\fancyfoot[c]{\thepage\\--- CCL BY-SA 3.0 ---}
}


\ifttm%
\def\MRelax{}%
\else%
\def\MRelax{\relax}%
\fi%

%--------------------------- Uebernahme von speziellen XML-Versionen einiger LaTeX-Kommandos aus xmlbefehle.tex vom alten Kasseler Konverter ---------------

\newcommand{\MSep}{\left\|{\phantom{\frac1g}}\right.}

\newcommand{\ML}{L}

\newcommand{\MGGT}{\mathrm{ggT}}


\ifttm
% Verhindert dass die subsection-nummer doppelt in der toccaption auftaucht (sollte ggf. in toccaption gefixt werden so dass diese Ueberschreibung nicht notwendig ist)
\renewcommand{\thesubsection}{}
% Kommandos die ttm nicht kennt
\newcommand{\binomial}[2]{{#1 \choose #2}} %  Binomialkoeffizienten
\newcommand{\eur}{\begin{html}&euro;\end{html}}
\newcommand{\square}{\begin{html}&square;\end{html}}
\newcommand{\glqq}{"'}  \newcommand{\grqq}{"'}
\newcommand{\nRightarrow}{\special{html: &nrArr; }}
\newcommand{\nmid}{\special{html: &nmid; }}
\newcommand{\nparallel}{\begin{html}&nparallel;\end{html}}
\newcommand{\mapstoo}{\begin{html}<mo>&map;</mo>\end{html}}

% Schnitt und Vereinigungssymbole von Mengen haben zu kleine Abstaende; korrigiert:
\newcommand{\ccup}{\,\!\cup\,\!}
\newcommand{\ccap}{\,\!\cap\,\!}


% Umsetzung von mathbb im HTML
\renewcommand{\mathbb}[1]{\begin{html}<mo>&#1opf;</mo>\end{html}}
\fi

%---------------------- Strukturierung ----------------------------------------------------------------------------------------------------------------------

%---------------------- Kapselung des sectioning findet auf drei Ebenen statt:
% 1. Die LateX-Befehl
% 2. Die D-Versionen der Befehle, die nur die Grade der Abschnitte umhaengen falls notwendig
% 3. Die M-Versionen der Befehle, die zusaetzliche Formatierungen vornehmen, Skripten starten und das HTML codieren
% Im Modultext duerfen nur die M-Befehle verwendet werden!

\ifttm

  \def\Dsubsubsubsection#1{\subsubsubsection{#1}}
  \def\Dsubsubsection#1{\subsubsection{#1}\addtocounter{subsubsection}{1}} % ttm-Fehler korrigieren
  \def\Dsubsection#1{\subsection{#1}}
  \def\Dsection#1{\section{#1}} % Im HTML wird nur der Sektionstitel gegeben
  \def\Dchapter#1{\chapter{#1}}
  \def\Dsubsubsubsectionx#1{\subsubsubsection*{#1}}
  \def\Dsubsubsectionx#1{\subsubsection*{#1}}
  \def\Dsubsectionx#1{\subsection*{#1}}
  \def\Dsectionx#1{\section*{#1}}
  \def\Dchapterx#1{\chapter*{#1}}

\else

  \def\Dsubsubsubsection#1{\subsubsection{#1}}
  \def\Dsubsubsection#1{\subsection{#1}}
  \def\Dsubsection#1{\section{#1}}
  \def\Dsection#1{\chapter{#1}}
  \def\Dchapter#1{\title{#1}}
  \def\Dsubsubsubsectionx#1{\subsubsection*{#1}}
  \def\Dsubsubsectionx#1{\subsection*{#1}}
  \def\Dsubsectionx#1{\section*{#1}}
  \def\Dsectionx#1{\chapter*{#1}}

\fi

\newcommand{\MStdPoints}{4}
\newcommand{\MSetPoints}[1]{\renewcommand{\MStdPoints}{#1}}

% Befehl zum Abbruch der Erstellung (nur PDF)
\newcommand{\MAbort}[1]{\err{#1}}

% Prefix vor Dateieinbindungen, wird in der Baumdatei mit \renewcommand modifiziert
% und auf das Verzeichnisprefix gesetzt, in dem das gerade bearbeitete tex-Dokument liegt.
% Im HTML wird es auf das Verzeichnis der HTML-Datei gesetzt.
% Das Prefix muss mit / enden !
\newcommand{\MDPrefix}{.}

% MRegisterFile notiert eine Datei zur Einbindung in den HTML-Baum. Grafiken mit MGraphics werden automatisch eingebunden.
% Mit MLastFile erhaelt man eine Markierung fuer die zuletzt registrierte Datei.
% Diese Markierung wird im postprocessing durch den physikalischen Dateinamen ersetzt, aber nur den Namen (d.h. \MMaterial gehoert noch davor, vgl Definition von MGraphics)
% Parameter: Pfad/Name der Datei bzw. des Ordners, relativ zur Position des Modul-Tex-Dokuments.
\ifttm
\newcommand{\MRegisterFile}[1]{\addtocounter{MFileNumber}{1}\special{html:<!-- registerfile;;}#1\special{html:;;}\MDPrefix\special{html:;;}\arabic{MFileNumber}\special{html:; //-->}}
\else
\newcommand{\MRegisterFile}[1]{\addtocounter{MFileNumber}{1}}
\fi

% Testen welcher Uebersetzer hier am Werk ist

\ifttm
\setcounter{MOutput}{3}
\else
\ifx\pdfoutput\undefined
  \pdffalse
  \setcounter{MOutput}{\MOutputDVI}
  \message{Verarbeitung mit latex, Ausgabe in dvi.}
\else
  \setcounter{MOutput}{\MOutputPDF}
  \message{Verarbeitung mit pdflatex, Ausgabe in pdf.}
  \ifnum \pdfoutput=0
    \pdffalse
  \setcounter{MOutput}{\MOutputDVI}
  \message{Verarbeitung mit pdflatex, Ausgabe in dvi.}
  \else
    \ifnum\pdfoutput=1
    \pdftrue
  \setcounter{MOutput}{\MOutputPDF}
  \message{Verarbeitung mit pdflatex, Ausgabe in pdf.}
    \fi
  \fi
\fi
\fi

\ifnum\value{MOutput}=\MOutputPDF
\DeclareGraphicsExtensions{.pdf,.png,.jpg}
\fi

\ifnum\value{MOutput}=\MOutputDVI
\DeclareGraphicsExtensions{.eps,.png,.jpg}
\fi

\ifnum\value{MOutput}=\MOutputHTML
% Wird vom Konverter leider nicht erkannt und daher in split.pm hardcodiert!
\DeclareGraphicsExtensions{.png,.jpg,.gif}
\fi

% Umdefinition der hyperref-Nummerierung im PDF-Modus
\ifttm
\else
\renewcommand{\theHfigure}{\arabic{chapter}.\arabic{section}.\arabic{figure}}
\fi

% Makro, um in der HTML-Ausgabe die zuerst zu oeffnende Datei zu kennzeichnen
\ifttm
\newcommand{\MGlobalStart}{\special{html:<!-- mglobalstarttag -->}}
\else
\newcommand{\MGlobalStart}{}
\fi

% Makro, um bei scormlogin ein pullen des Benutzers bei Aufruf der Seite zu erzwingen (typischerweise auf der Einstiegsseite)
\ifttm
\newcommand{\MPullSite}{\special{html:<!-- pullsite //-->}}
\else
\newcommand{\MPullSite}{}
\fi

% Makro, um in der HTML-Ausgabe die Kapiteluebersicht zu kennzeichnen
\ifttm
\newcommand{\MGlobalChapterTag}{\special{html:<!-- mglobalchaptertag -->}}
\else
\newcommand{\MGlobalChapterTag}{}
\fi

% Makro, um in der HTML-Ausgabe die Konfiguration zu kennzeichnen
\ifttm
\newcommand{\MGlobalConfTag}{\special{html:<!-- mglobalconfigtag -->}}
\else
\newcommand{\MGlobalConfTag}{}
\fi

% Makro, um in der HTML-Ausgabe die Standortbeschreibung zu kennzeichnen
\ifttm
\newcommand{\MGlobalLocationTag}{\special{html:<!-- mgloballocationtag -->}}
\else
\newcommand{\MGlobalLocationTag}{}
\fi

% Makro, um in der HTML-Ausgabe die persoenlichen Daten zu kennzeichnen
\ifttm
\newcommand{\MGlobalDataTag}{\special{html:<!-- mglobaldatatag -->}}
\else
\newcommand{\MGlobalDataTag}{}
\fi

% Makro, um in der HTML-Ausgabe die Suchseite zu kennzeichnen
\ifttm
\newcommand{\MGlobalSearchTag}{\special{html:<!-- mglobalsearchtag -->}}
\else
\newcommand{\MGlobalSearchTag}{}
\fi

% Makro, um in der HTML-Ausgabe die Favoritenseite zu kennzeichnen
\ifttm
\newcommand{\MGlobalFavoTag}{\special{html:<!-- mglobalfavoritestag -->}}
\else
\newcommand{\MGlobalFavoTag}{}
\fi

% Makro, um in der HTML-Ausgabe die Eingangstestseite zu kennzeichnen
\ifttm
\newcommand{\MGlobalSTestTag}{\special{html:<!-- mglobalstesttag -->}}
\else
\newcommand{\MGlobalSTestTag}{}
\fi

% Makro, um in der PDF-Ausgabe ein Wasserzeichen zu definieren
\ifttm
\newcommand{\MWatermarkSettings}{\relax}
\else
\newcommand{\MWatermarkSettings}{%
% \SetWatermarkText{(c) MINT-Kolleg Baden-W�rttemberg 2014}
% \SetWatermarkLightness{0.85}
% \SetWatermarkScale{1.5}
}
\fi

\ifttm
\newcommand{\MBinom}[2]{\left({\begin{array}{c} #1 \\ #2 \end{array}}\right)}
\else
\newcommand{\MBinom}[2]{\binom{#1}{#2}}
\fi

\ifttm
\newcommand{\DeclareMathOperator}[2]{\def#1{\mathrm{#2}}}
\newcommand{\operatorname}[1]{\mathrm{#1}}
\fi

%----------------- Makros fuer die gemischte HTML/PDF-Konvertierung ------------------------------

\newcommand{\MTestName}{\relax} % wird durch Test-Umgebung gesetzt

% Fuer experimentelle Kursinhalte, die im Release-Umsetzungsvorgang eine Fehlermeldung
% produzieren sollen aber sonst normal umgesetzt werden
\newenvironment{MExperimental}{%
}{%
}

% Wird von ttm nicht richtig umgesetzt!!
\newenvironment{MExerciseItems}{%
\renewcommand\theenumi{\alph{enumi}}%
\begin{enumerate}%
}{%
\end{enumerate}%
}


\definecolor{infoshadecolor}{rgb}{0.75,0.75,0.75}
\definecolor{exmpshadecolor}{rgb}{0.875,0.875,0.875}
\definecolor{expeshadecolor}{rgb}{0.95,0.95,0.95}
\definecolor{framecolor}{rgb}{0.2,0.2,0.2}

% Bei PDF-Uebersetzung wird hinter den Start jeder Satz/Info-aehnlichen Umgebung eine leere mbox gesetzt, damit
% fuehrende Listen oder enums nicht den Zeilenumbruch kaputtmachen
%\ifttm
\def\MTB{}
%\else
%\def\MTB{\mbox{}}
%\fi


\ifttm
\newcommand{\MRelates}{\special{html:<mi>&wedgeq;</mi>}}
\else
\def\MRelates{\stackrel{\scriptscriptstyle\wedge}{=}}
\fi

\def\MInch{\text{''}}
\def\Mdd{\textit{''}}

\ifttm
\def\MNL{ \newline }
\newenvironment{MArray}[1]{\begin{array}{#1}}{\end{array}}
\else
\def\MNL{ \\ }
\newenvironment{MArray}[1]{\begin{array}{#1}}{\end{array}}
\fi

\newcommand{\MBox}[1]{$\mathrm{#1}$}
\newcommand{\MMBox}[1]{\mathrm{#1}}


\ifttm%
\newcommand{\Mtfrac}[2]{{\textstyle \frac{#1}{#2}}}
\newcommand{\Mdfrac}[2]{{\displaystyle \frac{#1}{#2}}}
\newcommand{\Mmeasuredangle}{\special{html:<mi>&angmsd;</mi>}}
\else%
\newcommand{\Mtfrac}[2]{\tfrac{#1}{#2}}
\newcommand{\Mdfrac}[2]{\dfrac{#1}{#2}}
\newcommand{\Mmeasuredangle}{\measuredangle}
\relax
\fi

% Matrizen und Vektoren

% Inhalt wird in der Form a & b \\ c & d erwartet
% Vorsicht: MVector = Komponentenspalte, MVec = Variablensymbol
\ifttm%
\newcommand{\MVector}[1]{\left({\begin{array}{c}#1\end{array}}\right)}
\else%
\newcommand{\MVector}[1]{\begin{pmatrix}#1\end{pmatrix}}
\fi



\newcommand{\MVec}[1]{\vec{#1}}
\newcommand{\MDVec}[1]{\overrightarrow{#1}}

%----------------- Umgebungen fuer Definitionen und Saetze ----------------------------------------

% Fuegt einen Tabellen-Zeilenumbruch ein im PDF, aber nicht im HTML
\newcommand{\TSkip}{\ifttm \else&\ \\\fi}

\newenvironment{infoshaded}{%
\def\FrameCommand{\fboxsep=\FrameSep \fcolorbox{framecolor}{infoshadecolor}}%
\MakeFramed {\advance\hsize-\width \FrameRestore}}%
{\endMakeFramed}

\newenvironment{expeshaded}{%
\def\FrameCommand{\fboxsep=\FrameSep \fcolorbox{framecolor}{expeshadecolor}}%
\MakeFramed {\advance\hsize-\width \FrameRestore}}%
{\endMakeFramed}

\newenvironment{exmpshaded}{%
\def\FrameCommand{\fboxsep=\FrameSep \fcolorbox{framecolor}{exmpshadecolor}}%
\MakeFramed {\advance\hsize-\width \FrameRestore}}%
{\endMakeFramed}

\def\STDCOLOR{black}

\ifttm%
\else%
\newtheoremstyle{MSatzStyle}
  {1cm}                   %Space above
  {1cm}                   %Space below
  {\normalfont\itshape}   %Body font
  {}                      %Indent amount (empty = no indent,
                          %\parindent = para indent)
  {\normalfont\bfseries}  %Thm head font
  {}                      %Punctuation after thm head
  {\newline}              %Space after thm head: " " = normal interword
                          %space; \newline = linebreak
  {\thmname{#1}\thmnumber{ #2}\thmnote{ (#3)}}
                          %Thm head spec (can be left empty, meaning
                          %`normal')
                          %
\newtheoremstyle{MDefStyle}
  {1cm}                   %Space above
  {1cm}                   %Space below
  {\normalfont}           %Body font
  {}                      %Indent amount (empty = no indent,
                          %\parindent = para indent)
  {\normalfont\bfseries}  %Thm head font
  {}                      %Punctuation after thm head
  {\newline}              %Space after thm head: " " = normal interword
                          %space; \newline = linebreak
  {\thmname{#1}\thmnumber{ #2}\thmnote{ (#3)}}
                          %Thm head spec (can be left empty, meaning
                          %`normal')
\fi%

\newcommand{\MInfoText}{Info}

\newcounter{MHintCounter}
\newcounter{MCodeEditCounter}

\newcounter{MLastIndex}  % Enthaelt die dritte Stelle (Indexnummer) des letzten angelegten Objekts
\newcounter{MLastType}   % Enthaelt den Typ des letzten angelegten Objekts (mithilfe der unten definierten Konstanten). Die Entscheidung, wie der Typ dargstellt wird, wird in split.pm beim Postprocessing getroffen.
\newcounter{MLastTypeEq} % =1 falls das Label in einer Matheumgebung (equation, eqnarray usw.) steht, =2 falls das Label in einer table-Umgebung steht

% Da ttm keine Zahlmakros verarbeiten kann, werden diese Nummern in den Zuweisungen hardcodiert!
\def\MTypeSection{1}          %# Zaehler ist section
\def\MTypeSubsection{2}       %# Zaehler ist subsection
\def\MTypeSubsubsection{3}    %# Zaehler ist subsubsection
\def\MTypeInfo{4}             %# Eine Infobox, Separatzaehler fuer die Chemie (auch wenn es dort nicht nummeriert wird) ist MInfoCounter
\def\MTypeExercise{5}         %# Eine Aufgabe, Separatzaehler fuer die Chemie ist MExerciseCounter
\def\MTypeExample{6}          %# Eine Beispielbox, Separatzaehler fuer die Chemie ist MExampleCounter
\def\MTypeExperiment{7}       %# Eine Versuchsbox, Separatzaehler fuer die Chemie ist MExperimentCounter
\def\MTypeGraphics{8}         %# Eine Graphik, Separatzaehler fuer alle FB ist MGraphicsCounter
\def\MTypeTable{9}            %# Eine Tabellennummer, hat keinen Zaehler da durch table gezaehlt wird
\def\MTypeEquation{10}        %# Eine Gleichungsnummer, hat keinen Zaehler da durch equation/eqnarray gezaehlt wird
\def\MTypeTheorem{11}         % Ein theorem oder xtheorem, Separatzaehler fuer die Chemie ist MTheoremCounter
\def\MTypeVideo{12}           %# Ein Video,Separatzaehler fuer alle FB ist MVideoCounter
\def\MTypeEntry{13}           %# Ein Eintrag fuer die Stichwortliste, wird nicht gezaehlt sondern erhaelt im preparsing ein unique-label 

% Zaehler fuer das Labelsystem sind prefixcounter, jeder Zaehler wird VOR dem gezaehlten Objekt inkrementiert und zaehlt daher das aktuelle Objekt
\newcounter{MInfoCounter}
\newcounter{MExerciseCounter}
\newcounter{MExampleCounter}
\newcounter{MExperimentCounter}
\newcounter{MGraphicsCounter}
\newcounter{MTableCounter}
\newcounter{MEquationCounter}  % Nur im HTML, sonst durch "equation"-counter von latex realisiert
\newcounter{MTheoremCounter}
\newcounter{MObjectCounter}   % Gemeinsamer Zaehler fuer Objekte (ausser Grafiken/Tabellen) in Mathe/Info/Physik
\newcounter{MVideoCounter}
\newcounter{MEntryCounter}

\newcounter{MTestSite} % 1 = Subsubsection ist eine Pruefungsseite, 0 = ist eine normale Seite (inkl. Hilfeseite)

\def\MCell{$\phantom{a}$}

\newenvironment{MExportExercise}{\begin{MExercise}}{\end{MExercise}} % wird von mconvert abgefangen

\def\MGenerateExNumber{%
\ifnum\value{MSepNumbers}=0%
\arabic{section}.\arabic{subsection}.\arabic{MObjectCounter}\setcounter{MLastIndex}{\value{MObjectCounter}}%
\else%
\arabic{section}.\arabic{subsection}.\arabic{MExerciseCounter}\setcounter{MLastIndex}{\value{MExerciseCounter}}%
\fi%
}%

\def\MGenerateExmpNumber{%
\ifnum\value{MSepNumbers}=0%
\arabic{section}.\arabic{subsection}.\arabic{MObjectCounter}\setcounter{MLastIndex}{\value{MObjectCounter}}%
\else%
\arabic{section}.\arabic{subsection}.\arabic{MExerciseCounter}\setcounter{MLastIndex}{\value{MExampleCounter}}%
\fi%
}%

\def\MGenerateInfoNumber{%
\ifnum\value{MSepNumbers}=0%
\arabic{section}.\arabic{subsection}.\arabic{MObjectCounter}\setcounter{MLastIndex}{\value{MObjectCounter}}%
\else%
\arabic{section}.\arabic{subsection}.\arabic{MExerciseCounter}\setcounter{MLastIndex}{\value{MInfoCounter}}%
\fi%
}%

\def\MGenerateSiteNumber{%
\arabic{section}.\arabic{subsection}.\arabic{subsubsection}%
}%

% Funktionalitaet fuer Auswahlaufgaben

\newcounter{MExerciseCollectionCounter} % = 0 falls nicht in collection-Umgebung, ansonsten Schachtelungstiefe
\newcounter{MExerciseCollectionTextCounter} % wird von MExercise-Umgebung inkrementiert und von MExerciseCollection-Umgebung auf Null gesetzt

\ifttm
% MExerciseCollection gruppiert Aufgaben, die dynamisch aus der Datenbank gezogen werden und nicht direkt in der HTML-Seite stehen
% Parameter: #1 = ID der Collection, muss eindeutig fuer alle IN DER DB VORHANDENEN collections sein unabhaengig vom Kurs
%            #2 = Optionsargument (im Moment: 1 = Iterative Auswahl, 2 = Zufallsbasierte Auswahl)
\newenvironment{MExerciseCollection}[2]{%
\addtocounter{MExerciseCollectionCounter}{1}
\setcounter{MExerciseCollectionTextCounter}{0}
\special{html:<!-- mexercisecollectionstart;;}#1\special{html:;;}#2\special{html:;; //-->}%
}{%
\special{html:<!-- mexercisecollectionstop //-->}%
\addtocounter{MExerciseCollectionCounter}{-1}
}
\else
\newenvironment{MExerciseCollection}[2]{%
\addtocounter{MExerciseCollectionCounter}{1}
\setcounter{MExerciseCollectionTextCounter}{0}
}{%
\addtocounter{MExerciseCollectionCounter}{-1}
}
\fi

% Bei Uebersetzung nach PDF werden die theorem-Umgebungen verwendet, bei Uebersetzung in HTML ein manuelles Makro
\ifttm%

  \newenvironment{MHint}[1]{  \special{html:<button name="Name_MHint}\arabic{MHintCounter}\special{html:" class="hintbutton_closed" id="MHint}\arabic{MHintCounter}\special{html:_button" %
  type="button" onclick="toggle_hint('MHint}\arabic{MHintCounter}\special{html:');">}#1\special{html:</button>}
  \special{html:<div class="hint" style="display:none" id="MHint}\arabic{MHintCounter}\special{html:"> }}{\begin{html}</div>\end{html}\addtocounter{MHintCounter}{1}}

  \newenvironment{MCOSHZusatz}{  \special{html:<button name="Name_MHint}\arabic{MHintCounter}\special{html:" class="chintbutton_closed" id="MHint}\arabic{MHintCounter}\special{html:_button" %
  type="button" onclick="toggle_hint('MHint}\arabic{MHintCounter}\special{html:');">}Weiterf�hrende Inhalte\special{html:</button>}
  \special{html:<div class="hintc" style="display:none" id="MHint}\arabic{MHintCounter}\special{html:">
  <div class="coshwarn">Diese Inhalte gehen �ber das Kursniveau hinaus und werden in den Aufgaben und Tests nicht abgefragt.</div><br />}
  \addtocounter{MHintCounter}{1}}{\begin{html}</div>\end{html}}

  
  \newenvironment{MDefinition}{\begin{definition}\setcounter{MLastIndex}{\value{definition}}\ \\}{\end{definition}}

  
  \newenvironment{MExercise}{
  \renewcommand{\MStdPoints}{4}
  \addtocounter{MExerciseCounter}{1}
  \addtocounter{MObjectCounter}{1}
  \setcounter{MLastType}{5}

  \ifnum\value{MExerciseCollectionCounter}=0\else\addtocounter{MExerciseCollectionTextCounter}{1}\special{html:<!-- mexercisetextstart;;}\arabic{MExerciseCollectionTextCounter}\special{html:;; //-->}\fi
  \special{html:<div class="aufgabe" id="ADIV_}\MGenerateExNumber\special{html:">}%
  \textbf{Aufgabe \MGenerateExNumber
  } \ \\}{
  \special{html:</div><!-- mfeedbackbutton;Aufgabe;}\arabic{MTestSite}\special{html:;}\MGenerateExNumber\special{html:; //-->}
  \ifnum\value{MExerciseCollectionCounter}=0\else\special{html:<!-- mexercisetextstop //-->}\fi
  }

  % Stellt eine Kombination aus Aufgabe, Loesungstext und Eingabefeld bereit,
  % bei der Aufgabentext und Musterloesung sowie die zugehoerigen Feldelemente
  % extern bezogen und div-aktualisiert werden, das Eingabefeld aber immer das gleiche ist.
  \newenvironment{MFetchExercise}{
  \addtocounter{MExerciseCounter}{1}
  \addtocounter{MObjectCounter}{1}
  \setcounter{MLastType}{5}

  \special{html:<div class="aufgabe" id="ADIV_}\MGenerateExNumber\special{html:">}%
  \textbf{Aufgabe \MGenerateExNumber
  } \ \\%
  \special{html:</div><div class="exfetch_text" id="ADIVTEXT_}\MGenerateExNumber\special{html:">}%
  \special{html:</div><div class="exfetch_sol" id="ADIVSOL_}\MGenerateExNumber\special{html:">}%
  \special{html:</div><div class="exfetch_input" id="ADIVINPUT_}\MGenerateExNumber\special{html:">}%
  }{
  \special{html:</div>}
  }

  \newenvironment{MExample}{
  \addtocounter{MExampleCounter}{1}
  \addtocounter{MObjectCounter}{1}
  \setcounter{MLastType}{6}
  \begin{html}
  <div class="exmp">
  <div class="exmprahmen">
  \end{html}\textbf{Beispiel
  \ifnum\value{MSepNumbers}=0
  \arabic{section}.\arabic{subsection}.\arabic{MObjectCounter}\setcounter{MLastIndex}{\value{MObjectCounter}}
  \else
  \arabic{section}.\arabic{subsection}.\arabic{MExampleCounter}\setcounter{MLastIndex}{\value{MExampleCounter}}
  \fi
  } \ \\}{\begin{html}</div>
  </div>
  \end{html}
  \special{html:<!-- mfeedbackbutton;Beispiel;}\arabic{MTestSite}\special{html:;}\MGenerateExmpNumber\special{html:; //-->}
  }

  \newenvironment{MExperiment}{
  \addtocounter{MExperimentCounter}{1}
  \addtocounter{MObjectCounter}{1}
  \setcounter{MLastType}{7}
  \begin{html}
  <div class="expe">
  <div class="experahmen">
  \end{html}\textbf{Versuch
  \ifnum\value{MSepNumbers}=0
  \arabic{section}.\arabic{subsection}.\arabic{MObjectCounter}\setcounter{MLastIndex}{\value{MObjectCounter}}
  \else
%  \arabic{MExperimentCounter}\setcounter{MLastIndex}{\value{MExperimentCounter}}
  \arabic{section}.\arabic{subsection}.\arabic{MExperimentCounter}\setcounter{MLastIndex}{\value{MExperimentCounter}}
  \fi
  } \ \\}{\begin{html}</div>
  </div>
  \end{html}}

  \newenvironment{MChemInfo}{
  \setcounter{MLastType}{4}
  \begin{html}
  <div class="info">
  <div class="inforahmen">
  \end{html}}{\begin{html}</div>
  </div>
  \end{html}}

  \newenvironment{MXInfo}[1]{
  \addtocounter{MInfoCounter}{1}
  \addtocounter{MObjectCounter}{1}
  \setcounter{MLastType}{4}
  \begin{html}
  <div class="info">
  <div class="inforahmen">
  \end{html}\textbf{#1
  \ifnum\value{MInfoNumbers}=0
  \else
    \ifnum\value{MSepNumbers}=0
    \arabic{section}.\arabic{subsection}.\arabic{MObjectCounter}\setcounter{MLastIndex}{\value{MObjectCounter}}
    \else
    \arabic{MInfoCounter}\setcounter{MLastIndex}{\value{MInfoCounter}}
    \fi
  \fi
  } \ \\}{\begin{html}</div>
  </div>
  \end{html}
  \special{html:<!-- mfeedbackbutton;Info;}\arabic{MTestSite}\special{html:;}\MGenerateInfoNumber\special{html:; //-->}
  }

  \newenvironment{MInfo}{\ifnum\value{MInfoNumbers}=0\begin{MChemInfo}\else\begin{MXInfo}{Info}\ \\ \fi}{\ifnum\value{MInfoNumbers}=0\end{MChemInfo}\else\end{MXInfo}\fi}

\else%

  \theoremstyle{MSatzStyle}
  \newtheorem{thm}{Satz}[section]
  \newtheorem{thmc}{Satz}
  \theoremstyle{MDefStyle}
  \newtheorem{defn}[thm]{Definition}
  \newtheorem{exmp}[thm]{Beispiel}
  \newtheorem{info}[thm]{\MInfoText}
  \theoremstyle{MDefStyle}
  \newtheorem{defnc}{Definition}
  \theoremstyle{MDefStyle}
  \newtheorem{exmpc}{Beispiel}[section]
  \theoremstyle{MDefStyle}
  \newtheorem{infoc}{\MInfoText}
  \theoremstyle{MDefStyle}
  \newtheorem{exrc}{Aufgabe}[section]
  \theoremstyle{MDefStyle}
  \newtheorem{verc}{Versuch}[section]
  
  \newenvironment{MFetchExercise}{}{} % kann im PDF nicht dargestellt werden
  
  \newenvironment{MExercise}{\begin{exrc}\renewcommand{\MStdPoints}{1}\MTB}{\end{exrc}}
  \newenvironment{MHint}[1]{\ \\ \underline{#1:}\\}{}
  \newenvironment{MCOSHZusatz}{\ \\ \underline{Weiterf�hrende Inhalte:}\\}{}
  \newenvironment{MDefinition}{\ifnum\value{MInfoNumbers}=0\begin{defnc}\else\begin{defn}\fi\MTB}{\ifnum\value{MInfoNumbers}=0\end{defnc}\else\end{defn}\fi}
%  \newenvironment{MExample}{\begin{exmp}}{\ \linebreak[1] \ \ \ \ $\phantom{a}$ \ \hfill $\blacklozenge$\end{exmp}}
  \newenvironment{MExample}{
    \ifnum\value{MInfoNumbers}=0\begin{exmpc}\else\begin{exmp}\fi
    \MTB
    \begin{exmpshaded}
    \ \newline
}{
    \end{exmpshaded}
    \ifnum\value{MInfoNumbers}=0\end{exmpc}\else\end{exmp}\fi
}
  \newenvironment{MChemInfo}{\begin{infoshaded}}{\end{infoshaded}}

  \newenvironment{MInfo}{\ifnum\value{MInfoNumbers}=0\begin{MChemInfo}\else\renewcommand{\MInfoText}{Info}\begin{info}\begin{infoshaded}
  \MTB
   \ \newline
    \fi
  }{\ifnum\value{MInfoNumbers}=0\end{MChemInfo}\else\end{infoshaded}\end{info}\fi}

  \newenvironment{MXInfo}[1]{
    \renewcommand{\MInfoText}{#1}
    \ifnum\value{MInfoNumbers}=0\begin{infoc}\else\begin{info}\fi%
    \MTB
    \begin{infoshaded}
    \ \newline
  }{\end{infoshaded}\ifnum\value{MInfoNumbers}=0\end{infoc}\else\end{info}\fi}

  \newenvironment{MExperiment}{
    \renewcommand{\MInfoText}{Versuch}
    \ifnum\value{MInfoNumbers}=0\begin{verc}\else\begin{info}\fi
    \MTB
    \begin{expeshaded}
    \ \newline
  }{
    \end{expeshaded}
    \ifnum\value{MInfoNumbers}=0\end{verc}\else\end{info}\fi
  }
\fi%

% MHint sollte nicht direkt fuer Loesungen benutzt werden wegen solutionselect
\newenvironment{MSolution}{\begin{MHint}{L"osung}}{\end{MHint}}

\newcounter{MCodeCounter}

\ifttm
\newenvironment{MCode}{\special{html:<!-- mcodestart -->}\ttfamily\color{blue}}{\special{html:<!-- mcodestop -->}}
\else
\newenvironment{MCode}{\begin{flushleft}\ttfamily\addtocounter{MCodeCounter}{1}}{\addtocounter{MCodeCounter}{-1}\end{flushleft}}
% Ohne color-Statement da inkompatible mit framed/shaded-Boxen aus dem framed-package
\fi

%----------------- Sonderdefinitionen fuer Symbole, die der Konverter nicht kann ----------------------------------------------

\ifttm%
\newcommand{\MUnderset}[2]{\underbrace{#2}_{#1}}%
\else%
\newcommand{\MUnderset}[2]{\underset{#1}{#2}}%
\fi%

\ifttm
\newcommand{\MThinspace}{\special{html:<mi>&#x2009;</mi>}}
\else
\newcommand{\MThinspace}{\,}
\fi

\ifttm
\newcommand{\glq}{\begin{html}&sbquo;\end{html}}
\newcommand{\grq}{\begin{html}&lsquo;\end{html}}
\newcommand{\glqq}{\begin{html}&bdquo;\end{html}}
\newcommand{\grqq}{\begin{html}&ldquo;\end{html}}
\fi

\ifttm
\newcommand{\MNdash}{\begin{html}&ndash;\end{html}}
\else
\newcommand{\MNdash}{--}
\fi

%\ifttm\def\MIU{\special{html:<mi>&#8520;</mi>}}\else\def\MIU{\mathrm{i}}\fi
\def\MIU{\mathrm{i}}
\def\MEU{e} % TU9-Onlinekurs: italic-e
%\def\MEU{\mathrm{e}} % Alte Onlinemodule: roman-e
\def\MD{d} % Kursives d in Integralen im TU9-Onlinekurs
%\def\MD{\mathrm{d}} % roman-d in den alten Onlinemodulen
\def\MDB{\|}

%zusaetzlicher Leerraum vor "\MD"
\ifttm%
\def\MDSpace{\special{html:<mi>&#x2009;</mi>}}
\else%
\def\MDSpace{\,}
\fi%
\newcommand{\MDwSp}{\MDSpace\MD}%

\ifttm
\def\Mdq{\dq}
\else
\def\Mdq{\dq}
\fi

\def\MSpan#1{\left<{#1}\right>}
\def\MSetminus{\setminus}
\def\MIM{I}

\ifttm
\newcommand{\ld}{\text{ld}}
\newcommand{\lg}{\text{lg}}
\else
\DeclareMathOperator{\ld}{ld}
%\newcommand{\lg}{\text{lg}} % in latex schon definiert
\fi


\def\Mmapsto{\ifttm\special{html:<mi>&mapsto;</mi>}\else\mapsto\fi} 
\def\Mvarphi{\ifttm\phi\else\varphi\fi}
\def\Mphi{\ifttm\varphi\else\phi\fi}
\ifttm%
\newcommand{\MEumu}{\special{html:<mi>&#x3BC;</mi>}}%
\else%
\newcommand{\MEumu}{\textrm{\textmu}}%
\fi
\def\Mvarepsilon{\ifttm\epsilon\else\varepsilon\fi}
\def\Mepsilon{\ifttm\varepsilon\else\epsilon\fi}
\def\Mvarkappa{\ifttm\kappa\else\varkappa\fi}
\def\Mkappa{\ifttm\varkappa\else\kappa\fi}
\def\Mcomplement{\ifttm\special{html:<mi>&comp;</mi>}\else\complement\fi} 
\def\MWW{\mathrm{WW}}
\def\Mmod{\ifttm\special{html:<mi>&nbsp;mod&nbsp;</mi>}\else\mod\fi} 

\ifttm%
\def\mod{\text{\;mod\;}}%
\def\MNEquiv{\special{html:<mi>&NotCongruent;</mi>}}% 
\def\MNSubseteq{\special{html:<mi>&NotSubsetEqual;</mi>}}%
\def\MEmptyset{\special{html:<mi>&empty;</mi>}}%
\def\MVDots{\special{html:<mi>&#x22EE;</mi>}}%
\def\MHDots{\special{html:<mi>&#x2026;</mi>}}%
\def\Mddag{\special{html:<mi>&#x1202;</mi>}}%
\def\sphericalangle{\special{html:<mi>&measuredangle;</mi>}}%
\def\nparallel{\special{html:<mi>&nparallel;</mi>}}%
\def\MProofEnd{\special{html:<mi>&#x25FB;</mi>}}%
\newenvironment{MProof}[1]{\underline{#1}:\MCR\MCR}{\hfill $\MProofEnd$}%
\else%
\def\MNEquiv{\not\equiv}%
\def\MNSubseteq{\not\subseteq}%
\def\MEmptyset{\emptyset}%
\def\MVDots{\vdots}%
\def\MHDots{\hdots}%
\def\Mddag{\ddag}%
\newenvironment{MProof}[1]{\begin{proof}[#1]}{\end{proof}}%
\fi%



% Spaces zum Auffuellen von Tabellenbreiten, die nur im HTML wirken
\ifttm%
\def\MTSP{\:}%
\else%
\def\MTSP{}%
\fi%

\DeclareMathOperator{\arsinh}{arsinh}
\DeclareMathOperator{\arcosh}{arcosh}
\DeclareMathOperator{\artanh}{artanh}
\DeclareMathOperator{\arcoth}{arcoth}


\newcommand{\MMathSet}[1]{\mathbb{#1}}
\def\N{\MMathSet{N}}
\def\Z{\MMathSet{Z}}
\def\Q{\MMathSet{Q}}
\def\R{\MMathSet{R}}
\def\C{\MMathSet{C}}

\newcounter{MForLoopCounter}
\newcommand{\MForLoop}[2]{\setcounter{MForLoopCounter}{#1}\ifnum\value{MForLoopCounter}=0{}\else{{#2}\addtocounter{MForLoopCounter}{-1}\MForLoop{\value{MForLoopCounter}}{#2}}\fi}

\newcounter{MSiteCounter}
\newcounter{MFieldCounter} % Kombination section.subsection.site.field ist eindeutig in allen Modulen, field alleine nicht

\newcounter{MiniMarkerCounter}

\ifttm
\newenvironment{MMiniPageP}[1]{\begin{minipage}{#1\linewidth}\special{html:<!-- minimarker;;}\arabic{MiniMarkerCounter}\special{html:;;#1; //-->}}{\end{minipage}\addtocounter{MiniMarkerCounter}{1}}
\else
\newenvironment{MMiniPageP}[1]{\begin{minipage}{#1\linewidth}}{\end{minipage}\addtocounter{MiniMarkerCounter}{1}}
\fi

\newcounter{AlignCounter}

\newcommand{\MStartJustify}{\ifttm\special{html:<!-- startalign;;}\arabic{AlignCounter}\special{html:;;justify; //-->}\fi}
\newcommand{\MStopJustify}{\ifttm\special{html:<!-- stopalign;;}\arabic{AlignCounter}\special{html:; //-->}\fi\addtocounter{AlignCounter}{1}}

\newenvironment{MJTabular}[1]{
\MStartJustify
\begin{tabular}{#1}
}{
\end{tabular}
\MStopJustify
}

\newcommand{\MImageLeft}[2]{
\begin{center}
\begin{tabular}{lc}
\MStartJustify
\begin{MMiniPageP}{0.65}
#1
\end{MMiniPageP}
\MStopJustify
&
\begin{MMiniPageP}{0.3}
#2  
\end{MMiniPageP}
\end{tabular}
\end{center}
}

\newcommand{\MImageHalf}[2]{
\begin{center}
\begin{tabular}{lc}
\MStartJustify
\begin{MMiniPageP}{0.45}
#1
\end{MMiniPageP}
\MStopJustify
&
\begin{MMiniPageP}{0.45}
#2  
\end{MMiniPageP}
\end{tabular}
\end{center}
}

\newcommand{\MBigImageLeft}[2]{
\begin{center}
\begin{tabular}{lc}
\MStartJustify
\begin{MMiniPageP}{0.25}
#1
\end{MMiniPageP}
\MStopJustify
&
\begin{MMiniPageP}{0.7}
#2  
\end{MMiniPageP}
\end{tabular}
\end{center}
}

\ifttm
\def\No{\mathbb{N}_0}
\else
\def\No{\ensuremath{\N_0}}
\fi
\def\MT{\textrm{\tiny T}}
\newcommand{\MTranspose}[1]{{#1}^{\MT}}
\ifttm
\newcommand{\MRe}{\mathsf{Re}}
\newcommand{\MIm}{\mathsf{Im}}
\else
\DeclareMathOperator{\MRe}{Re}
\DeclareMathOperator{\MIm}{Im}
\fi

\newcommand{\Mid}{\mathrm{id}}
\newcommand{\MFeinheit}{\mathrm{feinh}}

\ifttm
\newcommand{\Msubstack}[1]{\begin{array}{c}{#1}\end{array}}
\else
\newcommand{\Msubstack}[1]{\substack{#1}}
\fi

% Typen von Fragefeldern:
% 1 = Alphanumerisch, case-sensitive-Vergleich
% 2 = Ja/Nein-Checkbox, Loesung ist 0 oder 1   (OPTION = Image-id fuer Rueckmeldung)
% 3 = Reelle Zahlen Geparset
% 4 = Funktionen Geparset (mit Stuetzstellen zur ueberpruefung)

% Dieser Befehl erstellt ein interaktives Aufgabenfeld. Parameter:
% - #1 Laenge in Zeichen
% - #2 Loesungstext (alphanumerisch, case sensitive)
% - #3 AufgabenID (alphanumerisch, case sensitive)
% - #4 Typ (Kennnummer)
% - #5 String fuer Optionen (ggf. mit Semikolon getrennte Einzelstrings)
% - #6 Anzahl Punkte
% - #7 uxid (kann z.B. Loesungsstring sein)
% ACHTUNG: Die langen Zeilen bitte so lassen, Zeilenumbrueche im tex werden in div's umgesetzt
\newcommand{\MQuestionID}[7]{
\ifttm
\special{html:<!-- mdeclareuxid;;}UX#7\special{html:;;}\arabic{section}\special{html:;;}#3\special{html:;; //-->}%
\special{html:<!-- mdeclarepoints;;}\arabic{section}\special{html:;;}#3\special{html:;;}#6\special{html:;;}\arabic{MTestSite}\special{html:;;}\arabic{chapter}%
\special{html:;; //--><!-- onloadstart //-->CreateQuestionObj("}#7\special{html:",}\arabic{MFieldCounter}\special{html:,"}#2%
\special{html:","}#3\special{html:",}#4\special{html:,"}#5\special{html:",}#6\special{html:,}\arabic{MTestSite}\special{html:,}\arabic{section}%
\special{html:);<!-- onloadstop //-->}%
\special{html:<input mfieldtype="}#4\special{html:" name="Name_}#3\special{html:" id="}#3\special{html:" type="text" size="}#1\special{html:" maxlength="}#1%
\special{html:" }\ifnum\value{MGroupActive}=0\special{html:onfocus="handlerFocus(}\arabic{MFieldCounter}%
\special{html:);" onblur="handlerBlur(}\arabic{MFieldCounter}\special{html:);" onkeyup="handlerChange(}\arabic{MFieldCounter}\special{html:,0);" onpaste="handlerChange(}\arabic{MFieldCounter}\special{html:,0);" oninput="handlerChange(}\arabic{MFieldCounter}\special{html:,0);" onpropertychange="handlerChange(}\arabic{MFieldCounter}\special{html:,0);"/>}%
\special{html:<img src="images/questionmark.gif" width="20" height="20" border="0" align="absmiddle" id="}QM#3\special{html:"/>}
\else%
\special{html:onblur="handlerBlur(}\arabic{MFieldCounter}%
\special{html:);" onfocus="handlerFocus(}\arabic{MFieldCounter}\special{html:);" onkeyup="handlerChange(}\arabic{MFieldCounter}\special{html:,1);" onpaste="handlerChange(}\arabic{MFieldCounter}\special{html:,1);" oninput="handlerChange(}\arabic{MFieldCounter}\special{html:,1);" onpropertychange="handlerChange(}\arabic{MFieldCounter}\special{html:,1);"/>}%
\special{html:<img src="images/questionmark.gif" width="20" height="20" border="0" align="absmiddle" id="}QM#3\special{html:"/>}\fi%
\else%
\ifnum\value{QBoxFlag}=1\fbox{$\phantom{\MForLoop{#1}{b}}$}\else$\phantom{\MForLoop{#1}{b}}$\fi%
\fi%
}

% ACHTUNG: Die langen Zeilen bitte so lassen, Zeilenumbrueche im tex werden in div's umgesetzt
% QuestionCheckbox macht ausserhalb einer QuestionGroup keinen Sinn!
% #1 = solution (1 oder 0), ggf. mit ::smc abgetrennt auszuschliessende single-choice-boxen (UXIDs durch , getrennt), #2 = id, #3 = points, #4 = uxid
\newcommand{\MQuestionCheckbox}[4]{
\ifttm
\special{html:<!-- mdeclareuxid;;}UX#4\special{html:;;}\arabic{section}\special{html:;;}#2\special{html:;; //-->}%
\ifnum\value{MGroupActive}=0\MDebugMessage{ERROR: Checkbox Nr. \arabic{MFieldCounter}\ ist nicht in einer Kontrollgruppe, es wird niemals eine Loesung angezeigt!}\fi
\special{html: %
<!-- mdeclarepoints;;}\arabic{section}\special{html:;;}#2\special{html:;;}#3\special{html:;;}\arabic{MTestSite}\special{html:;;}\arabic{chapter}%
\special{html:;; //--><!-- onloadstart //-->CreateQuestionObj("}#4\special{html:",}\arabic{MFieldCounter}\special{html:,"}#1\special{html:","}#2\special{html:",2,"IMG}#2%
\special{html:",}#3\special{html:,}\arabic{MTestSite}\special{html:,}\arabic{section}\special{html:);<!-- onloadstop //-->}%
\special{html:<input mfieldtype="2" type="checkbox" name="Name_}#2\special{html:" id="}#2\special{html:" onchange="handlerChange(}\arabic{MFieldCounter}\special{html:,1);"/><img src="images/questionmark.gif" name="}Name_IMG#2%
\special{html:" width="20" height="20" border="0" align="absmiddle" id="}IMG#2\special{html:"/> }%
\else%
\ifnum\value{QBoxFlag}=1\fbox{$\phantom{X}$}\else$\phantom{X}$\fi%
\fi%
}

\def\MGenerateID{QFELD_\arabic{section}.\arabic{subsection}.\arabic{MSiteCounter}.QF\arabic{MFieldCounter}}

% #1 = 0/1 ggf. mit ::smc abgetrennt auszuschliessende single-choice-boxen (UXIDs durch , getrennt ohne UX), #2 = uxid ohne UX
\newcommand{\MCheckbox}[2]{
\MQuestionCheckbox{#1}{\MGenerateID}{\MStdPoints}{#2}
\addtocounter{MFieldCounter}{1}
}

% Erster Parameter: Zeichenlaenge der Eingabebox, zweiter Parameter: Loesungstext
\newcommand{\MQuestion}[2]{
\MQuestionID{#1}{#2}{\MGenerateID}{1}{0}{\MStdPoints}{#2}
\addtocounter{MFieldCounter}{1}
}

% Erster Parameter: Zeichenlaenge der Eingabebox, zweiter Parameter: Loesungstext
\newcommand{\MLQuestion}[3]{
\MQuestionID{#1}{#2}{\MGenerateID}{1}{0}{\MStdPoints}{#3}
\addtocounter{MFieldCounter}{1}
}

% Parameter: Laenge des Feldes, Loesung (wird auch geparsed), Stellen Genauigkeit hinter dem Komma, weitere Stellen werden mathematisch gerundet vor Vergleich
\newcommand{\MParsedQuestion}[3]{
\MQuestionID{#1}{#2}{\MGenerateID}{3}{#3}{\MStdPoints}{#2}
\addtocounter{MFieldCounter}{1}
}

% Parameter: Laenge des Feldes, Loesung (wird auch geparsed), Stellen Genauigkeit hinter dem Komma, weitere Stellen werden mathematisch gerundet vor Vergleich
\newcommand{\MLParsedQuestion}[4]{
\MQuestionID{#1}{#2}{\MGenerateID}{3}{#3}{\MStdPoints}{#4}
\addtocounter{MFieldCounter}{1}
}

% Parameter: Laenge des Feldes, Loesungsfunktion, Anzahl Stuetzstellen, Funktionsvariablen durch Kommata getrennt (nicht case-sensitive), Anzahl Nachkommastellen im Vergleich
\newcommand{\MFunctionQuestion}[5]{
\MQuestionID{#1}{#2}{\MGenerateID}{4}{#3;#4;#5;0}{\MStdPoints}{#2}
\addtocounter{MFieldCounter}{1}
}

% Parameter: Laenge des Feldes, Loesungsfunktion, Anzahl Stuetzstellen, Funktionsvariablen durch Kommata getrennt (nicht case-sensitive), Anzahl Nachkommastellen im Vergleich, UXID
\newcommand{\MLFunctionQuestion}[6]{
\MQuestionID{#1}{#2}{\MGenerateID}{4}{#3;#4;#5;0}{\MStdPoints}{#6}
\addtocounter{MFieldCounter}{1}
}

% Parameter: Laenge des Feldes, Loesungsintervall, Genauigkeit der Zahlenwertpruefung
\newcommand{\MIntervalQuestion}[3]{
\MQuestionID{#1}{#2}{\MGenerateID}{6}{#3}{\MStdPoints}{#2}
\addtocounter{MFieldCounter}{1}
}

% Parameter: Laenge des Feldes, Loesungsintervall, Genauigkeit der Zahlenwertpruefung, UXID
\newcommand{\MLIntervalQuestion}[4]{
\MQuestionID{#1}{#2}{\MGenerateID}{6}{#3}{\MStdPoints}{#4}
\addtocounter{MFieldCounter}{1}
}

% Parameter: Laenge des Feldes, Loesungsfunktion, Anzahl Stuetzstellen, Funktionsvariable (nicht case-sensitive), Anzahl Nachkommastellen im Vergleich, Vereinfachungsbedingung
% Vereinfachungsbedingung ist eine der Folgenden:
% 0 = Keine Vereinfachungsbedingung
% 1 = Keine Klammern (runde oder eckige) mehr im vereinfachten Ausdruck
% 2 = Faktordarstellung (Term hat Produkte als letzte Operation, Summen als vorgeschaltete Operation)
% 3 = Summendarstellung (Term hat Summen als letzte Operation, Produkte als vorgeschaltete Operation)
% Flag 512: Besondere Stuetzstellen (nur >1 und nur schwach rational), sonst symmetrisch um Nullpunkt und ganze Zahlen inkl. Null werden getroffen
\newcommand{\MSimplifyQuestion}[6]{
\MQuestionID{#1}{#2}{\MGenerateID}{4}{#3;#4;#5;#6}{\MStdPoints}{#2}
\addtocounter{MFieldCounter}{1}
}

\newcommand{\MLSimplifyQuestion}[7]{
\MQuestionID{#1}{#2}{\MGenerateID}{4}{#3;#4;#5;#6}{\MStdPoints}{#7}
\addtocounter{MFieldCounter}{1}
}

% Parameter: Laenge des Feldes, Loesung (optionaler Ausdruck), Anzahl Stuetzstellen, Funktionsvariable (nicht case-sensitive), Anzahl Nachkommastellen im Vergleich, Spezialtyp (string-id)
\newcommand{\MLSpecialQuestion}[7]{
\MQuestionID{#1}{#2}{\MGenerateID}{7}{#3;#4;#5;#6}{\MStdPoints}{#7}
\addtocounter{MFieldCounter}{1}
}

\newcounter{MGroupStart}
\newcounter{MGroupEnd}
\newcounter{MGroupActive}

\newenvironment{MQuestionGroup}{
\setcounter{MGroupStart}{\value{MFieldCounter}}
\setcounter{MGroupActive}{1}
}{
\setcounter{MGroupActive}{0}
\setcounter{MGroupEnd}{\value{MFieldCounter}}
\addtocounter{MGroupEnd}{-1}
}

\newcommand{\MGroupButton}[1]{
\ifttm
\special{html:<button name="Name_Group}\arabic{MGroupStart}\special{html:to}\arabic{MGroupEnd}\special{html:" id="Group}\arabic{MGroupStart}\special{html:to}\arabic{MGroupEnd}\special{html:" %
type="button" onclick="group_button(}\arabic{MGroupStart}\special{html:,}\arabic{MGroupEnd}\special{html:);">}#1\special{html:</button>}
\else
\phantom{#1}
\fi
}

%----------------- Makros fuer die modularisierte Darstellung ------------------------------------

\def\MyText#1{#1}

% is used internally by the conversion package, should not be used by original tex documents
\def\MOrgLabel#1{\relax}

\ifttm

% Ein MLabel wird im html codiert durch das tag <!-- mmlabel;;Labelbezeichner;;SubjectArea;;chapter;;section;;subsection;;Index;;Objekttyp; //-->
\def\MLabel#1{%
\ifnum\value{MLastType}=8%
\ifnum\value{MCaptionOn}=0%
\MDebugMessage{ERROR: Grafik \arabic{MGraphicsCounter} hat separates label: #1 (Grafiklabels sollten nur in der Caption stehen)}%
\fi
\fi
\ifnum\value{MLastType}=12%
\ifnum\value{MCaptionOn}=0%
\MDebugMessage{ERROR: Video \arabic{MVideoCounter} hat separates label: #1 (Videolabels sollten nur in der Caption stehen}%
\fi
\fi
\ifnum\value{MLastType}=10\setcounter{MLastIndex}{\value{equation}}\fi
\label{#1}\begin{html}<!-- mmlabel;;#1;;\end{html}\arabic{MSubjectArea}\special{html:;;}\arabic{chapter}\special{html:;;}\arabic{section}\special{html:;;}\arabic{subsection}\special{html:;;}\arabic{MLastIndex}\special{html:;;}\arabic{MLastType}\special{html:; //-->}}%

\else

% Sonderbehandlung im PDF fuer Abbildungen in separater aux-Datei, da MGraphics die figure-Umgebung nicht verwendet
\def\MLabel#1{%
\ifnum\value{MLastType}=8%
\ifnum\value{MCaptionOn}=0%
\MDebugMessage{ERROR: Grafik \arabic{MGraphicsCounter} hat separates label: #1 (Grafiklabels sollten nur in der Caption stehen}%
\fi
\fi
\ifnum\value{MLastType}=12%
\ifnum\value{MCaptionOn}=0%
\MDebugMessage{ERROR: Video \arabic{MVideoCounter} hat separates label: #1 (Videolabels sollten nur in der Caption stehen}%
\fi
\fi
\label{#1}%
}%

\fi

% Gibt Begriff des referenzierten Objekts mit aus, aber nur im HTML, daher nur in Ausnahmefaellen (z.B. Copyrightliste) sinnvoll
\def\MCRef#1{\ifttm\special{html:<!-- mmref;;}#1\special{html:;;1; //-->}\else\vref{#1}\fi}


\def\MRef#1{\ifttm\special{html:<!-- mmref;;}#1\special{html:;;0; //-->}\else\vref{#1}\fi}
\def\MERef#1{\ifttm\special{html:<!-- mmref;;}#1\special{html:;;0; //-->}\else\eqref{#1}\fi}
\def\MNRef#1{\ifttm\special{html:<!-- mmref;;}#1\special{html:;;0; //-->}\else\ref{#1}\fi}
\def\MSRef#1#2{\ifttm\special{html:<!-- msref;;}#1\special{html:;;}#2\special{html:; //-->}\else \if#2\empty \ref{#1} \else \hyperref[#1]{#2}\fi\fi} 

\def\MRefRange#1#2{\ifttm\MRef{#1} bis 
\MRef{#2}\else\vrefrange[\unskip]{#1}{#2}\fi}

\def\MRefTwo#1#2{\ifttm\MRef{#1} und \MRef{#2}\else%
\let\vRefTLRsav=\reftextlabelrange\let\vRefTPRsav=\reftextpagerange%
\def\reftextlabelrange##1##2{\ref{##1} und~\ref{##2}}%
\def\reftextpagerange##1##2{auf den Seiten~\pageref{#1} und~\pageref{#2}}%
\vrefrange[\unskip]{#1}{#2}%
\let\reftextlabelrange=\vRefTLRsav\let\reftextpagerange=\vRefTPRsav\fi}

% MSectionChapter definiert falls notwendig das Kapitel vor der section. Das ist notwendig, wenn nur ein Einzelmodul uebersetzt wird.
% MChaptersGiven ist ein Counter, der von mconvert.pl vordefiniert wird.
\ifttm
\newcommand{\MSectionChapter}{\ifnum\value{MChaptersGiven}=0{\Dchapter{Modul}}\else{}\fi}
\else
\newcommand{\MSectionChapter}{\ifnum\value{chapter}=0{\Dchapter{Modul}}\else{}\fi}
\fi


\def\MChapter#1{\ifnum\value{MSSEnd}>0{\MSubsectionEndMacros}\addtocounter{MSSEnd}{-1}\fi\Dchapter{#1}}
\def\MSubject#1{\MChapter{#1}} % Schluesselwort HELPSECTION ist reserviert fuer Hilfesektion

\newcommand{\MSectionID}{UNKNOWNID}

\ifttm
\newcommand{\MSetSectionID}[1]{\renewcommand{\MSectionID}{#1}}
\else
\newcommand{\MSetSectionID}[1]{\renewcommand{\MSectionID}{#1}\tikzsetexternalprefix{#1}}
\fi


\newcommand{\MSection}[1]{\MSetSectionID{MODULID}\ifnum\value{MSSEnd}>0{\MSubsectionEndMacros}\addtocounter{MSSEnd}{-1}\fi\MSectionChapter\Dsection{#1}\MSectionStartMacros{#1}\setcounter{MLastIndex}{-1}\setcounter{MLastType}{1}} % Sections werden ueber das section-Feld im mmlabel-Tag identifiziert, nicht ueber das Indexfeld

\def\MSubsection#1{\ifnum\value{MSSEnd}>0{\MSubsectionEndMacros}\addtocounter{MSSEnd}{-1}\fi\ifttm\else\clearpage\fi\Dsubsection{#1}\MSubsectionStartMacros\setcounter{MLastIndex}{-1}\setcounter{MLastType}{2}\addtocounter{MSSEnd}{1}}% Subsections werden ueber das subsection-Feld im mmlabel-Tag identifiziert, nicht ueber das Indexfeld
\def\MSubsectionx#1{\Dsubsectionx{#1}} % Nur zur Verwendung in MSectionStart gedacht
\def\MSubsubsection#1{\Dsubsubsection{#1}\setcounter{MLastIndex}{\value{subsubsection}}\setcounter{MLastType}{3}\ifttm\special{html:<!-- sectioninfo;;}\arabic{section}\special{html:;;}\arabic{subsection}\special{html:;;}\arabic{subsubsection}\special{html:;;1;;}\arabic{MTestSite}\special{html:; //-->}\fi}
\def\MSubsubsectionx#1{\Dsubsubsectionx{#1}\ifttm\special{html:<!-- sectioninfo;;}\arabic{section}\special{html:;;}\arabic{subsection}\special{html:;;}\arabic{subsubsection}\special{html:;;0;;}\arabic{MTestSite}\special{html:; //-->}\else\addcontentsline{toc}{subsection}{#1}\fi}

\ifttm
\def\MSubsubsubsectionx#1{\ \newline\textbf{#1}\special{html:<br />}}
\else
\def\MSubsubsubsectionx#1{\ \newline
\textbf{#1}\ \\
}
\fi


% Dieses Skript wird zu Beginn jedes Modulabschnitts (=Webseite) ausgefuehrt und initialisiert den Aufgabenfeldzaehler
\newcommand{\MPageScripts}{
\setcounter{MFieldCounter}{1}
\addtocounter{MSiteCounter}{1}
\setcounter{MHintCounter}{1}
\setcounter{MCodeEditCounter}{1}
\setcounter{MGroupActive}{0}
\DoQBoxes
% Feldvariablen werden im HTML-Header in conv.pl eingestellt
}

% Dieses Skript wird zum Ende jedes Modulabschnitts (=Webseite) ausgefuehrt
\ifttm
\newcommand{\MEndScripts}{\special{html:<br /><!-- mfeedbackbutton;Seite;}\arabic{MTestSite}\special{html:;}\MGenerateSiteNumber\special{html:; //-->}
}
\else
\newcommand{\MEndScripts}{\relax}
\fi


\newcounter{QBoxFlag}
\newcommand{\DoQBoxes}{\setcounter{QBoxFlag}{1}}
\newcommand{\NoQBoxes}{\setcounter{QBoxFlag}{0}}

\newcounter{MXCTest}
\newcounter{MXCounter}
\newcounter{MSCounter}



\ifttm

% Struktur des sectioninfo-Tags: <!-- sectioninfo;;section;;subsection;;subsubsection;;nr_ausgeben;;testpage; //-->

%Fuegt eine zusaetzliche html-Seite an hinter ALLEN bisherigen und zukuenftigen content-Seiten ausserhalb der vor-zurueck-Schleife (d.h. nur durch Button oder MIntLink erreichbar!)
% #1 = Titel des Modulabschnitts, #2 = Kurztitel fuer die Buttons, #3 = Buttonkennung (STD = default nehmen, NONE = Ohne Button in der Navigation)
\newenvironment{MSContent}[3]{\special{html:<div class="xcontent}\arabic{MSCounter}\special{html:"><!-- scontent;-;}\arabic{MSCounter};-;#1;-;#2;-;#3\special{html: //-->}\MPageScripts\MSubsubsectionx{#1}}{\MEndScripts\special{html:<!-- endscontent;;}\arabic{MSCounter}\special{html: //--></div>}\addtocounter{MSCounter}{1}}

% Fuegt eine zusaetzliche html-Seite ein hinter den bereits vorhandenen content-Seiten (oder als erste Seite) innerhalb der vor-zurueck-Schleife der Navigation
% #1 = Titel des Modulabschnitts, #2 = Kurztitel fuer die Buttons, #3 = Buttonkennung (STD = Defaultbutton, NONE = Ohne Button in der Navigation)
\newenvironment{MXContent}[3]{\special{html:<div class="xcontent}\arabic{MXCounter}\special{html:"><!-- xcontent;-;}\arabic{MXCounter};-;#1;-;#2;-;#3\special{html: //-->}\MPageScripts\MSubsubsection{#1}}{\MEndScripts\special{html:<!-- endxcontent;;}\arabic{MXCounter}\special{html: //--></div>}\addtocounter{MXCounter}{1}}

% Fuegt eine zusaetzliche html-Seite ein die keine subsubsection-Nummer bekommt, nur zur internen Verwendung in mintmod.tex gedacht!
% #1 = Titel des Modulabschnitts, #2 = Kurztitel fuer die Buttons, #3 = Buttonkennung (STD = Defaultbutton, NONE = Ohne Button in der Navigation)
% \newenvironment{MUContent}[3]{\special{html:<div class="xcontent}\arabic{MXCounter}\special{html:"><!-- xcontent;-;}\arabic{MXCounter};-;#1;-;#2;-;#3\special{html: //-->}\MPageScripts\MSubsubsectionx{#1}}{\MEndScripts\special{html:<!-- endxcontent;;}\arabic{MXCounter}\special{html: //--></div>}\addtocounter{MXCounter}{1}}

\newcommand{\MDeclareSiteUXID}[1]{\special{html:<!-- mdeclaresiteuxid;;}#1\special{html:;;}\arabic{chapter}\special{html:;;}\arabic{section}\special{html:;; //-->}}

\else

%\newcommand{\MSubsubsection}[1]{\refstepcounter{subsubsection} \addcontentsline{toc}{subsubsection}{\thesubsubsection. #1}}


% Fuegt eine zusaetzliche html-Seite an hinter den bereits vorhandenen content-Seiten
% #1 = Titel des Modulabschnitts, #2 = Kurztitel fuer die Buttons, #3 = Iconkennung (im PDF wirkungslos)
%\newenvironment{MUContent}[3]{\ifnum\value{MXCTest}>0{\MDebugMessage{ERROR: Geschachtelter SContent}}\fi\MPageScripts\MSubsubsectionx{#1}\addtocounter{MXCTest}{1}}{\addtocounter{MXCounter}{1}\addtocounter{MXCTest}{-1}}
\newenvironment{MXContent}[3]{\ifnum\value{MXCTest}>0{\MDebugMessage{ERROR: Geschachtelter SContent}}\fi\MPageScripts\MSubsubsection{#1}\addtocounter{MXCTest}{1}}{\addtocounter{MXCounter}{1}\addtocounter{MXCTest}{-1}}
\newenvironment{MSContent}[3]{\ifnum\value{MXCTest}>0{\MDebugMessage{ERROR: Geschachtelter XContent}}\fi\MPageScripts\MSubsubsectionx{#1}\addtocounter{MXCTest}{1}}{\addtocounter{MSCounter}{1}\addtocounter{MXCTest}{-1}}

\newcommand{\MDeclareSiteUXID}[1]{\relax}

\fi 

% GHEADER und GFOOTER werden von split.pm gefunden, aber nur, wenn nicht HELPSITE oder TESTSITE
\ifttm
\newenvironment{MSectionStart}{\special{html:<div class="xcontent0">}\MSubsubsectionx{Modul\"ubersicht}}{\setcounter{MSSEnd}{0}\special{html:</div>}}
% Darf nicht als XContent nummeriert werden, darf nicht als XContent gelabelt werden, wird aber in eine xcontent-div gesetzt fuer Python-parsing
\else
\newenvironment{MSectionStart}{\MSubsectionx{Modul\"ubersicht}}{\setcounter{MSSEnd}{0}}
\fi

\newenvironment{MIntro}{\begin{MXContent}{Einf\"uhrung}{Einf\"uhrung}{genetisch}}{\end{MXContent}}
\newenvironment{MContent}{\begin{MXContent}{Inhalt}{Inhalt}{beweis}}{\end{MXContent}}
\newenvironment{MExercises}{\ifttm\else\clearpage\fi\begin{MXContent}{Aufgaben}{Aufgaben}{aufgb}\special{html:<!-- declareexcsymb //-->}}{\end{MXContent}}

% #1 = Lesbare Testbezeichnung
\newenvironment{MTest}[1]{%
\renewcommand{\MTestName}{#1}
\ifttm\else\clearpage\fi%
\addtocounter{MTestSite}{1}%
\begin{MXContent}{#1}{#1}{STD} % {aufgb}%
\special{html:<!-- declaretestsymb //-->}
\begin{MQuestionGroup}%
\MInTestHeader
}%
{%
\end{MQuestionGroup}%
\ \\ \ \\%
\MInTestFooter
\end{MXContent}\addtocounter{MTestSite}{-1}%
}

\newenvironment{MExtra}{\ifttm\else\clearpage\fi\begin{MXContent}{Zus\"atzliche Inhalte}{Zusatz}{weiterfhrg}}{\end{MXContent}}

\makeindex

\ifttm
\def\MPrintIndex{
\ifnum\value{MSSEnd}>0{\MSubsectionEndMacros}\addtocounter{MSSEnd}{-1}\fi
\renewcommand{\indexname}{Stichwortverzeichnis}
\special{html:<p><!-- printindex //--></p>}
}
\else
\def\MPrintIndex{
\ifnum\value{MSSEnd}>0{\MSubsectionEndMacros}\addtocounter{MSSEnd}{-1}\fi
\renewcommand{\indexname}{Stichwortverzeichnis}
\addcontentsline{toc}{section}{Stichwortverzeichnis}
\printindex
}
\fi


% Konstanten fuer die Modulfaecher

\def\MINTMathematics{1}
\def\MINTInformatics{2}
\def\MINTChemistry{3}
\def\MINTPhysics{4}
\def\MINTEngineering{5}

\newcounter{MSubjectArea}
\newcounter{MInfoNumbers} % Gibt an, ob die Infoboxen nummeriert werden sollen
\newcounter{MSepNumbers} % Gibt an, ob Beispiele und Experimente separat nummeriert werden sollen
\newcommand{\MSetSubject}[1]{
 % ttm kapiert setcounter mit Parametern nicht, also per if abragen und einsetzen
\ifnum#1=1\setcounter{MSubjectArea}{1}\setcounter{MInfoNumbers}{1}\setcounter{MSepNumbers}{0}\fi
\ifnum#1=2\setcounter{MSubjectArea}{2}\setcounter{MInfoNumbers}{1}\setcounter{MSepNumbers}{0}\fi
\ifnum#1=3\setcounter{MSubjectArea}{3}\setcounter{MInfoNumbers}{0}\setcounter{MSepNumbers}{1}\fi
\ifnum#1=4\setcounter{MSubjectArea}{4}\setcounter{MInfoNumbers}{0}\setcounter{MSepNumbers}{0}\fi
\ifnum#1=5\setcounter{MSubjectArea}{5}\setcounter{MInfoNumbers}{1}\setcounter{MSepNumbers}{0}\fi
% Separate Nummerntechnik fuer unsere Chemiker: alles dreistellig
\ifnum#1=3
  \ifttm
  \renewcommand{\theequation}{\arabic{section}.\arabic{subsection}.\arabic{equation}}
  \renewcommand{\thetable}{\arabic{section}.\arabic{subsection}.\arabic{table}} 
  \renewcommand{\thefigure}{\arabic{section}.\arabic{subsection}.\arabic{figure}} 
  \else
  \renewcommand{\theequation}{\arabic{chapter}.\arabic{section}.\arabic{equation}}
  \renewcommand{\thetable}{\arabic{chapter}.\arabic{section}.\arabic{table}}
  \renewcommand{\thefigure}{\arabic{chapter}.\arabic{section}.\arabic{figure}}
  \fi
\else
  \ifttm
  \renewcommand{\theequation}{\arabic{section}.\arabic{subsection}.\arabic{equation}}
  \renewcommand{\thetable}{\arabic{table}}
  \renewcommand{\thefigure}{\arabic{figure}}
  \else
  \renewcommand{\theequation}{\arabic{chapter}.\arabic{section}.\arabic{equation}}
  \renewcommand{\thetable}{\arabic{table}}
  \renewcommand{\thefigure}{\arabic{figure}}
  \fi
\fi
}

% Fuer tikz Autogenerierung
\newcounter{MTIKZAutofilenumber}

% Spezielle Counter fuer die Bentz-Module
\newcounter{mycounter}
\newcounter{chemapplet}
\newcounter{physapplet}

\newcounter{MSSEnd} % Ist 1 falls ein MSubsection aktiv ist, der einen MSubsectionEndMacro-Aufruf verursacht
\newcounter{MFileNumber}
\def\MLastFile{\special{html:[[!-- mfileref;;}\arabic{MFileNumber}\special{html:; //--]]}}

% Vollstaendiger Pfad ist \MMaterial / \MLastFilePath / \MLastFileName    ==   \MMaterial / \MLastFile

% Wird nur bei kompletter Baum-Erstellung ausgefuehrt!
% #1 = Lesbare Modulbezeichnung
\newcommand{\MSectionStartMacros}[1]{
\setcounter{MTestSite}{0}
\setcounter{MCaptionOn}{0}
\setcounter{MLastTypeEq}{0}
\setcounter{MSSEnd}{0}
\setcounter{MFileNumber}{0} % Preinkrekement-Counter
\setcounter{MTIKZAutofilenumber}{0}
\setcounter{mycounter}{1}
\setcounter{physapplet}{1}
\setcounter{chemapplet}{0}
\ifttm
\special{html:<!-- mdeclaresection;;}\arabic{chapter}\special{html:;;}\arabic{section}\special{html:;;}#1\special{html:;; //-->}%
\else
\setcounter{thmc}{0}
\setcounter{exmpc}{0}
\setcounter{verc}{0}
\setcounter{infoc}{0}
\fi
\setcounter{MiniMarkerCounter}{1}
\setcounter{AlignCounter}{1}
\setcounter{MXCTest}{0}
\setcounter{MCodeCounter}{0}
\setcounter{MEntryCounter}{0}
}

% Wird immer ausgefuehrt
\newcommand{\MSubsectionStartMacros}{
\ifttm\else\MPageHeaderDef\fi
\MWatermarkSettings
\setcounter{MXCounter}{0}
\setcounter{MSCounter}{0}
\setcounter{MSiteCounter}{1}
\setcounter{MExerciseCollectionCounter}{0}
% Zaehler fuer das Labelsystem zuruecksetzen (prefix-Zaehler)
\setcounter{MInfoCounter}{0}
\setcounter{MExerciseCounter}{0}
\setcounter{MExampleCounter}{0}
\setcounter{MExperimentCounter}{0}
\setcounter{MGraphicsCounter}{0}
\setcounter{MTableCounter}{0}
\setcounter{MTheoremCounter}{0}
\setcounter{MObjectCounter}{0}
\setcounter{MEquationCounter}{0}
\setcounter{MVideoCounter}{0}
\setcounter{equation}{0}
\setcounter{figure}{0}
}

\newcommand{\MSubsectionEndMacros}{
% Bei Chemiemodulen das PSE einhaengen, es soll als SContent am Ende erscheinen
\special{html:<!-- subsectionend //-->}
\ifnum\value{MSubjectArea}=3{\MIncludePSE}\fi
}


\ifttm
%\newcommand{\MEmbed}[1]{\MRegisterFile{#1}\begin{html}<embed src="\end{html}\MMaterial/\MLastFile\begin{html}" width="192" height="189"></embed>\end{html}}
\newcommand{\MEmbed}[1]{\MRegisterFile{#1}\begin{html}<embed src="\end{html}\MMaterial/\MLastFile\begin{html}"></embed>\end{html}}
\fi

%----------------- Makros fuer die Textdarstellung -----------------------------------------------

\ifttm
% MUGraphics bindet eine Grafik ein:
% Parameter 1: Dateiname der Grafik, relativ zur Position des Modul-Tex-Dokuments
% Parameter 2: Skalierungsoptionen fuer PDF (fuer includegraphics)
% Parameter 3: Titel fuer die Grafik, wird unter die Grafik mit der Grafiknummer gesetzt und kann MLabel bzw. MCopyrightLabel enthalten
% Parameter 4: Skalierungsoptionen fuer HTML (css-styles)

% ERSATZ: <img alt="My Image" src="data:image/png;base64,iVBORwA<MoreBase64SringHere>" />


\newcommand{\MUGraphics}[4]{\MRegisterFile{#1}\begin{html}
<div class="imagecenter">
<center>
<div>
<img src="\end{html}\MMaterial/\MLastFile\begin{html}" style="#4" alt="\end{html}\MMaterial/\MLastFile\begin{html}"/>
</div>
<div class="bildtext">
\end{html}
\addtocounter{MGraphicsCounter}{1}
\setcounter{MLastIndex}{\value{MGraphicsCounter}}
\setcounter{MLastType}{8}
\addtocounter{MCaptionOn}{1}
\ifnum\value{MSepNumbers}=0
\textbf{Abbildung \arabic{MGraphicsCounter}:} #3
\else
\textbf{Abbildung \arabic{section}.\arabic{subsection}.\arabic{MGraphicsCounter}:} #3
\fi
\addtocounter{MCaptionOn}{-1}
\begin{html}
</div>
</center>
</div>
<br />
\end{html}%
\special{html:<!-- mfeedbackbutton;Abbildung;}\arabic{MGraphicsCounter}\special{html:;}\arabic{section}.\arabic{subsection}.\arabic{MGraphicsCounter}\special{html:; //-->}%
}

% MVideo bindet ein Video als Einzeldatei ein:
% Parameter 1: Dateiname des Videos, relativ zur Position des Modul-Tex-Dokuments, ohne die Endung ".mp4"
% Parameter 2: Titel fuer das Video (kann MLabel oder MCopyrightLabel enthalten), wird unter das Video mit der Videonummer gesetzt
\newcommand{\MVideo}[2]{\MRegisterFile{#1.mp4}\begin{html}
<div class="imagecenter">
<center>
<div>
<video width="95\%" controls="controls"><source src="\end{html}\MMaterial/#1.mp4\begin{html}" type="video/mp4">Ihr Browser kann keine MP4-Videos abspielen!</video>
</div>
<div class="bildtext">
\end{html}
\addtocounter{MVideoCounter}{1}
\setcounter{MLastIndex}{\value{MVideoCounter}}
\setcounter{MLastType}{12}
\addtocounter{MCaptionOn}{1}
\ifnum\value{MSepNumbers}=0
\textbf{Video \arabic{MVideoCounter}:} #2
\else
\textbf{Video \arabic{section}.\arabic{subsection}.\arabic{MVideoCounter}:} #2
\fi
\addtocounter{MCaptionOn}{-1}
\begin{html}
</div>
</center>
</div>
<br />
\end{html}}

\newcommand{\MDVideo}[2]{\MRegisterFile{#1.mp4}\MRegisterFile{#1.ogv}\begin{html}
<div class="imagecenter">
<center>
<div>
<video width="70\%" controls><source src="\end{html}\MMaterial/#1.mp4\begin{html}" type="video/mp4"><source src="\end{html}\MMaterial/#1.ogv\begin{html}" type="video/ogg">Ihr Browser kann keine MP4-Videos abspielen!</video>
</div>
<br />
#2
</center>
</div>
<br />
\end{html}
}

\newcommand{\MGraphics}[3]{\MUGraphics{#1}{#2}{#3}{}}

\else

\newcommand{\MVideo}[2]{%
% Kein Video im PDF darstellbar, trotzdem so tun als ob da eines waere
\begin{center}
(Video nicht darstellbar)
\end{center}
\addtocounter{MVideoCounter}{1}
\setcounter{MLastIndex}{\value{MVideoCounter}}
\setcounter{MLastType}{12}
\addtocounter{MCaptionOn}{1}
\ifnum\value{MSepNumbers}=0
\textbf{Video \arabic{MVideoCounter}:} #2
\else
\textbf{Video \arabic{section}.\arabic{subsection}.\arabic{MVideoCounter}:} #2
\fi
\addtocounter{MCaptionOn}{-1}
}


% MGraphics bindet eine Grafik ein:
% Parameter 1: Dateiname der Grafik, relativ zur Position des Modul-Tex-Dokuments
% Parameter 2: Skalierungsoptionen fuer PDF (fuer includegraphics)
% Parameter 3: Titel fuer die Grafik, wird unter die Grafik mit der Grafiknummer gesetzt
\newcommand{\MGraphics}[3]{%
\MRegisterFile{#1}%
\ %
\begin{figure}[H]%
\centering{%
\includegraphics[#2]{\MDPrefix/#1}%
\addtocounter{MCaptionOn}{1}%
\caption{#3}%
\addtocounter{MCaptionOn}{-1}%
}%
\end{figure}%
\addtocounter{MGraphicsCounter}{1}\setcounter{MLastIndex}{\value{MGraphicsCounter}}\setcounter{MLastType}{8}\ %
%\ \\Abbildung \ifnum\value{MSepNumbers}=0\else\arabic{chapter}.\arabic{section}.\fi\arabic{MGraphicsCounter}: #3%
}

\newcommand{\MUGraphics}[4]{\MGraphics{#1}{#2}{#3}}


\fi

\newcounter{MCaptionOn} % = 1 falls eine Grafikcaption aktiv ist, = 0 sonst


% MGraphicsSolo bindet eine Grafik pur ein ohne Titel
% Parameter 1: Dateiname der Grafik, relativ zur Position des Modul-Tex-Dokuments
% Parameter 2: Skalierungsoptionen (wirken nur im PDF)
\newcommand{\MGraphicsSolo}[2]{\MUGraphicsSolo{#1}{#2}{}}

% MUGraphicsSolo bindet eine Grafik pur ein ohne Titel, aber mit HTML-Skalierung
% Parameter 1: Dateiname der Grafik, relativ zur Position des Modul-Tex-Dokuments
% Parameter 2: Skalierungsoptionen (wirken nur im PDF)
% Parameter 3: Skalierungsoptionen (wirken nur im HTML), als style-format: "width=???, height=???"
\ifttm
\newcommand{\MUGraphicsSolo}[3]{\MRegisterFile{#1}\begin{html}
<img src="\end{html}\MMaterial/\MLastFile\begin{html}" style="\end{html}#3\begin{html}" alt="\end{html}\MMaterial/\MLastFile\begin{html}"/>
\end{html}%
\special{html:<!-- mfeedbackbutton;Abbildung;}#1\special{html:;}\MMaterial/\MLastFile\special{html:; //-->}%
}
\else
\newcommand{\MUGraphicsSolo}[3]{\MRegisterFile{#1}\includegraphics[#2]{\MDPrefix/#1}}
\fi

% Externer Link mit URL
% Erster Parameter: Vollstaendige(!) URL des Links
% Zweiter Parameter: Text fuer den Link
\newcommand{\MExtLink}[2]{\ifttm\special{html:<a target="_new" href="}#1\special{html:">}#2\special{html:</a>}\else\href{#1}{#2}\fi} % ohne MINTERLINK!


% Interner Link, die verlinkte Datei muss im gleichen Verzeichnis liegen wie die Modul-Texdatei
% Erster Parameter: Dateiname
% Zweiter Parameter: Text fuer den Link
\newcommand{\MIntLink}[2]{\ifttm\MRegisterFile{#1}\special{html:<a class="MINTERLINK" target="_new" href="}\MMaterial/\MLastFile\special{html:">}#2\special{html:</a>}\else{\href{#1}{#2}}\fi}


\ifttm
\def\MMaterial{:localmaterial:}
\else
\def\MMaterial{\MDPrefix}
\fi

\ifttm
\def\MNoFile#1{:directmaterial:#1}
\else
\def\MNoFile#1{#1}
\fi

\newcommand{\MChem}[1]{$\mathrm{#1}$}

\newcommand{\MApplet}[3]{
% Bindet ein Java-Applet ein, die Parameter sind:
% (wird nur im HTML, aber nicht im PDF erstellt)
% #1 Dateiname des Applets (muss mit ".class" enden)
% #2 = Breite in Pixeln
% #3 = Hoehe in Pixeln
\ifttm
\MRegisterFile{#1}
\begin{html}
<applet code="\end{html}\MMaterial/\MLastFile\begin{html}" width="#2" height="#3" alt="[Java-Applet kann nicht gestartet werden]"></applet>
\end{html}
\fi
}

\newcommand{\MScriptPage}[2]{
% Bindet eine JavaScript-Datei ein, die eine eigene Seite bekommt
% (wird nur im HTML, aber nicht im PDF erstellt)
% #1 Dateiname des Programms (sollte mit ".js" enden)
% #2 = Kurztitel der Seite
\ifttm
\begin{MSContent}{#2}{#2}{puzzle}
\MRegisterFile{#1}
\begin{html}
<script src="\MMaterial/\MLastFile" type="text/javascript"></script>
\end{html}
\end{MSContent}
\fi
}

\newcommand{\MIncludePSE}{
% Bindet bei Chemie-Modulen das PSE ein
% (wird nur im HTML, aber nicht im PDF erstellt)
\ifttm
\special{html:<!-- includepse //-->}
\begin{MSContent}{Periodensystem der Elemente}{PSE}{table}
\MRegisterFile{../files/pse.js}
\MRegisterFile{../files/radio.png}
\begin{html}
<script src="\MMaterial/../files/pse.js" type="text/javascript"></script>
<p id="divid"><br /><br />
<script language="javascript" type="text/javascript">
    startpse("divid","\MMaterial/../files"); 
</script>
</p>
<br />
<br />
<br />
<p>Die Farben der Elementsymbole geben an: <font style="color:Red">gasf&ouml;rmig </font> <font style="color:Blue">fl&uuml;ssig </font> fest</p>
<p>Die Elemente der Gruppe 1 A, 2 A, 3 A usw. geh&ouml;ren zu den Hauptgruppenelementen.</p>
<p>Die Elemente der Gruppe 1 B, 2 B, 3 B usw. geh&ouml;ren zu den Nebengruppenelementen.</p>
<p>() kennzeichnet die Masse des stabilsten Isotops</p>
\end{html}
\end{MSContent}
\fi
}

\newcommand{\MAppletArchive}[4]{
% Bindet ein Java-Applet ein, die Parameter sind:
% (wird nur im HTML, aber nicht im PDF erstellt)
% #1 Dateiname der Klasse mit Appletaufruf (muss mit ".class" enden)
% #2 Dateiname des Archivs (muss mit ".jar" enden)
% #3 = Breite in Pixeln
% #4 = Hoehe in Pixeln
\ifttm
\MRegisterFile{#2}
\begin{html}
<applet code="#1" archive="\end{html}\MMaterial/\MLastFile\begin{html}" codebase="." width="#3" height="#4" alt="[Java-Archiv kann nicht gestartet werden]"></applet>
\end{html}
\fi
}

% Bindet in der Haupttexdatei ein MINT-Modul ein. Parameter 1 ist das Verzeichnis (relativ zur Haupttexdatei), Parameter 2 ist der Dateinahme ohne Pfad.
\newcommand{\IncludeModule}[2]{
\renewcommand{\MDPrefix}{#1}
\input{#1/#2}
\ifnum\value{MSSEnd}>0{\MSubsectionEndMacros}\addtocounter{MSSEnd}{-1}\fi
}

% Der ttm-Konverter setzt keine Makros im \input um, also muss hier getrickst werden:
% Das MDPrefix muss in den einzelnen Modulen manuell eingesetzt werden
\newcommand{\MInputFile}[1]{
\ifttm
\input{#1}
\else
\input{#1}
\fi
}


\newcommand{\MCases}[1]{\left\lbrace{\begin{array}{rl} #1 \end{array}}\right.}

\ifttm
\newenvironment{MCaseEnv}{\left\lbrace\begin{array}{rl}}{\end{array}\right.}
\else
\newenvironment{MCaseEnv}{\left\lbrace\begin{array}{rl}}{\end{array}\right.}
\fi

\def\MSkip{\ifttm\MCR\fi}

\ifttm
\def\MCR{\special{html:<br />}}
\else
\def\MCR{\ \\}
\fi


% Pragmas - Sind Schluesselwoerter, die dem Preprocessing sowie dem Konverter uebergeben werden und bestimmte
%           Aktionen ausloesen. Im Output (PDF und HTML) tauchen sie nicht auf.
\newcommand{\MPragma}[1]{%
\ifttm%
\special{html:<!-- mpragma;-;}#1\special{html:;; -->}%
\else%
% MPragmas werden vom Preprozessor direkt im LaTeX gefunden
\fi%
}

% Ersatz der Befehle textsubscript und textsuperscript, die ttm nicht kennt
\ifttm%
\newcommand{\MTextsubscript}[1]{\special{html:<sub>}#1\special{html:</sub>}}%
\newcommand{\MTextsuperscript}[1]{\special{html:<sup>}#1\special{html:</sup>}}%
\else%
\newcommand{\MTextsubscript}[1]{\textsubscript{#1}}%
\newcommand{\MTextsuperscript}[1]{\textsuperscript{#1}}%
\fi

%------------------ Einbindung von dia-Diagrammen ----------------------------------------------
% Beim preprocessing wird aus jeder dia-Datei eine tex-Datei und eine pdf-Datei erzeugt,
% diese werden hier jeweils im PDF und HTML eingebunden
% Parameter: Dateiname der mit dia erstellten Datei (OHNE die Endung .dia)
\ifttm%
\newcommand{\MDia}[1]{%
\MGraphicsSolo{#1minthtml.png}{}%
}
\else%
\newcommand{\MDia}[1]{%
\MGraphicsSolo{#1mintpdf.png}{scale=0.1667}%
}
\fi%

% subsup funktioniert im Ausdruck $D={\R}^+_0$, also \R geklammert und sup zuerst
% \ifttm
% \def\MSubsup#1#2#3{\special{html:<msubsup>} #1 #2 #3\special{html:</msubsup>}}
% \else
% \def\MSubsup#1#2#3{{#1}^{#3}_{#2}}
% \fi

%\input{local.tex}

% \ifttm
% \else
% \newwrite\mintlog
% \immediate\openout\mintlog=mintlog.txt
% \fi

% ----------------------- tikz autogenerator -------------------------------------------------------------------

\newcommand{\Mtikzexternalize}{\tikzexternalize}% wird bei Konvertierung ueber mconvert ggf. ausgehebelt!

\ifttm
\else
\tikzset%
{
  % Defines a custom style which generates pdf and converts to (low and hi-res quality) png and svg, then deletes the pdf
  % Important: DO NOT directly convert from pdf to hires-png or from svg to png with GraphViz convert as it has some problems and memory leaks
  png export/.style=%
  {
    external/system call/.add={}{; 
      pdf2svg "\image.pdf" "\image.svg" ; 
      convert -density 112.5 -transparent white "\image.pdf" "\image.png"; 
      inkscape --export-png="\image.4x.png" --export-dpi=450 --export-background-opacity=0 --without-gui "\image.svg"; 
      rm "\image.pdf"; rm "\image.log"; rm "\image.dpth"; rm "\image.idx"
    },
    external/force remake,
  }
}
\tikzset{png export}
\tikzsetexternalprefix{}
% PNGs bei externer Erzeugung in "richtiger" Groesse einbinden
\pgfkeys{/pgf/images/include external/.code={\includegraphics[scale=0.64]{#1}}}
\fi

% Spezielle Umgebung fuer Autogenerierung, Bildernamen sind nur innerhalb eines Moduls (einer MSection) eindeutig)

\newcommand{\MTIKZautofilename}{tikzautofile}

\ifttm
% HTML-Version: Vom Autogenerator erzeugte png-Datei einbinden, tikz selbst nicht ausfuehren (sprich: #1 schlucken)
\newcommand{\MTikzAuto}[1]{%
\addtocounter{MTIKZAutofilenumber}{1}
\renewcommand{\MTIKZautofilename}{mtikzauto_\arabic{MTIKZAutofilenumber}}
\MUGraphicsSolo{\MSectionID\MTIKZautofilename.4x.png}{scale=1}{\special{html:[[!-- svgstyle;}\MSectionID\MTIKZautofilename\special{html: //--]]}} % Styleinfos werden aus original-png, nicht 4x-png geholt!
%\MRegisterFile{\MSectionID\MTIKZautofilename.png} % not used right now
%\MRegisterFile{\MSectionID\MTIKZautofilename.svg}
}
\else%
% PDF-Version: Falls Autogenerator aktiv wird Datei automatisch benannt und exportiert
\newcommand{\MTikzAuto}[1]{%
\addtocounter{MTIKZAutofilenumber}{1}%
\renewcommand{\MTIKZautofilename}{mtikzauto_\arabic{MTIKZAutofilenumber}}
\tikzsetnextfilename{\MTIKZautofilename}%
#1%
}
\fi

% In einer reinen LaTeX-Uebersetzung kapselt der Preambelinclude-Befehl nur input,
% in einer konvertergesteuerten PDF/HTML-Uebersetzung wird er dagegen entfernt und
% die Preambeln an mintmod angehaengt, die Ersetzung wird von mconvert.pl vorgenommen.

\newcommand{\MPreambleInclude}[1]{\input{#1}}

% Globale Watermarksettings (werden auch nochmal zu Beginn jedes subsection gesetzt,
% muessen hier aber auch global ausgefuehrt wegen Einfuehrungsseiten und Inhaltsverzeichnis

\MWatermarkSettings
% ---------------------------------- Parametrisierte Aufgaben ----------------------------------------

\ifttm
\newenvironment{MPExercise}{%
\begin{MExercise}%
}{%
\special{html:<button name="Name_MPEX}\arabic{MExerciseCounter}\special{html:" id="MPEX}\arabic{MExerciseCounter}%
\special{html:" type="button" onclick="reroll('}\arabic{MExerciseCounter}\special{html:');">Neue Aufgabe erzeugen</button>}%
\end{MExercise}%
}
\else
\newenvironment{MPExercise}{%
\begin{MExercise}%
}{%
\end{MExercise}%
}
\fi

% Parameter: Name, Min, Max, PDF-Standard. Name in Deklaration OHNE backslash, im Code MIT Backslash
\ifttm
\newcommand{\MGlobalInteger}[4]{\special{html:%
<!-- onloadstart //-->%
MVAR.push(createGlobalInteger("}#1\special{html:",}#2\special{html:,}#3\special{html:,}#4\special{html:)); %
<!-- onloadstop //-->%
<!-- viewmodelstart //-->%
ob}#1\special{html:: ko.observable(rerollMVar("}#1\special{html:")),%
<!-- viewmodelstop //-->%
}%
}%
\else%
\newcommand{\MGlobalInteger}[4]{\newcounter{mvc_#1}\setcounter{mvc_#1}{#4}}
\fi

% Parameter: Name, Min, Max, PDF-Standard. Name in Deklaration OHNE backslash, im Code MIT Backslash, Wert ist Wurzel von value
\ifttm
\newcommand{\MGlobalSqrt}[4]{\special{html:%
<!-- onloadstart //-->%
MVAR.push(createGlobalSqrt("}#1\special{html:",}#2\special{html:,}#3\special{html:,}#4\special{html:)); %
<!-- onloadstop //-->%
<!-- viewmodelstart //-->%
ob}#1\special{html:: ko.observable(rerollMVar("}#1\special{html:")),%
<!-- viewmodelstop //-->%
}%
}%
\else%
\newcommand{\MGlobalSqrt}[4]{\newcounter{mvc_#1}\setcounter{mvc_#1}{#4}}% Funktioniert nicht als Wurzel !!!
\fi

% Parameter: Name, Min, Max, PDF-Standard zaehler, PDF-Standard nenner. Name in Deklaration OHNE backslash, im Code MIT Backslash
\ifttm
\newcommand{\MGlobalFraction}[5]{\special{html:%
<!-- onloadstart //-->%
MVAR.push(createGlobalFraction("}#1\special{html:",}#2\special{html:,}#3\special{html:,}#4\special{html:,}#5\special{html:)); %
<!-- onloadstop //-->%
<!-- viewmodelstart //-->%
ob}#1\special{html:: ko.observable(rerollMVar("}#1\special{html:")),%
<!-- viewmodelstop //-->%
}%
}%
\else%
\newcommand{\MGlobalFraction}[5]{\newcounter{mvc_#1}\setcounter{mvc_#1}{#4}} % Funktioniert nicht als Bruch !!!
\fi

% MVar darf im HTML nur in MEvalMathDisplay-Umgebungen genutzt werden oder in Strings die an den Parser uebergeben werden
\ifttm%
\newcommand{\MVar}[1]{\special{html:[var_}#1\special{html:]}}%
\else%
\newcommand{\MVar}[1]{\arabic{mvc_#1}}%
\fi

\ifttm%
\newcommand{\MRerollButton}[2]{\special{html:<button type="button" onclick="rerollMVar('}#1\special{html:');">}#2\special{html:</button>}}%
\else%
\newcommand{\MRerollButton}[2]{\relax}% Keine sinnvolle Entsprechung im PDF
\fi

% MEvalMathDisplay fuer HTML wird in mconvert.pl im preprocessing realisiert
% PDF: eine equation*-Umgebung (ueber amsmath)
% HTML: Eine Mathjax-Tex-Umgebung, deren Auswertung mit knockout-obervablen gekoppelt ist
% PDF-Version hier nur fuer pdflatex-only-Uebersetzung gegeben

\ifttm\else\newenvironment{MEvalMathDisplay}{\begin{equation*}}{\end{equation*}}\fi

% ---------------------------------- Spezialbefehle fuer AD ------------------------------------------

%Abk�rzung f�r \longrightarrow:
\newcommand{\lto}{\ensuremath{\longrightarrow}}

%Makro f�r Funktionen:
\newcommand{\exfunction}[5]
{\begin{array}{rrcl}
 #1 \colon  & #2 &\lto & #3 \\[.05cm]  
  & #4 &\longmapsto  & #5 
\end{array}}

\newcommand{\function}[5]{%
#1:\;\left\lbrace{\begin{array}{rcl}
 #2 &\lto & #3 \\
 #4 &\longmapsto  & #5 \end{array}}\right.}


%Die Identit�t:
\DeclareMathOperator{\Id}{Id}

%Die Signumfunktion:
\DeclareMathOperator{\sgn}{sgn}

%Zwei Betonungskommandos (k�nnen angepasst werden):
\newcommand{\highlight}[1]{#1}
\newcommand{\modstextbf}[1]{#1}
\newcommand{\modsemph}[1]{#1}


% ---------------------------------- Spezialbefehle fuer JL ------------------------------------------


\def\jccolorfkt{green!50!black} %Farbe des Funktionsgraphen
\def\jccolorfktarea{green!25!white} %Farbe der Fl"ache unter dem Graphen
\def\jccolorfktareahell{green!12!white} %helle Einf"arbung der Fl"ache unter dem Graphen
\def\jccolorfktwert{green!50!black} %Farbe einzelner Punkte des Graphen

\newcommand{\MPfadBilder}{Bilder}

\ifttm%
\newcommand{\jMD}{\,\MD}%
\else%
\newcommand{\jMD}{\;\MD}%
\fi%

\def\jHTMLHinweisBedienung{\MInputHint{%
Mit Hilfe der Symbole am oberen Rand des Fensters
k"onnen Sie durch die einzelnen Abschnitte navigieren.}}

\def\jHTMLHinweisEingabeText{\MInputHint{%
Geben Sie jeweils ein Wort oder Zeichen als Antwort ein.}}

\def\jHTMLHinweisEingabeTerm{\MInputHint{%
Klammern Sie Ihre Terme, um eine eindeutige Eingabe zu erhalten. 
Beispiel: Der Term $\frac{3x+1}{x-2}$ soll in der Form
\texttt{(3*x+1)/((x+2)^2}$ eingegeben werden (wobei auch Leerzeichen 
eingegeben werden k"onnen, damit eine Formel besser lesbar ist).}}

\def\jHTMLHinweisEingabeIntervalle{\MInputHint{%
Intervalle werden links mit einer "offnenden Klammer und rechts mit einer 
schlie"senden Klammer angegeben. Eine runde Klammer wird verwendet, wenn der 
Rand nicht dazu geh"ort, eine eckige, wenn er dazu geh"ort. 
Als Trennzeichen wird ein Komma oder ein Semikolon akzeptiert.
Beispiele: $(a, b)$ offenes Intervall,
$[a; b)$ links abgeschlossenes, rechts offenes Intervall von $a$ bis $b$. 
Die Eingabe $]a;b[$ f"ur ein offenes Intervall wird nicht akzeptiert.
F"ur $\infty$ kann \texttt{infty} oder \texttt{unendlich} geschrieben werden.}}

\def\jHTMLHinweisEingabeFunktionen{\MInputHint{%
Schreiben Sie Malpunkte (geschrieben als \texttt{*}) aus und setzen Sie Klammern um Argumente f�r Funktionen.
Beispiele: Polynom: \texttt{3*x + 0.1}, Sinusfunktion: \texttt{sin(x)}, 
Verkettung von cos und Wurzel: \texttt{cos(sqrt(3*x))}.}}

\def\jHTMLHinweisEingabeFunktionenSinCos{\MInputHint{%
Die Sinusfunktion $\sin x$ wird in der Form \texttt{sin(x)} angegeben, %
$\cos\left(\sqrt{3 x}\right)$ durch \texttt{cos(sqrt(3*x))}.}}

\def\jHTMLHinweisEingabeFunktionenExp{\MInputHint{%
Die Exponentialfunktion $\MEU^{3x^4 + 5}$ wird als
\texttt{exp(3 * x^4 + 5)} angegeben, %
$\ln\left(\sqrt{x} + 3.2\right)$ durch \texttt{ln(sqrt(x) + 3.2)}.}}

% ---------------------------------- Spezialbefehle fuer Fachbereich Physik --------------------------

\newcommand{\E}{{e}}
\newcommand{\ME}[1]{\cdot 10^{#1}}
\newcommand{\MU}[1]{\;\mathrm{#1}}
\newcommand{\MPG}[3]{%
  \ifnum#2=0%
    #1\ \mathrm{#3}%
  \else%
    #1\cdot 10^{#2}\ \mathrm{#3}%
  \fi}%
%

\newcommand{\MMul}{\MExponentensymbXYZl} % Nur eine Abkuerzung


% ---------------------------------- Stichwortfunktionialitaet ---------------------------------------

% mpreindexentry wird durch Auswahlroutine in conv.pl durch mindexentry substitutiert
\ifttm%
\def\MIndex#1{\index{#1}\special{html:<!-- mpreindexentry;;}#1\special{html:;;}\arabic{MSubjectArea}\special{html:;;}%
\arabic{chapter}\special{html:;;}\arabic{section}\special{html:;;}\arabic{subsection}\special{html:;;}\arabic{MEntryCounter}\special{html:; //-->}%
\setcounter{MLastIndex}{\value{MEntryCounter}}%
\addtocounter{MEntryCounter}{1}%
}%
% Copyrightliste wird als tex-Datei im preprocessing von conv.pl erzeugt und unter converter/tex/entrycollection.tex abgelegt
% Der input-Befehl funktioniert nur, wenn die aufrufende tex-Datei auf der obersten Ebene liegt (d.h. selbst kein input/include ist, insbesondere keine Moduldatei)
\def\MEntryList{} % \input funktioniert nicht, weil ttm (und damit das \input) ausgefuehrt wird, bevor Datei da ist
\else%
\def\MIndex#1{\index{#1}}
\def\MEntryList{\MAbort{Stichwortliste nur im HTML realisierbar}}%
\fi%

\def\MEntry#1#2{\textbf{#1}\MIndex{#2}} % Idee: MLastType auf neuen Entry-Typ und dann ein MLabel vergeben mit autogen-Nummer

% ---------------------------------- Befehle fuer Tests ----------------------------------------------

% MEquationItem stellt eine Eingabezeile der Form Vorgabe = Antwortfeld her, der zweite Parameter kann z.B. MSimplifyQuestion-Befehl sein
\ifttm
\newcommand{\MEquationItem}[2]{{#1}$\,=\,${#2}}%
\else%
\newcommand{\MEquationItem}[2]{{#1}$\;\;=\,${#2}}%
\fi

\ifttm
\newcommand{\MInputHint}[1]{%
\ifnum%
\if\value{MTestSite}>0%
\else%
{\color{blue}#1}%
\fi%
\fi%
}
\else
\newcommand{\MInputHint}[1]{\relax}
\fi

\ifttm
\newcommand{\MInTestHeader}{%
Dies ist ein einreichbarer Test:
\begin{itemize}
\item{Im Gegensatz zu den offenen Aufgaben werden beim Eingeben keine Hinweise zur Formulierung der mathematischen Ausdr�cke gegeben.}
\item{Der Test kann jederzeit neu gestartet oder verlassen werden.}
\item{Der Test kann durch die Buttons am Ende der Seite beendet und abgeschickt, oder zur�ckgesetzt werden.}
\item{Der Test kann mehrfach probiert werden. F�r die Statistik z�hlt die zuletzt abgeschickte Version.}
\end{itemize}
}
\else
\newcommand{\MInTestHeader}{%
\relax
}
\fi

\ifttm
\newcommand{\MInTestFooter}{%
\special{html:<button name="Name_TESTFINISH" id="TESTFINISH" type="button" onclick="finish_button('}\MTestName\special{html:');">Test auswerten</button>}%
\begin{html}
&nbsp;&nbsp;&nbsp;&nbsp;&nbsp;&nbsp;&nbsp;&nbsp;
<button name="Name_TESTRESET" id="TESTRESET" type="button" onclick="reset_button();">Test zur�cksetzen</button>
<br />
<br />
<div class="xreply">
<p name="Name_TESTEVAL" id="TESTEVAL">
Hier erscheint die Testauswertung!
<br />
</p>
</div>
\end{html}
}
\else
\newcommand{\MInTestFooter}{%
\relax
}
\fi


% ---------------------------------- Notationsmakros -------------------------------------------------------------

% Notationsmakros die nicht von der Kursvariante abhaengig sind

\newcommand{\MZahltrennzeichen}[1]{\renewcommand{\MZXYZhltrennzeichen}{#1}}

\ifttm
\newcommand{\MZahl}[3][\MZXYZhltrennzeichen]{\edef\MZXYZtemp{\noexpand\special{html:<mn>#2#1#3</mn>}}\MZXYZtemp}
\else
\newcommand{\MZahl}[3][\MZXYZhltrennzeichen]{{}#2{#1}#3}
\fi

\newcommand{\MEinheitenabstand}[1]{\renewcommand{\MEinheitenabstXYZnd}{#1}}
\ifttm
\newcommand{\MEinheit}[2][\MEinheitenabstXYZnd]{{}#1\edef\MEINHtemp{\noexpand\special{html:<mi mathvariant="normal">#2</mi>}}\MEINHtemp} 
\else
\newcommand{\MEinheit}[2][\MEinheitenabstXYZnd]{{}#1 \mathrm{#2}} 
\fi

\newcommand{\MExponentensymbol}[1]{\renewcommand{\MExponentensymbXYZl}{#1}}
\newcommand{\MExponent}[2][\MExponentensymbXYZl]{{}#1{} 10^{#2}} 

%Punkte in 2 und 3 Dimensionen
\newcommand{\MPointTwo}[3][]{#1(#2\MCoordPointSep #3{}#1)}
\newcommand{\MPointThree}[4][]{#1(#2\MCoordPointSep #3\MCoordPointSep #4{}#1)}
\newcommand{\MPointTwoAS}[2]{\left(#1\MCoordPointSep #2\right)}
\newcommand{\MPointThreeAS}[3]{\left(#1\MCoordPointSep #2\MCoordPointSep #3\right)}

% Masseinheit, Standardabstand: \,
\newcommand{\MEinheitenabstXYZnd}{\MThinspace} 

% Horizontaler Leerraum zwischen herausgestellter Formel und Interpunktion
\ifttm
\newcommand{\MDFPSpace}{\,}
\newcommand{\MDFPaSpace}{\,\,}
\newcommand{\MBlank}{\ }
\else
\newcommand{\MDFPSpace}{\;}
\newcommand{\MDFPaSpace}{\;\;}
\newcommand{\MBlank}{\ }
\fi

% Satzende in herausgestellter Formel mit horizontalem Leerraum
\newcommand{\MDFPeriod}{\MDFPSpace .}

% Separation von Aufzaehlung und Bedingung in Menge
\newcommand{\MCondSetSep}{\,:\,} %oder '\mid'

% Konverter kennt mathopen nicht
\ifttm
\def\mathopen#1{}
\fi

% -----------------------------------START Rouletteaufgaben ------------------------------------------------------------

\ifttm
% #1 = Dateiname, #2 = eindeutige ID fuer das Roulette im Kurs
\newcommand{\MDirectRouletteExercises}[2]{
\begin{MExercise}
\texttt{Im HTML erscheinen hier Aufgaben aus einer Aufgabenliste...}
\end{MExercise}
}
\else
\newcommand{\MDirectRouletteExercises}[2]{\relax} % wird durch mconvert.pl gefunden und ersetzt
\fi


% ---------------------------------- START Makros, die von der Kursvariante abhaengen ----------------------------------

\ifvariantunotation
  % unotation = An Universitaeten uebliche Notation
  \def\MVariant{unotation}

  % Trennzeichen fuer Dezimalzahlen
  \newcommand{\MZXYZhltrennzeichen}{.}

  % Exponent zur Basis 10 in der Exponentialschreibweise, 
  % Standardmalzeichen: \times
  \newcommand{\MExponentensymbXYZl}{\times} 

  % Begrenzungszeichen fuer offene Intervalle
  \newcommand{\MoIl}[1][]{\mbox{}#1(\mathopen{}} % bzw. ']'
  \newcommand{\MoIr}[1][]{#1)\mbox{}} % bzw. '['

  % Zahlen-Separation im IntervaLL
  \newcommand{\MIntvlSep}{,} %oder ';'

  % Separation von Elementen in Mengen
  \newcommand{\MElSetSep}{,} %oder ';'

  % Separation von Koordinaten in Punkten
  \newcommand{\MCoordPointSep}{,} %oder ';' oder '|', '\MThinspace|\MThinspace'

\else
  % An dieser Stelle wird angenommen, dass std-Variante aktiv ist
  % std = beschlossene Notation im TU9-Onlinekurs 
  \def\MVariant{std}

  % Trennzeichen fuer Dezimalzahlen
  \newcommand{\MZXYZhltrennzeichen}{,}

  % Exponent zur Basis 10 in der Exponentialschreibweise, 
  % Standardmalzeichen: \times
  \newcommand{\MExponentensymbXYZl}{\times} 

  % Begrenzungszeichen fuer offene Intervalle
  \newcommand{\MoIl}[1][]{\mbox{}#1]\mathopen{}} % bzw. '('
  \newcommand{\MoIr}[1][]{#1[\mbox{}} % bzw. ')'

  % Zahlen-Separation im IntervaLL
  \newcommand{\MIntvlSep}{;} %oder ','
  
  % Separation von Elementen in Mengen
  \newcommand{\MElSetSep}{;} %oder ','

  % Separation von Koordinaten in Punkten
  \newcommand{\MCoordPointSep}{;} %oder '|', '\MThinspace|\MThinspace'

\fi



% ---------------------------------- ENDE Makros, die von der Kursvariante abhaengen ----------------------------------


% diese Kommandos setzen Mathemodus vorraus
\newcommand{\MGeoAbstand}[2]{[\overline{{#1}{#2}}]}
\newcommand{\MGeoGerade}[2]{{#1}{#2}}
\newcommand{\MGeoStrecke}[2]{\overline{{#1}{#2}}}
\newcommand{\MGeoDreieck}[3]{{#1}{#2}{#3}}

%
\ifttm
\newcommand{\MOhm}{\special{html:<mn>&#x3A9;</mn>}}
\else
\newcommand{\MOhm}{\Omega} %\varOmega
\fi


\def\PERCTAG{\MAbort{PERCTAG ist zur internen verwendung in mconvert.pl reserviert, dieses Makro darf sonst nicht benutzt werden.}}

% Im Gegensatz zu einfachen html-Umgebungen werden MDirectHTML-Umgebungen von mconvert.pl am ganzen ttm-Prozess vorbeigeschleust und aus dem PDF komplett ausgeschnitten
\ifttm%
\newenvironment{MDirectHTML}{\begin{html}}{\end{html}}%
\else%
\newenvironment{MDirectHTML}{\begin{html}}{\end{html}}%
\fi

% Im Gegensatz zu einfachen Mathe-Umgebungen werden MDirectMath-Umgebungen von mconvert.pl am ganzen ttm-Prozess vorbeigeschleust, ueber MathJax realisiert, und im PDF als $$ ... $$ gesetzt
\ifttm%
\newenvironment{MDirectMath}{\begin{html}}{\end{html}}%
\else%
\newenvironment{MDirectMath}{\begin{equation*}}{\end{equation*}}% Vorsicht, auch \[ und \] werden in amsmath durch equation* redefiniert
\fi

% ---------------------------------- Location Management ---------------------------------------------

% #1 = buttonname (muss in files/images liegen und Format 48x48 haben), #2 = Vollstaendiger Einrichtungsname, #3 = Kuerzel der Einrichtung,  #4 = Name der include-texdatei
\ifttm
\newcommand{\MLocationSite}[3]{\special{html:<!-- mlocation;;}#1\special{html:;;}#2\special{html:;;}#3\special{html:;; //-->}}
\else
\newcommand{\MLocationSite}[3]{\relax}
\fi

% ---------------------------------- Copyright Management --------------------------------------------

\newcommand{\MCCLicense}{%
{\color{green}\textbf{CC BY-SA 3.0}}
}

\newcommand{\MCopyrightLabel}[1]{ (\MSRef{L_COPYRIGHTCOLLECTION}{Lizenz})\MLabel{#1}}

% Copyrightliste wird als tex-Datei im preprocessing erzeugt und unter converter/tex/copyrightcollection.tex abgelegt
% Der input-Befehl funktioniert nur, wenn die aufrufende tex-Datei auf der obersten Ebene liegt (d.h. selbst kein input/include ist, insbesondere keine Moduldatei)
\newcommand{\MCopyrightCollection}{\input{copyrightcollection.tex}}

% MCopyrightNotice fuegt eine Copyrightnotiz ein, der parser ersetzt diese durch CopyrightNoticePOST im preparsing, diese Definition wird nur fuer reine pdflatex-Uebersetzungen gebraucht
% Parameter: #1: Kurze Lizenzbeschreibung (typischerweise \MCCLicense)
%            #2: Link zum Original (http://...) oder NONE falls das Bild selbst ein Original ist, oder TIKZ falls das Bild aus einer tikz-Umgebung stammt
%            #3: Link zum Autor (http://...) oder MINT falls Original im MINT-Kolleg erstellt oder NONE falls Autor unbekannt
%            #4: Bemerkung (z.B. dass Datei mit Maple exportiert wurde)
%            #5: Labelstring fuer existierendes Label auf das copyrighted Objekt, mit MCopyrightLabel erzeugt
%            Keines der Felder darf leer sein!
\newcommand{\MCopyrightNotice}[5]{\MCopyrightNoticePOST{#1}{#2}{#3}{#4}{#5}}

\ifttm%
\newcommand{\MCopyrightNoticePOST}[5]{\relax}%
\else%
\newcommand{\MCopyrightNoticePOST}[5]{\relax}%
\fi%

% ---------------------------------- Meldungen fuer den Benutzer des Konverters ----------------------
\MPragma{mintmodversion;P0.1.0}
\MPragma{usercomment;This is file mintmod.tex version P0.1.0}


% ----------------------------------- Spezialelemente fuer Konfigurationsseite, werden nicht von mintscripts.js verwaltet --

% #1 = DOM-id der Box
\ifttm\newcommand{\MConfigbox}[1]{\special{html:<input cfieldtype="2" type="checkbox" name="Name_}#1\special{html:" id="}#1\special{html:" onchange="confHandlerChange('}#1\special{html:');"/>}}\fi % darf im PDF nicht aufgerufen werden!


\Mtikzexternalize
\MPragma{MathSkip}


\begin{document}

%\MSetSubject{MINTMathematics}
\MSection{Differentialrechnung}
\MLabel{VBKM07}
\MSetSectionID{VBKM07}

\begin{MSectionStart}
\MDeclareSiteUXID{VBKM07_START}

\MModstartBox
\end{MSectionStart}

%%%Abschnitt
\MSubsection{Ableitung einer Funktion}\MLabel{M07_Ableitung}

\begin{MIntro}
\MDeclareSiteUXID{VBKM07_Ableitung_Intro}
Eine Familie ist mit dem Auto unterwegs in den Urlaub. Der Wagen fährt mit einer Geschwindigkeit von $60\MEinheit{km}/\MEinheit[]{h}$ durch eine Baustelle.
Am Ende der Baustelle steht ein Schild, das ab sofort wieder eine Geschwindigkeit von $120\MEinheit{km}/\MEinheit[]{h}$ erlaubt.
Auch wenn die Fahrerin oder der Fahrer so kräftig wie nur irgend möglich auf das Gaspedal tritt, die Geschwindigkeit des Wagens wird sich nicht sprunghaft ändern,
sondern in Abhängigkeit von der Zeit steigen. Wird die Geschwindigkeit innerhalb von 5 Sekunden von $60\MEinheit{km}/\MEinheit[]{h}$ auf $120\MEinheit{km}/\MEinheit[]{h}$
mit einer konstanten Änderungsrate erhöht, dann ist die \emph{Beschleunigung} (= Geschwindigkeitsänderung pro Zeit) im vorliegenden Fall diese konstante Änderungsrate der Geschwindigkeit:
Die Beschleunigung ergibt sich als Quotient aus der Geschwindigkeitsänderung und der dafür benötigten Zeit. Ihr Wert ist hier also $12$ Kilometer pro Stunde pro Sekunde.
In der Realität wird die Geschwindigkeit des Autos jedoch nicht mit einer konstanten Änderungsrate erhöht werden können, sondern mit einer 
\emph{zeitabhängigen} Änderungsrate. Beschreibt man die Geschwindigkeit als Funktion $v$ der Zeit $t$, erhält man die Beschleunigung als Steigung dieser Funktion,
unabhängig davon, ob diese Steigung (zeitlich) konstant ist oder nicht. Mit anderen Worten: Die Beschleunigung ist die \emph{Ableitung} der Geschwindigkeits\emph{funktion} $v$
nach der Zeit $t$.

Ähnliche Zusammenhänge finden sich auch in anderen technischen Bereichen, z.B. bei der Berechnung von inneren Kräften, die in Stahlgerüsten von Bauwerken wirken,
der Vorhersage von Atmosphären- oder Meeresströmungen oder auch bei der heute so wichtigen Modellierung der Finanzmärkte.

Dieses Kapitel wiederholt die grundlegenden Ideen, die hinter diesen Berechnungen stecken, Gegenstand ist also die \textbf{Differentialrechnung}.
Mit anderen Worten: Es werden Ableitungen von Funktionen gebildet und so deren Steigungen bzw. Änderungsraten bestimmt.
Auch wenn hier diese Berechnungen streng mathematisch durchgeführt werden, ist die Motivation dafür nicht rein mathematischer Natur.
Ableitungen nehmen in vielen wissenschaftlichen Bereichen in der Interpretation als Änderungsraten verschiedener
Funktionen eine wichtige Rolle ein und werden oft als herausragende Größen untersucht.

\end{MIntro}

\begin{MXContent}{Relative Änderungsrate einer Funktion}{Relative Änderungsrate}{STD}
\MDeclareSiteUXID{VBKM07_Ableitung_Aenderungsrate}

Es sollen eine Funktion $f: [a\MIntvlSep  b] \rightarrow \R$, $x \mapsto f(x)$ sowie eine Skizze des Graphen von $f$ (siehe unten) betrachtet werden.
Das Ziel ist die Beschreibung der Änderungsrate dieser Funktion an einer beliebigen Stelle $x_0$ zwischen $a$ und $b$.
Dies wird auf den Begriff der Ableitung einer Funktion führen. Generell sollen möglichst einfache Rechenregeln Anwendung finden.

\begin{center}
\MTikzAuto{%
\begin{small}
\begin{tikzpicture}[line width=1.5pt,scale=0.7, %
declare function={
  x0 = 2;
  x1 = 4;
  fkt(\x) = 1/4 * (\x - 1)*(\x - 1) + 0.75;
  rT = 1.6; % relative Translation der Beschriftung $\Delta(f)$
%  Tangente(\x) = 1/2 * (\x - 2) + 1;
}
] %[every node/.style={fill=white}] 
%Koordinatenachsen:
\draw[->] (-0.6, 0) -- ({x1+1}, 0) node[below left]{$x$}; %x-Achse
\draw[->] (0, -0.6) -- (0, {fkt(x1)+1}) node[below left]{$y$}; %y-Achse
%Achsenbeschriftung:
\foreach \x in {1, 2, 3, 4} \draw (\x, 0) -- ++(0, -0.1); %
% node[below] {$\x$}; 
\foreach \y in {1, 2, 3} \draw (0, \y) -- ++(-0.1, 0); %
% node[left] {$\y$};
%\node[below left] at (0, 0) {$0$};
%Hilfslinien:
\draw[color=black!50!white,style=dashed] (x1, {fkt(x0)}) -- (x1, {fkt(x1)});
\draw[color=black!50!white,style=dotted] (x0, {fkt(x0)}) -- (x1, {fkt(x0)});
%Funktion:
\draw[domain=0.5:x1,samples=120,color=\jccolorfkt] %
 plot (\x, {fkt(\x)});
%Punkte und Hilfslinien:
\filldraw[color=black,fill=black] (x0, 0) circle (1pt);
\node[below] at (x0, -0.1) {$x_0$};
\filldraw[color=black,fill=black] (x0, {fkt(x0)}) circle (1pt);
%
\filldraw[color=black,fill=black] (x1, 0) circle (1pt);
\node[below] at (x1, -0.1) {$x$};
%
\filldraw[color=black,fill=black] (x1, {fkt(x0)}) circle (1pt);
\filldraw[color=black,fill=black] (x1, {fkt(x1)}) circle (1pt);
%Beschriftung:
\draw[line width=0.8pt, rounded corners=4pt] %
 ({x1 + rT + 0.0}, 1.0) -- ({x1 + rT + 0.2}, 1.0) -- ({x1 + rT + 0.2}, 2.0) %
 -- ({x1 + rT + 0.4}, 2.0);
\draw[line width=0.8pt, rounded corners=4pt] %
 ({x1 + rT + 0.4}, 2.0) %
 -- ({x1 + rT + 0.2}, 2.0) -- ({x1 + rT + 0.2}, 3.0) -- ({x1 + rT + 0.0}, 3.0);
%
\node[right] at ({x1+rT+0.5}, 2) {$\Delta(f) = f(x) - f(x_0)$};
%
\node[left] at (-0.5, {fkt(x0)}) {$f(x_0)$};
\node[left] at (-0.5, {fkt(x1)}) {$f(x)$};
\end{tikzpicture}
\end{small}
}
\end{center}

%----------------------------------------------------------------------------------------

Werden $x_0$ und der entsprechende Funktionswert $f\left(x_0\right)$ festgehalten und eine weitere beliebige, aber variable Stelle $x$ zwischen $a$ und $b$ sowie ihr
Funktionswert $f\left(x\right)$ ausgewählt, so lässt sich 
durch diese beiden Punkte, also durch $\MPointTwoAS{x_0}{f(x_0)}$ und $\MPointTwoAS{x}{f(x)}$,
eine Gerade legen, die durch ihre Steigung und ihren $y$-Achsenabschnitt charakterisiert wird.
Als Steigung dieser Geraden erhält man den sogenannten \textbf{Differenzenquotienten}

\[
\frac{\Delta(f)}{\Delta(x)} = \frac{f(x) - f(x_0)}{x - x_0}\MDFPSpace,
\]

der beschreibt, wie sich die Funktionswerte von $f$\ \textbf{im Mittel} zwischen $x_0$ und $x$ ändern. %###
Damit ist eine mittlere Änderungsrate der Funktion $f$ im Intervall $[x_0\MIntvlSep  x]$ gefunden. Dieser Quotient wird auch als \textbf{relative Änderung} bezeichnet.

Strebt nun die variable Stelle $x$ gegen die Stelle $x_0$, so stellt man fest, dass die Gerade, die den Graphen der Funktion
in den Punkten $\MPointTwoAS{x_0}{f\left(x_0\right)}$ und $\MPointTwoAS{x}{f\left(x\right)}$ schneidet, immer mehr zu einer Tangente
an den Graphen im Punkt $\MPointTwoAS{x_0}{f\left(x_0\right)}$ wird. Auf diese Weise kann die Änderungsrate
der Funktion $f$ - oder die \textbf{Steigung} des Graphen von $f$ - an der Stelle $x_0$\ \textbf{selbst} bestimmt werden. %###
Führt der geschilderte Prozess der Annäherung von $x$ an $x_0$ bildlich gesprochen
auf eine eindeutige Tangente (mit einer eindeutigen Steigung, die insbesondere nicht unendlich sein darf), so spricht man in der Mathematik davon,
dass der \MEntry{Grenzwert}{Grenzwert} des Differenzenquotienten \textbf{existiert}. Beschrieben wird dieser Grenzwertprozess, dass $x$ gegen $x_0$ strebt,
hier und im Folgenden mit dem Symbol
\[
\lim_{x \rightarrow x_0} \MDFPSpace,
\]
wobei $\lim$ abkürzend für \emph{Limes}, das lateinische Wort für Grenze, steht.
Existiert der Grenzwert des Differenzenquotienten, so bezeichnet

\[
f'(x_0) = \lim_{x \rightarrow x_0} \frac{\Delta(f)}{\Delta(x)} 
 = \lim_{x \rightarrow x_0} \frac{f(x) - f(x_0)}{x - x_0} %%
\] 

den Wert der \MEntry{Ableitung}{Ableitung} von $f$ in $x_0$. Die Funktion $f$ ist dann an der Stelle $x_0$\ \textbf{ableitbar} bzw. \textbf{differenzierbar}. %###

\begin{MExample}
  Für $f(x)=\sqrt{x}$ ist die relative Änderung an der Stelle $x_0=1$ gegeben durch
  \[
    \frac{f(x) - f(x_0)}{x - x_0} \;=\;
    \frac{\sqrt{x}-\sqrt{1}}{x-1} \;=\; \frac{\sqrt{x}-1}{(\sqrt{x}-1)(\sqrt{x}+1)} \;=\; \frac1{\sqrt{x}+1} \MDFPeriod
  \]
  Bewegt sich nun $x$ auf $x_0=1$ zu, so resultiert der Grenzwert
  $$
  \lim_{x \rightarrow x_0} \frac{\Delta(f)}{\Delta(x)} \;=\; \frac12 \MDFPeriod
  $$
  Für den Wert der Ableitung von $f$ an der Stelle $x_0=1$ schreibt man $f'(1)=\frac12$.
\end{MExample}

\begin{MExercise}
  Es sei $f: \R \rightarrow \R$ mit $x \mapsto f(x)=x^2$ und $x_0=1$. In diesem Punkt beträgt die relative Änderung für ein reelles $x$
  
  \MEquationItem{$\displaystyle \frac{f(x) - f(1)}{x - 1}$}{\MLFunctionQuestion{8}{x+1}{4}{x}{4}{OR30}}.\\
  \MInputHint{Rechnen Sie den Quotienten direkt aus, ohne bekannte Ableitungswerte und Regeln aus der Schule einzusetzen.}
 
  Bewegt sich $x$ auf $x_0=1$ zu, so erhält man die Steigung \MLParsedQuestion{5}{2}{5}{DG10} des Graphen von $f$ an der Stelle $x_0=1$.

  \begin{MHint}{\iSolution}
  Für $f(x)=x^2$ ist die relative Änderung an der Stelle $x_0=1$ gegeben durch
  \[
    \frac{f(x) - f(1)}{x - 1} \;=\;
    \frac{x^2-1}{x-1} \;=\; \frac{(x-1)(x+1)}{x-1} \;=\; x+1 \MDFPeriod
  \]
  Bewegt sich nun $x$ auf $x_0$ zu, so führt dies auf den Grenzwert
  $$
  \lim_{x \rightarrow 1} \frac{\Delta(f)}{\Delta(x)} \;=\; 2 \MDFPeriod
  $$
  Das ist die Steigung der Tangente an den Graphen von $f$ im Punkt $\MPointTwo{x_0}{f(x_0)} = \MPointTwo{1}{1}$.
  Für den Wert der Ableitung von $f$ an der Stelle $x_0=1$ schreibt man $f'(1)=2$.
  \end{MHint}  
\end{MExercise}

Über die Formel für die relative Änderungsrate kann man die Ableitung nur sehr mühsam und
auch nur für sehr einfache Funktionen ausrechnen. Typischerweise bestimmt man die Ableitung
durch Anwenden von Rechenregeln und durch Einsetzen bekannter Ableitungswerte für die einzelnen
Bausteine.

\end{MXContent}


%\MSubsubsection{Ableitung}
\begin{MXContent}{Ableitung}{Ableitung}{STD}
\MDeclareSiteUXID{VBKM07_Ableitung_Ableitung}

\begin{MXInfo}{Schreibweisen der Ableitung}
In der Mathematik sowie auch in den Natur- und Ingenieurwissenschaften werden verschiedene Schreibweisen der Ableitung äquivalent verwendet:
\[
f'(x_0) = \frac{\MD f}{\MD x}(x_0) = \frac{\MD}{\MD x}f(x_0) \MDFPeriod %%
\]
Diese Schreibweisen haben jeweils die Bedeutung der Ableitung der Funktion $f$ an der Stelle $x_0$.
\end{MXInfo}

Wenn die Ableitung mithilfe des Differenzenquotienten $\frac{f(x) - f(x_0)}{x - x_0}$ berechnet werden muss,
bietet es sich oft an, den Differenzenquotienten anders aufzuschreiben. Verwendet man die Differenz von $x$ und $x_0$ und bezeichnet sie als $h := x - x_0$, 

%Bild:
\begin{center}
\MTikzAuto{%
\begin{small}
\begin{tikzpicture}[line width=1.5pt,scale=0.7, %
declare function={
  x0 = 0;
  x1 = 4;
}
] %[every node/.style={fill=white}] 
%Koordinatenachsen:
\draw[->] (-0.6, 0) -- ({x1+1}, 0); % node[below left]{$x$}; %x-Achse
%Achsenbeschriftung:
\foreach \x in {0, 1, 2, 3, 4} \draw (\x, 0.0) -- ++(0, -0.1); %
\foreach \x in {0, 4} \draw (\x, 0.0) -- ++(0, +0.08); %
% node[below] {$\x$}; 
%Beschriftung:
%\draw[line width=0.8pt, rounded corners=4pt] %
% (0, 0.3) -- (0, 0.4) -- (2, 0.4) -- (2, 0.5);
\draw[line width=0.8pt, rounded corners=4pt] %
 ({x0}, 0.3) -- ({x0}, 0.4) -- ({x1/2}, 0.4) %
 -- ({x1/2}, 0.5);
\draw[line width=0.8pt, rounded corners=4pt] %
 ({x1/2}, 0.5)
 -- ({x1/2}, 0.4) -- ({x1}, 0.4) -- ({x1}, 0.3);
%
\node[above] at ({x1/2}, 0.6) {$h = x - x_0$};
%
\node[below] at ({x0}, -0.3) {$x_0$};
\node[below] at ({x1}, -0.3) {$x$};
\end{tikzpicture}
\end{small}
}
\end{center}

kann der Differenzenquotient mit $x = x_0 + h$ umgeschrieben werden zu
\[
\frac{f(x) - f(x_0)}{x - x_0} = \frac{f(x_0 + h) - f(x_0)}{h} \MDFPeriod
\]
Es wurde keine Voraussetzung darüber getroffen, ob $x$ größer oder kleiner als $x_0$ ist.
Die Größe $h$ kann daher positive oder negative Werte annehmen. Um die Ableitung der Funktion $f$ zu bestimmen, muss nun der Grenzwert für $h \rightarrow 0$ berechnet werden: 
\[
f'(x_0) = \lim_{x \rightarrow x_0} \frac{f(x) - f(x_0)}{x - x_0} %
 = \lim_{h \rightarrow 0} \frac{f(x_0 + h) - f(x_0)}{h} \MDFPeriod
\]
Wenn dieser Grenzwert \textbf{für alle} Stellen $x_0$ aus dem Definitionsbereich einer Funktion existiert, so nennt man diese Funktion (insgesamt)
\textbf{differenzierbar}. Viele der häufig benutzten Funktionen sind differenzierbar.
Ein einfaches Beispiel dafür, dass eine Funktion nicht unbedingt differenzierbar ist,
ist die Betragsfunktion $f: \R \rightarrow \R$ mit $x \mapsto f(x) := |x|$.

\begin{MExample}
Die Betragsfunktion (siehe Modul \MRef{VBKM06}, Abschnitt \MRef{VBKM06_sec:betrag}) ist an der Stelle $x_0 = 0$ nicht differenzierbar.
Der Differenzenquotient für $f$ an der Stelle $x_0 = 0$ lautet:
\[
\frac{f(0+h) - f(0)}{h} = \frac{|h| - |0|}{h} = \frac{|h|}{h} \MDFPeriod
\]
Da $h$ größer oder kleiner als $0$ sein kann, sind zwei Fälle zu unterscheiden:
Im Fall $h > 0$ ist $\frac{|h|}{h} = \frac{h}{h} = 1$, im Fall $h < 0$ erhält man $\frac{|h|}{h} = \frac{-h}{h} = -1$.
Der Grenzwertprozess, dass $h$ sich $0$ nähert, führt in den beiden Fällen also auf zwei verschiedene Ergebnisse ($1$ und $-1$).
Daher existiert \textbf{der} Grenzwert des Differenzenquotienten an der Stelle $x_0 = 0$ nicht. Als Folge davon ist die Betragsfunktion an der Stelle $x_0 = 0$ nicht differenzierbar.

Der Verlauf des Graphen ändert seine Richtung im Punkt $\MPointTwo{0}{0}$ sprunghaft: Salopp ausgedrückt, weist der Funktionsgraph im Punkt $\MPointTwo{0}{0}$ einen Knick auf.
%Bild:
\begin{center}
\MTikzAuto{%
\begin{tikzpicture}[line width=1.5pt,scale=1.0]
\draw[->] (-3.6, 0) -- (4, 0) node[below left]{$x$}; %x-Achse
\draw[->] (0, -0.6) -- (0, 4.6) node[below left]{$y$}; %y-Achse
%Achsenbeschriftung:
\foreach \x in {-3, -2, -1, 1, 2, 3} \draw (\x, 0) -- ++(0, -0.1) %
 node[below] {$\x$};
\foreach \y in {1, 2, 3, 4} \draw (0, \y) -- ++(-0.1, 0) node[left] {$\y$};
%\node[below left] at (0, 0) {$0$};
%Funktion:
\draw[domain=-3.2:3.2,samples=120,color=\jccolorfkt] %
 plot (\x, {abs(\x)});
%Tangenten im Nullpunkt, wenn $f$ für $x \leq 0$ bzw. $x \geq 0$ betrachtet 
%wird:
\draw[samples=120,color=blue!50!black] %
 (-0.5, 0.5) -- (0.5, -0.5);
\draw[samples=120,color=blue!50!black] %
 (-0.5, -0.5) -- (0.5, 0.5);
%Punkt: Markierung der Stelle $0$:
\filldraw[color=black,fill=black] (0, 0) circle (1pt);
%end of file
\end{tikzpicture}
}
\end{center}
\end{MExample}

Auch wenn eine Funktion eine Sprungstelle hat, gibt es keine eindeutige Tangente an den Graphen und somit keine Ableitung.

\end{MXContent}

\begin{MExercises}
\MDeclareSiteUXID{VBKM07_Ableitung_Exercises}

\begin{MExercise}
Berechnen Sie mittels Differenzenquotient die Ableitung von 
%$f(x) := |3 x|$ für $x_1 := 5$ und für $x_2 := -4$.
$f: \R \rightarrow \R$, $x \mapsto f(x) := 4 - x^2$ an den Stellen $x_1 = -2$ und $x_2 = 1$.

Antwort: 
\begin{MExerciseItems}
\item Der Differenzenquotient von $f$ %zu $x$
an der Stelle $x_1 = -2$ ist
\MLSimplifyQuestion{30}{2 - x}{10}{x}{4}{0}{SIMPLE4}
und hat für $x \rightarrow -2$ den Grenzwert
 $f'(-2) = $\MLParsedQuestion{8}{4}{3}{DG11}.
\item Der Differenzenquotient von $f$ %zu $x$
an der Stelle $x_2 = 1$ ist
\MLFunctionQuestion{30}{-1 - x}{10}{x}{4}{DG13X}
und hat für $x \rightarrow 1$ den Grenzwert
 $f'(1) = $\MLParsedQuestion{8}{-2}{3}{DG12}.
\end{MExerciseItems}
\begin{MHint}{\iSolution}
 \begin{MExerciseItems}
  \item An der Stelle $x_1 = -2$ gilt für den Differenzenquotienten
  \[
  \frac{\Delta (f)}{\Delta (x)} = \frac{f(x) - f(x_1)}{x - x_1} = \frac{f(x) - f(-2)}{x - (-2)} =
  \frac{4 - x^2 - 0}{x + 2} = \frac{(2-x)(2+x)}{2+x} = 2-x \MDFPeriod
  \]
  Für $x \rightarrow x_1$, also $x \rightarrow -2$, strebt dieser Differenzenquotient gegen $2 - (-2) = 4$; daher $f'(-2) = 4$.
  \item An der Stelle $x_2 = 1$ gilt für den Differenzenquotienten
  \[
  \frac{\Delta (f)}{\Delta (x)} = \frac{f(x) - f(x_2)}{x - x_2} = \frac{f(x) - f(1)}{x - 1} =
  \frac{4 - x^2 - 3}{x - 1} = \frac{1 - x^2}{x - 1} = - \frac{(x-1)(x+1)}{x-1} = - x - 1 \MDFPeriod
  \]
  Für $x \rightarrow x_2$, also $x \rightarrow 1$, besitzt dieser Differenzenquotient den Grenzwert $- 1 - 1 = -2$; daher $f'(1) = - 2$.
 \end{MExerciseItems}
\end{MHint}
\end{MExercise}


\begin{MExercise}
Erläutern Sie, warum
\begin{MExerciseItems}
\item $f: [-3 \MIntvlSep \infty \MoIr \rightarrow \R$ mit $f(x) := \sqrt{x+3}$ in $x_0 = -3$ und
\item $g: \R \rightarrow \R$ mit $g(x) := 6 \cdot |2 x - 10|$ in $x_0 = 5$
\end{MExerciseItems}
nicht differenzierbar sind.

Antwort:
%Die Tangente an den Funktionsgraphen der stetigen Funktion
\begin{MExerciseItems}
\item Die Ableitung von $f$ existiert an der Stelle $x_0 = -3$ nicht, 
da der Differenzenquotient 
\MLFunctionQuestion{30}{1/sqrt(h)}{1}{h}{20}{DG13}
für $h \rightarrow 0$ nicht konvergiert.
% in $x_0 = -3$ verläuft parallel zur $y$-Achse,
\item Die Ableitung von $g$ existiert an der Stelle $x_0 = 5$ nicht, da der 
Differenzenquotient für $h < 0$ den Wert 
\MLParsedQuestion{8}{-12}{3}{DG14} und für $h > 0$ den 
Wert {\MLParsedQuestion{8}{12}{3}{DG15}} hat. Somit existiert der Grenzwert für 
$h \rightarrow 0$ nicht.
%Für $x < 5$ haben alle Tangenten die Steigung $-12$, 
%und für $x > 5$ die Steigung $21$.
\end{MExerciseItems}
\begin{MHint}{\iSolution}
 \begin{MExerciseItems}
  \item Der Differenzenquotient von $f$ an der Stelle $x_0 = - 3$ ist
  \[
  \frac{\Delta (f)}{\Delta (x)} = \frac{f(x_0 + h) - f(x_0)}{h} = \frac{\sqrt{-3 + h + 3} - \sqrt{-3 + 3}}{h}
  = \frac{\sqrt{h} - 0}{h} = \frac{1}{\sqrt{h}} \MDFPeriod
  \]
  Für $h \rightarrow 0$ ($h > 0$) wächst dieser Differenzenquotient über alle Ma{\ss}en, d.h. der Grenzwert des Differenzenquotienten existiert nicht.
  \item Der Differenzenquotient von $g$ an der Stelle $x_0 = 5$ ist
  \[
  \frac{\Delta (g)}{\Delta (x)} = \frac{g(x_0 + h) - g(x_0)}{h} = \frac{6 \cdot |2 (5+h) - 10| - 6 \cdot |2 \cdot 5 - 10|}{h}
  = \frac{12 |h| - 0}{h} = \frac{12 |h|}{h} \MDFPeriod
  \]
  Für $h<0$ hat der Differenzenquotient wegen $|h| = -h$ also den Wert $-12$, für $h>0$ dagegen wegen $|h|=h$ den Wert $12$. Somit existiert kein Grenzwert
  des Differenzenquotienten. (Ein Grenzwert ist immer eindeutig.)
 \end{MExerciseItems}
\end{MHint}
\end{MExercise}

%\begin{MExercise}
%ei $x_0 \in \R$ mit $x_0 \neq 0$. Berechnen Sie mittels Differenzenquotienten 
%ie Ableitung von $f(x) := \frac{1}{x}$ an der Stelle $x_0$.
%end{MExercise}

\end{MExercises}



%%%Abschnitt
%\MSubsection{Ableitungen elementarer Funktionen}\MLabel{M07_Ableitung_elementareFunktionen}
\MSubsection{Standardableitungen}\MLabel{M07_Standardableitungen}

\begin{MIntro}
\MDeclareSiteUXID{VBKM07_StdAbleitungen_Intro}
Die meisten bekannten Funktionen, wie z.B. Polynome, trigonometrische Funktionen und Exponentialfunktionen (siehe Modul \MRef{VBKM06}),
sind differenzierbar. Im Folgenden werden die Ableitungsregeln für diese Funktionen wiederholt.
\end{MIntro}


\begin{MXContent}{Ableitung von Potenzfunktionen}{Ableitung von Potenzfunktionen}{STD}
\MDeclareSiteUXID{VBKM07_StdAbleitungen_Polynome}
 
Aus der Einführung der Ableitung als Grenzwert des Differenzenquotienten ergibt sich für eine affin lineare Funktion
(siehe Modul \MRef{VBKM06}, Abschnitt \MRef{VBKM06_sec:linear-affin})
$f: \R \rightarrow \R$, $x \mapsto f\left(x\right) = m x + b$, wobei $m$ und $b$ gegebene Zahlen sind, dass der Wert der Ableitung von $f$
an der Stelle $x_0$ gleich $f'(x_0) = m$ ist. (Die geneigte Leserin, der geneigte Leser möge dies gerne selbst überprüfen!)

Für Monome $x^n$ mit $n \geq 1$ ist es am einfachsten,
die Ableitung über den Differenzenquotienten zu bestimmen. Ohne hier eine detaillierte Rechnung oder einen Beweis anzugeben,
erhält man die folgenden Aussagen:

\begin{MXInfo}{Ableitung von $x^n$}
Gegeben sind eine natürliche Zahl $n$ und eine reelle Zahl $r$.

Die konstante Funktion $f: \R \rightarrow \R$ mit $x \mapsto f(x) := r = r \cdot x^0$ besitzt die Ableitung
\[
f': \R \rightarrow \R \MDFPSpace \mbox{ mit } \MDFPSpace x \mapsto f'(x) = 0\MDFPeriod %%
\] %###
Die Funktion $f: \R \rightarrow \R$ mit $x \mapsto f(x) := r \cdot x^n$ besitzt die Ableitung 
\[
f': \R \rightarrow \R \MDFPSpace \mbox{ mit } \MDFPSpace x \mapsto f'(x) = r \cdot n \cdot x^{n-1} \MDFPeriod %%
\]
Diese Ableitungsregel gilt im Übrigen auch für $n\in\mathbb{R}\setminus\{0\}$.
\end{MXInfo}

Auch die Überprüfung dieser Aussagen sei dem Selbststudium überlassen!

\begin{MExample}
Die folgende Untersuchung gilt der Funktion $f: \R \rightarrow \R$ mit $x \mapsto f(x) = 5 x^3$.
Der Vergleich der gegebenen Funktion mit den oben verwendeten Bezeichnungen ergibt $r = 5$ und $n = 3$. Damit erhält man für den Wert der 
Ableitung an der Stelle $x$
\[
f'(x) = 5 \cdot 3 x^{3 - 1} = 15 x^2 \MDFPeriod %%
\]
\end{MExample}


Für Wurzelfunktionen ergibt sich eine entsprechende Aussage. Allerdings 
ist zu beachten, dass Wurzelfunktionen nur für $x > 0$ differenzierbar sind.
Denn die Tangente an den Funktionsgraphen durch den Punkt $\MPointTwo{0}{0}$ verläuft 
parallel zur $y$-Achse und ist somit kein Schaubild einer Funktion. 

\begin{MXInfo}{Ableitung von $x^{\frac{1}{n}}$}
Für $n \in \Z$ mit $n \neq 0$ ist die Funktion
$f: [0 \MIntvlSep \infty \MoIr \rightarrow \R$, $x \mapsto f(x) := x^{\frac{1}{n}}$ für $x>0$ differenzierbar, und es gilt
\[
f': \MoIl 0 \MIntvlSep \infty \MoIr \rightarrow \R \MDFPSpace , \MDFPaSpace x \mapsto f'(x) = \frac{1}{n} \cdot x^{\frac{1}{n}-1} \MDFPeriod
\]
\end{MXInfo}

Für $n\in\mathbb{N}$ werden durch $f(x) = x^{\frac{1}{n}}$ Wurzelfunktionen beschrieben. Die hier wiedergegebene Ableitungsregel gilt natürlich auch für $n = 1$ oder $n = -1$, da sie für ganze Zahlen gültig ist. %###
\begin{MExample}
Die Wurzelfunktion $f: [0\MIntvlSep \infty\MoIr \rightarrow \R$ mit
$x \mapsto f(x) := \sqrt{x} = x^{\frac{1}{2}}$
ist für $x > 0$ differenzierbar. Der Wert der Ableitung an einer beliebigen Stelle $x>0$ ist durch
\[
f'(x) = \frac{1}{2} \cdot x^{\frac{1}{2}-1} %
= \frac{1}{2} \cdot x^{-\frac{1}{2}} = \frac{1}{2 \cdot \sqrt{x}} %%
\] 
gegeben. Die Ableitung in $x_0 = 0$ existiert nicht, da die Tangente an den Graphen von $f$ dort eine unendliche Steigung hätte.

\begin{center}
\MTikzAuto{%
%{Graph von $\sqrt{x}$ mit Tangente in $x_0 = 1$}{}
\begin{small}
\begin{tikzpicture}[line width=1.5pt,scale=0.8]
%Koordinatenachsen:
\draw[->] (-0.6, 0) -- (4.8, 0) node[below left]{$x$}; %x-Achse
\draw[->] (0, -0.6) -- (0, 3) node[below left]{$y$}; %y-Achse
%Achsenbeschriftung:
\foreach \x in {1, 2, 3, 4} \draw (\x, 0) -- ++(0, -0.1) %
 node[below] {$\x$};
\foreach \y in {1, 2} \draw (0, \y) -- ++(-0.1, 0) node[left] {$\y$};
%\node[below left] at (0, 0) {$0$};
%Funktion:
\draw[domain=0:4,samples=120,color=\jccolorfkt] %
 plot (\x, {sqrt(\x)});
%Tangenten y = f(x_0) + f'(x_0) * (x - x_0) im Punkt $(x_0, f(x_0)$ des Graphen:
%$x_0 := 1$:
\draw[samples=120,color=blue!50!black] %
({-1/2}, {1/4}) -- ++({7/2}, {7/4});
%Punkt:
\filldraw[color=black, fill=black] (1, 0) circle (2pt);
%end of file
\end{tikzpicture}
\end{small}
}
\end{center}

Die Tangente im Punkt $\MPointTwo{1}{1}$ an den Graphen der vorliegenden Wurzelfunktion weist 
die Steigung $\frac{1}{2 \sqrt{1}} = \frac{1}{2}$ auf.
\end{MExample}

Die bisherigen Aussagen können für $x>0$ auf Exponenten $p \in \R$ mit 
$p \neq 0$ ausgedehnt werden:
Der Wert $f'(x)$ der Ableitung einer Funktion $f$ mit Funktionsterm $f(x) = x^p$ für $x > 0$ ist
\[
f'(x) = p \cdot x^{p-1} \MDFPeriod %%
\]
\end{MXContent}

\begin{MXContent}{Ableitung spezieller Funktionen}{Ableitung spezieller Funktionen}{STD}
\MDeclareSiteUXID{VBKM07_SpezielleFunktionen}

\MSubsubsectionx{Ableitung trigonometrischer Funktionen}

Die Sinusfunktion $f: \R \rightarrow \R$, $x \mapsto f(x) = \sin(x)$ ist periodisch mit Periode $2 \pi$. Somit genügt es, die
Funktion auf einem Intervall der Länge $2 \pi$ zu betrachten. Einen Ausschnitt 
des Graphen für $-\pi \leq x \leq \pi$ zeigt die folgende Abbildung:

\begin{center}
\MTikzAuto{%
\begin{small}
\begin{tikzpicture}[line width=1.5pt,scale=1.0, %
declare function={
  fkt(\x) = sin(\x r);
  fktabl(\x) = cos(\x r);
}
] %[every node/.style={fill=white}] 
%
%Graph der Sinusfunktion:
\node[right] at (-6,1) {Sinusfunktion};
\begin{scope}%[xshift=-6]
%Koordinatenachsen:
\draw[->] (-3.6, 0) -- (4, 0) node[below left]{$x$}; %x-Achse
\draw[->] (0, -1.6) -- (0, 1.6) node[below left]{$y$}; %y-Achse
%Achsenbeschriftung:
\foreach \x in {{-pi}, {-pi/2}} \draw (\x, 0) -- ++(0, -0.1);
\node[below] at ({-pi}, 0) {$-\pi$};
\node[below] at ({-pi/2}, 0) {$-\frac{\pi}{2}$};
\foreach \x in {{pi/2}, pi} \draw (\x, 0) -- ++(0, -0.1);
\node[below] at ({pi/2}, 0) {$-\frac{\pi}{2}$};
\node[below] at ({pi}, 0) {$-\pi$};
\foreach \y in {-1} \draw (0, \y) -- ++(-0.1, 0) %
 node[left] {$\y$};
\foreach \y in {1} \draw (0, \y) -- ++(-0.1, 0) %
 node[left] {$\y$};
%\node[below left] at (0, 0) {$0$};
%Funktion:
\draw[domain=-3.14:3.14,samples=120,color=\jccolorfkt] %
 plot (\x, {fkt(\x)});
%Tangenten in verschiedenen Punkten:
\draw[samples=120,color=blue!50!black] %
 plot (-0.5,-0.5) -- (0.5,0.5);
\draw[samples=120,color=blue!50!black] %
 plot ({pi/3},1) -- ({2*pi/3},1);
\node[above] at ({pi/2}, 1) {$\sin'(\pi/2) = 0$};
\draw[samples=120,color=blue!50!black] %
 plot ({-2*pi/3},-1) -- ({-pi/3},-1);
\node[below] at ({-pi/2}, -1) {$\sin'(-\pi/2) = 0$};
\end{scope}
\node[right] at (-6,-2.8) {Ableitung};
\begin{scope}[yshift=-3.8cm]
%Koordinatenachsen:
\draw[->] (-3.6, 0) -- (4, 0) node[below left]{$x$}; %x-Achse
\draw[->] (0, -1.6) -- (0, 1.6) node[below left]{$y$}; %y-Achse
%Achsenbeschriftung:
\foreach \x in {{-pi}, {-pi/2}} \draw (\x, 0) -- ++(0, 0.1); 
\node[below] at ({-pi}, 0) {$-\pi$}; 
\node[below] at ({-pi/2}, 0) {$-\frac{\pi}{2}$}; 
\foreach \x in {{pi/2}, {pi}} \draw (\x, 0) -- ++(0, -0.1);
\node[below] at ({pi/2}, 0) {$\frac{\pi}{2}$}; 
\node[below] at ({pi}, 0) {$\pi$}; 
\foreach \y in {-1} \draw (0, \y) -- ++(0.1, 0);
\foreach \y in {1} \draw (0, \y) -- ++(-0.1, 0) %
 node[below left] {$\y$};
%\node[below left] at (0, 0) {$0$};
%Funktion:
\draw[domain=-3.14:3.14,samples=120,color=blue!50!white] %
 plot (\x, {fktabl(\x)});
%Punkte
\filldraw[color=black,fill=black] ({-pi}, -1) circle (2pt);
\filldraw[color=black,fill=black] ({-pi/2}, 0) circle (2pt);
\filldraw[color=black,fill=black] (0, 1) circle (2pt);
\filldraw[color=black,fill=black] ({pi/2}, 0) circle (2pt);
\filldraw[color=black,fill=black] ({pi}, -1) circle (2pt);
\end{scope}
\end{tikzpicture}
\end{small}
}
\end{center}

Wie in der Abbildung zu sehen, ist die Steigung des Sinus bei $x_0 = \pm\frac{\pi}{2}$ gerade $f'(\pm\frac{\pi}{2}) = 0$.
Legt man eine Tangente an der Stelle $x_0 = 0$ an den Graphen der Sinusfunktion, erhält man als deren Steigung $f'(0) = 1$.
Untersucht man die Stellen $x_0 = \pm\pi$, so findet man, dass dort die gleiche Steigung wie bei $x_0 = 0$, aber mit umgedrehtem Vorzeichen, vorliegt.
Die Steigung ist dort also $f'(\pm\pi) = -1$. Die Ableitung des Sinus ist also eine Funktion, die genau diese Eigenschaften erfüllt.
Eine genaue Untersuchung der Bereiche zwischen diesen speziell ausgesuchten Stellen ergibt, dass der Kosinus die Ableitung des Sinus darstellt:

\begin{MXInfo}{Ableitung trigonometrischer Funktionen}
Für die Sinusfunktion $f: \R \rightarrow \R$, $x \mapsto f(x) := \sin(x)$ gilt 
\[
f': \R \rightarrow \R \MDFPSpace , \MDFPaSpace x \mapsto f'(x) = \cos(x) \MDFPeriod %%
\]
Für die Kosinusfunktion $g: \R \rightarrow \R$, $x  \mapsto g(x) := \cos(x)$ gilt 
\[
g': \R \rightarrow \R \MDFPSpace , \MDFPaSpace x \mapsto g'(x) = -\sin(x) \MDFPeriod %%
\]
Für die Tangensfunktion $h: \R \setminus \{ \frac{\pi}{2} + k \pi \MCondSetSep k \in \Z \} \rightarrow \R$,
$x \mapsto h(x) := \tan(x)$ gilt
\[
h': \R \setminus \{ \frac{\pi}{2} + k \pi \MCondSetSep k \in \Z \} \rightarrow \R \MDFPSpace , \MDFPaSpace
x \mapsto h'(x) = 1 + (\tan(x))^2 = \frac{1}{\cos^2(x)} \MDFPeriod %%
\]
\end{MXInfo}
Letzteres ergibt sich auch aus den nachfolgend erläuterten Rechenregeln und der 
Definition des Tangens als Quotient von Sinus und Kosinus.

\MSubsubsectionx{Ableitung der Exponentialfunktion}

\begin{MInfo}
Die Exponentialfunktion $f: \R \rightarrow \R$, $x \mapsto f(x) := \MEU^{x} = \exp(x)$ hat die besondere Eigenschaft, 
dass ihre Ableitung $f'$ wiederum die Exponentialfunktion ist, also $f'(x) = \MEU^{x} = \exp(x)$ gilt.
\end{MInfo}

\MSubsubsectionx{Ableitung der Logarithmusfunktion}

%Die Logarithmusfunktion $f(y) := \ln(y)$ für $y > 0$ ist die Umkehrfunktion
%der Exponentialfunktion $y = \MEU^x$. Die Ableitung und damit die Steigung $m$ einer Tangente an den 
%Graphen der Exponentialfunktion ist $m = \MEU^x$. 
%In der folgenden Abbildung ist die Tangente an den Graphen der Exponentialfunktion im
%Punkt $(x_0, \MEU^{x_0})$ für $x_0 = \ln(2)$ angedeutet, sodass die 
%Tangentensteigung $m = 2$ ist. Die Umkehrabbildung der Tangente hat dann 
%die Steigung $\frac{1}{m} = \frac{1}{2}$. 
%Geometrisch ist es die Steigung der gespiegelten Tangente, also die Tangente
%an die Umkehrfunktion $\ln$.
%Verwenden wir wieder die gewohnten Bezeichnungen für die unabhängige Variable, können wir ohne Beweis die Ableitung der Logarithmusfunktion angeben. Es gilt: $f'(x) = \frac{1}{x}$ ist die Ableitung der 
%Logarithmusfunktion $f(x) = \ln(x)$.
Die Ableitung der Logarithmusfunktion wird hier ohne Beweis angegeben. Für $f: \MoIl 0 \MIntvlSep \infty\MoIr  \rightarrow \R$ mit $x \mapsto f(x) = \ln(x)$
erhält man $f': \MoIl 0 \MIntvlSep \infty\MoIr \rightarrow \R$, $x \mapsto f'(x) = \frac1x$.
%\begin{center}
%\MTikzAuto{%
%%%{Tangenten an $\exp$ und an die Umkehrfunktion $\ln$}{}
%\begin{small}
%\begin{tikzpicture}[line width=1.5pt,scale=0.8]
%%%Koordinatenachsen:
%\draw[->] (-3.2, 0) -- (4.6, 0) node[below left]{$x$}; %x-Achse
%\draw[->] (0, -3.2) -- (0, 4) node[below left]{$y$}; %y-Achse
%%%Achsenbeschriftung:
%\foreach \x in {-3, -2, -1} \draw (\x, 0) -- ++(0, -0.1); % node[below] {$\x$};
%\foreach \x in {1, 2, 3} \draw (\x, 0) -- ++(0, -0.1) node[below] {$\x$};
%\foreach \y in {-3, -2, -1} \draw (0, \y) -- ++(-0.1, 0); % node[left] {$\y$};
%\foreach \y in {1, 2, 3} \draw (0, \y) -- ++(-0.1, 0) node[left] {$\y$};
%%%\node[below left] at (0, 0) {$0$};
%%%
%%%Funktion exp:
%\draw[domain=-2.4:1.3,samples=120,color=white!50!black] %\jccolorfkt] %
% plot (\x, {exp(\x)});
%%%Tangenten in verschiedenen Punkten:
%%%x_0 = 0:
%%%\draw[samples=120,color=blue!50!black] %
%%% (-0.45, 0.45) -- (0.55, 1.55);
%%%
%%%x_0 = 1:
%%% ({3/4}, {exp(1) - 1/4 * exp(1)}) -- ++({2/4}, {2/4*exp(1)});
%%%
%%%x_0 = ln(2): y = exp(ln(2)) + 2 * (x - ln(2)) = 2 + 2 * (x - ln(2)):
%\draw[style=dashed,samples=120,color=blue!50!black] %
% ({ln(2) - 3/8}, {5/4}) -- ++({6/8}, {6/8*2});
%%%
%%%Punkte:
%%%\filldraw[color=black] (0, 1) circle (2pt);
%\filldraw[color=black] (0, 2) circle (2pt);
%\filldraw[color=black] ({ln(2)}, 2) circle (2pt);
%%%
%%%Spiegelachse (erste Winkelhalbierende):
%\draw[samples=120,style=dotted,color=white!50!black] %
% (-2.1, -2.1) -- (3.1, 3.1);
%%%
%%%Umkehrfunktion ln:
%\draw[domain=0.1:{exp(1.3},samples=120,color=\jccolorfkt] %
% plot (\x, {ln(\x)});
%%%Tangenten in verschiedenen Punkten:
%%%\draw[samples=120,color=blue!50!black] %
%%% (0.45, -0.45) -- (1.55, 0.55);
%%%
%%%\draw[style=dotted,samples=120,color=blue!50!black] %
%%% ({exp(1) - 1/2}, {1 - 1/2 * 1/exp(1)}) -- ++(1, {1/exp(1)});
%%%\draw[style=dotted,samples=120,color=blue!50!black] %
%%% ({exp(1) - 1/4*exp(1)}, {3/4}) -- ++({2/4*exp(1)}, {2/4});
%%%x_0 = 2: y = ln(2) + 1/2 * (x - 2):
%\draw[samples=120,color=blue!50!black] %
% ({2 - 3/4}, {ln(2) - 1/2 * 3/4}) -- ++({2*3/4}, {2*3/4*1/2});
%%%Punkte:
%%%\filldraw[color=black] (1, 0) circle (2pt);
%\filldraw[color=black] (2, 0) circle (2pt);
%\filldraw[color=black] (2, {ln(2)}) circle (2pt);
%\end{tikzpicture}
%\end{small}
%}
%\end{center}

\end{MXContent}


%%%Uebungen zum Abschnitt:
\begin{MExercises}
\MDeclareSiteUXID{VBKM07_Ableitung_Speziell_Exercises}

\begin{MExercise}
Bestimmen Sie die Ableitung, indem Sie die Funktionsterme vereinfachen und 
dann Ihre Kenntnisse über die Ableitung bekannter Funktionen anwenden
($x > 0$):
\begin{MExerciseItems}
\item $f(x) := x^6 \cdot x^{\frac{7}{2}} = $
\MLSimplifyQuestion{40}{x^(19/2)}{5}{x}{4}{512}{SIMPLE5}. % wird immer noch nicht richtig erkannt
\item $g(x) := \frac{x^{-\frac{3}{2}}}{\sqrt{x}} = $
\MLSimplifyQuestion{40}{1/x^2}{4}{x}{4}{512}{SIMPLE6}.
\end{MExerciseItems}
Damit ist:
\begin{MExerciseItems}
\item $f'(x) = $ \MLSimplifyQuestion{40}{19/2*x^(17/2)}{4}{x}{4}{512}{SIMPLE7}.
\item $g'(x) = $ \MLSimplifyQuestion{40}{-2/x^3}{4}{x}{4}{512}{SIMPLE8}.
\end{MExerciseItems}
\begin{MHint}{\iSolution}
 \begin{MExerciseItems}
  \item Es ist $f(x) = x^6 \cdot x^\frac72 = x^{6 + \frac72} = x^\frac{19}{2}$.
  Damit gilt $f'(x) = \frac{19}{2} x^{\frac{19}{2} - 1} = \frac{19}{2} x^\frac{17}{2}$.
  \item Es ist $g(x) = \frac{x^{-\frac32}}{\sqrt{x}} = x^{- \frac32} \cdot x^{- \frac12} = x^{- \frac32 - \frac12} = x^{-2}$.
  Damit gilt $g'(x) = (-2) \cdot x^{-2 -1} = -2 \cdot x^{-3} = - \frac{2}{x^3}$.
 \end{MExerciseItems}
\end{MHint}
\end{MExercise}

\begin{MExercise}
Vereinfachen Sie die Funktionsterme, um dann die Ableitung zu bestimmen:
\begin{MExerciseItems}
\item
 $f(x) := 2 \sin\left(\frac{x}{2}\right) \cdot \cos\left(\frac{x}{2}\right) = $ %
\MLSimplifyQuestion{40}{sin(x)}{10}{x}{4}{0}{DS1}.
\item $g(x) := \cos^2(3 x) + \sin^2(3 x) = $ %
\MLParsedQuestion{40}{1}{3}{DG16}.
\end{MExerciseItems}
Damit ist:
\begin{MExerciseItems}
\item $f'(x) = $ \MLFunctionQuestion{40}{cos(x)}{10}{x}{4}{DG17}.
\item $g'(x) = $ \MLParsedQuestion{40}{0}{3}{DG18}.
\end{MExerciseItems}
\begin{MHint}{\iSolution}
 \begin{MExerciseItems}
  \item Es ist allgemein
  \[
   \sin (u) \cdot \cos (v) = \frac12 \left( \sin (u-v) + \sin (u+v) \right) \MDFPeriod
  \]
  Im vorliegenden Fall gilt somit $f(x) = 2 \cdot \frac12 ( \sin(0) + \sin(x) ) = \sin (x)$ und daher $f'(x) =  \cos (x)$.
  \item Wegen $\sin^2 (u) + \cos^2 (u) = 1$ gilt $g(x) = 1$ und daher $g'(x) = 0$.
 \end{MExerciseItems}
\end{MHint}
\end{MExercise}

\begin{MExercise}
Vereinfachen Sie die Funktionsterme, um dann die Ableitung zu bestimmen
(für $x > 0$ in der ersten Teilaufgabe):
\begin{MExerciseItems}
\item $f(x) := 3 \ln(x) + \ln\left(\frac{1}{x}\right) = $
\MLSimplifyQuestion{40}{2*ln(x)}{10}{x}{4}{512}{SIMPLE9}.
\item $g(x) := \left(\MEU^x\right)^2 \cdot \MEU^{-x} = $
\MLFunctionQuestion{40}{exp(x)}{10}{x}{4}{DZ10}.
\end{MExerciseItems}
%\MInputHint{Schreiben Sie Exponentialfunktionen mit \texttt{exp}, beispielsweise schreiben Sie $\MEU^{4x^2}$ als \texttt{exp(4*x^2)}.}
Damit ist:
\begin{MExerciseItems}
\item $f'(x) = $ \MLSimplifyQuestion{40}{2/x}{1}{x}{20}{512}{TFR1}.
\item $g'(x) = $ \MLSimplifyQuestion{40}{exp(x)}{10}{x}{4}{0}{TFR2}.
\end{MExerciseItems}
\MInputHint{Schreiben Sie Exponentialfunktionen mit \texttt{exp}, beispielsweise schreiben Sie $\MEU^{4x^2}$ als \texttt{exp(4*x^2)}.}

\begin{MHint}{\iSolution}
 \begin{MExerciseItems}
  \item Es ist
  \[
   f(x) = 3 \ln (x) + \ln \left( \frac1x \right) = \ln (x^3) + \ln \left( \frac1x \right) = \ln \left( x^3 \cdot \frac1x \right)
   = \ln \left( x^2 \right) \MDFPeriod
  \]
  Für den Wert der Ableitung von $f$ an der Stelle $x$ ($x>0$) folgt daher mit Hilfe der Kettenregel
  $f'(x) = \frac{1}{x^2} \cdot 2 x = \frac{2}{x}$. (Die Kettenregel wird im Abschnitt \MRef{Kettenregel} ausführlich erläutert.)
  \item Es ist
  \[
  g(x) = \left( \MEU^x \right)^2 \cdot \MEU^{-x} = \MEU^x \cdot \MEU^x \cdot \MEU^{-x} = \MEU^{x+x-x} = \MEU^x \MDFPeriod
  \]
  Daher folgt $g'(x) = \MEU^x$.
 \end{MExerciseItems}
\end{MHint}
\end{MExercise}

\end{MExercises}



%%%Abschnitt
\MSubsection{Rechenregeln}\MLabel{M07_Rechenregeln}

\begin{MIntro}
\MDeclareSiteUXID{VBKM07_Rechenregeln_Intro}
Zusammen mit einigen wenigen Rechenregeln und den im letzten Abschnitt 
vorgestellten Ableitungen lässt sich eine Vielzahl an Funktionen 
differenzieren.
\end{MIntro}

\begin{MXContent}{Vielfaches und Summe von Funktionen}{Vielfaches und Summe}{STD}
\MDeclareSiteUXID{VBKM07_Rechenregeln_Summen}

Im Folgenden bezeichnen $u, v: D \rightarrow \R$ zwei beliebige differenzierbare Funktionen sowie $r$ eine beliebige reelle
Zahl.


\begin{MXInfo}{Summenregel und Vielfaches von Funktionen}
Sind $u$ und $v$ als differenzierbare Funktionen vorgegeben, dann ist auch die Summe
$f := u+v$ mit $f(x) = (u+v)(x) := u(x) + v(x)$ differenzierbar, und es gilt
$$
f'(x) = u'(x) + v'(x) \MDFPeriod%%
$$

Auch das $r$-fache einer Funktion, also $f := r \cdot u$ mit $f(x) = (r \cdot u)(x) := r \cdot u(x)$ ist differenzierbar,
und es gilt
$$
f'(x) = r \cdot u'(x) \MDFPeriod%%
$$
\end{MXInfo}

Mit diesen beiden Rechenregeln und der Ableitungsregel für Monome $x^n$ lässt sich z.B. jedes beliebige Polynom ableiten.
Es folgen einige Beispiele:

\begin{MExample}
Das Polynom $f$ mit dem Funktionsterm $f(x) = \frac{1}{4} x^3 - 2 x^{2} +5 $ ist differenzierbar, und man erhält
\[
f'(x) = \frac{3}{4} x^2 - 4 x \MDFPeriod%%
\]
Die Ableitung der Funktion $g: \MoIl 0 \MIntvlSep \infty\MoIr \rightarrow \R$ mit $g(x) = x^3 + \ln(x)$ ist
\[
g': \MoIl 0 \MIntvlSep \infty \MoIr \rightarrow \R \MDFPSpace \text{ mit } \MDFPSpace g'(x) = 3 x^2 + \frac{1}{x} = \frac{3 x^3 + 1}{x} \MDFPeriod%%
\]
%Mit $\ln(x^3) = 3 \ln(x)$ ergibt sich die Ableitung von 
%$g(x) = \ln(x^3) = 3 \ln(x)$ für $x > 0$ zu
%\[
%g'(x) = \frac{3}{x} %%
%\]
Ableiten der Funktion
$h: [0 \MIntvlSep \infty\MoIr \rightarrow \R$ mit $h(x) = 4^{-1} \cdot x^2 - \sqrt{x} %
 = \frac{1}{4} x^2 + (-1) \cdot x^{\frac{1}{2}}$ führt für $x>0$ auf
\[
h'(x) = \frac{1}{2} x - \frac{1}{2} x^{-\frac{1}{2}} %
 = \frac{x^{\frac{3}{2}} - 1}{2 \sqrt{x}} \MDFPeriod%%
\]
\end{MExample}

\end{MXContent}

\begin{MXContent}{Produkt und Quotient von Funktionen}{Produkt und Quotient}{STD}
\MDeclareSiteUXID{VBKM07_Produktregel}

\begin{MXInfo}{Produkt- und Quotientenregel}
Auch das Produkt von Funktionen, $f := u \cdot v$ mit $f(x) = (u \cdot v)(x) := u(x) \cdot v(x)$, ist differenzierbar,
und es gilt die \textbf{Produktregel}
$$
f'(x) = u'(x) \cdot v(x) +  u(x) \cdot v'(x) \MDFPeriod%%
$$

Der Quotient von Funktionen, $f := \frac{u}{v}$ mit $f(x) = \left( \frac{u}{v} \right) (x) := \frac{u(x)}{v(x)}$,
ist für alle $x$ mit $v(x) \neq 0$ definiert und differenzierbar, und es gilt die \textbf{Quotientenregel}
\ifttm
$$
f'(x) = \frac{u'(x) \cdot v(x) - u(x) \cdot v'(x)}{\left( v(x) \right)^2} \MDFPeriod%%
$$
\else
$$
f'(x) = %
\frac{u'(x) \cdot v(x) \,\textcolor{red}{\mathbf{-}}\, u(x) \cdot v'(x)}%
{\left(v(x)\right)^2} \MDFPeriod%%
$$
\fi
\end{MXInfo}

Auch diese Rechenregeln sollen anhand einiger Beispiele veranschaulicht werden:

\begin{MExample}
Gesucht ist die Ableitung von $f: \R \rightarrow \R$ mit $f(x) = x^2 \cdot \MEU^x$.
Anwenden der Produktregel mit (z.B.) $u(x) = x^2$ und $v(x) = \MEU^x$ führt auf
$u'(x) = 2x$ und $v'(x) = \MEU^x$. Werden diese Teilergebnisse mit der Produktregel zusammengesetzt, dann resultiert die Ableitung der Funktion $f$:
\[
f': \R \rightarrow \R \MDFPSpace , \MDFPaSpace x \mapsto f'(x) = 2 x \MEU^x + x^2 \MEU^x = (x^2 + 2x) \MEU^x \MDFPeriod%%
\]
Als Nächstes soll die Tangensfunktion $g$ mit $g(x) = \tan(x) = \frac{\sin(x)}{\cos(x)}$ ($\cos(x) \neq 0$) untersucht werden.
Durch Vergleich mit der Quotientenregel liest man $u(x) = \sin(x)$ und $v(x) = \cos(x)$ ab. Mit $u'(x) = \cos(x)$ und $v'(x) = -\sin(x)$
können die Teilergebnisse mit Hilfe der Quotientenregel zur Ableitung der Funktion $g$ zusammengefügt werden; es folgt:
\[
g'(x) = \frac{\cos(x) \cdot \cos(x) - \sin(x) \cdot (-\sin(x))}{\cos^2(x)} \MDFPeriod%
\]
Dieses Ergebnis lässt sich zu einem der folgenden Ausdrücke zusammenfassen:
\[
g'(x) = 1 + \left(\frac{\sin(x)}{\cos(x)}\right)^2 %
 = 1 + \tan^2(x) %
 = \frac{1}{\cos^2(x)} \MDFPeriod%%
\]
Für das letzte Gleichheitszeichen wurde der in Modul \MNRef{VBKM05} (Abschnitt \MRef{Abschnitt:TrigonometrieAmDreieck}) besprochene Zusammenhang
$\sin^2(x) + \cos^2(x) = 1$ verwendet.
\end{MExample}

\begin{MExercise}
Berechnen Sie die Ableitung von $f: \R \rightarrow \R$ mit $f(x)=\sin(x)\cdot x^3$, indem Sie dieses Produkt in zwei Faktoren zerlegen,
die Ableitungen bilden und diese mit Hilfe der Produktregel zusammensetzen:
\begin{MExerciseItems}
\item{Der linke Faktor \MEquationItem{$u(x)$}{\MLFunctionQuestion{10}{sin(x)}{5}{x}{5}{DS2}} führt auf \MEquationItem{$u'(x)$}{\MLFunctionQuestion{10}{cos(x)}{5}{x}{5}{DS5}}.}
\item{Der rechte Faktor \MEquationItem{$v(x)$}{\MLFunctionQuestion{10}{x^3}{5}{x}{5}{DS3}} führt auf \MEquationItem{$v'(x)$}{\MLFunctionQuestion{10}{3*x^2}{5}{x}{5}{DS4}}.}
\item{Für das Produkt $f$ gilt daher \MEquationItem{$f'(x)$}{\MLFunctionQuestion{30}{(cos(x)*x^3)+(sin(x)*3*x^2)}{5}{x}{5}{DS6}}.}
\end{MExerciseItems}
\begin{MHint}{\iSolution}
Die vier Bausteine sind
$$
u(x) \;=\; \sin(x) \MDFPSpace , \MDFPaSpace
u'(x) \;=\; \cos(x) \MDFPSpace , \MDFPaSpace
v(x) \;=\; x^3 \MDFPSpace , \MDFPaSpace
v'(x) \;=\; 3x^2 \MDFPSpace ,
$$
und Zusammensetzen mit der Produktregel ergibt
$$
f'(x) \;=\; \cos(x)\cdot x^3+\sin(x)\cdot 3x^2 \MDFPeriod
$$
\end{MHint}
\end{MExercise}

\begin{MExercise}
Berechnen Sie die Ableitung von $f: \MoIl 0 \MIntvlSep \infty\MoIr \rightarrow \R$ mit $f(x)=\frac{\ln(x)}{x^2}$, indem Sie
diesen Quotienten in Zähler und Nenner zerlegen, die Ableitungen bilden
und diese mit Hilfe der Quotientenregel zusammensetzen:
\begin{MExerciseItems}
\item{Der Zähler \MEquationItem{$u(x)$}{\MLSimplifyQuestion{10}{ln(x)}{5}{x}{5}{512}{DS7}} führt auf \MEquationItem{$u'(x)$}{\MLSimplifyQuestion{10}{1/x}{5}{x}{5}{512}{DS8}}.}
\item{Der Nenner \MEquationItem{$v(x)$}{\MLFunctionQuestion{10}{x^2}{5}{x}{5}{DZ2}} führt auf \MEquationItem{$v'(x)$}{\MLFunctionQuestion{10}{2*x}{5}{x}{5}{DZ1}}.}
\item{Für den Quotienten $f$ gilt \MEquationItem{$f'(x)$}{\MLSimplifyQuestion{30}{(1/x*x^2-ln(x)*2*x)/(x^4)}{5}{x}{5}{512}{DS9}}.}
\end{MExerciseItems}
\begin{MHint}{\iSolution}
Die vier Bausteine sind
$$
u(x) \;=\; \ln(x) \MDFPSpace , \MDFPaSpace
u'(x) \;=\; \frac1x \MDFPSpace , \MDFPaSpace
v(x) \;=\; x^2 \MDFPSpace , \MDFPaSpace
v'(x) \;=\; 2x \MDFPSpace ,
$$
und Zusammensetzen mit der Quotientenregel ergibt
$$
f'(x) \;=\; \frac{\frac1x\cdot x^2-\ln(x)\cdot 2x}{x^4} \; =\; \frac{1-2\ln(x)}{x^3} \MDFPSpace,
$$
wobei der letzte Umformungsschritt (Kürzen um $x$) nur der Vereinfachung dient.
\end{MHint}
\end{MExercise}
\end{MXContent}

\begin{MXContent}{Verkettung von Funktionen}{Verkettung}{STD}
\MDeclareSiteUXID{VBKM07_Verkettung}
\MLabel{Kettenregel}

Zum Abschluss wird die Verkettung (Modul \MRef{VBKM06}, Abschnitt \MRef{Verkettung}) von Funktionen untersucht:
Was passiert, wenn eine Funktion $u$ (die innere Funktion) in eine andere Funktion $v$
(die äußere Funktion) sozusagen eingesetzt wird? Eine solche Verkettung wird in der Mathematik durch die Schreibweise $f := v \circ u$ mit
$f(x) = (v \circ u)(x) := v(u(x))$ dargestellt. Dies bedeutet, dass zunächst der Wert einer Funktion $u$ in Abhängigkeit von der Variable $x$ bestimmt wird.
Dieser so berechnete Wert $u(x)$  wird dann als Argument der Funktion $v$ verwendet. Auf diese Weise entsteht der endgültige Funktionswert $v(u(x))$.

\begin{MXInfo}{Kettenregel}
Die Ableitung der Funktion $f := v \circ u$ mit $f(x) = (v \circ u)(x) := v(u(x))$ kann mit der \textbf{Kettenregel} bestimmt werden; es gilt:
$$
f'(x) = v'(u(x)) \cdot u'(x) \MDFPeriod%%
$$
Hierbei ist $v'(u(x))$ so zu verstehen, dass man $v$ als Funktion von $u$ auffasst und dementsprechend nach $u$ differenziert; anschließend wertet man
$v'(u)$ für $u = u(x)$ aus.\\
Hilfreich ist der Merksatz: Ableitung der Verkettung ist äußere Ableitung mal innere Ableitung.
\end{MXInfo}

Diese Ableitungsregel soll anhand einiger Beispiele verdeutlicht werden:

\begin{MExample}
Gesucht wird die Ableitung der Funktion $f: \R \rightarrow \R$ mit $f(x) = (3 - 2 x)^5$.
Soll die Kettenregel angewendet werden, sind eine innere und eine äußere Funktion zu identifizieren.
Setzt man für die innere Funktion $u$ den Funktionsterm $u(x) = 3 - 2x$, dann ist die äußere Funktion $v$ durch $v(u) = u^5$ gegeben.
Damit gilt wie verlangt $v(u(x)) = f(x)$.

Ableiten der inneren Funktion $u$ nach $x$ liefert $u'(x) = -2$.
Für die äußere Ableitung differenziert man $v$ nach $u$ und findet $v'(u) = 5 u^4$.
Einsetzen dieser Zwischenergebnisse in die Kettenregel führt auf die Ableitung $f'$ der Funktion $f$ mit:
\[
f'(x) = 5 (u(x))^4 \cdot (-2) = 5 (3 - 2 x)^4 \cdot (-2) = -10 (3 - 2 x)^4 \MDFPeriod %%
\]

Als zweites Beispiel soll die Ableitung von $g: \R \rightarrow \R$ mit $g(x) = \MEU^{x^3}$ berechnet werden.
Dazu bieten sich die Zuordnungen $x \mapsto u(x) = x^3$ für die innere Funktion $u$ und $u \mapsto v(u) = \MEU^u$ für die äußere Funktion $v$ an.
Die Bestimmung der inneren und der äußeren Ableitung führt auf $u'(x) = 3 x^2$ und $v'(u) = \MEU^u$.
Setzt man beides in die Kettenregel ein, erhält man die Ableitung der Funktion $g$:
\[
g': \R \rightarrow \R \MDFPSpace , \MDFPaSpace x \mapsto g'(x) = \MEU^{u(x)} \cdot 3 x^2 = \MEU^{x^3} \cdot 3 x^2 = 3 x^2 \MEU^{x^3} \MDFPeriod %%
\]
\end{MExample}

\end{MXContent}


%%%Uebungen zum Abschnitt:
\begin{MExercises}
\MDeclareSiteUXID{VBKM07_Rechenregeln_Exercises}

\begin{MExercise}
Berechnen Sie die Ableitungen der Funktionen $f,g$ und $h$ mit den gegebenen Funktionstermen:
\begin{MExerciseItems}
\item $f(x) := 3 + 5 x$
führt auf $f'(x) = $\MLParsedQuestion{30}{5}{4}{DG19}.
%
\item $g(x) := \frac{1}{4 x} - x^3$
führt auf $g'(x) = $\MLSimplifyQuestion{30}{-1/(4*x^2) - 3*x^2}{1}{x}{20}{512}{SIMPLE10}.
%
\item $h(x) := 2 \sqrt{x} + 4 x^{-3}$
führt auf $h'(x) = $\MLSimplifyQuestion{30}{1/sqrt(x) - 12/(x^4)}{1}{x}{20}{512}{SIMPLE11}.
\end{MExerciseItems}
\begin{MHint}{\iSolution}
 \begin{MExerciseItems}
  \item Es ist $f'(x) = 0 + 5 \cdot 1 \cdot x^0 = 0 + 5 = 5$.
  \item Wegen $g(x) = \frac{1}{4x} - x^3 = \frac14 x^{-1} - x^3$ folgt
  $g'(x) = \frac14 \cdot (-1) \cdot x^{-2} - 3 \cdot x^2 = - \frac{1}{4 x^2} - 3 x^2$.
  \item Wegen $h(x) = 2 \sqrt{x} + 4 x^{-3} = 2 x^\frac12 + 4 x^{-3}$ folgt
  $h'(x) = 2 \cdot \frac12 \cdot x^{-\frac12} + 4 \cdot (-3) \cdot x^{-4} = \frac{1}{\sqrt{x}} - \frac{12}{x^4}$.
 \end{MExerciseItems}
\end{MHint}
\end{MExercise}

\begin{MExercise}
Berechnen und vereinfachen Sie die Ableitungen der Funktionen $f,g$ und $h$ mit den gegebenen Funktionstermen:
\begin{MExerciseItems}
\item $f(x) := \cot x = \frac{\cos x}{\sin x}$
führt auf $f'(x) = $\MLSimplifyQuestion{30}{-1/(sin(x) * sin(x))}{10}{x}{4}{512}{DG20}.
%
\item $g(x) := \sin(3 x) \cdot \cos(3 x)$
führt auf $g'(x) = $\MLFunctionQuestion{30}{3*cos(6*x)}{10}{x}{4}{DG21}.
%
\item $h(x) := \frac{\sin(3 x)}{\sin(6 x)}$
führt auf $h'(x) = $\MLSimplifyQuestion{30}{3/2 * tan(3*x)/cos(3*x)}{10}{x}{4}{512}{SIMPLE12}.
\end{MExerciseItems}
\begin{MHint}{\iSolution}
 \begin{MExerciseItems}
  \item Es ist mit Hilfe der Quotientenregel
  \[
  f'(x) = \frac{(- \sin (x)) \cdot \sin (x) - \cos (x) \cdot \cos (x)}{\left( \sin (x) \right)^2}
  = - \frac{\sin^2 (x) + \cos^2 (x)}{\sin^2 (x)} = - \frac{1}{\sin^2 (x)} \MDFPeriod
  \]
  \item Produkt- und Kettenregel liefern
  \[
   g'(x) = \cos (3x) \cdot 3 \cdot \cos (3x) + \sin (3x) \cdot \left( - \sin (3x) \right) \cdot 3
   = 3 \left( \cos^2 (3x) - \sin^2 (3x) \right) \MDFPeriod
  \]
  Da allgemein $\cos^2 (u) - \sin^2 (u) = \cos (2u)$ gilt, folgt also $g'(x) = 3 \cos (6x)$.
  \item Aufgrund der allgemeinen Relation $\sin (2u) = 2 \sin(u) \cos(u)$ ist
  \[
  h(x) = \frac{\sin (3x)}{\sin (6x)} = \frac{\sin (3x)}{2 \sin (3x) \cos (3x)} = \frac{1}{2 \cos (3x)}
  = \frac12 \cdot \left( \cos (3x) \right)^{-1} \MDFPeriod
  \]
  Mehrmaliges Anwenden der Kettenregel führt dann auf
  \[
  h'(x) = \frac12 \cdot (-1) \cdot \left( \cos (3x) \right)^{-2} \cdot \left( - \sin (3x) \right) \cdot 3
  = \frac{3 \sin (3x)}{2 \cos^2 (3x)} = \frac{3 \tan (3x)}{2 \cos (3x)} \MDFPeriod
  \]
 \end{MExerciseItems}
\end{MHint}
\end{MExercise}

\begin{MExercise}
Berechnen Sie die Ableitungen der Funktionen $f,g$ und $h$ mit den gegebenen Funktionstermen und fassen Sie soweit wie möglich zusammen: %###
\begin{MExerciseItems}
\item $f(x) := \MEU^{5 x}$
führt auf $f'(x) = $\MLFunctionQuestion{30}{5*exp(5*x)}{10}{x}{4}{DG22}.
%
\item $g(x) := x \cdot \MEU^{6 x}$
führt auf $g'(x) = $\MLFunctionQuestion{30}{(6*x+1)*exp(6*x)}{10}{x}{4}{DG23}.
%
\item $h(x) := (x^2 - x) \cdot \MEU^{-2 x}$
führt auf $h'(x) = $\MLFunctionQuestion{30}{-(2*x^2 - 4*x + 1) * exp(-2*x)}{10}{x}{4}{DG24}.
\end{MExerciseItems}
\begin{MHint}{\iSolution}
 \begin{MExerciseItems}
  \item Die Kettenregel liefert sofort $f'(x) = 5 \MEU^{5x}$.
  \item Produkt- und Kettenregel führen auf $g'(x) = 1 \cdot \MEU^{6x} + x \cdot \MEU^{6x} \cdot 6 = \MEU^{6x} (1 + 6x)$.
  \item Produkt- und Kettenregel führen auf $h'(x) = (2x - 1) \cdot \MEU^{-2x} + (x^2 - x) \cdot \MEU^{-2x} \cdot (-2)
  = - (2x^2 - 4x + 1) \MEU^{-2x}$.
 \end{MExerciseItems}
\end{MHint}
\end{MExercise}

\begin{MExercise}
Berechnen Sie die ersten vier Ableitungen von $f: \R \rightarrow \R$ mit $f(x) := \sin(1 - 2x)$.

Antwort: 
Die $k$-te Ableitung von $f$ wird mit $f^{(k)}$ bezeichnet. Dabei ist $f^{(1)} = f'$, $f^{(2)}$ die Ableitung von $f^{(1)}$,
$f^{(3)}$ die Ableitung von $f^{(2)}$ usw.
Damit:
\begin{itemize}
\item
 $f^{(1)}(x) = $\MLFunctionQuestion{30}{-2 * cos(1 - 2*x)}{10}{x}{4}{DG25}.
\item
 $f^{(2)}(x) = $\MLFunctionQuestion{30}{-4 * sin(1 - 2*x)}{10}{x}{4}{DG26}.
\item
 $f^{(3)}(x) = $\MLFunctionQuestion{30}{8 * cos(1 - 2*x)}{10}{x}{4}{DG27}.
\item
 $f^{(4)}(x) = $\MLFunctionQuestion{30}{16 * sin(1 - 2*x)}{10}{x}{4}{DG28}.
\end{itemize}
\begin{MHint}{\iSolution}
 Mit Hilfe der Kettenregel bestimmt man sukzessive:
 \begin{eqnarray*}
  f^{(1)} (x) & = & \cos(1 - 2x) \cdot (-2) = -2 \cos(1 - 2x) \MDFPSpace , \\
  f^{(2)} (x) & = & -2 \cdot \left( - \sin (1 - 2x) \right) \cdot (-2) = -4 \sin (1 - 2x) \MDFPSpace , \\
  f^{(3)} (x) & = & -4 \cdot \cos (1 - 2x) \cdot (-2) = 8 \cos (1 - 2x) \MDFPSpace , \\
  f^{(4)} (x) & = & 8 \cdot \left( - \sin (1 - 2x) \right) \cdot (-2) = 16 \sin (1 - 2x) \MDFPeriod
 \end{eqnarray*}
\end{MHint}
\end{MExercise}

\end{MExercises}



%%%Abschnitt
\MSubsection{Eigenschaften von Funktionen}\MLabel{M07_Eigenschaften}

\begin{MIntro}
\MDeclareSiteUXID{VBKM07_Eigenschaften_Intro}
Die Ableitung wurde weiter oben mittels einer Tangente an den Graphen einer Funktion eingeführt. Diese Tangente beschreibt die gegebene Funktion
in einem gewissen Bereich {\glqq}näherungsweise{\grqq}. Aus den Eigenschaften dieser Geraden kann auch auf Eigenschaften der
angenäherten Funktion geschlossen werden.
\end{MIntro}

\begin{MXContent}{Monotonie}{Monotonie}{STD}
\MDeclareSiteUXID{VBKM07_Monotonie}

Mit der Ableitung kann das lokale Wachstumsverhalten untersucht werden, das 
heißt, ob für steigende Argumente die zugehörigen Funktionswerte größer oder 
kleiner werden.
Dazu wird eine Funktion $f: D \rightarrow \R$ betrachtet, die auf $\MoIl a\MIntvlSep b\MoIr \subseteq D$ 
differenzierbar ist:
\begin{center}
\MTikzAuto{%
\begin{small}
\begin{tikzpicture}[line width=1.5pt,scale=1.0, %
declare function={
  x1 = 2;
  x0 = 4;
  fkt(\x) = 1/4*(\x - 3)*(\x - 3) + 0.75;
  TangenteAblplus(\x) = 1 + 1/2*(\x - x0); % $f(x_0) = f(4) = 1$.
  TangenteAblminus(\x) = 1 - 1/2*(\x - x1); % $f(x_1) = f(2) = 1$.
}
] %[every node/.style={fill=white}] 
%,every node/.style={fill=white}] 
%Koordinatenachsen:
\draw[->] (-0.6, 0) -- (6, 0) node[below left]{$x$}; %x-Achse
\draw[->] (0, -0.6) -- (0, 3) node[below left]{$y$}; %y-Achse
%Achsenbeschriftung:
\foreach \x in {1, 2, 3, 4, 5} \draw (\x, 0) -- ++(0, -0.1); %
% node[below] {$\x$}; 
\foreach \y in {1, 2} \draw (0, \y) -- ++(-0.1, 0); %
% node[below left] {$\y$};
%\node[below left] at (0, 0) {$0$};
%Hilfslinien:
\draw[color=black!50!white] (x0, {fkt(x0)}) -- ({x0+1}, {fkt(x0)});
\draw[->,color=black!50!white] %
 ({x0+1}, {fkt(x0)}) -- ({x0+1}, {TangenteAblplus(x0+1)});
%
\draw[color=black!50!white] (x1, {fkt(x1)}) -- ({x1-1}, {fkt(x1)});
\draw[->,color=black!50!white] %
 ({x1-1}, {TangenteAblminus(x1-1)}) -- ({x1-1}, {fkt(x1)});
%Funktion:
\draw[domain=0.8:5.2,samples=120,color=\jccolorfkt] %
 plot (\x, {fkt(\x)});
%Tangenten:
\draw[domain={x1-1.2}:{x1+0.8},samples=120,color=blue!50!black] %
 plot (\x, {TangenteAblminus(\x)});
\filldraw[color=blue!50!black] (x0, {fkt(x0)}) circle (1pt); % Berührpunkt.
%
\draw[domain={x0-0.8}:{x0+1.2},samples=120,color=blue!50!black] %
 plot (\x, {TangenteAblplus(\x)});
\filldraw[color=blue!50!black] (x1, {fkt(x1)}) circle (1pt); % Berührpunkt.
%Beschriftung:
\node[right] at (5.1, 1.25) {$f'(x_0) = m_0 > 0$};
\node[style={fill=white},left] at (0.9, 1.25) {$f'(x_1) = m_1 < 0$};
\node[below] at (x0, -0.1) {$x_0$};
\node[below] at (x1, -0.1) {$x_1$};
\end{tikzpicture}
\end{small}
}
\end{center}

%\begin{MXInfo}{Monotonie}
Wenn $f'(x) \leq 0$ für alle $x$ zwischen $a$ und $b$ gilt, dann ist $f$ 
auf dem Intervall $\MoIl a\MIntvlSep b\MoIr$ monoton fallend.

Wenn $f'(x) \geq 0$ für alle $x$ zwischen $a$ und $b$ gilt, dann ist $f$ 
auf dem Intervall $\MoIl a\MIntvlSep b\MoIr$ monoton wachsend.
%\end{MXInfo}

Somit genügt es, das Vorzeichen der Ableitung $f'$ zu bestimmen, um zu 
erkennen, ob eine Funktion auf dem Intervall $\MoIl a \MIntvlSep b\MoIr$ monoton wachsend oder monoton fallend ist. Beachten muss man lediglich, dass sich die Monotonie der Funktion $f$ an den Nullstellen der Ableitung $f'$ ändern kann. %###
%\end{MXContent}

\begin{MExample}
Die Funktion $f: \R \rightarrow \R, x \mapsto x^3$ ist differenzierbar mit 
$f'(x) = 3 x^2$. Da $x^2 \geq 0$ für alle $x \in \R$ gilt, ist 
$f'(x) \geq 0$ und damit $f$ monoton wachsend.

Für $g: \R \rightarrow \R$ mit $g(x) = 2 x^3 + 6 x^2 - 18 x + 10$ besitzt
$g'(x) = 6 x^2 + 12 x - 18 = 6 (x + 3) (x - 1)$ die Nullstellen $x_1 = -3$ 
und $x_2 = 1$. Zur Untersuchung des Monotonieverhaltens werden also drei Bereiche unterschieden, in denen
die Ableitung $g'$ jeweils ein anderes Vorzeichen hat.

Mit Hilfe folgender Tabelle wird bestimmt, in welchen Bereichen die Ableitung 
von $g$ positiv bzw. negativ ist. Diese Bereiche entsprechen den Monotoniebereichen von 
$g$. Die untersuchten Terme ergeben sich aus den einzelnen Faktoren von $g$. Der Eintrag $+$ besagt, dass der betrachtete Term im angegebenen %###
Intervall positiv ist. Wenn er negativ ist, wird $-$ eingetragen:
\[
\begin{array}{c||c|c|c}
x & x < -3 & -3 < x < 1 & 1 < x \\
\hline\hline
x + 3 & - & + & + \\\hline
x - 1 & - & - & + \\\hline
g'(x) & + & - & + \\\hline\hline
g \text{ ist monoton } & \text{ wachsend } & \text{ fallend } & \text{ wachsend } \\
\end{array}
\]
Die Funktion $g$ ist im Intervall $\MoIl -\infty \MIntvlSep -3\MoIr$ monoton wachsend, im Intervall $\MoIl -3 \MIntvlSep 1\MoIr$ monoton fallend und im Intervall $\MoIl 1 \MIntvlSep \infty\MoIr$ wieder monoton wachsend.%###
\end{MExample}%###
%###
\begin{MExample}%###
Für die Funktion $h: \R \setminus \{ 0 \} \rightarrow \R$ mit $h(x) = \frac{1}{x}$
gilt $h'(x) = - \frac{1}{x^2}$. Hier ist 
$h'(x) < 0$ für alle $x \neq 0$.

Auch wenn für die beiden Teilbereiche $x<0$ und $x>0$ dasselbe Monotonieverhalten auftritt,
ist $h$ nicht über den gesamten Definitionsbereich monoton fallend. Als Gegenbeispiel kann
$h(-2) = -\frac{1}{2}$ und $h(1) = 1$ angeführt werden. Hier gilt $-2 < 1$, aber auch $h(-2) < h(1)$. Dies
entspricht einem wachsenden Verhalten beim Übergang vom einen zum anderen Teilbereich. Deshalb ist es wichtig deutlich zu unterscheiden, dass die Funktion $h$ 
auf dem Intervall $\MoIl -\infty\MIntvlSep 0\MoIr$ und ebenfalls auf dem Intervall $\MoIl 0\MIntvlSep \infty\MoIr$ monoton fallend ist. %###
\end{MExample}
\end{MXContent}

%\MSubsubsection{Zweite Ableitung und Krümmungseigenschaften}
\begin{MXContent}{Zweite Ableitung und Krümmungseigenschaften}{Zweite Ableitung und Krümmungseigenschaften}{STD}
\MDeclareSiteUXID{VBKM07_Kruemmungseigenschaften}

%Seien $a,b \in D$ mit $a<b$.
Gegenstand der Untersuchung ist eine Funktion $f: D \rightarrow \R$, die auf dem Intervall $\MoIl a\MIntvlSep b\MoIr \subseteq D$
differenzierbar ist. Ist deren Ableitung $f'$ ebenfalls auf dem Intervall $\MoIl a\MIntvlSep b\MoIr \subseteq D$ differenzierbar, so hei{\ss}t $f$ \textbf{zweimal differenzierbar}.
Bildet man die Ableitung der ersten Ableitung von $f$, dann nennt man $(f')' = {f'}'$ die \textbf{zweite Ableitung} der Funktion $f$.

Die zweite Ableitung der Funktion $f$ kann verwendet werden, um das Krümmungsverhalten der Funktion zu untersuchen:

\begin{MXInfo}{Krümmungseigenschaften}
 Ist ${f'}'(x) \geq 0$ für alle $x$ zwischen $a$ und $b$, dann hei{\ss}t $f$ auf dem Intervall $\MoIl a\MIntvlSep b\MoIr$ \textbf{konvex}
 (\textbf{linksgekrümmt}).

 Ist ${f'}'(x) \leq 0$ für alle $x$ zwischen $a$ und $b$, dann hei{\ss}t $f$ auf dem Intervall $\MoIl a\MIntvlSep b\MoIr$ \textbf{konkav}
 (\textbf{rechtsgekrümmt}).
\end{MXInfo}

Somit genügt es, das Vorzeichen der zweiten Ableitung ${f'}'$ zu bestimmen, 
um zu erkennen, ob eine Funktion konvex (linksgekrümmt) oder konkav 
(rechtsgekrümmt) ist.
%\end{MXContent}

\begin{MXInfo}{Anmerkung zur Notation}
Die zweite Ableitung und weitere {\glqq}höhere{\grqq} Ableitungen werden oft mit hochgestellten natürlichen 
Zahlen in runden Klammern notiert: $f^{(k)}$ bezeichnet dann die $k$-te Ableitung von $f$.
Diese Notation wird besonders in allgemein gehaltenen Formeln auch für 
die (erste) Ableitung ($k=1$) und für die Funktion $f$ selbst ($k=0$) verwendet.

Damit bezeichnet 
\begin{itemize}
\item $f^{(0)} = f$ die Funktion $f$, 
\item $f^{(1)} = f'$ die (erste) Ableitung,
\item $f^{(2)} = {f'}'$ die zweite Ableitung,
\item $f^{(3)}$ die dritte Ableitung,
\item $f^{(4)}$ die vierte Ableitung von $f$.
\end{itemize}
Diese Liste kann selbstverständlich beliebig lange fortgeführt werden, solange die Ableitungen von $f$ existieren.
\end{MXInfo} 

Das folgende Beispiel zeigt, dass eine monoton wachsende Funktion in einem 
Bereich konvex und in einem anderen konkav sein kann.

\begin{MExample}
Die Funktion $f: \R \rightarrow \R, x \mapsto x^3$ ist sicherlich mindestens zweimal differenzierbar. Wegen
$f'(x) = 3 x^2 \geq 0$ für alle $x \in \R$ ist $f$ auf dem gesamten Definitionsbereich monoton wachsend.

Weiter ist ${f'}'(x) = 6 x$. Somit ist für $x < 0$ auch ${f'}'(x) < 0$
und damit $f$ hier konkav (nach rechts gekrümmt). Für $x > 0$ ist
${f'}'(x) > 0$, sodass $f$ für $x > 0$ konvex (nach links gekrümmt) ist.
\end{MExample}
%\end{MContent}
\end{MXContent}


%%%Uebungen zum Abschnitt:
\begin{MExercises}
\MDeclareSiteUXID{VBKM07_Verhalten_Exercises}

\begin{MExercise}
In welchen möglichst großen offenen Intervallen ist die Funktion $f$ mit
$f(x) := \frac{x^2 - 1}{x^2 + 1}$
monoton wachsend beziehungsweise monoton fallend?
%In welchen möglichst großen offenen Intervallen sind die Funktionen
%\begin{MExerciseItems}
%\item $f(x) := x^2 - 12\right) \cdot (x^2 + 5)$
%\item $g(x) := \frac{x^2 + 1}{x^2 - 1}$
%\end{MExerciseItems}
%monoton wachsend beziehungsweise monoton fallend?

Antwort:
\begin{itemize}
\item $f$ ist auf $\MoIl -\infty\MIntvlSep 0\MoIr$ monoton \MLQuestion{12}{fallend}{DFA1}.
%\item $f$ auf $(-infty,0)$ monoton \MQuestion{12}{fallend}.
%\item $f$ auf \MIntervalQuestion{20}{(-infty,0)}{4} monoton fallend.
%\item $f$ auf $(-infty$\MLParsedQuestion{8}{0}{4}{PARSEDQUEST5}$)$ monoton fallend.
%
\item $f$ ist auf $\MoIl 0\MIntvlSep \infty\MoIr$ monoton \MLQuestion{12}{wachsend}{DFA2}.
%\item $f$ auf $(0, infty)$ monoton \MQuestion{12}{wachsend}.
%\item $f$ auf \MIntervalQuestion{20}{(0,infty)}{4} monoton wachsend.
%\item $f$ auf $($\MIntervalQuestion{8}{0}{4}$, infty)$ monoton wachsend.
\end{itemize}
\begin{MHint}{\iSolution}
 Die Ableitung $f'$ der Funktion $f$ besitzt den Funktionsterm
 \[
 f'(x) = \frac{2x \cdot (x^2 + 1) - (x^2 - 1) \cdot 2x}{(x^2 + 1)^2}
 = \frac{2x \left( x^2 + 1 - x^2 + 1 \right)}{(x^2 + 1)^2} = \frac{4x}{(x^2 + 1)^2} \MDFPeriod
 \]
 Da der Nenner von $f'(x)$ immer positiv ist, bestimmt ausschließlich der Zähler das Vorzeichen von $f'(x)$: Für alle negativen $x \in \R$ ist
 $f'(x) < 0$ und $f$ daher dort monoton fallend; für alle positiven $x \in \R$ dagegen ist $f'(x) > 0$ und $f$ daher dort monoton wachsend.
\end{MHint}
\end{MExercise}


\begin{MExercise}
In welchen möglichst großen offenen Intervallen $\MoIl c\MIntvlSep d\MoIr$ ist die Funktion $f$ mit
$f(x) := \frac{x^2 - 1}{x^2 + 1}$ für $x > 0$
konvex beziehungsweise konkav?
%In welchen möglichst großen offenen Intervallen sind die Funktionen
%konvex beziehungsweise konkav?
%\begin{MExerciseItems}
%\item $f(x) := \frac{6 x}{x^2 + 4}$
%\item $g(x) := \left(x - 2 \sqrt{3}\right) \cdot ($
%\end{MExerciseItems}
Antwort:
\begin{itemize}
\item $f$ ist auf \MLIntervalQuestion{35}{(0, 1/sqrt(3))}{4}{DEG1} konvex.
%
\item $f$ ist auf \MLIntervalQuestion{35}{(1/sqrt(3),infty)}{4}{DEG2} konkav.
\end{itemize}
\MInputHint{Offene Intervalle können in der Form $(a;b)$ eingetippt werden, geschlossene Intervalle als $[a;b]$, $a$ und $b$ dürfen
beliebige Ausdrücke sein. Verwenden Sie bei der Intervalleingabe nicht die Notation $]a;b[$ für offene Intervalle. Schreiben Sie \texttt{infty} oder
\texttt{unendlich} für $\infty$ in Ihrer Antwort.}

\begin{MHint}{\iSolution}
 Man berechnet für die erste und die zweite Ableitung von $f$ mit Hilfe der Quotientenregel
 \begin{eqnarray*}
  f'(x) & = & \frac{4x}{(x^2 + 1)^2} \MDFPSpace , \\
  {f'}'(x) & = & \frac{4 \cdot (x^2 + 1)^2 - 4x \cdot 2 (x^2 + 1) \cdot 2x}{(x^2 + 1)^4}
  = \frac{4x^2 + 4 - 16x^2}{(x^2 + 1)^3} = \frac{4 (1-3x^2)}{(x^2 + 1)^3} \MDFPeriod
 \end{eqnarray*}
 Da $4/(x^2 + 1)^3$ stets positiv ist, wird das Vorzeichen von ${f'}'(x)$ einzig durch den Faktor $(1 - 3x^2)$ bestimmt. Die einfachen
 Nullstellen von ${f'}'(x)$ liegen bei $x_0 = \pm \frac{1}{\sqrt{3}}$. Daher ist für $x>0$ die zweite Ableitung ${f'}'(x)$ auf dem %###
 offenen Intervall $\MoIl[\left] 0 \MIntvlSep \frac{1}{\sqrt{3}} \MoIr[\right] $
 echt größer $0$ und $f$ in diesem Intervall linksgekrümmt (konvex). Auf $\MoIl[\left] \frac{1}{\sqrt{3}} \MIntvlSep \infty \MoIr[\right]$ gilt ${f'}'(x) < 0$; somit ist $f$ dort %###
 rechtsgekrümmt (konkav).
 \end{MHint}
\end{MExercise}


\begin{MExercise}
Gegeben ist eine Funktion $f: [-\MZahl{4}{5} \MIntvlSep 4] \rightarrow \R$ mit $f(0) := 2$; deren 
Ableitung $f'$ besitze folgenden Graphen:

\begin{center}
\MTikzAuto{%
%{Ableitung von $f$}
\begin{small}
\begin{tikzpicture}[line width=1.5pt,scale=0.6, %
declare function={
  fkt(\x) = 1/40*(\x + 4)*(\x)*(\x - 3)*(\x - 3);
}
] %[every node/.style={fill=white}] 
%Koordinatenachsen:
\draw[->] (-5, 0) -- (5, 0) node[below left]{$x$}; %x-Achse
\draw[->] (0, -3.5) -- (0, 4) node[below left]{$y$}; %y-Achse
%Achsenbeschriftung:
\foreach \x in {-4, -3, -2, -1} \draw (\x, 0) -- ++(0, 0.1) %
 node[above] {$\x$}; 
\foreach \x in {1, 2, 3, 4} \draw (\x, 0) -- ++(0, -0.1) %
 node[below] {$\x$}; 
\foreach \y in {-3, -2, -1} \draw (0, \y) -- ++(0.1, 0) %
 node[right] {$\y$};
\foreach \y in {1, 2, 3} \draw (0, \y) -- ++(-0.1, 0) %
 node[left] {$\y$};
%\node[below left] at (0, 0) {$0$};
%Funktion:
\draw[domain=-4.5:4,samples=120,color=\jccolorfkt] %
 plot (\x, {fkt(\x)});
\end{tikzpicture}
\end{small}
}
\end{center}

\begin{MExerciseItems}
\item Wo ist $f$ monoton wachsend, wo monoton fallend? 
Gesucht sind jeweils möglichst große offene Intervalle $\MoIl c\MIntvlSep d\MoIr$, auf 
denen $f$ diese Eigenschaft hat.
%
\item Welche Aussagen erhalten 
Sie über die Maximal- beziehungsweise Minimalstellen der Funktion $f$?
% für $f$ zu deren Maximal- beziehungsweise Minimalstellen?
\end{MExerciseItems}

Antwort:
\begin{itemize}
\item $f$ ist auf $\MoIl[\big] -\MZahl{4}{5}\MIntvlSep $\MLParsedQuestion{8}{-4}{4}{DEG3}$\MoIr[\big]$ monoton
\MLQuestion{16}{wachsend}{DFA6}.
%
\item $f$ ist auf $\MoIl[\big]$\MLParsedQuestion{16}{-4}{4}{DEG4}$\MIntvlSep 0\MoIr[\big]$ monoton 
\MLQuestion{16}{fallend}{DFA7}.
%
\item $f$ ist auf $\MoIl 0\MIntvlSep 3\MoIr$ monoton \MLQuestion{16}{wachsend}{DEG5}.
%
\item $f$ ist auf $\MoIl 3\MIntvlSep 4\MoIr$ monoton \MLQuestion{16}{wachsend}{DEG6}.
\end{itemize}
Die Maximalstelle von $f$ ist bei \MLParsedQuestion{8}{-4}{4}{DEG7}.
Die Minimalstelle von $f$ ist bei \MLParsedQuestion{8}{0}{4}{DEG8}.

\begin{MHint}{\iSolution}
 Das Monotonieverhalten wird durch die Ableitung $f'$ der Funktion $f$ bestimmt. Da der Graph der Ableitung $f'$ in der Aufgabenstellung
 als Schaubild gegeben ist, muss man nur ablesen, in welchen Intervallen der Graph oberhalb (bzw. unterhalb) der $x$-Achse verläuft:
 Auf den Intervallen $\MoIr -\MZahl{4}{5} \MIntvlSep -4\MoIr$, $\MoIl 0 \MIntvlSep 3\MoIr$ und $\MoIl 3 \MIntvlSep 4\MoIr$ ist $f'(x) > 0$ und $f$ daher dort monoton wachsend;
 auf dem Intervall $\MoIl -4 \MIntvlSep 0\MoIr$ dagegen ist $f'(x) < 0$ und $f$ daher dort monoton fallend.\newline
 An einer Extremstelle $x_e$ (Maximal- oder Minimalstelle) einer Funktion $f$ (die nicht auf dem Rand des Definitionsbereichs liegt)
 verschwindet die erste Ableitung: $f'(x_e) = 0$. Anschaulich bedeutet dies, dass an einer solchen Stelle die Tangente an den Graphen
 von $f$ waagrecht verläuft. Die Nullstellen von $f'(x)$ liegen laut Aufgabenstellung bei $x_ 1 = -4$, $x_2 = 0$ und $x_3 = 3$. Da $f$ auf
 $\MoIl -\MZahl{4}{5} \MIntvlSep -4\MoIr$ monoton wächst und auf $\MoIl -4 \MIntvlSep 0\MoIr$ monoton fällt, ist $x_1 = - 4$ eine Maximalstelle.
 Analog begründet man, dass bei $x_2 = 0$ eine Minimalstelle vorliegt. (Bei $x_3 = 3$ handelt es sich um eine Sattelstelle.)
\end{MHint}
\end{MExercise}
\end{MExercises}

\MSubsection{Anwendungen}
\MLabel{M07_Anwendungen}


\begin{MXContent}{Kurvendiskussion}{Kurvendiskussion}{STD}
\MLabel{VBKM07_Kurvendiskussion}
\MDeclareSiteUXID{VBKM07_Kurvendiskussion}


Gegeben ist eine differenzierbare Funktion $f: \MoIl a\MIntvlSep b\MoIr \rightarrow \R$ 
mit Abbildungsvorschrift $x \mapsto y = f(x)$ für $x \in \MoIl a\MIntvlSep b\MoIr$.
Eine vollständige Kurvendiskussion für $f$ besteht in diesem Kurs aus folgenden Angaben:

\begin{itemize}
\item Maximaler Definitionsbereich
\item Achsenschnittpunkte des Graphen
\item Symmetrie des Graphen
\item Grenzverhalten/Asymptoten
\item Den ersten beiden Ableitungen %###
\item Extremwerte
\item Monotonieverhalten
\item Wendestellen
\item Krümmungsverhalten
\item Skizze des Graphen
\end{itemize}

Viele dieser Punkte wurden bereits in Modul \MRef{VBKM06} behandelt. Daher
wiederholt das Folgende nur kurz, was unter den einzelnen
Schritten der Kurvendiskussion zu verstehen ist. Im Anschluss wird
eine Kurvendiskussion detailliert an einem Beispiel durchgesprochen.

Der erste Teil der Kurvendiskussion besteht aus algebraischen und geometrischen Aspekten von $f$:

\begin{description}
\item[Maximaler Definitionsbereich]
Es werden alle reellen Zahlen $x$ bestimmt, für die $f(x)$ existiert. Die 
Menge $D$ all dieser Zahlen wird maximaler Definitionsbereich genannt.

\item[Schnittpunkte mit den Achsen] ~\relax 
\begin{itemize}
\item $x$-Achse: Alle Nullstellen von $f$ werden bestimmt.
\item $y$-Achse: Der Funktionswert $f(0)$ (falls $0 \in D$) wird berechnet.
\end{itemize}

\item[Symmetrie des Graphen]
Der Graph der Funktion ist symmetrisch zur $y$-Achse, wenn $f(-x) = f(x)$ für
alle $x \in D$ ist. Dann heißt die Funktion $f$ auch \textbf{gerade}.\\%###
Ist $f(-x) = -f(x)$ für alle $x \in D$, so ist der Graph
zum Ursprung $\MPointTwo{0}{0}$ des Koordinatensystems punktsymmetrisch.
In diesem Falle nennt man die Funktion $f$ auch \textbf{ungerade}.

\item[Asymptotisches Verhalten an den Rändern des Definitionsbereichs]
Die Grenzwerte der Funktion $f$ an den Grenzen ihres Definitionsbereichs werden untersucht.
\end{description} 

Im zweiten Teil wird die Funktion mittels Folgerungen aus der Ableitung 
analytisch untersucht. Dazu müssen natürlich zunächst die erste und die zweite Ableitung 
berechnet werden, sofern diese existieren.

\begin{description}
\item[Ableitungen]
Berechnung der ersten und zweiten Ableitung (soweit vorhanden).

\item[Extremwerte und Monotonie]
Notwendige Bedingung für Extremstellen $x$ (sofern $x \in D$ kein Randpunkt von $D$ ist): $f'(x) = 0$\newline
Wir berechnen also diejenigen Stellen $x_0$, an denen die Ableitung $f'$ den Wert Null annimmt. Wenn an diesen Stellen auch die zweite
Ableitung ${f'}'$ existiert, gilt:
\begin{itemize}
 \item ${f'}'(x_0) > 0$: $x_0$ ist eine Minimalstelle von $f$.
\item ${f'}'(x_0) < 0$: $x_0$ ist eine Maximalstelle von $f$.
\end{itemize}
Die Funktion $f$ ist auf denjenigen Intervallen des Definitionsbereichs monoton
wachsend, auf denen $f'(x) \geq 0$ gilt. Sie ist dort monoton fallend, wo $f'(x) \leq 0$ 
ist.

\item[Wendestellen und Krümmungseigenschaften]
Notwendige Bedingung für Wendestellen (sofern die zweite Ableitung ${f'}'$ existiert): ${f'}'(x) = 0$\newline
Wenn ${f'}'(w_0) = 0$ und $f^{(3)}(w_0) \neq 0$ ist, dann ist $w_0$ eine Wendestelle, d.h. $f$ ändert an dieser Stelle das Krümmungsverhalten.\newline
Die Funktion $f$ ist auf denjenigen Intervallen des Definitionsbereichs konvex
(linksgekrümmt), auf denen $f^{(2)}(x) \geq 0$ gilt. Sie ist konkav (rechtsgekrümmt) dort, 
wo $f^{(2)}(x) \leq 0$ ist.

\item[Skizze des Graphen] Eine Skizze des Graphen in einem geeigneten Koordinatensystem wird angefertigt, und zwar unter
Berücksichtigung der während der Kurvendiskussion gewonnenen Daten.
\end{description}

\MSubsubsection{Ausführliches Beispiel}
Es soll eine Funktion $f$ mit dem Funktionsterm
$$
f(x) \;=\; \frac{4x}{x^2+2}
$$
untersucht werden.

\textbf{Maximaler Definitionsbereich}\\
Der maximale Definitionsbereich dieser Funktion ist $D_f=\R$, da der Nenner der Funktion $x^2+2\geq 2$ ist, also niemals Null wird,
und daher keine Stellen ausgeschlossen werden müssen.
\ \\ \ \\
\textbf{Achsenschnittpunkte}\\
Die Nullstellen der Funktion entsprechen den Nullstellen des Zählers. Daher schneidet der Graph von $f$ die $x$-Achse nur
im Nullpunkt $\MPointTwo{0}{0}$, denn der Zähler wird nur für $x=0$ zu Null. Dies ist auch der einzige Schnittpunkt mit der $y$-Achse, da $f(0)=0$ ist.
\ \\ \ \\
\textbf{Symmetrie}\\
Um das Symmetrieverhalten zu untersuchen, wird das Argument $x$ durch $(-x)$ ersetzt. Es gilt
$$
f(-x) \;=\;\frac{4\cdot (-x)}{(-x)^2+2} \;=\; -\frac{4x}{x^2+2} \;=\; -f(x)
$$
für alle $x\in\R$. Der Graph von $f$ ist folglich punktsymmetrisch zum Ursprung.
\ \\ \ \\
\textbf{Grenzverhalten}\\
Die Funktion ist auf ganz $\R$ definiert, daher ist nur das Grenzverhalten für $x\rightarrow \infty$ und $x\rightarrow -\infty$ zu untersuchen.
Da $f(x)$ ein Bruch aus zwei Polynomen ist und der Nenner die höhere Potenz besitzt, ist die $x$-Achse die waagerechte Asymptote in beide Richtungen:
\[
\lim_{x \rightarrow \pm \infty} f(x) = 0 \MDFPeriod
\]
\ \\ \ \\
\textbf{Ableitungen}\\
Die ersten beiden Ableitungen der Funktion folgen mit Hilfe der Quotientenregel:
$$
f'(x) \;=\; \frac{4\cdot (x^2+2)-4x\cdot 2x}{(x^2+2)^2} \;=\; 4\cdot \frac{-x^2+2}{(x^2+2)^2} \MDFPeriod %###
$$
Erneutes Ableiten und Vereinfachen ergibt
\begin{eqnarray*}
{f'}'(x) & = & 4 \cdot % 
\frac{-2x (x^2 + 2)^2 - (-x^2 + 2) \cdot 2 (x^2 + 2) \cdot 2 x}{(x^2 + 2)^4} \\
& = & 4 \cdot \frac{-2x (x^2 + 2) - (-x^2 + 2) \cdot 4 x}{(x^2 + 2)^3} \\
& = & 4 \cdot \frac{-2x^3 - 4x + 4x^3 - 8 x}{(x^2 + 2)^3} \\
& = & 4 \cdot \frac{2x^3 - 12x}{(x^2 + 2)^3} \\
& = & 8 \cdot \frac{x (x^2 - 6)}{(x^2 + 2)^3} \MDFPeriod
\end{eqnarray*}
\ \\
\textbf{Extremwerte}\\
Die notwendige Bedingung für eine Extremstelle, $f'(x)=0$, ist hier gleichbedeutend mit $-x^2+2=0$.
Man erhält also $x_1=\sqrt2$ und $x_2=-\sqrt2$. Es muss noch das Verhalten der zweiten Ableitung an diesen Stellen untersucht werden:
$$
{f'}'(x_1) \;=\; 8\frac{\sqrt2\cdot(2-6)}{(2+2)^3}<0 \MDFPSpace , \MDFPaSpace
{f'}'(x_2) \;=\; 8\frac{(-\sqrt2)\cdot(2-6)}{(2+2)^3}>0 \MDFPeriod
$$
Folglich ist $x_1$ eine Maximalstelle und $x_2$ eine Minimalstelle von $f$. Durch Einsetzen in $f$ resultieren das
Maximum $\MPointTwoAS{\sqrt2}{\sqrt2}$ und das Minimum $\MPointTwoAS{-\sqrt2}{-\sqrt2}$ von $f$.
\ \\ \ \\
\textbf{Monotonieverhalten}\\
Da $f$ auf ganz $\R$ definiert ist,
kann das Monotonieverhalten aus der Lage der Extremstellen und aus deren Typ abgelesen werden: $f$ ist monoton fallend auf
$\MoIl[\left]-\infty\MIntvlSep -\sqrt2\MoIr[\right]$, monoton wachsend auf $\MoIl[\left]-\sqrt2\MIntvlSep \sqrt2\MoIr[\right]$
und monoton fallend auf $\MoIl[\left]\sqrt2\MIntvlSep \infty\MoIr[\right]$. Monotonieintervalle werden stets in offener Form angegeben.
\ \\ \ \\
\textbf{Wendestellen}\\
Aus der notwendigen Bedingung für Wendestellen ${f'}'(x)=0$ erhält man die Gleichung $8x(x^2-6)=0$.
Somit sind $w_0=0$, $w_1=\sqrt6$ und $w_2=-\sqrt6$ die einzigen Lösungen.
Das Polynom im Nenner von ${f'}'$ ist stets größer als Null. Da das Zählerpolynom nur einfache Nullstellen besitzt, ändert ${f'}'(x)$
in allen diesen Stellen das Vorzeichen. Es handelt sich
daher um Wendestellen von $f$. Die Wendepunkte $\MPointTwo{0}{0}$, $\MPointTwoAS{\sqrt6}{\frac12\sqrt6}$, $\MPointTwoAS{-\sqrt6}{-\frac12\sqrt6}$ ergeben sich durch
Einsetzen der Wendestellen in $f$.
\ \\ \ \\
\textbf{Krümmungsverhalten}\\
Die zweimal differenzierbare Funktion $f$ ist konvex, wenn die zweite Ableitung größer oder gleich Null ist. Sie ist konkav,
wenn die zweite Ableitung kleiner oder gleich Null ist.
Da das Polynom im Nenner von ${f'}'(x)$ stets positiv ist, genügt es, das Vorzeichen des Polynoms $p(x)=8x(x-\sqrt6)(x+\sqrt6)$ im Zähler
zu untersuchen. Für $0<x<\sqrt6$ ist es negativ (dort ist $f$ konkav). Für $x>\sqrt6$ ist es positiv (dort ist $f$ konvex).
Da $f$ punktsymmetrisch ist, folgt, dass $f$ auf den Intervallen $\MoIl[\left]-\sqrt 6\MIntvlSep 0\MoIr[\right]$ und $\MoIl[\left]\sqrt6\MIntvlSep \infty\MoIr[\right]$ konvex
sowie auf $\MoIl[\left]-\infty\MIntvlSep -\sqrt6\MoIr[\right]$ und $\MoIl[\left]0\MIntvlSep \sqrt6\MoIr[\right]$ konkav ist.
\ \\ \ \\
\textbf{Skizze des Graphen}\\
\MUGraphics{BildKurvendiskussion1.png}{width=0.5\linewidth}{Der Graph der Funktion $f$, skizziert auf dem Intervall $[-8\MIntvlSep 8]$.}{width:600px}

\end{MXContent}



\begin{MExercises}
\MDeclareSiteUXID{VBKM07_Kurvendiskussion_Exercises}

Mit dieser Trainingsaufgabe können die Bestandteile der Kurvendiskussion geübt werden:

\MDirectRouletteExercises{curves_polynomials.rtex}{VBKM07_CURVES_POLYNOMIALS}


\begin{MExercise}
Führen Sie eine vollständige Kurvendiskussion für eine Funktion $f$ mit $f(x) = (2 x - x^2) \MEU^x$ durch und
füllen Sie die Eingabefelder mit Ihren Ergebnissen aus:
\ \\ \ \\
Maximaler Definitionsbereich: \MLIntervalQuestion{25}{(-infty,infty)}{5}{DXG18} \MInputHint{(in Intervallschreibweise \texttt{(a;b)})}.\\
\ \\
Menge der Schnittpunkte mit der $x$-Achse (Nullstellen von $f(x)$): \MLParsedQuestion{5}{0,2}{5}{DXG1} \MInputHint{(in Mengenschreibweise \texttt{$\lbrace$a;b;c$\rbrace$}, nur $x$-Komponenten)}.\\
\ \\
Der Schnittpunkt mit der $y$-Achse ist bei \MEquationItem{$y$}{\MLParsedQuestion{5}{0}{5}{DXG2}}.\\
\ \\
Symmetrie: Die Funktion ist\\
\begin{MQuestionGroup}
\begin{tabular}{lll}
\MLCheckbox{0}{JCA1} & \ \ & achsensymmetrisch zur $y$-Achse,\\
\MLCheckbox{0}{JCA2} & \ \ & punktsymmetrisch zum Ursprung.
\end{tabular}
\end{MQuestionGroup}
\MGroupButton{Auswahl überprüfen}
\ \\ \ \\
Grenzverhalten: Für $x\rightarrow \infty$ streben die Funktionswerte $f(x)$ gegen \MLFunctionQuestion{20}{-infty}{4}{infty}{4}{DXG3}
und für $x\rightarrow-\infty$ gegen \MLParsedQuestion{15}{0}{3}{DXG5}.\\
\ \\
Ableitungen: Es ist $f'(x)$ = \MLFunctionQuestion{20}{-(x^2-2)*exp(x)}{4}{x}{4}{DXG6} sowie
${f'}'(x)$ = \MLFunctionQuestion{30}{-(x^2+2*x-2)*exp(x)}{5}{x}{5}{DXG7} .\\
\ \\
Monotonieverhalten: Die Funktion ist auf dem Intervall \MLIntervalQuestion{26}{(-sqrt(2),sqrt(2))}{4}{DXG8} monoton wachsend und ansonsten monoton fallend.\\
\ \\
Extremwerte: Die Stelle $x_1$ = \MLParsedQuestion{13}{-sqrt(2)}{3}{DXG9} ist eine Minimalstelle und
$x_2$ = \MLParsedQuestion{13}{sqrt(2)}{3}{DXG10} ist eine Maximalstelle.\\
\ \\
Wendepunkte: Die Menge der Wendestellen ist
\MLParsedQuestion{30}{-1-sqrt(3),-1+sqrt(3)}{3}{DXG11}\\\MInputHint{(in Mengenschreibweise, Wurzeln dürfen stehenbleiben)}.\\
\ \\
Kr\"mmungsverhalten: Die Funktion ist auf dem Intervall \MLIntervalQuestion{26}{(-1-sqrt(3),-1+sqrt(3))}{4}{DXG8} konvex und ansonsten konkav.\\%###
\ \\%###
Skizzieren Sie den Graphen und vergleichen Sie ihn anschließend mit der Musterlösung.

\begin{MHint}{\iSolution}
\textbf{Maximaler Definitionsbereich}\\
Es gilt $f(x) = - x (x - 2) \MEU^x$ und $\MEU^x > 0$ für alle  $x \in \R$; damit ist $D_f = \R = \MoIl -\infty \MIntvlSep \infty\MoIr$
der maximale Definitionsbereich.
\ \\ \ \\
\textbf{Achsenschnittpunkte}\\
Schnittpunkte mit der $x$-Achse (Nullstellen der Funktion) liegen bei $x_1=0$ und $x_2=2$, was auf die Punkte $\MPointTwo{0}{0}$ und $\MPointTwo{2}{0}$ führt.
Der Schnittpunkt mit der $y$-Achse ist $\MPointTwo{0}{0}$.
\ \\ \ \\
\textbf{Symmetrie}\\
Die Funktion $f$ ist weder gerade noch ungerade, und damit ist der Graph von $f$ weder achsensymmetrisch zur $y$-Achse noch
punktsymmetrisch bzgl. des Ursprungs.
\ \\ \ \\
\textbf{Grenzverhalten}\\
Da die Funktion für alle reellen Zahlen definiert ist, müssen nur die Asymptoten bei $\pm \infty$ untersucht werden:
$$
\lim_{x \rightarrow \infty} - x (x - 2) \MEU^x \; =\; - \infty \;\text{und}\;
\lim_{x \rightarrow -\infty} - x (x - 2) \MEU^x \;=\;  0 \MDFPeriod
$$
Somit ist $y = 0$ für $x \rightarrow -\infty$ eine Asymptote.
\ \\ \ \\
\textbf{Ableitungen}\\
Die ersten beiden Ableitungen von $f$ führen auf
\begin{eqnarray*}
f'(x) & = & (2 - 2 x) \MEU^x + (2 x - x^2) \MEU^x %
 = (2 - x^2) \MEU^x = - (x^2 - 2) \MEU^x \MDFPSpace ,\\
{f'}'(x) & = & - 2 x \MEU^x + (2 - x^2) \MEU^x %
 = - (x^2 + 2 x - 2) \MEU^x \MDFPeriod
\end{eqnarray*}
\ \\
\textbf{Monotonieverhalten und Extremwerte}\\
Lösungen von $f'(x) = 0$ sind $x_1 = -\sqrt{2}$ und $x_2 = \sqrt{2}$. 
Weiter ist $x_1 < x_2$ und 
$$
f'(x)\; =\; - (x + \sqrt{2}) (x - \sqrt{2}) \MEU^x \MDFPeriod
$$
Auf $\MoIl[\left]-\infty\MIntvlSep  -\sqrt{2}\MoIr[\right]$ ist $f'$ negativ und damit $f$ monoton fallend.\\ %###
Auf $\MoIl[\left]-\sqrt{2}\MIntvlSep  \sqrt{2}\MoIr[\right]$ ist $f'$ positiv und damit $f$ monoton wachsend.\\%###
Auf $\MoIl[\left]\sqrt{2}\MIntvlSep  \infty\MoIr[\right]$ ist $f'$ negativ und damit $f$ monoton fallend.\\%###
Somit ist $x_1 = -\sqrt{2}$ Minimalstelle und $x_2 = \sqrt{2}$ Maximalstelle.
\ \\ \ \\
\textbf{Wendepunkte}\\
Die notwendige Bedingung für Wendestellen ${f'}'(x) = 0$ führt auf die quadratische Gleichung
$x^2 + 2 x - 2 = 0$. Die Lösungen sind $w_1 = \frac{-2 - \sqrt{4 + 8}}{2} = -1 - \sqrt{3}$
und $w_2 = \frac{-2 + \sqrt{4 + 8}}{2} = -1 + \sqrt{3}$.
\ \\ \ \\
\textbf{Kr\"mmungsverhalten}\\%###
Da die Wendepunkte bekannt sind, muss nur noch bestimmt werden, wann ${f'}'(x)$ grö\ss er oder kleiner Null ist. Die Ableitungen der Exponentialfunktion sind immer grö\ss er Null, von daher bestimmt der quadratische Term das Vorzeichen.\\%###
Auf $\MoIl[\left]-\infty\MIntvlSep  -1 - \sqrt{3}\MoIr[\right]$ ist ${f'}'(x)$ negativ und damit $f$ konkav.\\%###
Auf $\MoIl[\left]-1 - \sqrt{3}\MIntvlSep  -1 + \sqrt{3}\MoIr[\right]$ ist ${f'}'(x)$ positiv und damit $f$ konvex.\\%###
Auf $\MoIl[\left]-1 + \sqrt{3}\MIntvlSep  \infty\MoIr[\right]$ ist ${f'}'(x)$ negativ und damit $f$ konkav.%###
\ \\ \ \\%###
\textbf{Skizze des Graphen}\\
\MUGraphics{BildKurvendiskussion3.png}{width=0.5\linewidth}{Der Graph der Funktion $f$,
skizziert auf dem Intervall $\MoIl -\MZahl{6}{2}\MIntvlSep 3\MoIr$.}{width:400px}
\end{MHint}

\end{MExercise}

\end{MExercises}



%\MSubsubsection{Optimierungsaufgaben}
\begin{MXContent}{Optimierungsaufgaben}{Optimierungsaufgaben}{STD}
\MDeclareSiteUXID{VBKM07_Optimierungsaufgaben}

In vielen Anwendungen der Technik oder Wirtschaft findet man Problemlösungen, die nicht eindeutig sind. Häufig hängen sie von variablen Bedingungen ab. Um eine ideale Lösung zu finden, werden zusätzliche Eigenschaften (Nebenbedingungen) definiert, die von der Lösung erfüllt werden müssen. Dies führt sehr oft zu sogenannten \textbf{Optimierungsaufgaben}, bei denen aus einer Schar von Lösungen diejenige gesucht werden muss, die eine vorab festgelegte Eigenschaft am besten erfüllt.

Als Beispiel werde die Aufgabe betrachtet, eine zylinderförmige Dose zu konstruieren. Die Dose soll zusätzlich die Bedingung erfüllen, ein 
Fassungsvermögen (Volumen) $V$ von einem Liter, also einem Kubikdezimeter ($1 \MEinheit{dm}^3$), zu haben. Wählt man für $V$ die
Einheit $\MEinheit[]{dm}^3$ und
sind $r$ der Radius und $h$ die Höhe der Dose, jeweils gemessen in Dezimetern ($\MEinheit[]{dm}$), so soll also $V = \pi r^2 \cdot h = 1$ sein.
%(Auf die physikalischen Einheiten wurde der mathematischen Einfachheit halber verzichtet -- in der Praxis ist es allerdings unumgänglich, mit
%Einheiten zu rechnen.)
Um Arbeitsmaterial zu sparen, wird nach derjenigen Dose gesucht, die eine möglichst kleine Oberfläche
$O = 2 \cdot \pi r^2 + 2 \pi r h$ hat. Hier ist die Oberfläche $O$ in Quadratdezimetern ($\MEinheit[]{dm}^2$)
eine Funktion in Abhängigkeit vom Radius $r$ und von der Höhe $h$ der Dose.

Mathematisch formuliert, führt die Aufgabe auf die Suche nach einem Minimum für die Funktion $O$ der Oberfläche, wobei für die Berechnung des Minimums nur 
Werte für $r$ und $h$ zugelassen werden, für die auch die Bedingung über das Volumen $V = \pi r^2 \cdot h = 1$ erfüllt ist. Eine solche zusätzliche
Bedingung bei der Suche nach Extremstellen wird auch \textbf{Nebenbedingung} genannt.

\begin{MXInfo}{Optimierungsaufgabe}
In einer \MEntry{Optimierungsaufgabe}{Optimierungsaufgabe} wird eine
Extremstelle $x_{\text{ext}}$ einer Funktion $f$ gesucht, die eine 
gegebene Gleichung $g(x_{\text{ext}}) = b$ erfüllt.

Wird ein Minimum gesucht, spricht man auch von einer 
\MEntry{Minimierungsaufgabe}{Minimierungsaufgabe}. Wenn ein Maximum gesucht
wird, heißt die Optimierungsaufgabe eine 
\MEntry{Maximierungsaufgabe}{Maximierungsaufgabe}.

Die Funktion $f$ heißt \MEntry{Zielfunktion}{Zielfunktion}, und die 
Gleichung $g(x) = b$ wird \MEntry{Nebenbedingung}{Nebenbedingung} der 
Optimierungsaufgabe genannt.
\end{MXInfo}


\end{MXContent}

\begin{MXContent}{Beispiel}{Beispiel}{STD}
\MDeclareSiteUXID{VBKM07_WeitereBeispiele}

Dazu soll ein Beispiel etwas genauer betrachtet werden: Es geht also um die Minimierung der Oberfläche einer zylinderförmigen Dose bei einem vorgegebenen Volumen %###
(Grundfläche mal Höhe)
%
\begin{eqnarray*}
V = \pi r^2 h = 1 \MDFPSpace,
\end{eqnarray*}
%
wobei $r$ der Radius der Grundfläche und $h$ die Höhe der Dose sind. Die Oberfläche setzt sich aus dem Deckel und dem Boden
(jeweils mit einer Fläche der Größe $\pi r^2$) und der Mantelfläche (der Größe $2 \pi r h$) zusammen,
und man erhält $O = 2 \pi r^2 + 2 \pi r h$. Die Oberfläche der Dose ist eine Funktion vom Radius $r$ und von der Höhe $h$.
Im Gegensatz dazu wird dem Volumen ein fester Wert zugeordnet (Nebenbedingung). Es kann also geschrieben werden:
%
\begin{eqnarray*}
O\left(r, h\right) = 2 \pi r^2 + 2 \pi r h \MDFPeriod
\end{eqnarray*}
%
Wegen der Nebenbedingung, dass $V = \pi r^2 h = 1$ sein soll, kann man dieses Problem in ursprünglich zwei Variablen ($r$ und $h$)
auf ein Problem mit nur noch einer Variablen reduzieren. Umformen des Volumens nach der Höhe der Dose führt auf:
%
\begin{eqnarray*}
\pi r^2 h &=& 1\\
\Leftrightarrow\quad h &=& \frac{1}{\pi r^2} \MDFPeriod
\end{eqnarray*}
%
Nach dem Einsetzen dieser Formel in die Funktion $O(r,h)$ resultiert eine Funktion, die nur noch von einer Variablen abhängt und die \textbf{der
 Einfachheit halber} ebenfalls $O$ genannt werde:
%
\begin{eqnarray*}
O\left(r, h\right) = 2 \pi r^2 + 2 \pi r h = 2 \pi r^2 + 2 \pi r \frac{1}{\pi r^2} = 2 \left(\pi r^2 + \frac{1}{r}\right) = O\left(r\right) \MDFPeriod
\end{eqnarray*}
%
Nach dieser Manipulation kann die Frage nach der minimalen Oberfläche der Dose ganz analog zu den Extremwertaufgaben von Funktionen bearbeitet werden.
Es wird also die erste Ableitung der Funktion $O$ nach der Variablen $r$ gebildet und diese gleich Null gesetzt:
%
\begin{eqnarray*}
O'\left(r\right) = 2 \left(2\pi r - \frac{1}{r^2}\right) &=& 0\\
\Leftrightarrow\quad 2\pi r &=& \frac{1}{r^2}\\
\Leftrightarrow\quad 2\pi r^3 &=& 1\\
\Leftrightarrow\quad r^3 &=& \frac{1}{2\pi}\\
\Leftrightarrow\quad r &=& \sqrt[3]{\frac{1}{2\pi}} \MDFPeriod
\end{eqnarray*}
%
Für die letzte Äquivalenzumformung wurde verwendet, dass der Radius $r$ keine negativen Werte annehmen kann.
Einsetzen dieses Ergebnisses in die zweite Ableitung von $O$  dient der Überprüfung, ob wirklich ein Minimum gefunden wurde
(${O'}'(r) = 4\pi + 4/r^3$):
\begin{eqnarray*}
{O'}'\left(\sqrt[3]{\frac{1}{2\pi}}\right) = 4\pi + \frac{4}{\left(\sqrt[3]{\frac{1}{2\pi}}\right)^3} = 12\pi > 0 \MDFPeriod
\end{eqnarray*}
%
Für den Radius $r = \sqrt[3]{\frac{1}{2\pi}}$ wird die Oberfläche der zylindrischen Dose mit dem gegebenen Volumen $V = 1$ minimal.
Die Höhe erhält man in diesem Fall zu $h = \frac{1}{\pi \left(\sqrt[3]{\frac{1}{2\pi}}\right)^2} = \sqrt[3]{\frac{4}{\pi}}$.
Wird eine Dose mit diesen Maßen hergestellt, wird für das im Beispiel vorgegebene Volumen der Materialverbrauch minimiert.

\end{MXContent}

\MSubsection{Abschlusstest}
\MLabel{M07_Abschlusstest}

\begin{MTest}{Abschlusstest Kapitel \arabic{section}}
\MDeclareSiteUXID{VBKM07_Abschlusstest}
\begin{MExercise}
In einem Behälter wird um $9$ Uhr eine Temperatur von $-10^{\circ}{\MEinheit[]{C}}$ 
gemessen. Um $15$ Uhr beträgt die Temperatur $-58^{\circ}{\MEinheit[]{C}}$.
Nach weiteren vierzehn Stunden ist die Temperatur auf 
$-140^{\circ}{\MEinheit[]{C}}$ gefallen. 
\begin{MExerciseItems}
\item Wie groß ist die mittlere Änderungsrate pro Stunde der Temperatur aufgrund der %###
ersten und zweiten Messung?

Antwort: \MLParsedQuestion{5}{-8}{3}{DYG12}
%Ergebnis: $(-58 - (-10)) / (15 - 9) = -48 / 6 = -8$.
%
%\item Wodurch drückt sich in der (mittleren) Änderungsrate aus, dass die 
%Temperatur fällt?
%
\item In der (mittleren) Änderungsrate drückt sich die Eigenschaft der fallenden Temperatur dadurch aus, dass die Änderungsrate
{\MLQuestion{20}{negativ}{DFA10}} ist.
\begin{MHint}{Hinweis}
Geben Sie ein Adjektiv an.
\end{MHint}
%{\MQuestion{20}{kleiner null bzw. negativ}} ist.
%
\item Berechnen Sie die mittlere Änderungsrate der Temperatur über die gesamte
Messdauer.

Antwort: \MLParsedQuestion{5}{-6.5}{3}{DXG12}
%Ergebnis: $(-140 - (-10)) / (29 - 9) = -130 / 20 = -6.5$.
\end{MExerciseItems}
\end{MExercise}

\begin{MExercise}
Zu einer Funktion $f: [-3\MIntvlSep 2] \rightarrow \R$, $x \mapsto f(x)$ gehört die Ableitung $f'$, deren 
Graph hier gezeichnet ist:

\begin{center}
\MTikzAuto{%
\begin{small}
\begin{tikzpicture}[line width=1.5pt,scale=0.6, %
declare function={
  fkt(\x) = (\x + abs(\x) + 2)/2;
}
] %[every node/.style={fill=white}] 
%Koordinatenachsen:
\draw[->] (-3.6, 0) -- (3, 0) node[below left]{$x$}; %x-Achse
\draw[->] (0, -0.6) -- (0, 3) node[below left]{$y$}; %y-Achse
%Achsenbeschriftung:
\foreach \x in {-3, -2, -1, 1, 2} \draw (\x, 0) -- ++(0, -0.1) %
 node[below] {$\x$}; 
\foreach \y in {1, 2} \draw (0, \y) -- ++(0.2, 0) %
 node[right] {$\y$};
%\node[below left] at (0, 0) {$0$};
%Funktion:
\draw[domain=-3:2,samples=120,color=\jccolorfkt] %
 plot (\x, {fkt(\x)});
\end{tikzpicture}
\end{small}
}
\end{center}

Die Funktionswerte von $f$ zwischen $-3$ und $0$\\
\begin{tabular}{lll}
\MLCheckbox{0}{JCC1} & \ \ & sind konstant,\\
\MLCheckbox{1}{JCC2} & \ \ & nehmen um $3$ zu,\\
\MLCheckbox{0}{JCC3} & \ \ & nehmen ab.
\end{tabular}
\ \\

Die Funktion $f$ besitzt an der Stelle $0$\\
\begin{tabular}{lll}
\MLCheckbox{0}{JCC4} & \ \ & eine Sprungstelle,\\
\MLCheckbox{0}{JCC5} & \ \ & keine Ableitung,\\
\MLCheckbox{1}{JCC6} & \ \ & $1$ als Wert der Ableitung.
\end{tabular}
\end{MExercise}

\begin{MExercise} %Rechenregeln
Berechnen Sie für
\begin{MExerciseItems}
 \item $f: \{ x \in \R \MCondSetSep x > 0 \} \rightarrow \R$ mit $f(x) := \ln\left(x^3 + x^2\right)$ den Wert der Ableitung $f'$ an der Stelle $x$:\newline
$f'(x) = $\MLSimplifyQuestion{30}{(3 x + 2)/(x^2 + x)}{1}{x}{20}{0}{SIMPLE13} \MDFPeriod
\item $g: \R \rightarrow \R$ mit $g(x) := x \cdot \MEU^{-x}$ den Wert der zweiten Ableitung ${g'}'$ an der Stelle $x$:\newline
${g'}'(x) = $\MLSimplifyQuestion{30}{(x - 2) * exp(-x)}{10}{x}{4}{0}{SIMPLE14} \MDFPeriod
\end{MExerciseItems}
\MInputHint{Klammern Sie die Terme, um Missverständnisse zu vermeiden, z.B. schreiben Sie $\frac{x+1}{(x+2)^2}$ als \texttt{(x+1)/((x+2)^2)}. Geben Sie $\MEU^{-x}$ als \texttt{exp(-x)} ein.} %###
\end{MExercise}

\begin{MExercise} %Eigenschaften
Betrachten Sie die Funktion $f: \MoIl 0\MIntvlSep  \infty\MoIr \rightarrow \R$, $x \mapsto f(x)$ mit $f'(x) = x \cdot \ln x$.
In welchen Bereichen ist $f$ monoton fallend, in welchen Bereichen ist $f$ konkav?
%
Geben Sie als Bereiche möglichst große offene Intervalle $\MoIl c\MIntvlSep  d\MoIr$ an:
\begin{MExerciseItems}
\item $f$ ist auf \MLIntervalQuestion{12}{(0, 1)}{4}{INT2} monoton fallend.
\item $f$ ist auf \MLIntervalQuestion{12}{(0, 1/e)}{4}{INT3} konkav.
\end{MExerciseItems}
\MInputHint{Offene Intervalle können in der Form $(a;b)$ eingetippt werden, geschlossene Intervalle als $[a;b]$, $a$ und $b$ dürfen
beliebige Ausdrücke sein. Verwenden Sie bei der Intervalleingabe nicht die Notation $]a;b[$ für offene Intervalle. Schreiben Sie \texttt{infty} oder
\texttt{unendlich} für $\infty$ in Ihrer Antwort.}
\end{MExercise}

%TODO Eingabe und Loesung fehlt
%\begin{MExercise} %Anwendungen
%Berechnen Sie alle lokalen und globalen Extremstellen und Wendestellen der 
%Funktion $f: \R \rightarrow \R$ mit $f(x) := (x^2 - 3 x) \cdot \MEU^{2 x}$ für 
%$x \in \R$.
%\end{MExercise}


\end{MTest}


\clearpage
\MPrintIndex

\end{document}

%Dateiende.

