%LaTeX-2e-File, Liedtke, 20140731.
%Inhalt: Einf"uhrung in die Differentialrechnung, Abschnitt 4.
%zuletzt bearbeitet: 20140922.

%Abschnitt 4: Eigenschaften
\begin{MContent} 

\MSubsubsection{Stetigkeit differenzierbarer Funktionen}
%\begin{MXContent}{Stetigkeit differenzierbarer Funktionen}{Stetigkeit}{STD}
 
%\begin{MXInfo}{Stetigkeit differenzierbarer Funktionen}
Wenn eine Funktion $f$ an der Stelle $x_0$ differenzierbar ist, dann ist $f$ an 
der Stelle $x_0$ auch stetig.
%\end{MXInfo}

Folglich ist eine ("uberall) differenzierbare Funktion ("uberall) stetig.
Insbesondere hat eine auf einem Intervall definierte differenzierbare Funktion
keine Sprungstellen.

Andererseits zeigt das Beispiel der stetigen Funktion $f: \R \to \R$ mit 
$f(x) := |x|$ f"ur $x \in \R$, dass aus der Stetigkeit allein nicht folgt, dass
$f$ auch differenzierbar ist. Denn $f$ in an der Stelle $x_0 = 0$ nicht 
differenzierbar.
%\end{MXContent}

\MSubsubsection{Monotonie}
%\begin{MXContent}{Monotonie}

Mit der Ableitung kann das lokale Wachstumsverhalten untersucht werden, das 
hei"st ob f"ur gr"o"ser werdende $x$-Werte die Funktionswerte gr"o"ser oder 
kleiner werden.
Dazu wird eine Funktion $f: D \to \R$ betrachtet, die auf $(a, b) \subseteq D$ 
differenzierbar ist.
%Bild:
\begin{center}
\ifttm
\MGraphicsSolo{\MPfadBilder/jb07A4_Monotonie.png}{scale=0.5}
%\MGraphicsSolo{jb07A4_Monotonie.png}{scale=1}
\else
\renewcommand{\jTikZScale}{1.0}
%Bild: {\MPfadBilder/jb07A4_Monotonie.tex}

%LaTeX-File, Liedtke, 20140829.
%VBKM-Modul 7 Differentialrechnung: Bild zur Monotonie.
%Bildname: jb07A4_Monotonie.tex.
%Erstellt: 20140829, Liedtke.
%Bearbeitet: 20140901, Liedtke (Dateiname ohne Endung erfasst).

\begin{small}
\tikzsetnextfilename{jb07A4_Monotonie}
\begin{tikzpicture}[line width=1.5pt,scale=\jTikZScale, %
declare function={
  x1 = 2;
  x0 = 4;
  fkt(\x) = 1/4*(\x - 3)*(\x - 3) + 0.75;
  TangenteAblplus(\x) = 1 + 1/2*(\x - x0); % $f(x_0) = f(4) = 1$.
  TangenteAblminus(\x) = 1 - 1/2*(\x - x1); % $f(x_1) = f(2) = 1$.
}
] %[every node/.style={fill=white}] 
%,every node/.style={fill=white}] 
%Koordinatenachsen:
\draw[->] (-0.6, 0) -- (6, 0) node[below left]{$x$}; %x-Achse
\draw[->] (0, -0.6) -- (0, 3) node[below left]{$y$}; %y-Achse
%Achsenbeschriftung:
\foreach \x in {1, 2, 3, 4, 5} \draw (\x, 0) -- ++(0, -0.1); %
% node[below] {$\x$}; 
\foreach \y in {1, 2} \draw (0, \y) -- ++(-0.1, 0); %
% node[below left] {$\y$};
%\node[below left] at (0, 0) {$0$};
%Hilfslinien:
\draw[color=black!50!white] (x0, {fkt(x0)}) -- ({x0+1}, {fkt(x0)});
\draw[->,color=black!50!white] %
 ({x0+1}, {fkt(x0)}) -- ({x0+1}, {TangenteAblplus(x0+1)});
%
\draw[color=black!50!white] (x1, {fkt(x1)}) -- ({x1-1}, {fkt(x1)});
\draw[->,color=black!50!white] %
 ({x1-1}, {TangenteAblminus(x1-1)}) -- ({x1-1}, {fkt(x1)});
%Funktion:
\draw[domain=0.8:5.2,samples=120,color=\jccolorfkt] %
 plot (\x, {fkt(\x)});
%Tangenten:
\draw[domain={x1-1.2}:{x1+0.8},samples=120,color=blue!50!black] %
 plot (\x, {TangenteAblminus(\x)});
\filldraw[color=blue!50!black] (x0, {fkt(x0)}) circle (1pt); % Ber"uhrpunkt.
%
\draw[domain={x0-0.8}:{x0+1.2},samples=120,color=blue!50!black] %
 plot (\x, {TangenteAblplus(\x)});
\filldraw[color=blue!50!black] (x1, {fkt(x1)}) circle (1pt); % Ber"uhrpunkt.
%Beschriftung:
\node[right] at (5.1, 1.25) {$f'(x_0) = m_0 > 0$};
\node[style={fill=white},left] at (0.9, 1.25) {$f'(x_1) = m_1 < 0$};
\end{tikzpicture}
\end{small}
%end of file
\fi
\end{center}
%Bildende.

%\begin{MXInfo}{Monotonie}
Wenn $f'(x) \leq 0$ f"ur alle $x$ zwischen $c$ und $d$ gilt, dann ist $f$ 
auf dem Intervall $(c; d)$ monoton fallend.

Wenn $f'(x) \geq 0$ f"ur alle $x$ zwischen $c$ und $d$ gilt, dann ist $f$ 
auf dem Intervall $(c; d)$ monoton wachsend.
%\end{MXInfo}

Somit gen"ugt es, das Vorzeichen der Ableitung $f'$ zu bestimmen, um zu 
erkennen, ob eine Funktion monoton wachsend oder monoton fallend ist.
%\end{MXContent}

\begin{MExample}
Die Funktion $f: \R \to \R, x \mapsto x^3$ ist differenzierbar mit 
$f(x) = 3 x^2$. Da $x^2 \geq 0$ f"ur alle $x \in \R$ gilt, ist 
$f'(x) \geq 0$ und damit $f$ monoton wachsend.

F"ur $g: \R \to \R$ mit $g(x) = 2 x^3 + 6 x^2 - 18 x + 10$ hat die Ableitung
$g'(x) = 6 x^2 + 12 x - 18 = 6 (x + 3) (x - 1)$ die Nullstellen $x_1 = -3$ 
und $x_2 = 1$.
Mit Hilfe folgender Tabelle wird bestimmt, in welchen Bereichen die Ableitung 
von $g$ positiv bzw. negativ ist, woraus sich dann die Monotoniebereiche von 
$g$ ergeben. Der Eintrag $+$ besagt, dass der betrachtete Term im angegebenen
Intervall positiv ist. Wenn er negativ ist, wird $-$ eingetragen.
\[
\begin{array}{cccc}
x & x < -3 & -3 < x < 1 & 1 < x \\
\hline
x + 3 & - & + & + \\
x - 1 & - & - & + \\
g'(x) & - & - & + \\
g \text{ monoton} & \text{wachsend} & \text{fallend} & \text{wachsend} \\
\end{array}
\]
Die Funktion $h: \R \setminus \{ 0 \} \to \R$ mit $h(x) = \frac{1}{x}$
hat die Ableitung $h'(x) = - \frac{1}{x^2}$. Hier gilt 
$h'(x) < 0$ f"ur alle $x \neq 0$.

Dennoch ist $h$ nicht monoton fallend, da beispielsweise 
$h(-2) = -\frac{1}{2} < 1 = h(1)$ gilt. Der Grund f"ur dieses "uberraschende 
Ergebnis ist, dass der Definitionsbereich von $h$ kein Intervall ist. Die 
Funktion $h$ ist auf $(-\infty; 0)$ monoton fallend, das hei"st, die 
Einschr"ankung von $h$ auf dieses Intervall ist monoton fallend. Zudem ist 
$h$ f"ur alle $x > 0$ monoton fallend.
\end{MExample}

\MSubsubsection{Kr"ummungseigenschaften}
%\begin{MXContent}{Kr"ummungseigenschaften}

Gegeben ist eine Funktion $f: D \to \R$, die auf $(a; b) \subseteq D$ 
zweimal differenzierbar ist.

%\begin{MXInfo}{Kr"ummungseigenschaften}
Wenn ${f'}'(x) \geq 0$ f"ur alle $x$ zwischen $c$ und $d$ gilt, dann ist $f$ 
auf dem Intervall $(c; d)$ konvex.

Wenn ${f'}'(x) \leq 0$ f"ur alle $x$ zwischen $c$ und $d$ gilt, dann ist $f$ 
auf dem Intervall $(c; d)$ konkav.
%\end{MXInfo}

Somit gen"ugt es, das Vorzeichen der zweiten Ableitung ${f'}'$ zu bestimmen, 
um zu erkennen, ob eine Funktion konvex (linksgekr"ummt) oder konkav 
(rechtsgekr"ummt) ist.
%\end{MXContent}

\begin{MXInfo}{Anmerkung zur Notation}
Die zweite und weitere {\glqq}h"ohere{\grqq} Ableitungen oft mit nat"urlichen 
Zahlen in runden Klammern gekennzeichnet: Die $k$-te Ableitung wird dann mit
$f^{(k)}$ bezeichnet. Diese Bezeichnung wird besonders in Formeln auch f"ur 
die (erste) Ableitung f"ur $k=1$ und f"ur die Funktion $f$ selbst f"ur $k=0$
verwendet.

Damit bezeichnet 
\begin{itemize}
\item $f^{(0)} = f$ die Funktion $f$, 
\item $f^{(1)} = f'$ die (erste) Ableitung,
\item $f^{(2)} = {f'}'$ die zweite Ableitung,
\item $f^{(3)}$ die dritte Ableitung von $f$, etc. (sofern diese existieren).
\end{itemize}
\end{MXInfo} 

Das folgende Beispiel zeigt, dass eine monoton wachsende Funktion in einem 
Bereich konvex und in einem anderen konkav sein kann.

\begin{MExample}
Die Funktion $f: \R \to \R, x \mapsto x^3$ ist zweimal differenzierbar. Wegen
$f'(x) = 3 x^2 \geq 0$ ist $f$ monoton wachsend.

Weiter ist $f^{(2)}(x) = 6 x$. Somit ist f"ur $x < 0$ auch $f^{(2)} < 0$
und damit $f$ hier konkav (nach rechts gekr"ummt), und f"ur $x > 0$ ist
$f^{(2)}(x) > 0$, sodass $f$ f"ur $x > 0$ konvex (nach links gekr"ummt) ist.
\end{MExample}
\end{MContent}

%end of file.

