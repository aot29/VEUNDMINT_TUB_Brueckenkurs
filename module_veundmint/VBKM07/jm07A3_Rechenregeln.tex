%LaTeX-2e-File, Liedtke, 20140731.
%Inhalt: Einf"uhrung in die Differentialrechnung, Abschnitt 3.
%zuletzt bearbeitet: 20140922.

%\MSubsubsection{Rechenregeln}
\begin{MContent}

Gegeben sind beliebige differenzierbare Funktionen $u, v: D \rightarrow \R$ und eine reelle
Zahl $r$.


\MSubsubsection{Vielfache und Summen von Funktionen}
 
\begin{MXInfo}{Summenregel und vielfache von Funktionen}
Sind $u$ und $v$ als differenzierbare Funktionen vorgegeben, dann ist auch die Summe der Funktionen $f(x) := u(x) + v(x)$ differenzierbar,
und es gilt
\begin{equation}
f'(x) = u'(x) + v'(x) %%
\end{equation}

Auch das $r$-fache der Funktion $f(x) := r \cdot u(x)$ ist differenzierbar,
und es gilt
\begin{equation}
f'(x) = r \cdot u'(x) %%
\end{equation}
\end{MXInfo}

Mit diesen beiden Rechenregeln und der Ableitungsregel f�r Monome l�sst sich z.B. jedes beliebige Polynom ableiten.

Sehen wir uns einige Beispiele an:

\begin{MExample}
Das Polynom $f(x) = \frac{1}{4} x^3 - 2 x^{2} +5 $ ist differenzierbar und wir erhalten
\[
f'(x) = \frac{3}{4} x^2 - 4 x %%
\]
Die Ableitung der Funktion $g(x) = x^3 + \ln(x)$ f"ur $x > 0$ ist
\[
g'(x) = 3 x^2 + \frac{1}{x} = \frac{3 x^3 + 1}{x} %%
\]
%Mit $\ln(x^3) = 3 \ln(x)$ ergibt sich die Ableitung von 
%$g(x) = \ln(x^3) = 3 \ln(x)$ f"ur $x > 0$ zu
%\[
%g'(x) = \frac{3}{x} %%
%\]
Leiten wir die Funktion
$h(x) = 4^{-1} \cdot x^2 - \sqrt{x} %
 = \frac{1}{4} x^2 + (-1) \cdot x^{\frac{1}{2}}$ ist f"ur $x > 0$ ab, dann erhalten wir
\[
h'(x) = \frac{1}{2} x - \frac{1}{2} x^{-\frac{1}{2}} %
 = \frac{x^{\frac{3}{2}} - 1}{2 \sqrt{x}} %%
\]
\end{MExample}


\MSubsubsection{Produkt und Quotient von Funktionen}

\begin{MXInfo}{Produkt- und Quotientenregel}
Auch das Produkt der Funktionen $f(x) := u(x) \cdot v(x)$ ist differenzierbar, und es gilt die \textbf{Produktregel}
\begin{equation}
f'(x) = u'(x) \cdot v(x) +  u(x) \cdot v'(x) %%
\end{equation}

Der Quotient der Funktionen $f(x) := \frac{u(x)}{v(x)}$ ist f"ur alle $x$ mit $v(x) \neq 0$ definiert und differenzierbar, und es gilt die \textbf{Quotientenregel}
\ifttm
\begin{equation}
f'(x) = \frac{u'(x) \cdot v(x) - u(x) \cdot v'(x)}{\left(v'(x)\right)^2} %%
\end{equation}
\else
\begin{equation}
f'(x) = %
\frac{u'(x) \cdot v(x) \,\textcolor{red}{\mathbf{-}}\, u(x) \cdot v'(x)}%
{\left(v'(x)\right)^2} %%
\end{equation}
\end{MXInfo}

Veranschaulichen wir auch diese Rechenregeln anhand einiger Beispiele:

\begin{MExample}
Bestimmen wir die Ableitung von $f(x) = x^2 \cdot \MEU^x$. Wenden wir die Produktregel an, k�nnen wir $u(x) = x^2$ und $v(x) = \MEU^x$ ablesen und die Ableitungen dieser Funktionen bestimmen. Dann gilt $u'(x) = 2x$ und $v'(x) = \MEU^x$. Setzen wir diese Teilergebnisse mit der Produktregel zusammen, dann erhalten wir die Ableitung der Funktion $f$:
\[
f'(x) = 2 x \MEU^x + x^2 \MEU^x = (x^2 + 2x) \MEU^x %%
\]
Untersuchen wir die Funktion $g(x) = \tan(x) = \frac{\sin(x)}{\cos(x)}$. Hier k�nnen wir durch Vergleich $u(x) = \sin(x)$ und $v(x) = \cos(x)$ ablesen. Mit $u'(x) = \cos(x)$ und $v'(x) = -\sin(x)$ k�nnen wir auch hier die Teilergebnisse mit der Quotientenregel zur Ableigung der Funktion $g$ zusammensetzen:
\[
g'(x) = \frac{\cos(x) \cdot \cos(x) - \sin(x) \cdot (-\sin(x))}{\cos^2(x)} %
\]
Mit dem in Modul \MRef{VBKM05} (Abschnitt \MRef{VBKM05_Abschnitt:TrigonometrieAmDreieck}) abgeleiteten Zusammenhang $\sin^2(x) + \cos^2(x) = 1$ k�nnen wir dieses Ergebnis zu einem der folgenden Ausdr�cke zusammenfassen:
\[
g'(x) = 1 + \left(\frac{\sin(x)}{\cos(x)}\right)^2 %
 = 1 + \tan^2(x) %
 = \frac{1}{\cos^2(x)} %%
\]
\end{MExample}

\MSubsubsection{Verkettung von Funktionen}

Zum Abschluss wollen wir die Verkettung (Modul \MRef{VBKM06}, Abschnitt \MRef{Verkettung}) von Funktionen untersuchen. Wir wollen also wissen, was passiert, wenn wir eine Funktion $u$ (innere Funktion) in eine andere Funktion $v$ (�u�ere Funktion) einsetzen. Eine solche Verkettung wird in der Mathematik durch die Schreibweise $f(x) := (v \circ u)(x) = v(u(x))$ dargestellt. Diese bedeutet, dass wir den Wert einer Funktion $u$ in Abh�ngigkeit von der Variable $x$ bestimmen. Diesen so berechneten Wert $u$ verwenden wir als neue Variable, die wir in $v$ einsetzen. Aus dieser Kombination berechnen wir schlie�lich den endg�ltigen Funktionswert $v(u(x))$.

\begin{MXInfo}{Kettenregel}
Die Ableitung der Funktion $f(x) := (v \circ u)(x) = v(u(x))$ kann mit der \textbf{Kettenregel} bestimmt werden:
\begin{equation}
f'(x) = v'(u) \cdot u'(x) %%
\end{equation}
Hierbei ist $v'(u)$ als Ableitung der Funktion $v$ nach $u$ zu verstehen. Bei diesem Teil der Ableitung wird die Abh�ngigkeit der Funktion $u$ von der Variable $x$ nicht ber�cksichtigt. Es gilt der Merksatz: Innere Ableitung $\times$ �u�ere Ableitung.
\end{MXInfo}

Machen wir uns diese Ableitungsregel an einigen Beispielen klar:

\begin{MExample}
Bestimmen wir die Ableitung der Funktion $f(x) = (3 - 2 x)^5$. Im Vergleich mit der Kettenregel m�ssen wir eine innere und eine �u�ere Funktion ablesen. Verwenden wir $u(x) = 3 - 2x$ als innere Funktion, dann k�nnen wir $v(u) = u^5$ als �u�ere Funktion ablesen. Setzen wir $u(x)$ f�r $u$ in $v(u)$ ein, erhalten wir die urspr�ngliche Funktion $f$.

Um die Ableitung von $f$ zu bestimmen, leiten wir zun�chst $u$ nach $x$ ab und erhalten $u'(x) = -2$ als innere Ableitung. F�r die �u�ere Ableitung leiten wir $v$ nach $u$ ab und finden $v'(u) = 5 u^4$. Setzen wir diese Zwischenergebnisse in die Kettenregel ein, so bekommen wir die Ableitung $f'$ der Funktion $f$:
\[
f'(x) = 5 u^4 \cdot (-2) = 5 (3 - 2 x)^4 \cdot (-2) = -10 (3 - 2 x)^4 %%
\]

Sehen wir uns ein zweites Beispiel an und bestimmen die Ableitung von $g(x) = \MEU^{x^3}$ und verwenden die Zuordnung $u(x) = x^3$ f�r die innere Funktion und $v(u) = \MEU^u$ f�r die �u�ere Funktion. Bestimmen wir die innere und �u�ere Ableitung, dann erhalten wir $u'(x) = 3 x^2$ und $v'(u) = \MEU^u$. Setzen wir beides in die Kettenregel ein, dann finden wir die Ableitung der Funktion $g$:
\[
g'(x) = \MEU^{u} \cdot 3 x^2 = \MEU^{x^3} \cdot 3 x^2 = 3 x^2 \MEU^{x^3} %%
\]
\end{MExample}

\end{MContent}

%end of file.

