%LaTeX-2e-File, Liedtke, 20140731.
%Inhalt: Einf"uhrung in die Differentialrechnung, Abschnitt 3.
%zuletzt bearbeitet: 20140922.

%\MSubsubsection{Rechenregeln}
\begin{MContent}

Gegeben sind differenzierbare Funktionen $u, v: D \to \R$ und eine reelle
Zahl $r$.


\MSubsubsection{Vielfache und Summen von Funktionen}
 
%\begin{MXInfo}{Ableitung von Summen und von Vielfachen von Funktionen}
Dann ist auch die Summe der Funktionen $f(x) := u(x) + v(x)$ differenzierbar,
und es gilt
\begin{equation}
f'(x) = u'(x) + v'(x) %%
\end{equation}

Es ist auch das $r$-fache der Funktion $f(x) := r \cdot u(x)$ differenzierbar,
und es gilt
\begin{equation}
f'(x) = r \cdot u'(x) %%
\end{equation}
%\end{MXInfo}

\begin{MExample}
Die Ableitung von $f(x) = x^3 + \ln(x)$ f"ur $x > 0$ ist
\[
f'(x) = 3 x^2 + \frac{1}{x} = \frac{3 x^3 + 1}{x} %%
\]
Mit $\ln(x^3) = 3 \ln(x)$ ergibt sich die Ableitung von 
$g(x) = \ln(x^3) = 3 \ln(x)$ f"ur $x > 0$ zu
\[
g'(x) = \frac{3}{x} %%
\]
Die Ableitung von 
$h(x) = 4^{-1} \cdot x^2 - \sqrt{x} %
 = \frac{1}{4} x^2 + (-1) \cdot x^{\frac{1}{2}}$ ist f"ur $x > 0$ dann
\[
h'(x) = \frac{1}{2} x - \frac{1}{2} x^{-\frac{1}{2}} %
 = \frac{x^{\frac{3}{2}} - 1}{2 \sqrt{x}} %%
\]
\end{MExample}


\MSubsubsection{Produkt und Quotient von Funktionen}

Dann ist auch das Produkt der Funktionen $f(x) := u(x) \cdot v(x)$ 
differenzierbar, und es gilt
\begin{equation}
f'(x) = u'(x) \cdot v(x) +  u(x) \cdot v'(x) %%
\end{equation}

Es ist auch der Quotient der Funktionen $f(x) := \frac{u(x)}{v(x)}$ f"ur 
alle $x$ mit $v(x) \neq 0$ definiert und differenzierbar, und es gilt
\ifttm
\begin{equation}
f'(x) = \frac{u'(x) \cdot v(x) - u(x) \cdot v'(x)}{\left(v'(x)\right)^2} %%
\end{equation}
\else
\begin{equation}
f'(x) = %
\frac{u'(x) \cdot v(x) \,\textcolor{red}{\mathbf{-}}\, u(x) \cdot v'(x)}%
{\left(v'(x)\right)^2} %%
\end{equation}
\fi

\begin{MExample}
Die Ableitung von $f(x) = x^2 \cdot \MEU^x$ ist
\[
f'(x) = 2 x \MEU^x + x^2 \MEU^x = (x^2 + 2x) \MEU^x %%
\]
Die Ableitung von $g(x) = \tan(x) = \frac{\sin(x)}{\cos(x)}$ ist
mit $\sin^2(x) + \cos^2(x) = 1$ dann
\[
g'(x) = \frac{\cos(x) \cdot \cos(x) - \sin(x) \cdot (-\sin(x))}{\cos^2(x)} %
 = 1 + \left(\frac{\sin(x)}{\cos(x)}\right)^2 %
 = 1 + \tan^2(x) %
 = \frac{1}{\cos^2(x)} %%
\]
\end{MExample}


\MSubsubsection{Verkettung von Funktionen}

Wenn die Funktion $u$ mit der Funktion $v$ verkettet werden kann (wenn $u$ in 
$v$ eingesetzt werden kann), dann ist auch die Verkettung
$f(x) := (v \circ u)(x) = v(u(x))$ differenzierbar, und es gilt
\begin{equation}
f'(x) = v'(u(x)) \cdot u'(x) %%
\end{equation}
Hier ist $v'(u(x))$ der Wert der Ableitung $v'$, der sich an der Stelle $u(x)$ 
ergibt. %ausgewertet wird.

\begin{MExample}
Die Ableitung von $f(x) = (3 - 2 x)^5$ ist
\[
f'(x) = 5 (3 - 2 x)^4 \cdot (-2) = -10 (3 - 2 x)^4 %%
\]
Die Ableitung von $g(x) = \MEU^{x^3}$ ist
\[
g'(x) = \MEU^{x^3} \cdot 3 x^2 = 3 x^2 \MEU^{x^3} %%
\]
\end{MExample}


\MSubsubsection{Umkehrfunktion}

Wenn die Funktion $u$ umkehrbar ist und ihre Ableitung an der Stelle $x_0$ 
ungleich null ist, also $u'(x_0) \neq 0$ gilt, dann ist die 
Umkehrfunktion $f(y) := u^{-1}(y)$ an der Stelle $y_0 := u(x_0)$ differenzierbar.
%Mit der Kettenregel, angewandt auf $x = u^{-1}(u(x)) = f(u(x))$ gilt -- da 
%die Ableitung der identischen Abbildung $x \mapsto x$ konstant $1$ ist -- dann
Mit der Kettenregel, angewandt auf $y = u(u^{-1}(y)) = u(f(y))$ gilt -- da 
die Ableitung 
%der identischen Abbildung $y \mapsto y$ konstant $1$ ist 
des Terms $y$ auf der linken Seite -- dann
\begin{equation}\MLabel{eq:AblUmkehrfkta}
%1 = f'(u(x_0)) \cdot u'(x_0) %
%\quad \text{und damit} \quad %
%f'(y_0) = f'(u(x_0)) = \frac{1}{u'(x_0)} = \frac{1}{u'(f(y_0))}. %%
1 = u'(f(y_0)) \cdot f'(y_0) %
\quad \text{und damit} \quad %
f'(y_0) = \frac{1}{u'(f(y_0))}. %%
\end{equation}
Ebenso kann von $x = u^{-1}(u(x)) = f(u(x))$ ausgegangen werden: Mit der
Kettenregel und $x_0 = u^{-1}(y_0) = f(y_0)$ ergibt sich dann
\begin{equation}\MLabel{eq:AblUmkehrfktb}
1 = f'(u(x_0)) \cdot u'(x_0) %
\quad \text{und damit} \quad %
f'(y_0) = f'(u(x_0)) = \frac{1}{u'(x_0)} = \frac{1}{u'(f(y_0))}. %%
\end{equation}

\begin{MXInfo}{Ergebnis: Ableitung der Umkehrfunktion}
Ersetzt man in den Gleichungen (\MNRef{eq:AblUmkehrfkta}) bzw. 
(\MNRef{eq:AblUmkehrfktb}) dann $f$ durch $u^{-1}$ und schreibt $y$ f"ur $y_0$, 
so ergibt sich die Regel
\[
\left(u^{-1}\right)'(y) = \frac{1}{u'\left(u^{-1}(y)\right)}. %%
\]
f"ur die Ableitung der Umkehrfunktion (an der Stelle $y$).

Wie "ublich, kann die unabh"angige Variable der Umkehrfunktion auch wieder 
mit $x$ bezeichnet werden.
\end{MXInfo}

\begin{MExample}
Die Funktion $u\colon (0; \infty) \to \R$ mit $u(x) = x^2$ ist umkehrbar und 
differenzierbar mit $u'(x) = 2 x$.
Damit ist die Ableitung der Umkehrfunktion $u^{-1}(y) = \sqrt{y}$ 
f"ur $y > 0$ dann
\[
\left(u^{-1}\right)'(y) = \frac{1}{u'\left(u^{-1}(y)\right)} %
 = \frac{1}{2 \left(u^{-1}(y)\right)} %%
 = \frac{1}{2 \sqrt{y}} %%
\]
"Ublicherweise wird die unabh"angige Variable der Umkehrfunktion noch mit $x$ 
bezeichnet.
Damit ist die Ableitung f"ur die Wurzelfunktion hergeleitet. 
In derselben Weise erh"alt man die Ableitungsregel von $x^{\frac{1}{n}}$ 
f"ur $n \neq 0$.
\end{MExample}
\end{MContent}

%end of file.

