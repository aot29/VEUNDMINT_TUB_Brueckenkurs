%LaTeX-2e-Datei, Liedtke, 20140729.
%Inhalt: Modul 7 VEMINT-Basiskurs Mathematik.
%Thema:  Differentialrechnung.
%zuletzt bearbeitet: 20140930.

%\input{mintmod_ed.tex} % anderer Name nur(!) f"ur PDF-Version m"oglich.

% ENCODING: ASCII (important as must be compatible both with utf8 and latin1 based authors)
% MINTMOD Version P0.1.0, needs to be consistent with preprocesser object in tex2x and MPragma-Version at the end of this file

% Parameter aus Konvertierungsprozess (PDF und HTML-Erzeugung wenn vom Konverter aus gestartet) werden hier eingefuegt, Preambleincludes werden am Schluss angehaengt

\newif\ifttm                % gesetzt falls Uebersetzung in HTML stattfindet, sonst uebersetzung in PDF

% Wahl der Notationsvariante ist im PDF immer std, in der HTML-Uebersetzung wird vom Konverter die Auswahl modifiziert
\newif\ifvariantstd
\newif\ifvariantunotation
\variantstdtrue % this string was added by tex2x VEUNDMINT preprocessor
 % Diese Zeile wird vom Konverter erkannt und ggf. modifiziert, daher nicht veraendern!


\def\MOutputDVI{1}
\def\MOutputPDF{2}
\def\MOutputHTML{3}
\newcounter{MOutput}

\ifttm
%\usepackage{german}
\usepackage{array}
\usepackage{amsmath}
\usepackage{amssymb}
\usepackage{amsthm}
\else
\documentclass[oneside]{scrbook}
\usepackage{etex}
\usepackage[utf8]{inputenc}
\usepackage{textcomp}
\usepackage[ngerman,english]{babel}
\usepackage[pdftex]{color}
\usepackage{xcolor}
\usepackage{graphicx}
\usepackage[all]{xy}
\usepackage{fancyhdr}
\usepackage{verbatim}
\usepackage{array}
\usepackage{float}
\usepackage{makeidx}
\usepackage{amsmath}
\usepackage{amstext}
\usepackage{amssymb}
\usepackage{amsthm}
\usepackage[ngerman,english]{varioref}
\usepackage{framed}
\usepackage{supertabular}
\usepackage{longtable}
%\usepackage{maxpage}
\usepackage{tikz}
\usepackage{tikzscale}
\usepackage{tikz-3dplot}
\usepackage{bibgerm}
\usepackage{chemarrow}
\usepackage{polynom}
%\usepackage{draftwatermark}
\usepackage{pdflscape}
\usetikzlibrary{calc}
\usetikzlibrary{through}
\usetikzlibrary{shapes.geometric}
\usetikzlibrary{arrows}
\usetikzlibrary{intersections}
\usetikzlibrary{decorations.pathmorphing}
\usetikzlibrary{external}
\usetikzlibrary{patterns}
\usetikzlibrary{fadings}
\usepackage[colorlinks=true,linkcolor=blue]{hyperref}
\usepackage[all]{hypcap}
%\usepackage[colorlinks=true,linkcolor=blue,bookmarksopen=true]{hyperref}
\usepackage{ifpdf}

\usepackage{movie15}

\setcounter{tocdepth}{2} % In Inhaltsverzeichnis bis subsection
\setcounter{secnumdepth}{3} % Nummeriert bis subsubsection

\setlength{\LTpost}{0pt} % Fuer longtable
\setlength{\parindent}{0pt}
\setlength{\parskip}{8pt}
%\setlength{\parskip}{9pt plus 2pt minus 1pt}
\setlength{\abovecaptionskip}{-0.25ex}
\setlength{\belowcaptionskip}{-0.25ex}
\fi

\ifttm
\newcommand{\MDebugMessage}[1]{\special{html:<!-- debugprint;;}#1\special{html:; //-->}}
\else
%\newcommand{\MDebugMessage}[1]{\immediate\write\mintlog{#1}}
\newcommand{\MDebugMessage}[1]{}
\fi

\def\MPageHeaderDef{%
\pagestyle{fancy}%
\fancyhead[r]{(C) VE\&MINT-Project}
\fancyfoot[c]{\thepage\\--- CCL BY-SA 3.0 ---}
}


\ifttm%
\def\MRelax{}%
\else%
\def\MRelax{\relax}%
\fi%

%--------------------------- Uebernahme von speziellen XML-Versionen einiger LaTeX-Kommandos aus xmlbefehle.tex vom alten Kasseler Konverter ---------------

\newcommand{\MSep}{\left\|{\phantom{\frac1g}}\right.}

\newcommand{\ML}{L}

\newcommand{\MGGT}{\mathrm{gcd}}

\ifttm
% Verhindert dass die subsection-nummer doppelt in der toccaption auftaucht (sollte ggf. in toccaption gefixt werden so dass diese Ueberschreibung nicht notwendig ist)
\renewcommand{\thesubsection}{}
% Kommandos die ttm nicht kennt
\newcommand{\binomial}[2]{{#1 \choose #2}} %  Binomialkoeffizienten
\newcommand{\eur}{\begin{html}&euro;\end{html}}
\newcommand{\square}{\begin{html}&square;\end{html}}
\newcommand{\glqq}{"'}  \newcommand{\grqq}{"'}
\newcommand{\nRightarrow}{\special{html: &nrArr; }}
\newcommand{\nmid}{\special{html: &nmid; }}
\newcommand{\nparallel}{\begin{html}&nparallel;\end{html}}
\newcommand{\mapstoo}{\begin{html}<mo>&map;</mo>\end{html}}

% Schnitt und Vereinigungssymbole von Mengen haben zu kleine Abstaende; korrigiert:
\newcommand{\ccup}{\,\!\cup\,\!}
\newcommand{\ccap}{\,\!\cap\,\!}


% Umsetzung von mathbb im HTML
\renewcommand{\mathbb}[1]{\begin{html}<mo>&#1opf;</mo>\end{html}}
\fi

%---------------------- Strukturierung ----------------------------------------------------------------------------------------------------------------------

%---------------------- Kapselung des sectioning findet auf drei Ebenen statt:
% 1. Die LateX-Befehl
% 2. Die D-Versionen der Befehle, die nur die Grade der Abschnitte umhaengen falls notwendig
% 3. Die M-Versionen der Befehle, die zusaetzliche Formatierungen vornehmen, Skripten starten und das HTML codieren
% Im Modultext duerfen nur die M-Befehle verwendet werden!

\ifttm

  \def\Dsubsubsubsection#1{\subsubsubsection{#1}}
  \def\Dsubsubsection#1{\subsubsection{#1}\addtocounter{subsubsection}{1}} % ttm-Fehler korrigieren
  \def\Dsubsection#1{\subsection{#1}}
  \def\Dsection#1{\section{#1}} % Im HTML wird nur der Sektionstitel gegeben
  \def\Dchapter#1{\chapter{#1}}
  \def\Dsubsubsubsectionx#1{\subsubsubsection*{#1}}
  \def\Dsubsubsectionx#1{\subsubsection*{#1}}
  \def\Dsubsectionx#1{\subsection*{#1}}
  \def\Dsectionx#1{\section*{#1}}
  \def\Dchapterx#1{\chapter*{#1}}

\else

  \def\Dsubsubsubsection#1{\subsubsection{#1}}
  \def\Dsubsubsection#1{\subsection{#1}}
  \def\Dsubsection#1{\section{#1}}
  \def\Dsection#1{\chapter{#1}}
  \def\Dchapter#1{\title{#1}}
  \def\Dsubsubsubsectionx#1{\subsubsection*{#1}}
  \def\Dsubsubsectionx#1{\subsection*{#1}}
  \def\Dsubsectionx#1{\section*{#1}}
  \def\Dsectionx#1{\chapter*{#1}}

\fi

\newcommand{\MStdPoints}{4}
\newcommand{\MSetPoints}[1]{\renewcommand{\MStdPoints}{#1}}

% Befehl zum Abbruch der Erstellung (nur PDF)
\newcommand{\MAbort}[1]{\err{#1}}

% Prefix vor Dateieinbindungen, wird in der Baumdatei mit \renewcommand modifiziert
% und auf das Verzeichnisprefix gesetzt, in dem das gerade bearbeitete tex-Dokument liegt.
% Im HTML wird es auf das Verzeichnis der HTML-Datei gesetzt.
% Das Prefix muss mit / enden !
\newcommand{\MDPrefix}{.}

% MRegisterFile notiert eine Datei zur Einbindung in den HTML-Baum. Grafiken mit MGraphics werden automatisch eingebunden.
% Mit MLastFile erhaelt man eine Markierung fuer die zuletzt registrierte Datei.
% Diese Markierung wird im postprocessing durch den physikalischen Dateinamen ersetzt, aber nur den Namen (d.h. \MMaterial gehoert noch davor, vgl Definition von MGraphics)
% Parameter: Pfad/Name der Datei bzw. des Ordners, relativ zur Position des Modul-Tex-Dokuments.
\ifttm
\newcommand{\MRegisterFile}[1]{\addtocounter{MFileNumber}{1}\special{html:<!-- registerfile;;}#1\special{html:;;}\MDPrefix\special{html:;;}\arabic{MFileNumber}\special{html:; //-->}}
\else
\newcommand{\MRegisterFile}[1]{\addtocounter{MFileNumber}{1}}
\fi

% Testen welcher Uebersetzer hier am Werk ist

\ifttm
\setcounter{MOutput}{3}
\else
\ifx\pdfoutput\undefined
  \pdffalse
  \setcounter{MOutput}{\MOutputDVI}
  \message{Verarbeitung mit latex, Ausgabe in dvi.}
\else
  \setcounter{MOutput}{\MOutputPDF}
  \message{Verarbeitung mit pdflatex, Ausgabe in pdf.}
  \ifnum \pdfoutput=0
    \pdffalse
  \setcounter{MOutput}{\MOutputDVI}
  \message{Verarbeitung mit pdflatex, Ausgabe in dvi.}
  \else
    \ifnum\pdfoutput=1
    \pdftrue
  \setcounter{MOutput}{\MOutputPDF}
  \message{Verarbeitung mit pdflatex, Ausgabe in pdf.}
    \fi
  \fi
\fi
\fi

\ifnum\value{MOutput}=\MOutputPDF
\DeclareGraphicsExtensions{.pdf,.png,.jpg}
\fi

\ifnum\value{MOutput}=\MOutputDVI
\DeclareGraphicsExtensions{.eps,.png,.jpg}
\fi

\ifnum\value{MOutput}=\MOutputHTML
% Wird vom Konverter leider nicht erkannt und daher in split.pm hardcodiert!
\DeclareGraphicsExtensions{.png,.jpg,.gif}
\fi

% Umdefinition der hyperref-Nummerierung im PDF-Modus
\ifttm
\else
\renewcommand{\theHfigure}{\arabic{chapter}.\arabic{section}.\arabic{figure}}
\fi

% Makro, um in der HTML-Ausgabe die zuerst zu oeffnende Datei zu kennzeichnen
\ifttm
\newcommand{\MGlobalStart}{\special{html:<!-- mglobalstarttag -->}}
\else
\newcommand{\MGlobalStart}{}
\fi

% Makro, um bei scormlogin ein pullen des Benutzers bei Aufruf der Seite zu erzwingen (typischerweise auf der Einstiegsseite)
\ifttm
\newcommand{\MPullSite}{\special{html:<!-- pullsite //-->}}
\else
\newcommand{\MPullSite}{}
\fi

% Makro, um in der HTML-Ausgabe die Kapiteluebersicht zu kennzeichnen
\ifttm
\newcommand{\MGlobalChapterTag}{\special{html:<!-- mglobalchaptertag -->}}
\else
\newcommand{\MGlobalChapterTag}{}
\fi

% Makro, um in der HTML-Ausgabe die Konfiguration zu kennzeichnen
\ifttm
\newcommand{\MGlobalConfTag}{\special{html:<!-- mglobalconfigtag -->}}
\else
\newcommand{\MGlobalConfTag}{}
\fi

% Makro, um in der HTML-Ausgabe die Standortbeschreibung zu kennzeichnen
\ifttm
\newcommand{\MGlobalLocationTag}{\special{html:<!-- mgloballocationtag -->}}
\else
\newcommand{\MGlobalLocationTag}{}
\fi

% Makro, um in der HTML-Ausgabe die persoenlichen Daten zu kennzeichnen
\ifttm
\newcommand{\MGlobalDataTag}{\special{html:<!-- mglobaldatatag -->}}
\else
\newcommand{\MGlobalDataTag}{}
\fi

% Makro, um in der HTML-Ausgabe die Suchseite zu kennzeichnen
\ifttm
\newcommand{\MGlobalSearchTag}{\special{html:<!-- mglobalsearchtag -->}}
\else
\newcommand{\MGlobalSearchTag}{}
\fi

% Makro, um in der HTML-Ausgabe die Favoritenseite zu kennzeichnen
\ifttm
\newcommand{\MGlobalFavoTag}{\special{html:<!-- mglobalfavoritestag -->}}
\else
\newcommand{\MGlobalFavoTag}{}
\fi

% Makro, um in der HTML-Ausgabe die Eingangstestseite zu kennzeichnen
\ifttm
\newcommand{\MGlobalSTestTag}{\special{html:<!-- mglobalstesttag -->}}
\else
\newcommand{\MGlobalSTestTag}{}
\fi

% Makro, um in der PDF-Ausgabe ein Wasserzeichen zu definieren
\ifttm
\newcommand{\MWatermarkSettings}{\relax}
\else
\newcommand{\MWatermarkSettings}{%
% \SetWatermarkText{(c) MINT-Kolleg Baden-Württemberg 2014}
% \SetWatermarkLightness{0.85}
% \SetWatermarkScale{1.5}
}
\fi

\ifttm
\newcommand{\MBinom}[2]{\left({\begin{array}{c} #1 \\ #2 \end{array}}\right)}
\else
\newcommand{\MBinom}[2]{\binom{#1}{#2}}
\fi

\ifttm
\newcommand{\DeclareMathOperator}[2]{\def#1{\mathrm{#2}}}
\newcommand{\operatorname}[1]{\mathrm{#1}}
\fi

%----------------- Makros fuer die gemischte HTML/PDF-Konvertierung ------------------------------

\newcommand{\MTestName}{\relax} % wird durch Test-Umgebung gesetzt

% Fuer experimentelle Kursinhalte, die im Release-Umsetzungsvorgang eine Fehlermeldung
% produzieren sollen aber sonst normal umgesetzt werden
\newenvironment{MExperimental}{%
}{%
}

% Wird von ttm nicht richtig umgesetzt!!
\newenvironment{MExerciseItems}{%
\renewcommand\theenumi{\alph{enumi}}%
\begin{enumerate}%
}{%
\end{enumerate}%
}


\definecolor{infoshadecolor}{rgb}{0.75,0.75,0.75}
\definecolor{exmpshadecolor}{rgb}{0.875,0.875,0.875}
\definecolor{expeshadecolor}{rgb}{0.95,0.95,0.95}
\definecolor{framecolor}{rgb}{0.2,0.2,0.2}

% Bei PDF-Uebersetzung wird hinter den Start jeder Satz/Info-aehnlichen Umgebung eine leere mbox gesetzt, damit
% fuehrende Listen oder enums nicht den Zeilenumbruch kaputtmachen
%\ifttm
\def\MTB{}
%\else
%\def\MTB{\mbox{}}
%\fi


\ifttm
\newcommand{\MRelates}{\special{html:<mi>&wedgeq;</mi>}}
\else
\def\MRelates{\stackrel{\scriptscriptstyle\wedge}{=}}
\fi

\def\MInch{\text{''}}
\def\Mdd{\textit{''}}

\ifttm
\def\MNL{ \newline }
\newenvironment{MArray}[1]{\begin{array}{#1}}{\end{array}}
\else
\def\MNL{ \\ }
\newenvironment{MArray}[1]{\begin{array}{#1}}{\end{array}}
\fi

\newcommand{\MBox}[1]{$\mathrm{#1}$}
\newcommand{\MMBox}[1]{\mathrm{#1}}


\ifttm%
\newcommand{\Mtfrac}[2]{{\textstyle \frac{#1}{#2}}}
\newcommand{\Mdfrac}[2]{{\displaystyle \frac{#1}{#2}}}
\newcommand{\Mmeasuredangle}{\special{html:<mi>&angmsd;</mi>}}
\else%
\newcommand{\Mtfrac}[2]{\tfrac{#1}{#2}}
\newcommand{\Mdfrac}[2]{\dfrac{#1}{#2}}
\newcommand{\Mmeasuredangle}{\measuredangle}
\relax
\fi

% Matrizen und Vektoren

% Inhalt wird in der Form a & b \\ c & d erwartet
% Vorsicht: MVector = Komponentenspalte, MVec = Variablensymbol
\ifttm%
\newcommand{\MVector}[1]{\left({\begin{array}{c}#1\end{array}}\right)}
\else%
\newcommand{\MVector}[1]{\begin{pmatrix}#1\end{pmatrix}}
\fi



\newcommand{\MVec}[1]{\vec{#1}}
\newcommand{\MDVec}[1]{\overrightarrow{#1}}

%----------------- Umgebungen fuer Definitionen und Saetze ----------------------------------------

% Fuegt einen Tabellen-Zeilenumbruch ein im PDF, aber nicht im HTML
\newcommand{\TSkip}{\ifttm \else&\ \\\fi}

\newenvironment{infoshaded}{%
\def\FrameCommand{\fboxsep=\FrameSep \fcolorbox{framecolor}{infoshadecolor}}%
\MakeFramed {\advance\hsize-\width \FrameRestore}}%
{\endMakeFramed}

\newenvironment{expeshaded}{%
\def\FrameCommand{\fboxsep=\FrameSep \fcolorbox{framecolor}{expeshadecolor}}%
\MakeFramed {\advance\hsize-\width \FrameRestore}}%
{\endMakeFramed}

\newenvironment{exmpshaded}{%
\def\FrameCommand{\fboxsep=\FrameSep \fcolorbox{framecolor}{exmpshadecolor}}%
\MakeFramed {\advance\hsize-\width \FrameRestore}}%
{\endMakeFramed}

\def\STDCOLOR{black}

\ifttm%
\else%
\newtheoremstyle{MSatzStyle}
  {1cm}                   %Space above
  {1cm}                   %Space below
  {\normalfont\itshape}   %Body font
  {}                      %Indent amount (empty = no indent,
                          %\parindent = para indent)
  {\normalfont\bfseries}  %Thm head font
  {}                      %Punctuation after thm head
  {\newline}              %Space after thm head: " " = normal interword
                          %space; \newline = linebreak
  {\thmname{#1}\thmnumber{#2}\thmnote{(#3)} }
                          %Thm head spec (can be left empty, meaning
                          %`normal')
                          %
\newtheoremstyle{MDefStyle}
  {1cm}                   %Space above
  {1cm}                   %Space below
  {\normalfont}           %Body font
  {}                      %Indent amount (empty = no indent,
                          %\parindent = para indent)
  {\normalfont\bfseries}  %Thm head font
  {}                      %Punctuation after thm head
  {\newline}              %Space after thm head: " " = normal interword
                          %space; \newline = linebreak
  {\thmname{#1}\thmnumber{#2}\thmnote{(#3)}}
                          %Thm head spec (can be left empty, meaning
                          %`normal')
\fi%

\newcommand{\MInfoText}{Info}

\newcounter{MHintCounter}
\newcounter{MCodeEditCounter}

\newcounter{MLastIndex}  % Enthaelt die dritte Stelle (Indexnummer) des letzten angelegten Objekts
\newcounter{MLastType}   % Enthaelt den Typ des letzten angelegten Objekts (mithilfe der unten definierten Konstanten). Die Entscheidung, wie der Typ dargstellt wird, wird in split.pm beim Postprocessing getroffen.
\newcounter{MLastTypeEq} % =1 falls das Label in einer Matheumgebung (equation, eqnarray usw.) steht, =2 falls das Label in einer table-Umgebung steht

% Da ttm keine Zahlmakros verarbeiten kann, werden diese Nummern in den Zuweisungen hardcodiert!
\def\MTypeSection{1}          %# Zaehler ist section
\def\MTypeSubsection{2}       %# Zaehler ist subsection
\def\MTypeSubsubsection{3}    %# Zaehler ist subsubsection
\def\MTypeInfo{4}             %# Eine Infobox, Separatzaehler fuer die Chemie (auch wenn es dort nicht nummeriert wird) ist MInfoCounter
\def\MTypeExercise{5}         %# Eine Aufgabe, Separatzaehler fuer die Chemie ist MExerciseCounter
\def\MTypeExample{6}          %# Eine Beispielbox, Separatzaehler fuer die Chemie ist MExampleCounter
\def\MTypeExperiment{7}       %# Eine Versuchsbox, Separatzaehler fuer die Chemie ist MExperimentCounter
\def\MTypeGraphics{8}         %# Eine Graphik, Separatzaehler fuer alle FB ist MGraphicsCounter
\def\MTypeTable{9}            %# Eine Tabellennummer, hat keinen Zaehler da durch table gezaehlt wird
\def\MTypeEquation{10}        %# Eine Gleichungsnummer, hat keinen Zaehler da durch equation/eqnarray gezaehlt wird
\def\MTypeTheorem{11}         % Ein theorem oder xtheorem, Separatzaehler fuer die Chemie ist MTheoremCounter
\def\MTypeVideo{12}           %# Ein Video,Separatzaehler fuer alle FB ist MVideoCounter
\def\MTypeEntry{13}           %# Ein Eintrag fuer die Stichwortliste, wird nicht gezaehlt sondern erhaelt im preparsing ein unique-label

% Zaehler fuer das Labelsystem sind prefixcounter, jeder Zaehler wird VOR dem gezaehlten Objekt inkrementiert und zaehlt daher das aktuelle Objekt
\newcounter{MInfoCounter}
\newcounter{MExerciseCounter}
\newcounter{MExampleCounter}
\newcounter{MExperimentCounter}
\newcounter{MGraphicsCounter}
\newcounter{MTableCounter}
\newcounter{MEquationCounter}  % Nur im HTML, sonst durch "equation"-counter von latex realisiert
\newcounter{MTheoremCounter}
\newcounter{MObjectCounter}   % Gemeinsamer Zaehler fuer Objekte (ausser Grafiken/Tabellen) in Mathe/Info/Physik
\newcounter{MVideoCounter}
\newcounter{MEntryCounter}

\newcounter{MTestSite} % 1 = Subsubsection ist eine Pruefungsseite, 0 = ist eine normale Seite (inkl. Hilfeseite)

\def\MCell{$\phantom{a}$}

\newenvironment{MExportExercise}{\begin{MExercise}}{\end{MExercise}} % wird von mconvert abgefangen

\def\MGenerateExNumber{%
\ifnum\value{MSepNumbers}=0%
\arabic{section}.\arabic{subsection}.\arabic{MObjectCounter}\setcounter{MLastIndex}{\value{MObjectCounter}}%
\else%
\arabic{section}.\arabic{subsection}.\arabic{MExerciseCounter}\setcounter{MLastIndex}{\value{MExerciseCounter}}%
\fi%
}%

\def\MGenerateExmpNumber{%
\ifnum\value{MSepNumbers}=0%
\arabic{section}.\arabic{subsection}.\arabic{MObjectCounter}\setcounter{MLastIndex}{\value{MObjectCounter}}%
\else%
\arabic{section}.\arabic{subsection}.\arabic{MExerciseCounter}\setcounter{MLastIndex}{\value{MExampleCounter}}%
\fi%
}%

\def\MGenerateInfoNumber{%
\ifnum\value{MSepNumbers}=0%
\arabic{section}.\arabic{subsection}.\arabic{MObjectCounter}\setcounter{MLastIndex}{\value{MObjectCounter}}%
\else%
\arabic{section}.\arabic{subsection}.\arabic{MExerciseCounter}\setcounter{MLastIndex}{\value{MInfoCounter}}%
\fi%
}%

\def\MGenerateSiteNumber{%
\arabic{section}.\arabic{subsection}.\arabic{subsubsection}%
}%

% Funktionalitaet fuer Auswahlaufgaben

\newcounter{MExerciseCollectionCounter} % = 0 falls nicht in collection-Umgebung, ansonsten Schachtelungstiefe
\newcounter{MExerciseCollectionTextCounter} % wird von MExercise-Umgebung inkrementiert und von MExerciseCollection-Umgebung auf Null gesetzt

\ifttm
% MExerciseCollection gruppiert Aufgaben, die dynamisch aus der Datenbank gezogen werden und nicht direkt in der HTML-Seite stehen
% Parameter: #1 = ID der Collection, muss eindeutig fuer alle IN DER DB VORHANDENEN collections sein unabhaengig vom Kurs
%            #2 = Optionsargument (im Moment: 1 = Iterative Auswahl, 2 = Zufallsbasierte Auswahl)
\newenvironment{MExerciseCollection}[2]{%
\addtocounter{MExerciseCollectionCounter}{1}
\setcounter{MExerciseCollectionTextCounter}{0}
\special{html:<!-- mexercisecollectionstart;;}#1\special{html:;;}#2\special{html:;; //-->}%
}{%
\special{html:<!-- mexercisecollectionstop //-->}%
\addtocounter{MExerciseCollectionCounter}{-1}
}
\else
\newenvironment{MExerciseCollection}[2]{%
\addtocounter{MExerciseCollectionCounter}{1}
\setcounter{MExerciseCollectionTextCounter}{0}
}{%
\addtocounter{MExerciseCollectionCounter}{-1}
}
\fi

% Bei Uebersetzung nach PDF werden die theorem-Umgebungen verwendet, bei Uebersetzung in HTML ein manuelles Makro
\ifttm%

  \newenvironment{MHint}[1]{  \special{html:<button name="Name_MHint}\arabic{MHintCounter}\special{html:" class="hintbutton_closed btn btn-default" id="MHint}\arabic{MHintCounter}\special{html:_button" %
  type="button" onclick="toggle_hint('MHint}\arabic{MHintCounter}\special{html:');">}#1\special{html:</button>}
  \special{html:<div class="hint well" style="display:none" id="MHint}\arabic{MHintCounter}\special{html:"> }}{\begin{html}</div>\end{html}\addtocounter{MHintCounter}{1}}

  \newenvironment{MCOSHZusatz}{  \special{html:<button name="Name_MHint}\arabic{MHintCounter}\special{html:" class="chintbutton_closed" id="MHint}\arabic{MHintCounter}\special{html:_button" %
  type="button" onclick="toggle_hint('MHint}\arabic{MHintCounter}\special{html:');">}\iCoshAdd{}\special{html:</button>}
  \special{html:<div class="hintc" style="display:none" id="MHint}\arabic{MHintCounter}\special{html:">
  <div class="coshwarn">}\iCoshWarn{}\special{html:</div><br />}
  \addtocounter{MHintCounter}{1}}{\begin{html}</div>\end{html}}


  \newenvironment{MDefinition}{\begin{definition}\setcounter{MLastIndex}{\value{definition}}\ \\}{\end{definition}}


  \newenvironment{MExercise}{
  \renewcommand{\MStdPoints}{4}
  \addtocounter{MExerciseCounter}{1}
  \addtocounter{MObjectCounter}{1}
  \setcounter{MLastType}{5}

\ifnum\value{MExerciseCollectionCounter}=0\else\addtocounter{MExerciseCollectionTextCounter}{1}\special{html:<!-- mexercisetextstart;;}\arabic{MExerciseCollectionTextCounter}\special{html:;; //-->}\fi
  \special{html:<div class="aufgabe"
  id="ADIV_}\MGenerateExNumber\special{html:"><div class="panel
  panel-warning"><div class="panel-heading"><h5 class="title">}%
  \textbf{\iExercise{} \MGenerateExNumber
  } \begin{html}</h5></div><div class="panel-body">\end{html}}{\special{html:</div></div></div><!--
  mfeedbackbutton;Exercise;}\arabic{MTestSite}\special{html:;}\MGenerateExNumber\special{html:; //-->} \ifnum\value{MExerciseCollectionCounter}=0\else\special{html:<!-- mexercisetextstop //-->}\fi
  }
  % Stellt eine Kombination aus Aufgabe, Loesungstext und Eingabefeld bereit,
  % bei der Aufgabentext und Musterloesung sowie die zugehoerigen Feldelemente
  % extern bezogen und div-aktualisiert werden, das Eingabefeld aber immer das gleiche ist.
  \newenvironment{MFetchExercise}{
  \addtocounter{MExerciseCounter}{1}
  \addtocounter{MObjectCounter}{1}
  \setcounter{MLastType}{5}

  \special{html:<div class="aufgabe" id="ADIV_}\MGenerateExNumber\special{html:">}%
  \textbf{\iExercise{} \MGenerateExNumber
  } \ \\%
  \special{html:</div><div class="exfetch_text" id="ADIVTEXT_}\MGenerateExNumber\special{html:">}%
  \special{html:</div><div class="exfetch_sol" id="ADIVSOL_}\MGenerateExNumber\special{html:">}%
  \special{html:</div><div class="exfetch_input" id="ADIVINPUT_}\MGenerateExNumber\special{html:">}%
  }{
  \special{html:</div>}
  }

  \newenvironment{MExample}{
  \addtocounter{MExampleCounter}{1}
  \addtocounter{MObjectCounter}{1}
  \setcounter{MLastType}{6}
  \begin{html}
  <div class="exmp">
  <div class="exmprahmen panel panel-default">
  <div class="panel-heading"><h5 class="panel-title">
  \end{html}\textbf{\iExample{}
  \ifnum\value{MSepNumbers}=0
  \arabic{section}.\arabic{subsection}.\arabic{MObjectCounter}\setcounter{MLastIndex}{\value{MObjectCounter}}
  \else
  \arabic{section}.\arabic{subsection}.\arabic{MExampleCounter}\setcounter{MLastIndex}{\value{MExampleCounter}}
  \fi
  } \begin{html}</h5></div><div class="panel-body">\end{html}}
  {\begin{html}</div></div></div>\end{html}
  \special{html:<!-- mfeedbackbutton;Example;}\arabic{MTestSite}\special{html:;}\MGenerateExmpNumber\special{html:; //-->}
  }

  \newenvironment{MExperiment}{
  \addtocounter{MExperimentCounter}{1}
  \addtocounter{MObjectCounter}{1}
  \setcounter{MLastType}{7}
  \begin{html}
  <div class="expe">
  <div class="experahmen">
  \end{html}\textbf{\iExperiment{}
  \ifnum\value{MSepNumbers}=0
  \arabic{section}.\arabic{subsection}.\arabic{MObjectCounter}\setcounter{MLastIndex}{\value{MObjectCounter}}
  \else
%  \arabic{MExperimentCounter}\setcounter{MLastIndex}{\value{MExperimentCounter}}
  \arabic{section}.\arabic{subsection}.\arabic{MExperimentCounter}\setcounter{MLastIndex}{\value{MExperimentCounter}}
  \fi
  } \ \\}{\begin{html}</div>
  </div>
  \end{html}}

  \newenvironment{MChemInfo}{
  \setcounter{MLastType}{4}
  \begin{html}
  <div class="info">
  <div class="inforahmen">
  \end{html}}{\begin{html}</div>
  </div>
  \end{html}}

  \newenvironment{MXInfo}[1]{
  \addtocounter{MInfoCounter}{1}
  \addtocounter{MObjectCounter}{1}
  \setcounter{MLastType}{4}
  \begin{html}
  <div class="info">
  <div class="inforahmen panel panel-info">
  <div class="panel-heading"><h5 class="panel-title">
  \end{html}\textbf{#1
  \ifnum\value{MInfoNumbers}=0
  \else
    \ifnum\value{MSepNumbers}=0
    \arabic{section}.\arabic{subsection}.\arabic{MObjectCounter}\setcounter{MLastIndex}{\value{MObjectCounter}}
    \else
    \arabic{MInfoCounter}\setcounter{MLastIndex}{\value{MInfoCounter}}
    \fi
  \fi
  } \begin{html}</h5></div><div class="panel-body">\end{html}}
  {\begin{html}</div></div></div>\end{html}
  \special{html:<!-- mfeedbackbutton;Info;}\arabic{MTestSite}\special{html:;}\MGenerateInfoNumber\special{html:; //-->}
  }

  \newenvironment{MInfo}
  {\ifnum\value{MInfoNumbers}=0\begin{MChemInfo}\else\begin{MXInfo}{Info}\ \\ \fi}
  {\ifnum\value{MInfoNumbers}=0\end{MChemInfo}\else\end{MXInfo}\fi}

\else%

  \theoremstyle{MSatzStyle}
  \newtheorem{thm}{Satz}[section]
  \newtheorem{thmc}{Satz}
  \theoremstyle{MDefStyle}
  \newtheorem{defn}[thm]{Definition}
  \newtheorem{exmp}[thm]{\iExample{} }
  \newtheorem{info}[thm]{\MInfoText}
  \theoremstyle{MDefStyle}
  \newtheorem{defnc}{Definition}
  \theoremstyle{MDefStyle}
  \newtheorem{exmpc}{\iExample{} } [section]
  \theoremstyle{MDefStyle}
  \newtheorem{infoc}{\MInfoText}
  \theoremstyle{MDefStyle}
  \newtheorem{exrc}{\iExercise{} } [section]
  \theoremstyle{MDefStyle}
  \newtheorem{verc}{Versuch} [section]

  \newenvironment{MFetchExercise}{}{} % kann im PDF nicht dargestellt werden

  \newenvironment{MExercise}{\begin{exrc}\renewcommand{\MStdPoints}{1}\MTB}{\end{exrc}}
  \newenvironment{MHint}[1]{\ \\ \underline{#1:}\\}{}
  \newenvironment{MCOSHZusatz}{\ \\ \underline{\iCoshAdd{}:}\\}{}
%  \newenvironment{MDefinition}{\ifnum\value{MInfoNumbers}=0\begin{defnc}\else\begin{defn}\fi\MTB}{\ifnum\value{MInfoNumbers}=0\end{defnc}\else\end{defn}\fi}
%  \newenvironment{MExample}{\begin{exmp}}{\ \linebreak[1] \ \ \ \ $\phantom{a}$ \ \hfill $\blacklozenge$\end{exmp}}
  \newenvironment{MExample}{
    \ifnum\value{MInfoNumbers}=0\begin{exmpc}\else\begin{exmp}\fi
    \MTB
    \begin{exmpshaded}
    \ \newline
}{
    \end{exmpshaded}
    \ifnum\value{MInfoNumbers}=0\end{exmpc}\else\end{exmp}\fi
}
  \newenvironment{MChemInfo}{\begin{infoshaded}}{\end{infoshaded}}
%  \newenvironment{MXInfo}{\begin{infoshaded}}{\end{infoshaded}}

  \newenvironment{MInfo}{\ifnum\value{MInfoNumbers}=0\begin{MChemInfo}\else\renewcommand{\MInfoText}{Info}\begin{info}\begin{infoshaded}
  \MTB
   \ \newline
    \fi
  }{\ifnum\value{MInfoNumbers}=0\end{MChemInfo}\else\end{infoshaded}\end{info}\fi}

  \newenvironment{MXInfo}[1]{
    \renewcommand{\MInfoText}{#1}
    \ifnum\value{MInfoNumbers}=0\begin{infoc}\else\begin{info}\fi%
    \MTB
    \begin{infoshaded}
    \ \newline
  }{\end{infoshaded}\ifnum\value{MInfoNumbers}=0\end{infoc}\else\end{info}\fi}

  \newenvironment{MExperiment}{
    \renewcommand{\MInfoText}{\iExperiment{}}
    \ifnum\value{MInfoNumbers}=0\begin{verc}\else\begin{info}\fi
    \MTB
    \begin{expeshaded}
    \ \newline
  }{
    \end{expeshaded}
    \ifnum\value{MInfoNumbers}=0\end{verc}\else\end{info}\fi
  }
\fi%

% MHint sollte nicht direkt fuer Loesungen benutzt werden wegen solutionselect
\newenvironment{MSolution}{\begin{MHint}{Solution}}{\end{MHint}}

\newcounter{MCodeCounter}

\ifttm
\newenvironment{MCode}{\special{html:<!-- mcodestart -->}\ttfamily\color{blue}}{\special{html:<!-- mcodestop -->}}
\else
\newenvironment{MCode}{\begin{flushleft}\ttfamily\addtocounter{MCodeCounter}{1}}{\addtocounter{MCodeCounter}{-1}\end{flushleft}}
% Ohne color-Statement da inkompatible mit framed/shaded-Boxen aus dem framed-package
\fi

%----------------- Sonderdefinitionen fuer Symbole, die der Konverter nicht kann ----------------------------------------------

\ifttm%
\newcommand{\MUnderset}[2]{\underbrace{#2}_{#1}}%
\else%
\newcommand{\MUnderset}[2]{\underset{#1}{#2}}%
\fi%

\ifttm
\newcommand{\MThinspace}{\special{html:<mi>&#x2009;</mi>}}
\else
\newcommand{\MThinspace}{\,}
\fi

\ifttm
\newcommand{\glq}{\begin{html}&sbquo;\end{html}}
\newcommand{\grq}{\begin{html}&lsquo;\end{html}}
\newcommand{\glqq}{\begin{html}&bdquo;\end{html}}
\newcommand{\grqq}{\begin{html}&ldquo;\end{html}}
\fi

% Backsim (PhysikBK)
\ifttm
\newcommand{\backsim}{\begin{html}<mi>&backsim;</mi>\end{html}}
\fi
% Cancel (PhysikBK)
\ifttm
\newcommand{\cancel}[1]{\begin{html}<menclose notation="horizontalstrike"><mi>#1</mi></menclose>\end{html}}
\fi


\ifttm
\newcommand{\MNdash}{\begin{html}&ndash;\end{html}}
\else
\newcommand{\MNdash}{--}
\fi

%\ifttm\def\MIU{\special{html:<mi>&#8520;</mi>}}\else\def\MIU{\mathrm{i}}\fi
\def\MIU{\mathrm{i}}
\def\MEU{e} % TU9-Onlinekurs: italic-e
%\def\MEU{\mathrm{e}} % Alte Onlinemodule: roman-e
\def\MD{d} % Kursives d in Integralen im TU9-Onlinekurs
%\def\MD{\mathrm{d}} % roman-d in den alten Onlinemodulen
\def\MDB{\|}

%zusaetzlicher Leerraum vor "\MD"
\ifttm%
\def\MDSpace{\special{html:<mi>&#x2009;</mi>}}
\else%
\def\MDSpace{\,}
\fi%
\newcommand{\MDwSp}{\MDSpace\MD}%

\ifttm
\def\Mdq{\dq}
\else
\def\Mdq{\dq}
\fi

\def\MSpan#1{\left<{#1}\right>}
\def\MSetminus{\setminus}
\def\MIM{I}

\ifttm
\newcommand{\ld}{\text{ld}}
\newcommand{\lg}{\text{lg}}
\else
\DeclareMathOperator{\ld}{ld}
%\newcommand{\lg}{\text{lg}} % in latex schon definiert
\fi


\def\Mmapsto{\ifttm\special{html:<mi>&mapsto;</mi>}\else\mapsto\fi}
\def\Mvarphi{\ifttm\phi\else\varphi\fi}
\def\Mphi{\ifttm\varphi\else\phi\fi}
\ifttm%
\newcommand{\MEumu}{\special{html:<mi>&#x3BC;</mi>}}%
\else%
\newcommand{\MEumu}{\textrm{\textmu}}%
\fi
\def\Mvarepsilon{\ifttm\epsilon\else\varepsilon\fi}
\def\Mepsilon{\ifttm\varepsilon\else\epsilon\fi}
\def\Mvarkappa{\ifttm\kappa\else\varkappa\fi}
\def\Mkappa{\ifttm\varkappa\else\kappa\fi}
\def\Mcomplement{\ifttm\special{html:<mi>&comp;</mi>}\else\complement\fi}
\def\MWW{\mathrm{WW}}
\def\Mmod{\ifttm\special{html:<mi>&nbsp;mod&nbsp;</mi>}\else\mod\fi}

\ifttm%
\def\mod{\text{\;mod\;}}%
\def\MNEquiv{\special{html:<mi>&NotCongruent;</mi>}}%
\def\MNSubseteq{\special{html:<mi>&NotSubsetEqual;</mi>}}%
\def\MEmptyset{\special{html:<mi>&empty;</mi>}}%
\def\MVDots{\special{html:<mi>&#x22EE;</mi>}}%
\def\MHDots{\special{html:<mi>&#x2026;</mi>}}%
\def\Mddag{\special{html:<mi>&#x1202;</mi>}}%
\def\sphericalangle{\special{html:<mi>&measuredangle;</mi>}}%
\def\nparallel{\special{html:<mi>&nparallel;</mi>}}%
\def\MProofEnd{\special{html:<mi>&#x25FB;</mi>}}%
\newenvironment{MProof}[1]{\underline{#1}:\MCR\MCR}{\hfill $\MProofEnd$}%
\else%
\def\MNEquiv{\not\equiv}%
\def\MNSubseteq{\not\subseteq}%
\def\MEmptyset{\emptyset}%
\def\MVDots{\vdots}%
\def\MHDots{\hdots}%
\def\Mddag{\ddag}%
\newenvironment{MProof}[1]{\begin{proof}[#1]}{\end{proof}}%
\fi%



% Spaces zum Auffuellen von Tabellenbreiten, die nur im HTML wirken
\ifttm%
\def\MTSP{\:}%
\else%
\def\MTSP{}%
\fi%

\DeclareMathOperator{\arsinh}{arsinh}
\DeclareMathOperator{\arcosh}{arcosh}
\DeclareMathOperator{\artanh}{artanh}
\DeclareMathOperator{\arcoth}{arcoth}


\newcommand{\MMathSet}[1]{\mathbb{#1}}
\def\N{\MMathSet{N}}
\def\Z{\MMathSet{Z}}
\def\Q{\MMathSet{Q}}
\def\R{\MMathSet{R}}
\def\C{\MMathSet{C}}

\newcounter{MForLoopCounter}
\newcommand{\MForLoop}[2]{\setcounter{MForLoopCounter}{#1}\ifnum\value{MForLoopCounter}=0{}\else{{#2}\addtocounter{MForLoopCounter}{-1}\MForLoop{\value{MForLoopCounter}}{#2}}\fi}

\newcounter{MSiteCounter}
\newcounter{MFieldCounter} % Kombination section.subsection.site.field ist eindeutig in allen Modulen, field alleine nicht

\newcounter{MiniMarkerCounter}

\ifttm
\newenvironment{MMiniPageP}[1]{\begin{minipage}{#1\linewidth}\special{html:<!-- minimarker;;}\arabic{MiniMarkerCounter}\special{html:;;#1; //-->}}{\end{minipage}\addtocounter{MiniMarkerCounter}{1}}
\else
\newenvironment{MMiniPageP}[1]{\begin{minipage}{#1\linewidth}}{\end{minipage}\addtocounter{MiniMarkerCounter}{1}}
\fi

\newcounter{AlignCounter}

\newcommand{\MStartJustify}{\ifttm\special{html:<!-- startalign;;}\arabic{AlignCounter}\special{html:;;justify; //-->}\fi}
\newcommand{\MStopJustify}{\ifttm\special{html:<!-- stopalign;;}\arabic{AlignCounter}\special{html:; //-->}\fi\addtocounter{AlignCounter}{1}}

\newenvironment{MJTabular}[1]{
\MStartJustify
\begin{tabular}{#1}
}{
\end{tabular}
\MStopJustify
}

\newcommand{\MImageLeft}[2]{
\begin{center}
\begin{tabular}{lc}
\MStartJustify
\begin{MMiniPageP}{0.65}
#1
\end{MMiniPageP}
\MStopJustify
&
\begin{MMiniPageP}{0.3}
#2
\end{MMiniPageP}
\end{tabular}
\end{center}
}

\newcommand{\MImageHalf}[2]{
\begin{center}
\begin{tabular}{lc}
\MStartJustify
\begin{MMiniPageP}{0.45}
#1
\end{MMiniPageP}
\MStopJustify
&
\begin{MMiniPageP}{0.45}
#2
\end{MMiniPageP}
\end{tabular}
\end{center}
}

\newcommand{\MBigImageLeft}[2]{
\begin{center}
\begin{tabular}{lc}
\MStartJustify
\begin{MMiniPageP}{0.25}
#1
\end{MMiniPageP}
\MStopJustify
&
\begin{MMiniPageP}{0.7}
#2
\end{MMiniPageP}
\end{tabular}
\end{center}
}

\ifttm
\def\No{\mathbb{N}_0}
\else
\def\No{\ensuremath{\N_0}}
\fi
\def\MT{\textrm{\tiny T}}
\newcommand{\MTranspose}[1]{{#1}^{\MT}}
\ifttm
\newcommand{\MRe}{\mathsf{Re}}
\newcommand{\MIm}{\mathsf{Im}}
\else
\DeclareMathOperator{\MRe}{Re}
\DeclareMathOperator{\MIm}{Im}
\fi

\newcommand{\Mid}{\mathrm{id}}
\newcommand{\MFeinheit}{\mathrm{feinh}}

\ifttm
\newcommand{\Msubstack}[1]{\begin{array}{c}{#1}\end{array}}
\else
\newcommand{\Msubstack}[1]{\substack{#1}}
\fi

% Typen von Fragefeldern:
% 1 = Alphanumerisch, case-sensitive-Vergleich
% 2 = Ja/Nein-Checkbox, Loesung ist 0 oder 1   (OPTION = Image-id fuer Rueckmeldung)
% 3 = Reelle Zahlen Geparset
% 4 = Funktionen Geparset (mit Stuetzstellen zur ueberpruefung)

% Dieser Befehl erstellt ein interaktives Aufgabenfeld. Parameter:
% - #1 Laenge in Zeichen
% - #2 Loesungstext (alphanumerisch, case sensitive)
% - #3 AufgabenID (alphanumerisch, case sensitive)
% - #4 Typ (Kennnummer)
% - #5 String fuer Optionen (ggf. mit Semikolon getrennte Einzelstrings)
% - #6 Anzahl Punkte
% - #7 uxid (kann z.B. Loesungsstring sein)
% ACHTUNG: Die langen Zeilen bitte so lassen, Zeilenumbrueche im tex werden in div's umgesetzt
\ifttm
\newcommand{\MQuestionID}[7]{
\special{html:<!-- mdeclareuxid;;}UX#7\special{html:;;}\arabic{section}\special{html:;;}#3\special{html:;; //-->}%
\special{html:<!-- mdeclarepoints;;}\arabic{section}\special{html:;;}#3\special{html:;;}#6\special{html:;;}\arabic{MTestSite}\special{html:;;}\arabic{chapter}%
\special{html:;; //--><!-- onloadstart //-->CreateQuestionObj("}#7\special{html:",}\arabic{MFieldCounter}\special{html:,"}#2%
\special{html:","}#3\special{html:",}#4\special{html:,"}#5\special{html:",}#6\special{html:,}\arabic{MTestSite}\special{html:,}\arabic{section}%
\special{html:);<!-- onloadstop //-->}%
\special{html:<input mfieldtype="}#4\special{html:" name="Name_}#3\special{html:" id="}#3\special{html:" type="text" size="}#1\special{html:" maxlength="}#1%
\special{html:" }
\ifnum\value{MGroupActive}=0\special{html:onfocus="handlerFocus(}\arabic{MFieldCounter}\special{html:);" onblur="handlerBlur(}\arabic{MFieldCounter}\special{html:);" onkeyup="handlerChange(}\arabic{MFieldCounter}\special{html:,0);" onpaste="handlerChange(}\arabic{MFieldCounter}\special{html:,0);" oninput="handlerChange(}\arabic{MFieldCounter}\special{html:,0);" onpropertychange="handlerChange(}\arabic{MFieldCounter}\special{html:,0);"/>}%
\special{html:<span><span class="glyphicon glyphicon-question-sign" id="}QM#3\special{html:"/></span>} % needs to be nested spans here, as they get garbled-up somehow
\else%
\special{html:onblur="handlerBlur(}\arabic{MFieldCounter}%
\special{html:);" onfocus="handlerFocus(}\arabic{MFieldCounter}\special{html:);" onkeyup="handlerChange(}\arabic{MFieldCounter}\special{html:,1);" onpaste="handlerChange(}\arabic{MFieldCounter}\special{html:,1);" oninput="handlerChange(}\arabic{MFieldCounter}\special{html:,1);" onpropertychange="handlerChange(}\arabic{MFieldCounter}\special{html:,1);"/>}%
\special{html:<span><span class="glyphicon glyphicon-question-sign" id="}QM#3\special{html:"/></span>} % needs to be nested spans here, as they get garbled-up somehow
\fi
}
\else
\newcommand{\MQuestionID}[7]{
\ifnum\value{QBoxFlag}=1\fbox{$\phantom{\MForLoop{#1}{b}}$}
\else
$\phantom{\MForLoop{#1}{b}}$
\fi%
}
\fi%

% ACHTUNG: Die langen Zeilen bitte so lassen, Zeilenumbrueche im tex werden in div's umgesetzt
% QuestionCheckbox macht ausserhalb einer QuestionGroup keinen Sinn!
% #1 = solution (1 oder 0), ggf. mit ::smc abgetrennt auszuschliessende single-choice-boxen (UXIDs durch , getrennt), #2 = id, #3 = points, #4 = uxid
\ifttm
\newcommand{\MQuestionCheckbox}[4]{
\special{html:<!-- mdeclareuxid;;}UX#4\special{html:;;}\arabic{section}\special{html:;;}#2\special{html:;; //-->}%
\ifnum\value{MGroupActive}=0\MDebugMessage{ERROR: Checkbox Nr. \arabic{MFieldCounter}\ ist nicht in einer Kontrollgruppe, es wird niemals eine Loesung angezeigt!}\fi
\special{html: %
<!-- mdeclarepoints;;}\arabic{section}\special{html:;;}#2\special{html:;;}#3\special{html:;;}\arabic{MTestSite}\special{html:;;}\arabic{chapter}%
\special{html:;; //--><!-- onloadstart //-->CreateQuestionObj("}#4\special{html:",}\arabic{MFieldCounter}\special{html:,"}#1\special{html:","}#2\special{html:",2,"IMG}#2%
\special{html:",}#3\special{html:,}\arabic{MTestSite}\special{html:,}\arabic{section}\special{html:);<!-- onloadstop //-->}%
\special{html:<input mtristate="1" cval="0" mfieldtype="2" type="checkbox" name="Name_}#2\special{html:" id="}#2\special{html:" onchange="handlerChange(}\arabic{MFieldCounter}\special{html:,1);"/>}
\special{html:<span><span class="glyphicon glyphicon-question-sign" id="}IMG#2\special{html:"/></span>} % needs to be nested spans here, as they get garbled-up somehow
}
\else%
\newcommand{\MQuestionCheckbox}[4]{
\ifnum\value{QBoxFlag}=1\fbox{$\phantom{X}$}\else$\phantom{X}$\fi%
}
\fi%

\def\MGenerateID{QFELD_\arabic{section}.\arabic{subsection}.\arabic{MSiteCounter}.QF\arabic{MFieldCounter}}

% #1 = 0/1 ggf. mit ::smc abgetrennt auszuschliessende single-choice-boxen (UXIDs durch , getrennt ohne UX), #2 = uxid ohne UX
\newcommand{\MCheckbox}[2]{
\MQuestionCheckbox{#1}{\MGenerateID}{\MStdPoints}{#2}
\addtocounter{MFieldCounter}{1}
}

% #1 = 0/1 ggf. mit ::smc abgetrennt auszuschliessende single-choice-boxen (UXIDs durch , getrennt ohne UX), #2 = uxid ohne UX
\newcommand{\MLCheckbox}[2]{
\MQuestionCheckbox{#1}{\MGenerateID}{\MStdPoints}{#2}
\addtocounter{MFieldCounter}{1}
}

% Erster Parameter: Zeichenlaenge der Eingabebox, zweiter Parameter: Loesungstext
\newcommand{\MQuestion}[2]{
\MQuestionID{#1}{#2}{\MGenerateID}{1}{0}{\MStdPoints}{#2}
\addtocounter{MFieldCounter}{1}
}

% Erster Parameter: Zeichenlaenge der Eingabebox, zweiter Parameter: Loesungstext
\newcommand{\MLQuestion}[3]{
\MQuestionID{#1}{#2}{\MGenerateID}{1}{0}{\MStdPoints}{#3}
\addtocounter{MFieldCounter}{1}
}

% Parameter: Laenge des Feldes, Loesung (wird auch geparsed), Stellen Genauigkeit hinter dem Komma, weitere Stellen werden mathematisch gerundet vor Vergleich
\newcommand{\MParsedQuestion}[3]{
\MQuestionID{#1}{#2}{\MGenerateID}{3}{#3}{\MStdPoints}{#2}
\addtocounter{MFieldCounter}{1}
}

% Parameter: Laenge des Feldes, Loesung (wird auch geparsed), Stellen Genauigkeit hinter dem Komma, weitere Stellen werden mathematisch gerundet vor Vergleich
\newcommand{\MLParsedQuestion}[4]{
\MQuestionID{#1}{#2}{\MGenerateID}{3}{#3}{\MStdPoints}{#4}
\addtocounter{MFieldCounter}{1}
}

% Parameter: Laenge des Feldes, Loesungsfunktion, Anzahl Stuetzstellen, Funktionsvariablen durch Kommata getrennt (nicht case-sensitive), Anzahl Nachkommastellen im Vergleich
\newcommand{\MFunctionQuestion}[5]{
\MQuestionID{#1}{#2}{\MGenerateID}{4}{#3;#4;#5;0}{\MStdPoints}{#2}
\addtocounter{MFieldCounter}{1}
}

% Parameter: Laenge des Feldes, Loesungsfunktion, Anzahl Stuetzstellen, Funktionsvariablen durch Kommata getrennt (nicht case-sensitive), Anzahl Nachkommastellen im Vergleich, UXID
\newcommand{\MLFunctionQuestion}[6]{
\MQuestionID{#1}{#2}{\MGenerateID}{4}{#3;#4;#5;0}{\MStdPoints}{#6}
\addtocounter{MFieldCounter}{1}
}

% Parameter: Laenge des Feldes, Loesungsintervall, Genauigkeit der Zahlenwertpruefung
\newcommand{\MIntervalQuestion}[3]{
\MQuestionID{#1}{#2}{\MGenerateID}{6}{#3}{\MStdPoints}{#2}
\addtocounter{MFieldCounter}{1}
}

% Parameter: Laenge des Feldes, Loesungsintervall, Genauigkeit der Zahlenwertpruefung, UXID
\newcommand{\MLIntervalQuestion}[4]{
\MQuestionID{#1}{#2}{\MGenerateID}{6}{#3}{\MStdPoints}{#4}
\addtocounter{MFieldCounter}{1}
}

% Parameter: Laenge des Feldes, Loesungsfunktion, Anzahl Stuetzstellen, Funktionsvariable (nicht case-sensitive), Anzahl Nachkommastellen im Vergleich, Vereinfachungsbedingung
% Vereinfachungsbedingung ist eine der Folgenden:
% 0 = Keine Vereinfachungsbedingung
% 1 = Keine Klammern (runde oder eckige) mehr im vereinfachten Ausdruck
% 2 = Faktordarstellung (Term hat Produkte als letzte Operation, Summen als vorgeschaltete Operation)
% 3 = Summendarstellung (Term hat Summen als letzte Operation, Produkte als vorgeschaltete Operation)
% Flag 512: Besondere Stuetzstellen (nur >1 und nur schwach rational), sonst symmetrisch um Nullpunkt und ganze Zahlen inkl. Null werden getroffen
\newcommand{\MSimplifyQuestion}[6]{
\MQuestionID{#1}{#2}{\MGenerateID}{4}{#3;#4;#5;#6}{\MStdPoints}{#2}
\addtocounter{MFieldCounter}{1}
}

\newcommand{\MLSimplifyQuestion}[7]{
\MQuestionID{#1}{#2}{\MGenerateID}{4}{#3;#4;#5;#6}{\MStdPoints}{#7}
\addtocounter{MFieldCounter}{1}
}

% Parameter: Laenge des Feldes, Loesung (optionaler Ausdruck), Anzahl Stuetzstellen, Funktionsvariable (nicht case-sensitive), Anzahl Nachkommastellen im Vergleich, Spezialtyp (string-id)
\newcommand{\MLSpecialQuestion}[7]{
\MQuestionID{#1}{#2}{\MGenerateID}{7}{#3;#4;#5;#6}{\MStdPoints}{#7}
\addtocounter{MFieldCounter}{1}
}

\newcounter{MGroupStart}
\newcounter{MGroupEnd}
\newcounter{MGroupActive}

\newenvironment{MQuestionGroup}{
\setcounter{MGroupStart}{\value{MFieldCounter}}
\setcounter{MGroupActive}{1}
}{
\setcounter{MGroupActive}{0}
\setcounter{MGroupEnd}{\value{MFieldCounter}}
\addtocounter{MGroupEnd}{-1}
}

\newcommand{\MGroupButton}[1]{
\ifttm
\special{html:<button name="Name_Group}\arabic{MGroupStart}\special{html:to}\arabic{MGroupEnd}\special{html:" class="groupbutton" id="Group}\arabic{MGroupStart}\special{html:to}\arabic{MGroupEnd}\special{html:" %
type="button" onclick="group_button(}\arabic{MGroupStart}\special{html:,}\arabic{MGroupEnd}\special{html:);">}#1\special{html:</button>}
\else
\phantom{#1}
\fi
}

%----------------- Makros fuer die modularisierte Darstellung ------------------------------------

\def\MyText#1{#1}

% is used internally by the conversion package, should not be used by original tex documents
\def\MOrgLabel#1{\relax}

\ifttm

% Ein MLabel wird im html codiert durch das tag <!-- mmlabel;;Labelbezeichner;;SubjectArea;;chapter;;section;;subsection;;Index;;Objekttyp; //-->
\def\MLabel#1{%
\ifnum\value{MLastType}=8%
\ifnum\value{MCaptionOn}=0%
\MDebugMessage{ERROR: Grafik \arabic{MGraphicsCounter} hat separates label: #1 (Grafiklabels sollten nur in der Caption stehen)}%
\fi
\fi
\ifnum\value{MLastType}=12%
\ifnum\value{MCaptionOn}=0%
\MDebugMessage{ERROR: Video \arabic{MVideoCounter} hat separates label: #1 (Videolabels sollten nur in der Caption stehen}%
\fi
\fi
\ifnum\value{MLastType}=10\setcounter{MLastIndex}{\value{equation}}\fi
\label{#1}\begin{html}<!-- mmlabel;;#1;;\end{html}\arabic{MSubjectArea}\special{html:;;}\arabic{chapter}\special{html:;;}\arabic{section}\special{html:;;}\arabic{subsection}\special{html:;;}\arabic{MLastIndex}\special{html:;;}\arabic{MLastType}\special{html:; //-->}}%

\else

% Sonderbehandlung im PDF fuer Abbildungen in separater aux-Datei, da MGraphics die figure-Umgebung nicht verwendet
\def\MLabel#1{%
\ifnum\value{MLastType}=8%
\ifnum\value{MCaptionOn}=0%
\MDebugMessage{ERROR: Grafik \arabic{MGraphicsCounter} hat separates label: #1 (Grafiklabels sollten nur in der Caption stehen}%
\fi
\fi
\ifnum\value{MLastType}=12%
\ifnum\value{MCaptionOn}=0%
\MDebugMessage{ERROR: Video \arabic{MVideoCounter} hat separates label: #1 (Videolabels sollten nur in der Caption stehen}%
\fi
\fi
\label{#1}%
}%

\fi

% Gibt Begriff des referenzierten Objekts mit aus, aber nur im HTML, daher nur in Ausnahmefaellen (z.B. Copyrightliste) sinnvoll
\def\MCRef#1{\ifttm\special{html:<!-- mmref;;}#1\special{html:;;1; //-->}\else\vref{#1}\fi}


\def\MRef#1{\ifttm\special{html:<!-- mmref;;}#1\special{html:;;0; //-->}\else\vref{#1}\fi}
\def\MERef#1{\ifttm\special{html:<!-- mmref;;}#1\special{html:;;0; //-->}\else\eqref{#1}\fi}
\def\MNRef#1{\ifttm\special{html:<!-- mmref;;}#1\special{html:;;0; //-->}\else\ref{#1}\fi}
\def\MSRef#1#2{\ifttm\special{html:<!-- msref;;}#1\special{html:;;}#2\special{html:; //-->}\else \if#2\empty \ref{#1} \else \hyperref[#1]{#2}\fi\fi}

\def\MRefRange#1#2{\ifttm\MRef{#1} bis
\MRef{#2}\else\vrefrange[\unskip]{#1}{#2}\fi}

\def\MRefTwo#1#2{\ifttm\MRef{#1} und \MRef{#2}\else%
\let\vRefTLRsav=\reftextlabelrange\let\vRefTPRsav=\reftextpagerange%
\def\reftextlabelrange##1##2{\ref{##1} und~\ref{##2}}%
\def\reftextpagerange##1##2{auf den Seiten~\pageref{#1} und~\pageref{#2}}%
\vrefrange[\unskip]{#1}{#2}%
\let\reftextlabelrange=\vRefTLRsav\let\reftextpagerange=\vRefTPRsav\fi}

% MSectionChapter definiert falls notwendig das Kapitel vor der section. Das ist notwendig, wenn nur ein Einzelmodul uebersetzt wird.
% MChaptersGiven ist ein Counter, der von mconvert.pl vordefiniert wird.
\ifttm
\newcommand{\MSectionChapter}{\ifnum\value{MChaptersGiven}=0{\Dchapter{Modul}}\else{}\fi}
\else
\newcommand{\MSectionChapter}{\ifnum\value{chapter}=0{\Dchapter{Modul}}\else{}\fi}
\fi


\def\MChapter#1{\ifnum\value{MSSEnd}>0{\MSubsectionEndMacros}\addtocounter{MSSEnd}{-1}\fi\Dchapter{#1}}
\def\MSubject#1{\MChapter{#1}} % Schluesselwort HELPSECTION ist reserviert fuer Hilfesektion

\newcommand{\MSectionID}{UNKNOWNID}

\ifttm
\newcommand{\MSetSectionID}[1]{\renewcommand{\MSectionID}{#1}}
\else
\newcommand{\MSetSectionID}[1]{\renewcommand{\MSectionID}{#1}\tikzsetexternalprefix{#1}}
\fi


\newcommand{\MSection}[1]{\MSetSectionID{MODULID}\ifnum\value{MSSEnd}>0{\MSubsectionEndMacros}\addtocounter{MSSEnd}{-1}\fi\MSectionChapter\Dsection{#1}\MSectionStartMacros{#1}\setcounter{MLastIndex}{-1}\setcounter{MLastType}{1}} % Sections werden ueber das section-Feld im mmlabel-Tag identifiziert, nicht ueber das Indexfeld

\def\MSubsection#1{\ifnum\value{MSSEnd}>0{\MSubsectionEndMacros}\addtocounter{MSSEnd}{-1}\fi\ifttm\else\clearpage\fi\Dsubsection{#1}\MSubsectionStartMacros\setcounter{MLastIndex}{-1}\setcounter{MLastType}{2}\addtocounter{MSSEnd}{1}}% Subsections werden ueber das subsection-Feld im mmlabel-Tag identifiziert, nicht ueber das Indexfeld
\def\MSubsectionx#1{\Dsubsectionx{#1}} % Nur zur Verwendung in MSectionStart gedacht
\def\MSubsubsection#1{\Dsubsubsection{#1}\setcounter{MLastIndex}{\value{subsubsection}}\setcounter{MLastType}{3}\ifttm\special{html:<!-- sectioninfo;;}\arabic{section}\special{html:;;}\arabic{subsection}\special{html:;;}\arabic{subsubsection}\special{html:;;1;;}\arabic{MTestSite}\special{html:; //-->}\fi}
\def\MSubsubsectionx#1{\Dsubsubsectionx{#1}\ifttm\special{html:<!-- sectioninfo;;}\arabic{section}\special{html:;;}\arabic{subsection}\special{html:;;}\arabic{subsubsection}\special{html:;;0;;}\arabic{MTestSite}\special{html:; //-->}\else\addcontentsline{toc}{subsection}{#1}\fi}

\ifttm
\def\MSubsubsubsectionx#1{\ \newline\textbf{#1}\special{html:<br />}}
\else
\def\MSubsubsubsectionx#1{\ \newline
\textbf{#1}\ \\
}
\fi


% Dieses Skript wird zu Beginn jedes Modulabschnitts (=Webseite) ausgefuehrt und initialisiert den Aufgabenfeldzaehler
\newcommand{\MPageScripts}{
\setcounter{MFieldCounter}{1}
\addtocounter{MSiteCounter}{1}
\setcounter{MHintCounter}{1}
\setcounter{MCodeEditCounter}{1}
\setcounter{MGroupActive}{0}
\DoQBoxes
% Feldvariablen werden im HTML-Header in conv.pl eingestellt
}

% Dieses Skript wird zum Ende jedes Modulabschnitts (=Webseite) ausgefuehrt
\ifttm
\newcommand{\MEndScripts}{\special{html:<br /><!-- mfeedbackbutton;Seite;}\arabic{MTestSite}\special{html:;}\MGenerateSiteNumber\special{html:; //-->}
}
\else
\newcommand{\MEndScripts}{\relax}
\fi


\newcounter{QBoxFlag}
\newcommand{\DoQBoxes}{\setcounter{QBoxFlag}{1}}
\newcommand{\NoQBoxes}{\setcounter{QBoxFlag}{0}}

\newcounter{MXCTest}
\newcounter{MXCounter}
\newcounter{MSCounter}



\ifttm

% Struktur des sectioninfo-Tags: <!-- sectioninfo;;section;;subsection;;subsubsection;;nr_ausgeben;;testpage; //-->

%Fuegt eine zusaetzliche html-Seite an hinter ALLEN bisherigen und zukuenftigen content-Seiten ausserhalb der vor-zurueck-Schleife (d.h. nur durch Button oder MIntLink erreichbar!)
% #1 = Titel des Modulabschnitts, #2 = Kurztitel fuer die Buttons, #3 = Buttonkennung (STD = default nehmen, NONE = Ohne Button in der Navigation)
\newenvironment{MSContent}[3]{\special{html:<div class="xcontent}\arabic{MSCounter}\special{html:"><!-- scontent;-;}\arabic{MSCounter};-;#1;-;#2;-;#3\special{html: //-->}\MPageScripts\MSubsubsectionx{#1}}{\MEndScripts\special{html:<!-- endscontent;;}\arabic{MSCounter}\special{html: //--></div>}\addtocounter{MSCounter}{1}}

% Fuegt eine zusaetzliche html-Seite ein hinter den bereits vorhandenen content-Seiten (oder als erste Seite) innerhalb der vor-zurueck-Schleife der Navigation
% #1 = Titel des Modulabschnitts, #2 = Kurztitel fuer die Buttons, #3 = Buttonkennung (STD = Defaultbutton, NONE = Ohne Button in der Navigation)
\newenvironment{MXContent}[3]{\special{html:<div class="xcontent}\arabic{MXCounter}\special{html:"><!-- xcontent;-;}\arabic{MXCounter};-;#1;-;#2;-;#3\special{html: //-->}\MPageScripts\MSubsubsection{#1}}{\MEndScripts\special{html:<!-- endxcontent;;}\arabic{MXCounter}\special{html: //--></div>}\addtocounter{MXCounter}{1}}

% Fuegt eine zusaetzliche html-Seite ein die keine subsubsection-Nummer bekommt, nur zur internen Verwendung in mintmod.tex gedacht!
% #1 = Titel des Modulabschnitts, #2 = Kurztitel fuer die Buttons, #3 = Buttonkennung (STD = Defaultbutton, NONE = Ohne Button in der Navigation)
% \newenvironment{MUContent}[3]{\special{html:<div class="xcontent}\arabic{MXCounter}\special{html:"><!-- xcontent;-;}\arabic{MXCounter};-;#1;-;#2;-;#3\special{html: //-->}\MPageScripts\MSubsubsectionx{#1}}{\MEndScripts\special{html:<!-- endxcontent;;}\arabic{MXCounter}\special{html: //--></div>}\addtocounter{MXCounter}{1}}

\newcommand{\MDeclareSiteUXID}[1]{\special{html:<!-- mdeclaresiteuxid;;}#1\special{html:;;}\arabic{chapter}\special{html:;;}\arabic{section}\special{html:;; //-->}}

\else

%\newcommand{\MSubsubsection}[1]{\refstepcounter{subsubsection} \addcontentsline{toc}{subsubsection}{\thesubsubsection. #1}}


% Fuegt eine zusaetzliche html-Seite an hinter den bereits vorhandenen content-Seiten
% #1 = Titel des Modulabschnitts, #2 = Kurztitel fuer die Buttons, #3 = Iconkennung (im PDF wirkungslos)
%\newenvironment{MUContent}[3]{\ifnum\value{MXCTest}>0{\MDebugMessage{ERROR: Geschachtelter SContent}}\fi\MPageScripts\MSubsubsectionx{#1}\addtocounter{MXCTest}{1}}{\addtocounter{MXCounter}{1}\addtocounter{MXCTest}{-1}}
\newenvironment{MXContent}[3]{\ifnum\value{MXCTest}>0{\MDebugMessage{ERROR: Geschachtelter SContent}}\fi\MPageScripts\MSubsubsection{#1}\addtocounter{MXCTest}{1}}{\addtocounter{MXCounter}{1}\addtocounter{MXCTest}{-1}}
\newenvironment{MSContent}[3]{\ifnum\value{MXCTest}>0{\MDebugMessage{ERROR: Geschachtelter XContent}}\fi\MPageScripts\MSubsubsectionx{#1}\addtocounter{MXCTest}{1}}{\addtocounter{MSCounter}{1}\addtocounter{MXCTest}{-1}}

\newcommand{\MDeclareSiteUXID}[1]{\relax}

\fi

% GHEADER und GFOOTER werden von split.pm gefunden, aber nur, wenn nicht HELPSITE oder TESTSITE
\ifttm
\newenvironment{MSectionStart}{\special{html:<div class="xcontent0">}\MSubsubsectionx{\iModuleOverview{}}}{\setcounter{MSSEnd}{0}\special{html:</div>}}
% Darf nicht als XContent nummeriert werden, darf nicht als XContent gelabelt werden, wird aber in eine xcontent-div gesetzt fuer Python-parsing
\else
\newenvironment{MSectionStart}{\MSubsectionx{\iModuleOverview{}}}{\setcounter{MSSEnd}{0}}
\fi

\ifttm
\newenvironment{MIntro}{\begin{MXContent}{\iIntroduction{}}{\iIntroduction{}}{genetisch}\begin{html}<div
class="intro">\end{html}}{\begin{html}</div>\end{html}\end{MXContent}}
\else
\newenvironment{MIntro}{\begin{MXContent}{\iIntroduction{}}{\iIntroduction{}}{genetisch}}{\end{MXContent}}
\fi

\newenvironment{MContent}{\begin{MXContent}{\iContents{}}{\iContents{}}{beweis}}{\end{MXContent}}
\newenvironment{MExercises}{\ifttm\else\clearpage\fi\begin{MXContent}{\iExercises{}}{\iExercises{}}{aufgb}\special{html:<!-- declareexcsymb //-->}}{\end{MXContent}}

% #1 = Lesbare Testbezeichnung
\newenvironment{MTest}[1]{%
\renewcommand{\MTestName}{#1}
\ifttm\else\clearpage\fi%
\addtocounter{MTestSite}{1}%
\begin{MXContent}{#1}{#1}{STD} % {aufgb}%
\special{html:<!-- declaretestsymb //-->}
\begin{MQuestionGroup}%
\MInTestHeader
}%
{%
\end{MQuestionGroup}%
\ \\ \ \\%
\MInTestFooter
\end{MXContent}\addtocounter{MTestSite}{-1}%
}

\newenvironment{MExtra}{\ifttm\else\clearpage\fi\begin{MXContent}{Zusätzliche Inhalte}{Zusatz}{weiterfhrg}}{\end{MXContent}}

\makeindex

\ifttm
\def\MPrintIndex{
\ifnum\value{MSSEnd}>0{\MSubsectionEndMacros}\addtocounter{MSSEnd}{-1}\fi
\renewcommand{\indexname}{Index}
\special{html:<p><!-- printindex //--></p>}
}
\else
\def\MPrintIndex{
\ifnum\value{MSSEnd}>0{\MSubsectionEndMacros}\addtocounter{MSSEnd}{-1}\fi
\renewcommand{\indexname}{Index}
\addcontentsline{toc}{section}{Index}
\printindex
}
\fi


% Konstanten fuer die Modulfaecher

\def\MINTMathematics{1}
\def\MINTInformatics{2}
\def\MINTChemistry{3}
\def\MINTPhysics{4}
\def\MINTEngineering{5}

\newcounter{MSubjectArea}
\newcounter{MInfoNumbers} % Gibt an, ob die Infoboxen nummeriert werden sollen
\newcounter{MSepNumbers} % Gibt an, ob Beispiele und Experimente separat nummeriert werden sollen
\newcommand{\MSetSubject}[1]{
 % ttm kapiert setcounter mit Parametern nicht, also per if abragen und einsetzen
\ifnum#1=1\setcounter{MSubjectArea}{1}\setcounter{MInfoNumbers}{1}\setcounter{MSepNumbers}{0}\fi
\ifnum#1=2\setcounter{MSubjectArea}{2}\setcounter{MInfoNumbers}{1}\setcounter{MSepNumbers}{0}\fi
\ifnum#1=3\setcounter{MSubjectArea}{3}\setcounter{MInfoNumbers}{0}\setcounter{MSepNumbers}{1}\fi
\ifnum#1=4\setcounter{MSubjectArea}{4}\setcounter{MInfoNumbers}{0}\setcounter{MSepNumbers}{0}\fi
\ifnum#1=5\setcounter{MSubjectArea}{5}\setcounter{MInfoNumbers}{1}\setcounter{MSepNumbers}{0}\fi
% Separate Nummerntechnik fuer unsere Chemiker: alles dreistellig
\ifnum#1=3
  \ifttm
  \renewcommand{\theequation}{\arabic{section}.\arabic{subsection}.\arabic{equation}}
  \renewcommand{\thetable}{\arabic{section}.\arabic{subsection}.\arabic{table}}
  \renewcommand{\thefigure}{\arabic{section}.\arabic{subsection}.\arabic{figure}}
  \else
  \renewcommand{\theequation}{\arabic{chapter}.\arabic{section}.\arabic{equation}}
  \renewcommand{\thetable}{\arabic{chapter}.\arabic{section}.\arabic{table}}
  \renewcommand{\thefigure}{\arabic{chapter}.\arabic{section}.\arabic{figure}}
  \fi
\else
  \ifttm
  \renewcommand{\theequation}{\arabic{section}.\arabic{subsection}.\arabic{equation}}
  \renewcommand{\thetable}{\arabic{table}}
  \renewcommand{\thefigure}{\arabic{figure}}
  \else
  \renewcommand{\theequation}{\arabic{chapter}.\arabic{section}.\arabic{equation}}
  \renewcommand{\thetable}{\arabic{table}}
  \renewcommand{\thefigure}{\arabic{figure}}
  \fi
\fi
}

% Fuer tikz Autogenerierung
\newcounter{MTIKZAutofilenumber}

% Spezielle Counter fuer die Bentz-Module
\newcounter{mycounter}
\newcounter{chemapplet}
\newcounter{physapplet}

\newcounter{MSSEnd} % Ist 1 falls ein MSubsection aktiv ist, der einen MSubsectionEndMacro-Aufruf verursacht
\newcounter{MFileNumber}
\def\MLastFile{\special{html:[[!-- mfileref;;}\arabic{MFileNumber}\special{html:; //--]]}}

% Vollstaendiger Pfad ist \MMaterial / \MLastFilePath / \MLastFileName    ==   \MMaterial / \MLastFile

% Wird nur bei kompletter Baum-Erstellung ausgefuehrt!
% #1 = Lesbare Modulbezeichnung
\newcommand{\MSectionStartMacros}[1]{
\setcounter{MTestSite}{0}
\setcounter{MCaptionOn}{0}
\setcounter{MLastTypeEq}{0}
\setcounter{MSSEnd}{0}
\setcounter{MFileNumber}{0} % Preinkrekement-Counter
\setcounter{MTIKZAutofilenumber}{0}
\setcounter{mycounter}{1}
\setcounter{physapplet}{1}
\setcounter{chemapplet}{0}
\ifttm
\special{html:<!-- mdeclaresection;;}\arabic{chapter}\special{html:;;}\arabic{section}\special{html:;;}#1\special{html:;; //-->}%
\else
\setcounter{thmc}{0}
\setcounter{exmpc}{0}
\setcounter{verc}{0}
\setcounter{infoc}{0}
\fi
\setcounter{MiniMarkerCounter}{1}
\setcounter{AlignCounter}{1}
\setcounter{MXCTest}{0}
\setcounter{MCodeCounter}{0}
\setcounter{MEntryCounter}{0}
}

% Wird immer ausgefuehrt
\newcommand{\MSubsectionStartMacros}{
\ifttm\else\MPageHeaderDef\fi
\MWatermarkSettings
\setcounter{MXCounter}{0}
\setcounter{MSCounter}{0}
\setcounter{MSiteCounter}{1}
\setcounter{MExerciseCollectionCounter}{0}
% Zaehler fuer das Labelsystem zuruecksetzen (prefix-Zaehler)
\setcounter{MInfoCounter}{0}
\setcounter{MExerciseCounter}{0}
\setcounter{MExampleCounter}{0}
\setcounter{MExperimentCounter}{0}
\setcounter{MGraphicsCounter}{0}
\setcounter{MTableCounter}{0}
\setcounter{MTheoremCounter}{0}
\setcounter{MObjectCounter}{0}
\setcounter{MEquationCounter}{0}
\setcounter{MVideoCounter}{0}
\setcounter{equation}{0}
\setcounter{figure}{0}
}

\newcommand{\MSubsectionEndMacros}{
% Bei Chemiemodulen das PSE einhaengen, es soll als SContent am Ende erscheinen
\special{html:<!-- subsectionend //-->}
\ifnum\value{MSubjectArea}=3{\MIncludePSE}\fi
}


\ifttm
%\newcommand{\MEmbed}[1]{\MRegisterFile{#1}\begin{html}<embed src="\end{html}\MMaterial/\MLastFile\begin{html}" width="192" height="189"></embed>\end{html}}
\newcommand{\MEmbed}[1]{\MRegisterFile{#1}\begin{html}<embed src="\end{html}\MMaterial/\MLastFile\begin{html}"></embed>\end{html}}
\fi

%----------------- Makros fuer die Textdarstellung -----------------------------------------------

\ifttm
% MUGraphics bindet eine Grafik ein:
% Parameter 1: Dateiname der Grafik, relativ zur Position des Modul-Tex-Dokuments
% Parameter 2: Skalierungsoptionen fuer PDF (fuer includegraphics)
% Parameter 3: Titel fuer die Grafik, wird unter die Grafik mit der Grafiknummer gesetzt und kann MLabel bzw. MCopyrightLabel enthalten
% Parameter 4: Skalierungsoptionen fuer HTML (css-styles)

% ERSATZ: <img alt="My Image" src="data:image/png;base64,iVBORwA<MoreBase64SringHere>" />


\newcommand{\MUGraphics}[4]{\MRegisterFile{#1}\begin{html}
<div class="imagecenter">
<center>
<div>
%<img src="\end{html}\MMaterial/\MLastFile\begin{html}" style="#4" alt="\end{html}\MMaterial/\MLastFile\begin{html}"/>
<img class="mintmodimage" src="../../images/\end{html}#1\begin{html}"/>
</div>
<div class="bildtext">
\end{html}
\addtocounter{MGraphicsCounter}{1}
\setcounter{MLastIndex}{\value{MGraphicsCounter}}
\setcounter{MLastType}{8}
\addtocounter{MCaptionOn}{1}
\ifnum\value{MSepNumbers}=0
\textbf{Abbildung \arabic{MGraphicsCounter}:} #3
\else
\textbf{Abbildung \arabic{section}.\arabic{subsection}.\arabic{MGraphicsCounter}:} #3
\fi
\addtocounter{MCaptionOn}{-1}
\begin{html}
</div>
</center>
</div>
<br />
\end{html}%
\special{html:<!-- mfeedbackbutton;Abbildung;}\arabic{MGraphicsCounter}\special{html:;}\arabic{section}.\arabic{subsection}.\arabic{MGraphicsCounter}\special{html:; //-->}%
}

% MVideo bindet ein Video als Einzeldatei ein:
% Parameter 1: Dateiname des Videos, relativ zur Position des Modul-Tex-Dokuments, ohne die Endung ".mp4"
% Parameter 2: Titel fuer das Video (kann MLabel oder MCopyrightLabel enthalten), wird unter das Video mit der Videonummer gesetzt
\newcommand{\MVideo}[2]{\MRegisterFile{#1.mp4}\begin{html}
<div class="imagecenter">
<center>
<div>
<video width="95\%" controls="controls"><source src="\end{html}\MMaterial/#1.mp4\begin{html}" type="video/mp4">Ihr Browser kann keine MP4-Videos abspielen!</video>
</div>
<div class="bildtext">
\end{html}
\addtocounter{MVideoCounter}{1}
\setcounter{MLastIndex}{\value{MVideoCounter}}
\setcounter{MLastType}{12}
\addtocounter{MCaptionOn}{1}
\ifnum\value{MSepNumbers}=0
\textbf{Video \arabic{MVideoCounter}:} #2
\else
\textbf{Video \arabic{section}.\arabic{subsection}.\arabic{MVideoCounter}:} #2
\fi
\addtocounter{MCaptionOn}{-1}
\begin{html}
</div>
</center>
</div>
<br />
\end{html}}

\newcommand{\MDVideo}[2]{\MRegisterFile{#1.mp4}\MRegisterFile{#1.ogv}\begin{html}
<div class="imagecenter">
<center>
<div>
<video width="70\%" controls><source src="\end{html}\MMaterial/#1.mp4\begin{html}" type="video/mp4"><source src="\end{html}\MMaterial/#1.ogv\begin{html}" type="video/ogg">Ihr Browser kann keine MP4-Videos abspielen!</video>
</div>
<br />
#2
</center>
</div>
<br />
\end{html}
}

\newcommand{\MGraphics}[3]{\MUGraphics{#1}{#2}{#3}{}}

\else

\newcommand{\MVideo}[2]{%
% Kein Video im PDF darstellbar, trotzdem so tun als ob da eines waere
\begin{center}
(\iVideoWarn{})
\end{center}
\addtocounter{MVideoCounter}{1}
\setcounter{MLastIndex}{\value{MVideoCounter}}
\setcounter{MLastType}{12}
\addtocounter{MCaptionOn}{1}
\ifnum\value{MSepNumbers}=0
\textbf{Video \arabic{MVideoCounter}:} #2
\else
\textbf{Video \arabic{section}.\arabic{subsection}.\arabic{MVideoCounter}:} #2
\fi
\addtocounter{MCaptionOn}{-1}
}


% MGraphics bindet eine Grafik ein:
% Parameter 1: Dateiname der Grafik, relativ zur Position des Modul-Tex-Dokuments
% Parameter 2: Skalierungsoptionen fuer PDF (fuer includegraphics)
% Parameter 3: Titel fuer die Grafik, wird unter die Grafik mit der Grafiknummer gesetzt
\newcommand{\MGraphics}[3]{%
\MRegisterFile{#1}%
\ %
\begin{figure}[H]%
\centering{%
\includegraphics[#2]{\MDPrefix/#1}%
\addtocounter{MCaptionOn}{1}%
\caption{#3}%
\addtocounter{MCaptionOn}{-1}%
}%
\end{figure}%
\addtocounter{MGraphicsCounter}{1}\setcounter{MLastIndex}{\value{MGraphicsCounter}}\setcounter{MLastType}{8}\ %
%\ \\Abbildung \ifnum\value{MSepNumbers}=0\else\arabic{chapter}.\arabic{section}.\fi\arabic{MGraphicsCounter}: #3%
}

\newcommand{\MUGraphics}[4]{\MGraphics{#1}{#2}{#3}}


\fi

\newcounter{MCaptionOn} % = 1 falls eine Grafikcaption aktiv ist, = 0 sonst


% MGraphicsSolo bindet eine Grafik pur ein ohne Titel
% Parameter 1: Dateiname der Grafik, relativ zur Position des Modul-Tex-Dokuments
% Parameter 2: Skalierungsoptionen (wirken nur im PDF)
\newcommand{\MGraphicsSolo}[2]{\MUGraphicsSolo{#1}{#2}{}}

% MUGraphicsSolo bindet eine Grafik pur ein ohne Titel, aber mit HTML-Skalierung
% Parameter 1: Dateiname der Grafik, relativ zur Position des Modul-Tex-Dokuments
% Parameter 2: Skalierungsoptionen (wirken nur im PDF)
% Parameter 3: Skalierungsoptionen (wirken nur im HTML), als style-format: "width=???, height=???"
\ifttm
\newcommand{\MUGraphicsSolo}[3]{\MRegisterFile{#1}
% unecessarily convoluted. Use deploy script for images, not tex converter.
%\begin{html}
%<img src="\end{html}\MMaterial/\MLastFile\begin{html}" style="\end{html}#3\begin{html}" alt="\end{html}\MMaterial/\MLastFile\begin{html}"/>
\begin{html}<img class="mintmodimage" src="../../images/\end{html}#1\begin{html}"/>\end{html}
%\end{html}%
\special{html:<!-- mfeedbackbutton;Abbildung;}#1\special{html:;}\MMaterial/\MLastFile\special{html:; //-->}%
}
\else
\newcommand{\MUGraphicsSolo}[3]{\MRegisterFile{#1}\includegraphics[#2]{\MDPrefix/#1}}
\fi

% Externer Link mit URL
% Erster Parameter: Vollstaendige(!) URL des Links
% Zweiter Parameter: Text fuer den Link
\newcommand{\MExtLink}[2]{\ifttm\special{html:<a target="_new" href="}#1\special{html:">}#2\special{html:</a>}\else\href{#1}{#2}\fi} % ohne MINTERLINK!


% Interner Link, die verlinkte Datei muss im gleichen Verzeichnis liegen wie die Modul-Texdatei
% Erster Parameter: Dateiname
% Zweiter Parameter: Text fuer den Link
\newcommand{\MIntLink}[2]{\ifttm\MRegisterFile{#1}\special{html:<a class="MINTERLINK" target="_new" href="}\MMaterial/\MLastFile\special{html:">}#2\special{html:</a>}\else{\href{#1}{#2}}\fi}


\ifttm
\def\MMaterial{:localmaterial:}
\else
\def\MMaterial{\MDPrefix}
\fi

\ifttm
\def\MNoFile#1{:directmaterial:#1}
\else
\def\MNoFile#1{#1}
\fi

\newcommand{\MChem}[1]{$\mathrm{#1}$}

\newcommand{\MApplet}[3]{
% Bindet ein Java-Applet ein, die Parameter sind:
% (wird nur im HTML, aber nicht im PDF erstellt)
% #1 Dateiname des Applets (muss mit ".class" enden)
% #2 = Breite in Pixeln
% #3 = Hoehe in Pixeln
\ifttm
\MRegisterFile{#1}
\begin{html}
<applet code="\end{html}\MMaterial/\MLastFile\begin{html}" width="#2" height="#3" alt="[Java-Applet kann nicht gestartet werden]"></applet>
\end{html}
\fi
}

\newcommand{\MScriptPage}[2]{
% Bindet eine JavaScript-Datei ein, die eine eigene Seite bekommt
% (wird nur im HTML, aber nicht im PDF erstellt)
% #1 Dateiname des Programms (sollte mit ".js" enden)
% #2 = Kurztitel der Seite
\ifttm
\begin{MSContent}{#2}{#2}{puzzle}
\MRegisterFile{#1}
\begin{html}
<script src="\MMaterial/\MLastFile" type="text/javascript"></script>
\end{html}
\end{MSContent}
\fi
}

% Bindet in der Haupttexdatei ein MINT-Modul ein. Parameter 1 ist das Verzeichnis (relativ zur Haupttexdatei), Parameter 2 ist der Dateinahme ohne Pfad.
\newcommand{\IncludeModule}[2]{
\renewcommand{\MDPrefix}{#1}
\input{#1/#2}
\ifnum\value{MSSEnd}>0{\MSubsectionEndMacros}\addtocounter{MSSEnd}{-1}\fi
}

% Der ttm-Konverter setzt keine Makros im \input um, also muss hier getrickst werden:
% Das MDPrefix muss in den einzelnen Modulen manuell eingesetzt werden
\newcommand{\MInputFile}[1]{
\ifttm
\input{#1}
\else
\input{#1}
\fi
}


\newcommand{\MCases}[1]{\left\lbrace{\begin{array}{rl} #1 \end{array}}\right.}

\ifttm
\newenvironment{MCaseEnv}{\left\lbrace\begin{array}{rl}}{\end{array}\right.}
\else
\newenvironment{MCaseEnv}{\left\lbrace\begin{array}{rl}}{\end{array}\right.}
\fi

\def\MSkip{\ifttm\MCR\fi}

\ifttm
\def\MCR{\special{html:<br />}}
\else
\def\MCR{\ \\}
\fi


% Pragmas - Sind Schluesselwoerter, die dem Preprocessing sowie dem Konverter uebergeben werden und bestimmte
%           Aktionen ausloesen. Im Output (PDF und HTML) tauchen sie nicht auf.
\newcommand{\MPragma}[1]{%
\ifttm%
\special{html:<!-- mpragma;-;}#1\special{html:;; -->}%
\else%
% MPragmas werden vom Preprozessor direkt im LaTeX gefunden
\fi%
}

% Ersatz der Befehle textsubscript und textsuperscript, die ttm nicht kennt
\ifttm%
\newcommand{\MTextsubscript}[1]{\special{html:<sub>}#1\special{html:</sub>}}%
\newcommand{\MTextsuperscript}[1]{\special{html:<sup>}#1\special{html:</sup>}}%
\else%
\newcommand{\MTextsubscript}[1]{\textsubscript{#1}}%
\newcommand{\MTextsuperscript}[1]{\textsuperscript{#1}}%
\fi

%------------------ Einbindung von dia-Diagrammen ----------------------------------------------
% Beim preprocessing wird aus jeder dia-Datei eine tex-Datei und eine pdf-Datei erzeugt,
% diese werden hier jeweils im PDF und HTML eingebunden
% Parameter: Dateiname der mit dia erstellten Datei (OHNE die Endung .dia)
\ifttm%
\newcommand{\MDia}[1]{%
\MGraphicsSolo{#1minthtml.png}{}%
}
\else%
\newcommand{\MDia}[1]{%
\MGraphicsSolo{#1mintpdf.png}{scale=0.1667}%
}
\fi%

% subsup funktioniert im Ausdruck $D={\R}^+_0$, also \R geklammert und sup zuerst
% \ifttm
% \def\MSubsup#1#2#3{\special{html:<msubsup>} #1 #2 #3\special{html:</msubsup>}}
% \else
% \def\MSubsup#1#2#3{{#1}^{#3}_{#2}}
% \fi

%\input{local.tex}

% \ifttm
% \else
% \newwrite\mintlog
% \immediate\openout\mintlog=mintlog.txt
% \fi

% ----------------------- tikz autogenerator -------------------------------------------------------------------

\newcommand{\Mtikzexternalize}{\tikzexternalize}% wird bei Konvertierung ueber mconvert ggf. ausgehebelt!

\ifttm
\else
\tikzset%
{
  % Defines a custom style which generates pdf and converts to (low and hi-res quality) png and svg, then deletes the pdf
  % Important: DO NOT directly convert from pdf to hires-png or from svg to png with GraphViz convert as it has some problems and memory leaks
  png export/.style=%
  {
    external/system call/.add={}{;
      pdf2svg "\image.pdf" "\image.svg" ;
      convert -density 112.5 -transparent white "\image.pdf" "\image.png";
      inkscape --export-png="\image.4x.png" --export-dpi=450 --export-background-opacity=0 --without-gui "\image.svg";
      rm "\image.pdf"; rm "\image.log"; rm "\image.dpth"; rm "\image.idx"
    },
    external/force remake,
  }
}
\tikzset{png export}
\tikzsetexternalprefix{}
% PNGs bei externer Erzeugung in "richtiger" Groesse einbinden
\pgfkeys{/pgf/images/include external/.code={\includegraphics[scale=0.64]{#1}}}
\fi

% Spezielle Umgebung fuer Autogenerierung, Bildernamen sind nur innerhalb eines Moduls (einer MSection) eindeutig)

\newcommand{\MTIKZautofilename}{tikzautofile}

\ifttm
% HTML-Version: Vom Autogenerator erzeugte png-Datei einbinden, tikz selbst nicht ausfuehren (sprich: #1 schlucken)
\newcommand{\MTikzAuto}[1]{%
\addtocounter{MTIKZAutofilenumber}{1}
\renewcommand{\MTIKZautofilename}{mtikzauto_\arabic{MTIKZAutofilenumber}}
\MUGraphicsSolo{\MSectionID\MTIKZautofilename.4x.png}{scale=1}{\special{html:[[!-- svgstyle;}\MSectionID\MTIKZautofilename\special{html: //--]]}} % Styleinfos werden aus original-png, nicht 4x-png geholt!
%\MRegisterFile{\MSectionID\MTIKZautofilename.png} % not used right now
%\MRegisterFile{\MSectionID\MTIKZautofilename.svg}
}
\else%
% PDF-Version: Falls Autogenerator aktiv wird Datei automatisch benannt und exportiert
\newcommand{\MTikzAuto}[1]{%
\addtocounter{MTIKZAutofilenumber}{1}%
\renewcommand{\MTIKZautofilename}{mtikzauto_\arabic{MTIKZAutofilenumber}}
\tikzsetnextfilename{\MTIKZautofilename}%
#1%
}
\fi

% In einer reinen LaTeX-Uebersetzung kapselt der Preambelinclude-Befehl nur input,
% in einer konvertergesteuerten PDF/HTML-Uebersetzung wird er dagegen entfernt und
% die Preambeln an mintmod angehaengt, die Ersetzung wird von mconvert.pl vorgenommen.

\newcommand{\MPreambleInclude}[1]{\input{#1}}

% Globale Watermarksettings (werden auch nochmal zu Beginn jedes subsection gesetzt,
% muessen hier aber auch global ausgefuehrt wegen Einfuehrungsseiten und Inhaltsverzeichnis

\MWatermarkSettings

% ---------------------------------- Spezialbefehle fuer AD ------------------------------------------

%Abkuerzung fuer \longrightarrow:
\newcommand{\lto}{\ensuremath{\longrightarrow}}

%Makro fuer Funktionen:
\newcommand{\exfunction}[5]
{\begin{array}{rrcl}
 #1 \colon  & #2 &\lto & #3 \\[.05cm]
  & #4 &\longmapsto  & #5
\end{array}}

\newcommand{\function}[5]{%
#1:\;\left\lbrace{\begin{array}{rcl}
 #2 &\lto & #3 \\
 #4 &\longmapsto  & #5 \end{array}}\right.}


%Die Identitaet:
\DeclareMathOperator{\Id}{Id}

%Die Signumfunktion:
\DeclareMathOperator{\sgn}{sgn}

%Zwei Betonungskommandos (koennen angepasst werden):
\newcommand{\highlight}[1]{#1}
\newcommand{\modstextbf}[1]{#1}
\newcommand{\modsemph}[1]{#1}


\ifttm%
\newcommand{\MModstartBox}{\special{html:<!-- modstartbox //-->}}
\else%
\newcommand{\MModstartBox}{%
\relax
}
\fi

% ---------------------------------- Spezialbefehle fuer JL ------------------------------------------


\def\jccolorfkt{green!50!black} %Farbe des Funktionsgraphen
\def\jccolorfktarea{green!25!white} %Farbe der Fläche unter dem Graphen
\def\jccolorfktareahell{green!12!white} %helle Einfärbung der Fläche unter dem Graphen
\def\jccolorfktwert{green!50!black} %Farbe einzelner Punkte des Graphen

\newcommand{\MPfadBilder}{Bilder}

\ifttm%
\newcommand{\jMD}{\,\MD}%
\else%
\newcommand{\jMD}{\;\MD}%
\fi%

\def\MFormelZoomHint{\MInputHint{%
\iFormZoomHint{}}}

\def\jHTMLHinweisBedienung{\MInputHint{%
\iUsageHint{}}}

\def\jHTMLHinweisEingabeText{\MInputHint{%
\iTextInputHint{}}}

\def\jHTMLHinweisEingabeTerm{\MInputHint{%
\iTermInputHint{}}}

\def\jHTMLHinweisEingabeIntervalle{\MInputHint{%
\iIntervalInputHint{}}}

\def\jHTMLHinweisEingabeFunktionen{\MInputHint{%
\iFunctionInputHint{}}}

\def\jHTMLHinweisEingabeFunktionenSinCos{\MInputHint{%
\iSincosInputHint{}}}

\def\jHTMLHinweisEingabeFunktionenExp{\MInputHint{%
iExpInputHint{}}}

% ---------------------------------- Spezialbefehle fuer Fachbereich Physik --------------------------

\newcommand{\E}{{e}}
\newcommand{\ME}[1]{\cdot 10^{#1}}
\newcommand{\MU}[1]{\;\mathrm{#1}}
\newcommand{\MPG}[3]{%
  \ifnum#2=0%
    #1\ \mathrm{#3}%
  \else%
    #1\cdot 10^{#2}\ \mathrm{#3}%
  \fi}%
%

\newcommand{\MMul}{\MExponentensymbXYZl} % Nur eine Abkuerzung


% ---------------------------------- Stichwortfunktionialitaet ---------------------------------------

% mpreindexentry wird durch Auswahlroutine in conv.pl durch mindexentry substitutiert
\ifttm%
\def\MIndex#1{\index{#1}\special{html:<!-- mpreindexentry;;}#1\special{html:;;}\arabic{MSubjectArea}\special{html:;;}%
\arabic{chapter}\special{html:;;}\arabic{section}\special{html:;;}\arabic{subsection}\special{html:;;}\arabic{MEntryCounter}\special{html:; //-->}%
\setcounter{MLastIndex}{\value{MEntryCounter}}%
\addtocounter{MEntryCounter}{1}%
}%
% Copyrightliste wird als tex-Datei im preprocessing von conv.pl erzeugt und unter converter/tex/entrycollection.tex abgelegt
% Der input-Befehl funktioniert nur, wenn die aufrufende tex-Datei auf der obersten Ebene liegt (d.h. selbst kein input/include ist, insbesondere keine Moduldatei)
\def\MEntryList{} % \input funktioniert nicht, weil ttm (und damit das \input) ausgefuehrt wird, bevor Datei da ist
\else%
\def\MIndex#1{\index{#1}}
\def\MEntryList{\MAbort{Stichwortliste nur im HTML realisierbar}}%
\fi%

\def\MEntry#1#2{\textbf{#1}\MIndex{#2}} % Idee: MLastType auf neuen Entry-Typ und dann ein MLabel vergeben mit autogen-Nummer

% ---------------------------------- Befehle fuer Tests ----------------------------------------------

% MEquationItem stellt eine Eingabezeile der Form Vorgabe = Antwortfeld her, der zweite Parameter kann z.B. MSimplifyQuestion-Befehl sein
\ifttm
\newcommand{\MEquationItem}[2]{{#1}$\,=\,${#2}}%
\else%
\newcommand{\MEquationItem}[2]{{#1}$\;\;=\,${#2}}%
\fi

\ifttm
\newcommand{\MInputHint}[1]{%
\ifnum%
\if\value{MTestSite}>0%
\else%
\begin{html}<span class="text-info">\end{html}{#1}\begin{html}</span>\end{html}
\fi%
\fi%
}
\else
\newcommand{\MInputHint}[1]{\relax}
\fi

\ifttm
\newcommand{\MInTestHeader}{%
\iTest{}}
\else
\newcommand{\MInTestHeader}{%
\relax
}
\fi

\ifttm
\newcommand{\MInTestFooter}{%
\special{html:<button name="Name_TESTFINISH" id="TESTFINISH" type="button" onclick="finish_button('}\MTestName\special{html:');" class="testsbutton">}\iTestSubmit{}\special{html:</button>}%
\special{html:
&nbsp;&nbsp;&nbsp;&nbsp;&nbsp;&nbsp;&nbsp;&nbsp;
<button name="Name_TESTRESET" id="TESTRESET" type="button" onclick="reset_button();" class="testsbutton">}\iTestReset{}\special{html:</button>
<br />
<br />
<div class="xreply">
<p name="Name_TESTEVAL" id="TESTEVAL">}
\iTestEval{}
\special{html:<br />
</p>
</div>}
}
\else
\newcommand{\MInTestFooter}{%
\relax
}
\fi


% ---------------------------------- Notationsmakros -------------------------------------------------------------

% Notationsmakros die nicht von der Kursvariante abhaengig sind

\ifttm%
\def\MTextSF#1{\textrm{#1}}% im HTML zerreisst ein font-Wechsel auf sans-serif die Zeilenhoehe
\else%
\def\MTextSF#1{\textsf{#1}}%
\fi%

\newcommand{\MZahltrennzeichen}[1]{\renewcommand{\MZXYZhltrennzeichen}{#1}}
\newcommand{\MGrad}{{}^{\circ}} % Fuer Winkel im Gradmass

\ifttm
\newcommand{\MZahl}[3][\MZXYZhltrennzeichen]{\edef\MZXYZtemp{\noexpand\special{html:<mn>#2#1#3</mn>}}\MZXYZtemp}
\else
\newcommand{\MZahl}[3][\MZXYZhltrennzeichen]{{}#2{#1}#3}
\fi

\newcommand{\MEinheitenabstand}[1]{\renewcommand{\MEinheitenabstXYZnd}{#1}}
\ifttm
\newcommand{\MEinheit}[2][\MEinheitenabstXYZnd]{{}#1\edef\MEINHtemp{\noexpand\special{html:<mi mathvariant="normal">#2</mi>}}\MEINHtemp}
\else
\newcommand{\MEinheit}[2][\MEinheitenabstXYZnd]{{}#1 \mathrm{#2}}
\fi

\newcommand{\MExponentensymbol}[1]{\renewcommand{\MExponentensymbXYZl}{#1}}
\newcommand{\MExponent}[2][\MExponentensymbXYZl]{{}#1{} 10^{#2}}

%Punkte in 2 und 3 Dimensionen
\newcommand{\MPointTwo}[3][]{#1(#2\MCoordPointSep #3{}#1)}
\newcommand{\MPointThree}[4][]{#1(#2\MCoordPointSep #3\MCoordPointSep #4{}#1)}
\newcommand{\MPointTwoAS}[2]{\left(#1\MCoordPointSep #2\right)}
\newcommand{\MPointThreeAS}[3]{\left(#1\MCoordPointSep #2\MCoordPointSep #3\right)}

% Masseinheit, Standardabstand: \,
\newcommand{\MEinheitenabstXYZnd}{\MThinspace}

% Horizontaler Leerraum zwischen herausgestellter Formel und Interpunktion
\ifttm
\newcommand{\MDFPSpace}{\,}
\newcommand{\MDFPaSpace}{\,\,}
\newcommand{\MBlank}{\ }
\else
\newcommand{\MDFPSpace}{\;}
\newcommand{\MDFPaSpace}{\;\;}
\newcommand{\MBlank}{\ }
\fi

% Satzende in herausgestellter Formel mit horizontalem Leerraum
\newcommand{\MDFPeriod}{\MDFPSpace .}

% Separation von Aufzaehlung und Bedingung in Menge
\newcommand{\MCondSetSep}{\,:\,} %oder '\mid'

% Konverter kennt mathopen nicht
\ifttm
\def\mathopen#1{}
\fi

% -----------------------------------START Rouletteaufgaben ------------------------------------------------------------

\ifttm
% #1 = Dateiname, #2 = eindeutige ID fuer das Roulette im Kurs
\newcommand{\MDirectRouletteExercises}[2]{
\begin{MExercise}
\texttt{Im HTML erscheinen hier Aufgaben aus einer Aufgabenliste...}
\end{MExercise}
}
\else
\newcommand{\MDirectRouletteExercises}[2]{\relax} % wird durch mconvert.pl gefunden und ersetzt
\fi


% ---------------------------------- START Makros, die von der Kursvariante abhaengen ----------------------------------

\ifvariantunotation
  % unotation = An Universitaeten uebliche Notation
  \def\MVariant{unotation}

  % Trennzeichen fuer Dezimalzahlen
  \newcommand{\MZXYZhltrennzeichen}{.}

  % Exponent zur Basis 10 in der Exponentialschreibweise,
  % Standardmalzeichen: \times
  \newcommand{\MExponentensymbXYZl}{\times}

  % Begrenzungszeichen fuer offene Intervalle
  \newcommand{\MoIl}[1][]{\mbox{}#1(\mathopen{}} % bzw. ']'
  \newcommand{\MoIr}[1][]{#1)\mbox{}} % bzw. '['

  % Zahlen-Separation im IntervaLL
  \newcommand{\MIntvlSep}{,} %oder ';'

  % Separation von Elementen in Mengen
  \newcommand{\MElSetSep}{,} %oder ';'

  % Separation von Koordinaten in Punkten
  \newcommand{\MCoordPointSep}{,} %oder ';' oder '|', '\MThinspace|\MThinspace'

\else
  % An dieser Stelle wird angenommen, dass std-Variante aktiv ist
  % std = beschlossene Notation im TU9-Onlinekurs
  \def\MVariant{std}

  % Trennzeichen fuer Dezimalzahlen
  \newcommand{\MZXYZhltrennzeichen}{.}

  % Exponent zur Basis 10 in der Exponentialschreibweise,
  % Standardmalzeichen: \times
  \newcommand{\MExponentensymbXYZl}{\times}

  % Begrenzungszeichen fuer offene Intervalle
  \newcommand{\MoIl}[1][]{\mbox{}#1]\mathopen{}} % bzw. '('
  \newcommand{\MoIr}[1][]{#1[\mbox{}} % bzw. ')'

  % Zahlen-Separation im IntervaLL
  \newcommand{\MIntvlSep}{;} %oder ','

  % Separation von Elementen in Mengen
  \newcommand{\MElSetSep}{;} %oder ','

  % Separation von Koordinaten in Punkten
  \newcommand{\MCoordPointSep}{;} %oder '|', '\MThinspace|\MThinspace'

\fi



% ---------------------------------- ENDE Makros, die von der Kursvariante abhaengen ----------------------------------


% diese Kommandos setzen Mathemodus vorraus
\newcommand{\MGeoAbstand}[2]{[\overline{{#1}{#2}}]}
\newcommand{\MGeoGerade}[2]{{#1}{#2}}
\newcommand{\MGeoStrecke}[2]{\overline{{#1}{#2}}}
\newcommand{\MGeoDreieck}[3]{{#1}{#2}{#3}}

%
\ifttm
\newcommand{\MOhm}{\special{html:<mn>&#x3A9;</mn>}}
\else
\newcommand{\MOhm}{\Omega} %\varOmega
\fi


\def\PERCTAG{\MAbort{PERCTAG ist zur internen verwendung in mconvert.pl reserviert, dieses Makro darf sonst nicht benutzt werden.}}

% Im Gegensatz zu einfachen html-Umgebungen werden MDirectHTML-Umgebungen von mconvert.pl am ganzen ttm-Prozess vorbeigeschleust und aus dem PDF komplett ausgeschnitten
\ifttm%
\newenvironment{MDirectHTML}{\begin{html}}{\end{html}}%
\else%
\newenvironment{MDirectHTML}{\begin{html}}{\end{html}}%
\fi

% Im Gegensatz zu einfachen Mathe-Umgebungen werden MDirectMath-Umgebungen von mconvert.pl am ganzen ttm-Prozess vorbeigeschleust, ueber MathJax realisiert, und im PDF als $$ ... $$ gesetzt
\ifttm%
\newenvironment{MDirectMath}{\begin{html}}{\end{html}}%
\else%
\newenvironment{MDirectMath}{\begin{equation*}}{\end{equation*}}% Vorsicht, auch \[ und \] werden in amsmath durch equation* redefiniert
\fi

% ---------------------------------- Location Management ---------------------------------------------

% #1 = buttonname (muss in files/images liegen und Format 48x48 haben), #2 = Vollstaendiger Einrichtungsname, #3 = Kuerzel der Einrichtung,  #4 = Name der include-texdatei
\ifttm
\newcommand{\MLocationSite}[3]{\special{html:<!-- mlocation;;}#1\special{html:;;}#2\special{html:;;}#3\special{html:;; //-->}}
\else
\newcommand{\MLocationSite}[3]{\relax}
\fi

% ---------------------------------- Copyright Management --------------------------------------------

\newcommand{\MCCLicense}{%
{\color{green}\textbf{CC BY-SA 3.0}}
}

\newcommand{\MCopyrightLabel}[1]{ (\MSRef{L_COPYRIGHTCOLLECTION}{Lizenz})\MLabel{#1}}
\newcommand{\MSilentCopyrightLabel}[1]{\MLabel{#1}} % label is added to tex code but not visible

% Copyrightliste wird als tex-Datei im preprocessing erzeugt und unter converter/tex/copyrightcollection.tex abgelegt
% Der input-Befehl funktioniert nur, wenn die aufrufende tex-Datei auf der obersten Ebene liegt (d.h. selbst kein input/include ist, insbesondere keine Moduldatei)
\newcommand{\MCopyrightCollection}{\input{copyrightcollection.tex}}

% MCopyrightNotice fuegt eine Copyrightnotiz ein, der parser ersetzt diese durch CopyrightNoticePOST im preparsing, diese Definition wird nur fuer reine pdflatex-Uebersetzungen gebraucht
% Parameter: #1: Kurze Lizenzbeschreibung (typischerweise \MCCLicense)
%            #2: Link zum Original (http://...) oder NONE falls das Bild selbst ein Original ist, oder TIKZ falls das Bild aus einer tikz-Umgebung stammt
%            #3: Link zum Autor (http://...) oder MINT falls Original im MINT-Kolleg erstellt oder NONE falls Autor unbekannt
%            #4: Bemerkung (z.B. dass Datei mit Maple exportiert wurde)
%            #5: Labelstring fuer existierendes Label auf das copyrighted Objekt, mit MCopyrightLabel erzeugt
%            Keines der Felder darf leer sein!
\newcommand{\MCopyrightNotice}[5]{\MCopyrightNoticePOST{#1}{#2}{#3}{#4}{#5}}

\ifttm%
\newcommand{\MCopyrightNoticePOST}[5]{\relax}%
\else%
\newcommand{\MCopyrightNoticePOST}[5]{\relax}%
\fi%

% ---------------------------------- Meldungen fuer den Benutzer des Konverters ----------------------
\MPragma{mintmodversion;P0.1.0}
\MPragma{usercomment;This is file mintmod.tex version P0.1.0}


% ----------------------------------- Spezialelemente fuer Konfigurationsseite, werden nicht von mintscripts.js verwaltet --

% #1 = DOM-id der Box
\ifttm\newcommand{\MConfigbox}[1]{\special{html:<input cfieldtype="2" type="checkbox" name="Name_}#1\special{html:" id="}#1\special{html:" onchange="confHandlerChange('}#1\special{html:');"/>}}\fi % darf im PDF nicht aufgerufen werden!
   % f"ur HTML-Version (der Name muss mintmod.tex sein).
\MPragma{MathSkip}

%\MRegisterFile{Bilder}
\MRegisterFile{Bildquelle/}

\def\MPfadBilder{Bildquelle}
%\input{\MPfadBilder/jdefBilder.tex}


\def\jTikZScale{1.0}
\def\jccolorfkt{green!50!black} %Farbe des Funktionsgraphen
\def\jccolorfktarea{green!25!black} %Farbe der Fl"ache unter dem Graphen
\def\jccolorfktareahell{green!12!black} %helle Einf"arbung der Fl"ache unter dem Graphen
\def\jccolorfktwert{green!50!black} %Farbe einzelner Punkte des Graphen

%Export von TikZ-Bildern:
%\tikzexternalize % Export der Bilder
%\tikzsetextenalprefix{filenameprefex} % Namenspr"afix, Bsp.: vbkm08_
%Hinweis: Bildnamen mit \tikzsetnextfilename{filename} % Name ohne Endung!
%Hinweis: LaTeX-Aufruf via pdflatex -shell-escape filename.tex
%
%%%%%%%%%% Beachte: Befehl \MRegisterFile deaktivieren %%%%%%%%%%%%%%%%%%%%%%%%%
%\tikzexternalize
%%\tikzsetexternalprefix{vbkm07b_} % Wenn m"oglich, ohne Pr"afix.


\begin{document}

%\MSetSubject{MINTMathematics}
\MSection{Differentialrechnung}\MLabel{VBKM07}

\begin{MSectionStart}\MLabel{L_SART7}

Das Modul umfasst folgende Abschnitte:
\begin{itemize}
\item{\MSRef{M07_Eingangstest}{Eingangstest},}
\item{\MSRef{M07_Ableitung}{Ableitung einer Funktion},}
\item{\MSRef{M07_Standardableitungen}{Ableitung elementarer Funktionen},}
\item{\MSRef{M07_Rechenregeln}{Rechenregeln},}
\item{\MSRef{M07_Eigenschaften}{Eigenschaften},}
\item{\MSRef{M07_Anwendungen}{Anwendungen},}
\item{\MSRef{M07_Zusammenfassung}{Zusammenfassung},}
\item{\MSRef{M07_Ausgangstest}{Ausgangstest}.}
\end{itemize}
\end{MSectionStart}



\MSubsection{Eingangstest}\MLabel{M07_Eingangstest}

\begin{MIntro}
Der Test umfasst Aufgaben zu den Inhalten dieses Moduls. 
Das unverbindliche Ergebnis ist Grundlage einer Empfehlung, ob und in welcher 
Weise die Inhalte des Moduls bearbeitet werden sollen.
\end{MIntro}

\begin{MTest}
\begin{MExercise} %"Anderungsraten und Ableitung
Ein Lieferwagen, dessen Kilometerz"ahler $20$ km anzeigt, startet seine Tour 
um sechs Uhr. Er erreicht sein Ziel vier Stunden sp"ater. Der Kilometerz"ahler
zeigt jetzt $280$ km.
Berechnen Sie die mittlere Geschwindigkeit $v$ , also die mittlere "Anderungsrate 
zwischen Start- und Zielort. Setzten Sie dazu die fehlenden Zahlen und 
mathematischen Symbole $+$, $-$, $\cdot$, $/$ in die folgende Rechnung ein:
\ifttm
\begin{center}
$v = \bigg(280\,$\MQuestion{2}{-}$\,$\MParsedQuestion{6}{20}{3}$\bigg)\;$%
\MQuestion{2}{/}$\;\bigg($\MParsedQuestion{6}{10}{3}$\, - \, 6\bigg) %
 = \,$\MParsedQuestion{6}{65}{3}%%
\end{center}
\else
\[
v = \left(280 \;\;\; \MQuestion{3}{-} \;\;\; \MParsedQuestion{5}{20}{3}\right) %
\;\;\; \MQuestion{3}{/} \;\;\; \left(\,\MParsedQuestion{5}{10}{3} - 6\right)%
 = \,\MParsedQuestion{5}{65}{3}%%
\]
\fi
\end{MExercise}

\begin{MExercise} %"Anderungsraten und Ableitung
Gegeben ist die Funktion $f: [-3; 4] \to \R$, deren Graph hier gezeichnet ist.

%Bild:
\ifttm
\MUGraphics{\MPfadBilder/jb07A0_ETest_AgAbleitung.png}{scale=0.4}{Ableitung}{width:200px}
%\MUGraphics{jb07A0_ETest_AgAbleitung.png}{scale=0.4}{Ableitung}{}
%\MUGraphicsSolo{jb07A0_ETest_AgAbleitung.png}{scale=1}{}
\else
\begin{center}
%%LaTeX-File, Liedtke, 20140825.
%VBKM-Modul 7 Differentialrechnung: Bild Funktionsgraph zum Eingangstest
%Bildname: jb07A0_ETest_AgAbleitung.tex.
%Erstellt: 20140825, Liedtke.
%Bearbeitet: 20140829, Liedtke (Dateiname angepasst).
%Bearbeitet: 20140901, Liedtke (Dateiname ohne Endung erfasst).

%\tikzsetnextfilename{jb07A0_ETest_AgAbleitung}
\begin{small}
\begin{tikzpicture}[line width=1.5pt,scale=\jTikZScale, %
declare function={
  fkt(\x) = \x/2 * sqrt((\x + 3)*(4 - \x));
}
] %[every node/.style={fill=white}] 
%Koordinatenachsen:
\draw[->] (-3.6, 0) -- (5, 0) node[below left]{$x$}; %x-Achse
\draw[->] (0, -3) -- (0, 4) node[below left]{$y$}; %y-Achse
%Achsenbeschriftung:
\foreach \x in {-3, -2, -1} \draw (\x, 0) -- ++(0, 0.1) %
 node[above] {$\x$}; 
\foreach \x in {1, 2, 3, 4} \draw (\x, 0) -- ++(0, -0.1) %
 node[below] {$\x$}; 
\foreach \y in {-2, -1} \draw (0, \y) -- ++(0.1, 0) %
 node[right] {$\y$};
\foreach \y in {1, 2, 3} \draw (0, \y) -- ++(-0.1, 0) %
 node[left] {$\y$};
%\node[below left] at (0, 0) {$0$};
%Funktion:
\draw[domain=-3:4,samples=120,color=\jccolorfkt] %
 plot (\x, {fkt(\x)});
%Punkte
\filldraw[color=black,fill=black] (0, 0) circle (1pt);
%\node[below left] at (0, 0) {$x_2 = 0$};
\filldraw[color=black,fill=black] (4, 0) circle (1pt);
%\node[left] at (4, -0.1) {$x_1 = 4$};
\end{tikzpicture}
\end{small}

%end of file
 
%LaTeX-File, Liedtke, 20140825.
%VBKM-Modul 7 Differentialrechnung: Bild Funktionsgraph zum Eingangstest
%Bildname: jb07A0_ETest_AgAbleitung.tex.
%Erstellt: 20140825, Liedtke.
%Bearbeitet: 20140829, Liedtke (Dateiname angepasst).
%Bearbeitet: 20140901, Liedtke (Dateiname ohne Endung erfasst).
\begin{small}
\renewcommand{\jTikZScale}{0.6}
\tikzsetnextfilename{jb07A0_ETest_AgAbleitung}
\begin{tikzpicture}[line width=1.5pt,scale=\jTikZScale, %
declare function={
  fkt(\x) = \x/2 * sqrt((\x + 3)*(4 - \x));
}
] %[every node/.style={fill=white}] 
%Koordinatenachsen:
\draw[->] (-3.6, 0) -- (5, 0) node[below left]{$x$}; %x-Achse
\draw[->] (0, -3) -- (0, 4) node[below left]{$y$}; %y-Achse
%Achsenbeschriftung:
\foreach \x in {-3, -2, -1} \draw (\x, 0) -- ++(0, 0.1) %
 node[above] {$\x$}; 
\foreach \x in {1, 2, 3, 4} \draw (\x, 0) -- ++(0, -0.1) %
 node[below] {$\x$}; 
\foreach \y in {-2, -1} \draw (0, \y) -- ++(0.1, 0) %
 node[right] {$\y$};
\foreach \y in {1, 2, 3} \draw (0, \y) -- ++(-0.1, 0) %
 node[left] {$\y$};
%\node[below left] at (0, 0) {$0$};
%Funktion:
\draw[domain=-3:4,samples=120,color=\jccolorfkt] %
 plot (\x, {fkt(\x)});
%Punkte
\filldraw[color=black,fill=black] (0, 0) circle (1pt);
%\node[below left] at (0, 0) {$x_2 = 0$};
\filldraw[color=black,fill=black] (4, 0) circle (1pt);
%\node[left] at (4, -0.1) {$x_1 = 4$};
\end{tikzpicture}
\end{small}
\end{center}
%end of file
\fi
%Bildende.

\begin{MQuestionGroup}
\begin{MExerciseItems}
\item In $x_1 = 4$ ist die Ableitung
\MCheckbox{0} gleich $0$,
\MCheckbox{1} nicht definiert,
\MCheckbox{0} unendlich.
\item In $x_2 = 0$ ist die Ableitung
\MCheckbox{1} positiv,
\MCheckbox{0} gleich $0$,
\MCheckbox{0} negativ.
\end{MExerciseItems}
\end{MQuestionGroup}
\MGroupButton{Auswahl pr"ufen}
\end{MExercise}

\begin{MExercise}
Berechnen Sie die erste und zweite Ableitung der Funktion
%$f: \R \to \R, x \mapsto \frac{x+1}{2} \cdot \MEU^{4x + 2}$,
%$f: \R \to \R, x \mapsto \frac{3 x + 1}{x^2 + 2}$,
%$f: \R \to \R, x \mapsto \frac{6 x + 2}{2 x^2 + 4}$,

%$f: \R \to \R, x \mapsto \frac{2 x^2 - 6}{x - 1}$,
$f: \R \to \R, x \mapsto \frac{5 x}{3 + x^2}$,
und geben Sie Ihr Resultat gek"urzt und zusammengefasst an:
\begin{MExerciseItems}
\item Erste Ableitung $f'(x) = $%
\MSimplifyQuestion{40}{5 * (3 - x^2) / (3 + x^2)^2}{10}{x}{4}{0}
\item Zweite Ableitung ${f'}'(x) = $%
\MSimplifyQuestion{40}{10 * x * (x^2 - 9) / (3 + x^2)^3}{10}{x}{4}{0}
\end{MExerciseItems}
\end{MExercise}

\begin{MExercise} %Eigenschaften
In welchen Bereichen ist die Funktion $f$ mit $f(x) := \frac{\ln x}{x}$ f"ur
$x > 0$ monoton fallend beziehungsweise monoton wachsend?
Geben Sie Ihre Antwort in Form m"oglichst gro"ser offener Intervalle $(c; d)$ an
(hier steht \texttt{infty} f"ur $\infty$):
\begin{MExerciseItems}
\item $f$ ist auf \MIntervalQuestion{20}{(0;e)}{4} monoton wachsend.
\item $f$ ist auf \MIntervalQuestion{20}{(e;infty)}{4} monoton fallend.
\end{MExerciseItems}

Welche der Stellen $x_1 = 1$, $x_2 = 2$ oder $x_3 = 6$ geh"oren zu einem
Bereich, in dem $f$ konvex ist?

Antwort: %
%Es ist $f$ auf $(\MEU^{\frac{3}{2}}, \infty)$ konvex.
\MParsedQuestion{8}{6}{3}
\end{MExercise}
\end{MTest}


%%%Abschnitt
\MSubsection{Ableitung}\MLabel{M07_Ableitung}

\begin{MIntro}
Es wird erl"autert, in welcher Weise die Ableitung einer Funktion 
beschreibt, wie stark sich die Funktionswerte "andern.
Die Ableitung wird im Folgenden zun"achst anhand der analytischen Frage 
eingef"uhrt, den Verlauf der Funktionswerte {\glqq}n"aherungsweise{\grqq}
durch eine ganz einfache Funktion, n"amlich eine Gerade, zu beschreiben.

%Diese Sichtweise ist m"oglicherweise f"ur manche Leserinnen und Leser neu. 
Dadurch soll eine erste Idee vermittelt werden, weshalb Eigenschaften von 
Funktionen so einfach mittels ihrer Ableitung beschrieben werden k"onnen.
Zudem haben Sie hier die Gelegenheit, Fragestellungen der Analysis in einem 
anschaulichen Kontext kennen zu lernen, die auch f"ur weitergehende 
mathematische Betrachtungen und auch f"ur naturwissenschaftlich-technische 
Anwendungen relevant sind.
\end{MIntro}

%%%Inhalt Teil A:
%LaTeX-2e-File, Liedtke, 20140724.
%Inhalt: Einf"uhrung der Ableitung.
%zuletzt bearbeitet: 20140922.

%\begin{MContent}

%%\MSubsubsection{Ableitung--ein Ma"s f"ur die "Anderungsrate der Funktionswerte}
%\MSubsubsection{Zur Idee der Ableitung als ein Ma"s f"ur die "Anderungsrate %
% der Funktionswerte}
\begin{MXContent}{Zur Idee der Ableitung als Ma"s f"ur die "Anderungsrate %
 der Funktionswerte}{Zur Idee der Ableitung}{STD}

Gegeben ist eine Funktion $f: [a, b] \to \R$ und eine Stelle $x_0$ zwischen 
$a$ und $b$.

Wir wollen m"oglichst einfach beschreiben, wie sich die Funktionwerte von 
$f$ lokal, also in der N"ahe einer Stelle $x_0$, "andern.

Eine Gerade beschreibt in einfacher Weise, wie wir von einem Punkt zu einem
anderen kommen k"onnen. Sie kann als vereinfachte, aber weiterhin 
aussagekr"aftige Beschreibung irgendeines {\glqq}krummlinigen{\grqq} Graphen
in einer kleinen Umgebung von einem Punkt angesehen werden.
%In einem ersten Schritt ist eine Gerade $g$ derart gesucht, dass sich die 

Unser Ziel ist es, eine Gerade $g$ derart zu suchen, dass sich die 
Funktionswerte und die "Anderungsraten von $g$ an $f$ \emph{stetig} ann"ahern, 
wenn sich die $x$-Werte an $x_0$ ann"ahern.

Zuerst "uberlegen wir uns, wie wir "Anderungen beschreiben k"onnen. In diesem
Kontext wird dann auch die Idee der stetigen Ann"aherung erl"autert.
%Wir wollen beschreiben, wie sich die Funktionswerte einer Funktion $f$ "andern.
\end{MXContent}

%\MSubsubsection{Absolute und relative "Anderungen}
%\MSubsubsection{Absolute "Anderung}
\begin{MXContent}{Absolute "Anderung}{Absolute "Anderung}{STD}

%Bild:
\begin{center}
\ifttm
\MGraphicsSolo{\MPfadBilder/jb07A1_Funktionsgraph.png}{scale=0.5}
\else
%Datei: {\MPfadBilder/jb07A1_Funktionsgraph.tex}
\begin{small}
\renewcommand{\jTikZScale}{0.7}
\tikzsetnextfilename{jb07A1_Funktionsgraph}
\begin{tikzpicture}[line width=1.5pt,scale=\jTikZScale, %
declare function={
  x0 = 2;
  x1 = 4;
  fkt(\x) = 1/4 * (\x - 1)*(\x - 1) + 0.75;
  rT = 1.6; % relative Translation der Beschriftung $\Delta(f)$
%  Tangente(\x) = 1/2 * (\x - 2) + 1;
}
] %[every node/.style={fill=white}] 
%Koordinatenachsen:
\draw[->] (-0.6, 0) -- ({x1+1}, 0) node[below left]{$x$}; %x-Achse
\draw[->] (0, -0.6) -- (0, {fkt(x1)+1}) node[below left]{$y$}; %y-Achse
%Achsenbeschriftung:
\foreach \x in {1, 2, 3, 4} \draw (\x, 0) -- ++(0, -0.1); %
% node[below] {$\x$}; 
\foreach \y in {1, 2, 3} \draw (0, \y) -- ++(-0.1, 0); %
% node[left] {$\y$};
%\node[below left] at (0, 0) {$0$};
%Hilfslinien:
\draw[color=black!50!white,style=dashed] (x1, {fkt(x0)}) -- (x1, {fkt(x1)});
\draw[color=black!50!white,style=dotted] (x0, {fkt(x0)}) -- (x1, {fkt(x0)});
%Funktion:
\draw[domain=0.5:x1,samples=120,color=\jccolorfkt] %
 plot (\x, {fkt(\x)});
%Punkte und Hilfslinien:
\filldraw[color=black,fill=black] (x0, 0) circle (1pt);
\node[below] at (x0, -0.1) {$x_0$};
\filldraw[color=black,fill=black] (x0, {fkt(x0)}) circle (1pt);
%
\filldraw[color=black,fill=black] (x1, 0) circle (1pt);
\node[below] at (x1, -0.1) {$x$};
%
\filldraw[color=black,fill=black] (x1, {fkt(x0)}) circle (1pt);
\filldraw[color=black,fill=black] (x1, {fkt(x1)}) circle (1pt);
%Beschriftung:
\draw[line width=0.8pt, rounded corners=4pt] %
 ({x1 + rT + 0.0}, 1.0) -- ({x1 + rT + 0.2}, 1.0) -- ({x1 + rT + 0.2}, 2.0) %
 -- ({x1 + rT + 0.4}, 2.0);
\draw[line width=0.8pt, rounded corners=4pt] %
 ({x1 + rT + 0.4}, 2.0) %
 -- ({x1 + rT + 0.2}, 2.0) -- ({x1 + rT + 0.2}, 3.0) -- ({x1 + rT + 0.0}, 3.0);
%
\node[right] at ({x1+rT+0.5}, 2) {$\Delta(f) = f(x) - f(x_0)$};
%
\node[left] at (-0.5, {fkt(x0)}) {$f(x_0)$};
\node[left] at (-0.5, {fkt(x1)}) {$f(x)$};
\end{tikzpicture}
\end{small}
\fi
\end{center}
%Bildende.

Die "Anderung der Funktionswerte zwischen $x$ und $x_0$ ist 
\[
\Delta(f) := f(x) - f(x_0).
\]
Die Differenz $\Delta(f)$ wird auch 
\MEntry{absolute "Anderung}{absolute \"Anderung} von $f$ genannt.
Die Differenz $\Delta(f)$ gibt den Unterschied zwischen $f(x)$ und $f(x_0)$ an.

Viele Funktionen $f$ haben die Eigenschaft, dass die Differenz $\Delta(f)$ 
jedenfalls dann nahezu null ist, wenn nur $x$ nahe genug bei $x_0$ liegt, 
das hei"st, wenn $x - x_0$ nahezu null ist.

Funktionen mit dieser Eigenschaft hei"sen ("uberall) 
\MEntry{stetig}{stetig}, wenn dies f"ur jede Stelle des Definitionsbereichs 
gilt.
%\begin{MExample}
%Beispielsweise gilt dies f"ur $g(x) := 2 x - 1$. Wenn als Abweichung maximal 
%$r := 0,1$ akzeptiert wird, ist f"ur $x_0 := 1$ und alle $x$ mit 
%$1 - \frac{r}{2} < x < 1 + \frac{r}{2}$ dann tats"achlich
%$-0,1 < \Delta(g) < 0,1$.
%\end{MExample}

%Wenn wie hier im Beispiel f"ur beliebig kleine Werte $r > 0$ die Differenz 
%%$\Delta(f)$ immer dann zwischen $-r$ und $r$ liegt -- ihr Betrag also kleiner 
%als $r$ ist, wenn nur $x$ nahe genug bei $x_0$ liegt, dann wird die Funktion 
%$f$ an der Stelle $x_0$ stetig genannt. Und wenn dies f"ur alle Stellen $x_0$ 
%stimmt, hei"st $f$ ("uberall) \MEntry{stetig}{stetig}.

%Bild: Stetige Funktionen -- unstetige Funktionen
\begin{center}
\ifttm
\MGraphicsSolo{\MPfadBilder/jb07A1_DefStetigkeit.png}{scale=0.5}
\else
%Datei: {\MPfadBilder/jb07A1_DefStetigkeit.tex}
\begin{small}
\renewcommand{\jTikZScale}{0.6}
\tikzsetnextfilename{jb07A1_DefStetigkeit}
\begin{tikzpicture}[line width=1.5pt,scale=\jTikZScale]
\begin{scope}[xshift=-9cm]
%\input{\MPfadBilder/jb07A1_DefStetigkeitFolgeLinks.tex}
%LaTeX-File, Liedtke, 20111203.
%MINT-Modul Stetigkeit und Ableitung: Bild Folgenstetigkeit (Folge von links).
%Bildname: jb07A1_DefStetigkeitFolgeLinks.
%Quelle: 20121004, Liedtke.
%bearbeitet: 20140916.

%\tikzsetnextfilename{jb07A1_DefStetigkeitFolgeLinks}
%\begin{small}
%\begin{tikzpicture}[line width=1.5pt,scale=\jTikZScale] % [scale=0.8] 
%[line width=1.5pt, every node/.style={fill=white}] 
%\clip(-1.4,-0.7) rectangle (6.7, 5.7);
%\begin{scope}[xshift=-8cm]
%Koordinatenachsen:
\draw[->] (-0.3, 0) -- (6.6, 0) node[below left]{$x$}; %x-Achse
\draw[->] (0, -0.3) -- (0, 5.6) node[below left]{$y$}; %y-Achse
%Achsenbeschriftung:
%\foreach \x in {1, 2, 3, 4, 5, 6} \draw (\x, 0) -- ++(0, -0.1)%
\foreach \x in {1} \draw (\x, 0) -- ++(0, -0.1)%
 node[below] {$\x$}; 
%\foreach \y in {1, 2, 3, 4} \draw (0, \y) -- ++(-0.1, 0) node[left] {$\y$};
\foreach \y in {1} \draw (0, \y) -- ++(-0.1, 0); % node[left] {$\y$};
\node[below left] at (0, 0) {$0$};
%Funktionsgraph:
%\draw[domain=0:4,color=green!50!black] plot (\x + 1, {1/4*(\x^2) + 1}); 
%\draw[domain=0:5,color=green!50!black] plot (\x + 1/2, {-2*cos(\x/2 r) + 3}); 
\draw[domain=0.8:5.2,samples=200,color=\jccolorfkt] %
 plot (\x, {9/4*sin(3*(\x - 3)/4 r) + 2.7}); 
%Punkte $x$-Achse:
\draw[color=black] (3, 0) circle[radius=1pt];
\foreach \x in {-2, -1, -0.5, -0.25, -0.125} \draw[color=blue] %
 (3 + \x, 0) circle[radius=1pt,fill=blue];
%Punkte Funktionsgraph:
\draw[color=black] (3, 2.7) circle[radius=1pt];
\foreach \x in {-2, -1, -0.5, -0.25, -0.125} \draw[color=blue] %
% (3 + \x, {9/4*sin(3*(3 -\x - 3)/4 r) + 2.7}) circle[radius=1pt]; 
 (3 + \x, {9/4*sin(3*(\x)/4 r) + 2.7}) circle[radius=1pt]; 
%Punkt $y$-Achse:
\draw[color=black] (0, 2.7) circle[radius=1pt];
\foreach \x in {-2, -1, -0.5, -0.25, -0.125} \draw[color=blue] %
 (0, {9/4*sin(3*(\x)/4 r) + 2.7}) circle[radius=1pt,fill=blue];
%Beschriftung:
\node[below] at (2, -0.1) {$x_n$};
\node[below] at (2.5, -0.1) {$\rightarrow$};
\node[below] at (3, -0.1) {$x_0$};
%
\node[left] at (-0.1, {9/4*sin(3*(-1)/4 r) + 2.7}) {$f(x_n)$};
\node[left] at ({-0.1 - 0.25}, {9/4*sin(3*(-0.5)/4 r) + 2.7}) {$\uparrow$};
\node[left] at (-0.1, 2.7) {$f(x_0)$};
%\end{scope}
%\end{tikzpicture}
%\end{small}

%end of file
\end{scope}
\begin{scope}[xshift=0cm]
%\input{\MPfadBilder/jb07A1_DefStetigkeitFolgeRechts.tex}
%LaTeX-File, Liedtke, 20111203.
%MINT-Modul Stetigkeit und Ableitung: Bild Folgenstetigkeit (Folge von rechts).
%Bildname: jb07A1_DefStetigkeitFolgeRechts.
%Quelle 20121004: Liedtke.
%bearbeitet: 20140916, Liedtke.

%\tikzsetnextfilename{jb07A1_DefStetigkeitFolgeRechts}
%\begin{small}
%\begin{tikzpicture}[line width=1.5pt,scale=\jTikZScale] % [scale=0.8] 
%[line width=1.5pt, every node/.style={fill=white}] 
%\clip(-1.4,-0.7) rectangle (6.7, 5.7);
%Koordinatenachsen:
\draw[->] (-0.3, 0) -- (6.6, 0) node[below left]{$x$}; %x-Achse
\draw[->] (0, -0.3) -- (0, 5.6) node[below left]{$y$}; %y-Achse
%Achsenbeschriftung:
%\foreach \x in {1, 2, 3, 4, 5, 6} \draw (\x, 0) -- ++(0, -0.1)%
\foreach \x in {1} \draw (\x, 0) -- ++(0, -0.1)%
 node[below] {$\x$}; 
%\foreach \y in {1, 2, 3, 4} \draw (0, \y) -- ++(-0.1, 0) node[left] {$\y$};
\foreach \y in {1} \draw (0, \y) -- ++(-0.1, 0) node[below left] {$\y$};
\node[below left] at (0, 0) {$0$};
%Funktionsgraph:
%\draw[domain=0:4,color=green!50!black] plot (\x + 1, {1/4*(\x^2) + 1}); 
%\draw[domain=0:5,color=green!50!black] plot (\x + 1/2, {-2*cos(\x/2 r) + 3}); 
\draw[domain=0.8:5.2,samples=200,color=green!50!black] %
 plot (\x, {9/4*sin(3*(\x - 3)/4 r) + 2.7}); 
%Punkte $x$-Achse:
\draw[color=black] (3, 0) circle[radius=1pt];
\foreach \x in {2, 1, 0.5, 0.25, 0.125} \draw[color=blue] %
 (3 + \x, 0) circle[radius=1pt,fill=blue];
%Punkte Funktionsgraph:
\draw[color=black] (3, 2.7) circle[radius=1pt];
\foreach \x in {2, 1, 0.5, 0.25, 0.125} \draw[color=blue]%  
% (3 + \x, {9/4*sin(3*(3 \x - 3)/4 r) + 2.7}) circle[radius=1pt]; 
 (3 + \x, {9/4*sin(3*(\x)/4 r) + 2.7}) circle[radius=1pt]; 
%Punkt $y$-Achse:
\draw[color=black] (0, 2.7) circle[radius=1pt];
\foreach \x in {2, 1, 0.5, 0.25, 0.125} \draw[color=blue] 
 (0, {9/4*sin(3*(\x)/4 r) + 2.7}) circle[radius=1pt,fill=blue];
%Beschriftung:
\node[below] at ({4 + 0.2}, -0.1) {$x_n$};
\node[below] at (3.5, -0.1) {$\leftarrow$};
\node[below] at (3, -0.1) {$x_0$};
%
\node[left] at (-0.1, {9/4*sin(3*(1)/4 r) + 2.7}) {$f(x_n)$};
\node[left] at ({-0.1 - 0.25}, {9/4*sin(3*(0.5)/4 r) + 2.7}) {$\downarrow$};
\node[left] at (-0.1, 2.7) {$f(x_0)$};
%\end{tikzpicture}
%\end{small}

%end of file
\end{scope}
%
\begin{scope}[xshift=9cm]
%\input{\MPfadBilder/jb07A1_BspUnstetigeFktRechts.tex}
%LaTeX-File, Liedtke, 20111203.
%MINT-Modul Stetigkeit und Ableitung: Bild Folgenstetigkeit (Folge von rechts).
%Bildname: jb07A1_BspUnstetigeFktRechts.
%changed 20121004: TikZ-Skalierung via Variable eingef"uhrt.

%\tikzsetnextfilename{jb07A1_BspUnstetigeFktRechts}
%\begin{small}
%\begin{tikzpicture}[line width=1.5pt,scale=\jTikZScale] % [scale=0.8] 
%[line width=1.5pt, every node/.style={fill=white}] 
%\clip(-1.4,-0.7) rectangle (6.7, 5.7);
%Koordinatenachsen:
\draw[->] (-0.3, 0) -- (6.6, 0) node[below left]{$x$}; %x-Achse
\draw[-] (0, -0.3) -- (0, 2.8); %y-Achse
\draw[->] (0, 3.0) -- (0, 5.6) node[below left]{$y$}; %y-Achse
%Achsenbeschriftung:
%\foreach \x in {1, 2, 3, 4, 5, 6} \draw (\x, 0) -- ++(0, -0.1)%
\foreach \x in {1} \draw (\x, 0) -- ++(0, -0.1)%
 node[below] {$\x$}; 
%\foreach \y in {1, 2, 3, 4} \draw (0, \y) -- ++(-0.1, 0) node[left] {$\y$};
\foreach \y in {1} \draw (0, \y) -- ++(-0.1, 0) node[below left] {$\y$};
\node[below left] at (0, 0) {$0$};
%Funktionsgraph:
%\draw[domain=0:4,color=green!50!black] plot (\x + 1, {1/4*(\x^2) + 1}); 
%\draw[domain=0:5,color=green!50!black] plot (\x + 1/2, {-2*cos(\x/2 r) + 3}); 
\draw[domain=0.8:3.0,samples=200,color=green!50!black] %
 plot (\x, {9/4*sin(3*(\x - 3)/4 r) + 2.7}); 
\draw[domain=3.05:5.2,samples=200,color=green!50!black] %
 plot (\x, {9/4*sin(3*(\x - 3)/4 r) + 2.7 + 0.2}); 
%Punkte $x$-Achse:
\draw[color=black] (3, 0) circle[radius=1pt];
\foreach \x in {2, 1, 0.5, 0.25, 0.125} \draw[color=blue] %
 (3 + \x, 0) circle[radius=1pt,fill=blue];
%Punkte Funktionsgraph:
\draw[color=black] (3, 2.7) circle[radius=1pt];
\foreach \x in {2, 1, 0.5, 0.25, 0.125} \draw[color=blue]%  
% (3 + \x, {9/4*sin(3*(3 \x - 3)/4 r) + 2.7}) circle[radius=1pt]; 
 (3 + \x, {9/4*sin(3*(\x)/4 r) + 2.7 + 0.2}) circle[radius=1pt]; 
%Punkt $y$-Achse:
\draw[color=black] (0, 2.7) circle[radius=1pt];
\foreach \x in {2, 1, 0.5, 0.25, 0.125} \draw[color=blue] 
 (0, {9/4*sin(3*(\x)/4 r) + 2.7 + 0.2}) circle[radius=1pt,fill=blue];
%Beschriftung:
\node[below] at ({4 + 0.2}, -0.1) {$x_n$};
\node[below] at (3.5, -0.1) {$\leftarrow$};
\node[below] at (3, -0.1) {$x_0$};
%
\node[left] at (-0.1, {9/4*sin(3*(1)/4 r) + 2.7}) {$f(x_n)$};
\node[left] at ({-0.1 - 0.25}, {9/4*sin(3*(0.5)/4 r) + 2.7}) {$\downarrow$};
\draw[line width=0.5pt,color=red] (-0.82, 3.55) -- (-0.63, 3.65);
\draw[line width=0.8pt,color=red] (-0.3, 2.8) -- (0.3, 3.0);
\node[left] at (-0.1, 2.7) {$f(x_0)$};
%\end{tikzpicture}
%\end{small}

%end of file
\end{scope}
\end{tikzpicture}
\end{small}
\fi
\end{center}
%Bildende.
Das linke und das mittlere Bild zeigen eine stetige Funktion: Je n"aher 
die $x$-Werte bei $x_0$ liegen, desto mehr n"ahern sich die Funktionswerte 
$f(x)$ dem Funktionswert $f(x_0)$ an.
Im rechten Bild ist dies nicht der Fall: Die Funktion ist in $x_0$ nicht 
stetig, kurz unstetig in $x_0$.
\end{MXContent}


%\MSubsubsection{Relative "Anderung}
\begin{MXContent}{Relative "Anderung}{Relative "Anderung}{STD}
In der obigen Betrachtung der absoluten Differenz $\Delta(f)$ der Funktionswerte 
wird noch nicht ber"ucksichtigt, wie weit $x$ von $x_0$ entfernt ist.  
Der Quotient
\[
\frac{\Delta(f)}{\Delta(x)} = \frac{f(x) - f(x_0)}{x - x_0}.
\]
beschreibt, wie sich die Funktionswerte von $f$ im \emph{Vergleich} zum Abstand 
von $x$ zu $x_0$ "andern. 

Der Quotient wird auch \MEntry{relative "Anderung}{relative \"Anderung} genannt
%und wird hier mit \MEntry{diskreter "Anderungsrate}{diskrete \"Anderungsrate} 
und wird hier mit \MEntry{mittlerer "Anderungsrate}{mittlerer \"Anderungsrate} 
bezeichnet. 
Wenn der Grenzwert f"ur $x \to x_0$ exisitert, beschreibt dieser die lokale 
"Anderungsrate:
\[
\lim_{x \to x_0} \frac{\Delta(f)}{\Delta(x)} 
 = \lim_{x \to x_0} \frac{f(x) - f(x_0)}{x - x_0} %%
\] 
Dieser Grenzwert ist somit ein Ma"s daf"ur, wie sich die 
Funktionswerte in der N"ahe von $x_0$ "andern. Er wird Ableitung von $f$ in 
$x_0$ genannt. Wie die Formel mit der urspr"unglichen Idee zusammenh"angt, 
den Verlauf der Funktion n"aherungsweise durch eine Gerade zu beschreiben,
sehen wir uns nun an, nachdem alle wichtigen Begriffe vorgestellt wurden.
\end{MXContent}


%\MSubsubsection{Ableitung}
\begin{MXContent}{Ableitung}{Ableitung}{STD}

Gesucht ist eine Gerade $g(x) = m \cdot (x - x_0) + b$, die sich $f$ in $x_0$ 
stetig ann"ahert, und zwar
\begin{itemize}
\item sowohl mit Blick auf die Funktionswerte
\item als auch mit Blick auf die "Anderungsraten.
\end{itemize}

Somit soll $f(x) - g(x)$ in $x_0$ stetig gegen null streben, also
$0 = f(x_0) - g(x_0)$ und damit $b = g(x_0) = f(x_0)$ sein.

Zudem soll die Restfunktion
\[
r(x) := \frac{\Delta(f)}{\Delta(x)} - \frac{\Delta(g)}{\Delta(x)} %%
 = \frac{f(x) - f(x_0)}{x - x_0} - \frac{g(x) - g(x_0)}{x - x_0} %%
\]
stetig gegen null streben, wenn $x \to x_0$ strebt.

Mit $g(x) = f(x_0) + m (x - x_0)$ und damit $g(x_0) = f(x_0)$ ist
\begin{eqnarray*}
r(x) & = & \frac{\Delta(f)}{\Delta(x)} - \frac{\Delta(g)}{\Delta(x)} \\
 & = & \frac{f(x) - f(x_0)}{x - x_0} - \frac{g(x) - g(x_0)}{x - x_0} \\
 & = & \frac{f(x) - f(x_0) - m \cdot (x - x_0)}{x - x_0} \\
\end{eqnarray*}
Multiplikation mit $x - x_0$ f"uhrt auf
$(x - x_0) r(x) = f(x) - f(x_0) - m \cdot (x - x_0)$, 
woraus dann %%% f"ur $f(x)$ dann
\begin{eqnarray*}
f(x) & = & f(x_0) + m \cdot (x - x_0) + r(x) \cdot (x - x_0) %%
\end{eqnarray*}
folgt.

\begin{MXInfo}{Ableitung}
Wenn es eine in $x_0$ stetige Funktion $r$ mit $r(x_0) = 0$ und eine Gerade 
$g$ mit $g(x) = f(x_0) + m (x - x_0)$ derart gibt, dass die Funktion
%die Funktionswerte $f(x)$ f"ur alle $x$ aus dem Definitionsbereich von $f$ 
in der Form
\[
f(x) = g(x) + r(x) \cdot (x - x_0)
\]
%\[
%f(x) = f(x_0) + m \cdot (x - x_0) + r(x) \cdot (x - x_0)
%\]
%f"ur alle $x$ aus dem Definitionsbereich von $f$ gilt,
geschrieben werden kann,
dann hei"st $f$ in $x_0$ \MEntry{differenzierbar}{differenzierbar}. 
Der eindeutige Wert $m$ wird die \MEntry{Ableitung}{Ableitung} 
von $f$ in $x_0$ genannt.
\end{MXInfo}

%Bild:
\begin{center}
\ifttm
\MGraphicsSolo{\MPfadBilder/jb07A1_Ableitung.png}{scale=0.5}
\else
\begin{small}
\renewcommand{\jTikZScale}{1.0}
%Datei: {\MPfadBilder/jb07A1_Ableitung.tex}

%LaTeX-File, Liedtke, 20140825.
%VBKM-Modul 7 Differentialrechnung: Bild zur Definition der Ableitung
%Bildname: jb07A1_Ableitung.tex.
%Erstellt: 20140825, Liedtke.
%Bearbeitet: 20140829, Liedtke.
%Bearbeitet: 20140901, Liedtke (Dateiname ohne Endung erfasst).

\tikzsetnextfilename{jb07A1_Ableitung}
\begin{tikzpicture}[line width=1.5pt,scale=\jTikZScale, %
declare function={
  x0 = 2;
  x1 = 4;
  fkt(\x) = 1/4 * (\x - 1)*(\x - 1) + 0.75;
  Tangente(\x) = 1/2 * (\x - 2) + 1;
}
] %[every node/.style={fill=white}] 
%Koordinatenachsen:
\draw[->] (-0.6, 0) -- ({x1+1}, 0) node[below left]{$x$}; %x-Achse
\draw[->] (0, -0.6) -- (0, {fkt(x1)+1}) node[below left]{$y$}; %y-Achse
%Achsenbeschriftung:
\foreach \x in {1, 2, 3, 4} \draw (\x, 0) -- ++(0, -0.1); %
% node[below] {$\x$}; 
\foreach \y in {1, 2, 3} \draw (0, \y) -- ++(-0.1, 0); %
% node[left] {$\y$};
%\node[below left] at (0, 0) {$0$};
%Hilfslinien:
\draw[color=black!50!white,style=dashed] (x1, 0) -- (x1, {fkt(x1)});
\draw[color=black!50!white,style=dotted] (x0, {fkt(x0)}) -- (x1, {fkt(x0)});
%Funktion:
\draw[domain=0.5:x1,samples=120,color=\jccolorfkt] %
 plot (\x, {fkt(\x)});
%Tangente:
\draw[domain=0.5:x1,samples=120,color=blue!50!white] %
 plot (\x, {Tangente(\x)});
%Punkte und Hilfslinien:
\filldraw[color=black,fill=black] (x0, 0) circle (1pt);
\node[below] at (x0, -0.1) {$x_0$};
\filldraw[color=black,fill=black] (x0, {fkt(x0)}) circle (1pt);
%
\filldraw[color=black,fill=black] (x1, 0) circle (1pt);
\node[below] at (x1, -0.1) {$x$};
%
\filldraw[color=black,fill=black] (x1, {fkt(x0)}) circle (1pt);
\filldraw[color=black,fill=black] (x1, {Tangente(x1)}) circle (1pt);
\filldraw[color=black,fill=black] (x1, {fkt(x1)}) circle (1pt);
%Beschriftung:
\node[right] at ({x1 + 0.5}, 0.5) {$f(x_0)$};
\node[right] at ({x1 + 0.5}, 1.5) {$m\cdot(x-x_0)$};
\node[right] at ({x1 + 0.5}, 2.5) {$r(x)\cdot(x-x_0)$};
%
\draw[line width=0.8pt, rounded corners=3pt] %
 ({x1 + 0.2}, 0.0) -- ({x1 + 0.3}, 0.0) -- ({x1 + 0.3}, 0.5) %
 -- ({x1 + 0.4}, 0.5); %
\draw[line width=0.8pt, rounded corners=3pt] %
 ({x1 + 0.4}, 0.5)  %
 -- ({x1 + 0.3}, 0.5) -- ({x1 + 0.3}, 1.0) -- ({x1 + 0.2}, 1.0);
%
%
\draw[line width=0.8pt, rounded corners=3pt] %
 ({x1 + 0.2}, 1.0) -- ({x1 + 0.3}, 1.0) -- ({x1 + 0.3}, 1.5) %
 -- ({x1 + 0.4}, 1.5); %
\draw[line width=0.8pt, rounded corners=3pt] %
 ({x1 + 0.4}, 1.5)  %
 -- ({x1 + 0.3}, 1.5) -- ({x1 + 0.3}, 2.0) -- ({x1 + 0.2}, 2.0);
%
%
\draw[line width=0.8pt, rounded corners=4pt] %
 ({x1 + 0.2}, 2.0) -- ({x1 + 0.3}, 2.0) -- ({x1 + 0.3}, 2.5) %
 -- ({x1 + 0.4}, 2.5); %
\draw[line width=0.8pt, rounded corners=3pt] %
 ({x1 + 0.4}, 2.5) %
 -- ({x1 + 0.3}, 2.5) -- ({x1 + 0.3}, 3.0) -- ({x1 + 0.2}, 3.0);
%
%
\draw[line width=0.8pt, rounded corners=4pt] %
 ({x1 + 3.0}, 1.0) -- ({x1 + 3.2}, 1.0) -- ({x1 + 3.2}, 2.0) %
 -- ({x1 + 3.4}, 2.0) %
 -- ({x1 + 3.2}, 2.0) -- ({x1 + 3.2}, 3.0) -- ({x1 + 3.0}, 3.0);
%
\node[right] at ({x1+3.5}, 2) {$\Delta(f) = (m+r(x))\cdot(x-x_0)$};
%
\node[left] at (-0.5, {fkt(x0)}) {$f(x_0)$};
\node[left] at (-0.5, {Tangente(x1)}) {$g(x)$};
\node[left] at (-0.5, {fkt(x1)}) {$f(x)$};
\end{tikzpicture}
\end{small}

%end of file
\fi
\end{center}
%Bildende.

Der Wert $m$ ist tats"achlich eindeutig bestimmt, wie folgende "Uberlegung 
verdeutlicht: Da $r$ stetig ist, gilt
$r(x_0) = r\left(\lim_{x \to x_0} x\right) = \lim_{x \to x_0} r(x)$ und damit
\[
0 = r(x_0) = \lim_{x \to x_0} r(x) %
 = \lim_{x \to x_0} \frac{f(x) - f(x_0) - m \cdot (x - x_0)}{x - x_0} %
 =  \lim_{x \to x_0} \left[\frac{f(x) - f(x_0)}{x - x_0} - m\right]. %%
\]
Hieraus folgt, dass $m$ die eindeutige Zahl
$m = \displaystyle\lim_{x \to x_0} \frac{f(x) - f(x_0)}{x - x_0}$ ist.

\begin{MXInfo}{Schreibweisen der Ableitung}
Damit man sich nicht immer so kompliziert ausdr"ucken muss, wird eine kurze
pr"agnante Schreibweise eingef"uhrt:
Der eindeutig bestimmte Wert $m$ wird mit 
\[
\frac{\MD f}{\MD x}(x_0) := m \quad \text{oder noch k"urzer} \quad f'(x_0) := m %%
\]
bezeichnet.

Die erste Notation wird auch nach Leibniz benannt.
\end{MXInfo}

Wenn die Ableitung mittels des 
Differenzenquotienten $\frac{f(x) - f(x_0)}{x - x_0}$ berechnet werden muss,
bietet es sich oft an, den Differenzenquotienten anders aufzuschreiben. Indem
die Differenz von $x$ und $x_0$ mit $h := x - x_0$ bezeichnet wird, 

%Bild:
\begin{center}
\ifttm
\MGraphicsSolo{\MPfadBilder/jb07A1_AbstandReellerZahlen.png}{scale=0.5}
\else
%Datei: {\MPfadBilder/jb07A1_AbstandReellerZahlen.tex}
\begin{small}
\renewcommand{\jTikZScale}{0.7}
\tikzsetnextfilename{jb07A1_AbstandReellerZahlen}
\begin{tikzpicture}[line width=1.5pt,scale=\jTikZScale, %
declare function={
  x0 = 0;
  x1 = 4;
}
] %[every node/.style={fill=white}] 
%Koordinatenachsen:
\draw[->] (-0.6, 0) -- ({x1+1}, 0); % node[below left]{$x$}; %x-Achse
%Achsenbeschriftung:
\foreach \x in {0, 1, 2, 3, 4} \draw (\x, 0.0) -- ++(0, -0.1); %
\foreach \x in {0, 4} \draw (\x, 0.0) -- ++(0, +0.08); %
% node[below] {$\x$}; 
%Beschriftung:
%\draw[line width=0.8pt, rounded corners=4pt] %
% (0, 0.3) -- (0, 0.4) -- (2, 0.4) -- (2, 0.5);
\draw[line width=0.8pt, rounded corners=4pt] %
 ({x0}, 0.3) -- ({x0}, 0.4) -- ({x1/2}, 0.4) %
 -- ({x1/2}, 0.5);
\draw[line width=0.8pt, rounded corners=4pt] %
 ({x1/2}, 0.5)
 -- ({x1/2}, 0.4) -- ({x1}, 0.4) -- ({x1}, 0.3);
%
\node[above] at ({x1/2}, 0.6) {$h = x - x_0$};
%
\node[below] at ({x0}, -0.3) {$x_0$};
\node[below] at ({x1}, -0.3) {$x$};
\end{tikzpicture}
\end{small}
\fi
\end{center}
%Bildende.

ist wegen $x = x_0 + h$ dann
\[
\frac{f(x) - f(x_0)}{x - x_0} = \frac{f(x_0 + h) - f(x_0)}{h}
\]
Hier ist nun der Grenzwert f"ur $h \to 0$ zu betrachten, um die Ableitung zu
berechnen: 
\[
f'(x_0) = \lim_{x \to x_0} \frac{f(x) - f(x_0)}{x - x_0} %
 = \lim_{h \to 0} \frac{f(x_0 + h) - f(x_0)}{h}
\]
Viele der oft benutzten Funktionen sind differenzierbar, wie im Folgenden 
erl"autert wird. 

Die Betragsfunktion $f(x) := |x|$ ist ein einfaches Beispiel 
daf"ur, dass eine stetige Funktion nicht unbedingt differenzierbar zu sein 
braucht. 

\begin{MExample}
Die Betragsfunktion ist an der Stelle $x_0 = 0$ nicht differenzierbar.
Wenn n"amlich f"ur $f$ an der Stelle $x_0 = 0$ der Differenzenquotient
\[
\frac{f(0+h) - f(0)}{h} = \frac{|h|}{h}
\]
f"ur $h < 0$ betrachtet wird, ergibt sich $-1$, und f"ur $h > 0$ ist er 
gleich $1$, sodass der Grenzwert f"ur $h \to 0$ nicht existiert und somit die 
Betragsfunktion an der Stelle $x_0 = 0$ nicht differenzierbar ist.

Der Verlauf des Graphen "andert seine Richtung an der Stelle $(0, 0)$ sprunghaft: 
Salopp
ausgedr"uckt, weist der Funktionsgraph an der Stelle $(0, 0)$ einen Knick auf.
%Bild:
\begin{center}
\ifttm
\MUGraphicsSolo{\MPfadBilder/jb07A1_BspBetragsfunktion.png}{scale=0.5}{}
\else
%Datei: {\MPfadBilder/jb07A1_BspBetragsfunktion.tex}
\renewcommand{\jTikZScale}{1.0}
\tikzsetnextfilename{jb07A1_BspBetragsfunktion}
\begin{tikzpicture}[line width=1.5pt,scale=\jTikZScale]
%LaTeX-File, Liedtke, 20140916.
%VBKM-Modul 7 Differentialrechnung: Bild zur Betragsfunktion
%Bildname: jb07A2_BspBetragsfunktion
%Erstellt: 20140916, Liedtke.

%\begin{small}
%\begin{tikzpicture}[line width=1.5pt,scale=\jTikZScale, %
%declare function={
%  fkt(\x) = sin(\x r);
%}
%] %[every node/.style={fill=white}] 
%Koordinatenachsen:
\draw[->] (-3.6, 0) -- (4, 0) node[below left]{$x$}; %x-Achse
\draw[->] (0, -0.6) -- (0, 4.6) node[below left]{$y$}; %y-Achse
%Achsenbeschriftung:
\foreach \x in {-3, -2, -1, 1, 2, 3} \draw (\x, 0) -- ++(0, -0.1) %
 node[below] {$\x$};
\foreach \y in {1, 2, 3, 4} \draw (0, \y) -- ++(-0.1, 0) node[left] {$\y$};
%\node[below left] at (0, 0) {$0$};
%Funktion:
\draw[domain=-3.2:3.2,samples=120,color=\jccolorfkt] %
 plot (\x, {abs(\x)});
%Tangenten im Nullpunkt, wenn $f$ f"ur $x \leq 0$ bzw. $x \geq 0$ betrachtet 
%wird:
\draw[samples=120,color=blue!50!black] %
 (-0.5, 0.5) -- (0.5, -0.5);
\draw[samples=120,color=blue!50!black] %
 (-0.5, -0.5) -- (0.5, 0.5);
%Punkt: Markierung der Stelle $0$:
\filldraw[color=black,fill=black] (0, 0) circle (1pt);
%end of file
\end{tikzpicture}
\fi
\end{center}
%Bildende.
\end{MExample}

Auch wenn eine Funktion unstetig ist, besonders wenn sie eine Sprungstelle hat,
gibt es keine eindeutige Tangente an den Graphen und somit keine Ableitung.

%Bild:
\ifttm
\MUGraphics{\MPfadBilder/jb07A1_BspUnstetigeFkt.png}{scale=0.5}%
{Funktion mit Sprungstelle in $x_0 = 1$}{}
\else
\begin{center}
%Datei: {\MPfadBilder/jb07A1_BspUnstetigeFkt.tex}
\renewcommand{\jTikZScale}{1.0}
\tikzsetnextfilename{jb07A1_BspUnstetigeFkt}
\begin{tikzpicture}[line width=1.5pt,scale=\jTikZScale]
%LaTeX-File, Liedtke, 20140916.
%VBKM-Modul 7 Differentialrechnung: Bild einer unstetigen Funktion
%Bildname: jb07A2_BspUnstetigeFkt.
%Erstellt: 20140916, Liedtke.
%declare function={
%  fkt(\x) = (\x+1)*(\x+1)/4 + 0.5};
%}
%] %[every node/.style={fill=white}] 
%Koordinatenachsen:
\draw[->] (-3.6, 0) -- (4, 0) node[below left]{$x$}; %x-Achse
\draw[->] (0, -0.6) -- (0, 4.2) node[below left]{$y$}; %y-Achse
%Achsenbeschriftung:
\foreach \x in {-3, -2, -1, 1, 2, 3} \draw (\x, 0) -- ++(0, -0.1) %
 node[below] {$\x$};
\foreach \y in {2, 3} \draw (0, \y) -- ++(-0.1, 0) node[left] {$\y$};
%\node[below left] at (0, 0) {$0$};
%Funktion:
\draw[domain=-2.0:1.0,samples=120,color=\jccolorfkt] %
 plot (\x, {(\x+1)*(\x+1)/4 + 0.5});
\draw[domain=1.05:2.0,samples=120,color=\jccolorfkt] %
 plot (\x, {(\x+1)*(\x+1)/4 + 1.5});
%Tangenten in $x_0 = 1$, wenn die stetige Fortsetzung von $f$ f"ur $x \leq 1$ 
%bzw. $x \geq 1$ betrachtet wird:
\draw[samples=120,color=blue!50!black] %
 (0.5, 1.0) -- (1.5, 2.0);
\draw[samples=120,color=blue!50!black] %
 (0.5, 2.0) -- (1.5, 3.0);
%Punkt: Markierung der Stelle $x_0 = 1$:
\filldraw[color=black,fill=black] (1, 0) circle (1pt);
\end{tikzpicture}
\end{center}
\fi
%Bildende.

\end{MXContent}

%\end{MContent}

%end of file.

 

%%%Uebungen zum Abschnitt:
\begin{MExercises}

%\begin{MExercise}
%Sei $x_0 \in \R$. Begr"unden Sie, warum $g(x) := 3 x - 5$ in $x_0$ 
%differenzierbar ist.
%\end{MExercise}

\begin{MExercise}\MLabel{jm07A1Aufgabe:AblUmkehrFktGerade}
Die Gerade, die durch $g(x) := 4 x + 3$ beschrieben wird, hat die 
Steigung $4$ und damit $g$ die Ableitung $g'(x) = 4$.
Berechnen Sie die Umkehrabbildung von $g$, die wieder eine Gerade beschreibt,
und geben Sie deren Ableitung an. Was gilt f"ur das Produkt der beiden 
Ableitungen?

Antwort: Die Umkehrabbildung $g^{-1}$ hat die 
\begin{MExerciseItems}
\item Funktionsgleichung
 $(g^{-1})(x) = $\MSimplifyQuestion{20}{1/4 * x - 3/4}{10}{x}{4}{0} und die 
\item Ableitung
 $(g^{-1})'(x) =$\MParsedQuestion{8}{0.25}{4}.
\end{MExerciseItems}
F"ur das Produkt der Ableitungen gilt
 $g'(x) \cdot (g^{-1})'(x) =$\MParsedQuestion{8}{1}{4}.
\end{MExercise}


\begin{MExercise}
Berechnen Sie mittels Differenzenquotient die Ableitung von 
%$f(x) := |3 x|$ f"ur $x_1 := 5$ und f"ur $x_2 := -4$.
$f(x) := 4 - x^2$ f"ur $x_1 = -2$ und f"ur $x_2 = 1$.

Antwort: 
\begin{MExerciseItems}
\item Der Differenzenquotient von $f$ zu $x$ an der Stelle $x_1 = -2$ ist
\MSimplifyQuestion{30}{2 - x}{10}{x}{4}{0}
und hat f"ur $x \to -2$ den Grenzwert
 $f'(-2) = $\MParsedQuestion{8}{4}{3}.
\item Der Differenzenquotient von $f$ zu $x$ an der Stelle $x_2 = 1$ ist
\MSimplifyQuestion{30}{-1 - x}{10}{x}{4}{0}
und hat f"ur $x \to 1$ den Grenzwert
 $f'(1) = $\MParsedQuestion{8}{-2}{3}.
\end{MExerciseItems}
\end{MExercise}


\begin{MExercise}
Erl"autern Sie, warum
\begin{MExerciseItems}
\item $f(x) := \sqrt{x+3}$ in $x_0 = -3$,
\item $g(x) := 6 \cdot |2 x - 10|$ in $x_0 = 5$
\end{MExerciseItems}
nicht differenzierbar ist.

Antwort:
Die Ableitung von
%Die Tangente an den Funktionsgraphen der stetigen Funktion
\begin{MExerciseItems}
\item $f$ existiert an der Stelle $x_0 = -3$ nicht, 
da der Differenzenquotient 
\MSimplifyQuestion{30}{1/sqrt(h)}{1}{h}{20}{0}
f"ur $h \to 0$ nicht konvergiert.
% in $x_0 = -3$ verl"auft parallel zur $y$-Achse,
\item $g$ existiert an der Stelle $x_0 = 5$ nicht, da der 
Differenzenquotient f"ur $h < 0$ den Wert 
\MParsedQuestion{8}{-12}{3} hat, und f"ur $h > 0$ den 
Wert {\MParsedQuestion{8}{12}{3}}\!\!. Somit existiert der Grenzwert f"ur 
$h \to 0$ nicht.
%F"ur $x < 5$ haben alle Tangenten die Steigung $-12$, 
%und f"ur $x > 5$ die Steigung $21$.
\end{MExerciseItems}
\end{MExercise}

%\begin{MExercise}
%ei $x_0 \in \R$ mit $x_0 \neq 0$. Berechnen Sie mittels Differenzenquotienten 
%ie Ableitung von $f(x) := \frac{1}{x}$ an der Stelle $x_0$.
%end{MExercise}

\end{MExercises}



%%%Abschnitt
%\MSubsection{Ableitungen elementarer Funktionen}\MLabel{M07_Ableitung_elementareFunktionen}
\MSubsection{Standardableitungen}\MLabel{M07_Standardableitungen}

\begin{MIntro}
Die meisten der bisher vorgestellten Funktionen wie Polynome, trigonometrische
Funktionen und Exponentialfunktion sind differenzierbar. 
Hier wird ihre Ableitung vorgestellt.
\end{MIntro}


%%%Inhalt Teil A:
%LaTeX-2e-File, Liedtke, 20140731.
%Inhalt: Einf"uhrung in die Differentialrechnung, Abschnitt 2.
%zuletzt bearbeitet: 20140922.

%Abschnitt 2: Standardableitungen
%\MSubsection{Standardableitungen}
%\begin{MContent}

%\MSubsubsection{Ableitung von Polynomen}
\begin{MXContent}{Ableitung von Polynomen}{Ableitung von Polynomen}{STD}
 
Aus der Einf"uhrung der Ableitung ergibt sich f"ur eine Gerade
$f: \R \to \R, x \mapsto a_1 x + a_0$, wobei $a_1$ und $a_0$ gegebene Zahlen 
sind, dass die Ableitung von $f$ an der Stelle $x_0$ gleich $f'(x_0) = a_1$ ist.
Wenn beispielsweise $a_1 = 2$ und $a_0 = 1$, dann ist $f(x) = 2 x + 1$, und die
Ableitung ist $f'(x) = 2$.
Denn mit $f(x_0) = 2 x_0 + 1$ ist
\[
f(x) = 2 x + 1 = 2 x_0 - 2 x_0 + 2 x + 1 %
 = f(x_0) + 2 (x - x_0) + 0 \cdot (x - x_0).
\]
Wird die Restfunktion $r(x) := 0$ eingef"uhrt, sind die Bedingungen zur 
Differenzierbarkeit erf"ullt, und es ist $f'(x_0) = 2$ die Ableitung von $f$ 
in $x_0$.

F"ur Monome $x^n$ mit $n > 1$ ist es am einfachsten, die Ableitung "uber den 
Differenzenquotienten zu bestimmen. Damit ergibt sich folgende Aussage:

\begin{MXInfo}{Ableitung von $x^n$}
Gegeben sind eine nat"urliche Zahl $n$ und eine reelle Zahlen $r$.

Die konstante Funktion $f(x) := r = r \cdot x^0$ hat die Ableitung
$f'(x) = 0$.

Die Funktion $f(x) := r \cdot x^n$ hat die Ableitung 
\[
f'(x) = r \cdot n \cdot x^{n-1} %%
\]
\end{MXInfo}

\begin{MExample}
Es wird die Funktion $f: \R \to \R$ mit $f(x) = 5 x^3$ betrachtet.
Mit obigen Bezeichnungen ist $r = 5$ und $n = 3$. Damit erh"alt man die 
Ableitung 
\[
f(x) = 5 \cdot 3 x^{3 - 1} = 15 x^2. %%
\]
\end{MExample}


F"ur Wurzelfunktionen ergibt sich eine entsprechende Aussage. Allerdings 
ist zu beachten, dass Wurzelfunktionen nur f"ur $x > 0$ differenzierbar sind.
Denn die Tangente an den Funktionsgraphen durch den Punkt $(0;0)$ verl"auft 
parallel zur $y$-Achse und beschreibt somit keine Funktion. 

\begin{MXInfo}{Ableitung von $x^{\frac{1}{n}}$}
%F"ur $n \in \Z$ und $n \notin \{0, 1, -1 \}$ ist die Wurzelfunktion
F"ur $n \in \Z$ mit $n \neq 0$ ist die Funktion
$f(x) := x^{\frac{1}{n}}$ f"ur $x \geq 0$ differenzierbar, und es gilt
\[
f'(x) = \frac{1}{n} \cdot x^{\frac{1}{n}-1} \text{ f"ur } x > 0 %%
\]%
\end{MXInfo}
F"ur $n \geq 2$ haben wir es mit Wurzelfunktionen zu tun.
Und nat"urlich sind die Sonderf"alle, die identische Abbildung $g(x) = x$ 
und $h(x) = x^{-1}$, auf ihrem gesamten Definitionsbereich differenzierbar.
%differenzierbar, aber eben keine Wurzelfunktionen, weshalb sie oben nicht 
%aufgef"uhrt sind.

%Beispielsweise ist die Ableitung von $f(x) := \sqrt{x} = x^{\frac{1}{2}}$
%f"ur $x > 0$ durch
%$f'(x) = \frac{1}{2} x^{\frac{1}{2}-1} %
%= \frac{1}{2} x^{-\frac{1}{2}} = \frac{1}{2 \cdot \sqrt{x}}$ 
%gegeben.

\begin{MExample}
Die Wurzelfunktion $f: [0, \infty) \to \R$ mit
 $f(x) := \sqrt{x} = x^{\frac{1}{2}}$
ist f"ur $x > 0$ differenzierbar. Die Ableitung ist durch
\[
f'(x) = \frac{1}{2} \cdot x^{\frac{1}{2}-1} %
= \frac{1}{2} \cdot x^{-\frac{1}{2}} = \frac{1}{2 \cdot \sqrt{x}} %%
\] 
gegeben.

%Bild:
\ifttm
\MUGraphics{\MPfadBilder/jb07A2_Wurzelfunktion.png}{scale=0.5}%
{Graph von $\sqrt{x}$ mit Tangente in $x_0 = 1$}{}
\else
\begin{center}
%Datei: {\MPfadBilder/jb07A1_Wurzelfunktion.tex}
%\input{\MPfadBilder/jb07A1_Wurzelfunktion.tex}
%LaTeX-File, Liedtke, 20140916.
%VBKM-Modul 7 Differentialrechnung: Bild zur Exponentialfunktion
%Bildname: jb07A2_Wurzelfunktion
%Erstellt: 20140922, Liedtke.
\begin{small}
\renewcommand{\jTikZScale}{0.8}
\tikzsetnextfilename{jb07A2_Wurzelfunktion}
\begin{tikzpicture}[line width=1.5pt,scale=\jTikZScale]
%\begin{small}
%\begin{tikzpicture}[line width=1.5pt,scale=\jTikZScale, %
%declare function={
%  fkt(\x) = sqrt(\x);
%}
%] %[every node/.style={fill=white}] 
%Koordinatenachsen:
\draw[->] (-0.6, 0) -- (4.8, 0) node[below left]{$x$}; %x-Achse
\draw[->] (0, -0.6) -- (0, 3) node[below left]{$y$}; %y-Achse
%Achsenbeschriftung:
\foreach \x in {1, 2, 3, 4} \draw (\x, 0) -- ++(0, -0.1) %
 node[below] {$\x$};
\foreach \y in {1, 2} \draw (0, \y) -- ++(-0.1, 0) node[left] {$\y$};
%\node[below left] at (0, 0) {$0$};
%Funktion:
\draw[domain=0:4,samples=120,color=\jccolorfkt] %
 plot (\x, {sqrt(\x)});
%Tangenten y = f(x_0) + f'(x_0) * (x - x_0) im Punkt $(x_0, f(x_0)$ des Graphen:
%$x_0 := 1$:
\draw[samples=120,color=blue!50!black] %
 (0.5, {1 - 1/2 * 1/2}) -- ++(1, {1/2});
%Punkt:
\filldraw[color=black, fill=black] (1, 0) circle (2pt);
%end of file
\end{tikzpicture}
\end{small}
\end{center}
\fi
%Bildende.
Die Tangente in $x_0 = 1$ an den Graphen der Wurzelfunktion $f(x) = \sqrt{x}$ hat 
die Steigung $\frac{1}{2 \sqrt{1}} = \frac{1}{2}$.
\end{MExample}

Eine entsprechende Aussage gilt auch f"ur allgemeine Exponenten $p \in \R$ mit 
$p \neq 0$ f"ur $x > 0$:
Die Ableitung von $f(x) = x^p$ f"ur $x > 0$ ist
\[
f'(x) = p \cdot x^{p-1} %%
\]
\end{MXContent}


%\MSubsubsection{Ableitung trigonometrischer Funktionen}
\begin{MXContent}{Ableitung trigonometrischer Funktionen}{Ableitung trigonometrischer Funktionen}{STD}

Die Sinusfunktion ist periodisch mit Periode $2 \pi$. Somit gen"ugt es, die
Funktion auf einem Intervall der L"ange $2 \pi$ zu betrachten. Einen Ausschnitt 
des Graphen f"ur $-\pi \leq x \leq \pi$ zeigt die folgende Abbildung:

%Bild:
\ifttm
\begin{center}
\MUGraphicsSolo{\MPfadBilder/jb07A2_SinusFktUndAbl.png}{scale=0.8}{}
%\MUGraphicsSolo{\MPfadBilder/jb07A2_SinusFktGraph.png}{scale=0.8}{}
%\MUGraphicsSolo{\MPfadBilder/jb07A2_SinusAbl.png}{scale=0.8}{}
%\MUGraphicsSolo{jb07A2_SinusFktGraph.png}{scale=1}{}
%\MUGraphicsSolo{jb07A2_SinusAbl.png}{scale=1}{}
\end{center}
\else
\begin{center}
%Bild: {\MPfadBilder/jb07A2_SinusFktGraph.tex}%
%LaTeX-File, Liedtke, 20140826.
%VBKM-Modul 7 Differentialrechnung: Bild zur Sinusfunktion.
%Bildname: jb07A2_SinusFktGraph.tex.
%Erstellt: 20140827, Liedtke.
%Bearbeitet: 20140829, Liedtke (Dateiname angepasst).
%Bearbeitet: 20140901, Liedtke (Dateiname ohne Endung erfasst).
\begin{small}
\renewcommand{\jTikZScale}{1.0}
\tikzsetnextfilename{jb07A2_SinusFktUndAbl}
%\tikzsetnextfilename{jb07A2_SinusFktGraph}
\begin{tikzpicture}[line width=1.5pt,scale=\jTikZScale, %
declare function={
  fkt(\x) = sin(\x r);
  fktabl(\x) = cos(\x r);
}
] %[every node/.style={fill=white}] 
%
%Graph der Sinusfunktion:
\node[right] at (-6,1) {Sinusfunktion};
\begin{scope}%[xshift=-6]
%Koordinatenachsen:
\draw[->] (-3.6, 0) -- (4, 0) node[below left]{$x$}; %x-Achse
\draw[->] (0, -1.6) -- (0, 1.6) node[below left]{$y$}; %y-Achse
%Achsenbeschriftung:
\foreach \x in {{-pi}, {-pi/2}} \draw (\x, 0) -- ++(0, -0.1);
\node[below] at ({-pi}, 0) {$-\pi$};
\node[below] at ({-pi/2}, 0) {$-\frac{\pi}{2}$};
\foreach \x in {{pi/2}, pi} \draw (\x, 0) -- ++(0, -0.1);
\node[below] at ({pi/2}, 0) {$-\frac{\pi}{2}$};
\node[below] at ({pi}, 0) {$-\pi$};
\foreach \y in {-1} \draw (0, \y) -- ++(-0.1, 0) %
 node[left] {$\y$};
\foreach \y in {1} \draw (0, \y) -- ++(-0.1, 0) %
 node[left] {$\y$};
%\node[below left] at (0, 0) {$0$};
%Funktion:
\draw[domain=-3.14:3.14,samples=120,color=\jccolorfkt] %
 plot (\x, {fkt(\x)});
%Tangenten in verschiedenen Punkten:
\draw[samples=120,color=blue!50!black] %
 plot (-0.5,-0.5) -- (0.5,0.5);
\draw[samples=120,color=blue!50!black] %
 plot ({pi/3},1) -- ({2*pi/3},1);
\node[above] at ({pi/2}, 1) {$\sin'(\pi/2) = 0$};
\draw[samples=120,color=blue!50!black] %
 plot ({-2*pi/3},-1) -- ({-pi/3},-1);
\node[below] at ({-pi/2}, -1) {$\sin'(-\pi/2) = 0$};
\end{scope}
%end of file
%
%
%Graph der Ableitung der Sinusfunktion:
%Bild: {\MPfadBilder/jb07A2_SinusAbl.tex}
%LaTeX-File, Liedtke, 20140826.
%VBKM-Modul 7 Differentialrechnung: Bild zur Ableitung der Sinusfunktion.
%Bildname: jb07A2_SinusAbl.tex.
%Erstellt: 20140827, Liedtke.
%Bearbeitet: 20140829, Liedtke (Dateiname angepasst).
%Bearbeitet: 20140901, Liedtke (Dateiname ohne Endung erfasst).
%
%\tikzsetnextfilename{jb07A2_SinusAbl}
%\begin{tikzpicture}[line width=1.5pt,scale=\jTikZScale, %
%declare function={
%  fktabl(\x) = cos(\x r);
%}
%] %[every node/.style={fill=white}] 
\node[right] at (-6,-2.8) {Ableitung};
\begin{scope}[yshift=-3.8cm]
%Koordinatenachsen:
\draw[->] (-3.6, 0) -- (4, 0) node[below left]{$x$}; %x-Achse
\draw[->] (0, -1.6) -- (0, 1.6) node[below left]{$y$}; %y-Achse
%Achsenbeschriftung:
\foreach \x in {{-pi}, {-pi/2}} \draw (\x, 0) -- ++(0, 0.1); 
\node[below] at ({-pi}, 0) {$-\pi$}; 
\node[below] at ({-pi/2}, 0) {$-\frac{\pi}{2}$}; 
\foreach \x in {{pi/2}, {pi}} \draw (\x, 0) -- ++(0, -0.1);
\node[below] at ({pi/2}, 0) {$\frac{\pi}{2}$}; 
\node[below] at ({pi}, 0) {$\pi$}; 
\foreach \y in {-1} \draw (0, \y) -- ++(0.1, 0);
\foreach \y in {1} \draw (0, \y) -- ++(-0.1, 0) %
 node[below left] {$\y$};
%\node[below left] at (0, 0) {$0$};
%Funktion:
\draw[domain=-3.14:3.14,samples=120,color=blue!50!white] %
 plot (\x, {fktabl(\x)});
%Punkte
\filldraw[color=black,fill=black] ({-pi}, -1) circle (2pt);
\filldraw[color=black,fill=black] ({-pi/2}, 0) circle (2pt);
\filldraw[color=black,fill=black] (0, 1) circle (2pt);
\filldraw[color=black,fill=black] ({pi/2}, 0) circle (2pt);
\filldraw[color=black,fill=black] ({pi}, -1) circle (2pt);
\end{scope}
\end{tikzpicture}
\end{small}
\end{center}
%end of file
\fi
%Bildende.

Der Sinus ist als Quotient der Gegenkathede durch die Hypotenuse definiert.
F"ur kleine Winkel ist die Gegenkathede n"aherungsweise gleich lang wie der 
Winkel im Bogenma"s, wie aus der Definition am Einheitskreis zu 
sehen ist. Damit ist der Differenzenquotient ungef"ahr $1$. 
Anschaulich verst"andlich ist, dass die Steigung der 
Tangente im Nullpunkt somit $1$ ist, wie dies in der zweiten Abbildung 
eingetragen ist. An den Stellen $\frac{\pi}{2}$ und $-\frac{\pi}{2}$ 
ist die Tangentensteigung gleich $0$. Eine genaue Betrachtung best"atigt 
die Werte und zeigt, dass sich insgesamt als Ableitungsfunktion die 
Kosinusfunktion ergibt.

\begin{MXInfo}{Ableitung trigonometrischer Funktionen}
F"ur die Sinusfunktion $f(x) := \sin(x)$ gilt 
\[
f'(x) = \cos(x) %%
\]
F"ur die Kosinusfunktion $g(x) := \cos(x)$ gilt 
\[
g'(x) = -\sin(x) %%
\]
F"ur die Tangensfunktion $h(x) := \tan(x)$ f"ur $x \neq \frac{\pi}{2} + k \pi$ 
mit $k \in \Z$ gilt
\[
h'(x) = 1 + (\tan(x))^2 = \frac{1}{\cos^2(x)} %%
\]
\end{MXInfo}
Letzteres ergibt sich auch aus den nachfolgend erl"auterten Rechenregeln und der 
Definition des Tangens als Quotient von Sinus und Kosinus.
\end{MXContent}


%\MSubsubsection{Ableitung der Exponentialfunktion}
\begin{MXContent}{Ableitung der Exponentialfunktion}{Ableitung der Exponentialfunktion}{STD}

Die Exponentialfunktion $f(x) := \exp(x)$ hat die besondere Eigenschaft, 
dass ihre Ableitung wiederum $f'(x) = \exp(x)$ ist.

F"ur die Stelle $0$ ist dies anschaulich nachvollziehbar, wenn der 
Graph m"oglichst genau gezeichnet wird, dass die Steigung der Tangente
gleich $f'(0) = 1$ und somit gleich $\MEU^0$ ist.

Wenn der Diffenzenquotient mit der Rechenregel 
$\MEU^{x_0+h} = \MEU^{x_0} \cdot \MEU^{h}$ f"ur die Exponentialfunktion 
umgeformt wird, ergibt sich daraus obiges Ergebnis 
$f'(x) = \MEU^x \cdot f'(0) = \MEU^x$.

%Bild:
\ifttm
\begin{center}
\MUGraphicsSolo{\MPfadBilder/jb07A2_Exponentialfunktion.png}{scale=0.5}{}
\end{center}
\else
\begin{center}
%Datei: {\MPfadBilder/jb07A1_BspExponentialfunktion.tex}
%\input{\MPfadBilder/jb07A1_BspExponentialfunktion.tex}
%LaTeX-File, Liedtke, 20140916.
%VBKM-Modul 7 Differentialrechnung: Bild zur Exponentialfunktion
%Bildname: jb07A2_BspExponentialfunktion
%Erstellt: 20140916, Liedtke.
\begin{small}
\renewcommand{\jTikZScale}{0.8}
\tikzsetnextfilename{jb07A2_Exponentialfunktion}
\begin{tikzpicture}[line width=1.5pt,scale=\jTikZScale]
%\begin{small}
%\begin{tikzpicture}[line width=1.5pt,scale=\jTikZScale, %
%declare function={
%  fkt(\x) = sin(\x r);
%}
%] %[every node/.style={fill=white}] 
%Koordinatenachsen:
\draw[->] (-3.6, 0) -- (4, 0) node[below left]{$x$}; %x-Achse
\draw[->] (0, -0.6) -- (0, 4) node[below left]{$y$}; %y-Achse
%Achsenbeschriftung:
\foreach \x in {-3, -2, -1, 1, 2, 3} \draw (\x, 0) -- ++(0, -0.1) %
 node[below] {$\x$};
\foreach \y in {1, 2, 3} \draw (0, \y) -- ++(-0.1, 0) node[left] {$\y$};
%\node[below left] at (0, 0) {$0$};
%Funktion:
\draw[domain=-2.4:1.3,samples=120,color=\jccolorfkt] %
 plot (\x, {exp(\x)});
%Tangenten in verschiedenen Punkten:
\draw[samples=120,color=blue!50!black] %
 (-0.45, 0.45) -- (0.55, 1.55);
%end of file
\end{tikzpicture}
\end{small}
\end{center}
\fi
%Bildende.
\end{MXContent}


%\MSubsubsection{Ableitung der Logarithmusfunktion}
\begin{MXContent}{Ableitung der Logarithmusfunktion}{Ableitung der Logarithmusfunktion}{STD}

Die Logarithmusfunktion $f(y) := \ln(y)$ f"ur $y > 0$ ist die Umkehrfunktion
der Exponentialfunktion $y = \MEU^x$.
Die Ableitung und damit die Steigung $m$ der Tangente an den 
Graphen der Exponentialfunktion ist $m = \MEU^x$. 

Der Graph der Umkehrfunktion $\ln(y)$ ergibt sich aus der Spiegelung an der 
ersten Winkelhalbierenden. Die Steigung der gespiegelten Tangente an die 
Exponentialfunktion ist dann der Kehrwert $f'(y) = \frac{1}{\MEU^x}$ an der 
Stelle $y = \MEU^x$, also $f'(y) = \frac{1}{y}$ (vergleiche auch Aufgabe
\MNRef{jm07A1Aufgabe:AblUmkehrFktGerade}).

Indem wieder die gewohnte Bezeichnung f"ur die unabh"angige Variable verwendet
wird, ergibt sich $f'(x) = \frac{1}{x}$ als Ableitung der 
Logarithmusfunktion $f(x) = \ln(x)$.

%Bild:
\ifttm
\MUGraphics{\MPfadBilder/jb07A2_Logarithmusfunktion.png}{scale=0.5}%
{Tangenten an $\exp$ und an die Umkehrfunktion $\ln$}{}
\else
\begin{center}
%Datei: {\MPfadBilder/jb07A1_BspExponentialfunktion.tex}
%\input{\MPfadBilder/jb07A1_BspExponentialfunktion.tex}
%LaTeX-File, Liedtke, 20140916.
%VBKM-Modul 7 Differentialrechnung: Bild zur Exponentialfunktion
%Bildname: jb07A2_BspExponentialfunktion
%Erstellt: 20140916, Liedtke.
\begin{small}
\renewcommand{\jTikZScale}{0.8}
\tikzsetnextfilename{jb07A2_Logarithmusfunktion}
\begin{tikzpicture}[line width=1.5pt,scale=\jTikZScale]
%\begin{small}
%\begin{tikzpicture}[line width=1.5pt,scale=\jTikZScale, %
%declare function={
%  fkt(\x) = sin(\x r);
%}
%] %[every node/.style={fill=white}] 
%Koordinatenachsen:
\draw[->] (-3.2, 0) -- (4.6, 0) node[below left]{$x$}; %x-Achse
\draw[->] (0, -3.2) -- (0, 4) node[below left]{$y$}; %y-Achse
%Achsenbeschriftung:
\foreach \x in {-3, -2, -1} \draw (\x, 0) -- ++(0, -0.1); % node[below] {$\x$};
\foreach \x in {1, 2, 3} \draw (\x, 0) -- ++(0, -0.1) node[below] {$\x$};
\foreach \y in {-3, -2, -1} \draw (0, \y) -- ++(-0.1, 0); % node[left] {$\y$};
\foreach \y in {1, 2, 3} \draw (0, \y) -- ++(-0.1, 0) node[left] {$\y$};
%\node[below left] at (0, 0) {$0$};
%
%Funktion exp:
\draw[domain=-2.4:1.3,samples=120,color=white!50!black] %\jccolorfkt] %
 plot (\x, {exp(\x)});
%Tangenten in verschiedenen Punkten:
%x_0 = 0:
%\draw[samples=120,color=blue!50!black] %
% (-0.45, 0.45) -- (0.55, 1.55);
%
%x_0 = 1:
% ({3/4}, {exp(1) - 1/4 * exp(1)}) -- ++({2/4}, {2/4*exp(1)});
%
%x_0 = ln(2): y = exp(ln(2)) + 2 * (x - ln(2)) = 2 + 2 * (x - ln(2)):
\draw[style=dashed,samples=120,color=blue!50!black] %
 ({ln(2) - 3/8}, {5/4}) -- ++({6/8}, {6/8*2});
%
%Punkte:
%\filldraw[color=black] (0, 1) circle (2pt);
\filldraw[color=black] (0, 2) circle (2pt);
\filldraw[color=black] ({ln(2)}, 2) circle (2pt);
%
%Spiegelachse (erste Winkelhalbierende):
\draw[samples=120,style=dotted,color=white!50!black] %
 (-2.1, -2.1) -- (3.1, 3.1);
%
%Umkehrfunktion ln:
\draw[domain=0.1:{exp(1.3},samples=120,color=\jccolorfkt] %
 plot (\x, {ln(\x)});
%Tangenten in verschiedenen Punkten:
%\draw[samples=120,color=blue!50!black] %
% (0.45, -0.45) -- (1.55, 0.55);
%
%\draw[style=dotted,samples=120,color=blue!50!black] %
% ({exp(1) - 1/2}, {1 - 1/2 * 1/exp(1)}) -- ++(1, {1/exp(1)});
%\draw[style=dotted,samples=120,color=blue!50!black] %
% ({exp(1) - 1/4*exp(1)}, {3/4}) -- ++({2/4*exp(1)}, {2/4});
%x_0 = 2: y = ln(2) + 1/2 * (x - 2):
\draw[samples=120,color=blue!50!black] %
 ({2 - 3/4}, {ln(2) - 1/2 * 3/4}) -- ++({2*3/4}, {2*3/4*1/2});
%Punkte:
%\filldraw[color=black] (1, 0) circle (2pt);
\filldraw[color=black] (2, 0) circle (2pt);
\filldraw[color=black] (2, {ln(2)}) circle (2pt);
\end{tikzpicture}
\end{small}
\end{center}
\fi
%Bildende.
In der obigen Abbildung ist die Tangente an den Graphen von $\exp$ im
Punkt $(x_0, \MEU^{x_0})$ f"ur $x_0 = \ln(2)$ angedeutet, sodass die 
Tangentensteigung $m = 2$ ist. Die Umkehrabbildung der Tangente hat dann 
die Steigung $\frac{1}{m} = \frac{1}{2}$. 
Geometrisch ist es die Steigung der gespiegelten Tangente, also die Tangente
an die Umkehrfunktion $\ln$.
%Dann hat die Tangente an den gespiegelten 
%Graphen, also den Graphen von $\ln$, die Steigung $\frac{1}{m} = \frac{1}{2}$.

\end{MXContent}

%\end{MContent}

%end of file.

 


%%%Uebungen zum Abschnitt:
\begin{MExercises}

\begin{MExercise}
Bestimmen Sie die Ableitung, indem Sie die Funktionsterme vereinfachen und 
dann Ihre Kenntnisse "uber die Ableitung bekannter Funktionen anwenden
($x > 0$):
\begin{MExerciseItems}
\item $f(x) := x^6 \cdot x^{\frac{7}{2}} = $
\MSimplifyQuestion{40}{x^(19/2)}{10}{x}{4}{0}
\item $g(x) := \frac{x^{-\frac{3}{2}}}{\sqrt{x}} = $
\MSimplifyQuestion{40}{1/x^2}{1}{x}{20}{0}
\end{MExerciseItems}
Damit ist
\begin{MExerciseItems}
\item $f'(x) = $ \MSimplifyQuestion{40}{19/2*x^(17/2)}{10}{x}{4}{0}
\item $g'(x) = $ \MSimplifyQuestion{40}{-2/x^3}{1}{x}{20}{0}
\end{MExerciseItems}
\end{MExercise}

\begin{MExercise}
Vereinfachen Sie die Funktionsterme, um dann die Ableitung zu bestimmen:
\begin{MExerciseItems}
\item
 $f(x) := 2 \sin\left(\frac{x}{2}\right) \cdot \cos\left(\frac{x}{2}\right) = $ %
\MSimplifyQuestion{40}{sin(x)}{10}{x}{4}{0}
\item $g(x) := \cos^2(3 x) + \sin^2(3 x) = $ %
\MParsedQuestion{40}{1}{3}
\end{MExerciseItems}
Damit ist
\begin{MExerciseItems}
\item $f'(x) = $ \MSimplifyQuestion{40}{cos(x)}{10}{x}{4}{0}
\item $g'(x) = $ \MParsedQuestion{40}{0}{3} 
\end{MExerciseItems}
\end{MExercise}

\begin{MExercise}
Vereinfachen Sie die Funktionsterme, um dann die Ableitung zu bestimmen
(f"ur $x > 0$ in der ersten Teilaufgabe):
\begin{MExerciseItems}
\item $f(x) := 2 \ln(x) + \ln\left(\frac{1}{x}\right) = $
\MSimplifyQuestion{40}{ln(x)}{10}{x}{4}{0}
\item $g(x) := \left(\MEU^x\right)^2 \cdot \MEU^{-x} = $
\MSimplifyQuestion{40}{exp(x)}{10}{x}{4}{0}
\end{MExerciseItems}
Damit ist
\begin{MExerciseItems}
\item $f'(x) = $ \MSimplifyQuestion{40}{1/x}{1}{x}{20}{0}
\item $g'(x) = $ \MSimplifyQuestion{40}{exp(x)}{10}{x}{4}{0}
\end{MExerciseItems}
\end{MExercise}

\end{MExercises}



%%%Abschnitt
\MSubsection{Rechenregeln}\MLabel{M07_Rechenregeln}

\begin{MIntro}
Zusammen mit einigen wenigen Rechenregeln und den im letzten Abschnitt 
vorgestellten Ableitungen lassen sich eine Vielzahl an Funktionen 
differenzieren.
\end{MIntro}

%%%Inhalt Teil A:
%LaTeX-2e-File, Liedtke, 20140731.
%Inhalt: Einf"uhrung in die Differentialrechnung, Abschnitt 3.
%zuletzt bearbeitet: 20140922.

%\MSubsubsection{Rechenregeln}
\begin{MContent}

Gegeben sind differenzierbare Funktionen $u, v: D \to \R$ und eine reelle
Zahl $r$.


\MSubsubsection{Vielfache und Summen von Funktionen}
 
%\begin{MXInfo}{Ableitung von Summen und von Vielfachen von Funktionen}
Dann ist auch die Summe der Funktionen $f(x) := u(x) + v(x)$ differenzierbar,
und es gilt
\begin{equation}
f'(x) = u'(x) + v'(x) %%
\end{equation}

Es ist auch das $r$-fache der Funktion $f(x) := r \cdot u(x)$ differenzierbar,
und es gilt
\begin{equation}
f'(x) = r \cdot u'(x) %%
\end{equation}
%\end{MXInfo}

\begin{MExample}
Die Ableitung von $f(x) = x^3 + \ln(x)$ f"ur $x > 0$ ist
\[
f'(x) = 3 x^2 + \frac{1}{x} = \frac{3 x^3 + 1}{x} %%
\]
Mit $\ln(x^3) = 3 \ln(x)$ ergibt sich die Ableitung von 
$g(x) = \ln(x^3) = 3 \ln(x)$ f"ur $x > 0$ zu
\[
g'(x) = \frac{3}{x} %%
\]
Die Ableitung von 
$h(x) = 4^{-1} \cdot x^2 - \sqrt{x} %
 = \frac{1}{4} x^2 + (-1) \cdot x^{\frac{1}{2}}$ ist f"ur $x > 0$ dann
\[
h'(x) = \frac{1}{2} x - \frac{1}{2} x^{-\frac{1}{2}} %
 = \frac{x^{\frac{3}{2}} - 1}{2 \sqrt{x}} %%
\]
\end{MExample}


\MSubsubsection{Produkt und Quotient von Funktionen}

Dann ist auch das Produkt der Funktionen $f(x) := u(x) \cdot v(x)$ 
differenzierbar, und es gilt
\begin{equation}
f'(x) = u'(x) \cdot v(x) +  u(x) \cdot v'(x) %%
\end{equation}

Es ist auch der Quotient der Funktionen $f(x) := \frac{u(x)}{v(x)}$ f"ur 
alle $x$ mit $v(x) \neq 0$ definiert und differenzierbar, und es gilt
\ifttm
\begin{equation}
f'(x) = \frac{u'(x) \cdot v(x) - u(x) \cdot v'(x)}{\left(v'(x)\right)^2} %%
\end{equation}
\else
\begin{equation}
f'(x) = %
\frac{u'(x) \cdot v(x) \,\textcolor{red}{\mathbf{-}}\, u(x) \cdot v'(x)}%
{\left(v'(x)\right)^2} %%
\end{equation}
\fi

\begin{MExample}
Die Ableitung von $f(x) = x^2 \cdot \MEU^x$ ist
\[
f'(x) = 2 x \MEU^x + x^2 \MEU^x = (x^2 + 2x) \MEU^x %%
\]
Die Ableitung von $g(x) = \tan(x) = \frac{\sin(x)}{\cos(x)}$ ist
mit $\sin^2(x) + \cos^2(x) = 1$ dann
\[
g'(x) = \frac{\cos(x) \cdot \cos(x) - \sin(x) \cdot (-\sin(x))}{\cos^2(x)} %
 = 1 + \left(\frac{\sin(x)}{\cos(x)}\right)^2 %
 = 1 + \tan^2(x) %
 = \frac{1}{\cos^2(x)} %%
\]
\end{MExample}


\MSubsubsection{Verkettung von Funktionen}

Wenn die Funktion $u$ mit der Funktion $v$ verkettet werden kann (wenn $u$ in 
$v$ eingesetzt werden kann), dann ist auch die Verkettung
$f(x) := (v \circ u)(x) = v(u(x))$ differenzierbar, und es gilt
\begin{equation}
f'(x) = v'(u(x)) \cdot u'(x) %%
\end{equation}
Hier ist $v'(u(x))$ der Wert der Ableitung $v'$, der sich an der Stelle $u(x)$ 
ergibt. %ausgewertet wird.

\begin{MExample}
Die Ableitung von $f(x) = (3 - 2 x)^5$ ist
\[
f'(x) = 5 (3 - 2 x)^4 \cdot (-2) = -10 (3 - 2 x)^4 %%
\]
Die Ableitung von $g(x) = \MEU^{x^3}$ ist
\[
g'(x) = \MEU^{x^3} \cdot 3 x^2 = 3 x^2 \MEU^{x^3} %%
\]
\end{MExample}


\MSubsubsection{Umkehrfunktion}

Wenn die Funktion $u$ umkehrbar ist und ihre Ableitung an der Stelle $x_0$ 
ungleich null ist, also $u'(x_0) \neq 0$ gilt, dann ist die 
Umkehrfunktion $f(y) := u^{-1}(y)$ an der Stelle $y_0 := u(x_0)$ differenzierbar.
%Mit der Kettenregel, angewandt auf $x = u^{-1}(u(x)) = f(u(x))$ gilt -- da 
%die Ableitung der identischen Abbildung $x \mapsto x$ konstant $1$ ist -- dann
Mit der Kettenregel, angewandt auf $y = u(u^{-1}(y)) = u(f(y))$ gilt -- da 
die Ableitung 
%der identischen Abbildung $y \mapsto y$ konstant $1$ ist 
des Terms $y$ auf der linken Seite -- dann
\begin{equation}\MLabel{eq:AblUmkehrfkta}
%1 = f'(u(x_0)) \cdot u'(x_0) %
%\quad \text{und damit} \quad %
%f'(y_0) = f'(u(x_0)) = \frac{1}{u'(x_0)} = \frac{1}{u'(f(y_0))}. %%
1 = u'(f(y_0)) \cdot f'(y_0) %
\quad \text{und damit} \quad %
f'(y_0) = \frac{1}{u'(f(y_0))}. %%
\end{equation}
Ebenso kann von $x = u^{-1}(u(x)) = f(u(x))$ ausgegangen werden: Mit der
Kettenregel und $x_0 = u^{-1}(y_0) = f(y_0)$ ergibt sich dann
\begin{equation}\MLabel{eq:AblUmkehrfktb}
1 = f'(u(x_0)) \cdot u'(x_0) %
\quad \text{und damit} \quad %
f'(y_0) = f'(u(x_0)) = \frac{1}{u'(x_0)} = \frac{1}{u'(f(y_0))}. %%
\end{equation}

\begin{MXInfo}{Ergebnis: Ableitung der Umkehrfunktion}
Ersetzt man in den Gleichungen (\MNRef{eq:AblUmkehrfkta}) bzw. 
(\MNRef{eq:AblUmkehrfktb}) dann $f$ durch $u^{-1}$ und schreibt $y$ f"ur $y_0$, 
so ergibt sich die Regel
\[
\left(u^{-1}\right)'(y) = \frac{1}{u'\left(u^{-1}(y)\right)}. %%
\]
f"ur die Ableitung der Umkehrfunktion (an der Stelle $y$).

Wie "ublich, kann die unabh"angige Variable der Umkehrfunktion auch wieder 
mit $x$ bezeichnet werden.
\end{MXInfo}

\begin{MExample}
Die Funktion $u\colon (0; \infty) \to \R$ mit $u(x) = x^2$ ist umkehrbar und 
differenzierbar mit $u'(x) = 2 x$.
Damit ist die Ableitung der Umkehrfunktion $u^{-1}(y) = \sqrt{y}$ 
f"ur $y > 0$ dann
\[
\left(u^{-1}\right)'(y) = \frac{1}{u'\left(u^{-1}(y)\right)} %
 = \frac{1}{2 \left(u^{-1}(y)\right)} %%
 = \frac{1}{2 \sqrt{y}} %%
\]
"Ublicherweise wird die unabh"angige Variable der Umkehrfunktion noch mit $x$ 
bezeichnet.
Damit ist die Ableitung f"ur die Wurzelfunktion hergeleitet. 
In derselben Weise erh"alt man die Ableitungsregel von $x^{\frac{1}{n}}$ 
f"ur $n \neq 0$.
\end{MExample}
\end{MContent}

%end of file.




%%%Uebungen zum Abschnitt:
\begin{MExercises}

\begin{MExercise}
Berechnen Sie die Ableitung von
\begin{MExerciseItems}
\item $f(x) := 3 + 5 x$
zu $f'(x) = $\MParsedQuestion{30}{5}{4}
%
\item $g(x) := \frac{1}{4 x} - x^3$
zu $g'(x) = $\MSimplifyQuestion{30}{-1/(4*x^2) - 3*x^2}{1}{x}{20}{0}
%
\item $h(x) := 2 \sqrt{x} + 4 x^{-3}$
zu $h'(x) = $\MSimplifyQuestion{30}{1/2*sqrt(x) - 12/(x^4)}{1}{x}{20}{0}
\end{MExerciseItems}
\end{MExercise}

\begin{MExercise}
Berechnen und vereinfachen Sie die Ableitung von
\begin{MExerciseItems}
\item $f(x) := \cot x = \frac{\cos(x)}{\sin(x)}$
zu $f'(x) = $\MSimplifyQuestion{30}{-1/(sin(x) * sin(x))}{10}{x}{4}{0}
%
\item $g(x) := \sin(3 x) \cdot \cos(3 x)$
zu $g'(x) = $\MSimplifyQuestion{30}{3*cos(6*x)}{10}{x}{4}{0}
%
\item $h(x) := \frac{\sin(3 x)}{\sin(6 x)}$
zu $h'(x) = $\MSimplifyQuestion{30}{3/2 * tan(3*x)/cos(3*x)}{10}{x}{4}{0}
\end{MExerciseItems}
\end{MExercise}

\begin{MExercise}
Berechnen Sie die Ableitung von
\begin{MExerciseItems}
\item $f(x) := \MEU^{5 x}$
zu $f'(x) = $\MSimplifyQuestion{30}{5*e^(5*x)}{10}{x}{4}{0}
%
\item $g(x) := x \cdot \MEU^{6 x}$
zu $g'(x) = $\MSimplifyQuestion{30}{(6*x + 1) * e^(6*x)}{10}{x}{4}{0}
%
\item $h(x) := (x^2 - x) \cdot \MEU^{-2 x}$
zu $h'(x) = $\MSimplifyQuestion{30}{-(2*x^2 - 4*x + 1) * e^(-2*x)}{10}{x}{4}{0}
\end{MExerciseItems}
\end{MExercise}

\begin{MExercise}
Berechnen Sie die ersten vier Ableitungen von $f(x) := \sin(1 - 2x)$.

Antwort: 
Es wird mit $f^{(k)}$ die $k$-te Ableitung von $f$ bezeichnet.
Damit ist
\begin{itemize}
\item
 $f^{(1)}x) = $\MSimplifyQuestion{30}{-2 * cos(1 - 2*x)}{10}{x}{4}{0}
\item
 $f^{(2)}(x) = $\MSimplifyQuestion{30}{-4 * sin(1 - 2*x)}{10}{x}{4}{0}
\item
 $f^{(3)}x) = $\MSimplifyQuestion{30}{8 * cos(1 - 2*x)}{10}{x}{4}{0}
\item
 $f^{(4)}x) = $\MSimplifyQuestion{30}{16 * sin(1 - 2*x)}{10}{x}{4}{0}
\end{itemize}
\end{MExercise}

\begin{MExercise}
Berechnen Sie die Ableitung von $f(x) := \arctan(x)$, also  der Umkehrfunktion 
von $\tan(x)$.

Antwort: Es ist
 $f'(x) = $\MSimplifyQuestion{30}{1/(1 + x^2)}{10}{x}{4}{0}
\end{MExercise}

\end{MExercises}



%%%Abschnitt
\MSubsection{Eigenschaften}\MLabel{M07_Eigenschaften}

\begin{MIntro}
Die Ableitung wurde oben mittels einer Geraden eingef"uhrt, die den 
Funktionsverlauf {\glqq}n"aherungsweise{\grqq} beschreibt. Aus den 
Eigenschaften der Geraden kann nun auf Eigenschaften der Funktion 
geschlossen werden.
\end{MIntro}

%%%Inhalt Teil A:
%LaTeX-2e-File, Liedtke, 20140731.
%Inhalt: Einf"uhrung in die Differentialrechnung, Abschnitt 4.
%zuletzt bearbeitet: 20140922.

%Abschnitt 4: Eigenschaften
\begin{MContent} 

\MSubsubsection{Stetigkeit differenzierbarer Funktionen}
%\begin{MXContent}{Stetigkeit differenzierbarer Funktionen}{Stetigkeit}{STD}
 
%\begin{MXInfo}{Stetigkeit differenzierbarer Funktionen}
Wenn eine Funktion $f$ an der Stelle $x_0$ differenzierbar ist, dann ist $f$ an 
der Stelle $x_0$ auch stetig.
%\end{MXInfo}

Folglich ist eine ("uberall) differenzierbare Funktion ("uberall) stetig.
Insbesondere hat eine auf einem Intervall definierte differenzierbare Funktion
keine Sprungstellen.

Andererseits zeigt das Beispiel der stetigen Funktion $f: \R \to \R$ mit 
$f(x) := |x|$ f"ur $x \in \R$, dass aus der Stetigkeit allein nicht folgt, dass
$f$ auch differenzierbar ist. Denn $f$ in an der Stelle $x_0 = 0$ nicht 
differenzierbar.
%\end{MXContent}

\MSubsubsection{Monotonie}
%\begin{MXContent}{Monotonie}

Mit der Ableitung kann das lokale Wachstumsverhalten untersucht werden, das 
hei"st ob f"ur gr"o"ser werdende $x$-Werte die Funktionswerte gr"o"ser oder 
kleiner werden.
Dazu wird eine Funktion $f: D \to \R$ betrachtet, die auf $(a, b) \subseteq D$ 
differenzierbar ist.
%Bild:
\begin{center}
\ifttm
\MGraphicsSolo{\MPfadBilder/jb07A4_Monotonie.png}{scale=0.5}
%\MGraphicsSolo{jb07A4_Monotonie.png}{scale=1}
\else
\renewcommand{\jTikZScale}{1.0}
%Bild: {\MPfadBilder/jb07A4_Monotonie.tex}

%LaTeX-File, Liedtke, 20140829.
%VBKM-Modul 7 Differentialrechnung: Bild zur Monotonie.
%Bildname: jb07A4_Monotonie.tex.
%Erstellt: 20140829, Liedtke.
%Bearbeitet: 20140901, Liedtke (Dateiname ohne Endung erfasst).

\begin{small}
\tikzsetnextfilename{jb07A4_Monotonie}
\begin{tikzpicture}[line width=1.5pt,scale=\jTikZScale, %
declare function={
  x1 = 2;
  x0 = 4;
  fkt(\x) = 1/4*(\x - 3)*(\x - 3) + 0.75;
  TangenteAblplus(\x) = 1 + 1/2*(\x - x0); % $f(x_0) = f(4) = 1$.
  TangenteAblminus(\x) = 1 - 1/2*(\x - x1); % $f(x_1) = f(2) = 1$.
}
] %[every node/.style={fill=white}] 
%,every node/.style={fill=white}] 
%Koordinatenachsen:
\draw[->] (-0.6, 0) -- (6, 0) node[below left]{$x$}; %x-Achse
\draw[->] (0, -0.6) -- (0, 3) node[below left]{$y$}; %y-Achse
%Achsenbeschriftung:
\foreach \x in {1, 2, 3, 4, 5} \draw (\x, 0) -- ++(0, -0.1); %
% node[below] {$\x$}; 
\foreach \y in {1, 2} \draw (0, \y) -- ++(-0.1, 0); %
% node[below left] {$\y$};
%\node[below left] at (0, 0) {$0$};
%Hilfslinien:
\draw[color=black!50!white] (x0, {fkt(x0)}) -- ({x0+1}, {fkt(x0)});
\draw[->,color=black!50!white] %
 ({x0+1}, {fkt(x0)}) -- ({x0+1}, {TangenteAblplus(x0+1)});
%
\draw[color=black!50!white] (x1, {fkt(x1)}) -- ({x1-1}, {fkt(x1)});
\draw[->,color=black!50!white] %
 ({x1-1}, {TangenteAblminus(x1-1)}) -- ({x1-1}, {fkt(x1)});
%Funktion:
\draw[domain=0.8:5.2,samples=120,color=\jccolorfkt] %
 plot (\x, {fkt(\x)});
%Tangenten:
\draw[domain={x1-1.2}:{x1+0.8},samples=120,color=blue!50!black] %
 plot (\x, {TangenteAblminus(\x)});
\filldraw[color=blue!50!black] (x0, {fkt(x0)}) circle (1pt); % Ber"uhrpunkt.
%
\draw[domain={x0-0.8}:{x0+1.2},samples=120,color=blue!50!black] %
 plot (\x, {TangenteAblplus(\x)});
\filldraw[color=blue!50!black] (x1, {fkt(x1)}) circle (1pt); % Ber"uhrpunkt.
%Beschriftung:
\node[right] at (5.1, 1.25) {$f'(x_0) = m_0 > 0$};
\node[style={fill=white},left] at (0.9, 1.25) {$f'(x_1) = m_1 < 0$};
\end{tikzpicture}
\end{small}
%end of file
\fi
\end{center}
%Bildende.

%\begin{MXInfo}{Monotonie}
Wenn $f'(x) \leq 0$ f"ur alle $x$ zwischen $c$ und $d$ gilt, dann ist $f$ 
auf dem Intervall $(c; d)$ monoton fallend.

Wenn $f'(x) \geq 0$ f"ur alle $x$ zwischen $c$ und $d$ gilt, dann ist $f$ 
auf dem Intervall $(c; d)$ monoton wachsend.
%\end{MXInfo}

Somit gen"ugt es, das Vorzeichen der Ableitung $f'$ zu bestimmen, um zu 
erkennen, ob eine Funktion monoton wachsend oder monoton fallend ist.
%\end{MXContent}

\begin{MExample}
Die Funktion $f: \R \to \R, x \mapsto x^3$ ist differenzierbar mit 
$f(x) = 3 x^2$. Da $x^2 \geq 0$ f"ur alle $x \in \R$ gilt, ist 
$f'(x) \geq 0$ und damit $f$ monoton wachsend.

F"ur $g: \R \to \R$ mit $g(x) = 2 x^3 + 6 x^2 - 18 x + 10$ hat die Ableitung
$g'(x) = 6 x^2 + 12 x - 18 = 6 (x + 3) (x - 1)$ die Nullstellen $x_1 = -3$ 
und $x_2 = 1$.
Mit Hilfe folgender Tabelle wird bestimmt, in welchen Bereichen die Ableitung 
von $g$ positiv bzw. negativ ist, woraus sich dann die Monotoniebereiche von 
$g$ ergeben. Der Eintrag $+$ besagt, dass der betrachtete Term im angegebenen
Intervall positiv ist. Wenn er negativ ist, wird $-$ eingetragen.
\[
\begin{array}{cccc}
x & x < -3 & -3 < x < 1 & 1 < x \\
\hline
x + 3 & - & + & + \\
x - 1 & - & - & + \\
g'(x) & - & - & + \\
g \text{ monoton} & \text{wachsend} & \text{fallend} & \text{wachsend} \\
\end{array}
\]
Die Funktion $h: \R \setminus \{ 0 \} \to \R$ mit $h(x) = \frac{1}{x}$
hat die Ableitung $h'(x) = - \frac{1}{x^2}$. Hier gilt 
$h'(x) < 0$ f"ur alle $x \neq 0$.

Dennoch ist $h$ nicht monoton fallend, da beispielsweise 
$h(-2) = -\frac{1}{2} < 1 = h(1)$ gilt. Der Grund f"ur dieses "uberraschende 
Ergebnis ist, dass der Definitionsbereich von $h$ kein Intervall ist. Die 
Funktion $h$ ist auf $(-\infty; 0)$ monoton fallend, das hei"st, die 
Einschr"ankung von $h$ auf dieses Intervall ist monoton fallend. Zudem ist 
$h$ f"ur alle $x > 0$ monoton fallend.
\end{MExample}

\MSubsubsection{Kr"ummungseigenschaften}
%\begin{MXContent}{Kr"ummungseigenschaften}

Gegeben ist eine Funktion $f: D \to \R$, die auf $(a; b) \subseteq D$ 
zweimal differenzierbar ist.

%\begin{MXInfo}{Kr"ummungseigenschaften}
Wenn ${f'}'(x) \geq 0$ f"ur alle $x$ zwischen $c$ und $d$ gilt, dann ist $f$ 
auf dem Intervall $(c; d)$ konvex.

Wenn ${f'}'(x) \leq 0$ f"ur alle $x$ zwischen $c$ und $d$ gilt, dann ist $f$ 
auf dem Intervall $(c; d)$ konkav.
%\end{MXInfo}

Somit gen"ugt es, das Vorzeichen der zweiten Ableitung ${f'}'$ zu bestimmen, 
um zu erkennen, ob eine Funktion konvex (linksgekr"ummt) oder konkav 
(rechtsgekr"ummt) ist.
%\end{MXContent}

\begin{MXInfo}{Anmerkung zur Notation}
Die zweite und weitere {\glqq}h"ohere{\grqq} Ableitungen oft mit nat"urlichen 
Zahlen in runden Klammern gekennzeichnet: Die $k$-te Ableitung wird dann mit
$f^{(k)}$ bezeichnet. Diese Bezeichnung wird besonders in Formeln auch f"ur 
die (erste) Ableitung f"ur $k=1$ und f"ur die Funktion $f$ selbst f"ur $k=0$
verwendet.

Damit bezeichnet 
\begin{itemize}
\item $f^{(0)} = f$ die Funktion $f$, 
\item $f^{(1)} = f'$ die (erste) Ableitung,
\item $f^{(2)} = {f'}'$ die zweite Ableitung,
\item $f^{(3)}$ die dritte Ableitung von $f$, etc. (sofern diese existieren).
\end{itemize}
\end{MXInfo} 

Das folgende Beispiel zeigt, dass eine monoton wachsende Funktion in einem 
Bereich konvex und in einem anderen konkav sein kann.

\begin{MExample}
Die Funktion $f: \R \to \R, x \mapsto x^3$ ist zweimal differenzierbar. Wegen
$f'(x) = 3 x^2 \geq 0$ ist $f$ monoton wachsend.

Weiter ist $f^{(2)}(x) = 6 x$. Somit ist f"ur $x < 0$ auch $f^{(2)} < 0$
und damit $f$ hier konkav (nach rechts gekr"ummt), und f"ur $x > 0$ ist
$f^{(2)}(x) > 0$, sodass $f$ f"ur $x > 0$ konvex (nach links gekr"ummt) ist.
\end{MExample}
\end{MContent}

%end of file.

 


%%%Uebungen zum Abschnitt:
\begin{MExercises}

\begin{MExercise}
In welchen m"oglichst gro"sen offenen Intervallen ist die Funktion
$f(x) := \frac{x^2 - 1}{x^2 + 1}$
monoton wachsend beziehungsweise monoton fallend?
%In welchen m"oglichst gro"sen offenen Intervallen sind die Funktionen
%\begin{MExerciseItems}
%\item $f(x) := x^2 - 12\right) \cdot (x^2 + 5)$
%\item $g(x) := \frac{x^2 + 1}{x^2 - 1}$
%\end{MExerciseItems}
%monoton wachsend beziehungsweise monoton fallend?

Antwort: Es ist
\begin{itemize}
\item $f$ auf $(-\infty;0)$ monoton \MQuestion{12}{fallend}.
%\item $f$ auf $(-infty;0)$ monoton \MQuestion{12}{fallend}.
%\item $f$ auf \MIntervalQuestion{20}{(-infty;0)}{4} monoton fallend.
%\item $f$ auf $(-infty$\MParsedQuestion{8}{0}{4}$)$ monoton fallend.
%
\item $f$ auf $(0; \infty)$ monoton \MQuestion{12}{wachsend}.
%\item $f$ auf $(0; infty)$ monoton \MQuestion{12}{wachsend}.
%\item $f$ auf \MIntervalQuestion{20}{(0;infty)}{4} monoton wachsend.
%\item $f$ auf $($\MIntervalQuestion{8}{0}{4}$; infty)$ monoton wachsend.
\end{itemize}
\end{MExercise}


\begin{MExercise}
In welchen m"oglichst gro"sen offenen Intervallen $(c; d)$ ist die Funktion
$f(x) := \frac{x^2 - 1}{x^2 + 1}$ f"ur $x > 0$
konvex beziehungsweise konkav?
%In welchen m"oglichst gro"sen offenen Intervallen sind die Funktionen
%konvex beziehungsweise konkav?
%\begin{MExerciseItems}
%\item $f(x) := \frac{6 x}{x^2 + 4}$
%\item $g(x) := \left(x - 2 \sqrt{3}\right) \cdot ($
%\end{MExerciseItems}
\MInputHint{Schreiben Sie \texttt{infty} f"ur $\infty$ in Ihrer Antwort.}

Antwort: Es ist
\begin{itemize}
\item $f$ auf \MIntervalQuestion{20}{(0; 1/sqrt(e))}{4} konvex.
%
\item $f$ auf \MParsedQuestion{16}{(1/sqrt(3);infty)}{4} konkav.
\end{itemize}
\end{MExercise}


\begin{MExercise}
Gegeben ist eine Funktion $f: [-4.5; 4] \to \R$ mit $f(0) := 2$, deren 
Ableitung $f'$ in nachstehendem Graphen gezeichnet ist.

%Bild:
\ifttm
\MUGraphics{\MPfadBilder/jb07A4_AgAblGraph.png}{scale=0.5}%
{Ableitung von $f$}{}
%\MUGraphicsSolo{jb07A4_AgAblGraph.png}{scale=1}{}
\else
\begin{center}
%\input{\MPfadBilder/jb07A4_AgAblGraph.tex}
%LaTeX-File, Liedtke, 20140827.
%VBKM-Modul 7 Differentialrechnung: Bild zu einer Aufgabe zu 
% Funktionseigenschaften (Eigenschaften aus dem Graph der Ableitung bestimmen).
%Bildname: jb07A4_AgAblGraph.tex.
%Erstellt: 20140826, Liedtke.
%Bearbeitet: 20140829, Liedtke (Dateiname angepasst).
%Bearbeitet: 20140901, Liedtke (Dateiname ohne Endung erfasst).

\begin{small}
\renewcommand{\jTikZScale}{0.6}
\tikzsetnextfilename{jb07A4_AgAblGraph}
\begin{tikzpicture}[line width=1.5pt,scale=\jTikZScale, %
declare function={
  fkt(\x) = 1/40*(\x + 4)*(\x)*(\x - 3)*(\x - 3);
}
] %[every node/.style={fill=white}] 
%Koordinatenachsen:
\draw[->] (-5, 0) -- (5, 0) node[below left]{$x$}; %x-Achse
\draw[->] (0, -3.5) -- (0, 4) node[below left]{$y$}; %y-Achse
%Achsenbeschriftung:
\foreach \x in {-4, -3, -2, -1} \draw (\x, 0) -- ++(0, 0.1) %
 node[above] {$\x$}; 
\foreach \x in {1, 2, 3, 4} \draw (\x, 0) -- ++(0, -0.1) %
 node[below] {$\x$}; 
\foreach \y in {-3, -2, -1} \draw (0, \y) -- ++(0.1, 0) %
 node[right] {$\y$};
\foreach \y in {1, 2, 3} \draw (0, \y) -- ++(-0.1, 0) %
 node[left] {$\y$};
%\node[below left] at (0, 0) {$0$};
%Funktion:
\draw[domain=-4.5:4,samples=120,color=\jccolorfkt] %
 plot (\x, {fkt(\x)});
\end{tikzpicture}
\end{small}

%end of file
\end{center}
\fi
%Bildende.
\begin{MExerciseItems}
\item Wo ist $f$ monoton wachsend, wo monoton fallend? 
Gesucht sind jeweils m"oglichst gr"o"se offene Intervalle $(c; d)$, auf 
denen $f$ diese Eigenschaft hat.
%
\item Welche Aussagen erhalten 
Sie "uber die Maximal- beziehungsweise Minimalstellen der Funktion $f$?
% f"ur $f$ zu deren Maximal- beziehungsweise Minimalstellen?
\end{MExerciseItems}

Antwort: Es ist
\begin{itemize}
\item $f$ auf $(-4.5; $\MParsedQuestion{8}{-4}{4}$)$ monoton 
\MQuestion{16}{wachsend}.
%
\item $f$ auf $($\MParsedQuestion{16}{-4}{4}$; 0)$ monoton 
\MQuestion{16}{fallend}.
%
\item $f$ auf $(0; 3)$ monoton \MQuestion{16}{wachsend}.
%
\item $f$ auf $(3; 4)$ monoton \MQuestion{16}{wachsend}.
\end{itemize}
Maximalstelle ist \MParsedQuestion{8}{-4}{4}.
Minimalstelle ist \MParsedQuestion{8}{0}{4}.
\end{MExercise}
\end{MExercises}



%%%Abschnitt
\MSubsection{Anwendungen}\MLabel{M07_Anwendungen}

\begin{MIntro}
Im vorherigen Abschnitt wurde erl"autert, wie der Funktionsverlauf mit Hilfe
der Ableitung beschrieben werden kann. Daraus ergeben sich weitere Folgerungen
zu besonderen Funktionswerten wie Maxima oder Minima. 
%Beispielsweise folgt aus der Kenntnis, dass die Funktionswerte bis zu einer 
%Stelle $x_0$ monoton wachsen und danach monoton fallen, dass die Stelle $x_0$ 
%eine Maximalstelle ist. Entsprechend zu Aussagen "uber Maximal- und 
%Minimalstellen ergeben auch Aussagen "uber Wendestellen.
Damit liegt es nahe, Aussagen "uber Funktionen mittels der Ableitung zu 
gewinnen. Systematisch wird dies unter dem Stichwort Kurvendiskussion
durchgef"uhrt. 

%Als weitere Beispiele f"ur Anwendungen werden die Berechnung von Grenzwerten 
%sowie die Bearbeitung von Optimierungsaufgaben vorgestellt. 
%Letztere befassen 
Als weiteres Beispiel f"ur Anwendungen werden Optimierungsaufgaben vorgestellt. 
Sie befassen sich mit der Frage nach bestm"oglichen Werten einer Funktion, 
beispielsweise einem minimalen Materialverbrauch. Somit handelt es sich aus 
mathematischer Sicht um die Bestimmung von Extremstellen im Rahmen einer 
technischen oder wirtschaftlichen Produktion.
\end{MIntro}

%%%Inhalt Teil A:
%LaTeX-2e-File, Liedtke, 20140731.
%Inhalt: Einf"uhrung in die Differentialrechnung, Abschnitt 5.
%zuletzt bearbeitet: 20140930.

%Abschnitt 5: Anwendungen der Differentialrechnung
%\MSubsubsection{Anwendungen}
%\begin{MContent}


%\MSubsubsection{Grenzwerte}
%%%\begin{MXContent}{Grenzwerte}{Grenzwerte}{STD}
%%%Gegeben ist $a \in \R$, und es sind Funktionen 
%%%$g,h: ]r, s[\; \setminus \{ a \} \to \R$ gegeben.
%%%
%%%Weiter haben die Funktionen $g$ und $h$ die Eigenschaft, dass
%%%$\displaystyle \lim_{x \to a}\; g(x) = 0$ und
%%%$\displaystyle \lim_{x \to a}\; h(x) = 0$ 
%%%gilt.
%%%
%%%Wenn $g$ und $h$ in einer Umgebung von $a$ differenzierbar sind und der 
%%%Grenzwert $\lim_{x \to a}\; \frac{g'(x)}{h'(x)}$ existiert, dann gilt
%%%\begin{equation}
%%%\lim_{x \to a}\; \frac{g(x)}{h(x)}
%%%= \lim_{x \to a}\; \frac{g'(x)}{h'(x)}
%%%\end{equation}
%%%Eine entsprechende Aussage gilt, wenn 
%%%$\displaystyle \lim_{x \to a}\; g(x) = \infty$ und
%%%$\displaystyle \lim_{x \to a}\; h(x) = \infty$ gilt.
%%%
%%%\end{MXContent}
 

%\MSubsubsection{Kurvendiskussion}
\begin{MXContent}{Kurvendiskussion}{Kurvendiskussion}{STD}

Gegeben ist eine differenzierbare Funktion $f: D \to \R$ auf ihrem maximalen 
Definitionsbereich $D$ mit Zuordnungsvorschrift $y = f(x)$ f"ur $x \in D$.

Im ersten Teil werden algebraische und geometrische Aspekte von $f$ betrachtet:
\begin{description}
\item[Maximaler Definitionsbereich]
Es werden alle reellen Zahlen $x$ bestimmt, f"ur die $f(x)$ existiert. Die 
Menge $D$ all dieser Zahlen wird maximaler Definitionsbereich genannt.

\item[Symmetrie des Graphen]
Es wird untersucht, ob $f(-x) = f(x)$ bzw. $f(-x) = -f(x)$ f"ur alle $x \in D$ 
gilt. Im ersten Fall ist der Graph zur $y$-Achse symmetrisch, im zweiten ist 
er zum Nullpunkt $(0;0)$ des Koordinatensystems punktsymmetrisch.

\item[Periodizit"at]
Es wird untersucht, ob es eine Zahl $p > 0$ derart gibt, dass $f(x+p) = f(x)$ 
f"ur alle $x \in D$ gibt.

\item[Schnittpunkte mit den Achsen] Es werden die Schnittpunkte mit den 
Koordinatenachsen bestimmt:
\begin{itemize}
\item $x$-Achse: Es werden alle Nullstellen von $f$ berechnet.
\item $y$-Achse: Wenn $0 \in D$ gilt, wird $f(0)$ berechnet.
\end{itemize}
\item[Asymptotisches Verhalten an den R"andern des Definitionsbereichs]
Es wird das Verhalten in der N"ahe von Definitionsl"ucken und an den R"andern
untersucht: Gibt es eine stetige Fortsetzung oder existieren zumindest die 
einseitigen Grenzwerte? Sind die Grenzwerte gegen $\infty$ bzw. $-\infty$ 
vorhanden, wenn der Definitionsbereich entsprechend unbeschr"ankt ist?
\end{description} 

Im zweiten Teil wird die Funktion mittels Folgerungen aus der Ableitung 
analytisch untersucht. Dazu werden zun"achst die erste und zweite Ableitung 
berechnet, sofern diese existieren.
\begin{description}
\item[Ableitungen]
Berechnung der ersten und zweiten Ableitung (soweit vorhanden).

\item[Extemstellen und Monotonie]
%Wenn $x_0$ eine Extremstellen in $(a, b)$ ist, dann gilt $f'(x_0) = 0$.
Eine notwendige Bedingung f"ur eine Extremstellen $x_0$ im Innern des 
Definitionsbereichs ist $f'(x_0) = 0$.

Ob tats"achlich eine Extremstelle vorliegt, muss dann noch gepr"uft werden.
Dies ist oft mit einem der folgenden Kriterien m"oglich:

Unter der Voraussetzung, dass $f'(x_0) = 0$ gilt, liegt eine
\begin{itemize}
\item Minimalstelle in $x_0$ vor, wenn in einer Umgebung von $x_0$ f"ur $x < x_0$ 
dann $f'(x) < 0$ und f"ur $x > x_0$ dann $f'(x) > 0$ ist,
\item Maximalstelle in $x_0$ vor, wenn in einer Umgebung von $x_0$ f"ur $x < x_0$ 
dann $f'(x) > 0$ und f"ur $x > x_0$ dann $f'(x) < 0$ ist.
\end{itemize}

Wenn auch die zweite Ableitung $f^{(2)} = {f'}'$ existiert, gilt:
Wenn $f'(x_0) = 0$ und zudem 
\begin{itemize}
\item $f^{(2)}(x_0) > 0$ gilt, dann ist $x_0$ eine Minimalstelle von $f$.

\item $f^{(2)}(x_0) < 0$ gilt, dann ist $x_0$ eine Maximalstelle von $f$.
\end{itemize}
Die Funktion $f$ ist auf den Intervallen des Defitionsbereichs monoton
wachend, wo $f'(x) \geq 0$ gilt. Sie ist monoton fallend, wenn $f'(x) \leq 0$ 
gilt.

\item[Wendestellen und Kr"ummungseigenschaften]
%Wenn $x_0$ eine Wendestellen in $(a, b)$ ist, dann gilt $f''(x_0) = 0$.
Wenn die zweite Ableitung ${f'}' = f^{(2)}$ existiert, ist $f^{(2)}(w) = 0$
eine notwendige Bedingung daf"ur, dass $w_0$ eine Wendestelle im Innern des 
Definitionsbereichs sein kann.

Wiederum ist noch zu pr"ufen, ob tats"achlich eine Wendestelle vorliegt.

Wie f"ur Extremstellen kann dies anhand eines Vorzeichenwechsels der zweiten
Ableitung erfolgen oder mit Hilfe der dritten Ableitung:

Unter der Voraussetzung, dass $f^{(2)}(w_0) = 0$ gilt, liegt eine
Wendestelle in $w_0$ vor, wenn in einer Umgebung von $w_0$
\begin{itemize}
\item f"ur $x < w_0$ dann $f^{(2)}(x) < 0$ und f"ur $x > w_0$ dann 
$f^{(2)}(x) > 0$ ist, oder
\item f"ur $x < w_0$ dann $f^{(2)}(x) > 0$ und f"ur $x > w_0$ dann 
$f^{(2)}(x) < 0$ ist.
\end{itemize}

Wenn $f^{(2)}(w_0) = 0$ und $f^{(3)}(w_0) \neq 0$ gilt, dann ist $w_0$ eine 
Wendestelle.
%Hinreichend f"ur eine Wendestelle ist, dass $f^{(2)}(x_0) = 0$ und 
% $f^{(3)}(x_0) \neq 0$ gilt.
Der Vorteil des ersten Kriteriums besteht darin, dass die dritte Ableitung 
nicht berechnet werden muss. 

Die Funktion $f$ ist auf den Intervallen des Definitionsbereichs konvex
(linksgekr"ummt), wo $f^{(2)}(x) \geq 0$ gilt. Sie ist konkav (rechtsgekr"ummt), 
wenn $f^{(2)}(x) \leq 0$ gilt.

\item[Graph]
Anhand der gewonnenen Ergebnisse wird dann ein charakteristischer Ausschnitt des 
Funktionsgraphen gezeichnet, der diese Ergebnisse ber"ucksichtigt.
\end{description}

Hier nochmals die einzelnen Betrachtungen in einer kurzen "Ubersicht:

\begin{MXInfo}{Kurvendiskussion}
Unter einer Kurvendiskussion wird hier die Untersuchung einer differenzierbaren
reellen Funktion $f: D \to \R$ verstanden, die durch einen Funktionsterm
$y = f(x)$ gegeben ist, und folgende Aspekte umfasst:
\begin{center}
\begin{minipage}{0.40\textwidth}
\begin{itemize}
\item Maximaler Definitionsbereich $D_f$
\item Symmetrie
 und Periodizit"at
\item Schnittpunkte mit den Achsen
\item Randverhalten
\end{itemize}
\end{minipage}
%
\begin{minipage}{0.58\textwidth}
\begin{itemize}
\item Ableitungen
\item Extremstellen und Monotoniebereiche
\item Wendestellen und Kr"ummungseigenschaften
%\item Charakteristischer Ausschnitt des Graphen
\item Graph der Funktion
\end{itemize}
\end{minipage}
\end{center}
\end{MXInfo}

Im folgenden Beispiel wird die Vorgehensweise illustriert.

\begin{MExample}
Es wird $f: D_f \to \R$ mit $f(x) = \frac{x^5}{5} - x^3$ auf seinem maximalen 
Definitionsbereich $D_f \subseteq \R$ untersucht.
\begin{description}
\item[Maximaler Definitionsbereich]
Der Funktionsterm ist f"ur alle reellen Zahlen $x$ definiert, sodass 
$D_f = \R$ ist.
\item[Symmetrie des Graphen]
Wegen $f(-x) = \frac{1}{5} (-x)^5 - (-x)^3 %
 = - \left(\frac{1}{5} x^5 - x^3\right) = -f(x)$ ist der Graph von $f$ 
dann punktsymmetrisch zum Nullpunkt $(0, 0)$, die Funktion $f$ ungerade. 
(Da $f$ ein Polynom ist
und $(-x)^{2k} = (-1)^{2k} x^{2k} = x^{2k}$ sowie 
$(-x)^{2k+1} = (-1)^{2k+1} x^{2k+1} = -x^{2k+1}$ f"ur alle $k \in \No$ gilt, 
kann die Antwort auch an den Exponenten der Monome abgelesen werden. 
Im Beispiel sind es die ungeraden Zahlen $5$ und $3$, sodass $f$ ungerade ist.)

Aufgrund der Symmetrie w"urde es f"ur die meisten "Uberlegungen in der 
Kurvendiskussion gen"ugen, diese f"ur $x \geq 0$ vorzunehmen. 
In diesem Beispiel soll diese Eigenschaft einmal nicht von vornherein benutzt 
werden, um eine m"oglichst ausf"uhrliche Darstellung zu bieten. 

\item[Schnittpunkte mit den Koordinatenachsen]
Es ist $f(x) = \frac{1}{5} x^3 \cdot (x^2 - 5)$. Somit sind $u_0 = 0$ und 
$u_1 = -\sqrt{5}$ sowie $u_2 = \sqrt{5}$ die Nullstellen von $f$.


\item[Randverhalten]
Als nicht konstantes Polynom ist $f$ unbeschr"ankt. Wegen $\frac{1}{5} > 0$ ist
$\displaystyle\lim_{x \to \infty} f(x) = \infty$, und aufgrund dessen, dass der 
Grad von $f$ ungerade ist, dann $\displaystyle\lim_{x \to -\infty} f(x) = \infty$. 

\item[Ableitungen]
F"ur $x \in \R$ gilt
$f'(x) = x^4 - 3 x^2$ und $f^{(2)}(x) = 4 x^3 - 6 x$.

\item[Extremstellen]
Die Nullstellen von $f'(x) = x^2 (x^2 - 3)$ sind $x_0 = 0$, $x_1 = -\sqrt{3}$ 
und $x_2 = \sqrt{3}$.
Mit Hilfe folgender Tabelle wird bestimmt, in welchen Bereichen die Ableitung 
von $f$ positiv bzw. negativ ist, woraus sich dann die Monotoniebereiche von 
$f$ ergeben. Der Eintrag $+$ besagt, dass der betrachtete \emph{Faktor} von 
$f'(x)$ im angegebenen Intervall positiv ist. Wenn er negativ ist, wird $-$ 
eingetragen.
\ifttm
\[
\begin{array}{ccccc}
x & x < -\sqrt{3} & -\sqrt{3} < x < 0 & 0 < x < \sqrt{3} & \sqrt{3} < x  \\
\hline
x^2          & + & + & + & + \\
x + \sqrt{3} & - & + & + & + \\
x - \sqrt{3} & - & - & - & + \\
f'(x)        & + & + & - & + \\
f \mbox{\ monoton} & \mbox{wachsend} & \mbox{fallend} & \mbox{fallend} & \mbox{wachsend} %%
\end{array}
\]
\else
\[
\begin{array}{ccccc}
x & x < -\sqrt{3} & -\sqrt{3} < x < 0 & 0 < x < \sqrt{3} & \sqrt{3} < x  \\
\hline
x^2          & + & + & + & + \\
x + \sqrt{3} & - & + & + & + \\
x - \sqrt{3} & - & - & - & + \\
f'(x)        & + & + & - & + \\
f \text{ monoton} & \text{wachsend} & \text{fallend} & \text{fallend} & \text{wachsend} %%
\end{array}
\]
\fi
Somit ist $x_1 = -\sqrt{3}$ eine lokale Maximalstelle und $x_2 = \sqrt{3}$ eine
lokale Minimalstelle.

\item[Wendestellen]
Die Nullstellen von 
$f^{(2)}(x) = 4 x^3 - 6 x = 4 x \left(x^2 - \frac{3}{2}\right)$
sind $w_0 = 0$ und $w_1 = -\sqrt{\frac{3}{2}}$ sowie
$w_2 = \sqrt{\frac{3}{2}}$.

Es handelt sich auch um Wendestellen, da es einfache Nullstellen der zweiten
Ableitung $f^{(2)}$ sind und damit ein Vorzeichenwechsel vorliegt 
(oder da $f^{(3)}(x) = 12 x^2 - 6$ dort jeweils ungleich null ist: Es gilt 
n"amlich $f^{(3)}(w_0) = - 6 \neq 0$ und
$f^{(3)}(w_1) = f^{(3)}(w_2) = 12 \cdot \frac{3}{2} - 6 = 12 \neq 0$).
Wegen $f'(w_0) = 0$ ist somit $(w_0; f(w_0)) = (0; 0)$ ein Sattelpunkt.

\item[Funktionsgraph]
Mit obigen Ergebnissen kann dann ein Ausschnitt des 
Funktionsgraphen von $f$ f"ur $-2.5 \leq x \leq 2.5$ gezeichnet werden.
\ifttm
%\begin{center}
\MUGraphics{\MPfadBilder/jb07A5_BspKurvendiskussion.png}{scale=0.5}%
{Graph von $f(x) = \frac{1}{5} x^5 - x^3$ f"ur $-2.5 \leq x \leq 2.5$}{}
%\end{center}
\else
\begin{center}
%\input{jb07A5_BspKurvendiskussion.tex}
\begin{small}
\renewcommand{\jTikZScale}{0.6}
\tikzsetnextfilename{jb07A5_BspKurvendiskussion}
\begin{tikzpicture}[line width=1.5pt,scale=\jTikZScale, %
declare function={
  fkt(\x) = (1/5*\x*\x - 1)*(\x)*(\x)*(\x);
}
] %[every node/.style={fill=white}] 
%Koordinatenachsen:
\draw[->] (-5, 0) -- (5, 0) node[below left]{$x$}; %x-Achse
\draw[->] (0, -4.4) -- (0, 5) node[below left]{$y$}; %y-Achse
%Achsenbeschriftung:
\foreach \x in {-4, -3, -2, -1} \draw (\x, 0) -- ++(0, -0.1) %
 node[below] {$\x$}; 
\foreach \x in {1, 2, 3, 4} \draw (\x, 0) -- ++(0, 0.1) %
 node[above] {$\x$}; 
\foreach \y in {-4, -3, -2, -1} \draw (0, \y) -- ++(-0.1, 0) %
 node[left] {$\y$};
\foreach \y in {1, 2, 3, 4} \draw (0, \y) -- ++(-0.1, 0) %
 node[left] {$\y$};
%\node[below left] at (0, 0) {$0$};
%Funktion:
\draw[domain=-2.5:2.5,samples=120,color=\jccolorfkt] %
 plot (\x, {fkt(\x)});
\end{tikzpicture}
\end{small}
\end{center}
\fi
\end{description}
\end{MExample}
\end{MXContent}


%\MSubsubsection{Optimierungsaufgaben}
\begin{MXContent}{Optimierungsaufgaben}{Optimierungsaufgaben}{STD}

In der Optimierung wird in einer vorgegebenen Schar von L"osungen einer 
Aufgabe diejenige gesucht, die eine vorab festgelegte Eigenschaft am besten 
erf"ullt.

Als Beispiel wird die Aufgabe betrachtet, eine zylinderf"ormige Dose zu 
konstruieren. Die Dose soll zus"atzlich die Bedingung erf"ullen, ein 
Fassungsverm"ogen von einem Liter zu haben. Sind $r$ der Radius und $h$ die
H"ohe der Dose, so soll also $\pi r^2 \cdot h = 1$ sein (auf die physikalischen
Einheiten wurde der mathematischen Einfachheit halber verzichtet -- in der 
Praxis ist es allerdings oft hilfreich, die Einheiten 
%physikalisch richtig auch 
zur Kontrolle der Ergebnisse mit aufzuschreiben). 

Gesucht wird nach derjenigen Dose, die eine m"oglichst kleine Oberfl"ache 
$O = 2 \cdot \pi r^2 + 2 \pi r h$ hat. Dies bedeutet, dass in 
dieser vereinfachten Betrachtung auch wenig Material in der Herstellung 
ben"otigt wird.

Mathematisch formuliert f"uhrt die Aufgabe auf die Suche nach einem 
Minimum f"ur die Funktion $O$ der Oberfl"ache, wobei das Minimum nur 
unter den Werten f"ur $r$ und $h$ gesucht wird, f"ur die auch die Bedingung 
"uber das Volumen $\pi r^2 \cdot h = 1$ erf"ullt ist.

Eine solche zus"atzliche Bedingung bei der Suche nach Extremstellen wird 
auch Nebenbedingung genannt. Sie kann ebenfalls mit einer Funktion 
formuliert werden.

Wird n"amlich $g(r, h) := \pi r^2 \cdot h$ gesetzt, dann besagt obige 
Bedingung, dass nur solche Paare von $r$ und $h$ betrachtet werden, f"ur die 
$g(r, h) = 1$ ist.

Auf diese Weise ergibt sich mit Hilfe zweier Funktionen, die von mehreren
Variablen abh"angen k"onnen, eine einfache Formulierung einer 
Optimierungsaufgabe. 

\begin{MXInfo}{Optimierungsaufgabe}
In einer \MEntry{Optimierungsaufgabe}{Optimierungsaufgabe} wird eine
Extremstelle $x_{\text{ext}}$ einer Funktion $f$ gesucht, die eine 
gegebene Gleichung $g(x_{\text{ext}}) = b$ erf"ullt.

Wenn ein globales Minimum gesucht wird, spricht man auch von einer 
\MEntry{Minimierungsaufgabe}{Minimierungsaufgabe}. Wenn ein Maximum gesucht
wird, hei"st die Optimierungsaufgabe eine 
\MEntry{Maximierungsaufgabe}{Maximierungsaufgabe}.

Die Funktion $f$ hei"st \MEntry{Zielfunktion}{Zielfunktion}, und die 
Gleichung $g(x) = b$ wird \MEntry{Nebenbedingung}{Nebenbedingung} der 
Optimierungsaufgabe genannt.
\end{MXInfo}

Im obigen Beispiel ist $O$ die Zielfunktion der Minimierungsaufgabe, die 
unter der Nebenbedingung $g(r, h) = 1$ gel"ost werden soll. Da sich die 
durch $g(r, h) = \pi r^2 \cdot h = 1$ gegebene Gleichung nach $r$ oder nach 
$h$ aufl"osen l"asst, kann die Funktion $O = O(r, h)$ in eine Funktion einer 
Variablen "uberf"uhrt werden, sodass die Frage nach dem Minimum mit den Mitteln 
der Kurvendiskussion %von Funktionen einer reellen Variablen 
bestimmt werden kann.

Aufl"osen nach $h$ f"uhrt auf $h = \frac{1}{\pi r^2}$. Eingesetzt in $O$ 
ergibt sich, dass eine (globale) Minimalstelle von
\[
f(r) := O(r, h) = 2 \cdot \pi r^2 + \frac{2 \pi r}{\pi r^2} %
 = 2 \pi r^2 + \frac{2}{r} %%
\]
f"ur $r > 0$ gesucht wird (da der Radius $r$ der Dose positiv ist).
Die notwendige Bedingung $f'(x) = 0$ f"uhrt auf
\[
0 = f'(x) = 4 \pi r - \frac{2}{r^2}, %%
\]
woraus $r^3 = \frac{1}{2 \pi}$ folgt. Wegen $r > 0$ ist dann
\[
R = \sqrt[3]{\frac{1}{2 \pi}} %%
\]
die einzige L"osung der Gleichung. Da $f$ nach oben unbeschr"ankt ist 
(an den R"andern f"ur $r \to 0$ bzw. $r \to \infty$ gegen {\glqq}unendlich 
strebt{\grqq}), ist $R$ als einzige Extremstelle die Minimalstelle von $f$.
%
Die zugeh"orige H"ohe $H$ der Dose mit Radius $R$ ist dann
$H = \sqrt[3]{\frac{4}{\pi}}$.

Abschliesend soll noch auf eine Besonderheit dieser Dose hingewiesen werden:
F"ur den Durchmesser $D$ der Dose gilt
\[
D = 2 R = \sqrt[3]{8} \cdot \sqrt[3]{\frac{1}{2 \pi}} %
 = \sqrt[3]{\frac{8}{2 \pi}} = H. %%
\]
Die H"ohe der Dose ist gleich dem Durchmesser, sodass die Dosen in 
w"urfelf"ormige Kisten verpackt werden k"onnen.
\end{MXContent}

%\end{MContent}

%end of file.




%%%Uebungen zum Abschnitt:
\begin{MExercises}

\begin{MExercise} %Anwendungen
Berechnen Sie alle lokalen und globalen Extremstellen sowie alle Wendestellen 
der Funktion $f: \R \to \R, x \mapsto x^4 - 4 x^3 - 2 x^2 + 12 x - 7$.

Antwort: Die Extremstellen werden so mit $x_1$, $x_2$ und $x_3$ bezeichnet, dass 
$x_1 < x_2 < x_3$ gilt.
\begin{itemize}
\item F"ur die Extremstellen gilt dann:
\begin{center}
\begin{tabular}{lcccc}
Stelle & Maximal- & Minimal- & lokal & global \\
       & stelle   & stelle   &       &        \\
$x_1 = $\MParsedQuestion{16}{-1}{4} & \MCheckbox{0} & \MCheckbox{1} & \MCheckbox{0} & \MCheckbox{1} \\
$x_2 = $\MParsedQuestion{16}{1}{4} & \MCheckbox{1} & \MCheckbox{0} & \MCheckbox{1} & \MCheckbox{0} \\
$x_3 = $\MParsedQuestion{16}{3}{4} & \MCheckbox{0} & \MCheckbox{1} & \MCheckbox{0} & \MCheckbox{0} \\
\end{tabular}
\end{center}
%
%\item Es ist $x_1 = $\MParsedQuestion{8}{-1}{4} eine 
% \MCheckbox lokale \MCheckbox globale Extremstellen.
%\item Die Minimalstelle \MParsedQuestion{8}{3}{4} ist eine
% \MCheckbox lokale \MCheckbox globale Extremstelle.
\item Das globale Minimum ist \MParsedQuestion{8}{-16}{4}.
\item Wendestellen sind \MParsedQuestion{8}{1 - 2/sqrt(3)}{4} und 
\MParsedQuestion{8}{1 + 2/sqrt(3)}{4}.
\end{itemize}
%Ableitungen
%$f'(x) = 4 (x^3 - 3 x^2 - x + 3) = 4 (x - 1) (x - 3) (x + 1)$
%${f'}'(x) = 4 (3 x^2 - 6 x - 1) %
% = 12 \left(x - 1 + \frac{2}{3}\sqrt{3}\right) %
% \cdot \left(x - 1 - \frac{2}{3}\sqrt{3}\right)$.
\end{MExercise}

%\begin{MExercise}
%F"uhren Sie f"ur die Funktion 
%$f: \R \to \R, x \mapsto \frac{x^5}{2} - \frac{2 x^3}{3}$
%eine Kurvendiskussion durch.
%\end{MExercise}
%
%\begin{MExercise}
%F"uhren Sie f"ur die Funktion 
%$f: \R \to \R, x \mapsto (x^2 - x) \MEU^{-x}$
%eine Kurvendiskussion durch.
%\end{MExercise}
%
%\begin{MExercise}
%Berechnen Sie die Grenzwerte
%\begin{MExerciseItems}
%\item $\displaystyle\lim_{x \to 0} \frac{\sin(3 x)}{\tan(2 x)}$
%\item $\displaystyle\lim_{x \to 0} \frac{1}{\sin(x)} - \frac{3}{2 x}$
%\end{MExerciseItems}
%\end{MExercise}

\begin{MExercise}
Bestimmen Sie das achsenparallele Rechteck kleinsten Umfangs, 
dessen eine Ecke im Nullpunkt $(0; 0)$ und dessen gegen"uberliegende Ecke 
auf dem Graphen von 
$f: (0, \infty)\; \to \R, x \mapsto f(x) := \frac{x^2 - 4}{x^4}$
liegt.

Antwort: Die gesuchte gegen"uberliegende Ecke hat die $x$-Koordinate
\MParsedQuestion{16}{2*sqrt(3)}{4} und die $y$-Koordinate
\MParsedQuestion{16}{1/18}{4}.
\end{MExercise}

\end{MExercises}



%%%Abschnitt
\MSubsection{Zusammenfassung}\MLabel{M07_Zusammenfassung}

%%%Inhalt Zusammenfassung:
%LaTeX-2e-File, Liedtke, 20140731.
%Inhalt: Einf"uhrung der Ableitung, Abschnitt 6: Zusammenfassung.
%zuletzt bearbeitet: 20140930.

%\MSubsection{Zusammenfassung}

%\MSubsubsection{Ableitung}
\begin{MXContent}{Ableitung}{Ableitung}{STD}
Es ist $f: [a, b] \rightarrow \R$ in $x_0 \in (a, b)$ differenzierbar, wenn
\[
m = \lim_{h \rightarrow 0} \frac{f(x_0 + h) - f(x_0)}{h} %%
\]
existiert. Dann hei"st $m$ die Ableitung von $f$ an der Stelle $x_0$,
und wird mit $\frac{\MD f(x_0)}{\MD x} := f'(x) := m$ bezeichnet.

Im Folgenden werden die wichtigsten Rechenregeln und Aussagen zusammengefasst.
\end{MXContent}
 
%\MSubsubsection{Standardableitungen}
\begin{MXContent}{Standardableitungen}{Standardableitungen}{STD}
\begin{tabular}{lp{55mm}l}
Konstante Funktion & $f(x) := a$ & 
 $f'(x) = 0$ \\
Monom & $f(x) := x^n$ f"ur $n \in \Z$ und $n \neq 0$ & 
 $f'(x) = n \cdot x^{n-1}$ \\
Wurzel & $f(x) := x^{\frac{1}{n}}$ f"ur $x \geq 0$ und 
 f"ur $n \in \Z$ und $n \neq 0$ & 
 $f'(x) = n \cdot x^{n-1}$ f"ur $x > 0$ \\
Sinusfunktion & $f(x) := \sin(x)$ & $f'(x) = \cos(x)$ \\
Kosinusfunktion & $f(x) := \cos(x)$ & $f'(x) = -\sin(x)$ \\
Tangensfunktion & $f(x) := \tan(x)$ f"ur $x \neq \frac{\pi}{2} + k \pi$ %
 & $f'(x) = 1 + (\tan(x))^2 = \frac{1}{\cos^2(x)}$ \\
Exponentialfunktion & $f(x) := \exp(x)$ & $f'(x) = \exp(x)$ \\
Logarithmusfunktion & $f(x) := \ln(x)$ f"ur $x > 0$ & $f'(x) = \frac{1}{x}$ %%
\end{tabular}
\end{MXContent}

%\MSubsubsection{Rechenregeln}
\begin{MXContent}{Rechenregeln}{Rechenregeln}{STD}
Die Funktionen $g$, $h$, $u$ und $v$ seien differenzierbar. Dann ist 
auch die nachstehend definierte Funktion $f$ differenzierbar, und 
die Ableitung von $f$ kann wie angegeben berechnet werden:
\begin{description}
\item[Vielfache] F"ur $f(x) := r \cdot g(x)$ ist
 $f'(x) = r \cdot g'(x)$
\item[Summe] F"ur $f(x) := g(x) + h(x)$ ist
  $f'(x) = g'(x) + h'(x)$
\item[Produktregel] F"ur $f(x) := u(x) \cdot v(x)$ ist 
 $f'(x) = u'(x) \cdot v(x) + u(x) \cdot v'(x)$
\item[Quotientenregel] F"ur $f(x) := \frac{u(x)}{v(x)}$ ist 
 $f'(x) = \frac{u'(x) \cdot v(x) - u(x) \cdot v'(x)}{(v(x))^2}$
\item[Kettenregel] F"ur $f(x) := (g \circ u)(x) = g(u(x))$ ist
 $f'(x) = g'(u(x)) \cdot u'(x)$
\end{description}
\end{MXContent}

%\MSubsubsection{Eigenschaften}
\begin{MXContent}{Eigenschaften}{Eigenschaften}{STD}
Eine differenzierbare Funktion $f: [a, b] \rightarrow \R$ ist stetig, und es gilt:
\begin{itemize}
\item Wenn $f'(x) \geq 0$ f"ur alle $x \in (a, b)$ gilt, dann ist $f$ monoton 
wachsend.
\item Wenn $f'(x) \leq 0$ f"ur alle $x \in (a, b)$ gilt, dann ist $f$ monoton 
fallend.
\end{itemize}
Wenn $f$ zweimal differenzierbar ist, gilt zudem (wobei die zweite Ableitung
mit $f^{(2)}$ bezeichnet wird):
\begin{itemize}
\item Wenn $f^{(2)}(x) \geq 0$ f"ur alle $x \in (a, b)$ gilt, dann ist $f$ konvex 
(linksgekr"ummt). 
\item Wenn $f^{(2)}(x) \leq 0$ f"ur alle $x \in (a, b)$ gilt, dann ist $f$ konkav 
(rechtsgekr"ummt). 
\end{itemize}

\begin{MXInfo}{Extremstellen}
Wenn $x_0$ eine Extremstellen in $(a, b)$ ist, dann gilt $f'(x_0) = 0$. Somit 
k"onnen nur solche Stellen der differenzierbaren Funktion $f$ auf $(a, b)$ 
Extremstellen sein.

Wenn $f'(x_0) = 0$ ist und zudem ein Vorzeichenwechsel der Ableitung in $x_0$ 
vorliegt oder die zweite Ableitung ungleich null ist, dann ist $x_0$ eine 
Extremstellen. Au"serdem kann dann festgesetllt werden, ob es sich um eine 
Minimalstelle oder eine Maximalstelle von $f$ handelt:
Wenn $f'(x_0) = 0$ ist, und
\begin{itemize}
\item $f'(x) < 0$ f"ur $x < x_0$ und $f'(x) > 0$ f"ur $x > x_0$ gilt, 
 dann ist $x_0$ eine Minimalstelle von $f$.
\item $f'(x) > 0$ f"ur $x < x_0$ und $f'(x) < 0$ f"ur $x > <_0$ gilt,
 dann ist $x_0$ eine Maximalstelle von $f$.
\end{itemize}

Oder wenn $f'(x_0) = 0$ ist, die zweite Ableitung in $x_0$ existiert und
\begin{itemize}
\item $f^{(2)}(x_0) > 0$ gilt, dann ist $x_0$ eine Minimalstelle von $f$.
\item $f^{(2)}(x_0) < 0$ gilt, dann ist $x_0$ eine Maximalstelle von $f$.
\end{itemize}
\end{MXInfo}

\begin{MXInfo}{Wendestellen}
Wenn $w_0$ eine Wendestellen in $(a, b)$ ist und $f$ zweimal differenzierbar 
ist, muss $f^{(2)}(w_0) = 0$ gelten. Somit gen"ugt es, diese Stellen weiter
zu untersuchen.

Es ist $w_0$ eine Wendestelle in $(a, b)$, wenn  $f^{(2)}(w_0) = 0$ ist und 
eine der folgenden weiteren Eigenschaften gelten:
\begin{itemize}
\item In $w_0$ liegt ein Vorzeichenwechsel der zweiten Ableitung vor (das hei"st,
f"ur kleinere Werte $x < w_0$ ist $f^{(2)}(x) < 0$ und f"ur gr"o"sere Werte 
$x > w_0$ ist $f^{(2)}(x) > 0$ oder umgekehrt).
%
\item Oder es gilt $f^{(3)}(w_0) \neq 0$.
\end{itemize}
Das letzte Kriterium erfordert neben der Bedingung $f^{(2)}(w_0)$ nat"urlich, 
dass die dritte Ableitung existiert und in der Praxis dann erst mal berechnet 
werden muss.
%Hinreichend f"ur eine Wendestelle $w_0$ ist, dass $f^{(2)}(w_0) = 0$ und 
% $f^{(3)}(w_0) \neq 0$ gilt.
\end{MXInfo}
\end{MXContent}

%\MSubsubsection{Anwendungen}
\begin{MXContent}{Anwendungen}{Anwendungen}{STD}
F"ur differenzierbare Funktionen wurden folgende Anwendungen vorgestellt:
\begin{itemize}
%\item Regel von de l'Hospital
\item Kurvendiskussion
\item Optimierungsaufgabe
\end{itemize}
\end{MXContent}

%end of file.





%%%Ausgangstest:
\MSubsection{Ausgangstest}\MLabel{M07_Ausgangstest}

\begin{MIntro}
Der Test umfasst Aufgaben zu den Inhalten dieses Moduls. 
Das unverbindliche Ergebnis ist f"ur Sie eine Erfolgskontrolle.
\end{MIntro}

\begin{MTest}
\begin{MExercise}
In einem Beh"ahlter wird um $9$ Uhr eine Temperatur von $-10^{\circ}{\mathrm{C}}$ 
gemessen. Um $15$ Uhr betr"agt die Temperatur $-58^{\circ}{\mathrm{C}}$.
Nach weiteren vierzehn Stunden ist die Temperatur auf 
$-140^{\circ}{\mathrm{C}}$ gefallen. 
\begin{MExerciseItems}
\item Wie gro"s ist die mittlere "Anderungsrate der Temperatur aufgrund der 
ersten und zweiten Messung?

Antwort: \MParsedQuestion{5}{-8}{3}
%Ergebnis: $(-58 - (-10)) / (15 - 9) = -48 / 6 = -8$.
%
%\item Wodurch dr"uckt sich in der (mittleren) "Anderungsrate aus, dass die 
%Temperatur f"allt?
%
\item In der (mittleren) "Anderungsrate dr"uckt sich die Eigenschaft, dass die
Temperatur f"allt, dadurch aus, dass die "Anderungsrate
{\MQuestion{20}{negativ}} ist.
\MInputHint{Geben Sie ein Adjektiv an.}
%{\MQuestion{20}{kleiner null bzw. negativ}} ist.
%
\item Berechnen Sie die mittlere "Anderungsrate der Temperatur der gesamten
Messdauer, die sich anhand der ersten und letzten Messung ergibt.

Antwort: \MParsedQuestion{5}{-6.5}{3}
%Ergebnis: $(-140 - (-10)) / (29 - 9) = -130 / 20 = -6.5$.
\end{MExerciseItems}
\end{MExercise}

\begin{MExercise}
Zu einer Funktion $f: [-3;2] \to \R$ geh"ort die Ableitung $f'$, deren 
Graph hier gezeichnet ist:

%Bild:
\ifttm
\MUGraphics{\MPfadBilder/jb07A8_ATest_AgAbleitung.png}{scale=1}%
{Ableitung von $f$}{}
%\MUGraphicsSolo{jb07A8_ATest_AgAbleitung.png}{scale=1}{}
\else
\begin{center}
\renewcommand{\jTikZScale}{0.6}
%\input{\MPfadBilder/jb07A8_ATest_AgAbleitung.tex}
%LaTeX-File, Liedtke, 20140825.
%VBKM-Modul 7 Differentialrechnung: Bild Funktionsgraph zum Ausgangstest.
%Bildname: jb07A8_ATest_AgAbleitung.tex.
%Erstellt: 20140825, Liedtke.
%Bearbeitet: 20140826, Liedtke.
%Bearbeitet: 20140829, Liedtke (Dateiname angepasst).
%Bearbeitet: 20140901, Liedtke (Dateiname ohne Endung erfasst).

\tikzsetnextfilename{jb07A8_ATest_AgAbleitung}
\begin{small}
\begin{tikzpicture}[line width=1.5pt,scale=\jTikZScale, %
declare function={
  fkt(\x) = (\x + abs(\x) + 2)/2;
}
] %[every node/.style={fill=white}] 
%Koordinatenachsen:
\draw[->] (-3.6, 0) -- (3, 0) node[below left]{$x$}; %x-Achse
\draw[->] (0, -0.6) -- (0, 3) node[below left]{$y$}; %y-Achse
%Achsenbeschriftung:
\foreach \x in {-3, -2, -1, 1, 2} \draw (\x, 0) -- ++(0, -0.1) %
 node[below] {$\x$}; 
\foreach \y in {1, 2} \draw (0, \y) -- ++(0.2, 0) %
 node[right] {$\y$};
%\node[below left] at (0, 0) {$0$};
%Funktion:
\draw[domain=-3:2,samples=120,color=\jccolorfkt] %
 plot (\x, {fkt(\x)});
\end{tikzpicture}
\end{small}

%end of file
\end{center}
\fi
%Bildende.

\begin{MExerciseItems}
\item Die Funktionswerte von $f$ zwischen $-3$ und $0$ 

\MCheckbox{0} sind konstant,
\MCheckbox{1} nehmen um $3$ zu,
\MCheckbox{0} nehmen ab.

\item Die Funktion $f$ hat an der Stelle $0$

\MCheckbox{0} eine Sprungstelle,
\MCheckbox{0} keine Ableitung,
\MCheckbox{1} die Ableitung $1$.
\end{MExerciseItems}
\end{MExercise}

\begin{MExercise} %Rechenregeln
Berechnen Sie f"ur
\begin{MExerciseItems}
\item $f(x) := \ln\left(x^3 + x^2\right)$ f"ur $x > 0$ die Ableitung
$f'(x) = $\MSimplifyQuestion{30}{(3 x + 2)/(x^2 + x)}{1}{x}{20}{0}
\item $g(x) := x \cdot \MEU^{-x}$ die zweite Ableitung
${g'}'(x) = $\MSimplifyQuestion{30}{(x - 2) * e^(-x)}{10}{x}{4}{0}
\end{MExerciseItems}
\end{MExercise}

\begin{MExercise} %Eigenschaften
In welchen Bereichen ist $f: (0; \infty) \to \R$ monoton fallend, 
in welchen konkav, wenn $f'(x) = x \cdot \ln x$ gilt?
%
Geben Sie als Bereiche m"oglichst gro"se offene Intervalle $(c; d)$ an, 
wobei $\infty$ durch \texttt{infty} angegeben wird:
\begin{MExerciseItems}
\item $f$ ist auf \MIntervalQuestion{12}{(0; 1)}{4} monoton fallend,
\item $f$ ist auf \MIntervalQuestion{12}{(0; 1/e)}{4} konkav.
\end{MExerciseItems}
\end{MExercise}

%\begin{MExercise} %Anwendungen
%Berechnen Sie alle lokalen und globalen Extremstellen und Wendestellen der 
%Funktion $f: \R \to \R$ mit $f(x) := (x^2 - 3 x) \cdot \exp^{2 x}$ f"ur 
%$x \in \R$.
%\end{MExercise}
\end{MTest}


\clearpage
\MPrintIndex

\end{document}

%Dateiende.

