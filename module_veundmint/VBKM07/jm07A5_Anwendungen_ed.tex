%LaTeX-2e-File, Liedtke, 20140731.
%Inhalt: Einf"uhrung in die Differentialrechnung, Abschnitt 5.
%zuletzt bearbeitet: 20140930.

%Abschnitt 5: Anwendungen der Differentialrechnung
%\MSubsubsection{Anwendungen}
%\begin{MContent}


%\MSubsubsection{Grenzwerte}
%%%\begin{MXContent}{Grenzwerte}{Grenzwerte}{STD}
%%%Gegeben ist $a \in \R$, und es sind Funktionen 
%%%$g,h: ]r, s[\; \setminus \{ a \} \to \R$ gegeben.
%%%
%%%Weiter haben die Funktionen $g$ und $h$ die Eigenschaft, dass
%%%$\displaystyle \lim_{x \to a}\; g(x) = 0$ und
%%%$\displaystyle \lim_{x \to a}\; h(x) = 0$ 
%%%gilt.
%%%
%%%Wenn $g$ und $h$ in einer Umgebung von $a$ differenzierbar sind und der 
%%%Grenzwert $\lim_{x \to a}\; \frac{g'(x)}{h'(x)}$ existiert, dann gilt
%%%\begin{equation}
%%%\lim_{x \to a}\; \frac{g(x)}{h(x)}
%%%= \lim_{x \to a}\; \frac{g'(x)}{h'(x)}
%%%\end{equation}
%%%Eine entsprechende Aussage gilt, wenn 
%%%$\displaystyle \lim_{x \to a}\; g(x) = \infty$ und
%%%$\displaystyle \lim_{x \to a}\; h(x) = \infty$ gilt.
%%%
%%%\end{MXContent}
 

%\MSubsubsection{Kurvendiskussion}
\begin{MXContent}{Kurvendiskussion}{Kurvendiskussion}{STD}

Gegeben ist eine differenzierbare Funktion $f: D \to \R$ auf ihrem maximalen 
Definitionsbereich $D$ mit Zuordnungsvorschrift $y = f(x)$ f"ur $x \in D$.

Im ersten Teil werden algebraische und geometrische Aspekte von $f$ betrachtet:
\begin{description}
\item[Maximaler Definitionsbereich]
Es werden alle reellen Zahlen $x$ bestimmt, f"ur die $f(x)$ existiert. Die 
Menge $D$ all dieser Zahlen wird maximaler Definitionsbereich genannt.

\item[Symmetrie des Graphen]
Es wird untersucht, ob $f(-x) = f(x)$ bzw. $f(-x) = -f(x)$ f"ur alle $x \in D$ 
gilt. Im ersten Fall ist der Graph zur $y$-Achse symmetrisch, im zweiten ist 
er zum Nullpunkt $(0;0)$ des Koordinatensystems punktsymmetrisch.

\item[Periodizit"at]
Es wird untersucht, ob es eine Zahl $p > 0$ derart gibt, dass $f(x+p) = f(x)$ 
f"ur alle $x \in D$ gibt.

\item[Schnittpunkte mit den Achsen] Es werden die Schnittpunkte mit den 
Koordinatenachsen bestimmt:
\begin{itemize}
\item $x$-Achse: Es werden alle Nullstellen von $f$ berechnet.
\item $y$-Achse: Wenn $0 \in D$ gilt, wird $f(0)$ berechnet.
\end{itemize}
\item[Asymptotisches Verhalten an den R"andern des Definitionsbereichs]
Es wird das Verhalten in der N"ahe von Definitionsl"ucken und an den R"andern
untersucht: Gibt es eine stetige Fortsetzung oder existieren zumindest die 
einseitigen Grenzwerte? Sind die Grenzwerte gegen $\infty$ bzw. $-\infty$ 
vorhanden, wenn der Definitionsbereich entsprechend unbeschr"ankt ist?
\end{description} 

Im zweiten Teil wird die Funktion mittels Folgerungen aus der Ableitung 
analytisch untersucht. Dazu werden zun"achst die erste und zweite Ableitung 
berechnet, sofern diese existieren.
\begin{description}
\item[Ableitungen]
Berechnung der ersten und zweiten Ableitung (soweit vorhanden).

\item[Extemstellen und Monotonie]
%Wenn $x_0$ eine Extremstellen in $(a, b)$ ist, dann gilt $f'(x_0) = 0$.
Eine notwendige Bedingung f"ur eine Extremstellen $x_0$ im Innern des 
Definitionsbereichs ist $f'(x_0) = 0$.

Ob tats"achlich eine Extremstelle vorliegt, muss dann noch gepr"uft werden.
Dies ist oft mit einem der folgenden Kriterien m"oglich:

Unter der Voraussetzung, dass $f'(x_0) = 0$ gilt, liegt eine
\begin{itemize}
\item Minimalstelle in $x_0$ vor, wenn in einer Umgebung von $x_0$ f"ur $x < x_0$ 
dann $f'(x) < 0$ und f"ur $x > x_0$ dann $f'(x) > 0$ ist,
\item Maximalstelle in $x_0$ vor, wenn in einer Umgebung von $x_0$ f"ur $x < x_0$ 
dann $f'(x) > 0$ und f"ur $x > x_0$ dann $f'(x) < 0$ ist.
\end{itemize}

Wenn auch die zweite Ableitung $f^{(2)} = {f'}'$ existiert, gilt:
Wenn $f'(x_0) = 0$ und zudem 
\begin{itemize}
\item $f^{(2)}(x_0) > 0$ gilt, dann ist $x_0$ eine Minimalstelle von $f$.

\item $f^{(2)}(x_0) < 0$ gilt, dann ist $x_0$ eine Maximalstelle von $f$.
\end{itemize}
Die Funktion $f$ ist auf den Intervallen des Defitionsbereichs monoton
wachend, wo $f'(x) \geq 0$ gilt. Sie ist monoton fallend, wenn $f'(x) \leq 0$ 
gilt.

\item[Wendestellen und Kr"ummungseigenschaften]
%Wenn $x_0$ eine Wendestellen in $(a, b)$ ist, dann gilt $f''(x_0) = 0$.
Wenn die zweite Ableitung ${f'}' = f^{(2)}$ existiert, ist $f^{(2)}(w) = 0$
eine notwendige Bedingung daf"ur, dass $w_0$ eine Wendestelle im Innern des 
Definitionsbereichs sein kann.

Wiederum ist noch zu pr"ufen, ob tats"achlich eine Wendestelle vorliegt.

Wie f"ur Extremstellen kann dies anhand eines Vorzeichenwechsels der zweiten
Ableitung erfolgen oder mit Hilfe der dritten Ableitung:

Unter der Voraussetzung, dass $f^{(2)}(w_0) = 0$ gilt, liegt eine
Wendestelle in $w_0$ vor, wenn in einer Umgebung von $w_0$
\begin{itemize}
\item f"ur $x < w_0$ dann $f^{(2)}(x) < 0$ und f"ur $x > w_0$ dann 
$f^{(2)}(x) > 0$ ist, oder
\item f"ur $x < w_0$ dann $f^{(2)}(x) > 0$ und f"ur $x > w_0$ dann 
$f^{(2)}(x) < 0$ ist.
\end{itemize}

Wenn $f^{(2)}(w_0) = 0$ und $f^{(3)}(w_0) \neq 0$ gilt, dann ist $w_0$ eine 
Wendestelle.
%Hinreichend f"ur eine Wendestelle ist, dass $f^{(2)}(x_0) = 0$ und 
% $f^{(3)}(x_0) \neq 0$ gilt.
Der Vorteil des ersten Kriteriums besteht darin, dass die dritte Ableitung 
nicht berechnet werden muss. 

Die Funktion $f$ ist auf den Intervallen des Definitionsbereichs konvex
(linksgekr"ummt), wo $f^{(2)}(x) \geq 0$ gilt. Sie ist konkav (rechtsgekr"ummt), 
wenn $f^{(2)}(x) \leq 0$ gilt.

\item[Graph]
Anhand der gewonnenen Ergebnisse wird dann ein charakteristischer Ausschnitt des 
Funktionsgraphen gezeichnet, der diese Ergebnisse ber"ucksichtigt.
\end{description}

Hier nochmals die einzelnen Betrachtungen in einer kurzen "Ubersicht:

\begin{MXInfo}{Kurvendiskussion}
Unter einer Kurvendiskussion wird hier die Untersuchung einer differenzierbaren
reellen Funktion $f: D \to \R$ verstanden, die durch einen Funktionsterm
$y = f(x)$ gegeben ist, und folgende Aspekte umfasst:
\begin{center}
\begin{minipage}{0.40\textwidth}
\begin{itemize}
\item Maximaler Definitionsbereich $D_f$
\item Symmetrie
 und Periodizit"at
\item Schnittpunkte mit den Achsen
\item Randverhalten
\end{itemize}
\end{minipage}
%
\begin{minipage}{0.58\textwidth}
\begin{itemize}
\item Ableitungen
\item Extremstellen und Monotoniebereiche
\item Wendestellen und Kr"ummungseigenschaften
%\item Charakteristischer Ausschnitt des Graphen
\item Graph der Funktion
\end{itemize}
\end{minipage}
\end{center}
\end{MXInfo}

Im folgenden Beispiel wird die Vorgehensweise illustriert.

\begin{MExample}
Es wird $f: D_f \to \R$ mit $f(x) = \frac{x^5}{5} - x^3$ auf seinem maximalen 
Definitionsbereich $D_f \subseteq \R$ untersucht.
\begin{description}
\item[Maximaler Definitionsbereich]
Der Funktionsterm ist f"ur alle reellen Zahlen $x$ definiert, sodass 
$D_f = \R$ ist.
\item[Symmetrie des Graphen]
Wegen $f(-x) = \frac{1}{5} (-x)^5 - (-x)^3 %
 = - \left(\frac{1}{5} x^5 - x^3\right) = -f(x)$ ist der Graph von $f$ 
dann punktsymmetrisch zum Nullpunkt $(0, 0)$, die Funktion $f$ ungerade. 
(Da $f$ ein Polynom ist
und $(-x)^{2k} = (-1)^{2k} x^{2k} = x^{2k}$ sowie 
$(-x)^{2k+1} = (-1)^{2k+1} x^{2k+1} = -x^{2k+1}$ f"ur alle $k \in \No$ gilt, 
kann die Antwort auch an den Exponenten der Monome abgelesen werden. 
Im Beispiel sind es die ungeraden Zahlen $5$ und $3$, sodass $f$ ungerade ist.)

Aufgrund der Symmetrie w"urde es f"ur die meisten "Uberlegungen in der 
Kurvendiskussion gen"ugen, diese f"ur $x \geq 0$ vorzunehmen. 
In diesem Beispiel soll diese Eigenschaft einmal nicht von vornherein benutzt 
werden, um eine m"oglichst ausf"uhrliche Darstellung zu bieten. 

\item[Schnittpunkte mit den Koordinatenachsen]
Es ist $f(x) = \frac{1}{5} x^3 \cdot (x^2 - 5)$. Somit sind $u_0 = 0$ und 
$u_1 = -\sqrt{5}$ sowie $u_2 = \sqrt{5}$ die Nullstellen von $f$.


\item[Randverhalten]
Als nicht konstantes Polynom ist $f$ unbeschr"ankt. Wegen $\frac{1}{5} > 0$ ist
$\displaystyle\lim_{x \to \infty} f(x) = \infty$, und aufgrund dessen, dass der 
Grad von $f$ ungerade ist, dann $\displaystyle\lim_{x \to -\infty} f(x) = \infty$. 

\item[Ableitungen]
F"ur $x \in \R$ gilt
$f'(x) = x^4 - 3 x^2$ und $f^{(2)}(x) = 4 x^3 - 6 x$.

\item[Extremstellen]
Die Nullstellen von $f'(x) = x^2 (x^2 - 3)$ sind $x_0 = 0$, $x_1 = -\sqrt{3}$ 
und $x_2 = \sqrt{3}$.
Mit Hilfe folgender Tabelle wird bestimmt, in welchen Bereichen die Ableitung 
von $f$ positiv bzw. negativ ist, woraus sich dann die Monotoniebereiche von 
$f$ ergeben. Der Eintrag $+$ besagt, dass der betrachtete \emph{Faktor} von 
$f'(x)$ im angegebenen Intervall positiv ist. Wenn er negativ ist, wird $-$ 
eingetragen.
\ifttm
\[
\begin{array}{ccccc}
x & x < -\sqrt{3} & -\sqrt{3} < x < 0 & 0 < x < \sqrt{3} & \sqrt{3} < x  \\
\hline
x^2          & + & + & + & + \\
x + \sqrt{3} & - & + & + & + \\
x - \sqrt{3} & - & - & - & + \\
f'(x)        & + & + & - & + \\
f \mbox{\ monoton} & \mbox{wachsend} & \mbox{fallend} & \mbox{fallend} & \mbox{wachsend} %%
\end{array}
\]
\else
\[
\begin{array}{ccccc}
x & x < -\sqrt{3} & -\sqrt{3} < x < 0 & 0 < x < \sqrt{3} & \sqrt{3} < x  \\
\hline
x^2          & + & + & + & + \\
x + \sqrt{3} & - & + & + & + \\
x - \sqrt{3} & - & - & - & + \\
f'(x)        & + & + & - & + \\
f \text{ monoton} & \text{wachsend} & \text{fallend} & \text{fallend} & \text{wachsend} %%
\end{array}
\]
\fi
Somit ist $x_1 = -\sqrt{3}$ eine lokale Maximalstelle und $x_2 = \sqrt{3}$ eine
lokale Minimalstelle.

\item[Wendestellen]
Die Nullstellen von 
$f^{(2)}(x) = 4 x^3 - 6 x = 4 x \left(x^2 - \frac{3}{2}\right)$
sind $w_0 = 0$ und $w_1 = -\sqrt{\frac{3}{2}}$ sowie
$w_2 = \sqrt{\frac{3}{2}}$.

Es handelt sich auch um Wendestellen, da es einfache Nullstellen der zweiten
Ableitung $f^{(2)}$ sind und damit ein Vorzeichenwechsel vorliegt 
(oder da $f^{(3)}(x) = 12 x^2 - 6$ dort jeweils ungleich null ist: Es gilt 
n"amlich $f^{(3)}(w_0) = - 6 \neq 0$ und
$f^{(3)}(w_1) = f^{(3)}(w_2) = 12 \cdot \frac{3}{2} - 6 = 12 \neq 0$).
Wegen $f'(w_0) = 0$ ist somit $(w_0; f(w_0)) = (0; 0)$ ein Sattelpunkt.

\item[Funktionsgraph]
Mit obigen Ergebnissen kann dann ein Ausschnitt des 
Funktionsgraphen von $f$ f"ur $-2.5 \leq x \leq 2.5$ gezeichnet werden.
\ifttm
%\begin{center}
\MUGraphics{\MPfadBilder/jb07A5_BspKurvendiskussion.png}{scale=0.5}%
{Graph von $f(x) = \frac{1}{5} x^5 - x^3$ f"ur $-2.5 \leq x \leq 2.5$}{}
%\end{center}
\else
\begin{center}
%\input{jb07A5_BspKurvendiskussion.tex}
\begin{small}
\renewcommand{\jTikZScale}{0.6}
\tikzsetnextfilename{jb07A5_BspKurvendiskussion}
\begin{tikzpicture}[line width=1.5pt,scale=\jTikZScale, %
declare function={
  fkt(\x) = (1/5*\x*\x - 1)*(\x)*(\x)*(\x);
}
] %[every node/.style={fill=white}] 
%Koordinatenachsen:
\draw[->] (-5, 0) -- (5, 0) node[below left]{$x$}; %x-Achse
\draw[->] (0, -4.4) -- (0, 5) node[below left]{$y$}; %y-Achse
%Achsenbeschriftung:
\foreach \x in {-4, -3, -2, -1} \draw (\x, 0) -- ++(0, -0.1) %
 node[below] {$\x$}; 
\foreach \x in {1, 2, 3, 4} \draw (\x, 0) -- ++(0, 0.1) %
 node[above] {$\x$}; 
\foreach \y in {-4, -3, -2, -1} \draw (0, \y) -- ++(-0.1, 0) %
 node[left] {$\y$};
\foreach \y in {1, 2, 3, 4} \draw (0, \y) -- ++(-0.1, 0) %
 node[left] {$\y$};
%\node[below left] at (0, 0) {$0$};
%Funktion:
\draw[domain=-2.5:2.5,samples=120,color=\jccolorfkt] %
 plot (\x, {fkt(\x)});
\end{tikzpicture}
\end{small}
\end{center}
\fi
\end{description}
\end{MExample}
\end{MXContent}


%\MSubsubsection{Optimierungsaufgaben}
\begin{MXContent}{Optimierungsaufgaben}{Optimierungsaufgaben}{STD}

In der Optimierung wird in einer vorgegebenen Schar von L"osungen einer 
Aufgabe diejenige gesucht, die eine vorab festgelegte Eigenschaft am besten 
erf"ullt.

Als Beispiel wird die Aufgabe betrachtet, eine zylinderf"ormige Dose zu 
konstruieren. Die Dose soll zus"atzlich die Bedingung erf"ullen, ein 
Fassungsverm"ogen von einem Liter zu haben. Sind $r$ der Radius und $h$ die
H"ohe der Dose, so soll also $\pi r^2 \cdot h = 1$ sein (auf die physikalischen
Einheiten wurde der mathematischen Einfachheit halber verzichtet -- in der 
Praxis ist es allerdings oft hilfreich, die Einheiten 
%physikalisch richtig auch 
zur Kontrolle der Ergebnisse mit aufzuschreiben). 

Gesucht wird nach derjenigen Dose, die eine m"oglichst kleine Oberfl"ache 
$O = 2 \cdot \pi r^2 + 2 \pi r h$ hat. Dies bedeutet, dass in 
dieser vereinfachten Betrachtung auch wenig Material in der Herstellung 
ben"otigt wird.

Mathematisch formuliert f"uhrt die Aufgabe auf die Suche nach einem 
Minimum f"ur die Funktion $O$ der Oberfl"ache, wobei das Minimum nur 
unter den Werten f"ur $r$ und $h$ gesucht wird, f"ur die auch die Bedingung 
"uber das Volumen $\pi r^2 \cdot h = 1$ erf"ullt ist.

Eine solche zus"atzliche Bedingung bei der Suche nach Extremstellen wird 
auch Nebenbedingung genannt. Sie kann ebenfalls mit einer Funktion 
formuliert werden.

Wird n"amlich $g(r, h) := \pi r^2 \cdot h$ gesetzt, dann besagt obige 
Bedingung, dass nur solche Paare von $r$ und $h$ betrachtet werden, f"ur die 
$g(r, h) = 1$ ist.

Auf diese Weise ergibt sich mit Hilfe zweier Funktionen, die von mehreren
Variablen abh"angen k"onnen, eine einfache Formulierung einer 
Optimierungsaufgabe. 

\begin{MXInfo}{Optimierungsaufgabe}
In einer \MEntry{Optimierungsaufgabe}{Optimierungsaufgabe} wird eine
Extremstelle $x_{\text{ext}}$ einer Funktion $f$ gesucht, die eine 
gegebene Gleichung $g(x_{\text{ext}}) = b$ erf"ullt.

Wenn ein globales Minimum gesucht wird, spricht man auch von einer 
\MEntry{Minimierungsaufgabe}{Minimierungsaufgabe}. Wenn ein Maximum gesucht
wird, hei"st die Optimierungsaufgabe eine 
\MEntry{Maximierungsaufgabe}{Maximierungsaufgabe}.

Die Funktion $f$ hei"st \MEntry{Zielfunktion}{Zielfunktion}, und die 
Gleichung $g(x) = b$ wird \MEntry{Nebenbedingung}{Nebenbedingung} der 
Optimierungsaufgabe genannt.
\end{MXInfo}

Im obigen Beispiel ist $O$ die Zielfunktion der Minimierungsaufgabe, die 
unter der Nebenbedingung $g(r, h) = 1$ gel"ost werden soll. Da sich die 
durch $g(r, h) = \pi r^2 \cdot h = 1$ gegebene Gleichung nach $r$ oder nach 
$h$ aufl"osen l"asst, kann die Funktion $O = O(r, h)$ in eine Funktion einer 
Variablen "uberf"uhrt werden, sodass die Frage nach dem Minimum mit den Mitteln 
der Kurvendiskussion %von Funktionen einer reellen Variablen 
bestimmt werden kann.

Aufl"osen nach $h$ f"uhrt auf $h = \frac{1}{\pi r^2}$. Eingesetzt in $O$ 
ergibt sich, dass eine (globale) Minimalstelle von
\[
f(r) := O(r, h) = 2 \cdot \pi r^2 + \frac{2 \pi r}{\pi r^2} %
 = 2 \pi r^2 + \frac{2}{r} %%
\]
f"ur $r > 0$ gesucht wird (da der Radius $r$ der Dose positiv ist).
Die notwendige Bedingung $f'(x) = 0$ f"uhrt auf
\[
0 = f'(x) = 4 \pi r - \frac{2}{r^2}, %%
\]
woraus $r^3 = \frac{1}{2 \pi}$ folgt. Wegen $r > 0$ ist dann
\[
R = \sqrt[3]{\frac{1}{2 \pi}} %%
\]
die einzige L"osung der Gleichung. Da $f$ nach oben unbeschr"ankt ist 
(an den R"andern f"ur $r \to 0$ bzw. $r \to \infty$ gegen {\glqq}unendlich 
strebt{\grqq}), ist $R$ als einzige Extremstelle die Minimalstelle von $f$.
%
Die zugeh"orige H"ohe $H$ der Dose mit Radius $R$ ist dann
$H = \sqrt[3]{\frac{4}{\pi}}$.

Abschliesend soll noch auf eine Besonderheit dieser Dose hingewiesen werden:
F"ur den Durchmesser $D$ der Dose gilt
\[
D = 2 R = \sqrt[3]{8} \cdot \sqrt[3]{\frac{1}{2 \pi}} %
 = \sqrt[3]{\frac{8}{2 \pi}} = H. %%
\]
Die H"ohe der Dose ist gleich dem Durchmesser, sodass die Dosen in 
w"urfelf"ormige Kisten verpackt werden k"onnen.
\end{MXContent}

%\end{MContent}

%end of file.

