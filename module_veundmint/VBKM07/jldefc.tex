%LaTeX-File, julie, 20110427
%content: LaTeX-Definitionen
%changed: 20121029.
%changed: 20121128.
%changed: 20130218: Befehl \jintlimitsfdx.
%changed: 20130227: Befehle f"ur Funktionsbezeichner (partielle Integration).
%changed: 20130322: Befehle f"ur variable Bildnamen (partielle Integration).
%changed: 20130819: Neue Befehlsdefinitionen f"ur physikalischer Gr"o"sen.

%Besonderheiten zur Erstellung von HTML-Texten:
\newcommand{\jfootnote}[1]{\ifttm\begin{MHint}{Anmerkung}{#1}\end{MHint}%
\else\footnote{#1}}


%Listen
%Technische Anpassung (sonst gibt es Auswirkungen auf andere Module):
%\renewcommand{\labelenumi}{\textbf{(\alph{enumi})}}
%\renewcommand{\labelenumii}{\textbf{(\roman{enumii})}}

%Bilder
\newif\ifGraphikExtern
%\graphicspath{{D:/3mathe/Bilder/}}
\graphicspath{{../Bilder/}}

%Modul Integrationstechniken:
%Funktionsbezeichungen zur partiellen Integration 
\newcommand{\jBildVariableFktu}{u(x)}
\newcommand{\jBildVariableFktAblv}{v'(x)}
%
\newcommand{\jBildVariableFktAblu}{u'(x)}
\newcommand{\jBildVariableFktv}{v(x)}
%
\newcommand{\jBildPartielleIntegrationuvsBezierFktvariabel}%
{BildPartielleIntegrationuvsBezierFktvariabel}
\newcommand{\jBildPartielleIntegrationusvBezierFktvariabel}%
{BildPartielleIntegrationusvBezierFktvariabel}


%Library for PGF and TikZ:
\usepgflibrary{patterns}
\usetikzlibrary{patterns}



%Mathematik
\newcommand{\jds}{\displaystyle}


\newcommand{\mnumeq}{\approx} % numerisch "gleich"

\newcommand{\jKomposition}{\circ}

\newcommand{\jlimes}{\rightarrow}

\newcommand{\jmenge}{\; : \;}

\newcommand{\jGrad}{\mathrm{Grad}}
\newcommand{\jlGaussfkt}[1]{\left\lfloor #1 \right\rfloor}
\newcommand{\jexp}{\mathrm{e}}

\newcommand{\jFkt}[3][f]{#1\colon #2 \to #3}
\newcommand{\jreelleFkt}[2]{#1\colon #2 \to \R}

\newcommand{\jAbl}[2][']{{#2}^{#1}}
\newcommand{\jAbln}[2][n]{{#2}^{(#1)}}
\newcommand{\jAblOpdfx}{\frac{\mathrm{d}}{\mathrm{d}x} f} %
\newcommand{\jAblOpn}[3][\relax]{\frac{\mathrm{d^{#1}}}{\mathrm{d}{#2}^{#1}} #3} %

\newcommand{\jint}[2]{\int_{#1}{#2} \!\! }
\newcommand{\jintS}[2]{\int {#1} \, \MD{#2}}
\newcommand{\jintfdx}[4]{\int_{#1}^{#2} \!\! #3 \, \MD{#4}}
\newcommand{\jintlimitsfdx}[4]{\int\limits_{#1}^{#2} \!\! #3 \, \MD{#4}}
\newcommand{\jtintfdx}[4]{\int_{#1}^{#2} #3 \, \MD{#4}}

%\newcommand{\jIntStammD}[3]{\left. #1 \right|_{#2}^{#3}}
\newcommand{\jIntStammD}[3]{\left[ #1 \right]_{#2}^{#3}}

\newcommand{\jlpunktierteUmg}{\dot{U}_r(a)}


%Abbildungen

\newcommand{\Abbildung}[5][extgraphic]{%
%\ifGraphikExtern
\MUGraphics{#2}{scale=0.5}{#3\MLabel{#4}}{}
}

%Physikalische Gr"o"sen:

%Abw"artskompatibilit"at:
%Ma"sstab, Ma"seinheit im mathematischen Modus
%Beispiel: 5 m : $5 \jms{m}$.
\newcommand{\jms}[1]{\, \mathrm{#1}} 

%Physikalische Gr"o"sen:
% Zahlwert, Standard: mit Dezimalpunkt
\newcommand{\jmzahl}[3][{.}]{#2{#1}#3} 
% Ma"seinheit, Standardabstand: \,
\newcommand{\jmeinheit}[2][\,]{#1 \mathrm{#2}} 
% Exponent zur Basis 10 in der Exponentialschreibweise, 
% Standardmalzeichen: \times
\newcommand{\jmexponent}[2][\times]{#1 10^{#2}} 


%end of file

