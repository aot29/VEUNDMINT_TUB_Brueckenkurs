%LaTeX-2e-Datei, Liedtke, 20140729.
%Inhalt: Modul 7 VEMINT-Basiskurs Mathematik.
%Thema:  Differentialrechnung.
%zuletzt bearbeitet: 20140922.

% MINTMOD Version P0.1.0, needs to be consistent with preprocesser object in tex2x and MPragma-Version at the end of this file

% Parameter aus Konvertierungsprozess (PDF und HTML-Erzeugung wenn vom Konverter aus gestartet) werden hier eingefuegt, Preambleincludes werden am Schluss angehaengt

\newif\ifttm                % gesetzt falls Uebersetzung in HTML stattfindet, sonst uebersetzung in PDF

% Wahl der Notationsvariante ist im PDF immer std, in der HTML-Uebersetzung wird vom Konverter die Auswahl modifiziert
\newif\ifvariantstd
\newif\ifvariantunotation
\variantstdtrue % Diese Zeile wird vom Konverter erkannt und ggf. modifiziert, daher nicht veraendern!


\def\MOutputDVI{1}
\def\MOutputPDF{2}
\def\MOutputHTML{3}
\newcounter{MOutput}

\ifttm
\usepackage{german}
\usepackage{array}
\usepackage{amsmath}
\usepackage{amssymb}
\usepackage{amsthm}
\else
\documentclass[ngerman,oneside]{scrbook}
\usepackage{etex}
\usepackage[latin1]{inputenc}
\usepackage{textcomp}
\usepackage[ngerman]{babel}
\usepackage[pdftex]{color}
\usepackage{xcolor}
\usepackage{graphicx}
\usepackage[all]{xy}
\usepackage{fancyhdr}
\usepackage{verbatim}
\usepackage{array}
\usepackage{float}
\usepackage{makeidx}
\usepackage{amsmath}
\usepackage{amstext}
\usepackage{amssymb}
\usepackage{amsthm}
\usepackage[ngerman]{varioref}
\usepackage{framed}
\usepackage{supertabular}
\usepackage{longtable}
\usepackage{maxpage}
\usepackage{tikz}
\usepackage{tikzscale}
\usepackage{tikz-3dplot}
\usepackage{bibgerm}
\usepackage{chemarrow}
\usepackage{polynom}
%\usepackage{draftwatermark}
\usepackage{pdflscape}
\usetikzlibrary{calc}
\usetikzlibrary{through}
\usetikzlibrary{shapes.geometric}
\usetikzlibrary{arrows}
\usetikzlibrary{intersections}
\usetikzlibrary{decorations.pathmorphing}
\usetikzlibrary{external}
\usetikzlibrary{patterns}
\usetikzlibrary{fadings}
\usepackage[colorlinks=true,linkcolor=blue]{hyperref} 
\usepackage[all]{hypcap}
%\usepackage[colorlinks=true,linkcolor=blue,bookmarksopen=true]{hyperref} 
\usepackage{ifpdf}

\usepackage{movie15}

\setcounter{tocdepth}{2} % In Inhaltsverzeichnis bis subsection
\setcounter{secnumdepth}{3} % Nummeriert bis subsubsection

\setlength{\LTpost}{0pt} % Fuer longtable
\setlength{\parindent}{0pt}
\setlength{\parskip}{8pt}
%\setlength{\parskip}{9pt plus 2pt minus 1pt}
\setlength{\abovecaptionskip}{-0.25ex}
\setlength{\belowcaptionskip}{-0.25ex}
\fi

\ifttm
\newcommand{\MDebugMessage}[1]{\special{html:<!-- debugprint;;}#1\special{html:; //-->}}
\else
%\newcommand{\MDebugMessage}[1]{\immediate\write\mintlog{#1}}
\newcommand{\MDebugMessage}[1]{}
\fi

\def\MPageHeaderDef{%
\pagestyle{fancy}%
\fancyhead[r]{(C) VE\&MINT-Projekt}
\fancyfoot[c]{\thepage\\--- CCL BY-SA 3.0 ---}
}


\ifttm%
\def\MRelax{}%
\else%
\def\MRelax{\relax}%
\fi%

%--------------------------- Uebernahme von speziellen XML-Versionen einiger LaTeX-Kommandos aus xmlbefehle.tex vom alten Kasseler Konverter ---------------

\newcommand{\MSep}{\left\|{\phantom{\frac1g}}\right.}

\newcommand{\ML}{L}

\newcommand{\MGGT}{\mathrm{ggT}}


\ifttm
% Verhindert dass die subsection-nummer doppelt in der toccaption auftaucht (sollte ggf. in toccaption gefixt werden so dass diese Ueberschreibung nicht notwendig ist)
\renewcommand{\thesubsection}{}
% Kommandos die ttm nicht kennt
\newcommand{\binomial}[2]{{#1 \choose #2}} %  Binomialkoeffizienten
\newcommand{\eur}{\begin{html}&euro;\end{html}}
\newcommand{\square}{\begin{html}&square;\end{html}}
\newcommand{\glqq}{"'}  \newcommand{\grqq}{"'}
\newcommand{\nRightarrow}{\special{html: &nrArr; }}
\newcommand{\nmid}{\special{html: &nmid; }}
\newcommand{\nparallel}{\begin{html}&nparallel;\end{html}}
\newcommand{\mapstoo}{\begin{html}<mo>&map;</mo>\end{html}}

% Schnitt und Vereinigungssymbole von Mengen haben zu kleine Abstaende; korrigiert:
\newcommand{\ccup}{\,\!\cup\,\!}
\newcommand{\ccap}{\,\!\cap\,\!}


% Umsetzung von mathbb im HTML
\renewcommand{\mathbb}[1]{\begin{html}<mo>&#1opf;</mo>\end{html}}
\fi

%---------------------- Strukturierung ----------------------------------------------------------------------------------------------------------------------

%---------------------- Kapselung des sectioning findet auf drei Ebenen statt:
% 1. Die LateX-Befehl
% 2. Die D-Versionen der Befehle, die nur die Grade der Abschnitte umhaengen falls notwendig
% 3. Die M-Versionen der Befehle, die zusaetzliche Formatierungen vornehmen, Skripten starten und das HTML codieren
% Im Modultext duerfen nur die M-Befehle verwendet werden!

\ifttm

  \def\Dsubsubsubsection#1{\subsubsubsection{#1}}
  \def\Dsubsubsection#1{\subsubsection{#1}\addtocounter{subsubsection}{1}} % ttm-Fehler korrigieren
  \def\Dsubsection#1{\subsection{#1}}
  \def\Dsection#1{\section{#1}} % Im HTML wird nur der Sektionstitel gegeben
  \def\Dchapter#1{\chapter{#1}}
  \def\Dsubsubsubsectionx#1{\subsubsubsection*{#1}}
  \def\Dsubsubsectionx#1{\subsubsection*{#1}}
  \def\Dsubsectionx#1{\subsection*{#1}}
  \def\Dsectionx#1{\section*{#1}}
  \def\Dchapterx#1{\chapter*{#1}}

\else

  \def\Dsubsubsubsection#1{\subsubsection{#1}}
  \def\Dsubsubsection#1{\subsection{#1}}
  \def\Dsubsection#1{\section{#1}}
  \def\Dsection#1{\chapter{#1}}
  \def\Dchapter#1{\title{#1}}
  \def\Dsubsubsubsectionx#1{\subsubsection*{#1}}
  \def\Dsubsubsectionx#1{\subsection*{#1}}
  \def\Dsubsectionx#1{\section*{#1}}
  \def\Dsectionx#1{\chapter*{#1}}

\fi

\newcommand{\MStdPoints}{4}
\newcommand{\MSetPoints}[1]{\renewcommand{\MStdPoints}{#1}}

% Befehl zum Abbruch der Erstellung (nur PDF)
\newcommand{\MAbort}[1]{\err{#1}}

% Prefix vor Dateieinbindungen, wird in der Baumdatei mit \renewcommand modifiziert
% und auf das Verzeichnisprefix gesetzt, in dem das gerade bearbeitete tex-Dokument liegt.
% Im HTML wird es auf das Verzeichnis der HTML-Datei gesetzt.
% Das Prefix muss mit / enden !
\newcommand{\MDPrefix}{.}

% MRegisterFile notiert eine Datei zur Einbindung in den HTML-Baum. Grafiken mit MGraphics werden automatisch eingebunden.
% Mit MLastFile erhaelt man eine Markierung fuer die zuletzt registrierte Datei.
% Diese Markierung wird im postprocessing durch den physikalischen Dateinamen ersetzt, aber nur den Namen (d.h. \MMaterial gehoert noch davor, vgl Definition von MGraphics)
% Parameter: Pfad/Name der Datei bzw. des Ordners, relativ zur Position des Modul-Tex-Dokuments.
\ifttm
\newcommand{\MRegisterFile}[1]{\addtocounter{MFileNumber}{1}\special{html:<!-- registerfile;;}#1\special{html:;;}\MDPrefix\special{html:;;}\arabic{MFileNumber}\special{html:; //-->}}
\else
\newcommand{\MRegisterFile}[1]{\addtocounter{MFileNumber}{1}}
\fi

% Testen welcher Uebersetzer hier am Werk ist

\ifttm
\setcounter{MOutput}{3}
\else
\ifx\pdfoutput\undefined
  \pdffalse
  \setcounter{MOutput}{\MOutputDVI}
  \message{Verarbeitung mit latex, Ausgabe in dvi.}
\else
  \setcounter{MOutput}{\MOutputPDF}
  \message{Verarbeitung mit pdflatex, Ausgabe in pdf.}
  \ifnum \pdfoutput=0
    \pdffalse
  \setcounter{MOutput}{\MOutputDVI}
  \message{Verarbeitung mit pdflatex, Ausgabe in dvi.}
  \else
    \ifnum\pdfoutput=1
    \pdftrue
  \setcounter{MOutput}{\MOutputPDF}
  \message{Verarbeitung mit pdflatex, Ausgabe in pdf.}
    \fi
  \fi
\fi
\fi

\ifnum\value{MOutput}=\MOutputPDF
\DeclareGraphicsExtensions{.pdf,.png,.jpg}
\fi

\ifnum\value{MOutput}=\MOutputDVI
\DeclareGraphicsExtensions{.eps,.png,.jpg}
\fi

\ifnum\value{MOutput}=\MOutputHTML
% Wird vom Konverter leider nicht erkannt und daher in split.pm hardcodiert!
\DeclareGraphicsExtensions{.png,.jpg,.gif}
\fi

% Umdefinition der hyperref-Nummerierung im PDF-Modus
\ifttm
\else
\renewcommand{\theHfigure}{\arabic{chapter}.\arabic{section}.\arabic{figure}}
\fi

% Makro, um in der HTML-Ausgabe die zuerst zu oeffnende Datei zu kennzeichnen
\ifttm
\newcommand{\MGlobalStart}{\special{html:<!-- mglobalstarttag -->}}
\else
\newcommand{\MGlobalStart}{}
\fi

% Makro, um bei scormlogin ein pullen des Benutzers bei Aufruf der Seite zu erzwingen (typischerweise auf der Einstiegsseite)
\ifttm
\newcommand{\MPullSite}{\special{html:<!-- pullsite //-->}}
\else
\newcommand{\MPullSite}{}
\fi

% Makro, um in der HTML-Ausgabe die Kapiteluebersicht zu kennzeichnen
\ifttm
\newcommand{\MGlobalChapterTag}{\special{html:<!-- mglobalchaptertag -->}}
\else
\newcommand{\MGlobalChapterTag}{}
\fi

% Makro, um in der HTML-Ausgabe die Konfiguration zu kennzeichnen
\ifttm
\newcommand{\MGlobalConfTag}{\special{html:<!-- mglobalconfigtag -->}}
\else
\newcommand{\MGlobalConfTag}{}
\fi

% Makro, um in der HTML-Ausgabe die Standortbeschreibung zu kennzeichnen
\ifttm
\newcommand{\MGlobalLocationTag}{\special{html:<!-- mgloballocationtag -->}}
\else
\newcommand{\MGlobalLocationTag}{}
\fi

% Makro, um in der HTML-Ausgabe die persoenlichen Daten zu kennzeichnen
\ifttm
\newcommand{\MGlobalDataTag}{\special{html:<!-- mglobaldatatag -->}}
\else
\newcommand{\MGlobalDataTag}{}
\fi

% Makro, um in der HTML-Ausgabe die Suchseite zu kennzeichnen
\ifttm
\newcommand{\MGlobalSearchTag}{\special{html:<!-- mglobalsearchtag -->}}
\else
\newcommand{\MGlobalSearchTag}{}
\fi

% Makro, um in der HTML-Ausgabe die Favoritenseite zu kennzeichnen
\ifttm
\newcommand{\MGlobalFavoTag}{\special{html:<!-- mglobalfavoritestag -->}}
\else
\newcommand{\MGlobalFavoTag}{}
\fi

% Makro, um in der HTML-Ausgabe die Eingangstestseite zu kennzeichnen
\ifttm
\newcommand{\MGlobalSTestTag}{\special{html:<!-- mglobalstesttag -->}}
\else
\newcommand{\MGlobalSTestTag}{}
\fi

% Makro, um in der PDF-Ausgabe ein Wasserzeichen zu definieren
\ifttm
\newcommand{\MWatermarkSettings}{\relax}
\else
\newcommand{\MWatermarkSettings}{%
% \SetWatermarkText{(c) MINT-Kolleg Baden-W�rttemberg 2014}
% \SetWatermarkLightness{0.85}
% \SetWatermarkScale{1.5}
}
\fi

\ifttm
\newcommand{\MBinom}[2]{\left({\begin{array}{c} #1 \\ #2 \end{array}}\right)}
\else
\newcommand{\MBinom}[2]{\binom{#1}{#2}}
\fi

\ifttm
\newcommand{\DeclareMathOperator}[2]{\def#1{\mathrm{#2}}}
\newcommand{\operatorname}[1]{\mathrm{#1}}
\fi

%----------------- Makros fuer die gemischte HTML/PDF-Konvertierung ------------------------------

\newcommand{\MTestName}{\relax} % wird durch Test-Umgebung gesetzt

% Fuer experimentelle Kursinhalte, die im Release-Umsetzungsvorgang eine Fehlermeldung
% produzieren sollen aber sonst normal umgesetzt werden
\newenvironment{MExperimental}{%
}{%
}

% Wird von ttm nicht richtig umgesetzt!!
\newenvironment{MExerciseItems}{%
\renewcommand\theenumi{\alph{enumi}}%
\begin{enumerate}%
}{%
\end{enumerate}%
}


\definecolor{infoshadecolor}{rgb}{0.75,0.75,0.75}
\definecolor{exmpshadecolor}{rgb}{0.875,0.875,0.875}
\definecolor{expeshadecolor}{rgb}{0.95,0.95,0.95}
\definecolor{framecolor}{rgb}{0.2,0.2,0.2}

% Bei PDF-Uebersetzung wird hinter den Start jeder Satz/Info-aehnlichen Umgebung eine leere mbox gesetzt, damit
% fuehrende Listen oder enums nicht den Zeilenumbruch kaputtmachen
%\ifttm
\def\MTB{}
%\else
%\def\MTB{\mbox{}}
%\fi


\ifttm
\newcommand{\MRelates}{\special{html:<mi>&wedgeq;</mi>}}
\else
\def\MRelates{\stackrel{\scriptscriptstyle\wedge}{=}}
\fi

\def\MInch{\text{''}}
\def\Mdd{\textit{''}}

\ifttm
\def\MNL{ \newline }
\newenvironment{MArray}[1]{\begin{array}{#1}}{\end{array}}
\else
\def\MNL{ \\ }
\newenvironment{MArray}[1]{\begin{array}{#1}}{\end{array}}
\fi

\newcommand{\MBox}[1]{$\mathrm{#1}$}
\newcommand{\MMBox}[1]{\mathrm{#1}}


\ifttm%
\newcommand{\Mtfrac}[2]{{\textstyle \frac{#1}{#2}}}
\newcommand{\Mdfrac}[2]{{\displaystyle \frac{#1}{#2}}}
\newcommand{\Mmeasuredangle}{\special{html:<mi>&angmsd;</mi>}}
\else%
\newcommand{\Mtfrac}[2]{\tfrac{#1}{#2}}
\newcommand{\Mdfrac}[2]{\dfrac{#1}{#2}}
\newcommand{\Mmeasuredangle}{\measuredangle}
\relax
\fi

% Matrizen und Vektoren

% Inhalt wird in der Form a & b \\ c & d erwartet
% Vorsicht: MVector = Komponentenspalte, MVec = Variablensymbol
\ifttm%
\newcommand{\MVector}[1]{\left({\begin{array}{c}#1\end{array}}\right)}
\else%
\newcommand{\MVector}[1]{\begin{pmatrix}#1\end{pmatrix}}
\fi



\newcommand{\MVec}[1]{\vec{#1}}
\newcommand{\MDVec}[1]{\overrightarrow{#1}}

%----------------- Umgebungen fuer Definitionen und Saetze ----------------------------------------

% Fuegt einen Tabellen-Zeilenumbruch ein im PDF, aber nicht im HTML
\newcommand{\TSkip}{\ifttm \else&\ \\\fi}

\newenvironment{infoshaded}{%
\def\FrameCommand{\fboxsep=\FrameSep \fcolorbox{framecolor}{infoshadecolor}}%
\MakeFramed {\advance\hsize-\width \FrameRestore}}%
{\endMakeFramed}

\newenvironment{expeshaded}{%
\def\FrameCommand{\fboxsep=\FrameSep \fcolorbox{framecolor}{expeshadecolor}}%
\MakeFramed {\advance\hsize-\width \FrameRestore}}%
{\endMakeFramed}

\newenvironment{exmpshaded}{%
\def\FrameCommand{\fboxsep=\FrameSep \fcolorbox{framecolor}{exmpshadecolor}}%
\MakeFramed {\advance\hsize-\width \FrameRestore}}%
{\endMakeFramed}

\def\STDCOLOR{black}

\ifttm%
\else%
\newtheoremstyle{MSatzStyle}
  {1cm}                   %Space above
  {1cm}                   %Space below
  {\normalfont\itshape}   %Body font
  {}                      %Indent amount (empty = no indent,
                          %\parindent = para indent)
  {\normalfont\bfseries}  %Thm head font
  {}                      %Punctuation after thm head
  {\newline}              %Space after thm head: " " = normal interword
                          %space; \newline = linebreak
  {\thmname{#1}\thmnumber{ #2}\thmnote{ (#3)}}
                          %Thm head spec (can be left empty, meaning
                          %`normal')
                          %
\newtheoremstyle{MDefStyle}
  {1cm}                   %Space above
  {1cm}                   %Space below
  {\normalfont}           %Body font
  {}                      %Indent amount (empty = no indent,
                          %\parindent = para indent)
  {\normalfont\bfseries}  %Thm head font
  {}                      %Punctuation after thm head
  {\newline}              %Space after thm head: " " = normal interword
                          %space; \newline = linebreak
  {\thmname{#1}\thmnumber{ #2}\thmnote{ (#3)}}
                          %Thm head spec (can be left empty, meaning
                          %`normal')
\fi%

\newcommand{\MInfoText}{Info}

\newcounter{MHintCounter}
\newcounter{MCodeEditCounter}

\newcounter{MLastIndex}  % Enthaelt die dritte Stelle (Indexnummer) des letzten angelegten Objekts
\newcounter{MLastType}   % Enthaelt den Typ des letzten angelegten Objekts (mithilfe der unten definierten Konstanten). Die Entscheidung, wie der Typ dargstellt wird, wird in split.pm beim Postprocessing getroffen.
\newcounter{MLastTypeEq} % =1 falls das Label in einer Matheumgebung (equation, eqnarray usw.) steht, =2 falls das Label in einer table-Umgebung steht

% Da ttm keine Zahlmakros verarbeiten kann, werden diese Nummern in den Zuweisungen hardcodiert!
\def\MTypeSection{1}          %# Zaehler ist section
\def\MTypeSubsection{2}       %# Zaehler ist subsection
\def\MTypeSubsubsection{3}    %# Zaehler ist subsubsection
\def\MTypeInfo{4}             %# Eine Infobox, Separatzaehler fuer die Chemie (auch wenn es dort nicht nummeriert wird) ist MInfoCounter
\def\MTypeExercise{5}         %# Eine Aufgabe, Separatzaehler fuer die Chemie ist MExerciseCounter
\def\MTypeExample{6}          %# Eine Beispielbox, Separatzaehler fuer die Chemie ist MExampleCounter
\def\MTypeExperiment{7}       %# Eine Versuchsbox, Separatzaehler fuer die Chemie ist MExperimentCounter
\def\MTypeGraphics{8}         %# Eine Graphik, Separatzaehler fuer alle FB ist MGraphicsCounter
\def\MTypeTable{9}            %# Eine Tabellennummer, hat keinen Zaehler da durch table gezaehlt wird
\def\MTypeEquation{10}        %# Eine Gleichungsnummer, hat keinen Zaehler da durch equation/eqnarray gezaehlt wird
\def\MTypeTheorem{11}         % Ein theorem oder xtheorem, Separatzaehler fuer die Chemie ist MTheoremCounter
\def\MTypeVideo{12}           %# Ein Video,Separatzaehler fuer alle FB ist MVideoCounter
\def\MTypeEntry{13}           %# Ein Eintrag fuer die Stichwortliste, wird nicht gezaehlt sondern erhaelt im preparsing ein unique-label 

% Zaehler fuer das Labelsystem sind prefixcounter, jeder Zaehler wird VOR dem gezaehlten Objekt inkrementiert und zaehlt daher das aktuelle Objekt
\newcounter{MInfoCounter}
\newcounter{MExerciseCounter}
\newcounter{MExampleCounter}
\newcounter{MExperimentCounter}
\newcounter{MGraphicsCounter}
\newcounter{MTableCounter}
\newcounter{MEquationCounter}  % Nur im HTML, sonst durch "equation"-counter von latex realisiert
\newcounter{MTheoremCounter}
\newcounter{MObjectCounter}   % Gemeinsamer Zaehler fuer Objekte (ausser Grafiken/Tabellen) in Mathe/Info/Physik
\newcounter{MVideoCounter}
\newcounter{MEntryCounter}

\newcounter{MTestSite} % 1 = Subsubsection ist eine Pruefungsseite, 0 = ist eine normale Seite (inkl. Hilfeseite)

\def\MCell{$\phantom{a}$}

\newenvironment{MExportExercise}{\begin{MExercise}}{\end{MExercise}} % wird von mconvert abgefangen

\def\MGenerateExNumber{%
\ifnum\value{MSepNumbers}=0%
\arabic{section}.\arabic{subsection}.\arabic{MObjectCounter}\setcounter{MLastIndex}{\value{MObjectCounter}}%
\else%
\arabic{section}.\arabic{subsection}.\arabic{MExerciseCounter}\setcounter{MLastIndex}{\value{MExerciseCounter}}%
\fi%
}%

\def\MGenerateExmpNumber{%
\ifnum\value{MSepNumbers}=0%
\arabic{section}.\arabic{subsection}.\arabic{MObjectCounter}\setcounter{MLastIndex}{\value{MObjectCounter}}%
\else%
\arabic{section}.\arabic{subsection}.\arabic{MExerciseCounter}\setcounter{MLastIndex}{\value{MExampleCounter}}%
\fi%
}%

\def\MGenerateInfoNumber{%
\ifnum\value{MSepNumbers}=0%
\arabic{section}.\arabic{subsection}.\arabic{MObjectCounter}\setcounter{MLastIndex}{\value{MObjectCounter}}%
\else%
\arabic{section}.\arabic{subsection}.\arabic{MExerciseCounter}\setcounter{MLastIndex}{\value{MInfoCounter}}%
\fi%
}%

\def\MGenerateSiteNumber{%
\arabic{section}.\arabic{subsection}.\arabic{subsubsection}%
}%

% Funktionalitaet fuer Auswahlaufgaben

\newcounter{MExerciseCollectionCounter} % = 0 falls nicht in collection-Umgebung, ansonsten Schachtelungstiefe
\newcounter{MExerciseCollectionTextCounter} % wird von MExercise-Umgebung inkrementiert und von MExerciseCollection-Umgebung auf Null gesetzt

\ifttm
% MExerciseCollection gruppiert Aufgaben, die dynamisch aus der Datenbank gezogen werden und nicht direkt in der HTML-Seite stehen
% Parameter: #1 = ID der Collection, muss eindeutig fuer alle IN DER DB VORHANDENEN collections sein unabhaengig vom Kurs
%            #2 = Optionsargument (im Moment: 1 = Iterative Auswahl, 2 = Zufallsbasierte Auswahl)
\newenvironment{MExerciseCollection}[2]{%
\addtocounter{MExerciseCollectionCounter}{1}
\setcounter{MExerciseCollectionTextCounter}{0}
\special{html:<!-- mexercisecollectionstart;;}#1\special{html:;;}#2\special{html:;; //-->}%
}{%
\special{html:<!-- mexercisecollectionstop //-->}%
\addtocounter{MExerciseCollectionCounter}{-1}
}
\else
\newenvironment{MExerciseCollection}[2]{%
\addtocounter{MExerciseCollectionCounter}{1}
\setcounter{MExerciseCollectionTextCounter}{0}
}{%
\addtocounter{MExerciseCollectionCounter}{-1}
}
\fi

% Bei Uebersetzung nach PDF werden die theorem-Umgebungen verwendet, bei Uebersetzung in HTML ein manuelles Makro
\ifttm%

  \newenvironment{MHint}[1]{  \special{html:<button name="Name_MHint}\arabic{MHintCounter}\special{html:" class="hintbutton_closed" id="MHint}\arabic{MHintCounter}\special{html:_button" %
  type="button" onclick="toggle_hint('MHint}\arabic{MHintCounter}\special{html:');">}#1\special{html:</button>}
  \special{html:<div class="hint" style="display:none" id="MHint}\arabic{MHintCounter}\special{html:"> }}{\begin{html}</div>\end{html}\addtocounter{MHintCounter}{1}}

  \newenvironment{MCOSHZusatz}{  \special{html:<button name="Name_MHint}\arabic{MHintCounter}\special{html:" class="chintbutton_closed" id="MHint}\arabic{MHintCounter}\special{html:_button" %
  type="button" onclick="toggle_hint('MHint}\arabic{MHintCounter}\special{html:');">}Weiterf�hrende Inhalte\special{html:</button>}
  \special{html:<div class="hintc" style="display:none" id="MHint}\arabic{MHintCounter}\special{html:">
  <div class="coshwarn">Diese Inhalte gehen �ber das Kursniveau hinaus und werden in den Aufgaben und Tests nicht abgefragt.</div><br />}
  \addtocounter{MHintCounter}{1}}{\begin{html}</div>\end{html}}

  
  \newenvironment{MDefinition}{\begin{definition}\setcounter{MLastIndex}{\value{definition}}\ \\}{\end{definition}}

  
  \newenvironment{MExercise}{
  \renewcommand{\MStdPoints}{4}
  \addtocounter{MExerciseCounter}{1}
  \addtocounter{MObjectCounter}{1}
  \setcounter{MLastType}{5}

  \ifnum\value{MExerciseCollectionCounter}=0\else\addtocounter{MExerciseCollectionTextCounter}{1}\special{html:<!-- mexercisetextstart;;}\arabic{MExerciseCollectionTextCounter}\special{html:;; //-->}\fi
  \special{html:<div class="aufgabe" id="ADIV_}\MGenerateExNumber\special{html:">}%
  \textbf{Aufgabe \MGenerateExNumber
  } \ \\}{
  \special{html:</div><!-- mfeedbackbutton;Aufgabe;}\arabic{MTestSite}\special{html:;}\MGenerateExNumber\special{html:; //-->}
  \ifnum\value{MExerciseCollectionCounter}=0\else\special{html:<!-- mexercisetextstop //-->}\fi
  }

  % Stellt eine Kombination aus Aufgabe, Loesungstext und Eingabefeld bereit,
  % bei der Aufgabentext und Musterloesung sowie die zugehoerigen Feldelemente
  % extern bezogen und div-aktualisiert werden, das Eingabefeld aber immer das gleiche ist.
  \newenvironment{MFetchExercise}{
  \addtocounter{MExerciseCounter}{1}
  \addtocounter{MObjectCounter}{1}
  \setcounter{MLastType}{5}

  \special{html:<div class="aufgabe" id="ADIV_}\MGenerateExNumber\special{html:">}%
  \textbf{Aufgabe \MGenerateExNumber
  } \ \\%
  \special{html:</div><div class="exfetch_text" id="ADIVTEXT_}\MGenerateExNumber\special{html:">}%
  \special{html:</div><div class="exfetch_sol" id="ADIVSOL_}\MGenerateExNumber\special{html:">}%
  \special{html:</div><div class="exfetch_input" id="ADIVINPUT_}\MGenerateExNumber\special{html:">}%
  }{
  \special{html:</div>}
  }

  \newenvironment{MExample}{
  \addtocounter{MExampleCounter}{1}
  \addtocounter{MObjectCounter}{1}
  \setcounter{MLastType}{6}
  \begin{html}
  <div class="exmp">
  <div class="exmprahmen">
  \end{html}\textbf{Beispiel
  \ifnum\value{MSepNumbers}=0
  \arabic{section}.\arabic{subsection}.\arabic{MObjectCounter}\setcounter{MLastIndex}{\value{MObjectCounter}}
  \else
  \arabic{section}.\arabic{subsection}.\arabic{MExampleCounter}\setcounter{MLastIndex}{\value{MExampleCounter}}
  \fi
  } \ \\}{\begin{html}</div>
  </div>
  \end{html}
  \special{html:<!-- mfeedbackbutton;Beispiel;}\arabic{MTestSite}\special{html:;}\MGenerateExmpNumber\special{html:; //-->}
  }

  \newenvironment{MExperiment}{
  \addtocounter{MExperimentCounter}{1}
  \addtocounter{MObjectCounter}{1}
  \setcounter{MLastType}{7}
  \begin{html}
  <div class="expe">
  <div class="experahmen">
  \end{html}\textbf{Versuch
  \ifnum\value{MSepNumbers}=0
  \arabic{section}.\arabic{subsection}.\arabic{MObjectCounter}\setcounter{MLastIndex}{\value{MObjectCounter}}
  \else
%  \arabic{MExperimentCounter}\setcounter{MLastIndex}{\value{MExperimentCounter}}
  \arabic{section}.\arabic{subsection}.\arabic{MExperimentCounter}\setcounter{MLastIndex}{\value{MExperimentCounter}}
  \fi
  } \ \\}{\begin{html}</div>
  </div>
  \end{html}}

  \newenvironment{MChemInfo}{
  \setcounter{MLastType}{4}
  \begin{html}
  <div class="info">
  <div class="inforahmen">
  \end{html}}{\begin{html}</div>
  </div>
  \end{html}}

  \newenvironment{MXInfo}[1]{
  \addtocounter{MInfoCounter}{1}
  \addtocounter{MObjectCounter}{1}
  \setcounter{MLastType}{4}
  \begin{html}
  <div class="info">
  <div class="inforahmen">
  \end{html}\textbf{#1
  \ifnum\value{MInfoNumbers}=0
  \else
    \ifnum\value{MSepNumbers}=0
    \arabic{section}.\arabic{subsection}.\arabic{MObjectCounter}\setcounter{MLastIndex}{\value{MObjectCounter}}
    \else
    \arabic{MInfoCounter}\setcounter{MLastIndex}{\value{MInfoCounter}}
    \fi
  \fi
  } \ \\}{\begin{html}</div>
  </div>
  \end{html}
  \special{html:<!-- mfeedbackbutton;Info;}\arabic{MTestSite}\special{html:;}\MGenerateInfoNumber\special{html:; //-->}
  }

  \newenvironment{MInfo}{\ifnum\value{MInfoNumbers}=0\begin{MChemInfo}\else\begin{MXInfo}{Info}\ \\ \fi}{\ifnum\value{MInfoNumbers}=0\end{MChemInfo}\else\end{MXInfo}\fi}

\else%

  \theoremstyle{MSatzStyle}
  \newtheorem{thm}{Satz}[section]
  \newtheorem{thmc}{Satz}
  \theoremstyle{MDefStyle}
  \newtheorem{defn}[thm]{Definition}
  \newtheorem{exmp}[thm]{Beispiel}
  \newtheorem{info}[thm]{\MInfoText}
  \theoremstyle{MDefStyle}
  \newtheorem{defnc}{Definition}
  \theoremstyle{MDefStyle}
  \newtheorem{exmpc}{Beispiel}[section]
  \theoremstyle{MDefStyle}
  \newtheorem{infoc}{\MInfoText}
  \theoremstyle{MDefStyle}
  \newtheorem{exrc}{Aufgabe}[section]
  \theoremstyle{MDefStyle}
  \newtheorem{verc}{Versuch}[section]
  
  \newenvironment{MFetchExercise}{}{} % kann im PDF nicht dargestellt werden
  
  \newenvironment{MExercise}{\begin{exrc}\renewcommand{\MStdPoints}{1}\MTB}{\end{exrc}}
  \newenvironment{MHint}[1]{\ \\ \underline{#1:}\\}{}
  \newenvironment{MCOSHZusatz}{\ \\ \underline{Weiterf�hrende Inhalte:}\\}{}
  \newenvironment{MDefinition}{\ifnum\value{MInfoNumbers}=0\begin{defnc}\else\begin{defn}\fi\MTB}{\ifnum\value{MInfoNumbers}=0\end{defnc}\else\end{defn}\fi}
%  \newenvironment{MExample}{\begin{exmp}}{\ \linebreak[1] \ \ \ \ $\phantom{a}$ \ \hfill $\blacklozenge$\end{exmp}}
  \newenvironment{MExample}{
    \ifnum\value{MInfoNumbers}=0\begin{exmpc}\else\begin{exmp}\fi
    \MTB
    \begin{exmpshaded}
    \ \newline
}{
    \end{exmpshaded}
    \ifnum\value{MInfoNumbers}=0\end{exmpc}\else\end{exmp}\fi
}
  \newenvironment{MChemInfo}{\begin{infoshaded}}{\end{infoshaded}}

  \newenvironment{MInfo}{\ifnum\value{MInfoNumbers}=0\begin{MChemInfo}\else\renewcommand{\MInfoText}{Info}\begin{info}\begin{infoshaded}
  \MTB
   \ \newline
    \fi
  }{\ifnum\value{MInfoNumbers}=0\end{MChemInfo}\else\end{infoshaded}\end{info}\fi}

  \newenvironment{MXInfo}[1]{
    \renewcommand{\MInfoText}{#1}
    \ifnum\value{MInfoNumbers}=0\begin{infoc}\else\begin{info}\fi%
    \MTB
    \begin{infoshaded}
    \ \newline
  }{\end{infoshaded}\ifnum\value{MInfoNumbers}=0\end{infoc}\else\end{info}\fi}

  \newenvironment{MExperiment}{
    \renewcommand{\MInfoText}{Versuch}
    \ifnum\value{MInfoNumbers}=0\begin{verc}\else\begin{info}\fi
    \MTB
    \begin{expeshaded}
    \ \newline
  }{
    \end{expeshaded}
    \ifnum\value{MInfoNumbers}=0\end{verc}\else\end{info}\fi
  }
\fi%

% MHint sollte nicht direkt fuer Loesungen benutzt werden wegen solutionselect
\newenvironment{MSolution}{\begin{MHint}{L"osung}}{\end{MHint}}

\newcounter{MCodeCounter}

\ifttm
\newenvironment{MCode}{\special{html:<!-- mcodestart -->}\ttfamily\color{blue}}{\special{html:<!-- mcodestop -->}}
\else
\newenvironment{MCode}{\begin{flushleft}\ttfamily\addtocounter{MCodeCounter}{1}}{\addtocounter{MCodeCounter}{-1}\end{flushleft}}
% Ohne color-Statement da inkompatible mit framed/shaded-Boxen aus dem framed-package
\fi

%----------------- Sonderdefinitionen fuer Symbole, die der Konverter nicht kann ----------------------------------------------

\ifttm%
\newcommand{\MUnderset}[2]{\underbrace{#2}_{#1}}%
\else%
\newcommand{\MUnderset}[2]{\underset{#1}{#2}}%
\fi%

\ifttm
\newcommand{\MThinspace}{\special{html:<mi>&#x2009;</mi>}}
\else
\newcommand{\MThinspace}{\,}
\fi

\ifttm
\newcommand{\glq}{\begin{html}&sbquo;\end{html}}
\newcommand{\grq}{\begin{html}&lsquo;\end{html}}
\newcommand{\glqq}{\begin{html}&bdquo;\end{html}}
\newcommand{\grqq}{\begin{html}&ldquo;\end{html}}
\fi

\ifttm
\newcommand{\MNdash}{\begin{html}&ndash;\end{html}}
\else
\newcommand{\MNdash}{--}
\fi

%\ifttm\def\MIU{\special{html:<mi>&#8520;</mi>}}\else\def\MIU{\mathrm{i}}\fi
\def\MIU{\mathrm{i}}
\def\MEU{e} % TU9-Onlinekurs: italic-e
%\def\MEU{\mathrm{e}} % Alte Onlinemodule: roman-e
\def\MD{d} % Kursives d in Integralen im TU9-Onlinekurs
%\def\MD{\mathrm{d}} % roman-d in den alten Onlinemodulen
\def\MDB{\|}

%zusaetzlicher Leerraum vor "\MD"
\ifttm%
\def\MDSpace{\special{html:<mi>&#x2009;</mi>}}
\else%
\def\MDSpace{\,}
\fi%
\newcommand{\MDwSp}{\MDSpace\MD}%

\ifttm
\def\Mdq{\dq}
\else
\def\Mdq{\dq}
\fi

\def\MSpan#1{\left<{#1}\right>}
\def\MSetminus{\setminus}
\def\MIM{I}

\ifttm
\newcommand{\ld}{\text{ld}}
\newcommand{\lg}{\text{lg}}
\else
\DeclareMathOperator{\ld}{ld}
%\newcommand{\lg}{\text{lg}} % in latex schon definiert
\fi


\def\Mmapsto{\ifttm\special{html:<mi>&mapsto;</mi>}\else\mapsto\fi} 
\def\Mvarphi{\ifttm\phi\else\varphi\fi}
\def\Mphi{\ifttm\varphi\else\phi\fi}
\ifttm%
\newcommand{\MEumu}{\special{html:<mi>&#x3BC;</mi>}}%
\else%
\newcommand{\MEumu}{\textrm{\textmu}}%
\fi
\def\Mvarepsilon{\ifttm\epsilon\else\varepsilon\fi}
\def\Mepsilon{\ifttm\varepsilon\else\epsilon\fi}
\def\Mvarkappa{\ifttm\kappa\else\varkappa\fi}
\def\Mkappa{\ifttm\varkappa\else\kappa\fi}
\def\Mcomplement{\ifttm\special{html:<mi>&comp;</mi>}\else\complement\fi} 
\def\MWW{\mathrm{WW}}
\def\Mmod{\ifttm\special{html:<mi>&nbsp;mod&nbsp;</mi>}\else\mod\fi} 

\ifttm%
\def\mod{\text{\;mod\;}}%
\def\MNEquiv{\special{html:<mi>&NotCongruent;</mi>}}% 
\def\MNSubseteq{\special{html:<mi>&NotSubsetEqual;</mi>}}%
\def\MEmptyset{\special{html:<mi>&empty;</mi>}}%
\def\MVDots{\special{html:<mi>&#x22EE;</mi>}}%
\def\MHDots{\special{html:<mi>&#x2026;</mi>}}%
\def\Mddag{\special{html:<mi>&#x1202;</mi>}}%
\def\sphericalangle{\special{html:<mi>&measuredangle;</mi>}}%
\def\nparallel{\special{html:<mi>&nparallel;</mi>}}%
\def\MProofEnd{\special{html:<mi>&#x25FB;</mi>}}%
\newenvironment{MProof}[1]{\underline{#1}:\MCR\MCR}{\hfill $\MProofEnd$}%
\else%
\def\MNEquiv{\not\equiv}%
\def\MNSubseteq{\not\subseteq}%
\def\MEmptyset{\emptyset}%
\def\MVDots{\vdots}%
\def\MHDots{\hdots}%
\def\Mddag{\ddag}%
\newenvironment{MProof}[1]{\begin{proof}[#1]}{\end{proof}}%
\fi%



% Spaces zum Auffuellen von Tabellenbreiten, die nur im HTML wirken
\ifttm%
\def\MTSP{\:}%
\else%
\def\MTSP{}%
\fi%

\DeclareMathOperator{\arsinh}{arsinh}
\DeclareMathOperator{\arcosh}{arcosh}
\DeclareMathOperator{\artanh}{artanh}
\DeclareMathOperator{\arcoth}{arcoth}


\newcommand{\MMathSet}[1]{\mathbb{#1}}
\def\N{\MMathSet{N}}
\def\Z{\MMathSet{Z}}
\def\Q{\MMathSet{Q}}
\def\R{\MMathSet{R}}
\def\C{\MMathSet{C}}

\newcounter{MForLoopCounter}
\newcommand{\MForLoop}[2]{\setcounter{MForLoopCounter}{#1}\ifnum\value{MForLoopCounter}=0{}\else{{#2}\addtocounter{MForLoopCounter}{-1}\MForLoop{\value{MForLoopCounter}}{#2}}\fi}

\newcounter{MSiteCounter}
\newcounter{MFieldCounter} % Kombination section.subsection.site.field ist eindeutig in allen Modulen, field alleine nicht

\newcounter{MiniMarkerCounter}

\ifttm
\newenvironment{MMiniPageP}[1]{\begin{minipage}{#1\linewidth}\special{html:<!-- minimarker;;}\arabic{MiniMarkerCounter}\special{html:;;#1; //-->}}{\end{minipage}\addtocounter{MiniMarkerCounter}{1}}
\else
\newenvironment{MMiniPageP}[1]{\begin{minipage}{#1\linewidth}}{\end{minipage}\addtocounter{MiniMarkerCounter}{1}}
\fi

\newcounter{AlignCounter}

\newcommand{\MStartJustify}{\ifttm\special{html:<!-- startalign;;}\arabic{AlignCounter}\special{html:;;justify; //-->}\fi}
\newcommand{\MStopJustify}{\ifttm\special{html:<!-- stopalign;;}\arabic{AlignCounter}\special{html:; //-->}\fi\addtocounter{AlignCounter}{1}}

\newenvironment{MJTabular}[1]{
\MStartJustify
\begin{tabular}{#1}
}{
\end{tabular}
\MStopJustify
}

\newcommand{\MImageLeft}[2]{
\begin{center}
\begin{tabular}{lc}
\MStartJustify
\begin{MMiniPageP}{0.65}
#1
\end{MMiniPageP}
\MStopJustify
&
\begin{MMiniPageP}{0.3}
#2  
\end{MMiniPageP}
\end{tabular}
\end{center}
}

\newcommand{\MImageHalf}[2]{
\begin{center}
\begin{tabular}{lc}
\MStartJustify
\begin{MMiniPageP}{0.45}
#1
\end{MMiniPageP}
\MStopJustify
&
\begin{MMiniPageP}{0.45}
#2  
\end{MMiniPageP}
\end{tabular}
\end{center}
}

\newcommand{\MBigImageLeft}[2]{
\begin{center}
\begin{tabular}{lc}
\MStartJustify
\begin{MMiniPageP}{0.25}
#1
\end{MMiniPageP}
\MStopJustify
&
\begin{MMiniPageP}{0.7}
#2  
\end{MMiniPageP}
\end{tabular}
\end{center}
}

\ifttm
\def\No{\mathbb{N}_0}
\else
\def\No{\ensuremath{\N_0}}
\fi
\def\MT{\textrm{\tiny T}}
\newcommand{\MTranspose}[1]{{#1}^{\MT}}
\ifttm
\newcommand{\MRe}{\mathsf{Re}}
\newcommand{\MIm}{\mathsf{Im}}
\else
\DeclareMathOperator{\MRe}{Re}
\DeclareMathOperator{\MIm}{Im}
\fi

\newcommand{\Mid}{\mathrm{id}}
\newcommand{\MFeinheit}{\mathrm{feinh}}

\ifttm
\newcommand{\Msubstack}[1]{\begin{array}{c}{#1}\end{array}}
\else
\newcommand{\Msubstack}[1]{\substack{#1}}
\fi

% Typen von Fragefeldern:
% 1 = Alphanumerisch, case-sensitive-Vergleich
% 2 = Ja/Nein-Checkbox, Loesung ist 0 oder 1   (OPTION = Image-id fuer Rueckmeldung)
% 3 = Reelle Zahlen Geparset
% 4 = Funktionen Geparset (mit Stuetzstellen zur ueberpruefung)

% Dieser Befehl erstellt ein interaktives Aufgabenfeld. Parameter:
% - #1 Laenge in Zeichen
% - #2 Loesungstext (alphanumerisch, case sensitive)
% - #3 AufgabenID (alphanumerisch, case sensitive)
% - #4 Typ (Kennnummer)
% - #5 String fuer Optionen (ggf. mit Semikolon getrennte Einzelstrings)
% - #6 Anzahl Punkte
% - #7 uxid (kann z.B. Loesungsstring sein)
% ACHTUNG: Die langen Zeilen bitte so lassen, Zeilenumbrueche im tex werden in div's umgesetzt
\newcommand{\MQuestionID}[7]{
\ifttm
\special{html:<!-- mdeclareuxid;;}UX#7\special{html:;;}\arabic{section}\special{html:;;}#3\special{html:;; //-->}%
\special{html:<!-- mdeclarepoints;;}\arabic{section}\special{html:;;}#3\special{html:;;}#6\special{html:;;}\arabic{MTestSite}\special{html:;;}\arabic{chapter}%
\special{html:;; //--><!-- onloadstart //-->CreateQuestionObj("}#7\special{html:",}\arabic{MFieldCounter}\special{html:,"}#2%
\special{html:","}#3\special{html:",}#4\special{html:,"}#5\special{html:",}#6\special{html:,}\arabic{MTestSite}\special{html:,}\arabic{section}%
\special{html:);<!-- onloadstop //-->}%
\special{html:<input mfieldtype="}#4\special{html:" name="Name_}#3\special{html:" id="}#3\special{html:" type="text" size="}#1\special{html:" maxlength="}#1%
\special{html:" }\ifnum\value{MGroupActive}=0\special{html:onfocus="handlerFocus(}\arabic{MFieldCounter}%
\special{html:);" onblur="handlerBlur(}\arabic{MFieldCounter}\special{html:);" onkeyup="handlerChange(}\arabic{MFieldCounter}\special{html:,0);" onpaste="handlerChange(}\arabic{MFieldCounter}\special{html:,0);" oninput="handlerChange(}\arabic{MFieldCounter}\special{html:,0);" onpropertychange="handlerChange(}\arabic{MFieldCounter}\special{html:,0);"/>}%
\special{html:<img src="images/questionmark.gif" width="20" height="20" border="0" align="absmiddle" id="}QM#3\special{html:"/>}
\else%
\special{html:onblur="handlerBlur(}\arabic{MFieldCounter}%
\special{html:);" onfocus="handlerFocus(}\arabic{MFieldCounter}\special{html:);" onkeyup="handlerChange(}\arabic{MFieldCounter}\special{html:,1);" onpaste="handlerChange(}\arabic{MFieldCounter}\special{html:,1);" oninput="handlerChange(}\arabic{MFieldCounter}\special{html:,1);" onpropertychange="handlerChange(}\arabic{MFieldCounter}\special{html:,1);"/>}%
\special{html:<img src="images/questionmark.gif" width="20" height="20" border="0" align="absmiddle" id="}QM#3\special{html:"/>}\fi%
\else%
\ifnum\value{QBoxFlag}=1\fbox{$\phantom{\MForLoop{#1}{b}}$}\else$\phantom{\MForLoop{#1}{b}}$\fi%
\fi%
}

% ACHTUNG: Die langen Zeilen bitte so lassen, Zeilenumbrueche im tex werden in div's umgesetzt
% QuestionCheckbox macht ausserhalb einer QuestionGroup keinen Sinn!
% #1 = solution (1 oder 0), ggf. mit ::smc abgetrennt auszuschliessende single-choice-boxen (UXIDs durch , getrennt), #2 = id, #3 = points, #4 = uxid
\newcommand{\MQuestionCheckbox}[4]{
\ifttm
\special{html:<!-- mdeclareuxid;;}UX#4\special{html:;;}\arabic{section}\special{html:;;}#2\special{html:;; //-->}%
\ifnum\value{MGroupActive}=0\MDebugMessage{ERROR: Checkbox Nr. \arabic{MFieldCounter}\ ist nicht in einer Kontrollgruppe, es wird niemals eine Loesung angezeigt!}\fi
\special{html: %
<!-- mdeclarepoints;;}\arabic{section}\special{html:;;}#2\special{html:;;}#3\special{html:;;}\arabic{MTestSite}\special{html:;;}\arabic{chapter}%
\special{html:;; //--><!-- onloadstart //-->CreateQuestionObj("}#4\special{html:",}\arabic{MFieldCounter}\special{html:,"}#1\special{html:","}#2\special{html:",2,"IMG}#2%
\special{html:",}#3\special{html:,}\arabic{MTestSite}\special{html:,}\arabic{section}\special{html:);<!-- onloadstop //-->}%
\special{html:<input mfieldtype="2" type="checkbox" name="Name_}#2\special{html:" id="}#2\special{html:" onchange="handlerChange(}\arabic{MFieldCounter}\special{html:,1);"/><img src="images/questionmark.gif" name="}Name_IMG#2%
\special{html:" width="20" height="20" border="0" align="absmiddle" id="}IMG#2\special{html:"/> }%
\else%
\ifnum\value{QBoxFlag}=1\fbox{$\phantom{X}$}\else$\phantom{X}$\fi%
\fi%
}

\def\MGenerateID{QFELD_\arabic{section}.\arabic{subsection}.\arabic{MSiteCounter}.QF\arabic{MFieldCounter}}

% #1 = 0/1 ggf. mit ::smc abgetrennt auszuschliessende single-choice-boxen (UXIDs durch , getrennt ohne UX), #2 = uxid ohne UX
\newcommand{\MCheckbox}[2]{
\MQuestionCheckbox{#1}{\MGenerateID}{\MStdPoints}{#2}
\addtocounter{MFieldCounter}{1}
}

% Erster Parameter: Zeichenlaenge der Eingabebox, zweiter Parameter: Loesungstext
\newcommand{\MQuestion}[2]{
\MQuestionID{#1}{#2}{\MGenerateID}{1}{0}{\MStdPoints}{#2}
\addtocounter{MFieldCounter}{1}
}

% Erster Parameter: Zeichenlaenge der Eingabebox, zweiter Parameter: Loesungstext
\newcommand{\MLQuestion}[3]{
\MQuestionID{#1}{#2}{\MGenerateID}{1}{0}{\MStdPoints}{#3}
\addtocounter{MFieldCounter}{1}
}

% Parameter: Laenge des Feldes, Loesung (wird auch geparsed), Stellen Genauigkeit hinter dem Komma, weitere Stellen werden mathematisch gerundet vor Vergleich
\newcommand{\MParsedQuestion}[3]{
\MQuestionID{#1}{#2}{\MGenerateID}{3}{#3}{\MStdPoints}{#2}
\addtocounter{MFieldCounter}{1}
}

% Parameter: Laenge des Feldes, Loesung (wird auch geparsed), Stellen Genauigkeit hinter dem Komma, weitere Stellen werden mathematisch gerundet vor Vergleich
\newcommand{\MLParsedQuestion}[4]{
\MQuestionID{#1}{#2}{\MGenerateID}{3}{#3}{\MStdPoints}{#4}
\addtocounter{MFieldCounter}{1}
}

% Parameter: Laenge des Feldes, Loesungsfunktion, Anzahl Stuetzstellen, Funktionsvariablen durch Kommata getrennt (nicht case-sensitive), Anzahl Nachkommastellen im Vergleich
\newcommand{\MFunctionQuestion}[5]{
\MQuestionID{#1}{#2}{\MGenerateID}{4}{#3;#4;#5;0}{\MStdPoints}{#2}
\addtocounter{MFieldCounter}{1}
}

% Parameter: Laenge des Feldes, Loesungsfunktion, Anzahl Stuetzstellen, Funktionsvariablen durch Kommata getrennt (nicht case-sensitive), Anzahl Nachkommastellen im Vergleich, UXID
\newcommand{\MLFunctionQuestion}[6]{
\MQuestionID{#1}{#2}{\MGenerateID}{4}{#3;#4;#5;0}{\MStdPoints}{#6}
\addtocounter{MFieldCounter}{1}
}

% Parameter: Laenge des Feldes, Loesungsintervall, Genauigkeit der Zahlenwertpruefung
\newcommand{\MIntervalQuestion}[3]{
\MQuestionID{#1}{#2}{\MGenerateID}{6}{#3}{\MStdPoints}{#2}
\addtocounter{MFieldCounter}{1}
}

% Parameter: Laenge des Feldes, Loesungsintervall, Genauigkeit der Zahlenwertpruefung, UXID
\newcommand{\MLIntervalQuestion}[4]{
\MQuestionID{#1}{#2}{\MGenerateID}{6}{#3}{\MStdPoints}{#4}
\addtocounter{MFieldCounter}{1}
}

% Parameter: Laenge des Feldes, Loesungsfunktion, Anzahl Stuetzstellen, Funktionsvariable (nicht case-sensitive), Anzahl Nachkommastellen im Vergleich, Vereinfachungsbedingung
% Vereinfachungsbedingung ist eine der Folgenden:
% 0 = Keine Vereinfachungsbedingung
% 1 = Keine Klammern (runde oder eckige) mehr im vereinfachten Ausdruck
% 2 = Faktordarstellung (Term hat Produkte als letzte Operation, Summen als vorgeschaltete Operation)
% 3 = Summendarstellung (Term hat Summen als letzte Operation, Produkte als vorgeschaltete Operation)
% Flag 512: Besondere Stuetzstellen (nur >1 und nur schwach rational), sonst symmetrisch um Nullpunkt und ganze Zahlen inkl. Null werden getroffen
\newcommand{\MSimplifyQuestion}[6]{
\MQuestionID{#1}{#2}{\MGenerateID}{4}{#3;#4;#5;#6}{\MStdPoints}{#2}
\addtocounter{MFieldCounter}{1}
}

\newcommand{\MLSimplifyQuestion}[7]{
\MQuestionID{#1}{#2}{\MGenerateID}{4}{#3;#4;#5;#6}{\MStdPoints}{#7}
\addtocounter{MFieldCounter}{1}
}

% Parameter: Laenge des Feldes, Loesung (optionaler Ausdruck), Anzahl Stuetzstellen, Funktionsvariable (nicht case-sensitive), Anzahl Nachkommastellen im Vergleich, Spezialtyp (string-id)
\newcommand{\MLSpecialQuestion}[7]{
\MQuestionID{#1}{#2}{\MGenerateID}{7}{#3;#4;#5;#6}{\MStdPoints}{#7}
\addtocounter{MFieldCounter}{1}
}

\newcounter{MGroupStart}
\newcounter{MGroupEnd}
\newcounter{MGroupActive}

\newenvironment{MQuestionGroup}{
\setcounter{MGroupStart}{\value{MFieldCounter}}
\setcounter{MGroupActive}{1}
}{
\setcounter{MGroupActive}{0}
\setcounter{MGroupEnd}{\value{MFieldCounter}}
\addtocounter{MGroupEnd}{-1}
}

\newcommand{\MGroupButton}[1]{
\ifttm
\special{html:<button name="Name_Group}\arabic{MGroupStart}\special{html:to}\arabic{MGroupEnd}\special{html:" id="Group}\arabic{MGroupStart}\special{html:to}\arabic{MGroupEnd}\special{html:" %
type="button" onclick="group_button(}\arabic{MGroupStart}\special{html:,}\arabic{MGroupEnd}\special{html:);">}#1\special{html:</button>}
\else
\phantom{#1}
\fi
}

%----------------- Makros fuer die modularisierte Darstellung ------------------------------------

\def\MyText#1{#1}

% is used internally by the conversion package, should not be used by original tex documents
\def\MOrgLabel#1{\relax}

\ifttm

% Ein MLabel wird im html codiert durch das tag <!-- mmlabel;;Labelbezeichner;;SubjectArea;;chapter;;section;;subsection;;Index;;Objekttyp; //-->
\def\MLabel#1{%
\ifnum\value{MLastType}=8%
\ifnum\value{MCaptionOn}=0%
\MDebugMessage{ERROR: Grafik \arabic{MGraphicsCounter} hat separates label: #1 (Grafiklabels sollten nur in der Caption stehen)}%
\fi
\fi
\ifnum\value{MLastType}=12%
\ifnum\value{MCaptionOn}=0%
\MDebugMessage{ERROR: Video \arabic{MVideoCounter} hat separates label: #1 (Videolabels sollten nur in der Caption stehen}%
\fi
\fi
\ifnum\value{MLastType}=10\setcounter{MLastIndex}{\value{equation}}\fi
\label{#1}\begin{html}<!-- mmlabel;;#1;;\end{html}\arabic{MSubjectArea}\special{html:;;}\arabic{chapter}\special{html:;;}\arabic{section}\special{html:;;}\arabic{subsection}\special{html:;;}\arabic{MLastIndex}\special{html:;;}\arabic{MLastType}\special{html:; //-->}}%

\else

% Sonderbehandlung im PDF fuer Abbildungen in separater aux-Datei, da MGraphics die figure-Umgebung nicht verwendet
\def\MLabel#1{%
\ifnum\value{MLastType}=8%
\ifnum\value{MCaptionOn}=0%
\MDebugMessage{ERROR: Grafik \arabic{MGraphicsCounter} hat separates label: #1 (Grafiklabels sollten nur in der Caption stehen}%
\fi
\fi
\ifnum\value{MLastType}=12%
\ifnum\value{MCaptionOn}=0%
\MDebugMessage{ERROR: Video \arabic{MVideoCounter} hat separates label: #1 (Videolabels sollten nur in der Caption stehen}%
\fi
\fi
\label{#1}%
}%

\fi

% Gibt Begriff des referenzierten Objekts mit aus, aber nur im HTML, daher nur in Ausnahmefaellen (z.B. Copyrightliste) sinnvoll
\def\MCRef#1{\ifttm\special{html:<!-- mmref;;}#1\special{html:;;1; //-->}\else\vref{#1}\fi}


\def\MRef#1{\ifttm\special{html:<!-- mmref;;}#1\special{html:;;0; //-->}\else\vref{#1}\fi}
\def\MERef#1{\ifttm\special{html:<!-- mmref;;}#1\special{html:;;0; //-->}\else\eqref{#1}\fi}
\def\MNRef#1{\ifttm\special{html:<!-- mmref;;}#1\special{html:;;0; //-->}\else\ref{#1}\fi}
\def\MSRef#1#2{\ifttm\special{html:<!-- msref;;}#1\special{html:;;}#2\special{html:; //-->}\else \if#2\empty \ref{#1} \else \hyperref[#1]{#2}\fi\fi} 

\def\MRefRange#1#2{\ifttm\MRef{#1} bis 
\MRef{#2}\else\vrefrange[\unskip]{#1}{#2}\fi}

\def\MRefTwo#1#2{\ifttm\MRef{#1} und \MRef{#2}\else%
\let\vRefTLRsav=\reftextlabelrange\let\vRefTPRsav=\reftextpagerange%
\def\reftextlabelrange##1##2{\ref{##1} und~\ref{##2}}%
\def\reftextpagerange##1##2{auf den Seiten~\pageref{#1} und~\pageref{#2}}%
\vrefrange[\unskip]{#1}{#2}%
\let\reftextlabelrange=\vRefTLRsav\let\reftextpagerange=\vRefTPRsav\fi}

% MSectionChapter definiert falls notwendig das Kapitel vor der section. Das ist notwendig, wenn nur ein Einzelmodul uebersetzt wird.
% MChaptersGiven ist ein Counter, der von mconvert.pl vordefiniert wird.
\ifttm
\newcommand{\MSectionChapter}{\ifnum\value{MChaptersGiven}=0{\Dchapter{Modul}}\else{}\fi}
\else
\newcommand{\MSectionChapter}{\ifnum\value{chapter}=0{\Dchapter{Modul}}\else{}\fi}
\fi


\def\MChapter#1{\ifnum\value{MSSEnd}>0{\MSubsectionEndMacros}\addtocounter{MSSEnd}{-1}\fi\Dchapter{#1}}
\def\MSubject#1{\MChapter{#1}} % Schluesselwort HELPSECTION ist reserviert fuer Hilfesektion

\newcommand{\MSectionID}{UNKNOWNID}

\ifttm
\newcommand{\MSetSectionID}[1]{\renewcommand{\MSectionID}{#1}}
\else
\newcommand{\MSetSectionID}[1]{\renewcommand{\MSectionID}{#1}\tikzsetexternalprefix{#1}}
\fi


\newcommand{\MSection}[1]{\MSetSectionID{MODULID}\ifnum\value{MSSEnd}>0{\MSubsectionEndMacros}\addtocounter{MSSEnd}{-1}\fi\MSectionChapter\Dsection{#1}\MSectionStartMacros{#1}\setcounter{MLastIndex}{-1}\setcounter{MLastType}{1}} % Sections werden ueber das section-Feld im mmlabel-Tag identifiziert, nicht ueber das Indexfeld

\def\MSubsection#1{\ifnum\value{MSSEnd}>0{\MSubsectionEndMacros}\addtocounter{MSSEnd}{-1}\fi\ifttm\else\clearpage\fi\Dsubsection{#1}\MSubsectionStartMacros\setcounter{MLastIndex}{-1}\setcounter{MLastType}{2}\addtocounter{MSSEnd}{1}}% Subsections werden ueber das subsection-Feld im mmlabel-Tag identifiziert, nicht ueber das Indexfeld
\def\MSubsectionx#1{\Dsubsectionx{#1}} % Nur zur Verwendung in MSectionStart gedacht
\def\MSubsubsection#1{\Dsubsubsection{#1}\setcounter{MLastIndex}{\value{subsubsection}}\setcounter{MLastType}{3}\ifttm\special{html:<!-- sectioninfo;;}\arabic{section}\special{html:;;}\arabic{subsection}\special{html:;;}\arabic{subsubsection}\special{html:;;1;;}\arabic{MTestSite}\special{html:; //-->}\fi}
\def\MSubsubsectionx#1{\Dsubsubsectionx{#1}\ifttm\special{html:<!-- sectioninfo;;}\arabic{section}\special{html:;;}\arabic{subsection}\special{html:;;}\arabic{subsubsection}\special{html:;;0;;}\arabic{MTestSite}\special{html:; //-->}\else\addcontentsline{toc}{subsection}{#1}\fi}

\ifttm
\def\MSubsubsubsectionx#1{\ \newline\textbf{#1}\special{html:<br />}}
\else
\def\MSubsubsubsectionx#1{\ \newline
\textbf{#1}\ \\
}
\fi


% Dieses Skript wird zu Beginn jedes Modulabschnitts (=Webseite) ausgefuehrt und initialisiert den Aufgabenfeldzaehler
\newcommand{\MPageScripts}{
\setcounter{MFieldCounter}{1}
\addtocounter{MSiteCounter}{1}
\setcounter{MHintCounter}{1}
\setcounter{MCodeEditCounter}{1}
\setcounter{MGroupActive}{0}
\DoQBoxes
% Feldvariablen werden im HTML-Header in conv.pl eingestellt
}

% Dieses Skript wird zum Ende jedes Modulabschnitts (=Webseite) ausgefuehrt
\ifttm
\newcommand{\MEndScripts}{\special{html:<br /><!-- mfeedbackbutton;Seite;}\arabic{MTestSite}\special{html:;}\MGenerateSiteNumber\special{html:; //-->}
}
\else
\newcommand{\MEndScripts}{\relax}
\fi


\newcounter{QBoxFlag}
\newcommand{\DoQBoxes}{\setcounter{QBoxFlag}{1}}
\newcommand{\NoQBoxes}{\setcounter{QBoxFlag}{0}}

\newcounter{MXCTest}
\newcounter{MXCounter}
\newcounter{MSCounter}



\ifttm

% Struktur des sectioninfo-Tags: <!-- sectioninfo;;section;;subsection;;subsubsection;;nr_ausgeben;;testpage; //-->

%Fuegt eine zusaetzliche html-Seite an hinter ALLEN bisherigen und zukuenftigen content-Seiten ausserhalb der vor-zurueck-Schleife (d.h. nur durch Button oder MIntLink erreichbar!)
% #1 = Titel des Modulabschnitts, #2 = Kurztitel fuer die Buttons, #3 = Buttonkennung (STD = default nehmen, NONE = Ohne Button in der Navigation)
\newenvironment{MSContent}[3]{\special{html:<div class="xcontent}\arabic{MSCounter}\special{html:"><!-- scontent;-;}\arabic{MSCounter};-;#1;-;#2;-;#3\special{html: //-->}\MPageScripts\MSubsubsectionx{#1}}{\MEndScripts\special{html:<!-- endscontent;;}\arabic{MSCounter}\special{html: //--></div>}\addtocounter{MSCounter}{1}}

% Fuegt eine zusaetzliche html-Seite ein hinter den bereits vorhandenen content-Seiten (oder als erste Seite) innerhalb der vor-zurueck-Schleife der Navigation
% #1 = Titel des Modulabschnitts, #2 = Kurztitel fuer die Buttons, #3 = Buttonkennung (STD = Defaultbutton, NONE = Ohne Button in der Navigation)
\newenvironment{MXContent}[3]{\special{html:<div class="xcontent}\arabic{MXCounter}\special{html:"><!-- xcontent;-;}\arabic{MXCounter};-;#1;-;#2;-;#3\special{html: //-->}\MPageScripts\MSubsubsection{#1}}{\MEndScripts\special{html:<!-- endxcontent;;}\arabic{MXCounter}\special{html: //--></div>}\addtocounter{MXCounter}{1}}

% Fuegt eine zusaetzliche html-Seite ein die keine subsubsection-Nummer bekommt, nur zur internen Verwendung in mintmod.tex gedacht!
% #1 = Titel des Modulabschnitts, #2 = Kurztitel fuer die Buttons, #3 = Buttonkennung (STD = Defaultbutton, NONE = Ohne Button in der Navigation)
% \newenvironment{MUContent}[3]{\special{html:<div class="xcontent}\arabic{MXCounter}\special{html:"><!-- xcontent;-;}\arabic{MXCounter};-;#1;-;#2;-;#3\special{html: //-->}\MPageScripts\MSubsubsectionx{#1}}{\MEndScripts\special{html:<!-- endxcontent;;}\arabic{MXCounter}\special{html: //--></div>}\addtocounter{MXCounter}{1}}

\newcommand{\MDeclareSiteUXID}[1]{\special{html:<!-- mdeclaresiteuxid;;}#1\special{html:;;}\arabic{chapter}\special{html:;;}\arabic{section}\special{html:;; //-->}}

\else

%\newcommand{\MSubsubsection}[1]{\refstepcounter{subsubsection} \addcontentsline{toc}{subsubsection}{\thesubsubsection. #1}}


% Fuegt eine zusaetzliche html-Seite an hinter den bereits vorhandenen content-Seiten
% #1 = Titel des Modulabschnitts, #2 = Kurztitel fuer die Buttons, #3 = Iconkennung (im PDF wirkungslos)
%\newenvironment{MUContent}[3]{\ifnum\value{MXCTest}>0{\MDebugMessage{ERROR: Geschachtelter SContent}}\fi\MPageScripts\MSubsubsectionx{#1}\addtocounter{MXCTest}{1}}{\addtocounter{MXCounter}{1}\addtocounter{MXCTest}{-1}}
\newenvironment{MXContent}[3]{\ifnum\value{MXCTest}>0{\MDebugMessage{ERROR: Geschachtelter SContent}}\fi\MPageScripts\MSubsubsection{#1}\addtocounter{MXCTest}{1}}{\addtocounter{MXCounter}{1}\addtocounter{MXCTest}{-1}}
\newenvironment{MSContent}[3]{\ifnum\value{MXCTest}>0{\MDebugMessage{ERROR: Geschachtelter XContent}}\fi\MPageScripts\MSubsubsectionx{#1}\addtocounter{MXCTest}{1}}{\addtocounter{MSCounter}{1}\addtocounter{MXCTest}{-1}}

\newcommand{\MDeclareSiteUXID}[1]{\relax}

\fi 

% GHEADER und GFOOTER werden von split.pm gefunden, aber nur, wenn nicht HELPSITE oder TESTSITE
\ifttm
\newenvironment{MSectionStart}{\special{html:<div class="xcontent0">}\MSubsubsectionx{Modul\"ubersicht}}{\setcounter{MSSEnd}{0}\special{html:</div>}}
% Darf nicht als XContent nummeriert werden, darf nicht als XContent gelabelt werden, wird aber in eine xcontent-div gesetzt fuer Python-parsing
\else
\newenvironment{MSectionStart}{\MSubsectionx{Modul\"ubersicht}}{\setcounter{MSSEnd}{0}}
\fi

\newenvironment{MIntro}{\begin{MXContent}{Einf\"uhrung}{Einf\"uhrung}{genetisch}}{\end{MXContent}}
\newenvironment{MContent}{\begin{MXContent}{Inhalt}{Inhalt}{beweis}}{\end{MXContent}}
\newenvironment{MExercises}{\ifttm\else\clearpage\fi\begin{MXContent}{Aufgaben}{Aufgaben}{aufgb}\special{html:<!-- declareexcsymb //-->}}{\end{MXContent}}

% #1 = Lesbare Testbezeichnung
\newenvironment{MTest}[1]{%
\renewcommand{\MTestName}{#1}
\ifttm\else\clearpage\fi%
\addtocounter{MTestSite}{1}%
\begin{MXContent}{#1}{#1}{STD} % {aufgb}%
\special{html:<!-- declaretestsymb //-->}
\begin{MQuestionGroup}%
\MInTestHeader
}%
{%
\end{MQuestionGroup}%
\ \\ \ \\%
\MInTestFooter
\end{MXContent}\addtocounter{MTestSite}{-1}%
}

\newenvironment{MExtra}{\ifttm\else\clearpage\fi\begin{MXContent}{Zus\"atzliche Inhalte}{Zusatz}{weiterfhrg}}{\end{MXContent}}

\makeindex

\ifttm
\def\MPrintIndex{
\ifnum\value{MSSEnd}>0{\MSubsectionEndMacros}\addtocounter{MSSEnd}{-1}\fi
\renewcommand{\indexname}{Stichwortverzeichnis}
\special{html:<p><!-- printindex //--></p>}
}
\else
\def\MPrintIndex{
\ifnum\value{MSSEnd}>0{\MSubsectionEndMacros}\addtocounter{MSSEnd}{-1}\fi
\renewcommand{\indexname}{Stichwortverzeichnis}
\addcontentsline{toc}{section}{Stichwortverzeichnis}
\printindex
}
\fi


% Konstanten fuer die Modulfaecher

\def\MINTMathematics{1}
\def\MINTInformatics{2}
\def\MINTChemistry{3}
\def\MINTPhysics{4}
\def\MINTEngineering{5}

\newcounter{MSubjectArea}
\newcounter{MInfoNumbers} % Gibt an, ob die Infoboxen nummeriert werden sollen
\newcounter{MSepNumbers} % Gibt an, ob Beispiele und Experimente separat nummeriert werden sollen
\newcommand{\MSetSubject}[1]{
 % ttm kapiert setcounter mit Parametern nicht, also per if abragen und einsetzen
\ifnum#1=1\setcounter{MSubjectArea}{1}\setcounter{MInfoNumbers}{1}\setcounter{MSepNumbers}{0}\fi
\ifnum#1=2\setcounter{MSubjectArea}{2}\setcounter{MInfoNumbers}{1}\setcounter{MSepNumbers}{0}\fi
\ifnum#1=3\setcounter{MSubjectArea}{3}\setcounter{MInfoNumbers}{0}\setcounter{MSepNumbers}{1}\fi
\ifnum#1=4\setcounter{MSubjectArea}{4}\setcounter{MInfoNumbers}{0}\setcounter{MSepNumbers}{0}\fi
\ifnum#1=5\setcounter{MSubjectArea}{5}\setcounter{MInfoNumbers}{1}\setcounter{MSepNumbers}{0}\fi
% Separate Nummerntechnik fuer unsere Chemiker: alles dreistellig
\ifnum#1=3
  \ifttm
  \renewcommand{\theequation}{\arabic{section}.\arabic{subsection}.\arabic{equation}}
  \renewcommand{\thetable}{\arabic{section}.\arabic{subsection}.\arabic{table}} 
  \renewcommand{\thefigure}{\arabic{section}.\arabic{subsection}.\arabic{figure}} 
  \else
  \renewcommand{\theequation}{\arabic{chapter}.\arabic{section}.\arabic{equation}}
  \renewcommand{\thetable}{\arabic{chapter}.\arabic{section}.\arabic{table}}
  \renewcommand{\thefigure}{\arabic{chapter}.\arabic{section}.\arabic{figure}}
  \fi
\else
  \ifttm
  \renewcommand{\theequation}{\arabic{section}.\arabic{subsection}.\arabic{equation}}
  \renewcommand{\thetable}{\arabic{table}}
  \renewcommand{\thefigure}{\arabic{figure}}
  \else
  \renewcommand{\theequation}{\arabic{chapter}.\arabic{section}.\arabic{equation}}
  \renewcommand{\thetable}{\arabic{table}}
  \renewcommand{\thefigure}{\arabic{figure}}
  \fi
\fi
}

% Fuer tikz Autogenerierung
\newcounter{MTIKZAutofilenumber}

% Spezielle Counter fuer die Bentz-Module
\newcounter{mycounter}
\newcounter{chemapplet}
\newcounter{physapplet}

\newcounter{MSSEnd} % Ist 1 falls ein MSubsection aktiv ist, der einen MSubsectionEndMacro-Aufruf verursacht
\newcounter{MFileNumber}
\def\MLastFile{\special{html:[[!-- mfileref;;}\arabic{MFileNumber}\special{html:; //--]]}}

% Vollstaendiger Pfad ist \MMaterial / \MLastFilePath / \MLastFileName    ==   \MMaterial / \MLastFile

% Wird nur bei kompletter Baum-Erstellung ausgefuehrt!
% #1 = Lesbare Modulbezeichnung
\newcommand{\MSectionStartMacros}[1]{
\setcounter{MTestSite}{0}
\setcounter{MCaptionOn}{0}
\setcounter{MLastTypeEq}{0}
\setcounter{MSSEnd}{0}
\setcounter{MFileNumber}{0} % Preinkrekement-Counter
\setcounter{MTIKZAutofilenumber}{0}
\setcounter{mycounter}{1}
\setcounter{physapplet}{1}
\setcounter{chemapplet}{0}
\ifttm
\special{html:<!-- mdeclaresection;;}\arabic{chapter}\special{html:;;}\arabic{section}\special{html:;;}#1\special{html:;; //-->}%
\else
\setcounter{thmc}{0}
\setcounter{exmpc}{0}
\setcounter{verc}{0}
\setcounter{infoc}{0}
\fi
\setcounter{MiniMarkerCounter}{1}
\setcounter{AlignCounter}{1}
\setcounter{MXCTest}{0}
\setcounter{MCodeCounter}{0}
\setcounter{MEntryCounter}{0}
}

% Wird immer ausgefuehrt
\newcommand{\MSubsectionStartMacros}{
\ifttm\else\MPageHeaderDef\fi
\MWatermarkSettings
\setcounter{MXCounter}{0}
\setcounter{MSCounter}{0}
\setcounter{MSiteCounter}{1}
\setcounter{MExerciseCollectionCounter}{0}
% Zaehler fuer das Labelsystem zuruecksetzen (prefix-Zaehler)
\setcounter{MInfoCounter}{0}
\setcounter{MExerciseCounter}{0}
\setcounter{MExampleCounter}{0}
\setcounter{MExperimentCounter}{0}
\setcounter{MGraphicsCounter}{0}
\setcounter{MTableCounter}{0}
\setcounter{MTheoremCounter}{0}
\setcounter{MObjectCounter}{0}
\setcounter{MEquationCounter}{0}
\setcounter{MVideoCounter}{0}
\setcounter{equation}{0}
\setcounter{figure}{0}
}

\newcommand{\MSubsectionEndMacros}{
% Bei Chemiemodulen das PSE einhaengen, es soll als SContent am Ende erscheinen
\special{html:<!-- subsectionend //-->}
\ifnum\value{MSubjectArea}=3{\MIncludePSE}\fi
}


\ifttm
%\newcommand{\MEmbed}[1]{\MRegisterFile{#1}\begin{html}<embed src="\end{html}\MMaterial/\MLastFile\begin{html}" width="192" height="189"></embed>\end{html}}
\newcommand{\MEmbed}[1]{\MRegisterFile{#1}\begin{html}<embed src="\end{html}\MMaterial/\MLastFile\begin{html}"></embed>\end{html}}
\fi

%----------------- Makros fuer die Textdarstellung -----------------------------------------------

\ifttm
% MUGraphics bindet eine Grafik ein:
% Parameter 1: Dateiname der Grafik, relativ zur Position des Modul-Tex-Dokuments
% Parameter 2: Skalierungsoptionen fuer PDF (fuer includegraphics)
% Parameter 3: Titel fuer die Grafik, wird unter die Grafik mit der Grafiknummer gesetzt und kann MLabel bzw. MCopyrightLabel enthalten
% Parameter 4: Skalierungsoptionen fuer HTML (css-styles)

% ERSATZ: <img alt="My Image" src="data:image/png;base64,iVBORwA<MoreBase64SringHere>" />


\newcommand{\MUGraphics}[4]{\MRegisterFile{#1}\begin{html}
<div class="imagecenter">
<center>
<div>
<img src="\end{html}\MMaterial/\MLastFile\begin{html}" style="#4" alt="\end{html}\MMaterial/\MLastFile\begin{html}"/>
</div>
<div class="bildtext">
\end{html}
\addtocounter{MGraphicsCounter}{1}
\setcounter{MLastIndex}{\value{MGraphicsCounter}}
\setcounter{MLastType}{8}
\addtocounter{MCaptionOn}{1}
\ifnum\value{MSepNumbers}=0
\textbf{Abbildung \arabic{MGraphicsCounter}:} #3
\else
\textbf{Abbildung \arabic{section}.\arabic{subsection}.\arabic{MGraphicsCounter}:} #3
\fi
\addtocounter{MCaptionOn}{-1}
\begin{html}
</div>
</center>
</div>
<br />
\end{html}%
\special{html:<!-- mfeedbackbutton;Abbildung;}\arabic{MGraphicsCounter}\special{html:;}\arabic{section}.\arabic{subsection}.\arabic{MGraphicsCounter}\special{html:; //-->}%
}

% MVideo bindet ein Video als Einzeldatei ein:
% Parameter 1: Dateiname des Videos, relativ zur Position des Modul-Tex-Dokuments, ohne die Endung ".mp4"
% Parameter 2: Titel fuer das Video (kann MLabel oder MCopyrightLabel enthalten), wird unter das Video mit der Videonummer gesetzt
\newcommand{\MVideo}[2]{\MRegisterFile{#1.mp4}\begin{html}
<div class="imagecenter">
<center>
<div>
<video width="95\%" controls="controls"><source src="\end{html}\MMaterial/#1.mp4\begin{html}" type="video/mp4">Ihr Browser kann keine MP4-Videos abspielen!</video>
</div>
<div class="bildtext">
\end{html}
\addtocounter{MVideoCounter}{1}
\setcounter{MLastIndex}{\value{MVideoCounter}}
\setcounter{MLastType}{12}
\addtocounter{MCaptionOn}{1}
\ifnum\value{MSepNumbers}=0
\textbf{Video \arabic{MVideoCounter}:} #2
\else
\textbf{Video \arabic{section}.\arabic{subsection}.\arabic{MVideoCounter}:} #2
\fi
\addtocounter{MCaptionOn}{-1}
\begin{html}
</div>
</center>
</div>
<br />
\end{html}}

\newcommand{\MDVideo}[2]{\MRegisterFile{#1.mp4}\MRegisterFile{#1.ogv}\begin{html}
<div class="imagecenter">
<center>
<div>
<video width="70\%" controls><source src="\end{html}\MMaterial/#1.mp4\begin{html}" type="video/mp4"><source src="\end{html}\MMaterial/#1.ogv\begin{html}" type="video/ogg">Ihr Browser kann keine MP4-Videos abspielen!</video>
</div>
<br />
#2
</center>
</div>
<br />
\end{html}
}

\newcommand{\MGraphics}[3]{\MUGraphics{#1}{#2}{#3}{}}

\else

\newcommand{\MVideo}[2]{%
% Kein Video im PDF darstellbar, trotzdem so tun als ob da eines waere
\begin{center}
(Video nicht darstellbar)
\end{center}
\addtocounter{MVideoCounter}{1}
\setcounter{MLastIndex}{\value{MVideoCounter}}
\setcounter{MLastType}{12}
\addtocounter{MCaptionOn}{1}
\ifnum\value{MSepNumbers}=0
\textbf{Video \arabic{MVideoCounter}:} #2
\else
\textbf{Video \arabic{section}.\arabic{subsection}.\arabic{MVideoCounter}:} #2
\fi
\addtocounter{MCaptionOn}{-1}
}


% MGraphics bindet eine Grafik ein:
% Parameter 1: Dateiname der Grafik, relativ zur Position des Modul-Tex-Dokuments
% Parameter 2: Skalierungsoptionen fuer PDF (fuer includegraphics)
% Parameter 3: Titel fuer die Grafik, wird unter die Grafik mit der Grafiknummer gesetzt
\newcommand{\MGraphics}[3]{%
\MRegisterFile{#1}%
\ %
\begin{figure}[H]%
\centering{%
\includegraphics[#2]{\MDPrefix/#1}%
\addtocounter{MCaptionOn}{1}%
\caption{#3}%
\addtocounter{MCaptionOn}{-1}%
}%
\end{figure}%
\addtocounter{MGraphicsCounter}{1}\setcounter{MLastIndex}{\value{MGraphicsCounter}}\setcounter{MLastType}{8}\ %
%\ \\Abbildung \ifnum\value{MSepNumbers}=0\else\arabic{chapter}.\arabic{section}.\fi\arabic{MGraphicsCounter}: #3%
}

\newcommand{\MUGraphics}[4]{\MGraphics{#1}{#2}{#3}}


\fi

\newcounter{MCaptionOn} % = 1 falls eine Grafikcaption aktiv ist, = 0 sonst


% MGraphicsSolo bindet eine Grafik pur ein ohne Titel
% Parameter 1: Dateiname der Grafik, relativ zur Position des Modul-Tex-Dokuments
% Parameter 2: Skalierungsoptionen (wirken nur im PDF)
\newcommand{\MGraphicsSolo}[2]{\MUGraphicsSolo{#1}{#2}{}}

% MUGraphicsSolo bindet eine Grafik pur ein ohne Titel, aber mit HTML-Skalierung
% Parameter 1: Dateiname der Grafik, relativ zur Position des Modul-Tex-Dokuments
% Parameter 2: Skalierungsoptionen (wirken nur im PDF)
% Parameter 3: Skalierungsoptionen (wirken nur im HTML), als style-format: "width=???, height=???"
\ifttm
\newcommand{\MUGraphicsSolo}[3]{\MRegisterFile{#1}\begin{html}
<img src="\end{html}\MMaterial/\MLastFile\begin{html}" style="\end{html}#3\begin{html}" alt="\end{html}\MMaterial/\MLastFile\begin{html}"/>
\end{html}%
\special{html:<!-- mfeedbackbutton;Abbildung;}#1\special{html:;}\MMaterial/\MLastFile\special{html:; //-->}%
}
\else
\newcommand{\MUGraphicsSolo}[3]{\MRegisterFile{#1}\includegraphics[#2]{\MDPrefix/#1}}
\fi

% Externer Link mit URL
% Erster Parameter: Vollstaendige(!) URL des Links
% Zweiter Parameter: Text fuer den Link
\newcommand{\MExtLink}[2]{\ifttm\special{html:<a target="_new" href="}#1\special{html:">}#2\special{html:</a>}\else\href{#1}{#2}\fi} % ohne MINTERLINK!


% Interner Link, die verlinkte Datei muss im gleichen Verzeichnis liegen wie die Modul-Texdatei
% Erster Parameter: Dateiname
% Zweiter Parameter: Text fuer den Link
\newcommand{\MIntLink}[2]{\ifttm\MRegisterFile{#1}\special{html:<a class="MINTERLINK" target="_new" href="}\MMaterial/\MLastFile\special{html:">}#2\special{html:</a>}\else{\href{#1}{#2}}\fi}


\ifttm
\def\MMaterial{:localmaterial:}
\else
\def\MMaterial{\MDPrefix}
\fi

\ifttm
\def\MNoFile#1{:directmaterial:#1}
\else
\def\MNoFile#1{#1}
\fi

\newcommand{\MChem}[1]{$\mathrm{#1}$}

\newcommand{\MApplet}[3]{
% Bindet ein Java-Applet ein, die Parameter sind:
% (wird nur im HTML, aber nicht im PDF erstellt)
% #1 Dateiname des Applets (muss mit ".class" enden)
% #2 = Breite in Pixeln
% #3 = Hoehe in Pixeln
\ifttm
\MRegisterFile{#1}
\begin{html}
<applet code="\end{html}\MMaterial/\MLastFile\begin{html}" width="#2" height="#3" alt="[Java-Applet kann nicht gestartet werden]"></applet>
\end{html}
\fi
}

\newcommand{\MScriptPage}[2]{
% Bindet eine JavaScript-Datei ein, die eine eigene Seite bekommt
% (wird nur im HTML, aber nicht im PDF erstellt)
% #1 Dateiname des Programms (sollte mit ".js" enden)
% #2 = Kurztitel der Seite
\ifttm
\begin{MSContent}{#2}{#2}{puzzle}
\MRegisterFile{#1}
\begin{html}
<script src="\MMaterial/\MLastFile" type="text/javascript"></script>
\end{html}
\end{MSContent}
\fi
}

\newcommand{\MIncludePSE}{
% Bindet bei Chemie-Modulen das PSE ein
% (wird nur im HTML, aber nicht im PDF erstellt)
\ifttm
\special{html:<!-- includepse //-->}
\begin{MSContent}{Periodensystem der Elemente}{PSE}{table}
\MRegisterFile{../files/pse.js}
\MRegisterFile{../files/radio.png}
\begin{html}
<script src="\MMaterial/../files/pse.js" type="text/javascript"></script>
<p id="divid"><br /><br />
<script language="javascript" type="text/javascript">
    startpse("divid","\MMaterial/../files"); 
</script>
</p>
<br />
<br />
<br />
<p>Die Farben der Elementsymbole geben an: <font style="color:Red">gasf&ouml;rmig </font> <font style="color:Blue">fl&uuml;ssig </font> fest</p>
<p>Die Elemente der Gruppe 1 A, 2 A, 3 A usw. geh&ouml;ren zu den Hauptgruppenelementen.</p>
<p>Die Elemente der Gruppe 1 B, 2 B, 3 B usw. geh&ouml;ren zu den Nebengruppenelementen.</p>
<p>() kennzeichnet die Masse des stabilsten Isotops</p>
\end{html}
\end{MSContent}
\fi
}

\newcommand{\MAppletArchive}[4]{
% Bindet ein Java-Applet ein, die Parameter sind:
% (wird nur im HTML, aber nicht im PDF erstellt)
% #1 Dateiname der Klasse mit Appletaufruf (muss mit ".class" enden)
% #2 Dateiname des Archivs (muss mit ".jar" enden)
% #3 = Breite in Pixeln
% #4 = Hoehe in Pixeln
\ifttm
\MRegisterFile{#2}
\begin{html}
<applet code="#1" archive="\end{html}\MMaterial/\MLastFile\begin{html}" codebase="." width="#3" height="#4" alt="[Java-Archiv kann nicht gestartet werden]"></applet>
\end{html}
\fi
}

% Bindet in der Haupttexdatei ein MINT-Modul ein. Parameter 1 ist das Verzeichnis (relativ zur Haupttexdatei), Parameter 2 ist der Dateinahme ohne Pfad.
\newcommand{\IncludeModule}[2]{
\renewcommand{\MDPrefix}{#1}
\input{#1/#2}
\ifnum\value{MSSEnd}>0{\MSubsectionEndMacros}\addtocounter{MSSEnd}{-1}\fi
}

% Der ttm-Konverter setzt keine Makros im \input um, also muss hier getrickst werden:
% Das MDPrefix muss in den einzelnen Modulen manuell eingesetzt werden
\newcommand{\MInputFile}[1]{
\ifttm
\input{#1}
\else
\input{#1}
\fi
}


\newcommand{\MCases}[1]{\left\lbrace{\begin{array}{rl} #1 \end{array}}\right.}

\ifttm
\newenvironment{MCaseEnv}{\left\lbrace\begin{array}{rl}}{\end{array}\right.}
\else
\newenvironment{MCaseEnv}{\left\lbrace\begin{array}{rl}}{\end{array}\right.}
\fi

\def\MSkip{\ifttm\MCR\fi}

\ifttm
\def\MCR{\special{html:<br />}}
\else
\def\MCR{\ \\}
\fi


% Pragmas - Sind Schluesselwoerter, die dem Preprocessing sowie dem Konverter uebergeben werden und bestimmte
%           Aktionen ausloesen. Im Output (PDF und HTML) tauchen sie nicht auf.
\newcommand{\MPragma}[1]{%
\ifttm%
\special{html:<!-- mpragma;-;}#1\special{html:;; -->}%
\else%
% MPragmas werden vom Preprozessor direkt im LaTeX gefunden
\fi%
}

% Ersatz der Befehle textsubscript und textsuperscript, die ttm nicht kennt
\ifttm%
\newcommand{\MTextsubscript}[1]{\special{html:<sub>}#1\special{html:</sub>}}%
\newcommand{\MTextsuperscript}[1]{\special{html:<sup>}#1\special{html:</sup>}}%
\else%
\newcommand{\MTextsubscript}[1]{\textsubscript{#1}}%
\newcommand{\MTextsuperscript}[1]{\textsuperscript{#1}}%
\fi

%------------------ Einbindung von dia-Diagrammen ----------------------------------------------
% Beim preprocessing wird aus jeder dia-Datei eine tex-Datei und eine pdf-Datei erzeugt,
% diese werden hier jeweils im PDF und HTML eingebunden
% Parameter: Dateiname der mit dia erstellten Datei (OHNE die Endung .dia)
\ifttm%
\newcommand{\MDia}[1]{%
\MGraphicsSolo{#1minthtml.png}{}%
}
\else%
\newcommand{\MDia}[1]{%
\MGraphicsSolo{#1mintpdf.png}{scale=0.1667}%
}
\fi%

% subsup funktioniert im Ausdruck $D={\R}^+_0$, also \R geklammert und sup zuerst
% \ifttm
% \def\MSubsup#1#2#3{\special{html:<msubsup>} #1 #2 #3\special{html:</msubsup>}}
% \else
% \def\MSubsup#1#2#3{{#1}^{#3}_{#2}}
% \fi

%\input{local.tex}

% \ifttm
% \else
% \newwrite\mintlog
% \immediate\openout\mintlog=mintlog.txt
% \fi

% ----------------------- tikz autogenerator -------------------------------------------------------------------

\newcommand{\Mtikzexternalize}{\tikzexternalize}% wird bei Konvertierung ueber mconvert ggf. ausgehebelt!

\ifttm
\else
\tikzset%
{
  % Defines a custom style which generates pdf and converts to (low and hi-res quality) png and svg, then deletes the pdf
  % Important: DO NOT directly convert from pdf to hires-png or from svg to png with GraphViz convert as it has some problems and memory leaks
  png export/.style=%
  {
    external/system call/.add={}{; 
      pdf2svg "\image.pdf" "\image.svg" ; 
      convert -density 112.5 -transparent white "\image.pdf" "\image.png"; 
      inkscape --export-png="\image.4x.png" --export-dpi=450 --export-background-opacity=0 --without-gui "\image.svg"; 
      rm "\image.pdf"; rm "\image.log"; rm "\image.dpth"; rm "\image.idx"
    },
    external/force remake,
  }
}
\tikzset{png export}
\tikzsetexternalprefix{}
% PNGs bei externer Erzeugung in "richtiger" Groesse einbinden
\pgfkeys{/pgf/images/include external/.code={\includegraphics[scale=0.64]{#1}}}
\fi

% Spezielle Umgebung fuer Autogenerierung, Bildernamen sind nur innerhalb eines Moduls (einer MSection) eindeutig)

\newcommand{\MTIKZautofilename}{tikzautofile}

\ifttm
% HTML-Version: Vom Autogenerator erzeugte png-Datei einbinden, tikz selbst nicht ausfuehren (sprich: #1 schlucken)
\newcommand{\MTikzAuto}[1]{%
\addtocounter{MTIKZAutofilenumber}{1}
\renewcommand{\MTIKZautofilename}{mtikzauto_\arabic{MTIKZAutofilenumber}}
\MUGraphicsSolo{\MSectionID\MTIKZautofilename.4x.png}{scale=1}{\special{html:[[!-- svgstyle;}\MSectionID\MTIKZautofilename\special{html: //--]]}} % Styleinfos werden aus original-png, nicht 4x-png geholt!
%\MRegisterFile{\MSectionID\MTIKZautofilename.png} % not used right now
%\MRegisterFile{\MSectionID\MTIKZautofilename.svg}
}
\else%
% PDF-Version: Falls Autogenerator aktiv wird Datei automatisch benannt und exportiert
\newcommand{\MTikzAuto}[1]{%
\addtocounter{MTIKZAutofilenumber}{1}%
\renewcommand{\MTIKZautofilename}{mtikzauto_\arabic{MTIKZAutofilenumber}}
\tikzsetnextfilename{\MTIKZautofilename}%
#1%
}
\fi

% In einer reinen LaTeX-Uebersetzung kapselt der Preambelinclude-Befehl nur input,
% in einer konvertergesteuerten PDF/HTML-Uebersetzung wird er dagegen entfernt und
% die Preambeln an mintmod angehaengt, die Ersetzung wird von mconvert.pl vorgenommen.

\newcommand{\MPreambleInclude}[1]{\input{#1}}

% Globale Watermarksettings (werden auch nochmal zu Beginn jedes subsection gesetzt,
% muessen hier aber auch global ausgefuehrt wegen Einfuehrungsseiten und Inhaltsverzeichnis

\MWatermarkSettings
% ---------------------------------- Parametrisierte Aufgaben ----------------------------------------

\ifttm
\newenvironment{MPExercise}{%
\begin{MExercise}%
}{%
\special{html:<button name="Name_MPEX}\arabic{MExerciseCounter}\special{html:" id="MPEX}\arabic{MExerciseCounter}%
\special{html:" type="button" onclick="reroll('}\arabic{MExerciseCounter}\special{html:');">Neue Aufgabe erzeugen</button>}%
\end{MExercise}%
}
\else
\newenvironment{MPExercise}{%
\begin{MExercise}%
}{%
\end{MExercise}%
}
\fi

% Parameter: Name, Min, Max, PDF-Standard. Name in Deklaration OHNE backslash, im Code MIT Backslash
\ifttm
\newcommand{\MGlobalInteger}[4]{\special{html:%
<!-- onloadstart //-->%
MVAR.push(createGlobalInteger("}#1\special{html:",}#2\special{html:,}#3\special{html:,}#4\special{html:)); %
<!-- onloadstop //-->%
<!-- viewmodelstart //-->%
ob}#1\special{html:: ko.observable(rerollMVar("}#1\special{html:")),%
<!-- viewmodelstop //-->%
}%
}%
\else%
\newcommand{\MGlobalInteger}[4]{\newcounter{mvc_#1}\setcounter{mvc_#1}{#4}}
\fi

% Parameter: Name, Min, Max, PDF-Standard. Name in Deklaration OHNE backslash, im Code MIT Backslash, Wert ist Wurzel von value
\ifttm
\newcommand{\MGlobalSqrt}[4]{\special{html:%
<!-- onloadstart //-->%
MVAR.push(createGlobalSqrt("}#1\special{html:",}#2\special{html:,}#3\special{html:,}#4\special{html:)); %
<!-- onloadstop //-->%
<!-- viewmodelstart //-->%
ob}#1\special{html:: ko.observable(rerollMVar("}#1\special{html:")),%
<!-- viewmodelstop //-->%
}%
}%
\else%
\newcommand{\MGlobalSqrt}[4]{\newcounter{mvc_#1}\setcounter{mvc_#1}{#4}}% Funktioniert nicht als Wurzel !!!
\fi

% Parameter: Name, Min, Max, PDF-Standard zaehler, PDF-Standard nenner. Name in Deklaration OHNE backslash, im Code MIT Backslash
\ifttm
\newcommand{\MGlobalFraction}[5]{\special{html:%
<!-- onloadstart //-->%
MVAR.push(createGlobalFraction("}#1\special{html:",}#2\special{html:,}#3\special{html:,}#4\special{html:,}#5\special{html:)); %
<!-- onloadstop //-->%
<!-- viewmodelstart //-->%
ob}#1\special{html:: ko.observable(rerollMVar("}#1\special{html:")),%
<!-- viewmodelstop //-->%
}%
}%
\else%
\newcommand{\MGlobalFraction}[5]{\newcounter{mvc_#1}\setcounter{mvc_#1}{#4}} % Funktioniert nicht als Bruch !!!
\fi

% MVar darf im HTML nur in MEvalMathDisplay-Umgebungen genutzt werden oder in Strings die an den Parser uebergeben werden
\ifttm%
\newcommand{\MVar}[1]{\special{html:[var_}#1\special{html:]}}%
\else%
\newcommand{\MVar}[1]{\arabic{mvc_#1}}%
\fi

\ifttm%
\newcommand{\MRerollButton}[2]{\special{html:<button type="button" onclick="rerollMVar('}#1\special{html:');">}#2\special{html:</button>}}%
\else%
\newcommand{\MRerollButton}[2]{\relax}% Keine sinnvolle Entsprechung im PDF
\fi

% MEvalMathDisplay fuer HTML wird in mconvert.pl im preprocessing realisiert
% PDF: eine equation*-Umgebung (ueber amsmath)
% HTML: Eine Mathjax-Tex-Umgebung, deren Auswertung mit knockout-obervablen gekoppelt ist
% PDF-Version hier nur fuer pdflatex-only-Uebersetzung gegeben

\ifttm\else\newenvironment{MEvalMathDisplay}{\begin{equation*}}{\end{equation*}}\fi

% ---------------------------------- Spezialbefehle fuer AD ------------------------------------------

%Abk�rzung f�r \longrightarrow:
\newcommand{\lto}{\ensuremath{\longrightarrow}}

%Makro f�r Funktionen:
\newcommand{\exfunction}[5]
{\begin{array}{rrcl}
 #1 \colon  & #2 &\lto & #3 \\[.05cm]  
  & #4 &\longmapsto  & #5 
\end{array}}

\newcommand{\function}[5]{%
#1:\;\left\lbrace{\begin{array}{rcl}
 #2 &\lto & #3 \\
 #4 &\longmapsto  & #5 \end{array}}\right.}


%Die Identit�t:
\DeclareMathOperator{\Id}{Id}

%Die Signumfunktion:
\DeclareMathOperator{\sgn}{sgn}

%Zwei Betonungskommandos (k�nnen angepasst werden):
\newcommand{\highlight}[1]{#1}
\newcommand{\modstextbf}[1]{#1}
\newcommand{\modsemph}[1]{#1}


% ---------------------------------- Spezialbefehle fuer JL ------------------------------------------


\def\jccolorfkt{green!50!black} %Farbe des Funktionsgraphen
\def\jccolorfktarea{green!25!white} %Farbe der Fl"ache unter dem Graphen
\def\jccolorfktareahell{green!12!white} %helle Einf"arbung der Fl"ache unter dem Graphen
\def\jccolorfktwert{green!50!black} %Farbe einzelner Punkte des Graphen

\newcommand{\MPfadBilder}{Bilder}

\ifttm%
\newcommand{\jMD}{\,\MD}%
\else%
\newcommand{\jMD}{\;\MD}%
\fi%

\def\jHTMLHinweisBedienung{\MInputHint{%
Mit Hilfe der Symbole am oberen Rand des Fensters
k"onnen Sie durch die einzelnen Abschnitte navigieren.}}

\def\jHTMLHinweisEingabeText{\MInputHint{%
Geben Sie jeweils ein Wort oder Zeichen als Antwort ein.}}

\def\jHTMLHinweisEingabeTerm{\MInputHint{%
Klammern Sie Ihre Terme, um eine eindeutige Eingabe zu erhalten. 
Beispiel: Der Term $\frac{3x+1}{x-2}$ soll in der Form
\texttt{(3*x+1)/((x+2)^2}$ eingegeben werden (wobei auch Leerzeichen 
eingegeben werden k"onnen, damit eine Formel besser lesbar ist).}}

\def\jHTMLHinweisEingabeIntervalle{\MInputHint{%
Intervalle werden links mit einer "offnenden Klammer und rechts mit einer 
schlie"senden Klammer angegeben. Eine runde Klammer wird verwendet, wenn der 
Rand nicht dazu geh"ort, eine eckige, wenn er dazu geh"ort. 
Als Trennzeichen wird ein Komma oder ein Semikolon akzeptiert.
Beispiele: $(a, b)$ offenes Intervall,
$[a; b)$ links abgeschlossenes, rechts offenes Intervall von $a$ bis $b$. 
Die Eingabe $]a;b[$ f"ur ein offenes Intervall wird nicht akzeptiert.
F"ur $\infty$ kann \texttt{infty} oder \texttt{unendlich} geschrieben werden.}}

\def\jHTMLHinweisEingabeFunktionen{\MInputHint{%
Schreiben Sie Malpunkte (geschrieben als \texttt{*}) aus und setzen Sie Klammern um Argumente f�r Funktionen.
Beispiele: Polynom: \texttt{3*x + 0.1}, Sinusfunktion: \texttt{sin(x)}, 
Verkettung von cos und Wurzel: \texttt{cos(sqrt(3*x))}.}}

\def\jHTMLHinweisEingabeFunktionenSinCos{\MInputHint{%
Die Sinusfunktion $\sin x$ wird in der Form \texttt{sin(x)} angegeben, %
$\cos\left(\sqrt{3 x}\right)$ durch \texttt{cos(sqrt(3*x))}.}}

\def\jHTMLHinweisEingabeFunktionenExp{\MInputHint{%
Die Exponentialfunktion $\MEU^{3x^4 + 5}$ wird als
\texttt{exp(3 * x^4 + 5)} angegeben, %
$\ln\left(\sqrt{x} + 3.2\right)$ durch \texttt{ln(sqrt(x) + 3.2)}.}}

% ---------------------------------- Spezialbefehle fuer Fachbereich Physik --------------------------

\newcommand{\E}{{e}}
\newcommand{\ME}[1]{\cdot 10^{#1}}
\newcommand{\MU}[1]{\;\mathrm{#1}}
\newcommand{\MPG}[3]{%
  \ifnum#2=0%
    #1\ \mathrm{#3}%
  \else%
    #1\cdot 10^{#2}\ \mathrm{#3}%
  \fi}%
%

\newcommand{\MMul}{\MExponentensymbXYZl} % Nur eine Abkuerzung


% ---------------------------------- Stichwortfunktionialitaet ---------------------------------------

% mpreindexentry wird durch Auswahlroutine in conv.pl durch mindexentry substitutiert
\ifttm%
\def\MIndex#1{\index{#1}\special{html:<!-- mpreindexentry;;}#1\special{html:;;}\arabic{MSubjectArea}\special{html:;;}%
\arabic{chapter}\special{html:;;}\arabic{section}\special{html:;;}\arabic{subsection}\special{html:;;}\arabic{MEntryCounter}\special{html:; //-->}%
\setcounter{MLastIndex}{\value{MEntryCounter}}%
\addtocounter{MEntryCounter}{1}%
}%
% Copyrightliste wird als tex-Datei im preprocessing von conv.pl erzeugt und unter converter/tex/entrycollection.tex abgelegt
% Der input-Befehl funktioniert nur, wenn die aufrufende tex-Datei auf der obersten Ebene liegt (d.h. selbst kein input/include ist, insbesondere keine Moduldatei)
\def\MEntryList{} % \input funktioniert nicht, weil ttm (und damit das \input) ausgefuehrt wird, bevor Datei da ist
\else%
\def\MIndex#1{\index{#1}}
\def\MEntryList{\MAbort{Stichwortliste nur im HTML realisierbar}}%
\fi%

\def\MEntry#1#2{\textbf{#1}\MIndex{#2}} % Idee: MLastType auf neuen Entry-Typ und dann ein MLabel vergeben mit autogen-Nummer

% ---------------------------------- Befehle fuer Tests ----------------------------------------------

% MEquationItem stellt eine Eingabezeile der Form Vorgabe = Antwortfeld her, der zweite Parameter kann z.B. MSimplifyQuestion-Befehl sein
\ifttm
\newcommand{\MEquationItem}[2]{{#1}$\,=\,${#2}}%
\else%
\newcommand{\MEquationItem}[2]{{#1}$\;\;=\,${#2}}%
\fi

\ifttm
\newcommand{\MInputHint}[1]{%
\ifnum%
\if\value{MTestSite}>0%
\else%
{\color{blue}#1}%
\fi%
\fi%
}
\else
\newcommand{\MInputHint}[1]{\relax}
\fi

\ifttm
\newcommand{\MInTestHeader}{%
Dies ist ein einreichbarer Test:
\begin{itemize}
\item{Im Gegensatz zu den offenen Aufgaben werden beim Eingeben keine Hinweise zur Formulierung der mathematischen Ausdr�cke gegeben.}
\item{Der Test kann jederzeit neu gestartet oder verlassen werden.}
\item{Der Test kann durch die Buttons am Ende der Seite beendet und abgeschickt, oder zur�ckgesetzt werden.}
\item{Der Test kann mehrfach probiert werden. F�r die Statistik z�hlt die zuletzt abgeschickte Version.}
\end{itemize}
}
\else
\newcommand{\MInTestHeader}{%
\relax
}
\fi

\ifttm
\newcommand{\MInTestFooter}{%
\special{html:<button name="Name_TESTFINISH" id="TESTFINISH" type="button" onclick="finish_button('}\MTestName\special{html:');">Test auswerten</button>}%
\begin{html}
&nbsp;&nbsp;&nbsp;&nbsp;&nbsp;&nbsp;&nbsp;&nbsp;
<button name="Name_TESTRESET" id="TESTRESET" type="button" onclick="reset_button();">Test zur�cksetzen</button>
<br />
<br />
<div class="xreply">
<p name="Name_TESTEVAL" id="TESTEVAL">
Hier erscheint die Testauswertung!
<br />
</p>
</div>
\end{html}
}
\else
\newcommand{\MInTestFooter}{%
\relax
}
\fi


% ---------------------------------- Notationsmakros -------------------------------------------------------------

% Notationsmakros die nicht von der Kursvariante abhaengig sind

\newcommand{\MZahltrennzeichen}[1]{\renewcommand{\MZXYZhltrennzeichen}{#1}}

\ifttm
\newcommand{\MZahl}[3][\MZXYZhltrennzeichen]{\edef\MZXYZtemp{\noexpand\special{html:<mn>#2#1#3</mn>}}\MZXYZtemp}
\else
\newcommand{\MZahl}[3][\MZXYZhltrennzeichen]{{}#2{#1}#3}
\fi

\newcommand{\MEinheitenabstand}[1]{\renewcommand{\MEinheitenabstXYZnd}{#1}}
\ifttm
\newcommand{\MEinheit}[2][\MEinheitenabstXYZnd]{{}#1\edef\MEINHtemp{\noexpand\special{html:<mi mathvariant="normal">#2</mi>}}\MEINHtemp} 
\else
\newcommand{\MEinheit}[2][\MEinheitenabstXYZnd]{{}#1 \mathrm{#2}} 
\fi

\newcommand{\MExponentensymbol}[1]{\renewcommand{\MExponentensymbXYZl}{#1}}
\newcommand{\MExponent}[2][\MExponentensymbXYZl]{{}#1{} 10^{#2}} 

%Punkte in 2 und 3 Dimensionen
\newcommand{\MPointTwo}[3][]{#1(#2\MCoordPointSep #3{}#1)}
\newcommand{\MPointThree}[4][]{#1(#2\MCoordPointSep #3\MCoordPointSep #4{}#1)}
\newcommand{\MPointTwoAS}[2]{\left(#1\MCoordPointSep #2\right)}
\newcommand{\MPointThreeAS}[3]{\left(#1\MCoordPointSep #2\MCoordPointSep #3\right)}

% Masseinheit, Standardabstand: \,
\newcommand{\MEinheitenabstXYZnd}{\MThinspace} 

% Horizontaler Leerraum zwischen herausgestellter Formel und Interpunktion
\ifttm
\newcommand{\MDFPSpace}{\,}
\newcommand{\MDFPaSpace}{\,\,}
\newcommand{\MBlank}{\ }
\else
\newcommand{\MDFPSpace}{\;}
\newcommand{\MDFPaSpace}{\;\;}
\newcommand{\MBlank}{\ }
\fi

% Satzende in herausgestellter Formel mit horizontalem Leerraum
\newcommand{\MDFPeriod}{\MDFPSpace .}

% Separation von Aufzaehlung und Bedingung in Menge
\newcommand{\MCondSetSep}{\,:\,} %oder '\mid'

% Konverter kennt mathopen nicht
\ifttm
\def\mathopen#1{}
\fi

% -----------------------------------START Rouletteaufgaben ------------------------------------------------------------

\ifttm
% #1 = Dateiname, #2 = eindeutige ID fuer das Roulette im Kurs
\newcommand{\MDirectRouletteExercises}[2]{
\begin{MExercise}
\texttt{Im HTML erscheinen hier Aufgaben aus einer Aufgabenliste...}
\end{MExercise}
}
\else
\newcommand{\MDirectRouletteExercises}[2]{\relax} % wird durch mconvert.pl gefunden und ersetzt
\fi


% ---------------------------------- START Makros, die von der Kursvariante abhaengen ----------------------------------

\ifvariantunotation
  % unotation = An Universitaeten uebliche Notation
  \def\MVariant{unotation}

  % Trennzeichen fuer Dezimalzahlen
  \newcommand{\MZXYZhltrennzeichen}{.}

  % Exponent zur Basis 10 in der Exponentialschreibweise, 
  % Standardmalzeichen: \times
  \newcommand{\MExponentensymbXYZl}{\times} 

  % Begrenzungszeichen fuer offene Intervalle
  \newcommand{\MoIl}[1][]{\mbox{}#1(\mathopen{}} % bzw. ']'
  \newcommand{\MoIr}[1][]{#1)\mbox{}} % bzw. '['

  % Zahlen-Separation im IntervaLL
  \newcommand{\MIntvlSep}{,} %oder ';'

  % Separation von Elementen in Mengen
  \newcommand{\MElSetSep}{,} %oder ';'

  % Separation von Koordinaten in Punkten
  \newcommand{\MCoordPointSep}{,} %oder ';' oder '|', '\MThinspace|\MThinspace'

\else
  % An dieser Stelle wird angenommen, dass std-Variante aktiv ist
  % std = beschlossene Notation im TU9-Onlinekurs 
  \def\MVariant{std}

  % Trennzeichen fuer Dezimalzahlen
  \newcommand{\MZXYZhltrennzeichen}{,}

  % Exponent zur Basis 10 in der Exponentialschreibweise, 
  % Standardmalzeichen: \times
  \newcommand{\MExponentensymbXYZl}{\times} 

  % Begrenzungszeichen fuer offene Intervalle
  \newcommand{\MoIl}[1][]{\mbox{}#1]\mathopen{}} % bzw. '('
  \newcommand{\MoIr}[1][]{#1[\mbox{}} % bzw. ')'

  % Zahlen-Separation im IntervaLL
  \newcommand{\MIntvlSep}{;} %oder ','
  
  % Separation von Elementen in Mengen
  \newcommand{\MElSetSep}{;} %oder ','

  % Separation von Koordinaten in Punkten
  \newcommand{\MCoordPointSep}{;} %oder '|', '\MThinspace|\MThinspace'

\fi



% ---------------------------------- ENDE Makros, die von der Kursvariante abhaengen ----------------------------------


% diese Kommandos setzen Mathemodus vorraus
\newcommand{\MGeoAbstand}[2]{[\overline{{#1}{#2}}]}
\newcommand{\MGeoGerade}[2]{{#1}{#2}}
\newcommand{\MGeoStrecke}[2]{\overline{{#1}{#2}}}
\newcommand{\MGeoDreieck}[3]{{#1}{#2}{#3}}

%
\ifttm
\newcommand{\MOhm}{\special{html:<mn>&#x3A9;</mn>}}
\else
\newcommand{\MOhm}{\Omega} %\varOmega
\fi


\def\PERCTAG{\MAbort{PERCTAG ist zur internen verwendung in mconvert.pl reserviert, dieses Makro darf sonst nicht benutzt werden.}}

% Im Gegensatz zu einfachen html-Umgebungen werden MDirectHTML-Umgebungen von mconvert.pl am ganzen ttm-Prozess vorbeigeschleust und aus dem PDF komplett ausgeschnitten
\ifttm%
\newenvironment{MDirectHTML}{\begin{html}}{\end{html}}%
\else%
\newenvironment{MDirectHTML}{\begin{html}}{\end{html}}%
\fi

% Im Gegensatz zu einfachen Mathe-Umgebungen werden MDirectMath-Umgebungen von mconvert.pl am ganzen ttm-Prozess vorbeigeschleust, ueber MathJax realisiert, und im PDF als $$ ... $$ gesetzt
\ifttm%
\newenvironment{MDirectMath}{\begin{html}}{\end{html}}%
\else%
\newenvironment{MDirectMath}{\begin{equation*}}{\end{equation*}}% Vorsicht, auch \[ und \] werden in amsmath durch equation* redefiniert
\fi

% ---------------------------------- Location Management ---------------------------------------------

% #1 = buttonname (muss in files/images liegen und Format 48x48 haben), #2 = Vollstaendiger Einrichtungsname, #3 = Kuerzel der Einrichtung,  #4 = Name der include-texdatei
\ifttm
\newcommand{\MLocationSite}[3]{\special{html:<!-- mlocation;;}#1\special{html:;;}#2\special{html:;;}#3\special{html:;; //-->}}
\else
\newcommand{\MLocationSite}[3]{\relax}
\fi

% ---------------------------------- Copyright Management --------------------------------------------

\newcommand{\MCCLicense}{%
{\color{green}\textbf{CC BY-SA 3.0}}
}

\newcommand{\MCopyrightLabel}[1]{ (\MSRef{L_COPYRIGHTCOLLECTION}{Lizenz})\MLabel{#1}}

% Copyrightliste wird als tex-Datei im preprocessing erzeugt und unter converter/tex/copyrightcollection.tex abgelegt
% Der input-Befehl funktioniert nur, wenn die aufrufende tex-Datei auf der obersten Ebene liegt (d.h. selbst kein input/include ist, insbesondere keine Moduldatei)
\newcommand{\MCopyrightCollection}{\input{copyrightcollection.tex}}

% MCopyrightNotice fuegt eine Copyrightnotiz ein, der parser ersetzt diese durch CopyrightNoticePOST im preparsing, diese Definition wird nur fuer reine pdflatex-Uebersetzungen gebraucht
% Parameter: #1: Kurze Lizenzbeschreibung (typischerweise \MCCLicense)
%            #2: Link zum Original (http://...) oder NONE falls das Bild selbst ein Original ist, oder TIKZ falls das Bild aus einer tikz-Umgebung stammt
%            #3: Link zum Autor (http://...) oder MINT falls Original im MINT-Kolleg erstellt oder NONE falls Autor unbekannt
%            #4: Bemerkung (z.B. dass Datei mit Maple exportiert wurde)
%            #5: Labelstring fuer existierendes Label auf das copyrighted Objekt, mit MCopyrightLabel erzeugt
%            Keines der Felder darf leer sein!
\newcommand{\MCopyrightNotice}[5]{\MCopyrightNoticePOST{#1}{#2}{#3}{#4}{#5}}

\ifttm%
\newcommand{\MCopyrightNoticePOST}[5]{\relax}%
\else%
\newcommand{\MCopyrightNoticePOST}[5]{\relax}%
\fi%

% ---------------------------------- Meldungen fuer den Benutzer des Konverters ----------------------
\MPragma{mintmodversion;P0.1.0}
\MPragma{usercomment;This is file mintmod.tex version P0.1.0}


% ----------------------------------- Spezialelemente fuer Konfigurationsseite, werden nicht von mintscripts.js verwaltet --

% #1 = DOM-id der Box
\ifttm\newcommand{\MConfigbox}[1]{\special{html:<input cfieldtype="2" type="checkbox" name="Name_}#1\special{html:" id="}#1\special{html:" onchange="confHandlerChange('}#1\special{html:');"/>}}\fi % darf im PDF nicht aufgerufen werden!


\Mtikzexternalize
\MPragma{MathSkip}


\begin{document}

%\MSetSubject{MINTMathematics}
\MSection{Differential Calculus}
\MLabel{VBKM07}
\MSetSectionID{VBKM07}

\begin{MSectionStart}
\MDeclareSiteUXID{VBKM07_START}

\MModstartBox
\end{MSectionStart}

%%%Abschnitt
\MSubsection{Derivative of a Function}\MLabel{M07_Ableitung}

\begin{MIntro}
\MDeclareSiteUXID{VBKM07_Ableitung_Intro}

A family is going on holiday by car. The car is moving through roadworks with a velocity of 
$60\MEinheit{km}[]/\MEinheit[]{h}$. The sign
at the end of the roadworks says that the speed limit is, as of now,  
$120\MEinheit{km}[]/\MEinheit[]{h}$. Even though the car driver puts the pedal to the metal,
the velocity of the car will not jump up immediately but increase as a function of time.
If the velocity increases from $60\MEinheit{km}[]/\MEinheit[]{h}$ to $120\MEinheit{km}[]/\MEinheit[]{h}$ 
 in 5~seconds at a constant rate of change, then
the \emph{acceleration} (= change of velocity per time) equals 
this constant (in this case) rate of velocity change: the acceleration is the quotient of 
the velocity change and the time required for this change. Thus, its value is here
$12$~kilometre per hour per second. In reality, the velocity of the car will not increase
at a constant rate but at a \emph{time-dependent} rate.
If the velocity $v$ is described as a function of time $t$, then the 
acceleration is the slope of this function. This does not depend on the fact 
whether this slope is constant (in time) or not. On other words: The acceleration
is the \emph{derivative} of the velocity \emph{function} $v$ with respect to the time $t$.

Similar relations can also be found in other technical fields such as, for example, the calculation of 
internal forces acting in steel frames of buildings, the forecast of atmospheric and oceanic 
currents, or in the modelling of financial markets, which is currently highly relevant.

This chapter reviews the basic ideas underlying these calculations, i.e. it deals with 
\textbf{differential calculus}. In other words: we will take derivatives of functions 
to find their slopes or rates of change. Even thought these calculations 
will be carried out here in a strictly mathematical way, their motivation is not 
purely mathematical. Derivatives, interpreted as rates of change of different functions,
play an important role in many scientific fields and are often investigated as special
quantities.

\end{MIntro}

\begin{MXContent}{Relative Rate of Change of a Function}{Relative Rate of Change}{STD}
\MDeclareSiteUXID{VBKM07_Ableitung_Aenderungsrate}

Consider a function $f: [a\MIntvlSep  b] \rightarrow \R$, $x \mapsto f(x)$ and a sketch of 
the graph of $f$ (shown in the figure below). We would like to describe the rate of change of this
function at an arbitrary point $x_0$ between $a$ and $b$. This will lead us to the notion of 
a derivative of a function. Generally, calculation rules are to be applied that are as simple
as possible. 


\begin{center}
\MTikzAuto{%
\begin{small}
\begin{tikzpicture}[line width=1.5pt,scale=0.7, %
declare function={
  x0 = 2;
  x1 = 4;
  fkt(\x) = 1/4 * (\x - 1)*(\x - 1) + 0.75;
  rT = 1.6; % relative Translation der Beschriftung $\Delta(f)$
%  Tangente(\x) = 1/2 * (\x - 2) + 1;
}
] %[every node/.style={fill=white}] 
%Koordinatenachsen:
\draw[->] (-0.6, 0) -- ({x1+1}, 0) node[below left]{$x$}; %x-Achse
\draw[->] (0, -0.6) -- (0, {fkt(x1)+1}) node[below left]{$y$}; %y-Achse
%Achsenbeschriftung:
\foreach \x in {1, 2, 3, 4} \draw (\x, 0) -- ++(0, -0.1); %
% node[below] {$\x$}; 
\foreach \y in {1, 2, 3} \draw (0, \y) -- ++(-0.1, 0); %
% node[left] {$\y$};
%\node[below left] at (0, 0) {$0$};
%Hilfslinien:
\draw[color=black!50!white,style=dashed] (x1, {fkt(x0)}) -- (x1, {fkt(x1)});
\draw[color=black!50!white,style=dotted] (x0, {fkt(x0)}) -- (x1, {fkt(x0)});
%Funktion:
\draw[domain=0.5:x1,samples=120,color=\jccolorfkt] %
 plot (\x, {fkt(\x)});
%Punkte und Hilfslinien:
\filldraw[color=black,fill=black] (x0, 0) circle (1pt);
\node[below] at (x0, -0.1) {$x_0$};
\filldraw[color=black,fill=black] (x0, {fkt(x0)}) circle (1pt);
%
\filldraw[color=black,fill=black] (x1, 0) circle (1pt);
\node[below] at (x1, -0.1) {$x$};
%
\filldraw[color=black,fill=black] (x1, {fkt(x0)}) circle (1pt);
\filldraw[color=black,fill=black] (x1, {fkt(x1)}) circle (1pt);
%Beschriftung:
\draw[line width=0.8pt, rounded corners=4pt] %
 ({x1 + rT + 0.0}, 1.0) -- ({x1 + rT + 0.2}, 1.0) -- ({x1 + rT + 0.2}, 2.0) %
 -- ({x1 + rT + 0.4}, 2.0);
\draw[line width=0.8pt, rounded corners=4pt] %
 ({x1 + rT + 0.4}, 2.0) %
 -- ({x1 + rT + 0.2}, 2.0) -- ({x1 + rT + 0.2}, 3.0) -- ({x1 + rT + 0.0}, 3.0);
%
\node[right] at ({x1+rT+0.5}, 2) {$\Delta(f) = f(x) - f(x_0)$};
%
\node[left] at (-0.5, {fkt(x0)}) {$f(x_0)$};
\node[left] at (-0.5, {fkt(x1)}) {$f(x)$};
\end{tikzpicture}
\end{small}
}
\end{center}

%----------------------------------------------------------------------------------------


If $x_0$ and the corresponding function value $f\left(x_0\right)$ are fixed and 
another arbitrary, but variable point $x$ between $a$ and $b$ as well as the 
corresponding function value $f\left(x\right)$ are chosen, then through these 
two points, i.e. the points $\MPointTwoAS{x_0}{f(x_0)}$ and $\MPointTwoAS{x}{f(x)}$, 
a line can be drawn that is characterised by its slope and its $y$-intercept. For the 
slope of this line one obtains the so-called \textbf{difference quotient}
\[
\frac{\Delta(f)}{\Delta(x)} = \frac{f(x) - f(x_0)}{x - x_0}
\]
that describes how the function values of $f$ between $x_0$ and $x$ change \textbf{on average}.
Thus, an average rate of change of the function $f$ on the interval $[x_0\MIntvlSep  x]$ is found. 
This quotient is also called \textbf{relative change}.

If we let the variable point $x$ approach the point $x_0$, then we see that the line 
that intersects the graph of the function in the points $\MPointTwoAS{x_0}{f\left(x_0\right)}$ 
and $\MPointTwoAS{x}{f\left(x\right)}$ 
gradually becomes a tangent line to the graph in the point $\MPointTwoAS{x_0}{f\left(x_0\right)}$.
In this way, the rate of change of the function $f$ -- or the \textbf{slope} of the graph of 
$f$ -- at the point $x_0$ \textbf{itself} can be determined. If the approaching process 
of $x$ to $x_0$ described above leads to, figuratively speaking, a unique tangent line (i.e. a line with 
a unique slope that, in particular, must not be infinity), then in mathematical terms
one says that the \MEntry{limit}{limit} of the difference quotient does \textbf{exist}. 
This limiting process, i.e. letting $x$ approach $x_0$, is described here and in the following 
by the symbol 
\[
\lim_{x \rightarrow x_0} \MDFPSpace,
\]
where $\lim$ is an abbreviation for the Latin word \emph{limes}, meaning ``border'' or ``boundary''.
If the limit of the difference quotient exists, then

\[
f'(x_0) = \lim_{x \rightarrow x_0} \frac{\Delta(f)}{\Delta(x)} 
 = \lim_{x \rightarrow x_0} \frac{f(x) - f(x_0)}{x - x_0} %%
\] 
denotes the value of the \MEntry{derivative}{derivative} of $f$ at $x_0$. The function $f$ is then 
said to be \textbf{differentiable} at the point $x_0$.

\begin{MExample}
  For the function $f(x)=\sqrt{x}$ the relative change at the point $x_0=1$ is given by
  \[
    \frac{f(x) - f(x_0)}{x - x_0} \;=\;
    \frac{\sqrt{x}-\sqrt{1}}{x-1} \;=\; \frac{\sqrt{x}-1}{(\sqrt{x}-1)(\sqrt{x}+1)} \;=\; \frac1{\sqrt{x}+1} \MDFPeriod
  \]
  If $x$ approaches $x_0=1$, this results in the limit
  $$
  \lim_{x \rightarrow x_0} \frac{\Delta(f)}{\Delta(x)} \;=\; \frac12 \MDFPeriod
  $$
  The value of the derivative of the function $f$ at the point $x_0=1$ is denoted by 
  $f'(1)=\frac12$.
\end{MExample}

\begin{MExercise}
  Consider the function $f: \R \rightarrow \R$ with $x \mapsto f(x)=x^2$ and a point $x_0=1$. 
  At this point, the relative change for a real value of $x$ equals 
  \MEquationItem{$\displaystyle \frac{f(x) - f(1)}{x - 1}$}{\MLFunctionQuestion{8}{x+1}{4}{x}{4}{OR30}}.\\
  \MInputHint{Calculate the quotient directly without using any differentiation rules and other rules known 
  from school.}
 
  If $x$ approaches $x_0=1$, this results in the slope
  \MLParsedQuestion{5}{2}{5}{DG10} of the graph of the function $f$ at the point $x_0=1$.

  \begin{MHint}{Solution}
  For $f(x)=x^2$, the relative change at the point $x_0=1$ is given by
  \[
    \frac{f(x) - f(1)}{x - 1} \;=\;
    \frac{x^2-1}{x-1} \;=\; \frac{(x-1)(x+1)}{x-1} \;=\; x+1 \MDFPeriod
  \]
  Then, if $x$ approaches $x_0$, this results in the limit
  $$
  \lim_{x \rightarrow 1} \frac{\Delta(f)}{\Delta(x)} \;=\; 2 \MDFPeriod
  $$
  This is the slope of the tangent line to the graph of $f$ at the point 
  $\MPointTwo{x_0}{f(x_0)} = \MPointTwo{1}{1}$. The value of the derivative 
  of $f$ at the point $x_0=1$ is denoted by $f'(1)=2$.
  \end{MHint}  
\end{MExercise}

Using the formula for the relative rate of change, calculating the derivative can be
very cumbersome and also only works for very simple functions. Typically, 
the derivative is determined by applying calculation rules and inserting known 
derivatives for the individual terms. 
\end{MXContent}


%\MSubsubsection{Ableitung}
\begin{MXContent}{Derivative}{Derivative}{STD}
\MDeclareSiteUXID{VBKM07_Ableitung_Ableitung}

\begin{MXInfo}{Notation of the Derivative}
In mathematics, sciences and engineering, different but equivalent notations for 
derivatives are used:
\[
f'(x_0) = \frac{\MD f}{\MD x}(x_0) = \frac{\MD}{\MD x}f(x_0) \MDFPeriod %%
\]
These different notations all denote the derivative of the function $f$ at the point 
$x_0$.
\end{MXInfo}

If the derivative is to be calculated using the difference quotient $\frac{f(x) - f(x_0)}{x - x_0}$,
then it is often convenient to rewrite the difference quotient in another way. Denoting the 
difference of $x$ and $x_0$ by $h := x - x_0$ (see figure below),


%Bild:
\begin{center}
\MTikzAuto{%
\begin{small}
\begin{tikzpicture}[line width=1.5pt,scale=0.7, %
declare function={
  x0 = 0;
  x1 = 4;
}
] %[every node/.style={fill=white}] 
%Koordinatenachsen:
\draw[->] (-0.6, 0) -- ({x1+1}, 0); % node[below left]{$x$}; %x-Achse
%Achsenbeschriftung:
\foreach \x in {0, 1, 2, 3, 4} \draw (\x, 0.0) -- ++(0, -0.1); %
\foreach \x in {0, 4} \draw (\x, 0.0) -- ++(0, +0.08); %
% node[below] {$\x$}; 
%Beschriftung:
%\draw[line width=0.8pt, rounded corners=4pt] %
% (0, 0.3) -- (0, 0.4) -- (2, 0.4) -- (2, 0.5);
\draw[line width=0.8pt, rounded corners=4pt] %
 ({x0}, 0.3) -- ({x0}, 0.4) -- ({x1/2}, 0.4) %
 -- ({x1/2}, 0.5);
\draw[line width=0.8pt, rounded corners=4pt] %
 ({x1/2}, 0.5)
 -- ({x1/2}, 0.4) -- ({x1}, 0.4) -- ({x1}, 0.3);
%
\node[above] at ({x1/2}, 0.6) {$h = x - x_0$};
%
\node[below] at ({x0}, -0.3) {$x_0$};
\node[below] at ({x1}, -0.3) {$x$};
\end{tikzpicture}
\end{small}
}
\end{center}

the difference quotient can be rewritten as 
\[
\frac{f(x) - f(x_0)}{x - x_0} = \frac{f(x_0 + h) - f(x_0)}{h} \MDFPSpace ,
\]
where $x = x_0 + h$. There is no statement about whether $x$ has to be greater or less than $x_0$. 
Hence, the quantity $h$ can take positive or negative values. To determine the derivative 
of the function $f$, the limit for $h \rightarrow 0$ has to be calculated:
\[
f'(x_0) = \lim_{x \rightarrow x_0} \frac{f(x) - f(x_0)}{x - x_0} %
 = \lim_{h \rightarrow 0} \frac{f(x_0 + h) - f(x_0)}{h} \MDFPeriod
\]
If this limit exists \textbf{for all} points $x_0$ in a function's domain, then the
function is said to be \textbf{differentiable} (everywhere). Many of the common functions
are differentiable. However, a simple example of a function that is not differentiable everywhere 
is the absolute value function $f: \R \rightarrow \R$ with $x \mapsto f(x) := |x|$.

\begin{MExample}
The absolute value function (see Module~\MNRef{VBKM06}, Section~\MRef{VBKM06_sec:betrag})
is not differentiable at the point $x_0 = 0$. The difference quotient of $f$ at the point 
$x_0 = 0$ is:
\[
\frac{f(0+h) - f(0)}{h} = \frac{|h| - |0|}{h} = \frac{|h|}{h} \MDFPeriod
\]
Since $h$ can be greater or less than $0$, two cases are to be distinguished:
For $h > 0$, we have $\frac{|h|}{h} = \frac{h}{h} = 1$, and for $h < 0$, we 
have $\frac{|h|}{h} = \frac{-h}{h} = -1$. In these two cases, the limiting process, 
i.e. $h$ approaching $0$, results in two different values ($1$ and $-1$).
Thus, \textbf{the} limit of the difference quotient at the point $x_0 = 0$ does not exist. 
Hence, the absolute value function is not differentiable at the point $x_0 = 0$.

The graph changes its direction at the point $\MPointTwo{0}{0}$ abruptly: 
Casually speaking, one says that the graph of the function has a kink at the point $\MPointTwo{0}{0}$.
%Bild:
\begin{center}
\MTikzAuto{%
\begin{tikzpicture}[line width=1.5pt,scale=1.0]
\draw[->] (-3.6, 0) -- (4, 0) node[below left]{$x$}; %x-Achse
\draw[->] (0, -0.6) -- (0, 4.6) node[below left]{$y$}; %y-Achse
%Achsenbeschriftung:
\foreach \x in {-3, -2, -1, 1, 2, 3} \draw (\x, 0) -- ++(0, -0.1) %
 node[below] {$\x$};
\foreach \y in {1, 2, 3, 4} \draw (0, \y) -- ++(-0.1, 0) node[left] {$\y$};
%\node[below left] at (0, 0) {$0$};
%Funktion:
\draw[domain=-3.2:3.2,samples=120,color=\jccolorfkt] %
 plot (\x, {abs(\x)});
%Tangenten im Nullpunkt, wenn $f$ f"ur $x \leq 0$ bzw. $x \geq 0$ betrachtet 
%wird:
\draw[samples=120,color=blue!50!black] %
 (-0.5, 0.5) -- (0.5, -0.5);
\draw[samples=120,color=blue!50!black] %
 (-0.5, -0.5) -- (0.5, 0.5);
%Punkt: Markierung der Stelle $0$:
\filldraw[color=black,fill=black] (0, 0) circle (1pt);
%end of file
\end{tikzpicture}
}
\end{center}
\end{MExample}

Likewise, if a function has a jump at a certain point, a unique tangent line to the graph at 
this point does not exist and thus, the function has no derivative at this point.

\end{MXContent}

\begin{MExercises}
\MDeclareSiteUXID{VBKM07_Ableitung_Exercises}

\begin{MExercise}
Using the difference quotient, calculate the derivative of 
%$f(x) := |3 x|$ f"ur $x_1 := 5$ und f"ur $x_2 := -4$.
$f: \R \rightarrow \R$, $x \mapsto f(x) := 4 - x^2$ at the points $x_1 = -2$ and $x_2 = 1$.

Answer: 
\begin{MExerciseItems}
\item The difference quotient of $f$ %zu $x$
at the point $x_1 = -2$ is
\MLSimplifyQuestion{30}{2 - x}{10}{x}{4}{0}{SIMPLE4}
and has for $x \rightarrow -2$ the limit
 $f'(-2) = $\MLParsedQuestion{8}{4}{3}{DG11}.
\item The difference quotient of $f$ %zu $x$
at the point $x_2 = 1$ is
\MLFunctionQuestion{30}{-1 - x}{10}{x}{4}{DG13X}
and has for $x \rightarrow 1$ the limit
 $f'(1) = $\MLParsedQuestion{8}{-2}{3}{DG12}.
\end{MExerciseItems}
\begin{MHint}{Solution}
 \begin{MExerciseItems}
  \item At the point  $x_1 = -2$, we have for the difference quotient
  \[
  \frac{\Delta (f)}{\Delta (x)} = \frac{f(x) - f(x_1)}{x - x_1} = \frac{f(x) - f(-2)}{x - (-2)} =
  \frac{4 - x^2 - 0}{x + 2} = \frac{(2-x)(2+x)}{2+x} = 2-x \MDFPeriod
  \]
  For $x \rightarrow x_1$, i.e. for $x \rightarrow -2$, this difference quotient tends to
  $2 - (-2) = 4$; hence, $f'(-2) = 4$.
  \item At the point $x_2 = 1$, we have for the difference quotient
  \[
  \frac{\Delta (f)}{\Delta (x)} = \frac{f(x) - f(x_2)}{x - x_2} = \frac{f(x) - f(1)}{x - 1} =
  \frac{4 - x^2 - 3}{x - 1} = \frac{1 - x^2}{x - 1} = - \frac{(x-1)(x+1)}{x-1} = - x - 1 \MDFPeriod
  \]
  For $x \rightarrow x_2$, i.e. for $x \rightarrow 1$, 
  this difference quotient has the limit $- 1 - 1 = -2$; hence, $f'(1) = - 2$.
 \end{MExerciseItems}
\end{MHint}
\end{MExercise}


\begin{MExercise}
Explain why the functions
\begin{MExerciseItems}
\item $f: [-3 \MIntvlSep \infty \MoIr \rightarrow \R$ with $f(x) := \sqrt{x+3}$ at $x_0 = -3$ and
\item $g: \R \rightarrow \R$ with $g(x) := 6 \cdot |2 x - 10|$ at $x_0 = 5$
\end{MExerciseItems}
are not differentiable.

Answer:
%Die Tangente an den Funktionsgraphen der stetigen Funktion
\begin{MExerciseItems}
\item The derivative of the function $f$ at the point $x_0 = -3$ does not exist since
the difference quotient
\MLFunctionQuestion{30}{1/sqrt(h)}{1}{h}{20}{DG13}
does not converge for $h \rightarrow 0$.
% in $x_0 = -3$ verl"auft parallel zur $y$-Achse,
\item The derivative of the function $g$ at the point $x_0 = 5$ does not exist since
the difference quotient for $h < 0$ has the value
\MLParsedQuestion{8}{-12}{3}{DG14} and for $h > 0$ has
the value {\MLParsedQuestion{8}{12}{3}{DG15}}. Thus, the limit for $h \rightarrow 0$
does not exist.
%F"ur $x < 5$ haben alle Tangenten die Steigung $-12$, 
%und f"ur $x > 5$ die Steigung $21$.
\end{MExerciseItems}
\begin{MHint}{Solution}
 \begin{MExerciseItems}
  \item The difference quotient of the function $f$ at the point $x_0 = - 3$ is
  \[
  \frac{\Delta (f)}{\Delta (x)} = \frac{f(x_0 + h) - f(x_0)}{h} = \frac{\sqrt{-3 + h + 3} - \sqrt{-3 + 3}}{h}
  = \frac{\sqrt{h} - 0}{h} = \frac{1}{\sqrt{h}} \MDFPeriod
  \]
  For $h \rightarrow 0$ ($h > 0$), this difference quotient increases infinitely, i.e.\ the limit 
  of the difference quotient does not exist.
  \item The difference quotient of the function $g$ at the point $x_0 = 5$ is
  \[
  \frac{\Delta (g)}{\Delta (x)} = \frac{g(x_0 + h) - g(x_0)}{h} = \frac{6 \cdot |2 (5+h) - 10| - 6 \cdot |2 \cdot 5 - 10|}{h}
  = \frac{12 |h| - 0}{h} = \frac{12 |h|}{h} \MDFPeriod
  \]
  For $h<0$, since $|h| = -h$, the difference quotient has the value $-12$. In contrast, for 
  $h>0$, since $|h|=h$, it has the value $12$. Thus, the limit of the difference quotient 
  does not exist. (The limit has always to be unique.)
 \end{MExerciseItems}
\end{MHint}
\end{MExercise}

%\begin{MExercise}
%ei $x_0 \in \R$ mit $x_0 \neq 0$. Berechnen Sie mittels Differenzenquotienten 
%ie Ableitung von $f(x) := \frac{1}{x}$ an der Stelle $x_0$.
%end{MExercise}

\end{MExercises}



%%%Abschnitt
%\MSubsection{Ableitungen elementarer Funktionen}\MLabel{M07_Ableitung_elementareFunktionen}
\MSubsection{Standard Derivatives}\MLabel{M07_Standardableitungen}

\begin{MIntro}
\MDeclareSiteUXID{VBKM07_StdAbleitungen_Intro}

Most of the common functions, such as polynomials, trigonometric functions, and exponential functions
(see Module~\ref{VBKM06}) are differentiable. In the following, the differentiation rules for these
functions are repeated.
\end{MIntro}


\begin{MXContent}{Derivatives of Power Functions}{Derivatives of Power Functions}{STD}
\MDeclareSiteUXID{VBKM07_StdAbleitungen_Polynome}
 
In the last section, the derivative was introduced as the limit of the difference quotient. Accordingly,
for a linear affine function (see Module~\MNRef{VBKM06}, Section~\MRef{VBKM06_sec:linear-affin})
$f: \R \rightarrow \R$, $x \mapsto f\left(x\right) = m x + b$, where $m$ and $b$ are given numbers, we obtain 
for the derivative at the point $x_0$ the value $f'(x_0) = m$. 
(Readers are invited to verify that fact themselves.)

For monomials $x^n$ with $n \geq 1$, it is easiest to determine the derivative using the difference quotient. 
Without any detailed calculation or any proof we state the following rules:

\begin{MXInfo}{Derivative of $x^n$}
Let a natural number $n$ and a real number $r$ be given.

The constant function $f: \R \rightarrow \R$ with $x \mapsto f(x) := r = r \cdot x^0$
has the derivative $f': \R \rightarrow \R$ with $x \mapsto f'(x) = 0$.

The function $f: \R \rightarrow \R$ with $x \mapsto f(x) := r \cdot x^n$ has the derivative
\[
f': \R \rightarrow \R \MDFPSpace \mbox{ with } \MDFPSpace x \mapsto f'(x) = r \cdot n \cdot x^{n-1} \MDFPeriod %%
\]
This differentiation rule is true for all $n\in\mathbb{R}\setminus\{0\}$.
\end{MXInfo}

Again, we leave the verification of these statements to the reader.

\begin{MExample}
Let us consider the function $f: \R \rightarrow \R$ with $x \mapsto f(x) = 5 x^3$. 
According to the notation above, this is a function with $r = 5$ and $n = 3$. Thus for the value of the derivative at the point~$x$ , we have
\[
f'(x) = 5 \cdot 3 x^{3 - 1} = 15 x^2 \MDFPeriod %%
\]
\end{MExample}

For root functions, an equivalent statement holds. However, it should be noted that root functions 
are only differentiable for $x > 0$ since the tangent line to the graph of the function 
at the point $\MPointTwo{0}{0}$ is parallel to the $y$-axis and thus, it is not a graph of a function.

\begin{MXInfo}{Derivative of $x^{\frac{1}{n}}$}
For $n \in \Z$ with $n \neq 0$, the function 
$f: [0 \MIntvlSep \infty \MoIr \rightarrow \R$, $x \mapsto f(x) := x^{\frac{1}{n}}$ 
is differentiable for $x>0$, and we have
\[
f': \MoIl 0 \MIntvlSep \infty \MoIr \rightarrow \R \MDFPSpace , 
\MDFPaSpace x \mapsto f'(x) = \frac{1}{n} \cdot x^{\frac{1}{n}-1} \MDFPeriod
\]
\end{MXInfo}

For $n\in\mathbb{N}$, root functions are described by $f(x) = x^{\frac{1}{n}}$.
Of course, the differentiation rule given here also holds for $n = 1$ or $n = -1$.

\begin{MExample}
The root function $f: [0\MIntvlSep \infty\MoIr \rightarrow \R$ with
$x \mapsto f(x) := \sqrt{x} = x^{\frac{1}{2}}$
is differentiable for $x > 0$. The value of the derivative at an arbitrary point 
$x>0$ is given by
\[
f'(x) = \frac{1}{2} \cdot x^{\frac{1}{2}-1} %
= \frac{1}{2} \cdot x^{-\frac{1}{2}} = \frac{1}{2 \cdot \sqrt{x}} \MDFPeriod%%
\] 
The derivative at the point $x_0 = 0$ does not exist since the slope of the tangent line 
to the graph of $f$ would be infinite there.

\begin{center}
\MTikzAuto{%
%{Graph von $\sqrt{x}$ mit Tangente in $x_0 = 1$}{}
\begin{small}
\begin{tikzpicture}[line width=1.5pt,scale=0.8]
%Koordinatenachsen:
\draw[->] (-0.6, 0) -- (4.8, 0) node[below left]{$x$}; %x-Achse
\draw[->] (0, -0.6) -- (0, 3) node[below left]{$y$}; %y-Achse
%Achsenbeschriftung:
\foreach \x in {1, 2, 3, 4} \draw (\x, 0) -- ++(0, -0.1) %
 node[below] {$\x$};
\foreach \y in {1, 2} \draw (0, \y) -- ++(-0.1, 0) node[left] {$\y$};
%\node[below left] at (0, 0) {$0$};
%Funktion:
\draw[domain=0:4,samples=120,color=\jccolorfkt] %
 plot (\x, {sqrt(\x)});
%Tangenten y = f(x_0) + f'(x_0) * (x - x_0) im Punkt $(x_0, f(x_0)$ des Graphen:
%$x_0 := 1$:
\draw[samples=120,color=blue!50!black] %
({-1/2}, {1/4}) -- ++({7/2}, {7/4});
%Punkt:
\filldraw[color=black, fill=black] (1, 0) circle (2pt);
%end of file
\end{tikzpicture}
\end{small}
}
\end{center}

The tangent line to the graph of the given root function at the point $\MPointTwo{1}{1}$ 
has the slope $\frac{1}{2 \sqrt{1}} = \frac{1}{2}$.
\end{MExample}

For $x>0$, the statements above can be extended to exponents $p \in \R$ with 
$p \neq 0$:
The value $f'(x)$ of the derivative of the function $f$ with the mapping rule 
$f(x) = x^p$ is, for $x > 0$,
\[
f'(x) = p \cdot x^{p-1} \MDFPeriod %%
\]
\end{MXContent}

\begin{MXContent}{Derivatives of Special Functions}{Derivatives of Special Functions}{STD}
\MDeclareSiteUXID{VBKM07_SpezielleFunktionen}

\MSubsubsectionx{Derivatives of Trigonometric Functions}

The sine function $f: \R \rightarrow \R$, $x \mapsto f(x) = \sin(x)$ is periodic with period 
$2 \pi$. Thus, it is sufficient to consider the function on an interval of length $2 \pi$. 
A section of the graph for $-\pi \leq x \leq \pi$ is shown in the figure below:

\begin{center}
\MTikzAuto{%
\begin{small}
\begin{tikzpicture}[line width=1.5pt,scale=1.0, %
declare function={
  fkt(\x) = sin(\x r);
  fktabl(\x) = cos(\x r);
}
] %[every node/.style={fill=white}] 
%
%Graph der Sinusfunktion:
\node[right] at (-6,1) {Sine function};
\begin{scope}%[xshift=-6]
%Koordinatenachsen:
\draw[->] (-3.6, 0) -- (4, 0) node[below left]{$x$}; %x-Achse
\draw[->] (0, -1.6) -- (0, 1.6) node[below left]{$y$}; %y-Achse
%Achsenbeschriftung:
\foreach \x in {{-pi}, {-pi/2}} \draw (\x, 0) -- ++(0, -0.1);
\node[below] at ({-pi}, 0) {$-\pi$};
\node[below] at ({-pi/2}, 0) {$-\frac{\pi}{2}$};
\foreach \x in {{pi/2}, pi} \draw (\x, 0) -- ++(0, -0.1);
\node[below] at ({pi/2}, 0) {$\frac{\pi}{2}$};
\node[below] at ({pi}, 0) {$\pi$};
\foreach \y in {-1} \draw (0, \y) -- ++(-0.1, 0) %
 node[left] {$\y$};
\foreach \y in {1} \draw (0, \y) -- ++(-0.1, 0) %
 node[left] {$\y$};
%\node[below left] at (0, 0) {$0$};
%Funktion:
\draw[domain=-3.14:3.14,samples=120,color=\jccolorfkt] %
 plot (\x, {fkt(\x)});
%Tangenten in verschiedenen Punkten:
\draw[samples=120,color=blue!50!black] %
 plot (-0.5,-0.5) -- (0.5,0.5);
\draw[samples=120,color=blue!50!black] %
 plot ({pi/3},1) -- ({2*pi/3},1);
\node[above] at ({pi/2}, 1) {$\sin'(\pi/2) = 0$};
\draw[samples=120,color=blue!50!black] %
 plot ({-2*pi/3},-1) -- ({-pi/3},-1);
\node[below] at ({-pi/2}, -1) {$\sin'(-\pi/2) = 0$};
\end{scope}
\node[right] at (-6,-2.8) {Derivative};
\begin{scope}[yshift=-3.8cm]
%Koordinatenachsen:
\draw[->] (-3.6, 0) -- (4, 0) node[below left]{$x$}; %x-Achse
\draw[->] (0, -1.6) -- (0, 1.6) node[below left]{$y$}; %y-Achse
%Achsenbeschriftung:
\foreach \x in {{-pi}, {-pi/2}} \draw (\x, 0) -- ++(0, 0.1); 
\node[below] at ({-pi}, 0) {$-\pi$}; 
\node[below] at ({-pi/2}, 0) {$-\frac{\pi}{2}$}; 
\foreach \x in {{pi/2}, {pi}} \draw (\x, 0) -- ++(0, -0.1);
\node[below] at ({pi/2}, 0) {$\frac{\pi}{2}$}; 
\node[below] at ({pi}, 0) {$\pi$}; 
\foreach \y in {-1} \draw (0, \y) -- ++(0.1, 0);
\foreach \y in {1} \draw (0, \y) -- ++(-0.1, 0) %
 node[below left] {$\y$};
%\node[below left] at (0, 0) {$0$};
%Funktion:
\draw[domain=-3.14:3.14,samples=120,color=blue!50!white] %
 plot (\x, {fktabl(\x)});
%Punkte
\filldraw[color=black,fill=black] ({-pi}, -1) circle (2pt);
\filldraw[color=black,fill=black] ({-pi/2}, 0) circle (2pt);
\filldraw[color=black,fill=black] (0, 1) circle (2pt);
\filldraw[color=black,fill=black] ({pi/2}, 0) circle (2pt);
\filldraw[color=black,fill=black] ({pi}, -1) circle (2pt);
\end{scope}
\end{tikzpicture}
\end{small}
}
\end{center}

As we see from the figure above, the slope of the sine function at $x_0 = \pm\frac{\pi}{2}$ 
is $f'(\pm\frac{\pi}{2}) = 0$. The tangent line to the graph of the sine function at $x_0 = 0$ 
has the slope $f'(0) = 1$. At $x_0 = \pm\pi$, the tangent line has the same slope as the tangent line at 
$x_0 = 0$, but the sign is opposite. Hence, the slope at $x_0 = \pm\pi$ is $f'(\pm\pi) = -1$.
Thus, the derivative of the sine function is a function that exhibits exactly these properties. 
A detailed investigation of the regions between these specially chosen points shows that the derivative of 
the sine function is the cosine function:

\begin{MXInfo}{Derivatives of Trigonometric Functions}
For the sine function 
$f: \R \rightarrow \R$, $x \mapsto f(x) := \sin(x)$, we have
\[
f': \R \rightarrow \R \MDFPSpace , \MDFPaSpace x \mapsto f'(x) = \cos(x) \MDFPeriod %%
\]
For the cosine function $g: \R \rightarrow \R$, $x  \mapsto g(x) := \cos(x)$, we have
\[
g': \R \rightarrow \R \MDFPSpace , \MDFPaSpace x \mapsto g'(x) = -\sin(x) \MDFPeriod %%
\]
For the tangent function $h: \R \setminus \{ \frac{\pi}{2} + k \pi \MCondSetSep k \in \Z \} \rightarrow \R$,
$x \mapsto h(x) := \tan(x)$, we have
\[
h': \R \setminus \{ \frac{\pi}{2} + k \pi \MCondSetSep k \in \Z \} \rightarrow \R \MDFPSpace , \MDFPaSpace
x \mapsto h'(x) = 1 + (\tan(x))^2 = \frac{1}{\cos^2(x)} \MDFPeriod %%
\]
\end{MXInfo}

This last result comes from the calculation rules explained (explained below) and
the definition of the tangent function as the quotient of the sine function and the 
cosine function.

\MSubsubsectionx{Derivative of the Exponential Function}

\begin{MInfo}
The exponential function $f: \R \rightarrow \R$, $x \mapsto f(x) := \MEU^{x} = \exp(x)$ has the 
special property that its derivative  $f'$ is also the exponential function, i.e. 
$f'(x) = \MEU^{x} = \exp(x)$.
\end{MInfo}

\MSubsubsectionx{Derivative of the Logarithmic Function}

%Die Logarithmusfunktion $f(y) := \ln(y)$ f"ur $y > 0$ ist die Umkehrfunktion
%der Exponentialfunktion $y = \MEU^x$. Die Ableitung und damit die Steigung $m$ einer Tangente an den 
%Graphen der Exponentialfunktion ist $m = \MEU^x$. 
%In der folgenden Abbildung ist die Tangente an den Graphen der Exponentialfunktion im
%Punkt $(x_0, \MEU^{x_0})$ f"ur $x_0 = \ln(2)$ angedeutet, sodass die 
%Tangentensteigung $m = 2$ ist. Die Umkehrabbildung der Tangente hat dann 
%die Steigung $\frac{1}{m} = \frac{1}{2}$. 
%Geometrisch ist es die Steigung der gespiegelten Tangente, also die Tangente
%an die Umkehrfunktion $\ln$.
%Verwenden wir wieder die gewohnten Bezeichnungen für die unabhängige Variable, können wir ohne Beweis die Ableitung der Logarithmusfunktion angeben. Es gilt: $f'(x) = \frac{1}{x}$ ist die Ableitung der 
%Logarithmusfunktion $f(x) = \ln(x)$.

The derivative of the logarithmic function is given here without proof. For 
$f: \MoIl 0 \MIntvlSep \infty\MoIr  \rightarrow \R$ with $x \mapsto f(x) = \ln(x)$
one obtains $f': \MoIl 0 \MIntvlSep \infty\MoIr \rightarrow \R$, $x \mapsto f'(x) = \frac1x$.
%\begin{center}
%\MTikzAuto{%
%%%{Tangenten an $\exp$ und an die Umkehrfunktion $\ln$}{}
%\begin{small}
%\begin{tikzpicture}[line width=1.5pt,scale=0.8]
%%%Koordinatenachsen:
%\draw[->] (-3.2, 0) -- (4.6, 0) node[below left]{$x$}; %x-Achse
%\draw[->] (0, -3.2) -- (0, 4) node[below left]{$y$}; %y-Achse
%%%Achsenbeschriftung:
%\foreach \x in {-3, -2, -1} \draw (\x, 0) -- ++(0, -0.1); % node[below] {$\x$};
%\foreach \x in {1, 2, 3} \draw (\x, 0) -- ++(0, -0.1) node[below] {$\x$};
%\foreach \y in {-3, -2, -1} \draw (0, \y) -- ++(-0.1, 0); % node[left] {$\y$};
%\foreach \y in {1, 2, 3} \draw (0, \y) -- ++(-0.1, 0) node[left] {$\y$};
%%%\node[below left] at (0, 0) {$0$};
%%%
%%%Funktion exp:
%\draw[domain=-2.4:1.3,samples=120,color=white!50!black] %\jccolorfkt] %
% plot (\x, {exp(\x)});
%%%Tangenten in verschiedenen Punkten:
%%%x_0 = 0:
%%%\draw[samples=120,color=blue!50!black] %
%%% (-0.45, 0.45) -- (0.55, 1.55);
%%%
%%%x_0 = 1:
%%% ({3/4}, {exp(1) - 1/4 * exp(1)}) -- ++({2/4}, {2/4*exp(1)});
%%%
%%%x_0 = ln(2): y = exp(ln(2)) + 2 * (x - ln(2)) = 2 + 2 * (x - ln(2)):
%\draw[style=dashed,samples=120,color=blue!50!black] %
% ({ln(2) - 3/8}, {5/4}) -- ++({6/8}, {6/8*2});
%%%
%%%Punkte:
%%%\filldraw[color=black] (0, 1) circle (2pt);
%\filldraw[color=black] (0, 2) circle (2pt);
%\filldraw[color=black] ({ln(2)}, 2) circle (2pt);
%%%
%%%Spiegelachse (erste Winkelhalbierende):
%\draw[samples=120,style=dotted,color=white!50!black] %
% (-2.1, -2.1) -- (3.1, 3.1);
%%%
%%%Umkehrfunktion ln:
%\draw[domain=0.1:{exp(1.3},samples=120,color=\jccolorfkt] %
% plot (\x, {ln(\x)});
%%%Tangenten in verschiedenen Punkten:
%%%\draw[samples=120,color=blue!50!black] %
%%% (0.45, -0.45) -- (1.55, 0.55);
%%%
%%%\draw[style=dotted,samples=120,color=blue!50!black] %
%%% ({exp(1) - 1/2}, {1 - 1/2 * 1/exp(1)}) -- ++(1, {1/exp(1)});
%%%\draw[style=dotted,samples=120,color=blue!50!black] %
%%% ({exp(1) - 1/4*exp(1)}, {3/4}) -- ++({2/4*exp(1)}, {2/4});
%%%x_0 = 2: y = ln(2) + 1/2 * (x - 2):
%\draw[samples=120,color=blue!50!black] %
% ({2 - 3/4}, {ln(2) - 1/2 * 3/4}) -- ++({2*3/4}, {2*3/4*1/2});
%%%Punkte:
%%%\filldraw[color=black] (1, 0) circle (2pt);
%\filldraw[color=black] (2, 0) circle (2pt);
%\filldraw[color=black] (2, {ln(2)}) circle (2pt);
%\end{tikzpicture}
%\end{small}
%}
%\end{center}

\end{MXContent}


%%%Uebungen zum Abschnitt:
\begin{MExercises}
\MDeclareSiteUXID{VBKM07_Ableitung_Speziell_Exercises}

\begin{MExercise}
Find the following derivatives by simplifying the terms of the functions and then 
applying your knowledge of the differentiation of common functions
($x > 0$):
\begin{MExerciseItems}
\item $f(x) := x^6 \cdot x^{\frac{7}{2}} = $
\MLSimplifyQuestion{40}{x^(19/2)}{5}{x}{4}{512}{SIMPLE5}. % wird immer noch nicht richtig erkannt
\item $g(x) := \frac{x^{-\frac{3}{2}}}{\sqrt{x}} = $
\MLSimplifyQuestion{40}{1/x^2}{4}{x}{4}{512}{SIMPLE6}.
\end{MExerciseItems}
Thus, we have:
\begin{MExerciseItems}
\item $f'(x) = $ \MLSimplifyQuestion{40}{19/2*x^(17/2)}{4}{x}{4}{512}{SIMPLE7}.
\item $g'(x) = $ \MLSimplifyQuestion{40}{-2/x^3}{4}{x}{4}{512}{SIMPLE8}.
\end{MExerciseItems}
\begin{MHint}{Solution}
 \begin{MExerciseItems}
  \item We have $f(x) = x^6 \cdot x^\frac72 = x^{6 + \frac72} = x^\frac{19}{2}$, and hence $f'(x) = \frac{19}{2} x^{\frac{19}{2} - 1} = \frac{19}{2} x^\frac{17}{2}$.
  \item We have $g(x) = \frac{x^{-\frac32}}{\sqrt{x}} = x^{- \frac32} \cdot x^{- \frac12} = x^{- \frac32 - \frac12} = x^{-2}$,
  and hence $g'(x) = (-2) \cdot x^{-2 -1} = -2 \cdot x^{-3} = - \frac{2}{x^3}$.
 \end{MExerciseItems}
\end{MHint}
\end{MExercise}

\begin{MExercise}
Simplify the terms of the functions and find their derivatives:
\begin{MExerciseItems}
\item
 $f(x) := 2 \sin\left(\frac{x}{2}\right) \cdot \cos\left(\frac{x}{2}\right) = $ %
\MLSimplifyQuestion{40}{sin(x)}{10}{x}{4}{0}{DS1}.
\item $g(x) := \cos^2(3 x) + \sin^2(3 x) = $ %
\MLParsedQuestion{40}{1}{3}{DG16}.
\end{MExerciseItems}
Thus, we have:
\begin{MExerciseItems}
\item $f'(x) = $ \MLFunctionQuestion{40}{cos(x)}{10}{x}{4}{DG17}.
\item $g'(x) = $ \MLParsedQuestion{40}{0}{3}{DG18}.
\end{MExerciseItems}
\begin{MHint}{Solution}
 \begin{MExerciseItems}
  \item Generally, we have
  \[
   \sin (u) \cdot \cos (v) = \frac12 \left( \sin (u-v) + \sin (u+v) \right) \MDFPeriod
  \]
  Thus, in the present case we have $f(x) = 2 \cdot \frac12 ( \sin(0) + \sin(x) ) = \sin (x)$, 
  and hence $f'(x) =  \cos (x)$.
  \item Since $\sin^2 (u) + \cos^2 (u) = 1$, we have $g(x) = 1$, and thus $g'(x) = 0$.
 \end{MExerciseItems}
\end{MHint}
\end{MExercise}

\begin{MExercise}
Simplify the terms of the functions and find the derivatives
(for $x > 0$ in the first part of this exercise):
\begin{MExerciseItems}
\item $f(x) := 3 \ln(x) + \ln\left(\frac{1}{x}\right) = $
\MLSimplifyQuestion{40}{2*ln(x)}{10}{x}{4}{512}{SIMPLE9}.
\item $g(x) := \left(\MEU^x\right)^2 \cdot \MEU^{-x} = $
\MLFunctionQuestion{40}{exp(x)}{10}{x}{4}{DZ10}.
\end{MExerciseItems}
%\MInputHint{Schreiben Sie Exponentialfunktionen mit \texttt{exp}, beispielsweise schreiben Sie $\MEU^{4x^2}$ als \texttt{exp(4*x^2)}.}
Thus, we have:
\begin{MExerciseItems}
\item $f'(x) = $ \MLSimplifyQuestion{40}{2/x}{1}{x}{20}{512}{TFR1}.
\item $g'(x) = $ \MLSimplifyQuestion{40}{exp(x)}{10}{x}{4}{0}{TFR2}.
\end{MExerciseItems}
\MInputHint{Enter the exponential functions as \texttt{exp}, for example, enter $\MEU^{4x^2}$ as \texttt{exp(4*x^2)}.}

\begin{MHint}{Solution}
 \begin{MExerciseItems}
  \item We have
  \[
   f(x) = 3 \ln (x) + \ln \left( \frac1x \right) = \ln (x^3) + \ln \left( \frac1x \right) = \ln \left( x^3 \cdot \frac1x \right)
   = \ln \left( x^2 \right) \MDFPeriod
  \]
  For the value of the derivative at the point $x$ ($x>0$), it follows from the chain rule
  $f'(x) = \frac{1}{x^2} \cdot 2 x = \frac{2}{x}$. (The chain rule is explained in detail 
  in Section~\MRef{Kettenregel}.)
  \item We have
  \[
  g(x) = \left( \MEU^x \right)^2 \cdot \MEU^{-x} = \MEU^x \cdot \MEU^x \cdot \MEU^{-x} = \MEU^{x+x-x} = \MEU^x \MDFPeriod
  \]
  Hence, it follows that $g'(x) = \MEU^x$.
 \end{MExerciseItems}
\end{MHint}
\end{MExercise}

\end{MExercises}



%%%Abschnitt
\MSubsection{Calculation Rules}\MLabel{M07_Rechenregeln}

\begin{MIntro}
\MDeclareSiteUXID{VBKM07_Rechenregeln_Intro}
Using a few calculation rules and the derivatives presented in the last section, 
a variety of functions can be differentiated. 
\end{MIntro}

\begin{MXContent}{Multiples and Sums of Functions}{Multiples and Sums}{STD}
\MDeclareSiteUXID{VBKM07_Rechenregeln_Summen}

In the following, $u, v: D \rightarrow \R$ will denote two arbitrary differentiable 
functions, and $r$ denotes an arbitrary real number.


\begin{MXInfo}{Sum Rule and Constant Factor Rule}
Let two differentiable functions $u$ and $v$ be given. Then, the sum
$f := u+v$ with $f(x) = (u+v)(x) := u(x) + v(x)$ is also differentiable, and we 
have
$$
f'(x) = u'(x) + v'(x) \MDFPeriod%%
$$
Likewise, a function multiplied by a factor $r$, i.e. $f := r \cdot u$ with  
$f(x) = (r \cdot u)(x) := r \cdot u(x)$, is also differentiable, and we have
$$
f'(x) = r \cdot u'(x) \MDFPeriod%%
$$
\end{MXInfo}

Using these two rules together with the differentiation rules for monomials $x^n$,
any arbitrary polynomial can be differentiated. Here are some examples.

\begin{MExample}
The polynomial $f$ with the mapping rule $f(x) = \frac{1}{4} x^3 - 2 x^{2} +5 $ 
is differentiable, and we have
\[
f'(x) = \frac{3}{4} x^2 - 4 x \MDFPeriod%%
\]
The derivative of the function $g: \MoIl 0 \MIntvlSep \infty\MoIr \rightarrow \R$ 
with $g(x) = x^3 + \ln(x)$ is
\[
g': \MoIl 0 \MIntvlSep \infty \MoIr \rightarrow \R \MDFPSpace \text{ with } \MDFPSpace g'(x) = 3 x^2 + \frac{1}{x} = \frac{3 x^3 + 1}{x} \MDFPeriod%%
\]
%Mit $\ln(x^3) = 3 \ln(x)$ ergibt sich die Ableitung von 
%$g(x) = \ln(x^3) = 3 \ln(x)$ f"ur $x > 0$ zu
%\[
%g'(x) = \frac{3}{x} %%
%\]
Differentiating the function 
$h: [0 \MIntvlSep \infty\MoIr \rightarrow \R$ with $h(x) = 4^{-1} \cdot x^2 - \sqrt{x} %
 = \frac{1}{4} x^2 + (-1) \cdot x^{\frac{1}{2}}$ results, for $x>0$, in
\[
h'(x) = \frac{1}{2} x - \frac{1}{2} x^{-\frac{1}{2}} %
 = \frac{x^{\frac{3}{2}} - 1}{2 \sqrt{x}} \MDFPeriod%%
\]
\end{MExample}

\end{MXContent}

\begin{MXContent}{Product and Quotient of Functions}{Product and Quotient}{STD}
\MDeclareSiteUXID{VBKM07_Produktregel}

\begin{MXInfo}{Product and Quotient Rule}
Likewise, the product of functions, i.e. $f := u \cdot v$ with $f(x) = (u \cdot v)(x) := u(x) \cdot v(x)$, 
is differentiable, and the following \textbf{product rule} applies:
$$
f'(x) = u'(x) \cdot v(x) +  u(x) \cdot v'(x) \MDFPeriod%%
$$

The quotient of functions, i.e. $f := \frac{u}{v}$ with 
$f(x) = \left( \frac{u}{v} \right) (x) := \frac{u(x)}{v(x)}$,
is defined and differentiable for all $x$ with $v(x) \neq 0$, and the 
following \textbf{quotient rule} applies:
\ifttm
$$
f'(x) = \frac{u'(x) \cdot v(x) - u(x) \cdot v'(x)}{\left( v(x) \right)^2} \MDFPeriod%%
$$
\else
$$
f'(x) = %
\frac{u'(x) \cdot v(x) \,\textcolor{red}{\mathbf{-}}\, u(x) \cdot v'(x)}%
{\left(v(x)\right)^2} \MDFPeriod%%
$$
\fi
\end{MXInfo}

These calculation rules shall be illustrated by means of a few examples.


\begin{MExample}
Find the derivative of $f: \R \rightarrow \R$ with $f(x) = x^2 \cdot \MEU^x$.
The product rule can be applied choosing, for example, $u(x) = x^2$ and $v(x) = \MEU^x$. The corresponding 
derivatives are $u'(x) = 2x$ and $v'(x) = \MEU^x$. Combining these terms according to the 
product rule results in the derivative of the function~$f$:
\[
f': \R \rightarrow \R \MDFPSpace , \MDFPaSpace x \mapsto f'(x) = 2 x \MEU^x + x^2 \MEU^x = (x^2 + 2x) \MEU^x \MDFPeriod%%
\]

Next, we investigate the tangent function $g$ with $g(x) = \tan(x) = \frac{\sin(x)}{\cos(x)}$ 
($\cos(x) \neq 0$).In order to use the quotient rule we set  $u(x) = \sin(x)$ and $v(x) = \cos(x)$.
The corresponding derivatives are  $u'(x) = \cos(x)$ and $v'(x) = -\sin(x)$. Combining these terms 
and applying the quotient rule results in 
the derivative of the function~$g$:
\[
g'(x) = \frac{\cos(x) \cdot \cos(x) - \sin(x) \cdot (-\sin(x))}{\cos^2(x)} \MDFPeriod%
\]
This result can be transformed into any of the following expressions:
\[
g'(x) = 1 + \left(\frac{\sin(x)}{\cos(x)}\right)^2 %
 = 1 + \tan^2(x) %
 = \frac{1}{\cos^2(x)} \MDFPeriod%%
\]
For the last transformation, the relation $\sin^2(x) + \cos^2(x) = 1$ was used, which was
given in Module~\MNRef{VBKM05} (see Section~\MRef{Abschnitt:TrigonometrieAmDreieck}).
\end{MExample}

\begin{MExercise}
Calculate the derivative of $f: \R \rightarrow \R$ with $f(x)=\sin(x)\cdot x^3$
by factorising the product into two factors, taking the derivatives of each single 
factor, and finally combining the results according to the product rule.

\begin{MExerciseItems}
\item{The derivative of the left factor \MEquationItem{$u(x)$}{\MLFunctionQuestion{10}{sin(x)}{5}{x}{5}{DS2}} is \MEquationItem{$u'(x)$}{\MLFunctionQuestion{10}{cos(x)}{5}{x}{5}{DS5}}.}
\item{The derivative of the right factor \MEquationItem{$v(x)$}{\MLFunctionQuestion{10}{x^3}{5}{x}{5}{DS3}} is \MEquationItem{$v'(x)$}{\MLFunctionQuestion{10}{3*x^2}{5}{x}{5}{DS4}}.}
\item{Thus, applying the product rule to $f$ results in \MEquationItem{$f'(x)$}{\MLFunctionQuestion{30}{(cos(x)*x^3)+(sin(x)*3*x^2)}{5}{x}{5}{DS6}}.}
\end{MExerciseItems}
\begin{MHint}{Solution}
The four terms are 
$$
u(x) \;=\; \sin(x) \MDFPSpace , \MDFPaSpace
u'(x) \;=\; \cos(x) \MDFPSpace , \MDFPaSpace
v(x) \;=\; x^3 \MDFPSpace , \MDFPaSpace
v'(x) \;=\; 3x^2 \MDFPSpace ,
$$
and combining them according to the product rule results in
$$
f'(x) \;=\; \cos(x)\cdot x^3+\sin(x)\cdot 3x^2 \MDFPeriod
$$
\end{MHint}
\end{MExercise}

\begin{MExercise}
Calculate the derivative of $f: \MoIl 0 \MIntvlSep \infty\MoIr \rightarrow \R$ with $f(x)=\frac{\ln(x)}{x^2}$
by splitting the quotient up into numerator and denominator, taking the derivatives of both, 
and combining them according to the quotient rule.

\begin{MExerciseItems}
\item{The derivative of the numerator \MEquationItem{$u(x)$}{\MLSimplifyQuestion{10}{ln(x)}{5}{x}{5}{512}{DS7}} is \MEquationItem{$u'(x)$}{\MLSimplifyQuestion{10}{1/x}{5}{x}{5}{512}{DS8}}.}
\item{The derivative of the denominator \MEquationItem{$v(x)$}{\MLFunctionQuestion{10}{x^2}{5}{x}{5}{DZ2}} is \MEquationItem{$v'(x)$}{\MLFunctionQuestion{10}{2*x}{5}{x}{5}{DZ1}}.}
\item{Thus, applying the quotient rule to $f$ results in \MEquationItem{$f'(x)$}{\MLSimplifyQuestion{30}{(1/x*x^2-ln(x)*2*x)/(x^4)}{5}{x}{5}{512}{DS9}}.}
\end{MExerciseItems}
\begin{MHint}{Solution}
The four terms are 
$$
u(x) \;=\; \ln(x) \MDFPSpace , \MDFPaSpace
u'(x) \;=\; \frac1x \MDFPSpace , \MDFPaSpace
v(x) \;=\; x^2 \MDFPSpace , \MDFPaSpace
v'(x) \;=\; 2x \MDFPSpace ,
$$
and combining them according to the quotient rule results in
$$
f'(x) \;=\; \frac{\frac1x\cdot x^2-\ln(x)\cdot 2x}{x^4} \; =\; \frac{1-2\ln(x)}{x^3} \MDFPSpace,
$$
where the last transformation step (cancelling $x$) is just to simplify the expression.
\end{MHint}
\end{MExercise}
\end{MXContent}

\begin{MXContent}{Composition of Functions}{Composition}{STD}
\MDeclareSiteUXID{VBKM07_Verkettung}
\MLabel{Kettenregel}

Finally, we investigate composition of functions (see Module~\MNRef{VBKM06}, Section~\MRef{Verkettung}):
what happens if a function $u$ (the inner function) is substituted into another function $v$ (the outer function)?
In mathematics, such a composition is denoted by $f := v \circ u$ with $f(x) = (v \circ u)(x) := v(u(x))$.
That is, first the value of a function $u$ is determined depending on the variable $x$. The value $u(x)$ calculated 
this way is then used as an argument of the function $v$. This results in the final function value $v(u(x))$.

\begin{MXInfo}{Chain Rule}
The derivative of the function $f := v \circ u$ with $f(x) = (v \circ u)(x) := v(u(x))$ can be 
calculated applying the \textbf{chain rule}:
$$
f'(x) = v'(u(x)) \cdot u'(x) \MDFPeriod%%
$$
Here, the expression $v'(u(x))$ is considered in such a way that $v$ is a function of $u$ and thus,
the derivative is taken with respect to $u$; then $v'(u)$ is evaluated for $u = u(x)$.\\
The following phrase is a useful summary: the derivative of a composite function is the product 
of the outer derivative and the inner derivative.
\end{MXInfo}

This differentiation rule shall be illustrated by a few examples.

\begin{MExample}
Find the derivative of the function $f: \R \rightarrow \R$ with $f(x) = (3 - 2 x)^5$.
To apply the chain rule, inner and outer functions must be identified. If we 
take the function $u(x) = 3 - 2x$ as the inner function $u$, then the outer function 
$v$ is given by $v(u) = u^5$. With this, we have the required form $v(u(x)) = f(x)$.

Taking the derivative of the inner function $u$ with respect to $x$ results in $u'(x) = -2$.
For the outer derivative, the function $v$ is differentiated with respect to $u$, which results
in $v'(u) = 5 u^4$. Inserting these terms into the chain rule results in the 
derivative $f'$ of the function $f$ with
\[
f'(x) = 5 (u(x))^4 \cdot (-2) = 5 (3 - 2 x)^4 \cdot (-2) = -10 (3 - 2 x)^4 \MDFPeriod %%
\]

As a second example, let's calculate the derivative of $g: \R \rightarrow \R$ with $g(x) = \MEU^{x^3}$.
For the inner function $u$ the assignment $x \mapsto u(x) = x^3$ and 
for the outer function $v$ the assignment $u \mapsto v(u) = \MEU^u$ is appropriate. Taking 
the inner and the outer derivative results in $u'(x) = 3 x^2$ and $v'(u) = \MEU^u$. Inserting these 
terms into the chain rule results in the derivative of the function $g$:
\[
g': \R \rightarrow \R \MDFPSpace , \MDFPaSpace x \mapsto g'(x) = \MEU^{u(x)} 
\cdot 3 x^2 = \MEU^{x^3} \cdot 3 x^2 = 3 x^2 \MEU^{x^3} \MDFPeriod %%
\]
\end{MExample}

\end{MXContent}


%%%Uebungen zum Abschnitt:
\begin{MExercises}
\MDeclareSiteUXID{VBKM07_Rechenregeln_Exercises}

\begin{MExercise}
Calculate the derivatives of the functions $f$, $g$, and $h$ defined by the following mapping rules:
\begin{MExerciseItems}
\item The derivative of $f(x) := 3 + 5 x$
is $f'(x) = $\MLParsedQuestion{30}{5}{4}{DG19}.
%
\item The derivative of $g(x) := \frac{1}{4 x} - x^3$
is $g'(x) = $\MLSimplifyQuestion{30}{-1/(4*x^2) - 3*x^2}{1}{x}{20}{512}{SIMPLE10}.
%
\item The derivative of $h(x) := 2 \sqrt{x} + 4 x^{-3}$
is $h'(x) = $\MLSimplifyQuestion{30}{1/sqrt(x) - 12/(x^4)}{1}{x}{20}{512}{SIMPLE11}.
\end{MExerciseItems}
\begin{MHint}{Solution}
 \begin{MExerciseItems}
  \item We have $f'(x) = 0 + 5 \cdot 1 \cdot x^0 = 0 + 5 = 5$.
  \item Since $g(x) = \frac{1}{4x} - x^3 = \frac14 x^{-1} - x^3$, we have
  $g'(x) = \frac14 \cdot (-1) \cdot x^{-2} - 3 \cdot x^2 = - \frac{1}{4 x^2} - 3 x^2$.
  \item Since $h(x) = 2 \sqrt{x} + 4 x^{-3} = 2 x^\frac12 + 4 x^{-3}$, we have
  $h'(x) = 2 \cdot \frac12 \cdot x^{-\frac12} + 4 \cdot (-3) \cdot x^{-4} = \frac{1}{\sqrt{x}} - \frac{12}{x^4}$.
 \end{MExerciseItems}
\end{MHint}
\end{MExercise}

\begin{MExercise}
Calculate the derivatives of the functions $f$, $g$, and $h$ described by the following mapping rules, and simplify
the results.
\begin{MExerciseItems}
\item The derivative of $f(x) := \cot x = \frac{\cos x}{\sin x}$
is $f'(x) = $\MLSimplifyQuestion{30}{-1/(sin(x) * sin(x))}{10}{x}{4}{512}{DG20}.
%
\item The derivative of $g(x) := \sin(3 x) \cdot \cos(3 x)$
is $g'(x) = $\MLFunctionQuestion{30}{3*cos(6*x)}{10}{x}{4}{DG21}.
%
\item The derivative of $h(x) := \frac{\sin(3 x)}{\sin(6 x)}$
is $h'(x) = $\MLSimplifyQuestion{30}{3/2 * tan(3*x)/cos(3*x)}{10}{x}{4}{512}{SIMPLE12}.
\end{MExerciseItems}
\begin{MHint}{Solution}
 \begin{MExerciseItems}
  \item From the quotient rule, we find
  \[
  f'(x) = \frac{(- \sin (x)) \cdot \sin (x) - \cos (x) \cdot \cos (x)}{\left( \sin (x) \right)^2}
  = - \frac{\sin^2 (x) + \cos^2 (x)}{\sin^2 (x)} = - \frac{1}{\sin^2 (x)} \MDFPeriod
  \]
  \item From the product rule and the chain rule, we find
  \[
   g'(x) = \cos (3x) \cdot 3 \cdot \cos (3x) + \sin (3x) \cdot \left( - \sin (3x) \right) \cdot 3
   = 3 \left( \cos^2 (3x) - \sin^2 (3x) \right) \MDFPeriod
  \]
  Since generally $\cos^2 (u) - \sin^2 (u) = \cos (2u)$, we have $g'(x) = 3 \cos (6x)$.
  \item According to the general relation $\sin (2u) = 2 \sin(u) \cos(u)$, we have
  \[
  h(x) = \frac{\sin (3x)}{\sin (6x)} = \frac{\sin (3x)}{2 \sin (3x) \cos (3x)} = \frac{1}{2 \cos (3x)}
  = \frac12 \cdot \left( \cos (3x) \right)^{-1} \MDFPeriod
  \]
    Applying the chain rule several times results in
  \[
  h'(x) = \frac12 \cdot (-1) \cdot \left( \cos (3x) \right)^{-2} \cdot \left( - \sin (3x) \right) \cdot 3
  = \frac{3 \sin (3x)}{2 \cos^2 (3x)} = \frac{3 \tan (3x)}{2 \cos (3x)} \MDFPeriod
  \]
 \end{MExerciseItems}
\end{MHint}
\end{MExercise}

\begin{MExercise}
Calculate the derivatives of the functions $f$, $g$, and $h$ defined by the following mapping rules:
\begin{MExerciseItems}
\item The derivative of $f(x) := \MEU^{5 x}$
is $f'(x) = $\MLFunctionQuestion{30}{5*exp(5*x)}{10}{x}{4}{DG22}.
%
\item The derivative of $g(x) := x \cdot \MEU^{6 x}$
is $g'(x) = $\MLFunctionQuestion{30}{(6*x+1)*exp(6*x)}{10}{x}{4}{DG23}.
%
\item The derivative of $h(x) := (x^2 - x) \cdot \MEU^{-2 x}$
is $h'(x) = $\MLFunctionQuestion{30}{-(2*x^2 - 4*x + 1) * exp(-2*x)}{10}{x}{4}{DG24}.
\end{MExerciseItems}
\begin{MHint}{Solution}
 \begin{MExerciseItems}
  \item From the chain rule, we immediately find $f'(x) = 5 \MEU^{5x}$.
  \item From the product rule and the chain rule, we find
   $g'(x) = 1 \cdot \MEU^{6x} + x \cdot \MEU^{6x} \cdot 6 = \MEU^{6x} (1 + 6x)$.
  \item From the product rule and the chain rule, we find 
  $h'(x) = (2x - 1) \cdot \MEU^{-2x} + (x^2 - x) \cdot \MEU^{-2x} \cdot (-2)
  = - (2x^2 - 4x + 1) \MEU^{-2x}$.
 \end{MExerciseItems}
\end{MHint}
\end{MExercise}

\begin{MExercise}
Calculate the first four derivatives of $f: \R \rightarrow \R$ with $f(x) := \sin(1 - 2x)$.

Answer: 
The $k$th derivative of $f$ is denoted by $f^{(k)}$. Here, $f^{(1)} = f'$, $f^{(2)}$ is the derivative of $f^{(1)}$,
$f^{(3)}$ is the derivative of $f^{(2)}$, etc. Thus, we have:
\begin{itemize}
\item
 $f^{(1)}(x) = $\MLFunctionQuestion{30}{-2 * cos(1 - 2*x)}{10}{x}{4}{DG25}.
\item
 $f^{(2)}(x) = $\MLFunctionQuestion{30}{-4 * sin(1 - 2*x)}{10}{x}{4}{DG26}.
\item
 $f^{(3)}(x) = $\MLFunctionQuestion{30}{8 * cos(1 - 2*x)}{10}{x}{4}{DG27}.
\item
 $f^{(4)}(x) = $\MLFunctionQuestion{30}{16 * sin(1 - 2*x)}{10}{x}{4}{DG28}.
\end{itemize}
\begin{MHint}{Solution}
    From the chain rule, we find successively:
 \begin{eqnarray*}
  f^{(1)} (x) & = & \cos(1 - 2x) \cdot (-2) = -2 \cos(1 - 2x) \MDFPSpace , \\
  f^{(2)} (x) & = & -2 \cdot \left( - \sin (1 - 2x) \right) \cdot (-2) = -4 \sin (1 - 2x) \MDFPSpace , \\
  f^{(3)} (x) & = & -4 \cdot \cos (1 - 2x) \cdot (-2) = 8 \cos (1 - 2x) \MDFPSpace , \\
  f^{(4)} (x) & = & 8 \cdot \left( - \sin (1 - 2x) \right) \cdot (-2) = 16 \sin (1 - 2x) \MDFPeriod
 \end{eqnarray*}
\end{MHint}
\end{MExercise}

\end{MExercises}



%%%Abschnitt
\MSubsection{Properties of Functions}\MLabel{M07_Eigenschaften}

\begin{MIntro}
\MDeclareSiteUXID{VBKM07_Eigenschaften_Intro}

The derivative was introduced above by means of a tangent line to a graph of a function. This tangent line 
describes the given function ``approximately'' in a certain region. The properties of this tangent line give also
information on the properties of the approximated function in this region.
\end{MIntro}

\begin{MXContent}{Monotony}{Monotony}{STD}
\MDeclareSiteUXID{VBKM07_Monotonie}

The derivative of a function can be used to study the growth behaviour, i.e.\
whether the function values 
increase or decrease for increasing values of $x$. For this purpose, we consider a function $f: D \rightarrow \R$ 
that is differentiable on $\MoIl a\MIntvlSep b\MoIr \subseteq D$:
\begin{center}
\MTikzAuto{%
\begin{small}
\begin{tikzpicture}[line width=1.5pt,scale=1.0, %
declare function={
  x1 = 2;
  x0 = 4;
  fkt(\x) = 1/4*(\x - 3)*(\x - 3) + 0.75;
  TangenteAblplus(\x) = 1 + 1/2*(\x - x0); % $f(x_0) = f(4) = 1$.
  TangenteAblminus(\x) = 1 - 1/2*(\x - x1); % $f(x_1) = f(2) = 1$.
}
] %[every node/.style={fill=white}] 
%,every node/.style={fill=white}] 
%Koordinatenachsen:
\draw[->] (-0.6, 0) -- (6, 0) node[below left]{$x$}; %x-Achse
\draw[->] (0, -0.6) -- (0, 3) node[below left]{$y$}; %y-Achse
%Achsenbeschriftung:
\foreach \x in {1, 2, 3, 4, 5} \draw (\x, 0) -- ++(0, -0.1); %
% node[below] {$\x$}; 
\foreach \y in {1, 2} \draw (0, \y) -- ++(-0.1, 0); %
% node[below left] {$\y$};
%\node[below left] at (0, 0) {$0$};
%Hilfslinien:
\draw[color=black!50!white] (x0, {fkt(x0)}) -- ({x0+1}, {fkt(x0)});
\draw[->,color=black!50!white] %
 ({x0+1}, {fkt(x0)}) -- ({x0+1}, {TangenteAblplus(x0+1)});
%
\draw[color=black!50!white] (x1, {fkt(x1)}) -- ({x1-1}, {fkt(x1)});
\draw[->,color=black!50!white] %
 ({x1-1}, {TangenteAblminus(x1-1)}) -- ({x1-1}, {fkt(x1)});
%Funktion:
\draw[domain=0.8:5.2,samples=120,color=\jccolorfkt] %
 plot (\x, {fkt(\x)});
%Tangenten:
\draw[domain={x1-1.2}:{x1+0.8},samples=120,color=blue!50!black] %
 plot (\x, {TangenteAblminus(\x)});
\filldraw[color=blue!50!black] (x0, {fkt(x0)}) circle (1pt); % Ber"uhrpunkt.
%
\draw[domain={x0-0.8}:{x0+1.2},samples=120,color=blue!50!black] %
 plot (\x, {TangenteAblplus(\x)});
\filldraw[color=blue!50!black] (x1, {fkt(x1)}) circle (1pt); % Ber"uhrpunkt.
%Beschriftung:
\node[right] at (5.1, 1.25) {$f'(x_0) = m_0 > 0$};
\node[style={fill=white},left] at (0.9, 1.25) {$f'(x_1) = m_1 < 0$};
\node[below] at (x0, -0.1) {$x_0$};
\node[below] at (x1, -0.1) {$x_1$};
\end{tikzpicture}
\end{small}
}
\end{center}

%\begin{MXInfo}{Monotonie}
If $f'(x) \leq 0$ for all $x$ between $a$ and $b$, then $f$ is 
monotonically decreasing on the interval $\MoIl a\MIntvlSep b\MoIr$.

If $f'(x) \geq 0$ for all $x$ between $a$ and $b$, then $f$ is 
monotonically increasing on the interval $\MoIl a\MIntvlSep b\MoIr$.

%\end{MXInfo}

Thus, it is sufficient to determine the sign of the derivative $f'$ 
to decide whether a function is monotonically increasing or decreasing 
on the interval $\MoIl a \MIntvlSep b\MoIr$.
%\end{MXContent}

\begin{MExample}
The function $f: \R \rightarrow \R, x \mapsto x^3$ is differentiable with  
$f'(x) = 3 x^2$. Since $x^2 \geq 0$ for all $x \in \R$, we have
$f'(x) \geq 0$, and therefore $f$ is monotonically increasing.

For $g: \R \rightarrow \R$ with $g(x) = 2 x^3 + 6 x^2 - 18 x + 10$, the function
$g'(x) = 6 x^2 + 12 x - 18 = 6 (x + 3) (x - 1)$ has the roots $x_1 = -3$ 
and $x_2 = 1$. If the monotony of the function $g$ is investigated, then 
three regions are to be distinguished in which $g'$ has a different sign.

The following table is used to determine in which region the derivative 
of $g$ is positive or negative. These regions correspond to the monotony 
regions of $g$. The entry ``$+$'' says that the considered term is positive on 
the given interval. If the term is negative, then ``$-$'' is entered.
\[
\begin{array}{c||c|c|c}
x & x < -3 & -3 < x < 1 & 1 < x \\
\hline\hline
x + 3 & - & + & + \\\hline
x - 1 & - & - & + \\\hline
g'(x) & + & - & + \\\hline\hline
g \text{ is monotonically } & \text{ increasing } & \text{ decreasing } & \text{ increasing } \\
\end{array}
\]
For the function $h: \R \setminus \{ 0 \} \rightarrow \R$ with $h(x) = \frac{1}{x}$,
we have $h'(x) = - \frac{1}{x^2}$, that is $h'(x) < 0$ for all $x \neq 0$.

Even though the function $h$ exhibits the same monotony behaviour for the two subregions 
$x<0$ and $x>0$, it is not monotonically decreasing on the entire region. As a counterexample 
let us consider the function values $h(-2) = -\frac{1}{2}$ and $h(1) = 1$. Here, we have $-2 < 1$
but also $h(-2) < h(1)$. This corresponds to an increasing growth behaviour if we change from 
one subregion to the other. The statement that the function $h$ is monotonically decreasing on
$\MoIl -\infty\MIntvlSep 0\MoIr$ thus means that the restriction of $h$ on this interval 
is monotonically decreasing. Moreover, the function $h$ is also monotonically decreasing for all $x > 0$.
\end{MExample}
\end{MXContent}


%\MSubsubsection{Zweite Ableitung und Kr"ummungseigenschaften}
\begin{MXContent}{Second Derivative and Bending Properties (Curvature)}{Second Derivative and Bending Properties (Curvature)}{STD}
\MDeclareSiteUXID{VBKM07_Kruemmungseigenschaften}

%Seien $a,b \in D$ mit $a<b$.
Let us consider a function $f: D \rightarrow \R$ that is differentiable on the interval $\MoIl a\MIntvlSep b\MoIr \subseteq D$.
If its derivative $f'$ is also differentiable on the interval $\MoIl a\MIntvlSep b\MoIr \subseteq D$, then 
$f$ is called \textbf{twice-differentiable}. The derivative of the first derivative of $f$ ($(f')' = {f'}'$) is called the \textbf{second derivative} of the function $f$.

The second derivative of the function $f$ can be used to investigate the bending behaviour (curvature) of the function:

\begin{MXInfo}{Bending Properties (Curvature)}
  If ${f'}'(x) \geq 0$ for all $x$ between $a$ and $b$, then $f$ is called \textbf{convex} (\textbf{left curved}
  or \textbf{concave up}) on the interval $\MoIl a\MIntvlSep b\MoIr$.

  If ${f'}'(x) \leq 0$ for all $x$ between $a$ and $b$, then $f$ is called \textbf{concave} (\textbf{right curved}
  or \textbf{concave down}) on the interval $\MoIl a\MIntvlSep b\MoIr$.
\end{MXInfo}

Thus, it is sufficient to determine the sign of the second derivative ${f'}'$ to decide 
whether a function is convex (left curved) or concave (right curved).
%\end{MXContent}

\begin{MXInfo}{Comment on the Notation}
The second derivative and further {``higher''} derivatives are often denoted using superscript 
natural numbers in round brackets: $f^{(k)}$ then denotes the $k$th derivative of $f$. In particular, 
this notation is used in generally written formulas even for the (first) derivative ($k=1$)
and for the function $f$ itself ($k=0$).

Hence, 
\begin{itemize}
\item $f^{(0)} = f$ denotes the function $f$, 
\item $f^{(1)} = f'$ denotes the (first) derivative,
\item $f^{(2)} = {f'}'$ the second derivative,
\item $f^{(3)}$ the third derivative, and
\item $f^{(4)}$ the fourth derivative of $f$.
\end{itemize}
This list can be continued as long as the derivatives of $f$ exist.
\end{MXInfo} 

The following example shows that a monotonically increasing function can be 
convex on one region and concave on another.

\begin{MExample}
Certainly, the function $f: \R \rightarrow \R, x \mapsto x^3$ is at least twice-differentiable. 
Since $f'(x) = 3 x^2 \geq 0$ for all $x \in \R$, the function $f$ is monotonically increasing 
on its entire domain. Moreover, we have ${f'}'(x) = 6 x$. Thus, for all $x < 0$, we also have 
${f'}'(x) < 0$ and hence, the function $f$ is concave (right curved) on this region. For $x > 0$, 
we have ${f'}'(x) > 0$. Hence, for $x > 0$, the function $f$ is convex (left curved).
\end{MExample}
%\end{MContent}
\end{MXContent}


%%%Uebungen zum Abschnitt:
\begin{MExercises}
\MDeclareSiteUXID{VBKM07_Verhalten_Exercises}

\begin{MExercise}
Specify the (maximum) open intervals on which the function $f$ with $f(x) := \frac{x^2 - 1}{x^2 + 1}$
is monotonically increasing or decreasing.
%In welchen m"oglichst gro"sen offenen Intervallen sind die Funktionen
%\begin{MExerciseItems}
%\item $f(x) := x^2 - 12\right) \cdot (x^2 + 5)$
%\item $g(x) := \frac{x^2 + 1}{x^2 - 1}$
%\end{MExerciseItems}
%monoton wachsend beziehungsweise monoton fallend?

Answer:
\begin{itemize}
\item $f$ is monotonically \MLQuestion{12}{decreasing}{DFA1} on $\MoIl -\infty\MIntvlSep 0\MoIr$.
%\item $f$ auf $(-infty,0)$ monoton \MQuestion{12}{fallend}.
%\item $f$ auf \MIntervalQuestion{20}{(-infty,0)}{4} monoton fallend.
%\item $f$ auf $(-infty$\MLParsedQuestion{8}{0}{4}{PARSEDQUEST5}$)$ monoton fallend.
%
\item $f$ is monotonically \MLQuestion{12}{increasing}{DFA2} on $\MoIl 0\MIntvlSep \infty\MoIr$.
%\item $f$ auf $(0, infty)$ monoton \MQuestion{12}{wachsend}.
%\item $f$ auf \MIntervalQuestion{20}{(0,infty)}{4} monoton wachsend.
%\item $f$ auf $($\MIntervalQuestion{8}{0}{4}$, infty)$ monoton wachsend.
\end{itemize}
\begin{MHint}{Solution}
The derivative $f'$ of the function $f$ is given by
\[
f'(x) = \frac{2x \cdot (x^2 + 1) - (x^2 - 1) \cdot 2x}{(x^2 + 1)^2}
= \frac{2x \left( x^2 + 1 - x^2 + 1 \right)}{(x^2 + 1)^2} = \frac{4x}{(x^2 + 1)^2} \MDFPeriod
\]
Since the denominator of $f'(x)$ is always positive, the sign of $f'(x)$ is determined solely by the numerator.
For all negative $x \in \R$, we have $f'(x) < 0$, and hence the function $f$ is monotonically decreasing on 
this region. In contrast, for all positive $x \in \R$, we have $f'(x) > 0$, and hence the function $f$ is 
monotonically increasing on this region.
\end{MHint}
\end{MExercise}


\begin{MExercise}
Specify the (maximum) open intervals $\MoIl c\MIntvlSep d\MoIr$ on which the function $f$ with $f(x) := \frac{x^2 - 1}{x^2 + 1}$
for $x > 0$ is convex or concave.
%In welchen m"oglichst gro"sen offenen Intervallen sind die Funktionen
%konvex beziehungsweise konkav?
%\begin{MExerciseItems}
%\item $f(x) := \frac{6 x}{x^2 + 4}$
%\item $g(x) := \left(x - 2 \sqrt{3}\right) \cdot ($
%\end{MExerciseItems}
Answer:
\begin{itemize}
\item The function $f$ is convex on \MLIntervalQuestion{35}{(0, 1/sqrt(3))}{4}{DEG1}.
%
\item The function $f$ is concave on \MLIntervalQuestion{35}{(1/sqrt(3),infty)}{4}{DEG2}.
\end{itemize}
\MInputHint{Open intervals can be entered in the form $(a;b)$, closed intervals in the form $[a;b]$. $a$ and 
$b$ can be arbitrary expressions. For entering an interval, do not use the notation $]a;b[$ for open intervals.
In your answer, enter \texttt{infty} for $\infty$.}

\begin{MHint}{Solution}
  From the quotient rule, we have for the first and the second derivative of $f$:
 \begin{eqnarray*}
  f'(x) & = & \frac{4x}{(x^2 + 1)^2} \MDFPSpace , \\
  {f'}'(x) & = & \frac{4 \cdot (x^2 + 1)^2 - 4x \cdot 2 (x^2 + 1) \cdot 2x}{(x^2 + 1)^4}
  = \frac{4x^2 + 4 - 16x^2}{(x^2 + 1)^3} = \frac{4 (1-3x^2)}{(x^2 + 1)^3} \MDFPeriod
 \end{eqnarray*}
 Since $4/(x^2 + 1)^3$ is always positive, the sign of ${f'}'(x)$ is only determined by the factor $(1 - 3x^2)$. 
 The roots of ${f'}'(x)$ are at $x_0 = \pm \frac{1}{\sqrt{3}}$. Thus, for $x>0$, the derivative ${f'}'(x)$
 is greater than $0$ on the open interval $\MoIl[\left] 0 \MIntvlSep \frac{1}{\sqrt{3}} \MoIr[\right]$, so
 the function $f$ is convex (left curved) on this region. On the interval 
 $\MoIl[\left] \frac{1}{\sqrt{3}} \MIntvlSep \infty \MoIr[\right]$, we have ${f'}'(x) < 0$; the function
 $f$ is concave (right curved) on this region.
 \end{MHint}
\end{MExercise}


\begin{MExercise}
Consider the function $f: [-\MZahl{4}{5} \MIntvlSep 4] \rightarrow \R$ with $f(0) := 2$. Its derivative 
$f'$ has the graph shown in the figure below:

\begin{center}
\MTikzAuto{%
%{Ableitung von $f$}
\begin{small}
\begin{tikzpicture}[line width=1.5pt,scale=0.6, %
declare function={
  fkt(\x) = 1/40*(\x + 4)*(\x)*(\x - 3)*(\x - 3);
}
] %[every node/.style={fill=white}] 
%Koordinatenachsen:
\draw[->] (-5, 0) -- (5, 0) node[below left]{$x$}; %x-Achse
\draw[->] (0, -3.5) -- (0, 4) node[below left]{$y$}; %y-Achse
%Achsenbeschriftung:
\foreach \x in {-4, -3, -2, -1} \draw (\x, 0) -- ++(0, 0.1) %
 node[above] {$\x$}; 
\foreach \x in {1, 2, 3, 4} \draw (\x, 0) -- ++(0, -0.1) %
 node[below] {$\x$}; 
\foreach \y in {-3, -2, -1} \draw (0, \y) -- ++(0.1, 0) %
 node[right] {$\y$};
\foreach \y in {1, 2, 3} \draw (0, \y) -- ++(-0.1, 0) %
 node[left] {$\y$};
%\node[below left] at (0, 0) {$0$};
%Funktion:
\draw[domain=-4.5:4,samples=120,color=\jccolorfkt] %
 plot (\x, {fkt(\x)});
\end{tikzpicture}
\end{small}
}
\end{center}

\begin{MExerciseItems}
\item Where is the function $f$ monotonically increasing and where it is monotonically decreasing?
Find the maximum open intervals $\MoIl c\MIntvlSep d\MoIr$ on which $f$ has this property.
%
\item What can you say about the maximum and minimum points of the function $f$?
% f"ur $f$ zu deren Maximal- beziehungsweise Minimalstellen?
\end{MExerciseItems}

Answer:
\begin{itemize}
\item The function $f$ is monotonically \MLQuestion{16}{increasing}{DFA6} on 
  $\MoIl[\big] -\MZahl{4}{5}\MIntvlSep $\MLParsedQuestion{8}{-4}{4}{DEG3}$\MoIr[\big]$.
%
\item The function $f$ is monotonically \MLQuestion{16}{decreasing}{DFA7} on 
  $\MoIl[\big]$\MLParsedQuestion{16}{-4}{4}{DEG4}$\MIntvlSep 0\MoIr[\big]$.
%
\item The function $f$ is monotonically \MLQuestion{16}{increasing}{DEG5} on
  \MLQuestion{16}{wachsend}{DEG5}.
%
\item The function $f$ is monotonically \MLQuestion{16}{increasing}{DEG6}
  on $\MoIl 3\MIntvlSep 4\MoIr$.
\end{itemize}
The maximum point of $f$ is at \MLParsedQuestion{8}{-4}{4}{DEG7}.
The minimum point of $f$ is at \MLParsedQuestion{8}{0}{4}{DEG8}.

\begin{MHint}{Solution}
 The monotony behaviour is determined by the derivative $f'$ of the function $f$. Since the graph of 
 the derivative $f'$ is given in the exercise, we only have to read off on which intervals the graph 
 lies above (or below) the $x$-axis: On the intervals $\MoIr -\MZahl{4}{5} \MIntvlSep -4\MoIr$, $\MoIl 0 \MIntvlSep 3\MoIr$,
 and $\MoIl 3 \MIntvlSep 4\MoIr$, we have $f'(x) > 0$, and hence, the function $f$ is monotonically increasing there.
 However, on the interval $\MoIl -4 \MIntvlSep 0\MoIr$, we have $f'(x) < 0$, and hence the function $f$ is 
 monotonically decreasing there.\newline
 At an extremal point $x_e$ (maximum or minimum point) of a function $f$ (which does not lie on the boundary of the 
 domain) the first derivative is zero: $f'(x_e) = 0$. Graphically, this means that the tangent line to the graph of 
 $f$ is a horizontal line. According to the graph, the zeros of $f'(x)$ are at $x_ 1 = -4$, $x_2 = 0$, and $x_3 = 3$.
 Since $f$ is monotonically increasing on $\MoIl -\MZahl{4}{5} \MIntvlSep -4\MoIr$ and monotonically decreasing on
 $\MoIl -4 \MIntvlSep 0\MoIr$, $x_1 = - 4$ is a maximum point. This tells us that the function 
 has a minimum point at $x_2 = 0$. (At $x_3 = 3$ the function has a saddle point.)
\end{MHint}
\end{MExercise}
\end{MExercises}

\MSubsection{Applications}
\MLabel{M07_Anwendungen}


\begin{MXContent}{Curve Analysis}{Curve Analysis}{STD}
\MLabel{VBKM07_Kurvendiskussion}
\MDeclareSiteUXID{VBKM07_Kurvendiskussion}

Curve analysis is an established part of the German mathematics syllabus that consists of
curve sketching for a function together with the gathering of certain standardized qualitative
and quantitative information about the function graph.

Let a differentiable function $f: \MoIl a\MIntvlSep b\MoIr \rightarrow \R$ with the mapping rule
$x \mapsto y = f(x)$ for $x \in \MoIl a\MIntvlSep b\MoIr$ be given. In this course, a complete 
curve analysis of $f$ includes the following information:

\begin{itemize}
\item maximum domain
\item $x$- and $y$-intercepts of the graph
\item symmetry of the graph
\item limiting behaviour/asymptotes
\item first derivatives
\item extremal values
\item monotony behaviour
\item inflexion points
\item bending behaviour (curvature)
\item sketch of the graph
\end{itemize}

Many of these points were already discussed in Module~\MRef{VBKM06}. Therefore,
in this section we shall only briefly repeat what is meant by the different steps 
of curve analysis. Subsequently, we will discuss one example of a curve analysis in detail.

The first part of the curve analysis involves the algebraic and geometric aspects of $f$:

\begin{description}
\item[Maximum Domain]
All real numbers $x$ for which $f(x)$ exists are determined. The set $D$
of these numbers is called the maximum domain.

\item[$x$- and $y$-Intercepts] ~\relax
\begin{itemize}
\item $x$-axis: All zeros of $f$ are determined.
\item $y$-axis: The function value $f(0)$ (if $0 \in D$) is calculated.
\end{itemize}

\item[Symmetry of the Graph]
The graph of the function is symmetrical with respect to the $y$-axis if $f(-x) = f(x)$ for all
$x \in D$. Then the function $f$ is called \textbf{even}. If $f(-x) = -f(x)$
for all $x \in D$, the graph is centrally symmetric with respect to the origin 
$\MPointTwo{0}{0}$ of the coordinate system. In this case, the function 
is called \textbf{odd}.


\item[Asymptotic Behaviour at the Domain Boundary]
The limits of the function $f$ at the boundaries of its domain are investigated.
\end{description} 

In the second part, the function is investigated analytically by means of conclusions 
from the first derivatives. Of course, the first and the second derivative have to be 
calculated first, provided they exist.
\begin{description}
\item[Derivatives]

Calculation of the first and the second derivative (if they exist).

\item[Extremal Values and Monotony]

The necessary condition for $x$ to be an extremum (if $x \in D$ is not a boundary point of $D$), is $f'(x) = 0$.\newline
Thus, we calculate the points $x_0$ at which the derivative $f'$ takes the value zero. If at these points also 
the second derivative exists, then we have:
\begin{itemize}
 \item ${f'}'(x_0) > 0$: $x_0$ is a minimum point of $f$.
 \item ${f'}'(x_0) < 0$: $x_0$ is a maximum point of $f$.
\end{itemize}
The function $f$ is monotonically increasing on that intervals of the domain 
on which we have $f'(x) \geq 0$. It is monotonically decreasing on that intervals
where $f'(x) \leq 0$.

\item[Inflexion Points and Curvature Properties]
The necessary condition for $x$ to be an inflexion point (if the second derivative ${f'}'$ exists), is ${f'}'(x) = 0$\newline
If ${f'}'(w_0) = 0$ and $f^{(3)}(w_0) \neq 0$, then $w_0$ is an inflexion point, i.e.\ the bending behaviour of $f$ changes at this 
point.\newline
The function $f$ is convex (left curved) on that intervals of the domain in which we have $f^{(2)}(x) \geq 0$. It is 
concave (right curved) on that intervals on which we have $f^{(2)}(x) \leq 0$.


\item[Sketch of the Graph] A sketch of the graph is drawn based on the information gained during 
the previous steps.
\end{description}

\MSubsubsection{Detailed Example}
We investigate a function $f$ defined by the mapping rule
$$
f(x) \;=\; \frac{4x}{x^2+2}\MDFPeriod
$$

\textbf{Maximum Domain}\\
The maximum domain of this function is $D_f=\R$ since the denominator $x^2+2\geq 2$, i.e. it is always 
non-zero, and hence no points have to be excluded.
\ \\ \ \\
\textbf{$x$- and $y$-Intercepts}\\
The zeros of the function are the zeros of the numerator. Hence, the graph of $f$ intersects the $x$-axis only at the origin 
$\MPointTwo{0}{0}$ since the numerator is only zero for $x=0$. This is also the only intersection point 
with the $y$-axis since there we have $f(0)=0$.
\ \\ \ \\
\textbf{Symmetry}\\
To investigate the symmetry we replace the argument $x$ by $(-x)$. We have
$$
f(-x) \;=\;\frac{4\cdot (-x)}{(-x)^2+2} \;=\; -\frac{4x}{x^2+2} \;=\; -f(x)
$$
for all $x\in\R$. Hence, the graph of the function $f$ is centrally symmetric with respect to the origin.
\ \\ \ \\
\textbf{Limiting Behaviour}\\
The function is defined on the entire set of real numbers $\R$, so only the limiting behaviour
for $x\rightarrow \infty$ and $x\rightarrow -\infty$ has to be investigated. Since $f(x)$ is a fraction 
of two polynomials and the denominator has a greater power than the numerator, the $x$-axis is 
a horizontal asymptote in both directions:
\[
\lim_{x \rightarrow \pm \infty} f(x) = 0 \MDFPeriod
\]
\ \\ \ \\
\textbf{Derivatives}\\
The first two derivatives of the function are calculated from the quotient rule. For the first derivative, we have:
$$
f'(x) \;=\; 4\cdot\frac{1\cdot (x^2+2)-x\cdot 2x}{(x^2+2)^2} \;=\; 4\cdot \frac{-x^2+2}{(x^2+2)^2} \MDFPeriod
$$
Taking the derivative again and simplifying the terms results in:
\begin{eqnarray*}
{f'}'(x) & = & 4 \cdot % 
\frac{-2x (x^2 + 2)^2 - (-x^2 + 2) \cdot 2 (x^2 + 2) \cdot 2 x}{(x^2 + 2)^4} \\
& = & 4 \cdot \frac{-2x (x^2 + 2) - (-x^2 + 2) \cdot 4 x}{(x^2 + 2)^3} \\
& = & 4 \cdot \frac{-2x^3 - 4x + 4x^3 - 8 x}{(x^2 + 2)^3} \\
& = & 4 \cdot \frac{2x^3 - 12x}{(x^2 + 2)^3} \\
& = & 8 \cdot \frac{x (x^2 - 6)}{(x^2 + 2)^3} \MDFPeriod
\end{eqnarray*}
\ \\
\textbf{Extremal Values}\\
The necessary condition for an extremum at $x$ is $f'(x)=0$, in this case $-x^2+2=0$.
Thus, we obtain $x_1=\sqrt2$ and $x_2=-\sqrt2$. In addition, we have to investigate the 
behaviour of the second derivative at these points:
$$
{f'}'(x_1) \;=\; 8\frac{\sqrt2\cdot(2-6)}{(2+2)^3}<0 \MDFPSpace , \MDFPaSpace
{f'}'(x_2) \;=\; 8\frac{(-\sqrt2)\cdot(2-6)}{(2+2)^3}>0 \MDFPeriod
$$
Hence, $x_1$ is a maximum point and $x_2$ is a minimum point of $f$. Inserting these values for $x$ into $f$ results
in the maximum point $\MPointTwoAS{\sqrt2}{\sqrt2}$ and the minimum point $\MPointTwoAS{-\sqrt2}{-\sqrt2}$
of $f$. 
\ \\ \ \\
\textbf{Monotony Behaviour}\\
Since $f$ is defined on the entire set of real numbers $\R$, the monotony behaviour can be derived from 
the position of the extremal points and their types:
$f$ is monotonically decreasing on $\MoIl[\left]-\infty\MIntvlSep -\sqrt2\MoIr[\right]$, monotonically increasing 
on $\MoIl[\left]-\sqrt2\MIntvlSep \sqrt2\MoIr[\right]$, and monotonically decreasing on
$\MoIl[\left]\sqrt2\MIntvlSep \infty\MoIr[\right]$. Monotony intervals are always given as open intervals.
\ \\ \ \\
\textbf{Inflexion Points}\\
From the necessary condition ${f'}'(x)=0$ for $x$ to be an inflexion point, we have the equation
$8x(x^2-6)=0$. Thus, $w_0=0$, $w_1=\sqrt6$, and $w_2=-\sqrt6$ are the only solutions. The polynomial
in the denominator of ${f'}'$ is always greater than zero. Since the polynomial in the numerator has only
single roots, the second derivative ${f'}'(x)$ changes its sign at all these points. Hence, these points are 
inflexion points of $f$. The coordinates of the inflexion points $\MPointTwo{0}{0}$, $\MPointTwoAS{\sqrt6}{\frac12\sqrt6}$, 
$\MPointTwoAS{-\sqrt6}{-\frac12\sqrt6}$ are determined by inserting the corresponding values for $x$ 
into $f$. 
\ \\ \ \\
\textbf{Bending Behaviour}\\
The twice-differentiable function $f$ is convex if the second derivative is greater or equal to zero. It is concave 
if the second derivative is less or equal to zero. 
Since the polynomial in the denominator of ${f'}'(x)$ is always positive, it is sufficient to examine
the sign of the polynomial $p(x)=8x(x-\sqrt6)(x+\sqrt6)$ in the numerator. For $0<x<\sqrt6$, it is negative 
($f$ is concave there). For $x>\sqrt6$ it is positive ($f$ is  convex there). Since $f$ is centrally 
symmetric, it follows that $f$ is convex on the intervals $\MoIl[\left]-\sqrt 6\MIntvlSep 0\MoIr[\right]$ 
and $\MoIl[\left]\sqrt6\MIntvlSep \infty\MoIr[\right]$ and concave on the intervals 
$\MoIl[\left]-\infty\MIntvlSep -\sqrt6\MoIr[\right]$ and $\MoIl[\left]0\MIntvlSep \sqrt6\MoIr[\right]$.
\ \\ \ \\
\textbf{Sketch of the Graph}\\
\MUGraphics{BildKurvendiskussion1.png}{width=0.5\linewidth}{The graph of the function $f$, sketched on 
the interval $[-8\MIntvlSep 8]$.}{width:600px}
\end{MXContent}



\begin{MExercises}
\MDeclareSiteUXID{VBKM07_Kurvendiskussion_Exercises}

With the following exercise the elements of the curve analysis method can be trained:

\MDirectRouletteExercises{curves_polynomials.rtex}{VBKM07_CURVES_POLYNOMIALS}


\begin{MExercise}
Carry out a complete curve analysis for the function $f$ with $f(x) = (2 x - x^2) \MEU^x$ and enter 
your results into the input fields.
\ \\ \ \\
Maximum domain: \MLIntervalQuestion{25}{(-infty,infty)}{5}{DXG18} \MInputHint{(as an interval \texttt{(a;b)})}.\\
\ \\
Set of intersection points with the $x$-axis
(zeros of $f(x)$): \MLParsedQuestion{5}{0,2}{5}{DXG1} \MInputHint{(as a set \texttt{$\lbrace$a;b;c$\rbrace$}, only $x$-components)}.\\
\ \\
The $y$-intercept is at \MEquationItem{$y$}{\MLParsedQuestion{5}{0}{5}{DXG2}}.\\
\ \\
Symmetry: The function is\\
\begin{MQuestionGroup}
\begin{tabular}{lll}
\MLCheckbox{0}{JCA1} & \ \ & axially symmetric with respect to the $y$-axis,\\
\MLCheckbox{0}{JCA2} & \ \ & centrally symmetric with respect to the origin.
\end{tabular}
\end{MQuestionGroup}
\MGroupButton{Check selection}
\ \\ \ \\
Limiting behaviour: For $x\rightarrow \infty$, the functions values 
$f(x)$ tend to \MLFunctionQuestion{20}{-infty}{4}{infty}{4}{DXG3},
and for $x\rightarrow-\infty$, they tend to \MLParsedQuestion{15}{0}{3}{DXG5}.\\
\ \\
Derivatives: We have $f'(x)$ = \MLFunctionQuestion{20}{-(x^2-2)*exp(x)}{4}{x}{4}{DXG6} and
${f'}'(x)$ = \MLFunctionQuestion{30}{-(x^2+2*x-2)*exp(x)}{5}{x}{5}{DXG7} .\\
\ \\
Monotony behaviour: The function is monotonically increasing on the interval 
\MLIntervalQuestion{26}{(-sqrt(2),sqrt(2))}{4}{DXG8} and monotonically decreasing otherwise.\\
\ \\
Extremal values: The point $x_1$ = \MLParsedQuestion{13}{-sqrt(2)}{3}{DXG9} is a minimum point and 
the point $x_2$ = \MLParsedQuestion{13}{sqrt(2)}{3}{DXG10} is a maximum point.\\
\ \\
Inflexion points: The set of inflexion points consists of
\MLParsedQuestion{30}{-1-sqrt(3),-1+sqrt(3)}{3}{DXG11}\\\MInputHint{(as a set, roots can be entered)}.\\
\ \\
Sketch the graph and compare your result to the sample solution.

\begin{MHint}{Solution}
\textbf{Maximum Domain}\\
We have $f(x) = - x (x - 2) \MEU^x$ and $\MEU^x > 0$ for all $x \in \R$; thus, $D_f = \R = \MoIl -\infty \MIntvlSep \infty\MoIr$
is the maximum domain.
\ \\ \ \\
\textbf{$x$- and $y$-Intercepts}\\
The intersection points with the $x$-axis (roots of the function) lie at $x_1=0$ and $x_2=2$, i.e. 
the coordinates of the points are $\MPointTwo{0}{0}$ and $\MPointTwo{2}{0}$.
The $y$-intercept is the point $\MPointTwo{0}{0}$.
\ \\ \ \\
\textbf{Symmetry}\\
The function $f$ is neither even nor odd, and hence the graph of $f$ is neither axially symmetric with respect to 
the $y$-axis nor centrally symmetric with respect to the origin.
\ \\ \ \\
\textbf{Limiting Behaviour}\\
Since the function is defined for all real numbers, only the asymptotes for $\pm \infty$ have to be investigated:
$$
\lim_{x \rightarrow \infty} - x (x - 2) \MEU^x \; =\; - \infty \;\text{and}\;
\lim_{x \rightarrow -\infty} - x (x - 2) \MEU^x \;=\;  0 \MDFPeriod
$$
Hence, $y = 0$ is an asymptote for $x \rightarrow -\infty$.
\ \\ \ \\
\textbf{Derivatives}\\
The first two derivatives of $f$ are
\begin{eqnarray*}
f'(x) & = & (2 - 2 x) \MEU^x + (2 x - x^2) \MEU^x %
 = (2 - x^2) \MEU^x = - (x^2 - 2) \MEU^x \MDFPSpace ,\\
{f'}'(x) & = & - 2 x \MEU^x + (2 - x^2) \MEU^x %
 = - (x^2 + 2 x - 2) \MEU^x \MDFPeriod
\end{eqnarray*}
\ \\
\textbf{Monotony Behaviour and Extremal Values}\\
The solutions of $f'(x) = 0$ are $x_1 = -\sqrt{2}$ and $x_2 = \sqrt{2}$. 
Furthermore, we have $x_1 < x_2$ and 
$$
f'(x)\; =\; - (x + \sqrt{2}) (x - \sqrt{2}) \MEU^x \MDFPeriod
$$
On $\MoIl[\left]-\infty\MIntvlSep  -\sqrt{2}\MoIr[\right]$ the first derivative $f'$ is negative, so $f$ is monotonically decreasing there.
On $\MoIl[\left]-\sqrt{2}\MIntvlSep  \sqrt{2}\MoIr[\right]$ the first derivative $f'$ is positive, hence $f$ is monotonically increasing there.
On $\MoIl[\left]\sqrt{2}\MIntvlSep  \infty\MoIr[\right]$ the first derivative $f'$ is negative, so
$f$ is monotonically decreasing there.
Thus, $x_1 = -\sqrt{2}$ is a minimum point, and $x_2 = \sqrt{2}$ is a maximum point.
\ \\ \ \\
\textbf{Inflexion Points}\\
The necessary condition ${f'}'(x) = 0$ for $x$ to be an inflexion point results in the 
quadratic equation $x^2 + 2 x - 2 = 0$. It has the solutions $w_1 = \frac{-2 - \sqrt{4 + 8}}{2} = -1 - \sqrt{3}$
and $w_2 = \frac{-2 + \sqrt{4 + 8}}{2} = -1 + \sqrt{3}$.
\ \\ \ \\
\textbf{Sketch of the Graph}\\
\MUGraphics{BildKurvendiskussion3.png}{width=0.47\linewidth}{Graph of the function $f$,
sketched on the interval $\MoIl -\MZahl{6}{2}\MIntvlSep 3\MoIr$.}{width:400px}
\end{MHint}

\end{MExercise}

\end{MExercises}



%\MSubsubsection{Optimierungsaufgaben}
\begin{MXContent}{Optimisation Problems}{Optimisation Problems}{STD}
\MDeclareSiteUXID{VBKM07_Optimierungsaufgaben}


In many applications in engineering and business, solutions to problems can be found which are not unique. 
They often depend on variable conditions. To find an ideal solution, additional properties (constraints)
are defined that are to be satisfied by the solution. This very often results in so-called 
\textbf{optimisation problems}, in which one solution has to be selected from a family of solutions such that it 
best satisfies a previously specified property.

As an example, we consider the problem of constructing a cylindrical can. This can must satisfy the 
additional condition of having a capacity (volume) $V$ of one litre (a.k.a. one cubic decimetre, $1 \MEinheit{dm}^3$).
Thus, if $V$ is specified in $\MEinheit[]{dm}^3$ and $r$ is the radius and $h$ the height of the can, 
each measured in decimetre ($\MEinheit[]{dm}$), then the volume is $V = \pi r^2 \cdot h = 1$.
%(Auf die physikalischen Einheiten wurde der mathematischen Einfachheit halber verzichtet -- in der Praxis ist es allerdings unumgänglich, mit
%Einheiten zu rechnen.)
The can with the least surface area $O = 2 \cdot \pi r^2 + 2 \pi r h$ is required in order to save material.
Here, the surface area $O$, measured in square decimetres ($\MEinheit[]{dm}^2$), is a function of the 
radius $r$ and the height $h$ of the can. 

In mathematical terms, our question results in the problem of finding a minimum 
of the surface function $O$, where the minimum has to be found for values of $r$ and $h$ that also satisfy the 
additional condition for the volume: $V = \pi r^2 \cdot h = 1$. In the context of finding extrema, 
such an additional condition is also called a \textbf{constraint}.

\begin{MXInfo}{Optimisation Problem}
In an \MEntry{optimisation problem}{optimisation problem}, we search for an 
extremum $x_{\text{ext}}$ of a function $f$ satisfying a given equation 
$g(x_{\text{ext}}) = b$.

If we search for a minimum point, this problem is called a \MEntry{minimisation problem}{minimisation problem}. 
If we search for a maximum point, this problem is called a \MEntry{maximisation problem}{maximisation problem}.  

The function $f$ is called the \MEntry{target function}{target function}, and the equation $g(x) = b$
is called the \MEntry{constraint}{constraint} of the optimisation problem.
\end{MXInfo}
\end{MXContent}

\begin{MXContent}{Example}{Example}{STD}
\MDeclareSiteUXID{VBKM07_WeitereBeispiele}

Let us consider the example above in more detail. Obviously, the problem is to minimise the 
surface area of a cylindrical can with a given volume (base multiplied by height):
%
\begin{eqnarray*}
V = \pi r^2 h = 1 \MDFPSpace,
\end{eqnarray*}
%
where $r$ is the radius of the base and $h$ is the height of the can. The surface area consists of the 
lid and the base (both with an area of $\pi r^2$) and the lateral surface (with an area of $2 \pi r h$), 
which results in the equation $O = 2 \pi r^2 + 2 \pi r h$ for the surface area of the can. The surface
area of the can is a function of the radius $r$ and the height $h$. In contrast, a fixed volume (constraint) 
is assigned to the volume. Thus, it can be written:
%
\begin{eqnarray*}
O\left(r, h\right) = 2 \pi r^2 + 2 \pi r h \MDFPeriod
\end{eqnarray*}
%
Due to the constraint $V = \pi r^2 h = 1$, this problem that initially involves two variables ($r$ and $h$)
can be reduced to a problem that only involves one variable. Solving the constraint for the height 
of the can results in:
%
\begin{eqnarray*}
\pi r^2 h &=& 1\\
\Leftrightarrow\quad h &=& \frac{1}{\pi r^2} \MDFPeriod
\end{eqnarray*}
%
Substituting this formula into the function $O(r,h)$ results in a function that only depends on one variable. This function 
is also called $O$ \textbf{for simplicity}:
%
\begin{eqnarray*}
O\left(r, h\right) = 2 \pi r^2 + 2 \pi r h = 2 \pi r^2 + 2 \pi r \frac{1}{\pi r^2} = 2 \left(\pi r^2 + \frac{1}{r}\right) = O\left(r\right) \MDFPeriod
\end{eqnarray*}
%
After this manipulation, the problem of finding the cans minimal surface area can be solved analogously 
to normal extremal value problems of functions. Thus, we take the first derivative of the function $O$ with respect to the 
variable $r$ and set this derivative equal to zero:
%
\begin{eqnarray*}
O'\left(r\right) = 2 \left(2\pi r - \frac{1}{r^2}\right) &=& 0\\
\Leftrightarrow\quad 2\pi r &=& \frac{1}{r^2}\\
\Leftrightarrow\quad 2\pi r^3 &=& 1\\
\Leftrightarrow\quad r^3 &=& \frac{1}{2\pi}\\
\Leftrightarrow\quad r &=& \sqrt[3]{\frac{1}{2\pi}} \MDFPeriod
\end{eqnarray*}
%
The last equivalent transformation used the fact that the radius $r$ cannot take negative values. 
Substituting this result into the second derivative of $O$ shows whether a minimum was actually found 
(${O'}'(r) = 4\pi + 4/r^3$):
\begin{eqnarray*}
{O'}'\left(\sqrt[3]{\frac{1}{2\pi}}\right) = 4\pi + \frac{4}{\left(\sqrt[3]{\frac{1}{2\pi}}\right)^3} = 12\pi > 0 \MDFPeriod
\end{eqnarray*}
%
For the radius $r = \sqrt[3]{\frac{1}{2\pi}}$, the surface area of the cylindrical can with the given volume $V = 1$ is a
minimum. The corresponding height of the can is $h = \frac{1}{\pi \left(\sqrt[3]{\frac{1}{2\pi}}\right)^2} = \sqrt[3]{\frac{4}{\pi}}$.
If a can of these dimensions is manufactured, the material usage for the given volume is minimised.
\end{MXContent}

\MSubsection{Final Test}
\MLabel{M07_Abschlusstest}

\begin{MTest}{Final Test Module \arabic{section}}
\MDeclareSiteUXID{VBKM07_Abschlusstest}
\begin{MExercise}
In a container at 9 a.m.\ a temperature of $-10^{\circ}{\MEinheit[]{C}}$ is measured.
At 3 p.m. the measured temperature is $-58^{\circ}{\MEinheit[]{C}}$. After a period of 14 hours, the temperature 
has fallen to $-140^{\circ}{\MEinheit[]{C}}$.

\begin{MExerciseItems}
\item What is the average rate of temperature change between the first and second measurements?

Answer: \MLParsedQuestion{5}{-8}{3}{DYG12}
%Ergebnis: $(-58 - (-10)) / (15 - 9) = -48 / 6 = -8$.
%
%\item Wodurch dr"uckt sich in der (mittleren) "Anderungsrate aus, dass die 
%Temperatur f"allt?
%
\item The `falling' property of the temperature shows in the fact
that the rate of change is {\MLQuestion{20}{negative}{DFA10}}.
\begin{MHint}{Hint}
Enter an adjective.
\end{MHint}
%{\MQuestion{20}{kleiner null bzw. negativ}} ist.
%
\item Calculate the average rate of temperature change for the whole measuring period. 

Answer: \MLParsedQuestion{5}{-6.5}{3}{DXG12}
%Ergebnis: $(-140 - (-10)) / (29 - 9) = -130 / 20 = -6.5$.
\end{MExerciseItems}
\end{MExercise}

\begin{MExercise}
A function $f: [-3\MIntvlSep 2] \rightarrow \R$, $x \mapsto f(x)$ has a first derivative $f'$ whose graph is 
shown in the figure below.

\begin{center}
\MTikzAuto{%
\begin{small}
\begin{tikzpicture}[line width=1.5pt,scale=0.6, %
declare function={
  fkt(\x) = (\x + abs(\x) + 2)/2;
}
] %[every node/.style={fill=white}] 
%Koordinatenachsen:
\draw[->] (-3.6, 0) -- (3, 0) node[below left]{$x$}; %x-Achse
\draw[->] (0, -0.6) -- (0, 3) node[below left]{$y$}; %y-Achse
%Achsenbeschriftung:
\foreach \x in {-3, -2, -1, 1, 2} \draw (\x, 0) -- ++(0, -0.1) %
 node[below] {$\x$}; 
\foreach \y in {1, 2} \draw (0, \y) -- ++(0.2, 0) %
 node[right] {$\y$};
%\node[below left] at (0, 0) {$0$};
%Funktion:
\draw[domain=-3:2,samples=120,color=\jccolorfkt] %
 plot (\x, {fkt(\x)});
\end{tikzpicture}
\end{small}
}
\end{center}


The function values of $f$ between $-3$ and $0$ \\
\begin{tabular}{lll}
\MLCheckbox{0}{JCC1} & \ \ & are constant,\\
\MLCheckbox{1}{JCC2} & \ \ & increase by $3$,\\
\MLCheckbox{0}{JCC3} & \ \ & decrease.
\end{tabular}
\ \\

At the point $x=0$ the function $f$ has \\
\begin{tabular}{lll}
\MLCheckbox{0}{JCC4} & \ \ & a jump,\\
\MLCheckbox{0}{JCC5} & \ \ & no derivative,\\
\MLCheckbox{1}{JCC6} & \ \ & a derivative of $1$.
\end{tabular}
\end{MExercise}

\begin{MExercise} %Rechenregeln
Calculate for the function
\begin{MExerciseItems}
 \item $f: \{ x \in \R \MCondSetSep x > 0 \} \rightarrow \R$ with $f(x) := \ln\left(x^3 + x^2\right)$ the value of the 
first derivative $f'$ at $x$:\newline
$f'(x) = $\MLSimplifyQuestion{30}{(3 x + 2)/(x^2 + x)}{1}{x}{20}{0}{SIMPLE13} \MDFPeriod
\item $g: \R \rightarrow \R$ with $g(x) := x \cdot \MEU^{-x}$ the  value of the second derivative ${g'}'$ at $x$:\newline
${g'}'(x) = $\MLSimplifyQuestion{30}{(x - 2) * exp(-x)}{10}{x}{4}{0}{SIMPLE14} \MDFPeriod
\end{MExerciseItems}
\MInputHint{Bracket the terms for clarification, e.g. enter $\frac{x+1}{(x+2)^2}$ as \texttt{(x+1)/((x+2)^2)}.} 
\end{MExercise}

\begin{MExercise} %Eigenschaften
Consider the function $f: \MoIl 0\MIntvlSep  \infty\MoIr \rightarrow \R$, $x \mapsto f(x)$ with 
$f'(x) = x \cdot \ln x$. On which regions is $f$ monotonically decreasing, and on which regions is $f$ concave? 
%
Specify the regions as open intervals $\MoIl c\MIntvlSep  d\MoIr$ that are as large as possible:
\begin{MExerciseItems}
\item $f$ is monotonically decreasing on \MLIntervalQuestion{12}{(0, 1)}{4}{INT2}.
\item $f$ is concave on \MLIntervalQuestion{12}{(0, 1/e)}{4}{INT3}.
\end{MExerciseItems}
\MInputHint{Open intervals can be entered in the form $(a;b)$, closed intervals are entered as $[a;b]$, $a$ and $b$ can
be arbitrary expressions. Do not use the notation  $]a;b[$ to enter open intervals. Enter \texttt{infty}
for $\infty$ in your answer.}
\end{MExercise}

%TODO Eingabe und Loesung fehlt
%\begin{MExercise} %Anwendungen
%Berechnen Sie alle lokalen und globalen Extremstellen und Wendestellen der 
%Funktion $f: \R \rightarrow \R$ mit $f(x) := (x^2 - 3 x) \cdot \MEU^{2 x}$ f"ur 
%$x \in \R$.
%\end{MExercise}


\end{MTest}


\clearpage
\MPrintIndex

\end{document}

%Dateiende.

