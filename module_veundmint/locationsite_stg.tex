\ifttm
\MSetSubject{\MINTPhysics}
\MSubject{Standortbeschreibung}
\MSection{Universit�t Stuttgart}

\begin{MSectionStart}
\MLabel{L_LOCATION}
\MDeclareSiteUXID{LOCATION_STG_START}
Auf den folgenden Seiten finden sich
\begin{itemize}
\item{die Beschreibung der Universit�t Stuttgart,}
\item{die Angebote des MINT-Kollegs Baden-W�rttemberg f�r Studieninteressierte vor Ort.}
\end{itemize}
\end{MSectionStart}

\MSubsection{Die Universit�t Stuttgart}

\begin{MXContent}{Uni Stuttgart}{Uni Stuttgart}{STD}
\MGlobalLocationTag
Die \MExtLink{http://www.uni-stuttgart.de}{Universit�t Stuttgart} liegt inmitten einer hochdynamischen Wirtschaftsregion mit weltweiter Ausstrahlung,
einer Region, die sich auf den Gebieten Mobilit�t, Informationstechnologie, Produktions- und Fertigungstechnik sowie Biowissenschaften profiliert hat.
\ \\ \ \\
Die Universit�t Stuttgart bietet folgende MINT-Bachelorstudieng�nge an:
\begin{itemize}
\item{Bauingenieurwesen}
\item{Chemie}
\item{Elektrotechnik und Informationstechnik}
\item{Erneuerbare Energien}
\item{Fahrzeug- und Motorentechnik}
\item{Geod�sie und Geoinformatik}
\item{Immobilientechnik und Immobilienwirtschaft}
\item{Informatik}
\item{Lebensmittelchemie}
\item{Luft- und Raumfahrttechnik}
\item{Maschinelle Sprachverarbeitung}
\item{Maschinenbau}
\item{Materialwissenschaft}
\item{Mathematik}
\item{Mechatronik}
\item{Physik}
\item{Simulation Technology}
\item{Softwaretechnik}
\item{Technische Biologie}
\item{Technikp�dagogik}
\item{Technische Kybernetik}
\item{Technologiemanagement}
\item{Umweltschutztechnik}
\item{Verfahrenstechnik}
\item{Verkehrsingenieurwesen}
\item{Wirtschaftsinformatik}
\end{itemize}
F�r diese Studieng�nge stehen f�r Studieninteressierte die Vorbereitungsangebote des \MExtLink{http://www.mint-kolleg.de}{MINT-Kollegs Baden-W�rttemberg} zur Verf�gung.
\end{MXContent}

\begin{MXContent}{MINT-Kolleg}{MINT-Kolleg}{STD}
Das \MExtLink{http://www.mint-kolleg.de}{MINT-Kolleg Baden-W�rttemberg} an der Universit�t Stuttgart bietet eine Reihe von Pr�senz- und Onlineangeboten,
zur Vorbereitung auf ein Studium in einem MINT-Studiungang an der Universit�t Stuttgart.

F�r Studieninteressierte gibt es neben einem Onlinetest und diesem Onlinekurs ein umfangreiches Pr�senzangebot vom MINT-Kolleg.

Ziel der studienvorbereitenden Kurse (Prop�deutika) am MINT-Kolleg ist es, Studieninteressierte und Studienanf�ngerinnen und Studienanf�nger
optimal auf ein Studium der Fachrichtungen Natur- und Ingenieurwissenschaften sowie Mathematik und Informatik vorzubereiten.
Eine Teilnahme ist m�glich, ohne an der Universit�t eingeschrieben zu sein. Die Kurse in kleinen Gruppen erm�glichen eine intensive Vertiefung
und �bung des Stoffes und obendrein die Gelegenheit erste Kontakte zu zuk�nftigen Kommilitonen zu kn�pfen und Uni-Luft zu schnuppern. 

Angepasst an Ihre individuellen Bed�rfnisse stehen drei Varianten zur Auswahl:
\begin{itemize}
\item{Einsemestriges Prop�deutikum - Start im April}
\item{Zweisemestriges Prop�deutikum - Start im November}
\item{Sommerprop�deutikum - Start im Juli, Zielgruppe: Abiturientinnen und Abiturienten des Jahres 2015}
\end{itemize}

Genauere Informationen finden sich \MExtLink{http://www.mint-kolleg.de/stuttgart/angebote/praesenzkurse/index.html}{hier}.

Vor Beginn des Wintersemesters wird ein Pr�senzvorkurs Mathematik angeboten, der unabh�ngig von der Teilnahme am Onlinekurs belegt werden kann.

\end{MXContent}


\fi
