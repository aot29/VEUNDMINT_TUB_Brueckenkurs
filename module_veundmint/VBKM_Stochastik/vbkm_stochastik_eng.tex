% MINTMOD Version P0.1.0, needs to be consistent with preprocesser object in tex2x and MPragma-Version at the end of this file

% Parameter aus Konvertierungsprozess (PDF und HTML-Erzeugung wenn vom Konverter aus gestartet) werden hier eingefuegt, Preambleincludes werden am Schluss angehaengt

\newif\ifttm                % gesetzt falls Uebersetzung in HTML stattfindet, sonst uebersetzung in PDF

% Wahl der Notationsvariante ist im PDF immer std, in der HTML-Uebersetzung wird vom Konverter die Auswahl modifiziert
\newif\ifvariantstd
\newif\ifvariantunotation
\variantstdtrue % Diese Zeile wird vom Konverter erkannt und ggf. modifiziert, daher nicht veraendern!


\def\MOutputDVI{1}
\def\MOutputPDF{2}
\def\MOutputHTML{3}
\newcounter{MOutput}

\ifttm
\usepackage{german}
\usepackage{array}
\usepackage{amsmath}
\usepackage{amssymb}
\usepackage{amsthm}
\else
\documentclass[ngerman,oneside]{scrbook}
\usepackage{etex}
\usepackage[latin1]{inputenc}
\usepackage{textcomp}
\usepackage[ngerman]{babel}
\usepackage[pdftex]{color}
\usepackage{xcolor}
\usepackage{graphicx}
\usepackage[all]{xy}
\usepackage{fancyhdr}
\usepackage{verbatim}
\usepackage{array}
\usepackage{float}
\usepackage{makeidx}
\usepackage{amsmath}
\usepackage{amstext}
\usepackage{amssymb}
\usepackage{amsthm}
\usepackage[ngerman]{varioref}
\usepackage{framed}
\usepackage{supertabular}
\usepackage{longtable}
\usepackage{maxpage}
\usepackage{tikz}
\usepackage{tikzscale}
\usepackage{tikz-3dplot}
\usepackage{bibgerm}
\usepackage{chemarrow}
\usepackage{polynom}
%\usepackage{draftwatermark}
\usepackage{pdflscape}
\usetikzlibrary{calc}
\usetikzlibrary{through}
\usetikzlibrary{shapes.geometric}
\usetikzlibrary{arrows}
\usetikzlibrary{intersections}
\usetikzlibrary{decorations.pathmorphing}
\usetikzlibrary{external}
\usetikzlibrary{patterns}
\usetikzlibrary{fadings}
\usepackage[colorlinks=true,linkcolor=blue]{hyperref} 
\usepackage[all]{hypcap}
%\usepackage[colorlinks=true,linkcolor=blue,bookmarksopen=true]{hyperref} 
\usepackage{ifpdf}

\usepackage{movie15}

\setcounter{tocdepth}{2} % In Inhaltsverzeichnis bis subsection
\setcounter{secnumdepth}{3} % Nummeriert bis subsubsection

\setlength{\LTpost}{0pt} % Fuer longtable
\setlength{\parindent}{0pt}
\setlength{\parskip}{8pt}
%\setlength{\parskip}{9pt plus 2pt minus 1pt}
\setlength{\abovecaptionskip}{-0.25ex}
\setlength{\belowcaptionskip}{-0.25ex}
\fi

\ifttm
\newcommand{\MDebugMessage}[1]{\special{html:<!-- debugprint;;}#1\special{html:; //-->}}
\else
%\newcommand{\MDebugMessage}[1]{\immediate\write\mintlog{#1}}
\newcommand{\MDebugMessage}[1]{}
\fi

\def\MPageHeaderDef{%
\pagestyle{fancy}%
\fancyhead[r]{(C) VE\&MINT-Projekt}
\fancyfoot[c]{\thepage\\--- CCL BY-SA 3.0 ---}
}


\ifttm%
\def\MRelax{}%
\else%
\def\MRelax{\relax}%
\fi%

%--------------------------- Uebernahme von speziellen XML-Versionen einiger LaTeX-Kommandos aus xmlbefehle.tex vom alten Kasseler Konverter ---------------

\newcommand{\MSep}{\left\|{\phantom{\frac1g}}\right.}

\newcommand{\ML}{L}

\newcommand{\MGGT}{\mathrm{ggT}}


\ifttm
% Verhindert dass die subsection-nummer doppelt in der toccaption auftaucht (sollte ggf. in toccaption gefixt werden so dass diese Ueberschreibung nicht notwendig ist)
\renewcommand{\thesubsection}{}
% Kommandos die ttm nicht kennt
\newcommand{\binomial}[2]{{#1 \choose #2}} %  Binomialkoeffizienten
\newcommand{\eur}{\begin{html}&euro;\end{html}}
\newcommand{\square}{\begin{html}&square;\end{html}}
\newcommand{\glqq}{"'}  \newcommand{\grqq}{"'}
\newcommand{\nRightarrow}{\special{html: &nrArr; }}
\newcommand{\nmid}{\special{html: &nmid; }}
\newcommand{\nparallel}{\begin{html}&nparallel;\end{html}}
\newcommand{\mapstoo}{\begin{html}<mo>&map;</mo>\end{html}}

% Schnitt und Vereinigungssymbole von Mengen haben zu kleine Abstaende; korrigiert:
\newcommand{\ccup}{\,\!\cup\,\!}
\newcommand{\ccap}{\,\!\cap\,\!}


% Umsetzung von mathbb im HTML
\renewcommand{\mathbb}[1]{\begin{html}<mo>&#1opf;</mo>\end{html}}
\fi

%---------------------- Strukturierung ----------------------------------------------------------------------------------------------------------------------

%---------------------- Kapselung des sectioning findet auf drei Ebenen statt:
% 1. Die LateX-Befehl
% 2. Die D-Versionen der Befehle, die nur die Grade der Abschnitte umhaengen falls notwendig
% 3. Die M-Versionen der Befehle, die zusaetzliche Formatierungen vornehmen, Skripten starten und das HTML codieren
% Im Modultext duerfen nur die M-Befehle verwendet werden!

\ifttm

  \def\Dsubsubsubsection#1{\subsubsubsection{#1}}
  \def\Dsubsubsection#1{\subsubsection{#1}\addtocounter{subsubsection}{1}} % ttm-Fehler korrigieren
  \def\Dsubsection#1{\subsection{#1}}
  \def\Dsection#1{\section{#1}} % Im HTML wird nur der Sektionstitel gegeben
  \def\Dchapter#1{\chapter{#1}}
  \def\Dsubsubsubsectionx#1{\subsubsubsection*{#1}}
  \def\Dsubsubsectionx#1{\subsubsection*{#1}}
  \def\Dsubsectionx#1{\subsection*{#1}}
  \def\Dsectionx#1{\section*{#1}}
  \def\Dchapterx#1{\chapter*{#1}}

\else

  \def\Dsubsubsubsection#1{\subsubsection{#1}}
  \def\Dsubsubsection#1{\subsection{#1}}
  \def\Dsubsection#1{\section{#1}}
  \def\Dsection#1{\chapter{#1}}
  \def\Dchapter#1{\title{#1}}
  \def\Dsubsubsubsectionx#1{\subsubsection*{#1}}
  \def\Dsubsubsectionx#1{\subsection*{#1}}
  \def\Dsubsectionx#1{\section*{#1}}
  \def\Dsectionx#1{\chapter*{#1}}

\fi

\newcommand{\MStdPoints}{4}
\newcommand{\MSetPoints}[1]{\renewcommand{\MStdPoints}{#1}}

% Befehl zum Abbruch der Erstellung (nur PDF)
\newcommand{\MAbort}[1]{\err{#1}}

% Prefix vor Dateieinbindungen, wird in der Baumdatei mit \renewcommand modifiziert
% und auf das Verzeichnisprefix gesetzt, in dem das gerade bearbeitete tex-Dokument liegt.
% Im HTML wird es auf das Verzeichnis der HTML-Datei gesetzt.
% Das Prefix muss mit / enden !
\newcommand{\MDPrefix}{.}

% MRegisterFile notiert eine Datei zur Einbindung in den HTML-Baum. Grafiken mit MGraphics werden automatisch eingebunden.
% Mit MLastFile erhaelt man eine Markierung fuer die zuletzt registrierte Datei.
% Diese Markierung wird im postprocessing durch den physikalischen Dateinamen ersetzt, aber nur den Namen (d.h. \MMaterial gehoert noch davor, vgl Definition von MGraphics)
% Parameter: Pfad/Name der Datei bzw. des Ordners, relativ zur Position des Modul-Tex-Dokuments.
\ifttm
\newcommand{\MRegisterFile}[1]{\addtocounter{MFileNumber}{1}\special{html:<!-- registerfile;;}#1\special{html:;;}\MDPrefix\special{html:;;}\arabic{MFileNumber}\special{html:; //-->}}
\else
\newcommand{\MRegisterFile}[1]{\addtocounter{MFileNumber}{1}}
\fi

% Testen welcher Uebersetzer hier am Werk ist

\ifttm
\setcounter{MOutput}{3}
\else
\ifx\pdfoutput\undefined
  \pdffalse
  \setcounter{MOutput}{\MOutputDVI}
  \message{Verarbeitung mit latex, Ausgabe in dvi.}
\else
  \setcounter{MOutput}{\MOutputPDF}
  \message{Verarbeitung mit pdflatex, Ausgabe in pdf.}
  \ifnum \pdfoutput=0
    \pdffalse
  \setcounter{MOutput}{\MOutputDVI}
  \message{Verarbeitung mit pdflatex, Ausgabe in dvi.}
  \else
    \ifnum\pdfoutput=1
    \pdftrue
  \setcounter{MOutput}{\MOutputPDF}
  \message{Verarbeitung mit pdflatex, Ausgabe in pdf.}
    \fi
  \fi
\fi
\fi

\ifnum\value{MOutput}=\MOutputPDF
\DeclareGraphicsExtensions{.pdf,.png,.jpg}
\fi

\ifnum\value{MOutput}=\MOutputDVI
\DeclareGraphicsExtensions{.eps,.png,.jpg}
\fi

\ifnum\value{MOutput}=\MOutputHTML
% Wird vom Konverter leider nicht erkannt und daher in split.pm hardcodiert!
\DeclareGraphicsExtensions{.png,.jpg,.gif}
\fi

% Umdefinition der hyperref-Nummerierung im PDF-Modus
\ifttm
\else
\renewcommand{\theHfigure}{\arabic{chapter}.\arabic{section}.\arabic{figure}}
\fi

% Makro, um in der HTML-Ausgabe die zuerst zu oeffnende Datei zu kennzeichnen
\ifttm
\newcommand{\MGlobalStart}{\special{html:<!-- mglobalstarttag -->}}
\else
\newcommand{\MGlobalStart}{}
\fi

% Makro, um bei scormlogin ein pullen des Benutzers bei Aufruf der Seite zu erzwingen (typischerweise auf der Einstiegsseite)
\ifttm
\newcommand{\MPullSite}{\special{html:<!-- pullsite //-->}}
\else
\newcommand{\MPullSite}{}
\fi

% Makro, um in der HTML-Ausgabe die Kapiteluebersicht zu kennzeichnen
\ifttm
\newcommand{\MGlobalChapterTag}{\special{html:<!-- mglobalchaptertag -->}}
\else
\newcommand{\MGlobalChapterTag}{}
\fi

% Makro, um in der HTML-Ausgabe die Konfiguration zu kennzeichnen
\ifttm
\newcommand{\MGlobalConfTag}{\special{html:<!-- mglobalconfigtag -->}}
\else
\newcommand{\MGlobalConfTag}{}
\fi

% Makro, um in der HTML-Ausgabe die Standortbeschreibung zu kennzeichnen
\ifttm
\newcommand{\MGlobalLocationTag}{\special{html:<!-- mgloballocationtag -->}}
\else
\newcommand{\MGlobalLocationTag}{}
\fi

% Makro, um in der HTML-Ausgabe die persoenlichen Daten zu kennzeichnen
\ifttm
\newcommand{\MGlobalDataTag}{\special{html:<!-- mglobaldatatag -->}}
\else
\newcommand{\MGlobalDataTag}{}
\fi

% Makro, um in der HTML-Ausgabe die Suchseite zu kennzeichnen
\ifttm
\newcommand{\MGlobalSearchTag}{\special{html:<!-- mglobalsearchtag -->}}
\else
\newcommand{\MGlobalSearchTag}{}
\fi

% Makro, um in der HTML-Ausgabe die Favoritenseite zu kennzeichnen
\ifttm
\newcommand{\MGlobalFavoTag}{\special{html:<!-- mglobalfavoritestag -->}}
\else
\newcommand{\MGlobalFavoTag}{}
\fi

% Makro, um in der HTML-Ausgabe die Eingangstestseite zu kennzeichnen
\ifttm
\newcommand{\MGlobalSTestTag}{\special{html:<!-- mglobalstesttag -->}}
\else
\newcommand{\MGlobalSTestTag}{}
\fi

% Makro, um in der PDF-Ausgabe ein Wasserzeichen zu definieren
\ifttm
\newcommand{\MWatermarkSettings}{\relax}
\else
\newcommand{\MWatermarkSettings}{%
% \SetWatermarkText{(c) MINT-Kolleg Baden-W�rttemberg 2014}
% \SetWatermarkLightness{0.85}
% \SetWatermarkScale{1.5}
}
\fi

\ifttm
\newcommand{\MBinom}[2]{\left({\begin{array}{c} #1 \\ #2 \end{array}}\right)}
\else
\newcommand{\MBinom}[2]{\binom{#1}{#2}}
\fi

\ifttm
\newcommand{\DeclareMathOperator}[2]{\def#1{\mathrm{#2}}}
\newcommand{\operatorname}[1]{\mathrm{#1}}
\fi

%----------------- Makros fuer die gemischte HTML/PDF-Konvertierung ------------------------------

\newcommand{\MTestName}{\relax} % wird durch Test-Umgebung gesetzt

% Fuer experimentelle Kursinhalte, die im Release-Umsetzungsvorgang eine Fehlermeldung
% produzieren sollen aber sonst normal umgesetzt werden
\newenvironment{MExperimental}{%
}{%
}

% Wird von ttm nicht richtig umgesetzt!!
\newenvironment{MExerciseItems}{%
\renewcommand\theenumi{\alph{enumi}}%
\begin{enumerate}%
}{%
\end{enumerate}%
}


\definecolor{infoshadecolor}{rgb}{0.75,0.75,0.75}
\definecolor{exmpshadecolor}{rgb}{0.875,0.875,0.875}
\definecolor{expeshadecolor}{rgb}{0.95,0.95,0.95}
\definecolor{framecolor}{rgb}{0.2,0.2,0.2}

% Bei PDF-Uebersetzung wird hinter den Start jeder Satz/Info-aehnlichen Umgebung eine leere mbox gesetzt, damit
% fuehrende Listen oder enums nicht den Zeilenumbruch kaputtmachen
%\ifttm
\def\MTB{}
%\else
%\def\MTB{\mbox{}}
%\fi


\ifttm
\newcommand{\MRelates}{\special{html:<mi>&wedgeq;</mi>}}
\else
\def\MRelates{\stackrel{\scriptscriptstyle\wedge}{=}}
\fi

\def\MInch{\text{''}}
\def\Mdd{\textit{''}}

\ifttm
\def\MNL{ \newline }
\newenvironment{MArray}[1]{\begin{array}{#1}}{\end{array}}
\else
\def\MNL{ \\ }
\newenvironment{MArray}[1]{\begin{array}{#1}}{\end{array}}
\fi

\newcommand{\MBox}[1]{$\mathrm{#1}$}
\newcommand{\MMBox}[1]{\mathrm{#1}}


\ifttm%
\newcommand{\Mtfrac}[2]{{\textstyle \frac{#1}{#2}}}
\newcommand{\Mdfrac}[2]{{\displaystyle \frac{#1}{#2}}}
\newcommand{\Mmeasuredangle}{\special{html:<mi>&angmsd;</mi>}}
\else%
\newcommand{\Mtfrac}[2]{\tfrac{#1}{#2}}
\newcommand{\Mdfrac}[2]{\dfrac{#1}{#2}}
\newcommand{\Mmeasuredangle}{\measuredangle}
\relax
\fi

% Matrizen und Vektoren

% Inhalt wird in der Form a & b \\ c & d erwartet
% Vorsicht: MVector = Komponentenspalte, MVec = Variablensymbol
\ifttm%
\newcommand{\MVector}[1]{\left({\begin{array}{c}#1\end{array}}\right)}
\else%
\newcommand{\MVector}[1]{\begin{pmatrix}#1\end{pmatrix}}
\fi



\newcommand{\MVec}[1]{\vec{#1}}
\newcommand{\MDVec}[1]{\overrightarrow{#1}}

%----------------- Umgebungen fuer Definitionen und Saetze ----------------------------------------

% Fuegt einen Tabellen-Zeilenumbruch ein im PDF, aber nicht im HTML
\newcommand{\TSkip}{\ifttm \else&\ \\\fi}

\newenvironment{infoshaded}{%
\def\FrameCommand{\fboxsep=\FrameSep \fcolorbox{framecolor}{infoshadecolor}}%
\MakeFramed {\advance\hsize-\width \FrameRestore}}%
{\endMakeFramed}

\newenvironment{expeshaded}{%
\def\FrameCommand{\fboxsep=\FrameSep \fcolorbox{framecolor}{expeshadecolor}}%
\MakeFramed {\advance\hsize-\width \FrameRestore}}%
{\endMakeFramed}

\newenvironment{exmpshaded}{%
\def\FrameCommand{\fboxsep=\FrameSep \fcolorbox{framecolor}{exmpshadecolor}}%
\MakeFramed {\advance\hsize-\width \FrameRestore}}%
{\endMakeFramed}

\def\STDCOLOR{black}

\ifttm%
\else%
\newtheoremstyle{MSatzStyle}
  {1cm}                   %Space above
  {1cm}                   %Space below
  {\normalfont\itshape}   %Body font
  {}                      %Indent amount (empty = no indent,
                          %\parindent = para indent)
  {\normalfont\bfseries}  %Thm head font
  {}                      %Punctuation after thm head
  {\newline}              %Space after thm head: " " = normal interword
                          %space; \newline = linebreak
  {\thmname{#1}\thmnumber{ #2}\thmnote{ (#3)}}
                          %Thm head spec (can be left empty, meaning
                          %`normal')
                          %
\newtheoremstyle{MDefStyle}
  {1cm}                   %Space above
  {1cm}                   %Space below
  {\normalfont}           %Body font
  {}                      %Indent amount (empty = no indent,
                          %\parindent = para indent)
  {\normalfont\bfseries}  %Thm head font
  {}                      %Punctuation after thm head
  {\newline}              %Space after thm head: " " = normal interword
                          %space; \newline = linebreak
  {\thmname{#1}\thmnumber{ #2}\thmnote{ (#3)}}
                          %Thm head spec (can be left empty, meaning
                          %`normal')
\fi%

\newcommand{\MInfoText}{Info}

\newcounter{MHintCounter}
\newcounter{MCodeEditCounter}

\newcounter{MLastIndex}  % Enthaelt die dritte Stelle (Indexnummer) des letzten angelegten Objekts
\newcounter{MLastType}   % Enthaelt den Typ des letzten angelegten Objekts (mithilfe der unten definierten Konstanten). Die Entscheidung, wie der Typ dargstellt wird, wird in split.pm beim Postprocessing getroffen.
\newcounter{MLastTypeEq} % =1 falls das Label in einer Matheumgebung (equation, eqnarray usw.) steht, =2 falls das Label in einer table-Umgebung steht

% Da ttm keine Zahlmakros verarbeiten kann, werden diese Nummern in den Zuweisungen hardcodiert!
\def\MTypeSection{1}          %# Zaehler ist section
\def\MTypeSubsection{2}       %# Zaehler ist subsection
\def\MTypeSubsubsection{3}    %# Zaehler ist subsubsection
\def\MTypeInfo{4}             %# Eine Infobox, Separatzaehler fuer die Chemie (auch wenn es dort nicht nummeriert wird) ist MInfoCounter
\def\MTypeExercise{5}         %# Eine Aufgabe, Separatzaehler fuer die Chemie ist MExerciseCounter
\def\MTypeExample{6}          %# Eine Beispielbox, Separatzaehler fuer die Chemie ist MExampleCounter
\def\MTypeExperiment{7}       %# Eine Versuchsbox, Separatzaehler fuer die Chemie ist MExperimentCounter
\def\MTypeGraphics{8}         %# Eine Graphik, Separatzaehler fuer alle FB ist MGraphicsCounter
\def\MTypeTable{9}            %# Eine Tabellennummer, hat keinen Zaehler da durch table gezaehlt wird
\def\MTypeEquation{10}        %# Eine Gleichungsnummer, hat keinen Zaehler da durch equation/eqnarray gezaehlt wird
\def\MTypeTheorem{11}         % Ein theorem oder xtheorem, Separatzaehler fuer die Chemie ist MTheoremCounter
\def\MTypeVideo{12}           %# Ein Video,Separatzaehler fuer alle FB ist MVideoCounter
\def\MTypeEntry{13}           %# Ein Eintrag fuer die Stichwortliste, wird nicht gezaehlt sondern erhaelt im preparsing ein unique-label 

% Zaehler fuer das Labelsystem sind prefixcounter, jeder Zaehler wird VOR dem gezaehlten Objekt inkrementiert und zaehlt daher das aktuelle Objekt
\newcounter{MInfoCounter}
\newcounter{MExerciseCounter}
\newcounter{MExampleCounter}
\newcounter{MExperimentCounter}
\newcounter{MGraphicsCounter}
\newcounter{MTableCounter}
\newcounter{MEquationCounter}  % Nur im HTML, sonst durch "equation"-counter von latex realisiert
\newcounter{MTheoremCounter}
\newcounter{MObjectCounter}   % Gemeinsamer Zaehler fuer Objekte (ausser Grafiken/Tabellen) in Mathe/Info/Physik
\newcounter{MVideoCounter}
\newcounter{MEntryCounter}

\newcounter{MTestSite} % 1 = Subsubsection ist eine Pruefungsseite, 0 = ist eine normale Seite (inkl. Hilfeseite)

\def\MCell{$\phantom{a}$}

\newenvironment{MExportExercise}{\begin{MExercise}}{\end{MExercise}} % wird von mconvert abgefangen

\def\MGenerateExNumber{%
\ifnum\value{MSepNumbers}=0%
\arabic{section}.\arabic{subsection}.\arabic{MObjectCounter}\setcounter{MLastIndex}{\value{MObjectCounter}}%
\else%
\arabic{section}.\arabic{subsection}.\arabic{MExerciseCounter}\setcounter{MLastIndex}{\value{MExerciseCounter}}%
\fi%
}%

\def\MGenerateExmpNumber{%
\ifnum\value{MSepNumbers}=0%
\arabic{section}.\arabic{subsection}.\arabic{MObjectCounter}\setcounter{MLastIndex}{\value{MObjectCounter}}%
\else%
\arabic{section}.\arabic{subsection}.\arabic{MExerciseCounter}\setcounter{MLastIndex}{\value{MExampleCounter}}%
\fi%
}%

\def\MGenerateInfoNumber{%
\ifnum\value{MSepNumbers}=0%
\arabic{section}.\arabic{subsection}.\arabic{MObjectCounter}\setcounter{MLastIndex}{\value{MObjectCounter}}%
\else%
\arabic{section}.\arabic{subsection}.\arabic{MExerciseCounter}\setcounter{MLastIndex}{\value{MInfoCounter}}%
\fi%
}%

\def\MGenerateSiteNumber{%
\arabic{section}.\arabic{subsection}.\arabic{subsubsection}%
}%

% Funktionalitaet fuer Auswahlaufgaben

\newcounter{MExerciseCollectionCounter} % = 0 falls nicht in collection-Umgebung, ansonsten Schachtelungstiefe
\newcounter{MExerciseCollectionTextCounter} % wird von MExercise-Umgebung inkrementiert und von MExerciseCollection-Umgebung auf Null gesetzt

\ifttm
% MExerciseCollection gruppiert Aufgaben, die dynamisch aus der Datenbank gezogen werden und nicht direkt in der HTML-Seite stehen
% Parameter: #1 = ID der Collection, muss eindeutig fuer alle IN DER DB VORHANDENEN collections sein unabhaengig vom Kurs
%            #2 = Optionsargument (im Moment: 1 = Iterative Auswahl, 2 = Zufallsbasierte Auswahl)
\newenvironment{MExerciseCollection}[2]{%
\addtocounter{MExerciseCollectionCounter}{1}
\setcounter{MExerciseCollectionTextCounter}{0}
\special{html:<!-- mexercisecollectionstart;;}#1\special{html:;;}#2\special{html:;; //-->}%
}{%
\special{html:<!-- mexercisecollectionstop //-->}%
\addtocounter{MExerciseCollectionCounter}{-1}
}
\else
\newenvironment{MExerciseCollection}[2]{%
\addtocounter{MExerciseCollectionCounter}{1}
\setcounter{MExerciseCollectionTextCounter}{0}
}{%
\addtocounter{MExerciseCollectionCounter}{-1}
}
\fi

% Bei Uebersetzung nach PDF werden die theorem-Umgebungen verwendet, bei Uebersetzung in HTML ein manuelles Makro
\ifttm%

  \newenvironment{MHint}[1]{  \special{html:<button name="Name_MHint}\arabic{MHintCounter}\special{html:" class="hintbutton_closed" id="MHint}\arabic{MHintCounter}\special{html:_button" %
  type="button" onclick="toggle_hint('MHint}\arabic{MHintCounter}\special{html:');">}#1\special{html:</button>}
  \special{html:<div class="hint" style="display:none" id="MHint}\arabic{MHintCounter}\special{html:"> }}{\begin{html}</div>\end{html}\addtocounter{MHintCounter}{1}}

  \newenvironment{MCOSHZusatz}{  \special{html:<button name="Name_MHint}\arabic{MHintCounter}\special{html:" class="chintbutton_closed" id="MHint}\arabic{MHintCounter}\special{html:_button" %
  type="button" onclick="toggle_hint('MHint}\arabic{MHintCounter}\special{html:');">}Weiterf�hrende Inhalte\special{html:</button>}
  \special{html:<div class="hintc" style="display:none" id="MHint}\arabic{MHintCounter}\special{html:">
  <div class="coshwarn">Diese Inhalte gehen �ber das Kursniveau hinaus und werden in den Aufgaben und Tests nicht abgefragt.</div><br />}
  \addtocounter{MHintCounter}{1}}{\begin{html}</div>\end{html}}

  
  \newenvironment{MDefinition}{\begin{definition}\setcounter{MLastIndex}{\value{definition}}\ \\}{\end{definition}}

  
  \newenvironment{MExercise}{
  \renewcommand{\MStdPoints}{4}
  \addtocounter{MExerciseCounter}{1}
  \addtocounter{MObjectCounter}{1}
  \setcounter{MLastType}{5}

  \ifnum\value{MExerciseCollectionCounter}=0\else\addtocounter{MExerciseCollectionTextCounter}{1}\special{html:<!-- mexercisetextstart;;}\arabic{MExerciseCollectionTextCounter}\special{html:;; //-->}\fi
  \special{html:<div class="aufgabe" id="ADIV_}\MGenerateExNumber\special{html:">}%
  \textbf{Aufgabe \MGenerateExNumber
  } \ \\}{
  \special{html:</div><!-- mfeedbackbutton;Aufgabe;}\arabic{MTestSite}\special{html:;}\MGenerateExNumber\special{html:; //-->}
  \ifnum\value{MExerciseCollectionCounter}=0\else\special{html:<!-- mexercisetextstop //-->}\fi
  }

  % Stellt eine Kombination aus Aufgabe, Loesungstext und Eingabefeld bereit,
  % bei der Aufgabentext und Musterloesung sowie die zugehoerigen Feldelemente
  % extern bezogen und div-aktualisiert werden, das Eingabefeld aber immer das gleiche ist.
  \newenvironment{MFetchExercise}{
  \addtocounter{MExerciseCounter}{1}
  \addtocounter{MObjectCounter}{1}
  \setcounter{MLastType}{5}

  \special{html:<div class="aufgabe" id="ADIV_}\MGenerateExNumber\special{html:">}%
  \textbf{Aufgabe \MGenerateExNumber
  } \ \\%
  \special{html:</div><div class="exfetch_text" id="ADIVTEXT_}\MGenerateExNumber\special{html:">}%
  \special{html:</div><div class="exfetch_sol" id="ADIVSOL_}\MGenerateExNumber\special{html:">}%
  \special{html:</div><div class="exfetch_input" id="ADIVINPUT_}\MGenerateExNumber\special{html:">}%
  }{
  \special{html:</div>}
  }

  \newenvironment{MExample}{
  \addtocounter{MExampleCounter}{1}
  \addtocounter{MObjectCounter}{1}
  \setcounter{MLastType}{6}
  \begin{html}
  <div class="exmp">
  <div class="exmprahmen">
  \end{html}\textbf{Beispiel
  \ifnum\value{MSepNumbers}=0
  \arabic{section}.\arabic{subsection}.\arabic{MObjectCounter}\setcounter{MLastIndex}{\value{MObjectCounter}}
  \else
  \arabic{section}.\arabic{subsection}.\arabic{MExampleCounter}\setcounter{MLastIndex}{\value{MExampleCounter}}
  \fi
  } \ \\}{\begin{html}</div>
  </div>
  \end{html}
  \special{html:<!-- mfeedbackbutton;Beispiel;}\arabic{MTestSite}\special{html:;}\MGenerateExmpNumber\special{html:; //-->}
  }

  \newenvironment{MExperiment}{
  \addtocounter{MExperimentCounter}{1}
  \addtocounter{MObjectCounter}{1}
  \setcounter{MLastType}{7}
  \begin{html}
  <div class="expe">
  <div class="experahmen">
  \end{html}\textbf{Versuch
  \ifnum\value{MSepNumbers}=0
  \arabic{section}.\arabic{subsection}.\arabic{MObjectCounter}\setcounter{MLastIndex}{\value{MObjectCounter}}
  \else
%  \arabic{MExperimentCounter}\setcounter{MLastIndex}{\value{MExperimentCounter}}
  \arabic{section}.\arabic{subsection}.\arabic{MExperimentCounter}\setcounter{MLastIndex}{\value{MExperimentCounter}}
  \fi
  } \ \\}{\begin{html}</div>
  </div>
  \end{html}}

  \newenvironment{MChemInfo}{
  \setcounter{MLastType}{4}
  \begin{html}
  <div class="info">
  <div class="inforahmen">
  \end{html}}{\begin{html}</div>
  </div>
  \end{html}}

  \newenvironment{MXInfo}[1]{
  \addtocounter{MInfoCounter}{1}
  \addtocounter{MObjectCounter}{1}
  \setcounter{MLastType}{4}
  \begin{html}
  <div class="info">
  <div class="inforahmen">
  \end{html}\textbf{#1
  \ifnum\value{MInfoNumbers}=0
  \else
    \ifnum\value{MSepNumbers}=0
    \arabic{section}.\arabic{subsection}.\arabic{MObjectCounter}\setcounter{MLastIndex}{\value{MObjectCounter}}
    \else
    \arabic{MInfoCounter}\setcounter{MLastIndex}{\value{MInfoCounter}}
    \fi
  \fi
  } \ \\}{\begin{html}</div>
  </div>
  \end{html}
  \special{html:<!-- mfeedbackbutton;Info;}\arabic{MTestSite}\special{html:;}\MGenerateInfoNumber\special{html:; //-->}
  }

  \newenvironment{MInfo}{\ifnum\value{MInfoNumbers}=0\begin{MChemInfo}\else\begin{MXInfo}{Info}\ \\ \fi}{\ifnum\value{MInfoNumbers}=0\end{MChemInfo}\else\end{MXInfo}\fi}

\else%

  \theoremstyle{MSatzStyle}
  \newtheorem{thm}{Satz}[section]
  \newtheorem{thmc}{Satz}
  \theoremstyle{MDefStyle}
  \newtheorem{defn}[thm]{Definition}
  \newtheorem{exmp}[thm]{Beispiel}
  \newtheorem{info}[thm]{\MInfoText}
  \theoremstyle{MDefStyle}
  \newtheorem{defnc}{Definition}
  \theoremstyle{MDefStyle}
  \newtheorem{exmpc}{Beispiel}[section]
  \theoremstyle{MDefStyle}
  \newtheorem{infoc}{\MInfoText}
  \theoremstyle{MDefStyle}
  \newtheorem{exrc}{Aufgabe}[section]
  \theoremstyle{MDefStyle}
  \newtheorem{verc}{Versuch}[section]
  
  \newenvironment{MFetchExercise}{}{} % kann im PDF nicht dargestellt werden
  
  \newenvironment{MExercise}{\begin{exrc}\renewcommand{\MStdPoints}{1}\MTB}{\end{exrc}}
  \newenvironment{MHint}[1]{\ \\ \underline{#1:}\\}{}
  \newenvironment{MCOSHZusatz}{\ \\ \underline{Weiterf�hrende Inhalte:}\\}{}
  \newenvironment{MDefinition}{\ifnum\value{MInfoNumbers}=0\begin{defnc}\else\begin{defn}\fi\MTB}{\ifnum\value{MInfoNumbers}=0\end{defnc}\else\end{defn}\fi}
%  \newenvironment{MExample}{\begin{exmp}}{\ \linebreak[1] \ \ \ \ $\phantom{a}$ \ \hfill $\blacklozenge$\end{exmp}}
  \newenvironment{MExample}{
    \ifnum\value{MInfoNumbers}=0\begin{exmpc}\else\begin{exmp}\fi
    \MTB
    \begin{exmpshaded}
    \ \newline
}{
    \end{exmpshaded}
    \ifnum\value{MInfoNumbers}=0\end{exmpc}\else\end{exmp}\fi
}
  \newenvironment{MChemInfo}{\begin{infoshaded}}{\end{infoshaded}}

  \newenvironment{MInfo}{\ifnum\value{MInfoNumbers}=0\begin{MChemInfo}\else\renewcommand{\MInfoText}{Info}\begin{info}\begin{infoshaded}
  \MTB
   \ \newline
    \fi
  }{\ifnum\value{MInfoNumbers}=0\end{MChemInfo}\else\end{infoshaded}\end{info}\fi}

  \newenvironment{MXInfo}[1]{
    \renewcommand{\MInfoText}{#1}
    \ifnum\value{MInfoNumbers}=0\begin{infoc}\else\begin{info}\fi%
    \MTB
    \begin{infoshaded}
    \ \newline
  }{\end{infoshaded}\ifnum\value{MInfoNumbers}=0\end{infoc}\else\end{info}\fi}

  \newenvironment{MExperiment}{
    \renewcommand{\MInfoText}{Versuch}
    \ifnum\value{MInfoNumbers}=0\begin{verc}\else\begin{info}\fi
    \MTB
    \begin{expeshaded}
    \ \newline
  }{
    \end{expeshaded}
    \ifnum\value{MInfoNumbers}=0\end{verc}\else\end{info}\fi
  }
\fi%

% MHint sollte nicht direkt fuer Loesungen benutzt werden wegen solutionselect
\newenvironment{MSolution}{\begin{MHint}{L"osung}}{\end{MHint}}

\newcounter{MCodeCounter}

\ifttm
\newenvironment{MCode}{\special{html:<!-- mcodestart -->}\ttfamily\color{blue}}{\special{html:<!-- mcodestop -->}}
\else
\newenvironment{MCode}{\begin{flushleft}\ttfamily\addtocounter{MCodeCounter}{1}}{\addtocounter{MCodeCounter}{-1}\end{flushleft}}
% Ohne color-Statement da inkompatible mit framed/shaded-Boxen aus dem framed-package
\fi

%----------------- Sonderdefinitionen fuer Symbole, die der Konverter nicht kann ----------------------------------------------

\ifttm%
\newcommand{\MUnderset}[2]{\underbrace{#2}_{#1}}%
\else%
\newcommand{\MUnderset}[2]{\underset{#1}{#2}}%
\fi%

\ifttm
\newcommand{\MThinspace}{\special{html:<mi>&#x2009;</mi>}}
\else
\newcommand{\MThinspace}{\,}
\fi

\ifttm
\newcommand{\glq}{\begin{html}&sbquo;\end{html}}
\newcommand{\grq}{\begin{html}&lsquo;\end{html}}
\newcommand{\glqq}{\begin{html}&bdquo;\end{html}}
\newcommand{\grqq}{\begin{html}&ldquo;\end{html}}
\fi

\ifttm
\newcommand{\MNdash}{\begin{html}&ndash;\end{html}}
\else
\newcommand{\MNdash}{--}
\fi

%\ifttm\def\MIU{\special{html:<mi>&#8520;</mi>}}\else\def\MIU{\mathrm{i}}\fi
\def\MIU{\mathrm{i}}
\def\MEU{e} % TU9-Onlinekurs: italic-e
%\def\MEU{\mathrm{e}} % Alte Onlinemodule: roman-e
\def\MD{d} % Kursives d in Integralen im TU9-Onlinekurs
%\def\MD{\mathrm{d}} % roman-d in den alten Onlinemodulen
\def\MDB{\|}

%zusaetzlicher Leerraum vor "\MD"
\ifttm%
\def\MDSpace{\special{html:<mi>&#x2009;</mi>}}
\else%
\def\MDSpace{\,}
\fi%
\newcommand{\MDwSp}{\MDSpace\MD}%

\ifttm
\def\Mdq{\dq}
\else
\def\Mdq{\dq}
\fi

\def\MSpan#1{\left<{#1}\right>}
\def\MSetminus{\setminus}
\def\MIM{I}

\ifttm
\newcommand{\ld}{\text{ld}}
\newcommand{\lg}{\text{lg}}
\else
\DeclareMathOperator{\ld}{ld}
%\newcommand{\lg}{\text{lg}} % in latex schon definiert
\fi


\def\Mmapsto{\ifttm\special{html:<mi>&mapsto;</mi>}\else\mapsto\fi} 
\def\Mvarphi{\ifttm\phi\else\varphi\fi}
\def\Mphi{\ifttm\varphi\else\phi\fi}
\ifttm%
\newcommand{\MEumu}{\special{html:<mi>&#x3BC;</mi>}}%
\else%
\newcommand{\MEumu}{\textrm{\textmu}}%
\fi
\def\Mvarepsilon{\ifttm\epsilon\else\varepsilon\fi}
\def\Mepsilon{\ifttm\varepsilon\else\epsilon\fi}
\def\Mvarkappa{\ifttm\kappa\else\varkappa\fi}
\def\Mkappa{\ifttm\varkappa\else\kappa\fi}
\def\Mcomplement{\ifttm\special{html:<mi>&comp;</mi>}\else\complement\fi} 
\def\MWW{\mathrm{WW}}
\def\Mmod{\ifttm\special{html:<mi>&nbsp;mod&nbsp;</mi>}\else\mod\fi} 

\ifttm%
\def\mod{\text{\;mod\;}}%
\def\MNEquiv{\special{html:<mi>&NotCongruent;</mi>}}% 
\def\MNSubseteq{\special{html:<mi>&NotSubsetEqual;</mi>}}%
\def\MEmptyset{\special{html:<mi>&empty;</mi>}}%
\def\MVDots{\special{html:<mi>&#x22EE;</mi>}}%
\def\MHDots{\special{html:<mi>&#x2026;</mi>}}%
\def\Mddag{\special{html:<mi>&#x1202;</mi>}}%
\def\sphericalangle{\special{html:<mi>&measuredangle;</mi>}}%
\def\nparallel{\special{html:<mi>&nparallel;</mi>}}%
\def\MProofEnd{\special{html:<mi>&#x25FB;</mi>}}%
\newenvironment{MProof}[1]{\underline{#1}:\MCR\MCR}{\hfill $\MProofEnd$}%
\else%
\def\MNEquiv{\not\equiv}%
\def\MNSubseteq{\not\subseteq}%
\def\MEmptyset{\emptyset}%
\def\MVDots{\vdots}%
\def\MHDots{\hdots}%
\def\Mddag{\ddag}%
\newenvironment{MProof}[1]{\begin{proof}[#1]}{\end{proof}}%
\fi%



% Spaces zum Auffuellen von Tabellenbreiten, die nur im HTML wirken
\ifttm%
\def\MTSP{\:}%
\else%
\def\MTSP{}%
\fi%

\DeclareMathOperator{\arsinh}{arsinh}
\DeclareMathOperator{\arcosh}{arcosh}
\DeclareMathOperator{\artanh}{artanh}
\DeclareMathOperator{\arcoth}{arcoth}


\newcommand{\MMathSet}[1]{\mathbb{#1}}
\def\N{\MMathSet{N}}
\def\Z{\MMathSet{Z}}
\def\Q{\MMathSet{Q}}
\def\R{\MMathSet{R}}
\def\C{\MMathSet{C}}

\newcounter{MForLoopCounter}
\newcommand{\MForLoop}[2]{\setcounter{MForLoopCounter}{#1}\ifnum\value{MForLoopCounter}=0{}\else{{#2}\addtocounter{MForLoopCounter}{-1}\MForLoop{\value{MForLoopCounter}}{#2}}\fi}

\newcounter{MSiteCounter}
\newcounter{MFieldCounter} % Kombination section.subsection.site.field ist eindeutig in allen Modulen, field alleine nicht

\newcounter{MiniMarkerCounter}

\ifttm
\newenvironment{MMiniPageP}[1]{\begin{minipage}{#1\linewidth}\special{html:<!-- minimarker;;}\arabic{MiniMarkerCounter}\special{html:;;#1; //-->}}{\end{minipage}\addtocounter{MiniMarkerCounter}{1}}
\else
\newenvironment{MMiniPageP}[1]{\begin{minipage}{#1\linewidth}}{\end{minipage}\addtocounter{MiniMarkerCounter}{1}}
\fi

\newcounter{AlignCounter}

\newcommand{\MStartJustify}{\ifttm\special{html:<!-- startalign;;}\arabic{AlignCounter}\special{html:;;justify; //-->}\fi}
\newcommand{\MStopJustify}{\ifttm\special{html:<!-- stopalign;;}\arabic{AlignCounter}\special{html:; //-->}\fi\addtocounter{AlignCounter}{1}}

\newenvironment{MJTabular}[1]{
\MStartJustify
\begin{tabular}{#1}
}{
\end{tabular}
\MStopJustify
}

\newcommand{\MImageLeft}[2]{
\begin{center}
\begin{tabular}{lc}
\MStartJustify
\begin{MMiniPageP}{0.65}
#1
\end{MMiniPageP}
\MStopJustify
&
\begin{MMiniPageP}{0.3}
#2  
\end{MMiniPageP}
\end{tabular}
\end{center}
}

\newcommand{\MImageHalf}[2]{
\begin{center}
\begin{tabular}{lc}
\MStartJustify
\begin{MMiniPageP}{0.45}
#1
\end{MMiniPageP}
\MStopJustify
&
\begin{MMiniPageP}{0.45}
#2  
\end{MMiniPageP}
\end{tabular}
\end{center}
}

\newcommand{\MBigImageLeft}[2]{
\begin{center}
\begin{tabular}{lc}
\MStartJustify
\begin{MMiniPageP}{0.25}
#1
\end{MMiniPageP}
\MStopJustify
&
\begin{MMiniPageP}{0.7}
#2  
\end{MMiniPageP}
\end{tabular}
\end{center}
}

\ifttm
\def\No{\mathbb{N}_0}
\else
\def\No{\ensuremath{\N_0}}
\fi
\def\MT{\textrm{\tiny T}}
\newcommand{\MTranspose}[1]{{#1}^{\MT}}
\ifttm
\newcommand{\MRe}{\mathsf{Re}}
\newcommand{\MIm}{\mathsf{Im}}
\else
\DeclareMathOperator{\MRe}{Re}
\DeclareMathOperator{\MIm}{Im}
\fi

\newcommand{\Mid}{\mathrm{id}}
\newcommand{\MFeinheit}{\mathrm{feinh}}

\ifttm
\newcommand{\Msubstack}[1]{\begin{array}{c}{#1}\end{array}}
\else
\newcommand{\Msubstack}[1]{\substack{#1}}
\fi

% Typen von Fragefeldern:
% 1 = Alphanumerisch, case-sensitive-Vergleich
% 2 = Ja/Nein-Checkbox, Loesung ist 0 oder 1   (OPTION = Image-id fuer Rueckmeldung)
% 3 = Reelle Zahlen Geparset
% 4 = Funktionen Geparset (mit Stuetzstellen zur ueberpruefung)

% Dieser Befehl erstellt ein interaktives Aufgabenfeld. Parameter:
% - #1 Laenge in Zeichen
% - #2 Loesungstext (alphanumerisch, case sensitive)
% - #3 AufgabenID (alphanumerisch, case sensitive)
% - #4 Typ (Kennnummer)
% - #5 String fuer Optionen (ggf. mit Semikolon getrennte Einzelstrings)
% - #6 Anzahl Punkte
% - #7 uxid (kann z.B. Loesungsstring sein)
% ACHTUNG: Die langen Zeilen bitte so lassen, Zeilenumbrueche im tex werden in div's umgesetzt
\newcommand{\MQuestionID}[7]{
\ifttm
\special{html:<!-- mdeclareuxid;;}UX#7\special{html:;;}\arabic{section}\special{html:;;}#3\special{html:;; //-->}%
\special{html:<!-- mdeclarepoints;;}\arabic{section}\special{html:;;}#3\special{html:;;}#6\special{html:;;}\arabic{MTestSite}\special{html:;;}\arabic{chapter}%
\special{html:;; //--><!-- onloadstart //-->CreateQuestionObj("}#7\special{html:",}\arabic{MFieldCounter}\special{html:,"}#2%
\special{html:","}#3\special{html:",}#4\special{html:,"}#5\special{html:",}#6\special{html:,}\arabic{MTestSite}\special{html:,}\arabic{section}%
\special{html:);<!-- onloadstop //-->}%
\special{html:<input mfieldtype="}#4\special{html:" name="Name_}#3\special{html:" id="}#3\special{html:" type="text" size="}#1\special{html:" maxlength="}#1%
\special{html:" }\ifnum\value{MGroupActive}=0\special{html:onfocus="handlerFocus(}\arabic{MFieldCounter}%
\special{html:);" onblur="handlerBlur(}\arabic{MFieldCounter}\special{html:);" onkeyup="handlerChange(}\arabic{MFieldCounter}\special{html:,0);" onpaste="handlerChange(}\arabic{MFieldCounter}\special{html:,0);" oninput="handlerChange(}\arabic{MFieldCounter}\special{html:,0);" onpropertychange="handlerChange(}\arabic{MFieldCounter}\special{html:,0);"/>}%
\special{html:<img src="images/questionmark.gif" width="20" height="20" border="0" align="absmiddle" id="}QM#3\special{html:"/>}
\else%
\special{html:onblur="handlerBlur(}\arabic{MFieldCounter}%
\special{html:);" onfocus="handlerFocus(}\arabic{MFieldCounter}\special{html:);" onkeyup="handlerChange(}\arabic{MFieldCounter}\special{html:,1);" onpaste="handlerChange(}\arabic{MFieldCounter}\special{html:,1);" oninput="handlerChange(}\arabic{MFieldCounter}\special{html:,1);" onpropertychange="handlerChange(}\arabic{MFieldCounter}\special{html:,1);"/>}%
\special{html:<img src="images/questionmark.gif" width="20" height="20" border="0" align="absmiddle" id="}QM#3\special{html:"/>}\fi%
\else%
\ifnum\value{QBoxFlag}=1\fbox{$\phantom{\MForLoop{#1}{b}}$}\else$\phantom{\MForLoop{#1}{b}}$\fi%
\fi%
}

% ACHTUNG: Die langen Zeilen bitte so lassen, Zeilenumbrueche im tex werden in div's umgesetzt
% QuestionCheckbox macht ausserhalb einer QuestionGroup keinen Sinn!
% #1 = solution (1 oder 0), ggf. mit ::smc abgetrennt auszuschliessende single-choice-boxen (UXIDs durch , getrennt), #2 = id, #3 = points, #4 = uxid
\newcommand{\MQuestionCheckbox}[4]{
\ifttm
\special{html:<!-- mdeclareuxid;;}UX#4\special{html:;;}\arabic{section}\special{html:;;}#2\special{html:;; //-->}%
\ifnum\value{MGroupActive}=0\MDebugMessage{ERROR: Checkbox Nr. \arabic{MFieldCounter}\ ist nicht in einer Kontrollgruppe, es wird niemals eine Loesung angezeigt!}\fi
\special{html: %
<!-- mdeclarepoints;;}\arabic{section}\special{html:;;}#2\special{html:;;}#3\special{html:;;}\arabic{MTestSite}\special{html:;;}\arabic{chapter}%
\special{html:;; //--><!-- onloadstart //-->CreateQuestionObj("}#4\special{html:",}\arabic{MFieldCounter}\special{html:,"}#1\special{html:","}#2\special{html:",2,"IMG}#2%
\special{html:",}#3\special{html:,}\arabic{MTestSite}\special{html:,}\arabic{section}\special{html:);<!-- onloadstop //-->}%
\special{html:<input mfieldtype="2" type="checkbox" name="Name_}#2\special{html:" id="}#2\special{html:" onchange="handlerChange(}\arabic{MFieldCounter}\special{html:,1);"/><img src="images/questionmark.gif" name="}Name_IMG#2%
\special{html:" width="20" height="20" border="0" align="absmiddle" id="}IMG#2\special{html:"/> }%
\else%
\ifnum\value{QBoxFlag}=1\fbox{$\phantom{X}$}\else$\phantom{X}$\fi%
\fi%
}

\def\MGenerateID{QFELD_\arabic{section}.\arabic{subsection}.\arabic{MSiteCounter}.QF\arabic{MFieldCounter}}

% #1 = 0/1 ggf. mit ::smc abgetrennt auszuschliessende single-choice-boxen (UXIDs durch , getrennt ohne UX), #2 = uxid ohne UX
\newcommand{\MCheckbox}[2]{
\MQuestionCheckbox{#1}{\MGenerateID}{\MStdPoints}{#2}
\addtocounter{MFieldCounter}{1}
}

% Erster Parameter: Zeichenlaenge der Eingabebox, zweiter Parameter: Loesungstext
\newcommand{\MQuestion}[2]{
\MQuestionID{#1}{#2}{\MGenerateID}{1}{0}{\MStdPoints}{#2}
\addtocounter{MFieldCounter}{1}
}

% Erster Parameter: Zeichenlaenge der Eingabebox, zweiter Parameter: Loesungstext
\newcommand{\MLQuestion}[3]{
\MQuestionID{#1}{#2}{\MGenerateID}{1}{0}{\MStdPoints}{#3}
\addtocounter{MFieldCounter}{1}
}

% Parameter: Laenge des Feldes, Loesung (wird auch geparsed), Stellen Genauigkeit hinter dem Komma, weitere Stellen werden mathematisch gerundet vor Vergleich
\newcommand{\MParsedQuestion}[3]{
\MQuestionID{#1}{#2}{\MGenerateID}{3}{#3}{\MStdPoints}{#2}
\addtocounter{MFieldCounter}{1}
}

% Parameter: Laenge des Feldes, Loesung (wird auch geparsed), Stellen Genauigkeit hinter dem Komma, weitere Stellen werden mathematisch gerundet vor Vergleich
\newcommand{\MLParsedQuestion}[4]{
\MQuestionID{#1}{#2}{\MGenerateID}{3}{#3}{\MStdPoints}{#4}
\addtocounter{MFieldCounter}{1}
}

% Parameter: Laenge des Feldes, Loesungsfunktion, Anzahl Stuetzstellen, Funktionsvariablen durch Kommata getrennt (nicht case-sensitive), Anzahl Nachkommastellen im Vergleich
\newcommand{\MFunctionQuestion}[5]{
\MQuestionID{#1}{#2}{\MGenerateID}{4}{#3;#4;#5;0}{\MStdPoints}{#2}
\addtocounter{MFieldCounter}{1}
}

% Parameter: Laenge des Feldes, Loesungsfunktion, Anzahl Stuetzstellen, Funktionsvariablen durch Kommata getrennt (nicht case-sensitive), Anzahl Nachkommastellen im Vergleich, UXID
\newcommand{\MLFunctionQuestion}[6]{
\MQuestionID{#1}{#2}{\MGenerateID}{4}{#3;#4;#5;0}{\MStdPoints}{#6}
\addtocounter{MFieldCounter}{1}
}

% Parameter: Laenge des Feldes, Loesungsintervall, Genauigkeit der Zahlenwertpruefung
\newcommand{\MIntervalQuestion}[3]{
\MQuestionID{#1}{#2}{\MGenerateID}{6}{#3}{\MStdPoints}{#2}
\addtocounter{MFieldCounter}{1}
}

% Parameter: Laenge des Feldes, Loesungsintervall, Genauigkeit der Zahlenwertpruefung, UXID
\newcommand{\MLIntervalQuestion}[4]{
\MQuestionID{#1}{#2}{\MGenerateID}{6}{#3}{\MStdPoints}{#4}
\addtocounter{MFieldCounter}{1}
}

% Parameter: Laenge des Feldes, Loesungsfunktion, Anzahl Stuetzstellen, Funktionsvariable (nicht case-sensitive), Anzahl Nachkommastellen im Vergleich, Vereinfachungsbedingung
% Vereinfachungsbedingung ist eine der Folgenden:
% 0 = Keine Vereinfachungsbedingung
% 1 = Keine Klammern (runde oder eckige) mehr im vereinfachten Ausdruck
% 2 = Faktordarstellung (Term hat Produkte als letzte Operation, Summen als vorgeschaltete Operation)
% 3 = Summendarstellung (Term hat Summen als letzte Operation, Produkte als vorgeschaltete Operation)
% Flag 512: Besondere Stuetzstellen (nur >1 und nur schwach rational), sonst symmetrisch um Nullpunkt und ganze Zahlen inkl. Null werden getroffen
\newcommand{\MSimplifyQuestion}[6]{
\MQuestionID{#1}{#2}{\MGenerateID}{4}{#3;#4;#5;#6}{\MStdPoints}{#2}
\addtocounter{MFieldCounter}{1}
}

\newcommand{\MLSimplifyQuestion}[7]{
\MQuestionID{#1}{#2}{\MGenerateID}{4}{#3;#4;#5;#6}{\MStdPoints}{#7}
\addtocounter{MFieldCounter}{1}
}

% Parameter: Laenge des Feldes, Loesung (optionaler Ausdruck), Anzahl Stuetzstellen, Funktionsvariable (nicht case-sensitive), Anzahl Nachkommastellen im Vergleich, Spezialtyp (string-id)
\newcommand{\MLSpecialQuestion}[7]{
\MQuestionID{#1}{#2}{\MGenerateID}{7}{#3;#4;#5;#6}{\MStdPoints}{#7}
\addtocounter{MFieldCounter}{1}
}

\newcounter{MGroupStart}
\newcounter{MGroupEnd}
\newcounter{MGroupActive}

\newenvironment{MQuestionGroup}{
\setcounter{MGroupStart}{\value{MFieldCounter}}
\setcounter{MGroupActive}{1}
}{
\setcounter{MGroupActive}{0}
\setcounter{MGroupEnd}{\value{MFieldCounter}}
\addtocounter{MGroupEnd}{-1}
}

\newcommand{\MGroupButton}[1]{
\ifttm
\special{html:<button name="Name_Group}\arabic{MGroupStart}\special{html:to}\arabic{MGroupEnd}\special{html:" id="Group}\arabic{MGroupStart}\special{html:to}\arabic{MGroupEnd}\special{html:" %
type="button" onclick="group_button(}\arabic{MGroupStart}\special{html:,}\arabic{MGroupEnd}\special{html:);">}#1\special{html:</button>}
\else
\phantom{#1}
\fi
}

%----------------- Makros fuer die modularisierte Darstellung ------------------------------------

\def\MyText#1{#1}

% is used internally by the conversion package, should not be used by original tex documents
\def\MOrgLabel#1{\relax}

\ifttm

% Ein MLabel wird im html codiert durch das tag <!-- mmlabel;;Labelbezeichner;;SubjectArea;;chapter;;section;;subsection;;Index;;Objekttyp; //-->
\def\MLabel#1{%
\ifnum\value{MLastType}=8%
\ifnum\value{MCaptionOn}=0%
\MDebugMessage{ERROR: Grafik \arabic{MGraphicsCounter} hat separates label: #1 (Grafiklabels sollten nur in der Caption stehen)}%
\fi
\fi
\ifnum\value{MLastType}=12%
\ifnum\value{MCaptionOn}=0%
\MDebugMessage{ERROR: Video \arabic{MVideoCounter} hat separates label: #1 (Videolabels sollten nur in der Caption stehen}%
\fi
\fi
\ifnum\value{MLastType}=10\setcounter{MLastIndex}{\value{equation}}\fi
\label{#1}\begin{html}<!-- mmlabel;;#1;;\end{html}\arabic{MSubjectArea}\special{html:;;}\arabic{chapter}\special{html:;;}\arabic{section}\special{html:;;}\arabic{subsection}\special{html:;;}\arabic{MLastIndex}\special{html:;;}\arabic{MLastType}\special{html:; //-->}}%

\else

% Sonderbehandlung im PDF fuer Abbildungen in separater aux-Datei, da MGraphics die figure-Umgebung nicht verwendet
\def\MLabel#1{%
\ifnum\value{MLastType}=8%
\ifnum\value{MCaptionOn}=0%
\MDebugMessage{ERROR: Grafik \arabic{MGraphicsCounter} hat separates label: #1 (Grafiklabels sollten nur in der Caption stehen}%
\fi
\fi
\ifnum\value{MLastType}=12%
\ifnum\value{MCaptionOn}=0%
\MDebugMessage{ERROR: Video \arabic{MVideoCounter} hat separates label: #1 (Videolabels sollten nur in der Caption stehen}%
\fi
\fi
\label{#1}%
}%

\fi

% Gibt Begriff des referenzierten Objekts mit aus, aber nur im HTML, daher nur in Ausnahmefaellen (z.B. Copyrightliste) sinnvoll
\def\MCRef#1{\ifttm\special{html:<!-- mmref;;}#1\special{html:;;1; //-->}\else\vref{#1}\fi}


\def\MRef#1{\ifttm\special{html:<!-- mmref;;}#1\special{html:;;0; //-->}\else\vref{#1}\fi}
\def\MERef#1{\ifttm\special{html:<!-- mmref;;}#1\special{html:;;0; //-->}\else\eqref{#1}\fi}
\def\MNRef#1{\ifttm\special{html:<!-- mmref;;}#1\special{html:;;0; //-->}\else\ref{#1}\fi}
\def\MSRef#1#2{\ifttm\special{html:<!-- msref;;}#1\special{html:;;}#2\special{html:; //-->}\else \if#2\empty \ref{#1} \else \hyperref[#1]{#2}\fi\fi} 

\def\MRefRange#1#2{\ifttm\MRef{#1} bis 
\MRef{#2}\else\vrefrange[\unskip]{#1}{#2}\fi}

\def\MRefTwo#1#2{\ifttm\MRef{#1} und \MRef{#2}\else%
\let\vRefTLRsav=\reftextlabelrange\let\vRefTPRsav=\reftextpagerange%
\def\reftextlabelrange##1##2{\ref{##1} und~\ref{##2}}%
\def\reftextpagerange##1##2{auf den Seiten~\pageref{#1} und~\pageref{#2}}%
\vrefrange[\unskip]{#1}{#2}%
\let\reftextlabelrange=\vRefTLRsav\let\reftextpagerange=\vRefTPRsav\fi}

% MSectionChapter definiert falls notwendig das Kapitel vor der section. Das ist notwendig, wenn nur ein Einzelmodul uebersetzt wird.
% MChaptersGiven ist ein Counter, der von mconvert.pl vordefiniert wird.
\ifttm
\newcommand{\MSectionChapter}{\ifnum\value{MChaptersGiven}=0{\Dchapter{Modul}}\else{}\fi}
\else
\newcommand{\MSectionChapter}{\ifnum\value{chapter}=0{\Dchapter{Modul}}\else{}\fi}
\fi


\def\MChapter#1{\ifnum\value{MSSEnd}>0{\MSubsectionEndMacros}\addtocounter{MSSEnd}{-1}\fi\Dchapter{#1}}
\def\MSubject#1{\MChapter{#1}} % Schluesselwort HELPSECTION ist reserviert fuer Hilfesektion

\newcommand{\MSectionID}{UNKNOWNID}

\ifttm
\newcommand{\MSetSectionID}[1]{\renewcommand{\MSectionID}{#1}}
\else
\newcommand{\MSetSectionID}[1]{\renewcommand{\MSectionID}{#1}\tikzsetexternalprefix{#1}}
\fi


\newcommand{\MSection}[1]{\MSetSectionID{MODULID}\ifnum\value{MSSEnd}>0{\MSubsectionEndMacros}\addtocounter{MSSEnd}{-1}\fi\MSectionChapter\Dsection{#1}\MSectionStartMacros{#1}\setcounter{MLastIndex}{-1}\setcounter{MLastType}{1}} % Sections werden ueber das section-Feld im mmlabel-Tag identifiziert, nicht ueber das Indexfeld

\def\MSubsection#1{\ifnum\value{MSSEnd}>0{\MSubsectionEndMacros}\addtocounter{MSSEnd}{-1}\fi\ifttm\else\clearpage\fi\Dsubsection{#1}\MSubsectionStartMacros\setcounter{MLastIndex}{-1}\setcounter{MLastType}{2}\addtocounter{MSSEnd}{1}}% Subsections werden ueber das subsection-Feld im mmlabel-Tag identifiziert, nicht ueber das Indexfeld
\def\MSubsectionx#1{\Dsubsectionx{#1}} % Nur zur Verwendung in MSectionStart gedacht
\def\MSubsubsection#1{\Dsubsubsection{#1}\setcounter{MLastIndex}{\value{subsubsection}}\setcounter{MLastType}{3}\ifttm\special{html:<!-- sectioninfo;;}\arabic{section}\special{html:;;}\arabic{subsection}\special{html:;;}\arabic{subsubsection}\special{html:;;1;;}\arabic{MTestSite}\special{html:; //-->}\fi}
\def\MSubsubsectionx#1{\Dsubsubsectionx{#1}\ifttm\special{html:<!-- sectioninfo;;}\arabic{section}\special{html:;;}\arabic{subsection}\special{html:;;}\arabic{subsubsection}\special{html:;;0;;}\arabic{MTestSite}\special{html:; //-->}\else\addcontentsline{toc}{subsection}{#1}\fi}

\ifttm
\def\MSubsubsubsectionx#1{\ \newline\textbf{#1}\special{html:<br />}}
\else
\def\MSubsubsubsectionx#1{\ \newline
\textbf{#1}\ \\
}
\fi


% Dieses Skript wird zu Beginn jedes Modulabschnitts (=Webseite) ausgefuehrt und initialisiert den Aufgabenfeldzaehler
\newcommand{\MPageScripts}{
\setcounter{MFieldCounter}{1}
\addtocounter{MSiteCounter}{1}
\setcounter{MHintCounter}{1}
\setcounter{MCodeEditCounter}{1}
\setcounter{MGroupActive}{0}
\DoQBoxes
% Feldvariablen werden im HTML-Header in conv.pl eingestellt
}

% Dieses Skript wird zum Ende jedes Modulabschnitts (=Webseite) ausgefuehrt
\ifttm
\newcommand{\MEndScripts}{\special{html:<br /><!-- mfeedbackbutton;Seite;}\arabic{MTestSite}\special{html:;}\MGenerateSiteNumber\special{html:; //-->}
}
\else
\newcommand{\MEndScripts}{\relax}
\fi


\newcounter{QBoxFlag}
\newcommand{\DoQBoxes}{\setcounter{QBoxFlag}{1}}
\newcommand{\NoQBoxes}{\setcounter{QBoxFlag}{0}}

\newcounter{MXCTest}
\newcounter{MXCounter}
\newcounter{MSCounter}



\ifttm

% Struktur des sectioninfo-Tags: <!-- sectioninfo;;section;;subsection;;subsubsection;;nr_ausgeben;;testpage; //-->

%Fuegt eine zusaetzliche html-Seite an hinter ALLEN bisherigen und zukuenftigen content-Seiten ausserhalb der vor-zurueck-Schleife (d.h. nur durch Button oder MIntLink erreichbar!)
% #1 = Titel des Modulabschnitts, #2 = Kurztitel fuer die Buttons, #3 = Buttonkennung (STD = default nehmen, NONE = Ohne Button in der Navigation)
\newenvironment{MSContent}[3]{\special{html:<div class="xcontent}\arabic{MSCounter}\special{html:"><!-- scontent;-;}\arabic{MSCounter};-;#1;-;#2;-;#3\special{html: //-->}\MPageScripts\MSubsubsectionx{#1}}{\MEndScripts\special{html:<!-- endscontent;;}\arabic{MSCounter}\special{html: //--></div>}\addtocounter{MSCounter}{1}}

% Fuegt eine zusaetzliche html-Seite ein hinter den bereits vorhandenen content-Seiten (oder als erste Seite) innerhalb der vor-zurueck-Schleife der Navigation
% #1 = Titel des Modulabschnitts, #2 = Kurztitel fuer die Buttons, #3 = Buttonkennung (STD = Defaultbutton, NONE = Ohne Button in der Navigation)
\newenvironment{MXContent}[3]{\special{html:<div class="xcontent}\arabic{MXCounter}\special{html:"><!-- xcontent;-;}\arabic{MXCounter};-;#1;-;#2;-;#3\special{html: //-->}\MPageScripts\MSubsubsection{#1}}{\MEndScripts\special{html:<!-- endxcontent;;}\arabic{MXCounter}\special{html: //--></div>}\addtocounter{MXCounter}{1}}

% Fuegt eine zusaetzliche html-Seite ein die keine subsubsection-Nummer bekommt, nur zur internen Verwendung in mintmod.tex gedacht!
% #1 = Titel des Modulabschnitts, #2 = Kurztitel fuer die Buttons, #3 = Buttonkennung (STD = Defaultbutton, NONE = Ohne Button in der Navigation)
% \newenvironment{MUContent}[3]{\special{html:<div class="xcontent}\arabic{MXCounter}\special{html:"><!-- xcontent;-;}\arabic{MXCounter};-;#1;-;#2;-;#3\special{html: //-->}\MPageScripts\MSubsubsectionx{#1}}{\MEndScripts\special{html:<!-- endxcontent;;}\arabic{MXCounter}\special{html: //--></div>}\addtocounter{MXCounter}{1}}

\newcommand{\MDeclareSiteUXID}[1]{\special{html:<!-- mdeclaresiteuxid;;}#1\special{html:;;}\arabic{chapter}\special{html:;;}\arabic{section}\special{html:;; //-->}}

\else

%\newcommand{\MSubsubsection}[1]{\refstepcounter{subsubsection} \addcontentsline{toc}{subsubsection}{\thesubsubsection. #1}}


% Fuegt eine zusaetzliche html-Seite an hinter den bereits vorhandenen content-Seiten
% #1 = Titel des Modulabschnitts, #2 = Kurztitel fuer die Buttons, #3 = Iconkennung (im PDF wirkungslos)
%\newenvironment{MUContent}[3]{\ifnum\value{MXCTest}>0{\MDebugMessage{ERROR: Geschachtelter SContent}}\fi\MPageScripts\MSubsubsectionx{#1}\addtocounter{MXCTest}{1}}{\addtocounter{MXCounter}{1}\addtocounter{MXCTest}{-1}}
\newenvironment{MXContent}[3]{\ifnum\value{MXCTest}>0{\MDebugMessage{ERROR: Geschachtelter SContent}}\fi\MPageScripts\MSubsubsection{#1}\addtocounter{MXCTest}{1}}{\addtocounter{MXCounter}{1}\addtocounter{MXCTest}{-1}}
\newenvironment{MSContent}[3]{\ifnum\value{MXCTest}>0{\MDebugMessage{ERROR: Geschachtelter XContent}}\fi\MPageScripts\MSubsubsectionx{#1}\addtocounter{MXCTest}{1}}{\addtocounter{MSCounter}{1}\addtocounter{MXCTest}{-1}}

\newcommand{\MDeclareSiteUXID}[1]{\relax}

\fi 

% GHEADER und GFOOTER werden von split.pm gefunden, aber nur, wenn nicht HELPSITE oder TESTSITE
\ifttm
\newenvironment{MSectionStart}{\special{html:<div class="xcontent0">}\MSubsubsectionx{Modul\"ubersicht}}{\setcounter{MSSEnd}{0}\special{html:</div>}}
% Darf nicht als XContent nummeriert werden, darf nicht als XContent gelabelt werden, wird aber in eine xcontent-div gesetzt fuer Python-parsing
\else
\newenvironment{MSectionStart}{\MSubsectionx{Modul\"ubersicht}}{\setcounter{MSSEnd}{0}}
\fi

\newenvironment{MIntro}{\begin{MXContent}{Einf\"uhrung}{Einf\"uhrung}{genetisch}}{\end{MXContent}}
\newenvironment{MContent}{\begin{MXContent}{Inhalt}{Inhalt}{beweis}}{\end{MXContent}}
\newenvironment{MExercises}{\ifttm\else\clearpage\fi\begin{MXContent}{Aufgaben}{Aufgaben}{aufgb}\special{html:<!-- declareexcsymb //-->}}{\end{MXContent}}

% #1 = Lesbare Testbezeichnung
\newenvironment{MTest}[1]{%
\renewcommand{\MTestName}{#1}
\ifttm\else\clearpage\fi%
\addtocounter{MTestSite}{1}%
\begin{MXContent}{#1}{#1}{STD} % {aufgb}%
\special{html:<!-- declaretestsymb //-->}
\begin{MQuestionGroup}%
\MInTestHeader
}%
{%
\end{MQuestionGroup}%
\ \\ \ \\%
\MInTestFooter
\end{MXContent}\addtocounter{MTestSite}{-1}%
}

\newenvironment{MExtra}{\ifttm\else\clearpage\fi\begin{MXContent}{Zus\"atzliche Inhalte}{Zusatz}{weiterfhrg}}{\end{MXContent}}

\makeindex

\ifttm
\def\MPrintIndex{
\ifnum\value{MSSEnd}>0{\MSubsectionEndMacros}\addtocounter{MSSEnd}{-1}\fi
\renewcommand{\indexname}{Stichwortverzeichnis}
\special{html:<p><!-- printindex //--></p>}
}
\else
\def\MPrintIndex{
\ifnum\value{MSSEnd}>0{\MSubsectionEndMacros}\addtocounter{MSSEnd}{-1}\fi
\renewcommand{\indexname}{Stichwortverzeichnis}
\addcontentsline{toc}{section}{Stichwortverzeichnis}
\printindex
}
\fi


% Konstanten fuer die Modulfaecher

\def\MINTMathematics{1}
\def\MINTInformatics{2}
\def\MINTChemistry{3}
\def\MINTPhysics{4}
\def\MINTEngineering{5}

\newcounter{MSubjectArea}
\newcounter{MInfoNumbers} % Gibt an, ob die Infoboxen nummeriert werden sollen
\newcounter{MSepNumbers} % Gibt an, ob Beispiele und Experimente separat nummeriert werden sollen
\newcommand{\MSetSubject}[1]{
 % ttm kapiert setcounter mit Parametern nicht, also per if abragen und einsetzen
\ifnum#1=1\setcounter{MSubjectArea}{1}\setcounter{MInfoNumbers}{1}\setcounter{MSepNumbers}{0}\fi
\ifnum#1=2\setcounter{MSubjectArea}{2}\setcounter{MInfoNumbers}{1}\setcounter{MSepNumbers}{0}\fi
\ifnum#1=3\setcounter{MSubjectArea}{3}\setcounter{MInfoNumbers}{0}\setcounter{MSepNumbers}{1}\fi
\ifnum#1=4\setcounter{MSubjectArea}{4}\setcounter{MInfoNumbers}{0}\setcounter{MSepNumbers}{0}\fi
\ifnum#1=5\setcounter{MSubjectArea}{5}\setcounter{MInfoNumbers}{1}\setcounter{MSepNumbers}{0}\fi
% Separate Nummerntechnik fuer unsere Chemiker: alles dreistellig
\ifnum#1=3
  \ifttm
  \renewcommand{\theequation}{\arabic{section}.\arabic{subsection}.\arabic{equation}}
  \renewcommand{\thetable}{\arabic{section}.\arabic{subsection}.\arabic{table}} 
  \renewcommand{\thefigure}{\arabic{section}.\arabic{subsection}.\arabic{figure}} 
  \else
  \renewcommand{\theequation}{\arabic{chapter}.\arabic{section}.\arabic{equation}}
  \renewcommand{\thetable}{\arabic{chapter}.\arabic{section}.\arabic{table}}
  \renewcommand{\thefigure}{\arabic{chapter}.\arabic{section}.\arabic{figure}}
  \fi
\else
  \ifttm
  \renewcommand{\theequation}{\arabic{section}.\arabic{subsection}.\arabic{equation}}
  \renewcommand{\thetable}{\arabic{table}}
  \renewcommand{\thefigure}{\arabic{figure}}
  \else
  \renewcommand{\theequation}{\arabic{chapter}.\arabic{section}.\arabic{equation}}
  \renewcommand{\thetable}{\arabic{table}}
  \renewcommand{\thefigure}{\arabic{figure}}
  \fi
\fi
}

% Fuer tikz Autogenerierung
\newcounter{MTIKZAutofilenumber}

% Spezielle Counter fuer die Bentz-Module
\newcounter{mycounter}
\newcounter{chemapplet}
\newcounter{physapplet}

\newcounter{MSSEnd} % Ist 1 falls ein MSubsection aktiv ist, der einen MSubsectionEndMacro-Aufruf verursacht
\newcounter{MFileNumber}
\def\MLastFile{\special{html:[[!-- mfileref;;}\arabic{MFileNumber}\special{html:; //--]]}}

% Vollstaendiger Pfad ist \MMaterial / \MLastFilePath / \MLastFileName    ==   \MMaterial / \MLastFile

% Wird nur bei kompletter Baum-Erstellung ausgefuehrt!
% #1 = Lesbare Modulbezeichnung
\newcommand{\MSectionStartMacros}[1]{
\setcounter{MTestSite}{0}
\setcounter{MCaptionOn}{0}
\setcounter{MLastTypeEq}{0}
\setcounter{MSSEnd}{0}
\setcounter{MFileNumber}{0} % Preinkrekement-Counter
\setcounter{MTIKZAutofilenumber}{0}
\setcounter{mycounter}{1}
\setcounter{physapplet}{1}
\setcounter{chemapplet}{0}
\ifttm
\special{html:<!-- mdeclaresection;;}\arabic{chapter}\special{html:;;}\arabic{section}\special{html:;;}#1\special{html:;; //-->}%
\else
\setcounter{thmc}{0}
\setcounter{exmpc}{0}
\setcounter{verc}{0}
\setcounter{infoc}{0}
\fi
\setcounter{MiniMarkerCounter}{1}
\setcounter{AlignCounter}{1}
\setcounter{MXCTest}{0}
\setcounter{MCodeCounter}{0}
\setcounter{MEntryCounter}{0}
}

% Wird immer ausgefuehrt
\newcommand{\MSubsectionStartMacros}{
\ifttm\else\MPageHeaderDef\fi
\MWatermarkSettings
\setcounter{MXCounter}{0}
\setcounter{MSCounter}{0}
\setcounter{MSiteCounter}{1}
\setcounter{MExerciseCollectionCounter}{0}
% Zaehler fuer das Labelsystem zuruecksetzen (prefix-Zaehler)
\setcounter{MInfoCounter}{0}
\setcounter{MExerciseCounter}{0}
\setcounter{MExampleCounter}{0}
\setcounter{MExperimentCounter}{0}
\setcounter{MGraphicsCounter}{0}
\setcounter{MTableCounter}{0}
\setcounter{MTheoremCounter}{0}
\setcounter{MObjectCounter}{0}
\setcounter{MEquationCounter}{0}
\setcounter{MVideoCounter}{0}
\setcounter{equation}{0}
\setcounter{figure}{0}
}

\newcommand{\MSubsectionEndMacros}{
% Bei Chemiemodulen das PSE einhaengen, es soll als SContent am Ende erscheinen
\special{html:<!-- subsectionend //-->}
\ifnum\value{MSubjectArea}=3{\MIncludePSE}\fi
}


\ifttm
%\newcommand{\MEmbed}[1]{\MRegisterFile{#1}\begin{html}<embed src="\end{html}\MMaterial/\MLastFile\begin{html}" width="192" height="189"></embed>\end{html}}
\newcommand{\MEmbed}[1]{\MRegisterFile{#1}\begin{html}<embed src="\end{html}\MMaterial/\MLastFile\begin{html}"></embed>\end{html}}
\fi

%----------------- Makros fuer die Textdarstellung -----------------------------------------------

\ifttm
% MUGraphics bindet eine Grafik ein:
% Parameter 1: Dateiname der Grafik, relativ zur Position des Modul-Tex-Dokuments
% Parameter 2: Skalierungsoptionen fuer PDF (fuer includegraphics)
% Parameter 3: Titel fuer die Grafik, wird unter die Grafik mit der Grafiknummer gesetzt und kann MLabel bzw. MCopyrightLabel enthalten
% Parameter 4: Skalierungsoptionen fuer HTML (css-styles)

% ERSATZ: <img alt="My Image" src="data:image/png;base64,iVBORwA<MoreBase64SringHere>" />


\newcommand{\MUGraphics}[4]{\MRegisterFile{#1}\begin{html}
<div class="imagecenter">
<center>
<div>
<img src="\end{html}\MMaterial/\MLastFile\begin{html}" style="#4" alt="\end{html}\MMaterial/\MLastFile\begin{html}"/>
</div>
<div class="bildtext">
\end{html}
\addtocounter{MGraphicsCounter}{1}
\setcounter{MLastIndex}{\value{MGraphicsCounter}}
\setcounter{MLastType}{8}
\addtocounter{MCaptionOn}{1}
\ifnum\value{MSepNumbers}=0
\textbf{Abbildung \arabic{MGraphicsCounter}:} #3
\else
\textbf{Abbildung \arabic{section}.\arabic{subsection}.\arabic{MGraphicsCounter}:} #3
\fi
\addtocounter{MCaptionOn}{-1}
\begin{html}
</div>
</center>
</div>
<br />
\end{html}%
\special{html:<!-- mfeedbackbutton;Abbildung;}\arabic{MGraphicsCounter}\special{html:;}\arabic{section}.\arabic{subsection}.\arabic{MGraphicsCounter}\special{html:; //-->}%
}

% MVideo bindet ein Video als Einzeldatei ein:
% Parameter 1: Dateiname des Videos, relativ zur Position des Modul-Tex-Dokuments, ohne die Endung ".mp4"
% Parameter 2: Titel fuer das Video (kann MLabel oder MCopyrightLabel enthalten), wird unter das Video mit der Videonummer gesetzt
\newcommand{\MVideo}[2]{\MRegisterFile{#1.mp4}\begin{html}
<div class="imagecenter">
<center>
<div>
<video width="95\%" controls="controls"><source src="\end{html}\MMaterial/#1.mp4\begin{html}" type="video/mp4">Ihr Browser kann keine MP4-Videos abspielen!</video>
</div>
<div class="bildtext">
\end{html}
\addtocounter{MVideoCounter}{1}
\setcounter{MLastIndex}{\value{MVideoCounter}}
\setcounter{MLastType}{12}
\addtocounter{MCaptionOn}{1}
\ifnum\value{MSepNumbers}=0
\textbf{Video \arabic{MVideoCounter}:} #2
\else
\textbf{Video \arabic{section}.\arabic{subsection}.\arabic{MVideoCounter}:} #2
\fi
\addtocounter{MCaptionOn}{-1}
\begin{html}
</div>
</center>
</div>
<br />
\end{html}}

\newcommand{\MDVideo}[2]{\MRegisterFile{#1.mp4}\MRegisterFile{#1.ogv}\begin{html}
<div class="imagecenter">
<center>
<div>
<video width="70\%" controls><source src="\end{html}\MMaterial/#1.mp4\begin{html}" type="video/mp4"><source src="\end{html}\MMaterial/#1.ogv\begin{html}" type="video/ogg">Ihr Browser kann keine MP4-Videos abspielen!</video>
</div>
<br />
#2
</center>
</div>
<br />
\end{html}
}

\newcommand{\MGraphics}[3]{\MUGraphics{#1}{#2}{#3}{}}

\else

\newcommand{\MVideo}[2]{%
% Kein Video im PDF darstellbar, trotzdem so tun als ob da eines waere
\begin{center}
(Video nicht darstellbar)
\end{center}
\addtocounter{MVideoCounter}{1}
\setcounter{MLastIndex}{\value{MVideoCounter}}
\setcounter{MLastType}{12}
\addtocounter{MCaptionOn}{1}
\ifnum\value{MSepNumbers}=0
\textbf{Video \arabic{MVideoCounter}:} #2
\else
\textbf{Video \arabic{section}.\arabic{subsection}.\arabic{MVideoCounter}:} #2
\fi
\addtocounter{MCaptionOn}{-1}
}


% MGraphics bindet eine Grafik ein:
% Parameter 1: Dateiname der Grafik, relativ zur Position des Modul-Tex-Dokuments
% Parameter 2: Skalierungsoptionen fuer PDF (fuer includegraphics)
% Parameter 3: Titel fuer die Grafik, wird unter die Grafik mit der Grafiknummer gesetzt
\newcommand{\MGraphics}[3]{%
\MRegisterFile{#1}%
\ %
\begin{figure}[H]%
\centering{%
\includegraphics[#2]{\MDPrefix/#1}%
\addtocounter{MCaptionOn}{1}%
\caption{#3}%
\addtocounter{MCaptionOn}{-1}%
}%
\end{figure}%
\addtocounter{MGraphicsCounter}{1}\setcounter{MLastIndex}{\value{MGraphicsCounter}}\setcounter{MLastType}{8}\ %
%\ \\Abbildung \ifnum\value{MSepNumbers}=0\else\arabic{chapter}.\arabic{section}.\fi\arabic{MGraphicsCounter}: #3%
}

\newcommand{\MUGraphics}[4]{\MGraphics{#1}{#2}{#3}}


\fi

\newcounter{MCaptionOn} % = 1 falls eine Grafikcaption aktiv ist, = 0 sonst


% MGraphicsSolo bindet eine Grafik pur ein ohne Titel
% Parameter 1: Dateiname der Grafik, relativ zur Position des Modul-Tex-Dokuments
% Parameter 2: Skalierungsoptionen (wirken nur im PDF)
\newcommand{\MGraphicsSolo}[2]{\MUGraphicsSolo{#1}{#2}{}}

% MUGraphicsSolo bindet eine Grafik pur ein ohne Titel, aber mit HTML-Skalierung
% Parameter 1: Dateiname der Grafik, relativ zur Position des Modul-Tex-Dokuments
% Parameter 2: Skalierungsoptionen (wirken nur im PDF)
% Parameter 3: Skalierungsoptionen (wirken nur im HTML), als style-format: "width=???, height=???"
\ifttm
\newcommand{\MUGraphicsSolo}[3]{\MRegisterFile{#1}\begin{html}
<img src="\end{html}\MMaterial/\MLastFile\begin{html}" style="\end{html}#3\begin{html}" alt="\end{html}\MMaterial/\MLastFile\begin{html}"/>
\end{html}%
\special{html:<!-- mfeedbackbutton;Abbildung;}#1\special{html:;}\MMaterial/\MLastFile\special{html:; //-->}%
}
\else
\newcommand{\MUGraphicsSolo}[3]{\MRegisterFile{#1}\includegraphics[#2]{\MDPrefix/#1}}
\fi

% Externer Link mit URL
% Erster Parameter: Vollstaendige(!) URL des Links
% Zweiter Parameter: Text fuer den Link
\newcommand{\MExtLink}[2]{\ifttm\special{html:<a target="_new" href="}#1\special{html:">}#2\special{html:</a>}\else\href{#1}{#2}\fi} % ohne MINTERLINK!


% Interner Link, die verlinkte Datei muss im gleichen Verzeichnis liegen wie die Modul-Texdatei
% Erster Parameter: Dateiname
% Zweiter Parameter: Text fuer den Link
\newcommand{\MIntLink}[2]{\ifttm\MRegisterFile{#1}\special{html:<a class="MINTERLINK" target="_new" href="}\MMaterial/\MLastFile\special{html:">}#2\special{html:</a>}\else{\href{#1}{#2}}\fi}


\ifttm
\def\MMaterial{:localmaterial:}
\else
\def\MMaterial{\MDPrefix}
\fi

\ifttm
\def\MNoFile#1{:directmaterial:#1}
\else
\def\MNoFile#1{#1}
\fi

\newcommand{\MChem}[1]{$\mathrm{#1}$}

\newcommand{\MApplet}[3]{
% Bindet ein Java-Applet ein, die Parameter sind:
% (wird nur im HTML, aber nicht im PDF erstellt)
% #1 Dateiname des Applets (muss mit ".class" enden)
% #2 = Breite in Pixeln
% #3 = Hoehe in Pixeln
\ifttm
\MRegisterFile{#1}
\begin{html}
<applet code="\end{html}\MMaterial/\MLastFile\begin{html}" width="#2" height="#3" alt="[Java-Applet kann nicht gestartet werden]"></applet>
\end{html}
\fi
}

\newcommand{\MScriptPage}[2]{
% Bindet eine JavaScript-Datei ein, die eine eigene Seite bekommt
% (wird nur im HTML, aber nicht im PDF erstellt)
% #1 Dateiname des Programms (sollte mit ".js" enden)
% #2 = Kurztitel der Seite
\ifttm
\begin{MSContent}{#2}{#2}{puzzle}
\MRegisterFile{#1}
\begin{html}
<script src="\MMaterial/\MLastFile" type="text/javascript"></script>
\end{html}
\end{MSContent}
\fi
}

\newcommand{\MIncludePSE}{
% Bindet bei Chemie-Modulen das PSE ein
% (wird nur im HTML, aber nicht im PDF erstellt)
\ifttm
\special{html:<!-- includepse //-->}
\begin{MSContent}{Periodensystem der Elemente}{PSE}{table}
\MRegisterFile{../files/pse.js}
\MRegisterFile{../files/radio.png}
\begin{html}
<script src="\MMaterial/../files/pse.js" type="text/javascript"></script>
<p id="divid"><br /><br />
<script language="javascript" type="text/javascript">
    startpse("divid","\MMaterial/../files"); 
</script>
</p>
<br />
<br />
<br />
<p>Die Farben der Elementsymbole geben an: <font style="color:Red">gasf&ouml;rmig </font> <font style="color:Blue">fl&uuml;ssig </font> fest</p>
<p>Die Elemente der Gruppe 1 A, 2 A, 3 A usw. geh&ouml;ren zu den Hauptgruppenelementen.</p>
<p>Die Elemente der Gruppe 1 B, 2 B, 3 B usw. geh&ouml;ren zu den Nebengruppenelementen.</p>
<p>() kennzeichnet die Masse des stabilsten Isotops</p>
\end{html}
\end{MSContent}
\fi
}

\newcommand{\MAppletArchive}[4]{
% Bindet ein Java-Applet ein, die Parameter sind:
% (wird nur im HTML, aber nicht im PDF erstellt)
% #1 Dateiname der Klasse mit Appletaufruf (muss mit ".class" enden)
% #2 Dateiname des Archivs (muss mit ".jar" enden)
% #3 = Breite in Pixeln
% #4 = Hoehe in Pixeln
\ifttm
\MRegisterFile{#2}
\begin{html}
<applet code="#1" archive="\end{html}\MMaterial/\MLastFile\begin{html}" codebase="." width="#3" height="#4" alt="[Java-Archiv kann nicht gestartet werden]"></applet>
\end{html}
\fi
}

% Bindet in der Haupttexdatei ein MINT-Modul ein. Parameter 1 ist das Verzeichnis (relativ zur Haupttexdatei), Parameter 2 ist der Dateinahme ohne Pfad.
\newcommand{\IncludeModule}[2]{
\renewcommand{\MDPrefix}{#1}
\input{#1/#2}
\ifnum\value{MSSEnd}>0{\MSubsectionEndMacros}\addtocounter{MSSEnd}{-1}\fi
}

% Der ttm-Konverter setzt keine Makros im \input um, also muss hier getrickst werden:
% Das MDPrefix muss in den einzelnen Modulen manuell eingesetzt werden
\newcommand{\MInputFile}[1]{
\ifttm
\input{#1}
\else
\input{#1}
\fi
}


\newcommand{\MCases}[1]{\left\lbrace{\begin{array}{rl} #1 \end{array}}\right.}

\ifttm
\newenvironment{MCaseEnv}{\left\lbrace\begin{array}{rl}}{\end{array}\right.}
\else
\newenvironment{MCaseEnv}{\left\lbrace\begin{array}{rl}}{\end{array}\right.}
\fi

\def\MSkip{\ifttm\MCR\fi}

\ifttm
\def\MCR{\special{html:<br />}}
\else
\def\MCR{\ \\}
\fi


% Pragmas - Sind Schluesselwoerter, die dem Preprocessing sowie dem Konverter uebergeben werden und bestimmte
%           Aktionen ausloesen. Im Output (PDF und HTML) tauchen sie nicht auf.
\newcommand{\MPragma}[1]{%
\ifttm%
\special{html:<!-- mpragma;-;}#1\special{html:;; -->}%
\else%
% MPragmas werden vom Preprozessor direkt im LaTeX gefunden
\fi%
}

% Ersatz der Befehle textsubscript und textsuperscript, die ttm nicht kennt
\ifttm%
\newcommand{\MTextsubscript}[1]{\special{html:<sub>}#1\special{html:</sub>}}%
\newcommand{\MTextsuperscript}[1]{\special{html:<sup>}#1\special{html:</sup>}}%
\else%
\newcommand{\MTextsubscript}[1]{\textsubscript{#1}}%
\newcommand{\MTextsuperscript}[1]{\textsuperscript{#1}}%
\fi

%------------------ Einbindung von dia-Diagrammen ----------------------------------------------
% Beim preprocessing wird aus jeder dia-Datei eine tex-Datei und eine pdf-Datei erzeugt,
% diese werden hier jeweils im PDF und HTML eingebunden
% Parameter: Dateiname der mit dia erstellten Datei (OHNE die Endung .dia)
\ifttm%
\newcommand{\MDia}[1]{%
\MGraphicsSolo{#1minthtml.png}{}%
}
\else%
\newcommand{\MDia}[1]{%
\MGraphicsSolo{#1mintpdf.png}{scale=0.1667}%
}
\fi%

% subsup funktioniert im Ausdruck $D={\R}^+_0$, also \R geklammert und sup zuerst
% \ifttm
% \def\MSubsup#1#2#3{\special{html:<msubsup>} #1 #2 #3\special{html:</msubsup>}}
% \else
% \def\MSubsup#1#2#3{{#1}^{#3}_{#2}}
% \fi

%\input{local.tex}

% \ifttm
% \else
% \newwrite\mintlog
% \immediate\openout\mintlog=mintlog.txt
% \fi

% ----------------------- tikz autogenerator -------------------------------------------------------------------

\newcommand{\Mtikzexternalize}{\tikzexternalize}% wird bei Konvertierung ueber mconvert ggf. ausgehebelt!

\ifttm
\else
\tikzset%
{
  % Defines a custom style which generates pdf and converts to (low and hi-res quality) png and svg, then deletes the pdf
  % Important: DO NOT directly convert from pdf to hires-png or from svg to png with GraphViz convert as it has some problems and memory leaks
  png export/.style=%
  {
    external/system call/.add={}{; 
      pdf2svg "\image.pdf" "\image.svg" ; 
      convert -density 112.5 -transparent white "\image.pdf" "\image.png"; 
      inkscape --export-png="\image.4x.png" --export-dpi=450 --export-background-opacity=0 --without-gui "\image.svg"; 
      rm "\image.pdf"; rm "\image.log"; rm "\image.dpth"; rm "\image.idx"
    },
    external/force remake,
  }
}
\tikzset{png export}
\tikzsetexternalprefix{}
% PNGs bei externer Erzeugung in "richtiger" Groesse einbinden
\pgfkeys{/pgf/images/include external/.code={\includegraphics[scale=0.64]{#1}}}
\fi

% Spezielle Umgebung fuer Autogenerierung, Bildernamen sind nur innerhalb eines Moduls (einer MSection) eindeutig)

\newcommand{\MTIKZautofilename}{tikzautofile}

\ifttm
% HTML-Version: Vom Autogenerator erzeugte png-Datei einbinden, tikz selbst nicht ausfuehren (sprich: #1 schlucken)
\newcommand{\MTikzAuto}[1]{%
\addtocounter{MTIKZAutofilenumber}{1}
\renewcommand{\MTIKZautofilename}{mtikzauto_\arabic{MTIKZAutofilenumber}}
\MUGraphicsSolo{\MSectionID\MTIKZautofilename.4x.png}{scale=1}{\special{html:[[!-- svgstyle;}\MSectionID\MTIKZautofilename\special{html: //--]]}} % Styleinfos werden aus original-png, nicht 4x-png geholt!
%\MRegisterFile{\MSectionID\MTIKZautofilename.png} % not used right now
%\MRegisterFile{\MSectionID\MTIKZautofilename.svg}
}
\else%
% PDF-Version: Falls Autogenerator aktiv wird Datei automatisch benannt und exportiert
\newcommand{\MTikzAuto}[1]{%
\addtocounter{MTIKZAutofilenumber}{1}%
\renewcommand{\MTIKZautofilename}{mtikzauto_\arabic{MTIKZAutofilenumber}}
\tikzsetnextfilename{\MTIKZautofilename}%
#1%
}
\fi

% In einer reinen LaTeX-Uebersetzung kapselt der Preambelinclude-Befehl nur input,
% in einer konvertergesteuerten PDF/HTML-Uebersetzung wird er dagegen entfernt und
% die Preambeln an mintmod angehaengt, die Ersetzung wird von mconvert.pl vorgenommen.

\newcommand{\MPreambleInclude}[1]{\input{#1}}

% Globale Watermarksettings (werden auch nochmal zu Beginn jedes subsection gesetzt,
% muessen hier aber auch global ausgefuehrt wegen Einfuehrungsseiten und Inhaltsverzeichnis

\MWatermarkSettings
% ---------------------------------- Parametrisierte Aufgaben ----------------------------------------

\ifttm
\newenvironment{MPExercise}{%
\begin{MExercise}%
}{%
\special{html:<button name="Name_MPEX}\arabic{MExerciseCounter}\special{html:" id="MPEX}\arabic{MExerciseCounter}%
\special{html:" type="button" onclick="reroll('}\arabic{MExerciseCounter}\special{html:');">Neue Aufgabe erzeugen</button>}%
\end{MExercise}%
}
\else
\newenvironment{MPExercise}{%
\begin{MExercise}%
}{%
\end{MExercise}%
}
\fi

% Parameter: Name, Min, Max, PDF-Standard. Name in Deklaration OHNE backslash, im Code MIT Backslash
\ifttm
\newcommand{\MGlobalInteger}[4]{\special{html:%
<!-- onloadstart //-->%
MVAR.push(createGlobalInteger("}#1\special{html:",}#2\special{html:,}#3\special{html:,}#4\special{html:)); %
<!-- onloadstop //-->%
<!-- viewmodelstart //-->%
ob}#1\special{html:: ko.observable(rerollMVar("}#1\special{html:")),%
<!-- viewmodelstop //-->%
}%
}%
\else%
\newcommand{\MGlobalInteger}[4]{\newcounter{mvc_#1}\setcounter{mvc_#1}{#4}}
\fi

% Parameter: Name, Min, Max, PDF-Standard. Name in Deklaration OHNE backslash, im Code MIT Backslash, Wert ist Wurzel von value
\ifttm
\newcommand{\MGlobalSqrt}[4]{\special{html:%
<!-- onloadstart //-->%
MVAR.push(createGlobalSqrt("}#1\special{html:",}#2\special{html:,}#3\special{html:,}#4\special{html:)); %
<!-- onloadstop //-->%
<!-- viewmodelstart //-->%
ob}#1\special{html:: ko.observable(rerollMVar("}#1\special{html:")),%
<!-- viewmodelstop //-->%
}%
}%
\else%
\newcommand{\MGlobalSqrt}[4]{\newcounter{mvc_#1}\setcounter{mvc_#1}{#4}}% Funktioniert nicht als Wurzel !!!
\fi

% Parameter: Name, Min, Max, PDF-Standard zaehler, PDF-Standard nenner. Name in Deklaration OHNE backslash, im Code MIT Backslash
\ifttm
\newcommand{\MGlobalFraction}[5]{\special{html:%
<!-- onloadstart //-->%
MVAR.push(createGlobalFraction("}#1\special{html:",}#2\special{html:,}#3\special{html:,}#4\special{html:,}#5\special{html:)); %
<!-- onloadstop //-->%
<!-- viewmodelstart //-->%
ob}#1\special{html:: ko.observable(rerollMVar("}#1\special{html:")),%
<!-- viewmodelstop //-->%
}%
}%
\else%
\newcommand{\MGlobalFraction}[5]{\newcounter{mvc_#1}\setcounter{mvc_#1}{#4}} % Funktioniert nicht als Bruch !!!
\fi

% MVar darf im HTML nur in MEvalMathDisplay-Umgebungen genutzt werden oder in Strings die an den Parser uebergeben werden
\ifttm%
\newcommand{\MVar}[1]{\special{html:[var_}#1\special{html:]}}%
\else%
\newcommand{\MVar}[1]{\arabic{mvc_#1}}%
\fi

\ifttm%
\newcommand{\MRerollButton}[2]{\special{html:<button type="button" onclick="rerollMVar('}#1\special{html:');">}#2\special{html:</button>}}%
\else%
\newcommand{\MRerollButton}[2]{\relax}% Keine sinnvolle Entsprechung im PDF
\fi

% MEvalMathDisplay fuer HTML wird in mconvert.pl im preprocessing realisiert
% PDF: eine equation*-Umgebung (ueber amsmath)
% HTML: Eine Mathjax-Tex-Umgebung, deren Auswertung mit knockout-obervablen gekoppelt ist
% PDF-Version hier nur fuer pdflatex-only-Uebersetzung gegeben

\ifttm\else\newenvironment{MEvalMathDisplay}{\begin{equation*}}{\end{equation*}}\fi

% ---------------------------------- Spezialbefehle fuer AD ------------------------------------------

%Abk�rzung f�r \longrightarrow:
\newcommand{\lto}{\ensuremath{\longrightarrow}}

%Makro f�r Funktionen:
\newcommand{\exfunction}[5]
{\begin{array}{rrcl}
 #1 \colon  & #2 &\lto & #3 \\[.05cm]  
  & #4 &\longmapsto  & #5 
\end{array}}

\newcommand{\function}[5]{%
#1:\;\left\lbrace{\begin{array}{rcl}
 #2 &\lto & #3 \\
 #4 &\longmapsto  & #5 \end{array}}\right.}


%Die Identit�t:
\DeclareMathOperator{\Id}{Id}

%Die Signumfunktion:
\DeclareMathOperator{\sgn}{sgn}

%Zwei Betonungskommandos (k�nnen angepasst werden):
\newcommand{\highlight}[1]{#1}
\newcommand{\modstextbf}[1]{#1}
\newcommand{\modsemph}[1]{#1}


% ---------------------------------- Spezialbefehle fuer JL ------------------------------------------


\def\jccolorfkt{green!50!black} %Farbe des Funktionsgraphen
\def\jccolorfktarea{green!25!white} %Farbe der Fl"ache unter dem Graphen
\def\jccolorfktareahell{green!12!white} %helle Einf"arbung der Fl"ache unter dem Graphen
\def\jccolorfktwert{green!50!black} %Farbe einzelner Punkte des Graphen

\newcommand{\MPfadBilder}{Bilder}

\ifttm%
\newcommand{\jMD}{\,\MD}%
\else%
\newcommand{\jMD}{\;\MD}%
\fi%

\def\jHTMLHinweisBedienung{\MInputHint{%
Mit Hilfe der Symbole am oberen Rand des Fensters
k"onnen Sie durch die einzelnen Abschnitte navigieren.}}

\def\jHTMLHinweisEingabeText{\MInputHint{%
Geben Sie jeweils ein Wort oder Zeichen als Antwort ein.}}

\def\jHTMLHinweisEingabeTerm{\MInputHint{%
Klammern Sie Ihre Terme, um eine eindeutige Eingabe zu erhalten. 
Beispiel: Der Term $\frac{3x+1}{x-2}$ soll in der Form
\texttt{(3*x+1)/((x+2)^2}$ eingegeben werden (wobei auch Leerzeichen 
eingegeben werden k"onnen, damit eine Formel besser lesbar ist).}}

\def\jHTMLHinweisEingabeIntervalle{\MInputHint{%
Intervalle werden links mit einer "offnenden Klammer und rechts mit einer 
schlie"senden Klammer angegeben. Eine runde Klammer wird verwendet, wenn der 
Rand nicht dazu geh"ort, eine eckige, wenn er dazu geh"ort. 
Als Trennzeichen wird ein Komma oder ein Semikolon akzeptiert.
Beispiele: $(a, b)$ offenes Intervall,
$[a; b)$ links abgeschlossenes, rechts offenes Intervall von $a$ bis $b$. 
Die Eingabe $]a;b[$ f"ur ein offenes Intervall wird nicht akzeptiert.
F"ur $\infty$ kann \texttt{infty} oder \texttt{unendlich} geschrieben werden.}}

\def\jHTMLHinweisEingabeFunktionen{\MInputHint{%
Schreiben Sie Malpunkte (geschrieben als \texttt{*}) aus und setzen Sie Klammern um Argumente f�r Funktionen.
Beispiele: Polynom: \texttt{3*x + 0.1}, Sinusfunktion: \texttt{sin(x)}, 
Verkettung von cos und Wurzel: \texttt{cos(sqrt(3*x))}.}}

\def\jHTMLHinweisEingabeFunktionenSinCos{\MInputHint{%
Die Sinusfunktion $\sin x$ wird in der Form \texttt{sin(x)} angegeben, %
$\cos\left(\sqrt{3 x}\right)$ durch \texttt{cos(sqrt(3*x))}.}}

\def\jHTMLHinweisEingabeFunktionenExp{\MInputHint{%
Die Exponentialfunktion $\MEU^{3x^4 + 5}$ wird als
\texttt{exp(3 * x^4 + 5)} angegeben, %
$\ln\left(\sqrt{x} + 3.2\right)$ durch \texttt{ln(sqrt(x) + 3.2)}.}}

% ---------------------------------- Spezialbefehle fuer Fachbereich Physik --------------------------

\newcommand{\E}{{e}}
\newcommand{\ME}[1]{\cdot 10^{#1}}
\newcommand{\MU}[1]{\;\mathrm{#1}}
\newcommand{\MPG}[3]{%
  \ifnum#2=0%
    #1\ \mathrm{#3}%
  \else%
    #1\cdot 10^{#2}\ \mathrm{#3}%
  \fi}%
%

\newcommand{\MMul}{\MExponentensymbXYZl} % Nur eine Abkuerzung


% ---------------------------------- Stichwortfunktionialitaet ---------------------------------------

% mpreindexentry wird durch Auswahlroutine in conv.pl durch mindexentry substitutiert
\ifttm%
\def\MIndex#1{\index{#1}\special{html:<!-- mpreindexentry;;}#1\special{html:;;}\arabic{MSubjectArea}\special{html:;;}%
\arabic{chapter}\special{html:;;}\arabic{section}\special{html:;;}\arabic{subsection}\special{html:;;}\arabic{MEntryCounter}\special{html:; //-->}%
\setcounter{MLastIndex}{\value{MEntryCounter}}%
\addtocounter{MEntryCounter}{1}%
}%
% Copyrightliste wird als tex-Datei im preprocessing von conv.pl erzeugt und unter converter/tex/entrycollection.tex abgelegt
% Der input-Befehl funktioniert nur, wenn die aufrufende tex-Datei auf der obersten Ebene liegt (d.h. selbst kein input/include ist, insbesondere keine Moduldatei)
\def\MEntryList{} % \input funktioniert nicht, weil ttm (und damit das \input) ausgefuehrt wird, bevor Datei da ist
\else%
\def\MIndex#1{\index{#1}}
\def\MEntryList{\MAbort{Stichwortliste nur im HTML realisierbar}}%
\fi%

\def\MEntry#1#2{\textbf{#1}\MIndex{#2}} % Idee: MLastType auf neuen Entry-Typ und dann ein MLabel vergeben mit autogen-Nummer

% ---------------------------------- Befehle fuer Tests ----------------------------------------------

% MEquationItem stellt eine Eingabezeile der Form Vorgabe = Antwortfeld her, der zweite Parameter kann z.B. MSimplifyQuestion-Befehl sein
\ifttm
\newcommand{\MEquationItem}[2]{{#1}$\,=\,${#2}}%
\else%
\newcommand{\MEquationItem}[2]{{#1}$\;\;=\,${#2}}%
\fi

\ifttm
\newcommand{\MInputHint}[1]{%
\ifnum%
\if\value{MTestSite}>0%
\else%
{\color{blue}#1}%
\fi%
\fi%
}
\else
\newcommand{\MInputHint}[1]{\relax}
\fi

\ifttm
\newcommand{\MInTestHeader}{%
Dies ist ein einreichbarer Test:
\begin{itemize}
\item{Im Gegensatz zu den offenen Aufgaben werden beim Eingeben keine Hinweise zur Formulierung der mathematischen Ausdr�cke gegeben.}
\item{Der Test kann jederzeit neu gestartet oder verlassen werden.}
\item{Der Test kann durch die Buttons am Ende der Seite beendet und abgeschickt, oder zur�ckgesetzt werden.}
\item{Der Test kann mehrfach probiert werden. F�r die Statistik z�hlt die zuletzt abgeschickte Version.}
\end{itemize}
}
\else
\newcommand{\MInTestHeader}{%
\relax
}
\fi

\ifttm
\newcommand{\MInTestFooter}{%
\special{html:<button name="Name_TESTFINISH" id="TESTFINISH" type="button" onclick="finish_button('}\MTestName\special{html:');">Test auswerten</button>}%
\begin{html}
&nbsp;&nbsp;&nbsp;&nbsp;&nbsp;&nbsp;&nbsp;&nbsp;
<button name="Name_TESTRESET" id="TESTRESET" type="button" onclick="reset_button();">Test zur�cksetzen</button>
<br />
<br />
<div class="xreply">
<p name="Name_TESTEVAL" id="TESTEVAL">
Hier erscheint die Testauswertung!
<br />
</p>
</div>
\end{html}
}
\else
\newcommand{\MInTestFooter}{%
\relax
}
\fi


% ---------------------------------- Notationsmakros -------------------------------------------------------------

% Notationsmakros die nicht von der Kursvariante abhaengig sind

\newcommand{\MZahltrennzeichen}[1]{\renewcommand{\MZXYZhltrennzeichen}{#1}}

\ifttm
\newcommand{\MZahl}[3][\MZXYZhltrennzeichen]{\edef\MZXYZtemp{\noexpand\special{html:<mn>#2#1#3</mn>}}\MZXYZtemp}
\else
\newcommand{\MZahl}[3][\MZXYZhltrennzeichen]{{}#2{#1}#3}
\fi

\newcommand{\MEinheitenabstand}[1]{\renewcommand{\MEinheitenabstXYZnd}{#1}}
\ifttm
\newcommand{\MEinheit}[2][\MEinheitenabstXYZnd]{{}#1\edef\MEINHtemp{\noexpand\special{html:<mi mathvariant="normal">#2</mi>}}\MEINHtemp} 
\else
\newcommand{\MEinheit}[2][\MEinheitenabstXYZnd]{{}#1 \mathrm{#2}} 
\fi

\newcommand{\MExponentensymbol}[1]{\renewcommand{\MExponentensymbXYZl}{#1}}
\newcommand{\MExponent}[2][\MExponentensymbXYZl]{{}#1{} 10^{#2}} 

%Punkte in 2 und 3 Dimensionen
\newcommand{\MPointTwo}[3][]{#1(#2\MCoordPointSep #3{}#1)}
\newcommand{\MPointThree}[4][]{#1(#2\MCoordPointSep #3\MCoordPointSep #4{}#1)}
\newcommand{\MPointTwoAS}[2]{\left(#1\MCoordPointSep #2\right)}
\newcommand{\MPointThreeAS}[3]{\left(#1\MCoordPointSep #2\MCoordPointSep #3\right)}

% Masseinheit, Standardabstand: \,
\newcommand{\MEinheitenabstXYZnd}{\MThinspace} 

% Horizontaler Leerraum zwischen herausgestellter Formel und Interpunktion
\ifttm
\newcommand{\MDFPSpace}{\,}
\newcommand{\MDFPaSpace}{\,\,}
\newcommand{\MBlank}{\ }
\else
\newcommand{\MDFPSpace}{\;}
\newcommand{\MDFPaSpace}{\;\;}
\newcommand{\MBlank}{\ }
\fi

% Satzende in herausgestellter Formel mit horizontalem Leerraum
\newcommand{\MDFPeriod}{\MDFPSpace .}

% Separation von Aufzaehlung und Bedingung in Menge
\newcommand{\MCondSetSep}{\,:\,} %oder '\mid'

% Konverter kennt mathopen nicht
\ifttm
\def\mathopen#1{}
\fi

% -----------------------------------START Rouletteaufgaben ------------------------------------------------------------

\ifttm
% #1 = Dateiname, #2 = eindeutige ID fuer das Roulette im Kurs
\newcommand{\MDirectRouletteExercises}[2]{
\begin{MExercise}
\texttt{Im HTML erscheinen hier Aufgaben aus einer Aufgabenliste...}
\end{MExercise}
}
\else
\newcommand{\MDirectRouletteExercises}[2]{\relax} % wird durch mconvert.pl gefunden und ersetzt
\fi


% ---------------------------------- START Makros, die von der Kursvariante abhaengen ----------------------------------

\ifvariantunotation
  % unotation = An Universitaeten uebliche Notation
  \def\MVariant{unotation}

  % Trennzeichen fuer Dezimalzahlen
  \newcommand{\MZXYZhltrennzeichen}{.}

  % Exponent zur Basis 10 in der Exponentialschreibweise, 
  % Standardmalzeichen: \times
  \newcommand{\MExponentensymbXYZl}{\times} 

  % Begrenzungszeichen fuer offene Intervalle
  \newcommand{\MoIl}[1][]{\mbox{}#1(\mathopen{}} % bzw. ']'
  \newcommand{\MoIr}[1][]{#1)\mbox{}} % bzw. '['

  % Zahlen-Separation im IntervaLL
  \newcommand{\MIntvlSep}{,} %oder ';'

  % Separation von Elementen in Mengen
  \newcommand{\MElSetSep}{,} %oder ';'

  % Separation von Koordinaten in Punkten
  \newcommand{\MCoordPointSep}{,} %oder ';' oder '|', '\MThinspace|\MThinspace'

\else
  % An dieser Stelle wird angenommen, dass std-Variante aktiv ist
  % std = beschlossene Notation im TU9-Onlinekurs 
  \def\MVariant{std}

  % Trennzeichen fuer Dezimalzahlen
  \newcommand{\MZXYZhltrennzeichen}{,}

  % Exponent zur Basis 10 in der Exponentialschreibweise, 
  % Standardmalzeichen: \times
  \newcommand{\MExponentensymbXYZl}{\times} 

  % Begrenzungszeichen fuer offene Intervalle
  \newcommand{\MoIl}[1][]{\mbox{}#1]\mathopen{}} % bzw. '('
  \newcommand{\MoIr}[1][]{#1[\mbox{}} % bzw. ')'

  % Zahlen-Separation im IntervaLL
  \newcommand{\MIntvlSep}{;} %oder ','
  
  % Separation von Elementen in Mengen
  \newcommand{\MElSetSep}{;} %oder ','

  % Separation von Koordinaten in Punkten
  \newcommand{\MCoordPointSep}{;} %oder '|', '\MThinspace|\MThinspace'

\fi



% ---------------------------------- ENDE Makros, die von der Kursvariante abhaengen ----------------------------------


% diese Kommandos setzen Mathemodus vorraus
\newcommand{\MGeoAbstand}[2]{[\overline{{#1}{#2}}]}
\newcommand{\MGeoGerade}[2]{{#1}{#2}}
\newcommand{\MGeoStrecke}[2]{\overline{{#1}{#2}}}
\newcommand{\MGeoDreieck}[3]{{#1}{#2}{#3}}

%
\ifttm
\newcommand{\MOhm}{\special{html:<mn>&#x3A9;</mn>}}
\else
\newcommand{\MOhm}{\Omega} %\varOmega
\fi


\def\PERCTAG{\MAbort{PERCTAG ist zur internen verwendung in mconvert.pl reserviert, dieses Makro darf sonst nicht benutzt werden.}}

% Im Gegensatz zu einfachen html-Umgebungen werden MDirectHTML-Umgebungen von mconvert.pl am ganzen ttm-Prozess vorbeigeschleust und aus dem PDF komplett ausgeschnitten
\ifttm%
\newenvironment{MDirectHTML}{\begin{html}}{\end{html}}%
\else%
\newenvironment{MDirectHTML}{\begin{html}}{\end{html}}%
\fi

% Im Gegensatz zu einfachen Mathe-Umgebungen werden MDirectMath-Umgebungen von mconvert.pl am ganzen ttm-Prozess vorbeigeschleust, ueber MathJax realisiert, und im PDF als $$ ... $$ gesetzt
\ifttm%
\newenvironment{MDirectMath}{\begin{html}}{\end{html}}%
\else%
\newenvironment{MDirectMath}{\begin{equation*}}{\end{equation*}}% Vorsicht, auch \[ und \] werden in amsmath durch equation* redefiniert
\fi

% ---------------------------------- Location Management ---------------------------------------------

% #1 = buttonname (muss in files/images liegen und Format 48x48 haben), #2 = Vollstaendiger Einrichtungsname, #3 = Kuerzel der Einrichtung,  #4 = Name der include-texdatei
\ifttm
\newcommand{\MLocationSite}[3]{\special{html:<!-- mlocation;;}#1\special{html:;;}#2\special{html:;;}#3\special{html:;; //-->}}
\else
\newcommand{\MLocationSite}[3]{\relax}
\fi

% ---------------------------------- Copyright Management --------------------------------------------

\newcommand{\MCCLicense}{%
{\color{green}\textbf{CC BY-SA 3.0}}
}

\newcommand{\MCopyrightLabel}[1]{ (\MSRef{L_COPYRIGHTCOLLECTION}{Lizenz})\MLabel{#1}}

% Copyrightliste wird als tex-Datei im preprocessing erzeugt und unter converter/tex/copyrightcollection.tex abgelegt
% Der input-Befehl funktioniert nur, wenn die aufrufende tex-Datei auf der obersten Ebene liegt (d.h. selbst kein input/include ist, insbesondere keine Moduldatei)
\newcommand{\MCopyrightCollection}{\input{copyrightcollection.tex}}

% MCopyrightNotice fuegt eine Copyrightnotiz ein, der parser ersetzt diese durch CopyrightNoticePOST im preparsing, diese Definition wird nur fuer reine pdflatex-Uebersetzungen gebraucht
% Parameter: #1: Kurze Lizenzbeschreibung (typischerweise \MCCLicense)
%            #2: Link zum Original (http://...) oder NONE falls das Bild selbst ein Original ist, oder TIKZ falls das Bild aus einer tikz-Umgebung stammt
%            #3: Link zum Autor (http://...) oder MINT falls Original im MINT-Kolleg erstellt oder NONE falls Autor unbekannt
%            #4: Bemerkung (z.B. dass Datei mit Maple exportiert wurde)
%            #5: Labelstring fuer existierendes Label auf das copyrighted Objekt, mit MCopyrightLabel erzeugt
%            Keines der Felder darf leer sein!
\newcommand{\MCopyrightNotice}[5]{\MCopyrightNoticePOST{#1}{#2}{#3}{#4}{#5}}

\ifttm%
\newcommand{\MCopyrightNoticePOST}[5]{\relax}%
\else%
\newcommand{\MCopyrightNoticePOST}[5]{\relax}%
\fi%

% ---------------------------------- Meldungen fuer den Benutzer des Konverters ----------------------
\MPragma{mintmodversion;P0.1.0}
\MPragma{usercomment;This is file mintmod.tex version P0.1.0}


% ----------------------------------- Spezialelemente fuer Konfigurationsseite, werden nicht von mintscripts.js verwaltet --

% #1 = DOM-id der Box
\ifttm\newcommand{\MConfigbox}[1]{\special{html:<input cfieldtype="2" type="checkbox" name="Name_}#1\special{html:" id="}#1\special{html:" onchange="confHandlerChange('}#1\special{html:');"/>}}\fi % darf im PDF nicht aufgerufen werden!


\MPragma{MathSkip}
\Mtikzexternalize

\begin{document}

\MSection{Language of Descriptive Statistics}
\MLabel{VBKM_STOCH}
\MSetSectionID{VBKM_STOCHASTIK} % wird fuer tikz-Dateien verwendet

\begin{MSectionStart}
\MDeclareSiteUXID{VBKM_STOCHASTIK_START}

In this Module we discuss the most important basics of descriptive statistics. In particular, we will 
discuss rounding and percentage calculations (which actually are not subjects of descriptive 
statistics, but will be needed there). The competent handling of percentage calculation
is essential in economics. Experience tells that these elementary subjects are already taught at secondary school
but often not thoroughly enuogh. For example, a test has shown that 
half of first year students are not able to calculate the amount of VAT in a gross invoice. 
This module consists of the following sections:


\begin{itemize}
\item{\MSRef{VBKM_STOCH_1_1}{Terminology}: we introduce the fundamental concepts of statistics and explain different 
rounding methods for numbers.}
\item{\MSRef{VBKM_STOCH_2_1}{Frequency Distributions and Percentage Calculation}: we introduce frequency distributions and explain 
the percentage calculation involved as well as the visualisation of the results using typical types of diagrams.}
\item{\MSRef{VBKM_STOCH_3_1}{Statistical Measures}: we explain the fundamental statistical measures of descriptive statistics 
such as arithmetic mean and sample variance.}
\item{\MSRef{VBKM_STOCH_Abschlusstest}{Final Test}.}
\end{itemize}

\end{MSectionStart}


\MSubsection{Terminology and Language}

\begin{MIntro}
\MLabel{VBKM_STOCH_1_1}
\MDeclareSiteUXID{VBKM_STOCHASTIK_1_1}

For statistical observations (surveys) of appropriately chosen units of observation
(a.k.a. units of investigation or experimental units), the values or attributes of a property or properties are determined. 
Here, a property is a characteristic of the observation unit to be investigated. The terminology of descriptive 
statistics is as follows:

\begin{itemize}
\item{The \MEntry{unit of investigation}{unit of investigation} (also: \MEntry{unit of observation}{unit of observation}) 
is the smallest unit on which the observations are made.}
\item{The \MEntry{characteristic}{characteristics} or \MEntry{property}{property} 
is the statistical variable of the unit to be investigated. Characteristics are often denoted by upper-case Latin letters 
($X, Y, Z, \ldots$).}
\item{\MEntry{Characteristic attributes}{characteristic attributes} or \MEntry{property values}{property values} are values
that properties can take. They are often denoted by lower-case Latin letters 
($a, b,\ldots, x, y, z, a_1, a_2,\ldots$).}
\item{The set of units of observation that is investigated with respect to a property of interest 
is called \MEntry{universe}{universe} or also \MEntry{population}{population}. It is the set of all possible 
observation units.}
\item{A \MEntry{sample}{sample} is a ``random finite subset'' of a certain population of interest. If this 
set consists of $n$~elements, then this set is called a ``sample of size $n$''.}
\item{\MEntry{Data}{data} are the observed values (attributes) of one or more characteristics or properties 
of a sample unit of observation of a certain population.}
\item{The \MEntry{original list}{original list} is the protocol that lists the sampled data in 
chronological order. Thus, the original list is a $n$-tuple (or vector, written here mostly in coordinate form):
$$
x \;=\; (x_1,\ldots,x_n)\: .
$$
This $n$-tuple is often called a ``sample of size $n$''.}
\end{itemize}

\begin{MExample}
From a daily production of components in a factory, $n=20$~samples of $15$~parts each are taken and the number of defective parts in each sample 
is determined. Here, $x_i$ is the number of defective parts in the $i$th sample, $i=1,\ldots,20$. The original list (sample of 
size $n=20$) contains the following data:
$$
x\; =\; (0,4,2,1,1,0,0,2,3,1,0,5,3,1,1,2,0,0,1,0)\: .
$$
In the second sample, $x_2=4$ defective parts were found. The population in this example is the set of all 
$15$-element subsets of the daily production. The property of interest is in this case
$$
X\;=\; \text{Number of defective workpieces in a sample of 15~elements}\: .
$$
\end{MExample}

\begin{MInfo}

The variables in a statistical observation are called \MEntry{characteristics}{characteristics} or \MEntry{properties}{property}.
Values that the properties can take are called \MEntry{property values}{property values} or 
\MEntry{characteristic attributes}{characteristic attributes}.
\end{MInfo}

Properties are roughly classified into qualitative properties (that can be ascertained in a descriptive way) and quantitative properties 
(that can naturally be ascertained numerically):
\begin{itemize}
\item{\MEntry{Qualitative properties}{property (qualitative)}:\begin{itemize}
  \item{Nominal properties: attributes classified according to purely qualitative aspects. Examples: 
  skin colour, nationality, blood type.}
  \item{Ordinal properties: attributes with a natural hierarchy, i.e. they can be ordered or sorted.
        Examples: grades, ranks, surnames.}
  \end{itemize}
  }
\item{\MEntry{Quantitative properties}{property (quantitative)}:\begin{itemize}
  \item{Discrete properties: property values are isolated values (e.g. integers).
        Examples: numbers, years, age in years.}
  \item{Continuous properties: property values can (at least in principle) take any value. 
        Examples: body size, weight, length.}
  \end{itemize}
}
\end{itemize}

The transition between continuous and discrete properties is partly fluid, once we consider the possibility of rounding.
\end{MIntro}


\begin{MXContent}{Rounding}{Rounding}{STD}
\MDeclareSiteUXID{VBKM_STOCHASTIK_1_2}
The \MEntry{rounding}{rounding} of measurement values is an everyday process.

\begin{MInfo}

In principle, there are three ways of rounding:

\begin{itemize}
\item{Rounding (off) using the $\MTextSF{floor}$ function $\left\lfloor{x}\right\rfloor$.}
\item{Rounding (up) using the $\MTextSF{ceil}$ function $\left\lceil{ x}\right\rceil$.}
\item{Rounding using the $\MTextSF{round}$ function (sometimes also called $\MTextSF{rnd}$ function).}
\end{itemize}
\end{MInfo}

The \MEntry{$\MTextSF{floor}$ function}{floor function} is defined as
$$
\MTextSF{floor}:\; \R\longrightarrow\R\;\; , \;\;
x\;\longmapsto\; \MTextSF{floor}(x)\;=\; \left\lfloor{ x}\right\rfloor\;=\; \max\lbrace k\in\Z\: :\: k\leq x\rbrace\: .
$$
If $x\in\R$ is a real number, then $\MTextSF{floor}(x)=\left\lfloor{ x}\right\rfloor$ is the largest integer 
that is smaller than or equal to $x$. It results from rounding off the value of $x$. If a positive 
real number $x$ is written as a decimal, then $\left\lfloor{ x}\right\rfloor$ equals the 
integer on the left of the decimal point: rounding (off) cuts off the digits on the right of the decimal point. 
For example $\left\lfloor{ 3.142}\right\rfloor=3$ but $\left\lfloor{ -2.124}\right\rfloor=-3$.
The $\MTextSF{floor}$ function is a step function with jumps (in more mathematical terms, jump discontinuities) of height $1$ at all points 
$x\in \Z$. The function values at the jumps always lie a step up. They are indicated by the small circles in 
the figure below, which shows the graph of the $\MTextSF{floor}$ function. 

\begin{center}
\MUGraphicsSolo{floor_400_0.png}{width=0.7\linewidth}{width:650px}\\
Graph of the $\MTextSF{floor}$ function
\end{center}

Let a real number $a\geq 0$ be given, written as a \MSRef{Mathematik_Grundlagen_UDB}{decimal number}
$$
a \;=\; g_n\, g_{n-1}\, \ldots\, g_1\, g_0\: .\: a_1\,a_2\, a_3\, \ldots
$$
This number $a$ can be rounded to $r$ fractional digits ($r\in\N_0$) using the $\MTextSF{floor}$ function by
$$
\tilde a \;=\; \frac1{10^r}\cdot \left\lfloor{ 10^r\cdot a}\right\rfloor\: .
$$
This process of rounding cuts off the decimal after the $r$th fractional digit. Thus, rounding 
using the $\MTextSF{floor}$ function is in general a rounding off.

\begin{MExample}
Rounding the number $a_1=2.3727$ to 2 fractional digits using the $\MTextSF{floor}$ function results in
$$
\tilde a_1 \;=\; \frac1{10^2}\cdot \left\lfloor{10^2\cdot2.3727}\right\rfloor\; =\; \frac1{10^2}\cdot \left\lfloor{ 237.27}\right\rfloor \;=\; \frac1{10^2}\cdot 237\;=\; =2.37\: .
$$
Alternatively, it can be rounded by cutting off the decimal after the second fractional digit 
(however, this is only possible if the number is given as a decimal which is rarely the case in a computer program).


Rounding the number $a_2=\sqrt{2}=1.414213562\ldots$ to 4 fractional digits using the $\MTextSF{floor}$ function results in
$$
\tilde a_2 \;=\; \frac1{10^4}\cdot \left\lfloor{ 10^4\cdot \sqrt2}\right\rfloor \;=\; \frac1{10^4}\cdot \left\lfloor{ 14142.1\ldots}\right\rfloor\;=\; \frac1{10^4}\cdot 14142 \;=\; 1.4142\: .
$$
Rounding the number
$$
a_3 \;=\; \pi \;=\; 3,141592654\ldots
$$
to 2 fractional digits using the $\MTextSF{floor}$ function results in
$$
\tilde a_3 \;=\; \frac1{10^2}\cdot \left\lfloor{ 10^2\cdot\pi}\right\rfloor\;=\; \frac1{10^2}\cdot \left\lfloor{ 314.159\ldots}\right\rfloor \;=\; \frac1{10^2}\cdot 314 \;=\; 3.14\: .
$$
\end{MExample}

The rounding method using the $\MTextSF{floor}$ function is often applied for calculating final grades in certificates 
(``academic rounding''). If a mathematics student has the individual grades
\begin{center}
\begin{tabular}{|c|c|}
\hline
Subject & Grade \\ \hline
Mathematics 1 & $1.3$ \\ 
Mathematics 2 & $2.3$ \\ 
Mathematics 3 & $2.0$ \\ \hline
\end{tabular}
\end{center}
then the arithmetic mean of these grades is calculated by 
$$
\frac{1.3 + 2.3 + 2.0}{3}\; =\; \frac{5.6}{3} \; =\; 1.8\overline{6}\: .
$$
Rounding to the first fractional digit using the $\MTextSF{floor}$ function would result in the final 
grade of $\tilde a=1.8$. The rounding methods for calculating final grades always have to be described exactly 
in the examination regulations.

The counterpart to the $\MTextSF{floor}$ function is the $\MTextSF{ceil}$ (a.k.a. ceiling) function:

\begin{MInfo}
The \MEntry{$\MTextSF{ceil}$ function}{ceil function} is defined as
$$
\MTextSF{ceil}:\; \R\longrightarrow\R\;\; , \;\;
x\;\longmapsto\; \MTextSF{ceil}(x)\;=\; \left\lceil{ x}\right\rceil\;=\; \min\lbrace k\in\Z\: :\: k\geq x\rbrace\: .
$$
\end{MInfo}

If $x\in \mathbb{R}$ is a real number, then $\MTextSF{ceil}(x)=\left\lceil{ x }\right\rceil$ is the smallest integer 
that is greater than or equal to $x$. The $\MTextSF{ceil}$ function is a step function with jumps (jump discontinuities) of height $1$ at all points  $x\in\Z$. The function values at the jumps always lie at the bottom. 
They are indicated by the small circles in the figure below showing the graph of the $\MTextSF{ceil}$ function. 

\begin{center}
\MUGraphicsSolo{ceil_400_0.png}{width=0.7\linewidth}{width:650px}\\
Graph of the $\MTextSF{ceil}$ function
\end{center}

Let a real number $a\geq 0$ be given as a \MSRef{Mathematik_Grundlagen_UDB}{decimal number}
$$
a \;=\; g_n\, g_{n-1}\, \ldots\, g_1\, g_0\: .\: a_1\,a_2\, a_3\, \ldots
$$
This number $a$ can be rounded to $r$ fractional digits ($r\in\N_0$) using the $\MTextSF{ceil}$ function by
$$
\hat{a} \;=\; \frac{1}{10^{r}}\cdot \left\lceil{ 10^{r}\cdot a }\right\rceil \: .
$$
Rounding using the $\MTextSF{ceil}$ function is in general a rounding up to the next decimal digit.

\begin{MExample}
Rounding the number $a_1=2.3727$ to $2$ fractional digits using the $\MTextSF{ceil}$ function results in
$$
\hat{a}_{1}\;=\; \frac{1}{10^{2}}\cdot \left\lceil{ 10^{2}\cdot 2.3727}\right\rceil\; =\; \frac{1}{10^{2}}\cdot \left\lceil{ 237.27}\right\rceil \;=\; \frac{1}{10^{2}}\cdot 238 \;=\; 2.38\: .
$$
Analogously, rounding the number $a_{2}=\sqrt{2}=1.414213562\ldots$ to $4$ fractional digits using the $\MTextSF{ceil}$ function results in
$$
\hat{a}_{2}\; =\; \frac{1}{10^{4}}\cdot \left\lceil{ 10^{4}\cdot \sqrt{2}}\right\rceil\; =\; \frac{1}{10^{4}}\cdot \left\lceil{ 14142.1\ldots}\right\rceil \;=\; \frac{1}{10^{4}}\cdot 14143\; =\; 1.4143\: .
$$
Rounding the number $a_{3}=\pi=3.141592654\ldots$ to $2$ fractional digits using the $\MTextSF{ceil}$ function results in
$$
\hat{a}_{3}\; =\; \frac{1}{10^{2}}\cdot \left\lceil{ 10^{2}\cdot \pi }\right\rceil\; =\; \frac{1}{10^{2}}\cdot \left\lceil{ 314.15\ldots }\right\rceil\; =\;\frac{1}{10^{2}}\cdot 315 \;=\; 3.15\: .
$$
\end{MExample}

The rounding method using the $\MTextSF{ceil}$ function is often applied, for example, in craftsmen's invoices. 
A craftsman is mostly paid by the hour. If a repair takes 50~minutes (i.e. $0.8\overline{3}$ hours as 
a decimal), then a craftsmen will round up and invoice a full working hour. Colloquially, rounding 
mostly means mathematical rounding:

\begin{MInfo}
The \MEntry{$\MTextSF{round}$ function}{round function} (or mathematical rounding) is defined as
$$
\MTextSF{round}:\; \R\longrightarrow\R\;\; , \;\;
x\;\longmapsto\; \MTextSF{round}(x)\;=\; \MTextSF{floor}\left({x + \frac{1}{2}}\right) \; =\; \left\lfloor{ x + \frac{1}{2} }\right\rfloor\: .
$$
In contrast to rounding up or rounding off, the maximum change to the number by this rounding is $0.5$.
\end{MInfo}

The $\MTextSF{round}$ function is a step function with jumps (jump discontinuities) of height $1$ at all points $x+\frac{1}{2},\; x\in \Z$. The function values at the jumps always
lie a step up. They are indicated by the small circles in the figure below showing the graph of the $\MTextSF{round}$ function.

\begin{center}
\MUGraphicsSolo{round_400_0.png}{width=0.7\linewidth}{width:650px}\\
Graph of the $\MTextSF{round}$ function
\end{center}

Let a real number $a\geq 0$ be given as a \MSRef{Mathematik_Grundlagen_UDB}{decimal number}
$$
a \;=\; g_n\, g_{n-1}\, \ldots\, g_1\, g_0\: .\: a_1\,a_2\, a_3\, \ldots
$$
This number $a$ can be rounded to $r$ fractional digits ($r\in\N_0$) using the $\MTextSF{round}$ function:
$$
\overline{a}\; =\; \frac{1}{10^{r}}\cdot \MTextSF{round}(10^{r}\cdot a)\; =\; \frac{1}{10^{r}}\cdot \left\lfloor{ 10^{r}\cdot a + \frac{1}{2} }\right\rfloor\: .
$$
This rounding method is called mathematical rounding and corresponds to the ``normal'' rounding process.

\begin{MExample}
The number $a_{1}=1.49$ is rounded to one fractional digit using the $\MTextSF{round}$ function to
\begin{eqnarray*}
\overline{a}_1 & = & \frac{1}{10}\cdot \mathsf{round}(10 \cdot 1.49) \; =\; \frac{1}{10}\cdot \left\lfloor{10\cdot 1.49 + 0.5}\right\rfloor\\
&=& \frac{1}{10}\cdot \left\lfloor{14.9+0.5}\right\rfloor \;=\; \frac{1}{10}\cdot \left\lfloor{ 15.4 }\right\rfloor \;=\; \frac{1}{10}\cdot 15 \;=\; 1.5\: .
\end{eqnarray*}
The number $a_{2}=1.52$ is rounded to one fractional digit using the  $\MTextSF{round}$ function to
\begin{eqnarray*}
\overline{a}_{2}& = & \frac{1}{10}\cdot \mathsf{round}(10 \cdot 1.52)\; =\; \frac{1}{10}\cdot \left\lfloor{ 10\cdot 1.52 + 0.5 }\right\rfloor \\
& = & \frac{1}{10}\cdot \left\lfloor{ 15.2+0.5}\right\rfloor \;=\; \frac{1}{10}\cdot \left\lfloor{ 15.7 }\right\rfloor\; =\; \frac{1}{10}\cdot 15 \;=\; 1.5\: .
\end{eqnarray*}
The number $a_{3}=2.3727$ is rounded to two fractional digits using the  $\MTextSF{round}$ function to
\begin{eqnarray*}
\overline{a}_{3} & = & \frac{1}{10^{2}}\cdot \mathsf{round}(10^{2}\cdot 2.3727)\; =\; \frac{1}{100}\cdot \left\lfloor{ 100\cdot 2.3727 + 0.5 }\right\rfloor \\
& = & \frac{1}{100}\cdot \left\lfloor{ 237.27+0.5}\right\rfloor\; =\; \frac{1}{100}\cdot \left\lfloor{ 237.77 }\right\rfloor \; =\;  \frac{1}{100}\cdot 237\; =\; 2.37\: .
\end{eqnarray*}
The number  $a_{4}=\sqrt{2}=1.414213562\ldots$ is rounded to seven fractional digits using the  $\MTextSF{round}$ function to
\begin{eqnarray*}
\overline{a}_{3} & = & \frac{1}{10^{7}}\cdot \mathsf{round}(10^{7}\cdot \sqrt{2})\; =\; \frac{1}{10^{7}}\cdot \left\lfloor{ 10^{7}\cdot 1.414213562\ldots + 0.5 }\right\rfloor \\
& = & \frac{1}{10^{7}}\cdot \left\lfloor{ 14142135.62\ldots+0.5}\right\rfloor \;=\; \frac{1}{10^{7}}\cdot \left\lfloor{ 14142136.12\ldots }\right\rfloor \\
& = & \frac{1}{10^{7}}\cdot 14142136\; =\;  1.4142136\: .
\end{eqnarray*}
\end{MExample}

\begin{MExercise}
Using the $\MTextSF{round}$ function, round the number $\pi=3.141592654\ldots$ to four fractional digits:
\MEquationItem{$\overline{\pi}$}{\MLParsedQuestion{20}{3.1416}{7}{STOCHROUND1}}.
\ \\ \ \\
\begin{MHint}{Solution}
\begin{eqnarray*}
\overline{\pi} & = & \frac{1}{10^{4}}\cdot \mathsf{round}(10^{4}\cdot \pi)\; =\; \frac{1}{10^{4}}\cdot \left\lfloor{ 10^{4}\cdot 3.141592654\ldots + 0.5 }\right\rfloor \ \\
& = & \frac{1}{10^{4}}\cdot \left\lfloor{ 31415.92654\ldots+0.5}\right\rfloor \;=\; \frac{1}{10^{4}}\cdot \left\lfloor{ 31416.42654\ldots }\right\rfloor\ \\
& = & \frac{1}{10^{4}}\cdot 31416\; =\;  3.1416\: .
\end{eqnarray*}
\end{MHint}
\end{MExercise}


\begin{MExercise}
Let the numbers 
$$
a\; =\; \frac{47}{17} \;\;\text{and}\;\;  b\; =\; 3.7861
$$
be given.
\begin{MExerciseItems}
\item{Round each of the numbers $a$ and $b$ to $2$ fractional digits using the $\MTextSF{floor}$ function.\\The roundings result in
\MEquationItem{$\tilde a$}{\MLParsedQuestion{10}{2.76}{7}{STOCHROUND2}} and \MEquationItem{$\tilde b$}{\MLParsedQuestion{10}{3.78}{7}{STOCHROUND3}}.}
\item{Round each of the numbers $a$ and $b$ to $2$ fractional digits using the $\MTextSF{ceil}$ function.\\The roundings result in
\MEquationItem{$\hat a$}{\MLParsedQuestion{10}{2.77}{7}{STOCHROUND4}} and \MEquationItem{$\hat b$}{\MLParsedQuestion{10}{3.79}{7}{STOCHROUND5}}.}
\item{Round each of the numbers $a$ and $b$ to $2$ fractional digits using the $\MTextSF{round}$ function.\\The roundings result in
\MEquationItem{$\overline{a}$}{\MLParsedQuestion{10}{2.76}{7}{STOCHROUND6}} and \MEquationItem{$\overline{b}$}{\MLParsedQuestion{10}{3.79}{7}{STOCHROUND7}}.}
\end{MExerciseItems}
\ \\
\begin{MHint}{Solution}
First, we transform the fraction into an appropriate decimal fraction by dividing successively with remainder
and substituting the results of the division as digits into the decimal:
$$
\frac{47}{17} \;=\; 2+\frac{13}{17}.
$$
The digit left to the decimal point is $2$. We have
$$
2+\frac{13}{17} \;=\; 2+\frac1{10}\cdot \frac{130}{17} \;=\; 2+\frac1{10}\cdot\left({7+\frac{11}{17}}\right).
$$
The first fractional digit is $7$. Proceeding further this way results in $a\;=\;  2.764705\ldots$, 
we only need three fractional digits for the required rounding. The complete calculation using the 
$\MTextSF{floor}$ function results in
$$
\tilde a \;=\; \frac1{10^2}\cdot \left\lfloor{ 10^2\cdot 2.764705\ldots}\right\rfloor \;=\; \frac1{10^2}\cdot \left\lfloor{ 276.4705\ldots}\right\rfloor \;=\;  2.76\: .
$$
This result can be obtained more quickly by simply cutting off the decimal after the second decimal digit:
$$
\tilde a \;=\; 2.76\;\; \text{and}\;\; \tilde b \;=\; 3.78\: .
$$
However, this is only allowed in this exercise since $a$ and $b$ are non-negative. 
A simple rounding up after the second decimal digit or a complete calculation using the $\MTextSF{ceil}$ function 
results in
$$
\hat a \;=\; 2.77\;\; \text{and}\;\; \hat b \;=\; 3.79\: .
$$
The results for the mathematical rounding are obtained either by a complete calculation as in the examples above, or 
by rounding to the decimal with two fractional digits with the smallest difference from the original number:
$$
\overline{a} \; = \; 2.76 \;\; \text{and}\;\; \overline{b} \,=\; 3.79\: .
$$
\end{MHint}
\end{MExercise}

\end{MXContent}

\begin{MXContent}{Remarks on the Rounding Processes}{Remarks}{STD}
\MDeclareSiteUXID{VBKM_STOCHASTIK_1_3}

As the following considerations and examples will show, we have to be very careful when calculating with rounded results. 
Let us consider the set $M=\R_{\geq0}$ of all non-negative real numbers. On this set, let us define the 
multiplication
$$
M\times M \;\longrightarrow M\;\; , \;\;
(a,b)\;\longmapsto\; a\odot b
$$
by the calculation rule
$$
a\odot b \; =\; \frac{1}{10^{2}}\cdot \MTextSF{round}( 10^{2}\cdot a \cdot b)\; =\; \frac{1}{10^{2}}\cdot \left\lfloor{ 10^{2}\cdot a \cdot b + \frac{1}{2} }\right\rfloor\: ,
$$
i.e. the product $a\odot b$ is calculated by calculating the usual product $a\cdot b$ first and subsequently rounding the result mathematically 
to two fractional digits.
\ \\ \ \\
The law of associativity no longer applies to the rounded multiplication. For the numbers 
$a=2.11$, $b=3.35$, and $c=2,61$, we have, for example,
$$
a\odot b \;=\; 2.11 \odot 3.35 \; =\; 7.07\;\text{and}\;
(a\odot b)\odot c \; =\; 7.07 \odot 2.61 \;=\;  18.45\: .
$$
Changing the brackets, however, results in
$$
b\odot c\;  =\; 3.35 \odot 2.61 \; =\; 8.74\;\;\text{and}\;\;a\odot (b\odot c)\; =\;2.11 \odot 8.74 \;=\; 18.44\: .
$$

\begin{MInfo}
Since calculators (and computers) always calculate using rounded results, this means that the 
law of associativity does not apply unrestrictedly to multiplication on calculators.
\end{MInfo}

Likewise, false results can be caused by careless rounding. Let us consider the numbers $a=4.98$ and $b=1.001$. Then, we have
$$
a\cdot b\; =\; 4.98 \cdot 1.001\; =\; 4.98498\;\text{ , i.e.}\;
a\odot b \; =\; 4.98 \odot 1.001\; =\; 4.98\; =\; a\: .
$$
Furthermore, we have
$$
a\cdot b^{1000}\; =\; 4.98 \cdot 1.001^{1000} \;=\; a\cdot \underbrace{b\cdot \ldots \cdot b}_{1000 \text{ factors}} \approx 13.53028118\: .
$$
Rounding after each multiplication to $2$ fractional digits results (due to $a\odot b=a$) in the wrong result:
$$
(\ldots ((a\odot \underbrace{b) \odot b) \ldots \odot b}_{1000 \text{ factors}}) \;=\;  a\; =\;4.98 \: .
$$

\begin{MInfo}
\MLabel{L_STOCH_RP}
In practice, this means that you must calculate at least with double precision (twice the required digits) and round the result only finally 
to the required digits.
\end{MInfo}

\end{MXContent}

\MSubsection{Frequency Distributions and Percentage Calculation}
\MLabel{VBKM_STOCH_2}

\begin{MIntro}
\MDeclareSiteUXID{VBKM_STOCHASTIK_2_1}
\MLabel{VBKM_STOCH_2_1}

Let $X$ be a given property. A sample of size $n$ resulted in the original list (sample)
$$
x \;=\; (x_{1},x_{2},\ldots ,x_{n})\: .
$$

\begin{MInfo}
\MLabel{L_STOCH_H}
If $a$ is a possible property value, then 
$$
H_{x}(a)\; =\; \text{number of}\;x_{j}\;\text{ within the original list}\;x\;\text{with}\;x_{j}=a
$$
is called the \MEntry{absolute frequency}{frequency (absolute)} of the property 
$a$ in the original list $x=(x_{1},x_{2},\ldots ,x_{n})$.
\end{MInfo}

If $a_{1}, a_{2},\ldots a_{k}$ are the possible property values in the original list
$x=(x_{1},x_{2},\ldots ,x_{n})$, then we have
$$
H_{x}(a_{1})+H_{x}(a_{2})+\ldots + H_{x}(a_{k})\; =\; n
$$
or in words: each of the $n$ values is counted by exactly one of the frequencies.

\begin{MInfo}
The \MEntry{relative frequency}{frequency (relative)} of the property value $a$ in the original list  
$x=(x_{1},x_{2},\ldots ,x_{n})$ is defined by
$$
h_{x}(a)\; =\;\frac{1}{n}\cdot H_{x}(a)\: .
$$
\end{MInfo}

If $a_{1}, a_{2},\ldots a_{k}$ are the possible property values in the original list
$x=(x_{1},x_{2},\ldots ,x_{n})$, then we have
$$
h_{x}(a_{1})+h_{x}(a_{2})+\ldots + h_{x}(a_{k})\; =\;1\: .
$$
Relative frequencies always lie in the interval $[0;1]$ and are often specified in percentages,e.g. $h_x(a_1)=34\%$ instead of $h_x(a_1)=0.34$.

\begin{MInfo}
Collecting the absolute or relative frequencies of all occurring (or possible) property values 
in the original list (sample) $x=(x_{1},x_{2},\ldots ,x_{n})$ in a table results in the
\MEntry{empirical frequency distribution}{frequency distribution}.
\end{MInfo}

\begin{MExample}

In a data centre, the processing time (in seconds, rounded to one fractional digit)
of $20$~program jobs was determined. This resulted in the following original list of a sample of size $n=20$:
\begin{center}
\begin{tabular}{|l|l|l|l|l|}
\hline
$3.9$ & $3.3$ & $4.6$ & $4.0$ & $3.8$ \\ \hline
$3.8$ & $3.6$ & $4.6$ & $4.0$ & $3.9$ \\ \hline
$3.9$ & $3.9$ & $4.1$ & $3.7$ & $3.6$ \\ \hline
$4.6$ & $4.0$ & $4.0$ & $3.8$ & $4.1$ \\ \hline
\end{tabular}
\end{center}

The smallest value is $3.3$~s, the largest value is $4.6$~s, the increment is $0.1$~s. Thus, we have the 
empirical frequency distribution listed (in tabular form) below. To keep the table short all values 
less than $3.3$ and greater that $4.6$ are not listed.
\begin{center}
\begin{tabular}{|c|c|c|c|}
\hline
Result $a$ & $H_{x}(a)$ & $h_{x}(a)$ & Percentage \\ \hline
$3.3$ & $1$ & $\frac{1}{20}=0.05$ & $5\%$ \\
$3.4$ & $0$ & $0$ & $0\%$ \\
$3.5$ & $0$ & $0$ & $0\%$ \\
$3.6$ & $2$ & $\frac{2}{20}=0.1$ & $10\%$\\
$3.7$ & $1$ & $\frac{1}{20}=0.05$ & $5\%$\\
$3.8$ & $3$ & $\frac{3}{20}=0.15$ & $15\%$\\
$3.9$ & $4$ & $\frac{4}{20}=0.2$ & $20\%$ \\
$4.0$ & $4$ & $\frac{4}{20}=0.2$ & $20\%$ \\
$4.1$ & $2$ & $\frac{2}{20}=0.1$ & $10\%$\\
$4.2$ & $0$ & $0$ & $0\%$ \\
$4.3$ & $0$ & $0$ & $0\%$ \\
$4.4$ & $0$ & $0$ & $0\%$ \\
$4.5$ & $0$ & $0$ & $0\%$\\
$4.6$ & $3$ & $\frac{3}{20}=0.15$ & $15\%$\\ \hline
Sum & $20$ & $1$ & $100\%$\\ \hline
\end{tabular}
\end{center}
\end{MExample}

\end{MIntro}

\begin{MXContent}{Percentage Calculation}{Percentage Calculation}{STD}
\MDeclareSiteUXID{VBKM_STOCHASTIK_2_2}
In descriptive statistics, numerical values are often specified in percentages, so we will review the most relevant 
elements of percentage calculations in this section. Numbers given as percentages (``percent, hundredth'') serve to 
illustrate ratios and to make them comparable by putting the numbers into relation to a unified  base value (hundred).

\begin{MInfo}
Let $a\geq 0$ be a real number. Then, we have $a\,\% = \frac{a}{100}$, i.e. the symbol $\%$ can be interpreted as 
``divided by 100'' (just as the symbol $\circ$ with respect to angles was interpreted in Module~\MNRef{VBKM05} as a multiplication 
by $\frac{\pi}{180}$).
\end{MInfo}


For example:
\begin{itemize}
\item{One percent is one hundredth: $1\,\%=\frac{1}{100}=0.01$}
\item{Ten percent is one tenth: $10\,\%=\frac{10}{100}=0.1$}
\item{25 percent is one quarter: $25\,\%=\frac{25}{100}=0.25$}
\item{One hundred percent is a whole: $100\,\%=\frac{100}{100}=1$}
\item{150 percent is a factor of 1.5: $150\,\%=\frac{150}{100}=1.5$}
\end{itemize}

In general, percentages describe ratios and relate to a certain base value. The base value is the initial 
value the percentage relates to. The percentage is expressed in percent and denotes a ratio with respect to
the base value. The real value of this quantity is called the percent value. The percent value has the same 
unit as the base value. 

\begin{MInfo}
The rule of three applies for the percent value, base value and percentage :
$$
\text{percentage}\;\cdot\;\text{base value}\;\; =\;\;\text{percent value}\: .
$$
\end{MInfo}

\end{MXContent}

\begin{MXContent}{Calculation of Interest}{Calculation of Interest}{STD}
\MDeclareSiteUXID{VBKM_STOCH_2_3}

In the calculation of interest, we distinguish between simple interest and
compound interest. For simple interest, the interest is paid at the end of the 
interest period. For compound interest, the interest is also paid on the previously accumulated interest.

\begin{MInfo}
When simple interest is applied, a quantity $K$ that increases by $p\,\%$ every year will increase after $t$ years 
($t\in \N$) to
$$
K_{t}\; =\; K\cdot \left({ 1+ t\cdot \frac{p}{100}}\right).
$$

\end{MInfo}

Note that $p$ itself can be a decimal, for example, for $p=2.5$, the percent value is $2.5\%=0.025$. 

\begin{MExercise}
What is the final capital for an initial capital of $K=4{,}000$~EUR after an interest period of 
$t=10$~years, when simple interest is applied at a rate of $p=2.5\%$ p.\,a.?
\ \\ \ \\
Answer: \MEquationItem{$K_{10}$}{\MLParsedQuestion{10}{5000}{4}{STOCHPRO1} EUR}.
\ \\ \ \\
\begin{MHint}{Solution}
Substituting the corresponding values into the interest formula for simple interest results in
$$
K_{10}\; =\; K\cdot \left({ 1 + 10\cdot 0.025}\right) =4{,}000\; \text{EUR} \cdot 1.25\; =\; 5{,}000\;\text{EUR}\: .
$$
\end{MHint}
\end{MExercise}

\begin{MExercise}
What is the initial capital $K$ that had been deposited at the 1$st$ of January, 2000 to get paid 
an end capital of $K_{12}=10{,}000$ EUR at the 31$st$ of December, 2011 when simple interest is applied
at a rate of $p=5\%$ p.\,a.?
\ \\ \ \\
Answer: \MEquationItem{$K$}{\MLParsedQuestion{10}{6250}{4}{STOCHPRO2} EUR}.
\ \\ \ \\
\begin{MHint}{Solution}
Substituting the corresponding values into the interest formula for simple interest results in
$$
K_{12}\; =\; 10{,}000\;\text{EUR}\;  = \; K\cdot \left({1+12\cdot 0.05}\right)\; =\; K\cdot 1.6\: .
$$
Solving this equation for $K$ results in
$$
K\; =\; \frac{10{,}000\; \text{EUR}}{1.6}\;  =\; \frac{10{,}0000\;\text{EUR}}{16}\;=\;  6{,}250\;\text{EUR after cancelling the fraction}\: .
$$
\end{MHint}
\end{MExercise}

While simple\MEntry{interest}{interest (simple)} is simply paid after an interest period,  
compound interest carries over into the next interest period, i.e. the interest will be 
added to the initial capital or will be \MEntry{capitalised}{interest (capitalised)}:

\begin{MExample}
For a bank account at the end of an interest period, an initial capital of $1{,}000$ EUR is deposited 
at an interest rate of $8\%$. After one year the deposit (in EUR) in the bank account is 
\begin{itemize}
\item{$1{,}000+\frac{1{,}000\cdot 8}{100}=1{,}000\cdot \left(1+\frac{8}{100} \right)=1{,}000\cdot 1.08=1{,}080$.}
\item{This deposit is invested for an additional year at the same interest rate of $8\%$. Then, the deposit (in EUR) 
after two years is $1{,}080\cdot 1.08=1{,}000\cdot 1.08^{2}=1{,}000\cdot \left( 1+\frac{8}{100}\right)^{2}$.}
\item{The deposit increases by a factor of $1.08$ per year. Hence, the deposit (in EUR) after $t$ years ($t\in\N_0$) is
$$
1{,}000\cdot 1.08^{t} \; =\; 1{,}000\cdot \left( 1+\frac{8}{100}\right)^{t}.
$$}
\end{itemize}
\end{MExample}

Thus, the compound interest is based on the following formula:

\begin{MInfo}
A quantity $K$ that increases every year by an amount of $p\,\%$ will have been increased after $t$ years ($t\in \N_0$) to
$$
K\cdot \left( 1+ \frac{p}{100}\right)^{t}\: .
$$
Here, $1+\frac{p}{100}$ is called the growth factor for a growth of $p\,\%$.
\end{MInfo}

In an advert offering deposit accounts or loans, the interest is usually given as a rate per year, even if the 
actual interest period differs. This interest period is the time between two successive dates at which the interest payments 
are due. On a deposit account the interest period is one year, though it is becoming more common to offer 
other interest periods. For example, for short-term loans the interest is paid daily or monthly.

If a bank offers a yearly interest rate of $9\,\%$ with monthly interest payments, then 
at the end of each month $\left(\frac{1}{12} \right)\cdot 9\% =0,75\%$ of the capital 
will be credited.  

\begin{MInfo}
The yearly rate is divided by the number of interest periods to obtain the 
periodic rate (the interest rate per period).
\end{MInfo}

Suppose an investment of $S_{0}$~EUR yields $p\,\%$ interest per interest period. After 
$t$~periods ($t\in \N_0$) the investment will have been increased to
$$
S_{t} \;=\; S_{0}\cdot (1+r)^{t} \;\;\text{with}\;\; r\;=\;\frac{p}{100}\: .
$$
In every period the investment increases by a factor of $1+r$, and we say ``the 
\MEntry{interest rate}{interest rate} equals $p\,\%$'' or ``the 
\MEntry{periodic rate}{periodic rate} equals $r$''. Suppose interest is credited  at a rate of 
$\frac{p}{n}\,\%$ to the capital at $n$ different times, more or less evenly distributed 
over the year.  Then, the capital is multiplied by a factor of 
$$
\left(1+\frac{r}{n}\right)^{n}
$$
every year. After $t$~years the capital has increased to
$$
S_{0}\cdot \left(1+\frac{r}{n} \right)^{n\cdot t}\: .
$$

\begin{MExample}
A capital of $5{,}000$ EUR is deposited for $t=8$~years in a bank account at a yearly interest rate of 
$9\,\%$, where the interest is paid quarterly. The periodic rate $\frac{r}{n}$ here is 
$$
\frac{r}{n} \;=\;\frac{0.09}{4} \; =\; 0.0225 \: ,
$$
and for the number of periods $n\cdot t$ we have 
$n\cdot t = 4\cdot 8=32$. Thus, after $t=8$~years the deposit has increased to
$$
5000\cdot (1+0.0225)^{32}\; \approx \;10190.52 \text{ EUR}\: .
$$
\end{MExample}


\begin{MExercise}
A capital of $K_0=8{,}750$ EUR is deposited for $t=4$~years at an interest rate of $p=3,5\%$ p.\,a., and
the interest is capitalised.
\begin{MExerciseItems}
\item{After one year the amount of capital is \MEquationItem{$K_1$}{\MLParsedQuestion{10}{9056.25}{7}{STOCHPRO7}}.}
\item{After two years the amount of capital is \MEquationItem{$K_2$}{\MLParsedQuestion{10}{9373.22}{7}{STOCHPRO8}}.}
\item{After three years the amount of capital is \MEquationItem{$K_3$}{\MLParsedQuestion{10}{9701.28}{7}{STOCHPRO9}}.}
\item{The final capital is \MEquationItem{$K_4$}{\MLParsedQuestion{10}{10040.83}{7}{STOCHPRO10}}.}
\end{MExerciseItems}
Specify all values rounded mathematically to the second fractional digit. Round only \textit{after}
carrying out the calculations. For these calculations, you are allowed to use a calculator.


\begin{MHint}{Solution}
Substituting the values into the compound interest formula results in
\begin{eqnarray*}
K_1 &=& K_0\cdot \left({1+0.035}\right)^1 \;=\; 9056.25\\
K_2 &=& K_0\cdot \left({1+0.035}\right)^2 \;=\; 9373.22\;\text{(rounded)}\\
K_3 &=& K_0\cdot \left({1+0.035}\right)^3 \;=\; 9701.28\;\text{(rounded)}\\
K_4 &=& K_0\cdot \left({1+0.035}\right)^4 \;=\; 10040.83\;\text{(rounded)}.
\end{eqnarray*}
The above mentioned \MSRef{L_STOCH_RP}{error propagation for rounding} requires calculating the power first and then 
rounding. For example, it is wrong to multiply the rounded value $K_3$ by $1.035$ to obtain $K_4$.
\end{MHint}
\end{MExercise}

A consumer who wants to take out a loan always faces several offers from competing banks. Thus, it is extremely useful to 
compare the different offers.

\begin{MExample}
Let us consider an offer providing an yearly interest rate of $9\%$, where the interest is charged at
a monthly ($12$ times per year) rate of $0.75\%$. If no interest is paid off in the meantime, 
the initial debt will increase to
$$
S_{0}\cdot \left({1+\frac{0.09}{12} }\right)^{12}\; \approx \; S_{0}\cdot 1.094
$$
after one year. The interest to be paid off is approximately
$$
1.094 \cdot S_{0}-S_{0}\; =\; 0.094 \cdot S_{0}\: .
$$
\end{MExample}

As long as no interest is paid off, the debt will increase at a constant rate which is approximately 
$9.4\%$ per year. This is why we may speak of an ``effective'' annual interest rate. In the example above the 
effective annual interest rate is $9.4\%$.


\begin{MInfo}
If interest is paid $n$ times per year at a periodic rate of $\frac{r}{n}$ per period, then the 
\MEntry{effective annual interest rate}{interest (effective)} $R$ is defined by 
$$
R\; =\;\left({ 1+\frac{r}{n}}\right)^{n}-1\: .
$$
\end{MInfo}

\end{MXContent}

\begin{MXContent}{Continuous Compounding Interest}{Continuous Compounding Interest}{STD}
\MDeclareSiteUXID{VBKM_STOCH_2_4}

The expression $a_{n}=\left({1+\frac{r}{n}}\right)^{n}$ with $r\in\R$ can also be interpreted 
as a \MSRef{VBKM06_Abbildungen}{map} depending on $n\in\N$
$$
a:\; \N\:\longrightarrow\: \R\;\; , \;\;  n\:\longmapsto\: a(n)\; =\; a_{n}\; =\;\left({ 1+\frac{r}{n}}\right)^{n}\: .
$$
A map $\N \ni n \mapsto a_{n}\in \R$ is called a real \MEntry{sequence}{sequence}. The pairs $(n,a_{n})$ 
can be interpreted as points in the Euclidean plane. In this sense, the sequence $a_{n}=\left({1+\frac{0.4}{n}} \right)^{n}$ 
is shown in the figure below as a sequence of points in the Euclidean plane.

\begin{center}
\MGraphicsSolo{exp1.png}{scale=1}\MLabel{G_VBKM_STOCH_EXP1}
\end{center}

Two properties of this sequence can immediately be seen from the figure above:
\begin{itemize}
\item{The sequence $a_{n}$, $n\in \N$ is monotonically increasing, i.e. for $i\leq j$ $a_i\leq a_j$, for all $i,j\in \N$.}
\item{The sequence approaches the value $a\in \R$ as $n\in \N$ increases. This number $a$ is called limit of the sequence 
$a_{n}$, and is written
$$
\lim\limits_{n\rightarrow \infty} a_{n}\; =\;a\: .
$$}
\end{itemize}
In the lecture mathematics 1, the natural \MSRef{M06_e_fkt}{exponential function}
$$
\exp:\;\R\:\longrightarrow\: \R\;\; , \;\; x\:\longmapsto\: \exp(x)\,=\; e^x
$$
will be studied in detail.


\begin{center}
\MGraphicsSolo{exp2.png}{scale=1}\MLabel{G_VBKM_STOCH_EXP2}\\
The natural exponential function
\end{center}

There, the following statement will be shown:

\begin{MInfo}
For an arbitrary number $x\in \R$, we have
$$
\lim\limits_{n\rightarrow \infty}\left({1 + \frac{x}{n}} \right)^{n}\; =\; e^{x}\: .
$$
\end{MInfo}

For $x=1$, the limit of this sequence is Euler's number (named after the Swiss mathematician Leonhard Euler, 1707--1783):
$$
\lim\limits_{n\rightarrow \infty}\left({1 + \frac{1}{n}} \right)^{n}\; =\; e\;  \approx \; 2.7182\ldots \: .
$$
It can be shown (with some difficulty) that Euler's number $e$ is an irrational number, and hence it cannot be written as a fraction.

The \MSRef{VBKM01_Potenzgesetze}{exponent rules} apply to the natural exponential function with arbitrary real 
numbers as its exponents:
\begin{itemize}
\item{$\exp(x+y)=e^{x+y}=e^{x}\cdot e^{y}=\exp (x)\cdot \exp (y)$ for $x,y\in\R$.}
\item{$\exp (x\cdot y)=e^{x\cdot y}=\left( e^{x} \right)^{y}= \left( e^{y} \right)^{x}$ for $x,y\in\R$.}
\end{itemize}

Information on the compound interest process can be gained if the number of times $n$ gets very large
using the exponential function and the relation to the sequence $(1+\frac{x}{n})^n$:
the capital is multiplied by a factor of $\left(1+\frac{r}{n} \right)^{n}$ every year if the interest at a rate of 
$\frac{r}{n}$ is credited to the initial capital $S_{0}$ at $n$ different times in the year. After $t$~years, 
$t\in\N$, the initial capital has increased to
$$
S_{0}\cdot \left({1+ \frac{r}{n}}\right)^{n\cdot t}\: .
$$
If $n\rightarrow \infty$, the limit of this sequence is
$$
\lim\limits_{n\rightarrow \infty} \left({S_{0}\cdot \left({1+\frac{r}{n}} \right)^{n\cdot t}}\right) \; =\; S_{0}\cdot e^{r\cdot t}\: .
$$
For increasing $n\in \N$ the interest is paid more and more frequently:

\begin{MInfo}
The limiting case is called the \MEntry{continuous compounding interest}{interest (continuous)}. 
For positive real numbers $t$, the formula
$$
s(t)\; =\; S_{0}\cdot e^{r\cdot t}
$$
specifies to which amount an initial capital $S_{0}$ has increased after $t$~years
if continuous compounding interest is applied at a rate $r$ per year.
\end{MInfo}

\begin{MExample}
An investment of $5{,}000$ EUR is deposited for $t=8$~years in a bank account where continuous compounding interest is applied 
at a yearly interest rate of $9\,\%$. After $t=8$~years, this results in an investment of
$$
5{,}000\cdot e^{0.09\cdot 8}\; =\; 5{,}000\cdot e^{0.72}\; \approx\; 10{,}272.17 \text{ EUR}\: .
$$
\end{MExample}

\end{MXContent}

\begin{MXContent}{Types of Diagrams}{Types of Diagrams}{STD}
\MDeclareSiteUXID{VBKM_STOCHASTIK_Diagrammarten}

Qualitative and quantitative discrete data gained from a sample are often presented graphically by 
\MEntry{bar charts}{bar chart}.

\begin{MInfo}
The bar chart shows the absolute or relative frequencies as a function of a finite number of property values 
in the sample. The bar lengths are proportional to the values they represent.
\end{MInfo}

This is now illustrated by an example. The species of $10$ trees at the forest's edge was determined. 
The possible characteristic attributes are:
\begin{eqnarray*}
a_{1}& =& \text{Oak} \;\; , \\
a_{2}& =& \text{Beech}\;\; , \\
a_{3}& =& \text{Spruce}\;\; , \\
a_{4}& =& \text{Pine, etc.}\: .
\end{eqnarray*}
A sample resulted in the following original list:

\begin{center}
\begin{tabular}{|c|c|c|c|c|c|c|c|c|c|c|}
\hline
$i$ & $1$ & $2$ & $3$ & $4$ & $5$ & $6$ & $7$ & $8$ & $9$ & $10$ \\ \hline
$x_i$ & $a_2$ & $a_1$ & $a_1$ & $a_3$ & $a_1$ & $a_2$ & $a_1$ & $a_1$ & $a_3$ & $a_3$ \\ \hline
\end{tabular}
\end{center}

This original list results in the following  empirical frequency table:

\begin{center}
\begin{tabular}{|c|c|c|c|}
\hline
Attribute & absolute & relative & in $\%$\\ \hline
Oak & $5$ & $0.5$ & 50 \\ \hline
Beech & $2$ & $0.2$ & 20 \\ \hline
Spruce & $3$ & $0.3$ & 30 \\ \hline
\end{tabular}
\end{center}

The bar chart corresponding to this empirical frequency table is shown in the figure below. 

\begin{center}
\MUGraphicsSolo{Stab_450_0.png}{width=0.4\linewidth}{width:450px}\\
Bar chart
\end{center}

Qualitative properties are often represented by \MEntry{pie charts}{pie chart}:

\begin{MInfo}
A slice is assigned to each characteristic attribute according to its relative frequency, where
$$
h_{j} \; =\; \frac{H_{j}}{n}\;=\;  \frac{\alpha_{j}}{360^{\circ}}\: .
$$
Here, $\alpha_j$ is the \MSRef{VBKM05_Winkel}{angle} (in degree measure) of the slice (circular sector) that corresponds
to the attribute $j$ within the original list $x=(x_{1},x_{2},\ldots ,x_{n})$.
\end{MInfo}

This is again illustrated by an example. 

A number $n=1000$ of households were queried as to how satisfied they were with a new kind of barbecue. The possible answers were:
very satisfied (1), satisfied (2), less satisfied (3) and not satisfied (4). 
\ \\ \ \\
The survey resulted in the following empirical frequency table.

\begin{center}
\begin{tabular}{|c|c|c|c|}
\hline
Attribute & Absolute frequencies & Relative frequencies & Percentage \\ \hline
Very satisfied &  $100$ & $0.1$ & $10\%$ \\ \hline
Satisfied & $240$ & $0.24$ & $24\%$ \\ \hline
Less satisfied &$480$ & $0.48$ & $48\%$\\ \hline
Not satisfied & $180$ & $0.18$ & $18\%$ \\ \hline
Sum & $1000$ &  $1$ & $100\%$\\ \hline
\end{tabular}
\end{center}

The corresponding angles are, according to the Info Box above,

\begin{itemize}
\item{$\alpha_{1}=360^{\circ}\cdot 0.1=36^{\circ}$,}
\item{$\alpha_{2}=360^{\circ}\cdot 0.24=86.4^{\circ}$,}
\item{$\alpha_{3}=360^{\circ}\cdot 0.48=172.8^{\circ}$,}
\item{$\alpha_{4}=360^{\circ}\cdot 0.18=64.8^{\circ}$.}
\end{itemize}

This results in the following pie chart:

\begin{center}
\MUGraphicsSolo{Kreis_1.png}{width=0.3\linewidth}{width:500px}
\end{center}

It is often pointless to present all possible attributes in a diagram. It is more convenient to
classify them and draw only the frequencies of the classes into a diagram. This is the only way 
to visualise the frequencies of continuous characteristics in a bar or pie chart.

Let $X$ be a quantitative (continuous) property, and $x=(x_{1},x_{2},\ldots ,x_{n})$ the original list for a sample 
of size $n$. An empirical frequency distribution is obtained according to the following approach:
\begin{itemize}
\item{Find the minimum and the maximum sample value, i.e. 
$$
x_{(1)} \; =\; \min\lbrace x_{1},x_{2},\ldots ,x_{n}\rbrace\;\; \text{and}\;\;
x_{(n)} \; =\; \max\lbrace x_{1},x_{2},\ldots ,x_{n}\rbrace\: .
$$}
\item{List these and all values in between, rounded to the required fractional digit and sorted by size. This converts 
the (continuous) property $X$ into a discrete property.}
\item{Prepare a tally sheet and draw the corresponding empirical frequency distribution.}
\end{itemize}

The empirical frequency distribution of a continuous property can be very broad. In particular
zeros may appear, caused by measurement values that do not occur in the original list (sample). 
Due to this, the empirical frequency table gets very confusing and bulky. 
Hence, a \MEntry{classification}{classification} is carried out to reduce the amount of data (data reduction).
In fact, this corresponds to a reduction of measurement accuracy (rounding!).

\begin{MInfo}
\MEntry{Classes}{classes} are half-open intervals of the form
$$
(a;b]\; =\; \lbrace x\in\R\:  : \: a<x\leq b\rbrace\;\; \text{with}\;\; a,b\in \R\cup\lbrace\pm \infty\rbrace\: .
$$
\end{MInfo}

There is no general rule defining the number $k$ of classes or the size of a class. However, the following 
guidelines are recommended:
\begin{itemize}
\item{Uniform classification: Find $x_{(1)}=\min\lbrace x_{1},x_{2},\ldots ,x_{n}\rbrace$ and
$x_{(n)}=\max\lbrace x_{1},x_{2},\ldots ,x_{n}\rbrace$. Then divide the interval 
$(x_{(1)}-\varepsilon;x_{(n)}+\varepsilon]$ with small $\varepsilon >0$ into $k$ uniform, non-overlapping, 
half-open subintervals.}
\item{Avoid classes that are too small or too large.}
\item{If possible, avoid classes with only a few observations.}
\item{Find approximately $k\approx \sqrt{n}$ equally sized classes, where $n$ is the number of samples.}
\end{itemize}

\begin{MInfo}
The \MEntry{histogram}{histogram} is used for the graphical representation of quantitative data (variable). It represents 
the relative frequencies of the data in the class $(a,b]$ by a rectangle with base $(a,b]$ whose area 
represents the class.
\end{MInfo}

A histogram is obtained through the following approach: let 
$$
x\;=\; (x_{1},x_{2},\ldots x_{n})
$$
be an original list for a sample of size $n$ of a quantitative property $X$.
\begin{itemize}
\item{Use a classification into $k$~classes. Let the interval of the $j$th class $j=1,2,\ldots ,k$ be 
$(t_{j};t_{j+1}]$.}
\item{Let $H_{j}$ be the number of sample values in the interval $(t_{j};t_{j+1}]$ for $j=1,2,\ldots ,k$.
The numbers $H_{j}$ are also called absolute class frequencies.}
\item{For each $j\in \{1,2,\ldots ,k\}$ draw a rectangle over the base $(t_{j};t_{j+1}]$ of height 
  $d_{j}$ with the area $d_{j}\cdot (t_{j+1}-t_{j})=h_{j}=\frac{H_{j}}{n}$.  The areas $h_{j}$ are 
  the relative frequencies.}
\end{itemize}

The total area of these rectangles equals $1$. 

This approach is now illustrated by a detailed example. In a data centre, the processing time (in s, rounded 
to one fractional digit) of $20$~program jobs was determined. This resulted in the following original 
list of a sample of size $n=20$:
\begin{center}
\begin{tabular}{|c|c|c|c|c|}
\hline
3.9 & 3.3 & 4.6 & 4.0 & 3.8\\ \hline
3.8 & 3.6 & 4.6 & 4.0 & 3.9\\ \hline
3.9 & 3.9 & 4.1 & 3.7 & 3.6\\ \hline
4.6 & 4.0 & 4.0 & 3.8 & 4.1\\ \hline
\end{tabular}
\end{center}

The smallest value is $3.3$~s, the largest value is $4.6$~s, the increment is $0.1$~s. According to the guidelines above, 
we should find approximately $k\approx \sqrt{n}$ equally sized classes. Here, we use the following
classification into $k=4$ classes.


\begin{center}
\begin{tabular}{|c|c|c|}
\hline
Class & $(t_{j};t_{j+1}],\;j=1,2,3,4$ & Data \\ \hline
Class 1 & $(3.25;3.65]$ &  ``From $3.3$ to $3.6$'' \\ \hline
Class 2 & $(3.65;3.95]$ &  ``From $3.7$ to $3.9$'' \\ \hline
Class 3 & $(3.95;4.25]$ &  ``From $4.0$ to $4.2$'' \\ \hline
Class 4 & $(4.25;4.65]$ &  ``From $4.3$ to $4.6$'' \\ \hline
\end{tabular}
\end{center}

The table of the absolute and relative frequencies has the following form:

\begin{center}
\begin{tabular}{|c|c|c|}
\hline
Class & abs. Class frequency $H_{j}$ & rel. Class frequency $h_{j}$ \\ \hline
Class 1 & $3$ & $0.15$ \\ \hline
Class 2 & $8$ & $0.4$ \\ \hline
Class 3 & $6$ & $0.3$ \\ \hline
Class 4 & $3$ & $0.15$ \\ \hline
\end{tabular}
\end{center}

The heights of the $k=4$ rectangles are as follows:
\begin{itemize}
\item{Class 1: $d_{1}\cdot(t_{2}-t_{1})=d_{1}\cdot 0.4=h_{1}=0.15$, i.e. $d_{1}=\frac{3}{8}=0.375$.}
\item{Class 2: $d_{2}\cdot(t_{3}-t_{2})=d_{2}\cdot 0.3=h_{2}=0.4$, i.e. $d_{2}=\frac{4}{3}=1.\overline{3}$.}
\item{Class 3: $d_{3}\cdot(t_{4}-t_{3})=d_{3}\cdot 0.3=h_{3}=0.3$, i.e. $d_{3}=1$.}
\item{Class 4: $d_{4}\cdot(t_{5}-t_{4})=d_{4}\cdot 0.4=h_{4}=0.15$, i.e. $d_{4}=\frac{3}{8}=0.375$.}
\end{itemize}

Thus, we have the following histogram:

\begin{center}
\MUGraphicsSolo{histogramm_500_0.png}{width=0.3\linewidth}{width:500px}
\end{center}

\end{MXContent}

\MSubsection{Statistical Measures}
\MLabel{VBKM_STOCH_3}

\begin{MIntro}
\MLabel{VBKM_STOCH_3_1}
\MDeclareSiteUXID{VBKM_STOCH_3_1}

Suppose a sample of size $n$ is given for some quantitative property $X$. Let the original list be given by
$$
x\; =\; (x_{1},x_{2},\ldots ,x_{n})\: .
$$

\begin{MInfo}
\MLabel{L_Mittelwert}
The \MEntry{arithmetic mean}{mean (arithmetic)} $\overline{x}$ (also called sample mean) of $x_{1},x_{2},\ldots ,x_{n}$ is defined as
$$
\overline{x}\; =\; \frac{1}{n}\cdot \sum\limits_{k=1} ^{n} x_{k}\;=\;\frac{x_{1}+x_{2}+\ldots +x_{n}}{n}\: .
$$
\end{MInfo}

In physical terms, $\overline{x}$ describes the centre of mass of a mass distribution given by equal masses 
at $x_{1},x_{2},\ldots ,x_{n}$ on the massless number line.


\begin{MExample}
We have the following original list for a sample of size $n=20$:
\begin{center}
\begin{tabular}{|c|c|c|c|c|}
\hline
10 & 11 & 9 & 7 & 9 \\ \hline
11 & 22 & 12 & 13 & 9 \\ \hline
11 & 9 & 10 & 12 & 13 \\ \hline
12 & 11 & 10 & 10 & 12\\ \hline
\end{tabular}
\end{center}

The investigated property could be, for example, the length of study (measured in semesters) of $20$~mathematics 
students at the TU Berlin. Summing up the values results in
$$
\sum\limits_{k=1}^{20}x_{k}\; =\;223\: ,
$$
so for the arithmetic mean we have
$$
\overline{x}\;=\; \frac{1}{20}\cdot \sum\limits_{k=1}^{20}x_{k}\;=\; \frac{223}{20}\; =\; 11.15\: .
$$
\end{MExample}

The arithmetic mean is rather sensitive to so-called statistical outliers: measurement values that 
vary strongly form the other data can significantly affect the arithmetic mean.

\begin{MExample}
Let us again consider the original list for the sample of size $n=20$ above. If we drop the value
$x_{7}=22$, then for the arithmetic mean of the remaining $19$ data values we have
$$
\frac{1}{19}\cdot \sum\limits_{k=1,k\neq 7}^{n}x_{k}\;=\; \frac{201}{19}\; \approx\; 10.58\: .
$$
\end{MExample}

If a multiplicative or relative relation exists among the values in an original list (for example, 
for growth processes or continuous compounding interest), the arithmetic (additive) mean is not an 
appropriate measure. For such data values, the geometric mean is used:

\begin{MInfo}
Let data of the form $x_{1}>0,\;x_{2}>0,\;\ldots ,x_{n}>0$ be given. 
Then, the \MEntry{geometric mean}{mean (geometric)} $\overline{x}_{G}$ of $x_{1},x_{2},\ldots ,x_{n}$ is given by
$$
\overline{x}_{G}\; =\; \sqrt[n]{x_{1}\cdot x_{2}\cdot \ldots \cdot x_{n}}\: .
$$
\end{MInfo}

\begin{MExample}
Let us consider a population that consists of $50$~animals at time $t_{0}$. Every two years, the number 
of animals is counted again.

\begin{center}
\begin{tabular}{|l|l|l|l|l|}
\hline
Year & & Number of animals & & Growth rate \\ \hline
$t_{0}$ & & 50 \\ \hline
$t_{0}+2$ & & 100 & & doubled ($x_{1}=2$)\\ \hline
$t_{0}+4$ & & 400 & & quadrupled  ($x_{2}=4$)\\ \hline
$t_{0}+6$ & & 1200 & & tripled ($x_{3}=3$)\\ \hline
\end{tabular}
\end{center}

For the (geometric) mean growth rate, we have
$$
\overline{x}_G\; =\; \sqrt[3]{2\cdot 4\cdot 3}\; =\; \sqrt[3]{24}\; \approx\; 2.8845\: .
$$
\end{MExample}

This example illustrates that applying the arithmetic mean to growth processes gives misleading results. We would get
$$
\overline{x}\; =\;\frac{1}{3}\cdot (2+4+3)\; =\; \frac{9}{3}=\; 3\: .
$$
However, a theoretical tripling of the population size every two years would imply 
that the number of animals after six years would be $1{,}350$ which is obviously not the case. 
From an average growth rate of $2.8845$, we obtain the correct result: $50\cdot (2.8845)^{3}\approx 1{,}200$.

\begin{MExercise}
The growth rates per year of an investment are as follows:
\begin{center}
\begin{tabular}{|c|c|c|c|c|c|}
\hline
Year         & 2011 & 2012 & 2013 & 2014 & 2015 \\ \hline
Growth rate  & $0.5\%$ & $1.1\%$ & $0.8\%$ & $1.2\%$ &  $0.7\%$ \\ \hline
\end{tabular}
\end{center}
Calculate the mean growth rate over five years in percent:
\MEquationItem{$\overline{x}_G$}{\MLParsedQuestion{10}{0.82}{7}{STOCHGEOMM}} $\%$, 
rounded mathematically to two fractional digits.
\ \\ \ \\
In this exercise you are allowed to use a calculator.
\ \\ \ \\
\begin{MHint}{Solution}
Taking the geometric mean results in 
$$
\overline{x}_G\; =\; \sqrt[5]{0.5\cdot 1.1\cdot 0.8\cdot 1.2\cdot 0.7}\; =\; \sqrt[5]{0.3696}\; =\; 0.819495159191\ldots
$$
which is mathematically rounded to $0.82$.
\end{MHint}
\end{MExercise}

\end{MIntro}

\begin{MXContent}{Robust Measures}{Robust Measures}{STD}
\MLabel{VBKM_STOCH_3_2}
\MDeclareSiteUXID{VBKM_STOCH_3_2}

The measures presented in this section are robust with respect to outliers: large deviations 
of single data values do not affect this measures (or only affect it slightly).

Consider an original list
$$
x\; =\;(x_{1},x_{2},\ldots ,x_{n})
$$
for a sample of size $n$. Let the data $x_{i}$ be the property values of a quantitative property $X$. 

\begin{MInfo}

The list $x_{(\; )}=(x_{(1)},x_{(2)},\ldots ,x_{(n)})$ gained by ascending sorting
$$
x_{(1)}\;\leq\; x_{(2)}\;\leq \;\ldots\;\leq\; x_{(n)}
$$
of the original list is called an ordered list or ordered sample (of the original list $x$). 
The i$th$ entry $x_{(i)}$ in the ordered list is the $i$th smallest value in the original list.
\end{MInfo}

\begin{MExample}
Let us again consider the original list $x=(x_{1},x_{2},\ldots ,x_{20})$ for the sample of size 
$n=20$ from the examples above. Ascending sorting 
$x_{(\; )}=(x_{(1)},x_{(2)},\ldots ,x_{(20)})$  results in the following ordered sample:
$$
\begin{array}{cccccccccccccccccccc} 7 & 9 & 9 & 9 & 9 & 10 & 10 & 10 & 10 & 11 & 11 & 11 & 11 & 12 & 12 & 12 & 12 & 13 & 13 & 22 \end{array}
$$
\end{MExample}

\begin{MInfo}
\MLabel{L_Median}
The (empirical) \MEntry{median}{median} $\tilde{x}$ of $x_{1},x_{2},\ldots ,x_{n}$ is defined as
$$
\tilde{x}\;=\;\left\lbrace{\begin{array}{lll}x_{\left(\frac{n+1}{2}\right)} & \text{for} & n\;\text{ odd}\\
\frac{1}{2}\cdot \left( x_{(\frac{n}{2})}+ x_{(\frac{n}{2}+1)} \right) & \text{for} &n\;\text{ even\: .}\end{array}}\right.
$$
\end{MInfo}

In contrast to the arithmetic mean, the (empirical) mean is not sensitive to outliers. For example, the largest 
value in the ordered original list can be arbitrarily enlarged without changing the median. 

\begin{MExample}
In the example above, the sample size $n=20$ is even. Thus, we have for the median
$$
\tilde{x}\; =\; \frac{1}{2}\cdot \left({x_{(10)}+x_{(11)} }\right)\; =\;\frac{1}{2}\cdot (11 + 11)\; =\; 11\: .
$$
\end{MExample}

Approximately half of the values in the original list are less than or equal to the median, and half of the 
values are greater than or equal to the median $\tilde{x}$. This principle can be generalised to define quantiles. 
For this purpose, take an original list $x=(x_{1},x_{2},\ldots ,x_{n})$ for a sample of size $n$ of a 
quantitative property $X$.

\begin{MInfo}
Let
$$
x_{(\; )}\; =\; (x_{(1)},x_{(2)},\ldots ,x_{(n)})
$$
be the corresponding ordered sample and 
$$
\alpha \in (0,1) \;\;\text{and} \;\; k = \MTextSF{floor}(n\cdot \alpha) \;= \;\lfloor n\cdot \alpha \rfloor\: .
$$
Then 
$$
\tilde{x}_{\alpha }\; =\; \left\lbrace{\begin{array}{lll}x_{(k+1)} & \text{if} & n\cdot \alpha \notin \N \\ \frac{1}{2} \cdot \left(x_{(k)}+x_{(k+1)}\right) & \text{if} & n\cdot \alpha \in \mathbb{N}\end{array}}\right.
$$
is called a sample $\alpha$-quantile or simply $\alpha$-quantile of $x_{1},x_{2}\ldots ,x_{n}$.
\end{MInfo}

The $0.25$-quantile is also called the lower \MEntry{quartile}{quartile}. It splits off approximately 
the lowest 25\,\% of data values from the highest 75\,\%. Accordingly, the $0.75$-quantile is called 
the upper quartile. For $\alpha = 0.5$ we have the median, i.e. $\tilde{x}=\tilde{x}_{0.5}$.
If $\alpha \in (0,1)$, the ordered list $x_{1},x_{2},\ldots ,x_{n}$ is split so that 
approximately $\alpha\cdot 100\%$ of the data value are less or equal to $\tilde{x}_{\alpha}$ and 
approximately  $(1-\alpha)\cdot 100\%$ of the data values are greater or equal to $\tilde{x}_{\alpha}$.


\begin{MExample}
Consider again the original list $x=(x_{1},x_{2},\ldots ,x_{20})$ for the sample of size $n=20$ from the examples 
above together with the ordered sample $x_{(\; )}=(x_{(1)},x_{(2)},\ldots ,x_{(20)})$
$$
\begin{array}{cccccccccccccccccccc} 7 & 9 & 9 & 9 & 9 & 10 & 10 & 10 & 10 & 11 & 11 & 11 & 11 & 12 & 12 & 12 & 12 & 13 & 13 & 22 \end{array}
$$
For $\alpha = 0.25$, the $25\%$-quantile is defined by $n\cdot \alpha = \frac{20}{4}=5\in \N$, i.e. for the 
lower quartile we have
$$
\tilde{x}_{0.25}\;=\;  \frac{1}{2}\cdot\left( x_{(5)}+x_{(6)} \right)\; =\;\frac{1}{2}\cdot (9+10)\; =\; \frac{19}{2}\; =\; 9.5\: .
$$
For the upper quartile, we set $\alpha = 0.75$ and obtain 
$n\cdot \alpha = \frac{20\cdot 3}{4}=15\in \N$, hence
$$
\tilde{x}_{0,75}\;=\;\frac{1}{2}\cdot\left( x_{(15)}+x_{(16)} \right) \;=\;\frac{1}{2}\cdot (12+12)\;=\;12\: .
$$
\end{MExample}

again, let a sample of size $n$ be given to a quantitative property $X$ with the corresponding 
ordered sample  
$$
x_{(\; )}\;=\;(x_{(1)},x_{(2)},\ldots ,x_{(n)})
$$
and
$$
\alpha \in [0,\;0.5)\;\; \text{and}\;\;k\;=\;\MTextSF{floor}(n\cdot \alpha) \;= \;\lfloor n\cdot \alpha \rfloor \: .
$$

\begin{MInfo}
The $\alpha$-trimmed (or $\alpha$-truncated) sample mean is defined as
$$
\overline{x}_{\alpha}\;=\;
\frac{1}{n-2\cdot k} \cdot \sum\limits_{j=k+1}^{n-k}x_{(j)}\;=\; \frac{1}{n-2\cdot k}\cdot \left(x_{(k+1)}+ \ldots + x_{(n-k)} \right)\: .
$$
\end{MInfo}

The $\alpha$-trimmed mean is an arithmetic mean that discards the $\alpha \cdot 100\%$ largest 
and $\alpha \cdot 100\%$ smallest data points from the calculation. Thus, it is a flexible protection tool 
against outliers at the boundaries of the data range. However, we mustn't forget that we no longer take all data into account when we use this tool.

\begin{MExample}
In the already much considered data set, the ordered sample 
 $x_{()}=(x_{(1)},x_{(2)},\ldots ,x_{(20)})$ is given by 
$$
\begin{array}{cccccccccccccccccccc} 7 & 9 & 9 & 9 & 9 & 10 & 10 & 10 & 10 & 11 & 11 & 11 & 11 & 12 & 12 & 12 & 12 & 13 & 13 & 22 \: ,\end{array}
$$
and for
$\alpha = 0.12$ and $k=\lfloor 20\cdot 0.12 \rfloor = \lfloor 2.4 \rfloor =2$ we obtain for the $12\%$-trimmed mean of the sample
$$
\overline{x}_{0.12} \;=\; \frac{1}{16}\cdot \sum\limits_{j=3}^{18}x_{(j)}\;=\; \frac{1}{16}\cdot 172\;=\;10.75\: .
$$
It is less than the arithmetic mean $\overline{x}=11.15$ since outliers, such as $x_{(20)}=22$, were ignored.
\end{MExample}

\end{MXContent}

\begin{MXContent}{Measures of Dispersion}{Measures of Dispersion}{STD}
\MLabel{VBKM_STOCH_3_3}
\MDeclareSiteUXID{VBKM_STOCH_3_3}

Means and quantiles are measures of position, i.e. they give information on the absolute position of the 
qualitative values $x_j$. If we add a constant $c$ to every value $x_j$, then the position measures also increase 
by $c$. In contrast, measures of dispersion are measures that give information on the dispersion or relative 
distribution of the data values independent of their absolute position. Consider a sample of size $n\geq 2$ 
of a quantitative property $X$. Let the original list be given by $x=(x_{1},x_{2},\ldots ,x_{n})\in \mathbb{R}^{n}$.

\begin{MInfo}
\MLabel{L_Varianz}
The \MEntry{sample variance}{sample variance} of the original list is defined as
$$
s_{x}^{2}\;=\; \frac{1}{n-1}\cdot \sum\limits_{k=1}^{n}(x_{k}-\overline{x})^{2}\; =\; \frac{(x_{1}-\overline{x})^{2}+ \ldots +(x_{n}-\overline{x})^{2}}{n-1}\: .
$$
The \MEntry{sample standard deviation}{sample standard deviation} is defined by $s_{x}=+\sqrt{s_{x}^{2}}$.
\end{MInfo}

The sample variance is a measure of dispersion that describes the variability of the observation sample. 
The smaller the variance the ``closer'' the data values lie to each other. A variance $s_x^2=0$ is only 
possible if all data values are equal. Typically, it strongly increases with increasing $n$. The  
standard deviation is a more appropriate measure for the ``broadness'' of the distribution of data values. 
The two formulas given above have a few pitfalls:

\begin{itemize}
\item{Before the variance can be calculated the mean $\overline{x}$ must already be known.}
\item{The fact that in the definition of $s_{x}^{2}$ is divided by $n-1$ and not by $n$ is for 
deeper mathematical reasons that can only be discussed in a statistics lecture.}
\item{The notation $s_{x}=+\sqrt{s_{x}^{2}}$ is a little misleading. You must not cancel the square by 
the square root, since the \textit{sum} $s^2_x$ must be calculated (and this value is not defined as a single square)
to determine $s_x$.}
\item{Be careful using a scientific calculator with statistical functions: the sample variance 
is available via the $s^2$ key. The $\sigma^2$ key, however, provides the sum with denominator 
$n$ instead of $n-1$. This is not the sample standard deviation.}
\end{itemize}

\begin{MExample}
The data sequence $x=(-1,0,1)$ has the mean $\overline{x}=0$ and the sample standard deviation 
$$
s^2_x \;=\;\frac{1}{n-1}\cdot \sum\limits_{k=1}^{n}(x_{k}-\overline{x})^{2}\;=\; \frac1{3-1}\cdot \left({(-1-0)^2+(0-0)^2+(1-0)^2}\right)\;=\; 1\: .
$$
Adding further zeros to the data sequence does not change the position measure $\overline{x}$, but the 
measure of deviation $s^2_x$,does change since the data values here are more strongly concentrated at the mean. 
In contrast, shifting all data values by a constant does not change the variance. For example, the data 
sequence $(-5,-4,-3)$ has also variance $1$.
\end{MExample}

\begin{MExercise}
A data sequence (with an unknown number $n$ of values) has the measures $\overline{x}=4$, $s^2_x=10$, 
and the median $\tilde x=3$. Suppose the values of a second data sequence satisfy the equation 
$y_k = (-2)\cdot x_k$ for every $k$. What are its measures?
\ \\ \ \\
Answer: the measures are \MEquationItem{$\overline{y}$}{\MLParsedQuestion{7}{-8}{3}{STOCHVAR1}}, \MEquationItem{$s_y^2$}{\MLParsedQuestion{7}{40}{3}{STOCHVAR2}},
and \MEquationItem{$\tilde y$}{\MLParsedQuestion{7}{-6}{3}{STOCHVAR3}}.
\ \\ \ \\
Hint: recall the definitions of the \MSRef{L_Mittelwert}{mean}, the \MSRef{L_Varianz}{sample variance}, and 
the \MSRef{L_Median}{median} consider how multiplying all $x$-values by a factor of $(-2)$ influences the entire expression.
\ \\ \ \\
\begin{MHint}{Solution}
Substituting the new $x$-values results in
\begin{eqnarray*}
\overline{y} &=& \frac1n\sum_{k=1}^n y_k \;=\;\frac1n\sum_{k=1}^n (-2)\cdot x_k \;=\; (-2)\cdot \frac1n\sum_{k=1}^n y_k \;=\; (-2)\cdot \overline{x} \;=\; -8\:,\\
s^2_y &=& \frac1{n-1}\sum_{k=1}^n \left({y_k-\overline{y}}\right)^2 \;=\;\frac1{n-1}\sum_{k=1}^n \left({(-2)x_k-(-2)\overline{x}}\right)^2\ \\
&=& \frac{(-2)^2}{n-1}\sum_{k=1}^n \left({x_k-\overline{x}}\right)^2 \;=\; (-2)^2\cdot s_x^2 \;=\;40\: ,\ \\
\tilde{y} &=& (-2)\tilde{x} \;=\; -6\: .
\end{eqnarray*}
The conversion of the median uses the fact that a multiplication by a factor of $(-2)$ reverses the ordering 
of the ordered original list, but the value at the mid position (for an odd number) or the two values at the mid positions 
(for an even number) stay at their positions and are multiplied by $(-2)$ each.
\end{MHint}
\end{MExercise}
\end{MXContent}


\MSubsection{Final Test}

\begin{MTest}{Final Test Module \arabic{section}}
\MLabel{VBKM_STOCH_Abschlusstest}
\MDeclareSiteUXID{VBKM_STOCHASTIK_Abschlusstest}


\begin{MExercise}
For bonds (e.g. government bonds),we distinguish between the nominal value and the issue price (quote).
Bonds can be issued at nominal value, under nominal value or over nominal value. The issue price is closer to the 
nominal value the more the bond interest corresponds to the current market rate. A customer buys bonds 
with a nominal value of $K=10{,}000$ EUR, an issue price of $100\%$, an interest rate of $p=4,5\%$ p.\,a., and 
a period of $t=10$~years. 

\begin{MExerciseItems}
\item{How much interest is paid at the end of each interest period for an issue price of $100\%$
when simple interest is applied. Answer: the annually paid interest is \MLParsedQuestion{10}{450}{5}{STOCHPRO3} EUR.}
\item{Specify the amount of the totally paid capital at the end of the period if simple interest is applied.
Answer: the amount of capital paid at the end of the period is \MLParsedQuestion{10}{14500}{5}{STOCHPRO4} EUR.}
\end{MExerciseItems}
\end{MExercise}

\begin{MExercise}
A capital of $K=25{,}000$ EUR will be invested with an annually paid interest rate of  $p=3,5\%$ p.\,a. until the amount of the capital has doubled. How many years does the capital have to be 
invested, if continuous compounding interest is applied?
\ \\ \ \\
Answer: the required investment period is \MEquationItem{$t$}{\MLParsedQuestion{10}{21}{5}{STOCHPRO11}} years.
\ \\ \ \\
Round up your result to the text integer.
\end{MExercise}

% \begin{MExercise}
% Der Ausgabekurs einer Anleihe betrage $97,5\%$. Es sollen $K=10000$ EUR investiert werden. Der Nominalzins der Anleihe betrage $p=3,5\%$ p.a. .
% \begin{MExerciseItems}
% \item{Bestimmen Sie den Kurswert der Anleihe zu Beginn der Laufzeit. Antwort: \MLParsedQuestion{10}{9750}{4}{STOCHPRO5} EUR.}
% \item{Bestimmen Sie die Höhe des insgesamt ausgezahlten Kapitals bei einer Laufzeit von $t=10$ Jahren. Antwort: \MLParsedQuestion{10}{13500}{4}{STOCHPRO6} EUR.}
% \item{Bestimmen Sie für diese Anleihe den Effektivzins. Antwort: \MLParsedQuestion{10}{3.85}{6} Prozent.}
% \end{MExerciseItems}
% Hinweis: Geben Sie den Effektivzinssatz mathematisch gerundet auf 2 Nachkommastellen an. Runden Sie erst \textit{nachdem} Sie den Effektivzins als Produkt ausgerechnet haben.
% \end{MExercise}

\begin{MExercise}
Read off the properties of the described sample from the histogram shown in the figure below.

\begin{center}
\MUGraphicsSolo{Histo2.png}{width=0.3\linewidth}{width:466px}
% CC BY-SA 3.0, Quelle: Histo.png aus der Wikipedia (Seite "Histogramm") von Benutzer https://de.wikipedia.org/wiki/Benutzer:Philipendula
\ \\
Histogram of the sample $x=(x_1,\ldots,x_n)$.
\end{center}

Specify the interval boundaries of the five classes and the corresponding relative frequencies. 
Fill in the frequency table. For this purpose, calculate the ratios of the areas of the single bars in the diagram 
to the total area.
\begin{center}
\begin{tabular}{|c|c|c|}
\hline
Class & Interval & rel. Class frequencies $h_{j}$ \\ \hline
Class 1 & $[0;200)$ & $0,16$ \\ \hline
Class 2 & \MLIntervalQuestion{12}{[200,300)}{5}{STOCHIT1} & \MLParsedQuestion{8}{0.19}{5}{STOCHIT5} \\ \hline
Class 3 & \MLIntervalQuestion{12}{[300,400)}{5}{STOCHIT2} & \MLParsedQuestion{8}{0.19}{5}{STOCHIT6} \\ \hline
Class 4 & \MLIntervalQuestion{12}{[400,500)}{5}{STOCHIT3} & \MLParsedQuestion{8}{0.28}{5}{STOCHIT7} \\ \hline
Class 5 & \MLIntervalQuestion{12}{[500,700)}{5}{STOCHIT4} & \MLParsedQuestion{8}{0.19}{5}{STOCHIT8} \\ \hline
\hline
\end{tabular}
\end{center}

\end{MExercise}


\begin{MExercise}
The measurement of the weight of $n=11$ watermelons (in kilogram) resulted in the following values:

\begin{center}
\begin{tabular}{|c|c|c|c|c|c|c|c|c|c|c|c|}
\hline
Number $j$ & $1$ & $2$ & $3$ & $4$ & $5$ & $6$ & $7$ & $8$ & $9$ & $10$ & $11$ \\ \hline
Weight $x_j$ in kilogram & $6.2$ &  $5.5$ & $7.3$ & $6.8$ & $6.3$ & $5.5$ & $4.5$ & $6.5$ & $7.3$ & $5.7$ & $5.6$\\ \hline
\end{tabular}
\end{center}

\begin{MExerciseItems}
\item{Find the arithmetic mean of the $11$ sample values: \MEquationItem{$\overline{x}$}{\MLParsedQuestion{10}{(6.2+5.5+7.3+6.8+6.3+5.5+4.5+6.5+7.3+5.7+5.6)/11}{6}{STOCHVAR4}}.}
\item{Find the median of the $11$ sample values: \MEquationItem{$\tilde{x}$}{\MLParsedQuestion{10}{6.2}{6}{STOCHVAR5}}.}
%\item{Bestimmen Sie die Stichprobenvarianz der $11$ Stichprobenwerte: \MEquationItem{$s^2_x$}{\MLParsedQuestion{10}{???}{6}{STOCHVAR6}}.}
\end{MExerciseItems}

Enter your result rounded to two fractional digits. Do not use a calculator but try to find the values by hand.
\end{MExercise}


\end{MTest}

\newpage
\MPrintIndex

\end{document}
