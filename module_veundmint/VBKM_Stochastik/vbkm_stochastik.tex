% MINTMOD Version P0.1.0, needs to be consistent with preprocesser object in tex2x and MPragma-Version at the end of this file

% Parameter aus Konvertierungsprozess (PDF und HTML-Erzeugung wenn vom Konverter aus gestartet) werden hier eingefuegt, Preambleincludes werden am Schluss angehaengt

\newif\ifttm                % gesetzt falls Uebersetzung in HTML stattfindet, sonst uebersetzung in PDF

% Wahl der Notationsvariante ist im PDF immer std, in der HTML-Uebersetzung wird vom Konverter die Auswahl modifiziert
\newif\ifvariantstd
\newif\ifvariantunotation
\variantstdtrue % Diese Zeile wird vom Konverter erkannt und ggf. modifiziert, daher nicht veraendern!


\def\MOutputDVI{1}
\def\MOutputPDF{2}
\def\MOutputHTML{3}
\newcounter{MOutput}

\ifttm
\usepackage{german}
\usepackage{array}
\usepackage{amsmath}
\usepackage{amssymb}
\usepackage{amsthm}
\else
\documentclass[ngerman,oneside]{scrbook}
\usepackage{etex}
\usepackage[latin1]{inputenc}
\usepackage{textcomp}
\usepackage[ngerman]{babel}
\usepackage[pdftex]{color}
\usepackage{xcolor}
\usepackage{graphicx}
\usepackage[all]{xy}
\usepackage{fancyhdr}
\usepackage{verbatim}
\usepackage{array}
\usepackage{float}
\usepackage{makeidx}
\usepackage{amsmath}
\usepackage{amstext}
\usepackage{amssymb}
\usepackage{amsthm}
\usepackage[ngerman]{varioref}
\usepackage{framed}
\usepackage{supertabular}
\usepackage{longtable}
\usepackage{maxpage}
\usepackage{tikz}
\usepackage{tikzscale}
\usepackage{tikz-3dplot}
\usepackage{bibgerm}
\usepackage{chemarrow}
\usepackage{polynom}
%\usepackage{draftwatermark}
\usepackage{pdflscape}
\usetikzlibrary{calc}
\usetikzlibrary{through}
\usetikzlibrary{shapes.geometric}
\usetikzlibrary{arrows}
\usetikzlibrary{intersections}
\usetikzlibrary{decorations.pathmorphing}
\usetikzlibrary{external}
\usetikzlibrary{patterns}
\usetikzlibrary{fadings}
\usepackage[colorlinks=true,linkcolor=blue]{hyperref} 
\usepackage[all]{hypcap}
%\usepackage[colorlinks=true,linkcolor=blue,bookmarksopen=true]{hyperref} 
\usepackage{ifpdf}

\usepackage{movie15}

\setcounter{tocdepth}{2} % In Inhaltsverzeichnis bis subsection
\setcounter{secnumdepth}{3} % Nummeriert bis subsubsection

\setlength{\LTpost}{0pt} % Fuer longtable
\setlength{\parindent}{0pt}
\setlength{\parskip}{8pt}
%\setlength{\parskip}{9pt plus 2pt minus 1pt}
\setlength{\abovecaptionskip}{-0.25ex}
\setlength{\belowcaptionskip}{-0.25ex}
\fi

\ifttm
\newcommand{\MDebugMessage}[1]{\special{html:<!-- debugprint;;}#1\special{html:; //-->}}
\else
%\newcommand{\MDebugMessage}[1]{\immediate\write\mintlog{#1}}
\newcommand{\MDebugMessage}[1]{}
\fi

\def\MPageHeaderDef{%
\pagestyle{fancy}%
\fancyhead[r]{(C) VE\&MINT-Projekt}
\fancyfoot[c]{\thepage\\--- CCL BY-SA 3.0 ---}
}


\ifttm%
\def\MRelax{}%
\else%
\def\MRelax{\relax}%
\fi%

%--------------------------- Uebernahme von speziellen XML-Versionen einiger LaTeX-Kommandos aus xmlbefehle.tex vom alten Kasseler Konverter ---------------

\newcommand{\MSep}{\left\|{\phantom{\frac1g}}\right.}

\newcommand{\ML}{L}

\newcommand{\MGGT}{\mathrm{ggT}}


\ifttm
% Verhindert dass die subsection-nummer doppelt in der toccaption auftaucht (sollte ggf. in toccaption gefixt werden so dass diese Ueberschreibung nicht notwendig ist)
\renewcommand{\thesubsection}{}
% Kommandos die ttm nicht kennt
\newcommand{\binomial}[2]{{#1 \choose #2}} %  Binomialkoeffizienten
\newcommand{\eur}{\begin{html}&euro;\end{html}}
\newcommand{\square}{\begin{html}&square;\end{html}}
\newcommand{\glqq}{"'}  \newcommand{\grqq}{"'}
\newcommand{\nRightarrow}{\special{html: &nrArr; }}
\newcommand{\nmid}{\special{html: &nmid; }}
\newcommand{\nparallel}{\begin{html}&nparallel;\end{html}}
\newcommand{\mapstoo}{\begin{html}<mo>&map;</mo>\end{html}}

% Schnitt und Vereinigungssymbole von Mengen haben zu kleine Abstaende; korrigiert:
\newcommand{\ccup}{\,\!\cup\,\!}
\newcommand{\ccap}{\,\!\cap\,\!}


% Umsetzung von mathbb im HTML
\renewcommand{\mathbb}[1]{\begin{html}<mo>&#1opf;</mo>\end{html}}
\fi

%---------------------- Strukturierung ----------------------------------------------------------------------------------------------------------------------

%---------------------- Kapselung des sectioning findet auf drei Ebenen statt:
% 1. Die LateX-Befehl
% 2. Die D-Versionen der Befehle, die nur die Grade der Abschnitte umhaengen falls notwendig
% 3. Die M-Versionen der Befehle, die zusaetzliche Formatierungen vornehmen, Skripten starten und das HTML codieren
% Im Modultext duerfen nur die M-Befehle verwendet werden!

\ifttm

  \def\Dsubsubsubsection#1{\subsubsubsection{#1}}
  \def\Dsubsubsection#1{\subsubsection{#1}\addtocounter{subsubsection}{1}} % ttm-Fehler korrigieren
  \def\Dsubsection#1{\subsection{#1}}
  \def\Dsection#1{\section{#1}} % Im HTML wird nur der Sektionstitel gegeben
  \def\Dchapter#1{\chapter{#1}}
  \def\Dsubsubsubsectionx#1{\subsubsubsection*{#1}}
  \def\Dsubsubsectionx#1{\subsubsection*{#1}}
  \def\Dsubsectionx#1{\subsection*{#1}}
  \def\Dsectionx#1{\section*{#1}}
  \def\Dchapterx#1{\chapter*{#1}}

\else

  \def\Dsubsubsubsection#1{\subsubsection{#1}}
  \def\Dsubsubsection#1{\subsection{#1}}
  \def\Dsubsection#1{\section{#1}}
  \def\Dsection#1{\chapter{#1}}
  \def\Dchapter#1{\title{#1}}
  \def\Dsubsubsubsectionx#1{\subsubsection*{#1}}
  \def\Dsubsubsectionx#1{\subsection*{#1}}
  \def\Dsubsectionx#1{\section*{#1}}
  \def\Dsectionx#1{\chapter*{#1}}

\fi

\newcommand{\MStdPoints}{4}
\newcommand{\MSetPoints}[1]{\renewcommand{\MStdPoints}{#1}}

% Befehl zum Abbruch der Erstellung (nur PDF)
\newcommand{\MAbort}[1]{\err{#1}}

% Prefix vor Dateieinbindungen, wird in der Baumdatei mit \renewcommand modifiziert
% und auf das Verzeichnisprefix gesetzt, in dem das gerade bearbeitete tex-Dokument liegt.
% Im HTML wird es auf das Verzeichnis der HTML-Datei gesetzt.
% Das Prefix muss mit / enden !
\newcommand{\MDPrefix}{.}

% MRegisterFile notiert eine Datei zur Einbindung in den HTML-Baum. Grafiken mit MGraphics werden automatisch eingebunden.
% Mit MLastFile erhaelt man eine Markierung fuer die zuletzt registrierte Datei.
% Diese Markierung wird im postprocessing durch den physikalischen Dateinamen ersetzt, aber nur den Namen (d.h. \MMaterial gehoert noch davor, vgl Definition von MGraphics)
% Parameter: Pfad/Name der Datei bzw. des Ordners, relativ zur Position des Modul-Tex-Dokuments.
\ifttm
\newcommand{\MRegisterFile}[1]{\addtocounter{MFileNumber}{1}\special{html:<!-- registerfile;;}#1\special{html:;;}\MDPrefix\special{html:;;}\arabic{MFileNumber}\special{html:; //-->}}
\else
\newcommand{\MRegisterFile}[1]{\addtocounter{MFileNumber}{1}}
\fi

% Testen welcher Uebersetzer hier am Werk ist

\ifttm
\setcounter{MOutput}{3}
\else
\ifx\pdfoutput\undefined
  \pdffalse
  \setcounter{MOutput}{\MOutputDVI}
  \message{Verarbeitung mit latex, Ausgabe in dvi.}
\else
  \setcounter{MOutput}{\MOutputPDF}
  \message{Verarbeitung mit pdflatex, Ausgabe in pdf.}
  \ifnum \pdfoutput=0
    \pdffalse
  \setcounter{MOutput}{\MOutputDVI}
  \message{Verarbeitung mit pdflatex, Ausgabe in dvi.}
  \else
    \ifnum\pdfoutput=1
    \pdftrue
  \setcounter{MOutput}{\MOutputPDF}
  \message{Verarbeitung mit pdflatex, Ausgabe in pdf.}
    \fi
  \fi
\fi
\fi

\ifnum\value{MOutput}=\MOutputPDF
\DeclareGraphicsExtensions{.pdf,.png,.jpg}
\fi

\ifnum\value{MOutput}=\MOutputDVI
\DeclareGraphicsExtensions{.eps,.png,.jpg}
\fi

\ifnum\value{MOutput}=\MOutputHTML
% Wird vom Konverter leider nicht erkannt und daher in split.pm hardcodiert!
\DeclareGraphicsExtensions{.png,.jpg,.gif}
\fi

% Umdefinition der hyperref-Nummerierung im PDF-Modus
\ifttm
\else
\renewcommand{\theHfigure}{\arabic{chapter}.\arabic{section}.\arabic{figure}}
\fi

% Makro, um in der HTML-Ausgabe die zuerst zu oeffnende Datei zu kennzeichnen
\ifttm
\newcommand{\MGlobalStart}{\special{html:<!-- mglobalstarttag -->}}
\else
\newcommand{\MGlobalStart}{}
\fi

% Makro, um bei scormlogin ein pullen des Benutzers bei Aufruf der Seite zu erzwingen (typischerweise auf der Einstiegsseite)
\ifttm
\newcommand{\MPullSite}{\special{html:<!-- pullsite //-->}}
\else
\newcommand{\MPullSite}{}
\fi

% Makro, um in der HTML-Ausgabe die Kapiteluebersicht zu kennzeichnen
\ifttm
\newcommand{\MGlobalChapterTag}{\special{html:<!-- mglobalchaptertag -->}}
\else
\newcommand{\MGlobalChapterTag}{}
\fi

% Makro, um in der HTML-Ausgabe die Konfiguration zu kennzeichnen
\ifttm
\newcommand{\MGlobalConfTag}{\special{html:<!-- mglobalconfigtag -->}}
\else
\newcommand{\MGlobalConfTag}{}
\fi

% Makro, um in der HTML-Ausgabe die Standortbeschreibung zu kennzeichnen
\ifttm
\newcommand{\MGlobalLocationTag}{\special{html:<!-- mgloballocationtag -->}}
\else
\newcommand{\MGlobalLocationTag}{}
\fi

% Makro, um in der HTML-Ausgabe die persoenlichen Daten zu kennzeichnen
\ifttm
\newcommand{\MGlobalDataTag}{\special{html:<!-- mglobaldatatag -->}}
\else
\newcommand{\MGlobalDataTag}{}
\fi

% Makro, um in der HTML-Ausgabe die Suchseite zu kennzeichnen
\ifttm
\newcommand{\MGlobalSearchTag}{\special{html:<!-- mglobalsearchtag -->}}
\else
\newcommand{\MGlobalSearchTag}{}
\fi

% Makro, um in der HTML-Ausgabe die Favoritenseite zu kennzeichnen
\ifttm
\newcommand{\MGlobalFavoTag}{\special{html:<!-- mglobalfavoritestag -->}}
\else
\newcommand{\MGlobalFavoTag}{}
\fi

% Makro, um in der HTML-Ausgabe die Eingangstestseite zu kennzeichnen
\ifttm
\newcommand{\MGlobalSTestTag}{\special{html:<!-- mglobalstesttag -->}}
\else
\newcommand{\MGlobalSTestTag}{}
\fi

% Makro, um in der PDF-Ausgabe ein Wasserzeichen zu definieren
\ifttm
\newcommand{\MWatermarkSettings}{\relax}
\else
\newcommand{\MWatermarkSettings}{%
% \SetWatermarkText{(c) MINT-Kolleg Baden-W�rttemberg 2014}
% \SetWatermarkLightness{0.85}
% \SetWatermarkScale{1.5}
}
\fi

\ifttm
\newcommand{\MBinom}[2]{\left({\begin{array}{c} #1 \\ #2 \end{array}}\right)}
\else
\newcommand{\MBinom}[2]{\binom{#1}{#2}}
\fi

\ifttm
\newcommand{\DeclareMathOperator}[2]{\def#1{\mathrm{#2}}}
\newcommand{\operatorname}[1]{\mathrm{#1}}
\fi

%----------------- Makros fuer die gemischte HTML/PDF-Konvertierung ------------------------------

\newcommand{\MTestName}{\relax} % wird durch Test-Umgebung gesetzt

% Fuer experimentelle Kursinhalte, die im Release-Umsetzungsvorgang eine Fehlermeldung
% produzieren sollen aber sonst normal umgesetzt werden
\newenvironment{MExperimental}{%
}{%
}

% Wird von ttm nicht richtig umgesetzt!!
\newenvironment{MExerciseItems}{%
\renewcommand\theenumi{\alph{enumi}}%
\begin{enumerate}%
}{%
\end{enumerate}%
}


\definecolor{infoshadecolor}{rgb}{0.75,0.75,0.75}
\definecolor{exmpshadecolor}{rgb}{0.875,0.875,0.875}
\definecolor{expeshadecolor}{rgb}{0.95,0.95,0.95}
\definecolor{framecolor}{rgb}{0.2,0.2,0.2}

% Bei PDF-Uebersetzung wird hinter den Start jeder Satz/Info-aehnlichen Umgebung eine leere mbox gesetzt, damit
% fuehrende Listen oder enums nicht den Zeilenumbruch kaputtmachen
%\ifttm
\def\MTB{}
%\else
%\def\MTB{\mbox{}}
%\fi


\ifttm
\newcommand{\MRelates}{\special{html:<mi>&wedgeq;</mi>}}
\else
\def\MRelates{\stackrel{\scriptscriptstyle\wedge}{=}}
\fi

\def\MInch{\text{''}}
\def\Mdd{\textit{''}}

\ifttm
\def\MNL{ \newline }
\newenvironment{MArray}[1]{\begin{array}{#1}}{\end{array}}
\else
\def\MNL{ \\ }
\newenvironment{MArray}[1]{\begin{array}{#1}}{\end{array}}
\fi

\newcommand{\MBox}[1]{$\mathrm{#1}$}
\newcommand{\MMBox}[1]{\mathrm{#1}}


\ifttm%
\newcommand{\Mtfrac}[2]{{\textstyle \frac{#1}{#2}}}
\newcommand{\Mdfrac}[2]{{\displaystyle \frac{#1}{#2}}}
\newcommand{\Mmeasuredangle}{\special{html:<mi>&angmsd;</mi>}}
\else%
\newcommand{\Mtfrac}[2]{\tfrac{#1}{#2}}
\newcommand{\Mdfrac}[2]{\dfrac{#1}{#2}}
\newcommand{\Mmeasuredangle}{\measuredangle}
\relax
\fi

% Matrizen und Vektoren

% Inhalt wird in der Form a & b \\ c & d erwartet
% Vorsicht: MVector = Komponentenspalte, MVec = Variablensymbol
\ifttm%
\newcommand{\MVector}[1]{\left({\begin{array}{c}#1\end{array}}\right)}
\else%
\newcommand{\MVector}[1]{\begin{pmatrix}#1\end{pmatrix}}
\fi



\newcommand{\MVec}[1]{\vec{#1}}
\newcommand{\MDVec}[1]{\overrightarrow{#1}}

%----------------- Umgebungen fuer Definitionen und Saetze ----------------------------------------

% Fuegt einen Tabellen-Zeilenumbruch ein im PDF, aber nicht im HTML
\newcommand{\TSkip}{\ifttm \else&\ \\\fi}

\newenvironment{infoshaded}{%
\def\FrameCommand{\fboxsep=\FrameSep \fcolorbox{framecolor}{infoshadecolor}}%
\MakeFramed {\advance\hsize-\width \FrameRestore}}%
{\endMakeFramed}

\newenvironment{expeshaded}{%
\def\FrameCommand{\fboxsep=\FrameSep \fcolorbox{framecolor}{expeshadecolor}}%
\MakeFramed {\advance\hsize-\width \FrameRestore}}%
{\endMakeFramed}

\newenvironment{exmpshaded}{%
\def\FrameCommand{\fboxsep=\FrameSep \fcolorbox{framecolor}{exmpshadecolor}}%
\MakeFramed {\advance\hsize-\width \FrameRestore}}%
{\endMakeFramed}

\def\STDCOLOR{black}

\ifttm%
\else%
\newtheoremstyle{MSatzStyle}
  {1cm}                   %Space above
  {1cm}                   %Space below
  {\normalfont\itshape}   %Body font
  {}                      %Indent amount (empty = no indent,
                          %\parindent = para indent)
  {\normalfont\bfseries}  %Thm head font
  {}                      %Punctuation after thm head
  {\newline}              %Space after thm head: " " = normal interword
                          %space; \newline = linebreak
  {\thmname{#1}\thmnumber{ #2}\thmnote{ (#3)}}
                          %Thm head spec (can be left empty, meaning
                          %`normal')
                          %
\newtheoremstyle{MDefStyle}
  {1cm}                   %Space above
  {1cm}                   %Space below
  {\normalfont}           %Body font
  {}                      %Indent amount (empty = no indent,
                          %\parindent = para indent)
  {\normalfont\bfseries}  %Thm head font
  {}                      %Punctuation after thm head
  {\newline}              %Space after thm head: " " = normal interword
                          %space; \newline = linebreak
  {\thmname{#1}\thmnumber{ #2}\thmnote{ (#3)}}
                          %Thm head spec (can be left empty, meaning
                          %`normal')
\fi%

\newcommand{\MInfoText}{Info}

\newcounter{MHintCounter}
\newcounter{MCodeEditCounter}

\newcounter{MLastIndex}  % Enthaelt die dritte Stelle (Indexnummer) des letzten angelegten Objekts
\newcounter{MLastType}   % Enthaelt den Typ des letzten angelegten Objekts (mithilfe der unten definierten Konstanten). Die Entscheidung, wie der Typ dargstellt wird, wird in split.pm beim Postprocessing getroffen.
\newcounter{MLastTypeEq} % =1 falls das Label in einer Matheumgebung (equation, eqnarray usw.) steht, =2 falls das Label in einer table-Umgebung steht

% Da ttm keine Zahlmakros verarbeiten kann, werden diese Nummern in den Zuweisungen hardcodiert!
\def\MTypeSection{1}          %# Zaehler ist section
\def\MTypeSubsection{2}       %# Zaehler ist subsection
\def\MTypeSubsubsection{3}    %# Zaehler ist subsubsection
\def\MTypeInfo{4}             %# Eine Infobox, Separatzaehler fuer die Chemie (auch wenn es dort nicht nummeriert wird) ist MInfoCounter
\def\MTypeExercise{5}         %# Eine Aufgabe, Separatzaehler fuer die Chemie ist MExerciseCounter
\def\MTypeExample{6}          %# Eine Beispielbox, Separatzaehler fuer die Chemie ist MExampleCounter
\def\MTypeExperiment{7}       %# Eine Versuchsbox, Separatzaehler fuer die Chemie ist MExperimentCounter
\def\MTypeGraphics{8}         %# Eine Graphik, Separatzaehler fuer alle FB ist MGraphicsCounter
\def\MTypeTable{9}            %# Eine Tabellennummer, hat keinen Zaehler da durch table gezaehlt wird
\def\MTypeEquation{10}        %# Eine Gleichungsnummer, hat keinen Zaehler da durch equation/eqnarray gezaehlt wird
\def\MTypeTheorem{11}         % Ein theorem oder xtheorem, Separatzaehler fuer die Chemie ist MTheoremCounter
\def\MTypeVideo{12}           %# Ein Video,Separatzaehler fuer alle FB ist MVideoCounter
\def\MTypeEntry{13}           %# Ein Eintrag fuer die Stichwortliste, wird nicht gezaehlt sondern erhaelt im preparsing ein unique-label 

% Zaehler fuer das Labelsystem sind prefixcounter, jeder Zaehler wird VOR dem gezaehlten Objekt inkrementiert und zaehlt daher das aktuelle Objekt
\newcounter{MInfoCounter}
\newcounter{MExerciseCounter}
\newcounter{MExampleCounter}
\newcounter{MExperimentCounter}
\newcounter{MGraphicsCounter}
\newcounter{MTableCounter}
\newcounter{MEquationCounter}  % Nur im HTML, sonst durch "equation"-counter von latex realisiert
\newcounter{MTheoremCounter}
\newcounter{MObjectCounter}   % Gemeinsamer Zaehler fuer Objekte (ausser Grafiken/Tabellen) in Mathe/Info/Physik
\newcounter{MVideoCounter}
\newcounter{MEntryCounter}

\newcounter{MTestSite} % 1 = Subsubsection ist eine Pruefungsseite, 0 = ist eine normale Seite (inkl. Hilfeseite)

\def\MCell{$\phantom{a}$}

\newenvironment{MExportExercise}{\begin{MExercise}}{\end{MExercise}} % wird von mconvert abgefangen

\def\MGenerateExNumber{%
\ifnum\value{MSepNumbers}=0%
\arabic{section}.\arabic{subsection}.\arabic{MObjectCounter}\setcounter{MLastIndex}{\value{MObjectCounter}}%
\else%
\arabic{section}.\arabic{subsection}.\arabic{MExerciseCounter}\setcounter{MLastIndex}{\value{MExerciseCounter}}%
\fi%
}%

\def\MGenerateExmpNumber{%
\ifnum\value{MSepNumbers}=0%
\arabic{section}.\arabic{subsection}.\arabic{MObjectCounter}\setcounter{MLastIndex}{\value{MObjectCounter}}%
\else%
\arabic{section}.\arabic{subsection}.\arabic{MExerciseCounter}\setcounter{MLastIndex}{\value{MExampleCounter}}%
\fi%
}%

\def\MGenerateInfoNumber{%
\ifnum\value{MSepNumbers}=0%
\arabic{section}.\arabic{subsection}.\arabic{MObjectCounter}\setcounter{MLastIndex}{\value{MObjectCounter}}%
\else%
\arabic{section}.\arabic{subsection}.\arabic{MExerciseCounter}\setcounter{MLastIndex}{\value{MInfoCounter}}%
\fi%
}%

\def\MGenerateSiteNumber{%
\arabic{section}.\arabic{subsection}.\arabic{subsubsection}%
}%

% Funktionalitaet fuer Auswahlaufgaben

\newcounter{MExerciseCollectionCounter} % = 0 falls nicht in collection-Umgebung, ansonsten Schachtelungstiefe
\newcounter{MExerciseCollectionTextCounter} % wird von MExercise-Umgebung inkrementiert und von MExerciseCollection-Umgebung auf Null gesetzt

\ifttm
% MExerciseCollection gruppiert Aufgaben, die dynamisch aus der Datenbank gezogen werden und nicht direkt in der HTML-Seite stehen
% Parameter: #1 = ID der Collection, muss eindeutig fuer alle IN DER DB VORHANDENEN collections sein unabhaengig vom Kurs
%            #2 = Optionsargument (im Moment: 1 = Iterative Auswahl, 2 = Zufallsbasierte Auswahl)
\newenvironment{MExerciseCollection}[2]{%
\addtocounter{MExerciseCollectionCounter}{1}
\setcounter{MExerciseCollectionTextCounter}{0}
\special{html:<!-- mexercisecollectionstart;;}#1\special{html:;;}#2\special{html:;; //-->}%
}{%
\special{html:<!-- mexercisecollectionstop //-->}%
\addtocounter{MExerciseCollectionCounter}{-1}
}
\else
\newenvironment{MExerciseCollection}[2]{%
\addtocounter{MExerciseCollectionCounter}{1}
\setcounter{MExerciseCollectionTextCounter}{0}
}{%
\addtocounter{MExerciseCollectionCounter}{-1}
}
\fi

% Bei Uebersetzung nach PDF werden die theorem-Umgebungen verwendet, bei Uebersetzung in HTML ein manuelles Makro
\ifttm%

  \newenvironment{MHint}[1]{  \special{html:<button name="Name_MHint}\arabic{MHintCounter}\special{html:" class="hintbutton_closed" id="MHint}\arabic{MHintCounter}\special{html:_button" %
  type="button" onclick="toggle_hint('MHint}\arabic{MHintCounter}\special{html:');">}#1\special{html:</button>}
  \special{html:<div class="hint" style="display:none" id="MHint}\arabic{MHintCounter}\special{html:"> }}{\begin{html}</div>\end{html}\addtocounter{MHintCounter}{1}}

  \newenvironment{MCOSHZusatz}{  \special{html:<button name="Name_MHint}\arabic{MHintCounter}\special{html:" class="chintbutton_closed" id="MHint}\arabic{MHintCounter}\special{html:_button" %
  type="button" onclick="toggle_hint('MHint}\arabic{MHintCounter}\special{html:');">}Weiterf�hrende Inhalte\special{html:</button>}
  \special{html:<div class="hintc" style="display:none" id="MHint}\arabic{MHintCounter}\special{html:">
  <div class="coshwarn">Diese Inhalte gehen �ber das Kursniveau hinaus und werden in den Aufgaben und Tests nicht abgefragt.</div><br />}
  \addtocounter{MHintCounter}{1}}{\begin{html}</div>\end{html}}

  
  \newenvironment{MDefinition}{\begin{definition}\setcounter{MLastIndex}{\value{definition}}\ \\}{\end{definition}}

  
  \newenvironment{MExercise}{
  \renewcommand{\MStdPoints}{4}
  \addtocounter{MExerciseCounter}{1}
  \addtocounter{MObjectCounter}{1}
  \setcounter{MLastType}{5}

  \ifnum\value{MExerciseCollectionCounter}=0\else\addtocounter{MExerciseCollectionTextCounter}{1}\special{html:<!-- mexercisetextstart;;}\arabic{MExerciseCollectionTextCounter}\special{html:;; //-->}\fi
  \special{html:<div class="aufgabe" id="ADIV_}\MGenerateExNumber\special{html:">}%
  \textbf{Aufgabe \MGenerateExNumber
  } \ \\}{
  \special{html:</div><!-- mfeedbackbutton;Aufgabe;}\arabic{MTestSite}\special{html:;}\MGenerateExNumber\special{html:; //-->}
  \ifnum\value{MExerciseCollectionCounter}=0\else\special{html:<!-- mexercisetextstop //-->}\fi
  }

  % Stellt eine Kombination aus Aufgabe, Loesungstext und Eingabefeld bereit,
  % bei der Aufgabentext und Musterloesung sowie die zugehoerigen Feldelemente
  % extern bezogen und div-aktualisiert werden, das Eingabefeld aber immer das gleiche ist.
  \newenvironment{MFetchExercise}{
  \addtocounter{MExerciseCounter}{1}
  \addtocounter{MObjectCounter}{1}
  \setcounter{MLastType}{5}

  \special{html:<div class="aufgabe" id="ADIV_}\MGenerateExNumber\special{html:">}%
  \textbf{Aufgabe \MGenerateExNumber
  } \ \\%
  \special{html:</div><div class="exfetch_text" id="ADIVTEXT_}\MGenerateExNumber\special{html:">}%
  \special{html:</div><div class="exfetch_sol" id="ADIVSOL_}\MGenerateExNumber\special{html:">}%
  \special{html:</div><div class="exfetch_input" id="ADIVINPUT_}\MGenerateExNumber\special{html:">}%
  }{
  \special{html:</div>}
  }

  \newenvironment{MExample}{
  \addtocounter{MExampleCounter}{1}
  \addtocounter{MObjectCounter}{1}
  \setcounter{MLastType}{6}
  \begin{html}
  <div class="exmp">
  <div class="exmprahmen">
  \end{html}\textbf{Beispiel
  \ifnum\value{MSepNumbers}=0
  \arabic{section}.\arabic{subsection}.\arabic{MObjectCounter}\setcounter{MLastIndex}{\value{MObjectCounter}}
  \else
  \arabic{section}.\arabic{subsection}.\arabic{MExampleCounter}\setcounter{MLastIndex}{\value{MExampleCounter}}
  \fi
  } \ \\}{\begin{html}</div>
  </div>
  \end{html}
  \special{html:<!-- mfeedbackbutton;Beispiel;}\arabic{MTestSite}\special{html:;}\MGenerateExmpNumber\special{html:; //-->}
  }

  \newenvironment{MExperiment}{
  \addtocounter{MExperimentCounter}{1}
  \addtocounter{MObjectCounter}{1}
  \setcounter{MLastType}{7}
  \begin{html}
  <div class="expe">
  <div class="experahmen">
  \end{html}\textbf{Versuch
  \ifnum\value{MSepNumbers}=0
  \arabic{section}.\arabic{subsection}.\arabic{MObjectCounter}\setcounter{MLastIndex}{\value{MObjectCounter}}
  \else
%  \arabic{MExperimentCounter}\setcounter{MLastIndex}{\value{MExperimentCounter}}
  \arabic{section}.\arabic{subsection}.\arabic{MExperimentCounter}\setcounter{MLastIndex}{\value{MExperimentCounter}}
  \fi
  } \ \\}{\begin{html}</div>
  </div>
  \end{html}}

  \newenvironment{MChemInfo}{
  \setcounter{MLastType}{4}
  \begin{html}
  <div class="info">
  <div class="inforahmen">
  \end{html}}{\begin{html}</div>
  </div>
  \end{html}}

  \newenvironment{MXInfo}[1]{
  \addtocounter{MInfoCounter}{1}
  \addtocounter{MObjectCounter}{1}
  \setcounter{MLastType}{4}
  \begin{html}
  <div class="info">
  <div class="inforahmen">
  \end{html}\textbf{#1
  \ifnum\value{MInfoNumbers}=0
  \else
    \ifnum\value{MSepNumbers}=0
    \arabic{section}.\arabic{subsection}.\arabic{MObjectCounter}\setcounter{MLastIndex}{\value{MObjectCounter}}
    \else
    \arabic{MInfoCounter}\setcounter{MLastIndex}{\value{MInfoCounter}}
    \fi
  \fi
  } \ \\}{\begin{html}</div>
  </div>
  \end{html}
  \special{html:<!-- mfeedbackbutton;Info;}\arabic{MTestSite}\special{html:;}\MGenerateInfoNumber\special{html:; //-->}
  }

  \newenvironment{MInfo}{\ifnum\value{MInfoNumbers}=0\begin{MChemInfo}\else\begin{MXInfo}{Info}\ \\ \fi}{\ifnum\value{MInfoNumbers}=0\end{MChemInfo}\else\end{MXInfo}\fi}

\else%

  \theoremstyle{MSatzStyle}
  \newtheorem{thm}{Satz}[section]
  \newtheorem{thmc}{Satz}
  \theoremstyle{MDefStyle}
  \newtheorem{defn}[thm]{Definition}
  \newtheorem{exmp}[thm]{Beispiel}
  \newtheorem{info}[thm]{\MInfoText}
  \theoremstyle{MDefStyle}
  \newtheorem{defnc}{Definition}
  \theoremstyle{MDefStyle}
  \newtheorem{exmpc}{Beispiel}[section]
  \theoremstyle{MDefStyle}
  \newtheorem{infoc}{\MInfoText}
  \theoremstyle{MDefStyle}
  \newtheorem{exrc}{Aufgabe}[section]
  \theoremstyle{MDefStyle}
  \newtheorem{verc}{Versuch}[section]
  
  \newenvironment{MFetchExercise}{}{} % kann im PDF nicht dargestellt werden
  
  \newenvironment{MExercise}{\begin{exrc}\renewcommand{\MStdPoints}{1}\MTB}{\end{exrc}}
  \newenvironment{MHint}[1]{\ \\ \underline{#1:}\\}{}
  \newenvironment{MCOSHZusatz}{\ \\ \underline{Weiterf�hrende Inhalte:}\\}{}
  \newenvironment{MDefinition}{\ifnum\value{MInfoNumbers}=0\begin{defnc}\else\begin{defn}\fi\MTB}{\ifnum\value{MInfoNumbers}=0\end{defnc}\else\end{defn}\fi}
%  \newenvironment{MExample}{\begin{exmp}}{\ \linebreak[1] \ \ \ \ $\phantom{a}$ \ \hfill $\blacklozenge$\end{exmp}}
  \newenvironment{MExample}{
    \ifnum\value{MInfoNumbers}=0\begin{exmpc}\else\begin{exmp}\fi
    \MTB
    \begin{exmpshaded}
    \ \newline
}{
    \end{exmpshaded}
    \ifnum\value{MInfoNumbers}=0\end{exmpc}\else\end{exmp}\fi
}
  \newenvironment{MChemInfo}{\begin{infoshaded}}{\end{infoshaded}}

  \newenvironment{MInfo}{\ifnum\value{MInfoNumbers}=0\begin{MChemInfo}\else\renewcommand{\MInfoText}{Info}\begin{info}\begin{infoshaded}
  \MTB
   \ \newline
    \fi
  }{\ifnum\value{MInfoNumbers}=0\end{MChemInfo}\else\end{infoshaded}\end{info}\fi}

  \newenvironment{MXInfo}[1]{
    \renewcommand{\MInfoText}{#1}
    \ifnum\value{MInfoNumbers}=0\begin{infoc}\else\begin{info}\fi%
    \MTB
    \begin{infoshaded}
    \ \newline
  }{\end{infoshaded}\ifnum\value{MInfoNumbers}=0\end{infoc}\else\end{info}\fi}

  \newenvironment{MExperiment}{
    \renewcommand{\MInfoText}{Versuch}
    \ifnum\value{MInfoNumbers}=0\begin{verc}\else\begin{info}\fi
    \MTB
    \begin{expeshaded}
    \ \newline
  }{
    \end{expeshaded}
    \ifnum\value{MInfoNumbers}=0\end{verc}\else\end{info}\fi
  }
\fi%

% MHint sollte nicht direkt fuer Loesungen benutzt werden wegen solutionselect
\newenvironment{MSolution}{\begin{MHint}{L"osung}}{\end{MHint}}

\newcounter{MCodeCounter}

\ifttm
\newenvironment{MCode}{\special{html:<!-- mcodestart -->}\ttfamily\color{blue}}{\special{html:<!-- mcodestop -->}}
\else
\newenvironment{MCode}{\begin{flushleft}\ttfamily\addtocounter{MCodeCounter}{1}}{\addtocounter{MCodeCounter}{-1}\end{flushleft}}
% Ohne color-Statement da inkompatible mit framed/shaded-Boxen aus dem framed-package
\fi

%----------------- Sonderdefinitionen fuer Symbole, die der Konverter nicht kann ----------------------------------------------

\ifttm%
\newcommand{\MUnderset}[2]{\underbrace{#2}_{#1}}%
\else%
\newcommand{\MUnderset}[2]{\underset{#1}{#2}}%
\fi%

\ifttm
\newcommand{\MThinspace}{\special{html:<mi>&#x2009;</mi>}}
\else
\newcommand{\MThinspace}{\,}
\fi

\ifttm
\newcommand{\glq}{\begin{html}&sbquo;\end{html}}
\newcommand{\grq}{\begin{html}&lsquo;\end{html}}
\newcommand{\glqq}{\begin{html}&bdquo;\end{html}}
\newcommand{\grqq}{\begin{html}&ldquo;\end{html}}
\fi

\ifttm
\newcommand{\MNdash}{\begin{html}&ndash;\end{html}}
\else
\newcommand{\MNdash}{--}
\fi

%\ifttm\def\MIU{\special{html:<mi>&#8520;</mi>}}\else\def\MIU{\mathrm{i}}\fi
\def\MIU{\mathrm{i}}
\def\MEU{e} % TU9-Onlinekurs: italic-e
%\def\MEU{\mathrm{e}} % Alte Onlinemodule: roman-e
\def\MD{d} % Kursives d in Integralen im TU9-Onlinekurs
%\def\MD{\mathrm{d}} % roman-d in den alten Onlinemodulen
\def\MDB{\|}

%zusaetzlicher Leerraum vor "\MD"
\ifttm%
\def\MDSpace{\special{html:<mi>&#x2009;</mi>}}
\else%
\def\MDSpace{\,}
\fi%
\newcommand{\MDwSp}{\MDSpace\MD}%

\ifttm
\def\Mdq{\dq}
\else
\def\Mdq{\dq}
\fi

\def\MSpan#1{\left<{#1}\right>}
\def\MSetminus{\setminus}
\def\MIM{I}

\ifttm
\newcommand{\ld}{\text{ld}}
\newcommand{\lg}{\text{lg}}
\else
\DeclareMathOperator{\ld}{ld}
%\newcommand{\lg}{\text{lg}} % in latex schon definiert
\fi


\def\Mmapsto{\ifttm\special{html:<mi>&mapsto;</mi>}\else\mapsto\fi} 
\def\Mvarphi{\ifttm\phi\else\varphi\fi}
\def\Mphi{\ifttm\varphi\else\phi\fi}
\ifttm%
\newcommand{\MEumu}{\special{html:<mi>&#x3BC;</mi>}}%
\else%
\newcommand{\MEumu}{\textrm{\textmu}}%
\fi
\def\Mvarepsilon{\ifttm\epsilon\else\varepsilon\fi}
\def\Mepsilon{\ifttm\varepsilon\else\epsilon\fi}
\def\Mvarkappa{\ifttm\kappa\else\varkappa\fi}
\def\Mkappa{\ifttm\varkappa\else\kappa\fi}
\def\Mcomplement{\ifttm\special{html:<mi>&comp;</mi>}\else\complement\fi} 
\def\MWW{\mathrm{WW}}
\def\Mmod{\ifttm\special{html:<mi>&nbsp;mod&nbsp;</mi>}\else\mod\fi} 

\ifttm%
\def\mod{\text{\;mod\;}}%
\def\MNEquiv{\special{html:<mi>&NotCongruent;</mi>}}% 
\def\MNSubseteq{\special{html:<mi>&NotSubsetEqual;</mi>}}%
\def\MEmptyset{\special{html:<mi>&empty;</mi>}}%
\def\MVDots{\special{html:<mi>&#x22EE;</mi>}}%
\def\MHDots{\special{html:<mi>&#x2026;</mi>}}%
\def\Mddag{\special{html:<mi>&#x1202;</mi>}}%
\def\sphericalangle{\special{html:<mi>&measuredangle;</mi>}}%
\def\nparallel{\special{html:<mi>&nparallel;</mi>}}%
\def\MProofEnd{\special{html:<mi>&#x25FB;</mi>}}%
\newenvironment{MProof}[1]{\underline{#1}:\MCR\MCR}{\hfill $\MProofEnd$}%
\else%
\def\MNEquiv{\not\equiv}%
\def\MNSubseteq{\not\subseteq}%
\def\MEmptyset{\emptyset}%
\def\MVDots{\vdots}%
\def\MHDots{\hdots}%
\def\Mddag{\ddag}%
\newenvironment{MProof}[1]{\begin{proof}[#1]}{\end{proof}}%
\fi%



% Spaces zum Auffuellen von Tabellenbreiten, die nur im HTML wirken
\ifttm%
\def\MTSP{\:}%
\else%
\def\MTSP{}%
\fi%

\DeclareMathOperator{\arsinh}{arsinh}
\DeclareMathOperator{\arcosh}{arcosh}
\DeclareMathOperator{\artanh}{artanh}
\DeclareMathOperator{\arcoth}{arcoth}


\newcommand{\MMathSet}[1]{\mathbb{#1}}
\def\N{\MMathSet{N}}
\def\Z{\MMathSet{Z}}
\def\Q{\MMathSet{Q}}
\def\R{\MMathSet{R}}
\def\C{\MMathSet{C}}

\newcounter{MForLoopCounter}
\newcommand{\MForLoop}[2]{\setcounter{MForLoopCounter}{#1}\ifnum\value{MForLoopCounter}=0{}\else{{#2}\addtocounter{MForLoopCounter}{-1}\MForLoop{\value{MForLoopCounter}}{#2}}\fi}

\newcounter{MSiteCounter}
\newcounter{MFieldCounter} % Kombination section.subsection.site.field ist eindeutig in allen Modulen, field alleine nicht

\newcounter{MiniMarkerCounter}

\ifttm
\newenvironment{MMiniPageP}[1]{\begin{minipage}{#1\linewidth}\special{html:<!-- minimarker;;}\arabic{MiniMarkerCounter}\special{html:;;#1; //-->}}{\end{minipage}\addtocounter{MiniMarkerCounter}{1}}
\else
\newenvironment{MMiniPageP}[1]{\begin{minipage}{#1\linewidth}}{\end{minipage}\addtocounter{MiniMarkerCounter}{1}}
\fi

\newcounter{AlignCounter}

\newcommand{\MStartJustify}{\ifttm\special{html:<!-- startalign;;}\arabic{AlignCounter}\special{html:;;justify; //-->}\fi}
\newcommand{\MStopJustify}{\ifttm\special{html:<!-- stopalign;;}\arabic{AlignCounter}\special{html:; //-->}\fi\addtocounter{AlignCounter}{1}}

\newenvironment{MJTabular}[1]{
\MStartJustify
\begin{tabular}{#1}
}{
\end{tabular}
\MStopJustify
}

\newcommand{\MImageLeft}[2]{
\begin{center}
\begin{tabular}{lc}
\MStartJustify
\begin{MMiniPageP}{0.65}
#1
\end{MMiniPageP}
\MStopJustify
&
\begin{MMiniPageP}{0.3}
#2  
\end{MMiniPageP}
\end{tabular}
\end{center}
}

\newcommand{\MImageHalf}[2]{
\begin{center}
\begin{tabular}{lc}
\MStartJustify
\begin{MMiniPageP}{0.45}
#1
\end{MMiniPageP}
\MStopJustify
&
\begin{MMiniPageP}{0.45}
#2  
\end{MMiniPageP}
\end{tabular}
\end{center}
}

\newcommand{\MBigImageLeft}[2]{
\begin{center}
\begin{tabular}{lc}
\MStartJustify
\begin{MMiniPageP}{0.25}
#1
\end{MMiniPageP}
\MStopJustify
&
\begin{MMiniPageP}{0.7}
#2  
\end{MMiniPageP}
\end{tabular}
\end{center}
}

\ifttm
\def\No{\mathbb{N}_0}
\else
\def\No{\ensuremath{\N_0}}
\fi
\def\MT{\textrm{\tiny T}}
\newcommand{\MTranspose}[1]{{#1}^{\MT}}
\ifttm
\newcommand{\MRe}{\mathsf{Re}}
\newcommand{\MIm}{\mathsf{Im}}
\else
\DeclareMathOperator{\MRe}{Re}
\DeclareMathOperator{\MIm}{Im}
\fi

\newcommand{\Mid}{\mathrm{id}}
\newcommand{\MFeinheit}{\mathrm{feinh}}

\ifttm
\newcommand{\Msubstack}[1]{\begin{array}{c}{#1}\end{array}}
\else
\newcommand{\Msubstack}[1]{\substack{#1}}
\fi

% Typen von Fragefeldern:
% 1 = Alphanumerisch, case-sensitive-Vergleich
% 2 = Ja/Nein-Checkbox, Loesung ist 0 oder 1   (OPTION = Image-id fuer Rueckmeldung)
% 3 = Reelle Zahlen Geparset
% 4 = Funktionen Geparset (mit Stuetzstellen zur ueberpruefung)

% Dieser Befehl erstellt ein interaktives Aufgabenfeld. Parameter:
% - #1 Laenge in Zeichen
% - #2 Loesungstext (alphanumerisch, case sensitive)
% - #3 AufgabenID (alphanumerisch, case sensitive)
% - #4 Typ (Kennnummer)
% - #5 String fuer Optionen (ggf. mit Semikolon getrennte Einzelstrings)
% - #6 Anzahl Punkte
% - #7 uxid (kann z.B. Loesungsstring sein)
% ACHTUNG: Die langen Zeilen bitte so lassen, Zeilenumbrueche im tex werden in div's umgesetzt
\newcommand{\MQuestionID}[7]{
\ifttm
\special{html:<!-- mdeclareuxid;;}UX#7\special{html:;;}\arabic{section}\special{html:;;}#3\special{html:;; //-->}%
\special{html:<!-- mdeclarepoints;;}\arabic{section}\special{html:;;}#3\special{html:;;}#6\special{html:;;}\arabic{MTestSite}\special{html:;;}\arabic{chapter}%
\special{html:;; //--><!-- onloadstart //-->CreateQuestionObj("}#7\special{html:",}\arabic{MFieldCounter}\special{html:,"}#2%
\special{html:","}#3\special{html:",}#4\special{html:,"}#5\special{html:",}#6\special{html:,}\arabic{MTestSite}\special{html:,}\arabic{section}%
\special{html:);<!-- onloadstop //-->}%
\special{html:<input mfieldtype="}#4\special{html:" name="Name_}#3\special{html:" id="}#3\special{html:" type="text" size="}#1\special{html:" maxlength="}#1%
\special{html:" }\ifnum\value{MGroupActive}=0\special{html:onfocus="handlerFocus(}\arabic{MFieldCounter}%
\special{html:);" onblur="handlerBlur(}\arabic{MFieldCounter}\special{html:);" onkeyup="handlerChange(}\arabic{MFieldCounter}\special{html:,0);" onpaste="handlerChange(}\arabic{MFieldCounter}\special{html:,0);" oninput="handlerChange(}\arabic{MFieldCounter}\special{html:,0);" onpropertychange="handlerChange(}\arabic{MFieldCounter}\special{html:,0);"/>}%
\special{html:<img src="images/questionmark.gif" width="20" height="20" border="0" align="absmiddle" id="}QM#3\special{html:"/>}
\else%
\special{html:onblur="handlerBlur(}\arabic{MFieldCounter}%
\special{html:);" onfocus="handlerFocus(}\arabic{MFieldCounter}\special{html:);" onkeyup="handlerChange(}\arabic{MFieldCounter}\special{html:,1);" onpaste="handlerChange(}\arabic{MFieldCounter}\special{html:,1);" oninput="handlerChange(}\arabic{MFieldCounter}\special{html:,1);" onpropertychange="handlerChange(}\arabic{MFieldCounter}\special{html:,1);"/>}%
\special{html:<img src="images/questionmark.gif" width="20" height="20" border="0" align="absmiddle" id="}QM#3\special{html:"/>}\fi%
\else%
\ifnum\value{QBoxFlag}=1\fbox{$\phantom{\MForLoop{#1}{b}}$}\else$\phantom{\MForLoop{#1}{b}}$\fi%
\fi%
}

% ACHTUNG: Die langen Zeilen bitte so lassen, Zeilenumbrueche im tex werden in div's umgesetzt
% QuestionCheckbox macht ausserhalb einer QuestionGroup keinen Sinn!
% #1 = solution (1 oder 0), ggf. mit ::smc abgetrennt auszuschliessende single-choice-boxen (UXIDs durch , getrennt), #2 = id, #3 = points, #4 = uxid
\newcommand{\MQuestionCheckbox}[4]{
\ifttm
\special{html:<!-- mdeclareuxid;;}UX#4\special{html:;;}\arabic{section}\special{html:;;}#2\special{html:;; //-->}%
\ifnum\value{MGroupActive}=0\MDebugMessage{ERROR: Checkbox Nr. \arabic{MFieldCounter}\ ist nicht in einer Kontrollgruppe, es wird niemals eine Loesung angezeigt!}\fi
\special{html: %
<!-- mdeclarepoints;;}\arabic{section}\special{html:;;}#2\special{html:;;}#3\special{html:;;}\arabic{MTestSite}\special{html:;;}\arabic{chapter}%
\special{html:;; //--><!-- onloadstart //-->CreateQuestionObj("}#4\special{html:",}\arabic{MFieldCounter}\special{html:,"}#1\special{html:","}#2\special{html:",2,"IMG}#2%
\special{html:",}#3\special{html:,}\arabic{MTestSite}\special{html:,}\arabic{section}\special{html:);<!-- onloadstop //-->}%
\special{html:<input mfieldtype="2" type="checkbox" name="Name_}#2\special{html:" id="}#2\special{html:" onchange="handlerChange(}\arabic{MFieldCounter}\special{html:,1);"/><img src="images/questionmark.gif" name="}Name_IMG#2%
\special{html:" width="20" height="20" border="0" align="absmiddle" id="}IMG#2\special{html:"/> }%
\else%
\ifnum\value{QBoxFlag}=1\fbox{$\phantom{X}$}\else$\phantom{X}$\fi%
\fi%
}

\def\MGenerateID{QFELD_\arabic{section}.\arabic{subsection}.\arabic{MSiteCounter}.QF\arabic{MFieldCounter}}

% #1 = 0/1 ggf. mit ::smc abgetrennt auszuschliessende single-choice-boxen (UXIDs durch , getrennt ohne UX), #2 = uxid ohne UX
\newcommand{\MCheckbox}[2]{
\MQuestionCheckbox{#1}{\MGenerateID}{\MStdPoints}{#2}
\addtocounter{MFieldCounter}{1}
}

% Erster Parameter: Zeichenlaenge der Eingabebox, zweiter Parameter: Loesungstext
\newcommand{\MQuestion}[2]{
\MQuestionID{#1}{#2}{\MGenerateID}{1}{0}{\MStdPoints}{#2}
\addtocounter{MFieldCounter}{1}
}

% Erster Parameter: Zeichenlaenge der Eingabebox, zweiter Parameter: Loesungstext
\newcommand{\MLQuestion}[3]{
\MQuestionID{#1}{#2}{\MGenerateID}{1}{0}{\MStdPoints}{#3}
\addtocounter{MFieldCounter}{1}
}

% Parameter: Laenge des Feldes, Loesung (wird auch geparsed), Stellen Genauigkeit hinter dem Komma, weitere Stellen werden mathematisch gerundet vor Vergleich
\newcommand{\MParsedQuestion}[3]{
\MQuestionID{#1}{#2}{\MGenerateID}{3}{#3}{\MStdPoints}{#2}
\addtocounter{MFieldCounter}{1}
}

% Parameter: Laenge des Feldes, Loesung (wird auch geparsed), Stellen Genauigkeit hinter dem Komma, weitere Stellen werden mathematisch gerundet vor Vergleich
\newcommand{\MLParsedQuestion}[4]{
\MQuestionID{#1}{#2}{\MGenerateID}{3}{#3}{\MStdPoints}{#4}
\addtocounter{MFieldCounter}{1}
}

% Parameter: Laenge des Feldes, Loesungsfunktion, Anzahl Stuetzstellen, Funktionsvariablen durch Kommata getrennt (nicht case-sensitive), Anzahl Nachkommastellen im Vergleich
\newcommand{\MFunctionQuestion}[5]{
\MQuestionID{#1}{#2}{\MGenerateID}{4}{#3;#4;#5;0}{\MStdPoints}{#2}
\addtocounter{MFieldCounter}{1}
}

% Parameter: Laenge des Feldes, Loesungsfunktion, Anzahl Stuetzstellen, Funktionsvariablen durch Kommata getrennt (nicht case-sensitive), Anzahl Nachkommastellen im Vergleich, UXID
\newcommand{\MLFunctionQuestion}[6]{
\MQuestionID{#1}{#2}{\MGenerateID}{4}{#3;#4;#5;0}{\MStdPoints}{#6}
\addtocounter{MFieldCounter}{1}
}

% Parameter: Laenge des Feldes, Loesungsintervall, Genauigkeit der Zahlenwertpruefung
\newcommand{\MIntervalQuestion}[3]{
\MQuestionID{#1}{#2}{\MGenerateID}{6}{#3}{\MStdPoints}{#2}
\addtocounter{MFieldCounter}{1}
}

% Parameter: Laenge des Feldes, Loesungsintervall, Genauigkeit der Zahlenwertpruefung, UXID
\newcommand{\MLIntervalQuestion}[4]{
\MQuestionID{#1}{#2}{\MGenerateID}{6}{#3}{\MStdPoints}{#4}
\addtocounter{MFieldCounter}{1}
}

% Parameter: Laenge des Feldes, Loesungsfunktion, Anzahl Stuetzstellen, Funktionsvariable (nicht case-sensitive), Anzahl Nachkommastellen im Vergleich, Vereinfachungsbedingung
% Vereinfachungsbedingung ist eine der Folgenden:
% 0 = Keine Vereinfachungsbedingung
% 1 = Keine Klammern (runde oder eckige) mehr im vereinfachten Ausdruck
% 2 = Faktordarstellung (Term hat Produkte als letzte Operation, Summen als vorgeschaltete Operation)
% 3 = Summendarstellung (Term hat Summen als letzte Operation, Produkte als vorgeschaltete Operation)
% Flag 512: Besondere Stuetzstellen (nur >1 und nur schwach rational), sonst symmetrisch um Nullpunkt und ganze Zahlen inkl. Null werden getroffen
\newcommand{\MSimplifyQuestion}[6]{
\MQuestionID{#1}{#2}{\MGenerateID}{4}{#3;#4;#5;#6}{\MStdPoints}{#2}
\addtocounter{MFieldCounter}{1}
}

\newcommand{\MLSimplifyQuestion}[7]{
\MQuestionID{#1}{#2}{\MGenerateID}{4}{#3;#4;#5;#6}{\MStdPoints}{#7}
\addtocounter{MFieldCounter}{1}
}

% Parameter: Laenge des Feldes, Loesung (optionaler Ausdruck), Anzahl Stuetzstellen, Funktionsvariable (nicht case-sensitive), Anzahl Nachkommastellen im Vergleich, Spezialtyp (string-id)
\newcommand{\MLSpecialQuestion}[7]{
\MQuestionID{#1}{#2}{\MGenerateID}{7}{#3;#4;#5;#6}{\MStdPoints}{#7}
\addtocounter{MFieldCounter}{1}
}

\newcounter{MGroupStart}
\newcounter{MGroupEnd}
\newcounter{MGroupActive}

\newenvironment{MQuestionGroup}{
\setcounter{MGroupStart}{\value{MFieldCounter}}
\setcounter{MGroupActive}{1}
}{
\setcounter{MGroupActive}{0}
\setcounter{MGroupEnd}{\value{MFieldCounter}}
\addtocounter{MGroupEnd}{-1}
}

\newcommand{\MGroupButton}[1]{
\ifttm
\special{html:<button name="Name_Group}\arabic{MGroupStart}\special{html:to}\arabic{MGroupEnd}\special{html:" id="Group}\arabic{MGroupStart}\special{html:to}\arabic{MGroupEnd}\special{html:" %
type="button" onclick="group_button(}\arabic{MGroupStart}\special{html:,}\arabic{MGroupEnd}\special{html:);">}#1\special{html:</button>}
\else
\phantom{#1}
\fi
}

%----------------- Makros fuer die modularisierte Darstellung ------------------------------------

\def\MyText#1{#1}

% is used internally by the conversion package, should not be used by original tex documents
\def\MOrgLabel#1{\relax}

\ifttm

% Ein MLabel wird im html codiert durch das tag <!-- mmlabel;;Labelbezeichner;;SubjectArea;;chapter;;section;;subsection;;Index;;Objekttyp; //-->
\def\MLabel#1{%
\ifnum\value{MLastType}=8%
\ifnum\value{MCaptionOn}=0%
\MDebugMessage{ERROR: Grafik \arabic{MGraphicsCounter} hat separates label: #1 (Grafiklabels sollten nur in der Caption stehen)}%
\fi
\fi
\ifnum\value{MLastType}=12%
\ifnum\value{MCaptionOn}=0%
\MDebugMessage{ERROR: Video \arabic{MVideoCounter} hat separates label: #1 (Videolabels sollten nur in der Caption stehen}%
\fi
\fi
\ifnum\value{MLastType}=10\setcounter{MLastIndex}{\value{equation}}\fi
\label{#1}\begin{html}<!-- mmlabel;;#1;;\end{html}\arabic{MSubjectArea}\special{html:;;}\arabic{chapter}\special{html:;;}\arabic{section}\special{html:;;}\arabic{subsection}\special{html:;;}\arabic{MLastIndex}\special{html:;;}\arabic{MLastType}\special{html:; //-->}}%

\else

% Sonderbehandlung im PDF fuer Abbildungen in separater aux-Datei, da MGraphics die figure-Umgebung nicht verwendet
\def\MLabel#1{%
\ifnum\value{MLastType}=8%
\ifnum\value{MCaptionOn}=0%
\MDebugMessage{ERROR: Grafik \arabic{MGraphicsCounter} hat separates label: #1 (Grafiklabels sollten nur in der Caption stehen}%
\fi
\fi
\ifnum\value{MLastType}=12%
\ifnum\value{MCaptionOn}=0%
\MDebugMessage{ERROR: Video \arabic{MVideoCounter} hat separates label: #1 (Videolabels sollten nur in der Caption stehen}%
\fi
\fi
\label{#1}%
}%

\fi

% Gibt Begriff des referenzierten Objekts mit aus, aber nur im HTML, daher nur in Ausnahmefaellen (z.B. Copyrightliste) sinnvoll
\def\MCRef#1{\ifttm\special{html:<!-- mmref;;}#1\special{html:;;1; //-->}\else\vref{#1}\fi}


\def\MRef#1{\ifttm\special{html:<!-- mmref;;}#1\special{html:;;0; //-->}\else\vref{#1}\fi}
\def\MERef#1{\ifttm\special{html:<!-- mmref;;}#1\special{html:;;0; //-->}\else\eqref{#1}\fi}
\def\MNRef#1{\ifttm\special{html:<!-- mmref;;}#1\special{html:;;0; //-->}\else\ref{#1}\fi}
\def\MSRef#1#2{\ifttm\special{html:<!-- msref;;}#1\special{html:;;}#2\special{html:; //-->}\else \if#2\empty \ref{#1} \else \hyperref[#1]{#2}\fi\fi} 

\def\MRefRange#1#2{\ifttm\MRef{#1} bis 
\MRef{#2}\else\vrefrange[\unskip]{#1}{#2}\fi}

\def\MRefTwo#1#2{\ifttm\MRef{#1} und \MRef{#2}\else%
\let\vRefTLRsav=\reftextlabelrange\let\vRefTPRsav=\reftextpagerange%
\def\reftextlabelrange##1##2{\ref{##1} und~\ref{##2}}%
\def\reftextpagerange##1##2{auf den Seiten~\pageref{#1} und~\pageref{#2}}%
\vrefrange[\unskip]{#1}{#2}%
\let\reftextlabelrange=\vRefTLRsav\let\reftextpagerange=\vRefTPRsav\fi}

% MSectionChapter definiert falls notwendig das Kapitel vor der section. Das ist notwendig, wenn nur ein Einzelmodul uebersetzt wird.
% MChaptersGiven ist ein Counter, der von mconvert.pl vordefiniert wird.
\ifttm
\newcommand{\MSectionChapter}{\ifnum\value{MChaptersGiven}=0{\Dchapter{Modul}}\else{}\fi}
\else
\newcommand{\MSectionChapter}{\ifnum\value{chapter}=0{\Dchapter{Modul}}\else{}\fi}
\fi


\def\MChapter#1{\ifnum\value{MSSEnd}>0{\MSubsectionEndMacros}\addtocounter{MSSEnd}{-1}\fi\Dchapter{#1}}
\def\MSubject#1{\MChapter{#1}} % Schluesselwort HELPSECTION ist reserviert fuer Hilfesektion

\newcommand{\MSectionID}{UNKNOWNID}

\ifttm
\newcommand{\MSetSectionID}[1]{\renewcommand{\MSectionID}{#1}}
\else
\newcommand{\MSetSectionID}[1]{\renewcommand{\MSectionID}{#1}\tikzsetexternalprefix{#1}}
\fi


\newcommand{\MSection}[1]{\MSetSectionID{MODULID}\ifnum\value{MSSEnd}>0{\MSubsectionEndMacros}\addtocounter{MSSEnd}{-1}\fi\MSectionChapter\Dsection{#1}\MSectionStartMacros{#1}\setcounter{MLastIndex}{-1}\setcounter{MLastType}{1}} % Sections werden ueber das section-Feld im mmlabel-Tag identifiziert, nicht ueber das Indexfeld

\def\MSubsection#1{\ifnum\value{MSSEnd}>0{\MSubsectionEndMacros}\addtocounter{MSSEnd}{-1}\fi\ifttm\else\clearpage\fi\Dsubsection{#1}\MSubsectionStartMacros\setcounter{MLastIndex}{-1}\setcounter{MLastType}{2}\addtocounter{MSSEnd}{1}}% Subsections werden ueber das subsection-Feld im mmlabel-Tag identifiziert, nicht ueber das Indexfeld
\def\MSubsectionx#1{\Dsubsectionx{#1}} % Nur zur Verwendung in MSectionStart gedacht
\def\MSubsubsection#1{\Dsubsubsection{#1}\setcounter{MLastIndex}{\value{subsubsection}}\setcounter{MLastType}{3}\ifttm\special{html:<!-- sectioninfo;;}\arabic{section}\special{html:;;}\arabic{subsection}\special{html:;;}\arabic{subsubsection}\special{html:;;1;;}\arabic{MTestSite}\special{html:; //-->}\fi}
\def\MSubsubsectionx#1{\Dsubsubsectionx{#1}\ifttm\special{html:<!-- sectioninfo;;}\arabic{section}\special{html:;;}\arabic{subsection}\special{html:;;}\arabic{subsubsection}\special{html:;;0;;}\arabic{MTestSite}\special{html:; //-->}\else\addcontentsline{toc}{subsection}{#1}\fi}

\ifttm
\def\MSubsubsubsectionx#1{\ \newline\textbf{#1}\special{html:<br />}}
\else
\def\MSubsubsubsectionx#1{\ \newline
\textbf{#1}\ \\
}
\fi


% Dieses Skript wird zu Beginn jedes Modulabschnitts (=Webseite) ausgefuehrt und initialisiert den Aufgabenfeldzaehler
\newcommand{\MPageScripts}{
\setcounter{MFieldCounter}{1}
\addtocounter{MSiteCounter}{1}
\setcounter{MHintCounter}{1}
\setcounter{MCodeEditCounter}{1}
\setcounter{MGroupActive}{0}
\DoQBoxes
% Feldvariablen werden im HTML-Header in conv.pl eingestellt
}

% Dieses Skript wird zum Ende jedes Modulabschnitts (=Webseite) ausgefuehrt
\ifttm
\newcommand{\MEndScripts}{\special{html:<br /><!-- mfeedbackbutton;Seite;}\arabic{MTestSite}\special{html:;}\MGenerateSiteNumber\special{html:; //-->}
}
\else
\newcommand{\MEndScripts}{\relax}
\fi


\newcounter{QBoxFlag}
\newcommand{\DoQBoxes}{\setcounter{QBoxFlag}{1}}
\newcommand{\NoQBoxes}{\setcounter{QBoxFlag}{0}}

\newcounter{MXCTest}
\newcounter{MXCounter}
\newcounter{MSCounter}



\ifttm

% Struktur des sectioninfo-Tags: <!-- sectioninfo;;section;;subsection;;subsubsection;;nr_ausgeben;;testpage; //-->

%Fuegt eine zusaetzliche html-Seite an hinter ALLEN bisherigen und zukuenftigen content-Seiten ausserhalb der vor-zurueck-Schleife (d.h. nur durch Button oder MIntLink erreichbar!)
% #1 = Titel des Modulabschnitts, #2 = Kurztitel fuer die Buttons, #3 = Buttonkennung (STD = default nehmen, NONE = Ohne Button in der Navigation)
\newenvironment{MSContent}[3]{\special{html:<div class="xcontent}\arabic{MSCounter}\special{html:"><!-- scontent;-;}\arabic{MSCounter};-;#1;-;#2;-;#3\special{html: //-->}\MPageScripts\MSubsubsectionx{#1}}{\MEndScripts\special{html:<!-- endscontent;;}\arabic{MSCounter}\special{html: //--></div>}\addtocounter{MSCounter}{1}}

% Fuegt eine zusaetzliche html-Seite ein hinter den bereits vorhandenen content-Seiten (oder als erste Seite) innerhalb der vor-zurueck-Schleife der Navigation
% #1 = Titel des Modulabschnitts, #2 = Kurztitel fuer die Buttons, #3 = Buttonkennung (STD = Defaultbutton, NONE = Ohne Button in der Navigation)
\newenvironment{MXContent}[3]{\special{html:<div class="xcontent}\arabic{MXCounter}\special{html:"><!-- xcontent;-;}\arabic{MXCounter};-;#1;-;#2;-;#3\special{html: //-->}\MPageScripts\MSubsubsection{#1}}{\MEndScripts\special{html:<!-- endxcontent;;}\arabic{MXCounter}\special{html: //--></div>}\addtocounter{MXCounter}{1}}

% Fuegt eine zusaetzliche html-Seite ein die keine subsubsection-Nummer bekommt, nur zur internen Verwendung in mintmod.tex gedacht!
% #1 = Titel des Modulabschnitts, #2 = Kurztitel fuer die Buttons, #3 = Buttonkennung (STD = Defaultbutton, NONE = Ohne Button in der Navigation)
% \newenvironment{MUContent}[3]{\special{html:<div class="xcontent}\arabic{MXCounter}\special{html:"><!-- xcontent;-;}\arabic{MXCounter};-;#1;-;#2;-;#3\special{html: //-->}\MPageScripts\MSubsubsectionx{#1}}{\MEndScripts\special{html:<!-- endxcontent;;}\arabic{MXCounter}\special{html: //--></div>}\addtocounter{MXCounter}{1}}

\newcommand{\MDeclareSiteUXID}[1]{\special{html:<!-- mdeclaresiteuxid;;}#1\special{html:;;}\arabic{chapter}\special{html:;;}\arabic{section}\special{html:;; //-->}}

\else

%\newcommand{\MSubsubsection}[1]{\refstepcounter{subsubsection} \addcontentsline{toc}{subsubsection}{\thesubsubsection. #1}}


% Fuegt eine zusaetzliche html-Seite an hinter den bereits vorhandenen content-Seiten
% #1 = Titel des Modulabschnitts, #2 = Kurztitel fuer die Buttons, #3 = Iconkennung (im PDF wirkungslos)
%\newenvironment{MUContent}[3]{\ifnum\value{MXCTest}>0{\MDebugMessage{ERROR: Geschachtelter SContent}}\fi\MPageScripts\MSubsubsectionx{#1}\addtocounter{MXCTest}{1}}{\addtocounter{MXCounter}{1}\addtocounter{MXCTest}{-1}}
\newenvironment{MXContent}[3]{\ifnum\value{MXCTest}>0{\MDebugMessage{ERROR: Geschachtelter SContent}}\fi\MPageScripts\MSubsubsection{#1}\addtocounter{MXCTest}{1}}{\addtocounter{MXCounter}{1}\addtocounter{MXCTest}{-1}}
\newenvironment{MSContent}[3]{\ifnum\value{MXCTest}>0{\MDebugMessage{ERROR: Geschachtelter XContent}}\fi\MPageScripts\MSubsubsectionx{#1}\addtocounter{MXCTest}{1}}{\addtocounter{MSCounter}{1}\addtocounter{MXCTest}{-1}}

\newcommand{\MDeclareSiteUXID}[1]{\relax}

\fi 

% GHEADER und GFOOTER werden von split.pm gefunden, aber nur, wenn nicht HELPSITE oder TESTSITE
\ifttm
\newenvironment{MSectionStart}{\special{html:<div class="xcontent0">}\MSubsubsectionx{Modul\"ubersicht}}{\setcounter{MSSEnd}{0}\special{html:</div>}}
% Darf nicht als XContent nummeriert werden, darf nicht als XContent gelabelt werden, wird aber in eine xcontent-div gesetzt fuer Python-parsing
\else
\newenvironment{MSectionStart}{\MSubsectionx{Modul\"ubersicht}}{\setcounter{MSSEnd}{0}}
\fi

\newenvironment{MIntro}{\begin{MXContent}{Einf\"uhrung}{Einf\"uhrung}{genetisch}}{\end{MXContent}}
\newenvironment{MContent}{\begin{MXContent}{Inhalt}{Inhalt}{beweis}}{\end{MXContent}}
\newenvironment{MExercises}{\ifttm\else\clearpage\fi\begin{MXContent}{Aufgaben}{Aufgaben}{aufgb}\special{html:<!-- declareexcsymb //-->}}{\end{MXContent}}

% #1 = Lesbare Testbezeichnung
\newenvironment{MTest}[1]{%
\renewcommand{\MTestName}{#1}
\ifttm\else\clearpage\fi%
\addtocounter{MTestSite}{1}%
\begin{MXContent}{#1}{#1}{STD} % {aufgb}%
\special{html:<!-- declaretestsymb //-->}
\begin{MQuestionGroup}%
\MInTestHeader
}%
{%
\end{MQuestionGroup}%
\ \\ \ \\%
\MInTestFooter
\end{MXContent}\addtocounter{MTestSite}{-1}%
}

\newenvironment{MExtra}{\ifttm\else\clearpage\fi\begin{MXContent}{Zus\"atzliche Inhalte}{Zusatz}{weiterfhrg}}{\end{MXContent}}

\makeindex

\ifttm
\def\MPrintIndex{
\ifnum\value{MSSEnd}>0{\MSubsectionEndMacros}\addtocounter{MSSEnd}{-1}\fi
\renewcommand{\indexname}{Stichwortverzeichnis}
\special{html:<p><!-- printindex //--></p>}
}
\else
\def\MPrintIndex{
\ifnum\value{MSSEnd}>0{\MSubsectionEndMacros}\addtocounter{MSSEnd}{-1}\fi
\renewcommand{\indexname}{Stichwortverzeichnis}
\addcontentsline{toc}{section}{Stichwortverzeichnis}
\printindex
}
\fi


% Konstanten fuer die Modulfaecher

\def\MINTMathematics{1}
\def\MINTInformatics{2}
\def\MINTChemistry{3}
\def\MINTPhysics{4}
\def\MINTEngineering{5}

\newcounter{MSubjectArea}
\newcounter{MInfoNumbers} % Gibt an, ob die Infoboxen nummeriert werden sollen
\newcounter{MSepNumbers} % Gibt an, ob Beispiele und Experimente separat nummeriert werden sollen
\newcommand{\MSetSubject}[1]{
 % ttm kapiert setcounter mit Parametern nicht, also per if abragen und einsetzen
\ifnum#1=1\setcounter{MSubjectArea}{1}\setcounter{MInfoNumbers}{1}\setcounter{MSepNumbers}{0}\fi
\ifnum#1=2\setcounter{MSubjectArea}{2}\setcounter{MInfoNumbers}{1}\setcounter{MSepNumbers}{0}\fi
\ifnum#1=3\setcounter{MSubjectArea}{3}\setcounter{MInfoNumbers}{0}\setcounter{MSepNumbers}{1}\fi
\ifnum#1=4\setcounter{MSubjectArea}{4}\setcounter{MInfoNumbers}{0}\setcounter{MSepNumbers}{0}\fi
\ifnum#1=5\setcounter{MSubjectArea}{5}\setcounter{MInfoNumbers}{1}\setcounter{MSepNumbers}{0}\fi
% Separate Nummerntechnik fuer unsere Chemiker: alles dreistellig
\ifnum#1=3
  \ifttm
  \renewcommand{\theequation}{\arabic{section}.\arabic{subsection}.\arabic{equation}}
  \renewcommand{\thetable}{\arabic{section}.\arabic{subsection}.\arabic{table}} 
  \renewcommand{\thefigure}{\arabic{section}.\arabic{subsection}.\arabic{figure}} 
  \else
  \renewcommand{\theequation}{\arabic{chapter}.\arabic{section}.\arabic{equation}}
  \renewcommand{\thetable}{\arabic{chapter}.\arabic{section}.\arabic{table}}
  \renewcommand{\thefigure}{\arabic{chapter}.\arabic{section}.\arabic{figure}}
  \fi
\else
  \ifttm
  \renewcommand{\theequation}{\arabic{section}.\arabic{subsection}.\arabic{equation}}
  \renewcommand{\thetable}{\arabic{table}}
  \renewcommand{\thefigure}{\arabic{figure}}
  \else
  \renewcommand{\theequation}{\arabic{chapter}.\arabic{section}.\arabic{equation}}
  \renewcommand{\thetable}{\arabic{table}}
  \renewcommand{\thefigure}{\arabic{figure}}
  \fi
\fi
}

% Fuer tikz Autogenerierung
\newcounter{MTIKZAutofilenumber}

% Spezielle Counter fuer die Bentz-Module
\newcounter{mycounter}
\newcounter{chemapplet}
\newcounter{physapplet}

\newcounter{MSSEnd} % Ist 1 falls ein MSubsection aktiv ist, der einen MSubsectionEndMacro-Aufruf verursacht
\newcounter{MFileNumber}
\def\MLastFile{\special{html:[[!-- mfileref;;}\arabic{MFileNumber}\special{html:; //--]]}}

% Vollstaendiger Pfad ist \MMaterial / \MLastFilePath / \MLastFileName    ==   \MMaterial / \MLastFile

% Wird nur bei kompletter Baum-Erstellung ausgefuehrt!
% #1 = Lesbare Modulbezeichnung
\newcommand{\MSectionStartMacros}[1]{
\setcounter{MTestSite}{0}
\setcounter{MCaptionOn}{0}
\setcounter{MLastTypeEq}{0}
\setcounter{MSSEnd}{0}
\setcounter{MFileNumber}{0} % Preinkrekement-Counter
\setcounter{MTIKZAutofilenumber}{0}
\setcounter{mycounter}{1}
\setcounter{physapplet}{1}
\setcounter{chemapplet}{0}
\ifttm
\special{html:<!-- mdeclaresection;;}\arabic{chapter}\special{html:;;}\arabic{section}\special{html:;;}#1\special{html:;; //-->}%
\else
\setcounter{thmc}{0}
\setcounter{exmpc}{0}
\setcounter{verc}{0}
\setcounter{infoc}{0}
\fi
\setcounter{MiniMarkerCounter}{1}
\setcounter{AlignCounter}{1}
\setcounter{MXCTest}{0}
\setcounter{MCodeCounter}{0}
\setcounter{MEntryCounter}{0}
}

% Wird immer ausgefuehrt
\newcommand{\MSubsectionStartMacros}{
\ifttm\else\MPageHeaderDef\fi
\MWatermarkSettings
\setcounter{MXCounter}{0}
\setcounter{MSCounter}{0}
\setcounter{MSiteCounter}{1}
\setcounter{MExerciseCollectionCounter}{0}
% Zaehler fuer das Labelsystem zuruecksetzen (prefix-Zaehler)
\setcounter{MInfoCounter}{0}
\setcounter{MExerciseCounter}{0}
\setcounter{MExampleCounter}{0}
\setcounter{MExperimentCounter}{0}
\setcounter{MGraphicsCounter}{0}
\setcounter{MTableCounter}{0}
\setcounter{MTheoremCounter}{0}
\setcounter{MObjectCounter}{0}
\setcounter{MEquationCounter}{0}
\setcounter{MVideoCounter}{0}
\setcounter{equation}{0}
\setcounter{figure}{0}
}

\newcommand{\MSubsectionEndMacros}{
% Bei Chemiemodulen das PSE einhaengen, es soll als SContent am Ende erscheinen
\special{html:<!-- subsectionend //-->}
\ifnum\value{MSubjectArea}=3{\MIncludePSE}\fi
}


\ifttm
%\newcommand{\MEmbed}[1]{\MRegisterFile{#1}\begin{html}<embed src="\end{html}\MMaterial/\MLastFile\begin{html}" width="192" height="189"></embed>\end{html}}
\newcommand{\MEmbed}[1]{\MRegisterFile{#1}\begin{html}<embed src="\end{html}\MMaterial/\MLastFile\begin{html}"></embed>\end{html}}
\fi

%----------------- Makros fuer die Textdarstellung -----------------------------------------------

\ifttm
% MUGraphics bindet eine Grafik ein:
% Parameter 1: Dateiname der Grafik, relativ zur Position des Modul-Tex-Dokuments
% Parameter 2: Skalierungsoptionen fuer PDF (fuer includegraphics)
% Parameter 3: Titel fuer die Grafik, wird unter die Grafik mit der Grafiknummer gesetzt und kann MLabel bzw. MCopyrightLabel enthalten
% Parameter 4: Skalierungsoptionen fuer HTML (css-styles)

% ERSATZ: <img alt="My Image" src="data:image/png;base64,iVBORwA<MoreBase64SringHere>" />


\newcommand{\MUGraphics}[4]{\MRegisterFile{#1}\begin{html}
<div class="imagecenter">
<center>
<div>
<img src="\end{html}\MMaterial/\MLastFile\begin{html}" style="#4" alt="\end{html}\MMaterial/\MLastFile\begin{html}"/>
</div>
<div class="bildtext">
\end{html}
\addtocounter{MGraphicsCounter}{1}
\setcounter{MLastIndex}{\value{MGraphicsCounter}}
\setcounter{MLastType}{8}
\addtocounter{MCaptionOn}{1}
\ifnum\value{MSepNumbers}=0
\textbf{Abbildung \arabic{MGraphicsCounter}:} #3
\else
\textbf{Abbildung \arabic{section}.\arabic{subsection}.\arabic{MGraphicsCounter}:} #3
\fi
\addtocounter{MCaptionOn}{-1}
\begin{html}
</div>
</center>
</div>
<br />
\end{html}%
\special{html:<!-- mfeedbackbutton;Abbildung;}\arabic{MGraphicsCounter}\special{html:;}\arabic{section}.\arabic{subsection}.\arabic{MGraphicsCounter}\special{html:; //-->}%
}

% MVideo bindet ein Video als Einzeldatei ein:
% Parameter 1: Dateiname des Videos, relativ zur Position des Modul-Tex-Dokuments, ohne die Endung ".mp4"
% Parameter 2: Titel fuer das Video (kann MLabel oder MCopyrightLabel enthalten), wird unter das Video mit der Videonummer gesetzt
\newcommand{\MVideo}[2]{\MRegisterFile{#1.mp4}\begin{html}
<div class="imagecenter">
<center>
<div>
<video width="95\%" controls="controls"><source src="\end{html}\MMaterial/#1.mp4\begin{html}" type="video/mp4">Ihr Browser kann keine MP4-Videos abspielen!</video>
</div>
<div class="bildtext">
\end{html}
\addtocounter{MVideoCounter}{1}
\setcounter{MLastIndex}{\value{MVideoCounter}}
\setcounter{MLastType}{12}
\addtocounter{MCaptionOn}{1}
\ifnum\value{MSepNumbers}=0
\textbf{Video \arabic{MVideoCounter}:} #2
\else
\textbf{Video \arabic{section}.\arabic{subsection}.\arabic{MVideoCounter}:} #2
\fi
\addtocounter{MCaptionOn}{-1}
\begin{html}
</div>
</center>
</div>
<br />
\end{html}}

\newcommand{\MDVideo}[2]{\MRegisterFile{#1.mp4}\MRegisterFile{#1.ogv}\begin{html}
<div class="imagecenter">
<center>
<div>
<video width="70\%" controls><source src="\end{html}\MMaterial/#1.mp4\begin{html}" type="video/mp4"><source src="\end{html}\MMaterial/#1.ogv\begin{html}" type="video/ogg">Ihr Browser kann keine MP4-Videos abspielen!</video>
</div>
<br />
#2
</center>
</div>
<br />
\end{html}
}

\newcommand{\MGraphics}[3]{\MUGraphics{#1}{#2}{#3}{}}

\else

\newcommand{\MVideo}[2]{%
% Kein Video im PDF darstellbar, trotzdem so tun als ob da eines waere
\begin{center}
(Video nicht darstellbar)
\end{center}
\addtocounter{MVideoCounter}{1}
\setcounter{MLastIndex}{\value{MVideoCounter}}
\setcounter{MLastType}{12}
\addtocounter{MCaptionOn}{1}
\ifnum\value{MSepNumbers}=0
\textbf{Video \arabic{MVideoCounter}:} #2
\else
\textbf{Video \arabic{section}.\arabic{subsection}.\arabic{MVideoCounter}:} #2
\fi
\addtocounter{MCaptionOn}{-1}
}


% MGraphics bindet eine Grafik ein:
% Parameter 1: Dateiname der Grafik, relativ zur Position des Modul-Tex-Dokuments
% Parameter 2: Skalierungsoptionen fuer PDF (fuer includegraphics)
% Parameter 3: Titel fuer die Grafik, wird unter die Grafik mit der Grafiknummer gesetzt
\newcommand{\MGraphics}[3]{%
\MRegisterFile{#1}%
\ %
\begin{figure}[H]%
\centering{%
\includegraphics[#2]{\MDPrefix/#1}%
\addtocounter{MCaptionOn}{1}%
\caption{#3}%
\addtocounter{MCaptionOn}{-1}%
}%
\end{figure}%
\addtocounter{MGraphicsCounter}{1}\setcounter{MLastIndex}{\value{MGraphicsCounter}}\setcounter{MLastType}{8}\ %
%\ \\Abbildung \ifnum\value{MSepNumbers}=0\else\arabic{chapter}.\arabic{section}.\fi\arabic{MGraphicsCounter}: #3%
}

\newcommand{\MUGraphics}[4]{\MGraphics{#1}{#2}{#3}}


\fi

\newcounter{MCaptionOn} % = 1 falls eine Grafikcaption aktiv ist, = 0 sonst


% MGraphicsSolo bindet eine Grafik pur ein ohne Titel
% Parameter 1: Dateiname der Grafik, relativ zur Position des Modul-Tex-Dokuments
% Parameter 2: Skalierungsoptionen (wirken nur im PDF)
\newcommand{\MGraphicsSolo}[2]{\MUGraphicsSolo{#1}{#2}{}}

% MUGraphicsSolo bindet eine Grafik pur ein ohne Titel, aber mit HTML-Skalierung
% Parameter 1: Dateiname der Grafik, relativ zur Position des Modul-Tex-Dokuments
% Parameter 2: Skalierungsoptionen (wirken nur im PDF)
% Parameter 3: Skalierungsoptionen (wirken nur im HTML), als style-format: "width=???, height=???"
\ifttm
\newcommand{\MUGraphicsSolo}[3]{\MRegisterFile{#1}\begin{html}
<img src="\end{html}\MMaterial/\MLastFile\begin{html}" style="\end{html}#3\begin{html}" alt="\end{html}\MMaterial/\MLastFile\begin{html}"/>
\end{html}%
\special{html:<!-- mfeedbackbutton;Abbildung;}#1\special{html:;}\MMaterial/\MLastFile\special{html:; //-->}%
}
\else
\newcommand{\MUGraphicsSolo}[3]{\MRegisterFile{#1}\includegraphics[#2]{\MDPrefix/#1}}
\fi

% Externer Link mit URL
% Erster Parameter: Vollstaendige(!) URL des Links
% Zweiter Parameter: Text fuer den Link
\newcommand{\MExtLink}[2]{\ifttm\special{html:<a target="_new" href="}#1\special{html:">}#2\special{html:</a>}\else\href{#1}{#2}\fi} % ohne MINTERLINK!


% Interner Link, die verlinkte Datei muss im gleichen Verzeichnis liegen wie die Modul-Texdatei
% Erster Parameter: Dateiname
% Zweiter Parameter: Text fuer den Link
\newcommand{\MIntLink}[2]{\ifttm\MRegisterFile{#1}\special{html:<a class="MINTERLINK" target="_new" href="}\MMaterial/\MLastFile\special{html:">}#2\special{html:</a>}\else{\href{#1}{#2}}\fi}


\ifttm
\def\MMaterial{:localmaterial:}
\else
\def\MMaterial{\MDPrefix}
\fi

\ifttm
\def\MNoFile#1{:directmaterial:#1}
\else
\def\MNoFile#1{#1}
\fi

\newcommand{\MChem}[1]{$\mathrm{#1}$}

\newcommand{\MApplet}[3]{
% Bindet ein Java-Applet ein, die Parameter sind:
% (wird nur im HTML, aber nicht im PDF erstellt)
% #1 Dateiname des Applets (muss mit ".class" enden)
% #2 = Breite in Pixeln
% #3 = Hoehe in Pixeln
\ifttm
\MRegisterFile{#1}
\begin{html}
<applet code="\end{html}\MMaterial/\MLastFile\begin{html}" width="#2" height="#3" alt="[Java-Applet kann nicht gestartet werden]"></applet>
\end{html}
\fi
}

\newcommand{\MScriptPage}[2]{
% Bindet eine JavaScript-Datei ein, die eine eigene Seite bekommt
% (wird nur im HTML, aber nicht im PDF erstellt)
% #1 Dateiname des Programms (sollte mit ".js" enden)
% #2 = Kurztitel der Seite
\ifttm
\begin{MSContent}{#2}{#2}{puzzle}
\MRegisterFile{#1}
\begin{html}
<script src="\MMaterial/\MLastFile" type="text/javascript"></script>
\end{html}
\end{MSContent}
\fi
}

\newcommand{\MIncludePSE}{
% Bindet bei Chemie-Modulen das PSE ein
% (wird nur im HTML, aber nicht im PDF erstellt)
\ifttm
\special{html:<!-- includepse //-->}
\begin{MSContent}{Periodensystem der Elemente}{PSE}{table}
\MRegisterFile{../files/pse.js}
\MRegisterFile{../files/radio.png}
\begin{html}
<script src="\MMaterial/../files/pse.js" type="text/javascript"></script>
<p id="divid"><br /><br />
<script language="javascript" type="text/javascript">
    startpse("divid","\MMaterial/../files"); 
</script>
</p>
<br />
<br />
<br />
<p>Die Farben der Elementsymbole geben an: <font style="color:Red">gasf&ouml;rmig </font> <font style="color:Blue">fl&uuml;ssig </font> fest</p>
<p>Die Elemente der Gruppe 1 A, 2 A, 3 A usw. geh&ouml;ren zu den Hauptgruppenelementen.</p>
<p>Die Elemente der Gruppe 1 B, 2 B, 3 B usw. geh&ouml;ren zu den Nebengruppenelementen.</p>
<p>() kennzeichnet die Masse des stabilsten Isotops</p>
\end{html}
\end{MSContent}
\fi
}

\newcommand{\MAppletArchive}[4]{
% Bindet ein Java-Applet ein, die Parameter sind:
% (wird nur im HTML, aber nicht im PDF erstellt)
% #1 Dateiname der Klasse mit Appletaufruf (muss mit ".class" enden)
% #2 Dateiname des Archivs (muss mit ".jar" enden)
% #3 = Breite in Pixeln
% #4 = Hoehe in Pixeln
\ifttm
\MRegisterFile{#2}
\begin{html}
<applet code="#1" archive="\end{html}\MMaterial/\MLastFile\begin{html}" codebase="." width="#3" height="#4" alt="[Java-Archiv kann nicht gestartet werden]"></applet>
\end{html}
\fi
}

% Bindet in der Haupttexdatei ein MINT-Modul ein. Parameter 1 ist das Verzeichnis (relativ zur Haupttexdatei), Parameter 2 ist der Dateinahme ohne Pfad.
\newcommand{\IncludeModule}[2]{
\renewcommand{\MDPrefix}{#1}
\input{#1/#2}
\ifnum\value{MSSEnd}>0{\MSubsectionEndMacros}\addtocounter{MSSEnd}{-1}\fi
}

% Der ttm-Konverter setzt keine Makros im \input um, also muss hier getrickst werden:
% Das MDPrefix muss in den einzelnen Modulen manuell eingesetzt werden
\newcommand{\MInputFile}[1]{
\ifttm
\input{#1}
\else
\input{#1}
\fi
}


\newcommand{\MCases}[1]{\left\lbrace{\begin{array}{rl} #1 \end{array}}\right.}

\ifttm
\newenvironment{MCaseEnv}{\left\lbrace\begin{array}{rl}}{\end{array}\right.}
\else
\newenvironment{MCaseEnv}{\left\lbrace\begin{array}{rl}}{\end{array}\right.}
\fi

\def\MSkip{\ifttm\MCR\fi}

\ifttm
\def\MCR{\special{html:<br />}}
\else
\def\MCR{\ \\}
\fi


% Pragmas - Sind Schluesselwoerter, die dem Preprocessing sowie dem Konverter uebergeben werden und bestimmte
%           Aktionen ausloesen. Im Output (PDF und HTML) tauchen sie nicht auf.
\newcommand{\MPragma}[1]{%
\ifttm%
\special{html:<!-- mpragma;-;}#1\special{html:;; -->}%
\else%
% MPragmas werden vom Preprozessor direkt im LaTeX gefunden
\fi%
}

% Ersatz der Befehle textsubscript und textsuperscript, die ttm nicht kennt
\ifttm%
\newcommand{\MTextsubscript}[1]{\special{html:<sub>}#1\special{html:</sub>}}%
\newcommand{\MTextsuperscript}[1]{\special{html:<sup>}#1\special{html:</sup>}}%
\else%
\newcommand{\MTextsubscript}[1]{\textsubscript{#1}}%
\newcommand{\MTextsuperscript}[1]{\textsuperscript{#1}}%
\fi

%------------------ Einbindung von dia-Diagrammen ----------------------------------------------
% Beim preprocessing wird aus jeder dia-Datei eine tex-Datei und eine pdf-Datei erzeugt,
% diese werden hier jeweils im PDF und HTML eingebunden
% Parameter: Dateiname der mit dia erstellten Datei (OHNE die Endung .dia)
\ifttm%
\newcommand{\MDia}[1]{%
\MGraphicsSolo{#1minthtml.png}{}%
}
\else%
\newcommand{\MDia}[1]{%
\MGraphicsSolo{#1mintpdf.png}{scale=0.1667}%
}
\fi%

% subsup funktioniert im Ausdruck $D={\R}^+_0$, also \R geklammert und sup zuerst
% \ifttm
% \def\MSubsup#1#2#3{\special{html:<msubsup>} #1 #2 #3\special{html:</msubsup>}}
% \else
% \def\MSubsup#1#2#3{{#1}^{#3}_{#2}}
% \fi

%\input{local.tex}

% \ifttm
% \else
% \newwrite\mintlog
% \immediate\openout\mintlog=mintlog.txt
% \fi

% ----------------------- tikz autogenerator -------------------------------------------------------------------

\newcommand{\Mtikzexternalize}{\tikzexternalize}% wird bei Konvertierung ueber mconvert ggf. ausgehebelt!

\ifttm
\else
\tikzset%
{
  % Defines a custom style which generates pdf and converts to (low and hi-res quality) png and svg, then deletes the pdf
  % Important: DO NOT directly convert from pdf to hires-png or from svg to png with GraphViz convert as it has some problems and memory leaks
  png export/.style=%
  {
    external/system call/.add={}{; 
      pdf2svg "\image.pdf" "\image.svg" ; 
      convert -density 112.5 -transparent white "\image.pdf" "\image.png"; 
      inkscape --export-png="\image.4x.png" --export-dpi=450 --export-background-opacity=0 --without-gui "\image.svg"; 
      rm "\image.pdf"; rm "\image.log"; rm "\image.dpth"; rm "\image.idx"
    },
    external/force remake,
  }
}
\tikzset{png export}
\tikzsetexternalprefix{}
% PNGs bei externer Erzeugung in "richtiger" Groesse einbinden
\pgfkeys{/pgf/images/include external/.code={\includegraphics[scale=0.64]{#1}}}
\fi

% Spezielle Umgebung fuer Autogenerierung, Bildernamen sind nur innerhalb eines Moduls (einer MSection) eindeutig)

\newcommand{\MTIKZautofilename}{tikzautofile}

\ifttm
% HTML-Version: Vom Autogenerator erzeugte png-Datei einbinden, tikz selbst nicht ausfuehren (sprich: #1 schlucken)
\newcommand{\MTikzAuto}[1]{%
\addtocounter{MTIKZAutofilenumber}{1}
\renewcommand{\MTIKZautofilename}{mtikzauto_\arabic{MTIKZAutofilenumber}}
\MUGraphicsSolo{\MSectionID\MTIKZautofilename.4x.png}{scale=1}{\special{html:[[!-- svgstyle;}\MSectionID\MTIKZautofilename\special{html: //--]]}} % Styleinfos werden aus original-png, nicht 4x-png geholt!
%\MRegisterFile{\MSectionID\MTIKZautofilename.png} % not used right now
%\MRegisterFile{\MSectionID\MTIKZautofilename.svg}
}
\else%
% PDF-Version: Falls Autogenerator aktiv wird Datei automatisch benannt und exportiert
\newcommand{\MTikzAuto}[1]{%
\addtocounter{MTIKZAutofilenumber}{1}%
\renewcommand{\MTIKZautofilename}{mtikzauto_\arabic{MTIKZAutofilenumber}}
\tikzsetnextfilename{\MTIKZautofilename}%
#1%
}
\fi

% In einer reinen LaTeX-Uebersetzung kapselt der Preambelinclude-Befehl nur input,
% in einer konvertergesteuerten PDF/HTML-Uebersetzung wird er dagegen entfernt und
% die Preambeln an mintmod angehaengt, die Ersetzung wird von mconvert.pl vorgenommen.

\newcommand{\MPreambleInclude}[1]{\input{#1}}

% Globale Watermarksettings (werden auch nochmal zu Beginn jedes subsection gesetzt,
% muessen hier aber auch global ausgefuehrt wegen Einfuehrungsseiten und Inhaltsverzeichnis

\MWatermarkSettings
% ---------------------------------- Parametrisierte Aufgaben ----------------------------------------

\ifttm
\newenvironment{MPExercise}{%
\begin{MExercise}%
}{%
\special{html:<button name="Name_MPEX}\arabic{MExerciseCounter}\special{html:" id="MPEX}\arabic{MExerciseCounter}%
\special{html:" type="button" onclick="reroll('}\arabic{MExerciseCounter}\special{html:');">Neue Aufgabe erzeugen</button>}%
\end{MExercise}%
}
\else
\newenvironment{MPExercise}{%
\begin{MExercise}%
}{%
\end{MExercise}%
}
\fi

% Parameter: Name, Min, Max, PDF-Standard. Name in Deklaration OHNE backslash, im Code MIT Backslash
\ifttm
\newcommand{\MGlobalInteger}[4]{\special{html:%
<!-- onloadstart //-->%
MVAR.push(createGlobalInteger("}#1\special{html:",}#2\special{html:,}#3\special{html:,}#4\special{html:)); %
<!-- onloadstop //-->%
<!-- viewmodelstart //-->%
ob}#1\special{html:: ko.observable(rerollMVar("}#1\special{html:")),%
<!-- viewmodelstop //-->%
}%
}%
\else%
\newcommand{\MGlobalInteger}[4]{\newcounter{mvc_#1}\setcounter{mvc_#1}{#4}}
\fi

% Parameter: Name, Min, Max, PDF-Standard. Name in Deklaration OHNE backslash, im Code MIT Backslash, Wert ist Wurzel von value
\ifttm
\newcommand{\MGlobalSqrt}[4]{\special{html:%
<!-- onloadstart //-->%
MVAR.push(createGlobalSqrt("}#1\special{html:",}#2\special{html:,}#3\special{html:,}#4\special{html:)); %
<!-- onloadstop //-->%
<!-- viewmodelstart //-->%
ob}#1\special{html:: ko.observable(rerollMVar("}#1\special{html:")),%
<!-- viewmodelstop //-->%
}%
}%
\else%
\newcommand{\MGlobalSqrt}[4]{\newcounter{mvc_#1}\setcounter{mvc_#1}{#4}}% Funktioniert nicht als Wurzel !!!
\fi

% Parameter: Name, Min, Max, PDF-Standard zaehler, PDF-Standard nenner. Name in Deklaration OHNE backslash, im Code MIT Backslash
\ifttm
\newcommand{\MGlobalFraction}[5]{\special{html:%
<!-- onloadstart //-->%
MVAR.push(createGlobalFraction("}#1\special{html:",}#2\special{html:,}#3\special{html:,}#4\special{html:,}#5\special{html:)); %
<!-- onloadstop //-->%
<!-- viewmodelstart //-->%
ob}#1\special{html:: ko.observable(rerollMVar("}#1\special{html:")),%
<!-- viewmodelstop //-->%
}%
}%
\else%
\newcommand{\MGlobalFraction}[5]{\newcounter{mvc_#1}\setcounter{mvc_#1}{#4}} % Funktioniert nicht als Bruch !!!
\fi

% MVar darf im HTML nur in MEvalMathDisplay-Umgebungen genutzt werden oder in Strings die an den Parser uebergeben werden
\ifttm%
\newcommand{\MVar}[1]{\special{html:[var_}#1\special{html:]}}%
\else%
\newcommand{\MVar}[1]{\arabic{mvc_#1}}%
\fi

\ifttm%
\newcommand{\MRerollButton}[2]{\special{html:<button type="button" onclick="rerollMVar('}#1\special{html:');">}#2\special{html:</button>}}%
\else%
\newcommand{\MRerollButton}[2]{\relax}% Keine sinnvolle Entsprechung im PDF
\fi

% MEvalMathDisplay fuer HTML wird in mconvert.pl im preprocessing realisiert
% PDF: eine equation*-Umgebung (ueber amsmath)
% HTML: Eine Mathjax-Tex-Umgebung, deren Auswertung mit knockout-obervablen gekoppelt ist
% PDF-Version hier nur fuer pdflatex-only-Uebersetzung gegeben

\ifttm\else\newenvironment{MEvalMathDisplay}{\begin{equation*}}{\end{equation*}}\fi

% ---------------------------------- Spezialbefehle fuer AD ------------------------------------------

%Abk�rzung f�r \longrightarrow:
\newcommand{\lto}{\ensuremath{\longrightarrow}}

%Makro f�r Funktionen:
\newcommand{\exfunction}[5]
{\begin{array}{rrcl}
 #1 \colon  & #2 &\lto & #3 \\[.05cm]  
  & #4 &\longmapsto  & #5 
\end{array}}

\newcommand{\function}[5]{%
#1:\;\left\lbrace{\begin{array}{rcl}
 #2 &\lto & #3 \\
 #4 &\longmapsto  & #5 \end{array}}\right.}


%Die Identit�t:
\DeclareMathOperator{\Id}{Id}

%Die Signumfunktion:
\DeclareMathOperator{\sgn}{sgn}

%Zwei Betonungskommandos (k�nnen angepasst werden):
\newcommand{\highlight}[1]{#1}
\newcommand{\modstextbf}[1]{#1}
\newcommand{\modsemph}[1]{#1}


% ---------------------------------- Spezialbefehle fuer JL ------------------------------------------


\def\jccolorfkt{green!50!black} %Farbe des Funktionsgraphen
\def\jccolorfktarea{green!25!white} %Farbe der Fl"ache unter dem Graphen
\def\jccolorfktareahell{green!12!white} %helle Einf"arbung der Fl"ache unter dem Graphen
\def\jccolorfktwert{green!50!black} %Farbe einzelner Punkte des Graphen

\newcommand{\MPfadBilder}{Bilder}

\ifttm%
\newcommand{\jMD}{\,\MD}%
\else%
\newcommand{\jMD}{\;\MD}%
\fi%

\def\jHTMLHinweisBedienung{\MInputHint{%
Mit Hilfe der Symbole am oberen Rand des Fensters
k"onnen Sie durch die einzelnen Abschnitte navigieren.}}

\def\jHTMLHinweisEingabeText{\MInputHint{%
Geben Sie jeweils ein Wort oder Zeichen als Antwort ein.}}

\def\jHTMLHinweisEingabeTerm{\MInputHint{%
Klammern Sie Ihre Terme, um eine eindeutige Eingabe zu erhalten. 
Beispiel: Der Term $\frac{3x+1}{x-2}$ soll in der Form
\texttt{(3*x+1)/((x+2)^2}$ eingegeben werden (wobei auch Leerzeichen 
eingegeben werden k"onnen, damit eine Formel besser lesbar ist).}}

\def\jHTMLHinweisEingabeIntervalle{\MInputHint{%
Intervalle werden links mit einer "offnenden Klammer und rechts mit einer 
schlie"senden Klammer angegeben. Eine runde Klammer wird verwendet, wenn der 
Rand nicht dazu geh"ort, eine eckige, wenn er dazu geh"ort. 
Als Trennzeichen wird ein Komma oder ein Semikolon akzeptiert.
Beispiele: $(a, b)$ offenes Intervall,
$[a; b)$ links abgeschlossenes, rechts offenes Intervall von $a$ bis $b$. 
Die Eingabe $]a;b[$ f"ur ein offenes Intervall wird nicht akzeptiert.
F"ur $\infty$ kann \texttt{infty} oder \texttt{unendlich} geschrieben werden.}}

\def\jHTMLHinweisEingabeFunktionen{\MInputHint{%
Schreiben Sie Malpunkte (geschrieben als \texttt{*}) aus und setzen Sie Klammern um Argumente f�r Funktionen.
Beispiele: Polynom: \texttt{3*x + 0.1}, Sinusfunktion: \texttt{sin(x)}, 
Verkettung von cos und Wurzel: \texttt{cos(sqrt(3*x))}.}}

\def\jHTMLHinweisEingabeFunktionenSinCos{\MInputHint{%
Die Sinusfunktion $\sin x$ wird in der Form \texttt{sin(x)} angegeben, %
$\cos\left(\sqrt{3 x}\right)$ durch \texttt{cos(sqrt(3*x))}.}}

\def\jHTMLHinweisEingabeFunktionenExp{\MInputHint{%
Die Exponentialfunktion $\MEU^{3x^4 + 5}$ wird als
\texttt{exp(3 * x^4 + 5)} angegeben, %
$\ln\left(\sqrt{x} + 3.2\right)$ durch \texttt{ln(sqrt(x) + 3.2)}.}}

% ---------------------------------- Spezialbefehle fuer Fachbereich Physik --------------------------

\newcommand{\E}{{e}}
\newcommand{\ME}[1]{\cdot 10^{#1}}
\newcommand{\MU}[1]{\;\mathrm{#1}}
\newcommand{\MPG}[3]{%
  \ifnum#2=0%
    #1\ \mathrm{#3}%
  \else%
    #1\cdot 10^{#2}\ \mathrm{#3}%
  \fi}%
%

\newcommand{\MMul}{\MExponentensymbXYZl} % Nur eine Abkuerzung


% ---------------------------------- Stichwortfunktionialitaet ---------------------------------------

% mpreindexentry wird durch Auswahlroutine in conv.pl durch mindexentry substitutiert
\ifttm%
\def\MIndex#1{\index{#1}\special{html:<!-- mpreindexentry;;}#1\special{html:;;}\arabic{MSubjectArea}\special{html:;;}%
\arabic{chapter}\special{html:;;}\arabic{section}\special{html:;;}\arabic{subsection}\special{html:;;}\arabic{MEntryCounter}\special{html:; //-->}%
\setcounter{MLastIndex}{\value{MEntryCounter}}%
\addtocounter{MEntryCounter}{1}%
}%
% Copyrightliste wird als tex-Datei im preprocessing von conv.pl erzeugt und unter converter/tex/entrycollection.tex abgelegt
% Der input-Befehl funktioniert nur, wenn die aufrufende tex-Datei auf der obersten Ebene liegt (d.h. selbst kein input/include ist, insbesondere keine Moduldatei)
\def\MEntryList{} % \input funktioniert nicht, weil ttm (und damit das \input) ausgefuehrt wird, bevor Datei da ist
\else%
\def\MIndex#1{\index{#1}}
\def\MEntryList{\MAbort{Stichwortliste nur im HTML realisierbar}}%
\fi%

\def\MEntry#1#2{\textbf{#1}\MIndex{#2}} % Idee: MLastType auf neuen Entry-Typ und dann ein MLabel vergeben mit autogen-Nummer

% ---------------------------------- Befehle fuer Tests ----------------------------------------------

% MEquationItem stellt eine Eingabezeile der Form Vorgabe = Antwortfeld her, der zweite Parameter kann z.B. MSimplifyQuestion-Befehl sein
\ifttm
\newcommand{\MEquationItem}[2]{{#1}$\,=\,${#2}}%
\else%
\newcommand{\MEquationItem}[2]{{#1}$\;\;=\,${#2}}%
\fi

\ifttm
\newcommand{\MInputHint}[1]{%
\ifnum%
\if\value{MTestSite}>0%
\else%
{\color{blue}#1}%
\fi%
\fi%
}
\else
\newcommand{\MInputHint}[1]{\relax}
\fi

\ifttm
\newcommand{\MInTestHeader}{%
Dies ist ein einreichbarer Test:
\begin{itemize}
\item{Im Gegensatz zu den offenen Aufgaben werden beim Eingeben keine Hinweise zur Formulierung der mathematischen Ausdr�cke gegeben.}
\item{Der Test kann jederzeit neu gestartet oder verlassen werden.}
\item{Der Test kann durch die Buttons am Ende der Seite beendet und abgeschickt, oder zur�ckgesetzt werden.}
\item{Der Test kann mehrfach probiert werden. F�r die Statistik z�hlt die zuletzt abgeschickte Version.}
\end{itemize}
}
\else
\newcommand{\MInTestHeader}{%
\relax
}
\fi

\ifttm
\newcommand{\MInTestFooter}{%
\special{html:<button name="Name_TESTFINISH" id="TESTFINISH" type="button" onclick="finish_button('}\MTestName\special{html:');">Test auswerten</button>}%
\begin{html}
&nbsp;&nbsp;&nbsp;&nbsp;&nbsp;&nbsp;&nbsp;&nbsp;
<button name="Name_TESTRESET" id="TESTRESET" type="button" onclick="reset_button();">Test zur�cksetzen</button>
<br />
<br />
<div class="xreply">
<p name="Name_TESTEVAL" id="TESTEVAL">
Hier erscheint die Testauswertung!
<br />
</p>
</div>
\end{html}
}
\else
\newcommand{\MInTestFooter}{%
\relax
}
\fi


% ---------------------------------- Notationsmakros -------------------------------------------------------------

% Notationsmakros die nicht von der Kursvariante abhaengig sind

\newcommand{\MZahltrennzeichen}[1]{\renewcommand{\MZXYZhltrennzeichen}{#1}}

\ifttm
\newcommand{\MZahl}[3][\MZXYZhltrennzeichen]{\edef\MZXYZtemp{\noexpand\special{html:<mn>#2#1#3</mn>}}\MZXYZtemp}
\else
\newcommand{\MZahl}[3][\MZXYZhltrennzeichen]{{}#2{#1}#3}
\fi

\newcommand{\MEinheitenabstand}[1]{\renewcommand{\MEinheitenabstXYZnd}{#1}}
\ifttm
\newcommand{\MEinheit}[2][\MEinheitenabstXYZnd]{{}#1\edef\MEINHtemp{\noexpand\special{html:<mi mathvariant="normal">#2</mi>}}\MEINHtemp} 
\else
\newcommand{\MEinheit}[2][\MEinheitenabstXYZnd]{{}#1 \mathrm{#2}} 
\fi

\newcommand{\MExponentensymbol}[1]{\renewcommand{\MExponentensymbXYZl}{#1}}
\newcommand{\MExponent}[2][\MExponentensymbXYZl]{{}#1{} 10^{#2}} 

%Punkte in 2 und 3 Dimensionen
\newcommand{\MPointTwo}[3][]{#1(#2\MCoordPointSep #3{}#1)}
\newcommand{\MPointThree}[4][]{#1(#2\MCoordPointSep #3\MCoordPointSep #4{}#1)}
\newcommand{\MPointTwoAS}[2]{\left(#1\MCoordPointSep #2\right)}
\newcommand{\MPointThreeAS}[3]{\left(#1\MCoordPointSep #2\MCoordPointSep #3\right)}

% Masseinheit, Standardabstand: \,
\newcommand{\MEinheitenabstXYZnd}{\MThinspace} 

% Horizontaler Leerraum zwischen herausgestellter Formel und Interpunktion
\ifttm
\newcommand{\MDFPSpace}{\,}
\newcommand{\MDFPaSpace}{\,\,}
\newcommand{\MBlank}{\ }
\else
\newcommand{\MDFPSpace}{\;}
\newcommand{\MDFPaSpace}{\;\;}
\newcommand{\MBlank}{\ }
\fi

% Satzende in herausgestellter Formel mit horizontalem Leerraum
\newcommand{\MDFPeriod}{\MDFPSpace .}

% Separation von Aufzaehlung und Bedingung in Menge
\newcommand{\MCondSetSep}{\,:\,} %oder '\mid'

% Konverter kennt mathopen nicht
\ifttm
\def\mathopen#1{}
\fi

% -----------------------------------START Rouletteaufgaben ------------------------------------------------------------

\ifttm
% #1 = Dateiname, #2 = eindeutige ID fuer das Roulette im Kurs
\newcommand{\MDirectRouletteExercises}[2]{
\begin{MExercise}
\texttt{Im HTML erscheinen hier Aufgaben aus einer Aufgabenliste...}
\end{MExercise}
}
\else
\newcommand{\MDirectRouletteExercises}[2]{\relax} % wird durch mconvert.pl gefunden und ersetzt
\fi


% ---------------------------------- START Makros, die von der Kursvariante abhaengen ----------------------------------

\ifvariantunotation
  % unotation = An Universitaeten uebliche Notation
  \def\MVariant{unotation}

  % Trennzeichen fuer Dezimalzahlen
  \newcommand{\MZXYZhltrennzeichen}{.}

  % Exponent zur Basis 10 in der Exponentialschreibweise, 
  % Standardmalzeichen: \times
  \newcommand{\MExponentensymbXYZl}{\times} 

  % Begrenzungszeichen fuer offene Intervalle
  \newcommand{\MoIl}[1][]{\mbox{}#1(\mathopen{}} % bzw. ']'
  \newcommand{\MoIr}[1][]{#1)\mbox{}} % bzw. '['

  % Zahlen-Separation im IntervaLL
  \newcommand{\MIntvlSep}{,} %oder ';'

  % Separation von Elementen in Mengen
  \newcommand{\MElSetSep}{,} %oder ';'

  % Separation von Koordinaten in Punkten
  \newcommand{\MCoordPointSep}{,} %oder ';' oder '|', '\MThinspace|\MThinspace'

\else
  % An dieser Stelle wird angenommen, dass std-Variante aktiv ist
  % std = beschlossene Notation im TU9-Onlinekurs 
  \def\MVariant{std}

  % Trennzeichen fuer Dezimalzahlen
  \newcommand{\MZXYZhltrennzeichen}{,}

  % Exponent zur Basis 10 in der Exponentialschreibweise, 
  % Standardmalzeichen: \times
  \newcommand{\MExponentensymbXYZl}{\times} 

  % Begrenzungszeichen fuer offene Intervalle
  \newcommand{\MoIl}[1][]{\mbox{}#1]\mathopen{}} % bzw. '('
  \newcommand{\MoIr}[1][]{#1[\mbox{}} % bzw. ')'

  % Zahlen-Separation im IntervaLL
  \newcommand{\MIntvlSep}{;} %oder ','
  
  % Separation von Elementen in Mengen
  \newcommand{\MElSetSep}{;} %oder ','

  % Separation von Koordinaten in Punkten
  \newcommand{\MCoordPointSep}{;} %oder '|', '\MThinspace|\MThinspace'

\fi



% ---------------------------------- ENDE Makros, die von der Kursvariante abhaengen ----------------------------------


% diese Kommandos setzen Mathemodus vorraus
\newcommand{\MGeoAbstand}[2]{[\overline{{#1}{#2}}]}
\newcommand{\MGeoGerade}[2]{{#1}{#2}}
\newcommand{\MGeoStrecke}[2]{\overline{{#1}{#2}}}
\newcommand{\MGeoDreieck}[3]{{#1}{#2}{#3}}

%
\ifttm
\newcommand{\MOhm}{\special{html:<mn>&#x3A9;</mn>}}
\else
\newcommand{\MOhm}{\Omega} %\varOmega
\fi


\def\PERCTAG{\MAbort{PERCTAG ist zur internen verwendung in mconvert.pl reserviert, dieses Makro darf sonst nicht benutzt werden.}}

% Im Gegensatz zu einfachen html-Umgebungen werden MDirectHTML-Umgebungen von mconvert.pl am ganzen ttm-Prozess vorbeigeschleust und aus dem PDF komplett ausgeschnitten
\ifttm%
\newenvironment{MDirectHTML}{\begin{html}}{\end{html}}%
\else%
\newenvironment{MDirectHTML}{\begin{html}}{\end{html}}%
\fi

% Im Gegensatz zu einfachen Mathe-Umgebungen werden MDirectMath-Umgebungen von mconvert.pl am ganzen ttm-Prozess vorbeigeschleust, ueber MathJax realisiert, und im PDF als $$ ... $$ gesetzt
\ifttm%
\newenvironment{MDirectMath}{\begin{html}}{\end{html}}%
\else%
\newenvironment{MDirectMath}{\begin{equation*}}{\end{equation*}}% Vorsicht, auch \[ und \] werden in amsmath durch equation* redefiniert
\fi

% ---------------------------------- Location Management ---------------------------------------------

% #1 = buttonname (muss in files/images liegen und Format 48x48 haben), #2 = Vollstaendiger Einrichtungsname, #3 = Kuerzel der Einrichtung,  #4 = Name der include-texdatei
\ifttm
\newcommand{\MLocationSite}[3]{\special{html:<!-- mlocation;;}#1\special{html:;;}#2\special{html:;;}#3\special{html:;; //-->}}
\else
\newcommand{\MLocationSite}[3]{\relax}
\fi

% ---------------------------------- Copyright Management --------------------------------------------

\newcommand{\MCCLicense}{%
{\color{green}\textbf{CC BY-SA 3.0}}
}

\newcommand{\MCopyrightLabel}[1]{ (\MSRef{L_COPYRIGHTCOLLECTION}{Lizenz})\MLabel{#1}}

% Copyrightliste wird als tex-Datei im preprocessing erzeugt und unter converter/tex/copyrightcollection.tex abgelegt
% Der input-Befehl funktioniert nur, wenn die aufrufende tex-Datei auf der obersten Ebene liegt (d.h. selbst kein input/include ist, insbesondere keine Moduldatei)
\newcommand{\MCopyrightCollection}{\input{copyrightcollection.tex}}

% MCopyrightNotice fuegt eine Copyrightnotiz ein, der parser ersetzt diese durch CopyrightNoticePOST im preparsing, diese Definition wird nur fuer reine pdflatex-Uebersetzungen gebraucht
% Parameter: #1: Kurze Lizenzbeschreibung (typischerweise \MCCLicense)
%            #2: Link zum Original (http://...) oder NONE falls das Bild selbst ein Original ist, oder TIKZ falls das Bild aus einer tikz-Umgebung stammt
%            #3: Link zum Autor (http://...) oder MINT falls Original im MINT-Kolleg erstellt oder NONE falls Autor unbekannt
%            #4: Bemerkung (z.B. dass Datei mit Maple exportiert wurde)
%            #5: Labelstring fuer existierendes Label auf das copyrighted Objekt, mit MCopyrightLabel erzeugt
%            Keines der Felder darf leer sein!
\newcommand{\MCopyrightNotice}[5]{\MCopyrightNoticePOST{#1}{#2}{#3}{#4}{#5}}

\ifttm%
\newcommand{\MCopyrightNoticePOST}[5]{\relax}%
\else%
\newcommand{\MCopyrightNoticePOST}[5]{\relax}%
\fi%

% ---------------------------------- Meldungen fuer den Benutzer des Konverters ----------------------
\MPragma{mintmodversion;P0.1.0}
\MPragma{usercomment;This is file mintmod.tex version P0.1.0}


% ----------------------------------- Spezialelemente fuer Konfigurationsseite, werden nicht von mintscripts.js verwaltet --

% #1 = DOM-id der Box
\ifttm\newcommand{\MConfigbox}[1]{\special{html:<input cfieldtype="2" type="checkbox" name="Name_}#1\special{html:" id="}#1\special{html:" onchange="confHandlerChange('}#1\special{html:');"/>}}\fi % darf im PDF nicht aufgerufen werden!


\MPragma{MathSkip}
\Mtikzexternalize

\begin{document}

\MSection{Sprechweisen der Statistik}
\MLabel{VBKM_STOCH}
\MSetSectionID{VBKM_STOCHASTIK} % wird fuer tikz-Dateien verwendet

\begin{MSectionStart}
\MDeclareSiteUXID{VBKM_STOCHASTIK_START}

In diesem Modul werden die wichtigsten Grundlagen der deskriptiven (beschreibenden) Statistik behandelt.
Dabei werden den Themen Rundung und Prozentrechnung (diese sind eigentlich keine Themen der deskriptiven Statistik,
werden dort aber gebraucht) relativ viel Raum eingeräumt. Vor allem ist der sichere Umgang mit der Prozentrechnung in den Wirtschaftswissenschaften
unverzichtbar. Die Erfahrung lehrt, dass diese elementaren Dinge, die schon in der Mittelstufe des Gymnasiums unterrichtet werden, sehr oft auf die leichte
Schulter genommen werden. Ein Test hat z.B. ergeben, dass die Hälfte der Studienanfänger nicht in der Lage sind, aus dem Stegreif heraus von einer
Bruttorechnung die Mehrwertsteuer auszuweisen. Das Modul besteht aus

\begin{itemize}
\item{dem Abschnitt \MSRef{VBKM_STOCH_1_1}{Begriffe und Sprechweisen}, in dem die statistischen Grundbegriffe eingeführt und die verschiedenen Verfahren zur Rundung von Werten erklärt werden,}
\item{dem Abschnitt \MSRef{VBKM_STOCH_2_1}{Häufigkeitsverteilungen und Prozentrechnung}, in dem der Häufigkeitsbegriff eingeführt und die damit verbundene
Prozentrechnung sowie die Darstellung der gefundenen Werte mit Hilfe typischer Diagrammtypen erklärt wird,}
\item{dem Abschnitt \MSRef{VBKM_STOCH_3_1}{Statistische Maßzahlen}, in dem die wichtigsten Maßzahlen der beschreibenden Statistik wie Mittelwerte und Stichprobenvarianz erklärt werden,}
\item{dem \MSRef{VBKM_STOCH_Abschlusstest}{Abschlusstest}.}
\end{itemize}

\end{MSectionStart}


\MSubsection{Begriffe und Sprechweisen}

\begin{MIntro}
\MLabel{VBKM_STOCH_1_1}
\MDeclareSiteUXID{VBKM_STOCHASTIK_1_1}

Bei statistischen Untersuchungen (Erhebungen) werden an geeignet ausgewählten Untersuchungseinheiten (Beobachtungseinheiten, Versuchseinheiten) jeweils die Werte eines oder mehrerer Merkmale festgestellt. Dabei ist ein Merkmal eine zu untersuchende Größe der Beobachtungseinheit.
Begriffe und Sprechweisen:
\begin{itemize}
\item{Die \MEntry{Untersuchungseinheit}{Untersuchungseinheit} (auch: \MEntry{Beobachtungseinheit}{Beobachtungseinheit} oder \MEntry{Merkmalträger}{Merkmalträger})
ist die kleinste Einheit, an der Beobachtungen durchgeführt werden.}
\item{Das \MEntry{Merkmal}{Merkmal} ist die zu untersuchende Größe der Untersuchungseinheit. Merkmale werden meist mit großen lateinischen Buchstaben ($X, Y, Z, \ldots$)
bezeichnet.}
\item{\MEntry{Merkmalsausprägungen}{Merkmalsausprägungen} oder \MEntry{Merkmalswerte}{Merkmalswerte} sind die Werte, die von Merkmalen angenommen werden können.
Merkmalswerte werden meist mit kleinen, lateinischen Buchstaben ($a, b,\ldots, x, y, z, a_1, a_2,\ldots$) bezeichnet.}
\item{Die Menge der Untersuchungseinheiten, über die hinsichtlich eines interessierenden Merkmals eine Aussage gemacht werden soll,
heißt \MEntry{Grundgesamtheit}{Grundgesamtheit} oder auch \MEntry{Population}{Population}. Sie ist die Menge aller denkbaren Untersuchungseinheiten einer Untersuchung.}
\item{Eine \MEntry{Stichprobe}{Stichprobe} ist eine \glqq zufällig gewonnene\grqq\ endliche Teilmenge aus einer bestimmten,
interessierenden Grundgesamtheit. Hat diese Teilmenge $n$ Elemente, so spricht man von einer \glqq Stichprobe vom Umfang $n$\grqq.}
\item{\MEntry{Daten}{Daten} sind die beobachteten Werte (Ausprägungen) eines oder mehrerer Merkmale einer Stichprobe von Untersuchungseinheiten einer bestimmten Grundgesamtheit.}
\item{Die \MEntry{Urliste}{Urliste} ist das Protokoll, auf dem die Daten einer Untersuchung in der beobachteten Reihenfolge stehen.
Die Urliste ist also ein $n$-Tupel (bzw. Vektor, hier meist als Zeile statt Spalte geschrieben):
$$
x \;=\; (x_1,\ldots,x_n)\: .
$$
aus Daten. Oft nennt man auch dieses $n$-Tupel eine \glqq Stichprobe vom Umfang $n$\grqq.}
\end{itemize}

\begin{MExample}
Bei einer Tagesproduktion von Werkstücken werden $n=20$ Proben zu je $15$ Teilen entnommen und jeweils die Anzahl defekter Teile festgestellt.
Dabei sei $x_i$ die Anzahl der defekten Teile in der $i$-ten Probe, $i=1,\ldots,20$. Die Urliste (Stichprobe vom Umfang $n=20$) sei
$$
x\; =\; (0,4,2,1,1,0,0,2,3,1,0,5,3,1,1,2,0,0,1,0)\: .
$$
In der zweiten Probe wurden z.B. $x_2=4$ defekte Teile gefunden. Die Grundgesamtheit in diesem Beispiel wäre also die Menge aller
$15$-elementigen Teilmengen der Tagesproduktion. Das interessierende Merkmal ist
$$
X\;=\; \text{Anzahl der defekten Werkstücke in einer Probe mit 15 Elementen}\: .
$$
\end{MExample}

\begin{MInfo}
Die bei einer statistischen Untersuchung beobachtbaren Größen heißen \MEntry{Merkmale}{Merkmal}.
Werte, die von Merkmalen angenommen werden können, heißen \MEntry{Merkmalsausprägungen}{Merkmalsausprägung} oder \MEntry{Merkmalswerte}{Merkmalswert}.
\end{MInfo}

Bei Merkmalen wird grob zwischen qualitativen (artmäßig erfassbaren) und quantitativen (in natürlicher Weise zahlenmäßig erfassbaren) Merkmalen unterschieden:
\begin{itemize}
\item{\MEntry{Qualitative Merkmale}{Merkmal (qualitativ)}:\begin{itemize}
  \item{Nominale Merkmale: Die Klassifizierung der Merkmalsausprägungen erfolgt nach rein qualitativen Gesichtspunkten.
        Beispiele: Hautfarbe, Nationalität, Blutgruppe.}
  \item{Ordinale Merkmale: Eine natürliche Rangfolge der Merkmalsausprägungen ist vorhanden, man kann die Werte anordnen bzw. sortieren.
        Beispiele: Schulnote, Dienstgrade, Nachnamen.}
  \end{itemize}
  }
\item{\MEntry{Quantitative Merkmale}{Merkmal (quantitativ)}:\begin{itemize}
  \item{Diskrete Merkmale: Die Merkmalsausprägungen sind isolierte Zahlwerte.
        Beispiele: Anzahlen, Jahreszahlen, Alter in Jahren.}
  \item{Stetige Merkmale: Die Merkmalsausprägungen können (zumindestens prinzipiell) jeden Wert in einem Intervall annehmen.
        Beispiele: Größe, Gewicht, Länge.}
  \end{itemize}
}
\end{itemize}

Der Übergang zwischen stetigen und diskreten Merkmalen ist durch Rundungsmöglichkeiten zum Teil fließend.

\end{MIntro}


\begin{MXContent}{Rundung}{Rundung}{STD}
\MDeclareSiteUXID{VBKM_STOCHASTIK_1_2}
Das \MEntry{Runden}{Rundung} von Messergebnissen ist ein alltäglicher Vorgang.

\begin{MInfo}
Für Rundungsvorgänge gibt es prinzipiell drei Möglichkeiten:
\begin{itemize}
\item{Das (Ab)Runden mit Hilfe der $\MTextSF{floor}$-Funktion $\left\lfloor{x}\right\rfloor$,}
\item{das (Auf)Runden mit Hilfe der $\MTextSF{ceil}$-Funktion $\left\lceil{ x}\right\rceil$,}
\item{das Runden mit Hilfe der $\MTextSF{round}$-Funktion, manchmal auch als $\MTextSF{rnd}$-Funktion bezeichnet.}
\end{itemize}
\end{MInfo}

Die \MEntry{$\MTextSF{floor}$-Funktion}{floor-Funktion} (engl. floor = Fußboden, Diele) ist definiert durch
$$
\MTextSF{floor}:\; \R\longrightarrow\R\;\; , \;\;
x\;\longmapsto\; \MTextSF{floor}(x)\;=\; \left\lfloor{x}\right\rfloor\;=\; \max\lbrace k\in\Z\: :\: k\leq x\rbrace\: .
$$
Ist $x\in\R$ eine reelle Zahl, so ist $\MTextSF{floor}(x)=\left\lfloor{ x}\right\rfloor$ die größte ganze Zahl, die kleiner oder gleich x ist.
Sie entsteht durch Abrundung von $x$. Schreibt man eine positive reelle Zahl $x$ als Dezimalbruch, so ist $\left\lfloor{ x}\right\rfloor$ die ganze Zahl vor dem
Dezimalkomma: Die Abrundung schneidet die Nachkommastellen ab. Beispielsweise ist $\left\lfloor{ 3,142}\right\rfloor=3$, aber $\left\lfloor{ -2,124}\right\rfloor=-3$.
Die Floor-Funktion ist eine Treppenfunktion mit Sprungstellen in allen Punkten $x\in \Z$ der Sprunghöhe $1$.
Die Funktionswerte in den Sprungstellen liegen immer oben. Dies ist in dem folgenden Schaubild des Graphen angedeutet durch eingezeichnete Punkte:

\begin{center}
\MUGraphicsSolo{floor_400_0.png}{width=0.7\linewidth}{width:650px}\\
Graph der $\MTextSF{floor}$-Funktion.
\end{center}

Gegeben sei eine reelle Zahl $a\geq 0$, dargestellt als \MSRef{Mathematik_Grundlagen_UDB}{Dezimalbruch}
$$
a \;=\; g_n\, g_{n-1}\, \ldots\, g_1\, g_0\: .\: a_1\,a_2\, a_3\, \ldots
$$
Will man mit Hilfe der $\MTextSF{floor}$-Funktion die Zahl $a$ auf $r$ Stellen ($r\in\N_0$) nach dem Komma runden, so bildet man
$$
\tilde a \;=\; \frac1{10^r}\cdot \left\lfloor{ 10^r\cdot a}\right\rfloor\: .
$$
Dieser Rundungsvorgang ist die Rundung durch Abschneiden nach der r-ten Nachkommastelle. Beim Rundungsverfahren mit Hilfe der
$\MTextSF{floor}$-Funktion wird also grundsätzlich abgerundet.

\begin{MExample}
$a_1=2,3727$ wird mit Hilfe der $\MTextSF{floor}$-Funktion gerundet auf 2 Nachkommastellen zu
$$
\tilde a_1 \;=\; \frac1{10^2}\cdot \left\lfloor{10^2\cdot2,3727}\right\rfloor\; =\; \frac1{10^2}\cdot \left\lfloor{ 237,27}\right\rfloor \;=\; \frac1{10^2}\cdot 237\;=\; =2,37\: .
$$
oder einfach durch Abschneiden aller Nachkommastellen nach der zweiten Stelle (das geht aber nur, wenn die Zahl schon als Dezimalbruch vorliegt, was beispielsweise
in einem Computerprogramm oft nicht der Fall ist).
Die Zahl $a_2=\sqrt{2}=1,414213562\ldots$ wird mit Hilfe der $\MTextSF{floor}$-Funktion auf 4 Nachkommastellen zu
$$
\tilde a_2 \;=\; \frac1{10^4}\cdot \left\lfloor{ 10^4\cdot \sqrt2}\right\rfloor \;=\; \frac1{10^4}\cdot \left\lfloor{ 14142,1\ldots}\right\rfloor\;=\; \frac1{10^4}\cdot 14142 \;=\; 1,4142
$$
gerundet. Die Kreiszahl
$$
a_3 \;=\; \pi \;=\; 3,141592654\ldots
$$
wird mit Hilfe der $\MTextSF{floor}$-Funktion gerundet auf 2 Stellen nach dem Komma zu
$$
\tilde a_3 \;=\; \frac1{10^2}\cdot \left\lfloor{ 10^2\cdot\pi}\right\rfloor\;=\; \frac1{10^2}\cdot \left\lfloor{ 314,159\ldots}\right\rfloor \;=\; \frac1{10^2}\cdot 314 \;=\; 3,14\: .
$$
\end{MExample}

Das Rundungsverfahren mit Hilfe der $\MTextSF{floor}$-Funktion findet oft Anwendung bei der Bestimmung von Gesamtnoten in Zeugnissen (\glqq akademische Rundung\grqq).
Hat ein Studierender im Fach Mathematik z.B. die Einzelnoten

\begin{center}
\begin{tabular}{|c|c|}
\hline
Fach & Note \\ \hline
Mathematik 1 & $1,3$ \\ 
Mathematik 2 & $2,3$ \\ 
Mathematik 3 & $2,0$ \\ \hline
\end{tabular}
\end{center}

so wird das arithmetische Mittel dieser drei Noten
$$
\frac{1,3 + 2,3 + 2,0}{3}\; =\; \frac{5,6}{3} \; =\; 1,8\overline{6}\: .
$$
gebildet. Die Rundung mit Hilfe der $\MTextSF{floor}$-Funktion auf eine Nachkommastelle würde die Gesamtnote Mathematik $\tilde a=1,8$ ergeben.
Die zur Bildung von Gesamtnoten verwendeten Rundungsverfahren müssen in den jeweiligen Prüfungsordnungen stets genau beschrieben sein.
Das Gegenstück zur $\MTextSF{floor}$-Funktion ist die $\MTextSF{ceil}$-Funktion:

\begin{MInfo}
Die \MEntry{$\MTextSF{ceil}$-Funktion}{ceil-Funktion} (engl. ceil = Zimmerdecke, Decke) ist definiert durch
$$
\MTextSF{ceil}:\; \R\longrightarrow\R\;\; , \;\;
x\;\longmapsto\; \MTextSF{ceil}(x)\;=\; \left\lceil{ x}\right\rceil\;=\; \min\lbrace k\in\Z\: :\: k\geq x\rbrace\: .
$$
\end{MInfo}

Ist $x\in \mathbb{R}$ eine reelle Zahl, so ist
$\MTextSF{ceil}(x)=\left\lceil{ x }\right\rceil$ die kleinste ganze Zahl, die größer oder gleich $x$ ist.
Die $\MTextSF{ceil}$-Funktion ist eine Treppenfunktion mit Sprungstellen in allen Punkten $x\in\Z$ der Sprunghöhe $1$.
Die Funktionswerte in den Sprungstellen liegen immer unten. Dies ist in dem folgenden Schaubild des Graphen der $\MTextSF{ceil}$-Funktion angedeutet durch eingezeichnete Punkte:

\begin{center}
\MUGraphicsSolo{ceil_400_0.png}{width=0.7\linewidth}{width:650px}\\
Graph der $\MTextSF{ceil}$-Funktion.
\end{center}

Gegeben sei eine reelle Zahl $a\geq 0$, dargestellt als \MSRef{Mathematik_Grundlagen_UDB}{Dezimalbruch}
$$
a \;=\; g_n\, g_{n-1}\, \ldots\, g_1\, g_0\: .\: a_1\,a_2\, a_3\, \ldots
$$
Will man mit Hilfe der $\MTextSF{ceil}$-Funktion die Zahl $a$ auf $r$ Stellen ($r\in\N_0$) nach dem Komma runden, so bildet man
$$
\hat{a} \;=\; \frac{1}{10^{r}}\cdot \left\lceil{ 10^{r}\cdot a }\right\rceil \: .
$$
Bei diesem Rundungsvorgang wird stets aufgerundet auf die nächsthöhere Dezimalstelle.

\begin{MExample}
Die Zahl $a_1=2,3727$ wird mit Hilfe der $\MTextSF{ceil}$-Funktion gerundet auf $2$ Nachkommastellen zu
$$
\hat{a}_{1}\;=\; \frac{1}{10^{2}}\cdot \left\lceil{ 10^{2}\cdot 2,3727}\right\rceil\; =\; \frac{1}{10^{2}}\cdot \left\lceil{ 237,27}\right\rceil \;=\; \frac{1}{10^{2}}\cdot 238 \;=\; 2,38\: .
$$
Analog wird $a_{2}=\sqrt{2}=1,414213562\ldots$ mit Hilfe der $\MTextSF{ceil}$-Funktion auf $4$ Nachkommastellen zu
$$
\hat{a}_{2}\; =\; \frac{1}{10^{4}}\cdot \left\lceil{ 10^{4}\cdot \sqrt{2}}\right\rceil\; =\; \frac{1}{10^{4}}\cdot \left\lceil{ 14142,1\ldots}\right\rceil \;=\; \frac{1}{10^{4}}\cdot 14143\; =\; 1,4143
$$
gerundet. Die Kreiszahl $a_{3}=\pi=3,141592654\ldots$ wird mit Hilfe der $\MTextSF{ceil}$-Funktion gerundet auf $2$ Stellen nach dem Komma zu
$$
\hat{a}_{3}\; =\; \frac{1}{10^{2}}\cdot \left\lceil{ 10^{2}\cdot \pi }\right\rceil\; =\; \frac{1}{10^{2}}\cdot \left\lceil{ 314,15\ldots }\right\rceil\; =\;\frac{1}{10^{2}}\cdot 315 \;=\; 3,15\: .
$$
\end{MExample}

Rundungsverfahren mit Hilfe der $\MTextSF{ceil}$-Funktion findet man z.B. häufig bei Handwerkerrechnungen. Ein Handwerker wird meistens nach Arbeitsstunden bezahlt.
Dauert eine Reparatur 50 Minuten (das sind $\MZahl{0}{8}\overline{3}$ Stunden im Dezimalsystem), so wird trotzdem aufgerundet und eine volle Arbeitsstunde berechnet.
Spricht man umgangssprachlich von Runden, so ist meist die mathematische Rundung gemeint:

\begin{MInfo}
Die \MEntry{$\MTextSF{round}$-Funktion}{round-Funktion} (oder mathematische Rundung) ist definiert durch
$$
\MTextSF{round}:\; \R\longrightarrow\R\;\; , \;\;
x\;\longmapsto\; \MTextSF{round}(x)\;=\; \MTextSF{floor}\left({x + \frac{1}{2}}\right) \; =\; \left\lfloor{ x + \frac{1}{2} }\right\rfloor\: .
$$
Im Gegensatz zur Auf- und Abrundung beträgt die durch diese Rundung an der Zahl vorgenommene Veränderung höchstens $0,5$.
\end{MInfo}

Die $\MTextSF{round}$-Funktion ist eine Treppenfunktion mit Sprungstellen in allen Punkten $x+\frac{1}{2},\; x\in \Z$, der Sprunghöhe $1$.
Die Funktionswerte in den Sprungstellen liegen immer oben. Dies ist in dem folgenden Schaubild des Graphen der $\MTextSF{round}$-Funktion angedeutet durch eingezeichnete Punkte:

\begin{center}
\MUGraphicsSolo{round_400_0.png}{width=0.7\linewidth}{width:650px}\\
Graph der $\MTextSF{round}$-Funktion.
\end{center}

Gegeben sei eine reelle Zahl $a\geq 0$, dargestellt als \MSRef{Mathematik_Grundlagen_UDB}{Dezimalbruch}
$$
a \;=\; g_n\, g_{n-1}\, \ldots\, g_1\, g_0\: .\: a_1\,a_2\, a_3\, \ldots
$$
Will man mit Hilfe der Round-Funktion die Zahl $a$ auf $r$ ($r\in\N_0$) Stellen nach dem Komma runden, so bildet man
$$
\overline{a}\; =\; \frac{1}{10^{r}}\cdot \MTextSF{round}(10^{r}\cdot a)\; =\; \frac{1}{10^{r}}\cdot \left\lfloor{ 10^{r}\cdot a + \frac{1}{2} }\right\rfloor\: .
$$
Dieser Rundungsvorgang entspricht dem üblichen sogenannten mathematischen Runden.

\begin{MExample}
Die Zahl $a_{1}=1,49$ wird mit Hilfe der $\MTextSF{round}$-Funktion auf eine Nachkommestelle gerundet zu
\begin{eqnarray*}
\overline{a}_1 & = & \frac{1}{10}\cdot \mathsf{round}(10 \cdot 1,49) \; =\; \frac{1}{10}\cdot \left\lfloor{10\cdot 1,49 + 0,5}\right\rfloor\\
&=& \frac{1}{10}\cdot \left\lfloor{14,9+0,5}\right\rfloor \;=\; \frac{1}{10}\cdot \left\lfloor{ 15,4 }\right\rfloor \;=\; \frac{1}{10}\cdot 15 \;=\; 1,5\: .
\end{eqnarray*}
Die Zahl $a_{2}=1,52$ wird mit Hilfe der $\MTextSF{round}$-Funktion auf eine Nachkommestelle gerundet zu
\begin{eqnarray*}
\overline{a}_{2}& = & \frac{1}{10}\cdot \mathsf{round}(10 \cdot 1,52)\; =\; \frac{1}{10}\cdot \left\lfloor{ 10\cdot 1,52 + 0,5 }\right\rfloor \\
& = & \frac{1}{10}\cdot \left\lfloor{ 15,2+0,5}\right\rfloor \;=\; \frac{1}{10}\cdot \left\lfloor{ 15,7 }\right\rfloor\; =\; \frac{1}{10}\cdot 15 \;=\; 1,5\: .
\end{eqnarray*}
Die Zahl $a_{3}=2,3727$ wird mit Hilfe der $\MTextSF{round}$-Funktion gerundet auf $2$ Nachkommastellen zu
\begin{eqnarray*}
\overline{a}_{3} & = & \frac{1}{10^{2}}\cdot \MTextSF{round}(10^{2}\cdot 2,3727)\; =\; \frac{1}{100}\cdot \left\lfloor{ 100\cdot 2,3727 + 0,5 }\right\rfloor \\
& = & \frac{1}{100}\cdot \left\lfloor{ 237,27+0,5}\right\rfloor\; =\; \frac{1}{100}\cdot \left\lfloor{ 237,77 }\right\rfloor \; =\;  \frac{1}{100}\cdot 237\; =\; 2,37\: .
\end{eqnarray*}
Die Zahl  $a_{4}=\sqrt{2}=1,414213562\ldots$ wird mit Hilfe der $\MTextSF{round}$-Funktion gerundet auf $7$ Nachkommastellen zu
\begin{eqnarray*}
\overline{a}_{3} & = & \frac{1}{10^{7}}\cdot \mathsf{round}(10^{7}\cdot \sqrt{2})\; =\; \frac{1}{10^{7}}\cdot \left\lfloor{ 10^{7}\cdot 1,414213562\ldots + 0,5 }\right\rfloor \\
& = & \frac{1}{10^{7}}\cdot \left\lfloor{ 14142135,62\ldots+0,5}\right\rfloor \;=\; \frac{1}{10^{7}}\cdot \left\lfloor{ 14142136,12\ldots }\right\rfloor \\
& = & \frac{1}{10^{7}}\cdot 14142136\; =\;  1,4142136\: .
\end{eqnarray*}
\end{MExample}

\begin{MExercise}
Berechnen Sie mit Hilfe der $\MTextSF{round}$-Funktion die Rundung von $\pi=3,141592654\ldots$ auf $4$ Nachkommastellen:
\MEquationItem{$\overline{\pi}$}{\MLParsedQuestion{20}{3.1416}{7}{STOCHROUND1}}.
\ \\ \ \\
\begin{MHint}{\iSolution}
\begin{eqnarray*}
\overline{\pi} & = & \frac{1}{10^{4}}\cdot \mathsf{round}(10^{4}\cdot \pi)\; =\; \frac{1}{10^{4}}\cdot \left\lfloor{ 10^{4}\cdot 3,141592654\ldots + 0,5 }\right\rfloor \ \\
& = & \frac{1}{10^{4}}\cdot \left\lfloor{ 31415,92654\ldots+0,5}\right\rfloor \;=\; \frac{1}{10^{4}}\cdot \left\lfloor{ 31416,42654\ldots }\right\rfloor\ \\
& = & \frac{1}{10^{4}}\cdot 31416\; =\;  3,1416\: .
\end{eqnarray*}
\end{MHint}
\end{MExercise}

\begin{MExercise}
Vorgegeben seien die Zahlen
$$
a\; =\; \frac{47}{17} \;\;\text{und}\;\;  b\; =\; 3,7861\: .
$$
\begin{MExerciseItems}
\item{Runden Sie die Zahlen $a$ und $b$ mit Hilfe der $\MTextSF{floor}$-Funktion auf jeweils $2$ Nachkommastellen:\\Die Rundungen ergeben
\MEquationItem{$\tilde a$}{\MLParsedQuestion{10}{2,76}{7}{STOCHROUND2}} sowie \MEquationItem{$\tilde b$}{\MLParsedQuestion{10}{3,78}{7}{STOCHROUND3}}.}
\item{Runden Sie die Zahlen $a$ und $b$ mit Hilfe der $\MTextSF{ceil}$-Funktion auf jeweils $2$ Nachkommastellen:\\Die Rundungen ergeben
\MEquationItem{$\hat a$}{\MLParsedQuestion{10}{2,77}{7}{STOCHROUND4}} sowie \MEquationItem{$\hat b$}{\MLParsedQuestion{10}{3,79}{7}{STOCHROUND5}}.}
\item{Runden Sie die Zahlen $a$ und $b$ mit Hilfe der $\MTextSF{round}$-Funktion auf jeweils $2$ Nachkommastellen:\\Die Rundungen ergeben
\MEquationItem{$\overline{a}$}{\MLParsedQuestion{10}{2,76}{7}{STOCHROUND6}} sowie \MEquationItem{$\overline{b}$}{\MLParsedQuestion{10}{3,79}{7}{STOCHROUND7}}.}
\end{MExerciseItems}
\ \\
\begin{MHint}{\iSolution}
Wir formen den Bruch zunächst in eineb möglichst guteb Dezimalbruch um, indem wir sukzessive Division mit Rest durchführen und die Divisionsergebnisse als Stellen im Dezimalbruch einsetzen:
$$
\frac{47}{17} \;=\; 2+\frac{13}{17}
$$
also haben wir eine $2$ vor dem Dezimalkomma. Dann ist
$$
2+\frac{13}{17} \;=\; 2+\frac1{10}\cdot \frac{130}{17} \;=\; 2+\frac1{10}\cdot\left({7+\frac{11}{17}}\right)
$$
also erscheint eine $7$ als erste Nachkommastelle. Zerlegt man die Brüche weiter, so erhält man  $a\;=\;  2,764705\ldots$ wobei für die gefragte Rundung nur drei Nachkommastellen benötigt werden.
Die vollständige Rechnung mit der $\MTextSF{floor}$-Funktion ergibt
$$
\tilde a \;=\; \frac1{10^2}\cdot \left\lfloor{ 10^2\cdot 2,764705\ldots}\right\rfloor \;=\; \frac1{10^2}\cdot \left\lfloor{ 276,4705\ldots}\right\rfloor \;=\;  2,76\: .
$$
Auf die Ergebnisse kommt man aber auch schneller durch einfaches Abrunden bzw. Abschneiden nach der $2$ Nachkommastelle:
$$
\tilde a \;=\; 2,76\;\; \text{und}\;\; \tilde b \;=\; 3,78\: .
$$
Das ist in dieser Aufgabe aber nur erlaubt, weil $a$ und $b$ nicht negativ sind.
Durch einfaches Aufrunden nach der $2$ Nachkommastelle oder eine vollständige Rechnung erhalten wir für die $\MTextSF{ceil}$-Rundungen
$$
\hat a \;=\; 2,77\;\; \text{und}\;\; \hat b \;=\; 3,79\: .
$$
Die $\MTextSF{round}$-basierte Rundung erhält man entweder durch die vollständige Rechnung wie in den Beispielen
oder durch Rundung auf diejenige Zahl mit zwei Stellen hinter dem Dezimalkomma, die den geringsten Abstand zu den ursprünglichen Zahlen besitzt:
$$
\overline{a} \; = \; 2,76 \;\; \text{und}\;\; \overline{b} \,=\; 3,79\: .
$$
\end{MHint}
\end{MExercise}

\end{MXContent}

\begin{MXContent}{Bemerkungen zu den Rundungsvorgängen}{Bemerkungen}{STD}
\MDeclareSiteUXID{VBKM_STOCHASTIK_1_3}
Wie die folgenden Überlegungen und Beispiele zeigen, muss man beim Rechnen mit gerundeten Ergebnissen sehr kritisch sein.
Man betrachte die Menge $M=\R_{\geq0}$ aller nicht negativen rellen Zahlen und führe auf der Menge $M$ die Multiplikation
$$
M\times M \;\longrightarrow M\;\; , \;\;
(a,b)\;\longmapsto\; a\odot b
$$
über die Berechnungsvorschrift
$$
a\odot b \; =\; \frac{1}{10^{2}}\cdot \MTextSF{round}( 10^{2}\cdot a \cdot b)\; =\; \frac{1}{10^{2}}\cdot \left\lfloor{ 10^{2}\cdot a \cdot b + \frac{1}{2} }\right\rfloor
$$
ein. Anschaulich bedeutet dies, dass das Produkt $a\odot b$ berechnet wird, indem zunächst das gewöhnliche Produnkt $a\cdot b$ berechnet wird und
das Ergebnis dann auf zwei Nachkommastellen mathematisch gerundet wird.
\ \\ \ \\
Das Assoziativgesetz der Multiplikation gilt nicht mehr für die gerundete Multiplikation:
Für die Zahlen $a=2,11$, $b=3,35$ und $c=2,61$ gilt beispielsweise
$$
a\odot b \;=\; 2,11 \odot 3,35 \; =\; 7,07\;\text{und}\;
(a\odot b)\odot c \; =\; 7,07 \odot 2,61 \;=\;  18,45\: .
$$
Setzt man die Klammern um, so erhält man aber
$$
b\odot c\;  =\; 3,35 \odot 2,61 \; =\; 8,74\;\;\text{und}\;\;a\odot (b\odot c)\; =\;2,11 \odot 8,74 \;=\; 18,44\: .
$$

\begin{MInfo}
Da Taschenrechner (und Computer) stets mit gerundeten Ergebnissen rechnen, bedeutet dies, dass das Assoziativgesetz der Multiplikation auf Taschenrechnern nicht uneingeschränkt gültig ist.
\end{MInfo}

Ebenso können falsche Ergebnisse durch ungeschicktes Runden entstehen:
Dazu seien $a=4,98$ und $b=1,001$. Dann gilt
$$
a\cdot b\; =\; 4,98 \cdot 1,001\; =\; 4,98498\;\text{also}\;
a\odot b \; =\; 4,98 \odot 1,001\; =\; 4,98\; =\; a\: .
$$
Weiter gilt
$$
a\cdot b^{1000}\; =\; 4,98 \cdot 1,001^{1000} \;=\; a\cdot \underbrace{b\cdot \ldots \cdot b}_{1000 \text{ Faktoren}} \approx 13,53028118\: .
$$
Rundet man jetzt nach jeder durchgeführten Multiplikation auf $2$ Nachkommastellen, so erhält man wegen $a\odot b=a$ das völlig falsche Resultat
$$
(\ldots ((a\odot \underbrace{b) \odot b) \ldots \odot b}_{1000 \text{ Faktoren}}) \;=\;  a\; =\;4,98 \: .
$$

\begin{MInfo}
\MLabel{L_STOCH_RP}
Dies bedeutet für die Praxis, dass man grundsätzlich mit mindestens der doppelten Genauigkeit rechnen muss und erst das Endergebnis auf die vorgeschriebene Genauigkeit runden darf.
\end{MInfo}

\end{MXContent}

\MSubsection{Häufigkeitsverteilungen und Prozentrechnung}
\MLabel{VBKM_STOCH_2}

\begin{MIntro}
\MDeclareSiteUXID{VBKM_STOCHASTIK_2_1}
\MLabel{VBKM_STOCH_2_1}
Vorgegeben sei ein Merkmal $X$. Aufgrund einer Stichprobe vom Umfang $n$ ergab sich die Urliste (Stichprobe)
$$
x \;=\; (x_{1},x_{2},\ldots ,x_{n})\: .
$$

\begin{MInfo}
\MLabel{L_STOCH_H}
Ist $a$ ein möglicher Merkmalswert, so heißt
$$
H_{x}(a)\; =\; \text{Anzahl der}\;x_{j}\;\text{ in der Urliste}\;x\;\text{mit}\;x_{j}=a
$$
die \MEntry{absolute Häufigkeit}{Häufigkeit (absolut)} des Merkmalwertes $a$ in der Urliste $x=(x_{1},x_{2},\ldots ,x_{n})$.
\end{MInfo}

Sind $a_{1}, a_{2},\ldots a_{k}$ die möglichen Merkmalswerte in der Urliste $x=(x_{1},x_{2},\ldots ,x_{n})$, so gilt
$$
H_{x}(a_{1})+H_{x}(a_{2})+\ldots + H_{x}(a_{k})\; =\; n
$$
oder umgangssprachlich ausgedrückt: Jeder der $n$ Werte wird von genau einer der Häufigkeiten gezählt.

\begin{MInfo}
Die \MEntry{relative Häufigkeit}{Häufigkeit (relativ)} für den Merkmalswert $a$ in der Urliste $x=(x_{1},x_{2},\ldots ,x_{n})$ ist definiert durch
$$
h_{x}(a)\; =\;\frac{1}{n}\cdot H_{x}(a)\: .
$$
\end{MInfo}

Sind $a_{1}, a_{2},\ldots a_{k}$ die möglichen Merkmalswerte in der Urliste $x=(x_{1},x_{2},\ldots ,x_{n})$, so gilt
$$
h_{x}(a_{1})+h_{x}(a_{2})+\ldots + h_{x}(a_{k})\; =\;1\: .
$$
Relative Häufigkeiten liegen stets im Intervall $[0;1]$ und werden oft in Prozent geschrieben, also beispielsweise $h_x(a_1)=34\%$ statt $h_x(a_1)=0,34$.

\begin{MInfo}
Durch Zusammenfassung der absoluten bzw. relativen Häufigkeiten aller auftretenden (bzw. möglichen) Ergebnisse in der
Urliste (Stichprobe) $x=(x_{1},x_{2},\ldots ,x_{n})$ in einer Tabelle erhält man die \MEntry{empirische Häufigkeitsverteilung}{Häufigkeitsverteilung}.
\end{MInfo}

\begin{MExample}
In einem Rechenzentrum wurde die Bearbeitungsdauer (in sec., gerundet auf eine Nachkommastelle) von 20 Programmjobs bestimmt.
Es ergab sich die folgende Urliste zu einer Stichprobe vom Umfang $n=20$:
\begin{center}
\begin{tabular}{|l|l|l|l|l|}
\hline
$3,9$ & $3,3$ & $4,6$ & $4,0$ & $3,8$ \\ \hline
$3,8$ & $3,6$ & $4,6$ & $4,0$ & $3,9$ \\ \hline
$3,9$ & $3,9$ & $4,1$ & $3,7$ & $3,6$ \\ \hline
$4,6$ & $4,0$ & $4,0$ & $3,8$ & $4,1$ \\ \hline
\end{tabular}
\end{center}

Der kleinste Wert beträgt $3,3$ sec., der größte Wert beträgt $4,6$ sec., die Abstufung beträgt $0,1$sec., damit ergibt sich die folgende empirische Häufigkeitsverteilung (tabellarisch).
Damit die Tabelle nicht unnötig lang wird, sind alle Werte kleiner als $3,3$ sec. bzw. größer als $4,6$ sec. nicht aufgeführt:
\begin{center}
\begin{tabular}{|c|c|c|c|}
\hline
Ergebnis $a$ & $H_{x}(a)$ & $h_{x}(a)$ & in Prozent \\ \hline
$3,3$ & $1$ & $\frac{1}{20}=0,05$ & $5\%$ \\
$3,4$ & $0$ & $0$ & $0\%$ \\
$3,5$ & $0$ & $0$ & $0\%$ \\
$3,6$ & $2$ & $\frac{2}{20}=0.1$ & $10\%$\\
$3,7$ & $1$ & $\frac{1}{20}=0.05$ & $5\%$\\
$3,8$ & $3$ & $\frac{3}{20}=0.15$ & $15\%$\\
$3,9$ & $4$ & $\frac{4}{20}=0.2$ & $20\%$ \\
$4,0$ & $4$ & $\frac{4}{20}=0.2$ & $20\%$ \\
$4,1$ & $2$ & $\frac{2}{20}=0.1$ & $10\%$\\
$4,2$ & $0$ & $0$ & $0\%$ \\
$4,3$ & $0$ & $0$ & $0\%$ \\
$4,4$ & $0$ & $0$ & $0\%$ \\
$4,5$ & $0$ & $0$ & $0\%$\\
$4,6$ & $3$ & $\frac{3}{20}=0,15$ & $15\%$\\ \hline
Summe & $20$ & $1$ & $100\%$\\ \hline
\end{tabular}
\end{center}
\end{MExample}

\end{MIntro}

\begin{MXContent}{Prozentrechnung}{Prozentrechnung}{STD}
\MDeclareSiteUXID{VBKM_STOCHASTIK_2_2}
In der deskriptiven Statistik werden sehr oft Zahlenangaben in Prozent verwendet, daher werden in diesem Abschnitt
die wichtigsten Grundlagen aus der Prozentrechnung wiederholt. Zahlenangaben in Prozent (\glqq von Hundert, Hundertstel\grqq)
dienen der Veranschaulichung und dem Vergleichbarmachen von Größenverhältnissen, indem die Größen zu einem einheitlichen
Grundwert (Hundert) ins Verhältnis gesetzt werden.

\begin{MInfo}
Ist $a\geq 0$ eine reelle Zahl, so ist $a\,\% = \frac{a}{100}$, man kann das Symbol $\%$ also \glqq dividiere durch $100$\grqq\ interpretieren (ebenso wie im Modul \MNRef{VBKM05} das Gradsymbol
$\circ$ für Winkel als Multiplikation mit $\frac{\pi}{180}$ interpretiert wurde).
\end{MInfo}


Es gilt beispielsweise
\begin{itemize}
\item{Ein Prozent ist ein Hundertstel: $1\,\%=\frac{1}{100}=0,01$,}
\item{Zehn Prozent ist ein Zehntel: $10\,\%=\frac{10}{100}=0,1,$}
\item{25 Prozent sind ein Viertel: $25\,\%=\frac{25}{100}=0,25,$}
\item{Hundert Prozent sind ein Ganzes: $100\,\%=\frac{100}{100}=1,$}
\item{150 Prozent sind das 1,5-fache: $150\,\%=\frac{150}{100}=1,5.$}
\end{itemize}

Prozentangaben beschreiben im Allgemeinen Größenverhältnisse und beziehen sich dabei auf einen Grundwert.
Der Grundwert ist die Ausgangsgröße, auf die sich die Prozentangabe bezieht.
Der Prozentsatz wird in Prozent ausgedrückt und bezeichnet ein Größenverhältnis relativ zum Grundwert.
Die absolute Bestimmung dieser Größe nennt man Prozentwert. Der Prozentwert hat dieselbe Einheit wie der Grundwert.

\begin{MInfo}
Für Prozentwert, Grundwert und Prozentsatz gilt der Dreisatz
$$
\text{Prozentsatz}\;\cdot\;\text{Grundwert}\;\; =\;\;\text{Prozentwert}\: .
$$
\end{MInfo}

\end{MXContent}

\begin{MXContent}{Zinsrechnung}{Zinsrechnung}{STD}
\MDeclareSiteUXID{VBKM_STOCH_2_3}
In der Zinsrechnung unterscheidet man die einfache Verzinsung und die Zinseszinsrechnung.
Bei der einfachen Verzinsung werden die Zinsen am Ende der jeweiligen Zinsperioden ausgezahlt.
Dagegen werden bei der Verzinsung mit Zinseszins die Zinsen wiederum verzinst.

\begin{MInfo}
Eine Größe $K$, die jedes Jahr um $p\,\%$ anwächst, wird bei einfacher Verzinsung nach $t$ Jahren, $t\in \N$, auf
$$
K_{t}\; =\; K\cdot \left({ 1+ t\cdot \frac{p}{100}}\right)
$$
anwachsen.
\end{MInfo}

Dabei ist zu beachten, dass $p$ selbst eine Zahl mit Nachkommastellen sein kann, beispielsweise $p=2,5$. Dann ist $2,5\%=0,025$ der Prozentwert.

\begin{MExercise}
Welches Endkapital ergibt sich bei einfacher Verzinsung in Höhe von $p=2,5\%$ p. a. nach $t=10$ Jahren Laufzeit bei einem Anfangskapital von $K=4000$ EUR?
\ \\ \ \\
Antwort: \MEquationItem{$K_{10}$}{\MLParsedQuestion{10}{5000}{4}{STOCHPRO1} EUR}.
\ \\ \ \\
\begin{MHint}{\iSolution}
Einsetzen der entsprechenden Werte in die Zinsformel für die einfache Verzinsung ergibt
$$
K_{10}\; =\; K\cdot \left({ 1 + 10\cdot 0,025}\right) =4000\; \text{EUR} \cdot 1,25\; =\; 5000\;\text{EUR}\: .
$$
\end{MHint}
\end{MExercise}

\begin{MExercise}
Welcher Betrag $K$ hätte am 1.1.2000 einbezahlt werden müssen, um bei einfacher Verzinsung zu $p=5\%$ p. a. am 31.12.2011 ein Kapital von $K_{12}=10000$ EUR zu erhalten?
\ \\ \ \\
Antwort: \MEquationItem{$K$}{\MLParsedQuestion{10}{6250}{4}{STOCHPRO2} EUR}.
\ \\ \ \\
\begin{MHint}{\iSolution}
Einsetzen der entsprechenden Werte in die Zinsformel für die einfache Verzinsung ergibt
$$
K_{12}\; =\; 10000\;\text{EUR}\;  = \; K\cdot \left({1+12\cdot 0,05}\right)\; =\; K\cdot 1,6.
$$
Löst man diese Gleichung nach $K$ auf, so ergibt sich
$$
K\; =\; \frac{10000\; \text{EUR}}{1,6}\;  =\; \frac{100000\;\text{EUR}}{16}\;=\;  6250\;\text{EUR nach Kürzen des Bruchs}\: .
$$
\end{MHint}
\end{MExercise}

Während bei der einfachen Verzinsung die \MEntry{Zinsen}{Zinsen (einfach)} ausgezahlt werden, verzinst man diese bei der Zinseszinsrechnung in der folgenden Zinsperiode mit,
d.h. die Zinsen werden dem Kapital zugeschlagen bzw. \MEntry{kapitalisiert}{Zinsen (kapitalisiert)}:

\begin{MExample}
Auf einem Bankkonto werde ein Guthaben von $1000$ Euro zu $8\%$ Zinsen am Ende des Anlagejahres angelegt. Nach einem Jahr beträgt das Guthaben auf dem Konto (in EUR)
\begin{itemize}
\item{$1000+\frac{1000\cdot 8}{100}=1000\cdot \left(1+\frac{8}{100} \right)=1000\cdot 1,08=1080$.}
\item{Das Guthaben werde ein weiteres Jahr zum selben Zinssatz von $8\%$ angelegt. Dann beläuft sich das Gesamtguthaben nach zwei Jahren (in EUR) auf
$1080\cdot 1,08=1000\cdot 1,08^{2}=1000\cdot \left( 1+\frac{8}{100}\right)^{2}$.}
\item{Jedes Jahr wächst das Guthaben um den Faktor $1,08$. Folglich beträgt das Guthaben (in EUR) nach $t$ Jahren ($t\in\N_0$)
$$
1000\cdot 1,08^{t} \; =\; 1000\cdot \left( 1+\frac{8}{100}\right)^{t}.
$$}
\end{itemize}
\end{MExample}

Die Zinsenszinsrechnung basiert daher auf folgender Formel:

\begin{MInfo}
Eine Größe $K$, die jedes Jahr um $p\,\%$ anwächst, wird nach $t$ Jahren ($t\in \N_0$) auf
$$
K\cdot \left( 1+ \frac{p}{100}\right)^{t}
$$
anwachsen. Dabei wird $1+\frac{p}{100}$ der Wachstumsfaktor für ein Wachstum von $p\,\%$ genannt.
\end{MInfo}

In einer Werbung, die Sparkonten oder Kredite anbietet, wird der Zins gewöhnlich als jährliche Rate angegeben, auch dann,
wenn die aktuelle Zinsperiode verschieden davon ist. Diese Zinsperiode ist die Zeit, die zwischen zwei aufeinanderfolgenden Zeitpunkten liegt, zu denen Zinszahlungen fällig sind.
Für einige Sparkonten ist die Zinsperiode ein Jahr, aber es wird üblicher, andere Zinsperioden anzubieten.
So werden z.B. bei Spareinlagen die Zinsen täglich oder monatlich gutgeschrieben.

Falls eine Bank eine jährliche Zinsrate von $9\,\%$ mit monatlicher Zinsgutschrift anbietet, so werden $\left(\frac{1}{12} \right)\cdot 9\% =0,75\%$ des Kapitals am
Ende eines jeden Monats dem Konto gutgeschrieben. 

\begin{MInfo}
Die jährliche Rate muss dividiert werden durch die Anzahl der Zinsperioden,
um die periodische Rate, das ist der Zinssatz pro Periode, zu erhalten.
\end{MInfo}

Angenommen eine Kapitalanlage von $S_{0}$ EUR bringt $p\,\%$ Zinsen pro Periode ein. Nach $t$ Perioden ($t\in \N_0$) wird das Kapital auf den Betrag
$$
S_{t} \;=\; S_{0}\cdot (1+r)^{t} \;\;\text{mit}\;\; r\;=\;\frac{p}{100}
$$
angewachsen sein. In jeder Periode wächst das Kapital um den Faktor $1+r$, und man sagt: der \MEntry{Zinssatz}{Zinssatz} beträgt $p\,\%$ oder die \MEntry{Zinsrate}{Zinsrate} ist gleich $r$.
Es werde angenommen, dass die Zinsen zur Rate $\frac{p}{n}\,\%$ dem Kapital zu $n$ verschiedenen Zeitpunkten, die mehr oder weniger gleichmäßig über das Jahr verteilt sind,
gutgeschrieben werden. Dann wird das Kapital jedes Jahr mit einem Faktor
$$
\left(1+\frac{r}{n}\right)^{n}
$$
multipliziert. Nach $t$ Jahren ist das Kapital angewachsen auf
$$
S_{0}\cdot \left(1+\frac{r}{n} \right)^{n\cdot t}\: .
$$

\begin{MExample}
Ein Guthaben von $5000$ EUR wird für $t=8$ Jahren auf einem Konto angelegt zu einem jährlichen Zinssatz von $9\,\%$,
wobei die Zinsen vierteljährlich gutgeschrieben werden. Die periodische Rate $\frac{r}{n}$ ist in diesem Fall
$$
\frac{r}{n} \;=\;\frac{0.09}{4} \; =\; 0,0225
$$
und für die Anzahl der Perioden $n\cdot t$ ergibt sich
$n\cdot t = 4\cdot 8=32$. Damit wächst das Guthaben nach $t=8$ Jahren auf
$$
5000\cdot (1+0,0225)^{32}\; \approx \;10190,52 \text{ EUR}\: .
$$
\end{MExample}


\begin{MExercise}
Ein Kapital von $K_0=8750$ EUR wird $t=4$ Jahre zu $p=3,5\%$ p.a. und Zinskapitalisierung angelegt.
\begin{MExerciseItems}
\item{Die Höhe des Kapitals nach einem Jahr ist \MEquationItem{$K_1$}{\MLParsedQuestion{10}{9056.25}{7}{STOCHPRO7}}.}
\item{Die Höhe des Kapitals nach zwei Jahren ist \MEquationItem{$K_2$}{\MLParsedQuestion{10}{9373.22}{7}{STOCHPRO8}}.}
\item{Die Höhe des Kapitals nach drei Jahren ist \MEquationItem{$K_3$}{\MLParsedQuestion{10}{9701.28}{7}{STOCHPRO9}}.}
\item{Die Höhe des Endkapitals ist \MEquationItem{$K_4$}{\MLParsedQuestion{10}{10040.83}{7}{STOCHPRO10}}.}
\end{MExerciseItems}
Geben Sie alle Werte mathematisch gerundet auf zwei Nachkommastellen an, runden Sie erst \textit{nach} Ausführung der Rechenoperationen.
Sie können für diese Aufgabe einen Taschenrechner einsetzen.

\begin{MHint}{\iSolution}
Einsetzen in die Zinseszinsformel ergibt die Werte
\begin{eqnarray*}
K_1 &=& K_0\cdot \left({1+0,035}\right)^1 \;=\; 9056,25\\
K_2 &=& K_0\cdot \left({1+0,035}\right)^2 \;=\; 9373,22\;\text{(gerundet)}\\
K_3 &=& K_0\cdot \left({1+0,035}\right)^3 \;=\; 9701,28\;\text{(gerundet)}\\
K_4 &=& K_0\cdot \left({1+0,035}\right)^4 \;=\; 10040,83\;\text{(gerundet)}
\end{eqnarray*}
wobei es wegen \MSRef{L_STOCH_RP}{der Fehlerfortpflanzung bei Rundung} notwendig ist, erst die Potenz auszurechnen und dann zu runden. Es ist beispielsweise nicht richtig,
die gerundeten Jahreswert $K_3$ mit $1,035$ zu multiplizieren um $K_4$ zu erhalten.
\end{MHint}
\end{MExercise}

Ein Verbraucher, der einen Kredit aufnehmen möchte, steht möglicherweise vor mehreren Angeboten konkurrierender Geldinstitute.
Es ist daher von großer Bedeutung, die verschiedenen Angebote zu vergleichen.

\begin{MExample}
Betrachtet wird ein Angebot mit einem jährlichen Zinssatz von $9\%$, wobei Zinsen zur Rate $0,75\%$ monatlich, also $12$-mal im Jahr,
berechnet werden. Wenn zwischenzeitlich keine Zinsen abgezahlt werden, wird eine Anfangsschuld $S_{0}$ nach einem Jahr anwachsen auf eine Schuld
$$
S_{0}\cdot \left({1+\frac{0.09}{12} }\right)^{12}\; \approx \; S_{0}\cdot 1,094\: .
$$
Die zu zahlenden Zinsen sind ungefähr
$$
1,094 \cdot S_{0}-S_{0}\; =\; 0,094 \cdot S_{0}\: .
$$
\end{MExample}

Die Schuld wird, solange keine Zinsen zwischenzeitlich abgezahlt werden, mit einer konstanten proportionalen Rate wachsen, die ungefähr $9,4\%$ pro Jahr beträgt.
Aus diesem Grund spricht man vom effektiven jährlichen Zinssatz. Im Beispiel beträgt der effektive jährliche Zinssatz $9,4\%$.

\begin{MInfo}
Wenn die Zinsen $n$-mal im Jahr zum Zinssatz $\frac{r}{n}$ pro Periode gutgeschrieben werden, so ist der \MEntry{effektive jährliche Zinssatz}{Zinsen (effektiv)} $R$ definiert durch
$$
R\; =\;\left({ 1+\frac{r}{n}}\right)^{n}-1\: .
$$
\end{MInfo}

\end{MXContent}

\begin{MXContent}{Stetige Verzinsung}{Stetige Verzinsung}{STD}
\MDeclareSiteUXID{VBKM_STOCH_2_4}

Der Ausdruck $a_{n}=\left({1+\frac{r}{n}}\right)^{n}$ mit $r\in\R$ lässt sich in Abhängigkeit von $n\in\N$ auch auffassen als \MSRef{VBKM06_Abbildungen}{Abbildung}
$$
a:\; \N\:\longrightarrow\: \R\;\; , \;\;  n\:\longmapsto\: a(n)\; =\; a_{n}\; =\;\left({ 1+\frac{r}{n}}\right)^{n}\: .
$$
Eine Abbildung $\N \ni n \mapsto a_{n}\in \R$ nennt man eine reelle \MEntry{Zahlenfolge}{Zahlenfolge}.
Die Paare $(n,a_{n})$ für $n\in\N$ lassen sich interpretieren als Punkte in der euklidischen Ebene. Im folgenden Bild ist die Folge
$a_{n}=\left({1+\frac{0.4}{n}} \right)^{n}$ in diesem Sinn als Punktfolge in der euklidischen Ebene dargestellt:

\begin{center}
\MGraphicsSolo{exp1.png}{scale=1}\MLabel{G_VBKM_STOCH_EXP1}
\end{center}

An diesem Bild erkennt man zwei Eigenschaften dieser Folge:
\begin{itemize}
\item{Die Folge $a_{n}$, $n\in \N$, ist monoton wachsend, d.h. aus $i\leq j$ folgt $a_i\leq a_j$ für $i,j\in \N$.}
\item{Die Folge nähert sich für wachsende $n\in \N$ beliebig genau einem Wert $a\in \R$ an. Diese Zahl $a$ nennt man den Grenzwert der Folge $a_{n}$ und man schreibt
$$
\lim\limits_{n\rightarrow \infty} a_{n}\; =\;a\: .
$$}
\end{itemize}
In der Vorlesung Mathematik 1 wird die natürliche \MSRef{M06_e_fkt}{Exponentialfunktion}
$$
\exp:\;\R\:\longrightarrow\: \R\;\; , \;\; x\:\longmapsto\: \exp(x)\,=\; e^x
$$
vollständig eingeführt.

\begin{center}
\MGraphicsSolo{exp2.png}{scale=1}\MLabel{G_VBKM_STOCH_EXP2}\\
Die natürliche Exponentialfunktion.
\end{center}

Dort wird die folgende Aussage gezeigt:

\begin{MInfo}
Für beliebiges $x\in \R$ gilt
$$
\lim\limits_{n\rightarrow \infty}\left({1 + \frac{x}{n}} \right)^{n}\; =\; e^{x}\: .
$$
\end{MInfo}

Für $x=1$ ergibt dieser Grenzwert die Eulersche Zahl, benannt nach dem Schweizer Mathematiker Leonhard Euler (1707 - 1783):
$$
\lim\limits_{n\rightarrow \infty}\left({1 + \frac{1}{n}} \right)^{n}\; =\; e\;  \approx \; 2,7182\ldots \: .
$$
Man kann zeigen (schwierig), dass die Eulersche Zahl $e$ eine irrationale Zahl ist und sich daher nicht als Bruch schreiben lässt.

Die Exponentialfunktion erfüllt die \MSRef{VBKM01_Potenzgesetze}{Potenzgesetze} für beliebige reelle Zahlen als Exponenten:
\begin{itemize}
\item{$\exp(x+y)=e^{x+y}=e^{x}\cdot e^{y}=\exp (x)\cdot \exp (y)$ für $x,y\in\R$.}
\item{$\exp (x\cdot y)=e^{x\cdot y}=\left( e^{x} \right)^{y}= \left( e^{y} \right)^{x}$ für $x,y\in\R$.}
\end{itemize}

Mit Hilfe der Exponentialfunktion und dem Zusammenhang mit der Folge $(1+\frac{x}{n})^n$ kann man
Informationen über den Verzinsungsvorgang erhalten wenn die Anzahl der Zeitpunkte $n$ sehr groß wird: Das Kapital wird jährlich mit einem Faktor
$\left(1+\frac{r}{n} \right)^{n}$ multipliziert, wenn die Zinsen mit der Rate $\frac{r}{n}$ dem Anfangskapital $S_{0}$ an $n$ verschiedenen Zeitpunkten des Jahres gutgeschrieben werden.
Nach $t$ Jahren, $t\in\N$, ist das Kapital angewachsen auf
$$
S_{0}\cdot \left({1+ \frac{r}{n}}\right)^{n\cdot t}\: .
$$
Für $n\rightarrow \infty$ ergibt sich der Grenzwert
$$
\lim\limits_{n\rightarrow \infty} \left({S_{0}\cdot \left({1+\frac{r}{n}} \right)^{n\cdot t}}\right) \; =\; S_{0}\cdot e^{r\cdot t}\: .
$$
Bei wachsendem $n\in \N$ werden die Zinsen immer häufiger gutgeschrieben:

\begin{MInfo}
Den Grenzfall nennt man den Fall \MEntry{stetiger Verzinsung}{Zinsen (stetig)}.
Die Formel
$$
s(t)\; =\; S_{0}\cdot e^{r\cdot t}
$$
gibt für positive reelle Zahlen $t$ an, auf welchen Betrag ein Kapital $S_{0}$ nach $t$ Jahren bei stetiger Verzinsung angewachsen ist, wenn der jährliche Zinssatz $r$ ist.
\end{MInfo}

\begin{MExample}
Ein Guthaben von $5000$ EUR wird auf einem Konto bei einem jährlichen Zinssatz von $9\%$ und stetiger Verzinsung angelegt. Nach $t=8$ Jahren ergibt sich dann ein Guthaben von
$$
5000\cdot e^{0,09\cdot 8}\; =\; 5000\cdot e^{0,72}\; \approx\; 10272,17 \text{ EUR}\: .
$$
\end{MExample}

\end{MXContent}

\begin{MXContent}{Diagrammarten}{Diagrammarten}{STD}
\MDeclareSiteUXID{VBKM_STOCHASTIK_Diagrammarten}

Die graphische Darstellung von qualitativen bzw. quantitativ-diskreten Daten, welche durch eine Stichpobe gewonnen wurden,
erfolgt oft mittels \MEntry{Stabdiagrammen}{Stabdiagramm} (bzw. \MEntry{Balkendiagrammen}{Balkendiagramm}).

\begin{MInfo}
Das Stabdiagramm zeigt die absoluten bzw. relativen Häufigkeiten als Funktion der endlich vielen Merkmalswerte in der Stichprobe an.
Das Darstellungsmittel ist die Länge der Stäbe bzw. Balken.
\end{MInfo}

Dazu ein Beispiel: Bei $10$ Bäumen am Waldrand wurde die Baumart bestimmt. Die möglichen Merkmalsausprägungen des Merkmals
$X=$ Baumart sind: 
\begin{eqnarray*}
a_{1}& =& \text{Eiche} \;\; , \\
a_{2}& =& \text{Buche}\;\; , \\
a_{3}& =& \text{Fichte}\;\; , \\
a_{4}& =& \text{Kiefer, etc.}\: .
\end{eqnarray*}
Es ergab sich die folgende Urliste:

\begin{center}
\begin{tabular}{|c|c|c|c|c|c|c|c|c|c|c|}
\hline
$i$ & $1$ & $2$ & $3$ & $4$ & $5$ & $6$ & $7$ & $8$ & $9$ & $10$ \\ \hline
$x_i$ & $a_2$ & $a_1$ & $a_1$ & $a_3$ & $a_1$ & $a_2$ & $a_1$ & $a_1$ & $a_3$ & $a_3$ \\ \hline
\end{tabular}
\end{center}

Daraus ergibt sich die folgende empirische Häufigkeitstabelle:

\begin{center}
\begin{tabular}{|c|c|c|c|}
\hline
Ausprägung & absolut & relativ & in $\%$\\ \hline
Eiche & $5$ & $0,5$ & 50 \\ \hline
Buche & $2$ & $0,2$ & 20 \\ \hline
Fichte & $3$ & $0,3$ & 30 \\ \hline
\end{tabular}
\end{center}

Zu dieser empirischen Häufigkeitstabelle gehört das folgende Stabdiagramm:

\begin{center}
\MUGraphicsSolo{Stab_450_0.png}{width=0.4\linewidth}{width:450px}\\
Ein Stabdiagramm.
\end{center}

Zur Darstellung qualitativer Merkmale finden dagegen meist \MEntry{Kreisdiagramme}{Kreisdiagramm} Verwendung:

\begin{MInfo}
Den Merkmalsausprägungen werden entsprechend ihren relativen Häufigkeiten Kreissektoren zugeordnet, wobei gilt:
$$
h_{j} \; =\; \frac{H_{j}}{n}\;=\;  \frac{\alpha_{j}}{360^{\circ}}
$$
wobei $\alpha_j$ der \MSRef{VBKM05_Winkel}{Winkel} (in Grad) des Kreissektors zur Ausprägung $j$ ist.
Dabei bezeichnet $H_{j}$ die absolute und $h_{j}$ die relative \MSRef{L_STOCH_H}{Häufigkeit} der Ausprägung $j$ in der Urliste $x=(x_{1},x_{2},\ldots ,x_{n})$.
\end{MInfo}

Auch dazu ein Beispiel: Es wurden $n=1000$ Haushalte befragt, wie zufrieden sie sind mit einem neuartigen Gartengrill. Es gab die Anwortmöglichkeiten
sehr zufrieden (1), zufrieden (2), weniger zufrieden (3), unzufrieden (4).
\ \\ \ \\
Die Befragung ergab das folgende Ergebnis.

\begin{center}
\begin{tabular}{|c|c|c|c|}
\hline
Ausprägung & absolute Häufigkeiten & relative Häufigkeiten & in Prozent \\ \hline
sehr zufrieden &  $100$ & $0,1$ & $10\%$ \\ \hline
zufrieden & $240$ & $0,24$ & $24\%$ \\ \hline
weniger zufrieden &$480$ & $0,48$ & $48\%$\\ \hline
unzufrieden & $180$ & $0,18$ & $18\%$ \\ \hline
Summe & $1000$ &  $1$ & $100\%$\\ \hline
\end{tabular}
\end{center}

Die Umrechnung für die Winkel ergibt
\begin{itemize}
\item{$\alpha_{1}=360^{\circ}\cdot 0,1=36^{\circ}$,}
\item{$\alpha_{2}=360^{\circ}\cdot 0,24=86,4^{\circ}$,}
\item{$\alpha_{3}=360^{\circ}\cdot 0,48=172,8^{\circ}$,}
\item{$\alpha_{4}=360^{\circ}\cdot 0,18=64,8^{\circ}$.}
\end{itemize}

Damit ergibt sich das folgende Kreisdiagramm:

\begin{center}
\MUGraphicsSolo{Kreis_1.png}{width=0.3\linewidth}{width:500px}
\end{center}

Meist ist es nicht sinnvoll, alle möglichen Ausprägungen in einem Diagramm aufzuführen, sondern sie zu Klassen zu gruppieren und nur die Häufigkeiten der Klassen im Diagramm einzutragen.
Dies ist auch die einzige Möglichkeit, die Häufigkeiten stetiger Merkmale in einem Stab- oder Kreisdiagramm zu visualisieren.

Es sei $X$ ein quantitatives (stetiges) Merkmal und $x=(x_{1},x_{2},\ldots ,x_{n})$ die Urliste zu einer Stichprobe vom Umfang $n$.
Um eine empirische Häufigkeitsverteilung zu erhalten, ergibt sich das folgende Vorgehen:
\begin{itemize}
\item{Man bestimme den kleinsten und den größten Stichprobenwert, also
$$
x_{(1)} \; =\; \min\lbrace x_{1},x_{2},\ldots ,x_{n}\rbrace\;\; \text{und}\;\;
x_{(n)} \; =\; \max\lbrace x_{1},x_{2},\ldots ,x_{n}\rbrace\: .
$$}
\item{Man schreibe diese und alle anderen dazwischenliegenden Werte in der vorgeschriebenen Messgenauigkeit der Größe nach sortiert auf.
Hierdurch wird das Merkmal $X$ de facto ein diskretes Merkmal.}
\item{Man fertige eine Strichliste und eine empirische Häufigkeitsverteilung an.}
\end{itemize}

Die empirische Häufigkeitsverteilung eines stetigen Merkmals kann sehr umfangreich sein, vor allem können sehr viele Nullen auftreten
durch Messwerte, die in der Urliste (Stichprobe) nicht vorkommen. Dies macht die empirische Häufigkeitstabelle sehr
unübersichtlich und unhandlich. Daher führt man eine \MEntry{Klassenbildung}{Klassenbildung} zur Verringerung der Datenmengen durch (Datenreduktion).
Dies entspricht praktisch der Herabsetzung der Meßgenauigkeit (Rundung!).

\begin{MInfo}
\MEntry{Klassen}{Klassen} sind halboffene Intervalle der Form
$$
(a;b]\; =\; \lbrace x\in\R\:  : \: a<x\leq b\rbrace\;\; \text{mit}\;\; a,b\in \R\cup\lbrace\pm \infty\rbrace\: .
$$
\end{MInfo}

Es gibt keine allgemeinen Vorschriften für die Anzahl $k$ der Klassen und für die Klassengrößen, aber folgende
Richtlinien sind empfehlenswert:
\begin{itemize}
\item{Gleichmäßige Einteilung: Man berechnet $x_{(1)}=\min\lbrace x_{1},x_{2},\ldots ,x_{n}\rbrace$ und $x_{(n)}=\max\lbrace x_{1},x_{2},\ldots ,x_{n}\rbrace$.
Dann teilt man das Intervall $(x_{(1)}-\varepsilon;x_{(n)}+\varepsilon]$ mit $\varepsilon >0$ klein,
in etwa $k$ gleichgroße, sich nicht überlappende halboffene Teilintervalle ein.}
\item{Man vermeide zu kleine und zu große Klassen.}
\item{Man vermeide, wenn möglich, Klassen, welche sehr wenige Beobachtungen enthalten.}
\item{Man bilde etwa $k\approx \sqrt{n}$ etwa gleichgroße Klassen, dabei bezeichne $n$ den Stichprobenumfang.}
\end{itemize}

\begin{MInfo}
Das \MEntry{Histogramm}{Histogramm} dient zur graphischen Darstellung quantitativer Daten.
Es zeigt die relative Häufigkeit der Daten in der Klasse $(a,b]$ durch ein Rechteck mi Grundfläche $(a,b]$ an, dessen Flächeninhalt das Darstellungsmittel der Klasse ist.
\end{MInfo}

Es empfiehlt sich folgendes Vorgehen für die Erstellung eines Histogramms:
Es sei
$$
x\;=\; (x_{1},x_{2},\ldots x_{n})
$$
eine Urliste zu einer Stichprobe vom Umfang $n$ eines quantitativen Merkmals $X$.
\begin{itemize}
\item{Man verwendet eine Klasseneinteilung in $k$ Klassen. Es sei $(t_{j};t_{j+1}]$ das Intervall für die $j-$te Klasse, $j=1,2,\ldots ,k$.}
\item{Weiter sei $H_{j}$ die Anzahl der Datenwerte im Intervall $(t_{j};t_{j+1}]$ für $j=1,2,\ldots ,k$.
Die Zahlen $H_{j}$ nennt man auch die absoluten Klassenhäufigkeiten}
\item{Bilde für jedes $j\in \{1,2,\ldots ,k\}$ über der Grundseite $(t_{j};t_{j+1}]$ ein Rechteck der Höhe $d_{j}$ mit dem Flächeninhalt
$d_{j}\cdot (t_{j+1}-t_{j})=h_{j}=\frac{H_{j}}{n}$. Die Flächeninhalte $h_{j}$ sind die relativen Klassenhäufigkeiten.}
\end{itemize}

Die Gesamtfläche aller dieser Rechtecke ist dann gleich $1$.

Ein ausführliches Beispiel dazu:
In einem Rechenzentrum wurde die Bearbeitungsdauer (in sec., gerundet auf eine Nachkommastelle) von 20 Programmjobs bestimmt.
Es ergab sich die folgende Urliste zu einer Stichprobe vom Umfang $n=20$:

\begin{center}
\begin{tabular}{|c|c|c|c|c|}
\hline
3,9 & 3,3 & 4,6 & 4,0 & 3,8\\ \hline
3,8 & 3,6 & 4,6 & 4,0 & 3,9\\ \hline
3,9 & 3,9 & 4,1 & 3,7 & 3,6\\ \hline
4,6 & 4,0 & 4,0 & 3,8 & 4,1\\ \hline
\end{tabular}
\end{center}

Der kleinste Wert beträgt $3,3 \sec .$, der größte Wert $4,6 \sec .$, die Abstufung ist $0,1\sec .$.
Aufgrund der Empfehlung sind etwa $k\approx \sqrt{20}$ etwa gleichgroße Klassen zu wählen.
Es wird die folgende Klasseneinteilung mit $k=4$ Klassen gewählt.

\begin{center}
\begin{tabular}{|c|c|c|}
\hline
Klassen & $(t_{j};t_{j+1}],\;j=1,2,3,4$ & Daten \\ \hline
Klasse 1 & $(3,25;3,65]$ &  \glqq Von $3,3$ bis $3,6$\grqq \\ \hline
Klasse 2 & $(3,65;3,95]$ &  \glqq Von $3,7$ bis $3,9$\grqq \\ \hline
Klasse 3 & $(3,95;4,25]$ &  \glqq Von $4,0$ bis $4,2$\grqq \\ \hline
Klasse 4 & $(4,25;4,65]$ &  \glqq Von $4,3$ bis $4,6$\grqq \\ \hline
\end{tabular}
\end{center}

Die Tabelle der absoluten und relativen Häufigkeiten hat die folgende Gestalt.

\begin{center}
\begin{tabular}{|c|c|c|}
\hline
Klasse & abs. Klassenhäufigkeiten $H_{j}$ & rel. Klassenhäufigkeiten $h_{j}$ \\ \hline
Klasse 1 & $3$ & $0,15$ \\ \hline
Klasse 2 & $8$ & $0,4$ \\ \hline
Klasse 3 & $6$ & $0,3$ \\ \hline
Klasse 4 & $3$ & $0,15$ \\ \hline
\end{tabular}
\end{center}

Die Höhen der $k=4$ Rechtecke ergeben sich wie folgt:
\begin{itemize}
\item{1. Klasse: $d_{1}\cdot(t_{2}-t_{1})=d_{1}\cdot 0,4=h_{1}=0,15$, also $d_{1}=\frac{3}{8}=0,375$.}
\item{2. Klasse: $d_{2}\cdot(t_{3}-t_{2})=d_{2}\cdot 0,3=h_{2}=0,4$, also $d_{2}=\frac{4}{3}=1,\overline{3}$.}
\item{3. Klasse: $d_{3}\cdot(t_{4}-t_{3})=d_{3}\cdot 0,3=h_{3}=0,3$, also $d_{3}=1$.}
\item{4. Klasse: $d_{4}\cdot(t_{5}-t_{4})=d_{4}\cdot 0,4=h_{4}=0,15$, also $d_{4}=\frac{3}{8}=0,375$.}
\end{itemize}

Damit ergibt sich das folgende Histogramm:

\begin{center}
\MUGraphicsSolo{histogramm_500_0.png}{width=0.3\linewidth}{width:500px}
\end{center}

\end{MXContent}

\MSubsection{Statistische Maßzahlen}
\MLabel{VBKM_STOCH_3}

\begin{MIntro}
\MLabel{VBKM_STOCH_3_1}
\MDeclareSiteUXID{VBKM_STOCH_3_1}

Vorgegeben sei eine Stichprobe vom Umfang $n$ zu einem quantitativen Merkmal $X$. Die Urliste sei
$$
x\; =\; (x_{1},x_{2},\ldots ,x_{n})\: .
$$

\begin{MInfo}
\MLabel{L_Mittelwert}
Das \MEntry{arithmetische Mittel}{Mittelwert (arithmetisch)} $\overline{x}$, auch Stichprobenmittel genannt, von $x_{1},x_{2},\ldots ,x_{n}$ ist definiert durch
$$
\overline{x}\; =\; \frac{1}{n}\cdot \sum\limits_{k=1} ^{n} x_{k}\;=\;\frac{x_{1}+x_{2}+\ldots +x_{n}}{n}\: .
$$
\end{MInfo}

Physikalisch beschreibt $\overline{x}$ den Schwerpunkt der durch gleiche Massen in $x_{1},x_{2},\ldots ,x_{n}$ gegebenen Massenverteilung auf der als gewichtlos angenommenen Zahlengeraden.

\begin{MExample}
Vorgelegt sei die folgende Urliste zu einer Stichprobe vom Umfang $n=20$:

\begin{center}
\begin{tabular}{|c|c|c|c|c|}
\hline
10 & 11 & 9 & 7 & 9 \\ \hline
11 & 22 & 12 & 13 & 9 \\ \hline
11 & 9 & 10 & 12 & 13 \\ \hline
12 & 11 & 10 & 10 & 12\\ \hline
\end{tabular}
\end{center}

Das untersuchte Merkmal könnte z.B. die Studiendauer (in Semestern) von 20 Studierenden im Fach Mathematik am KIT sein. Aufsummieren der Werte ergibt
$$
\sum\limits_{k=1}^{20}x_{k}\; =\;223\: ,
$$
so dass sich für das arithmetische Mittel in diesem Beispiel

$$
\overline{x}\;=\; \frac{1}{20}\cdot \sum\limits_{k=1}^{20}x_{k}\;=\; \frac{223}{20}\; =\; 11,15
$$
ergibt.
\end{MExample}

Das arithmetische Mittel reagiert ziemlich stark auf sogenannte Ausreißerdaten. Dies bedeutet, dass ein stark von den übrigen Daten abweichender Messwert erhebliche
Auswirkungen auf den arithmetischen Mittelwert haben kann.

\begin{MExample}
Betrachtet man wieder die obige Urliste zu einer Stichprobe vom Umfang $n=20$ und lässt den Datenwert $x_{7}=22$ weg, so erhält man als arithmetisches Mittel der verbleibenden $19$ Datenwerte
$$
\frac{1}{19}\cdot \sum\limits_{k=1,k\neq 7}^{n}x_{k}\;=\; \frac{201}{19}\; \approx\; 10,58\: .
$$
\end{MExample}

Wird ein multiplikativer bzw. relativer Zusammenhang zwischen den Werten einer Urliste vermutet (beispielsweise bei Wachstumsprozessen oder Verzinsungen),
so ist das arithmetische (additive) Mittel keine geeignete Maßzahl. Für solche Datenwerte verwendet man das geometrische Mittel:

\begin{MInfo}
Für Daten $x_{1}>0,\;x_{2}>0,\;\ldots ,x_{n}>0$ ist das \MEntry{geometrische Mittel}{Mittelwert (geometrisch)} $\overline{x}_{G}$ von $x_{1},x_{2},\ldots ,x_{n}$ durch
$$
\overline{x}_{G}\; =\; \sqrt[n]{x_{1}\cdot x_{2}\cdot \ldots \cdot x_{n}}
$$
definiert.
\end{MInfo}

\begin{MExample}
Es wird eine Population beobachtet, die zum Zeitpunkt $t_{0}$ aus 50 Tieren besteht. Alle zwei Jahre wird die Zahl der Tiere neu beobachtet.

\begin{center}
\begin{tabular}{|l|l|l|l|l|}
\hline
Jahr & & Anzahl der Tiere & & Wachstumsrate \\ \hline
$t_{0}$ & & 50 \\ \hline
$t_{0}+2$ & & 100 & & verdoppelt ($x_{1}=2$)\\ \hline
$t_{0}+4$ & & 400 & & vervierfacht ($x_{2}=4$)\\ \hline
$t_{0}+6$ & & 1200 & & verdreifacht ($x_{3}=3$)\\ \hline
\end{tabular}
\end{center}

Die (geometrische) mittlere Wachtumsrate beträgt dann
$$
\overline{x}_G\; =\; \sqrt[3]{2\cdot 4\cdot 3}\; =\; \sqrt[3]{24}\; \approx\; 2,8845\: .
$$
\end{MExample}

An diesem Beispiel wird deutlich, dass die Anwendung des arithmetischen Mittels bei Wachstumsvorgängen zu falschen Ergebnissen führt. Es gilt
$$
\overline{x}\; =\;\frac{1}{3}\cdot (2+4+3)\; =\; \frac{9}{3}=\; 3\: ,
$$
aber eine theoretische Verdreifachung der Population alle zwei Jahre würde bedeuten, dass sie nach sechs Jahren $1350$ Tiere umfassen müsste, was ersichtlich falsch ist.
Bei einer durchschnittlichen Wachstumsrate von $2,8845$ erhält man das richtige Ergebnis: $50\cdot (2,8845)^{3}\approx 1200$.

\begin{MExercise}
Eine Kapitalanlage verzeichne die folgenden Wachstumsraten pro Jahr:
\begin{center}
\begin{tabular}{|c|c|c|c|c|c|}
\hline
Jahr           & 2011 & 2012 & 2013 & 2014 & 2015 \\ \hline
Wachstumsrate  & $0,5\%$ & $1,1\%$ & $0,8\%$ & $1,2\%$ &  $0,7\%$ \\ \hline
\end{tabular}
\end{center}
Bestimmen Sie die mittlere Wachstumsrate über die fünf Jahre in Prozent: \MEquationItem{$\overline{x}_G$}{\MLParsedQuestion{10}{0.82}{7}{STOCHGEOMM}} $\%$ mathematisch gerundet auf zwei Stellen hinter dem Komma.
\ \\ \ \\
Bei dieser Aufgabe dürfen Sie einen Taschenrechner für die Berechnungen verwenden.
\ \\ \ \\
\begin{MHint}{\iSolution}
Bildung des geometrischen Mittels ergibt
$$
\overline{x}_G\; =\; \sqrt[5]{0,5\cdot 1,1\cdot 0,8\cdot 1,2\cdot 0,7}\; =\; \sqrt[5]{0,3696}\; =\; 0,819495159191\ldots
$$
was auf den mathematisch gerundeten Wert $0,82$ führt.
\end{MHint}
\end{MExercise}

\end{MIntro}

\begin{MXContent}{Robuste Maßzahlen}{Robuste Maßzahlen}{STD}
\MLabel{VBKM_STOCH_3_2}
\MDeclareSiteUXID{VBKM_STOCH_3_2}
Die in diesem Abschnitt vorgestellten Maßzahlen sind robust gegenüber Ausreißern, d.h. starke Änderungen einzelner Datenwerte verändern diese Maßzahlen nicht oder nur wenig.

Vorgegeben sei eine Urliste
$$
x\; =\;(x_{1},x_{2},\ldots ,x_{n})
$$
zu einer Stichprobe vom Umfang $n$. Die Daten $x_{i}$ seien Merkmalswerte eines quantitativen Merkmals $X$.

\begin{MInfo}
Die durch aufsteigende Sortierung
$$
x_{(1)}\;\leq\; x_{(2)}\;\leq \;\ldots\;\leq\; x_{(n)}
$$
der Urliste gewonnene Liste $x_{(\; )}=(x_{(1)},x_{(2)},\ldots ,x_{(n)})$
heißt die geordnete Liste oder auch geordnete Stichprobe (zur Urliste $x$). Der $i$tte Eintrag $x_{(i)}$ in der geordneten Liste ist der $i$-te kleinste Wert in der Urliste.
\end{MInfo}

\begin{MExample}
Betrachtet man wieder die Urliste $x=(x_{1},x_{2},\ldots ,x_{20})$ zu der Stichprobe vom Umfang $n=20$
aus den vorangehenden Beispielen, so ergibt Sortieren die geordnete Stichprobe
$x_{(\; )}=(x_{(1)},x_{(2)},\ldots ,x_{(20)})$ zu
$$
\begin{array}{cccccccccc} 7 & 9 & 9 & 9 & 9 & 10 & 10 & 10 & 10 & 11 \\ 11 & 11 & 11 & 12 & 12 & 12 & 12 & 13 & 13 & 22 \end{array}
$$
\end{MExample}

\begin{MInfo}
\MLabel{L_Median}
Der (empirische) \MEntry{Median}{Median} $\tilde{x}$, auch Zentralwert genannt, von $x_{1},x_{2},\ldots ,x_{n}$ ist durch
$$
\tilde{x}\;=\;\left\lbrace{\begin{array}{lll}x_{\left(\frac{n+1}{2}\right)} & \text{falls} & n\;\text{ ungerade bzw.}\\
\frac{1}{2}\cdot \left( x_{(\frac{n}{2})}+ x_{(\frac{n}{2}+1)} \right) & \text{falls} &n\;\text{ gerade ist}\end{array}}\right.
$$
definiert.
\end{MInfo}

Im Gegensatz zum arithmetischen Mittel ist der (empirische) Median unempfindlich gegenüber Ausreißerdaten.
Es kann z.B. der größte Wert in der Urliste beliebig vergrößert werden, ohne dass sich der Median ändert.

\begin{MExample}
Im obigen Beispiel ist der Stichprobenumfang $n=20$ gerade, damit ergibt sich für den Median
$$
\tilde{x}\; =\; \frac{1}{2}\cdot \left({x_{(10)}+x_{(11)} }\right)\; =\;\frac{1}{2}\cdot (11 + 11)\; =\; 11\: .
$$
\end{MExample}

Etwa die Hälfte der Daten in der Urliste sind kleinergleich und etwa die Hälfte der Daten in der Urliste sind größergleich als der Median $\tilde{x}$.
Dieses Prinzip kann man verallgemeinern, um Quantile zu definieren.
Vorgegeben sei dazu eine Urliste $x=(x_{1},x_{2},\ldots ,x_{n})$
zu einer Stichprobe vom Umfang $n$ eines quantitativen Merkmals $X$.

\begin{MInfo}
Es sei
$$
x_{(\; )}\; =\; (x_{(1)},x_{(2)},\ldots ,x_{(n)})
$$
die zugehörige geordnete Stichprobe und
$$
\alpha \in (0,1) \;\;\text{und} \;\; k = \MTextSF{floor}(n\cdot \alpha) \;= \;\lfloor n\cdot \alpha \rfloor\: .
$$
Dann heißt
$$
\tilde{x}_{\alpha }\; =\; \left\lbrace{\begin{array}{lll}x_{(k+1)} & \text{falls} & n\cdot \alpha \notin \N \\ \frac{1}{2} \cdot \left(x_{(k)}+x_{(k+1)}\right) & \text{falls} & n\cdot \alpha \in \mathbb{N}\end{array}}\right.
$$
das Stichproben-$\alpha$-Quantil oder einfach das $\alpha$-Quantil von $x_{1},x_{2}\ldots ,x_{n}$.
\end{MInfo}

Das $0,25$-Quantil nennt man auch das untere \MEntry{Quartil}{Quartil}. Es trennt in etwa das untere Viertel der Datenwerte ab. Das $0,75$-Quantil nennt man entsprechend das obere Quartil.
Für $\alpha = 0,5$ ergibt sich der Median, also $\tilde{x}=\tilde{x}_{0,5}$.
Ist $\alpha \in (0,1)$, so wird die Datenreihe $x_{1},x_{2},\ldots ,x_{n}$ so aufgeteilt, dass etwa $\alpha\cdot 100\%$ der Daten kleinergleich $\tilde{x}_{\alpha}$ und etwa
$(1-\alpha)\cdot 100\%$ der Daten größergleich $\tilde{x}_{\alpha}$ sind.

\begin{MExample}
Vorgelegt sei wieder die Urliste $x=(x_{1},x_{2},\ldots ,x_{20})$ zu der Stichprobe vom Umfang $n=20$ aus den vorangehenden Beispielen mit der zugehörigen
geordneten Stichprobe $x_{(\; )}=(x_{(1)},x_{(2)},\ldots ,x_{(20)})$
$$
\begin{array}{cccccccccc} 7 & 9 & 9 & 9 & 9 & 10 & 10 & 10 & 10 & 11 \\ 11 & 11 & 11 & 12 & 12 & 12 & 12 & 13 & 13 & 22 \end{array}
$$
Für $\alpha = 0,25$ ist das $25\%$-Quantil bestimmt durch $n\cdot \alpha = \frac{20}{4}=5\in \N$, also
ergibt sich für das untere Quartil
$$
\tilde{x}_{0,25}\;=\;  \frac{1}{2}\cdot\left( x_{(5)}+x_{(6)} \right)\; =\;\frac{1}{2}\cdot (9+10)\; =\; \frac{19}{2}\; =\; 9,5\: .
$$
Für das obere Quartil setzen wir dagegen $\alpha = 0,75$ ein und erhalten
$n\cdot \alpha = \frac{20\cdot 3}{4}=15\in \N$, folglich
$$
\tilde{x}_{0,75}\;=\;\frac{1}{2}\cdot\left( x_{(15)}+x_{(16)} \right) \;=\;\frac{1}{2}\cdot (12+12)\;=\;12\: .
$$
\end{MExample}

Vorgegeben sei wieder eine Stichprobe vom Umfang $n$ zu einem quantitativen Merkmal $X$ mit 
zugehöriger geordneter Stichprobe
$$
x_{(\; )}\;=\;(x_{(1)},x_{(2)},\ldots ,x_{(n)})
$$
und
$$
\alpha \in [0,\;0.5)\;\; \text{und}\;\;k\;=\;\MTextSF{floor}(n\cdot \alpha) \;= \;\lfloor n\cdot \alpha \rfloor \: .
$$

\begin{MInfo}
Das $\alpha$-getrimmte (oder auch $\alpha$-gestutzte) Stichprobenmittel ist definiert durch
$$
\overline{x}_{\alpha}\;=\;
\frac{1}{n-2\cdot k} \cdot \sum\limits_{j=k+1}^{n-k}x_{(j)}\;=\; \frac{1}{n-2\cdot k}\cdot \left(x_{(k+1)}+ \ldots + x_{(n-k)} \right)\: .
$$
\end{MInfo}

Das $\alpha$-getrimmte Mittel ist ein arithmetischer Mittelwert, welcher die $\alpha \cdot 100\%$ größten und die $\alpha \cdot 100\%$ kleinsten Daten nicht in
die Rechnung mit einbezieht. Es stellt somit ein flexibles Instrument zum Schutz gegenüber Ausreißern an den Rändern des Datenbereichs dar.
Bei der Verwendung ist aber zu bedenken, dass nicht mehr alle ermittelten Daten in die Rechnung einfließen.

\begin{MExample}
In dem schon mehrfach betrachteten Datensatz ist die geordnete Stichprobe $x_{()}=(x_{(1)},x_{(2)},\ldots ,x_{(20)})$ gegeben durch
$$
\begin{array}{cccccccccc} 7 & 9 & 9 & 9 & 9 & 10 & 10 & 10 & 10 & 11 \\ 11 & 11 & 11 & 12 & 12 & 12 & 12 & 13 & 13 & 22 \end{array}
$$
und für
$\alpha = 0,12$ sowie $k=\lfloor 20\cdot 0,12 \rfloor = \lfloor 2,4 \rfloor =2$ erhalten wir das $12\%$-getrimmte Mittel der Stichprobe zu
$$
\overline{x}_{0.12} \;=\; \frac{1}{16}\cdot \sum\limits_{j=3}^{18}x_{(j)}\;=\; \frac{1}{16}\cdot 172\;=\;10,75\: .
$$
Es liegt niedriger als das arithmetische Mittel $\overline{x}=11,15$ da zum Beispiel der Ausreißer $x_{(20)}=22$ ignoriert wurde.
\end{MExample}

\end{MXContent}

\begin{MXContent}{Streuungsmaße}{Streuungsmaße}{STD}
\MLabel{VBKM_STOCH_3_3}
\MDeclareSiteUXID{VBKM_STOCH_3_3}
Mittelwerte und Quantile sind Lagemaße, d.h. sie sagen etwas über die absolute Lage der qualitativen Werte $x_j$ aus. Addiert man zu jedem $x_j$ eine Konstante $c$,
so erhöhen sich auch die Lagemaße um $c$. Streuungsmaße sind dagegen Maßzahlen, die etwas über die Streuung oder relative Verteilung der Datenwerte aussagen,
unabhängig von ihrer absoluten Lage. Vorgegeben sei eine Stichprobe vom Umfang $n\geq 2$ zu einem quantitativen Merkmal $X$.
Die Urliste sei $x=(x_{1},x_{2},\ldots ,x_{n})\in \mathbb{R}^{n}$.

\begin{MInfo}
\MLabel{L_Varianz}
Die \MEntry{Stichprobenvarianz}{Stichprobenvarianz} zur Urliste ist
$$
s_{x}^{2}\;=\; \frac{1}{n-1}\cdot \sum\limits_{k=1}^{n}(x_{k}-\overline{x})^{2}\; =\; \frac{(x_{1}-\overline{x})^{2}+ \ldots +(x_{n}-\overline{x})^{2}}{n-1}\: .
$$
Die \MEntry{Stichprobenstandardabweichung}{Stichprobenstandardabweichung} ist definiert durch $s_{x}=+\sqrt{s_{x}^{2}}$.
\end{MInfo}

Die Stichprobenvarianz ist ein Streuungsmaß, welches die Variabilität einer Beobachtungsreihe beschreibt. Je kleiner die Varianz, desto \glqq näher\grqq\ sind die Datenwerte beieinander.
die Varianz $s_x^2=0$ ist nur möglich, wenn alle Datenwerte gleich sind. Sie steigt mit zunehmendem $n$ typischerweise stark an, die Standardabweichung ist ein besserer Maßstab um
die \glqq Weite\grqq\ der Verteilung der Datenwerte einzuschätzen. Die beiden Formeln haben ein paar Tücken:
\begin{itemize}
\item{Um die Varianz ausrechnen zu können, muss der Mittelwert $\overline{x}$ schon bekannt sein.}
\item{Die Tatsache, dass in der Definition von $s_{x}^{2}$ durch $n-1$ und nicht durch das zunächst naheliegende $n$ dividiert wird, hat tieferliegende mathematische Gründe,
die erst in den Statistikvorlesungen behandelt werden können.}
\item{Die Schreibweise $s_{x}=+\sqrt{s_{x}^{2}}$ ist ein wenig irreführend, man darf das Quadrat nicht mit der Wurzel kürzen, denn tatsächlich muss man erst $s^2_x$ ausrechnen (und dieser Wert ist
nicht als Einzelquadrat definiert), um $s_x$ bestimmen zu können.}
\item{Vorsicht ist bei der Benutzung eines Taschenrechners mit Statistikfunktionen geboten: Die Stichprobenvarianz erhält man mit der Taste $s^2$, die Taste $\sigma^2$ liefert dagegen
die Summe mit Nenner $n$ statt $n-1$, das ist nicht die Stichprobenvarianz.}
\end{itemize}

\begin{MExample}
Die Datenreihe $x=(-1,0,1)$ besitzt den Mittelwert $\overline{x}=0$ und die Stichprobenvarianz
$$
s^2_x \;=\;\frac{1}{n-1}\cdot \sum\limits_{k=1}^{n}(x_{k}-\overline{x})^{2}\;=\; \frac1{3-1}\cdot \left({(-1-0)^2+(0-0)^2+(1-0)^2}\right)\;=\; 1\: .
$$
Hinzufügen von weiteren Nullen zur Datenreihe verändert das Lagemaß $\overline{x}$ nicht, sehr wohl aber das Streuungsmaß $s^2_x$, da dann mehr Datenwerte in der Mitte konzentriert sind.
Dagegen verändert das Verschieben aller Datenwerte um eine Konstante die Varianz nicht, beispielsweise besitzt auch die Datenreihe $(-5,-4,-3)$ die Varianz $1$.
\end{MExample}

\begin{MExercise}
Eine Datenreihe (mit einer unbekannten Anzahl $n$ von Werten habe die Maßzahlen $\overline{x}=4$, $s^2_x=10$ und den Median $\tilde x=3$.
Angenommen eine zweite Datenreihe erfüllt die Gleichung $y_k = (-2)\cdot x_k$ für jedes $k$, wie lauten dann ihre Maßzahlen?
\ \\ \ \\
Antwort: Die Maßzahlen sind \MEquationItem{$\overline{y}$}{\MLParsedQuestion{7}{-8}{3}{STOCHVAR1}}, \MEquationItem{$s_y^2$}{\MLParsedQuestion{7}{40}{3}{STOCHVAR2}}
und \MEquationItem{$\tilde y$}{\MLParsedQuestion{7}{-6}{3}{STOCHVAR3}}.
\ \\ \ \\
Hinweis: Untersuchen Sie in den Definitionen des \MSRef{L_Mittelwert}{Mittelwerts}, der \MSRef{L_Varianz}{Stichprobenvarianz} und des
\MSRef{L_Median}{Medians}, wie sich die Multiplikation aller $x$-Werte mit $(-2)$ auf den gesamten Term auswirkt.
\ \\ \ \\
\begin{MHint}{\iSolution}
Einsetzen der Veränderung in die $x$-Werte ergibt
\begin{eqnarray*}
\overline{y} &=& \frac1n\sum_{k=1}^n y_k \;=\;\frac1n\sum_{k=1}^n (-2)\cdot x_k \;=\; (-2)\cdot \frac1n\sum_{k=1}^n y_k \;=\; (-2)\cdot \overline{x} \;=\; -8\:,\\
s^2_y &=& \frac1{n-1}\sum_{k=1}^n \left({y_k-\overline{y}}\right)^2 \;=\;\frac1{n-1}\sum_{k=1}^n \left({(-2)x_k-(-2)\overline{x}}\right)^2\ \\
&=& \frac{(-2)^2}{n-1}\sum_{k=1}^n \left({x_k-\overline{x}}\right)^2 \;=\; (-2)^2\cdot s_x^2 \;=\;40\: ,\ \\
\tilde{y} &=& (-2)\tilde{x} \;=\; -6\: .
\end{eqnarray*}
Beim Umrechnen des Medians wird benutzt, dass die Multiplikation mit $(-2)$ die Sortierreihenfolge der geordneten Urliste umkehrt,
aber der Wert an der mittleren Position (bei ungerader Anzahl) bzw. die Werte an den beiden mittleren Positionen (bei gerader Anzahl) an diesen
Positionen bleiben und jeweils mit $(-2)$ multipliziert werden.
\end{MHint}
\end{MExercise}
\end{MXContent}


\MSubsection{Abschlusstest}

\begin{MTest}{Abschlusstest Kapitel \arabic{section}}
\MLabel{VBKM_STOCH_Abschlusstest}
\MDeclareSiteUXID{VBKM_STOCHASTIK_Abschlusstest}


\begin{MExercise}
Bei Anleihen (z.B. Bundesobligationen) unterscheidet man den Nennwert und den Ausgabekurs (Kurswert).
Anleihen können zum Nennwert, unter Nennwert oder über Nennwert herausgegeben werden.
Der Ausgabekurs liegt um so näher am Nennwert, je mehr der Anleihezins dem aktuellen Marktzins entspricht.
Ein Kunde/eine Kundin erwirbt Anleihen im Nennwert von $K=10000$ EUR mit einem Ausgabewert von $100\%$, einem Zins von $p=4,5\%$ p. a. und einer Laufzeit von $t=10$ Jahren.
\begin{MExerciseItems}
\item{Welche Zinszahlung erhält er/sie am Ende jeder Zinsperiode bei einem Ausgabewert von $100\%$ und einfacher Verzinsung? Antwort: Die jährliche Zinsauszahlung beträgt \MLParsedQuestion{10}{450}{5}{STOCHPRO3} EUR.}
\item{Bestimmen Sie die Höhe des insgesamt ausgezahlten Kapitals am Ende der Laufzeit bei einfacher Verzinsung. Antwort: Die Höhe des insgesamt ausgezahlten Kapitals am Ende der Laufzeit beträgt \MLParsedQuestion{10}{14500}{5}{STOCHPRO4} EUR.}
\end{MExerciseItems}
\end{MExercise}

\begin{MExercise}
Ein Betrag von $K=25000$ EUR soll bei jährlicher Verzinsung solange angelegt werden, bis der Wert der Anlage sich verdoppelt hat. Der Zinssatz betrage $p=3,5\%$ p. a.
Wie viele Jahre muss das Geld bei kapitalisierten Zinsen angelegt werden?
\ \\ \ \\
Antwort: Der notwendige Anlagezeitraum beträgt \MEquationItem{$t$}{\MLParsedQuestion{10}{21}{5}{STOCHPRO11}} Jahre.
\ \\ \ \\
Runden Sie das Ergebnis nach oben auf eine natürliche Zahl.
\end{MExercise}

% \begin{MExercise}
% Der Ausgabekurs einer Anleihe betrage $97,5\%$. Es sollen $K=10000$ EUR investiert werden. Der Nominalzins der Anleihe betrage $p=3,5\%$ p.a. .
% \begin{MExerciseItems}
% \item{Bestimmen Sie den Kurswert der Anleihe zu Beginn der Laufzeit. Antwort: \MLParsedQuestion{10}{9750}{4}{STOCHPRO5} EUR.}
% \item{Bestimmen Sie die Höhe des insgesamt ausgezahlten Kapitals bei einer Laufzeit von $t=10$ Jahren. Antwort: \MLParsedQuestion{10}{13500}{4}{STOCHPRO6} EUR.}
% \item{Bestimmen Sie für diese Anleihe den Effektivzins. Antwort: \MLParsedQuestion{10}{3.85}{6} Prozent.}
% \end{MExerciseItems}
% Hinweis: Geben Sie den Effektivzinssatz mathematisch gerundet auf 2 Nachkommastellen an. Runden Sie erst \textit{nachdem} Sie den Effektivzins als Produkt ausgerechnet haben.
% \end{MExercise}

\begin{MExercise}
Lesen Sie aus dem folgende Histogramm ab, welche Eigenschaften die beschriebene Stichprobe aufweist:
\begin{center}
\MUGraphicsSolo{Histo2.png}{width=0.3\linewidth}{width:466px}
% CC BY-SA 3.0, Quelle: Histo.png aus der Wikipedia (Seite "Histogramm") von Benutzer https://de.wikipedia.org/wiki/Benutzer:Philipendula
\ \\
Histogramm der Stichprobe $x=(x_1,\ldots,x_n)$.
\end{center}

Bestimmen Sie aus diesen Daten die Intervallgrenzen für die fünf eingeteilten Klassen sowie die relativen Häufigkeiten. Füllen Sie die Häufigkeitstabelle aus. Bestimmen Sie dazu die Flächeninhalte
der Balken im Diagramm im Verhältnis zum Gesamtflächeninhalt:
\begin{center}
\begin{tabular}{|c|c|c|}
\hline
Klasse & Werteintervall & rel. Klassenhäufigkeiten $h_{j}$ \\ \hline
Klasse 1 & $[0;200)$ & $0,16$ \\ \hline
Klasse 2 & \MLIntervalQuestion{12}{[200,300)}{5}{STOCHIT1} & \MLParsedQuestion{8}{0.19}{5}{STOCHIT5} \\ \hline
Klasse 3 & \MLIntervalQuestion{12}{[300,400)}{5}{STOCHIT2} & \MLParsedQuestion{8}{0.19}{5}{STOCHIT6} \\ \hline
Klasse 4 & \MLIntervalQuestion{12}{[400,500)}{5}{STOCHIT3} & \MLParsedQuestion{8}{0.28}{5}{STOCHIT7} \\ \hline
Klasse 5 & \MLIntervalQuestion{12}{[500,700)}{5}{STOCHIT4} & \MLParsedQuestion{8}{0.19}{5}{STOCHIT8} \\ \hline
\hline
\end{tabular}
\end{center}

\end{MExercise}


\begin{MExercise}
Die Messung des Gewichtes von $n=11$ Wassermelonen (in Kilogramm) ergab die folgenden Ergebnisse:

\begin{center}
\begin{tabular}{|c|c|c|c|c|c|c|c|c|c|c|c|}
\hline
Nummer der Melone $j$ & $1$ & $2$ & $3$ & $4$ & $5$ & $6$ & $7$ & $8$ & $9$ & $10$ & $11$ \\ \hline
Gewicht $x_j$ in Kilogramm & $6,2$ &  $5,5$ & $7,3$ & $6,8$ & $6,3$ & $5,5$ & $4,5$ & $6,5$ & $7,3$ & $5,7$ & $5,6$\\ \hline
\end{tabular}
\end{center}

\begin{MExerciseItems}
\item{Bestimmen Sie das arithmetische Mittel der $11$ Stichprobenwerte: \MEquationItem{$\overline{x}$}{\MLParsedQuestion{10}{(6.2+5.5+7.3+6.8+6.3+5.5+4.5+6.5+7.3+5.7+5.6)/11}{6}{STOCHVAR4}}.}
\item{Bestimmen Sie den Median der $11$ Stichprobenwerte: \MEquationItem{$\tilde{x}$}{\MLParsedQuestion{10}{6.2}{6}{STOCHVAR5}}.}
%\item{Bestimmen Sie die Stichprobenvarianz der $11$ Stichprobenwerte: \MEquationItem{$s^2_x$}{\MLParsedQuestion{10}{???}{6}{STOCHVAR6}}.}
\end{MExerciseItems}
Tragen Sie das Ergebnis gerundet auf zwei Nachkommastellen ein. Benutzen Sie keinen Taschenrechner, sondern versuchen Sie, die Zahlenwerte per Hand zu ermitteln.
\end{MExercise}


\end{MTest}

\newpage
\MPrintIndex

\end{document}
