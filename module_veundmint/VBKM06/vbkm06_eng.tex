% MINTMOD Version P0.1.0, needs to be consistent with preprocesser object in tex2x and MPragma-Version at the end of this file

% Parameter aus Konvertierungsprozess (PDF und HTML-Erzeugung wenn vom Konverter aus gestartet) werden hier eingefuegt, Preambleincludes werden am Schluss angehaengt

\newif\ifttm                % gesetzt falls Uebersetzung in HTML stattfindet, sonst uebersetzung in PDF

% Wahl der Notationsvariante ist im PDF immer std, in der HTML-Uebersetzung wird vom Konverter die Auswahl modifiziert
\newif\ifvariantstd
\newif\ifvariantunotation
\variantstdtrue % Diese Zeile wird vom Konverter erkannt und ggf. modifiziert, daher nicht veraendern!


\def\MOutputDVI{1}
\def\MOutputPDF{2}
\def\MOutputHTML{3}
\newcounter{MOutput}

\ifttm
\usepackage{german}
\usepackage{array}
\usepackage{amsmath}
\usepackage{amssymb}
\usepackage{amsthm}
\else
\documentclass[ngerman,oneside]{scrbook}
\usepackage{etex}
\usepackage[latin1]{inputenc}
\usepackage{textcomp}
\usepackage[ngerman]{babel}
\usepackage[pdftex]{color}
\usepackage{xcolor}
\usepackage{graphicx}
\usepackage[all]{xy}
\usepackage{fancyhdr}
\usepackage{verbatim}
\usepackage{array}
\usepackage{float}
\usepackage{makeidx}
\usepackage{amsmath}
\usepackage{amstext}
\usepackage{amssymb}
\usepackage{amsthm}
\usepackage[ngerman]{varioref}
\usepackage{framed}
\usepackage{supertabular}
\usepackage{longtable}
\usepackage{maxpage}
\usepackage{tikz}
\usepackage{tikzscale}
\usepackage{tikz-3dplot}
\usepackage{bibgerm}
\usepackage{chemarrow}
\usepackage{polynom}
%\usepackage{draftwatermark}
\usepackage{pdflscape}
\usetikzlibrary{calc}
\usetikzlibrary{through}
\usetikzlibrary{shapes.geometric}
\usetikzlibrary{arrows}
\usetikzlibrary{intersections}
\usetikzlibrary{decorations.pathmorphing}
\usetikzlibrary{external}
\usetikzlibrary{patterns}
\usetikzlibrary{fadings}
\usepackage[colorlinks=true,linkcolor=blue]{hyperref} 
\usepackage[all]{hypcap}
%\usepackage[colorlinks=true,linkcolor=blue,bookmarksopen=true]{hyperref} 
\usepackage{ifpdf}

\usepackage{movie15}

\setcounter{tocdepth}{2} % In Inhaltsverzeichnis bis subsection
\setcounter{secnumdepth}{3} % Nummeriert bis subsubsection

\setlength{\LTpost}{0pt} % Fuer longtable
\setlength{\parindent}{0pt}
\setlength{\parskip}{8pt}
%\setlength{\parskip}{9pt plus 2pt minus 1pt}
\setlength{\abovecaptionskip}{-0.25ex}
\setlength{\belowcaptionskip}{-0.25ex}
\fi

\ifttm
\newcommand{\MDebugMessage}[1]{\special{html:<!-- debugprint;;}#1\special{html:; //-->}}
\else
%\newcommand{\MDebugMessage}[1]{\immediate\write\mintlog{#1}}
\newcommand{\MDebugMessage}[1]{}
\fi

\def\MPageHeaderDef{%
\pagestyle{fancy}%
\fancyhead[r]{(C) VE\&MINT-Projekt}
\fancyfoot[c]{\thepage\\--- CCL BY-SA 3.0 ---}
}


\ifttm%
\def\MRelax{}%
\else%
\def\MRelax{\relax}%
\fi%

%--------------------------- Uebernahme von speziellen XML-Versionen einiger LaTeX-Kommandos aus xmlbefehle.tex vom alten Kasseler Konverter ---------------

\newcommand{\MSep}{\left\|{\phantom{\frac1g}}\right.}

\newcommand{\ML}{L}

\newcommand{\MGGT}{\mathrm{ggT}}


\ifttm
% Verhindert dass die subsection-nummer doppelt in der toccaption auftaucht (sollte ggf. in toccaption gefixt werden so dass diese Ueberschreibung nicht notwendig ist)
\renewcommand{\thesubsection}{}
% Kommandos die ttm nicht kennt
\newcommand{\binomial}[2]{{#1 \choose #2}} %  Binomialkoeffizienten
\newcommand{\eur}{\begin{html}&euro;\end{html}}
\newcommand{\square}{\begin{html}&square;\end{html}}
\newcommand{\glqq}{"'}  \newcommand{\grqq}{"'}
\newcommand{\nRightarrow}{\special{html: &nrArr; }}
\newcommand{\nmid}{\special{html: &nmid; }}
\newcommand{\nparallel}{\begin{html}&nparallel;\end{html}}
\newcommand{\mapstoo}{\begin{html}<mo>&map;</mo>\end{html}}

% Schnitt und Vereinigungssymbole von Mengen haben zu kleine Abstaende; korrigiert:
\newcommand{\ccup}{\,\!\cup\,\!}
\newcommand{\ccap}{\,\!\cap\,\!}


% Umsetzung von mathbb im HTML
\renewcommand{\mathbb}[1]{\begin{html}<mo>&#1opf;</mo>\end{html}}
\fi

%---------------------- Strukturierung ----------------------------------------------------------------------------------------------------------------------

%---------------------- Kapselung des sectioning findet auf drei Ebenen statt:
% 1. Die LateX-Befehl
% 2. Die D-Versionen der Befehle, die nur die Grade der Abschnitte umhaengen falls notwendig
% 3. Die M-Versionen der Befehle, die zusaetzliche Formatierungen vornehmen, Skripten starten und das HTML codieren
% Im Modultext duerfen nur die M-Befehle verwendet werden!

\ifttm

  \def\Dsubsubsubsection#1{\subsubsubsection{#1}}
  \def\Dsubsubsection#1{\subsubsection{#1}\addtocounter{subsubsection}{1}} % ttm-Fehler korrigieren
  \def\Dsubsection#1{\subsection{#1}}
  \def\Dsection#1{\section{#1}} % Im HTML wird nur der Sektionstitel gegeben
  \def\Dchapter#1{\chapter{#1}}
  \def\Dsubsubsubsectionx#1{\subsubsubsection*{#1}}
  \def\Dsubsubsectionx#1{\subsubsection*{#1}}
  \def\Dsubsectionx#1{\subsection*{#1}}
  \def\Dsectionx#1{\section*{#1}}
  \def\Dchapterx#1{\chapter*{#1}}

\else

  \def\Dsubsubsubsection#1{\subsubsection{#1}}
  \def\Dsubsubsection#1{\subsection{#1}}
  \def\Dsubsection#1{\section{#1}}
  \def\Dsection#1{\chapter{#1}}
  \def\Dchapter#1{\title{#1}}
  \def\Dsubsubsubsectionx#1{\subsubsection*{#1}}
  \def\Dsubsubsectionx#1{\subsection*{#1}}
  \def\Dsubsectionx#1{\section*{#1}}
  \def\Dsectionx#1{\chapter*{#1}}

\fi

\newcommand{\MStdPoints}{4}
\newcommand{\MSetPoints}[1]{\renewcommand{\MStdPoints}{#1}}

% Befehl zum Abbruch der Erstellung (nur PDF)
\newcommand{\MAbort}[1]{\err{#1}}

% Prefix vor Dateieinbindungen, wird in der Baumdatei mit \renewcommand modifiziert
% und auf das Verzeichnisprefix gesetzt, in dem das gerade bearbeitete tex-Dokument liegt.
% Im HTML wird es auf das Verzeichnis der HTML-Datei gesetzt.
% Das Prefix muss mit / enden !
\newcommand{\MDPrefix}{.}

% MRegisterFile notiert eine Datei zur Einbindung in den HTML-Baum. Grafiken mit MGraphics werden automatisch eingebunden.
% Mit MLastFile erhaelt man eine Markierung fuer die zuletzt registrierte Datei.
% Diese Markierung wird im postprocessing durch den physikalischen Dateinamen ersetzt, aber nur den Namen (d.h. \MMaterial gehoert noch davor, vgl Definition von MGraphics)
% Parameter: Pfad/Name der Datei bzw. des Ordners, relativ zur Position des Modul-Tex-Dokuments.
\ifttm
\newcommand{\MRegisterFile}[1]{\addtocounter{MFileNumber}{1}\special{html:<!-- registerfile;;}#1\special{html:;;}\MDPrefix\special{html:;;}\arabic{MFileNumber}\special{html:; //-->}}
\else
\newcommand{\MRegisterFile}[1]{\addtocounter{MFileNumber}{1}}
\fi

% Testen welcher Uebersetzer hier am Werk ist

\ifttm
\setcounter{MOutput}{3}
\else
\ifx\pdfoutput\undefined
  \pdffalse
  \setcounter{MOutput}{\MOutputDVI}
  \message{Verarbeitung mit latex, Ausgabe in dvi.}
\else
  \setcounter{MOutput}{\MOutputPDF}
  \message{Verarbeitung mit pdflatex, Ausgabe in pdf.}
  \ifnum \pdfoutput=0
    \pdffalse
  \setcounter{MOutput}{\MOutputDVI}
  \message{Verarbeitung mit pdflatex, Ausgabe in dvi.}
  \else
    \ifnum\pdfoutput=1
    \pdftrue
  \setcounter{MOutput}{\MOutputPDF}
  \message{Verarbeitung mit pdflatex, Ausgabe in pdf.}
    \fi
  \fi
\fi
\fi

\ifnum\value{MOutput}=\MOutputPDF
\DeclareGraphicsExtensions{.pdf,.png,.jpg}
\fi

\ifnum\value{MOutput}=\MOutputDVI
\DeclareGraphicsExtensions{.eps,.png,.jpg}
\fi

\ifnum\value{MOutput}=\MOutputHTML
% Wird vom Konverter leider nicht erkannt und daher in split.pm hardcodiert!
\DeclareGraphicsExtensions{.png,.jpg,.gif}
\fi

% Umdefinition der hyperref-Nummerierung im PDF-Modus
\ifttm
\else
\renewcommand{\theHfigure}{\arabic{chapter}.\arabic{section}.\arabic{figure}}
\fi

% Makro, um in der HTML-Ausgabe die zuerst zu oeffnende Datei zu kennzeichnen
\ifttm
\newcommand{\MGlobalStart}{\special{html:<!-- mglobalstarttag -->}}
\else
\newcommand{\MGlobalStart}{}
\fi

% Makro, um bei scormlogin ein pullen des Benutzers bei Aufruf der Seite zu erzwingen (typischerweise auf der Einstiegsseite)
\ifttm
\newcommand{\MPullSite}{\special{html:<!-- pullsite //-->}}
\else
\newcommand{\MPullSite}{}
\fi

% Makro, um in der HTML-Ausgabe die Kapiteluebersicht zu kennzeichnen
\ifttm
\newcommand{\MGlobalChapterTag}{\special{html:<!-- mglobalchaptertag -->}}
\else
\newcommand{\MGlobalChapterTag}{}
\fi

% Makro, um in der HTML-Ausgabe die Konfiguration zu kennzeichnen
\ifttm
\newcommand{\MGlobalConfTag}{\special{html:<!-- mglobalconfigtag -->}}
\else
\newcommand{\MGlobalConfTag}{}
\fi

% Makro, um in der HTML-Ausgabe die Standortbeschreibung zu kennzeichnen
\ifttm
\newcommand{\MGlobalLocationTag}{\special{html:<!-- mgloballocationtag -->}}
\else
\newcommand{\MGlobalLocationTag}{}
\fi

% Makro, um in der HTML-Ausgabe die persoenlichen Daten zu kennzeichnen
\ifttm
\newcommand{\MGlobalDataTag}{\special{html:<!-- mglobaldatatag -->}}
\else
\newcommand{\MGlobalDataTag}{}
\fi

% Makro, um in der HTML-Ausgabe die Suchseite zu kennzeichnen
\ifttm
\newcommand{\MGlobalSearchTag}{\special{html:<!-- mglobalsearchtag -->}}
\else
\newcommand{\MGlobalSearchTag}{}
\fi

% Makro, um in der HTML-Ausgabe die Favoritenseite zu kennzeichnen
\ifttm
\newcommand{\MGlobalFavoTag}{\special{html:<!-- mglobalfavoritestag -->}}
\else
\newcommand{\MGlobalFavoTag}{}
\fi

% Makro, um in der HTML-Ausgabe die Eingangstestseite zu kennzeichnen
\ifttm
\newcommand{\MGlobalSTestTag}{\special{html:<!-- mglobalstesttag -->}}
\else
\newcommand{\MGlobalSTestTag}{}
\fi

% Makro, um in der PDF-Ausgabe ein Wasserzeichen zu definieren
\ifttm
\newcommand{\MWatermarkSettings}{\relax}
\else
\newcommand{\MWatermarkSettings}{%
% \SetWatermarkText{(c) MINT-Kolleg Baden-W�rttemberg 2014}
% \SetWatermarkLightness{0.85}
% \SetWatermarkScale{1.5}
}
\fi

\ifttm
\newcommand{\MBinom}[2]{\left({\begin{array}{c} #1 \\ #2 \end{array}}\right)}
\else
\newcommand{\MBinom}[2]{\binom{#1}{#2}}
\fi

\ifttm
\newcommand{\DeclareMathOperator}[2]{\def#1{\mathrm{#2}}}
\newcommand{\operatorname}[1]{\mathrm{#1}}
\fi

%----------------- Makros fuer die gemischte HTML/PDF-Konvertierung ------------------------------

\newcommand{\MTestName}{\relax} % wird durch Test-Umgebung gesetzt

% Fuer experimentelle Kursinhalte, die im Release-Umsetzungsvorgang eine Fehlermeldung
% produzieren sollen aber sonst normal umgesetzt werden
\newenvironment{MExperimental}{%
}{%
}

% Wird von ttm nicht richtig umgesetzt!!
\newenvironment{MExerciseItems}{%
\renewcommand\theenumi{\alph{enumi}}%
\begin{enumerate}%
}{%
\end{enumerate}%
}


\definecolor{infoshadecolor}{rgb}{0.75,0.75,0.75}
\definecolor{exmpshadecolor}{rgb}{0.875,0.875,0.875}
\definecolor{expeshadecolor}{rgb}{0.95,0.95,0.95}
\definecolor{framecolor}{rgb}{0.2,0.2,0.2}

% Bei PDF-Uebersetzung wird hinter den Start jeder Satz/Info-aehnlichen Umgebung eine leere mbox gesetzt, damit
% fuehrende Listen oder enums nicht den Zeilenumbruch kaputtmachen
%\ifttm
\def\MTB{}
%\else
%\def\MTB{\mbox{}}
%\fi


\ifttm
\newcommand{\MRelates}{\special{html:<mi>&wedgeq;</mi>}}
\else
\def\MRelates{\stackrel{\scriptscriptstyle\wedge}{=}}
\fi

\def\MInch{\text{''}}
\def\Mdd{\textit{''}}

\ifttm
\def\MNL{ \newline }
\newenvironment{MArray}[1]{\begin{array}{#1}}{\end{array}}
\else
\def\MNL{ \\ }
\newenvironment{MArray}[1]{\begin{array}{#1}}{\end{array}}
\fi

\newcommand{\MBox}[1]{$\mathrm{#1}$}
\newcommand{\MMBox}[1]{\mathrm{#1}}


\ifttm%
\newcommand{\Mtfrac}[2]{{\textstyle \frac{#1}{#2}}}
\newcommand{\Mdfrac}[2]{{\displaystyle \frac{#1}{#2}}}
\newcommand{\Mmeasuredangle}{\special{html:<mi>&angmsd;</mi>}}
\else%
\newcommand{\Mtfrac}[2]{\tfrac{#1}{#2}}
\newcommand{\Mdfrac}[2]{\dfrac{#1}{#2}}
\newcommand{\Mmeasuredangle}{\measuredangle}
\relax
\fi

% Matrizen und Vektoren

% Inhalt wird in der Form a & b \\ c & d erwartet
% Vorsicht: MVector = Komponentenspalte, MVec = Variablensymbol
\ifttm%
\newcommand{\MVector}[1]{\left({\begin{array}{c}#1\end{array}}\right)}
\else%
\newcommand{\MVector}[1]{\begin{pmatrix}#1\end{pmatrix}}
\fi



\newcommand{\MVec}[1]{\vec{#1}}
\newcommand{\MDVec}[1]{\overrightarrow{#1}}

%----------------- Umgebungen fuer Definitionen und Saetze ----------------------------------------

% Fuegt einen Tabellen-Zeilenumbruch ein im PDF, aber nicht im HTML
\newcommand{\TSkip}{\ifttm \else&\ \\\fi}

\newenvironment{infoshaded}{%
\def\FrameCommand{\fboxsep=\FrameSep \fcolorbox{framecolor}{infoshadecolor}}%
\MakeFramed {\advance\hsize-\width \FrameRestore}}%
{\endMakeFramed}

\newenvironment{expeshaded}{%
\def\FrameCommand{\fboxsep=\FrameSep \fcolorbox{framecolor}{expeshadecolor}}%
\MakeFramed {\advance\hsize-\width \FrameRestore}}%
{\endMakeFramed}

\newenvironment{exmpshaded}{%
\def\FrameCommand{\fboxsep=\FrameSep \fcolorbox{framecolor}{exmpshadecolor}}%
\MakeFramed {\advance\hsize-\width \FrameRestore}}%
{\endMakeFramed}

\def\STDCOLOR{black}

\ifttm%
\else%
\newtheoremstyle{MSatzStyle}
  {1cm}                   %Space above
  {1cm}                   %Space below
  {\normalfont\itshape}   %Body font
  {}                      %Indent amount (empty = no indent,
                          %\parindent = para indent)
  {\normalfont\bfseries}  %Thm head font
  {}                      %Punctuation after thm head
  {\newline}              %Space after thm head: " " = normal interword
                          %space; \newline = linebreak
  {\thmname{#1}\thmnumber{ #2}\thmnote{ (#3)}}
                          %Thm head spec (can be left empty, meaning
                          %`normal')
                          %
\newtheoremstyle{MDefStyle}
  {1cm}                   %Space above
  {1cm}                   %Space below
  {\normalfont}           %Body font
  {}                      %Indent amount (empty = no indent,
                          %\parindent = para indent)
  {\normalfont\bfseries}  %Thm head font
  {}                      %Punctuation after thm head
  {\newline}              %Space after thm head: " " = normal interword
                          %space; \newline = linebreak
  {\thmname{#1}\thmnumber{ #2}\thmnote{ (#3)}}
                          %Thm head spec (can be left empty, meaning
                          %`normal')
\fi%

\newcommand{\MInfoText}{Info}

\newcounter{MHintCounter}
\newcounter{MCodeEditCounter}

\newcounter{MLastIndex}  % Enthaelt die dritte Stelle (Indexnummer) des letzten angelegten Objekts
\newcounter{MLastType}   % Enthaelt den Typ des letzten angelegten Objekts (mithilfe der unten definierten Konstanten). Die Entscheidung, wie der Typ dargstellt wird, wird in split.pm beim Postprocessing getroffen.
\newcounter{MLastTypeEq} % =1 falls das Label in einer Matheumgebung (equation, eqnarray usw.) steht, =2 falls das Label in einer table-Umgebung steht

% Da ttm keine Zahlmakros verarbeiten kann, werden diese Nummern in den Zuweisungen hardcodiert!
\def\MTypeSection{1}          %# Zaehler ist section
\def\MTypeSubsection{2}       %# Zaehler ist subsection
\def\MTypeSubsubsection{3}    %# Zaehler ist subsubsection
\def\MTypeInfo{4}             %# Eine Infobox, Separatzaehler fuer die Chemie (auch wenn es dort nicht nummeriert wird) ist MInfoCounter
\def\MTypeExercise{5}         %# Eine Aufgabe, Separatzaehler fuer die Chemie ist MExerciseCounter
\def\MTypeExample{6}          %# Eine Beispielbox, Separatzaehler fuer die Chemie ist MExampleCounter
\def\MTypeExperiment{7}       %# Eine Versuchsbox, Separatzaehler fuer die Chemie ist MExperimentCounter
\def\MTypeGraphics{8}         %# Eine Graphik, Separatzaehler fuer alle FB ist MGraphicsCounter
\def\MTypeTable{9}            %# Eine Tabellennummer, hat keinen Zaehler da durch table gezaehlt wird
\def\MTypeEquation{10}        %# Eine Gleichungsnummer, hat keinen Zaehler da durch equation/eqnarray gezaehlt wird
\def\MTypeTheorem{11}         % Ein theorem oder xtheorem, Separatzaehler fuer die Chemie ist MTheoremCounter
\def\MTypeVideo{12}           %# Ein Video,Separatzaehler fuer alle FB ist MVideoCounter
\def\MTypeEntry{13}           %# Ein Eintrag fuer die Stichwortliste, wird nicht gezaehlt sondern erhaelt im preparsing ein unique-label 

% Zaehler fuer das Labelsystem sind prefixcounter, jeder Zaehler wird VOR dem gezaehlten Objekt inkrementiert und zaehlt daher das aktuelle Objekt
\newcounter{MInfoCounter}
\newcounter{MExerciseCounter}
\newcounter{MExampleCounter}
\newcounter{MExperimentCounter}
\newcounter{MGraphicsCounter}
\newcounter{MTableCounter}
\newcounter{MEquationCounter}  % Nur im HTML, sonst durch "equation"-counter von latex realisiert
\newcounter{MTheoremCounter}
\newcounter{MObjectCounter}   % Gemeinsamer Zaehler fuer Objekte (ausser Grafiken/Tabellen) in Mathe/Info/Physik
\newcounter{MVideoCounter}
\newcounter{MEntryCounter}

\newcounter{MTestSite} % 1 = Subsubsection ist eine Pruefungsseite, 0 = ist eine normale Seite (inkl. Hilfeseite)

\def\MCell{$\phantom{a}$}

\newenvironment{MExportExercise}{\begin{MExercise}}{\end{MExercise}} % wird von mconvert abgefangen

\def\MGenerateExNumber{%
\ifnum\value{MSepNumbers}=0%
\arabic{section}.\arabic{subsection}.\arabic{MObjectCounter}\setcounter{MLastIndex}{\value{MObjectCounter}}%
\else%
\arabic{section}.\arabic{subsection}.\arabic{MExerciseCounter}\setcounter{MLastIndex}{\value{MExerciseCounter}}%
\fi%
}%

\def\MGenerateExmpNumber{%
\ifnum\value{MSepNumbers}=0%
\arabic{section}.\arabic{subsection}.\arabic{MObjectCounter}\setcounter{MLastIndex}{\value{MObjectCounter}}%
\else%
\arabic{section}.\arabic{subsection}.\arabic{MExerciseCounter}\setcounter{MLastIndex}{\value{MExampleCounter}}%
\fi%
}%

\def\MGenerateInfoNumber{%
\ifnum\value{MSepNumbers}=0%
\arabic{section}.\arabic{subsection}.\arabic{MObjectCounter}\setcounter{MLastIndex}{\value{MObjectCounter}}%
\else%
\arabic{section}.\arabic{subsection}.\arabic{MExerciseCounter}\setcounter{MLastIndex}{\value{MInfoCounter}}%
\fi%
}%

\def\MGenerateSiteNumber{%
\arabic{section}.\arabic{subsection}.\arabic{subsubsection}%
}%

% Funktionalitaet fuer Auswahlaufgaben

\newcounter{MExerciseCollectionCounter} % = 0 falls nicht in collection-Umgebung, ansonsten Schachtelungstiefe
\newcounter{MExerciseCollectionTextCounter} % wird von MExercise-Umgebung inkrementiert und von MExerciseCollection-Umgebung auf Null gesetzt

\ifttm
% MExerciseCollection gruppiert Aufgaben, die dynamisch aus der Datenbank gezogen werden und nicht direkt in der HTML-Seite stehen
% Parameter: #1 = ID der Collection, muss eindeutig fuer alle IN DER DB VORHANDENEN collections sein unabhaengig vom Kurs
%            #2 = Optionsargument (im Moment: 1 = Iterative Auswahl, 2 = Zufallsbasierte Auswahl)
\newenvironment{MExerciseCollection}[2]{%
\addtocounter{MExerciseCollectionCounter}{1}
\setcounter{MExerciseCollectionTextCounter}{0}
\special{html:<!-- mexercisecollectionstart;;}#1\special{html:;;}#2\special{html:;; //-->}%
}{%
\special{html:<!-- mexercisecollectionstop //-->}%
\addtocounter{MExerciseCollectionCounter}{-1}
}
\else
\newenvironment{MExerciseCollection}[2]{%
\addtocounter{MExerciseCollectionCounter}{1}
\setcounter{MExerciseCollectionTextCounter}{0}
}{%
\addtocounter{MExerciseCollectionCounter}{-1}
}
\fi

% Bei Uebersetzung nach PDF werden die theorem-Umgebungen verwendet, bei Uebersetzung in HTML ein manuelles Makro
\ifttm%

  \newenvironment{MHint}[1]{  \special{html:<button name="Name_MHint}\arabic{MHintCounter}\special{html:" class="hintbutton_closed" id="MHint}\arabic{MHintCounter}\special{html:_button" %
  type="button" onclick="toggle_hint('MHint}\arabic{MHintCounter}\special{html:');">}#1\special{html:</button>}
  \special{html:<div class="hint" style="display:none" id="MHint}\arabic{MHintCounter}\special{html:"> }}{\begin{html}</div>\end{html}\addtocounter{MHintCounter}{1}}

  \newenvironment{MCOSHZusatz}{  \special{html:<button name="Name_MHint}\arabic{MHintCounter}\special{html:" class="chintbutton_closed" id="MHint}\arabic{MHintCounter}\special{html:_button" %
  type="button" onclick="toggle_hint('MHint}\arabic{MHintCounter}\special{html:');">}Weiterf�hrende Inhalte\special{html:</button>}
  \special{html:<div class="hintc" style="display:none" id="MHint}\arabic{MHintCounter}\special{html:">
  <div class="coshwarn">Diese Inhalte gehen �ber das Kursniveau hinaus und werden in den Aufgaben und Tests nicht abgefragt.</div><br />}
  \addtocounter{MHintCounter}{1}}{\begin{html}</div>\end{html}}

  
  \newenvironment{MDefinition}{\begin{definition}\setcounter{MLastIndex}{\value{definition}}\ \\}{\end{definition}}

  
  \newenvironment{MExercise}{
  \renewcommand{\MStdPoints}{4}
  \addtocounter{MExerciseCounter}{1}
  \addtocounter{MObjectCounter}{1}
  \setcounter{MLastType}{5}

  \ifnum\value{MExerciseCollectionCounter}=0\else\addtocounter{MExerciseCollectionTextCounter}{1}\special{html:<!-- mexercisetextstart;;}\arabic{MExerciseCollectionTextCounter}\special{html:;; //-->}\fi
  \special{html:<div class="aufgabe" id="ADIV_}\MGenerateExNumber\special{html:">}%
  \textbf{Aufgabe \MGenerateExNumber
  } \ \\}{
  \special{html:</div><!-- mfeedbackbutton;Aufgabe;}\arabic{MTestSite}\special{html:;}\MGenerateExNumber\special{html:; //-->}
  \ifnum\value{MExerciseCollectionCounter}=0\else\special{html:<!-- mexercisetextstop //-->}\fi
  }

  % Stellt eine Kombination aus Aufgabe, Loesungstext und Eingabefeld bereit,
  % bei der Aufgabentext und Musterloesung sowie die zugehoerigen Feldelemente
  % extern bezogen und div-aktualisiert werden, das Eingabefeld aber immer das gleiche ist.
  \newenvironment{MFetchExercise}{
  \addtocounter{MExerciseCounter}{1}
  \addtocounter{MObjectCounter}{1}
  \setcounter{MLastType}{5}

  \special{html:<div class="aufgabe" id="ADIV_}\MGenerateExNumber\special{html:">}%
  \textbf{Aufgabe \MGenerateExNumber
  } \ \\%
  \special{html:</div><div class="exfetch_text" id="ADIVTEXT_}\MGenerateExNumber\special{html:">}%
  \special{html:</div><div class="exfetch_sol" id="ADIVSOL_}\MGenerateExNumber\special{html:">}%
  \special{html:</div><div class="exfetch_input" id="ADIVINPUT_}\MGenerateExNumber\special{html:">}%
  }{
  \special{html:</div>}
  }

  \newenvironment{MExample}{
  \addtocounter{MExampleCounter}{1}
  \addtocounter{MObjectCounter}{1}
  \setcounter{MLastType}{6}
  \begin{html}
  <div class="exmp">
  <div class="exmprahmen">
  \end{html}\textbf{Beispiel
  \ifnum\value{MSepNumbers}=0
  \arabic{section}.\arabic{subsection}.\arabic{MObjectCounter}\setcounter{MLastIndex}{\value{MObjectCounter}}
  \else
  \arabic{section}.\arabic{subsection}.\arabic{MExampleCounter}\setcounter{MLastIndex}{\value{MExampleCounter}}
  \fi
  } \ \\}{\begin{html}</div>
  </div>
  \end{html}
  \special{html:<!-- mfeedbackbutton;Beispiel;}\arabic{MTestSite}\special{html:;}\MGenerateExmpNumber\special{html:; //-->}
  }

  \newenvironment{MExperiment}{
  \addtocounter{MExperimentCounter}{1}
  \addtocounter{MObjectCounter}{1}
  \setcounter{MLastType}{7}
  \begin{html}
  <div class="expe">
  <div class="experahmen">
  \end{html}\textbf{Versuch
  \ifnum\value{MSepNumbers}=0
  \arabic{section}.\arabic{subsection}.\arabic{MObjectCounter}\setcounter{MLastIndex}{\value{MObjectCounter}}
  \else
%  \arabic{MExperimentCounter}\setcounter{MLastIndex}{\value{MExperimentCounter}}
  \arabic{section}.\arabic{subsection}.\arabic{MExperimentCounter}\setcounter{MLastIndex}{\value{MExperimentCounter}}
  \fi
  } \ \\}{\begin{html}</div>
  </div>
  \end{html}}

  \newenvironment{MChemInfo}{
  \setcounter{MLastType}{4}
  \begin{html}
  <div class="info">
  <div class="inforahmen">
  \end{html}}{\begin{html}</div>
  </div>
  \end{html}}

  \newenvironment{MXInfo}[1]{
  \addtocounter{MInfoCounter}{1}
  \addtocounter{MObjectCounter}{1}
  \setcounter{MLastType}{4}
  \begin{html}
  <div class="info">
  <div class="inforahmen">
  \end{html}\textbf{#1
  \ifnum\value{MInfoNumbers}=0
  \else
    \ifnum\value{MSepNumbers}=0
    \arabic{section}.\arabic{subsection}.\arabic{MObjectCounter}\setcounter{MLastIndex}{\value{MObjectCounter}}
    \else
    \arabic{MInfoCounter}\setcounter{MLastIndex}{\value{MInfoCounter}}
    \fi
  \fi
  } \ \\}{\begin{html}</div>
  </div>
  \end{html}
  \special{html:<!-- mfeedbackbutton;Info;}\arabic{MTestSite}\special{html:;}\MGenerateInfoNumber\special{html:; //-->}
  }

  \newenvironment{MInfo}{\ifnum\value{MInfoNumbers}=0\begin{MChemInfo}\else\begin{MXInfo}{Info}\ \\ \fi}{\ifnum\value{MInfoNumbers}=0\end{MChemInfo}\else\end{MXInfo}\fi}

\else%

  \theoremstyle{MSatzStyle}
  \newtheorem{thm}{Satz}[section]
  \newtheorem{thmc}{Satz}
  \theoremstyle{MDefStyle}
  \newtheorem{defn}[thm]{Definition}
  \newtheorem{exmp}[thm]{Beispiel}
  \newtheorem{info}[thm]{\MInfoText}
  \theoremstyle{MDefStyle}
  \newtheorem{defnc}{Definition}
  \theoremstyle{MDefStyle}
  \newtheorem{exmpc}{Beispiel}[section]
  \theoremstyle{MDefStyle}
  \newtheorem{infoc}{\MInfoText}
  \theoremstyle{MDefStyle}
  \newtheorem{exrc}{Aufgabe}[section]
  \theoremstyle{MDefStyle}
  \newtheorem{verc}{Versuch}[section]
  
  \newenvironment{MFetchExercise}{}{} % kann im PDF nicht dargestellt werden
  
  \newenvironment{MExercise}{\begin{exrc}\renewcommand{\MStdPoints}{1}\MTB}{\end{exrc}}
  \newenvironment{MHint}[1]{\ \\ \underline{#1:}\\}{}
  \newenvironment{MCOSHZusatz}{\ \\ \underline{Weiterf�hrende Inhalte:}\\}{}
  \newenvironment{MDefinition}{\ifnum\value{MInfoNumbers}=0\begin{defnc}\else\begin{defn}\fi\MTB}{\ifnum\value{MInfoNumbers}=0\end{defnc}\else\end{defn}\fi}
%  \newenvironment{MExample}{\begin{exmp}}{\ \linebreak[1] \ \ \ \ $\phantom{a}$ \ \hfill $\blacklozenge$\end{exmp}}
  \newenvironment{MExample}{
    \ifnum\value{MInfoNumbers}=0\begin{exmpc}\else\begin{exmp}\fi
    \MTB
    \begin{exmpshaded}
    \ \newline
}{
    \end{exmpshaded}
    \ifnum\value{MInfoNumbers}=0\end{exmpc}\else\end{exmp}\fi
}
  \newenvironment{MChemInfo}{\begin{infoshaded}}{\end{infoshaded}}

  \newenvironment{MInfo}{\ifnum\value{MInfoNumbers}=0\begin{MChemInfo}\else\renewcommand{\MInfoText}{Info}\begin{info}\begin{infoshaded}
  \MTB
   \ \newline
    \fi
  }{\ifnum\value{MInfoNumbers}=0\end{MChemInfo}\else\end{infoshaded}\end{info}\fi}

  \newenvironment{MXInfo}[1]{
    \renewcommand{\MInfoText}{#1}
    \ifnum\value{MInfoNumbers}=0\begin{infoc}\else\begin{info}\fi%
    \MTB
    \begin{infoshaded}
    \ \newline
  }{\end{infoshaded}\ifnum\value{MInfoNumbers}=0\end{infoc}\else\end{info}\fi}

  \newenvironment{MExperiment}{
    \renewcommand{\MInfoText}{Versuch}
    \ifnum\value{MInfoNumbers}=0\begin{verc}\else\begin{info}\fi
    \MTB
    \begin{expeshaded}
    \ \newline
  }{
    \end{expeshaded}
    \ifnum\value{MInfoNumbers}=0\end{verc}\else\end{info}\fi
  }
\fi%

% MHint sollte nicht direkt fuer Loesungen benutzt werden wegen solutionselect
\newenvironment{MSolution}{\begin{MHint}{L"osung}}{\end{MHint}}

\newcounter{MCodeCounter}

\ifttm
\newenvironment{MCode}{\special{html:<!-- mcodestart -->}\ttfamily\color{blue}}{\special{html:<!-- mcodestop -->}}
\else
\newenvironment{MCode}{\begin{flushleft}\ttfamily\addtocounter{MCodeCounter}{1}}{\addtocounter{MCodeCounter}{-1}\end{flushleft}}
% Ohne color-Statement da inkompatible mit framed/shaded-Boxen aus dem framed-package
\fi

%----------------- Sonderdefinitionen fuer Symbole, die der Konverter nicht kann ----------------------------------------------

\ifttm%
\newcommand{\MUnderset}[2]{\underbrace{#2}_{#1}}%
\else%
\newcommand{\MUnderset}[2]{\underset{#1}{#2}}%
\fi%

\ifttm
\newcommand{\MThinspace}{\special{html:<mi>&#x2009;</mi>}}
\else
\newcommand{\MThinspace}{\,}
\fi

\ifttm
\newcommand{\glq}{\begin{html}&sbquo;\end{html}}
\newcommand{\grq}{\begin{html}&lsquo;\end{html}}
\newcommand{\glqq}{\begin{html}&bdquo;\end{html}}
\newcommand{\grqq}{\begin{html}&ldquo;\end{html}}
\fi

\ifttm
\newcommand{\MNdash}{\begin{html}&ndash;\end{html}}
\else
\newcommand{\MNdash}{--}
\fi

%\ifttm\def\MIU{\special{html:<mi>&#8520;</mi>}}\else\def\MIU{\mathrm{i}}\fi
\def\MIU{\mathrm{i}}
\def\MEU{e} % TU9-Onlinekurs: italic-e
%\def\MEU{\mathrm{e}} % Alte Onlinemodule: roman-e
\def\MD{d} % Kursives d in Integralen im TU9-Onlinekurs
%\def\MD{\mathrm{d}} % roman-d in den alten Onlinemodulen
\def\MDB{\|}

%zusaetzlicher Leerraum vor "\MD"
\ifttm%
\def\MDSpace{\special{html:<mi>&#x2009;</mi>}}
\else%
\def\MDSpace{\,}
\fi%
\newcommand{\MDwSp}{\MDSpace\MD}%

\ifttm
\def\Mdq{\dq}
\else
\def\Mdq{\dq}
\fi

\def\MSpan#1{\left<{#1}\right>}
\def\MSetminus{\setminus}
\def\MIM{I}

\ifttm
\newcommand{\ld}{\text{ld}}
\newcommand{\lg}{\text{lg}}
\else
\DeclareMathOperator{\ld}{ld}
%\newcommand{\lg}{\text{lg}} % in latex schon definiert
\fi


\def\Mmapsto{\ifttm\special{html:<mi>&mapsto;</mi>}\else\mapsto\fi} 
\def\Mvarphi{\ifttm\phi\else\varphi\fi}
\def\Mphi{\ifttm\varphi\else\phi\fi}
\ifttm%
\newcommand{\MEumu}{\special{html:<mi>&#x3BC;</mi>}}%
\else%
\newcommand{\MEumu}{\textrm{\textmu}}%
\fi
\def\Mvarepsilon{\ifttm\epsilon\else\varepsilon\fi}
\def\Mepsilon{\ifttm\varepsilon\else\epsilon\fi}
\def\Mvarkappa{\ifttm\kappa\else\varkappa\fi}
\def\Mkappa{\ifttm\varkappa\else\kappa\fi}
\def\Mcomplement{\ifttm\special{html:<mi>&comp;</mi>}\else\complement\fi} 
\def\MWW{\mathrm{WW}}
\def\Mmod{\ifttm\special{html:<mi>&nbsp;mod&nbsp;</mi>}\else\mod\fi} 

\ifttm%
\def\mod{\text{\;mod\;}}%
\def\MNEquiv{\special{html:<mi>&NotCongruent;</mi>}}% 
\def\MNSubseteq{\special{html:<mi>&NotSubsetEqual;</mi>}}%
\def\MEmptyset{\special{html:<mi>&empty;</mi>}}%
\def\MVDots{\special{html:<mi>&#x22EE;</mi>}}%
\def\MHDots{\special{html:<mi>&#x2026;</mi>}}%
\def\Mddag{\special{html:<mi>&#x1202;</mi>}}%
\def\sphericalangle{\special{html:<mi>&measuredangle;</mi>}}%
\def\nparallel{\special{html:<mi>&nparallel;</mi>}}%
\def\MProofEnd{\special{html:<mi>&#x25FB;</mi>}}%
\newenvironment{MProof}[1]{\underline{#1}:\MCR\MCR}{\hfill $\MProofEnd$}%
\else%
\def\MNEquiv{\not\equiv}%
\def\MNSubseteq{\not\subseteq}%
\def\MEmptyset{\emptyset}%
\def\MVDots{\vdots}%
\def\MHDots{\hdots}%
\def\Mddag{\ddag}%
\newenvironment{MProof}[1]{\begin{proof}[#1]}{\end{proof}}%
\fi%



% Spaces zum Auffuellen von Tabellenbreiten, die nur im HTML wirken
\ifttm%
\def\MTSP{\:}%
\else%
\def\MTSP{}%
\fi%

\DeclareMathOperator{\arsinh}{arsinh}
\DeclareMathOperator{\arcosh}{arcosh}
\DeclareMathOperator{\artanh}{artanh}
\DeclareMathOperator{\arcoth}{arcoth}


\newcommand{\MMathSet}[1]{\mathbb{#1}}
\def\N{\MMathSet{N}}
\def\Z{\MMathSet{Z}}
\def\Q{\MMathSet{Q}}
\def\R{\MMathSet{R}}
\def\C{\MMathSet{C}}

\newcounter{MForLoopCounter}
\newcommand{\MForLoop}[2]{\setcounter{MForLoopCounter}{#1}\ifnum\value{MForLoopCounter}=0{}\else{{#2}\addtocounter{MForLoopCounter}{-1}\MForLoop{\value{MForLoopCounter}}{#2}}\fi}

\newcounter{MSiteCounter}
\newcounter{MFieldCounter} % Kombination section.subsection.site.field ist eindeutig in allen Modulen, field alleine nicht

\newcounter{MiniMarkerCounter}

\ifttm
\newenvironment{MMiniPageP}[1]{\begin{minipage}{#1\linewidth}\special{html:<!-- minimarker;;}\arabic{MiniMarkerCounter}\special{html:;;#1; //-->}}{\end{minipage}\addtocounter{MiniMarkerCounter}{1}}
\else
\newenvironment{MMiniPageP}[1]{\begin{minipage}{#1\linewidth}}{\end{minipage}\addtocounter{MiniMarkerCounter}{1}}
\fi

\newcounter{AlignCounter}

\newcommand{\MStartJustify}{\ifttm\special{html:<!-- startalign;;}\arabic{AlignCounter}\special{html:;;justify; //-->}\fi}
\newcommand{\MStopJustify}{\ifttm\special{html:<!-- stopalign;;}\arabic{AlignCounter}\special{html:; //-->}\fi\addtocounter{AlignCounter}{1}}

\newenvironment{MJTabular}[1]{
\MStartJustify
\begin{tabular}{#1}
}{
\end{tabular}
\MStopJustify
}

\newcommand{\MImageLeft}[2]{
\begin{center}
\begin{tabular}{lc}
\MStartJustify
\begin{MMiniPageP}{0.65}
#1
\end{MMiniPageP}
\MStopJustify
&
\begin{MMiniPageP}{0.3}
#2  
\end{MMiniPageP}
\end{tabular}
\end{center}
}

\newcommand{\MImageHalf}[2]{
\begin{center}
\begin{tabular}{lc}
\MStartJustify
\begin{MMiniPageP}{0.45}
#1
\end{MMiniPageP}
\MStopJustify
&
\begin{MMiniPageP}{0.45}
#2  
\end{MMiniPageP}
\end{tabular}
\end{center}
}

\newcommand{\MBigImageLeft}[2]{
\begin{center}
\begin{tabular}{lc}
\MStartJustify
\begin{MMiniPageP}{0.25}
#1
\end{MMiniPageP}
\MStopJustify
&
\begin{MMiniPageP}{0.7}
#2  
\end{MMiniPageP}
\end{tabular}
\end{center}
}

\ifttm
\def\No{\mathbb{N}_0}
\else
\def\No{\ensuremath{\N_0}}
\fi
\def\MT{\textrm{\tiny T}}
\newcommand{\MTranspose}[1]{{#1}^{\MT}}
\ifttm
\newcommand{\MRe}{\mathsf{Re}}
\newcommand{\MIm}{\mathsf{Im}}
\else
\DeclareMathOperator{\MRe}{Re}
\DeclareMathOperator{\MIm}{Im}
\fi

\newcommand{\Mid}{\mathrm{id}}
\newcommand{\MFeinheit}{\mathrm{feinh}}

\ifttm
\newcommand{\Msubstack}[1]{\begin{array}{c}{#1}\end{array}}
\else
\newcommand{\Msubstack}[1]{\substack{#1}}
\fi

% Typen von Fragefeldern:
% 1 = Alphanumerisch, case-sensitive-Vergleich
% 2 = Ja/Nein-Checkbox, Loesung ist 0 oder 1   (OPTION = Image-id fuer Rueckmeldung)
% 3 = Reelle Zahlen Geparset
% 4 = Funktionen Geparset (mit Stuetzstellen zur ueberpruefung)

% Dieser Befehl erstellt ein interaktives Aufgabenfeld. Parameter:
% - #1 Laenge in Zeichen
% - #2 Loesungstext (alphanumerisch, case sensitive)
% - #3 AufgabenID (alphanumerisch, case sensitive)
% - #4 Typ (Kennnummer)
% - #5 String fuer Optionen (ggf. mit Semikolon getrennte Einzelstrings)
% - #6 Anzahl Punkte
% - #7 uxid (kann z.B. Loesungsstring sein)
% ACHTUNG: Die langen Zeilen bitte so lassen, Zeilenumbrueche im tex werden in div's umgesetzt
\newcommand{\MQuestionID}[7]{
\ifttm
\special{html:<!-- mdeclareuxid;;}UX#7\special{html:;;}\arabic{section}\special{html:;;}#3\special{html:;; //-->}%
\special{html:<!-- mdeclarepoints;;}\arabic{section}\special{html:;;}#3\special{html:;;}#6\special{html:;;}\arabic{MTestSite}\special{html:;;}\arabic{chapter}%
\special{html:;; //--><!-- onloadstart //-->CreateQuestionObj("}#7\special{html:",}\arabic{MFieldCounter}\special{html:,"}#2%
\special{html:","}#3\special{html:",}#4\special{html:,"}#5\special{html:",}#6\special{html:,}\arabic{MTestSite}\special{html:,}\arabic{section}%
\special{html:);<!-- onloadstop //-->}%
\special{html:<input mfieldtype="}#4\special{html:" name="Name_}#3\special{html:" id="}#3\special{html:" type="text" size="}#1\special{html:" maxlength="}#1%
\special{html:" }\ifnum\value{MGroupActive}=0\special{html:onfocus="handlerFocus(}\arabic{MFieldCounter}%
\special{html:);" onblur="handlerBlur(}\arabic{MFieldCounter}\special{html:);" onkeyup="handlerChange(}\arabic{MFieldCounter}\special{html:,0);" onpaste="handlerChange(}\arabic{MFieldCounter}\special{html:,0);" oninput="handlerChange(}\arabic{MFieldCounter}\special{html:,0);" onpropertychange="handlerChange(}\arabic{MFieldCounter}\special{html:,0);"/>}%
\special{html:<img src="images/questionmark.gif" width="20" height="20" border="0" align="absmiddle" id="}QM#3\special{html:"/>}
\else%
\special{html:onblur="handlerBlur(}\arabic{MFieldCounter}%
\special{html:);" onfocus="handlerFocus(}\arabic{MFieldCounter}\special{html:);" onkeyup="handlerChange(}\arabic{MFieldCounter}\special{html:,1);" onpaste="handlerChange(}\arabic{MFieldCounter}\special{html:,1);" oninput="handlerChange(}\arabic{MFieldCounter}\special{html:,1);" onpropertychange="handlerChange(}\arabic{MFieldCounter}\special{html:,1);"/>}%
\special{html:<img src="images/questionmark.gif" width="20" height="20" border="0" align="absmiddle" id="}QM#3\special{html:"/>}\fi%
\else%
\ifnum\value{QBoxFlag}=1\fbox{$\phantom{\MForLoop{#1}{b}}$}\else$\phantom{\MForLoop{#1}{b}}$\fi%
\fi%
}

% ACHTUNG: Die langen Zeilen bitte so lassen, Zeilenumbrueche im tex werden in div's umgesetzt
% QuestionCheckbox macht ausserhalb einer QuestionGroup keinen Sinn!
% #1 = solution (1 oder 0), ggf. mit ::smc abgetrennt auszuschliessende single-choice-boxen (UXIDs durch , getrennt), #2 = id, #3 = points, #4 = uxid
\newcommand{\MQuestionCheckbox}[4]{
\ifttm
\special{html:<!-- mdeclareuxid;;}UX#4\special{html:;;}\arabic{section}\special{html:;;}#2\special{html:;; //-->}%
\ifnum\value{MGroupActive}=0\MDebugMessage{ERROR: Checkbox Nr. \arabic{MFieldCounter}\ ist nicht in einer Kontrollgruppe, es wird niemals eine Loesung angezeigt!}\fi
\special{html: %
<!-- mdeclarepoints;;}\arabic{section}\special{html:;;}#2\special{html:;;}#3\special{html:;;}\arabic{MTestSite}\special{html:;;}\arabic{chapter}%
\special{html:;; //--><!-- onloadstart //-->CreateQuestionObj("}#4\special{html:",}\arabic{MFieldCounter}\special{html:,"}#1\special{html:","}#2\special{html:",2,"IMG}#2%
\special{html:",}#3\special{html:,}\arabic{MTestSite}\special{html:,}\arabic{section}\special{html:);<!-- onloadstop //-->}%
\special{html:<input mfieldtype="2" type="checkbox" name="Name_}#2\special{html:" id="}#2\special{html:" onchange="handlerChange(}\arabic{MFieldCounter}\special{html:,1);"/><img src="images/questionmark.gif" name="}Name_IMG#2%
\special{html:" width="20" height="20" border="0" align="absmiddle" id="}IMG#2\special{html:"/> }%
\else%
\ifnum\value{QBoxFlag}=1\fbox{$\phantom{X}$}\else$\phantom{X}$\fi%
\fi%
}

\def\MGenerateID{QFELD_\arabic{section}.\arabic{subsection}.\arabic{MSiteCounter}.QF\arabic{MFieldCounter}}

% #1 = 0/1 ggf. mit ::smc abgetrennt auszuschliessende single-choice-boxen (UXIDs durch , getrennt ohne UX), #2 = uxid ohne UX
\newcommand{\MCheckbox}[2]{
\MQuestionCheckbox{#1}{\MGenerateID}{\MStdPoints}{#2}
\addtocounter{MFieldCounter}{1}
}

% Erster Parameter: Zeichenlaenge der Eingabebox, zweiter Parameter: Loesungstext
\newcommand{\MQuestion}[2]{
\MQuestionID{#1}{#2}{\MGenerateID}{1}{0}{\MStdPoints}{#2}
\addtocounter{MFieldCounter}{1}
}

% Erster Parameter: Zeichenlaenge der Eingabebox, zweiter Parameter: Loesungstext
\newcommand{\MLQuestion}[3]{
\MQuestionID{#1}{#2}{\MGenerateID}{1}{0}{\MStdPoints}{#3}
\addtocounter{MFieldCounter}{1}
}

% Parameter: Laenge des Feldes, Loesung (wird auch geparsed), Stellen Genauigkeit hinter dem Komma, weitere Stellen werden mathematisch gerundet vor Vergleich
\newcommand{\MParsedQuestion}[3]{
\MQuestionID{#1}{#2}{\MGenerateID}{3}{#3}{\MStdPoints}{#2}
\addtocounter{MFieldCounter}{1}
}

% Parameter: Laenge des Feldes, Loesung (wird auch geparsed), Stellen Genauigkeit hinter dem Komma, weitere Stellen werden mathematisch gerundet vor Vergleich
\newcommand{\MLParsedQuestion}[4]{
\MQuestionID{#1}{#2}{\MGenerateID}{3}{#3}{\MStdPoints}{#4}
\addtocounter{MFieldCounter}{1}
}

% Parameter: Laenge des Feldes, Loesungsfunktion, Anzahl Stuetzstellen, Funktionsvariablen durch Kommata getrennt (nicht case-sensitive), Anzahl Nachkommastellen im Vergleich
\newcommand{\MFunctionQuestion}[5]{
\MQuestionID{#1}{#2}{\MGenerateID}{4}{#3;#4;#5;0}{\MStdPoints}{#2}
\addtocounter{MFieldCounter}{1}
}

% Parameter: Laenge des Feldes, Loesungsfunktion, Anzahl Stuetzstellen, Funktionsvariablen durch Kommata getrennt (nicht case-sensitive), Anzahl Nachkommastellen im Vergleich, UXID
\newcommand{\MLFunctionQuestion}[6]{
\MQuestionID{#1}{#2}{\MGenerateID}{4}{#3;#4;#5;0}{\MStdPoints}{#6}
\addtocounter{MFieldCounter}{1}
}

% Parameter: Laenge des Feldes, Loesungsintervall, Genauigkeit der Zahlenwertpruefung
\newcommand{\MIntervalQuestion}[3]{
\MQuestionID{#1}{#2}{\MGenerateID}{6}{#3}{\MStdPoints}{#2}
\addtocounter{MFieldCounter}{1}
}

% Parameter: Laenge des Feldes, Loesungsintervall, Genauigkeit der Zahlenwertpruefung, UXID
\newcommand{\MLIntervalQuestion}[4]{
\MQuestionID{#1}{#2}{\MGenerateID}{6}{#3}{\MStdPoints}{#4}
\addtocounter{MFieldCounter}{1}
}

% Parameter: Laenge des Feldes, Loesungsfunktion, Anzahl Stuetzstellen, Funktionsvariable (nicht case-sensitive), Anzahl Nachkommastellen im Vergleich, Vereinfachungsbedingung
% Vereinfachungsbedingung ist eine der Folgenden:
% 0 = Keine Vereinfachungsbedingung
% 1 = Keine Klammern (runde oder eckige) mehr im vereinfachten Ausdruck
% 2 = Faktordarstellung (Term hat Produkte als letzte Operation, Summen als vorgeschaltete Operation)
% 3 = Summendarstellung (Term hat Summen als letzte Operation, Produkte als vorgeschaltete Operation)
% Flag 512: Besondere Stuetzstellen (nur >1 und nur schwach rational), sonst symmetrisch um Nullpunkt und ganze Zahlen inkl. Null werden getroffen
\newcommand{\MSimplifyQuestion}[6]{
\MQuestionID{#1}{#2}{\MGenerateID}{4}{#3;#4;#5;#6}{\MStdPoints}{#2}
\addtocounter{MFieldCounter}{1}
}

\newcommand{\MLSimplifyQuestion}[7]{
\MQuestionID{#1}{#2}{\MGenerateID}{4}{#3;#4;#5;#6}{\MStdPoints}{#7}
\addtocounter{MFieldCounter}{1}
}

% Parameter: Laenge des Feldes, Loesung (optionaler Ausdruck), Anzahl Stuetzstellen, Funktionsvariable (nicht case-sensitive), Anzahl Nachkommastellen im Vergleich, Spezialtyp (string-id)
\newcommand{\MLSpecialQuestion}[7]{
\MQuestionID{#1}{#2}{\MGenerateID}{7}{#3;#4;#5;#6}{\MStdPoints}{#7}
\addtocounter{MFieldCounter}{1}
}

\newcounter{MGroupStart}
\newcounter{MGroupEnd}
\newcounter{MGroupActive}

\newenvironment{MQuestionGroup}{
\setcounter{MGroupStart}{\value{MFieldCounter}}
\setcounter{MGroupActive}{1}
}{
\setcounter{MGroupActive}{0}
\setcounter{MGroupEnd}{\value{MFieldCounter}}
\addtocounter{MGroupEnd}{-1}
}

\newcommand{\MGroupButton}[1]{
\ifttm
\special{html:<button name="Name_Group}\arabic{MGroupStart}\special{html:to}\arabic{MGroupEnd}\special{html:" id="Group}\arabic{MGroupStart}\special{html:to}\arabic{MGroupEnd}\special{html:" %
type="button" onclick="group_button(}\arabic{MGroupStart}\special{html:,}\arabic{MGroupEnd}\special{html:);">}#1\special{html:</button>}
\else
\phantom{#1}
\fi
}

%----------------- Makros fuer die modularisierte Darstellung ------------------------------------

\def\MyText#1{#1}

% is used internally by the conversion package, should not be used by original tex documents
\def\MOrgLabel#1{\relax}

\ifttm

% Ein MLabel wird im html codiert durch das tag <!-- mmlabel;;Labelbezeichner;;SubjectArea;;chapter;;section;;subsection;;Index;;Objekttyp; //-->
\def\MLabel#1{%
\ifnum\value{MLastType}=8%
\ifnum\value{MCaptionOn}=0%
\MDebugMessage{ERROR: Grafik \arabic{MGraphicsCounter} hat separates label: #1 (Grafiklabels sollten nur in der Caption stehen)}%
\fi
\fi
\ifnum\value{MLastType}=12%
\ifnum\value{MCaptionOn}=0%
\MDebugMessage{ERROR: Video \arabic{MVideoCounter} hat separates label: #1 (Videolabels sollten nur in der Caption stehen}%
\fi
\fi
\ifnum\value{MLastType}=10\setcounter{MLastIndex}{\value{equation}}\fi
\label{#1}\begin{html}<!-- mmlabel;;#1;;\end{html}\arabic{MSubjectArea}\special{html:;;}\arabic{chapter}\special{html:;;}\arabic{section}\special{html:;;}\arabic{subsection}\special{html:;;}\arabic{MLastIndex}\special{html:;;}\arabic{MLastType}\special{html:; //-->}}%

\else

% Sonderbehandlung im PDF fuer Abbildungen in separater aux-Datei, da MGraphics die figure-Umgebung nicht verwendet
\def\MLabel#1{%
\ifnum\value{MLastType}=8%
\ifnum\value{MCaptionOn}=0%
\MDebugMessage{ERROR: Grafik \arabic{MGraphicsCounter} hat separates label: #1 (Grafiklabels sollten nur in der Caption stehen}%
\fi
\fi
\ifnum\value{MLastType}=12%
\ifnum\value{MCaptionOn}=0%
\MDebugMessage{ERROR: Video \arabic{MVideoCounter} hat separates label: #1 (Videolabels sollten nur in der Caption stehen}%
\fi
\fi
\label{#1}%
}%

\fi

% Gibt Begriff des referenzierten Objekts mit aus, aber nur im HTML, daher nur in Ausnahmefaellen (z.B. Copyrightliste) sinnvoll
\def\MCRef#1{\ifttm\special{html:<!-- mmref;;}#1\special{html:;;1; //-->}\else\vref{#1}\fi}


\def\MRef#1{\ifttm\special{html:<!-- mmref;;}#1\special{html:;;0; //-->}\else\vref{#1}\fi}
\def\MERef#1{\ifttm\special{html:<!-- mmref;;}#1\special{html:;;0; //-->}\else\eqref{#1}\fi}
\def\MNRef#1{\ifttm\special{html:<!-- mmref;;}#1\special{html:;;0; //-->}\else\ref{#1}\fi}
\def\MSRef#1#2{\ifttm\special{html:<!-- msref;;}#1\special{html:;;}#2\special{html:; //-->}\else \if#2\empty \ref{#1} \else \hyperref[#1]{#2}\fi\fi} 

\def\MRefRange#1#2{\ifttm\MRef{#1} bis 
\MRef{#2}\else\vrefrange[\unskip]{#1}{#2}\fi}

\def\MRefTwo#1#2{\ifttm\MRef{#1} und \MRef{#2}\else%
\let\vRefTLRsav=\reftextlabelrange\let\vRefTPRsav=\reftextpagerange%
\def\reftextlabelrange##1##2{\ref{##1} und~\ref{##2}}%
\def\reftextpagerange##1##2{auf den Seiten~\pageref{#1} und~\pageref{#2}}%
\vrefrange[\unskip]{#1}{#2}%
\let\reftextlabelrange=\vRefTLRsav\let\reftextpagerange=\vRefTPRsav\fi}

% MSectionChapter definiert falls notwendig das Kapitel vor der section. Das ist notwendig, wenn nur ein Einzelmodul uebersetzt wird.
% MChaptersGiven ist ein Counter, der von mconvert.pl vordefiniert wird.
\ifttm
\newcommand{\MSectionChapter}{\ifnum\value{MChaptersGiven}=0{\Dchapter{Modul}}\else{}\fi}
\else
\newcommand{\MSectionChapter}{\ifnum\value{chapter}=0{\Dchapter{Modul}}\else{}\fi}
\fi


\def\MChapter#1{\ifnum\value{MSSEnd}>0{\MSubsectionEndMacros}\addtocounter{MSSEnd}{-1}\fi\Dchapter{#1}}
\def\MSubject#1{\MChapter{#1}} % Schluesselwort HELPSECTION ist reserviert fuer Hilfesektion

\newcommand{\MSectionID}{UNKNOWNID}

\ifttm
\newcommand{\MSetSectionID}[1]{\renewcommand{\MSectionID}{#1}}
\else
\newcommand{\MSetSectionID}[1]{\renewcommand{\MSectionID}{#1}\tikzsetexternalprefix{#1}}
\fi


\newcommand{\MSection}[1]{\MSetSectionID{MODULID}\ifnum\value{MSSEnd}>0{\MSubsectionEndMacros}\addtocounter{MSSEnd}{-1}\fi\MSectionChapter\Dsection{#1}\MSectionStartMacros{#1}\setcounter{MLastIndex}{-1}\setcounter{MLastType}{1}} % Sections werden ueber das section-Feld im mmlabel-Tag identifiziert, nicht ueber das Indexfeld

\def\MSubsection#1{\ifnum\value{MSSEnd}>0{\MSubsectionEndMacros}\addtocounter{MSSEnd}{-1}\fi\ifttm\else\clearpage\fi\Dsubsection{#1}\MSubsectionStartMacros\setcounter{MLastIndex}{-1}\setcounter{MLastType}{2}\addtocounter{MSSEnd}{1}}% Subsections werden ueber das subsection-Feld im mmlabel-Tag identifiziert, nicht ueber das Indexfeld
\def\MSubsectionx#1{\Dsubsectionx{#1}} % Nur zur Verwendung in MSectionStart gedacht
\def\MSubsubsection#1{\Dsubsubsection{#1}\setcounter{MLastIndex}{\value{subsubsection}}\setcounter{MLastType}{3}\ifttm\special{html:<!-- sectioninfo;;}\arabic{section}\special{html:;;}\arabic{subsection}\special{html:;;}\arabic{subsubsection}\special{html:;;1;;}\arabic{MTestSite}\special{html:; //-->}\fi}
\def\MSubsubsectionx#1{\Dsubsubsectionx{#1}\ifttm\special{html:<!-- sectioninfo;;}\arabic{section}\special{html:;;}\arabic{subsection}\special{html:;;}\arabic{subsubsection}\special{html:;;0;;}\arabic{MTestSite}\special{html:; //-->}\else\addcontentsline{toc}{subsection}{#1}\fi}

\ifttm
\def\MSubsubsubsectionx#1{\ \newline\textbf{#1}\special{html:<br />}}
\else
\def\MSubsubsubsectionx#1{\ \newline
\textbf{#1}\ \\
}
\fi


% Dieses Skript wird zu Beginn jedes Modulabschnitts (=Webseite) ausgefuehrt und initialisiert den Aufgabenfeldzaehler
\newcommand{\MPageScripts}{
\setcounter{MFieldCounter}{1}
\addtocounter{MSiteCounter}{1}
\setcounter{MHintCounter}{1}
\setcounter{MCodeEditCounter}{1}
\setcounter{MGroupActive}{0}
\DoQBoxes
% Feldvariablen werden im HTML-Header in conv.pl eingestellt
}

% Dieses Skript wird zum Ende jedes Modulabschnitts (=Webseite) ausgefuehrt
\ifttm
\newcommand{\MEndScripts}{\special{html:<br /><!-- mfeedbackbutton;Seite;}\arabic{MTestSite}\special{html:;}\MGenerateSiteNumber\special{html:; //-->}
}
\else
\newcommand{\MEndScripts}{\relax}
\fi


\newcounter{QBoxFlag}
\newcommand{\DoQBoxes}{\setcounter{QBoxFlag}{1}}
\newcommand{\NoQBoxes}{\setcounter{QBoxFlag}{0}}

\newcounter{MXCTest}
\newcounter{MXCounter}
\newcounter{MSCounter}



\ifttm

% Struktur des sectioninfo-Tags: <!-- sectioninfo;;section;;subsection;;subsubsection;;nr_ausgeben;;testpage; //-->

%Fuegt eine zusaetzliche html-Seite an hinter ALLEN bisherigen und zukuenftigen content-Seiten ausserhalb der vor-zurueck-Schleife (d.h. nur durch Button oder MIntLink erreichbar!)
% #1 = Titel des Modulabschnitts, #2 = Kurztitel fuer die Buttons, #3 = Buttonkennung (STD = default nehmen, NONE = Ohne Button in der Navigation)
\newenvironment{MSContent}[3]{\special{html:<div class="xcontent}\arabic{MSCounter}\special{html:"><!-- scontent;-;}\arabic{MSCounter};-;#1;-;#2;-;#3\special{html: //-->}\MPageScripts\MSubsubsectionx{#1}}{\MEndScripts\special{html:<!-- endscontent;;}\arabic{MSCounter}\special{html: //--></div>}\addtocounter{MSCounter}{1}}

% Fuegt eine zusaetzliche html-Seite ein hinter den bereits vorhandenen content-Seiten (oder als erste Seite) innerhalb der vor-zurueck-Schleife der Navigation
% #1 = Titel des Modulabschnitts, #2 = Kurztitel fuer die Buttons, #3 = Buttonkennung (STD = Defaultbutton, NONE = Ohne Button in der Navigation)
\newenvironment{MXContent}[3]{\special{html:<div class="xcontent}\arabic{MXCounter}\special{html:"><!-- xcontent;-;}\arabic{MXCounter};-;#1;-;#2;-;#3\special{html: //-->}\MPageScripts\MSubsubsection{#1}}{\MEndScripts\special{html:<!-- endxcontent;;}\arabic{MXCounter}\special{html: //--></div>}\addtocounter{MXCounter}{1}}

% Fuegt eine zusaetzliche html-Seite ein die keine subsubsection-Nummer bekommt, nur zur internen Verwendung in mintmod.tex gedacht!
% #1 = Titel des Modulabschnitts, #2 = Kurztitel fuer die Buttons, #3 = Buttonkennung (STD = Defaultbutton, NONE = Ohne Button in der Navigation)
% \newenvironment{MUContent}[3]{\special{html:<div class="xcontent}\arabic{MXCounter}\special{html:"><!-- xcontent;-;}\arabic{MXCounter};-;#1;-;#2;-;#3\special{html: //-->}\MPageScripts\MSubsubsectionx{#1}}{\MEndScripts\special{html:<!-- endxcontent;;}\arabic{MXCounter}\special{html: //--></div>}\addtocounter{MXCounter}{1}}

\newcommand{\MDeclareSiteUXID}[1]{\special{html:<!-- mdeclaresiteuxid;;}#1\special{html:;;}\arabic{chapter}\special{html:;;}\arabic{section}\special{html:;; //-->}}

\else

%\newcommand{\MSubsubsection}[1]{\refstepcounter{subsubsection} \addcontentsline{toc}{subsubsection}{\thesubsubsection. #1}}


% Fuegt eine zusaetzliche html-Seite an hinter den bereits vorhandenen content-Seiten
% #1 = Titel des Modulabschnitts, #2 = Kurztitel fuer die Buttons, #3 = Iconkennung (im PDF wirkungslos)
%\newenvironment{MUContent}[3]{\ifnum\value{MXCTest}>0{\MDebugMessage{ERROR: Geschachtelter SContent}}\fi\MPageScripts\MSubsubsectionx{#1}\addtocounter{MXCTest}{1}}{\addtocounter{MXCounter}{1}\addtocounter{MXCTest}{-1}}
\newenvironment{MXContent}[3]{\ifnum\value{MXCTest}>0{\MDebugMessage{ERROR: Geschachtelter SContent}}\fi\MPageScripts\MSubsubsection{#1}\addtocounter{MXCTest}{1}}{\addtocounter{MXCounter}{1}\addtocounter{MXCTest}{-1}}
\newenvironment{MSContent}[3]{\ifnum\value{MXCTest}>0{\MDebugMessage{ERROR: Geschachtelter XContent}}\fi\MPageScripts\MSubsubsectionx{#1}\addtocounter{MXCTest}{1}}{\addtocounter{MSCounter}{1}\addtocounter{MXCTest}{-1}}

\newcommand{\MDeclareSiteUXID}[1]{\relax}

\fi 

% GHEADER und GFOOTER werden von split.pm gefunden, aber nur, wenn nicht HELPSITE oder TESTSITE
\ifttm
\newenvironment{MSectionStart}{\special{html:<div class="xcontent0">}\MSubsubsectionx{Modul\"ubersicht}}{\setcounter{MSSEnd}{0}\special{html:</div>}}
% Darf nicht als XContent nummeriert werden, darf nicht als XContent gelabelt werden, wird aber in eine xcontent-div gesetzt fuer Python-parsing
\else
\newenvironment{MSectionStart}{\MSubsectionx{Modul\"ubersicht}}{\setcounter{MSSEnd}{0}}
\fi

\newenvironment{MIntro}{\begin{MXContent}{Einf\"uhrung}{Einf\"uhrung}{genetisch}}{\end{MXContent}}
\newenvironment{MContent}{\begin{MXContent}{Inhalt}{Inhalt}{beweis}}{\end{MXContent}}
\newenvironment{MExercises}{\ifttm\else\clearpage\fi\begin{MXContent}{Aufgaben}{Aufgaben}{aufgb}\special{html:<!-- declareexcsymb //-->}}{\end{MXContent}}

% #1 = Lesbare Testbezeichnung
\newenvironment{MTest}[1]{%
\renewcommand{\MTestName}{#1}
\ifttm\else\clearpage\fi%
\addtocounter{MTestSite}{1}%
\begin{MXContent}{#1}{#1}{STD} % {aufgb}%
\special{html:<!-- declaretestsymb //-->}
\begin{MQuestionGroup}%
\MInTestHeader
}%
{%
\end{MQuestionGroup}%
\ \\ \ \\%
\MInTestFooter
\end{MXContent}\addtocounter{MTestSite}{-1}%
}

\newenvironment{MExtra}{\ifttm\else\clearpage\fi\begin{MXContent}{Zus\"atzliche Inhalte}{Zusatz}{weiterfhrg}}{\end{MXContent}}

\makeindex

\ifttm
\def\MPrintIndex{
\ifnum\value{MSSEnd}>0{\MSubsectionEndMacros}\addtocounter{MSSEnd}{-1}\fi
\renewcommand{\indexname}{Stichwortverzeichnis}
\special{html:<p><!-- printindex //--></p>}
}
\else
\def\MPrintIndex{
\ifnum\value{MSSEnd}>0{\MSubsectionEndMacros}\addtocounter{MSSEnd}{-1}\fi
\renewcommand{\indexname}{Stichwortverzeichnis}
\addcontentsline{toc}{section}{Stichwortverzeichnis}
\printindex
}
\fi


% Konstanten fuer die Modulfaecher

\def\MINTMathematics{1}
\def\MINTInformatics{2}
\def\MINTChemistry{3}
\def\MINTPhysics{4}
\def\MINTEngineering{5}

\newcounter{MSubjectArea}
\newcounter{MInfoNumbers} % Gibt an, ob die Infoboxen nummeriert werden sollen
\newcounter{MSepNumbers} % Gibt an, ob Beispiele und Experimente separat nummeriert werden sollen
\newcommand{\MSetSubject}[1]{
 % ttm kapiert setcounter mit Parametern nicht, also per if abragen und einsetzen
\ifnum#1=1\setcounter{MSubjectArea}{1}\setcounter{MInfoNumbers}{1}\setcounter{MSepNumbers}{0}\fi
\ifnum#1=2\setcounter{MSubjectArea}{2}\setcounter{MInfoNumbers}{1}\setcounter{MSepNumbers}{0}\fi
\ifnum#1=3\setcounter{MSubjectArea}{3}\setcounter{MInfoNumbers}{0}\setcounter{MSepNumbers}{1}\fi
\ifnum#1=4\setcounter{MSubjectArea}{4}\setcounter{MInfoNumbers}{0}\setcounter{MSepNumbers}{0}\fi
\ifnum#1=5\setcounter{MSubjectArea}{5}\setcounter{MInfoNumbers}{1}\setcounter{MSepNumbers}{0}\fi
% Separate Nummerntechnik fuer unsere Chemiker: alles dreistellig
\ifnum#1=3
  \ifttm
  \renewcommand{\theequation}{\arabic{section}.\arabic{subsection}.\arabic{equation}}
  \renewcommand{\thetable}{\arabic{section}.\arabic{subsection}.\arabic{table}} 
  \renewcommand{\thefigure}{\arabic{section}.\arabic{subsection}.\arabic{figure}} 
  \else
  \renewcommand{\theequation}{\arabic{chapter}.\arabic{section}.\arabic{equation}}
  \renewcommand{\thetable}{\arabic{chapter}.\arabic{section}.\arabic{table}}
  \renewcommand{\thefigure}{\arabic{chapter}.\arabic{section}.\arabic{figure}}
  \fi
\else
  \ifttm
  \renewcommand{\theequation}{\arabic{section}.\arabic{subsection}.\arabic{equation}}
  \renewcommand{\thetable}{\arabic{table}}
  \renewcommand{\thefigure}{\arabic{figure}}
  \else
  \renewcommand{\theequation}{\arabic{chapter}.\arabic{section}.\arabic{equation}}
  \renewcommand{\thetable}{\arabic{table}}
  \renewcommand{\thefigure}{\arabic{figure}}
  \fi
\fi
}

% Fuer tikz Autogenerierung
\newcounter{MTIKZAutofilenumber}

% Spezielle Counter fuer die Bentz-Module
\newcounter{mycounter}
\newcounter{chemapplet}
\newcounter{physapplet}

\newcounter{MSSEnd} % Ist 1 falls ein MSubsection aktiv ist, der einen MSubsectionEndMacro-Aufruf verursacht
\newcounter{MFileNumber}
\def\MLastFile{\special{html:[[!-- mfileref;;}\arabic{MFileNumber}\special{html:; //--]]}}

% Vollstaendiger Pfad ist \MMaterial / \MLastFilePath / \MLastFileName    ==   \MMaterial / \MLastFile

% Wird nur bei kompletter Baum-Erstellung ausgefuehrt!
% #1 = Lesbare Modulbezeichnung
\newcommand{\MSectionStartMacros}[1]{
\setcounter{MTestSite}{0}
\setcounter{MCaptionOn}{0}
\setcounter{MLastTypeEq}{0}
\setcounter{MSSEnd}{0}
\setcounter{MFileNumber}{0} % Preinkrekement-Counter
\setcounter{MTIKZAutofilenumber}{0}
\setcounter{mycounter}{1}
\setcounter{physapplet}{1}
\setcounter{chemapplet}{0}
\ifttm
\special{html:<!-- mdeclaresection;;}\arabic{chapter}\special{html:;;}\arabic{section}\special{html:;;}#1\special{html:;; //-->}%
\else
\setcounter{thmc}{0}
\setcounter{exmpc}{0}
\setcounter{verc}{0}
\setcounter{infoc}{0}
\fi
\setcounter{MiniMarkerCounter}{1}
\setcounter{AlignCounter}{1}
\setcounter{MXCTest}{0}
\setcounter{MCodeCounter}{0}
\setcounter{MEntryCounter}{0}
}

% Wird immer ausgefuehrt
\newcommand{\MSubsectionStartMacros}{
\ifttm\else\MPageHeaderDef\fi
\MWatermarkSettings
\setcounter{MXCounter}{0}
\setcounter{MSCounter}{0}
\setcounter{MSiteCounter}{1}
\setcounter{MExerciseCollectionCounter}{0}
% Zaehler fuer das Labelsystem zuruecksetzen (prefix-Zaehler)
\setcounter{MInfoCounter}{0}
\setcounter{MExerciseCounter}{0}
\setcounter{MExampleCounter}{0}
\setcounter{MExperimentCounter}{0}
\setcounter{MGraphicsCounter}{0}
\setcounter{MTableCounter}{0}
\setcounter{MTheoremCounter}{0}
\setcounter{MObjectCounter}{0}
\setcounter{MEquationCounter}{0}
\setcounter{MVideoCounter}{0}
\setcounter{equation}{0}
\setcounter{figure}{0}
}

\newcommand{\MSubsectionEndMacros}{
% Bei Chemiemodulen das PSE einhaengen, es soll als SContent am Ende erscheinen
\special{html:<!-- subsectionend //-->}
\ifnum\value{MSubjectArea}=3{\MIncludePSE}\fi
}


\ifttm
%\newcommand{\MEmbed}[1]{\MRegisterFile{#1}\begin{html}<embed src="\end{html}\MMaterial/\MLastFile\begin{html}" width="192" height="189"></embed>\end{html}}
\newcommand{\MEmbed}[1]{\MRegisterFile{#1}\begin{html}<embed src="\end{html}\MMaterial/\MLastFile\begin{html}"></embed>\end{html}}
\fi

%----------------- Makros fuer die Textdarstellung -----------------------------------------------

\ifttm
% MUGraphics bindet eine Grafik ein:
% Parameter 1: Dateiname der Grafik, relativ zur Position des Modul-Tex-Dokuments
% Parameter 2: Skalierungsoptionen fuer PDF (fuer includegraphics)
% Parameter 3: Titel fuer die Grafik, wird unter die Grafik mit der Grafiknummer gesetzt und kann MLabel bzw. MCopyrightLabel enthalten
% Parameter 4: Skalierungsoptionen fuer HTML (css-styles)

% ERSATZ: <img alt="My Image" src="data:image/png;base64,iVBORwA<MoreBase64SringHere>" />


\newcommand{\MUGraphics}[4]{\MRegisterFile{#1}\begin{html}
<div class="imagecenter">
<center>
<div>
<img src="\end{html}\MMaterial/\MLastFile\begin{html}" style="#4" alt="\end{html}\MMaterial/\MLastFile\begin{html}"/>
</div>
<div class="bildtext">
\end{html}
\addtocounter{MGraphicsCounter}{1}
\setcounter{MLastIndex}{\value{MGraphicsCounter}}
\setcounter{MLastType}{8}
\addtocounter{MCaptionOn}{1}
\ifnum\value{MSepNumbers}=0
\textbf{Abbildung \arabic{MGraphicsCounter}:} #3
\else
\textbf{Abbildung \arabic{section}.\arabic{subsection}.\arabic{MGraphicsCounter}:} #3
\fi
\addtocounter{MCaptionOn}{-1}
\begin{html}
</div>
</center>
</div>
<br />
\end{html}%
\special{html:<!-- mfeedbackbutton;Abbildung;}\arabic{MGraphicsCounter}\special{html:;}\arabic{section}.\arabic{subsection}.\arabic{MGraphicsCounter}\special{html:; //-->}%
}

% MVideo bindet ein Video als Einzeldatei ein:
% Parameter 1: Dateiname des Videos, relativ zur Position des Modul-Tex-Dokuments, ohne die Endung ".mp4"
% Parameter 2: Titel fuer das Video (kann MLabel oder MCopyrightLabel enthalten), wird unter das Video mit der Videonummer gesetzt
\newcommand{\MVideo}[2]{\MRegisterFile{#1.mp4}\begin{html}
<div class="imagecenter">
<center>
<div>
<video width="95\%" controls="controls"><source src="\end{html}\MMaterial/#1.mp4\begin{html}" type="video/mp4">Ihr Browser kann keine MP4-Videos abspielen!</video>
</div>
<div class="bildtext">
\end{html}
\addtocounter{MVideoCounter}{1}
\setcounter{MLastIndex}{\value{MVideoCounter}}
\setcounter{MLastType}{12}
\addtocounter{MCaptionOn}{1}
\ifnum\value{MSepNumbers}=0
\textbf{Video \arabic{MVideoCounter}:} #2
\else
\textbf{Video \arabic{section}.\arabic{subsection}.\arabic{MVideoCounter}:} #2
\fi
\addtocounter{MCaptionOn}{-1}
\begin{html}
</div>
</center>
</div>
<br />
\end{html}}

\newcommand{\MDVideo}[2]{\MRegisterFile{#1.mp4}\MRegisterFile{#1.ogv}\begin{html}
<div class="imagecenter">
<center>
<div>
<video width="70\%" controls><source src="\end{html}\MMaterial/#1.mp4\begin{html}" type="video/mp4"><source src="\end{html}\MMaterial/#1.ogv\begin{html}" type="video/ogg">Ihr Browser kann keine MP4-Videos abspielen!</video>
</div>
<br />
#2
</center>
</div>
<br />
\end{html}
}

\newcommand{\MGraphics}[3]{\MUGraphics{#1}{#2}{#3}{}}

\else

\newcommand{\MVideo}[2]{%
% Kein Video im PDF darstellbar, trotzdem so tun als ob da eines waere
\begin{center}
(Video nicht darstellbar)
\end{center}
\addtocounter{MVideoCounter}{1}
\setcounter{MLastIndex}{\value{MVideoCounter}}
\setcounter{MLastType}{12}
\addtocounter{MCaptionOn}{1}
\ifnum\value{MSepNumbers}=0
\textbf{Video \arabic{MVideoCounter}:} #2
\else
\textbf{Video \arabic{section}.\arabic{subsection}.\arabic{MVideoCounter}:} #2
\fi
\addtocounter{MCaptionOn}{-1}
}


% MGraphics bindet eine Grafik ein:
% Parameter 1: Dateiname der Grafik, relativ zur Position des Modul-Tex-Dokuments
% Parameter 2: Skalierungsoptionen fuer PDF (fuer includegraphics)
% Parameter 3: Titel fuer die Grafik, wird unter die Grafik mit der Grafiknummer gesetzt
\newcommand{\MGraphics}[3]{%
\MRegisterFile{#1}%
\ %
\begin{figure}[H]%
\centering{%
\includegraphics[#2]{\MDPrefix/#1}%
\addtocounter{MCaptionOn}{1}%
\caption{#3}%
\addtocounter{MCaptionOn}{-1}%
}%
\end{figure}%
\addtocounter{MGraphicsCounter}{1}\setcounter{MLastIndex}{\value{MGraphicsCounter}}\setcounter{MLastType}{8}\ %
%\ \\Abbildung \ifnum\value{MSepNumbers}=0\else\arabic{chapter}.\arabic{section}.\fi\arabic{MGraphicsCounter}: #3%
}

\newcommand{\MUGraphics}[4]{\MGraphics{#1}{#2}{#3}}


\fi

\newcounter{MCaptionOn} % = 1 falls eine Grafikcaption aktiv ist, = 0 sonst


% MGraphicsSolo bindet eine Grafik pur ein ohne Titel
% Parameter 1: Dateiname der Grafik, relativ zur Position des Modul-Tex-Dokuments
% Parameter 2: Skalierungsoptionen (wirken nur im PDF)
\newcommand{\MGraphicsSolo}[2]{\MUGraphicsSolo{#1}{#2}{}}

% MUGraphicsSolo bindet eine Grafik pur ein ohne Titel, aber mit HTML-Skalierung
% Parameter 1: Dateiname der Grafik, relativ zur Position des Modul-Tex-Dokuments
% Parameter 2: Skalierungsoptionen (wirken nur im PDF)
% Parameter 3: Skalierungsoptionen (wirken nur im HTML), als style-format: "width=???, height=???"
\ifttm
\newcommand{\MUGraphicsSolo}[3]{\MRegisterFile{#1}\begin{html}
<img src="\end{html}\MMaterial/\MLastFile\begin{html}" style="\end{html}#3\begin{html}" alt="\end{html}\MMaterial/\MLastFile\begin{html}"/>
\end{html}%
\special{html:<!-- mfeedbackbutton;Abbildung;}#1\special{html:;}\MMaterial/\MLastFile\special{html:; //-->}%
}
\else
\newcommand{\MUGraphicsSolo}[3]{\MRegisterFile{#1}\includegraphics[#2]{\MDPrefix/#1}}
\fi

% Externer Link mit URL
% Erster Parameter: Vollstaendige(!) URL des Links
% Zweiter Parameter: Text fuer den Link
\newcommand{\MExtLink}[2]{\ifttm\special{html:<a target="_new" href="}#1\special{html:">}#2\special{html:</a>}\else\href{#1}{#2}\fi} % ohne MINTERLINK!


% Interner Link, die verlinkte Datei muss im gleichen Verzeichnis liegen wie die Modul-Texdatei
% Erster Parameter: Dateiname
% Zweiter Parameter: Text fuer den Link
\newcommand{\MIntLink}[2]{\ifttm\MRegisterFile{#1}\special{html:<a class="MINTERLINK" target="_new" href="}\MMaterial/\MLastFile\special{html:">}#2\special{html:</a>}\else{\href{#1}{#2}}\fi}


\ifttm
\def\MMaterial{:localmaterial:}
\else
\def\MMaterial{\MDPrefix}
\fi

\ifttm
\def\MNoFile#1{:directmaterial:#1}
\else
\def\MNoFile#1{#1}
\fi

\newcommand{\MChem}[1]{$\mathrm{#1}$}

\newcommand{\MApplet}[3]{
% Bindet ein Java-Applet ein, die Parameter sind:
% (wird nur im HTML, aber nicht im PDF erstellt)
% #1 Dateiname des Applets (muss mit ".class" enden)
% #2 = Breite in Pixeln
% #3 = Hoehe in Pixeln
\ifttm
\MRegisterFile{#1}
\begin{html}
<applet code="\end{html}\MMaterial/\MLastFile\begin{html}" width="#2" height="#3" alt="[Java-Applet kann nicht gestartet werden]"></applet>
\end{html}
\fi
}

\newcommand{\MScriptPage}[2]{
% Bindet eine JavaScript-Datei ein, die eine eigene Seite bekommt
% (wird nur im HTML, aber nicht im PDF erstellt)
% #1 Dateiname des Programms (sollte mit ".js" enden)
% #2 = Kurztitel der Seite
\ifttm
\begin{MSContent}{#2}{#2}{puzzle}
\MRegisterFile{#1}
\begin{html}
<script src="\MMaterial/\MLastFile" type="text/javascript"></script>
\end{html}
\end{MSContent}
\fi
}

\newcommand{\MIncludePSE}{
% Bindet bei Chemie-Modulen das PSE ein
% (wird nur im HTML, aber nicht im PDF erstellt)
\ifttm
\special{html:<!-- includepse //-->}
\begin{MSContent}{Periodensystem der Elemente}{PSE}{table}
\MRegisterFile{../files/pse.js}
\MRegisterFile{../files/radio.png}
\begin{html}
<script src="\MMaterial/../files/pse.js" type="text/javascript"></script>
<p id="divid"><br /><br />
<script language="javascript" type="text/javascript">
    startpse("divid","\MMaterial/../files"); 
</script>
</p>
<br />
<br />
<br />
<p>Die Farben der Elementsymbole geben an: <font style="color:Red">gasf&ouml;rmig </font> <font style="color:Blue">fl&uuml;ssig </font> fest</p>
<p>Die Elemente der Gruppe 1 A, 2 A, 3 A usw. geh&ouml;ren zu den Hauptgruppenelementen.</p>
<p>Die Elemente der Gruppe 1 B, 2 B, 3 B usw. geh&ouml;ren zu den Nebengruppenelementen.</p>
<p>() kennzeichnet die Masse des stabilsten Isotops</p>
\end{html}
\end{MSContent}
\fi
}

\newcommand{\MAppletArchive}[4]{
% Bindet ein Java-Applet ein, die Parameter sind:
% (wird nur im HTML, aber nicht im PDF erstellt)
% #1 Dateiname der Klasse mit Appletaufruf (muss mit ".class" enden)
% #2 Dateiname des Archivs (muss mit ".jar" enden)
% #3 = Breite in Pixeln
% #4 = Hoehe in Pixeln
\ifttm
\MRegisterFile{#2}
\begin{html}
<applet code="#1" archive="\end{html}\MMaterial/\MLastFile\begin{html}" codebase="." width="#3" height="#4" alt="[Java-Archiv kann nicht gestartet werden]"></applet>
\end{html}
\fi
}

% Bindet in der Haupttexdatei ein MINT-Modul ein. Parameter 1 ist das Verzeichnis (relativ zur Haupttexdatei), Parameter 2 ist der Dateinahme ohne Pfad.
\newcommand{\IncludeModule}[2]{
\renewcommand{\MDPrefix}{#1}
\input{#1/#2}
\ifnum\value{MSSEnd}>0{\MSubsectionEndMacros}\addtocounter{MSSEnd}{-1}\fi
}

% Der ttm-Konverter setzt keine Makros im \input um, also muss hier getrickst werden:
% Das MDPrefix muss in den einzelnen Modulen manuell eingesetzt werden
\newcommand{\MInputFile}[1]{
\ifttm
\input{#1}
\else
\input{#1}
\fi
}


\newcommand{\MCases}[1]{\left\lbrace{\begin{array}{rl} #1 \end{array}}\right.}

\ifttm
\newenvironment{MCaseEnv}{\left\lbrace\begin{array}{rl}}{\end{array}\right.}
\else
\newenvironment{MCaseEnv}{\left\lbrace\begin{array}{rl}}{\end{array}\right.}
\fi

\def\MSkip{\ifttm\MCR\fi}

\ifttm
\def\MCR{\special{html:<br />}}
\else
\def\MCR{\ \\}
\fi


% Pragmas - Sind Schluesselwoerter, die dem Preprocessing sowie dem Konverter uebergeben werden und bestimmte
%           Aktionen ausloesen. Im Output (PDF und HTML) tauchen sie nicht auf.
\newcommand{\MPragma}[1]{%
\ifttm%
\special{html:<!-- mpragma;-;}#1\special{html:;; -->}%
\else%
% MPragmas werden vom Preprozessor direkt im LaTeX gefunden
\fi%
}

% Ersatz der Befehle textsubscript und textsuperscript, die ttm nicht kennt
\ifttm%
\newcommand{\MTextsubscript}[1]{\special{html:<sub>}#1\special{html:</sub>}}%
\newcommand{\MTextsuperscript}[1]{\special{html:<sup>}#1\special{html:</sup>}}%
\else%
\newcommand{\MTextsubscript}[1]{\textsubscript{#1}}%
\newcommand{\MTextsuperscript}[1]{\textsuperscript{#1}}%
\fi

%------------------ Einbindung von dia-Diagrammen ----------------------------------------------
% Beim preprocessing wird aus jeder dia-Datei eine tex-Datei und eine pdf-Datei erzeugt,
% diese werden hier jeweils im PDF und HTML eingebunden
% Parameter: Dateiname der mit dia erstellten Datei (OHNE die Endung .dia)
\ifttm%
\newcommand{\MDia}[1]{%
\MGraphicsSolo{#1minthtml.png}{}%
}
\else%
\newcommand{\MDia}[1]{%
\MGraphicsSolo{#1mintpdf.png}{scale=0.1667}%
}
\fi%

% subsup funktioniert im Ausdruck $D={\R}^+_0$, also \R geklammert und sup zuerst
% \ifttm
% \def\MSubsup#1#2#3{\special{html:<msubsup>} #1 #2 #3\special{html:</msubsup>}}
% \else
% \def\MSubsup#1#2#3{{#1}^{#3}_{#2}}
% \fi

%\input{local.tex}

% \ifttm
% \else
% \newwrite\mintlog
% \immediate\openout\mintlog=mintlog.txt
% \fi

% ----------------------- tikz autogenerator -------------------------------------------------------------------

\newcommand{\Mtikzexternalize}{\tikzexternalize}% wird bei Konvertierung ueber mconvert ggf. ausgehebelt!

\ifttm
\else
\tikzset%
{
  % Defines a custom style which generates pdf and converts to (low and hi-res quality) png and svg, then deletes the pdf
  % Important: DO NOT directly convert from pdf to hires-png or from svg to png with GraphViz convert as it has some problems and memory leaks
  png export/.style=%
  {
    external/system call/.add={}{; 
      pdf2svg "\image.pdf" "\image.svg" ; 
      convert -density 112.5 -transparent white "\image.pdf" "\image.png"; 
      inkscape --export-png="\image.4x.png" --export-dpi=450 --export-background-opacity=0 --without-gui "\image.svg"; 
      rm "\image.pdf"; rm "\image.log"; rm "\image.dpth"; rm "\image.idx"
    },
    external/force remake,
  }
}
\tikzset{png export}
\tikzsetexternalprefix{}
% PNGs bei externer Erzeugung in "richtiger" Groesse einbinden
\pgfkeys{/pgf/images/include external/.code={\includegraphics[scale=0.64]{#1}}}
\fi

% Spezielle Umgebung fuer Autogenerierung, Bildernamen sind nur innerhalb eines Moduls (einer MSection) eindeutig)

\newcommand{\MTIKZautofilename}{tikzautofile}

\ifttm
% HTML-Version: Vom Autogenerator erzeugte png-Datei einbinden, tikz selbst nicht ausfuehren (sprich: #1 schlucken)
\newcommand{\MTikzAuto}[1]{%
\addtocounter{MTIKZAutofilenumber}{1}
\renewcommand{\MTIKZautofilename}{mtikzauto_\arabic{MTIKZAutofilenumber}}
\MUGraphicsSolo{\MSectionID\MTIKZautofilename.4x.png}{scale=1}{\special{html:[[!-- svgstyle;}\MSectionID\MTIKZautofilename\special{html: //--]]}} % Styleinfos werden aus original-png, nicht 4x-png geholt!
%\MRegisterFile{\MSectionID\MTIKZautofilename.png} % not used right now
%\MRegisterFile{\MSectionID\MTIKZautofilename.svg}
}
\else%
% PDF-Version: Falls Autogenerator aktiv wird Datei automatisch benannt und exportiert
\newcommand{\MTikzAuto}[1]{%
\addtocounter{MTIKZAutofilenumber}{1}%
\renewcommand{\MTIKZautofilename}{mtikzauto_\arabic{MTIKZAutofilenumber}}
\tikzsetnextfilename{\MTIKZautofilename}%
#1%
}
\fi

% In einer reinen LaTeX-Uebersetzung kapselt der Preambelinclude-Befehl nur input,
% in einer konvertergesteuerten PDF/HTML-Uebersetzung wird er dagegen entfernt und
% die Preambeln an mintmod angehaengt, die Ersetzung wird von mconvert.pl vorgenommen.

\newcommand{\MPreambleInclude}[1]{\input{#1}}

% Globale Watermarksettings (werden auch nochmal zu Beginn jedes subsection gesetzt,
% muessen hier aber auch global ausgefuehrt wegen Einfuehrungsseiten und Inhaltsverzeichnis

\MWatermarkSettings
% ---------------------------------- Parametrisierte Aufgaben ----------------------------------------

\ifttm
\newenvironment{MPExercise}{%
\begin{MExercise}%
}{%
\special{html:<button name="Name_MPEX}\arabic{MExerciseCounter}\special{html:" id="MPEX}\arabic{MExerciseCounter}%
\special{html:" type="button" onclick="reroll('}\arabic{MExerciseCounter}\special{html:');">Neue Aufgabe erzeugen</button>}%
\end{MExercise}%
}
\else
\newenvironment{MPExercise}{%
\begin{MExercise}%
}{%
\end{MExercise}%
}
\fi

% Parameter: Name, Min, Max, PDF-Standard. Name in Deklaration OHNE backslash, im Code MIT Backslash
\ifttm
\newcommand{\MGlobalInteger}[4]{\special{html:%
<!-- onloadstart //-->%
MVAR.push(createGlobalInteger("}#1\special{html:",}#2\special{html:,}#3\special{html:,}#4\special{html:)); %
<!-- onloadstop //-->%
<!-- viewmodelstart //-->%
ob}#1\special{html:: ko.observable(rerollMVar("}#1\special{html:")),%
<!-- viewmodelstop //-->%
}%
}%
\else%
\newcommand{\MGlobalInteger}[4]{\newcounter{mvc_#1}\setcounter{mvc_#1}{#4}}
\fi

% Parameter: Name, Min, Max, PDF-Standard. Name in Deklaration OHNE backslash, im Code MIT Backslash, Wert ist Wurzel von value
\ifttm
\newcommand{\MGlobalSqrt}[4]{\special{html:%
<!-- onloadstart //-->%
MVAR.push(createGlobalSqrt("}#1\special{html:",}#2\special{html:,}#3\special{html:,}#4\special{html:)); %
<!-- onloadstop //-->%
<!-- viewmodelstart //-->%
ob}#1\special{html:: ko.observable(rerollMVar("}#1\special{html:")),%
<!-- viewmodelstop //-->%
}%
}%
\else%
\newcommand{\MGlobalSqrt}[4]{\newcounter{mvc_#1}\setcounter{mvc_#1}{#4}}% Funktioniert nicht als Wurzel !!!
\fi

% Parameter: Name, Min, Max, PDF-Standard zaehler, PDF-Standard nenner. Name in Deklaration OHNE backslash, im Code MIT Backslash
\ifttm
\newcommand{\MGlobalFraction}[5]{\special{html:%
<!-- onloadstart //-->%
MVAR.push(createGlobalFraction("}#1\special{html:",}#2\special{html:,}#3\special{html:,}#4\special{html:,}#5\special{html:)); %
<!-- onloadstop //-->%
<!-- viewmodelstart //-->%
ob}#1\special{html:: ko.observable(rerollMVar("}#1\special{html:")),%
<!-- viewmodelstop //-->%
}%
}%
\else%
\newcommand{\MGlobalFraction}[5]{\newcounter{mvc_#1}\setcounter{mvc_#1}{#4}} % Funktioniert nicht als Bruch !!!
\fi

% MVar darf im HTML nur in MEvalMathDisplay-Umgebungen genutzt werden oder in Strings die an den Parser uebergeben werden
\ifttm%
\newcommand{\MVar}[1]{\special{html:[var_}#1\special{html:]}}%
\else%
\newcommand{\MVar}[1]{\arabic{mvc_#1}}%
\fi

\ifttm%
\newcommand{\MRerollButton}[2]{\special{html:<button type="button" onclick="rerollMVar('}#1\special{html:');">}#2\special{html:</button>}}%
\else%
\newcommand{\MRerollButton}[2]{\relax}% Keine sinnvolle Entsprechung im PDF
\fi

% MEvalMathDisplay fuer HTML wird in mconvert.pl im preprocessing realisiert
% PDF: eine equation*-Umgebung (ueber amsmath)
% HTML: Eine Mathjax-Tex-Umgebung, deren Auswertung mit knockout-obervablen gekoppelt ist
% PDF-Version hier nur fuer pdflatex-only-Uebersetzung gegeben

\ifttm\else\newenvironment{MEvalMathDisplay}{\begin{equation*}}{\end{equation*}}\fi

% ---------------------------------- Spezialbefehle fuer AD ------------------------------------------

%Abk�rzung f�r \longrightarrow:
\newcommand{\lto}{\ensuremath{\longrightarrow}}

%Makro f�r Funktionen:
\newcommand{\exfunction}[5]
{\begin{array}{rrcl}
 #1 \colon  & #2 &\lto & #3 \\[.05cm]  
  & #4 &\longmapsto  & #5 
\end{array}}

\newcommand{\function}[5]{%
#1:\;\left\lbrace{\begin{array}{rcl}
 #2 &\lto & #3 \\
 #4 &\longmapsto  & #5 \end{array}}\right.}


%Die Identit�t:
\DeclareMathOperator{\Id}{Id}

%Die Signumfunktion:
\DeclareMathOperator{\sgn}{sgn}

%Zwei Betonungskommandos (k�nnen angepasst werden):
\newcommand{\highlight}[1]{#1}
\newcommand{\modstextbf}[1]{#1}
\newcommand{\modsemph}[1]{#1}


% ---------------------------------- Spezialbefehle fuer JL ------------------------------------------


\def\jccolorfkt{green!50!black} %Farbe des Funktionsgraphen
\def\jccolorfktarea{green!25!white} %Farbe der Fl"ache unter dem Graphen
\def\jccolorfktareahell{green!12!white} %helle Einf"arbung der Fl"ache unter dem Graphen
\def\jccolorfktwert{green!50!black} %Farbe einzelner Punkte des Graphen

\newcommand{\MPfadBilder}{Bilder}

\ifttm%
\newcommand{\jMD}{\,\MD}%
\else%
\newcommand{\jMD}{\;\MD}%
\fi%

\def\jHTMLHinweisBedienung{\MInputHint{%
Mit Hilfe der Symbole am oberen Rand des Fensters
k"onnen Sie durch die einzelnen Abschnitte navigieren.}}

\def\jHTMLHinweisEingabeText{\MInputHint{%
Geben Sie jeweils ein Wort oder Zeichen als Antwort ein.}}

\def\jHTMLHinweisEingabeTerm{\MInputHint{%
Klammern Sie Ihre Terme, um eine eindeutige Eingabe zu erhalten. 
Beispiel: Der Term $\frac{3x+1}{x-2}$ soll in der Form
\texttt{(3*x+1)/((x+2)^2}$ eingegeben werden (wobei auch Leerzeichen 
eingegeben werden k"onnen, damit eine Formel besser lesbar ist).}}

\def\jHTMLHinweisEingabeIntervalle{\MInputHint{%
Intervalle werden links mit einer "offnenden Klammer und rechts mit einer 
schlie"senden Klammer angegeben. Eine runde Klammer wird verwendet, wenn der 
Rand nicht dazu geh"ort, eine eckige, wenn er dazu geh"ort. 
Als Trennzeichen wird ein Komma oder ein Semikolon akzeptiert.
Beispiele: $(a, b)$ offenes Intervall,
$[a; b)$ links abgeschlossenes, rechts offenes Intervall von $a$ bis $b$. 
Die Eingabe $]a;b[$ f"ur ein offenes Intervall wird nicht akzeptiert.
F"ur $\infty$ kann \texttt{infty} oder \texttt{unendlich} geschrieben werden.}}

\def\jHTMLHinweisEingabeFunktionen{\MInputHint{%
Schreiben Sie Malpunkte (geschrieben als \texttt{*}) aus und setzen Sie Klammern um Argumente f�r Funktionen.
Beispiele: Polynom: \texttt{3*x + 0.1}, Sinusfunktion: \texttt{sin(x)}, 
Verkettung von cos und Wurzel: \texttt{cos(sqrt(3*x))}.}}

\def\jHTMLHinweisEingabeFunktionenSinCos{\MInputHint{%
Die Sinusfunktion $\sin x$ wird in der Form \texttt{sin(x)} angegeben, %
$\cos\left(\sqrt{3 x}\right)$ durch \texttt{cos(sqrt(3*x))}.}}

\def\jHTMLHinweisEingabeFunktionenExp{\MInputHint{%
Die Exponentialfunktion $\MEU^{3x^4 + 5}$ wird als
\texttt{exp(3 * x^4 + 5)} angegeben, %
$\ln\left(\sqrt{x} + 3.2\right)$ durch \texttt{ln(sqrt(x) + 3.2)}.}}

% ---------------------------------- Spezialbefehle fuer Fachbereich Physik --------------------------

\newcommand{\E}{{e}}
\newcommand{\ME}[1]{\cdot 10^{#1}}
\newcommand{\MU}[1]{\;\mathrm{#1}}
\newcommand{\MPG}[3]{%
  \ifnum#2=0%
    #1\ \mathrm{#3}%
  \else%
    #1\cdot 10^{#2}\ \mathrm{#3}%
  \fi}%
%

\newcommand{\MMul}{\MExponentensymbXYZl} % Nur eine Abkuerzung


% ---------------------------------- Stichwortfunktionialitaet ---------------------------------------

% mpreindexentry wird durch Auswahlroutine in conv.pl durch mindexentry substitutiert
\ifttm%
\def\MIndex#1{\index{#1}\special{html:<!-- mpreindexentry;;}#1\special{html:;;}\arabic{MSubjectArea}\special{html:;;}%
\arabic{chapter}\special{html:;;}\arabic{section}\special{html:;;}\arabic{subsection}\special{html:;;}\arabic{MEntryCounter}\special{html:; //-->}%
\setcounter{MLastIndex}{\value{MEntryCounter}}%
\addtocounter{MEntryCounter}{1}%
}%
% Copyrightliste wird als tex-Datei im preprocessing von conv.pl erzeugt und unter converter/tex/entrycollection.tex abgelegt
% Der input-Befehl funktioniert nur, wenn die aufrufende tex-Datei auf der obersten Ebene liegt (d.h. selbst kein input/include ist, insbesondere keine Moduldatei)
\def\MEntryList{} % \input funktioniert nicht, weil ttm (und damit das \input) ausgefuehrt wird, bevor Datei da ist
\else%
\def\MIndex#1{\index{#1}}
\def\MEntryList{\MAbort{Stichwortliste nur im HTML realisierbar}}%
\fi%

\def\MEntry#1#2{\textbf{#1}\MIndex{#2}} % Idee: MLastType auf neuen Entry-Typ und dann ein MLabel vergeben mit autogen-Nummer

% ---------------------------------- Befehle fuer Tests ----------------------------------------------

% MEquationItem stellt eine Eingabezeile der Form Vorgabe = Antwortfeld her, der zweite Parameter kann z.B. MSimplifyQuestion-Befehl sein
\ifttm
\newcommand{\MEquationItem}[2]{{#1}$\,=\,${#2}}%
\else%
\newcommand{\MEquationItem}[2]{{#1}$\;\;=\,${#2}}%
\fi

\ifttm
\newcommand{\MInputHint}[1]{%
\ifnum%
\if\value{MTestSite}>0%
\else%
{\color{blue}#1}%
\fi%
\fi%
}
\else
\newcommand{\MInputHint}[1]{\relax}
\fi

\ifttm
\newcommand{\MInTestHeader}{%
Dies ist ein einreichbarer Test:
\begin{itemize}
\item{Im Gegensatz zu den offenen Aufgaben werden beim Eingeben keine Hinweise zur Formulierung der mathematischen Ausdr�cke gegeben.}
\item{Der Test kann jederzeit neu gestartet oder verlassen werden.}
\item{Der Test kann durch die Buttons am Ende der Seite beendet und abgeschickt, oder zur�ckgesetzt werden.}
\item{Der Test kann mehrfach probiert werden. F�r die Statistik z�hlt die zuletzt abgeschickte Version.}
\end{itemize}
}
\else
\newcommand{\MInTestHeader}{%
\relax
}
\fi

\ifttm
\newcommand{\MInTestFooter}{%
\special{html:<button name="Name_TESTFINISH" id="TESTFINISH" type="button" onclick="finish_button('}\MTestName\special{html:');">Test auswerten</button>}%
\begin{html}
&nbsp;&nbsp;&nbsp;&nbsp;&nbsp;&nbsp;&nbsp;&nbsp;
<button name="Name_TESTRESET" id="TESTRESET" type="button" onclick="reset_button();">Test zur�cksetzen</button>
<br />
<br />
<div class="xreply">
<p name="Name_TESTEVAL" id="TESTEVAL">
Hier erscheint die Testauswertung!
<br />
</p>
</div>
\end{html}
}
\else
\newcommand{\MInTestFooter}{%
\relax
}
\fi


% ---------------------------------- Notationsmakros -------------------------------------------------------------

% Notationsmakros die nicht von der Kursvariante abhaengig sind

\newcommand{\MZahltrennzeichen}[1]{\renewcommand{\MZXYZhltrennzeichen}{#1}}

\ifttm
\newcommand{\MZahl}[3][\MZXYZhltrennzeichen]{\edef\MZXYZtemp{\noexpand\special{html:<mn>#2#1#3</mn>}}\MZXYZtemp}
\else
\newcommand{\MZahl}[3][\MZXYZhltrennzeichen]{{}#2{#1}#3}
\fi

\newcommand{\MEinheitenabstand}[1]{\renewcommand{\MEinheitenabstXYZnd}{#1}}
\ifttm
\newcommand{\MEinheit}[2][\MEinheitenabstXYZnd]{{}#1\edef\MEINHtemp{\noexpand\special{html:<mi mathvariant="normal">#2</mi>}}\MEINHtemp} 
\else
\newcommand{\MEinheit}[2][\MEinheitenabstXYZnd]{{}#1 \mathrm{#2}} 
\fi

\newcommand{\MExponentensymbol}[1]{\renewcommand{\MExponentensymbXYZl}{#1}}
\newcommand{\MExponent}[2][\MExponentensymbXYZl]{{}#1{} 10^{#2}} 

%Punkte in 2 und 3 Dimensionen
\newcommand{\MPointTwo}[3][]{#1(#2\MCoordPointSep #3{}#1)}
\newcommand{\MPointThree}[4][]{#1(#2\MCoordPointSep #3\MCoordPointSep #4{}#1)}
\newcommand{\MPointTwoAS}[2]{\left(#1\MCoordPointSep #2\right)}
\newcommand{\MPointThreeAS}[3]{\left(#1\MCoordPointSep #2\MCoordPointSep #3\right)}

% Masseinheit, Standardabstand: \,
\newcommand{\MEinheitenabstXYZnd}{\MThinspace} 

% Horizontaler Leerraum zwischen herausgestellter Formel und Interpunktion
\ifttm
\newcommand{\MDFPSpace}{\,}
\newcommand{\MDFPaSpace}{\,\,}
\newcommand{\MBlank}{\ }
\else
\newcommand{\MDFPSpace}{\;}
\newcommand{\MDFPaSpace}{\;\;}
\newcommand{\MBlank}{\ }
\fi

% Satzende in herausgestellter Formel mit horizontalem Leerraum
\newcommand{\MDFPeriod}{\MDFPSpace .}

% Separation von Aufzaehlung und Bedingung in Menge
\newcommand{\MCondSetSep}{\,:\,} %oder '\mid'

% Konverter kennt mathopen nicht
\ifttm
\def\mathopen#1{}
\fi

% -----------------------------------START Rouletteaufgaben ------------------------------------------------------------

\ifttm
% #1 = Dateiname, #2 = eindeutige ID fuer das Roulette im Kurs
\newcommand{\MDirectRouletteExercises}[2]{
\begin{MExercise}
\texttt{Im HTML erscheinen hier Aufgaben aus einer Aufgabenliste...}
\end{MExercise}
}
\else
\newcommand{\MDirectRouletteExercises}[2]{\relax} % wird durch mconvert.pl gefunden und ersetzt
\fi


% ---------------------------------- START Makros, die von der Kursvariante abhaengen ----------------------------------

\ifvariantunotation
  % unotation = An Universitaeten uebliche Notation
  \def\MVariant{unotation}

  % Trennzeichen fuer Dezimalzahlen
  \newcommand{\MZXYZhltrennzeichen}{.}

  % Exponent zur Basis 10 in der Exponentialschreibweise, 
  % Standardmalzeichen: \times
  \newcommand{\MExponentensymbXYZl}{\times} 

  % Begrenzungszeichen fuer offene Intervalle
  \newcommand{\MoIl}[1][]{\mbox{}#1(\mathopen{}} % bzw. ']'
  \newcommand{\MoIr}[1][]{#1)\mbox{}} % bzw. '['

  % Zahlen-Separation im IntervaLL
  \newcommand{\MIntvlSep}{,} %oder ';'

  % Separation von Elementen in Mengen
  \newcommand{\MElSetSep}{,} %oder ';'

  % Separation von Koordinaten in Punkten
  \newcommand{\MCoordPointSep}{,} %oder ';' oder '|', '\MThinspace|\MThinspace'

\else
  % An dieser Stelle wird angenommen, dass std-Variante aktiv ist
  % std = beschlossene Notation im TU9-Onlinekurs 
  \def\MVariant{std}

  % Trennzeichen fuer Dezimalzahlen
  \newcommand{\MZXYZhltrennzeichen}{,}

  % Exponent zur Basis 10 in der Exponentialschreibweise, 
  % Standardmalzeichen: \times
  \newcommand{\MExponentensymbXYZl}{\times} 

  % Begrenzungszeichen fuer offene Intervalle
  \newcommand{\MoIl}[1][]{\mbox{}#1]\mathopen{}} % bzw. '('
  \newcommand{\MoIr}[1][]{#1[\mbox{}} % bzw. ')'

  % Zahlen-Separation im IntervaLL
  \newcommand{\MIntvlSep}{;} %oder ','
  
  % Separation von Elementen in Mengen
  \newcommand{\MElSetSep}{;} %oder ','

  % Separation von Koordinaten in Punkten
  \newcommand{\MCoordPointSep}{;} %oder '|', '\MThinspace|\MThinspace'

\fi



% ---------------------------------- ENDE Makros, die von der Kursvariante abhaengen ----------------------------------


% diese Kommandos setzen Mathemodus vorraus
\newcommand{\MGeoAbstand}[2]{[\overline{{#1}{#2}}]}
\newcommand{\MGeoGerade}[2]{{#1}{#2}}
\newcommand{\MGeoStrecke}[2]{\overline{{#1}{#2}}}
\newcommand{\MGeoDreieck}[3]{{#1}{#2}{#3}}

%
\ifttm
\newcommand{\MOhm}{\special{html:<mn>&#x3A9;</mn>}}
\else
\newcommand{\MOhm}{\Omega} %\varOmega
\fi


\def\PERCTAG{\MAbort{PERCTAG ist zur internen verwendung in mconvert.pl reserviert, dieses Makro darf sonst nicht benutzt werden.}}

% Im Gegensatz zu einfachen html-Umgebungen werden MDirectHTML-Umgebungen von mconvert.pl am ganzen ttm-Prozess vorbeigeschleust und aus dem PDF komplett ausgeschnitten
\ifttm%
\newenvironment{MDirectHTML}{\begin{html}}{\end{html}}%
\else%
\newenvironment{MDirectHTML}{\begin{html}}{\end{html}}%
\fi

% Im Gegensatz zu einfachen Mathe-Umgebungen werden MDirectMath-Umgebungen von mconvert.pl am ganzen ttm-Prozess vorbeigeschleust, ueber MathJax realisiert, und im PDF als $$ ... $$ gesetzt
\ifttm%
\newenvironment{MDirectMath}{\begin{html}}{\end{html}}%
\else%
\newenvironment{MDirectMath}{\begin{equation*}}{\end{equation*}}% Vorsicht, auch \[ und \] werden in amsmath durch equation* redefiniert
\fi

% ---------------------------------- Location Management ---------------------------------------------

% #1 = buttonname (muss in files/images liegen und Format 48x48 haben), #2 = Vollstaendiger Einrichtungsname, #3 = Kuerzel der Einrichtung,  #4 = Name der include-texdatei
\ifttm
\newcommand{\MLocationSite}[3]{\special{html:<!-- mlocation;;}#1\special{html:;;}#2\special{html:;;}#3\special{html:;; //-->}}
\else
\newcommand{\MLocationSite}[3]{\relax}
\fi

% ---------------------------------- Copyright Management --------------------------------------------

\newcommand{\MCCLicense}{%
{\color{green}\textbf{CC BY-SA 3.0}}
}

\newcommand{\MCopyrightLabel}[1]{ (\MSRef{L_COPYRIGHTCOLLECTION}{Lizenz})\MLabel{#1}}

% Copyrightliste wird als tex-Datei im preprocessing erzeugt und unter converter/tex/copyrightcollection.tex abgelegt
% Der input-Befehl funktioniert nur, wenn die aufrufende tex-Datei auf der obersten Ebene liegt (d.h. selbst kein input/include ist, insbesondere keine Moduldatei)
\newcommand{\MCopyrightCollection}{\input{copyrightcollection.tex}}

% MCopyrightNotice fuegt eine Copyrightnotiz ein, der parser ersetzt diese durch CopyrightNoticePOST im preparsing, diese Definition wird nur fuer reine pdflatex-Uebersetzungen gebraucht
% Parameter: #1: Kurze Lizenzbeschreibung (typischerweise \MCCLicense)
%            #2: Link zum Original (http://...) oder NONE falls das Bild selbst ein Original ist, oder TIKZ falls das Bild aus einer tikz-Umgebung stammt
%            #3: Link zum Autor (http://...) oder MINT falls Original im MINT-Kolleg erstellt oder NONE falls Autor unbekannt
%            #4: Bemerkung (z.B. dass Datei mit Maple exportiert wurde)
%            #5: Labelstring fuer existierendes Label auf das copyrighted Objekt, mit MCopyrightLabel erzeugt
%            Keines der Felder darf leer sein!
\newcommand{\MCopyrightNotice}[5]{\MCopyrightNoticePOST{#1}{#2}{#3}{#4}{#5}}

\ifttm%
\newcommand{\MCopyrightNoticePOST}[5]{\relax}%
\else%
\newcommand{\MCopyrightNoticePOST}[5]{\relax}%
\fi%

% ---------------------------------- Meldungen fuer den Benutzer des Konverters ----------------------
\MPragma{mintmodversion;P0.1.0}
\MPragma{usercomment;This is file mintmod.tex version P0.1.0}


% ----------------------------------- Spezialelemente fuer Konfigurationsseite, werden nicht von mintscripts.js verwaltet --

% #1 = DOM-id der Box
\ifttm\newcommand{\MConfigbox}[1]{\special{html:<input cfieldtype="2" type="checkbox" name="Name_}#1\special{html:" id="}#1\special{html:" onchange="confHandlerChange('}#1\special{html:');"/>}}\fi % darf im PDF nicht aufgerufen werden!


\MPragma{MathSkip}

\Mtikzexternalize

\begin{document}
\MSetSubject{\MINTMathematics}

\MSection{Elementary Functions} 
\MSetSectionID{elfunktionen}

\begin{MSectionStart}
\MLabel{VBKM06}
\MDeclareSiteUXID{VBKM06_START}

\MModstartBox
\end{MSectionStart}

\MSubsection{Basics of Functions}
\MLabel{VBKM06_Grundlagen}

\begin{MIntro}
\MDeclareSiteUXID{VBKM06_Grundlagen_Intro}
From Module~\MNRef{VBKM01} we already know that \highlight{real numbers} are \highlight{sets} and 
\highlight{intervals} are \highlight{subsets} of real numbers.

\begin{MExample}
All real numbers $\R$, excluding the number $0\in\R$, are to be collected in a set. How is this 
set of numbers described? For this, the notation 
\[
 \R\setminus\{0\} 
\]
is used. This reads as ``$\R$ without $0$''. Alternatively, this set can be described 
as a union of two open intervals:
\[
 \R\setminus\{0\}=(-\infty\MIntvlSep 0)\cup(0\MIntvlSep \infty) \MDFPeriod
\]
In the same way, single numbers can be removed from any other sets. So, for example, the 
set
\[
 [1\MIntvlSep 3)\setminus\{2\} 
\]
contains all numbers of the half-open interval $[1\MIntvlSep 3)$, excluding the number $2$:

\MTikzAuto{%
\begin{tikzpicture}
% reelle Achse
\draw[->,color=black] (-1,0.0) -- (5,0.0);
\foreach \x in {-1, 0, 1, 2, 3, 4}
\draw[shift={(\x,0)},color=black] (0pt,2pt) -- (0pt,-2pt) node[below] {\footnotesize $\x$};
\draw (4.9,-0.3) node[] {$\mathbb{R}$};
% Intervall: [1,2):
\draw [line width=2.0pt,color=blue] (1,0.0)-- (2,0.0);
% Intervall: (2,3):
\draw [line width=2.0pt,color=blue] (2,0.0)-- (3,0.0);
% fehlende 2:
\draw [color = blue, fill = white] (2,0) circle (1.5pt);
% Enden:
\draw [fill = blue] (1,0) circle (1.5pt);
\draw [color = blue, fill = white] (3,0) circle (1.5pt);
\end{tikzpicture}
}%
\end{MExample}

\begin{MExercise}
Indicate the intervals $(-\infty\MIntvlSep \pi)$ and $(8\MIntvlSep \MZahl{8}{5}]$ on the number line.
\begin{MHint}{Solution}
\MTikzAuto{%
\begin{tikzpicture}
% reelle Achse
\draw[->,color=black] (-2.0,0.0) -- (6.0,0.0);
\foreach \x in {-1, 0, 1,2,3,4,5}
\draw[shift={(\x,0)},color=black] (0pt,2pt) -- (0pt,-2pt) node[below] {\footnotesize $\x$};
\draw (5.7,-0.3) node[] {$\mathbb{R}$};
% Intervall: [-\infty,\pi)
\draw [line width=2.0pt,color=blue] (-2,0.0)-- (3.14,0.0);
\draw [color = blue, fill = white] (3.14,0) circle (1.5pt);
\end{tikzpicture}
}%

\MTikzAuto{%
\begin{tikzpicture}
% reelle Achse
\draw[->,color=black] (5,0.0) -- (11,0.0);
\foreach \x in {6, 7, 8, 9, 10}
\draw[shift={(\x,0)},color=black] (0pt,2pt) -- (0pt,-2pt) node[below] {\footnotesize $\x$};
\draw (10.7,-0.3) node[] {$\mathbb{R}$};
% Intervall: (8,8.5]
\draw [line width=2.0pt,color=blue] (8,0.0)-- (8.5,0.0);
\draw [color = blue, fill = white] (8,0) circle (1.5pt);
\draw [fill = blue] (8.5,0) circle (1.5pt);
\end{tikzpicture}
}%

\end{MHint}
\end{MExercise}

For doing and applying mathematics, it is not sufficient to just consider sets and 
equations and inequalities for numbers of these sets, as done in the previous modules (for example, 
in \MSRef{VBKM01}{Module 1}). We also need \highlight{functions} 
(which are often also denoted as \modsemph{maps}).

\begin{MInfo}
\MLabel{VBKM06_Abbildungen}
\MEntry{Functions}{function} (or \MEntry{maps}{map}) are assignments between elements of two sets such that there is exactly one element in the second set assigned to each element in the first set.%%%
\end{MInfo}
\end{MIntro}

We will focus on the basic mathematical term of an assignment between numbers in 
Section~\MNRef{sec:zuordnungen}. In Section~\MNRef{sec:anwendungen} we will refer to  
applications of mathematics in other sciences, and we will illustrate how useful the 
mathematical notion of a function as a tool for the formalisation of relations between 
dependent quantities is. Finally, we will study the graphical representation of functions 
using graphs in Section~\MNRef{sec:graphen}. Later in this chapter we will consider 
the most relevant elementary functions together with their graphs. It is fundamental to 
know the behaviour of the graphs of the elementary functions.


\begin{MXContent}{Assignments between Sets}{Assignments}{STD}\MLabel{sec:zuordnungen}
\MDeclareSiteUXID{VBKM06_Zuordnungen}

We start with a first example of a function as an assignment between two sets. For this 
purpose, we consider the set of natural numbers $\N$ and the set of rational numbers
$\Q$, and visualise these two sets as two ``containers'' filled with numbers.

\MTikzAuto{%
\begin{tikzpicture}
% nat�rliche Zahlen:
\draw  (3.5,10.3) node[anchor=north west] {$\mathbb{N}$};
\draw [] (2.7,7.6) ellipse (1.7cm and 2.4cm);
\draw (2,6.3)  node[anchor=north west](eins) {1};
\draw (2.1,7.4)  node[anchor=north west](zwei) {2};
\draw (2.3,8.7)  node[anchor=north west] (drei){3};
\draw (2.5,9.5) node[anchor=north west] (vier) {4};
\draw (2.7,9.8) node[anchor=north west] {...};
% rationale Zahlen:
\draw (9.8,10.3) node[anchor=north west] {$\mathbb{Q}$};
\draw (8.8,7.5) ellipse (1.9cm and 2.6cm);
\draw (8,8) node[anchor=north west](Feins) {$\frac{1}{2}$};
\draw (9.8,8.4) node[anchor=north west] {$\frac{3}{4}$};
\draw (7.4,6.7) node[anchor=north west] {$-\frac{1}{2}$};
\draw (7.8,9.6) node[anchor=north west](Fvier) {$2$};
\draw (7.1,7.5) node[anchor=north west] {$0$};
\draw (8.6,9.9) node[anchor=north west] {$3$};
\draw (7.3,8.7) node[anchor=north west](Fzwei) {$1$};
\draw (8.6,6.9) node[anchor=north west] {$\frac{1}{3}$};
\draw (8.5,8.7) node[anchor=north west] {$\frac{2}{3}$};
\draw (9.0,7.8) node[anchor=north west] {$\frac{2}{5}$};
\draw (9.3,9.3) node[anchor=north west](Fdrei) {$\frac{3}{2}$};
\draw (9.3,6.5) node[anchor=north west] {...};
\end{tikzpicture} 
}%

Now, we want to create an assignment between the elements of these two sets as follows. 
To every number $n\in\N$ half of this number $\frac{n}{2}\in\Q$ is assigned, i.e. 
to the number $1\in\N$ we assign the number $\frac{1}{2}\in\Q$, to the number 
$2\in\N$ we assign the number $1\in\Q$, etc. In the figure below, this 
is illustrated by arrows that indicate which numbers in $\Q$ are assigned to
which numbers in $\N$.

\MTikzAuto{%
\begin{tikzpicture}
% nat�rliche Zahlen:
\draw  (3.5,10.3) node[anchor=north west] {$\mathbb{N}$};
\draw [] (2.7,7.6) ellipse (1.7cm and 2.4cm);
\draw [color = blue](2,6.3)  node[anchor=north west](eins) {1};
\draw [color = blue](2.1,7.4)  node[anchor=north west](zwei) {2};
\draw [color = blue](2.3,8.7)  node[anchor=north west] (drei){3};
\draw [color = blue](2.5,9.5) node[anchor=north west] (vier) {4};
\draw [color = blue](2.7,9.8) node[anchor=north west] {...};
% rationale Zahlen:
\draw (9.8,10.3) node[anchor=north west] {$\mathbb{Q}$};
\draw (8.8,7.5) ellipse (1.9cm and 2.6cm);
\draw (8,8) node[anchor=north west](Feins) {$\frac{1}{2}$};
\draw (9.8,8.4) node[anchor=north west] {$\frac{3}{4}$};
\draw (7.4,6.7) node[anchor=north west] {$-\frac{1}{2}$};
\draw (7.8,9.6) node[anchor=north west](Fvier) {$2$};
\draw (7.1,7.5) node[anchor=north west] {$0$};
\draw (8.6,9.9) node[anchor=north west] {$3$};
\draw (7.3,8.7) node[anchor=north west](Fzwei) {$1$};
\draw (8.6,6.9) node[anchor=north west] {$\frac{1}{3}$};
\draw (8.5,8.7) node[anchor=north west] {$\frac{2}{3}$};
\draw (9.0,7.8) node[anchor=north west] {$\frac{2}{5}$};
\draw (9.3,9.3) node[anchor=north west](Fdrei) {$\frac{3}{2}$};
\draw (9.3,6.5) node[anchor=north west] {...};
% Abbildung: 
\draw [|->] (eins) -- (Feins);
\draw [|->] (zwei) --(Fzwei);
\draw [|->] (drei) -- (Fdrei);
\draw [|->] (vier) --(Fvier);
\end{tikzpicture}
}%

For the \modsemph{assignment of single elements} of the sets, described above in words,
we use the so-called \highlight{assignment arrow}. This is an arrow that has a bar at its tail: $\longmapsto$.
It says that to the number on the side with the bar the number on the side with the arrow is assigned:
\[
\N\ni1\longmapsto\MZahl{0}{5}\in\Q \MDFPSpace,\MDFPaSpace \N\ni2\longmapsto1\in\Q \MDFPSpace, \MDFPaSpace \text{etc.} 
\]
With these assignments, we have now constructed a \modsemph{function} from the natural numbers $\N$ to the rational 
numbers $\Q$. In mathematics, this assignment is named, i.e. a symbol is allocated (often this is $f$ for \modsemph{f}unction),
that shall describe exactly this assignment. For this purpose, the sets of numbers \modsemph{from which}
and \modsemph{to which} the assignment will be done must be noted. In our case, to the elements of the set of natural 
numbers $\N$ the rational numbers are assigned. Mathematically, this is expressed by a so-called 
\highlight{mapping arrow} $\null\lto\null$, i.e. an arrow that has the target set at its head and the set, 
whose elements are assigned to, at its tail. Thus, in our case we have
\[
 f\colon \N\lto\Q \MDFPeriod
\]
This reads as ``the function $f$ maps $\N$ to $\Q$''. 

Furthermore, we can now ask whether the assignments of this function,
$1\longmapsto\frac{1}{2},\ 2\longmapsto1,\ \text{etc.}$, can be described in a more 
compact way. To do this, we recall the beginning of this example. We decided to assign 
to every natural number $n$ its half $\frac{n}{2}$. So we can write this arbitrary 
natural number $n$ and the corresponding rational number $\frac{n}{2}$ left and right to the 
assignment arrow, respectively:
\[
 n\longmapsto\frac{n}{2} \MDFPeriod
\]
This reads as ``$n$ is mapped to $\frac{n}{2}$''. This notation is also called 
\highlight{mapping rule} of the function. Another notation for the mapping rule uses 
the name of the function:
\[
 f(n) = \frac{n}{2} \MDFPeriod
\]
This reads as ``$f$ of $n$ equals $\frac{n}{2}$''. Altogether we can describe this function $f$
as follows:
\[
 \function{f}{\N}{\Q}{n}{\frac{n}{2} \MDFPeriod}
\]
Finally, this reads as ``the function $f$ maps $\N$ to $\Q$, each $n\in\N$ is mapped to $\frac{n}{2}\in\Q$''.
Throughout this module, we will continue to use this summarising notation of functions. 

Let us consider a few further simple examples for functions.

\begin{MExample}\MLabel{bsp1:grundlagen}
\begin{itemize}
 \item A function $g$ assigns to every real number $x$ its square: $x\cdot x=x^2$. 
This results in the so-called standard parabola (see Section~\MNRef{sec:monome}):
 \[
  \function{g}{\R}{\R}{x}{x^2 \MDFPeriod}
 \]
Hence, the mapping rule of $g$ is $g(x)=x^2$. Then, assignments can be calculated for specific numbers.
For example, $g(2)=2^2=4$ or $g(-\pi)=(-\pi)^2=\pi^2$, etc.

\item A function $\Mvarphi$ shall assign to every real number $y$ between $0$ and $1$ three times its value plus 
$1$. This is an example for a so-called linear affine function (see Section~\MNRef{VBKM06_sec:linear-affin}):
 \[
  \function{\Mvarphi}{(0\MIntvlSep 1)}{\R}{y}{3y+1 \MDFPeriod}
 \]
Hence, the mapping rule of $\Mvarphi$ is $\Mvarphi(y)=3y+1$. Thus, for example, 
$\Mvarphi(\frac{1}{3})=3\cdot\frac{1}{3}+1=2$, etc. 
However, in this case the values $\Mvarphi(8)$ or $\Mvarphi(1)$ \modsemph{cannot be calculated}
since $8$ and $1$ do not belong to the set $(0\MIntvlSep 1)$. 
 %Dies wird unten im Rest dieses Abschnitts nochmal genauer er�rtert werden. 
\end{itemize}

\end{MExample}

\begin{MExercise}\MLabel{ex1:grundlagen}
\begin{itemize}
 \item[(i)] Specify a function $h$ that assigns to every positive real number $x$ its reciprocal. 
  Calculate the values $h(2)$ and $h(1)$. Complete the two assignments
 \[
  3\longmapsto\text{ ?}\quad\text{and}\quad\text{? }\longmapsto2
 \]
 of $h$.
 \item[(ii)] Describe in words the assignment that is done by the following function:
 \[
  \function{w}{[4\MIntvlSep 9]}{\R}{\alpha}{\sqrt{\alpha} \MDFPeriod}
 \]
 Calculate $w(9)$ and $w(5)$. Can $w(10)$ also be calculated?
\end{itemize}

\begin{MHint}{Solution}
 \begin{itemize}
  \item[(i)] The reciprocal of $x$ is $\frac{1}{x}$. The set of positive real numbers is $(0\MIntvlSep \infty)$.
  Hence, the function $h$ can be written as
  \[
   \function{h}{(0\MIntvlSep \infty)}{\R}{x}{\frac{1}{x} \MDFPeriod}
  \]
  This is an example for a function of hyperbolic type that will be described in more detail in 
  Section~\MNRef{VBKM06_Potenz}. The mapping rule of $h$ is $h(x)=\frac{1}{x}$, hence $h(2)=\frac{1}{2}$
  and $h(1)=\frac{1}{1}=1$. Moreover, we have $h(3)=\frac{1}{3}$, hence $3\longmapsto\frac{1}{3}$.
  The observation $h(\frac{1}{2})=\frac{1}{\frac{1}{2}}=2$ gives us $\frac{1}{2}\longmapsto2$.
  \item[(ii)] The function $w$ assigns to every real number $\alpha$ greater or equal to $4$ and less or equal
  to $9$ its square root $\sqrt{\alpha}$. The mapping rule is $w(\alpha)=\sqrt{\alpha}$, hence 
  $w(9)=\sqrt{9}=3$ and $w(5)=\sqrt{5}$. The value $w(10)$ cannot be calculated since $10\notin[4\MIntvlSep 9]$.
 \end{itemize}

\end{MHint}
\end{MExercise}

The examples above show some basic properties of functions for which we will now introduce specific terminology.
\begin{MInfo}
For the definition of a function, a set of numbers is specified whose elements are to be assigned to 
other numbers by the function. This set is called the \MEntry{domain}{domain} or set of definition of the function.
If the function has a name, e.g. $f$, then the domain is denoted by the symbol $D_f$. For example, 
the domain of the function
\[
   \function{h}{(0\MIntvlSep \infty)}{\R}{x}{\frac{1}{x}}
\]
in Exercise~\MNRef{ex1:grundlagen} is the set $D_h=(0\MIntvlSep \infty)$. There is also a specific term for the elements of the 
domain. In this exercise, the numbers $x\in D_h$ are assigned by 
the mapping rule $h(x)=\frac{1}{x}$. Here, the variable $x$ is called the 
\MEntry{independent variable}{independent variable} of the function $h$.
\end{MInfo}

\begin{MExercise}
Specify the domain of function $w$ in Exercise~\MNRef{ex1:grundlagen} and function $g$ in 
Example~\MNRef{bsp1:grundlagen}.

\begin{MHint}{Solution}
We have
\[
  \function{w}{[4\MIntvlSep 9]}{\R}{\alpha}{\sqrt{\alpha}}
\]
and 
\[
  \function{g}{\R}{\R}{x}{x^2 \MDFPSpace,}
\]
hence $D_w=[4\MIntvlSep 9]$ and $D_g=\R$.
\end{MHint}
\end{MExercise}

If we consider the mapping rule $h(x)=\frac{1}{x}$ of the function, we see that there is no reason 
not to insert in $\frac{1}{x}$ any real number $x$, \modsemph{excluding} the value $x=0$, since 
the operation ``$\frac{1}{0}$'' has no solution. Therefore, in specifying the domain, we can distinguish 
between numbers that are excluded because they are 
\modsemph{not allowed to be inserted in the mapping rule at all} and those that are excluded because 
\modsemph{the function is just defined accordingly}. This now leads to the term of the 
\highlight{maximal domain} of a function, i.e. the maximum subset of real numbers $\R$ that can
be used as the domain of a function with a known mapping rule.
    

\begin{MExample}
The maximum domain $D_h\subset\R$ of the function
\[
 \function{h}{D_h}{\R}{x}{\frac{1}{x} \MDFPSpace,}
\]
is $D_h=\R\setminus\{0\}$.
\end{MExample}

\begin{MExercise}
Find the maximum domain of the function
\[
  \function{w}{D_w}{\R}{\alpha}{\sqrt{\alpha} \MDFPeriod}
\]

\begin{MHint}{Solution}
The square root has a real number as its result for all non-negative real numbers. Hence, $D_w=[0\MIntvlSep \infty)$.%
\end{MHint}%
\end{MExercise}%

For the definition of a function, a second set  (beside the domain) is required, the set that
is the target of the assignment by the function. This set is called \highlight{target set} or \highlight{codomain}. 
Let us again consider the function
\[
  \function{\Mvarphi}{(0\MIntvlSep 1)}{\R}{y}{3y+1}
\]
in Example~\MNRef{bsp1:grundlagen}. Its target set is the set of real numbers $\R$. The target set of the function
\[
 \function{f}{\N}{\Q}{n}{\frac{n}{2}}
\]
in the first example of this section is the set of rational numbers $\Q$. Here, we see an important difference
between the domain and target set of a function. The domain contains all numbers, and only those numbers, that are allowed 
to be inserted (and one wishes to insert) in the mapping rule of the function. However, the target set can contain
all numbers that can potentially appear as a result of the mapping function.

In this context the question arises: What is the smallest target set that can be used for a function 
with given domain and known mapping rule? The smallest target set is the set of numbers that -- for 
a given domain and mapping rule -- indeed appear as targets of the assignment. This set is 
called \MEntry{range}{range} or image, and its elements are called values of the function. For a function
$f$, the symbol $W_f$ is used for the image. For the values of a function $f$ of an independent variable $x$ 
we write generally $f(x)\in W_f$, as in the mapping rule, or we introduce another variable, e.g. $y=f(x)\in 
W_f$. 


\begin{MExample}
Let us consider again the example
\[
  \function{\Mvarphi}{(0\MIntvlSep 1)}{\R}{y}{3y+1 \MDFPeriod}
\]
The range of this function is
\[
W_{\Mvarphi}=(1\MIntvlSep 4) \MDFPeriod
\]
This can be seen by inserting some values from $D_{\Mvarphi}=(0\MIntvlSep 1)$ in the  mapping rule and calculating 
the results. In this way, a so-called \MEntry{table of values}{table of values} is obtained:

\begin{center}
\begin{tabular}{|c|c|c|c|c|c|}
\hline
$y$ & $\MZahl{0}{1}$ & $\MZahl{0}{3}$ & $\MZahl{0}{5}$ & $\MZahl{0}{7}$ & $\MZahl{0}{9}$ \\\hline 
$\Mvarphi(y)$ & $\MZahl{1}{3}$ & $\MZahl{1}{9}$ & $\MZahl{2}{5}$ & $\MZahl{3}{1}$ & $\MZahl{3}{7}$ \\ \hline
\end{tabular}
\end{center}
\end{MExample}

Such tables of values are useful to get an overview of the values of a function. However, from a 
mathematical point of view, they are not sufficient to be \modsemph{completely sure} what the 
actual range of a function is. One method to determine the range of a function is based on 
solving inequalities:
\begin{MExample}
For the function
\[
  \function{\Mvarphi}{(0\MIntvlSep 1)}{\R}{y}{3y+1 \MDFPSpace ,} 
\]
due to the domain $D_{\Mvarphi}=(0\MIntvlSep 1)$ for the independent variable $y$ we have
\[
 0<y<1 \MDFPeriod
\]
Now, we use equivalent transformations to create the mapping rule $\Mvarphi(y)=3y+1$ in this inequality:
\[
 0<y<1\,|\cdot 3\quad\Leftrightarrow\quad 0<3y<3\,|+1\quad\Leftrightarrow\quad 1<3y+1<4\quad\Leftrightarrow\quad 1<\Mvarphi(y)<4 \MDFPeriod
\]
Hence, we have for the values of the function $\Mvarphi(y)\in(1\MIntvlSep 4)$ and therefore $W_{\Mvarphi}=(1\MIntvlSep 4)$.
\end{MExample}

\end{MXContent}

\begin{MXContent}{Functions in Mathematics and Applications}{Mathematics and Applications}{STD}\MLabel{sec:anwendungen}
\MDeclareSiteUXID{VBKM06_FunktionenAnwendungen}
Mathematical functions often describe relations between quantities that arise from other sciences or everyday life. 
For example, the volume $V$ of a cube depends on the side length $a$ of the cube. The volume 
can be considered as a mathematical function that assigns the corresponding volume $V(a)=a^3$ to every possible side length $a>0$:
\[
 \function{V}{(0\MIntvlSep \infty)}{\R}{a}{V(a)=a^3 \MDFPeriod}
\]
The result is the cubic standard parabola (see Section~\MNRef{sec:monome}) for the relation between side length and
volume. In this way, many more examples can be found arising from sciences and everyday life: the position as 
a function of time in physics, the reaction rate as a function of concentration in chemistry, the amount 
of flour needed as a function of the desired amount of dough in a cake recipe, etc. 

To this end, let us consider an example.
\begin{MExample}\MLabel{bsp:anwendungen}%
The intensity of nuclear radiation is \highlight{inversely proportional} to the square of the distance to the 
source. This is called the \modsemph{inverse square law}. Using a physical \highlight{proportionality factor} $c>0$,
the relation between intensity $I$ of the radiation and distance $r>0$ from the source can be formulated 
mathematically as follows:
\[
 \function{I}{(0\MIntvlSep \infty)}{\R}{r}{\frac{c}{r^2} \MDFPeriod}
\]
Hence, for the intensity we have the mapping rule $I(r)=\frac{c}{r^2}$ that describes the relation between the 
quantities $I$ and $r$. 
\end{MExample}

\begin{MExercise}\MLabel{ex:anwendungen}
In the construction of wind turbines it is known that the wind turbine power is \highlight{proportional} to the 
cube of the wind velocity. Under the condition that the proportionality factor satisfies the relation $\rho>0$, 
which of the following mathematical functions describes this relation of physical quantities correctly?

a)
\[
 \function{P}{(0\MIntvlSep \infty)}{\R}{v}{P(v)=\frac{\rho}{v^3}}
\]
b)
\[
 \function{P}{\R}{\R}{v}{P(v)=\rho v^3}
\]
c)
\[
 \function{P}{[0\MIntvlSep \infty)}{\R}{v}{P(v)=\rho v^3}
\]
d)
\[
 \function{x}{[0\MIntvlSep \infty)}{\R}{f}{x(f)=\rho f^3}
\]
\begin{MHint}{Solution}
c) and d) are correct. a) is wrong since it describes an inverse proportionality. b) is not adequate to the problem.
Only non-negative wind velocities make sense in this context and hence, negative values should be excluded from the domain. 
c) is correct; in this case we have the relation $P(v)=\rho v^3$ between the power $P$ and the wind velocity $v$. We see that d) 
is equally correct. In this case we have a function that is completely identical to case c), except for the fact that in this 
case the power is denoted by $x$ and the wind velocity is denoted by $f$. This again clarifies that the letters used to denote the 
function and the variables are mathematically completely arbitrary. However, in the sciences there are conventions, which 
letters are \modsemph{generally} used to denote certain quantities. So, here it is more common to denote the velocity by $v$ and the power by $P$.
\end{MHint}

\end{MExercise}


\end{MXContent}

\begin{MXContent}{Invertability}{Invertability}{STD}\MLabel{sec:graphen}
\MDeclareSiteUXID{VBKM06_Umkehrbarkeit}
The visualisation of a function, as for example
\[
 \function{f}{\R}{\R}{x}{x^2 \MDFPSpace,}
\]
in form of a so-called Venn diagram (see Section~\MNRef{sec:zuordnungen})

% venn7 Quadratische Zuordnung = venn4
\MTikzAuto{%
\begin{tikzpicture}[scale = 0.5]
% Definitionsbereich
\draw  (3.5,10.3) node[anchor=north west] {$\mathbb{R}$};
\draw [] (2.7,7.6) ellipse (1.7cm and 2.4cm);
\draw [color = blue](2,6.3)  node[anchor=north west](vierMinus) {-4};
\draw [color = blue](2.1,7.4)  node[anchor=north west](eins) {1};
\draw [color = blue](2.3,8.7)  node[anchor=north west] (vier){4};
\draw [color = blue](2.5,9.5) node[anchor=north west]  {...};
% Wertebereich
\draw (9.8,10.3) node[anchor=north west] {$\mathbb{R}$};
\draw (8.8,7.5) ellipse (1.9cm and 2.6cm);
\draw (8,8) node[anchor=north west] {$\frac{1}{2}$};
\draw (7.4,6.7) node[anchor=north west] {$-\frac{1}{2}$};
\draw (7.1,7.5) node[anchor=north west](Feins) {$1$};
\draw (8.0,9.8) node[anchor=north west] (Fvier){$16$};
\draw (7.3,8.7) node[anchor=north west]{$0$};
\draw (9.0,7.8) node[anchor=north west] {$\pi$};
\draw (9.3,9.3) node[anchor=north west] {$\frac{3}{2}$};
\draw (9.3,6.5) node[anchor=north west] {...};
% Abbildung: 
\draw[|->] (eins) -- (Feins);
\draw[|->] (vier) -- (Fvier);
\draw[|->] (vierMinus) -- (Fvier);
\end{tikzpicture}
}%

is in fact useful to understand what a function really is, but it says nothing about the special properties 
of the function. For this purpose, another kind of graphical representation exists, namely the representation 
as the \highlight{graph} of the function. For this representation, we draw a two-dimensional coordinate system 
(see Module~\MNRef{VBKM09}) in which the numbers of the domain of the function are indicated on the 
horizontal axis and the numbers from the range are indicated on the vertical axis. In such a coordinate 
system we mark all points $(x|f(x))$ arising by the assignment of the function $x\longmapsto f(x)$. In 
this case, these are all points 
$(x|x^2)$, i.e.~$(1|1)$, $(-1|1)$, $(-\frac{\pi}{2}|\frac{\pi^2}{4})$, etc. This results in a curve that
is called \highlight{graph of $f$} and that is denoted by the symbol $G_f$:

% grund1
\MTikzAuto{%
\begin{tikzpicture}
%Koordinatensystem
\draw[->,color=black] (-4,0) -- (3.9,0);
\foreach \x in {-3,-2,-1,1,2,3}
\draw[shift={(\x,0)},color=black] (0pt,2pt) -- (0pt,-2pt) node[below] {\footnotesize $\x$};
\draw[->,color=black] (0,-0.5) -- (0,4.7);
\foreach \y in {1,2,3,4}
\draw[shift={(0,\y)},color=black] (2pt,0pt) -- (-2pt,0pt) node[left] {\footnotesize $\y$};
\draw[color=black] (0pt,-10pt) node[right] {\footnotesize $0$};
%Achsenbeschriftung
\draw (4,0) node[anchor=north west] {$x$};
\draw (-1,5.2) node[anchor=north west] {$f(x)$};
% Graph (x,x^2)
\draw[smooth,samples=50,domain=-2.1:2.1, line width=2pt,color=green] plot(\x,{(\x)^(2.0)});
%Ausgew�hlte Elemente des Definitionsbereiches:
\foreach \x in {-1.57, -1, 1, 2}{
	\fill [color = cyan](\x,0.0) circle (1.5pt);
	% f(x):
	\fill [color = orange] (0, \x*\x) circle (1.5pt);
	%Punktepaare
	\draw [fill=green] (\x,\x*\x) circle (1.5pt);
	%kreuzlinien
	\draw [dashed, color = cyan] (\x,0)--(\x, \x*\x);
	\draw [dashed, color = orange] (0,\x*\x)--(\x, \x*\x);
}
% Beschriftung:
\begin{scriptsize}
  \draw (-1.6,2.4) node[anchor = east]{$(-\frac{\pi}{2}| \frac{\pi^2}{4})$};
	\draw (-1,1) node[anchor = east]{$(-1| 1)$};
	\draw (1,1) node[anchor = west]{$(1| 1)$};
	\draw (2,4) node[anchor = west]{$(2| 4)$};
\end{scriptsize}
\draw (-2,4) node[color = green, anchor = east]{$G_f:$ $\color{black} ( \color{blue} x \color{black}| \color{orange} f(x)\color{black})$};
\end{tikzpicture}
}%


If we consider the function
\[
 \function{f}{\N}{\Q}{n}{\frac{n}{2}}
\]
from Section~\MNRef{sec:zuordnungen} and its graph

% grund2 f: \N\rightarrow \R, n \mapsto n/2
\MTikzAuto{%
\begin{tikzpicture}
%Koordinatensystem
\draw[->,color=black] (-4,0) -- (3.9,0);
\foreach \x in {-3,-2,-1,1,2,3}
\draw[shift={(\x,0)},color=black] (0pt,2pt) -- (0pt,-2pt) node[below] {\footnotesize $\x$};
\draw[->,color=black] (0,-2) -- (0,2);
\foreach \y in {-1, 1}
\draw[shift={(0,\y)},color=black] (2pt,0pt) -- (-2pt,0pt) node[left] {\footnotesize $\y$};
\draw[color=black] (0pt,-10pt) node[right] {\footnotesize $0$};
%Achsenbeschriftung
\draw (4,0) node[anchor=north west] {$n$};
\draw (-1,2) node[anchor=north west] {$f(n)$};
%älle" Elemente des Definitionsbereiches:
\foreach \n in {-3,-2,-1, 0, 1,2,3}{
	\fill [color = cyan](\n,0.0) circle (1.5pt);
	% f(x):
	\fill [color = orange] (0, \n/2) circle (1.5pt);
	%Punktepaare
	\draw [color=green, fill=green] (\n,\n/2) circle (1.5pt);
	%kreuzlinien
	\draw [dashed, color = cyan] (\n,0)--(\n, \n/2);
	\draw [dashed, color = orange] (0,\n/2)--(\n, \n/2);
}
\draw (3.1,1.5) node[color = green, anchor = west]{$G_f$};
\end{tikzpicture}
}%

then we see that graphs not always have to be continuous curves, but also can consist, as in this case, 
only of single points. 

Now, from the graph we can see many basic properties of the function. Recall the function 
\[
  \function{\Mvarphi}{(0\MIntvlSep 1)}{\R}{y}{3y+1}
\]
with domain $D_{\Mvarphi}=(0\MIntvlSep 1)$ and range $W_{\Mvarphi}=(1\MIntvlSep 4)$ from 
Section~\MNRef{sec:zuordnungen}. If we draw its graph, then we see that the domain and the range 
appear on the horizontal and on the vertical axes, respectively:

% grund3 f: (0,1) --> \R, y \mapsto 3y+1
\MTikzAuto{%
\begin{tikzpicture} 
%Koordinatensystem
\draw[->,color=black] (-1.5,0) -- (2,0);
\foreach \x in {-1,1}
\draw[shift={(\x,0)},color=black] (0pt,2pt) -- (0pt,-2pt) node[below] {\footnotesize $\x$};
\draw[->,color=black] (0,-0.5) -- (0,5);
\foreach \y in {1, 2, 3, 4}
\draw[shift={(0,\y)},color=black] (2pt,0pt) -- (-2pt,0pt) node[left] {\footnotesize $\y$};
\draw[color=black] (0pt,-10pt) node[right] {\footnotesize $0$};
%Achsenbeschriftung
\draw (2,0) node[anchor=north west] {$y$};
\draw (-1,5) node[anchor=north west] {$\Mvarphi(y)$};
%Graph: x \mapsto 3x+1
\draw (0.5,2.5) node[anchor=west, color = green] {$G_{\Mvarphi}$};
\draw[smooth,samples=50,domain=0:1, line width=2pt,color=green] plot(\x,{3*(\x)+1});
\draw[color=green, fill=white] (1,4) circle (1.5pt);
%Definitionsbereich
\draw (0.5,0) node[anchor=south, color = cyan] {$D_{\Mvarphi}$};
\draw [line width=2.0pt,color=cyan] (0,0)-- (1,0);
\draw [color = blue, fill = white]  (0,0) circle (1.5pt);
\draw [color = blue, fill = white]  (1,0) circle (1.5pt);
% Wertebereich
\draw (0,2.5) node[anchor=east, color = orange] {$W_{\Mvarphi}$};
\draw [line width=2.0pt,color=orange] (0,1)-- (0,4);
\draw [color = orange, fill = white]  (0,1) circle (1.5pt);
\draw [color = orange, fill = white]  (0,4) circle (1.5pt);
\end{tikzpicture}
}%

\begin{MExercise}
Consider again the graph of the function
\[
 \function{f}{\R}{\R}{x}{x^2\MDFPSpace,} 
\]
indicate domain and range on the the horizontal axis and vertical axes and label them.

\begin{MHint}{Solution}
% grund4
\MTikzAuto{%
\begin{tikzpicture}
%Koordinatensystem
\draw[->,color=black] (-4,0) -- (3.9,0);
\foreach \x in {-3,-2,-1,1,2,3}
\draw[shift={(\x,0)},color=black] (0pt,2pt) -- (0pt,-2pt) node[below] {\footnotesize $\x$};
\draw[->,color=black] (0,-0.5) -- (0,4.7);
\foreach \y in {1,2,3,4}
\draw[shift={(0,\y)},color=black] (2pt,0pt) -- (-2pt,0pt) node[left] {\footnotesize $\y$};
\draw[color=black] (0pt,-10pt) node[right] {\footnotesize $0$};
%Achsenbeschriftung
\draw (4,0) node[anchor=north west] {$x$};
\draw (-1,5.2) node[anchor=north west] {$f(x)$};
% Graph (x,x^2)
\draw[smooth,samples=50,domain=-2.1:2.1] plot(\x,{(\x)^(2.0)});
%Definitionsbereich
\draw (1,0) node[anchor=south, color = cyan] {$D_f$};
\draw [color=cyan] (-4,0) -- (3.9,0);
% Wertebereich
\draw (0,2.5) node[anchor=east, color = orange] {$W_f$};
\draw [color=orange] (0,0)-- (0,4.7);
\fill [color = orange]  (0,0) circle (1.5pt);
\end{tikzpicture}
}%

$D_f=\R$, $W_f=[0\MIntvlSep \infty)$
\end{MHint}
\end{MExercise}

Furthermore, the property of uniqueness of functions can be seen from the graph. To convince 
ourselves of this, we realise that a curve as shown in the figure below 


% notfunction
\MTikzAuto{%
\begin{tikzpicture}
% Koordinatensystem
\node (xMAX) at (5,0){};
\node (yMAX) at (0,2.8){};
\draw[->,color=black] (-3,0) -- (xMAX);
\foreach \x in {-2,-1,1,2,3,4}
\draw[shift={(\x,0)},color=black] (0pt,2pt) -- (0pt,-2pt) node[below] {\footnotesize $\x$};
\draw[->,color=black] (0,-2) -- (yMAX);
\foreach \y in {-1,1,2}
\draw[shift={(0,\y)},color=black] (2pt,0pt) -- (-2pt,0pt) node[left] {\footnotesize $\y$};
\draw[color=black] (0pt,-10pt) node[right] {\footnotesize $0$};
% Achsenbeschriftung
\draw (xMAX) node[anchor=north east] {$x$};
\draw (yMAX) node[anchor=east] {$y$};
% notfunktion
\draw [rotate around={90:(0,0)}, 
			smooth,samples=100,
			domain=-1.5:1.9] 
			plot(\x,{((\x)-1.0)*(\x)*((\x)+1.0)-2});
% Punkte
\fill [color = red]  (2,1) circle (1.5pt);
\fill [color = red]  (2,0) circle (1.5pt);
\fill [color = red]  (2,-1) circle (1.5pt);
% x
\begin{scriptsize}
	\draw (2,0) circle node[anchor = north west ](x){$x$};
	% y1,y2,y3
	\draw (0,1) circle  node[anchor = west ]{$y_1$};
	\draw (0,0) circle  node[anchor = south west ]{$y_2$};
	\draw (0,-1) circle node[anchor = west ]{$y_3$};
\end{scriptsize}
\draw [color = red, dotted] (2,-1)--(2,0) --(2,1);
\end{tikzpicture}
}%

cannot be a graph of a function. For a single $x$-value in the domain, several values 
$y_1,y_2,y_3$  had to exist in the range. Graphs of functions indicate 
the uniqueness by the fact that they ``cannot reverse in the horizontal direction''.

Another important property of a graph is its \modsemph{growing behaviour}. Let us consider the function
\[
 \function{f}{\R}{\R}{x}{x^2}
\]
and its graph.

%GRAPH: mon1:
\MTikzAuto{%
\begin{tikzpicture} 
%Koordinatensystem
\node (xMAX) at (2.8,0){};
\node (yMAX) at (0,4.8){};
%Achsen
\draw[->,color=green, line width=2pt] (0,0) -- (3,0);
\draw[->,color=red, line width=2pt] (-3,0) -- (-0.05,0);
\draw[->,color=black] (-2.5,0) -- (xMAX);
\foreach \x in {-2,-1,1,2}
\draw[shift={(\x,0)},color=black] (0pt,2pt) -- (0pt,-2pt) node[below] {\footnotesize $\x$};
\draw[->,color=black] (0,-0.5) -- (yMAX);
\foreach \y in {1,2,3,4}
\draw[shift={(0,\y)},color=black] (2pt,0pt) -- (-2pt,0pt) node[left] {\footnotesize $\y$};
\draw[color=black] (0pt,-10pt) node[right] {\footnotesize $0$};
%Achsenbeschriftung
\draw (xMAX) node[anchor=north east] {$x$};
\draw (yMAX) node[anchor=east] {$f(x)$};
% Graph (x,x^2)
\draw[smooth,samples=50,domain=-2.1:2.1] plot(\x,{(\x)^(2.0)});
%Markierungen
\draw (2.1,4) node[right] {$G_f$};
\draw[color = black, fill = black]  (0,0) circle (1.5pt);
\draw[->,color=green] (1.5,1) -- (2.5,3);
\draw[color=green] (2,2) node[right] {rises};
\draw[->,color=red] (-2.5,3) -- (-1.5,1);
\draw[color=red] (-2,2) node[left] {falls};
\end{tikzpicture}
}%

On the horizontal axis in this graph we see two regions in which the \modsemph{growing behaviour} of the 
graph is different. In the region of $x$-values with $x\in(-\infty\MIntvlSep 0)$ the graph 
\modsemph{falls}, i.e. if the $x$ values increase, then the corresponding function values on 
the vertical axis decrease. In the region of $x$-values with $x\in(0\MIntvlSep \infty)$ we 
observe the contrary behaviour. For increasing $x$-values the corresponding function values also increase: the graph \modsemph{rises}. At the $x$-value $0\in\R$ the region of decreasing growing behaviour merges into 
the region of increasing growing behaviour. Such values will be of particular importance for the study of vertices 
in Section~\MNRef{sec:polynome} and for the determination of \MSRef{VBKM07_Kurvendiskussion}{extreme values} 
in Module~\MNRef{VBKM07}. 

These two properties are referred to as \highlight{strictly decreasing} and \highlight{strictly increasing}, respectively, 
and they are defined mathematically as follows:

\begin{MInfo}
\begin{itemize}
 \item If for $x_1<x_2$ in a subset of the domain of a function $f$ we always have $f(x_1)>f(x_2)$, the function $f$ is 
  said to be strictly decreasing for every $x$ in this subset.

%GRAPH: mon2a:
\MTikzAuto{%
\begin{tikzpicture} 
%Koordinatensystem
\node (xMAX) at (3.8,0){};
\node (yMAX) at (0,3.8){};
%Achsen
\draw[->,color=black] (-.5,0) -- (xMAX);
\draw[->,color=black] (0,-.5) -- (yMAX);
\draw[color=black] (0pt,-10pt) node[right] {\footnotesize $0$};
%Achsenbeschriftung
\draw (xMAX) node[anchor=north east] {$x$};
\draw (yMAX) node[anchor=east] {$f(x)$};
% Graph (x,x^2)
\draw[smooth,samples=150,domain=0.5:2.8] plot(\x,{-sin(\x r)-\x +3.5});
\draw (0.5,2.52) node[anchor=south] {$G_f$};
%Markierungen
\draw[color = red, fill = red]  (1,1.659) circle (1.5pt);
\draw[color = red, fill = red]  (2,0.591) circle (1.5pt);
\draw[color = red, dotted] (1,1.659) -- (1,0);
\draw[color = red, dotted] (2,0.591) -- (2,0);
\draw[color = red, dotted] (1,1.659) -- (0,1.659);
\draw[color = red, dotted] (2,0.591) -- (0,0.591);
\draw[shift={(1,0)},color=red] (0pt,2pt) -- (0pt,-2pt) node[below] {\footnotesize{$x_1$}};
\draw[shift={(2,0)},color=red] (0pt,2pt) -- (0pt,-2pt) node[below] {\footnotesize{$x_2$}};
\draw[shift={(0,1.659)},color=red] (2pt,0pt) -- (-2pt,0pt) node[left] {\footnotesize{$f(x_1)$}};
\draw[shift={(0,0.591)},color=red] (2pt,0pt) -- (-2pt,0pt) node[left] {\footnotesize{$f(x_2)$}};
\draw[color=red] (1.5,-0.02) node[below] {\footnotesize{$<$}};
\draw[color=red] (-0.3,1.125) node[left] {\footnotesize{$\vee$}};
\end{tikzpicture}  
}%

 \item If for $x_1<x_2$ in a subset of the domain of a function $f$ we always have $f(x_1)<f(x_2)$, the function $f$ is 
  said to be strictly increasing for every $x$ in this subset.

%GRAPH: mon2b:
\MTikzAuto{%
\begin{tikzpicture} 
%Koordinatensystem
\node (xMAX) at (3.8,0){};
\node (yMAX) at (0,3.8){};
%Achsen
\draw[->,color=black] (-.5,0) -- (xMAX);
\draw[->,color=black] (0,-.5) -- (yMAX);
\draw[color=black] (0pt,-10pt) node[right] {\footnotesize $0$};
%Achsenbeschriftung
\draw (xMAX) node[anchor=north east] {$x$};
\draw (yMAX) node[anchor=east] {$f(x)$};
% Graph (x,x^2)
\draw[smooth,samples=150,domain=0.5:2.8] plot(\x,{sin(\x r)+\x-.5});
\draw (2.8,2.52) node[anchor=south] {$G_f$};
%Markierungen
\draw[color = olive, fill = olive]  (1,1.3415) circle (1.5pt);
\draw[color = olive, fill = olive]  (2,2.409) circle (1.5pt);
\draw[color = olive, dotted] (1,1.3415) -- (1,0);
\draw[color = olive, dotted] (2,2.409) -- (2,0);
\draw[color = olive, dotted] (1,1.3415) -- (0,1.3415);
\draw[color = olive, dotted] (2,2.409) -- (0,2.409);
\draw[shift={(1,0)},color=olive] (0pt,2pt) -- (0pt,-2pt) node[below] {\footnotesize{$x_1$}};
\draw[shift={(2,0)},color=olive] (0pt,2pt) -- (0pt,-2pt) node[below] {\footnotesize{$x_2$}};
\draw[shift={(0,1.3415)},color=olive] (2pt,0pt) -- (-2pt,0pt) node[left] {\footnotesize{$f(x_1)$}};
\draw[shift={(0,2.409)},color=olive] (2pt,0pt) -- (-2pt,0pt) node[left] {\footnotesize{$f(x_2)$}};
\draw[color=olive] (1.5,-0.02) node[below] {\footnotesize{$<$}};
\draw[color=olive] (-0.3,1.875) node[left] {\footnotesize{$\vee$}};
\end{tikzpicture}
}%

\end{itemize}
\end{MInfo}

This is true for all functions that will be investigated throughout this module. Often, the described 
monotonicity properties are only true for certain regions in the functions's domain, as we have seen for
the case of the standard parabola. However functions also exist which only have one of 
the monotonicity properties for the entire domain (see Example~\MNRef{bsp1:monoton} below). In this 
case the entire function is called strictly increasing or strictly decreasing. A function that is either strictly decreasing or strictly increasing is simply called 
\highlight{strictly monotonic}.

A further example shows how strict monotonicity can be checked explicitly for a function by solving 
\MSRef{M03_Ungleichungen}{inequalities} from Module~\MRef{VBKM03}.

\begin{MExample}\MLabel{bsp1:monoton}%

Consider the function
\[
 \function{h}{\R}{\R}{x}{-\frac{1}{2}x+1}.
\]
Determine whether the function $h$ is either strictly increasing or 
strictly decreasing.

First, we choose two arbitrary numbers $x_1,x_2\in D_h=\R$ for which
\[
 x_1<x_2 \MDFPeriod
\]
By \MSRef{VBKM03_AequivalenzumformungenUngleichungen}{equivalent transformations} of inequalities, we can 
transform $x_1<x_2$ either to $h(x_1)<h(x_2)$ or to $h(x_1)>h(x_2)$, and from this we can conclude that 
$h$ is strictly increasing or strictly decreasing. We apply the following equivalent transformations to
the inequality $x_1<x_2$:
\[
 x_1<x_2\,|\cdot(-\frac{1}{2})\quad\Leftrightarrow\quad -\frac{1}{2}x_1>-\frac{1}{2}x_2 \MDFPeriod
\]
Furthermore, we add $+1$ to the mapping rule. Thus, we obtain
\[
 -\frac{1}{2}x_1>-\frac{1}{2}x_2\,|+1\quad\Leftrightarrow\quad -\frac{1}{2}x_1+1>-\frac{1}{2}x_2+1\quad\Leftrightarrow\quad h(x_1)>h(x_2) \MDFPeriod
\]
Since $h(x_1)>h(x_2)$, the function $h$ is strictly decreasing. This can also be seen from the graph of~$h$:

%GRAPH: mon3:
\MTikzAuto{%
\begin{tikzpicture} 
%Koordinatensystem
\node (xMAX) at (4.8,0){};
\node (yMAX) at (0,2.8){};
%Achsen
\draw[->,color=black] (-1.5,0) -- (xMAX);
\foreach \x in {-1,1,2,3,4}
\draw[shift={(\x,0)},color=black] (0pt,2pt) -- (0pt,-2pt) node[below] {\footnotesize $\x$};
\draw[->,color=black] (0,-2.5) -- (yMAX);
\foreach \y in {-2,-1,1,2}
\draw[shift={(0,\y)},color=black] (2pt,0pt) -- (-2pt,0pt) node[left] {\footnotesize $\y$};
\draw[color=black] (0pt,-10pt) node[right] {\footnotesize $0$};
%Achsenbeschriftung
\draw (xMAX) node[anchor=north east] {$x$};
\draw (yMAX) node[anchor=east] {$h(x)$};
% Graph (x,x^2)
\draw[smooth,samples=150,domain=-1:4] plot(\x,{-0.5*\x+1});
\draw (-1,1.5) node[anchor=east] {$G_h$};
%Markierung
\draw[->,color=red] (1,1.2) -- (3,0.2);
\draw[color=red] (2,1) node[anchor=west] {falls};
\end{tikzpicture}
}%

\end{MExample}

\begin{MExercise}\MLabel{ex1:monoton}
Use equivalent transformations to check explicitly that the function
\[
 \function{\eta}{\R}{\R}{x}{2x+2}
\]
is strictly increasing.
\begin{MHint}{Solution}
We have
\[
 x_1<x_2\,|\cdot2\quad\Leftrightarrow\quad2x_1<2x_2\,|+2\quad\Leftrightarrow\quad2x_1+2<2x_2+2\quad\Leftrightarrow\quad\eta(x_1)<\eta(x_2) \MDFPSpace,
\]
hence the function $\eta$ is strictly increasing.
\end{MHint}
\end{MExercise}


\end{MXContent}

\MSubsection{Linear Functions and Polynomials}
\MLabel{VBKM06_linear}

\begin{MIntro}
\MDeclareSiteUXID{VBKM06_EinfacheFunktionen_Intro}
In this section we study the following types of functions: constant functions, linear functions, linear affine 
functions, monomials and polynomials.
\end{MIntro}


\begin{MXContent}{Constant Functions and the Identity}{Constants and Identity}{STD}
\MDeclareSiteUXID{VBKM06_EinfacheFunktionen_Konstanten}

so-called constant functions assign to every number in the domain $\R$ \modsemph{exactly the same} 
constant number in the target set $\R$, e.g. the constant number $2$, in the following way:
\[
 \function{f}{\R}{\R}{x}{2 \MDFPeriod}
\]
%BILD: venn-const
\MTikzAuto{%
\begin{tikzpicture}
%venn.Definitionsbereich
\draw (0,0) ellipse (1cm and 1.5cm);
\draw (-0.5,1.8) node[auto] {$\mathbb{R}$};
% x0, x1, x2
\draw(0.0, 1) node (x1){$-1$};
\draw (0.3,-1) node[auto](x2){$2$} ;
\draw (-0.3,-0.2)node[auto](xh){$\frac{1}{2}$};
\draw (0.6,0.2) node(xp) {$\pi$};
\draw (-.5,0.45) node {$\dots$};
%venn.Wertebereich
\draw (5,0) ellipse (1 cm and 1.5cm);
\draw (4.5,1.8) node[auto] {$\mathbb{R}$};
% f(x0), f(x1), f(x2)
\draw (5,1)  node[auto]{$-1$};
\draw (5,-1)  node[auto](Fx){$2$};
\draw (4.7,-0.2) node {$\frac{1}{2}$};
\draw (5.6,0.2) node {$\pi$};
\draw (4.6,0.45) node {$\dots$};
%Abbildung
\draw [|->] (x1) -- (Fx);
\draw [|->] (x2) -- (Fx);
\draw[|->] (xh) -- (Fx);
\draw[|->] (xp) -- (Fx);
\draw (2.5,0.3) node {$f$};
\end{tikzpicture}
}%
%Abstand:
\hspace{2cm}
%GRAPH: const1 
\MTikzAuto{%
\begin{tikzpicture} 
%Graph: konstant
\draw[color=red] (3,2) node[anchor=south] {$G_f$};
\draw[color=red] (-1.5,2)--(2.8,2);
%Koordinatensystem
\node (xMAX) at (2.8,0){};
\node (yMAX) at (0,3.5){};
\draw[->,color=black] (-1.5,0) -- (xMAX);
\foreach \x in {-1,1,2}
\draw[shift={(\x,0)},color=black] (0pt,2pt) -- (0pt,-2pt) node[below] {\footnotesize $\x$};
\draw[->,color=black] (0,-0.5) -- (yMAX);
\foreach \y in {1, 2, 3}
\draw[shift={(0,\y)},color=black] (2pt,0pt) -- (-2pt,0pt) node[left] {\footnotesize $\y$};
\draw[color=black] (0pt,-10pt) node[right] {\footnotesize $0$};
%Achsenbeschriftung
\draw (xMAX) node[anchor=north east] {$x$};
\draw (yMAX) node[anchor=east] {$f(x)$};
\end{tikzpicture}
}%

Here, we then have $f(x)=2$ for all $x\in\R$. Hence, the range of this function $f$ consists only 
of the set $W_f=\{2\}\subset\R$.

The identity function on $\R$ is the function that assigns each real number to itself. This is written as follows:
\[
 \function{\Id}{\R}{\R}{x}{x \MDFPeriod}
\]
%BILD: venn-id
\MTikzAuto{%
\begin{tikzpicture}
%venn.Definitionsbereich
\draw (0,0) ellipse (1cm and 1.5cm);
\draw (-0.5,1.8) node[auto] {$\mathbb{R}$};
% x0, x1, x2
\draw(0.0, 1) node (x1){$-1$};
\draw (0.3,-1) node[auto](x2){$2$} ;
\draw (-0.3,-0.2)node[auto](xh){$\frac{1}{2}$};
\draw (0.6,0.2) node(xp) {$\pi$};
\draw (-.5,0.45) node {$\dots$};
%venn.Wertebereich
\draw (5,0) ellipse (1 cm and 1.5cm);
\draw (4.5,1.8) node[auto] {$\mathbb{R}$};
% f(x0), f(x1), f(x2)
\draw (5,1)  node[auto](Fx1){$-1$};
\draw (5,-1)  node[auto](Fx2){$2$};
\draw (4.7,-0.2) node(Fxh) {$\frac{1}{2}$};
\draw (5.6,0.2) node(Fxp) {$\pi$};
\draw (4.6,0.45) node {$\dots$};
%Abbildung
\draw [|->] (x1) -- (Fx1);
\draw [|->] (x2) -- (Fx2);
\draw[|->] (xh) -- (Fxh);
\draw[|->] (xp) -- (Fxp);
\draw (2.5,1.3) node {$\Id$};
\end{tikzpicture}
}%
%Abstand:
\hspace{2cm}
%GRAPH: id1
\MTikzAuto{%
\begin{tikzpicture} 
%Graph: Id
\draw[color=red] (2.5,2.5) node[anchor=south] {$G_{\Id}$};
\draw[color=red] (-0.5,-0.5)--(2.5,2.5);
%Koordinatensystem
\node (xMAX) at (2.8,0){};
\node (yMAX) at (0,3.5){};
\draw[->,color=black] (-1.5,0) -- (xMAX);
\foreach \x in {-1,1,2}
\draw[shift={(\x,0)},color=black] (0pt,2pt) -- (0pt,-2pt) node[below] {\footnotesize $\x$};
\draw[->,color=black] (0,-0.5) -- (yMAX);
\foreach \y in {1, 2, 3}
\draw[shift={(0,\y)},color=black] (2pt,0pt) -- (-2pt,0pt) node[left] {\footnotesize $\y$};
\draw[color=black] (0pt,-10pt) node[right] {\footnotesize $0$};
%Achsenbeschriftung
\draw (xMAX) node[anchor=north east] {$x$};
\draw (yMAX) node[anchor=east] {$\operatorname{Id}(x)$};
\end{tikzpicture}
}%

Here, we then have $\Id(x)=x$ for all $x\in\R$. Hence, the range of $\Id$ is the set 
of real numbers ($W_{\Id}=\R$). Furthermore, the identity function is (obviously) a strictly 
increasing function.
\end{MXContent}


\begin{MXContent}{Linear Functions}{Linear}{STD}\MLabel{sec:linear}
\MDeclareSiteUXID{VBKM06_EinfacheFunktionen_Linear}

Starting from the identity function, more complex functions - so-called \highlight{linear functions} - can be constructed. So, for example, one can think of a function that assigns to every 
real number twice its value or $\pi$ times its value, etc., e.g.
\[
 \function{f}{\R}{\R}{x}{2x}
\]
or
\[
 \function{g}{\R}{\R}{x}{\pi x \MDFPeriod}
\]

%BILD: venn-lin1
\MTikzAuto{%
\begin{tikzpicture}
%venn.Definitionsbereich
\draw (0,0) ellipse (1cm and 1.5cm);
\draw (-0.5,1.8) node[auto] {$\mathbb{R}$};
% x0, x1, x2
\draw (0.0, 0.8) node(x1){$-1$};
\draw (0.6,0.2) node(xn) {$0$};
\draw (-0.3,-0.2)node(xh){$\frac{1}{2}$};
\draw (0.3,-0.9) node(x2){$2$} ;
\draw (-0.4,0.3) node {$\dots$};
%venn.Wertebereich
\draw (5,0) ellipse (1 cm and 1.5cm);
\draw (4.5,1.8) node[auto] {$\mathbb{R}$};
% f(x0), f(x1), f(x2)
\draw (5,1)  node(Fx1){$-2$};
\draw (5.6,0.2) node(Fxn) {$0$};
\draw (4.7,-0.4) node(Fxh) {$1$};
\draw (5,-1.2)  node(Fx2){$4$};
\draw (5.3,-0.6) node {$\dots$};
%Abbildung
\draw [|->] (x1) -- (Fx1);
\draw [|->] (x2) -- (Fx2);
\draw[|->] (xn) -- (Fxn);
\draw[|->] (xh) -- (Fxh);
\draw (2.5,1.3) node {$f$};
\end{tikzpicture}
}%
\hspace{2cm}
%GRAPH: lin1
\MTikzAuto{%
\begin{tikzpicture} 
%Koordinatensystem
% x-Achse
\node (xMAX) at (2.8,0){};
\draw[->,color=black] (-0.5,0) -- (xMAX);
\foreach \x in {1,2}
\draw[shift={(\x,0)},color=black] (0pt,2pt) -- (0pt,-2pt) node[below] {\footnotesize $\x$};
% y-Achse
\node (yMAX) at (0,4.8){};
\draw[->,color=black] (0,-0.5) -- (yMAX);
\foreach \y in {1, 2, 3, 4}
\draw[shift={(0,\y)},color=black] (2pt,0pt) -- (-2pt,0pt) node[left] {\footnotesize $\y$};
\draw[color=black] (0pt,-10pt) node[right] {\footnotesize $0$};
%Achsenbeschriftung
\draw (xMAX) node[anchor=north east] {$x$};
\draw (yMAX) node[anchor=east] {$f(x)$};
%Graph: x->2x
\draw[color=red] (-0.2,-0.4)--(2.2,4.4);
\draw[color=red] (2.2,4.4) node[anchor=south] {$G_f$};
\end{tikzpicture}
}%


%BILD: venn-lin2
\MTikzAuto{%
\begin{tikzpicture}
%venn.Definitionsbereich
\draw (0,0) ellipse (1cm and 1.5cm);
\draw (-0.5,1.8) node[auto] {$\mathbb{R}$};
% x0, x1, x2
\draw (0.0, 0.8) node(x1){$-1$};
\draw (0.6,0.2) node(xn) {$0$};
\draw (-0.3,-0.2)node(xh){$\frac{1}{2}$};
\draw (0.3,-0.9) node(x2){$2$} ;
\draw (-0.4,0.3) node {$\dots$};
%venn.Wertebereich
\draw (5,0) ellipse (1 cm and 1.5cm);
\draw (4.5,1.8) node[auto] {$\mathbb{R}$};
% f(x0), f(x1), f(x2)
\draw (5,1)  node(Fx1){$-\pi$};
\draw (5.6,0.2) node(Fxn) {$0$};
\draw (4.7,-0.4) node(Fxh) {$\frac{\pi}{2}$};
\draw (5,-1.2)  node(Fx2){$2\pi$};
\draw (5.3,-0.6) node {$\dots$};
%Abbildung
\draw [|->] (x1) -- (Fx1);
\draw [|->] (x2) -- (Fx2);
\draw[|->] (xn) -- (Fxn);
\draw[|->] (xh) -- (Fxh);
\draw (2.5,1.3) node {$g$};
\end{tikzpicture}
}%
\hspace{2cm}
%GRAPH: lin2
\MTikzAuto{%
\begin{tikzpicture} 
%Koordinatensystem
% x-Achse
\node (xMAX) at (2.8,0){};
\draw[->,color=black] (-0.5,0) -- (xMAX);
\foreach \x in {1,2}
\draw[shift={(\x,0)},color=black] (0pt,2pt) -- (0pt,-2pt) node[below] {\footnotesize $\x$};
% y-Achse
\node (yMAX) at (0,4.8){};
\draw[->,color=black] (0,-0.5) -- (yMAX);
\foreach \y in {1, 2, 3, 4}
\draw[shift={(0,\y)},color=black] (2pt,0pt) -- (-2pt,0pt) node[left] {\footnotesize $\y$};
\draw[color=black] (0pt,-10pt) node[right] {\footnotesize $0$};
%Achsenbeschriftung
\draw (xMAX) node[anchor=north east] {$x$};
\draw (yMAX) node[anchor=east] {$g(x)$};
%Graph: x->pi*x
\draw[color=red] (-0.1,-0.3)--(1.4,4.5);
\draw[color=red] (1.4,4.5) node[anchor=south] {$G_g$};
\end{tikzpicture}
}%

Hence, all linear functions (except for the zero function, see below) also have the entire set 
of real numbers as their ranges ($W_f,W_g=\R$). The factor that multiplies each real number in such 
a linear function is called \highlight{slope} of the linear function. Often, also for linear functions 
one does not like to specify a certain function with a specific slope, but an arbitrary function 
with an arbitrary slope $m\in\R$:
\[
 \function{f}{\R}{\R}{x}{m x \MDFPeriod}
\]

%GRAPH: lin3
\MTikzAuto{%
\begin{tikzpicture} 
%Koordinatensystem
% x-Achse
\node (xMAX) at (2.8,0){};
\draw[->,color=black] (-0.5,0) -- (xMAX);
\foreach \x in {1}
\draw[shift={(\x,0)},color=black] (0pt,2pt) -- (0pt,-2pt) node[below] {\footnotesize $\x$};
% y-Achse
\node (yMAX) at (0,1.8){};
\draw[->,color=black] (0,-0.5) -- (yMAX);
\draw[shift={(0,0.5)},color=black] (2pt,0pt) -- (-2pt,0pt) node[left] {$m$};
\draw[color=black] (0pt,-10pt) node[right] {\footnotesize $0$};
%Achsenbeschriftung
\draw (xMAX) node[anchor=north east] {$x$};
\draw (yMAX) node[anchor=east] {$f(x)$};
%Graph: x->m*x
\draw[color=red] (-0.4,-0.2)--(2.2,1.1);
\draw[color=red] (2.2,1.1) node[anchor=south] {$G_f$};
%Markierungen
\draw[color=red, dotted] (1,.5) -- (1,0); 
\draw[color=red, dotted] (1,.5) -- (0,.5);
\end{tikzpicture}
}%

Where does the term slope of a linear function come from? If the difference in height by which the graph is rising vertically 
is divided by the corresponding length in the horizontal direction, then one obtains the slope $m$. So $m = \frac{f(x_2) - f(x_1)}{x_2 - x_1}$ for $x_1 < x_2$.%%%


\begin{MInfo}
A linear function
\[
 \function{f}{\R}{\R}{x}{m x} 
\]
is strictly increasing if and only if its slope is positive, i.e. $m>0$; 
and it is strictly decreasing if and only if its slope is negative, i.e. $m<0$.
\end{MInfo}


\begin{MExercise}
Which of the linear functions above has the slope $m=1$?

\begin{MHint}{Solution}
We have $f(x)=1\cdot x=x=\Id(x)$, i.e. the identity function.
\end{MHint}
\end{MExercise}

\begin{MExercise}
Which of the linear functions above has the slope $m=0$?
\begin{MHint}{Solution}
We have $f(x)=0\cdot x=0$, i.e. the constant function that is always $0$. 
\end{MHint}
\end{MExercise}

\end{MXContent}
 
\begin{MXContent}{Linear Affine Functions}{Affine}{STD}
\MLabel{VBKM06_sec:linear-affin}
\MDeclareSiteUXID{VBKM06_EinfacheFunktionen_Affin}

Combining linear functions with constant functions results in so-called \highlight{linear affine functions}.
These are the sum of a linear function and a constant function. Generally, without any specification for 
the slope ($m\in\R$) this is written as follows:
\[
 \function{f}{\R}{\R}{x}{m x+c \MDFPeriod}
\]

%GRAPH: affin1
\MTikzAuto{%
\begin{tikzpicture} 
%Koordinatensystem
% x-Achse
\node (xMAX) at (3.8,0){};
\draw[->,color=black] (-0.5,0) -- (xMAX);
\foreach \x in {1,2}
\draw[shift={(\x,0)},color=black] (0pt,2pt) -- (0pt,-2pt) node[below] {\footnotesize $\x$};
% y-Achse
\node (yMAX) at (0,3.8){};
\draw[->,color=black] (0,-0.5) -- (yMAX);
\draw[shift={(0,1.5)},color=black] (2pt,0pt) -- (-2pt,0pt) node[left] {$c$};
\draw[shift={(0,2)},color=black] (2pt,0pt) -- (-2pt,0pt) node[left] {$c+m$};
\draw[color=black] (0pt,-10pt) node[right] {\footnotesize $0$};
%Achsenbeschriftung
\draw (xMAX) node[anchor=north east] {$x$};
\draw (yMAX) node[anchor=east] {$f(x)$};
%Graph: x->m*x+c
\draw[color=red] (-1,1)--(3,3);
\draw[color=red] (3,3) node[anchor=south] {$G_f$};
%Markierungen
\draw[color=red,dotted] (1,0) -- (1,2);
\draw[color=red,dotted] (0,2) -- (1,2);
\end{tikzpicture}	
}%

The graphs of linear affine functions are also called \highlight{lines}. For linear affine functions,
the constant $m$ is still called slope, and the constant $c\in\R$ is called \highlight{$y$-intercept}.
The reason for this term is as follows: if the intersection point of the graph of the linear affine function
with the vertical axis is considered, then this point has the distance $c$ from the origin (see figure above). 
So, for the linear affine function shown in the figure below
\[
 \function{f}{\R}{\R}{x}{-2x-1}
\]

%GRAPH: affin2
\MTikzAuto{%
\begin{tikzpicture} 
%Koordinatensystem
% x-Achse
\node (xMAX) at (2.8,0){};
\draw[->,color=black] (-1.5,0) -- (xMAX);
\foreach \x in {-1,1,2}
\draw[shift={(\x,0)},color=black] (0pt,2pt) -- (0pt,-2pt) node[below] {\footnotesize $\x$};
% y-Achse
\node (yMAX) at (0,1.8){};
\draw[->,color=black] (0,-5) -- (yMAX);
\foreach \y in {-5,-4,-3,-2,-2,1}
\draw[shift={(0,\y)},color=black] (2pt,0pt) -- (-2pt,0pt) node[left] {\footnotesize $\y$};
\draw[color=black] (0pt,-10pt) node[right] {\footnotesize $0$};
\draw[shift={(0,-1)},color=red] (2pt,0pt) -- (-2pt,0pt) node[left] {\footnotesize $c=-1$};
%Achsenbeschriftung
\draw (xMAX) node[anchor=north east] {$x$};
\draw (yMAX) node[anchor=east] {$f(x)$};
%Graph: x-> -2x-1
\draw[color=red] (-1,1)--(2,-5);
\draw[color=red] (2,-5) node[anchor=north] {$G_f$};
\end{tikzpicture}
}%

we have the slope $m=-2$ and the $y$-intercept $c=-1$. The $y$-intercept is the value of the function 
at $x=0$ and hence given by 
\[
 c=f(0)=-2\cdot 0-1 = -1 \MDFPeriod
\]

\begin{MExercise}
Find the slope and the $y$-intercept of the function
\[
 \function{f}{\R}{\R}{x}{\pi x-42 \MDFPeriod}
\]
\begin{MHint}{Solution}
Slope: $\pi$ and $y$-intercept: $-42$
\end{MHint}
\end{MExercise}


\begin{MExercise}
Which functions are the linear affine functions that have slope $m=0$, and 
which are the ones with $y$-intercept $c=0$?
\begin{MHint}{Solution}
If $m=0$, then $f(x)=0\cdot x+c=c$. 
Hence, the constant functions are the ones that have slope $0$. A zero 
$y$-intercept, i.e. $c=0$, implies $f(x)=m x+0=m x$. Hence, in this case
we obtain exactly the linear functions.
\end{MHint}
\end{MExercise}


\end{MXContent}
 
\begin{MXContent}{Absolute Value Functions}{Absolute Value}{STD}
\MLabel{VBKM06_sec:betrag}
\MDeclareSiteUXID{VBKM06_EinfacheFunktionen_Betrag}
In Module~\MNRef{VBKM02} the \MSRef{VBKM02_Betrag}{absolute value} of a real number $x$ was introduced in the following way:
\[
 |x| = \begin{MCaseEnv} x & \text{for}\;x\geq 0 \\ -x & \text{for}\;x<0 \MDFPeriod \end{MCaseEnv}
\]
In the context of this module, the absolute value can be regarded as a function. This results in 
the \MEntry{absolute value function}{absolute value function}:
\[
 \function{b}{\R}{\R}{x}{|x| \MDFPeriod}
\]

\begin{MExercise}
What is the range $W_b$ of the absolute value function $b$?
\begin{MHint}{Solution}
Since $b(x)=|x|\geq0$ for all numbers $x$ in $D_b=\R$, we have  $W_b=[0\MIntvlSep \infty)$. 
\end{MHint}
\end{MExercise}

Due to the definition by cases
\[
 b(x)=|x|=\begin{MCaseEnv} x & \text{for}\;x\geq0 \\ -x & \text{for}\;x<0 \MDFPSpace , \end{MCaseEnv}
\]
the absolute value function is an example of a \modsemph{piecewise defined} function. If absolute values are 
defined according to different cases, then it is also said that the absolute value is \highlight{resolved}.
Then, the graph of the absolute value function $b$ looks as follows:

%GRAPH: abs1
\MTikzAuto{%
\begin{tikzpicture} 
%Koordinatensystem
% x-Achse
\node (xMAX) at (2.8,0){};
\draw[->,color=black] (-2.5,0) -- (xMAX);
\foreach \x in {-2,-1,1,2}
\draw[shift={(\x,0)},color=black] (0pt,2pt) -- (0pt,-2pt) node[below] {\footnotesize $\x$};
% y-Achse
\node (yMAX) at (0,2.8){};
\draw[->,color=black] (0,-0.5) -- (yMAX);
\foreach \y in {1,2}
\draw[shift={(0,\y)},color=black] (2pt,0pt) -- (-2pt,0pt) node[left] {\footnotesize $\y$};
\draw[color=black] (0pt,-10pt) node[right] {\footnotesize $0$};
%Achsenbeschriftung
\draw (xMAX) node[anchor=north east] {$x$};
\draw (yMAX) node[anchor=east] {$b(x)$};
%Graph: x-> |x|
\draw[color=red] (-2,2)--(0,0)--(2,2);
\draw[color=red] (2,2) node[anchor=south] {$G_b$};
\end{tikzpicture}
}%

One property of the graph of the absolute value function, which most of the more general functions involving absolute
values have in common, is the kink at $x=0$. The absolute value function $b$ defined above is only the simplest case of 
a function involving an absolute value. More complicated examples of functions can be constructed, involving one or several absolute values, e.g.
\[
 \function{f}{\R}{\R}{x}{|2x-1| \MDFPeriod}
\]

For such functions, it is a relevant task to get an idea of how the graph of the function looks. 
To do this, we use the piecewise definition of the absolute value, and the approach is similar to 
the one for the solution of \MSRef{VBKM02_FallBetrag}{absolute value equations} and inequalities. Here,
we demonstrate this approach for the example of the function $f$ defined above:

\begin{MExample}
Consider the function
\[\function{f}{\R}{\R}{x}{|2x-1|}.\]
What does the graph look like?

We calculate:
\[
 f(x)=|2x-1|=\begin{MCaseEnv} 2x-1 & \text{for}\ 2x-1\geq0 \\ -(2x-1) & \text{for}\ 2x-1<0\end{MCaseEnv}=\begin{MCaseEnv} 2x-1 & \text{for}\ x\geq\frac{1}{2} \\ -2x+1 & \text{for}\ x<\frac{1}{2} \MDFPeriod\end{MCaseEnv}
\]
Thus, we obtain a piecewise defined function whose graph is an increasing line with the slope $2$ 
and the $y$-intercept $-1$ for $x$ in the region $x\geq\frac{1}{2}$ and a decreasing line with 
the slope $-2$ and the $y$-intercept $1$ for $x$ in the region $x<\frac{1}{2}$. With this information 
we can  draw the graph of $f$:

%GRAPH: abs2
\MTikzAuto{%
\begin{tikzpicture} 
%Koordinatensystem
% x-Achse
\node (xMAX) at (2.8,0){};
\draw[->,color=black] (-1.5,0) -- (xMAX);
\foreach \x in {-1,1,2}
\draw[shift={(\x,0)},color=black] (0pt,2pt) -- (0pt,-2pt) node[below] {\footnotesize $\x$};
% y-Achse
\node (yMAX) at (0,4.8){};
\draw[->,color=black] (0,-0.5) -- (yMAX);
\foreach \y in {1,2, 3, 4}
\draw[shift={(0,\y)},color=black] (2pt,0pt) -- (-2pt,0pt) node[left] {\footnotesize $\y$};
\draw[color=black] (0pt,-10pt) node[right] {\footnotesize $0$};
%Achsenbeschriftung
\draw (xMAX) node[anchor=north east] {$x$};
\draw (yMAX) node[anchor=east] {$f(x)$};
%Graph: x-> |2x-1|
\draw[color=red] (-1.5,4)--(0.5,0)--(2.5,4);
\draw[color=red] (2.5,4) node[anchor=south] {$G_{f}$};
\end{tikzpicture}
}%
\end{MExample}

\begin{MInfo}
\textcolor{red}{CAUTION!} 
If absolute values are resolved as in the calculation in the example above, two important calculation rules 
have to be observed:

\begin{enumerate}
 \item The \modsemph{regions} for the cases are defined by inequalities for the entire 
  \modsemph{expression} between the absolute value bars, here $2x-1\geq 0$ and $2x-1<0$, and \modsemph{not} only 
  by $x\geq 0$ and $x<0$. It is always that way if absolute values are resolved.

 \item For the case $<0$ the \modsemph{entire} expression gets a minus sign. Here, care has to be taken that
  the expression is bracketed. In the example above, we therefore have $-(2x-1)=-2x+1$ and \modsemph{not}
  $-2x-1$. This is always the case if absolute values are resolved.
\end{enumerate}
\end{MInfo}



\begin{MExercise}
Sketch the graph of the function
\[
 \function{\alpha}{\R}{\R}{x}{|-8x+1|-1 \MDFPeriod } 
\]
Moreover, specify its range $W_\alpha$.
\begin{MHint}{Solution}
We have
\[
 \alpha(x)=|-8x+1|-1=\begin{MCaseEnv} -8x+1-1 & \text{for}\ -8x+1\geq0 \\ -(-8x+1)-1 & \text{for}\ -8x+1<0 \end{MCaseEnv}=\begin{MCaseEnv} -8x & \text{for}\ x\leq\frac{1}{8} \\ 8x-2 & \text{for}\ x>\frac{1}{8} \MDFPSpace,\end{MCaseEnv}
\]
hence: 

%GRAPH: abs3
\MTikzAuto{%
\begin{tikzpicture}[x = 2cm] 
%Koordinatensystem
% x-Achse
\node (xMAX) at (1.8,0){};
\draw[->,color=black] (-1.5,0) -- (xMAX);
\foreach \x in {-1,1}
\draw[shift={(\x,0)},color=black] (0pt,2pt) -- (0pt,-2pt) node[below] {\footnotesize $\x$};
% y-Achse
\node (yMAX) at (0,4.5){};
\draw[->,color=black] (0,-1.5) -- (yMAX);
\foreach \y in {-1,1,2, 3, 4}
\draw[shift={(0,\y)},color=black] (2pt,0pt) -- (-2pt,0pt) node[left] {\footnotesize $\y$};
\draw[color=black] (0pt,-10pt) node[left] {\footnotesize $0$};
%Achsenbeschriftung
\draw (xMAX) node[anchor=north east] {$x$};
\draw (yMAX) node[anchor=east] {$\alpha(x)$};
%Graph: x-> |-8x+1|-1
\draw[color=red] (-0.5,4)--(0.125,-1)--(0.75,4);
\draw[color=red] (0.75,4) node[anchor=south] {$G_{\alpha}$};
\draw[shift={(1/8,0)},color=black] (0pt,2pt) -- (0pt,-2pt) node[anchor=north] {\footnotesize $\frac{1}{8}$};
\end{tikzpicture}
}%

Since $|-8x+1|\geq0$, it follows $|-8x+1|-1\geq -1$ which implies $W_\alpha=[-1\MIntvlSep \infty)$.
\end{MHint}
\end{MExercise}


\end{MXContent}


\begin{MXContent}{Monomials}{Monomials}{STD}\MLabel{sec:monome}
\MDeclareSiteUXID{VBKM06_EinfacheFunktionen_Monome}
In addition to the linear affine functions studied in the previous section, we can also
think of functions that assign to every real number a non-negative integer power
of the number. An example is the function
\[
 \function{g}{\R}{\R}{x}{x^2 \MDFPeriod}
\]

% venn-para (siehe venn4)
\MTikzAuto{%
\begin{tikzpicture}[scale = 0.5]
% Definitionsbereich
\draw  (3.5,10.3) node[anchor=north west] {$\mathbb{R}$};
\draw [] (2.7,7.6) ellipse (1.7cm and 2.4cm);
\draw [color = blue](2,6.3)  node[anchor=north west](vierMinus) {-4};
\draw [color = blue](2.1,7.4)  node[anchor=north west](eins) {1};
\draw [color = blue](2.3,8.7)  node[anchor=north west] (vier){4};
\draw [color = blue](2.5,9.5) node[anchor=north west]  {...};
% Wertebereich
\draw (9.8,10.3) node[anchor=north west] {$\mathbb{R}$};
\draw (8.8,7.5) ellipse (1.9cm and 2.6cm);
\draw (8,8) node[anchor=north west] {$\frac{1}{2}$};
\draw (7.4,6.7) node[anchor=north west] {$-\frac{1}{2}$};
\draw (7.1,7.5) node[anchor=north west](Feins) {$1$};
\draw (8.0,9.8) node[anchor=north west] (Fvier){$16$};
\draw (7.3,8.7) node[anchor=north west]{$0$};
\draw (9.0,7.8) node[anchor=north west] {$\pi$};
\draw (9.3,9.3) node[anchor=north west] {$\frac{3}{2}$};
\draw (9.3,6.5) node[anchor=north west] {...};
% Abbildung: 
\draw[|->] (eins) -- (Feins);
\draw[|->] (vier) -- (Fvier);
\draw[|->] (vierMinus) -- (Fvier);
\end{tikzpicture}
}%
\hspace{1cm}
%GRAPH: mon1
\MTikzAuto{%
\begin{tikzpicture} 
%Koordinatensystem
\node (xMAX) at (2.8,0){};
\node (yMAX) at (0,4.8){};
\draw[->,color=black] (-2.5,0) -- (xMAX);
\foreach \x in {-2,-1,1,2}
\draw[shift={(\x,0)},color=black] (0pt,2pt) -- (0pt,-2pt) node[below] {\footnotesize $\x$};
\draw[->,color=black] (0,-0.5) -- (yMAX);
\foreach \y in {1,2,3,4}
\draw[shift={(0,\y)},color=black] (2pt,0pt) -- (-2pt,0pt) node[left] {\footnotesize $\y$};
\draw[color=black] (0pt,-10pt) node[right] {\footnotesize $0$};
%Achsenbeschriftung
\draw (xMAX) node[anchor=north east] {$x$};
\draw (yMAX) node[anchor=east] {$g(x)$};
% Graph (x,x^2)
\draw[color=red,smooth,samples=50,domain=-2.1:2.1] plot(\x,{(\x)^(2.0)});
\draw[color=red] (2.1,4.41) node[anchor=south] {$G_g$};
\end{tikzpicture}
}%

This works for every non-negative integer exponent, and generally this function is written as 
\[
 \function{f}{\R}{\R}{x}{x^n}
\]
with $n\in\N_0$, and it is called a \highlight{monomial}. The exponent $n$ of a monomial is called 
the \highlight{degree} of the monomial. For example, the function $g$ described at the beginning 
of this section is a monomial of degree $2$.

\begin{MExercise}
Which functions are the monomials of degree $1$ and $0$? 

\begin{MHint}{Solution}
Since $x^1=x$ for all $x\in\R$, the identity function  $\Id$ is the monomial of degree $1$. Likewise, 
$x^0=1$ for all $x\in\R$, and thus, the constant function  $f\colon\R\lto\R$, $f(x)=1$ is the monomial 
of degree $0$.
\end{MHint}
\end{MExercise}
The monomial of degree $2$ is called the \highlight{standard parabola}. The monomial 
of degree $3$ is called the \highlight{cubic standard parabola}. The figure below shows 
the graphs of a few monomials.

%GRAPH: mon2
\MTikzAuto{%
\begin{tikzpicture} 
%Koordinatensystem
\node (xMAX) at (2.8,0){};
\node (yMAX) at (0,4.8){};
\draw[->,color=black] (-2.5,0) -- (xMAX);
\foreach \x in {-2,-1,1,2}
\draw[shift={(\x,0)},color=black] (0pt,2pt) -- (0pt,-2pt) node[below] {\footnotesize $\x$};
\draw[->,color=black] (0,-3.5) -- (yMAX);
\foreach \y in {-3,-2,-1,1,2,3,4}
\draw[shift={(0,\y)},color=black] (2pt,0pt) -- (-2pt,0pt) node[left] {\footnotesize $\y$};
\draw[color=black] (0pt,-10pt) node[right] {\footnotesize $0$};
%Achsenbeschriftung
\draw (xMAX) node[anchor=north east] {$x$};
\draw (yMAX) node[anchor=east] {$f(x)$};
\clip(-2.5,-3.5) rectangle (2.5,4.5);
% Monome
\draw[color = cyan,smooth,samples=50,domain=-2.5:2.5] plot(\x,{(\x)}); \draw[color = cyan](-1.4,-1.4)node[left]{$G_x$};
\draw[color = orange, smooth,samples=50,domain=-2.4:2.4] plot(\x,{(\x)^(2.0)}); \draw[color = orange](-1.5,2)node[left]{$G_{x^2}$};
\draw[color = blue, smooth,samples=50,domain=-1.9:1.9] plot(\x,{(\x)^(3.0)}); \draw[color = blue](-1.3,-3)node[left]{$G_{x^3}$};
\draw[color = red, smooth,samples=50,domain=-1.5:1.5] plot(\x,{(\x)^(4.0)}); \draw[color = red](-1.3,3)node[right]{$G_{x^4}$};
\draw[color = black, smooth,samples=50,domain=-1.4:1.4] plot(\x,{(\x)^(5.0)}); \draw[color = black](-1.2,-2.5)node[right]{$G_{x^5}$};
%Punkte
\draw[color=black,fill=black] (1,1) circle (1pt);
\draw[color=black,fill=black] (-1,1) circle (1pt);
\draw[color=black,fill=black] (-1,-1) circle (1pt);
\end{tikzpicture}
}%

On the basis of these graphs, we now summarise some conclusions on monomials: There is a fundamental difference 
between monomials (with the mapping rule $f(x)=x^n$, $n\in\N$) of even and odd degree. 
The range of monomials of an even non-zero degree is always the set $[0\MIntvlSep \infty)$, while monomials
of odd degree have the range $\R$. Furthermore, we always have 
\[
 f(1)=1^n=1 \MDFPSpace,
\]
\[
 f(0)=0^n=0
\]
and
\[
 f(-1)=\MCases{1 & \text{for}\ n\ \text{even} \\ -1 & \text{for}\ n\ \text{odd\MDFPeriod} }
\]
Moreover, we have
\[
 \MCases{x>x^2>x^3>x^4>\dots & \text{for}\ x\in(0\MIntvlSep 1) \\ x<x^2<x^3<x^4<\dots & \text{for}\ x\in(1\MIntvlSep \infty) \MDFPeriod}
\]

\begin{MExercise}
How can our conclusions concerning monomials be seen immediately from the exponent rules?

\begin{MHint}{Solution}

From the exponent rules, we know that $1^n=1$ and $0^n=0$ for arbitrary non-negative 
integers $n$. We have $(-1)^n=1$  if $n$ is an even number and $(-1)^n=-1$ if $n$ is an odd number. 
From this, the described conclusions result for all monomials.
$x>x^2>x^3>x^4>\dots$ for positive $x$ less than $1$ and $x<x^2<x^3<x^4<\dots$ for $x$ greater than $1$ result from the exponent
rules since higher powers of positive numbers less than $1$ will always have decreasing values while, in contrast,
higher powers of positive numbers greater than $1$ will always have increasing values.
\end{MHint}
\end{MExercise}

\end{MXContent}

\begin{MXContent}{Polynomials and Their Roots}{Roots}{STD}\MLabel{sec:polynome}
\MDeclareSiteUXID{VBKM06_EinfacheFunktionen_Polynome}

The monomials considered until now always involved only exactly one power of the 
independent variable. From these monomials we can easily construct more complex functions involving several different powers of the independent variable. These are sums 
of multiples of monomials. They are called \highlight{polynomials}. 
A few examples and their graphs are given below.  

\[
 \begin{array}{ll}
 \function{f_1}{\R}{\R}{x}{2x^3+4x^2-3x+42} & (\text{degree:}\;3) \\[3ex]

 \function{f_2}{\R}{\R}{x}{-x^{101}+3x-14} & (\text{degree:}\;101) \\[3ex]

 \function{f_3}{\R}{\R}{x}{9x^4+9x^3-2x^2-19x} & (\text{degree:}\;4) \\[3ex]

 \function{f_4}{\R}{\R}{x}{x^2+2x+2} & (\text{degree:}\;2) \\[3ex]

 \function{f_5}{\R}{\R}{x}{8x-2} & (\text{degree:}\;1) \\[3ex]

 \function{f_6}{\R}{\R}{x}{13} & (\text{degree:}\;0) \\
 \end{array}
\]

%GRAPHEN: poly1
\MTikzAuto{%
\begin{tikzpicture}[y=0.1cm] 
%Koordinatensystem
\node (xMAX) at (2.5,0){};
\node (yMAX) at (0,60){};
\draw[->,color=black] (-5,0) -- (xMAX);
\foreach \x in {-5, -4, -3, -2, -1,1, 2}
\draw[shift={(\x,0)},color=black] (0pt,2pt) -- (0pt,-2pt) node[below] {\footnotesize $\x$};
\draw[->,color=black] (0,-15) -- (yMAX);
\foreach \y in {-10,10,20,30,40,50}
\draw[shift={(0,\y)},color=black] (2pt,0pt) -- (-2pt,0pt) node[left] {\footnotesize $\y$};
\draw[color=black] (0pt,-10pt) node[right] {\footnotesize $0$};
%Achsenbeschriftung
\draw (xMAX) node[anchor=north east] {$x$};
\draw (yMAX) node[anchor=east] {$f_1(x)$};
%Beschriftung
%\draw(0,38)node[right]{$f_1(x)= 2x^3+ 4x^2 - 3x+ 42$};
% Polynom-Plot: 2*x^3+ 4*x^2 - 3*x+ 42
\clip(-5,-10) rectangle (5,55);
\draw[color=red,smooth,samples=50,domain=-4.5:2] plot(\x,{2*(\x)^3 + 4*(\x)^2 - 3*(\x)+ 42}); 
\draw[color=red] (-3.5,20) node[anchor=east] {$G_{f_1}$};
\end{tikzpicture}
}%
\hspace{1cm}
%GRAPHEN: poly2
\MTikzAuto{%
\begin{tikzpicture}[y=0.4cm] 
% Koordinatensystem
\node (xMAX) at (2.5,0){};
\node (yMAX) at (0,4){};
\draw[->,color=black] (-2,0) -- (xMAX);
\foreach \x in {-1,1,2}
\draw[shift={(\x,0)},color=black] (0pt,2pt) -- (0pt,-2pt) node[below] {\footnotesize $\x$};
\draw[->,color=black] (0,-20) -- (yMAX);
\foreach \y in {-18,-16,-14,-12,-10,-8,-6,-4,-2,2}
\draw[shift={(0,\y)},color=black] (2pt,0pt) -- (-2pt,0pt) node[left] {\footnotesize $\y$};
\draw[color=black] (0pt,-10pt) node[right] {\footnotesize $0$};
% Achsenbeschriftung
\draw (xMAX) node[anchor=north east] {$x$};
\draw (yMAX) node[anchor=east] {$f_2(x)$};
% Beschriftung
%\draw(0,-7)node[right]{$f_2(x)= -x^{101}+ 3x-14$};
% Polynom-Plot: -x^101+ 3x- 14
\clip(-2,-20) rectangle (2,3);
\draw[color=red,smooth,samples=100,domain=-1.05:1.05] plot(\x,{-(\x)^(101) + 3*(\x)-14}); 
\draw[color=red] (1,-12) node[anchor=west] {$G_{f_2}$};
\end{tikzpicture}
}%

%GRAPHEN: poly3
\MTikzAuto{%
\begin{tikzpicture}[y=0.4cm] 
% Koordinatensystem
% x-koordinate
\node (xMAX) at (2.8,0){};
\draw[->,color=black] (-2,0) -- (xMAX);
\foreach \x in {-1,1,2}
\draw[shift={(\x,0)},color=black] (0pt,2pt) -- (0pt,-2pt) node[below] {\footnotesize $\x$};
% y- koordinate
\node (yMAX) at (0,10){};
\draw[->,color=black] (0,-10) -- (yMAX);
\foreach \y in {-8,-6,-4,-2,2,4,6,8}
\draw[shift={(0,\y)},color=black] (2pt,0pt) -- (-2pt,0pt) node[left] {\footnotesize $\y$};
\draw[color=black] (0pt,-10pt) node[right] {\footnotesize $0$};
% Achsenbeschriftung
\draw (xMAX) node[anchor=north east] {$x$};
\draw (yMAX) node[anchor=east] {$f_3(x)$};
% Polynom-Plot:
\clip(-1,-10) rectangle (2,9);
\draw[color=red, smooth,samples=100,domain=-0.6:1.3] plot(\x,{9.0*(\x)^(4.0)+9.0*(\x)^(3.0)-2.0*(\x)^(2.0)-19.0*(\x)});
\draw[color=red] (1,-4) node[anchor=west] {$G_{f_3}$};
\end{tikzpicture}
}%
\hspace{1cm}
%GRAPHEN: poly4
\MTikzAuto{%
\begin{tikzpicture} 
% Koordinatensystem
% x-koordinate
\node (xMAX) at (2.8,0){};
\draw[->,color=black] (-5,0) -- (xMAX);
\foreach \x in {-4,-3,-2,-1,1,2}
\draw[shift={(\x,0)},color=black] (0pt,2pt) -- (0pt,-2pt) node[below] {\footnotesize $\x$};
% y- koordinate
\node (yMAX) at (0,9){};
\draw[->,color=black] (0,-1) -- (yMAX);
\foreach \y in {1, 2,3, 4,5, 6,7,8}
\draw[shift={(0,\y)},color=black] (2pt,0pt) -- (-2pt,0pt) node[left] {\footnotesize $\y$};
\draw[color=black] (0pt,-10pt) node[right] {\footnotesize $0$};
% Achsenbeschriftung
\draw (xMAX) node[anchor=north east] {$x$};
\draw (yMAX) node[anchor=east] {$f_4(x)$};
% Polynom-Plot: 
\clip(-5,0) rectangle (3,8.5);
\draw[color=red, smooth,samples=100,domain=-5:3]plot(\x,{(\x)^(2.0)+2.0*(\x)+2.0});
\draw[color=red] (1.5,6) node[anchor=west] {$G_{f_4}$};
\end{tikzpicture}
}%

%GRAPHEN: poly5
\MTikzAuto{%
\begin{tikzpicture} 
% Koordinatensystem
% x-koordinate
\node (xMAX) at (2.8,0){};
\draw[->,color=black] (-1.5,0) -- (xMAX);
\foreach \x in {-1,1,2}
\draw[shift={(\x,0)},color=black] (0pt,2pt) -- (0pt,-2pt) node[below] {\footnotesize $\x$};
% y- koordinate
\node (yMAX) at (0,6){};
\draw[->,color=black] (0,-4) -- (yMAX);
\foreach \y in {-3,-2,-1,1, 2,3, 4,5}
\draw[shift={(0,\y)},color=black] (2pt,0pt) -- (-2pt,0pt) node[left] {\footnotesize $\y$};
\draw[color=black] (0pt,-10pt) node[left] {\footnotesize $0$};
% Achsenbeschriftung
\draw (xMAX) node[anchor=north east] {$x$};
\draw (yMAX) node[anchor=east] {$f_5(x)$};
% Polynom-Plot: 
\clip(-2,-4) rectangle (2.8,5.8);
\draw[color=red, smooth,samples=100,domain=-5:3]plot(\x,{8*(\x)-2});
\draw[color=red] (1,5) node[anchor=west] {$G_{f_5}$};
\end{tikzpicture}
}%
\hspace{1cm}
%GRAPHEN: poly6
\MTikzAuto{%
\begin{tikzpicture}[y = 0.2cm]
% Koordinatensystem
% x-koordinate
\node (xMAX) at (2.8,0){};
\draw[->,color=black] (-1.5,0) -- (xMAX);
\foreach \x in {-1,1,2}
\draw[shift={(\x,0)},color=black] (0pt,2pt) -- (0pt,-2pt) node[below] {\footnotesize $\x$};
% y- koordinate
\node (yMAX) at (0,20){};
\draw[->,color=black] (0,-2) -- (yMAX);
\foreach \y in {10}
\draw[shift={(0,\y)},color=black] (2pt,0pt) -- (-2pt,0pt) node[left] {\footnotesize $\y$};
\draw[color=black] (0pt,-10pt) node[right] {\footnotesize $0$};
% Achsenbeschriftung
\draw (xMAX) node[anchor=north east] {$x$};
\draw (yMAX) node[anchor=east] {$f_6(x)$};
% Polynom-Plot: 
\draw[color=red] (-1.5,13)--(2.5, 13);
\draw[color=red] (2.5,13) node[anchor=south] {$G_{f_6}$};
\end{tikzpicture}
}%

Obviously, the degree of a polynomial is determined by the monomial with highest degree. Moreover,
we see that all types of functions (constant functions, linear functions and linear affine functions) 
studied so far -- as well as the monomials -- occur naturally as special cases of polynomials. Thus, 
the polynomials include all types of functions considered so far. 

An unspecific polynomial of degree $n\in\N$ is written as follows:
\[
 \function{f}{\R}{\R}{x}{a_nx^n+a_{n-1}x^{n-1}+a_{n-2}x^{n-2}\dots +a_{2}x^{2}+a_1x+a_0 \MDFPeriod}
\]
Here, $a_0,a_1,\dots,a_n$ with $a_n\neq0$ are real prefactors of the individual monomials, which
are called \highlight{coefficients} of the polynomial.

\begin{MExercise}
What is the polynomial $f(x)$ with the coefficients $a_0=-4$, $a_2=\pi$, and $a_4=9$, and what is its
range?
\ \\ \ \\
The polynomial is \MEquationItem{$f(x)$}{\MLFunctionQuestion{25}{9*x^4+pi*x^2-4}{5}{x}{5}{ELFP1}},\\ 
its range is \MEquationItem{$W_f$}{\MLIntervalQuestion{15}{[-4,infty)}{4}{ELFP1b}}.\\
\MInputHint{Enter the number $\pi$ as \texttt{pi} and the range as an interval.}

\begin{MHint}{Solution}
The polynomial is
\[
 \function{f}{\R}{\R}{x}{9x^4+\pi x^2-4 \MDFPSpace ,}
\]
and its range is $W_f=[-4\MIntvlSep \infty)$ since the even powers of $x$ can only 
take non-negative values.
\end{MHint}
\end{MExercise}

For general polynomials, the roots are of particular interest. The roots of a polynomial
can be found by solving equations of $n$th degree. In the case of degree polynomials 
of degree $n=2$ (which are also called general parabolas),  this is possible by solving a 
quadratic equation. In Module~\MNRef{VBKM02} the relevant terms and methods, i.e. 
\MSRef{VBKM02_QuadratischErgaenzung}{completing the square}, the \MSRef{VBKM02_pqFormel}{$pq$ formula}, 
and the \MSRef{VBKM02_Scheitelpunktform}{vertex form} of quadratic expressions are explained in detail. 



\begin{MExample}
Consider the parabola 
\[
 \function{\zeta}{\R}{\R}{y}{2y^2-8y+6 }.
\]
We find the roots and the vertex and then sketch the graph.

We complete the square in the mapping rule $\zeta(y)=2(y^2-4y+3)$:
\[
 y^2-4y+3=y^2-4y+4-1=(y-2)^2-1 \MDFPeriod
\]
Thus, the mapping rule can be written as 
\[
 \zeta(y)=2(y-2)^2-2 \MDFPeriod
\]
We see that the parabola is shifted with respect to the standard parabola by $2$~units to the right 
and $2$~units downwards. It can be seen that the vertex is at $(2,-2)$. The roots can be calculated according to:
\[
	\zeta(y)=2((y-2)^2-1)=0\quad\Leftrightarrow\quad (y-2)^2=1\quad\Leftrightarrow\quad y_{1,2}-2=\MCases{1\\-1}\quad\Leftrightarrow\quad y_{1,2}=\MCases{3\\1 \MDFPeriod}
\]
Finally, the graph of the function is as shown in the figure below.

%GRAPH: bsp-para
\MTikzAuto{%
\begin{tikzpicture} 
% Koordinatensystem
% x-koordinate
\node (xMAX) at (4.8,0){};
\draw[->,color=black] (-1.5,0) -- (xMAX);
\foreach \x in {-1,1,2,3,4}
\draw[shift={(\x,0)},color=black] (0pt,2pt) -- (0pt,-2pt) node[below] {\footnotesize $\x$};
% y- koordinate
\node (yMAX) at (0,6){};
\draw[->,color=black] (0,-2.5) -- (yMAX);
\foreach \y in {-2,-1,1, 2,3, 4,5}
\draw[shift={(0,\y)},color=black] (2pt,0pt) -- (-2pt,0pt) node[left] {\footnotesize $\y$};
\draw[color=black] (0pt,-10pt) node[left] {\footnotesize $0$};
% Achsenbeschriftung
\draw (xMAX) node[anchor=north east] {$y$};
\draw (yMAX) node[anchor=east] {$\zeta(y)$};
% Para-Plot: 
%\clip(-2,-4) rectangle (2.8,5.8);
\draw[color=red, smooth,samples=100,domain=0:4]plot(\x,{2*\x^2-8*\x+6});
\draw[color=red] (3.8,4) node[anchor=west] {$G_{\zeta}$};
\draw[color=orange, fill=orange] (2,-2) circle (1.5pt);
\draw[color=blue, fill=blue] (3,0) circle (1.5pt);
\draw[color=blue, fill=blue] (1,0) circle (1.5pt);
\draw[color=orange] (2,-2) node[anchor=north] {\footnotesize vertex};
\draw[color=blue] (1.1,0) node[anchor=south west] {\footnotesize roots};
\end{tikzpicture}
}%

\end{MExample}

\end{MXContent}


\begin{MXContent}{Hyperbolas}{Hyperbolas}{STD}\MLabel{sec:hyperbel}
\MDeclareSiteUXID{VBKM06_EinfacheFunktionen_Hyperbeln}

We consider functions which have a \highlight{reciprocal relation} in their mapping rule. For the determination of the \modsemph{maximum domain} of such a function,
note that the denominator must be non-zero. 

A few examples of reciprocal functions are listed below; these are reciprocals of monomials, and they 
are also called functions of \highlight{hyperbolic type}.
\[
 \function{f_1}{\R\setminus\{0\}}{\R}{x}{\frac{1}{x} \MDFPSpace,}
\]
\[
 \function{f_2}{\R\setminus\{0\}}{\R}{x}{\frac{1}{x^2} \MDFPSpace,}
\]
\[
 \function{f_3}{\R\setminus\{0\}}{\R}{x}{\frac{1}{x^3} \MDFPSpace,}
\]
etc. Their graphs are as follows. 

%GRAPHEN: hyp2
\MTikzAuto{%
\begin{tikzpicture}
% Koordinatensystem
% x-koordinate
\node (xMAX) at (4.8,0){};
\draw[->,color=black] (-3.5,0) -- (xMAX);
\foreach \x in {-3,-2,-1,1,2,3,4}
\draw[shift={(\x,0)},color=black] (0pt,2pt) -- (0pt,-2pt) node[below] {\footnotesize $\x$};
% y- koordinate
\node (yMAX) at (0,4.8){};
\draw[->,color=black] (0,-3.5) -- (yMAX);
\foreach \y in {-3,-2,-1,1,2,3,4}
\draw[shift={(0,\y)},color=black] (2pt,0pt) -- (-2pt,0pt) node[left] {\footnotesize $\y$};
\draw[color=black] (0pt,-10pt) node[right] {\footnotesize $0$};
% Achsenbeschriftung
\draw (xMAX) node[anchor=north east] {$x$};
\draw (yMAX) node[anchor=east] {$f_1(x),f_2(x),f_3(x)$};
% Graph: 
\clip(-3.5,-3.5) rectangle (4.5,4.5);
%1/x
\draw[color = cyan, smooth,samples=100,domain=0.02:4.5]plot(\x,{1/(\x)});
\draw[color = cyan, smooth,samples=100,domain=-4.5:-0.02]plot(\x,{1/(\x)});
\draw[color = cyan](-2,-0.5) node[below]{$G_{f_1}$};
%1/x^2
\draw[color = orange, smooth,samples=100,domain=0.3:4.5]plot(\x,{1/(\x)^2});
\draw[color = orange, smooth,samples=100,domain=-4.5:-0.3]plot(\x,{1/(\x)^2});
\draw[color = orange](-1,1) node[left]{$G_{f_2}$};
%1/x^3
\draw[color = blue, smooth,samples=100,domain=0.6:4.5]plot(\x,{1/(\x)^3});
\draw[color = blue, smooth,samples=100,domain=-4.5:-0.6]plot(\x,{1/(\x)^3});
\draw[color = blue](-0.9,-1.5) node[left]{$G_{f_3}$};
\end{tikzpicture}
}%

In particular, the graph of the function
\[
 \function{f_1}{\R\setminus\{0\}}{\R}{x}{\frac{1}{x} }
\]
is called the \highlight{hyperbola}.

Generally, for the reciprocal of an arbitrary monomial of degree $n\in\N$
a corresponding function of hyperbolic type can be specified.
\[
 \function{f_n}{\R\setminus\{0\}}{\R}{x}{\frac{1}{x^n} \MDFPeriod}
\]
\begin{MExercise}
What is the range $W_{f_n}$ of the function $f_n$ for even or odd $n\in\N$?

\begin{MHint}{Solution}
We always have $\frac{1}{x^n}\neq0$ since a quotient can only be zero if the numerator is zero. Thus, the range 
never contains $0\in\R$. Since $x^n\geq0$ for even $n\in\N$, we have $\frac{1}{x^n}>0$  for even $n\in\N$. 
However, for odd $n\in\N$ we can also have $\frac{1}{x^n}<0$. This results in
\[
 W_{f_n}=\MCases{\R\setminus\{0\} & \text{for}\ n\ \text{odd} \\ (0\MIntvlSep \infty) & \text{for}\ n\ \text{even} \MDFPeriod}
\]
This can also be seen from the graphs of the functions of hyperbolic type.
\end{MHint}
\end{MExercise}

Further examples for functions of hyperbolic type were already considered in Example~\MNRef{bsp:anwendungen} and in 
Exercise~\MNRef{ex:anwendungen} in Section~\MNRef{sec:anwendungen}.
\end{MXContent}

\begin{MXContent}{Rational Functions}{Rational}{STD}\MLabel{sec:gebrochen}
\MDeclareSiteUXID{VBKM06_EinfacheFunktionen_GebrochenRational}
A general rational function has a mapping rule that is the quotient of two polynomials. 
Some examples with their graphs are given below. Of course, for these functions numbers for which the denominator in the mapping rule equals zero must also be 
excluded from the domain.

\begin{MExample}\MLabel{einf_bsp_gebr_rat}
\[
 \function{f}{\R}{\R}{x}{\frac{8}{x^2+1} \MDFPSpace,}
\]
\[
 \function{g}{\R\setminus\{-4,\frac{3}{2}\}}{\R}{x}{\frac{-18x+3}{2x^2+5x-12} \MDFPSpace,}
\]
\[
 \function{h}{\R\setminus\{-1\}}{\R}{x}{\frac{x^3-x^2+x}{x+1} \MDFPeriod}
\] 

%GRAPHEN: rat1
\MTikzAuto{%
\begin{tikzpicture}[scale = 0.5]
% Koordinatensystem
% x-koordinate
\node (xMAX) at (10,0){};
\draw[->,color=black] (-9,0) -- (xMAX);
\foreach \x in {-8,-6,-4,-2,2,4,6,8}
\draw[shift={(\x,0)},color=black] (0pt,2pt) -- (0pt,-2pt) node[below] {\footnotesize $\x$};
% y- koordinate
\node (yMAX) at (0,9){};
\draw[->,color=black] (0,-2) -- (yMAX);
\foreach \y in {2,4,6,8}
\draw[shift={(0,\y)},color=black] (2pt,0pt) -- (-2pt,0pt) node[left] {\footnotesize $\y$};
\draw[color=black] (0pt,-10pt) node[right] {\footnotesize $0$};
% Achsenbeschriftung
\draw (xMAX) node[anchor=north east] {$x$};
\draw (yMAX) node[anchor=east] {$f(x)$};
% Graph: 
\clip(-9,-1) rectangle (9.5,8.5);
\draw[color=red,smooth,samples=100,domain=-9:9]plot(\x,{8/((\x)^2+1)}) node[above] {$G_f$};
\end{tikzpicture}
}%

%GRAPHEN: rat2
\MTikzAuto{%
\begin{tikzpicture}[scale = 0.5]
% Koordinatensystem
% x-koordinate
\node (xMAX) at (10,0){};
\draw[->,color=black] (-9,0) -- (xMAX);
\foreach \x in {-8,-6,-4,-2,2,4,6,8}
\draw[shift={(\x,0)},color=black] (0pt,2pt) -- (0pt,-2pt) node[below] {\footnotesize $\x$};
% y- koordinate
\node (yMAX) at (0,9){};
\draw[->,color=black] (0,-7) -- (yMAX);
\foreach \y in {-6,-4,-2,2,4,6,8}
\draw[shift={(0,\y)},color=black] (2pt,0pt) -- (-2pt,0pt) node[left] {\footnotesize $\y$};
\draw[color=black] (0pt,-10pt) node[right] {\footnotesize $0$};
% Achsenbeschriftung
\draw (xMAX) node[anchor=north east] {$x$};
\draw (yMAX) node[anchor=east] {$g(x)$};
% Graph: 
\clip(-9,-7) rectangle (9.5,8.5);
\draw[color=red,smooth,samples=100,domain=-9:-4.8]
		plot(\x,{(-18*(\x)+3)/(2*(\x)^2+5*(\x)-12)});	
\draw[color=red,smooth,samples=100,domain=-3.1:1.3]
		plot(\x,{(-18*(\x)+3)/(2*(\x)^2+5*(\x)-12)}) ;
\draw[color=red,smooth,samples=100,domain=1.8:9.5]
		plot(\x,{(-18*(\x)+3)/(2*(\x)^2+5*(\x)-12)});
\draw[color=red] (1.1,6) node[right] {$G_g$};
%Definitionsl�cken:		
\draw[color = blue, fill = white](-4,0) circle(3pt);		
\draw[color = blue, fill = white](1.5,0) circle(3pt);
\end{tikzpicture}
}%
\hspace{0.5cm}
%GRAPHEN: rat3
\MTikzAuto{%
\begin{tikzpicture}[y = 0.2cm, x = 0.5cm]
% Koordinatensystem
% x-koordinate
\node (xMAX) at (6,0){};
\draw[->,color=black] (-4,0) -- (xMAX);
\foreach \x in {-3,-2,-1,1,2,3,4,5}
\draw[shift={(\x,0)},color=black] (0pt,2pt) -- (0pt,-2pt) node[below] {\footnotesize $\x$};
% y- koordinate
\node (yMAX) at (0,20){};
\draw[->,color=black] (0,-10) -- (yMAX);
\foreach \y in {-5,5,10,15}
\draw[shift={(0,\y)},color=black] (2pt,0pt) -- (-2pt,0pt) node[left] {\footnotesize $\y$};
\draw[color=black] (0pt,-10pt) node[right] {\footnotesize $0$};
% Achsenbeschriftung
\draw (xMAX) node[anchor=north east] {$x$};
\draw (yMAX) node[anchor=east] {$h(x)$};
% Graph: 
\clip(-4,-10) rectangle (6,19);
\draw[color=red,smooth,samples=100,domain=-3.1:-1.2]
		plot(\x,{((\x)^3-(\x)^2+(\x))/((\x)+1)});
\draw[color=red,smooth,samples=100,domain=-0.8:5.3]
		plot(\x,{((\x)^3-(\x)^2+(\x))/((\x)+1)});
\draw[color=red] (4.5,15) node[left] {$G_h$};
%Definitionsl�cken:		
\draw[color = blue, fill = white](-1,0) circle(1.5pt);		
\end{tikzpicture}
}%

\end{MExample}


\begin{MExercise}
For the function
\[
 \function{\psi}{D_\psi}{\R}{x}{\frac{-42x}{x^2-\pi} }
\]
find the maximum domain $D_\psi\subset\R$ of $\psi$.

\begin{MHint}{Solution}
The roots of the denominator are 
\[
 x^2-\pi=0\quad\Leftrightarrow\quad x^2=\pi\quad\Leftrightarrow\quad x=\pm\sqrt{\pi} \MDFPeriod
\]
Thus, we have $D_\psi=\R\setminus\{-\sqrt{\pi},\sqrt{\pi}\}$.
\end{MHint}
\end{MExercise}

\begin{MExercise}
For the rational functions in Example~\MNRef{einf_bsp_gebr_rat}, specify the degree of the polynomials 
in the numerator and the denominator and find their roots.

\begin{MHint}{Solution}
The function $f$ has the a numerator of degree $0$ and denominator of degree $2$. The numerator does not have a root ($8\neq0$), nor does the denominator ($x^2+1=0$ has no solution).\\
The function $g$ has the a numerator of degree $1$ and denominator of degree $2$. The root of the numerator is at $x=\frac{1}{6}$ ($-18x+3=0\Leftrightarrow x=\frac{3}{18}$), and the roots of the denominator at $x_{1}=-4$, $x_{2} = \frac{3}{2}$ are obtained by solving the quadratic equation $2x^2+5x-12=0$, e.g. by means of the quadratic formula.\\
The function $h$ has a numerator of degree $3$ and denominator of degree $1$. The root of the denominator is simply at 
$x=-1$ ($x+1=0\Leftrightarrow x=-1$). To find the roots of the numerator, the equation $x^3-x^2+x=0$ has to be solved. 
By factoring out $x$ one obtains $x(x^2-x+1)=0$, and it can be seen immediately that one root is at $x=0$. Finally, 
the quadratic equation $x^2-x+1=0$ has to be solved by means of the quadratic formula. However, the discriminant 
$\Delta=1^2-4\cdot1\cdot1=-3$ is negative, so no other real solution of the equation -- and hence 
no other root of the numerator -- exists. 
\end{MHint}
\end{MExercise}

The \highlight{roots} of a rational function are the roots of the numerator. For example, the function 
\[
 \function{j}{\R\setminus\{-1\MElSetSep 3\}}{\R}{x}{\frac{x-1}{x^2-2x-3}}
\]
has a single root at $x=1$. The roots of the denominator of rational functions that must be excluded from 
the domain often have to be investigated further. It is of particular interest how the graphs of functions 
behave \modsemph{in the neighbourhood} of gaps in the domain. The roots of the denominator are also called 
\highlight{poles}. The next examples will illustrate the different types of poles that can occur. 

\begin{MExample}
\[
 \function{f_1}{\R\setminus\{2\}}{\R}{x}{\frac{3}{x-2}}
\]
\[
 \function{f_2}{\R\setminus\{-3\}}{\R}{x}{\frac{2}{(x+3)^2}}
\]
\[
 \function{f_3}{\R\setminus\{1\}}{\R}{x}{\frac{x^2-1}{x-1}}
\]

%GRAPHEN: rat4
\MTikzAuto{%
\begin{tikzpicture}
% Koordinatensystem
% x-koordinate
\node (xMAX) at (5.8,0){};
\draw[->,color=black] (-3,0) -- (xMAX);
\foreach \x in {-2,-1,1,2,3,4,5}
\draw[shift={(\x,0)},color=black] (0pt,2pt) -- (0pt,-2pt) node[below] {\footnotesize $\x$};
% y- koordinate
\node (yMAX) at (0,5.8){};
\draw[->,color=black] (0,-4) -- (yMAX);
\foreach \y in {-3,-2,-1,1,2,3,4,5}
\draw[shift={(0,\y)},color=black] (2pt,0pt) -- (-2pt,0pt) node[left] {\footnotesize $\y$};
\draw[color=black] (0pt,-10pt) node[right] {\footnotesize $0$};
% Achsenbeschriftung
\draw (xMAX) node[anchor=north east] {$x$};
\draw (yMAX) node[anchor=east] {$f_1(x)$};
% Graph: 
\clip(-3,-4) rectangle (5.5,5.5);
\draw[color=red,smooth,samples=100,domain=-3:1.3]
		plot(\x,{3/((\x)-2)}) ;
\draw[color=red,samples=100,domain=2.4:6]
		plot(\x,{3/((\x)-2)}) ;		
\draw[color=red] (2.7,5) node[right] {$G_{f_1}$};
%Definitionsl�cken:		
\draw[color = blue, fill = white](2,0) circle(1.5pt);
\draw[color = blue, dotted](2,-4)--(2,5.5);
\end{tikzpicture}
}%

% GRAPHEN: rat5
\MTikzAuto{%
\begin{tikzpicture}
% Koordinatensystem
% x-koordinate
\node (xMAX) at (1.8,0){};
\draw[->,color=black] (-6,0) -- (xMAX);
\foreach \x in {-5,-4,-3,-2,-1,1}
\draw[shift={(\x,0)},color=black] (0pt,2pt) -- (0pt,-2pt) node[below] {\footnotesize $\x$};
% y- koordinate
\node (yMAX) at (0,5.8){};
\draw[->,color=black] (0,-1) -- (yMAX);
\foreach \y in {1,2,3,4,5}
\draw[shift={(0,\y)},color=black] (2pt,0pt) -- (-2pt,0pt) node[left] {\footnotesize $\y$};
\draw[color=black] (0pt,-10pt) node[right] {\footnotesize $0$};
% Achsenbeschriftung
\draw (xMAX) node[anchor=north east] {$x$};
\draw (yMAX) node[anchor=east] {$f_2(x)$};
% Graph: 
\clip(-6,-1) rectangle (1.5,5.5);
\draw[color=red,smooth,samples=100,domain=-6:-3.5]
		plot(\x,{2/(((\x)+3)^2)});
\draw[color=red,smooth,samples=100,domain=-2.5:1.5]
		plot(\x,{2/(((\x)+3)^2)});
\draw[color=red] (1,0.2) node[above] {$G_{f_2}$};
%Definitionsl�cken:		
\draw[color = blue, fill = white](-3,0) circle(1.5pt);
\draw[color = blue, dotted](-3,-1)--(-3,5.5);
\end{tikzpicture}
}%
\hspace{.5cm}
% GRAPHEN: rat6
\MTikzAuto{%
\begin{tikzpicture}
% Koordinatensystem
% x-koordinate
\node (xMAX) at (3.8,0){};
\draw[->,color=black] (-3,0) -- (xMAX);
\foreach \x in {-2,-1,1,2,3}
\draw[shift={(\x,0)},color=black] (0pt,2pt) -- (0pt,-2pt) node[below] {\footnotesize $\x$};
% y- koordinate
\node (yMAX) at (0,5.8){};
\draw[->,color=black] (0,-2) -- (yMAX);
\foreach \y in {-1,1,2,3,4,5}
\draw[shift={(0,\y)},color=black] (2pt,0pt) -- (-2pt,0pt) node[left] {\footnotesize $\y$};
\draw[color=black] (0pt,-10pt) node[right] {\footnotesize $0$};
% Achsenbeschriftung
\draw (xMAX) node[anchor=north east] {$x$};
\draw (yMAX) node[anchor=east] {$f_3(x)$};
% Graph: 
\clip(-3,-2) rectangle (3.5,5.5);
\draw[color=red, smooth,samples=100,domain=-3:3.5]
		plot(\x,{(\x)+1}) node[above, left=2pt] {$G_{f_3}$};		
%hebbare Definitionsl�cke:		
\draw[color = blue, fill = white](1,0) circle(1.5pt);
\draw[color = blue, dotted](1,-1)--(1,5);
\draw[color = black, fill = white](1,2) circle(1pt);
\end{tikzpicture}
}%
\end{MExample}

At $x=2$ and $x=-3$ the functions $f_1$ and $f_2$, respectively, have so-called 
\modsemph{(proper) poles}, and at $x=1$ the function $f_3$ has a so-called \modsemph{removable singularity}.
Looking at the graphs, the difference between these types of poles becomes clear. For (proper) poles,
the graph rises or falls unboundedly in the neighbourhood of the pole, and for removable 
singularities the graph ends left and right in the ``gap''.

In the mapping rules of the three functions, this difference is expressed as follows: the values 
$x=2$ and $x=-3$ are \modsemph{roots of the denominator}, but they are not \modsemph{roots of the numerator}
of the functions $f_1$ and $f_2$, respectively. Actually, the functions $f_1$ and $f_2$ do not have 
any roots in the numerator. In such cases, the roots of the denominator are always (proper) poles.

\begin{MExercise}
Is the denominator's root of the function
\[
 \function{q}{\R\setminus\{\frac{1}{2}\}}{\R}{x}{\frac{x^4-1}{2x-1}}
\]
a proper pole? If so, give reasons for your answer.

\begin{MHint}{Solution}
The point $x=\frac{1}{2}$ is a root of the denominator:
\[
 2x-1=0\quad\Leftrightarrow\quad 2x=1\quad\Leftrightarrow\quad x=\frac{1}{2} \MDFPeriod
\]
However, for the numerator, we have:
\[
 x^4-1=0\quad\Leftrightarrow\quad x^4=1\quad\Leftrightarrow\quad x=\pm 1 \MDFPSpace,
\]
and thus, the roots of the numerator are at $x=-1$ and $x=1$. Hence, $x=\frac{1}{2}$ is not a root
of the numerator. Thus, $x=\frac{1}{2}$ is a proper pole.
\end{MHint}
\end{MExercise}

Between the two poles of $f_1$ and $f_2$ there is a further difference.
At the pole $x=2$ of $f_1$, the function has a \modsemph{change of sign}. The graph of $f_1$ falls 
left to the pole unboundedly to minus infinity and rises right to the pole (coming from the right) 
unboundedly to plus infinity.\\
The graph of $f_2$ rises on both sides of the pole at $x=-3$ (while approaching the pole) unboundedly
to plus infinity, and hence, there is no \modsemph{change of sign} in the function values.

However, in the mapping rule of $f_3$, the term responsible for the pole at $x=1$ can be cancelled
out. For rational functions that have a removable singularity, this is always possible.

\begin{MExercise}
Find all poles/singularities of the function
\[
 \function{\gamma}{D_\gamma}{\R}{x}{\frac{3x+6}{x^2-x-6}}
\]
and determine their type. Specify the maximum domain $D_\gamma\subset\R$ of the function.

\begin{MHint}{Solution}
The roots of the denominator are the solutions of the quadratic equation $x^2-x-6=0$, thus
\[
 x_{1,2}=\frac{-(-1)\pm\sqrt{(-1)^2-4\cdot(-6)\cdot1}}{2}=\frac{1\pm 5}{2}=\MCases{3\MDFPaSpace\MDFPSpace \\-2\MDFPeriod}
\]
Hence, the maximum domain is
\[
 D_\gamma=\R\setminus\{-2\MElSetSep 3\} \MDFPeriod
\]
The root of the numerator results from $3x+6=0$, i.e. the root of the numerator is also at $x=-2$. Thus, we 
can transform the mapping rule of $\gamma$ for $x\in D_\gamma$ as follows:
\[
 \gamma(x)=\frac{3x+6}{x^2-x-6}=\frac{3(x+2)}{(x-3)(x+2)}=\frac{3}{x-3} \MDFPeriod
\]
Hence, the function can also be written in the form
\[
 \function{\gamma}{\R\setminus\{-2\MElSetSep 3\}}{\R}{x}{\frac{3}{x-3}\MDFPSpace ,} 
\]
and thus, the function has a continuously removable singularity at $x=-2$ and a (proper) pole with 
change of sign at $x=3$.
\end{MHint}
\end{MExercise}
\end{MXContent}

\begin{MXContent}{Asymptotes}{Asymptotes}{STD}\MLabel{sec:asymptoten}
\MDeclareSiteUXID{VBKM06_EinfacheFunktionen_Asymptoten}

In this section we will study how rational functions behave as $x$ tends to infinity if 
the \modsemph{degree of the polynomial in the numerator is less than or equal to the degree of the 
polynomial in the denominator}. An example of this is the function
\[
 \function{f}{\R\setminus\{-\pi\}}{\R}{x}{\frac{x}{x+\pi} \MDFPeriod}
\]
In the function $f$, the degree of the polynomial in the numerator is $1$,
and degree of the polynomial in the denominator is $1$. Other examples were 
investigated in the previous Section~\MNRef{sec:gebrochen}.

\begin{MExample}\MLabel{ex1:polydiv}%
Let us consider the function
\[
 \function{g}{\R\setminus\{1\}}{\R}{x}{1+\frac{1}{x-1} \MDFPeriod} 
\]
Its mapping rule is a sum of a polynomial (of degree $0$) and a rational term. By finding the common
denominator it is easy to transform $g(x)$ into a rational form in which the degree of the polynomial in the 
numerator equals the degree of the polynomial in the denominator:
\[
 g(x)=1+\frac{1}{x-1}=\frac{x-1}{x-1}+\frac{1}{x-1}=\frac{x-1+1}{x-1}=\frac{x}{x-1} \MDFPeriod
\]
Thus, we can rewrite $g$ in the form
\[
 \function{g}{\R\setminus\{1\}}{\R}{x}{\frac{x}{x-1}\MDFPSpace ,} 
\]
and we now consider the corresponding graph.

% GRAPHEN: rat7
\MTikzAuto{%
\begin{tikzpicture}
% Koordinatensystem
% x-koordinate
\node (xMAX) at (7.8,0){};
\draw[->,color=black] (-5.5,0) -- (xMAX);
\foreach \x in {-5,-4,-3,-2,-1,1,2,3,4,5,6,7}
\draw[shift={(\x,0)},color=black] (0pt,2pt) -- (0pt,-2pt) node[below] {\footnotesize $\x$};
% y- koordinate
\node (yMAX) at (0,5.8){};
\draw[->,color=black] (0,-4.5) -- (yMAX);
\foreach \y in {-4,-3,-2,-1,1,2,3,4,5}
\draw[shift={(0,\y)},color=black] (2pt,0pt) -- (-2pt,0pt) node[left] {\footnotesize $\y$};
\draw[color=black] (0pt,-10pt) node[right] {\footnotesize $0$};
% Achsenbeschriftung
\draw (xMAX) node[anchor=north east] {$x$};
\draw (yMAX) node[anchor=east] {$g(x)$};
% Graph: 
\clip(-5,-4.5) rectangle (7,5.5);
\draw[color=red,smooth,samples=100,domain=-5:0.9]
		plot(\x,{\x/(\x-1)});
\draw[color=red,smooth,samples=100,domain=1.1:7]
		plot(\x,{\x/(\x-1)});
\draw[color=red] (5,1.5) node[above] {$G_{g}$};
%Definitionsl�cke:		
\draw[color = blue, fill = white](1,0) circle(1.5pt);
%Asymptote:
\draw[color=blue, dotted] (-5.5,1) -- (7.5,1);
\end{tikzpicture}
}%

Besides the pole and the singularity at $x=1$, we see that the value 
$y=1$ is of specific relevance. Obviously, this value is never taken by the function $g$. 
Thus, the range of $g$ is $W_g=\R\setminus\{1\}$. Instead, for ``very large'' values 
($x$ tends to plus infinity) and ``very small'' values ($x$ tends to minus infinity) of the 
independent variable $x$, the function $g$ \modsemph{approaches} its 
limiting value $1$ indefinitely \modsemph{without} ever reaching it for any real number $x$.

This can be seen from the mapping rule $g(x)=1+\frac{1}{x-1}$ as follows. For ``very large''
($50$, $100$, $1000$, etc.) or ``very small'' ($50$, $100$, $1000$, etc.) values of $x$, 
the rational part $\frac{1}{x-1}$ approaches $0$ since $x$ occurs in the 
denominator. In general, for these values of $x$, only the polynomial part $1$ 
of the mapping rule remains. This part can be described by a -- in this case constant --
function that is called an \highlight{asymptote} $g_{As}$ of the function $g$:
\[
 \function{g_{As}}{\R}{\R}{x}{1 \MDFPeriod}
\]
Since this is a constant function, it is also called 
a \modsemph{horizontal asymptote}.
 
\end{MExample}
\begin{MExercise}\MLabel{auf2:polydiv}
Identify the asymptotes of the function
\[
 \function{i}{\R\setminus\{-2\}}{\R}{x}{3-\frac{6}{x+2}}
\]
and the asymptote of the hyperbola described in Section~\MNRef{sec:hyperbel}. 
\begin{MHint}{Solution}
We have
\[
 i(x)=3-\frac{6}{x+2}
\]
with the rational part $\frac{6}{x+2}$. Hence, the horizontal asymptote of $i$ has the 
mapping rule $i_{As}(x)=3$. The hyperbola
\[
 \function{f}{\R\setminus\{0\}}{\R}{x}{\frac{1}{x}}
\]
also has an asymptote. We can write the mapping rule as 
\[
 f(x)=0+\frac{1}{x} \MDFPSpace,
\]
thus, we have $f_{As}(x)=0$ for the asymptote, i.e. the asymptote is the function that is constantly $0$:
the zero function or the horizontal axis of the coordinate system.
\end{MHint}
\end{MExercise}



\begin{MInfo}
A rational function $f$ with the polynomial $p(x)$ of degree $z\geq0$ in the numerator and 
the polynomial $q(x)$ of degree  $n\geq0$ in the denominator of the form 
\[
 \function{f}{\R\setminus\{\text{denominator's roots}\}}{\R}{x}{f(x)=\frac{p(x)}{q(x)}}
\]
has a constant function (or a horizontal line) as an asymptote if $z\leq n$.
In particular, the zero function is the asymptote in the case $z<n$.
\end{MInfo}

\end{MXContent}


\MSubsection{Power Functions}
\MLabel{VBKM06_Potenz}
\begin{MIntro}
\MDeclareSiteUXID{VBKM06_Potenzfunktionen_Intro}
In Section~\MNRef{sec:monome} and Section~\MNRef{sec:hyperbel} we studied monomials and functions of 
hyperbolic type. In summary, these can be described as the following type of functions:
\[
 \function{f}{D_f}{\R}{x}{x^k \MDFPSpace ,} 
\]
where $k\in\Z\setminus\{0\}$ and $D_f=\R$ for $k\in\N$ as well as $D_f=\R\setminus\{0\}$ for $k\in\Z$ 
with $k<0$. In this section, we will allow arbitrary rational for the exponent in the mapping rule.
This results in so-called \modsemph{power functions} that again include monomials and functions of
hyperbolic type as special cases. We will collect their fundamental properties and see some applications.
\end{MIntro}



\begin{MXContent}{Radical Functions}{Radical Functions}{STD}\MLabel{sec:wurzel}
\MDeclareSiteUXID{VBKM06_Potenzfunktionen_Wurzel}%
\begin{MExample}\MLabel{bsp1:wurzel}%

If an object falling in the homogeneous gravitational field of the Earth is observed, then 
the following relation between the falling time and the travelled distance can be found:

\renewcommand{\arraystretch}{1.5}
\begin{tabular}{c|c|c|c|c|c}
Falling time $t$ in seconds & $0$ & $\sqrt{\frac{2}{g}}$ & $\sqrt{\frac{2}{g}}\cdot \MZahl{1}{5}$ & $\sqrt{\frac{2}{g}}\cdot 2$ & $\sqrt{\frac{2}{g}}\cdot 3$ \\\hline
Travelled distance $s$ in metres & $0$ & $1$ & $\MZahl{2}{25}$ & $4$ & $9$
\end{tabular}

Here, $g\approx \MZahl{9}{81}\frac{\MEinheit[]{m}}{\MEinheit[]{s}^2}$ is the physical constant of the 
gravitational acceleration. Now, plotting these values in a diagram with the horizontal axis $s$ 
and the vertical axis $t$ results in the figure below.


% GRAPHEN: wurz-mess
\MTikzAuto{%
\begin{tikzpicture}[y = 2cm]
% Koordinatensystem
% x-koordinate
\node (xMAX) at (10,0){};
\draw[->,color=black] (-0.5,0) -- (xMAX);
\foreach \x in {0,1,2.25, 4, 9}
		\draw[shift={(\x,0)},color=black] 
		  (0pt,2pt) -- (0pt,-2pt) node[below] {\footnotesize $\x$};
% y- koordinate
\node (yMAX) at (0,1.8){};
\draw[->,color=black] (0,0) -- (yMAX);

\draw[shift={(0,0)},color=black] (2pt,0pt) -- (-2pt,0pt) node[above left] {\tiny $0$};
\draw[shift={(0,0.45)},color=black] (2pt,0pt) -- (-2pt,0pt) node[left] {\tiny $\sqrt{\frac{2}{g}}$};

\foreach \y in {1.5, 2, 3}
\draw[shift={(0,0.45*\y)},color=black] (2pt,0pt) -- (-2pt,0pt) node[left] {\tiny $\sqrt{\frac{2}{g}}\cdot\y$};

% Achsenbeschriftung
\draw (xMAX) node[anchor=north east] {$s$};
\draw (yMAX) node[anchor=east] {$t(s)$};	
% Punkte
\draw[fill = blue](0,0)circle(1.5pt);
\draw[fill = blue](1,0.45)circle(1.5pt);
\draw[fill = blue](2.25,0.68)circle(1.5pt);
\draw[fill = blue](4,0.9)circle(1.5pt);
\draw[fill = blue](9,1.35)circle(1.5pt);
\end{tikzpicture}
}%

This suggests that the relation between $t$ and $s$ can be described mathematically by the function
\[
 \function{t}{[0\MIntvlSep \infty)}{\R}{s}{\sqrt{\frac{2}{g}}\cdot \sqrt{s}}
\]
with $s$ being the independent variable. This is a function, in whose mapping rule a root 
(more specifically, a square root) of the independent variable occurs. Then, the graph of this function 
contains the measurement points listed above:

% GRAPHEN: wurz1
\MTikzAuto{%
\begin{tikzpicture}[y = 2cm]
% Koordinatensystem
% x-koordinate
\node (xMAX) at (10,0){};
\draw[->,color=black] (-0.5,0) -- (xMAX);
\foreach \x in {0,1,2.25, 4, 9}
		\draw[shift={(\x,0)},color=black] 
		  (0pt,2pt) -- (0pt,-2pt) node[below] {\footnotesize $\x$};
% y- koordinate
\node (yMAX) at (0,1.8){};
\draw[->,color=black] (0,0) -- (yMAX);
\draw[shift={(0,0)},color=black] (2pt,0pt) -- (-2pt,0pt) node[above left] {\tiny $0$};
\draw[shift={(0,0.45)},color=black] (2pt,0pt) -- (-2pt,0pt) node[left] {\tiny $\sqrt{\frac{2}{g}}$};

\foreach \y in {1.5, 2, 3}
\draw[shift={(0,0.45*\y)},color=black] (2pt,0pt) -- (-2pt,0pt) node[left] {\tiny $\sqrt{\frac{2}{g}}\cdot\y$};
% Achsenbeschriftung
\draw (xMAX) node[anchor=north east] {$s$};
\draw (yMAX) node[anchor=east] {$t(s)$};
%Graph
\draw[smooth,samples=100,domain=0.01:9.5,color=blue] plot(\x,{sqrt(2.0*(\x)/9.81)});	
% Punkte
\draw[fill = blue](0,0)circle(1.5pt);
\draw[fill = blue](1,0.45)circle(1.5pt);
\draw[fill = blue](2.25,0.68)circle(1.5pt);
\draw[fill = blue](4,0.9)circle(1.5pt);
\draw[fill = blue](9,1.35)circle(1.5pt);
\end{tikzpicture}
}%
\end{MExample}

This example shows that functions with mapping rules that contain roots of the independent variables 
occur naturally in applications of mathematics. 
 
For natural numbers $n\in\N$, $n>1$, the functions 
\[
 \function{f_n}{D_{f_n}}{\R}{x}{\sqrt[n]{x}=x^{\frac{1}{n}}}
\]
are called \highlight{radical functions}. Obviously, these include the square root $f_2(x)=\sqrt{x}$,
the cube root $f_3(x)=\sqrt[3]{x}$, the fourth root $f_4(x)=\sqrt[4]{x}$, etc., as mapping rules 
of functions (see \MSRef{VBKM01_Potenzgesetze}{exponent rules}). 


\begin{MExercise}
Transform the mapping rule of the radical functions using exponent rules such that only 
exponents still occur in the mapping rule.

\begin{MHint}{Solution}
According to the exponent rules, we have 
\[
 f_n(x)=\sqrt[n]{x}=x^{\frac{1}{n}}
\]
for all natural numbers $n$. Hence, for example
\[
 f_2(x)=\sqrt{x}=x^{\frac{1}{2}}\MDFPSpace, \MDFPaSpace f_3(x)=\sqrt[3]{x}=x^{\frac{1}{3}}\MDFPSpace, \MDFPaSpace f_4(x)=\sqrt[4]{x}=x^{\frac{1}{4}}\MDFPSpace, \MDFPaSpace \dots
\]
\end{MHint}
\end{MExercise}

\begin{MExercise}
What is the function $f_n$ with $n=1$?

\begin{MHint}{Solution}

According to the exponent rules, we have for $n=1$ 
\[
 f_1(x)=\sqrt[1]{x}=x^{\frac{1}{1}}=x=\Id(x) \MDFPeriod
\]
This is the identity function. Generally, this function is excluded from the class of radical functions.
\end{MHint}
\end{MExercise}

Of great relevance is now the maximum domain $D_{f_n}$ that a radical function can have. Obviously, 
it depends on the exponent $n$ of the root which values of $x$ are allowed to be inserted in the mapping 
rule to obtain real values as a result. So we see that the square root $\sqrt{\ }$ has a
real value as a result only for a non-negative number. However, if we consider the cube root $\sqrt[3]{\ }$, 
then we see that the cube root has a real value as a result for all real numbers, for example,  
$\sqrt[3]{-27}=-3$. Generally, we have:


\begin{MInfo}
The radical functions
\[
 \function{f_n}{D_{f_n}}{\R}{x}{\sqrt[n]{x}}
\]
with $n\in\N$ and $n>1$ have the maximum domain $D_{f_n}=[0\MIntvlSep \infty)$ if $n$ is even and 
the maximum domain $D_{f_n}=\R$ if $n$ is odd.
\end{MInfo}

Thus, the graphs of the first four radical functions, $f_2$, $f_3$, $f_4$, $f_5$, look like as in the figure below. 

% GRAPHEN: wurz2
\MTikzAuto{%
\begin{tikzpicture}
% Koordinatensystem
% x-koordinate
\node (xMAX) at (8.8,0){};
\draw[->,color=black] (-8,0) -- (xMAX);
\foreach \x in {-7,-6,-5,-4,-3,-2,-1,1,2,3,4,5,6,7,8}
		\draw[shift={(\x,0)},color=black] 
		  (0pt,2pt) -- (0pt,-2pt) node[below] {\footnotesize $\x$};
% y- koordinate
\node (yMAX) at (0,3.8){};
\draw[->,color=black] (0,-3) -- (yMAX);
\foreach \y in {-2,-1,1,2,3}
\draw[shift={(0,\y)},color=black] (2pt,0pt) -- (-2pt,0pt) node[left] {\footnotesize $\y$};
% Achsenbeschriftung
\draw (xMAX) node[anchor=north east] {$x$};
\draw (yMAX) node[anchor=east] {$f_2(x),f_3(x),f_4(x),f_5(x)$};
%Beschriftung
\draw[color = orange] (8.2,2.9) node[right] {$G_{\sqrt{x}}$};
\draw[color = cyan] (8.2,2.3) node[right] {$G_{\sqrt[3]{x}}$};
\draw[color = red] (8.2,1.8) node[right] {$G_{\sqrt[4]{x}}$};
\draw[color = blue] (8.2,1.3) node[right] {$G_{\sqrt[5]{x}}$};
%Graph
\clip(-8,-3) rectangle (8.2,3.5);
% f_2
\draw [rotate around={90:(0,0)}, 
      color = orange,
			smooth,samples=100,
			domain=0:3] 
			plot(\x,{-(\x)^2});	
%f_3
\draw [rotate around={90:(0,0)}, 
      color = cyan,
			smooth,samples=100,
			domain=-2.2:2.2] 
			plot(\x,{(-\x)^3});
%f_4
\draw [rotate around={90:(0,0)}, 
      color = red,
			smooth,samples=100,
			domain=0:1.8] 
			plot(\x,{-(\x)^4});
%f_3
\draw [rotate around={90:(0,0)}, 
      color = blue,
			smooth,samples=100,
			domain=-1.6:1.6] 
			plot(\x,{(-\x)^5});
\end{tikzpicture}
}%

From the graphs, it can be seen that all radical functions 
are \highlight{strictly increasing}.

\begin{MExercise}
For the radical functions
\[
 \function{f_n}{D_{f_n}}{\R}{x}{\sqrt[n]{x}}
\]
with $n\in\N$, $n>1$, find the range $W_{f_n}$ depending on whether $n$ is even or odd.

\begin{MHint}{Solution}
For radical functions with $n$ even, obviously only non-negative numbers can occur as results since, according to 
the exponent rules, the roots $\sqrt{x}$, $\sqrt[4]{x}$, $\sqrt[6]{x}$ are always non-negative for $x\geq 0$. 
In contrast, for radical functions with $n$ odd, all negative real numbers can occur as results. In fact, 
we have $\sqrt[3]{x}<0$, $\sqrt[5]{x}<0$ if and only if $x<0$. In summary, we then have, regarding the strict
monotonicity of the radical functions, $W_{f_n}=\R$ if $n$ is odd, and $W_{f_n}=[0\MIntvlSep \infty)$ if $n$ is even.
\end{MHint}


\end{MExercise}

\end{MXContent}


\MSubsection{Exponential and Logarithmic Functions}
\MLabel{VBKM06_exponential}

\begin{MIntro}
\MDeclareSiteUXID{VBKM06_Exponentialfunktionen_Intro}

In contrast to power functions, in \MEntry{exponential functions}{exponential function} the independent variable 
does not occur in the \modstextbf{base} of the exponential term but in the \modstextbf{exponent}.
Accordingly, we will consider mapping rules such as
$$x \longmapsto 2^x\; \text{or}\; x \longmapsto 10^x \MDFPeriod$$
\ \\ \ \\
Exponential functions are relevant in many different fields, for example, for the description of 
biological growth processes -- including diverse population models --, the processes of radioactive decay, or 
a certain kind of interest calculation. Let us consider an example.

\begin{MExample}
\MLabel{M06_bsp_bak_pop}

A bacterial culture starts with $500$ bacteria and doubles in size every $13$ minutes. We would like to know how many
bacteria will be in the culture after $1$ hour and $15$ minutes (i.e. after $75$ minutes).

As a first try we can create a simple table of values that lists the number of bacteria in the population at 
the beginning ($t=0$ min), after $t = 13$ min, after $t = 26$ min, etc., i.e. at multiples of the 
$13$ minutes duplication time:
\begin{center}
 \begin{tabular}{|c|r|r|r|r|r|r|r|r|c|}
 \hline
 Time $t$ in min & 0 & 13 & 26 & 39 & 52 & 65 & 78 & 91 & etc. \\ \hline
 Number of Bacteria & 500 & 1\,000 & 2\,000 & 4\,000 & 8\,000 & 16\,000 & 32\,000 & 64\,000 & etc. \\ 
 \hline
 \end{tabular}
\end{center}
From the table we can estimate that the answer to our question will be between $16\,000$ and $32\,000$, probably 
closer to $32\,000$. What about a (more) precise answer? For this, we need to know the 
\modstextbf{functional relation} between the values of $t$ and the number of bacteria. In the figure below 
the graph of a function $p$ is shown; so to speak, this graph closes the gaps between the isolated points that 
correspond to the pairs of values in the table and are plotted as well. The corresponding mapping rule 
assigns to every real valued point in time a number of bacteria. As we will see, the corresponding function 
is an \modstextbf{exponential function}.

\begin{center}
\MUGraphicsSolo{bak_pop.png}{scale=1}{width:400px}
\end{center}

From the graphical representation, the required number of bacteria can be read off a bit more precisely. However, 
for the exact specification we need the mapping rule underlying the graph, which we will first simply state here:
$$p: [0\MIntvlSep \infty) \longrightarrow (0\MIntvlSep  \infty) \;\text{with}\; t \longmapsto p(t) = 500 \cdot 2^{(t/13)} \MDFPeriod$$
(In Exercise~\MRef{M06_bsp_bak_pop_exercise} we will give a reason for this functional relation.)\\
For $t = 75$ (measured in minutes) we obtain the function value
$$p(75) = 500 \cdot 2^{(75/13)} \approx 500 \cdot \MZahl{54}{539545} \approx 27\,270 \MDFPeriod$$
Hence, after $75$~minutes approx. $27\,270$~bacteria live in the considered population.
\end{MExample}
\end{MIntro}


\begin{MContent}
\MLabel{M06_allgem_exp_fkt}
\MDeclareSiteUXID{VBKM06_Exponentialfunktionen_Allgemein}

In the previous \MSRef{M06_bsp_bak_pop}{example}, an \modstextbf{exponential function} with base $a = 2$ occurs, and the 
independent variable -- in this example this is the variable $t$ -- occurs in the exponent. We will now specify the general 
mapping rule for an exponential function with an arbitrary base $a$; however, we here assume $a > 0$:
$$
 \function{f}{\R}{(0;\infty)}{x}{f(x)=f_{0}\cdot a^{\lambda x}}
$$
Here, $f_0$ and $\lambda$ denote so-called parameters of the exponential function that will be introduced below. 

The domain of all exponential functions is the set of all real numbers, i.e. $D_f = \R$, whereas the range 
only consists of the positive real numbers, i.e. $W_f = (0\MIntvlSep  \infty)$, since every power of a positive number 
can only be positive.

\begin{MExercise}
 Why it is assumed that the base $a$ of the exponential function is greater than zero? 

\begin{MHint}{Solution}
  An exponential function will be defined not only for certain, specific, or isolated values of the 
  variable $x$ but, if possible, for all real numbers. If negative bases $a < 0$ were allowed, then 
  problems would immediately arise in extracting roots -- referring to $a^{(1/2)} = \sqrt{a}, a^{1/4}, 
  a^{1/12}$, etc. For example, square roots of negative numbers are not defined, see 
  Section~\MRef{VBKM06_Potenz}.
 \end{MHint}
\end{MExercise}

Some general properties can be seen from the figure below showing exponential functions 
$g: \R \rightarrow (0\MIntvlSep  \infty)$, $x \longmapsto g(x) = a^x$ for different values of 
$a$:
\begin{center}
\MUGraphicsSolo{expo_vgl.png}{scale=1}{width:400px}
\end{center}
\begin{itemize}
 \item All these exponential functions pass trough the point $(x = 0, y = 1)$, since 
  $g(x = 0) = a^0$ and $a^0 = 1$ for every number $a$.
 \item If $a > 1$, then the graph of $g$ rises from left to right (i.e. for increasing $x$-values); 
  one also says that the function $g$ is \modstextbf{strictly increasing}. The greater the value of $a$,
  the steeper the graph of $g$ rises for \modstextbf{positive} values of $x$. Moving towards ever larger 
  negative values of $x$ (i.e. approaching from right to left) the negative $x$-axis is an asymptote 
  of the graph.
 \item If $a < 1$, then the graph of $g$ falls from left to right (i.e. for increasing $x$-values);
  one also says that the function $d$ is \modstextbf{strictly decreasing}.  The greater the value of $a$,
  the slower the graph of $g$ falls for \modstextbf{negative} values of $x$. Moving towards ever larger 
  positive values of $x$ (i.e. approaching from left to right) the positive $x$-axis is an asymptote 
  of the graph.
\end{itemize}
What are the parameters $f_0$ and $\lambda$? The parameter $f_0$ is easily explained: 
if the value $x = 0$ is inserted in the general exponential function $f$, resulting in 
$$f(x = 0) = f_0 \cdot a^{\lambda \cdot 0} = f_0 \cdot a^0 = f_0 \cdot 1 = f_0 \MDFPSpace,$$
then it can be seen that $f_0$ is a kind of \modstextbf{starting point or initial value} (at least if the 
variable $x$ is taken for a time); the exponential progression $a^{\lambda x}$ is generally multiplied 
by the factor $f_0$ and thus weighted accordingly, i.e. stretched (for $|f_0| > 1$)  
or compressed (for $|f_0| < 1$).

The parameter $\lambda$ that occurs in the exponent is called \MEntry{growth rate}{growth rate}; it 
determines how strong the exponential function -- with the same base -- increases (for $\lambda > 0$) 
or decreases (for $\lambda < 0$).


\begin{MExercise}
 \MLabel{M06_bsp_bak_pop_exercise}
  Explain the form of the exponential function $f(t) = 500 \cdot 2^{(t/13)}$ that occurs in 
  Example~\MRef{M06_bsp_bak_pop}.

  \begin{MHint}{Solution}
  In every duplication period of $13$ minutes -- as the name suggests -- the population of bacteria is doubled.
  Compared to the initial value ($500$ bacteria), the number of bacteria doubles after a period of $13$~minutes, 
  quadruples after two such periods, increases by a factor of eight after $3 \cdot 13$~minutes (always compared
  to the initial value, as already said), etc. From that fact we see that a growth process involves powers 
  of $2$; accordingly, we take as the base of the functional relation $a = 2$. 

  This consideration also determines the exponent of the required exponential function: Our measurement 
  of time has to refer to the duplication period of $13$ minutes, hence, the exponent is $\Mtfrac{t}{13}$. 
  After $13$~minutes, the exponent has the value $\Mtfrac{13}{13} = 1$. Then, the growth factor is 
  $2^{(13/13)}=2$. After two duplication periods ($26$~minutes), the exponent has the value $\Mtfrac{26}{13} = 2$, 
  and thus the growth factor is altogether $2^{(26/13)} = 2^2 = 4$, etc. 

  Finally, we have to weight our exponential function $2^{(t/13)}$ by the correct initial value 
  ($500$ bacteria); this is done by the factor $500$.
 \end{MHint}
\end{MExercise}
\end{MContent}

\begin{MXContent}{The Natural Exponential Function}{Natural Exponential Function}{STD}
\MLabel{M06_e_fkt}
\MDeclareSiteUXID{VBKM06_Exponentialfunktionen_efkt}
  There is a very special exponential function, sometimes also called \textit{the} exponential function, that we 
  will study now. In fact, all other exponential functions can be reduced to this special exponential function. It has 		
  \MEntry{Euler's number}{Euler's number} $e$ as its base. Its value is (approximately) equal to
  $$\MEU = \MZahl{2}{718281828459045235} \dots \MDFPeriod$$
 
  So, let us consider the graph of \modstextbf{the} exponential function -- 
  for the time being without any additional parameters -- 
  $$
  \begin{array}{rrcl}
  g: & \R & \longrightarrow & (0\MIntvlSep  \infty) \\ & x & \longmapsto & g(x) = \MEU^x \MDFPSpace,
  \end{array}
  $$
  which is, because of its base $\MEU$, also called the \MEntry{$\MEU$ function}{e function} or 
  \MEntry{natural exponential function}{exponential function (natural)}:
  \begin{center}
  \MUGraphicsSolo{e_fkt.png}{scale=1}{width:400px}
  \end{center}
  Unsurprisingly, the natural exponential function shows the typical behaviour of exponential 
  functions $x \Mmapsto a^x$ $(a > 1)$ already discussed in Section~\MRef{M06_allgem_exp_fkt}, 
  after all we have only chosen a special value for the base, namely $a = \MEU$. In particular, we note 
  again that the natural exponential function is \modstextbf{strictly increasing}, i.e. for large negative 
  values of $x$, it approaches the negative $x$-axis, and for $x = 0$, it takes the value $1$.

  \begin{MExercise}
  \MLabel{M06_aufg_e_fkt_2}
  What does the graph of the function $h: \R \rightarrow (0\MIntvlSep  \infty)$, $x \Mmapsto h(x) = \MEU^{- x}$ 
  look like, and which general properties has this function?

  \begin{MHint}{Solution}
   \begin{center}
    \begin{tabular}{ccc}
     \MUGraphicsSolo{e_fkt_2.png}{scale=1}{width:300px} &
     \hspace*{1.5cm} &
     \begin{minipage}[b]{6cm}
      The function $h$ is strictly decreasing, for large positive values of $x$ the graph of $h$ 
      approaches the positive $x$-axis, and for $x = 0$ the function $h$ takes the value $h(x = 0) = 1$.
      \end{minipage}
      \end{tabular}
    \end{center}
    \end{MHint}
  \end{MExercise}
 
  At the beginning of this subsection we claimed that the exponential functions described above 
  can be reduced to the natural exponential function. This is done by means of the identity
  $$a^x = \MEU^{x \cdot\ln(a)} $$
  that is valid for any real number $a > 0$ and any real number $x$. Here, $\ln$ denotes the 
  \MSRef{VBKM06_LNDEF}{natural logarithmic function} that will be studied in detail in the 
  following Section~\MRef{M06_Logarithmus}.
 \begin{MExercise}
  Explain why the identity $a^x = \MEU^{x \cdot \ln(a)}$ is valid. 
  \begin{MHint}{Solution}
  According to the exponent rule $(b^r)^s = b^{r \cdot s}$ the right hand side of the identity in 
  question can be rewritten as $\MEU^{x \cdot\ln(a)} = (\MEU^{\ln(a)})^x$. Since the 
  natural logarithmic function $\ln$ is the inverse of the natural exponential function,
  we have $\MEU^{\ln(a)} = a$.  This implies $(\MEU^{\ln(a)})^x = a^x$ which is indeed the 
  left hand side of the identity.
  \end{MHint}
 \end{MExercise}
 
 
  In \modstextbf{general natural exponential functions}, the parameters $f_0$ and $\lambda$ occur
  that were already introduced in Section~\MRef{M06_allgem_exp_fkt}; thus, its functional description is as 
  follows: 
  $$
  \begin{array}{rrcl}
    f: & \R & \longrightarrow & (0\MIntvlSep  \infty) \\ & x & \longmapsto & f(x) = f_0 \cdot \MEU^{\lambda x} \MDFPeriod
  \end{array}
  $$
  Again, the parameter $f_0$ describes initial values different from $1$, and the factor 
  $\lambda$ in the exponent allows for different (positive or negative) growth rates. 
  This shall be finally illustrated by means of an example. 

  \begin{MExample}
  A series of experiments with radioactive iodine atoms ($^{131}\mathrm{I}$) results  
  in the following mean data:
  \begin{center}
   \begin{tabular}{|c|r|r|r|r|r|}
    \hline
    Number of Iodine Atoms & 10000 & 5000 & 2500 & 1250 & etc. \\ \hline
    Number of Days Elapsed & 0 & \MZahl{8}{04} & \MZahl{16}{08} & \MZahl{24}{12} & etc.\\ \hline
   \end{tabular}
  \end{center}
  In other words: every $\MZahl{8}{04}$~days the number of iodine atoms halves due to radioactive decay. For this reason 
  one says in this context that the \modstextbf{half-life} $h$ of $^{131}\mathrm{I}$ equals $\MZahl{8}{04}$~days.

  The radioactive decay follows an \modstextbf{exponential law}: 
  $$N(t) = N_0 \cdot \MEU^{\lambda t} \MDFPeriod$$
  Our exponential function is here denoted by $N$; it describes the number of remaining iodine atoms.
  Accordingly, $N_0$ denotes the number of iodine atoms at the beginning, i.e. $N_0 = 10000$. 
  The independent variable is in this case the time $t$ (measured in days). We expect the 
  parameter $\lambda$ to be negative since the exponential function describes a decay
  process, i.e. a process with a negative growth rate. We will determine $\lambda$
  from the measurement data.
  
  After $h = \MZahl{8}{04}$~days only $5000$ iodine atoms are still present, i.e. 
  $N(t = \MZahl{8}{04}) = 5000 = \Mtfrac{N_0}{2}$. Using the exponential law for the 
  radioactive decay, we obtain:
  $$\Mdfrac{N_0}{2} = N_0 \cdot \MEU^{\lambda \cdot h}  \MDFPeriod$$
  Now, we can cancel $N_0$ on both sides of the equation and subsequently take the
  natural logarithm of the equation (see Section~\MRef{M06_Logarithmus}): 
  $$\ln \left( \Mdfrac12 \right ) = \ln(\MEU^{\lambda \cdot h})  \MDFPeriod$$
  We transform the left hand side according to the calculation rules for logarithmic functions
  (see Section~\MRef{M06_Logarithmus}): $\ln(1/2) = \ln(1) - \ln(2) = 0 - \ln(2) = - \ln(2)$.
  For the right hand side we note that the natural logarithmic function is the inverse 
  function of the natural exponential function, i.e. $\ln(\MEU^{\lambda \cdot h}) = \lambda \cdot h$;
  thus we have:
  $$\begin{array}{crcl} & - \ln(2) & = & \lambda \cdot h \\[.5ex]
  \Leftrightarrow & \lambda & = & - \Mdfrac{\ln(2)}{h}  \MDFPeriod \end{array}$$
  Inserting the half-life $h = \MZahl{8}{04}$~days of $^{131}\mathrm{I}$ results in this case
  in 
  $$\lambda \approx - \MZahl{0}{0862} \; \Mdfrac{1}{\text{day}}  \MDFPeriod$$
  Other radioactive substances have different half-lifes, e.g. $^{239}\mathrm{Pu}$ has 
  a half-life of $24000$~years, and hence they result in different values of the 
  parameter $\lambda$ in the exponential law for the radioactive decay.
 \end{MExample}
\end{MXContent}



\begin{MXContent}{Logarithmic Function}{Logarithmic Function}{STD}
\MLabel{M06_Logarithmus}
\MDeclareSiteUXID{VBKM06_Logarithmusfunktion}

In Section~\MRef{M06_e_fkt} we studied the natural exponential function
 $$
 \begin{array}{rrcl} g : & \R & \longrightarrow & (0\MIntvlSep  \infty) \\
 & x & \longmapsto & g(x) = \MEU^x  \MDFPeriod
 \end{array}
 $$
In particular, we pointed out a very important property of the natural exponential function: it is strictly increasing. If the graph of this function is reflected about the angle 
bisector between the first and the third quadrant, one obtains the graph of the natural logarithmic function
that has the symbol $\ln$:
 
\begin{MInfo}
\MLabel{VBKM06_LNDEF}
The function defined by the equation $\MEU^{\ln(x)}=x$
$$
\function{\ln}{(0\MIntvlSep \infty)}{\R}{x}{\ln(x)}
$$
is called the \MEntry{natural logarithmic function}{logarithmic function (natural)}.
\end{MInfo}

Here, the equation shall be read in such a way that $\ln(x)=a$ is just the value $a$ with $\MEU^a=x$.
The construction mentioned above is shown in the figure below.
 \begin{center}
 \MUGraphicsSolo{ln_fkt.png}{scale=1}{}
 \end{center}

The following properties of the natural logarithmic function can be seen from 
the graph:
 \begin{itemize}
  \item The function $\ln$ is strictly increasing.
  \item Approaching zero from the right on the $x$-axis, $\ln(x)$ takes ever larger negative values: 
  We note that the graph of $\ln$ gets arbitrarily close to the negative vertical axis ($y$-axis).
  \item At the point $x = 1$ the natural logarithmic function takes the value $0$, i.e. $\ln(1) = 0$.
 \end{itemize}
 
As well as the natural logarithmic function there are other logarithmic functions, which each correspond to a certain 
exponent.

\begin{MInfo}
If $b>0$ is an arbitrary exponent, then the function defined by the equation 
$b^{\log_b(x)}=x$ (read as: ``$\log_b(x)=a$ is the exponent $a$ with $b^a=x$'')
$$
\function{\log_b}{(0\MIntvlSep \infty)}{\R}{x}{\log_b(x)}
$$
is called the \MEntry{general logarithmic function}{logarithmic function (general)} with base $b$.
\end{MInfo}

Normally, the logarithmic function cannot be calculated directly. Since it is defined as the inverse function 
of the exponential function, one generally tries to rewrite its input as a power and reads off the exponent. 

\begin{MExample}
Typical calculations for the natural logarithmic function are
$$
\ln(\MEU^5) \;=\; 5\MDFPSpace, \MDFPaSpace \ln(\sqrt{\MEU}) \;=\; \ln(\MEU^{\frac12}) \;=\; \frac12
$$
and for the general logarithmic function:
$$
\log_5(25) \;=\; \log_5(5^2) \;=\; 2 \MDFPSpace, \MDFPaSpace \log_3(81) \;=\; \log_3(3^4) \;=\; 4 \MDFPeriod
$$
\end{MExample}

Here, the base of the logarithmic function has to be observed, for example, we have
$$
\log_2(64) \;=\; \log_2(2^6) \;=\; 6 \MDFPSpace,\MDFPSpace \text{ but }\MDFPSpace
\log_4(64) \;=\; \log_4(4^3) \;=\; 3 \MDFPeriod
$$

\begin{MExercise}
Calculate the values of the following logarithmic functions:
\begin{MExerciseItems}
\item{\MEquationItem{$\displaystyle\ln(\sqrt[3]{\MEU})$}{\MLSimplifyQuestion{10}{1/3}{5}{}{5}{1}{LNA1}}. \begin{MHint}{Solution}We have $\ln(\sqrt[3]{\MEU})=\ln(\MEU^{\frac13})=\frac13$.\end{MHint}}
\item{\MEquationItem{$\displaystyle\log_2(256)$}{\MLSimplifyQuestion{10}{8}{5}{}{5}{1}{LNA2}}. \begin{MHint}{Solution}We have $\log_2(256)=\log_2(2^8)=8$.\end{MHint}}
\item{\MEquationItem{$\displaystyle\log_9(3)$}{\MLSimplifyQuestion{10}{1/2}{5}{}{5}{1}{LNA3}}. \begin{MHint}{Solution}We have $\log_9(3)=\log_9(9^{\frac12})=\frac12$.\end{MHint}}
\end{MExerciseItems}
\end{MExercise}

In mathematics and in the sciences the following logarithmic functions are frequently used and 
thus have their own dedicated names:
\begin{itemize}
\item{Logarithmic function to the base $10$: denoted by $\log_{10}(x)=\lg(x)$ or sometimes only by $\log(x)$. This 
logarithmic function is associated to the powers of ten and is used, for example, in chemistry for the calculation of 
the pH level.}
\item{Logarithmic function to the base $2$: denoted by $\log_2(x)=\ld(x)$. This 
logarithmic function is relevant in computer science.}
\item{Logarithmic function to the base $\MEU$: denoted by $\log_\MEU(x)=\ln(\MEU)$. The natural logarithmic function
is mostly inadequate for practical calculations (unless the expression is a power of $\MEU$). It is called 
natural since the exponential function with base $\MEU$, from a mathematical point of view, is simpler than 
the general exponential function (e.g. because $\MEU^x$ is its own derivative, but $b^x$ for $b\not=\MEU$ is not).}
\end{itemize}

There are several calculation rules for the logarithmic function, which will be explained in the next section.
\end{MXContent}

\begin{MXContent}{Logarithm Rules}{Logarithm Rules}{STD}
\MDeclareSiteUXID{VBKM06_Logarithmusgesetze}
 

For calculations involving logarithmic functions certain rules apply
that can be derived form the \MSRef{VBKM01_Potenzgesetze}{exponent rules}.
 
 
 \begin{MInfo}
  \MLabel{M06_log_ges}
  The following rules are called \MEntry{logarithm rules}{logarithm rules}:
  $$\begin{array}{rcll}
   \log (u \cdot v) & = & \log(u) + \log(v) & (u, v >0) \MDFPSpace, \\[.5ex]
   \log \left( \Mdfrac{u}{v} \right) & = & \log (u) - \log (v) & (u, v >0) \MDFPSpace, \\[.5ex]
   \log (u^x) & = & x \cdot \log(u) & (u>0, x \in \R) \MDFPeriod
  \end{array}$$
 \end{MInfo}
 
  These rules do not only apply to natural logarithmic functions but also to all other 
  logarithmic functions. They can be used to transform a given expression in such a way that 
  the power occurs only in the logarithmic terms.

 
 \begin{MExample}
 For example, the value $\ld(4^5)$ can be calculated applying the logarithm rules:
 $$
 \ld(8^5) \;=\; \log_2(8^5) \;=\;5\cdot \log_2(8) \;=\; 5\cdot \log_2(2^3) \;=\; 5\cdot 3 \;=\; 15 \MDFPeriod
 $$
 Products in logarithmic functions can be split into sums outside the logarithmic functions:
 $$
 \lg\left({100\cdot \sqrt{10}\cdot \frac1{10}}\right) \;=\; \lg(100)+\lg(\sqrt{10})-\lg(10) 
  \;=\; 2+\frac12-1 \;=\; \frac32 \MDFPeriod
 $$
 \end{MExample}

 
 Importantly, the splitting rule $\log(u\cdot v)=\log(u)+\log(v)$ transforms products into sums. 
 The other way round is impossible for logarithmic functions: the logarithm of a sum 
 cannot be transformed any further.
\end{MXContent}



\MSubsection{Trigonometric Functions}
\MLabel{VBKM06_trigonometrisch}

\begin{MIntro}
\MDeclareSiteUXID{VBKM06_TrigonometrischeFunktionen_Intro}
 

\MEntry{Trigonometry}{trigonometry}, according to its origin in the Greek words trigonon for ``triangle'' and metron 
for ``measure'', is the \modstextbf{measuring theory of (angles and sides of) triangles}. In this theory, 
the \MEntry{trigonometric functions}{trigonometric functions} \modstextbf{sine function}, \modstextbf{cosine function} 
and \modstextbf{tangent function}, play a central role. 

However, the field of applications of sine, cosine, tangent etc is not restricted 
to ``simple'' triangle calculations. In fact, the trigonometric functions show their real potential 
in the manifold fields of applications. The most relevant of those are probably in the description of 
oscillation processes and wave phenomena in physics and engineering. However, they are also applied 
in many other fields as, for example, in geodesy and astronomy.
 
\end{MIntro}

\begin{MXContent}{Sine Function}{Sine Function}{STD}
\MLabel{M06_Sinusfkt}
\MDeclareSiteUXID{VBKM06_TrigonometrischeFunktionen_Sinus}
In Module~\MRef{VBKM05} the trigonometric functions were introduced in Section~\MRef{M05_Trigonometrie}
on right triangles, for example by the relation
$$\begin{array}{rcccccl}
  \sin(\alpha) & = & \Mdfrac{\text{opposite side}}{\text{hypotenuse}\MDFPSpace ,} & & \\[2ex]
\end{array}$$
and explained using the \MSRef{VBKM05_Trigonometrie_Einheitskreis}{unit circle}. Starting from this 
definition of $\sin(\alpha)$ we arrive at the \MEntry{sine function}{sine function} if we declare  
the angle $\alpha$ to be the independent variable of a function named $\sin$. This can be illustrated 
by means of a family of right triangles $A B C$ that are inscribed the \MEntry{unit circle}{unit circle}, 
i.e. a circle with radius $r = 1$, in a certain way.
\begin{center}
 \MUGraphicsSolo{sin_fkt_im_kreis.png}{scale=0.83}{width:400px}
\end{center}

If we start with the angle $\alpha = 0^\circ$, corresponding to a triangle degenerated to a line segment, then 
the length of the line segment $\overline{B C}$ equals $0$. If we now rotate the point $B$ \modstextbf{counterclockwise}
around the circle, then the angle $\alpha$ -- and $\sin(\alpha)$ as well -- increases until for 
$\alpha = 90^\circ$ a \modstextbf{maximum value} ($\sin(90^\circ) = 1$) is reached. Afterwards, the angle 
$\alpha$ continues to increase while $\sin(\alpha)$ is starting to decrease again. For $\alpha = 180^\circ$, 
the triangle $A B C$ is again degenerated to a line segment and $\sin(180^\circ) = 0$. If 
$\alpha$ further increases, the triangle ``flips downwards'' and the line segment $\overline{B C}$ is 
oriented parallel to the negative vertical line ($y$-axis), hence, its length is \modstextbf{negative}.
For $\alpha = 270^\circ$ the \modstextbf{maximum negative value} occurs, before $\alpha$ approaches 
$0$ again. At $\alpha = 360^\circ$ the game starts again.
 \begin{center}
  \MUGraphicsSolo{sin_fkt.png}{scale=1}{width:400px}
 \end{center}
The figure above shows the graph of the sine function
 $$
 \begin{array}{rrcl}
  \sin : & \R & \longrightarrow & [- 1; + 1] \\ & \alpha & \longmapsto & \sin(\alpha) \MDFPeriod
 \end{array}
 $$
 In contrast to the discussion so far, the angle $\alpha$ is plotted on the horizontal axis ($\alpha$-axis)
 in \MSRef{VBKM05_Def_Bogenmass}{radian measure}, used more commonly in this context, and not in degree measure.
 
  Let us specify some of the most relevant properties of the sine function:
 \begin{itemize}
  \item{The sine function is defined on the entire set of real numbers $\R$. Hence, $D_{\sin} = \R$. In contrast, 
  the range only consists of the interval between $-1$ and $+ 1$, including these two endpoints: 
  $W_{\sin} = [- 1\MIntvlSep  + 1]$.}


  \item{After certain measures, the graph of the since function repeats its shape exactly; in this context 
  this is called the periodicity of the sine function. The \MEntry{period}{period (sine)} of the 
  sine function is $360^\circ$ or $2 \pi$. Mathematically, this relation can be expressed as 
  $$\sin(\alpha) = \sin(\alpha + 2 \pi)\MDFPeriod$$}
 \end{itemize}
 
  Just a quick look at the graph of the simple sine function suggests that we could use this function for the description 
  of wave phenomena. However, to be able to use the full potential of the sine function a few parameters will be 
  introduced. For example, the amplitude of the sine function can be amplified or damped by a 
  so-called \modstextbf{amplitude factor} $A$, the frequency of the oscillation can be affected by a 
  \modstextbf{frequency-like} factor $a$, and the entire path of the graph can be shifted to the left or to the right 
  by a \modstextbf{shifting constant} $b$. Thus, the \modstextbf{general sine function} has the following form:
 $$\begin{array}{rrcl}
  f : & \R & \longrightarrow & [- A\MIntvlSep  + A] \\
  & x & \longmapsto & f(x) = A \, \sin (a x + b) \MDFPeriod 
 \end{array}$$
 \begin{MExample}
  \MLabel{M06_Fadenpendel}
  Let us consider a simple pendulum. A small, heavy weight swings freely in the gravitational field of the 
  Earth at a long, very thin cord that is, for example, fixed at the ceiling of a (high) room. Under certain 
  idealised conditions and for small values of the displacement angle $\Mvarphi$ from the 
  \modstextbf{rest position} (the \modstextbf{vertical axis}), the relation between the angle $\Mvarphi$  
  and the independent variable $t$, i.e. the time, is described by a general sine function:
  $$\Mvarphi (t) = A \, \sin(\Mdfrac{2 \pi}{T} t + b) \MDFPeriod$$
  Here, $T$ denotes the so-called \modstextbf{period} of the pendulum, i.e. the period of time, required 
  by the pendulum for a full oscillation.
 \end{MExample}
\end{MXContent}

\begin{MXContent}{Cosine and Tangent Function}{Cosine, Tangent, and Cotangent}{STD}\MLabel{sec:cos}
\MDeclareSiteUXID{VBKM06_TrigonometrischeFunktionen_CosTan}
  Essentially, for the cosine function and the tangent function we have to do the same considerations 
  as for the sine function that we already know from Subsection~\MRef{M06_Sinusfkt}. As we have some 
  experience, we can shorten the discussion a bit. We start with the 
  \MEntry{cosine function}{cosine function} and consider again our triangles inscribed the unit circle.
 \begin{center}
  \MUGraphicsSolo{cos_fkt_im_kreis.png}{scale=1}{width:400px}
 \end{center}
  Again, all hypotenuses of these right triangles have the length $1$, such that the  
  cosines of the angles $\alpha$ occur as the lengths of the line segments $\overline{A C}$ in 
  the figure. If we uniformly rotate the point $B$ \modstextbf{counterclockwise} around the circle, varying the 
  angle $\alpha$, we finally obtain the \modstextbf{cosine function}:
 $$
 \begin{array}{rrcl}
  \cos : & \R & \longrightarrow & [- 1; + 1] \\ & \alpha & \longmapsto & \cos(\alpha)\MDFPeriod
 \end{array} 
 $$
 \begin{center}
  \MUGraphicsSolo{cos_fkt.png}{scale=1}{width:400px}
 \end{center}
  The figure above shows the graph of the cosine function(solid line) and the graph 
  of the sine function (dotted line) side by side for comparison. We see a very strong relationship, which we will
  discuss later.

  What are the relevant properties of the cosine function?
 
 \begin{itemize}
  \item The \modstextbf{cosine function} is also a \modstextbf{periodic function}. The \modstextbf{period} 
  is again $2 \pi$ or $360^\circ$.
  \item The domain of the cosine function consists of the entire set of the real numbers. Hence, $D_{\cos} = \R$.
  The range is the interval between $-1$ and $+1$, including the endpoints: $W_{\cos} = [- 1\MIntvlSep  + 1]$.
  \item From the figure above, showing $\cos(\alpha)$ and $\sin(\alpha)$, we immediately see that
  $$\cos(\alpha) = \sin \left( \alpha + \Mdfrac{\pi}{2} \right)$$
  for all real values of $\alpha$. Also true, but not as obvious, is the relation
    $$\cos(\alpha) = - \sin \left( \alpha - \Mdfrac{\pi}{2} \right) \MDFPeriod$$
 \end{itemize}
 \begin{MExercise}
  \MLabel{M06_cos_diskussion}
  At which points does the cosine function take its maximum positive value $1$, and at which points does it take its maximum negative value $- 1$?
  What are its roots (at which points is the function $0$)?
  \begin{MHint}{Solution}
   We have $\cos(0) = 1$. Because of the periodicity with period $2 \pi$ this is also true 
  for all $\alpha$ equal to $\pm 2 \pi, \pm 2 \cdot 2 \pi,
  \pm 3 \cdot 2 \pi, \dots$. Hence, the cosine function takes its maximum value $1$ for all integer 
  multiples of $2 \pi$ (or all even multiples of $\pi$). This can also be written as:
  $$\cos(\alpha) = 1 \Leftrightarrow \alpha \in \{ 2 k \cdot \pi\MCondSetSep k \in \Z \} \MDFPeriod $$
  The value $- 1$ is taken by the cosine function at the points $\dots, - 3 \pi, - \pi, \pi, 3 \pi, 5 \pi, \dots$, i.e.
  for all odd multiples of $\pi$:
   $$\cos(\alpha) = - 1 \Leftrightarrow \alpha \in \{ (2 k + 1) \cdot \pi\MCondSetSep k \in \Z \} \MDFPeriod $$
  Roots occur at $\dots, - \Mtfrac{3}{2} \pi, - \Mtfrac{1}{2} \pi, \Mtfrac{1}{2} \pi, \Mtfrac32 \pi, \dots$, i.e. 
  at half-integer multiples of $\pi$:
   $$\cos(\alpha) = 0 \Leftrightarrow \alpha \in \{ \Mtfrac{2k+1}{2} \cdot \pi; k \in \Z \} \MDFPeriod$$
  \end{MHint}
 \end{MExercise}
  
As in the case of the sine, the cosine has also a general cosine function. In its definition additional degrees of freedom
occur in form of parameters (\modstextbf{amplitude factor} $B$, \modstextbf{frequency factor} $c$, and
\modstextbf{shifting constant} $d$). In this way, it is possible to fit the function's graph 
to different situations (in application examples):
 $$
 \begin{array}{rrcl}
  g : & \R & \longrightarrow & [- A; + A] \\
  & x & \longmapsto & g(x) = B \, \cos (c x + d)  
 \end{array} \MDFPeriod
 $$
 \begin{MExercise}
  In Example~\MRef{M06_Fadenpendel} we briefly discussed the simple pendulum. In particular, 
  the displacement angle $\Mvarphi$ of the pendulum can be determined as a function of time under the condition that 
  the period $T$ equals $\pi$~seconds and that the pendulum at $t = 0$ is started with an initial 
  displacement angle of $30^\circ$:
  $$\Mvarphi (t) = \Mdfrac{\pi}{6} \cdot\sin \left(2 t + \Mdfrac{\pi}{2}\right) \MDFPeriod$$
  Can this situation also be described using the (general) cosine function (instead of the sine function), 
  and if so, what form does $\Mvarphi (t)$ take in this case?

\begin{MHint}{Solution}
  The answer to the first question is: ``Yes, it is possible to describe the present situation 
  using the cosine function'' (as we will see in a moment). 

  In principle, we could start with the general cosine function $g$ given above and determine the parameters 
  $B$, $c$ and $d$ for the present situation using the same consideration as in Example~\MRef{M06_Fadenpendel}. 
  However, it is easier to apply the relation $\cos(\alpha) = \sin(\alpha + \Mdfrac{\pi}{2})$ between 
  cosine and sine functions since then, it immediately follows that
   $$\sin(2 t + \Mdfrac{\pi}{2}) = \cos(2 t) \MDFPSpace,$$
   and thus:
   $$\Mvarphi (t) = \Mdfrac{\pi}{6}\cdot\cos(2 t) \MDFPeriod$$
  \end{MHint}
 \end{MExercise}
 
  The tangent is the ratio of sine to cosine. Thus, it follows immediately that the 
  \MEntry{tangent function}{tangent function} cannot be defined for all real numbers since 
  finally the cosine function has an infinite number of roots. This can be seen, for example, 
  in Exercise~\MRef{M06_cos_diskussion}. In Exercise~\MRef{M06_cos_diskussion} also the positions 
  of the roots of the cosine function are determined, namely $\cos(\alpha) = 0
 \Leftrightarrow \alpha \in \{ \Mtfrac{2k+1}{2} \cdot \pi; k \in \Z \}$. Thus, the domain 
  of the tangent function is $D_{\tan} = \R \setminus \{ \Mtfrac{2k+1}{2} \cdot \pi; k \in \Z \}$.

  And what about the range? At the roots of the cosine function the tangent function tends
  to infinite positive or negative values and has a pole, and at the root of the sine function
  the ratio of sine and cosine is zero. In between, all values can be taken by the tangent 
  function, and hence $W_{\tan} = \R$. All in all, for the tangent function we have
  $$
  \begin{array}{rrcl}
    \tan : & \R \setminus \{ \Mtfrac{2k+1}{2} \cdot \pi; k \in \Z \} & \longrightarrow & \R \\
    & \alpha & \longmapsto & \tan(\alpha)
  \end{array}
  $$
  The graph of the function is shown in the figure below. 
 \begin{center}
  \MUGraphicsSolo{tan_fkt.png}{scale=1}{width:400px}
 \end{center}

  In addition, the tangent function is periodic, however, the period is $\pi$ or $180^\circ$.
 
 \begin{MExercise}
 The so-called cotangent function (abbreviated to $\cot$) is defined by
 $\cot (\alpha) = \frac1{\tan(\alpha)}= \Mdfrac{\cos(\alpha)}{\sin(\alpha)}$.
 
  Specify the domain and the range of the cotangent function.

  \begin{MHint}{Solution}
  The poles of the cotangent functions are at the points where the sine function is $0$.
  This is the case if $\alpha$ is an integer multiple of $\pi$. Thus, we have to exclude these points 
  in the definition of the cotangent function:
   $$D_{\cot} = \R \setminus \{ k \cdot \pi \MCondSetSep k \in \Z \} \MDFPeriod$$

  For the determination of the range, the considerations are very similar to the ones for the range of the tangent function. 
  We thus have $W_{\cot} = \R$.
   \begin{center}
    \MUGraphicsSolo{cot_fkt.png}{scale=1}{width:400px}
   \end{center}
  \end{MHint}
 \end{MExercise}
\end{MXContent}


\MSubsection{Properties and Construction of Elementary Functions}
\MLabel{VBKM06_eigenschaften}

\begin{MIntro}
\MDeclareSiteUXID{VBKM06_Eigenschaften_Intro}

In this section, we will consider a further property of elementary functions that we have not 
discussed yet in previous sections: the symmetry of functions. Furthermore, 
we will investigate how to construct new functions from known elementary functions. For this
purpose sums, products and compositions of functions are introduced.
\end{MIntro}


\begin{MXContent}{Symmetry}{Symmetry}{STD}
\MDeclareSiteUXID{VBKM06_Eigenschaften_Symmetrie}

\begin{MInfo}
A function $f\colon \R\lto\R$ is called \highlight{even} or \highlight{axially symmetric} if, for all $x\in\R$, we have
\[
 f(x)=f(-x)
\MDFPeriod
\]
Analogously, a function is called \highlight{odd} or \highlight{centrally symmetric} if, for all $x\in\R$, we have
\[
 f(x)=-f(-x)\MDFPeriod
\]
\end{MInfo}

These two symmetry conditions for functions allow us to make conclusions about the behaviour of the graphs. 
For even functions, a reflection across the vertical axis does not change the graph, and for odd 
functions, a reflection across the origin does not change the graph. A few illustrative examples 
are listed below.


\begin{MExample}
\begin{itemize}
 \item The functions
 \[
  \function{f_1}{\R}{\R}{x}{x^2}
 \]
 and
 \[
  \function{f_2}{\R}{\R}{x}{|x|\MDFPSpace,}
 \]
 i.e. the standard parabola (see Section~\MNRef{sec:monome}) and the absolute value function
 (see Section~\MNRef{VBKM06_sec:betrag}), are examples of even functions. We have 
  $f_1(-x)=(-x)^2=x^2=f_1(x)$ and $f_2(-x)=|-x|=|x|=f_2(x)$ for all $x\in\R$. 
  The graphs of these two functions are symmetric under reflection across the vertical axis.

%GRAPH: sym1
\MTikzAuto{%
\begin{tikzpicture} 
%Koordinatensystem
\node (xMAX) at (2.8,0){};
\node (yMAX) at (0,4.8){};
\draw[->,color=black] (-2.5,0) -- (xMAX);
\foreach \x in {-2,-1,1,2}
\draw[shift={(\x,0)},color=black] (0pt,2pt) -- (0pt,-2pt) node[below] {\footnotesize $\x$};
\draw[->,color=black] (0,-0.5) -- (yMAX);
\foreach \y in {1,2,3,4}
\draw[shift={(0,\y)},color=black] (2pt,0pt) -- (-2pt,0pt) node[left] {\footnotesize $\y$};
\draw[color=black] (0pt,-10pt) node[right] {\footnotesize $0$};
%Achsenbeschriftung
\draw (xMAX) node[anchor=north east] {$x$};
\draw (yMAX) node[anchor=east] {$f_1(x),f_2(x)$};
% Graphen
\draw[color=red,smooth,samples=50,domain=-2.1:2.1] plot(\x,{(\x)^(2.0)});
\draw[color=red] (2.1,4.41) node[anchor=south] {$G_{f_1}$};
\draw[color=blue] (0,0) -- (2,2);
\draw[color=blue] (0,0) -- (-2,2);
\draw[color=blue] (-2.1,2.1) node[anchor=south] {$G_{f_2}$};
\end{tikzpicture}  
}%

 \item The function
 \[
  \function{g}{\R}{\R}{x}{x^3 \MDFPSpace,}
 \]
  i.e. the cubic parabola (see Section~\MNRef{sec:monome}), is an example of an odd function. We have 
  $g(-x)=(-x)^3=-x^3=-g(x)$ for all $x\in\R$. The graph of the function is centrally symmetric with respect 
  to the origin.

%GRAPH: sym2
\MTikzAuto{%
\begin{tikzpicture} 
%Koordinatensystem
\node (xMAX) at (2.8,0){};
\node (yMAX) at (0,4.8){};
\draw[->,color=black] (-2.5,0) -- (xMAX);
\foreach \x in {-2,-1,1,2}
\draw[shift={(\x,0)},color=black] (0pt,2pt) -- (0pt,-2pt) node[below] {\footnotesize $\x$};
\draw[->,color=black] (0,-3.5) -- (yMAX);
\foreach \y in {-3,-2,-1,1,2,3,4}
\draw[shift={(0,\y)},color=black] (2pt,0pt) -- (-2pt,0pt) node[left] {\footnotesize $\y$};
\draw[color=black] (0pt,-10pt) node[right] {\footnotesize $0$};
%Achsenbeschriftung
\draw (xMAX) node[anchor=north east] {$x$};
\draw (yMAX) node[anchor=east] {$g$};
\clip(-2.5,-3.5) rectangle (2.5,4.5);
% Graphen
\draw[color = red, smooth,samples=50,domain=-1.9:1.9] plot(\x,{(\x)^(3.0)}); \draw[color = red](-1.4,-3)node[left]{$G_{g}$};
%\draw[color=blue] (0,1) -- (2,1);
%\draw[color=blue] (0,-1) -- (-2,-1);
%\draw[color=blue, fill=blue] (0,0) circle (1.5pt);
%\draw[color=blue, fill=white] (0,1) circle (1.5pt);
%\draw[color=blue, fill=white] (0,-1) circle (1.5pt);
%\draw[color=blue] (2,1) node[anchor=south] {$G_{g_2}$};
\end{tikzpicture} 
}%

 

\end{itemize}
\end{MExample}

Of course, the symmetry properties of functions can also be used if the domain of the function is not the entire 
set of real numbers. However, then the domain must contain the number $0$ \modsemph{in the middle of the interval}. 
An example of this case is the tangent function in the exercise below.

\begin{MExercise}
Specify whether the following functions are even, odd or non-symmetric.

\begin{itemize}
 \item[a)] \[\function{f}{\R}{\R}{x}{\MEU^x}\]
 \item[b)] \[\function{g}{\R}{\R}{y}{\sin(y)}\]
 \item[c)] \[\function{h}{(-\frac{\pi}{2}\MIntvlSep \frac{\pi}{2})}{\R}{\alpha}{\tan(\alpha)}\]
 \item[d)] \[\function{i}{\R}{\R}{u}{\cos(u)}\]
 \item[e)] \[\function{j}{\R}{\R}{x}{42}\]
\end{itemize}
\begin{MHint}{Solution}
a) non-symmetric, b) odd, c) odd, d) even, e) even 
\end{MHint}

\end{MExercise}
\end{MXContent}



\begin{MXContent}{Sums, Products, Compositions}{Sums, Products, Compositions}{STD}\MLabel{Verkettung}
\MDeclareSiteUXID{VBKM06_SummeProduktVerkettung}


In this section, we will now use the large collection of elementary functions that we acquired in this 
module to create new more complex functions out of this elementary functions. At different places throughout this 
module we already studied functions those mapping rules were composed of sums and products of simpler mapping 
rules. Of course, one can also take differences and, under certain conditions, quotients from mapping 
rule. The example below lists a few such combined functions.

\begin{MExample}
\begin{itemize}
 \item The function 
 \[
  \function{f}{\R}{\R}{x}{x+\sin(x)}
 \]
  is the \modsemph{sum} of the identity function (see Section~\MNRef{sec:linear}) and the sine function 
  (see Section~\MNRef{VBKM06_trigonometrisch}). The graph of this function is shown in the figure below.

 %GRAPH: summ1
\MTikzAuto{%
\begin{tikzpicture}[scale=0.5]
%Koordinatensystem
% x-Achse
\node (xMAX) at (8,0){};
\draw[->,color=black] (-7.5,0) -- (xMAX);
\foreach \x in {-7,-6,-5,-4,-3,-2,-1,1,2,3,4,5,6,7}
\draw[shift={(\x,0)},color=black] (0pt,2pt) -- (0pt,-2pt) node[below] {\footnotesize $\x$};
% y-Achse
\node (yMAX) at (0,7.8){};
\draw[->,color=black] (0,-7.5) -- (yMAX);
\foreach \y in {-7,-6,-5,-4,-3,-2,-1,1,2,3,4,5,6,7}
\draw[shift={(0,\y)},color=black] (2pt,0pt) -- (-2pt,0pt) node[left] {\footnotesize $\y$};
\draw[color=black] (0pt,-10pt) node[right] {\footnotesize $0$};
%Achsenbeschriftung
\draw (xMAX) node[anchor=north east] {$x$};
\draw (yMAX) node[anchor=east] {$f(x)$};
%Graph
\draw[color=red,samples=100,domain=-6.2:6.2] plot(\x,{\x + sin(\x r)}); 
\draw[color=red] (2,2) node[anchor=north] {$G_f$};
\end{tikzpicture} 
}%

 \item The function
 \[
  \function{g}{[1\MIntvlSep 2]}{\R}{y}{y^2-\ln(y)}
 \]
  is the \modsemph{difference} of the standard parabola (see Section~\MNRef{sec:monome}) and the 
  natural logarithmic function (see Section~\MNRef{M06_Logarithmus}). The graph of this function is shown 
  in the figure below.%

  %GRAPH: summ2
\MTikzAuto{%
\begin{tikzpicture}
%Koordinatensystem
% x-Achse
\node (xMAX) at (3.8,0){};
\draw[->,color=black] (-0.5,0) -- (xMAX);
\foreach \x in {1,2,3}
\draw[shift={(\x,0)},color=black] (0pt,2pt) -- (0pt,-2pt) node[below] {\footnotesize $\x$};
% y-Achse
\node (yMAX) at (0,4.8){};
\draw[->,color=black] (0,-0.5) -- (yMAX);
\foreach \y in {1,2,3,4}
\draw[shift={(0,\y)},color=black] (2pt,0pt) -- (-2pt,0pt) node[left] {\footnotesize $\y$};
\draw[color=black] (0pt,-10pt) node[right] {\footnotesize $0$};
%Achsenbeschriftung
\draw (xMAX) node[anchor=north east] {$y$};
\draw (yMAX) node[anchor=east] {$g(y)$};
%Graph
\draw[color=red,samples=50,domain=1:2] plot(\x,{\x*\x - ln(\x)}); 
\draw[color=red,fill=red] (1,1) circle (1pt);
\draw[color=red,fill=red] (2,3.3) circle (1pt);
\draw[color=red] (.8,1) node[anchor=south] {$G_g$};
\end{tikzpicture} 
}%

 \item The function
 \[
  \function{h}{(0\MIntvlSep \infty)}{\R}{x}{\MEU^x\frac{1}{x}}
 \]
 is the \modsemph{product} of the natural exponential function with the mapping rule 
 $\MEU^x$ (see Section~\MNRef{M06_e_fkt}) and the hyperbola with the 
 mapping rule $\frac{1}{x}$ (see Section~\MNRef{sec:hyperbel}). The graph of this function is shown 
  in the figure below.
 
   %GRAPH: summ3
\MTikzAuto{%
\begin{tikzpicture}
%Koordinatensystem
% x-Achse
\node (xMAX) at (3.8,0){};
\draw[->,color=black] (-0.5,0) -- (xMAX);
\foreach \x in {1,2,3}
\draw[shift={(\x,0)},color=black] (0pt,2pt) -- (0pt,-2pt) node[below] {\footnotesize $\x$};
% y-Achse
\node (yMAX) at (0,7.8){};
\draw[->,color=black] (0,-0.5) -- (yMAX);
\foreach \y in {1,2,3,4,5,6,7}
\draw[shift={(0,\y)},color=black] (2pt,0pt) -- (-2pt,0pt) node[left] {\footnotesize $\y$};
\draw[color=black] (0pt,-10pt) node[right] {\footnotesize $0$};
%Achsenbeschriftung
\draw (xMAX) node[anchor=north east] {$x$};
\draw (yMAX) node[anchor=east] {$h(x)$};
%Graph
\draw[color=red,samples=100,domain=0.2:3] plot(\x,{exp(\x)/\x}); 
\draw[color=red] (3.1,6) node[anchor=north] {$G_h$};
\end{tikzpicture} 
}%

 \item The function
 \[
  \function{\Mvarphi}{\R}{\R}{z}{\frac{cos(z)}{z^2+1}}
 \]
  is the \modsemph{quotient} of the cosine function (see Section~\MNRef{sec:cos}) and the 
  polynomial of degree $2$ (see Section~\MNRef{sec:polynome}) with the mapping rule $z^2+1$. 
  The graph of this function is shown in the figure below.
 
  %GRAPH: summ4
\MTikzAuto{%
\begin{tikzpicture}[y=3cm]
%Koordinatensystem
% x-Achse
\node (xMAX) at (8,0){};
\draw[->,color=black] (-7.5,0) -- (xMAX);
\foreach \x in {-7,-6,-5,-4,-3,-2,-1,1,2,3,4,5,6,7}
\draw[shift={(\x,0)},color=black] (0pt,2pt) -- (0pt,-2pt) node[below] {\footnotesize $\x$};
% y-Achse
\node (yMAX) at (0,1.8){};
\draw[->,color=black] (0,-1.5) -- (yMAX);
\foreach \y in {-1,1}
\draw[shift={(0,\y)},color=black] (2pt,0pt) -- (-2pt,0pt) node[left] {\footnotesize $\y$};
\draw[color=black] (0pt,-10pt) node[right] {\footnotesize $0$};
%Achsenbeschriftung
\draw (xMAX) node[anchor=north east] {$z$};
\draw (yMAX) node[anchor=east] {$\Mvarphi(z)$};
%Graph
\draw[color=red,samples=100,domain=-7:7] plot(\x,{cos(\x r)/(\x*\x + 1)}); 
\draw[color=red] (1,0.7) node[anchor=south] {$G_{\Mvarphi}$};
\end{tikzpicture}
}%
\end{itemize}
 
\end{MExample}


\begin{MExercise}
Find further examples for elementary functions we studied in this module that were constructed from 
simpler elementary functions by combining sums, differences, product, or quotients. 

\begin{MHint}{Solution}
Further examples are:
\begin{itemize}
 \item Functions of hyperbolic type (see Section~\MNRef{sec:hyperbel}) are all quotients 
  of the constant function $1$ and a monomial.
 \item Monomials (see Section~\MNRef{sec:monome}) are all multiple products of the 
  identity function $\Id(x)=x$.
 \item Linear functions (see Section~\MNRef{sec:linear}) are products of constant functions 
  describing the slope and the identity function.
 \item Polynomials (see Section~\MNRef{sec:polynome}) are sums and differences of functions 
  that are themselves products of constant functions and monomials.
\end{itemize}
 
\end{MHint}

\end{MExercise}

Finally, there is another way to combine elementary functions to obtain new functions. This is the 
so-called \highlight{composition} of functions. 

Let us consider a few examples.

\begin{MExample}
\begin{itemize}
 \item The functions
 \[
  \function{f}{\R}{\R}{x}{f(x)=x^2+1}
 \]
 and
 \[
  \function{g}{\R}{\R}{x}{g(x)=\MEU^x}
 \]
 can be composed in two ways. We can compose the function $f\circ g\colon\R\lto\R$ or the 
function $g\circ f\colon \R\lto\R$. We obtain
 \[
  (f\circ g)(x)=f(g(x))=f(\MEU^x)=(\MEU^x)^2+1=\MEU^{2x}+1 \MDFPSpace,
 \]
 i.e.
 \[
  \function{f\circ g}{\R}{\R}{x}{\MEU^{2x}+1 \MDFPSpace,}
 \]
 and
 \[
  (g\circ f)(x)=g(f(x))=g(x^2+1)=\MEU^{x^2+1} \MDFPSpace,
 \]
 i.e.
 \[
  \function{g\circ f}{\R}{\R}{x}{\MEU^{x^2+1} \MDFPeriod}
 \]
  If we look at the graphs, we see that these are two completely different functions, i.e. 
the order of the composition is relevant.
 
   %GRAPH: kett1
\MTikzAuto{%
\begin{tikzpicture}[x=2cm,y=0.5cm]
%Koordinatensystem
% x-Achse
\node (xMAX) at (1.8,0){};
\draw[->,color=black] (-1.5,0) -- (xMAX);
\foreach \x in {-1,1}
\draw[shift={(\x,0)},color=black] (0pt,2pt) -- (0pt,-2pt) node[below] {\footnotesize $\x$};
% y-Achse
\node (yMAX) at (0,8.8){};
\draw[->,color=black] (0,-0.5) -- (yMAX);
\foreach \y in {1,2,3,4,5,6,7,8}
\draw[shift={(0,\y)},color=black] (2pt,0pt) -- (-2pt,0pt) node[left] {\footnotesize $\y$};
\draw[color=black] (0pt,-10pt) node[right] {\footnotesize $0$};
%Achsenbeschriftung
\draw (xMAX) node[anchor=north east] {$x$};
\draw (yMAX) node[anchor=east] {$(f\circ g)(x),(g\circ f)(x)$};
%Graph
\draw[color=red,samples=50,domain=-1:1] plot(\x,{exp(\x*\x + 1)}); 
\draw[color=blue,samples=50,domain=-1:1] plot(\x,{exp(2*\x) + 1}); 
\draw[color=red] (-0.7,4) node[anchor=east] {$G_{g\circ f}$};
\draw[color=blue] (-1,1) node[anchor=east] {$G_{f\circ g}$};
\end{tikzpicture}
}%

 \item If two functions such as 
 \[
  \function{h}{\R}{\R}{x}{\sin(x)}
 \]
 and 
 \[
  \function{w}{[0\MIntvlSep \infty)}{\R}{x}{\sqrt{x}}
 \]
  are composed, however, the domains of the functions have to be observed. For example, if we 
  want to consider the composed function $w\circ h$, then we have
 \[
  (w\circ h)(x)=w(h(x))=w(\sin(x))=\sqrt{\sin(x)} \MDFPeriod
 \]
  Since the values of the sine function can also be negative but the square root only accepts non-negative 
  values, the domain of the sine function has to be restricted accordingly such that 
  the corresponding function values of the sine function are always non-negative, for example, by 
  the restriction $x\in[0\MIntvlSep \pi]=D_{w\circ h}$. Thus, we have 
 \[
  \function{w\circ h}{[0\MIntvlSep \pi]}{\R}{x}{\sqrt{\sin(x)} \MDFPeriod}
 \]
\end{itemize}
\end{MExample}


\begin{MExercise}
Let the functions
\[
 \function{f}{\R}{\R}{x}{2x-3 \MDFPSpace}
\]
\[
 \function{g}{\R\setminus\{0\}}{\R}{x}{\frac{1}{x}}
\]

and
\[
 \function{h}{\R}{\R}{x}{\sin(x) }
\]
be given. Specify the compositions $f\circ g$, $g\circ f$, $h\circ f$, $h\circ g$, $f\circ f$, and $g\circ g$. 
Restrict the domains, if necessary, such that the composition is allowed. However, for the composed function 
always use the maximum domain.

\begin{MHint}{Solution}
\[
 \function{f\circ g}{\R\setminus\{0\}}{\R}{x}{\frac{2}{x}-3}
\]
\[
 \function{g\circ f}{\R\setminus\{\frac{3}{2}\}}{\R}{x}{\frac{1}{2x-3}}
\]
\[
 \function{h\circ f}{\R}{\R}{x}{\sin(2x-3)}
\]
\[
 \function{h\circ g}{\R\setminus\{0\}}{\R}{x}{\sin(\frac{1}{x})}
\] 
\[
 \function{f\circ f}{\R}{\R}{x}{4x-9}
\]
\[
 \function{g\circ g}{\R}{\R}{x}{x}
\]
\end{MHint}

\end{MExercise}


\end{MXContent}


\MSubsection{Final Test}
\MLabel{M06_Abschlusstest}

\begin{MTest}{Final Test Module \arabic{section}}
\MDeclareSiteUXID{VBKM06_Abschlusstest}

\begin{MExercise}
Specify the maximum domains $D_f$ and $D_g$ of the two functions
\[
 \function{f}{D_f}{\R}{x}{\frac{9x^2-\sin(x)+42}{x^2-2}}
\]
and
\[
 \function{g}{D_g}{\R}{y}{\frac{\ln(y)}{y^2+1}\MDFPeriod}
\]
% Intervallquestion
\end{MExercise}

\begin{MExercise}
Specify the range $W_i$ of the function
\[
 \function{i}{\R}{\R}{x}{x^2-4x+4+\pi\MDFPeriod}
\]
\end{MExercise}

\begin{MExercise}
Find the parameters  $A,\lambda\in\R$ in the exponential function 
\[
 \function{c}{\R}{\R}{x}{A\cdot \MEU^{\lambda x}-1\MDFPaSpace,}
\]
such that $c(0)=1$ and $c(4)=0$.
\ \\ \ \\
Answer: \MEquationItem{$A$}{\MLParsedQuestion{10}{2}{3}{EL1}}, \MEquationItem{$\lambda$}{\MLParsedQuestion{15}{-ln(2)/4}{3}{EL2}}.
\ \\
\MInputHint{Simple logarithms can be left as they are, e.g. $\ln(100)$ can be entered as \texttt{ln(100)} even though the exact value of $ln(100)$ is unknown.}
\end{MExercise}

\begin{MExercise}
Specify the composition $h=f\circ g\colon\R\to\R$ (note: $h(x)=(f\circ g)(x)=f(g(x))$)
of the functions 
\[
 \function{f}{\R}{\R}{x}{C\cdot\sin(x)}
\]
and
\[
 \function{g}{\R}{\R}{x}{B\cdot x+\pi \MDFPeriod}
\]
Answer: \MEquationItem{$h(x)$}{\MLFunctionQuestion{20}{C*sin(B*x+pi)}{10}{x,B,C}{4}{EL3}}.
\ \\ \ \\
Find the parameters such that the sine wave described by the function $h$ has the graph shown below.

 \begin{center}
 \MUGraphics{sinusfrage.png}{width=0.4\linewidth}{A sine wave.}{width:300px}
 %\MCopyrightNotice{\MCCLicense}{NONE}{MINT}{Funktionsplot erstellt und exportiert mit Maple15}{VBKM06_Abbildung_Sinusfrage}
 \end{center}
\ \\ \ \\
Answer: \MEquationItem{$h(x)$}{\MLFunctionQuestion{20}{3*sin(1/2*pi*x+pi)}{10}{x}{4}{EL4}}.
\end{MExercise}


\begin{MExercise}
Specify the inverse function $f=u^{-1}$ of
\[
 \function{u}{(0\MIntvlSep \infty)}{\R}{y}{-\log_2(y) \MDFPeriod}
\]
The function $f=u^{-1}$ has
\begin{MExerciseItems}
\item{the domain \MEquationItem{$D_f$}{\MLIntervalQuestion{20}{(-infty,infty)}{3}{TEID1}}.}
\item{the range \MEquationItem{$W_f$}{\MLIntervalQuestion{20}{(0,infty)}{3}{TEID2}}.}
\item{the mapping rule \MEquationItem{$f(y)=u^{-1}(y)$}{\MLSimplifyQuestion{13}{2^(-y)}{5}{y}{5}{512}{TEID3}}.}
\end{MExerciseItems}
\MInputHint{Enter the ranges as intervals of the form \texttt{(a;b)}, \texttt{infinity} can also be an endpoint.}
\end{MExercise}

\begin{MExercise}

Please indicate whether the following statements are right or wrong:
\ \\ \ \\

The function 
\[
 \function{f}{[0\MIntvlSep 3)}{\R}{x}{2x+1}
\]

% Safari kann Checkboxes in lists nicht anzeigen
\begin{tabular}{lll}
\MLCheckbox{0}{EF1} & \ \ & ... can be also written for short as $f(x)=2x+1$.\\
\MLCheckbox{1}{EF2} & \ \ & ... is a linear affine function.\\
\MLCheckbox{0}{EF3} & \ \ & ... has the range $\R$.\\
\MLCheckbox{1}{EF4} & \ \ & ... has the slope $2$.\\
\MLCheckbox{1}{EF5} & \ \ & ... can only take values greater or equal $1$ and less than $7$.\\
\MLCheckbox{1}{EF6} & \ \ & ... has a graph that is a piece of a line.\\
\MLCheckbox{1}{EF7} & \ \ & ... has at $x=0$ the value $1$.\\
\MLCheckbox{0}{EF8} & \ \ & ... has the domain $\R$.
\end{tabular}
\end{MExercise}

\begin{MExercise}
Calculate the following logarithms:
\begin{MExerciseItems}
\item{\MEquationItem{$\ln(\MEU^5\cdot\frac1{\sqrt{\MEU}})$}{\MLSimplifyQuestion{10}{9/2}{5}{}{5}{1}{TLG0}}.}
\item{\MEquationItem{$\log_{10}(\MZahl{0}{01})$}{\MLSimplifyQuestion{10}{-2}{5}{}{5}{1}{TLG1}}.}
\item{\MEquationItem{$\log_2(\sqrt{2\cdot 4\cdot 16\cdot 256\cdot 1024})$}{\MLSimplifyQuestion{10}{25/2}{5}{}{5}{1}{TLG2}}.}
\end{MExerciseItems}
\end{MExercise}
\end{MTest}


\newpage
\MPrintIndex

\end{document}
