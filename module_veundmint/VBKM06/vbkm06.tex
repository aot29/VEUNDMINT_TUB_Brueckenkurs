% MINTMOD Version P0.1.0, needs to be consistent with preprocesser object in tex2x and MPragma-Version at the end of this file

% Parameter aus Konvertierungsprozess (PDF und HTML-Erzeugung wenn vom Konverter aus gestartet) werden hier eingefuegt, Preambleincludes werden am Schluss angehaengt

\newif\ifttm                % gesetzt falls Uebersetzung in HTML stattfindet, sonst uebersetzung in PDF

% Wahl der Notationsvariante ist im PDF immer std, in der HTML-Uebersetzung wird vom Konverter die Auswahl modifiziert
\newif\ifvariantstd
\newif\ifvariantunotation
\variantstdtrue % Diese Zeile wird vom Konverter erkannt und ggf. modifiziert, daher nicht veraendern!


\def\MOutputDVI{1}
\def\MOutputPDF{2}
\def\MOutputHTML{3}
\newcounter{MOutput}

\ifttm
\usepackage{german}
\usepackage{array}
\usepackage{amsmath}
\usepackage{amssymb}
\usepackage{amsthm}
\else
\documentclass[ngerman,oneside]{scrbook}
\usepackage{etex}
\usepackage[latin1]{inputenc}
\usepackage{textcomp}
\usepackage[ngerman]{babel}
\usepackage[pdftex]{color}
\usepackage{xcolor}
\usepackage{graphicx}
\usepackage[all]{xy}
\usepackage{fancyhdr}
\usepackage{verbatim}
\usepackage{array}
\usepackage{float}
\usepackage{makeidx}
\usepackage{amsmath}
\usepackage{amstext}
\usepackage{amssymb}
\usepackage{amsthm}
\usepackage[ngerman]{varioref}
\usepackage{framed}
\usepackage{supertabular}
\usepackage{longtable}
\usepackage{maxpage}
\usepackage{tikz}
\usepackage{tikzscale}
\usepackage{tikz-3dplot}
\usepackage{bibgerm}
\usepackage{chemarrow}
\usepackage{polynom}
%\usepackage{draftwatermark}
\usepackage{pdflscape}
\usetikzlibrary{calc}
\usetikzlibrary{through}
\usetikzlibrary{shapes.geometric}
\usetikzlibrary{arrows}
\usetikzlibrary{intersections}
\usetikzlibrary{decorations.pathmorphing}
\usetikzlibrary{external}
\usetikzlibrary{patterns}
\usetikzlibrary{fadings}
\usepackage[colorlinks=true,linkcolor=blue]{hyperref} 
\usepackage[all]{hypcap}
%\usepackage[colorlinks=true,linkcolor=blue,bookmarksopen=true]{hyperref} 
\usepackage{ifpdf}

\usepackage{movie15}

\setcounter{tocdepth}{2} % In Inhaltsverzeichnis bis subsection
\setcounter{secnumdepth}{3} % Nummeriert bis subsubsection

\setlength{\LTpost}{0pt} % Fuer longtable
\setlength{\parindent}{0pt}
\setlength{\parskip}{8pt}
%\setlength{\parskip}{9pt plus 2pt minus 1pt}
\setlength{\abovecaptionskip}{-0.25ex}
\setlength{\belowcaptionskip}{-0.25ex}
\fi

\ifttm
\newcommand{\MDebugMessage}[1]{\special{html:<!-- debugprint;;}#1\special{html:; //-->}}
\else
%\newcommand{\MDebugMessage}[1]{\immediate\write\mintlog{#1}}
\newcommand{\MDebugMessage}[1]{}
\fi

\def\MPageHeaderDef{%
\pagestyle{fancy}%
\fancyhead[r]{(C) VE\&MINT-Projekt}
\fancyfoot[c]{\thepage\\--- CCL BY-SA 3.0 ---}
}


\ifttm%
\def\MRelax{}%
\else%
\def\MRelax{\relax}%
\fi%

%--------------------------- Uebernahme von speziellen XML-Versionen einiger LaTeX-Kommandos aus xmlbefehle.tex vom alten Kasseler Konverter ---------------

\newcommand{\MSep}{\left\|{\phantom{\frac1g}}\right.}

\newcommand{\ML}{L}

\newcommand{\MGGT}{\mathrm{ggT}}


\ifttm
% Verhindert dass die subsection-nummer doppelt in der toccaption auftaucht (sollte ggf. in toccaption gefixt werden so dass diese Ueberschreibung nicht notwendig ist)
\renewcommand{\thesubsection}{}
% Kommandos die ttm nicht kennt
\newcommand{\binomial}[2]{{#1 \choose #2}} %  Binomialkoeffizienten
\newcommand{\eur}{\begin{html}&euro;\end{html}}
\newcommand{\square}{\begin{html}&square;\end{html}}
\newcommand{\glqq}{"'}  \newcommand{\grqq}{"'}
\newcommand{\nRightarrow}{\special{html: &nrArr; }}
\newcommand{\nmid}{\special{html: &nmid; }}
\newcommand{\nparallel}{\begin{html}&nparallel;\end{html}}
\newcommand{\mapstoo}{\begin{html}<mo>&map;</mo>\end{html}}

% Schnitt und Vereinigungssymbole von Mengen haben zu kleine Abstaende; korrigiert:
\newcommand{\ccup}{\,\!\cup\,\!}
\newcommand{\ccap}{\,\!\cap\,\!}


% Umsetzung von mathbb im HTML
\renewcommand{\mathbb}[1]{\begin{html}<mo>&#1opf;</mo>\end{html}}
\fi

%---------------------- Strukturierung ----------------------------------------------------------------------------------------------------------------------

%---------------------- Kapselung des sectioning findet auf drei Ebenen statt:
% 1. Die LateX-Befehl
% 2. Die D-Versionen der Befehle, die nur die Grade der Abschnitte umhaengen falls notwendig
% 3. Die M-Versionen der Befehle, die zusaetzliche Formatierungen vornehmen, Skripten starten und das HTML codieren
% Im Modultext duerfen nur die M-Befehle verwendet werden!

\ifttm

  \def\Dsubsubsubsection#1{\subsubsubsection{#1}}
  \def\Dsubsubsection#1{\subsubsection{#1}\addtocounter{subsubsection}{1}} % ttm-Fehler korrigieren
  \def\Dsubsection#1{\subsection{#1}}
  \def\Dsection#1{\section{#1}} % Im HTML wird nur der Sektionstitel gegeben
  \def\Dchapter#1{\chapter{#1}}
  \def\Dsubsubsubsectionx#1{\subsubsubsection*{#1}}
  \def\Dsubsubsectionx#1{\subsubsection*{#1}}
  \def\Dsubsectionx#1{\subsection*{#1}}
  \def\Dsectionx#1{\section*{#1}}
  \def\Dchapterx#1{\chapter*{#1}}

\else

  \def\Dsubsubsubsection#1{\subsubsection{#1}}
  \def\Dsubsubsection#1{\subsection{#1}}
  \def\Dsubsection#1{\section{#1}}
  \def\Dsection#1{\chapter{#1}}
  \def\Dchapter#1{\title{#1}}
  \def\Dsubsubsubsectionx#1{\subsubsection*{#1}}
  \def\Dsubsubsectionx#1{\subsection*{#1}}
  \def\Dsubsectionx#1{\section*{#1}}
  \def\Dsectionx#1{\chapter*{#1}}

\fi

\newcommand{\MStdPoints}{4}
\newcommand{\MSetPoints}[1]{\renewcommand{\MStdPoints}{#1}}

% Befehl zum Abbruch der Erstellung (nur PDF)
\newcommand{\MAbort}[1]{\err{#1}}

% Prefix vor Dateieinbindungen, wird in der Baumdatei mit \renewcommand modifiziert
% und auf das Verzeichnisprefix gesetzt, in dem das gerade bearbeitete tex-Dokument liegt.
% Im HTML wird es auf das Verzeichnis der HTML-Datei gesetzt.
% Das Prefix muss mit / enden !
\newcommand{\MDPrefix}{.}

% MRegisterFile notiert eine Datei zur Einbindung in den HTML-Baum. Grafiken mit MGraphics werden automatisch eingebunden.
% Mit MLastFile erhaelt man eine Markierung fuer die zuletzt registrierte Datei.
% Diese Markierung wird im postprocessing durch den physikalischen Dateinamen ersetzt, aber nur den Namen (d.h. \MMaterial gehoert noch davor, vgl Definition von MGraphics)
% Parameter: Pfad/Name der Datei bzw. des Ordners, relativ zur Position des Modul-Tex-Dokuments.
\ifttm
\newcommand{\MRegisterFile}[1]{\addtocounter{MFileNumber}{1}\special{html:<!-- registerfile;;}#1\special{html:;;}\MDPrefix\special{html:;;}\arabic{MFileNumber}\special{html:; //-->}}
\else
\newcommand{\MRegisterFile}[1]{\addtocounter{MFileNumber}{1}}
\fi

% Testen welcher Uebersetzer hier am Werk ist

\ifttm
\setcounter{MOutput}{3}
\else
\ifx\pdfoutput\undefined
  \pdffalse
  \setcounter{MOutput}{\MOutputDVI}
  \message{Verarbeitung mit latex, Ausgabe in dvi.}
\else
  \setcounter{MOutput}{\MOutputPDF}
  \message{Verarbeitung mit pdflatex, Ausgabe in pdf.}
  \ifnum \pdfoutput=0
    \pdffalse
  \setcounter{MOutput}{\MOutputDVI}
  \message{Verarbeitung mit pdflatex, Ausgabe in dvi.}
  \else
    \ifnum\pdfoutput=1
    \pdftrue
  \setcounter{MOutput}{\MOutputPDF}
  \message{Verarbeitung mit pdflatex, Ausgabe in pdf.}
    \fi
  \fi
\fi
\fi

\ifnum\value{MOutput}=\MOutputPDF
\DeclareGraphicsExtensions{.pdf,.png,.jpg}
\fi

\ifnum\value{MOutput}=\MOutputDVI
\DeclareGraphicsExtensions{.eps,.png,.jpg}
\fi

\ifnum\value{MOutput}=\MOutputHTML
% Wird vom Konverter leider nicht erkannt und daher in split.pm hardcodiert!
\DeclareGraphicsExtensions{.png,.jpg,.gif}
\fi

% Umdefinition der hyperref-Nummerierung im PDF-Modus
\ifttm
\else
\renewcommand{\theHfigure}{\arabic{chapter}.\arabic{section}.\arabic{figure}}
\fi

% Makro, um in der HTML-Ausgabe die zuerst zu oeffnende Datei zu kennzeichnen
\ifttm
\newcommand{\MGlobalStart}{\special{html:<!-- mglobalstarttag -->}}
\else
\newcommand{\MGlobalStart}{}
\fi

% Makro, um bei scormlogin ein pullen des Benutzers bei Aufruf der Seite zu erzwingen (typischerweise auf der Einstiegsseite)
\ifttm
\newcommand{\MPullSite}{\special{html:<!-- pullsite //-->}}
\else
\newcommand{\MPullSite}{}
\fi

% Makro, um in der HTML-Ausgabe die Kapiteluebersicht zu kennzeichnen
\ifttm
\newcommand{\MGlobalChapterTag}{\special{html:<!-- mglobalchaptertag -->}}
\else
\newcommand{\MGlobalChapterTag}{}
\fi

% Makro, um in der HTML-Ausgabe die Konfiguration zu kennzeichnen
\ifttm
\newcommand{\MGlobalConfTag}{\special{html:<!-- mglobalconfigtag -->}}
\else
\newcommand{\MGlobalConfTag}{}
\fi

% Makro, um in der HTML-Ausgabe die Standortbeschreibung zu kennzeichnen
\ifttm
\newcommand{\MGlobalLocationTag}{\special{html:<!-- mgloballocationtag -->}}
\else
\newcommand{\MGlobalLocationTag}{}
\fi

% Makro, um in der HTML-Ausgabe die persoenlichen Daten zu kennzeichnen
\ifttm
\newcommand{\MGlobalDataTag}{\special{html:<!-- mglobaldatatag -->}}
\else
\newcommand{\MGlobalDataTag}{}
\fi

% Makro, um in der HTML-Ausgabe die Suchseite zu kennzeichnen
\ifttm
\newcommand{\MGlobalSearchTag}{\special{html:<!-- mglobalsearchtag -->}}
\else
\newcommand{\MGlobalSearchTag}{}
\fi

% Makro, um in der HTML-Ausgabe die Favoritenseite zu kennzeichnen
\ifttm
\newcommand{\MGlobalFavoTag}{\special{html:<!-- mglobalfavoritestag -->}}
\else
\newcommand{\MGlobalFavoTag}{}
\fi

% Makro, um in der HTML-Ausgabe die Eingangstestseite zu kennzeichnen
\ifttm
\newcommand{\MGlobalSTestTag}{\special{html:<!-- mglobalstesttag -->}}
\else
\newcommand{\MGlobalSTestTag}{}
\fi

% Makro, um in der PDF-Ausgabe ein Wasserzeichen zu definieren
\ifttm
\newcommand{\MWatermarkSettings}{\relax}
\else
\newcommand{\MWatermarkSettings}{%
% \SetWatermarkText{(c) MINT-Kolleg Baden-W�rttemberg 2014}
% \SetWatermarkLightness{0.85}
% \SetWatermarkScale{1.5}
}
\fi

\ifttm
\newcommand{\MBinom}[2]{\left({\begin{array}{c} #1 \\ #2 \end{array}}\right)}
\else
\newcommand{\MBinom}[2]{\binom{#1}{#2}}
\fi

\ifttm
\newcommand{\DeclareMathOperator}[2]{\def#1{\mathrm{#2}}}
\newcommand{\operatorname}[1]{\mathrm{#1}}
\fi

%----------------- Makros fuer die gemischte HTML/PDF-Konvertierung ------------------------------

\newcommand{\MTestName}{\relax} % wird durch Test-Umgebung gesetzt

% Fuer experimentelle Kursinhalte, die im Release-Umsetzungsvorgang eine Fehlermeldung
% produzieren sollen aber sonst normal umgesetzt werden
\newenvironment{MExperimental}{%
}{%
}

% Wird von ttm nicht richtig umgesetzt!!
\newenvironment{MExerciseItems}{%
\renewcommand\theenumi{\alph{enumi}}%
\begin{enumerate}%
}{%
\end{enumerate}%
}


\definecolor{infoshadecolor}{rgb}{0.75,0.75,0.75}
\definecolor{exmpshadecolor}{rgb}{0.875,0.875,0.875}
\definecolor{expeshadecolor}{rgb}{0.95,0.95,0.95}
\definecolor{framecolor}{rgb}{0.2,0.2,0.2}

% Bei PDF-Uebersetzung wird hinter den Start jeder Satz/Info-aehnlichen Umgebung eine leere mbox gesetzt, damit
% fuehrende Listen oder enums nicht den Zeilenumbruch kaputtmachen
%\ifttm
\def\MTB{}
%\else
%\def\MTB{\mbox{}}
%\fi


\ifttm
\newcommand{\MRelates}{\special{html:<mi>&wedgeq;</mi>}}
\else
\def\MRelates{\stackrel{\scriptscriptstyle\wedge}{=}}
\fi

\def\MInch{\text{''}}
\def\Mdd{\textit{''}}

\ifttm
\def\MNL{ \newline }
\newenvironment{MArray}[1]{\begin{array}{#1}}{\end{array}}
\else
\def\MNL{ \\ }
\newenvironment{MArray}[1]{\begin{array}{#1}}{\end{array}}
\fi

\newcommand{\MBox}[1]{$\mathrm{#1}$}
\newcommand{\MMBox}[1]{\mathrm{#1}}


\ifttm%
\newcommand{\Mtfrac}[2]{{\textstyle \frac{#1}{#2}}}
\newcommand{\Mdfrac}[2]{{\displaystyle \frac{#1}{#2}}}
\newcommand{\Mmeasuredangle}{\special{html:<mi>&angmsd;</mi>}}
\else%
\newcommand{\Mtfrac}[2]{\tfrac{#1}{#2}}
\newcommand{\Mdfrac}[2]{\dfrac{#1}{#2}}
\newcommand{\Mmeasuredangle}{\measuredangle}
\relax
\fi

% Matrizen und Vektoren

% Inhalt wird in der Form a & b \\ c & d erwartet
% Vorsicht: MVector = Komponentenspalte, MVec = Variablensymbol
\ifttm%
\newcommand{\MVector}[1]{\left({\begin{array}{c}#1\end{array}}\right)}
\else%
\newcommand{\MVector}[1]{\begin{pmatrix}#1\end{pmatrix}}
\fi



\newcommand{\MVec}[1]{\vec{#1}}
\newcommand{\MDVec}[1]{\overrightarrow{#1}}

%----------------- Umgebungen fuer Definitionen und Saetze ----------------------------------------

% Fuegt einen Tabellen-Zeilenumbruch ein im PDF, aber nicht im HTML
\newcommand{\TSkip}{\ifttm \else&\ \\\fi}

\newenvironment{infoshaded}{%
\def\FrameCommand{\fboxsep=\FrameSep \fcolorbox{framecolor}{infoshadecolor}}%
\MakeFramed {\advance\hsize-\width \FrameRestore}}%
{\endMakeFramed}

\newenvironment{expeshaded}{%
\def\FrameCommand{\fboxsep=\FrameSep \fcolorbox{framecolor}{expeshadecolor}}%
\MakeFramed {\advance\hsize-\width \FrameRestore}}%
{\endMakeFramed}

\newenvironment{exmpshaded}{%
\def\FrameCommand{\fboxsep=\FrameSep \fcolorbox{framecolor}{exmpshadecolor}}%
\MakeFramed {\advance\hsize-\width \FrameRestore}}%
{\endMakeFramed}

\def\STDCOLOR{black}

\ifttm%
\else%
\newtheoremstyle{MSatzStyle}
  {1cm}                   %Space above
  {1cm}                   %Space below
  {\normalfont\itshape}   %Body font
  {}                      %Indent amount (empty = no indent,
                          %\parindent = para indent)
  {\normalfont\bfseries}  %Thm head font
  {}                      %Punctuation after thm head
  {\newline}              %Space after thm head: " " = normal interword
                          %space; \newline = linebreak
  {\thmname{#1}\thmnumber{ #2}\thmnote{ (#3)}}
                          %Thm head spec (can be left empty, meaning
                          %`normal')
                          %
\newtheoremstyle{MDefStyle}
  {1cm}                   %Space above
  {1cm}                   %Space below
  {\normalfont}           %Body font
  {}                      %Indent amount (empty = no indent,
                          %\parindent = para indent)
  {\normalfont\bfseries}  %Thm head font
  {}                      %Punctuation after thm head
  {\newline}              %Space after thm head: " " = normal interword
                          %space; \newline = linebreak
  {\thmname{#1}\thmnumber{ #2}\thmnote{ (#3)}}
                          %Thm head spec (can be left empty, meaning
                          %`normal')
\fi%

\newcommand{\MInfoText}{Info}

\newcounter{MHintCounter}
\newcounter{MCodeEditCounter}

\newcounter{MLastIndex}  % Enthaelt die dritte Stelle (Indexnummer) des letzten angelegten Objekts
\newcounter{MLastType}   % Enthaelt den Typ des letzten angelegten Objekts (mithilfe der unten definierten Konstanten). Die Entscheidung, wie der Typ dargstellt wird, wird in split.pm beim Postprocessing getroffen.
\newcounter{MLastTypeEq} % =1 falls das Label in einer Matheumgebung (equation, eqnarray usw.) steht, =2 falls das Label in einer table-Umgebung steht

% Da ttm keine Zahlmakros verarbeiten kann, werden diese Nummern in den Zuweisungen hardcodiert!
\def\MTypeSection{1}          %# Zaehler ist section
\def\MTypeSubsection{2}       %# Zaehler ist subsection
\def\MTypeSubsubsection{3}    %# Zaehler ist subsubsection
\def\MTypeInfo{4}             %# Eine Infobox, Separatzaehler fuer die Chemie (auch wenn es dort nicht nummeriert wird) ist MInfoCounter
\def\MTypeExercise{5}         %# Eine Aufgabe, Separatzaehler fuer die Chemie ist MExerciseCounter
\def\MTypeExample{6}          %# Eine Beispielbox, Separatzaehler fuer die Chemie ist MExampleCounter
\def\MTypeExperiment{7}       %# Eine Versuchsbox, Separatzaehler fuer die Chemie ist MExperimentCounter
\def\MTypeGraphics{8}         %# Eine Graphik, Separatzaehler fuer alle FB ist MGraphicsCounter
\def\MTypeTable{9}            %# Eine Tabellennummer, hat keinen Zaehler da durch table gezaehlt wird
\def\MTypeEquation{10}        %# Eine Gleichungsnummer, hat keinen Zaehler da durch equation/eqnarray gezaehlt wird
\def\MTypeTheorem{11}         % Ein theorem oder xtheorem, Separatzaehler fuer die Chemie ist MTheoremCounter
\def\MTypeVideo{12}           %# Ein Video,Separatzaehler fuer alle FB ist MVideoCounter
\def\MTypeEntry{13}           %# Ein Eintrag fuer die Stichwortliste, wird nicht gezaehlt sondern erhaelt im preparsing ein unique-label 

% Zaehler fuer das Labelsystem sind prefixcounter, jeder Zaehler wird VOR dem gezaehlten Objekt inkrementiert und zaehlt daher das aktuelle Objekt
\newcounter{MInfoCounter}
\newcounter{MExerciseCounter}
\newcounter{MExampleCounter}
\newcounter{MExperimentCounter}
\newcounter{MGraphicsCounter}
\newcounter{MTableCounter}
\newcounter{MEquationCounter}  % Nur im HTML, sonst durch "equation"-counter von latex realisiert
\newcounter{MTheoremCounter}
\newcounter{MObjectCounter}   % Gemeinsamer Zaehler fuer Objekte (ausser Grafiken/Tabellen) in Mathe/Info/Physik
\newcounter{MVideoCounter}
\newcounter{MEntryCounter}

\newcounter{MTestSite} % 1 = Subsubsection ist eine Pruefungsseite, 0 = ist eine normale Seite (inkl. Hilfeseite)

\def\MCell{$\phantom{a}$}

\newenvironment{MExportExercise}{\begin{MExercise}}{\end{MExercise}} % wird von mconvert abgefangen

\def\MGenerateExNumber{%
\ifnum\value{MSepNumbers}=0%
\arabic{section}.\arabic{subsection}.\arabic{MObjectCounter}\setcounter{MLastIndex}{\value{MObjectCounter}}%
\else%
\arabic{section}.\arabic{subsection}.\arabic{MExerciseCounter}\setcounter{MLastIndex}{\value{MExerciseCounter}}%
\fi%
}%

\def\MGenerateExmpNumber{%
\ifnum\value{MSepNumbers}=0%
\arabic{section}.\arabic{subsection}.\arabic{MObjectCounter}\setcounter{MLastIndex}{\value{MObjectCounter}}%
\else%
\arabic{section}.\arabic{subsection}.\arabic{MExerciseCounter}\setcounter{MLastIndex}{\value{MExampleCounter}}%
\fi%
}%

\def\MGenerateInfoNumber{%
\ifnum\value{MSepNumbers}=0%
\arabic{section}.\arabic{subsection}.\arabic{MObjectCounter}\setcounter{MLastIndex}{\value{MObjectCounter}}%
\else%
\arabic{section}.\arabic{subsection}.\arabic{MExerciseCounter}\setcounter{MLastIndex}{\value{MInfoCounter}}%
\fi%
}%

\def\MGenerateSiteNumber{%
\arabic{section}.\arabic{subsection}.\arabic{subsubsection}%
}%

% Funktionalitaet fuer Auswahlaufgaben

\newcounter{MExerciseCollectionCounter} % = 0 falls nicht in collection-Umgebung, ansonsten Schachtelungstiefe
\newcounter{MExerciseCollectionTextCounter} % wird von MExercise-Umgebung inkrementiert und von MExerciseCollection-Umgebung auf Null gesetzt

\ifttm
% MExerciseCollection gruppiert Aufgaben, die dynamisch aus der Datenbank gezogen werden und nicht direkt in der HTML-Seite stehen
% Parameter: #1 = ID der Collection, muss eindeutig fuer alle IN DER DB VORHANDENEN collections sein unabhaengig vom Kurs
%            #2 = Optionsargument (im Moment: 1 = Iterative Auswahl, 2 = Zufallsbasierte Auswahl)
\newenvironment{MExerciseCollection}[2]{%
\addtocounter{MExerciseCollectionCounter}{1}
\setcounter{MExerciseCollectionTextCounter}{0}
\special{html:<!-- mexercisecollectionstart;;}#1\special{html:;;}#2\special{html:;; //-->}%
}{%
\special{html:<!-- mexercisecollectionstop //-->}%
\addtocounter{MExerciseCollectionCounter}{-1}
}
\else
\newenvironment{MExerciseCollection}[2]{%
\addtocounter{MExerciseCollectionCounter}{1}
\setcounter{MExerciseCollectionTextCounter}{0}
}{%
\addtocounter{MExerciseCollectionCounter}{-1}
}
\fi

% Bei Uebersetzung nach PDF werden die theorem-Umgebungen verwendet, bei Uebersetzung in HTML ein manuelles Makro
\ifttm%

  \newenvironment{MHint}[1]{  \special{html:<button name="Name_MHint}\arabic{MHintCounter}\special{html:" class="hintbutton_closed" id="MHint}\arabic{MHintCounter}\special{html:_button" %
  type="button" onclick="toggle_hint('MHint}\arabic{MHintCounter}\special{html:');">}#1\special{html:</button>}
  \special{html:<div class="hint" style="display:none" id="MHint}\arabic{MHintCounter}\special{html:"> }}{\begin{html}</div>\end{html}\addtocounter{MHintCounter}{1}}

  \newenvironment{MCOSHZusatz}{  \special{html:<button name="Name_MHint}\arabic{MHintCounter}\special{html:" class="chintbutton_closed" id="MHint}\arabic{MHintCounter}\special{html:_button" %
  type="button" onclick="toggle_hint('MHint}\arabic{MHintCounter}\special{html:');">}Weiterf�hrende Inhalte\special{html:</button>}
  \special{html:<div class="hintc" style="display:none" id="MHint}\arabic{MHintCounter}\special{html:">
  <div class="coshwarn">Diese Inhalte gehen �ber das Kursniveau hinaus und werden in den Aufgaben und Tests nicht abgefragt.</div><br />}
  \addtocounter{MHintCounter}{1}}{\begin{html}</div>\end{html}}

  
  \newenvironment{MDefinition}{\begin{definition}\setcounter{MLastIndex}{\value{definition}}\ \\}{\end{definition}}

  
  \newenvironment{MExercise}{
  \renewcommand{\MStdPoints}{4}
  \addtocounter{MExerciseCounter}{1}
  \addtocounter{MObjectCounter}{1}
  \setcounter{MLastType}{5}

  \ifnum\value{MExerciseCollectionCounter}=0\else\addtocounter{MExerciseCollectionTextCounter}{1}\special{html:<!-- mexercisetextstart;;}\arabic{MExerciseCollectionTextCounter}\special{html:;; //-->}\fi
  \special{html:<div class="aufgabe" id="ADIV_}\MGenerateExNumber\special{html:">}%
  \textbf{Aufgabe \MGenerateExNumber
  } \ \\}{
  \special{html:</div><!-- mfeedbackbutton;Aufgabe;}\arabic{MTestSite}\special{html:;}\MGenerateExNumber\special{html:; //-->}
  \ifnum\value{MExerciseCollectionCounter}=0\else\special{html:<!-- mexercisetextstop //-->}\fi
  }

  % Stellt eine Kombination aus Aufgabe, Loesungstext und Eingabefeld bereit,
  % bei der Aufgabentext und Musterloesung sowie die zugehoerigen Feldelemente
  % extern bezogen und div-aktualisiert werden, das Eingabefeld aber immer das gleiche ist.
  \newenvironment{MFetchExercise}{
  \addtocounter{MExerciseCounter}{1}
  \addtocounter{MObjectCounter}{1}
  \setcounter{MLastType}{5}

  \special{html:<div class="aufgabe" id="ADIV_}\MGenerateExNumber\special{html:">}%
  \textbf{Aufgabe \MGenerateExNumber
  } \ \\%
  \special{html:</div><div class="exfetch_text" id="ADIVTEXT_}\MGenerateExNumber\special{html:">}%
  \special{html:</div><div class="exfetch_sol" id="ADIVSOL_}\MGenerateExNumber\special{html:">}%
  \special{html:</div><div class="exfetch_input" id="ADIVINPUT_}\MGenerateExNumber\special{html:">}%
  }{
  \special{html:</div>}
  }

  \newenvironment{MExample}{
  \addtocounter{MExampleCounter}{1}
  \addtocounter{MObjectCounter}{1}
  \setcounter{MLastType}{6}
  \begin{html}
  <div class="exmp">
  <div class="exmprahmen">
  \end{html}\textbf{Beispiel
  \ifnum\value{MSepNumbers}=0
  \arabic{section}.\arabic{subsection}.\arabic{MObjectCounter}\setcounter{MLastIndex}{\value{MObjectCounter}}
  \else
  \arabic{section}.\arabic{subsection}.\arabic{MExampleCounter}\setcounter{MLastIndex}{\value{MExampleCounter}}
  \fi
  } \ \\}{\begin{html}</div>
  </div>
  \end{html}
  \special{html:<!-- mfeedbackbutton;Beispiel;}\arabic{MTestSite}\special{html:;}\MGenerateExmpNumber\special{html:; //-->}
  }

  \newenvironment{MExperiment}{
  \addtocounter{MExperimentCounter}{1}
  \addtocounter{MObjectCounter}{1}
  \setcounter{MLastType}{7}
  \begin{html}
  <div class="expe">
  <div class="experahmen">
  \end{html}\textbf{Versuch
  \ifnum\value{MSepNumbers}=0
  \arabic{section}.\arabic{subsection}.\arabic{MObjectCounter}\setcounter{MLastIndex}{\value{MObjectCounter}}
  \else
%  \arabic{MExperimentCounter}\setcounter{MLastIndex}{\value{MExperimentCounter}}
  \arabic{section}.\arabic{subsection}.\arabic{MExperimentCounter}\setcounter{MLastIndex}{\value{MExperimentCounter}}
  \fi
  } \ \\}{\begin{html}</div>
  </div>
  \end{html}}

  \newenvironment{MChemInfo}{
  \setcounter{MLastType}{4}
  \begin{html}
  <div class="info">
  <div class="inforahmen">
  \end{html}}{\begin{html}</div>
  </div>
  \end{html}}

  \newenvironment{MXInfo}[1]{
  \addtocounter{MInfoCounter}{1}
  \addtocounter{MObjectCounter}{1}
  \setcounter{MLastType}{4}
  \begin{html}
  <div class="info">
  <div class="inforahmen">
  \end{html}\textbf{#1
  \ifnum\value{MInfoNumbers}=0
  \else
    \ifnum\value{MSepNumbers}=0
    \arabic{section}.\arabic{subsection}.\arabic{MObjectCounter}\setcounter{MLastIndex}{\value{MObjectCounter}}
    \else
    \arabic{MInfoCounter}\setcounter{MLastIndex}{\value{MInfoCounter}}
    \fi
  \fi
  } \ \\}{\begin{html}</div>
  </div>
  \end{html}
  \special{html:<!-- mfeedbackbutton;Info;}\arabic{MTestSite}\special{html:;}\MGenerateInfoNumber\special{html:; //-->}
  }

  \newenvironment{MInfo}{\ifnum\value{MInfoNumbers}=0\begin{MChemInfo}\else\begin{MXInfo}{Info}\ \\ \fi}{\ifnum\value{MInfoNumbers}=0\end{MChemInfo}\else\end{MXInfo}\fi}

\else%

  \theoremstyle{MSatzStyle}
  \newtheorem{thm}{Satz}[section]
  \newtheorem{thmc}{Satz}
  \theoremstyle{MDefStyle}
  \newtheorem{defn}[thm]{Definition}
  \newtheorem{exmp}[thm]{Beispiel}
  \newtheorem{info}[thm]{\MInfoText}
  \theoremstyle{MDefStyle}
  \newtheorem{defnc}{Definition}
  \theoremstyle{MDefStyle}
  \newtheorem{exmpc}{Beispiel}[section]
  \theoremstyle{MDefStyle}
  \newtheorem{infoc}{\MInfoText}
  \theoremstyle{MDefStyle}
  \newtheorem{exrc}{Aufgabe}[section]
  \theoremstyle{MDefStyle}
  \newtheorem{verc}{Versuch}[section]
  
  \newenvironment{MFetchExercise}{}{} % kann im PDF nicht dargestellt werden
  
  \newenvironment{MExercise}{\begin{exrc}\renewcommand{\MStdPoints}{1}\MTB}{\end{exrc}}
  \newenvironment{MHint}[1]{\ \\ \underline{#1:}\\}{}
  \newenvironment{MCOSHZusatz}{\ \\ \underline{Weiterf�hrende Inhalte:}\\}{}
  \newenvironment{MDefinition}{\ifnum\value{MInfoNumbers}=0\begin{defnc}\else\begin{defn}\fi\MTB}{\ifnum\value{MInfoNumbers}=0\end{defnc}\else\end{defn}\fi}
%  \newenvironment{MExample}{\begin{exmp}}{\ \linebreak[1] \ \ \ \ $\phantom{a}$ \ \hfill $\blacklozenge$\end{exmp}}
  \newenvironment{MExample}{
    \ifnum\value{MInfoNumbers}=0\begin{exmpc}\else\begin{exmp}\fi
    \MTB
    \begin{exmpshaded}
    \ \newline
}{
    \end{exmpshaded}
    \ifnum\value{MInfoNumbers}=0\end{exmpc}\else\end{exmp}\fi
}
  \newenvironment{MChemInfo}{\begin{infoshaded}}{\end{infoshaded}}

  \newenvironment{MInfo}{\ifnum\value{MInfoNumbers}=0\begin{MChemInfo}\else\renewcommand{\MInfoText}{Info}\begin{info}\begin{infoshaded}
  \MTB
   \ \newline
    \fi
  }{\ifnum\value{MInfoNumbers}=0\end{MChemInfo}\else\end{infoshaded}\end{info}\fi}

  \newenvironment{MXInfo}[1]{
    \renewcommand{\MInfoText}{#1}
    \ifnum\value{MInfoNumbers}=0\begin{infoc}\else\begin{info}\fi%
    \MTB
    \begin{infoshaded}
    \ \newline
  }{\end{infoshaded}\ifnum\value{MInfoNumbers}=0\end{infoc}\else\end{info}\fi}

  \newenvironment{MExperiment}{
    \renewcommand{\MInfoText}{Versuch}
    \ifnum\value{MInfoNumbers}=0\begin{verc}\else\begin{info}\fi
    \MTB
    \begin{expeshaded}
    \ \newline
  }{
    \end{expeshaded}
    \ifnum\value{MInfoNumbers}=0\end{verc}\else\end{info}\fi
  }
\fi%

% MHint sollte nicht direkt fuer Loesungen benutzt werden wegen solutionselect
\newenvironment{MSolution}{\begin{MHint}{L"osung}}{\end{MHint}}

\newcounter{MCodeCounter}

\ifttm
\newenvironment{MCode}{\special{html:<!-- mcodestart -->}\ttfamily\color{blue}}{\special{html:<!-- mcodestop -->}}
\else
\newenvironment{MCode}{\begin{flushleft}\ttfamily\addtocounter{MCodeCounter}{1}}{\addtocounter{MCodeCounter}{-1}\end{flushleft}}
% Ohne color-Statement da inkompatible mit framed/shaded-Boxen aus dem framed-package
\fi

%----------------- Sonderdefinitionen fuer Symbole, die der Konverter nicht kann ----------------------------------------------

\ifttm%
\newcommand{\MUnderset}[2]{\underbrace{#2}_{#1}}%
\else%
\newcommand{\MUnderset}[2]{\underset{#1}{#2}}%
\fi%

\ifttm
\newcommand{\MThinspace}{\special{html:<mi>&#x2009;</mi>}}
\else
\newcommand{\MThinspace}{\,}
\fi

\ifttm
\newcommand{\glq}{\begin{html}&sbquo;\end{html}}
\newcommand{\grq}{\begin{html}&lsquo;\end{html}}
\newcommand{\glqq}{\begin{html}&bdquo;\end{html}}
\newcommand{\grqq}{\begin{html}&ldquo;\end{html}}
\fi

\ifttm
\newcommand{\MNdash}{\begin{html}&ndash;\end{html}}
\else
\newcommand{\MNdash}{--}
\fi

%\ifttm\def\MIU{\special{html:<mi>&#8520;</mi>}}\else\def\MIU{\mathrm{i}}\fi
\def\MIU{\mathrm{i}}
\def\MEU{e} % TU9-Onlinekurs: italic-e
%\def\MEU{\mathrm{e}} % Alte Onlinemodule: roman-e
\def\MD{d} % Kursives d in Integralen im TU9-Onlinekurs
%\def\MD{\mathrm{d}} % roman-d in den alten Onlinemodulen
\def\MDB{\|}

%zusaetzlicher Leerraum vor "\MD"
\ifttm%
\def\MDSpace{\special{html:<mi>&#x2009;</mi>}}
\else%
\def\MDSpace{\,}
\fi%
\newcommand{\MDwSp}{\MDSpace\MD}%

\ifttm
\def\Mdq{\dq}
\else
\def\Mdq{\dq}
\fi

\def\MSpan#1{\left<{#1}\right>}
\def\MSetminus{\setminus}
\def\MIM{I}

\ifttm
\newcommand{\ld}{\text{ld}}
\newcommand{\lg}{\text{lg}}
\else
\DeclareMathOperator{\ld}{ld}
%\newcommand{\lg}{\text{lg}} % in latex schon definiert
\fi


\def\Mmapsto{\ifttm\special{html:<mi>&mapsto;</mi>}\else\mapsto\fi} 
\def\Mvarphi{\ifttm\phi\else\varphi\fi}
\def\Mphi{\ifttm\varphi\else\phi\fi}
\ifttm%
\newcommand{\MEumu}{\special{html:<mi>&#x3BC;</mi>}}%
\else%
\newcommand{\MEumu}{\textrm{\textmu}}%
\fi
\def\Mvarepsilon{\ifttm\epsilon\else\varepsilon\fi}
\def\Mepsilon{\ifttm\varepsilon\else\epsilon\fi}
\def\Mvarkappa{\ifttm\kappa\else\varkappa\fi}
\def\Mkappa{\ifttm\varkappa\else\kappa\fi}
\def\Mcomplement{\ifttm\special{html:<mi>&comp;</mi>}\else\complement\fi} 
\def\MWW{\mathrm{WW}}
\def\Mmod{\ifttm\special{html:<mi>&nbsp;mod&nbsp;</mi>}\else\mod\fi} 

\ifttm%
\def\mod{\text{\;mod\;}}%
\def\MNEquiv{\special{html:<mi>&NotCongruent;</mi>}}% 
\def\MNSubseteq{\special{html:<mi>&NotSubsetEqual;</mi>}}%
\def\MEmptyset{\special{html:<mi>&empty;</mi>}}%
\def\MVDots{\special{html:<mi>&#x22EE;</mi>}}%
\def\MHDots{\special{html:<mi>&#x2026;</mi>}}%
\def\Mddag{\special{html:<mi>&#x1202;</mi>}}%
\def\sphericalangle{\special{html:<mi>&measuredangle;</mi>}}%
\def\nparallel{\special{html:<mi>&nparallel;</mi>}}%
\def\MProofEnd{\special{html:<mi>&#x25FB;</mi>}}%
\newenvironment{MProof}[1]{\underline{#1}:\MCR\MCR}{\hfill $\MProofEnd$}%
\else%
\def\MNEquiv{\not\equiv}%
\def\MNSubseteq{\not\subseteq}%
\def\MEmptyset{\emptyset}%
\def\MVDots{\vdots}%
\def\MHDots{\hdots}%
\def\Mddag{\ddag}%
\newenvironment{MProof}[1]{\begin{proof}[#1]}{\end{proof}}%
\fi%



% Spaces zum Auffuellen von Tabellenbreiten, die nur im HTML wirken
\ifttm%
\def\MTSP{\:}%
\else%
\def\MTSP{}%
\fi%

\DeclareMathOperator{\arsinh}{arsinh}
\DeclareMathOperator{\arcosh}{arcosh}
\DeclareMathOperator{\artanh}{artanh}
\DeclareMathOperator{\arcoth}{arcoth}


\newcommand{\MMathSet}[1]{\mathbb{#1}}
\def\N{\MMathSet{N}}
\def\Z{\MMathSet{Z}}
\def\Q{\MMathSet{Q}}
\def\R{\MMathSet{R}}
\def\C{\MMathSet{C}}

\newcounter{MForLoopCounter}
\newcommand{\MForLoop}[2]{\setcounter{MForLoopCounter}{#1}\ifnum\value{MForLoopCounter}=0{}\else{{#2}\addtocounter{MForLoopCounter}{-1}\MForLoop{\value{MForLoopCounter}}{#2}}\fi}

\newcounter{MSiteCounter}
\newcounter{MFieldCounter} % Kombination section.subsection.site.field ist eindeutig in allen Modulen, field alleine nicht

\newcounter{MiniMarkerCounter}

\ifttm
\newenvironment{MMiniPageP}[1]{\begin{minipage}{#1\linewidth}\special{html:<!-- minimarker;;}\arabic{MiniMarkerCounter}\special{html:;;#1; //-->}}{\end{minipage}\addtocounter{MiniMarkerCounter}{1}}
\else
\newenvironment{MMiniPageP}[1]{\begin{minipage}{#1\linewidth}}{\end{minipage}\addtocounter{MiniMarkerCounter}{1}}
\fi

\newcounter{AlignCounter}

\newcommand{\MStartJustify}{\ifttm\special{html:<!-- startalign;;}\arabic{AlignCounter}\special{html:;;justify; //-->}\fi}
\newcommand{\MStopJustify}{\ifttm\special{html:<!-- stopalign;;}\arabic{AlignCounter}\special{html:; //-->}\fi\addtocounter{AlignCounter}{1}}

\newenvironment{MJTabular}[1]{
\MStartJustify
\begin{tabular}{#1}
}{
\end{tabular}
\MStopJustify
}

\newcommand{\MImageLeft}[2]{
\begin{center}
\begin{tabular}{lc}
\MStartJustify
\begin{MMiniPageP}{0.65}
#1
\end{MMiniPageP}
\MStopJustify
&
\begin{MMiniPageP}{0.3}
#2  
\end{MMiniPageP}
\end{tabular}
\end{center}
}

\newcommand{\MImageHalf}[2]{
\begin{center}
\begin{tabular}{lc}
\MStartJustify
\begin{MMiniPageP}{0.45}
#1
\end{MMiniPageP}
\MStopJustify
&
\begin{MMiniPageP}{0.45}
#2  
\end{MMiniPageP}
\end{tabular}
\end{center}
}

\newcommand{\MBigImageLeft}[2]{
\begin{center}
\begin{tabular}{lc}
\MStartJustify
\begin{MMiniPageP}{0.25}
#1
\end{MMiniPageP}
\MStopJustify
&
\begin{MMiniPageP}{0.7}
#2  
\end{MMiniPageP}
\end{tabular}
\end{center}
}

\ifttm
\def\No{\mathbb{N}_0}
\else
\def\No{\ensuremath{\N_0}}
\fi
\def\MT{\textrm{\tiny T}}
\newcommand{\MTranspose}[1]{{#1}^{\MT}}
\ifttm
\newcommand{\MRe}{\mathsf{Re}}
\newcommand{\MIm}{\mathsf{Im}}
\else
\DeclareMathOperator{\MRe}{Re}
\DeclareMathOperator{\MIm}{Im}
\fi

\newcommand{\Mid}{\mathrm{id}}
\newcommand{\MFeinheit}{\mathrm{feinh}}

\ifttm
\newcommand{\Msubstack}[1]{\begin{array}{c}{#1}\end{array}}
\else
\newcommand{\Msubstack}[1]{\substack{#1}}
\fi

% Typen von Fragefeldern:
% 1 = Alphanumerisch, case-sensitive-Vergleich
% 2 = Ja/Nein-Checkbox, Loesung ist 0 oder 1   (OPTION = Image-id fuer Rueckmeldung)
% 3 = Reelle Zahlen Geparset
% 4 = Funktionen Geparset (mit Stuetzstellen zur ueberpruefung)

% Dieser Befehl erstellt ein interaktives Aufgabenfeld. Parameter:
% - #1 Laenge in Zeichen
% - #2 Loesungstext (alphanumerisch, case sensitive)
% - #3 AufgabenID (alphanumerisch, case sensitive)
% - #4 Typ (Kennnummer)
% - #5 String fuer Optionen (ggf. mit Semikolon getrennte Einzelstrings)
% - #6 Anzahl Punkte
% - #7 uxid (kann z.B. Loesungsstring sein)
% ACHTUNG: Die langen Zeilen bitte so lassen, Zeilenumbrueche im tex werden in div's umgesetzt
\newcommand{\MQuestionID}[7]{
\ifttm
\special{html:<!-- mdeclareuxid;;}UX#7\special{html:;;}\arabic{section}\special{html:;;}#3\special{html:;; //-->}%
\special{html:<!-- mdeclarepoints;;}\arabic{section}\special{html:;;}#3\special{html:;;}#6\special{html:;;}\arabic{MTestSite}\special{html:;;}\arabic{chapter}%
\special{html:;; //--><!-- onloadstart //-->CreateQuestionObj("}#7\special{html:",}\arabic{MFieldCounter}\special{html:,"}#2%
\special{html:","}#3\special{html:",}#4\special{html:,"}#5\special{html:",}#6\special{html:,}\arabic{MTestSite}\special{html:,}\arabic{section}%
\special{html:);<!-- onloadstop //-->}%
\special{html:<input mfieldtype="}#4\special{html:" name="Name_}#3\special{html:" id="}#3\special{html:" type="text" size="}#1\special{html:" maxlength="}#1%
\special{html:" }\ifnum\value{MGroupActive}=0\special{html:onfocus="handlerFocus(}\arabic{MFieldCounter}%
\special{html:);" onblur="handlerBlur(}\arabic{MFieldCounter}\special{html:);" onkeyup="handlerChange(}\arabic{MFieldCounter}\special{html:,0);" onpaste="handlerChange(}\arabic{MFieldCounter}\special{html:,0);" oninput="handlerChange(}\arabic{MFieldCounter}\special{html:,0);" onpropertychange="handlerChange(}\arabic{MFieldCounter}\special{html:,0);"/>}%
\special{html:<img src="images/questionmark.gif" width="20" height="20" border="0" align="absmiddle" id="}QM#3\special{html:"/>}
\else%
\special{html:onblur="handlerBlur(}\arabic{MFieldCounter}%
\special{html:);" onfocus="handlerFocus(}\arabic{MFieldCounter}\special{html:);" onkeyup="handlerChange(}\arabic{MFieldCounter}\special{html:,1);" onpaste="handlerChange(}\arabic{MFieldCounter}\special{html:,1);" oninput="handlerChange(}\arabic{MFieldCounter}\special{html:,1);" onpropertychange="handlerChange(}\arabic{MFieldCounter}\special{html:,1);"/>}%
\special{html:<img src="images/questionmark.gif" width="20" height="20" border="0" align="absmiddle" id="}QM#3\special{html:"/>}\fi%
\else%
\ifnum\value{QBoxFlag}=1\fbox{$\phantom{\MForLoop{#1}{b}}$}\else$\phantom{\MForLoop{#1}{b}}$\fi%
\fi%
}

% ACHTUNG: Die langen Zeilen bitte so lassen, Zeilenumbrueche im tex werden in div's umgesetzt
% QuestionCheckbox macht ausserhalb einer QuestionGroup keinen Sinn!
% #1 = solution (1 oder 0), ggf. mit ::smc abgetrennt auszuschliessende single-choice-boxen (UXIDs durch , getrennt), #2 = id, #3 = points, #4 = uxid
\newcommand{\MQuestionCheckbox}[4]{
\ifttm
\special{html:<!-- mdeclareuxid;;}UX#4\special{html:;;}\arabic{section}\special{html:;;}#2\special{html:;; //-->}%
\ifnum\value{MGroupActive}=0\MDebugMessage{ERROR: Checkbox Nr. \arabic{MFieldCounter}\ ist nicht in einer Kontrollgruppe, es wird niemals eine Loesung angezeigt!}\fi
\special{html: %
<!-- mdeclarepoints;;}\arabic{section}\special{html:;;}#2\special{html:;;}#3\special{html:;;}\arabic{MTestSite}\special{html:;;}\arabic{chapter}%
\special{html:;; //--><!-- onloadstart //-->CreateQuestionObj("}#4\special{html:",}\arabic{MFieldCounter}\special{html:,"}#1\special{html:","}#2\special{html:",2,"IMG}#2%
\special{html:",}#3\special{html:,}\arabic{MTestSite}\special{html:,}\arabic{section}\special{html:);<!-- onloadstop //-->}%
\special{html:<input mfieldtype="2" type="checkbox" name="Name_}#2\special{html:" id="}#2\special{html:" onchange="handlerChange(}\arabic{MFieldCounter}\special{html:,1);"/><img src="images/questionmark.gif" name="}Name_IMG#2%
\special{html:" width="20" height="20" border="0" align="absmiddle" id="}IMG#2\special{html:"/> }%
\else%
\ifnum\value{QBoxFlag}=1\fbox{$\phantom{X}$}\else$\phantom{X}$\fi%
\fi%
}

\def\MGenerateID{QFELD_\arabic{section}.\arabic{subsection}.\arabic{MSiteCounter}.QF\arabic{MFieldCounter}}

% #1 = 0/1 ggf. mit ::smc abgetrennt auszuschliessende single-choice-boxen (UXIDs durch , getrennt ohne UX), #2 = uxid ohne UX
\newcommand{\MCheckbox}[2]{
\MQuestionCheckbox{#1}{\MGenerateID}{\MStdPoints}{#2}
\addtocounter{MFieldCounter}{1}
}

% Erster Parameter: Zeichenlaenge der Eingabebox, zweiter Parameter: Loesungstext
\newcommand{\MQuestion}[2]{
\MQuestionID{#1}{#2}{\MGenerateID}{1}{0}{\MStdPoints}{#2}
\addtocounter{MFieldCounter}{1}
}

% Erster Parameter: Zeichenlaenge der Eingabebox, zweiter Parameter: Loesungstext
\newcommand{\MLQuestion}[3]{
\MQuestionID{#1}{#2}{\MGenerateID}{1}{0}{\MStdPoints}{#3}
\addtocounter{MFieldCounter}{1}
}

% Parameter: Laenge des Feldes, Loesung (wird auch geparsed), Stellen Genauigkeit hinter dem Komma, weitere Stellen werden mathematisch gerundet vor Vergleich
\newcommand{\MParsedQuestion}[3]{
\MQuestionID{#1}{#2}{\MGenerateID}{3}{#3}{\MStdPoints}{#2}
\addtocounter{MFieldCounter}{1}
}

% Parameter: Laenge des Feldes, Loesung (wird auch geparsed), Stellen Genauigkeit hinter dem Komma, weitere Stellen werden mathematisch gerundet vor Vergleich
\newcommand{\MLParsedQuestion}[4]{
\MQuestionID{#1}{#2}{\MGenerateID}{3}{#3}{\MStdPoints}{#4}
\addtocounter{MFieldCounter}{1}
}

% Parameter: Laenge des Feldes, Loesungsfunktion, Anzahl Stuetzstellen, Funktionsvariablen durch Kommata getrennt (nicht case-sensitive), Anzahl Nachkommastellen im Vergleich
\newcommand{\MFunctionQuestion}[5]{
\MQuestionID{#1}{#2}{\MGenerateID}{4}{#3;#4;#5;0}{\MStdPoints}{#2}
\addtocounter{MFieldCounter}{1}
}

% Parameter: Laenge des Feldes, Loesungsfunktion, Anzahl Stuetzstellen, Funktionsvariablen durch Kommata getrennt (nicht case-sensitive), Anzahl Nachkommastellen im Vergleich, UXID
\newcommand{\MLFunctionQuestion}[6]{
\MQuestionID{#1}{#2}{\MGenerateID}{4}{#3;#4;#5;0}{\MStdPoints}{#6}
\addtocounter{MFieldCounter}{1}
}

% Parameter: Laenge des Feldes, Loesungsintervall, Genauigkeit der Zahlenwertpruefung
\newcommand{\MIntervalQuestion}[3]{
\MQuestionID{#1}{#2}{\MGenerateID}{6}{#3}{\MStdPoints}{#2}
\addtocounter{MFieldCounter}{1}
}

% Parameter: Laenge des Feldes, Loesungsintervall, Genauigkeit der Zahlenwertpruefung, UXID
\newcommand{\MLIntervalQuestion}[4]{
\MQuestionID{#1}{#2}{\MGenerateID}{6}{#3}{\MStdPoints}{#4}
\addtocounter{MFieldCounter}{1}
}

% Parameter: Laenge des Feldes, Loesungsfunktion, Anzahl Stuetzstellen, Funktionsvariable (nicht case-sensitive), Anzahl Nachkommastellen im Vergleich, Vereinfachungsbedingung
% Vereinfachungsbedingung ist eine der Folgenden:
% 0 = Keine Vereinfachungsbedingung
% 1 = Keine Klammern (runde oder eckige) mehr im vereinfachten Ausdruck
% 2 = Faktordarstellung (Term hat Produkte als letzte Operation, Summen als vorgeschaltete Operation)
% 3 = Summendarstellung (Term hat Summen als letzte Operation, Produkte als vorgeschaltete Operation)
% Flag 512: Besondere Stuetzstellen (nur >1 und nur schwach rational), sonst symmetrisch um Nullpunkt und ganze Zahlen inkl. Null werden getroffen
\newcommand{\MSimplifyQuestion}[6]{
\MQuestionID{#1}{#2}{\MGenerateID}{4}{#3;#4;#5;#6}{\MStdPoints}{#2}
\addtocounter{MFieldCounter}{1}
}

\newcommand{\MLSimplifyQuestion}[7]{
\MQuestionID{#1}{#2}{\MGenerateID}{4}{#3;#4;#5;#6}{\MStdPoints}{#7}
\addtocounter{MFieldCounter}{1}
}

% Parameter: Laenge des Feldes, Loesung (optionaler Ausdruck), Anzahl Stuetzstellen, Funktionsvariable (nicht case-sensitive), Anzahl Nachkommastellen im Vergleich, Spezialtyp (string-id)
\newcommand{\MLSpecialQuestion}[7]{
\MQuestionID{#1}{#2}{\MGenerateID}{7}{#3;#4;#5;#6}{\MStdPoints}{#7}
\addtocounter{MFieldCounter}{1}
}

\newcounter{MGroupStart}
\newcounter{MGroupEnd}
\newcounter{MGroupActive}

\newenvironment{MQuestionGroup}{
\setcounter{MGroupStart}{\value{MFieldCounter}}
\setcounter{MGroupActive}{1}
}{
\setcounter{MGroupActive}{0}
\setcounter{MGroupEnd}{\value{MFieldCounter}}
\addtocounter{MGroupEnd}{-1}
}

\newcommand{\MGroupButton}[1]{
\ifttm
\special{html:<button name="Name_Group}\arabic{MGroupStart}\special{html:to}\arabic{MGroupEnd}\special{html:" id="Group}\arabic{MGroupStart}\special{html:to}\arabic{MGroupEnd}\special{html:" %
type="button" onclick="group_button(}\arabic{MGroupStart}\special{html:,}\arabic{MGroupEnd}\special{html:);">}#1\special{html:</button>}
\else
\phantom{#1}
\fi
}

%----------------- Makros fuer die modularisierte Darstellung ------------------------------------

\def\MyText#1{#1}

% is used internally by the conversion package, should not be used by original tex documents
\def\MOrgLabel#1{\relax}

\ifttm

% Ein MLabel wird im html codiert durch das tag <!-- mmlabel;;Labelbezeichner;;SubjectArea;;chapter;;section;;subsection;;Index;;Objekttyp; //-->
\def\MLabel#1{%
\ifnum\value{MLastType}=8%
\ifnum\value{MCaptionOn}=0%
\MDebugMessage{ERROR: Grafik \arabic{MGraphicsCounter} hat separates label: #1 (Grafiklabels sollten nur in der Caption stehen)}%
\fi
\fi
\ifnum\value{MLastType}=12%
\ifnum\value{MCaptionOn}=0%
\MDebugMessage{ERROR: Video \arabic{MVideoCounter} hat separates label: #1 (Videolabels sollten nur in der Caption stehen}%
\fi
\fi
\ifnum\value{MLastType}=10\setcounter{MLastIndex}{\value{equation}}\fi
\label{#1}\begin{html}<!-- mmlabel;;#1;;\end{html}\arabic{MSubjectArea}\special{html:;;}\arabic{chapter}\special{html:;;}\arabic{section}\special{html:;;}\arabic{subsection}\special{html:;;}\arabic{MLastIndex}\special{html:;;}\arabic{MLastType}\special{html:; //-->}}%

\else

% Sonderbehandlung im PDF fuer Abbildungen in separater aux-Datei, da MGraphics die figure-Umgebung nicht verwendet
\def\MLabel#1{%
\ifnum\value{MLastType}=8%
\ifnum\value{MCaptionOn}=0%
\MDebugMessage{ERROR: Grafik \arabic{MGraphicsCounter} hat separates label: #1 (Grafiklabels sollten nur in der Caption stehen}%
\fi
\fi
\ifnum\value{MLastType}=12%
\ifnum\value{MCaptionOn}=0%
\MDebugMessage{ERROR: Video \arabic{MVideoCounter} hat separates label: #1 (Videolabels sollten nur in der Caption stehen}%
\fi
\fi
\label{#1}%
}%

\fi

% Gibt Begriff des referenzierten Objekts mit aus, aber nur im HTML, daher nur in Ausnahmefaellen (z.B. Copyrightliste) sinnvoll
\def\MCRef#1{\ifttm\special{html:<!-- mmref;;}#1\special{html:;;1; //-->}\else\vref{#1}\fi}


\def\MRef#1{\ifttm\special{html:<!-- mmref;;}#1\special{html:;;0; //-->}\else\vref{#1}\fi}
\def\MERef#1{\ifttm\special{html:<!-- mmref;;}#1\special{html:;;0; //-->}\else\eqref{#1}\fi}
\def\MNRef#1{\ifttm\special{html:<!-- mmref;;}#1\special{html:;;0; //-->}\else\ref{#1}\fi}
\def\MSRef#1#2{\ifttm\special{html:<!-- msref;;}#1\special{html:;;}#2\special{html:; //-->}\else \if#2\empty \ref{#1} \else \hyperref[#1]{#2}\fi\fi} 

\def\MRefRange#1#2{\ifttm\MRef{#1} bis 
\MRef{#2}\else\vrefrange[\unskip]{#1}{#2}\fi}

\def\MRefTwo#1#2{\ifttm\MRef{#1} und \MRef{#2}\else%
\let\vRefTLRsav=\reftextlabelrange\let\vRefTPRsav=\reftextpagerange%
\def\reftextlabelrange##1##2{\ref{##1} und~\ref{##2}}%
\def\reftextpagerange##1##2{auf den Seiten~\pageref{#1} und~\pageref{#2}}%
\vrefrange[\unskip]{#1}{#2}%
\let\reftextlabelrange=\vRefTLRsav\let\reftextpagerange=\vRefTPRsav\fi}

% MSectionChapter definiert falls notwendig das Kapitel vor der section. Das ist notwendig, wenn nur ein Einzelmodul uebersetzt wird.
% MChaptersGiven ist ein Counter, der von mconvert.pl vordefiniert wird.
\ifttm
\newcommand{\MSectionChapter}{\ifnum\value{MChaptersGiven}=0{\Dchapter{Modul}}\else{}\fi}
\else
\newcommand{\MSectionChapter}{\ifnum\value{chapter}=0{\Dchapter{Modul}}\else{}\fi}
\fi


\def\MChapter#1{\ifnum\value{MSSEnd}>0{\MSubsectionEndMacros}\addtocounter{MSSEnd}{-1}\fi\Dchapter{#1}}
\def\MSubject#1{\MChapter{#1}} % Schluesselwort HELPSECTION ist reserviert fuer Hilfesektion

\newcommand{\MSectionID}{UNKNOWNID}

\ifttm
\newcommand{\MSetSectionID}[1]{\renewcommand{\MSectionID}{#1}}
\else
\newcommand{\MSetSectionID}[1]{\renewcommand{\MSectionID}{#1}\tikzsetexternalprefix{#1}}
\fi


\newcommand{\MSection}[1]{\MSetSectionID{MODULID}\ifnum\value{MSSEnd}>0{\MSubsectionEndMacros}\addtocounter{MSSEnd}{-1}\fi\MSectionChapter\Dsection{#1}\MSectionStartMacros{#1}\setcounter{MLastIndex}{-1}\setcounter{MLastType}{1}} % Sections werden ueber das section-Feld im mmlabel-Tag identifiziert, nicht ueber das Indexfeld

\def\MSubsection#1{\ifnum\value{MSSEnd}>0{\MSubsectionEndMacros}\addtocounter{MSSEnd}{-1}\fi\ifttm\else\clearpage\fi\Dsubsection{#1}\MSubsectionStartMacros\setcounter{MLastIndex}{-1}\setcounter{MLastType}{2}\addtocounter{MSSEnd}{1}}% Subsections werden ueber das subsection-Feld im mmlabel-Tag identifiziert, nicht ueber das Indexfeld
\def\MSubsectionx#1{\Dsubsectionx{#1}} % Nur zur Verwendung in MSectionStart gedacht
\def\MSubsubsection#1{\Dsubsubsection{#1}\setcounter{MLastIndex}{\value{subsubsection}}\setcounter{MLastType}{3}\ifttm\special{html:<!-- sectioninfo;;}\arabic{section}\special{html:;;}\arabic{subsection}\special{html:;;}\arabic{subsubsection}\special{html:;;1;;}\arabic{MTestSite}\special{html:; //-->}\fi}
\def\MSubsubsectionx#1{\Dsubsubsectionx{#1}\ifttm\special{html:<!-- sectioninfo;;}\arabic{section}\special{html:;;}\arabic{subsection}\special{html:;;}\arabic{subsubsection}\special{html:;;0;;}\arabic{MTestSite}\special{html:; //-->}\else\addcontentsline{toc}{subsection}{#1}\fi}

\ifttm
\def\MSubsubsubsectionx#1{\ \newline\textbf{#1}\special{html:<br />}}
\else
\def\MSubsubsubsectionx#1{\ \newline
\textbf{#1}\ \\
}
\fi


% Dieses Skript wird zu Beginn jedes Modulabschnitts (=Webseite) ausgefuehrt und initialisiert den Aufgabenfeldzaehler
\newcommand{\MPageScripts}{
\setcounter{MFieldCounter}{1}
\addtocounter{MSiteCounter}{1}
\setcounter{MHintCounter}{1}
\setcounter{MCodeEditCounter}{1}
\setcounter{MGroupActive}{0}
\DoQBoxes
% Feldvariablen werden im HTML-Header in conv.pl eingestellt
}

% Dieses Skript wird zum Ende jedes Modulabschnitts (=Webseite) ausgefuehrt
\ifttm
\newcommand{\MEndScripts}{\special{html:<br /><!-- mfeedbackbutton;Seite;}\arabic{MTestSite}\special{html:;}\MGenerateSiteNumber\special{html:; //-->}
}
\else
\newcommand{\MEndScripts}{\relax}
\fi


\newcounter{QBoxFlag}
\newcommand{\DoQBoxes}{\setcounter{QBoxFlag}{1}}
\newcommand{\NoQBoxes}{\setcounter{QBoxFlag}{0}}

\newcounter{MXCTest}
\newcounter{MXCounter}
\newcounter{MSCounter}



\ifttm

% Struktur des sectioninfo-Tags: <!-- sectioninfo;;section;;subsection;;subsubsection;;nr_ausgeben;;testpage; //-->

%Fuegt eine zusaetzliche html-Seite an hinter ALLEN bisherigen und zukuenftigen content-Seiten ausserhalb der vor-zurueck-Schleife (d.h. nur durch Button oder MIntLink erreichbar!)
% #1 = Titel des Modulabschnitts, #2 = Kurztitel fuer die Buttons, #3 = Buttonkennung (STD = default nehmen, NONE = Ohne Button in der Navigation)
\newenvironment{MSContent}[3]{\special{html:<div class="xcontent}\arabic{MSCounter}\special{html:"><!-- scontent;-;}\arabic{MSCounter};-;#1;-;#2;-;#3\special{html: //-->}\MPageScripts\MSubsubsectionx{#1}}{\MEndScripts\special{html:<!-- endscontent;;}\arabic{MSCounter}\special{html: //--></div>}\addtocounter{MSCounter}{1}}

% Fuegt eine zusaetzliche html-Seite ein hinter den bereits vorhandenen content-Seiten (oder als erste Seite) innerhalb der vor-zurueck-Schleife der Navigation
% #1 = Titel des Modulabschnitts, #2 = Kurztitel fuer die Buttons, #3 = Buttonkennung (STD = Defaultbutton, NONE = Ohne Button in der Navigation)
\newenvironment{MXContent}[3]{\special{html:<div class="xcontent}\arabic{MXCounter}\special{html:"><!-- xcontent;-;}\arabic{MXCounter};-;#1;-;#2;-;#3\special{html: //-->}\MPageScripts\MSubsubsection{#1}}{\MEndScripts\special{html:<!-- endxcontent;;}\arabic{MXCounter}\special{html: //--></div>}\addtocounter{MXCounter}{1}}

% Fuegt eine zusaetzliche html-Seite ein die keine subsubsection-Nummer bekommt, nur zur internen Verwendung in mintmod.tex gedacht!
% #1 = Titel des Modulabschnitts, #2 = Kurztitel fuer die Buttons, #3 = Buttonkennung (STD = Defaultbutton, NONE = Ohne Button in der Navigation)
% \newenvironment{MUContent}[3]{\special{html:<div class="xcontent}\arabic{MXCounter}\special{html:"><!-- xcontent;-;}\arabic{MXCounter};-;#1;-;#2;-;#3\special{html: //-->}\MPageScripts\MSubsubsectionx{#1}}{\MEndScripts\special{html:<!-- endxcontent;;}\arabic{MXCounter}\special{html: //--></div>}\addtocounter{MXCounter}{1}}

\newcommand{\MDeclareSiteUXID}[1]{\special{html:<!-- mdeclaresiteuxid;;}#1\special{html:;;}\arabic{chapter}\special{html:;;}\arabic{section}\special{html:;; //-->}}

\else

%\newcommand{\MSubsubsection}[1]{\refstepcounter{subsubsection} \addcontentsline{toc}{subsubsection}{\thesubsubsection. #1}}


% Fuegt eine zusaetzliche html-Seite an hinter den bereits vorhandenen content-Seiten
% #1 = Titel des Modulabschnitts, #2 = Kurztitel fuer die Buttons, #3 = Iconkennung (im PDF wirkungslos)
%\newenvironment{MUContent}[3]{\ifnum\value{MXCTest}>0{\MDebugMessage{ERROR: Geschachtelter SContent}}\fi\MPageScripts\MSubsubsectionx{#1}\addtocounter{MXCTest}{1}}{\addtocounter{MXCounter}{1}\addtocounter{MXCTest}{-1}}
\newenvironment{MXContent}[3]{\ifnum\value{MXCTest}>0{\MDebugMessage{ERROR: Geschachtelter SContent}}\fi\MPageScripts\MSubsubsection{#1}\addtocounter{MXCTest}{1}}{\addtocounter{MXCounter}{1}\addtocounter{MXCTest}{-1}}
\newenvironment{MSContent}[3]{\ifnum\value{MXCTest}>0{\MDebugMessage{ERROR: Geschachtelter XContent}}\fi\MPageScripts\MSubsubsectionx{#1}\addtocounter{MXCTest}{1}}{\addtocounter{MSCounter}{1}\addtocounter{MXCTest}{-1}}

\newcommand{\MDeclareSiteUXID}[1]{\relax}

\fi 

% GHEADER und GFOOTER werden von split.pm gefunden, aber nur, wenn nicht HELPSITE oder TESTSITE
\ifttm
\newenvironment{MSectionStart}{\special{html:<div class="xcontent0">}\MSubsubsectionx{Modul\"ubersicht}}{\setcounter{MSSEnd}{0}\special{html:</div>}}
% Darf nicht als XContent nummeriert werden, darf nicht als XContent gelabelt werden, wird aber in eine xcontent-div gesetzt fuer Python-parsing
\else
\newenvironment{MSectionStart}{\MSubsectionx{Modul\"ubersicht}}{\setcounter{MSSEnd}{0}}
\fi

\newenvironment{MIntro}{\begin{MXContent}{Einf\"uhrung}{Einf\"uhrung}{genetisch}}{\end{MXContent}}
\newenvironment{MContent}{\begin{MXContent}{Inhalt}{Inhalt}{beweis}}{\end{MXContent}}
\newenvironment{MExercises}{\ifttm\else\clearpage\fi\begin{MXContent}{Aufgaben}{Aufgaben}{aufgb}\special{html:<!-- declareexcsymb //-->}}{\end{MXContent}}

% #1 = Lesbare Testbezeichnung
\newenvironment{MTest}[1]{%
\renewcommand{\MTestName}{#1}
\ifttm\else\clearpage\fi%
\addtocounter{MTestSite}{1}%
\begin{MXContent}{#1}{#1}{STD} % {aufgb}%
\special{html:<!-- declaretestsymb //-->}
\begin{MQuestionGroup}%
\MInTestHeader
}%
{%
\end{MQuestionGroup}%
\ \\ \ \\%
\MInTestFooter
\end{MXContent}\addtocounter{MTestSite}{-1}%
}

\newenvironment{MExtra}{\ifttm\else\clearpage\fi\begin{MXContent}{Zus\"atzliche Inhalte}{Zusatz}{weiterfhrg}}{\end{MXContent}}

\makeindex

\ifttm
\def\MPrintIndex{
\ifnum\value{MSSEnd}>0{\MSubsectionEndMacros}\addtocounter{MSSEnd}{-1}\fi
\renewcommand{\indexname}{Stichwortverzeichnis}
\special{html:<p><!-- printindex //--></p>}
}
\else
\def\MPrintIndex{
\ifnum\value{MSSEnd}>0{\MSubsectionEndMacros}\addtocounter{MSSEnd}{-1}\fi
\renewcommand{\indexname}{Stichwortverzeichnis}
\addcontentsline{toc}{section}{Stichwortverzeichnis}
\printindex
}
\fi


% Konstanten fuer die Modulfaecher

\def\MINTMathematics{1}
\def\MINTInformatics{2}
\def\MINTChemistry{3}
\def\MINTPhysics{4}
\def\MINTEngineering{5}

\newcounter{MSubjectArea}
\newcounter{MInfoNumbers} % Gibt an, ob die Infoboxen nummeriert werden sollen
\newcounter{MSepNumbers} % Gibt an, ob Beispiele und Experimente separat nummeriert werden sollen
\newcommand{\MSetSubject}[1]{
 % ttm kapiert setcounter mit Parametern nicht, also per if abragen und einsetzen
\ifnum#1=1\setcounter{MSubjectArea}{1}\setcounter{MInfoNumbers}{1}\setcounter{MSepNumbers}{0}\fi
\ifnum#1=2\setcounter{MSubjectArea}{2}\setcounter{MInfoNumbers}{1}\setcounter{MSepNumbers}{0}\fi
\ifnum#1=3\setcounter{MSubjectArea}{3}\setcounter{MInfoNumbers}{0}\setcounter{MSepNumbers}{1}\fi
\ifnum#1=4\setcounter{MSubjectArea}{4}\setcounter{MInfoNumbers}{0}\setcounter{MSepNumbers}{0}\fi
\ifnum#1=5\setcounter{MSubjectArea}{5}\setcounter{MInfoNumbers}{1}\setcounter{MSepNumbers}{0}\fi
% Separate Nummerntechnik fuer unsere Chemiker: alles dreistellig
\ifnum#1=3
  \ifttm
  \renewcommand{\theequation}{\arabic{section}.\arabic{subsection}.\arabic{equation}}
  \renewcommand{\thetable}{\arabic{section}.\arabic{subsection}.\arabic{table}} 
  \renewcommand{\thefigure}{\arabic{section}.\arabic{subsection}.\arabic{figure}} 
  \else
  \renewcommand{\theequation}{\arabic{chapter}.\arabic{section}.\arabic{equation}}
  \renewcommand{\thetable}{\arabic{chapter}.\arabic{section}.\arabic{table}}
  \renewcommand{\thefigure}{\arabic{chapter}.\arabic{section}.\arabic{figure}}
  \fi
\else
  \ifttm
  \renewcommand{\theequation}{\arabic{section}.\arabic{subsection}.\arabic{equation}}
  \renewcommand{\thetable}{\arabic{table}}
  \renewcommand{\thefigure}{\arabic{figure}}
  \else
  \renewcommand{\theequation}{\arabic{chapter}.\arabic{section}.\arabic{equation}}
  \renewcommand{\thetable}{\arabic{table}}
  \renewcommand{\thefigure}{\arabic{figure}}
  \fi
\fi
}

% Fuer tikz Autogenerierung
\newcounter{MTIKZAutofilenumber}

% Spezielle Counter fuer die Bentz-Module
\newcounter{mycounter}
\newcounter{chemapplet}
\newcounter{physapplet}

\newcounter{MSSEnd} % Ist 1 falls ein MSubsection aktiv ist, der einen MSubsectionEndMacro-Aufruf verursacht
\newcounter{MFileNumber}
\def\MLastFile{\special{html:[[!-- mfileref;;}\arabic{MFileNumber}\special{html:; //--]]}}

% Vollstaendiger Pfad ist \MMaterial / \MLastFilePath / \MLastFileName    ==   \MMaterial / \MLastFile

% Wird nur bei kompletter Baum-Erstellung ausgefuehrt!
% #1 = Lesbare Modulbezeichnung
\newcommand{\MSectionStartMacros}[1]{
\setcounter{MTestSite}{0}
\setcounter{MCaptionOn}{0}
\setcounter{MLastTypeEq}{0}
\setcounter{MSSEnd}{0}
\setcounter{MFileNumber}{0} % Preinkrekement-Counter
\setcounter{MTIKZAutofilenumber}{0}
\setcounter{mycounter}{1}
\setcounter{physapplet}{1}
\setcounter{chemapplet}{0}
\ifttm
\special{html:<!-- mdeclaresection;;}\arabic{chapter}\special{html:;;}\arabic{section}\special{html:;;}#1\special{html:;; //-->}%
\else
\setcounter{thmc}{0}
\setcounter{exmpc}{0}
\setcounter{verc}{0}
\setcounter{infoc}{0}
\fi
\setcounter{MiniMarkerCounter}{1}
\setcounter{AlignCounter}{1}
\setcounter{MXCTest}{0}
\setcounter{MCodeCounter}{0}
\setcounter{MEntryCounter}{0}
}

% Wird immer ausgefuehrt
\newcommand{\MSubsectionStartMacros}{
\ifttm\else\MPageHeaderDef\fi
\MWatermarkSettings
\setcounter{MXCounter}{0}
\setcounter{MSCounter}{0}
\setcounter{MSiteCounter}{1}
\setcounter{MExerciseCollectionCounter}{0}
% Zaehler fuer das Labelsystem zuruecksetzen (prefix-Zaehler)
\setcounter{MInfoCounter}{0}
\setcounter{MExerciseCounter}{0}
\setcounter{MExampleCounter}{0}
\setcounter{MExperimentCounter}{0}
\setcounter{MGraphicsCounter}{0}
\setcounter{MTableCounter}{0}
\setcounter{MTheoremCounter}{0}
\setcounter{MObjectCounter}{0}
\setcounter{MEquationCounter}{0}
\setcounter{MVideoCounter}{0}
\setcounter{equation}{0}
\setcounter{figure}{0}
}

\newcommand{\MSubsectionEndMacros}{
% Bei Chemiemodulen das PSE einhaengen, es soll als SContent am Ende erscheinen
\special{html:<!-- subsectionend //-->}
\ifnum\value{MSubjectArea}=3{\MIncludePSE}\fi
}


\ifttm
%\newcommand{\MEmbed}[1]{\MRegisterFile{#1}\begin{html}<embed src="\end{html}\MMaterial/\MLastFile\begin{html}" width="192" height="189"></embed>\end{html}}
\newcommand{\MEmbed}[1]{\MRegisterFile{#1}\begin{html}<embed src="\end{html}\MMaterial/\MLastFile\begin{html}"></embed>\end{html}}
\fi

%----------------- Makros fuer die Textdarstellung -----------------------------------------------

\ifttm
% MUGraphics bindet eine Grafik ein:
% Parameter 1: Dateiname der Grafik, relativ zur Position des Modul-Tex-Dokuments
% Parameter 2: Skalierungsoptionen fuer PDF (fuer includegraphics)
% Parameter 3: Titel fuer die Grafik, wird unter die Grafik mit der Grafiknummer gesetzt und kann MLabel bzw. MCopyrightLabel enthalten
% Parameter 4: Skalierungsoptionen fuer HTML (css-styles)

% ERSATZ: <img alt="My Image" src="data:image/png;base64,iVBORwA<MoreBase64SringHere>" />


\newcommand{\MUGraphics}[4]{\MRegisterFile{#1}\begin{html}
<div class="imagecenter">
<center>
<div>
<img src="\end{html}\MMaterial/\MLastFile\begin{html}" style="#4" alt="\end{html}\MMaterial/\MLastFile\begin{html}"/>
</div>
<div class="bildtext">
\end{html}
\addtocounter{MGraphicsCounter}{1}
\setcounter{MLastIndex}{\value{MGraphicsCounter}}
\setcounter{MLastType}{8}
\addtocounter{MCaptionOn}{1}
\ifnum\value{MSepNumbers}=0
\textbf{Abbildung \arabic{MGraphicsCounter}:} #3
\else
\textbf{Abbildung \arabic{section}.\arabic{subsection}.\arabic{MGraphicsCounter}:} #3
\fi
\addtocounter{MCaptionOn}{-1}
\begin{html}
</div>
</center>
</div>
<br />
\end{html}%
\special{html:<!-- mfeedbackbutton;Abbildung;}\arabic{MGraphicsCounter}\special{html:;}\arabic{section}.\arabic{subsection}.\arabic{MGraphicsCounter}\special{html:; //-->}%
}

% MVideo bindet ein Video als Einzeldatei ein:
% Parameter 1: Dateiname des Videos, relativ zur Position des Modul-Tex-Dokuments, ohne die Endung ".mp4"
% Parameter 2: Titel fuer das Video (kann MLabel oder MCopyrightLabel enthalten), wird unter das Video mit der Videonummer gesetzt
\newcommand{\MVideo}[2]{\MRegisterFile{#1.mp4}\begin{html}
<div class="imagecenter">
<center>
<div>
<video width="95\%" controls="controls"><source src="\end{html}\MMaterial/#1.mp4\begin{html}" type="video/mp4">Ihr Browser kann keine MP4-Videos abspielen!</video>
</div>
<div class="bildtext">
\end{html}
\addtocounter{MVideoCounter}{1}
\setcounter{MLastIndex}{\value{MVideoCounter}}
\setcounter{MLastType}{12}
\addtocounter{MCaptionOn}{1}
\ifnum\value{MSepNumbers}=0
\textbf{Video \arabic{MVideoCounter}:} #2
\else
\textbf{Video \arabic{section}.\arabic{subsection}.\arabic{MVideoCounter}:} #2
\fi
\addtocounter{MCaptionOn}{-1}
\begin{html}
</div>
</center>
</div>
<br />
\end{html}}

\newcommand{\MDVideo}[2]{\MRegisterFile{#1.mp4}\MRegisterFile{#1.ogv}\begin{html}
<div class="imagecenter">
<center>
<div>
<video width="70\%" controls><source src="\end{html}\MMaterial/#1.mp4\begin{html}" type="video/mp4"><source src="\end{html}\MMaterial/#1.ogv\begin{html}" type="video/ogg">Ihr Browser kann keine MP4-Videos abspielen!</video>
</div>
<br />
#2
</center>
</div>
<br />
\end{html}
}

\newcommand{\MGraphics}[3]{\MUGraphics{#1}{#2}{#3}{}}

\else

\newcommand{\MVideo}[2]{%
% Kein Video im PDF darstellbar, trotzdem so tun als ob da eines waere
\begin{center}
(Video nicht darstellbar)
\end{center}
\addtocounter{MVideoCounter}{1}
\setcounter{MLastIndex}{\value{MVideoCounter}}
\setcounter{MLastType}{12}
\addtocounter{MCaptionOn}{1}
\ifnum\value{MSepNumbers}=0
\textbf{Video \arabic{MVideoCounter}:} #2
\else
\textbf{Video \arabic{section}.\arabic{subsection}.\arabic{MVideoCounter}:} #2
\fi
\addtocounter{MCaptionOn}{-1}
}


% MGraphics bindet eine Grafik ein:
% Parameter 1: Dateiname der Grafik, relativ zur Position des Modul-Tex-Dokuments
% Parameter 2: Skalierungsoptionen fuer PDF (fuer includegraphics)
% Parameter 3: Titel fuer die Grafik, wird unter die Grafik mit der Grafiknummer gesetzt
\newcommand{\MGraphics}[3]{%
\MRegisterFile{#1}%
\ %
\begin{figure}[H]%
\centering{%
\includegraphics[#2]{\MDPrefix/#1}%
\addtocounter{MCaptionOn}{1}%
\caption{#3}%
\addtocounter{MCaptionOn}{-1}%
}%
\end{figure}%
\addtocounter{MGraphicsCounter}{1}\setcounter{MLastIndex}{\value{MGraphicsCounter}}\setcounter{MLastType}{8}\ %
%\ \\Abbildung \ifnum\value{MSepNumbers}=0\else\arabic{chapter}.\arabic{section}.\fi\arabic{MGraphicsCounter}: #3%
}

\newcommand{\MUGraphics}[4]{\MGraphics{#1}{#2}{#3}}


\fi

\newcounter{MCaptionOn} % = 1 falls eine Grafikcaption aktiv ist, = 0 sonst


% MGraphicsSolo bindet eine Grafik pur ein ohne Titel
% Parameter 1: Dateiname der Grafik, relativ zur Position des Modul-Tex-Dokuments
% Parameter 2: Skalierungsoptionen (wirken nur im PDF)
\newcommand{\MGraphicsSolo}[2]{\MUGraphicsSolo{#1}{#2}{}}

% MUGraphicsSolo bindet eine Grafik pur ein ohne Titel, aber mit HTML-Skalierung
% Parameter 1: Dateiname der Grafik, relativ zur Position des Modul-Tex-Dokuments
% Parameter 2: Skalierungsoptionen (wirken nur im PDF)
% Parameter 3: Skalierungsoptionen (wirken nur im HTML), als style-format: "width=???, height=???"
\ifttm
\newcommand{\MUGraphicsSolo}[3]{\MRegisterFile{#1}\begin{html}
<img src="\end{html}\MMaterial/\MLastFile\begin{html}" style="\end{html}#3\begin{html}" alt="\end{html}\MMaterial/\MLastFile\begin{html}"/>
\end{html}%
\special{html:<!-- mfeedbackbutton;Abbildung;}#1\special{html:;}\MMaterial/\MLastFile\special{html:; //-->}%
}
\else
\newcommand{\MUGraphicsSolo}[3]{\MRegisterFile{#1}\includegraphics[#2]{\MDPrefix/#1}}
\fi

% Externer Link mit URL
% Erster Parameter: Vollstaendige(!) URL des Links
% Zweiter Parameter: Text fuer den Link
\newcommand{\MExtLink}[2]{\ifttm\special{html:<a target="_new" href="}#1\special{html:">}#2\special{html:</a>}\else\href{#1}{#2}\fi} % ohne MINTERLINK!


% Interner Link, die verlinkte Datei muss im gleichen Verzeichnis liegen wie die Modul-Texdatei
% Erster Parameter: Dateiname
% Zweiter Parameter: Text fuer den Link
\newcommand{\MIntLink}[2]{\ifttm\MRegisterFile{#1}\special{html:<a class="MINTERLINK" target="_new" href="}\MMaterial/\MLastFile\special{html:">}#2\special{html:</a>}\else{\href{#1}{#2}}\fi}


\ifttm
\def\MMaterial{:localmaterial:}
\else
\def\MMaterial{\MDPrefix}
\fi

\ifttm
\def\MNoFile#1{:directmaterial:#1}
\else
\def\MNoFile#1{#1}
\fi

\newcommand{\MChem}[1]{$\mathrm{#1}$}

\newcommand{\MApplet}[3]{
% Bindet ein Java-Applet ein, die Parameter sind:
% (wird nur im HTML, aber nicht im PDF erstellt)
% #1 Dateiname des Applets (muss mit ".class" enden)
% #2 = Breite in Pixeln
% #3 = Hoehe in Pixeln
\ifttm
\MRegisterFile{#1}
\begin{html}
<applet code="\end{html}\MMaterial/\MLastFile\begin{html}" width="#2" height="#3" alt="[Java-Applet kann nicht gestartet werden]"></applet>
\end{html}
\fi
}

\newcommand{\MScriptPage}[2]{
% Bindet eine JavaScript-Datei ein, die eine eigene Seite bekommt
% (wird nur im HTML, aber nicht im PDF erstellt)
% #1 Dateiname des Programms (sollte mit ".js" enden)
% #2 = Kurztitel der Seite
\ifttm
\begin{MSContent}{#2}{#2}{puzzle}
\MRegisterFile{#1}
\begin{html}
<script src="\MMaterial/\MLastFile" type="text/javascript"></script>
\end{html}
\end{MSContent}
\fi
}

\newcommand{\MIncludePSE}{
% Bindet bei Chemie-Modulen das PSE ein
% (wird nur im HTML, aber nicht im PDF erstellt)
\ifttm
\special{html:<!-- includepse //-->}
\begin{MSContent}{Periodensystem der Elemente}{PSE}{table}
\MRegisterFile{../files/pse.js}
\MRegisterFile{../files/radio.png}
\begin{html}
<script src="\MMaterial/../files/pse.js" type="text/javascript"></script>
<p id="divid"><br /><br />
<script language="javascript" type="text/javascript">
    startpse("divid","\MMaterial/../files"); 
</script>
</p>
<br />
<br />
<br />
<p>Die Farben der Elementsymbole geben an: <font style="color:Red">gasf&ouml;rmig </font> <font style="color:Blue">fl&uuml;ssig </font> fest</p>
<p>Die Elemente der Gruppe 1 A, 2 A, 3 A usw. geh&ouml;ren zu den Hauptgruppenelementen.</p>
<p>Die Elemente der Gruppe 1 B, 2 B, 3 B usw. geh&ouml;ren zu den Nebengruppenelementen.</p>
<p>() kennzeichnet die Masse des stabilsten Isotops</p>
\end{html}
\end{MSContent}
\fi
}

\newcommand{\MAppletArchive}[4]{
% Bindet ein Java-Applet ein, die Parameter sind:
% (wird nur im HTML, aber nicht im PDF erstellt)
% #1 Dateiname der Klasse mit Appletaufruf (muss mit ".class" enden)
% #2 Dateiname des Archivs (muss mit ".jar" enden)
% #3 = Breite in Pixeln
% #4 = Hoehe in Pixeln
\ifttm
\MRegisterFile{#2}
\begin{html}
<applet code="#1" archive="\end{html}\MMaterial/\MLastFile\begin{html}" codebase="." width="#3" height="#4" alt="[Java-Archiv kann nicht gestartet werden]"></applet>
\end{html}
\fi
}

% Bindet in der Haupttexdatei ein MINT-Modul ein. Parameter 1 ist das Verzeichnis (relativ zur Haupttexdatei), Parameter 2 ist der Dateinahme ohne Pfad.
\newcommand{\IncludeModule}[2]{
\renewcommand{\MDPrefix}{#1}
\input{#1/#2}
\ifnum\value{MSSEnd}>0{\MSubsectionEndMacros}\addtocounter{MSSEnd}{-1}\fi
}

% Der ttm-Konverter setzt keine Makros im \input um, also muss hier getrickst werden:
% Das MDPrefix muss in den einzelnen Modulen manuell eingesetzt werden
\newcommand{\MInputFile}[1]{
\ifttm
\input{#1}
\else
\input{#1}
\fi
}


\newcommand{\MCases}[1]{\left\lbrace{\begin{array}{rl} #1 \end{array}}\right.}

\ifttm
\newenvironment{MCaseEnv}{\left\lbrace\begin{array}{rl}}{\end{array}\right.}
\else
\newenvironment{MCaseEnv}{\left\lbrace\begin{array}{rl}}{\end{array}\right.}
\fi

\def\MSkip{\ifttm\MCR\fi}

\ifttm
\def\MCR{\special{html:<br />}}
\else
\def\MCR{\ \\}
\fi


% Pragmas - Sind Schluesselwoerter, die dem Preprocessing sowie dem Konverter uebergeben werden und bestimmte
%           Aktionen ausloesen. Im Output (PDF und HTML) tauchen sie nicht auf.
\newcommand{\MPragma}[1]{%
\ifttm%
\special{html:<!-- mpragma;-;}#1\special{html:;; -->}%
\else%
% MPragmas werden vom Preprozessor direkt im LaTeX gefunden
\fi%
}

% Ersatz der Befehle textsubscript und textsuperscript, die ttm nicht kennt
\ifttm%
\newcommand{\MTextsubscript}[1]{\special{html:<sub>}#1\special{html:</sub>}}%
\newcommand{\MTextsuperscript}[1]{\special{html:<sup>}#1\special{html:</sup>}}%
\else%
\newcommand{\MTextsubscript}[1]{\textsubscript{#1}}%
\newcommand{\MTextsuperscript}[1]{\textsuperscript{#1}}%
\fi

%------------------ Einbindung von dia-Diagrammen ----------------------------------------------
% Beim preprocessing wird aus jeder dia-Datei eine tex-Datei und eine pdf-Datei erzeugt,
% diese werden hier jeweils im PDF und HTML eingebunden
% Parameter: Dateiname der mit dia erstellten Datei (OHNE die Endung .dia)
\ifttm%
\newcommand{\MDia}[1]{%
\MGraphicsSolo{#1minthtml.png}{}%
}
\else%
\newcommand{\MDia}[1]{%
\MGraphicsSolo{#1mintpdf.png}{scale=0.1667}%
}
\fi%

% subsup funktioniert im Ausdruck $D={\R}^+_0$, also \R geklammert und sup zuerst
% \ifttm
% \def\MSubsup#1#2#3{\special{html:<msubsup>} #1 #2 #3\special{html:</msubsup>}}
% \else
% \def\MSubsup#1#2#3{{#1}^{#3}_{#2}}
% \fi

%\input{local.tex}

% \ifttm
% \else
% \newwrite\mintlog
% \immediate\openout\mintlog=mintlog.txt
% \fi

% ----------------------- tikz autogenerator -------------------------------------------------------------------

\newcommand{\Mtikzexternalize}{\tikzexternalize}% wird bei Konvertierung ueber mconvert ggf. ausgehebelt!

\ifttm
\else
\tikzset%
{
  % Defines a custom style which generates pdf and converts to (low and hi-res quality) png and svg, then deletes the pdf
  % Important: DO NOT directly convert from pdf to hires-png or from svg to png with GraphViz convert as it has some problems and memory leaks
  png export/.style=%
  {
    external/system call/.add={}{; 
      pdf2svg "\image.pdf" "\image.svg" ; 
      convert -density 112.5 -transparent white "\image.pdf" "\image.png"; 
      inkscape --export-png="\image.4x.png" --export-dpi=450 --export-background-opacity=0 --without-gui "\image.svg"; 
      rm "\image.pdf"; rm "\image.log"; rm "\image.dpth"; rm "\image.idx"
    },
    external/force remake,
  }
}
\tikzset{png export}
\tikzsetexternalprefix{}
% PNGs bei externer Erzeugung in "richtiger" Groesse einbinden
\pgfkeys{/pgf/images/include external/.code={\includegraphics[scale=0.64]{#1}}}
\fi

% Spezielle Umgebung fuer Autogenerierung, Bildernamen sind nur innerhalb eines Moduls (einer MSection) eindeutig)

\newcommand{\MTIKZautofilename}{tikzautofile}

\ifttm
% HTML-Version: Vom Autogenerator erzeugte png-Datei einbinden, tikz selbst nicht ausfuehren (sprich: #1 schlucken)
\newcommand{\MTikzAuto}[1]{%
\addtocounter{MTIKZAutofilenumber}{1}
\renewcommand{\MTIKZautofilename}{mtikzauto_\arabic{MTIKZAutofilenumber}}
\MUGraphicsSolo{\MSectionID\MTIKZautofilename.4x.png}{scale=1}{\special{html:[[!-- svgstyle;}\MSectionID\MTIKZautofilename\special{html: //--]]}} % Styleinfos werden aus original-png, nicht 4x-png geholt!
%\MRegisterFile{\MSectionID\MTIKZautofilename.png} % not used right now
%\MRegisterFile{\MSectionID\MTIKZautofilename.svg}
}
\else%
% PDF-Version: Falls Autogenerator aktiv wird Datei automatisch benannt und exportiert
\newcommand{\MTikzAuto}[1]{%
\addtocounter{MTIKZAutofilenumber}{1}%
\renewcommand{\MTIKZautofilename}{mtikzauto_\arabic{MTIKZAutofilenumber}}
\tikzsetnextfilename{\MTIKZautofilename}%
#1%
}
\fi

% In einer reinen LaTeX-Uebersetzung kapselt der Preambelinclude-Befehl nur input,
% in einer konvertergesteuerten PDF/HTML-Uebersetzung wird er dagegen entfernt und
% die Preambeln an mintmod angehaengt, die Ersetzung wird von mconvert.pl vorgenommen.

\newcommand{\MPreambleInclude}[1]{\input{#1}}

% Globale Watermarksettings (werden auch nochmal zu Beginn jedes subsection gesetzt,
% muessen hier aber auch global ausgefuehrt wegen Einfuehrungsseiten und Inhaltsverzeichnis

\MWatermarkSettings
% ---------------------------------- Parametrisierte Aufgaben ----------------------------------------

\ifttm
\newenvironment{MPExercise}{%
\begin{MExercise}%
}{%
\special{html:<button name="Name_MPEX}\arabic{MExerciseCounter}\special{html:" id="MPEX}\arabic{MExerciseCounter}%
\special{html:" type="button" onclick="reroll('}\arabic{MExerciseCounter}\special{html:');">Neue Aufgabe erzeugen</button>}%
\end{MExercise}%
}
\else
\newenvironment{MPExercise}{%
\begin{MExercise}%
}{%
\end{MExercise}%
}
\fi

% Parameter: Name, Min, Max, PDF-Standard. Name in Deklaration OHNE backslash, im Code MIT Backslash
\ifttm
\newcommand{\MGlobalInteger}[4]{\special{html:%
<!-- onloadstart //-->%
MVAR.push(createGlobalInteger("}#1\special{html:",}#2\special{html:,}#3\special{html:,}#4\special{html:)); %
<!-- onloadstop //-->%
<!-- viewmodelstart //-->%
ob}#1\special{html:: ko.observable(rerollMVar("}#1\special{html:")),%
<!-- viewmodelstop //-->%
}%
}%
\else%
\newcommand{\MGlobalInteger}[4]{\newcounter{mvc_#1}\setcounter{mvc_#1}{#4}}
\fi

% Parameter: Name, Min, Max, PDF-Standard. Name in Deklaration OHNE backslash, im Code MIT Backslash, Wert ist Wurzel von value
\ifttm
\newcommand{\MGlobalSqrt}[4]{\special{html:%
<!-- onloadstart //-->%
MVAR.push(createGlobalSqrt("}#1\special{html:",}#2\special{html:,}#3\special{html:,}#4\special{html:)); %
<!-- onloadstop //-->%
<!-- viewmodelstart //-->%
ob}#1\special{html:: ko.observable(rerollMVar("}#1\special{html:")),%
<!-- viewmodelstop //-->%
}%
}%
\else%
\newcommand{\MGlobalSqrt}[4]{\newcounter{mvc_#1}\setcounter{mvc_#1}{#4}}% Funktioniert nicht als Wurzel !!!
\fi

% Parameter: Name, Min, Max, PDF-Standard zaehler, PDF-Standard nenner. Name in Deklaration OHNE backslash, im Code MIT Backslash
\ifttm
\newcommand{\MGlobalFraction}[5]{\special{html:%
<!-- onloadstart //-->%
MVAR.push(createGlobalFraction("}#1\special{html:",}#2\special{html:,}#3\special{html:,}#4\special{html:,}#5\special{html:)); %
<!-- onloadstop //-->%
<!-- viewmodelstart //-->%
ob}#1\special{html:: ko.observable(rerollMVar("}#1\special{html:")),%
<!-- viewmodelstop //-->%
}%
}%
\else%
\newcommand{\MGlobalFraction}[5]{\newcounter{mvc_#1}\setcounter{mvc_#1}{#4}} % Funktioniert nicht als Bruch !!!
\fi

% MVar darf im HTML nur in MEvalMathDisplay-Umgebungen genutzt werden oder in Strings die an den Parser uebergeben werden
\ifttm%
\newcommand{\MVar}[1]{\special{html:[var_}#1\special{html:]}}%
\else%
\newcommand{\MVar}[1]{\arabic{mvc_#1}}%
\fi

\ifttm%
\newcommand{\MRerollButton}[2]{\special{html:<button type="button" onclick="rerollMVar('}#1\special{html:');">}#2\special{html:</button>}}%
\else%
\newcommand{\MRerollButton}[2]{\relax}% Keine sinnvolle Entsprechung im PDF
\fi

% MEvalMathDisplay fuer HTML wird in mconvert.pl im preprocessing realisiert
% PDF: eine equation*-Umgebung (ueber amsmath)
% HTML: Eine Mathjax-Tex-Umgebung, deren Auswertung mit knockout-obervablen gekoppelt ist
% PDF-Version hier nur fuer pdflatex-only-Uebersetzung gegeben

\ifttm\else\newenvironment{MEvalMathDisplay}{\begin{equation*}}{\end{equation*}}\fi

% ---------------------------------- Spezialbefehle fuer AD ------------------------------------------

%Abk�rzung f�r \longrightarrow:
\newcommand{\lto}{\ensuremath{\longrightarrow}}

%Makro f�r Funktionen:
\newcommand{\exfunction}[5]
{\begin{array}{rrcl}
 #1 \colon  & #2 &\lto & #3 \\[.05cm]  
  & #4 &\longmapsto  & #5 
\end{array}}

\newcommand{\function}[5]{%
#1:\;\left\lbrace{\begin{array}{rcl}
 #2 &\lto & #3 \\
 #4 &\longmapsto  & #5 \end{array}}\right.}


%Die Identit�t:
\DeclareMathOperator{\Id}{Id}

%Die Signumfunktion:
\DeclareMathOperator{\sgn}{sgn}

%Zwei Betonungskommandos (k�nnen angepasst werden):
\newcommand{\highlight}[1]{#1}
\newcommand{\modstextbf}[1]{#1}
\newcommand{\modsemph}[1]{#1}


% ---------------------------------- Spezialbefehle fuer JL ------------------------------------------


\def\jccolorfkt{green!50!black} %Farbe des Funktionsgraphen
\def\jccolorfktarea{green!25!white} %Farbe der Fl"ache unter dem Graphen
\def\jccolorfktareahell{green!12!white} %helle Einf"arbung der Fl"ache unter dem Graphen
\def\jccolorfktwert{green!50!black} %Farbe einzelner Punkte des Graphen

\newcommand{\MPfadBilder}{Bilder}

\ifttm%
\newcommand{\jMD}{\,\MD}%
\else%
\newcommand{\jMD}{\;\MD}%
\fi%

\def\jHTMLHinweisBedienung{\MInputHint{%
Mit Hilfe der Symbole am oberen Rand des Fensters
k"onnen Sie durch die einzelnen Abschnitte navigieren.}}

\def\jHTMLHinweisEingabeText{\MInputHint{%
Geben Sie jeweils ein Wort oder Zeichen als Antwort ein.}}

\def\jHTMLHinweisEingabeTerm{\MInputHint{%
Klammern Sie Ihre Terme, um eine eindeutige Eingabe zu erhalten. 
Beispiel: Der Term $\frac{3x+1}{x-2}$ soll in der Form
\texttt{(3*x+1)/((x+2)^2}$ eingegeben werden (wobei auch Leerzeichen 
eingegeben werden k"onnen, damit eine Formel besser lesbar ist).}}

\def\jHTMLHinweisEingabeIntervalle{\MInputHint{%
Intervalle werden links mit einer "offnenden Klammer und rechts mit einer 
schlie"senden Klammer angegeben. Eine runde Klammer wird verwendet, wenn der 
Rand nicht dazu geh"ort, eine eckige, wenn er dazu geh"ort. 
Als Trennzeichen wird ein Komma oder ein Semikolon akzeptiert.
Beispiele: $(a, b)$ offenes Intervall,
$[a; b)$ links abgeschlossenes, rechts offenes Intervall von $a$ bis $b$. 
Die Eingabe $]a;b[$ f"ur ein offenes Intervall wird nicht akzeptiert.
F"ur $\infty$ kann \texttt{infty} oder \texttt{unendlich} geschrieben werden.}}

\def\jHTMLHinweisEingabeFunktionen{\MInputHint{%
Schreiben Sie Malpunkte (geschrieben als \texttt{*}) aus und setzen Sie Klammern um Argumente f�r Funktionen.
Beispiele: Polynom: \texttt{3*x + 0.1}, Sinusfunktion: \texttt{sin(x)}, 
Verkettung von cos und Wurzel: \texttt{cos(sqrt(3*x))}.}}

\def\jHTMLHinweisEingabeFunktionenSinCos{\MInputHint{%
Die Sinusfunktion $\sin x$ wird in der Form \texttt{sin(x)} angegeben, %
$\cos\left(\sqrt{3 x}\right)$ durch \texttt{cos(sqrt(3*x))}.}}

\def\jHTMLHinweisEingabeFunktionenExp{\MInputHint{%
Die Exponentialfunktion $\MEU^{3x^4 + 5}$ wird als
\texttt{exp(3 * x^4 + 5)} angegeben, %
$\ln\left(\sqrt{x} + 3.2\right)$ durch \texttt{ln(sqrt(x) + 3.2)}.}}

% ---------------------------------- Spezialbefehle fuer Fachbereich Physik --------------------------

\newcommand{\E}{{e}}
\newcommand{\ME}[1]{\cdot 10^{#1}}
\newcommand{\MU}[1]{\;\mathrm{#1}}
\newcommand{\MPG}[3]{%
  \ifnum#2=0%
    #1\ \mathrm{#3}%
  \else%
    #1\cdot 10^{#2}\ \mathrm{#3}%
  \fi}%
%

\newcommand{\MMul}{\MExponentensymbXYZl} % Nur eine Abkuerzung


% ---------------------------------- Stichwortfunktionialitaet ---------------------------------------

% mpreindexentry wird durch Auswahlroutine in conv.pl durch mindexentry substitutiert
\ifttm%
\def\MIndex#1{\index{#1}\special{html:<!-- mpreindexentry;;}#1\special{html:;;}\arabic{MSubjectArea}\special{html:;;}%
\arabic{chapter}\special{html:;;}\arabic{section}\special{html:;;}\arabic{subsection}\special{html:;;}\arabic{MEntryCounter}\special{html:; //-->}%
\setcounter{MLastIndex}{\value{MEntryCounter}}%
\addtocounter{MEntryCounter}{1}%
}%
% Copyrightliste wird als tex-Datei im preprocessing von conv.pl erzeugt und unter converter/tex/entrycollection.tex abgelegt
% Der input-Befehl funktioniert nur, wenn die aufrufende tex-Datei auf der obersten Ebene liegt (d.h. selbst kein input/include ist, insbesondere keine Moduldatei)
\def\MEntryList{} % \input funktioniert nicht, weil ttm (und damit das \input) ausgefuehrt wird, bevor Datei da ist
\else%
\def\MIndex#1{\index{#1}}
\def\MEntryList{\MAbort{Stichwortliste nur im HTML realisierbar}}%
\fi%

\def\MEntry#1#2{\textbf{#1}\MIndex{#2}} % Idee: MLastType auf neuen Entry-Typ und dann ein MLabel vergeben mit autogen-Nummer

% ---------------------------------- Befehle fuer Tests ----------------------------------------------

% MEquationItem stellt eine Eingabezeile der Form Vorgabe = Antwortfeld her, der zweite Parameter kann z.B. MSimplifyQuestion-Befehl sein
\ifttm
\newcommand{\MEquationItem}[2]{{#1}$\,=\,${#2}}%
\else%
\newcommand{\MEquationItem}[2]{{#1}$\;\;=\,${#2}}%
\fi

\ifttm
\newcommand{\MInputHint}[1]{%
\ifnum%
\if\value{MTestSite}>0%
\else%
{\color{blue}#1}%
\fi%
\fi%
}
\else
\newcommand{\MInputHint}[1]{\relax}
\fi

\ifttm
\newcommand{\MInTestHeader}{%
Dies ist ein einreichbarer Test:
\begin{itemize}
\item{Im Gegensatz zu den offenen Aufgaben werden beim Eingeben keine Hinweise zur Formulierung der mathematischen Ausdr�cke gegeben.}
\item{Der Test kann jederzeit neu gestartet oder verlassen werden.}
\item{Der Test kann durch die Buttons am Ende der Seite beendet und abgeschickt, oder zur�ckgesetzt werden.}
\item{Der Test kann mehrfach probiert werden. F�r die Statistik z�hlt die zuletzt abgeschickte Version.}
\end{itemize}
}
\else
\newcommand{\MInTestHeader}{%
\relax
}
\fi

\ifttm
\newcommand{\MInTestFooter}{%
\special{html:<button name="Name_TESTFINISH" id="TESTFINISH" type="button" onclick="finish_button('}\MTestName\special{html:');">Test auswerten</button>}%
\begin{html}
&nbsp;&nbsp;&nbsp;&nbsp;&nbsp;&nbsp;&nbsp;&nbsp;
<button name="Name_TESTRESET" id="TESTRESET" type="button" onclick="reset_button();">Test zur�cksetzen</button>
<br />
<br />
<div class="xreply">
<p name="Name_TESTEVAL" id="TESTEVAL">
Hier erscheint die Testauswertung!
<br />
</p>
</div>
\end{html}
}
\else
\newcommand{\MInTestFooter}{%
\relax
}
\fi


% ---------------------------------- Notationsmakros -------------------------------------------------------------

% Notationsmakros die nicht von der Kursvariante abhaengig sind

\newcommand{\MZahltrennzeichen}[1]{\renewcommand{\MZXYZhltrennzeichen}{#1}}

\ifttm
\newcommand{\MZahl}[3][\MZXYZhltrennzeichen]{\edef\MZXYZtemp{\noexpand\special{html:<mn>#2#1#3</mn>}}\MZXYZtemp}
\else
\newcommand{\MZahl}[3][\MZXYZhltrennzeichen]{{}#2{#1}#3}
\fi

\newcommand{\MEinheitenabstand}[1]{\renewcommand{\MEinheitenabstXYZnd}{#1}}
\ifttm
\newcommand{\MEinheit}[2][\MEinheitenabstXYZnd]{{}#1\edef\MEINHtemp{\noexpand\special{html:<mi mathvariant="normal">#2</mi>}}\MEINHtemp} 
\else
\newcommand{\MEinheit}[2][\MEinheitenabstXYZnd]{{}#1 \mathrm{#2}} 
\fi

\newcommand{\MExponentensymbol}[1]{\renewcommand{\MExponentensymbXYZl}{#1}}
\newcommand{\MExponent}[2][\MExponentensymbXYZl]{{}#1{} 10^{#2}} 

%Punkte in 2 und 3 Dimensionen
\newcommand{\MPointTwo}[3][]{#1(#2\MCoordPointSep #3{}#1)}
\newcommand{\MPointThree}[4][]{#1(#2\MCoordPointSep #3\MCoordPointSep #4{}#1)}
\newcommand{\MPointTwoAS}[2]{\left(#1\MCoordPointSep #2\right)}
\newcommand{\MPointThreeAS}[3]{\left(#1\MCoordPointSep #2\MCoordPointSep #3\right)}

% Masseinheit, Standardabstand: \,
\newcommand{\MEinheitenabstXYZnd}{\MThinspace} 

% Horizontaler Leerraum zwischen herausgestellter Formel und Interpunktion
\ifttm
\newcommand{\MDFPSpace}{\,}
\newcommand{\MDFPaSpace}{\,\,}
\newcommand{\MBlank}{\ }
\else
\newcommand{\MDFPSpace}{\;}
\newcommand{\MDFPaSpace}{\;\;}
\newcommand{\MBlank}{\ }
\fi

% Satzende in herausgestellter Formel mit horizontalem Leerraum
\newcommand{\MDFPeriod}{\MDFPSpace .}

% Separation von Aufzaehlung und Bedingung in Menge
\newcommand{\MCondSetSep}{\,:\,} %oder '\mid'

% Konverter kennt mathopen nicht
\ifttm
\def\mathopen#1{}
\fi

% -----------------------------------START Rouletteaufgaben ------------------------------------------------------------

\ifttm
% #1 = Dateiname, #2 = eindeutige ID fuer das Roulette im Kurs
\newcommand{\MDirectRouletteExercises}[2]{
\begin{MExercise}
\texttt{Im HTML erscheinen hier Aufgaben aus einer Aufgabenliste...}
\end{MExercise}
}
\else
\newcommand{\MDirectRouletteExercises}[2]{\relax} % wird durch mconvert.pl gefunden und ersetzt
\fi


% ---------------------------------- START Makros, die von der Kursvariante abhaengen ----------------------------------

\ifvariantunotation
  % unotation = An Universitaeten uebliche Notation
  \def\MVariant{unotation}

  % Trennzeichen fuer Dezimalzahlen
  \newcommand{\MZXYZhltrennzeichen}{.}

  % Exponent zur Basis 10 in der Exponentialschreibweise, 
  % Standardmalzeichen: \times
  \newcommand{\MExponentensymbXYZl}{\times} 

  % Begrenzungszeichen fuer offene Intervalle
  \newcommand{\MoIl}[1][]{\mbox{}#1(\mathopen{}} % bzw. ']'
  \newcommand{\MoIr}[1][]{#1)\mbox{}} % bzw. '['

  % Zahlen-Separation im IntervaLL
  \newcommand{\MIntvlSep}{,} %oder ';'

  % Separation von Elementen in Mengen
  \newcommand{\MElSetSep}{,} %oder ';'

  % Separation von Koordinaten in Punkten
  \newcommand{\MCoordPointSep}{,} %oder ';' oder '|', '\MThinspace|\MThinspace'

\else
  % An dieser Stelle wird angenommen, dass std-Variante aktiv ist
  % std = beschlossene Notation im TU9-Onlinekurs 
  \def\MVariant{std}

  % Trennzeichen fuer Dezimalzahlen
  \newcommand{\MZXYZhltrennzeichen}{,}

  % Exponent zur Basis 10 in der Exponentialschreibweise, 
  % Standardmalzeichen: \times
  \newcommand{\MExponentensymbXYZl}{\times} 

  % Begrenzungszeichen fuer offene Intervalle
  \newcommand{\MoIl}[1][]{\mbox{}#1]\mathopen{}} % bzw. '('
  \newcommand{\MoIr}[1][]{#1[\mbox{}} % bzw. ')'

  % Zahlen-Separation im IntervaLL
  \newcommand{\MIntvlSep}{;} %oder ','
  
  % Separation von Elementen in Mengen
  \newcommand{\MElSetSep}{;} %oder ','

  % Separation von Koordinaten in Punkten
  \newcommand{\MCoordPointSep}{;} %oder '|', '\MThinspace|\MThinspace'

\fi



% ---------------------------------- ENDE Makros, die von der Kursvariante abhaengen ----------------------------------


% diese Kommandos setzen Mathemodus vorraus
\newcommand{\MGeoAbstand}[2]{[\overline{{#1}{#2}}]}
\newcommand{\MGeoGerade}[2]{{#1}{#2}}
\newcommand{\MGeoStrecke}[2]{\overline{{#1}{#2}}}
\newcommand{\MGeoDreieck}[3]{{#1}{#2}{#3}}

%
\ifttm
\newcommand{\MOhm}{\special{html:<mn>&#x3A9;</mn>}}
\else
\newcommand{\MOhm}{\Omega} %\varOmega
\fi


\def\PERCTAG{\MAbort{PERCTAG ist zur internen verwendung in mconvert.pl reserviert, dieses Makro darf sonst nicht benutzt werden.}}

% Im Gegensatz zu einfachen html-Umgebungen werden MDirectHTML-Umgebungen von mconvert.pl am ganzen ttm-Prozess vorbeigeschleust und aus dem PDF komplett ausgeschnitten
\ifttm%
\newenvironment{MDirectHTML}{\begin{html}}{\end{html}}%
\else%
\newenvironment{MDirectHTML}{\begin{html}}{\end{html}}%
\fi

% Im Gegensatz zu einfachen Mathe-Umgebungen werden MDirectMath-Umgebungen von mconvert.pl am ganzen ttm-Prozess vorbeigeschleust, ueber MathJax realisiert, und im PDF als $$ ... $$ gesetzt
\ifttm%
\newenvironment{MDirectMath}{\begin{html}}{\end{html}}%
\else%
\newenvironment{MDirectMath}{\begin{equation*}}{\end{equation*}}% Vorsicht, auch \[ und \] werden in amsmath durch equation* redefiniert
\fi

% ---------------------------------- Location Management ---------------------------------------------

% #1 = buttonname (muss in files/images liegen und Format 48x48 haben), #2 = Vollstaendiger Einrichtungsname, #3 = Kuerzel der Einrichtung,  #4 = Name der include-texdatei
\ifttm
\newcommand{\MLocationSite}[3]{\special{html:<!-- mlocation;;}#1\special{html:;;}#2\special{html:;;}#3\special{html:;; //-->}}
\else
\newcommand{\MLocationSite}[3]{\relax}
\fi

% ---------------------------------- Copyright Management --------------------------------------------

\newcommand{\MCCLicense}{%
{\color{green}\textbf{CC BY-SA 3.0}}
}

\newcommand{\MCopyrightLabel}[1]{ (\MSRef{L_COPYRIGHTCOLLECTION}{Lizenz})\MLabel{#1}}

% Copyrightliste wird als tex-Datei im preprocessing erzeugt und unter converter/tex/copyrightcollection.tex abgelegt
% Der input-Befehl funktioniert nur, wenn die aufrufende tex-Datei auf der obersten Ebene liegt (d.h. selbst kein input/include ist, insbesondere keine Moduldatei)
\newcommand{\MCopyrightCollection}{\input{copyrightcollection.tex}}

% MCopyrightNotice fuegt eine Copyrightnotiz ein, der parser ersetzt diese durch CopyrightNoticePOST im preparsing, diese Definition wird nur fuer reine pdflatex-Uebersetzungen gebraucht
% Parameter: #1: Kurze Lizenzbeschreibung (typischerweise \MCCLicense)
%            #2: Link zum Original (http://...) oder NONE falls das Bild selbst ein Original ist, oder TIKZ falls das Bild aus einer tikz-Umgebung stammt
%            #3: Link zum Autor (http://...) oder MINT falls Original im MINT-Kolleg erstellt oder NONE falls Autor unbekannt
%            #4: Bemerkung (z.B. dass Datei mit Maple exportiert wurde)
%            #5: Labelstring fuer existierendes Label auf das copyrighted Objekt, mit MCopyrightLabel erzeugt
%            Keines der Felder darf leer sein!
\newcommand{\MCopyrightNotice}[5]{\MCopyrightNoticePOST{#1}{#2}{#3}{#4}{#5}}

\ifttm%
\newcommand{\MCopyrightNoticePOST}[5]{\relax}%
\else%
\newcommand{\MCopyrightNoticePOST}[5]{\relax}%
\fi%

% ---------------------------------- Meldungen fuer den Benutzer des Konverters ----------------------
\MPragma{mintmodversion;P0.1.0}
\MPragma{usercomment;This is file mintmod.tex version P0.1.0}


% ----------------------------------- Spezialelemente fuer Konfigurationsseite, werden nicht von mintscripts.js verwaltet --

% #1 = DOM-id der Box
\ifttm\newcommand{\MConfigbox}[1]{\special{html:<input cfieldtype="2" type="checkbox" name="Name_}#1\special{html:" id="}#1\special{html:" onchange="confHandlerChange('}#1\special{html:');"/>}}\fi % darf im PDF nicht aufgerufen werden!


\MPragma{MathSkip}

\Mtikzexternalize

\begin{document}
\MSetSubject{\MINTMathematics}

\MSection{Elementare Funktionen} 
\MSetSectionID{elfunktionen}

\begin{MSectionStart}
\MLabel{VBKM06}
\MDeclareSiteUXID{VBKM06_START}

\MModstartBox
\end{MSectionStart}

\MSubsection{Grundlegendes zu Funktionen}
\MLabel{VBKM06_Grundlagen}

\begin{MIntro}
\MDeclareSiteUXID{VBKM06_Grundlagen_Intro}
Aus Modul \MNRef{VBKM01} kennen wir bereits die \highlight{reellen Zahlen} als \highlight{Menge} sowie \highlight{Intervalle} als wichtige \highlight{Teilmengen} der reellen Zahlen.


\begin{MExample}
Wir möchten die gesamten reellen Zahlen $\R$ außer der Zahl $0\in\R$ in einer Menge zusammenfassen. Wie schreiben wir eine solche Zahlenmenge auf? Hierfür gibt es die Schreibweise
\[
 \R\setminus\{0\} \MDFPeriod
\]
Diese wird gelesen als ``$\R$ ohne $0$''. Eine weitere Schreibweise für diese Menge ist die Vereinigung zweier offener Intervalle:
\[
 \R\setminus\{0\}=(-\infty\MIntvlSep 0)\cup(0\MIntvlSep \infty) \MDFPeriod
\]
Genauso kann man aus beliebigen anderen Mengen einzelne Zahlen entfernen. So beinhaltet etwa die Menge
\[
 [1\MIntvlSep 3)\setminus\{2\} \MDFPSpace,
\]
alle Zahlen aus dem halboffenen Intervall $[1\MIntvlSep 3)$ außer der Zahl $2$:

\MTikzAuto{%
\begin{tikzpicture}
% reelle Achse
\draw[->,color=black] (-1,0.0) -- (5,0.0);
\foreach \x in {-1, 0, 1, 2, 3, 4}
\draw[shift={(\x,0)},color=black] (0pt,2pt) -- (0pt,-2pt) node[below] {\footnotesize $\x$};
\draw (4.9,-0.3) node[] {$\mathbb{R}$};
% Intervall: [1,2):
\draw [line width=2.0pt,color=blue] (1,0.0)-- (2,0.0);
% Intervall: (2,3):
\draw [line width=2.0pt,color=blue] (2,0.0)-- (3,0.0);
% fehlende 2:
\draw [color = blue, fill = white] (2,0) circle (1.5pt);
% Enden:
\draw [fill = blue] (1,0) circle (1.5pt);
\draw [color = blue, fill = white] (3,0) circle (1.5pt);
\end{tikzpicture}
}%
\end{MExample}

\begin{MExercise}
	Wie sehen die Intervalle $(-\infty\MIntvlSep \pi)$ und $(8\MIntvlSep \MZahl{8}{5}]$ auf der Zahlengeraden aus?
\begin{MHint}{\iSolution}
\MTikzAuto{%
\begin{tikzpicture}
% reelle Achse
\draw[->,color=black] (-2.0,0.0) -- (6.0,0.0);
\foreach \x in {-1, 0, 1,2,3,4,5}
\draw[shift={(\x,0)},color=black] (0pt,2pt) -- (0pt,-2pt) node[below] {\footnotesize $\x$};
\draw (5.7,-0.3) node[] {$\mathbb{R}$};
% Intervall: [-\infty,\pi)
\draw [line width=2.0pt,color=blue] (-2,0.0)-- (3.14,0.0);
\draw [color = blue, fill = white] (3.14,0) circle (1.5pt);
\end{tikzpicture}
}%

\MTikzAuto{%
\begin{tikzpicture}
% reelle Achse
\draw[->,color=black] (5,0.0) -- (11,0.0);
\foreach \x in {6, 7, 8, 9, 10}
\draw[shift={(\x,0)},color=black] (0pt,2pt) -- (0pt,-2pt) node[below] {\footnotesize $\x$};
\draw (10.7,-0.3) node[] {$\mathbb{R}$};
% Intervall: (8,8.5]
\draw [line width=2.0pt,color=blue] (8,0.0)-- (8.5,0.0);
\draw [color = blue, fill = white] (8,0) circle (1.5pt);
\draw [fill = blue] (8.5,0) circle (1.5pt);
\end{tikzpicture}
}%

\end{MHint}
\end{MExercise}



Die Betrachtung von Mengen sowie Gleichungen und Ungleichungen für Zahlen aus diesen Mengen, wie in den vorhergehenden Modulen (etwa \MSRef{VBKM01}{Modul 1}) geschehen, reicht nicht aus um Mathematik zu betreiben und anzuwenden. Darüber hinaus brauchen wir sogenannte \highlight{Funktionen} (diese werden oft auch als \modsemph{Abbildungen} bezeichnet). 
\begin{MInfo}
\MLabel{VBKM06_Abbildungen}
\MEntry{Funktionen}{Funktion} (bzw. \MEntry{Abbildungen}{Abbildung}) sind Zuordnungen zwischen den Elementen zweier Mengen, die jedem Element der einen Menge genau ein Element der anderen Menge zuordnet.%%%
\end{MInfo}
\end{MIntro}

Diesem grundlegenden mathematischen Begriff der Zuordnung zwischen Mengen werden wir uns im ersten Abschnitt \MNRef{sec:zuordnungen} widmen. Im Abschnitt \MNRef{sec:anwendungen} stellen wir Bezüge zu Anwendungen der Mathematik in anderen Wissenschaften her und machen uns die Nützlichkeit des mathematischen Funktionsbegriffs, als Formalisierung von abhängigen Größen, bewusst. Schließlich untersuchen wir in Abschnitt \MNRef{sec:graphen} die bildliche Darstellung von Funktionen mittels Graphen. Im weiteren Verlauf dieses Moduls werden wir dann die wichtigsten elementaren Funktionen zusammen mit ihren Graphen betrachten. Es ist fundamental, den Verlauf der Graphen der elementaren Funktionen zu kennen. 


\begin{MXContent}{Zuordnungen zwischen Mengen}{Zuordnungen}{STD}\MLabel{sec:zuordnungen}
\MDeclareSiteUXID{VBKM06_Zuordnungen}

Wir beginnen mit einem ersten Beispiel einer Funktion als Zuordnung zwischen zwei Mengen.
Dazu betrachten wir die Menge der natürlichen Zahlen $\N$ sowie die Menge der rationalen Zahlen $\Q$
und veranschaulichen uns diese als zwei "`Container'' mit Zahlen.

\MTikzAuto{%
\begin{tikzpicture}
% natürliche Zahlen:
\draw  (3.5,10.3) node[anchor=north west] {$\mathbb{N}$};
\draw [] (2.7,7.6) ellipse (1.7cm and 2.4cm);
\draw (2,6.3)  node[anchor=north west](eins) {1};
\draw (2.1,7.4)  node[anchor=north west](zwei) {2};
\draw (2.3,8.7)  node[anchor=north west] (drei){3};
\draw (2.5,9.5) node[anchor=north west] (vier) {4};
\draw (2.7,9.8) node[anchor=north west] {...};
% rationale Zahlen:
\draw (9.8,10.3) node[anchor=north west] {$\mathbb{Q}$};
\draw (8.8,7.5) ellipse (1.9cm and 2.6cm);
\draw (8,8) node[anchor=north west](Feins) {$\frac{1}{2}$};
\draw (9.8,8.4) node[anchor=north west] {$\frac{3}{4}$};
\draw (7.4,6.7) node[anchor=north west] {$-\frac{1}{2}$};
\draw (7.8,9.6) node[anchor=north west](Fvier) {$2$};
\draw (7.1,7.5) node[anchor=north west] {$0$};
\draw (8.6,9.9) node[anchor=north west] {$3$};
\draw (7.3,8.7) node[anchor=north west](Fzwei) {$1$};
\draw (8.6,6.9) node[anchor=north west] {$\frac{1}{3}$};
\draw (8.5,8.7) node[anchor=north west] {$\frac{2}{3}$};
\draw (9.0,7.8) node[anchor=north west] {$\frac{2}{5}$};
\draw (9.3,9.3) node[anchor=north west](Fdrei) {$\frac{3}{2}$};
\draw (9.3,6.5) node[anchor=north west] {...};
\end{tikzpicture} 
}%

Nun wollen wir eine Zuordnung zwischen den Elementen dieser beiden Mengen auf folgende Art durchführen. Jeder beliebigen Zahl $n\in\N$ wird die Hälfte dieser Zahl $\frac{n}{2}\in\Q$ zugeordnet, also der Zahl $1\in\N$ die Zahl $\frac{1}{2}\in\Q$, der Zahl $2\in\N$ die Zahl $1\in\Q$ und immer so weiter. Dies können wir im Bild durch Pfeile veranschaulichen, die andeuten, welche Zahlen in $\N$ welchen Zahlen in $\Q$ zugeordnet werden.

\MTikzAuto{%
\begin{tikzpicture}
% natürliche Zahlen:
\draw  (3.5,10.3) node[anchor=north west] {$\mathbb{N}$};
\draw [] (2.7,7.6) ellipse (1.7cm and 2.4cm);
\draw [color = blue](2,6.3)  node[anchor=north west](eins) {1};
\draw [color = blue](2.1,7.4)  node[anchor=north west](zwei) {2};
\draw [color = blue](2.3,8.7)  node[anchor=north west] (drei){3};
\draw [color = blue](2.5,9.5) node[anchor=north west] (vier) {4};
\draw [color = blue](2.7,9.8) node[anchor=north west] {...};
% rationale Zahlen:
\draw (9.8,10.3) node[anchor=north west] {$\mathbb{Q}$};
\draw (8.8,7.5) ellipse (1.9cm and 2.6cm);
\draw (8,8) node[anchor=north west](Feins) {$\frac{1}{2}$};
\draw (9.8,8.4) node[anchor=north west] {$\frac{3}{4}$};
\draw (7.4,6.7) node[anchor=north west] {$-\frac{1}{2}$};
\draw (7.8,9.6) node[anchor=north west](Fvier) {$2$};
\draw (7.1,7.5) node[anchor=north west] {$0$};
\draw (8.6,9.9) node[anchor=north west] {$3$};
\draw (7.3,8.7) node[anchor=north west](Fzwei) {$1$};
\draw (8.6,6.9) node[anchor=north west] {$\frac{1}{3}$};
\draw (8.5,8.7) node[anchor=north west] {$\frac{2}{3}$};
\draw (9.0,7.8) node[anchor=north west] {$\frac{2}{5}$};
\draw (9.3,9.3) node[anchor=north west](Fdrei) {$\frac{3}{2}$};
\draw (9.3,6.5) node[anchor=north west] {...};
% Abbildung: 
\draw [|->] (eins) -- (Feins);
\draw [|->] (zwei) --(Fzwei);
\draw [|->] (drei) -- (Fdrei);
\draw [|->] (vier) --(Fvier);
\end{tikzpicture}
}%

Wir benutzen für die \modsemph{Zuordnung der einzelnen Elemente} der Mengen, die wir oben in Worten beschrieben haben, den sogenannten \highlight{Zuordnungspfeil}. Dies ist ein Pfeil, der auf einer Seite einen senkrechten Strich als Abschluss hat: $\longmapsto$. Er bedeutet, dass der Zahl auf der Seite mit dem senkrechten Strich die Zahl auf der Seite der Pfeilspitze zugeordnet wird: 
\[
\N\ni1\longmapsto\MZahl{0}{5}\in\Q \MDFPSpace,\MDFPaSpace \N\ni2\longmapsto1\in\Q \MDFPSpace, \MDFPaSpace \text{usw.} 
\]
Mit diesen Zuordnungen haben wir nun eine \modsemph{Funktion} von den natürlichen Zahlen $\N$ in die rationalen Zahlen $\Q$ konstruiert. In der Mathematik gibt man dieser Zuordnung nun einen Namen, d.h.~man reserviert ein Symbol (oft $f$ für \modsemph{F}unktion), das genau diese Zuordnung beschreiben soll. Dazu muss man die Zahlenmengen notieren, \modsemph{aus denen} und \modsemph{in die} zugeordnet werden soll. In diesem Fall werden den Elementen der natürlichen Zahlen $\N$ rationale Zahlen zugeordnet. Dies schreibt man mathematisch mit einem sogenannten \highlight{Abbildungspfeil} $\null\lto\null$, an dessen Spitze die Menge auftaucht, die das Ziel der Zuordnung ist und an dessen Basis die Menge steht, deren Elemente zugeordnet werden. In diesem Fall also
\[
 f\colon \N\lto\Q \MDFPeriod
\]
Man liest dies als "`die Funktion $f$ bildet von $\N$ nach $\Q$ ab''. 



Weiterhin können wir uns nun die Frage stellen, ob wir die Zuordnungen dieser Funktion $1\longmapsto\frac{1}{2},\ 2\longmapsto1,\ \text{usw.}$ kürzer
aufschreiben können. Dazu erinnern wir uns an den Beginn dieses Beispiels. Wir haben uns überlegt, jeder natürlichen Zahl $n$ ihre Hälfte $\frac{n}{2}$ zuzuordnen.
Damit können wir links und rechts des Zuordnungspfeils nun einfach diese beliebige natürliche Zahl $n$ bzw.~die sich daraus ergebende rationale Zahl $\frac{n}{2}$
hinschreiben:
\[
 n\longmapsto\frac{n}{2} \MDFPeriod
\]
Man liest dies als "`$n$ wird auf $\frac{n}{2}$ abgebildet''. Diese Schreibweise bezeichnet man auch als \highlight{Abbildungsvorschrift} der Funktion.
Eine weitere Schreibweise für die Abbildungsvorschrift benutzt den Namen der Funktion:
\[
 f(n) = \frac{n}{2} \MDFPeriod
\]
Man liest dies als "`$f$ von $n$ ist gleich $\frac{n}{2}$''.  Wir können also die hier
betrachtete Funktion $f$ nun zusammengefasst folgendermaßen schreiben:
\[
 \function{f}{\N}{\Q}{n}{\frac{n}{2} \MDFPeriod}
\]
Man liest dies nun als "`die Funktion $f$ bildet von $\N$ nach $\Q$ ab, jedes $n\in\N$ wird auf $\frac{n}{2}\in\Q$ abgebildet''.
Diese zusammenfassende Schreibweise werden wir im Rest dieses Moduls für Funktionen weiter verwenden.

Wir betrachten einige weitere einfache Beispiele für Funktionen:

\begin{MExample}\MLabel{bsp1:grundlagen}
\begin{itemize}
 \item Eine Funktion $g$ soll jeder reellen Zahl $x$ ihr Quadrat $x\cdot x=x^2$ zuordnen. Dies ergibt die sogenannte Standardparabel (siehe \MNRef{sec:monome}):
 \[
  \function{g}{\R}{\R}{x}{x^2 \MDFPeriod}
 \]
 Die Abbildungsvorschrift von $g$ lautet damit $g(x)=x^2$. Man kann dann die Zuordnungen für konkrete Zahlen ausrechnen. Zum Beispiel $g(2)=2^2=4$ oder $g(-\pi)=(-\pi)^2=\pi^2$, usw. 
 \item Eine Funktion $\Mvarphi$ soll jeder reellen Zahl $y$ zwischen $0$ und $1$ ihren dreifachen Wert plus $1$ zuordnen. Dies ist ein Beispiel für eine sogenannte affin-lineare Funktion (siehe \MNRef{VBKM06_sec:affin-linear}):
 \[
  \function{\Mvarphi}{(0\MIntvlSep 1)}{\R}{y}{3y+1 \MDFPeriod}
 \]
 Die Abbildungsvorschrift von $\Mvarphi$ lautet damit $\Mvarphi(y)=3y+1$. Somit errechnet man beispielsweise $\Mvarphi(\frac{1}{3})=3\cdot\frac{1}{3}+1=2$, usw. Allerdings kann man in diesem Fall $\Mvarphi(8)$ oder auch $\Mvarphi(1)$ \modsemph{nicht angeben}, da $8$ und $1$ keine Elemente der Menge $(0\MIntvlSep 1)$ sind.  %Dies wird unten im Rest dieses Abschnitts nochmal genauer erörtert werden. 
\end{itemize}

\end{MExample}

\begin{MExercise}\MLabel{ex1:grundlagen}
\begin{itemize}
 \item[(i)] Geben Sie eine Funktion $h$ an, die jeder positiven reellen Zahl $x$ ihren Kehrwert zuordnet. Berechnen Sie $h(2)$ und $h(1)$. Vervollständigen Sie die beiden Zuordnungen
 \[
  3\longmapsto\text{ ?}\quad\text{und}\quad\text{? }\longmapsto2
 \]
 von $h$.
 \item[(ii)] Beschreiben Sie in Worten die Zuordnung, die von folgender Funktion ausgeführt wird:
 \[
  \function{w}{[4\MIntvlSep 9]}{\R}{\alpha}{\sqrt{\alpha} \MDFPeriod}
 \]
 Berechnen Sie $w(9)$ und $w(5)$. Kann man auch $w(10)$ angeben?
\end{itemize}

\begin{MHint}{\iSolution}
 \begin{itemize}
  \item[(i)] Der Kehrwert von $x$ ist $\frac{1}{x}$. Die positiven reellen Zahlen sind die Menge $(0\MIntvlSep \infty)$. Damit kann man die Funktion $h$ schreiben als
  \[
   \function{h}{(0\MIntvlSep \infty)}{\R}{x}{\frac{1}{x} \MDFPeriod}
  \]
  Dies ist ein Beispiel für eine Funktion vom hyperbolischen Typ und wird in Abschnitt \MNRef{VBKM06_Potenz} genauer behandelt. Die Abbildungsvorschrift von $h$ ist $h(x)=\frac{1}{x}$, womit gilt $h(2)=\frac{1}{2}$ und $h(1)=\frac{1}{1}=1$. Außerdem berechnet man $h(3)=\frac{1}{3}$, womit $3\longmapsto\frac{1}{3}$ gilt. Weiterhin führt die Überlegung $h(\frac{1}{2})=\frac{1}{\frac{1}{2}}=2$ auf $\frac{1}{2}\longmapsto2$.
  \item[(ii)] Die Funktion $w$ ordnet jeder reellen Zahl $\alpha$, die größer oder gleich $4$ und kleiner oder gleich $9$ ist, ihre Quadratwurzel $\sqrt{\alpha}$ zu. Die Abbildungsvorschrift lautet $w(\alpha)=\sqrt{\alpha}$, womit $w(9)=\sqrt{9}=3$ sowie $w(5)=\sqrt{5}$ gilt. $w(10)$ kann nicht angegeben werden, da $10\notin[4\MIntvlSep 9]$.
 \end{itemize}

\end{MHint}
\end{MExercise}

Die obigen Beispiele zeigen einige Grundeigenschaften von Funktionen, für die wir nun spezielle Begriffe einführen wollen: 
\begin{MInfo}
Beim Aufschreiben einer Funktion gibt man eine Menge von Zahlen an, deren Elemente von der Funktion anderen Zahlen zugeordnet werden sollen.
Diese Menge heißt \MEntry{Definitionsbereich}{Definitionsbereich} oder Definitionsmenge der Funktion.
Hat die Funktion einen Namen, etwa $f$, so wird der Definitionsbereich mit dem Symbol $D_f$ bezeichnet. So ist zum Beispiel die Definitionsmenge der Funktion
\[
   \function{h}{(0\MIntvlSep \infty)}{\R}{x}{\frac{1}{x}}
\]
aus Aufgabe \MNRef{ex1:grundlagen} die Menge $D_h=(0\MIntvlSep \infty)$. Auch für die Elemente des Definitionsbereichs gibt es eine spezielle Bezeichnung.
In diesem Fall werden die Zahlen $x\in D_h$ mittels der Abbildungsvorschrift $h(x)=\frac{1}{x}$ zugeordnet.
Hierbei wird die Variable $x$ als die \MEntry{Veränderliche}{Veränderliche} der Funktion $h$ bezeichnet. 
\end{MInfo}

\begin{MExercise}
Geben Sie die Definitionsbereiche der Funktionen $w$ aus Aufgabe \MNRef{ex1:grundlagen} und $g$ aus Beispiel \MNRef{bsp1:grundlagen} an.
\begin{MHint}{\iSolution}
Es gilt
\[
  \function{w}{[4\MIntvlSep 9]}{\R}{\alpha}{\sqrt{\alpha}}
\]
und 
\[
  \function{g}{\R}{\R}{x}{x^2 \MDFPSpace,}
\]
womit man $D_w=[4\MIntvlSep 9]$ und $D_g=\R$ erhält.
\end{MHint}
\end{MExercise}

Betrachten wir die Abbildungsvorschrift $h(x)=\frac{1}{x}$ der Funktion $h$, so sehen wir, dass eigentlich nichts dagegen spricht,
jede beliebige reelle Zahl für $x$ in $\frac{1}{x}$ einzusetzen \modsemph{außer} der
Zahl $x=0$, da die Rechenoperation "`$\frac{1}{0}$'' kein Ergebnis liefert. Man kann bei der Angabe einer
Definitionsmenge also unterscheiden zwischen Zahlen, die ausgeschlossen sind, da man sie \modsemph{überhaupt nicht in die Abbildungsvorschrift einsetzen darf}, und
solchen, die ausgeschlossen sind, weil \modsemph{die Funktion eben so definiert ist.} Dies führt nun auf den Begriff des \highlight{größtmöglichen Definitionsbereichs}
einer Funktion, der größtmöglichen Teilmenge der reellen Zahlen $\R$, die man als Definitionsmenge einer Funktion mit bekannter Abbildungsvorschrift benutzen
kann.     

\begin{MExample}
Der größtmögliche Definitionsbereich $D_h\subset\R$ der Funktion
\[
 \function{h}{D_h}{\R}{x}{\frac{1}{x} \MDFPSpace,}
\]
ist $D_h=\R\setminus\{0\}$.
\end{MExample}

\begin{MExercise}
Geben Sie den größtmöglichen Definitionsbereich der Funktion
\[
  \function{w}{D_w}{\R}{\alpha}{\sqrt{\alpha}}
\]
an.
\begin{MHint}{\iSolution}
Die Wurzel liefert für alle nicht-negativen reellen Zahlen ein reelles Ergebnis. Somit gilt $D_w=[0\MIntvlSep \infty)$.%%%  
\end{MHint}
\end{MExercise}


Beim Aufschreiben von Funktionen ist neben dem Definitionsbereich noch eine zweite Menge notwendig, nämlich diejenige Menge, die das Ziel der durch die Funktion
beschriebenen Zuordnung ist. Diese wird als \highlight{Zielmenge} oder \highlight{Zielbereich}
bezeichnet. Betrachten wir nochmal die Funktion 
\[
  \function{\Mvarphi}{(0\MIntvlSep 1)}{\R}{y}{3y+1}
\]
aus Beispiel \MNRef{bsp1:grundlagen}. Deren Zielmenge sind die reellen Zahlen $\R$. Die Zielmenge der Funktion
\[
 \function{f}{\N}{\Q}{n}{\frac{n}{2}}
\]
aus dem einführenden Beispiel dieses Abschnitts sind die rationalen Zahlen $\Q$. Wir erkennen hier einen wichtigen Unterschied zwischen der Definitionsmenge
und der Zielmenge einer Funktion. Die Definitionsmenge enthält alle Zahlen, und nur diese, die man in die Abbildungsvorschrift der Funktion einsetzen darf
und möchte. Wohingegen die Zielmenge alle Zahlen enthalten kann, die potentiell als Ergebnis der Abbildungsvorschrift auftauchen können. 

In diesem Zusammenhang stellen wir uns die Frage, was denn der kleinstmögliche Zielbereich ist, den man für eine Funktion mit gegebenem
Definitionsbereich und bekannter Abbildungsvorschrift benutzen kann. Unter dem kleinstmöglichen Zielbereich verstehen wir all diejenigen Zahlen,
die -- bei gegebener Definitionsmenge und Abbildungsvorschrift -- tatsächlich als Ziele der Zuordnung auftauchen. Diese Menge bezeichnet man
als \MEntry{Wertebereich}{Wertebereich} oder Wertemenge und dessen Elemente als Werte der Funktion. Für eine Funktion $f$ benutzt man
das Symbol $W_f$ für die Wertemenge. Für die Werte einer Funktion $f$ mit Veränderlicher $x$
schreibt man allgemein meist $f(x)\in W_f$, wie in der Abbildungsvorschrift, oder führt eine weitere Variable ein, zum Beispiel $y=f(x)\in 
W_f$. 

\begin{MExample}Betrachten wir hierzu nochmal das Beispiel 
\[
  \function{\Mvarphi}{(0\MIntvlSep 1)}{\R}{y}{3y+1 \MDFPeriod}
\]
Der Wertebereich dieser Funktion ist
\[
W_{\Mvarphi}=(1\MIntvlSep 4) \MDFPeriod
\]
Dies sieht man ein, indem man einige Werte aus $D_{\Mvarphi}=(0\MIntvlSep 1)$ in die Abbildungsvorschrift einsetzt und die Ergebnisse berechnet.
Dies führt auf eine sogenannte \MEntry{Wertetabelle}{Wertetabelle}:

\begin{center}
\begin{tabular}{|c|c|c|c|c|c|}
\hline
$y$ & $\MZahl{0}{1}$ & $\MZahl{0}{3}$ & $\MZahl{0}{5}$ & $\MZahl{0}{7}$ & $\MZahl{0}{9}$ \\\hline 
$\Mvarphi(y)$ & $\MZahl{1}{3}$ & $\MZahl{1}{9}$ & $\MZahl{2}{5}$ & $\MZahl{3}{1}$ & $\MZahl{3}{7}$ \\ \hline
\end{tabular}
\end{center}
\end{MExample}

Solche Wertetabellen sind sinnvoll, um sich einen Überblick über die Werte einer Funktion zu verschaffen. Sie reichen aber nicht aus, um mathematisch \modsemph{ganz sicher} zu sein, was der tatsächliche Wertebereich einer Funktion ist. Eine Methode, den Wertebereich einer Funktion zu bestimmen, benutzt das Lösen von Ungleichungen:

\begin{MExample}
In der Funktion 
\[
  \function{\Mvarphi}{(0\MIntvlSep 1)}{\R}{y}{3y+1}
\]
gilt aufgrund des Definitionsbereichs $D_{\Mvarphi}=(0\MIntvlSep 1)$ für die Veränderliche:
\[
 0<y<1 \MDFPeriod
\]
Nun benutzen wir Äquivalenzumformungen, um in diesen Ungleichungen die Abbildungsvorschrift $\Mvarphi(y)=3y+1$ zu erzeugen:
\[
 0<y<1\,|\cdot 3\quad\Leftrightarrow\quad 0<3y<3\,|+1\quad\Leftrightarrow\quad 1<3y+1<4\quad\Leftrightarrow\quad 1<\Mvarphi(y)<4 \MDFPeriod
\]
Somit gilt für die Werte der Funktion $\Mvarphi(y)\in(1\MIntvlSep 4)$ und deshalb $W_{\Mvarphi}=(1\MIntvlSep 4)$.
\end{MExample}

\end{MXContent}

\begin{MXContent}{Funktionen in Mathematik und Anwendungen}{Mathe und Anwendungen}{STD}\MLabel{sec:anwendungen}
\MDeclareSiteUXID{VBKM06_FunktionenAnwendungen}
Mathematische Funktionen beschreiben oft Zusammenhänge zwischen Größen, die aus anderen Wissenschaften oder dem Alltagsbereich stammen. So ist etwa das Volumen $V$ eines Würfels abhängig von der Kantenlänge $a$ des Würfels. Damit kann das Volumen als mathematische Funktion aufgefasst werden, in der jeder Kantenlänge $a>0$ das entsprechende Volumen $V(a)=a^3$ zugeordnet wird:
\[
 \function{V}{(0\MIntvlSep \infty)}{\R}{a}{V(a)=a^3 \MDFPeriod}
\]
Es ergibt sich die kubische Standardparabel (siehe Abschnitt \MNRef{sec:monome}) als Zusammenhang zwischen Kantenlänge und Volumen. Auf diese Art lassen sich noch viele weitere Beispiele aus Naturwissenschaften und aus dem Alltagsbereich finden: Ort in Abhängigkeit von der Zeit in der Physik, Reaktionsgeschwindigkeit in Ab\-häng\-ig\-keit von der Konzentration in der Chemie, Mehlmenge in Abhängigkeit von der gewünschten Teigmenge in einem Rezept, usw. 



Wir betrachten dazu ein Beispiel. 
\begin{MExample}\MLabel{bsp:anwendungen}
Die Intensität radioaktiver Strahlung ist \highlight{umgekehrt proportional} zum Quadrat des Abstands von der Quelle. Dies wird auch als \modsemph{Abstandsgesetz} bezeichnet. Unter Benutzung einer physikalischen \highlight{Proportionalitätskonstanten} $c>0$ kann man den Zusammenhang zwischen Intensität $I$ der Strahlung und Abstand $r>0$ von der Quelle folgendermaßen als mathematische Funktion formulieren:
\[
 \function{I}{(0\MIntvlSep \infty)}{\R}{r}{\frac{c}{r^2} \MDFPeriod}
\]
Für die Intensität gilt also die Abbildungsvorschrift $I(r)=\frac{c}{r^2}$, die somit die Abhängigkeit der Größen $I$ und $r$ voneinander beschreibt. 
\end{MExample}

\begin{MExercise}\MLabel{ex:anwendungen}
Beim Bau von Windkraftanlagen gilt in guter Näherung, dass die Leistung \highlight{proportional} zur dritten Potenz der Windgeschwindigkeit ist. Unter Benutzung einer Proportionalitätskonstanten $\rho>0$, welche der folgenden mathematischen Funktionen gibt diese Abhängigkeit physikalischer Größen korrekt wieder?

a)
\[
 \function{P}{(0\MIntvlSep \infty)}{\R}{v}{P(v)=\frac{\rho}{v^3}}
\]
b)
\[
 \function{P}{\R}{\R}{v}{P(v)=\rho v^3}
\]
c)
\[
 \function{P}{[0\MIntvlSep \infty)}{\R}{v}{P(v)=\rho v^3}
\]
d)
\[
 \function{x}{[0\MIntvlSep \infty)}{\R}{f}{x(f)=\rho f^3}
\]
\begin{MHint}{\iSolution}
c) und d). a) ist falsch, da es sich um eine umgekehrte Proportionalität handelt. b) ist nicht der Problemstellung angepasst. Negative Windgeschwindigkeiten ergeben in diesem Zusammenhang keinen Sinn und sollten aus dem Definitionsbereich ausgeschlossen werden. c) ist korrekt; in diesem Fall ergibt sich die Abhängigkeit $P(v)=\rho v^3$ für die Leistung $P$ und die Windgeschwindigkeit $v$. Wir bemerken, dass d) genauso richtig ist. In diesem Fall haben wir eine Funktion, die völlig identisch zu Fall c) ist, nur die Leistung wird in diesem Fall mit $x$ und die Windgeschwindigkeit mit $f$ bezeichnet. Dies verdeutlicht nochmals, dass die Variablen, die für die Bezeichnung der Funktion und für die Veränderliche benutzt werden, mathematisch völlig willkürlich sind. Allerdings gibt es in den Naturwissenschaften Konventionen, welche Variablen \modsemph{üblicherweise} für bestimmte Größen verwendet werden. So ist es hier doch üblicher, die Geschwindigkeit mit $v$ (vom englischen \modsemph{velocity}) und die Leistung 
mit $P$ (
vom englischen \modsemph{power}) zu bezeichnen.
\end{MHint}

\end{MExercise}


\end{MXContent}

\begin{MXContent}{Umkehrbarkeit}{Umkehrbarkeit}{STD}\MLabel{sec:graphen}
\MDeclareSiteUXID{VBKM06_Umkehrbarkeit}
Die bildliche Darstellung einer Funktion, wie zum Beispiel
\[
 \function{f}{\R}{\R}{x}{x^2 \MDFPSpace,}
\]
als sogenanntes Venn-Diagramm (vgl.~Abschnitt \MNRef{sec:zuordnungen})

% venn7 Quadratische Zuordnung = venn4
\MTikzAuto{%
\begin{tikzpicture}[scale = 0.5]
% Definitionsbereich
\draw  (3.5,10.3) node[anchor=north west] {$\mathbb{R}$};
\draw [] (2.7,7.6) ellipse (1.7cm and 2.4cm);
\draw [color = blue](2,6.3)  node[anchor=north west](vierMinus) {-4};
\draw [color = blue](2.1,7.4)  node[anchor=north west](eins) {1};
\draw [color = blue](2.3,8.7)  node[anchor=north west] (vier){4};
\draw [color = blue](2.5,9.5) node[anchor=north west]  {...};
% Wertebereich
\draw (9.8,10.3) node[anchor=north west] {$\mathbb{R}$};
\draw (8.8,7.5) ellipse (1.9cm and 2.6cm);
\draw (8,8) node[anchor=north west] {$\frac{1}{2}$};
\draw (7.4,6.7) node[anchor=north west] {$-\frac{1}{2}$};
\draw (7.1,7.5) node[anchor=north west](Feins) {$1$};
\draw (8.0,9.8) node[anchor=north west] (Fvier){$16$};
\draw (7.3,8.7) node[anchor=north west]{$0$};
\draw (9.0,7.8) node[anchor=north west] {$\pi$};
\draw (9.3,9.3) node[anchor=north west] {$\frac{3}{2}$};
\draw (9.3,6.5) node[anchor=north west] {...};
% Abbildung: 
\draw[|->] (eins) -- (Feins);
\draw[|->] (vier) -- (Fvier);
\draw[|->] (vierMinus) -- (Fvier);
\end{tikzpicture}
}%

ist zwar nützlich, um den Funktionsbegriff zu verstehen, sagt aber nicht viel über die besonderen Eigenschaften der Funktion aus. Hierfür gibt es eine andere Möglichkeit der grafischen Darstellung, nämlich die des \highlight{Graphen} der Funktion. Dazu fertigen wir ein zweidimensionales Koordinatensystem (vgl.~Modul \MNRef{VBKM09}) an, in dem die Zahlen aus dem Definitionsbereich der Funktion auf der Querachse und die Zahlen aus dem Zielbereich auf der Hochachse eingetragen werden.
In einem solchen Koordinatensystem markieren wir alle Punkte $(x|f(x))$, die durch die Zuordnung der Funktion $x\longmapsto f(x)$ entstehen, in diesem Fall also alle Punkte $(x|x^2)$, d.h.~$(1|1)$, $(-1|1)$, $(-\frac{\pi}{2}|\frac{\pi^2}{4})$, usw. Dadurch entsteht eine Kurve, die \highlight{Graph von $f$} genannt und mit dem Symbol $G_f$ bezeichnet wird:

% grund1
\MTikzAuto{%
\begin{tikzpicture}
%Koordinatensystem
\draw[->,color=black] (-4,0) -- (3.9,0);
\foreach \x in {-3,-2,-1,1,2,3}
\draw[shift={(\x,0)},color=black] (0pt,2pt) -- (0pt,-2pt) node[below] {\footnotesize $\x$};
\draw[->,color=black] (0,-0.5) -- (0,4.7);
\foreach \y in {1,2,3,4}
\draw[shift={(0,\y)},color=black] (2pt,0pt) -- (-2pt,0pt) node[left] {\footnotesize $\y$};
\draw[color=black] (0pt,-10pt) node[right] {\footnotesize $0$};
%Achsenbeschriftung
\draw (4,0) node[anchor=north west] {$x$};
\draw (-1,5.2) node[anchor=north west] {$f(x)$};
% Graph (x,x^2)
\draw[smooth,samples=50,domain=-2.1:2.1, line width=2pt,color=green] plot(\x,{(\x)^(2.0)});
%Ausgewählte Elemente des Definitionsbereiches:
\foreach \x in {-1.57, -1, 1, 2}{
	\fill [color = cyan](\x,0.0) circle (1.5pt);
	% f(x):
	\fill [color = orange] (0, \x*\x) circle (1.5pt);
	%Punktepaare
	\draw [fill=green] (\x,\x*\x) circle (1.5pt);
	%kreuzlinien
	\draw [dashed, color = cyan] (\x,0)--(\x, \x*\x);
	\draw [dashed, color = orange] (0,\x*\x)--(\x, \x*\x);
}
% Beschriftung:
\begin{scriptsize}
  \draw (-1.6,2.4) node[anchor = east]{$(-\frac{\pi}{2}| \frac{\pi^2}{4})$};
	\draw (-1,1) node[anchor = east]{$(-1| 1)$};
	\draw (1,1) node[anchor = west]{$(1| 1)$};
	\draw (2,4) node[anchor = west]{$(2| 4)$};
\end{scriptsize}
\draw (-2,4) node[color = green, anchor = east]{$G_f:$ $\color{black} ( \color{blue} x \color{black}| \color{orange} f(x)\color{black})$};
\end{tikzpicture}
}%

Betrachten wir die Funktion 
\[
 \function{f}{\N}{\Q}{n}{\frac{n}{2}}
\]
aus Abschnitt \MNRef{sec:zuordnungen} und deren Graphen

% grund2 f: \N\rightarrow \R, n \mapsto n/2
\MTikzAuto{%
\begin{tikzpicture}
%Koordinatensystem
\draw[->,color=black] (-4,0) -- (3.9,0);
\foreach \x in {-3,-2,-1,1,2,3}
\draw[shift={(\x,0)},color=black] (0pt,2pt) -- (0pt,-2pt) node[below] {\footnotesize $\x$};
\draw[->,color=black] (0,-2) -- (0,2);
\foreach \y in {-1, 1}
\draw[shift={(0,\y)},color=black] (2pt,0pt) -- (-2pt,0pt) node[left] {\footnotesize $\y$};
\draw[color=black] (0pt,-10pt) node[right] {\footnotesize $0$};
%Achsenbeschriftung
\draw (4,0) node[anchor=north west] {$n$};
\draw (-1,2) node[anchor=north west] {$f(n)$};
%älle" Elemente des Definitionsbereiches:
\foreach \n in {-3,-2,-1, 0, 1,2,3}{
	\fill [color = cyan](\n,0.0) circle (1.5pt);
	% f(x):
	\fill [color = orange] (0, \n/2) circle (1.5pt);
	%Punktepaare
	\draw [color=green, fill=green] (\n,\n/2) circle (1.5pt);
	%kreuzlinien
	\draw [dashed, color = cyan] (\n,0)--(\n, \n/2);
	\draw [dashed, color = orange] (0,\n/2)--(\n, \n/2);
}
\draw (3.1,1.5) node[color = green, anchor = west]{$G_f$};
\end{tikzpicture}
}%

so stellen wir fest, dass Graphen nicht immer durchgehende Kurven sein müssen, sondern wie in diesem Fall auch nur aus einzelnen Punkten bestehen können. 

Anhand des Graphen sind nun viele Grundeigenschaften einer Funktion erkennbar. Rufen wir uns die Funktion 
\[
  \function{\Mvarphi}{(0\MIntvlSep 1)}{\R}{y}{3y+1}
\]
mit dem Definitionsbereich $D_{\Mvarphi}=(0\MIntvlSep 1)$ und dem Wertebereich $W_{\Mvarphi}=(1\MIntvlSep 4)$ aus Abschnitt \MNRef{sec:zuordnungen} ins Gedächtnis. Wenn wir ihren Graphen zeichnen, so erkennen wir, dass der Definitionsbereich und der Wertebereich auf der Quer- bzw.~Hochachse auftauchen:

% grund3 f: (0,1) --> \R, y \mapsto 3y+1
\MTikzAuto{%
\begin{tikzpicture} 
%Koordinatensystem
\draw[->,color=black] (-1.5,0) -- (2,0);
\foreach \x in {-1,1}
\draw[shift={(\x,0)},color=black] (0pt,2pt) -- (0pt,-2pt) node[below] {\footnotesize $\x$};
\draw[->,color=black] (0,-0.5) -- (0,5);
\foreach \y in {1, 2, 3, 4}
\draw[shift={(0,\y)},color=black] (2pt,0pt) -- (-2pt,0pt) node[left] {\footnotesize $\y$};
\draw[color=black] (0pt,-10pt) node[right] {\footnotesize $0$};
%Achsenbeschriftung
\draw (2,0) node[anchor=north west] {$y$};
\draw (-1,5) node[anchor=north west] {$\Mvarphi(y)$};
%Graph: x \mapsto 3x+1
\draw (0.5,2.5) node[anchor=west, color = green] {$G_{\Mvarphi}$};
\draw[smooth,samples=50,domain=0:1, line width=2pt,color=green] plot(\x,{3*(\x)+1});
\draw[color=green, fill=white] (1,4) circle (1.5pt);
%Definitionsbereich
\draw (0.5,0) node[anchor=south, color = cyan] {$D_{\Mvarphi}$};
\draw [line width=2.0pt,color=cyan] (0,0)-- (1,0);
\draw [color = blue, fill = white]  (0,0) circle (1.5pt);
\draw [color = blue, fill = white]  (1,0) circle (1.5pt);
% Wertebereich
\draw (0,2.5) node[anchor=east, color = orange] {$W_{\Mvarphi}$};
\draw [line width=2.0pt,color=orange] (0,1)-- (0,4);
\draw [color = orange, fill = white]  (0,1) circle (1.5pt);
\draw [color = orange, fill = white]  (0,4) circle (1.5pt);
\end{tikzpicture}
}%

\begin{MExercise}
Betrachten Sie nochmal den Graphen der Funktion
\[
 \function{f}{\R}{\R}{x}{x^2} \MDFPSpace,
\]
markieren Sie Definitions- und Wertebereiche auf Quer- und Hochachse und geben Sie diese an.
\begin{MHint}{\iSolution}
% grund4
\MTikzAuto{%
\begin{tikzpicture}
%Koordinatensystem
\draw[->,color=black] (-4,0) -- (3.9,0);
\foreach \x in {-3,-2,-1,1,2,3}
\draw[shift={(\x,0)},color=black] (0pt,2pt) -- (0pt,-2pt) node[below] {\footnotesize $\x$};
\draw[->,color=black] (0,-0.5) -- (0,4.7);
\foreach \y in {1,2,3,4}
\draw[shift={(0,\y)},color=black] (2pt,0pt) -- (-2pt,0pt) node[left] {\footnotesize $\y$};
\draw[color=black] (0pt,-10pt) node[right] {\footnotesize $0$};
%Achsenbeschriftung
\draw (4,0) node[anchor=north west] {$x$};
\draw (-1,5.2) node[anchor=north west] {$f(x)$};
% Graph (x,x^2)
\draw[smooth,samples=50,domain=-2.1:2.1] plot(\x,{(\x)^(2.0)});
%Definitionsbereich
\draw (1,0) node[anchor=south, color = cyan] {$D_f$};
\draw [color=cyan] (-4,0) -- (3.9,0);
% Wertebereich
\draw (0,2.5) node[anchor=east, color = orange] {$W_f$};
\draw [color=orange] (0,0)-- (0,4.7);
\fill [color = orange]  (0,0) circle (1.5pt);
\end{tikzpicture}
}%

$D_f=\R$, $W_f=[0\MIntvlSep \infty)$
\end{MHint}
\end{MExercise}

Weiterhin ist die Eigenschaft der Eindeutigkeit von Funktionen am Graphen zu erkennen. Um dies einzusehen, machen wir uns klar, dass eine Kurve wie in der folgenden Abbildung


% notfunction
\MTikzAuto{%
\begin{tikzpicture}
% Koordinatensystem
\node (xMAX) at (5,0){};
\node (yMAX) at (0,2.8){};
\draw[->,color=black] (-3,0) -- (xMAX);
\foreach \x in {-2,-1,1,2,3,4}
\draw[shift={(\x,0)},color=black] (0pt,2pt) -- (0pt,-2pt) node[below] {\footnotesize $\x$};
\draw[->,color=black] (0,-2) -- (yMAX);
\foreach \y in {-1,1,2}
\draw[shift={(0,\y)},color=black] (2pt,0pt) -- (-2pt,0pt) node[left] {\footnotesize $\y$};
\draw[color=black] (0pt,-10pt) node[right] {\footnotesize $0$};
% Achsenbeschriftung
\draw (xMAX) node[anchor=north east] {$x$};
\draw (yMAX) node[anchor=east] {$y$};
% notfunktion
\draw [rotate around={90:(0,0)}, 
			smooth,samples=100,
			domain=-1.5:1.9] 
			plot(\x,{((\x)-1.0)*(\x)*((\x)+1.0)-2});
% Punkte
\fill [color = red]  (2,1) circle (1.5pt);
\fill [color = red]  (2,0) circle (1.5pt);
\fill [color = red]  (2,-1) circle (1.5pt);
% x
\begin{scriptsize}
	\draw (2,0) circle node[anchor = north west ](x){$x$};
	% y1,y2,y3
	\draw (0,1) circle  node[anchor = west ]{$y_1$};
	\draw (0,0) circle  node[anchor = south west ]{$y_2$};
	\draw (0,-1) circle node[anchor = west ]{$y_3$};
\end{scriptsize}
\draw [color = red, dotted] (2,-1)--(2,0) --(2,1);
\end{tikzpicture}
}%

niemals als Graph einer Funktion auftauchen kann. Zu einem $x$-Wert aus dem Definitionsbereich müsste es hier mehrere Werte $y_1,y_2,y_3$ aus dem Wertebereich geben. Graphen von Funktionen geben die Eindeutigkeit also immer dadurch wieder, dass sie "`nicht in horizontaler Richtung zurücklaufen können''. 


Eine weitere wichtige Eigenschaft eines Graphen ist sein \modsemph{Wachstumsverhalten}.
Betrachten wir die Funktion 
\[
 \function{f}{\R}{\R}{x}{x^2}
\]
und ihren Graphen.

%GRAPH: mon1:
\MTikzAuto{%
\begin{tikzpicture} 
%Koordinatensystem
\node (xMAX) at (2.8,0){};
\node (yMAX) at (0,4.8){};
%Achsen
\draw[->,color=green, line width=2pt] (0,0) -- (3,0);
\draw[->,color=red, line width=2pt] (-3,0) -- (-0.05,0);
\draw[->,color=black] (-2.5,0) -- (xMAX);
\foreach \x in {-2,-1,1,2}
\draw[shift={(\x,0)},color=black] (0pt,2pt) -- (0pt,-2pt) node[below] {\footnotesize $\x$};
\draw[->,color=black] (0,-0.5) -- (yMAX);
\foreach \y in {1,2,3,4}
\draw[shift={(0,\y)},color=black] (2pt,0pt) -- (-2pt,0pt) node[left] {\footnotesize $\y$};
\draw[color=black] (0pt,-10pt) node[right] {\footnotesize $0$};
%Achsenbeschriftung
\draw (xMAX) node[anchor=north east] {$x$};
\draw (yMAX) node[anchor=east] {$f(x)$};
% Graph (x,x^2)
\draw[smooth,samples=50,domain=-2.1:2.1] plot(\x,{(\x)^(2.0)});
%Markierungen
\draw (2.1,4) node[right] {$G_f$};
\draw[color = black, fill = black]  (0,0) circle (1.5pt);
\draw[->,color=green] (1.5,1) -- (2.5,3);
\draw[color=green] (2,2) node[right] {wächst};
\draw[->,color=red] (-2.5,3) -- (-1.5,1);
\draw[color=red] (-2,2) node[left] {fällt};
\end{tikzpicture}
}%

Auf der Querachse in diesem Graphen erkennen wir zwei Bereiche, in denen der Graph unterschiedliches \modsemph{Wachstumsverhalten} zeigt. Im Bereich von $x$-Werten mit $x\in(-\infty\MIntvlSep 0)$ \modsemph{fällt} der Graph. Das heißt, werden die $x$-Werte größer, so werden die zugehörigen Funktionswerte auf der Hochachse kleiner. Im Bereich von $x$-Werten mit $x\in(0\MIntvlSep \infty)$ stellen wir das gegenteilige Verhalten fest. Bei größer werdenden $x$-Werten werden auch die zugehörigen Funktionswerte größer. Der Graph \modsemph{wächst}. Beim Wert $0\in\R$ geht der fallende Bereich in den wachsenden Bereich über. Solche Werte werden bei der Untersuchung von Scheitelpunkten in Abschnitt \MNRef{sec:polynome} und beim Bestimmen von \MSRef{VBKM07_Kurvendiskussion}{Extremwerten} in Modul \MNRef{VBKM07} besonders wichtig werden.

Wir bezeichnen diese beiden Eigenschaften als \highlight{streng monoton fallend} bzw.~\highlight{streng monoton wachsend} und schreiben sie mathematisch folgendermaßen auf:

\begin{MInfo}
\begin{itemize}
 \item Für $x_1<x_2$ aus einer Teilmenge des Definitionsbereichs einer Funktion $f$ gilt $f(x_1)>f(x_2)$. Dann heißt $f$ in dieser Teilmenge streng monoton fallend.

%GRAPH: mon2a:
\MTikzAuto{%
\begin{tikzpicture} 
%Koordinatensystem
\node (xMAX) at (3.8,0){};
\node (yMAX) at (0,3.8){};
%Achsen
\draw[->,color=black] (-.5,0) -- (xMAX);
\draw[->,color=black] (0,-.5) -- (yMAX);
\draw[color=black] (0pt,-10pt) node[right] {\footnotesize $0$};
%Achsenbeschriftung
\draw (xMAX) node[anchor=north east] {$x$};
\draw (yMAX) node[anchor=east] {$f(x)$};
% Graph (x,x^2)
\draw[smooth,samples=150,domain=0.5:2.8] plot(\x,{-sin(\x r)-\x +3.5});
\draw (0.5,2.52) node[anchor=south] {$G_f$};
%Markierungen
\draw[color = red, fill = red]  (1,1.659) circle (1.5pt);
\draw[color = red, fill = red]  (2,0.591) circle (1.5pt);
\draw[color = red, dotted] (1,1.659) -- (1,0);
\draw[color = red, dotted] (2,0.591) -- (2,0);
\draw[color = red, dotted] (1,1.659) -- (0,1.659);
\draw[color = red, dotted] (2,0.591) -- (0,0.591);
\draw[shift={(1,0)},color=red] (0pt,2pt) -- (0pt,-2pt) node[below] {\footnotesize{$x_1$}};
\draw[shift={(2,0)},color=red] (0pt,2pt) -- (0pt,-2pt) node[below] {\footnotesize{$x_2$}};
\draw[shift={(0,1.659)},color=red] (2pt,0pt) -- (-2pt,0pt) node[left] {\footnotesize{$f(x_1)$}};
\draw[shift={(0,0.591)},color=red] (2pt,0pt) -- (-2pt,0pt) node[left] {\footnotesize{$f(x_2)$}};
\draw[color=red] (1.5,-0.02) node[below] {\footnotesize{$<$}};
\draw[color=red] (-0.3,1.125) node[left] {\footnotesize{$\vee$}};
\end{tikzpicture}  
}%

 \item Für $x_1<x_2$ aus einer Teilmenge des Definitionsbereichs einer Funktion $f$ gilt $f(x_1)<f(x_2)$. Dann heißt $f$ in dieser Teilmenge streng monoton wachsend.

%GRAPH: mon2b:
\MTikzAuto{%
\begin{tikzpicture} 
%Koordinatensystem
\node (xMAX) at (3.8,0){};
\node (yMAX) at (0,3.8){};
%Achsen
\draw[->,color=black] (-.5,0) -- (xMAX);
\draw[->,color=black] (0,-.5) -- (yMAX);
\draw[color=black] (0pt,-10pt) node[right] {\footnotesize $0$};
%Achsenbeschriftung
\draw (xMAX) node[anchor=north east] {$x$};
\draw (yMAX) node[anchor=east] {$f(x)$};
% Graph (x,x^2)
\draw[smooth,samples=150,domain=0.5:2.8] plot(\x,{sin(\x r)+\x-.5});
\draw (2.8,2.52) node[anchor=south] {$G_f$};
%Markierungen
\draw[color = olive, fill = olive]  (1,1.3415) circle (1.5pt);
\draw[color = olive, fill = olive]  (2,2.409) circle (1.5pt);
\draw[color = olive, dotted] (1,1.3415) -- (1,0);
\draw[color = olive, dotted] (2,2.409) -- (2,0);
\draw[color = olive, dotted] (1,1.3415) -- (0,1.3415);
\draw[color = olive, dotted] (2,2.409) -- (0,2.409);
\draw[shift={(1,0)},color=olive] (0pt,2pt) -- (0pt,-2pt) node[below] {\footnotesize{$x_1$}};
\draw[shift={(2,0)},color=olive] (0pt,2pt) -- (0pt,-2pt) node[below] {\footnotesize{$x_2$}};
\draw[shift={(0,1.3415)},color=olive] (2pt,0pt) -- (-2pt,0pt) node[left] {\footnotesize{$f(x_1)$}};
\draw[shift={(0,2.409)},color=olive] (2pt,0pt) -- (-2pt,0pt) node[left] {\footnotesize{$f(x_2)$}};
\draw[color=olive] (1.5,-0.02) node[below] {\footnotesize{$<$}};
\draw[color=olive] (-0.3,1.875) node[left] {\footnotesize{$\vee$}};
\end{tikzpicture}
}%

\end{itemize}
\end{MInfo}

Dies gilt so für alle Funktionen, die wir in diesem Modul betrachten werden. Oft treffen die beschriebenen Monotonieeigenschaften nur in bestimmten Bereichen der Definitionsmenge der Funktion zu, wie oben bei der Standardparabel gesehen. Es gibt jedoch auch Funktionen, die nur eine der Monotonieeigenschaften im gesamten Definitionsbereich besitzen (siehe Beispiel \MNRef{bsp1:monoton} unten). In diesem Fall nennt man dann die gesamte Funktion streng monoton wachsend oder streng monoton fallend. Weiterhin heißt eine Funktion, die entweder streng monoton fallend oder streng monoton wachsend ist, \highlight{streng monoton}.

Ein weiteres Beispiel zeigt, wie man strenge Monotonie mit Hilfe des Lösens von \MSRef{M03_Ungleichungen}{Ungleichungen} aus Modul \MRef{VBKM03} bei einer Funktion explizit nachrechnen kann.

\begin{MExample}\MLabel{bsp1:monoton}%
Gegeben sei die Funktion
\[
 \function{h}{\R}{\R}{x}{-\frac{1}{2}x+1 \MDFPeriod}
\]
Ist $h$ streng monoton wachsend oder streng monoton fallend?

Wir beginnen, zwei beliebige Zahlen $x_1,x_2\in D_h=\R$ aufzuschreiben mit der Eigenschaft, dass 
\[
 x_1<x_2
\]
gilt. Durch 
\MSRef{VBKM03 AequivalenzumformungenUngleichungen}{{Ä}quivalenzumformungen} von Ungleichungen können wir $x_1<x_2$ nun entweder zu $h(x_1)<h(x_2)$ oder zu $h(x_1)>h(x_2)$ umformen und damit folgern, dass $h$ streng monoton wachsend oder streng monoton fallend ist. Es gelten die folgenden Äquivalenzumformungen für $x_1<x_2$:
\[
 x_1<x_2\,|\cdot(-\frac{1}{2})\quad\Leftrightarrow\quad -\frac{1}{2}x_1>-\frac{1}{2}x_2 \MDFPeriod
\]
Weiterhin wird in der Abbildungsvorschrift $+1$ addiert. Wir erhalten also
\[
 -\frac{1}{2}x_1>-\frac{1}{2}x_2\,|+1\quad\Leftrightarrow\quad -\frac{1}{2}x_1+1>-\frac{1}{2}x_2+1\quad\Leftrightarrow\quad h(x_1)>h(x_2) \MDFPeriod
\]
Da nun $h(x_1)>h(x_2)$ gilt, ist $h$ streng monoton fallend. Dies können wir auch am Graphen von $h$ erkennen:

%GRAPH: mon3:
\MTikzAuto{%
\begin{tikzpicture} 
%Koordinatensystem
\node (xMAX) at (4.8,0){};
\node (yMAX) at (0,2.8){};
%Achsen
\draw[->,color=black] (-1.5,0) -- (xMAX);
\foreach \x in {-1,1,2,3,4}
\draw[shift={(\x,0)},color=black] (0pt,2pt) -- (0pt,-2pt) node[below] {\footnotesize $\x$};
\draw[->,color=black] (0,-2.5) -- (yMAX);
\foreach \y in {-2,-1,1,2}
\draw[shift={(0,\y)},color=black] (2pt,0pt) -- (-2pt,0pt) node[left] {\footnotesize $\y$};
\draw[color=black] (0pt,-10pt) node[right] {\footnotesize $0$};
%Achsenbeschriftung
\draw (xMAX) node[anchor=north east] {$x$};
\draw (yMAX) node[anchor=east] {$h(x)$};
% Graph (x,x^2)
\draw[smooth,samples=150,domain=-1:4] plot(\x,{-0.5*\x+1});
\draw (-1,1.5) node[anchor=east] {$G_h$};
%Markierung
\draw[->,color=red] (1,1.2) -- (3,0.2);
\draw[color=red] (2,1) node[anchor=west] {fällt};
\end{tikzpicture}
}%

\end{MExample}

\begin{MExercise}\MLabel{ex1:monoton}
Rechnen Sie durch Äquivalenzumformungen von Ungleichungen explizit nach, dass
\[
 \function{\eta}{\R}{\R}{x}{2x+2}
\]
streng monoton wachsend ist.
\begin{MHint}{\iSolution}
Es gilt
\[
 x_1<x_2\,|\cdot2\quad\Leftrightarrow\quad2x_1<2x_2\,|+2\quad\Leftrightarrow\quad2x_1+2<2x_2+2\quad\Leftrightarrow\quad\eta(x_1)<\eta(x_2) \MDFPSpace,
\]
womit $\eta$ streng monoton wachsend ist.
\end{MHint}
\end{MExercise}




\end{MXContent}

\MSubsection{Lineare Funktionen und Polynome}
\MLabel{VBKM06_linear}

\begin{MIntro}
\MDeclareSiteUXID{VBKM06_EinfacheFunktionen_Intro}
In diesem Abschnitt untersuchen wir folgende Klassen von Funktionen: konstante, lineare, affin-lineare Funktionen
sowie Monome und Polynome. 
\end{MIntro}


\begin{MXContent}{Konstante Funktionen und die Identität}{Konstanten und Identität}{STD}
\MDeclareSiteUXID{VBKM06_EinfacheFunktionen_Konstanten}

Die sogenannten konstanten Funktionen ordnen jeder Zahl aus dem Definitionsbereich $\R$ \modsemph{eine} konstante Zahl aus dem Zielbereich $\R$ zu. Zum Beispiel die konstante Zahl $2$ auf folgende Art:
\[
 \function{f}{\R}{\R}{x}{2 \MDFPeriod}
\]
%BILD: venn-const
\MTikzAuto{%
\begin{tikzpicture}
%venn.Definitionsbereich
\draw (0,0) ellipse (1cm and 1.5cm);
\draw (-0.5,1.8) node[auto] {$\mathbb{R}$};
% x0, x1, x2
\draw(0.0, 1) node (x1){$-1$};
\draw (0.3,-1) node[auto](x2){$2$} ;
\draw (-0.3,-0.2)node[auto](xh){$\frac{1}{2}$};
\draw (0.6,0.2) node(xp) {$\pi$};
\draw (-.5,0.45) node {$\dots$};
%venn.Wertebereich
\draw (5,0) ellipse (1 cm and 1.5cm);
\draw (4.5,1.8) node[auto] {$\mathbb{R}$};
% f(x0), f(x1), f(x2)
\draw (5,1)  node[auto]{$-1$};
\draw (5,-1)  node[auto](Fx){$2$};
\draw (4.7,-0.2) node {$\frac{1}{2}$};
\draw (5.6,0.2) node {$\pi$};
\draw (4.6,0.45) node {$\dots$};
%Abbildung
\draw [|->] (x1) -- (Fx);
\draw [|->] (x2) -- (Fx);
\draw[|->] (xh) -- (Fx);
\draw[|->] (xp) -- (Fx);
\draw (2.5,0.3) node {$f$};
\end{tikzpicture}
}%
%Abstand:
\hspace{2cm}
%GRAPH: const1 
\MTikzAuto{%
\begin{tikzpicture} 
%Graph: konstant
\draw[color=red] (3,2) node[anchor=south] {$G_f$};
\draw[color=red] (-1.5,2)--(2.8,2);
%Koordinatensystem
\node (xMAX) at (2.8,0){};
\node (yMAX) at (0,3.5){};
\draw[->,color=black] (-1.5,0) -- (xMAX);
\foreach \x in {-1,1,2}
\draw[shift={(\x,0)},color=black] (0pt,2pt) -- (0pt,-2pt) node[below] {\footnotesize $\x$};
\draw[->,color=black] (0,-0.5) -- (yMAX);
\foreach \y in {1, 2, 3}
\draw[shift={(0,\y)},color=black] (2pt,0pt) -- (-2pt,0pt) node[left] {\footnotesize $\y$};
\draw[color=black] (0pt,-10pt) node[right] {\footnotesize $0$};
%Achsenbeschriftung
\draw (xMAX) node[anchor=north east] {$x$};
\draw (yMAX) node[anchor=east] {$f(x)$};
\end{tikzpicture}
}%

Es gilt hier also $f(x)=2$ für alle $x\in\R$, womit die Wertemenge dieser Funktion $f$ nur aus der Menge $W_f=\{2\}\subset\R$ besteht. 

Die Identität auf $\R$ ist die Funktion, welche jeder reellen Zahl wieder genau die identische reelle Zahl zuordnet. Man schreibt das so:
\[
 \function{\Id}{\R}{\R}{x}{x \MDFPeriod}
\]
%BILD: venn-id
\MTikzAuto{%
\begin{tikzpicture}
%venn.Definitionsbereich
\draw (0,0) ellipse (1cm and 1.5cm);
\draw (-0.5,1.8) node[auto] {$\mathbb{R}$};
% x0, x1, x2
\draw(0.0, 1) node (x1){$-1$};
\draw (0.3,-1) node[auto](x2){$2$} ;
\draw (-0.3,-0.2)node[auto](xh){$\frac{1}{2}$};
\draw (0.6,0.2) node(xp) {$\pi$};
\draw (-.5,0.45) node {$\dots$};
%venn.Wertebereich
\draw (5,0) ellipse (1 cm and 1.5cm);
\draw (4.5,1.8) node[auto] {$\mathbb{R}$};
% f(x0), f(x1), f(x2)
\draw (5,1)  node[auto](Fx1){$-1$};
\draw (5,-1)  node[auto](Fx2){$2$};
\draw (4.7,-0.2) node(Fxh) {$\frac{1}{2}$};
\draw (5.6,0.2) node(Fxp) {$\pi$};
\draw (4.6,0.45) node {$\dots$};
%Abbildung
\draw [|->] (x1) -- (Fx1);
\draw [|->] (x2) -- (Fx2);
\draw[|->] (xh) -- (Fxh);
\draw[|->] (xp) -- (Fxp);
\draw (2.5,1.3) node {$\Id$};
\end{tikzpicture}
}%
%Abstand:
\hspace{2cm}
%GRAPH: id1
\MTikzAuto{%
\begin{tikzpicture} 
%Graph: Id
\draw[color=red] (2.5,2.5) node[anchor=south] {$G_{\Id}$};
\draw[color=red] (-0.5,-0.5)--(2.5,2.5);
%Koordinatensystem
\node (xMAX) at (2.8,0){};
\node (yMAX) at (0,3.5){};
\draw[->,color=black] (-1.5,0) -- (xMAX);
\foreach \x in {-1,1,2}
\draw[shift={(\x,0)},color=black] (0pt,2pt) -- (0pt,-2pt) node[below] {\footnotesize $\x$};
\draw[->,color=black] (0,-0.5) -- (yMAX);
\foreach \y in {1, 2, 3}
\draw[shift={(0,\y)},color=black] (2pt,0pt) -- (-2pt,0pt) node[left] {\footnotesize $\y$};
\draw[color=black] (0pt,-10pt) node[right] {\footnotesize $0$};
%Achsenbeschriftung
\draw (xMAX) node[anchor=north east] {$x$};
\draw (yMAX) node[anchor=east] {$\operatorname{Id}(x)$};
\end{tikzpicture}
}%

Es gilt hier also $\Id(x)=x$ für alle $x\in\R$, womit der Wertebereich von $\Id$ die gesamten reellen Zahlen sind ($W_{\Id}=\R$). Weiterhin ist die Identität offenbar eine streng monoton wachsende Funktion. 



\end{MXContent}


\begin{MXContent}{Lineare Funktionen}{Linear}{STD}\MLabel{sec:linear}
\MDeclareSiteUXID{VBKM06_EinfacheFunktionen_Linear}
Ausgehend von der Identität, kann man sich nun komplexere Funktionen, die sogenannten \highlight{linearen Funktionen}, konstruieren. So kann man sich zum Beispiel überlegen, dass jede reelle Zahl ihrem doppelten Wert, oder ihrem $\pi$-fachen Wert, usw.~zugeordnet werden kann. Etwa
\[
 \function{f}{\R}{\R}{x}{2x}
\]
oder
\[
 \function{g}{\R}{\R}{x}{\pi x \MDFPeriod}
\]

%BILD: venn-lin1
\MTikzAuto{%
\begin{tikzpicture}
%venn.Definitionsbereich
\draw (0,0) ellipse (1cm and 1.5cm);
\draw (-0.5,1.8) node[auto] {$\mathbb{R}$};
% x0, x1, x2
\draw (0.0, 0.8) node(x1){$-1$};
\draw (0.6,0.2) node(xn) {$0$};
\draw (-0.3,-0.2)node(xh){$\frac{1}{2}$};
\draw (0.3,-0.9) node(x2){$2$} ;
\draw (-0.4,0.3) node {$\dots$};
%venn.Wertebereich
\draw (5,0) ellipse (1 cm and 1.5cm);
\draw (4.5,1.8) node[auto] {$\mathbb{R}$};
% f(x0), f(x1), f(x2)
\draw (5,1)  node(Fx1){$-2$};
\draw (5.6,0.2) node(Fxn) {$0$};
\draw (4.7,-0.4) node(Fxh) {$1$};
\draw (5,-1.2)  node(Fx2){$4$};
\draw (5.3,-0.6) node {$\dots$};
%Abbildung
\draw [|->] (x1) -- (Fx1);
\draw [|->] (x2) -- (Fx2);
\draw[|->] (xn) -- (Fxn);
\draw[|->] (xh) -- (Fxh);
\draw (2.5,1.3) node {$f$};
\end{tikzpicture}
}%
\hspace{2cm}
%GRAPH: lin1
\MTikzAuto{%
\begin{tikzpicture} 
%Koordinatensystem
% x-Achse
\node (xMAX) at (2.8,0){};
\draw[->,color=black] (-0.5,0) -- (xMAX);
\foreach \x in {1,2}
\draw[shift={(\x,0)},color=black] (0pt,2pt) -- (0pt,-2pt) node[below] {\footnotesize $\x$};
% y-Achse
\node (yMAX) at (0,4.8){};
\draw[->,color=black] (0,-0.5) -- (yMAX);
\foreach \y in {1, 2, 3, 4}
\draw[shift={(0,\y)},color=black] (2pt,0pt) -- (-2pt,0pt) node[left] {\footnotesize $\y$};
\draw[color=black] (0pt,-10pt) node[right] {\footnotesize $0$};
%Achsenbeschriftung
\draw (xMAX) node[anchor=north east] {$x$};
\draw (yMAX) node[anchor=east] {$f(x)$};
%Graph: x->2x
\draw[color=red] (-0.2,-0.4)--(2.2,4.4);
\draw[color=red] (2.2,4.4) node[anchor=south] {$G_f$};
\end{tikzpicture}
}%


%BILD: venn-lin2
\MTikzAuto{%
\begin{tikzpicture}
%venn.Definitionsbereich
\draw (0,0) ellipse (1cm and 1.5cm);
\draw (-0.5,1.8) node[auto] {$\mathbb{R}$};
% x0, x1, x2
\draw (0.0, 0.8) node(x1){$-1$};
\draw (0.6,0.2) node(xn) {$0$};
\draw (-0.3,-0.2)node(xh){$\frac{1}{2}$};
\draw (0.3,-0.9) node(x2){$2$} ;
\draw (-0.4,0.3) node {$\dots$};
%venn.Wertebereich
\draw (5,0) ellipse (1 cm and 1.5cm);
\draw (4.5,1.8) node[auto] {$\mathbb{R}$};
% f(x0), f(x1), f(x2)
\draw (5,1)  node(Fx1){$-\pi$};
\draw (5.6,0.2) node(Fxn) {$0$};
\draw (4.7,-0.4) node(Fxh) {$\frac{\pi}{2}$};
\draw (5,-1.2)  node(Fx2){$2\pi$};
\draw (5.3,-0.6) node {$\dots$};
%Abbildung
\draw [|->] (x1) -- (Fx1);
\draw [|->] (x2) -- (Fx2);
\draw[|->] (xn) -- (Fxn);
\draw[|->] (xh) -- (Fxh);
\draw (2.5,1.3) node {$g$};
\end{tikzpicture}
}%
\hspace{2cm}
%GRAPH: lin2
\MTikzAuto{%
\begin{tikzpicture} 
%Koordinatensystem
% x-Achse
\node (xMAX) at (2.8,0){};
\draw[->,color=black] (-0.5,0) -- (xMAX);
\foreach \x in {1,2}
\draw[shift={(\x,0)},color=black] (0pt,2pt) -- (0pt,-2pt) node[below] {\footnotesize $\x$};
% y-Achse
\node (yMAX) at (0,4.8){};
\draw[->,color=black] (0,-0.5) -- (yMAX);
\foreach \y in {1, 2, 3, 4}
\draw[shift={(0,\y)},color=black] (2pt,0pt) -- (-2pt,0pt) node[left] {\footnotesize $\y$};
\draw[color=black] (0pt,-10pt) node[right] {\footnotesize $0$};
%Achsenbeschriftung
\draw (xMAX) node[anchor=north east] {$x$};
\draw (yMAX) node[anchor=east] {$g(x)$};
%Graph: x->pi*x
\draw[color=red] (-0.1,-0.3)--(1.4,4.5);
\draw[color=red] (1.4,4.5) node[anchor=south] {$G_g$};
\end{tikzpicture}
}%

Alle linearen Funktionen (außer der Nullfunktion, s.u.) haben also als Wertebereich ebenfalls die gesamten reellen Zahlen ($W_f,W_g=\R$). Der Faktor, mit dem jede reelle Zahl in einer solchen linearen Funktion multipliziert wird, heißt \highlight{Steigung} der linearen Funktion. Oft möchte man auch bei linearen Funktionen nicht eine bestimmte Funktion mit spezifischer Steigung angeben, sondern irgendeine mit beliebiger Steigung $m\in\R$:
\[
 \function{f}{\R}{\R}{x}{m x \MDFPeriod}
\]

%GRAPH: lin3
\MTikzAuto{%
\begin{tikzpicture} 
%Koordinatensystem
% x-Achse
\node (xMAX) at (2.8,0){};
\draw[->,color=black] (-0.5,0) -- (xMAX);
\foreach \x in {1}
\draw[shift={(\x,0)},color=black] (0pt,2pt) -- (0pt,-2pt) node[below] {\footnotesize $\x$};
% y-Achse
\node (yMAX) at (0,1.8){};
\draw[->,color=black] (0,-0.5) -- (yMAX);
\draw[shift={(0,0.5)},color=black] (2pt,0pt) -- (-2pt,0pt) node[left] {$m$};
\draw[color=black] (0pt,-10pt) node[right] {\footnotesize $0$};
%Achsenbeschriftung
\draw (xMAX) node[anchor=north east] {$x$};
\draw (yMAX) node[anchor=east] {$f(x)$};
%Graph: x->m*x
\draw[color=red] (-0.4,-0.2)--(2.2,1.1);
\draw[color=red] (2.2,1.1) node[anchor=south] {$G_f$};
%Markierungen
\draw[color=red, dotted] (1,.5) -- (1,0); 
\draw[color=red, dotted] (1,.5) -- (0,.5);
\end{tikzpicture}
}%

Woher kommt der Begriff Steigung für eine lineare Funktion? Teilt man die Differenz, um welche der Graph in vertikaler Richtung anwächst, durch die entsprechende Längeneinheit in horizontaler Richtung, so erhält man die Steigung $m$. Also $m = \frac{f(x_2) - f(x_1)}{x_2 - x_1}$ für $x_1 < x_2$.%%%


\begin{MInfo}
Eine lineare Funktion
\[
 \function{f}{\R}{\R}{x}{m x} 
\]
ist genau dann streng monoton wachsend, wenn ihre Steigung positiv ist, also $m>0$ gilt; und sie ist genau dann streng monoton fallend, wenn ihre Steigung negativ ist, also $m<0$ gilt.
\end{MInfo}


\begin{MExercise}
Welche lineare Funktion ergibt sich für die Steigung $m=1$?
\begin{MHint}{\iSolution}
Es ergibt sich $f(x)=1\cdot x=x=\Id(x)$, also genau die Identität.
\end{MHint}
\end{MExercise}

\begin{MExercise}
Welche Funktion ergibt sich für die Steigung $m=0$?
\begin{MHint}{\iSolution}
Es ergibt sich $f(x)=0\cdot x=0$, also genau die konstante Funktion, die konstant $0$ ist.
\end{MHint}
\end{MExercise}

\end{MXContent}
 
\begin{MXContent}{Affin-lineare Funktionen}{Affin}{STD}
\MLabel{VBKM06_sec:affin-linear}
\MDeclareSiteUXID{VBKM06_EinfacheFunktionen_Affin}
Kombiniert man lineare Funktionen mit konstanten Funktionen, so erhält man die sogenannten \highlight{affin-lineare Funktionen}. Diese ergeben sich als die Summe einer linearen und einer konstanten Funktion. Im allgemeinen Fall, ohne konkret spezifizierte Steigung ($m\in\R$) und mit einer Konstanten ($c\in\R$) schreibt man das so:
\[
 \function{f}{\R}{\R}{x}{m x+c \MDFPeriod}
\]

%GRAPH: affin1
\MTikzAuto{%
\begin{tikzpicture} 
%Koordinatensystem
% x-Achse
\node (xMAX) at (3.8,0){};
\draw[->,color=black] (-0.5,0) -- (xMAX);
\foreach \x in {1,2}
\draw[shift={(\x,0)},color=black] (0pt,2pt) -- (0pt,-2pt) node[below] {\footnotesize $\x$};
% y-Achse
\node (yMAX) at (0,3.8){};
\draw[->,color=black] (0,-0.5) -- (yMAX);
\draw[shift={(0,1.5)},color=black] (2pt,0pt) -- (-2pt,0pt) node[left] {$c$};
\draw[shift={(0,2)},color=black] (2pt,0pt) -- (-2pt,0pt) node[left] {$c+m$};
\draw[color=black] (0pt,-10pt) node[right] {\footnotesize $0$};
%Achsenbeschriftung
\draw (xMAX) node[anchor=north east] {$x$};
\draw (yMAX) node[anchor=east] {$f(x)$};
%Graph: x->m*x+c
\draw[color=red] (-1,1)--(3,3);
\draw[color=red] (3,3) node[anchor=south] {$G_f$};
%Markierungen
\draw[color=red,dotted] (1,0) -- (1,2);
\draw[color=red,dotted] (0,2) -- (1,2);
\end{tikzpicture}	
}%

Die Graphen affin-linearer Funktionen werden auch als \highlight{Geraden} bezeichnet. Die Konstante $m$ wird für affin-lineare Funktionen weiterhin als Steigung bezeichnet, die Konstante $c\in\R$ als \highlight{Achsenabschnitt}. Der Grund für diese Bezeichnung ist folgender: Betrachtet man den Schnittpunkt des Graphen der affin-linearer Funktion mit der vertikalen Achse, so hat dieser vom Ursprung den Abstand $c$ (siehe Abbildung oben). So ergibt sich zum Beispiel für die unten abgebildete affin-linearer Funktion
\[
 \function{f}{\R}{\R}{x}{-2x-1}
\]

%GRAPH: affin2
\MTikzAuto{%
\begin{tikzpicture} 
%Koordinatensystem
% x-Achse
\node (xMAX) at (2.8,0){};
\draw[->,color=black] (-1.5,0) -- (xMAX);
\foreach \x in {-1,1,2}
\draw[shift={(\x,0)},color=black] (0pt,2pt) -- (0pt,-2pt) node[below] {\footnotesize $\x$};
% y-Achse
\node (yMAX) at (0,1.8){};
\draw[->,color=black] (0,-5) -- (yMAX);
\foreach \y in {-5,-4,-3,-2,-2,1}
\draw[shift={(0,\y)},color=black] (2pt,0pt) -- (-2pt,0pt) node[left] {\footnotesize $\y$};
\draw[color=black] (0pt,-10pt) node[right] {\footnotesize $0$};
\draw[shift={(0,-1)},color=red] (2pt,0pt) -- (-2pt,0pt) node[left] {\footnotesize $c=-1$};
%Achsenbeschriftung
\draw (xMAX) node[anchor=north east] {$x$};
\draw (yMAX) node[anchor=east] {$f(x)$};
%Graph: x-> -2x-1
\draw[color=red] (-1,1)--(2,-5);
\draw[color=red] (2,-5) node[anchor=north] {$G_f$};
\end{tikzpicture}
}%

die Steigung $m=-2$ und der Achsenabschnitt $c=-1$. Der Achsenabschnitt ergibt sich als Funktionswert bei $x=0$ und somit durch
\[
 c=f(0)=-2\cdot 0-1 = -1 \MDFPeriod
\]

\begin{MExercise}
Was sind die Steigung und der Achsenabschnitt von 
\[
 \function{f}{\R}{\R}{x}{\pi x-42 \MDFPSpace?}
\]
\begin{MHint}{\iSolution}
Steigung: $\pi$ und Achsenabschnitt: $-42$
\end{MHint}
\end{MExercise}


\begin{MExercise}
Welche Funktionen ergeben sich als affin-linearer Funktionen mit Steigung $m=0$ und welche mit Achsenabschnitt $c=0$ ?
\begin{MHint}{\iSolution}
Ist $m=0$, so gilt $f(x)=0\cdot x+c=c$. Damit ergeben sich die konstanten Funktionen als diejenigen mit Steigung $0$. Ein verschwindender Achsenabschnitt $c=0$ impliziert $f(x)=m x+0=m x$. Somit ergeben sich in diesem Fall genau die linearen Funktionen.
\end{MHint}
\end{MExercise}


\end{MXContent}
 
\begin{MXContent}{Betragsfunktionen}{Betrag}{STD}
\MLabel{VBKM06_sec:betrag}
\MDeclareSiteUXID{VBKM06_EinfacheFunktionen_Betrag}
In Modul \MNRef{VBKM02} wurde der \MSRef{VBKM02_Betrag}{Betrag} einer reellen Zahl $x$ auf folgende Art eingeführt:
\[
 |x| = \begin{MCaseEnv} x & \text{falls}\;x\geq 0 \\ -x & \text{falls}\;x<0 \MDFPeriod \end{MCaseEnv}
\]
Im Kontext dieses Moduls kann der Betrag nun als Funktion aufgefasst werden. Man erhält die \MEntry{Betragsfunktion}{Betragsfunktion}
\[
 \function{b}{\R}{\R}{x}{|x| \MDFPeriod}
\]

\begin{MExercise}
Was ist die Wertemenge $W_b$ der Betragsfunktion $b$?
\begin{MHint}{\iSolution}
Da für alle Zahlen $x$ aus $D_b=\R$ gilt $b(x)=|x|\geq0$, ergibt sich $W_b=[0\MIntvlSep \infty)$. 
\end{MHint}
\end{MExercise}

Durch die Fallunterscheidung
\[
 b(x)=|x|=\begin{MCaseEnv} x & \text{falls}\;x\geq0 \\ -x & \text{falls}\;x<0 \end{MCaseEnv}
\]
ist die Betragsfunktion ein Beispiel für eine \modsemph{abschnittsweise definierte} Funktion. Schreibt man Beträge mit Hilfe dieser Fallunterscheidung um, so spricht man auch vom \highlight{Auflösen} des Betrags. Der Graph der Betragsfunktion $b$ sieht dann so aus:

%GRAPH: abs1
\MTikzAuto{%
\begin{tikzpicture} 
%Koordinatensystem
% x-Achse
\node (xMAX) at (2.8,0){};
\draw[->,color=black] (-2.5,0) -- (xMAX);
\foreach \x in {-2,-1,1,2}
\draw[shift={(\x,0)},color=black] (0pt,2pt) -- (0pt,-2pt) node[below] {\footnotesize $\x$};
% y-Achse
\node (yMAX) at (0,2.8){};
\draw[->,color=black] (0,-0.5) -- (yMAX);
\foreach \y in {1,2}
\draw[shift={(0,\y)},color=black] (2pt,0pt) -- (-2pt,0pt) node[left] {\footnotesize $\y$};
\draw[color=black] (0pt,-10pt) node[right] {\footnotesize $0$};
%Achsenbeschriftung
\draw (xMAX) node[anchor=north east] {$x$};
\draw (yMAX) node[anchor=east] {$b(x)$};
%Graph: x-> |x|
\draw[color=red] (-2,2)--(0,0)--(2,2);
\draw[color=red] (2,2) node[anchor=south] {$G_b$};
\end{tikzpicture}
}%

Eine Eigenschaft des Graphen der Betragsfunktion, die auch bei den meisten allgemeineren Funktionen auftritt, in denen ein Betrag vorkommt, ist der "`Knick'' an der Stelle $x=0$. 
Die oben definierte Betragsfunktion $b$ ist nur der einfachste Fall einer Funktion, in der der Betrag vorkommt. Man kann sich kompliziertere Beispiele von Funktionen überlegen, in denen ein Betrag oder mehrere Beträge vorkommen, so etwa
\[
 \function{f}{\R}{\R}{x}{|2x-1| \MDFPeriod}
\]


Eine wichtige Aufgabe bei solchen gegebenen Funktionen ist, eine Vorstellung von deren Graphen zu bekommen. Dabei benutzt man die abschnittsweise Definition des Betrags und geht ähnlich vor wie beim Lösen von \MSRef{VBKM02_FallBetrag}{Betragsgleichungen} und -ungleichungen. Wir zeigen dies hier am Beispiel obiger Funktion $f$:

\begin{MExample}
Die Funktion
\[\function{f}{\R}{\R}{x}{|2x-1|}\]
ist gegeben. Wie sieht ihr Graph aus?

Wir berechnen:
\[
 f(x)=|2x-1|=\begin{MCaseEnv} 2x-1 & \text{falls}\ 2x-1\geq0 \\ -(2x-1) & \text{falls}\ 2x-1<0\end{MCaseEnv}=\begin{MCaseEnv} 2x-1 & \text{falls}\ x\geq\frac{1}{2} \\ -2x+1 & \text{falls}\ x<\frac{1}{2} \MDFPeriod\end{MCaseEnv}
\]
Somit erhalten wir eine abschnittsweise definierte Funktion, deren Graph eine steigende Gerade mit Steigung $2$ und Achsenabschnitt $-1$ im Bereich $x\geq\frac{1}{2}$ und eine fallende Gerade mit Steigung $-2$ und Achsenabschnitt $1$ im Bereich $x<\frac{1}{2}$ ist. Mit diesen Informationen können wir den Graphen von $f$ zeichnen:

%GRAPH: abs2
\MTikzAuto{%
\begin{tikzpicture} 
%Koordinatensystem
% x-Achse
\node (xMAX) at (2.8,0){};
\draw[->,color=black] (-1.5,0) -- (xMAX);
\foreach \x in {-1,1,2}
\draw[shift={(\x,0)},color=black] (0pt,2pt) -- (0pt,-2pt) node[below] {\footnotesize $\x$};
% y-Achse
\node (yMAX) at (0,4.8){};
\draw[->,color=black] (0,-0.5) -- (yMAX);
\foreach \y in {1,2, 3, 4}
\draw[shift={(0,\y)},color=black] (2pt,0pt) -- (-2pt,0pt) node[left] {\footnotesize $\y$};
\draw[color=black] (0pt,-10pt) node[right] {\footnotesize $0$};
%Achsenbeschriftung
\draw (xMAX) node[anchor=north east] {$x$};
\draw (yMAX) node[anchor=east] {$f(x)$};
%Graph: x-> |2x-1|
\draw[color=red] (-1.5,4)--(0.5,0)--(2.5,4);
\draw[color=red] (2.5,4) node[anchor=south] {$G_{f}$};
\end{tikzpicture}
}%
\end{MExample}

\begin{MInfo}
\textcolor{red}{WICHTIG!} Beim Auflösen von Beträgen wie in der Rechnung in obigem Beispiel sind folgende zwei wichtige Rechengesetze zu beachten:
\begin{enumerate}
 \item Die \modsemph{Bereiche} der Fallunterscheidung des Betrags ergeben sich als Ungleichungen für den \modsemph{gesamten} Ausdruck im Betrag, hier also $2x-1\geq 0$ und $2x-1<0$ und \modsemph{nicht} etwa nur $x\geq 0$ und $x<0$. Dies funktioniert beim Auflösen von Beträgen immer so. 
 \item Im Fall $<0$ erhält der \modsemph{gesamte} Ausdruck im Betrag ein Minuszeichen. Hier muss also auf das Setzen einer Klammer geachtet werden. Im obigen Beispiel ergibt sich deshalb $-(2x-1)=-2x+1$ und \modsemph{nicht} etwa $-2x-1$. Auch dies funktioniert beim Auflösen von Beträgen immer so. 
\end{enumerate}

\end{MInfo}

\begin{MExercise}
Wie sieht der Graph der Funktion
\[
 \function{\alpha}{\R}{\R}{x}{|-8x+1|-1}
\]
aus? Geben Sie außerdem $W_\alpha$ an.
\begin{MHint}{\iSolution}
Es gilt
\[
 \alpha(x)=|-8x+1|-1=\begin{MCaseEnv} -8x+1-1 & \text{falls}\ -8x+1\geq0 \\ -(-8x+1)-1 & \text{falls}\ -8x+1<0 \end{MCaseEnv}=\begin{MCaseEnv} -8x & \text{falls}\ x\leq\frac{1}{8} \\ 8x-2 & \text{falls}\ x>\frac{1}{8} \MDFPSpace,\end{MCaseEnv}
\]
folglich: 

%GRAPH: abs3
\MTikzAuto{%
\begin{tikzpicture}[x = 2cm] 
%Koordinatensystem
% x-Achse
\node (xMAX) at (1.8,0){};
\draw[->,color=black] (-1.5,0) -- (xMAX);
\foreach \x in {-1,1}
\draw[shift={(\x,0)},color=black] (0pt,2pt) -- (0pt,-2pt) node[below] {\footnotesize $\x$};
% y-Achse
\node (yMAX) at (0,4.5){};
\draw[->,color=black] (0,-1.5) -- (yMAX);
\foreach \y in {-1,1,2, 3, 4}
\draw[shift={(0,\y)},color=black] (2pt,0pt) -- (-2pt,0pt) node[left] {\footnotesize $\y$};
\draw[color=black] (0pt,-10pt) node[left] {\footnotesize $0$};
%Achsenbeschriftung
\draw (xMAX) node[anchor=north east] {$x$};
\draw (yMAX) node[anchor=east] {$\alpha(x)$};
%Graph: x-> |-8x+1|-1
\draw[color=red] (-0.5,4)--(0.125,-1)--(0.75,4);
\draw[color=red] (0.75,4) node[anchor=south] {$G_{\alpha}$};
\draw[shift={(1/8,0)},color=black] (0pt,2pt) -- (0pt,-2pt) node[anchor=north] {\footnotesize $\frac{1}{8}$};
\end{tikzpicture}
}%

Da $|-8x+1|\geq0$ gilt, folgt $|-8x+1|-1\geq -1$ und damit ist $W_\alpha=[-1\MIntvlSep \infty)$.
\end{MHint}
\end{MExercise}


\end{MXContent}


\begin{MXContent}{Monome}{Monome}{STD}\MLabel{sec:monome}
\MDeclareSiteUXID{VBKM06_EinfacheFunktionen_Monome}
Neben den affin-linearer Funktionen aus dem vorigen Abschnitt kann man sich nun auch Funktionen überlegen, die allen reellen Zahlen natürliche Potenzen ihrer selbst zuordnen. So zum Beispiel
\[
 \function{g}{\R}{\R}{x}{x^2 \MDFPeriod}
\]

% venn-para (siehe venn4)
\MTikzAuto{%
\begin{tikzpicture}[scale = 0.5]
% Definitionsbereich
\draw  (3.5,10.3) node[anchor=north west] {$\mathbb{R}$};
\draw [] (2.7,7.6) ellipse (1.7cm and 2.4cm);
\draw [color = blue](2,6.3)  node[anchor=north west](vierMinus) {-4};
\draw [color = blue](2.1,7.4)  node[anchor=north west](eins) {1};
\draw [color = blue](2.3,8.7)  node[anchor=north west] (vier){4};
\draw [color = blue](2.5,9.5) node[anchor=north west]  {...};
% Wertebereich
\draw (9.8,10.3) node[anchor=north west] {$\mathbb{R}$};
\draw (8.8,7.5) ellipse (1.9cm and 2.6cm);
\draw (8,8) node[anchor=north west] {$\frac{1}{2}$};
\draw (7.4,6.7) node[anchor=north west] {$-\frac{1}{2}$};
\draw (7.1,7.5) node[anchor=north west](Feins) {$1$};
\draw (8.0,9.8) node[anchor=north west] (Fvier){$16$};
\draw (7.3,8.7) node[anchor=north west]{$0$};
\draw (9.0,7.8) node[anchor=north west] {$\pi$};
\draw (9.3,9.3) node[anchor=north west] {$\frac{3}{2}$};
\draw (9.3,6.5) node[anchor=north west] {...};
% Abbildung: 
\draw[|->] (eins) -- (Feins);
\draw[|->] (vier) -- (Fvier);
\draw[|->] (vierMinus) -- (Fvier);
\end{tikzpicture}
}%
\hspace{1cm}
%GRAPH: mon1
\MTikzAuto{%
\begin{tikzpicture} 
%Koordinatensystem
\node (xMAX) at (2.8,0){};
\node (yMAX) at (0,4.8){};
\draw[->,color=black] (-2.5,0) -- (xMAX);
\foreach \x in {-2,-1,1,2}
\draw[shift={(\x,0)},color=black] (0pt,2pt) -- (0pt,-2pt) node[below] {\footnotesize $\x$};
\draw[->,color=black] (0,-0.5) -- (yMAX);
\foreach \y in {1,2,3,4}
\draw[shift={(0,\y)},color=black] (2pt,0pt) -- (-2pt,0pt) node[left] {\footnotesize $\y$};
\draw[color=black] (0pt,-10pt) node[right] {\footnotesize $0$};
%Achsenbeschriftung
\draw (xMAX) node[anchor=north east] {$x$};
\draw (yMAX) node[anchor=east] {$g(x)$};
% Graph (x,x^2)
\draw[color=red,smooth,samples=50,domain=-2.1:2.1] plot(\x,{(\x)^(2.0)});
\draw[color=red] (2.1,4.41) node[anchor=south] {$G_g$};
\end{tikzpicture}
}%

Dies funktioniert für jeden natürlichen Exponenten und man schreibt dann allgemein
\[
 \function{f}{\R}{\R}{x}{x^n}
\]
mit $n\in\N_0$ und bezeichnet diese Funktionen als \highlight{Monome}. Der Exponent $n$ eines Monoms wird als \highlight{Grad} des Monoms bezeichnet. So ist etwa die Funktion $g$ vom Anfang dieses Abschnitts ein Monom vom Grad $2$, usw.

\begin{MExercise}
Welche Funktion ergibt sich als Monom vom Grad $1$ bzw.~vom Grad $0$ ?
\begin{MHint}{\iSolution}
Da $x^1=x$ für alle $x\in\R$ gilt, ergibt sich die Identität $\Id$ als Monom vom Grad $1$. Genauso gilt $x^0=1$ für alle $x\in\R$ und damit ist die konstante Funktion $f\colon\R\lto\R$, $f(x)=1$ das Monom vom Grad $0$.
\end{MHint}
\end{MExercise}

Man bezeichnet das Monom vom Grad $2$ auch als die \highlight{Standardparabel}. Das Monom vom Grad $3$ wird auch als \highlight{kubische Standardparabel} bezeichnet. Hier einige Graphen von Monomen:



%GRAPH: mon2
\MTikzAuto{%
\begin{tikzpicture} 
%Koordinatensystem
\node (xMAX) at (2.8,0){};
\node (yMAX) at (0,4.8){};
\draw[->,color=black] (-2.5,0) -- (xMAX);
\foreach \x in {-2,-1,1,2}
\draw[shift={(\x,0)},color=black] (0pt,2pt) -- (0pt,-2pt) node[below] {\footnotesize $\x$};
\draw[->,color=black] (0,-3.5) -- (yMAX);
\foreach \y in {-3,-2,-1,1,2,3,4}
\draw[shift={(0,\y)},color=black] (2pt,0pt) -- (-2pt,0pt) node[left] {\footnotesize $\y$};
\draw[color=black] (0pt,-10pt) node[right] {\footnotesize $0$};
%Achsenbeschriftung
\draw (xMAX) node[anchor=north east] {$x$};
\draw (yMAX) node[anchor=east] {$f(x)$};
\clip(-2.5,-3.5) rectangle (2.5,4.5);
% Monome
\draw[color = cyan,smooth,samples=50,domain=-2.5:2.5] plot(\x,{(\x)}); \draw[color = cyan](-1.4,-1.4)node[left]{$G_x$};
\draw[color = orange, smooth,samples=50,domain=-2.4:2.4] plot(\x,{(\x)^(2.0)}); \draw[color = orange](-1.5,2)node[left]{$G_{x^2}$};
\draw[color = blue, smooth,samples=50,domain=-1.9:1.9] plot(\x,{(\x)^(3.0)}); \draw[color = blue](-1.3,-3)node[left]{$G_{x^3}$};
\draw[color = red, smooth,samples=50,domain=-1.5:1.5] plot(\x,{(\x)^(4.0)}); \draw[color = red](-1.3,3)node[right]{$G_{x^4}$};
\draw[color = black, smooth,samples=50,domain=-1.4:1.4] plot(\x,{(\x)^(5.0)}); \draw[color = black](-1.2,-2.5)node[right]{$G_{x^5}$};
%Punkte
\draw[color=black,fill=black] (1,1) circle (1pt);
\draw[color=black,fill=black] (-1,1) circle (1pt);
\draw[color=black,fill=black] (-1,-1) circle (1pt);
\end{tikzpicture}
}%

Auf Basis dieser Graphen fassen wir nun einige Erkenntisse über Monome zusammen: Es gibt einen grundlegenden Unterschied zwischen Monomen (mit Abbildungsvorschrift $f(x)=x^n$, $n\in\N$) von geradem und von ungeradem Grad. Die Monome von geradem Grad größer Null haben als Wertebereich immer die Menge $[0\MIntvlSep \infty)$, während die Monome von ungeradem Grad ganz $\R$ als Wertebereich besitzen. Weiterhin gilt stets
\[
 f(1)=1^n=1 \MDFPSpace,
\]
\[
 f(0)=0^n=0
\]
und
\[
 f(-1)=\MCases{1 & \text{falls}\ n\ \text{gerade} \\ -1 & \text{falls}\ n\ \text{ungerade \MDFPeriod} }
\]
Ferner gilt
\[
 \MCases{x>x^2>x^3>x^4>\dots & \text{für}\ x\in(0\MIntvlSep 1) \\ x<x^2<x^3<x^4<\dots & \text{für}\ x\in(1\MIntvlSep \infty) \MDFPeriod}
\]

\begin{MExercise}
Wie ergeben sich diese Erkenntnisse über Monome unmittelbar aus den Potenzrechengesetzen?
\begin{MHint}{\iSolution}
Nach den Potenzrechengesetzen gilt stets $1^n=1$ und $0^n=0$ für beliebige natürliche Zahlen $n$ und $(-1)^n=1$ für gerade $n$ sowie $(-1)^n=-1$ für ungerade $n$. Dadurch ergeben sich die beschriebenen Punkte, die alle Graphen der Monome gemeinsam haben. Auch die Ungleichungen $x>x^2>x^3>x^4>\dots$ für positive $x$ kleiner $1$ und $x<x^2<x^3<x^4<\dots$ für $x$ größer $1$ folgen aus den Potenzrechengesetzen, da höher werdende Potenzen für positive Zahlen kleiner $1$ stets kleiner werdende Ergebnisse liefern und für Zahlen größer $1$ umgekehrt stets größer werdende Ergebnisse.
\end{MHint}
\end{MExercise}

\end{MXContent}

\begin{MXContent}{Polynome und ihre Nullstellen}{Nullstellen}{STD}\MLabel{sec:polynome}
\MDeclareSiteUXID{VBKM06_EinfacheFunktionen_Polynome}
Während in den bis jetzt betrachteten Monomen immer nur genau eine Potenz der Veränderlichen in der Abbildungsvorschrift vorkommt,
lassen sich aus diesen Monomen problemlos komplexere Funktionen konstruieren in denen mehrere verschiedene Potenzen der Veränderlichen vorkommen.
Diese ergeben sich als Summen von Vielfachen von Monomen. Man spricht dann von sogenannten \highlight{Polynomen}; hier einige Beispiele und deren Graphen:
\[
 \begin{array}{ll}
 \function{f_1}{\R}{\R}{x}{2x^3+4x^2-3x+42} & (\text{Grad:}\;3) \\[3ex]

 \function{f_2}{\R}{\R}{x}{-x^{101}+3x-14} & (\text{Grad:}\;101) \\[3ex]

 \function{f_3}{\R}{\R}{x}{9x^4+9x^3-2x^2-19x} & (\text{Grad:}\;4) \\[3ex]

 \function{f_4}{\R}{\R}{x}{x^2+2x+2} & (\text{Grad:}\;2) \\[3ex]

 \function{f_5}{\R}{\R}{x}{8x-2} & (\text{Grad:}\;1) \\[3ex]

 \function{f_6}{\R}{\R}{x}{13} & (\text{Grad:}\;0) \\
 \end{array}
\]

%GRAPHEN: poly1
\MTikzAuto{%
\begin{tikzpicture}[y=0.1cm] 
%Koordinatensystem
\node (xMAX) at (2.5,0){};
\node (yMAX) at (0,60){};
\draw[->,color=black] (-5,0) -- (xMAX);
\foreach \x in {-5, -4, -3, -2, -1,1, 2}
\draw[shift={(\x,0)},color=black] (0pt,2pt) -- (0pt,-2pt) node[below] {\footnotesize $\x$};
\draw[->,color=black] (0,-15) -- (yMAX);
\foreach \y in {-10,10,20,30,40,50}
\draw[shift={(0,\y)},color=black] (2pt,0pt) -- (-2pt,0pt) node[left] {\footnotesize $\y$};
\draw[color=black] (0pt,-10pt) node[right] {\footnotesize $0$};
%Achsenbeschriftung
\draw (xMAX) node[anchor=north east] {$x$};
\draw (yMAX) node[anchor=east] {$f_1(x)$};
%Beschriftung
%\draw(0,38)node[right]{$f_1(x)= 2x^3+ 4x^2 - 3x+ 42$};
% Polynom-Plot: 2*x^3+ 4*x^2 - 3*x+ 42
\clip(-5,-10) rectangle (5,55);
\draw[color=red,smooth,samples=50,domain=-4.5:2] plot(\x,{2*(\x)^3 + 4*(\x)^2 - 3*(\x)+ 42}); 
\draw[color=red] (-3.5,20) node[anchor=east] {$G_{f_1}$};
\end{tikzpicture}
}%
\hspace{1cm}
%GRAPHEN: poly2
\MTikzAuto{%
\begin{tikzpicture}[y=0.4cm] 
% Koordinatensystem
\node (xMAX) at (2.5,0){};
\node (yMAX) at (0,4){};
\draw[->,color=black] (-2,0) -- (xMAX);
\foreach \x in {-1,1,2}
\draw[shift={(\x,0)},color=black] (0pt,2pt) -- (0pt,-2pt) node[below] {\footnotesize $\x$};
\draw[->,color=black] (0,-20) -- (yMAX);
\foreach \y in {-18,-16,-14,-12,-10,-8,-6,-4,-2,2}
\draw[shift={(0,\y)},color=black] (2pt,0pt) -- (-2pt,0pt) node[left] {\footnotesize $\y$};
\draw[color=black] (0pt,-10pt) node[right] {\footnotesize $0$};
% Achsenbeschriftung
\draw (xMAX) node[anchor=north east] {$x$};
\draw (yMAX) node[anchor=east] {$f_2(x)$};
% Beschriftung
%\draw(0,-7)node[right]{$f_2(x)= -x^{101}+ 3x-14$};
% Polynom-Plot: -x^101+ 3x- 14
\clip(-2,-20) rectangle (2,3);
\draw[color=red,smooth,samples=100,domain=-1.05:1.05] plot(\x,{-(\x)^(101) + 3*(\x)-14}); 
\draw[color=red] (1,-12) node[anchor=west] {$G_{f_2}$};
\end{tikzpicture}
}%

%GRAPHEN: poly3
\MTikzAuto{%
\begin{tikzpicture}[y=0.4cm] 
% Koordinatensystem
% x-koordinate
\node (xMAX) at (2.8,0){};
\draw[->,color=black] (-2,0) -- (xMAX);
\foreach \x in {-1,1,2}
\draw[shift={(\x,0)},color=black] (0pt,2pt) -- (0pt,-2pt) node[below] {\footnotesize $\x$};
% y- koordinate
\node (yMAX) at (0,10){};
\draw[->,color=black] (0,-10) -- (yMAX);
\foreach \y in {-8,-6,-4,-2,2,4,6,8}
\draw[shift={(0,\y)},color=black] (2pt,0pt) -- (-2pt,0pt) node[left] {\footnotesize $\y$};
\draw[color=black] (0pt,-10pt) node[right] {\footnotesize $0$};
% Achsenbeschriftung
\draw (xMAX) node[anchor=north east] {$x$};
\draw (yMAX) node[anchor=east] {$f_3(x)$};
% Polynom-Plot:
\clip(-1,-10) rectangle (2,9);
\draw[color=red, smooth,samples=100,domain=-0.6:1.3] plot(\x,{9.0*(\x)^(4.0)+9.0*(\x)^(3.0)-2.0*(\x)^(2.0)-19.0*(\x)});
\draw[color=red] (1,-4) node[anchor=west] {$G_{f_3}$};
\end{tikzpicture}
}%
\hspace{1cm}
%GRAPHEN: poly4
\MTikzAuto{%
\begin{tikzpicture} 
% Koordinatensystem
% x-koordinate
\node (xMAX) at (2.8,0){};
\draw[->,color=black] (-5,0) -- (xMAX);
\foreach \x in {-4,-3,-2,-1,1,2}
\draw[shift={(\x,0)},color=black] (0pt,2pt) -- (0pt,-2pt) node[below] {\footnotesize $\x$};
% y- koordinate
\node (yMAX) at (0,9){};
\draw[->,color=black] (0,-1) -- (yMAX);
\foreach \y in {1, 2,3, 4,5, 6,7,8}
\draw[shift={(0,\y)},color=black] (2pt,0pt) -- (-2pt,0pt) node[left] {\footnotesize $\y$};
\draw[color=black] (0pt,-10pt) node[right] {\footnotesize $0$};
% Achsenbeschriftung
\draw (xMAX) node[anchor=north east] {$x$};
\draw (yMAX) node[anchor=east] {$f_4(x)$};
% Polynom-Plot: 
\clip(-5,0) rectangle (3,8.5);
\draw[color=red, smooth,samples=100,domain=-5:3]plot(\x,{(\x)^(2.0)+2.0*(\x)+2.0});
\draw[color=red] (1.5,6) node[anchor=west] {$G_{f_4}$};
\end{tikzpicture}
}%

%GRAPHEN: poly5
\MTikzAuto{%
\begin{tikzpicture} 
% Koordinatensystem
% x-koordinate
\node (xMAX) at (2.8,0){};
\draw[->,color=black] (-1.5,0) -- (xMAX);
\foreach \x in {-1,1,2}
\draw[shift={(\x,0)},color=black] (0pt,2pt) -- (0pt,-2pt) node[below] {\footnotesize $\x$};
% y- koordinate
\node (yMAX) at (0,6){};
\draw[->,color=black] (0,-4) -- (yMAX);
\foreach \y in {-3,-2,-1,1, 2,3, 4,5}
\draw[shift={(0,\y)},color=black] (2pt,0pt) -- (-2pt,0pt) node[left] {\footnotesize $\y$};
\draw[color=black] (0pt,-10pt) node[left] {\footnotesize $0$};
% Achsenbeschriftung
\draw (xMAX) node[anchor=north east] {$x$};
\draw (yMAX) node[anchor=east] {$f_5(x)$};
% Polynom-Plot: 
\clip(-2,-4) rectangle (2.8,5.8);
\draw[color=red, smooth,samples=100,domain=-5:3]plot(\x,{8*(\x)-2});
\draw[color=red] (1,5) node[anchor=west] {$G_{f_5}$};
\end{tikzpicture}
}%
\hspace{1cm}
%GRAPHEN: poly6
\MTikzAuto{%
\begin{tikzpicture}[y = 0.2cm]
% Koordinatensystem
% x-koordinate
\node (xMAX) at (2.8,0){};
\draw[->,color=black] (-1.5,0) -- (xMAX);
\foreach \x in {-1,1,2}
\draw[shift={(\x,0)},color=black] (0pt,2pt) -- (0pt,-2pt) node[below] {\footnotesize $\x$};
% y- koordinate
\node (yMAX) at (0,20){};
\draw[->,color=black] (0,-2) -- (yMAX);
\foreach \y in {10}
\draw[shift={(0,\y)},color=black] (2pt,0pt) -- (-2pt,0pt) node[left] {\footnotesize $\y$};
\draw[color=black] (0pt,-10pt) node[right] {\footnotesize $0$};
% Achsenbeschriftung
\draw (xMAX) node[anchor=north east] {$x$};
\draw (yMAX) node[anchor=east] {$f_6(x)$};
% Polynom-Plot: 
\draw[color=red] (-1.5,13)--(2.5, 13);
\draw[color=red] (2.5,13) node[anchor=south] {$G_{f_6}$};
\end{tikzpicture}
}%

Der Grad eines Polynoms richtet sich also nach dem vorkommenden Monom mit dem höchsten Grad. Außerdem erkennen wir, dass die bisher behandelten Funktionstypen der konstanten, linearen und affin-lineareren Funktionen -- genauso wie die Monome -- auf natürliche Weise wieder als Spezialfälle der Polynome auftauchen. Die Polynome umfassen also alle bisher betrachteten Funktionstypen. 

Möchte man allgemein ein unspezifiziertes Polynom vom Grad $n\in\N$ angeben, so schreibt man dies folgendermaßen:
\[
 \function{f}{\R}{\R}{x}{a_nx^n+a_{n-1}x^{n-1}+a_{n-2}x^{n-2}\dots +a_{2}x^{2}+a_1x+a_0 \MDFPeriod}
\]
Dabei sind $a_0,a_1,\dots,a_n$ mit $a_n\neq0$ die reellen Vorfaktoren vor den einzelnen Monomen, die als \highlight{Koeffizienten} des Polynoms bezeichnet werden. 

\begin{MExercise}
Wie lautet das Polynom $f(x)$ mit den Koeffizienten $a_0=-4$, $a_2=\pi$ und $a_4=9$ und welchen Wertebereich besitzt es? 
\ \\ \ \\
Das Polynom lautet \MEquationItem{$f(x)$}{\MLFunctionQuestion{25}{9*x^4+pi*x^2-4}{5}{x}{5}{ELFP1}},\\
sein Wertebereich ist \MEquationItem{$W_f$}{\MLIntervalQuestion{15}{[-4,infty)}{4}{ELFP1b}}.\\
\MInputHint{Die Kreiszahl $\pi$ kann als \texttt{pi} und der Wertebereich als Intervall eingegeben werden.}

\begin{MHint}{\iSolution}
Das Polynom lautet
\[
 \function{f}{\R}{\R}{x}{9x^4+\pi x^2-4}
\]
und besitzt den Wertebereich $W_f=[-4\MIntvlSep \infty)$, da die geraden Potenzen von $x$ nur nichtnegative Werte annehmen können.
\end{MHint}
\end{MExercise}

Bei allgemeinen Polynomen sind insbesondere die Nullstellen von Interesse. Diese findet man durch das Lösen
von Gleichungen $n$-ten Grades. Für den Grad $n=2$, also für Polynome vom Grad $2$ (diese werden auch als allgemeine Parabeln bezeichnet),
ist dies durch das Lösen einer quadratischen Gleichung möglich. In Modul \MNRef{VBKM02} werden die relevanten Techniken
der \MSRef{VBKM02_QuadratischErgaenzung}{quadratischen Ergänzung}, der \MSRef{VBKM02_pqFormel}{$pq$-Formel} und der \MSRef{VBKM02_Scheitelpunktform}{Scheitelpunktsform}
quadratischer Ausdrücke genauer erklärt. 

\begin{MExample}
Gegeben ist die Parabel
\[
 \function{\zeta}{\R}{\R}{y}{2y^2-8y+6 \MDFPeriod}
\]
Wir bestimmen Nullstellen und Scheitelpunkt und zeichnen den Graphen.

Wir führen an der Abbildungsvorschrift $\zeta(y)=2(y^2-4y+3)$ eine quadratische Ergänzung durch:
\[
 y^2-4y+3=y^2-4y+4-1=(y-2)^2-1 \MDFPeriod
\]
Folglich lässt sich die Abbildungsvorschrift als
\[
 \zeta(y)=2(y-2)^2-2
\]
schreiben. Wir erkennen, dass die Parabel gegenüber der Standardparabel um $2$ Längeneinheiten nach rechts und nach unten verschoben ist.  Der Scheitelpunkt lässt sich bei $(2|-2)$ ablesen. Die Nullstellen lassen sich berechnen:
\[
	\zeta(y)=2((y-2)^2-1)=0\quad\Leftrightarrow\quad (y-2)^2=1\quad\Leftrightarrow\quad y_{1,2}-2=\MCases{1\\-1}\quad\Leftrightarrow\quad y_{1,2}=\MCases{3\\1}\MDFPeriod
\]
Der Graph ergibt sich schließlich zu:

%GRAPH: bsp-para
\MTikzAuto{%
\begin{tikzpicture} 
% Koordinatensystem
% x-koordinate
\node (xMAX) at (4.8,0){};
\draw[->,color=black] (-1.5,0) -- (xMAX);
\foreach \x in {-1,1,2,3,4}
\draw[shift={(\x,0)},color=black] (0pt,2pt) -- (0pt,-2pt) node[below] {\footnotesize $\x$};
% y- koordinate
\node (yMAX) at (0,6){};
\draw[->,color=black] (0,-2.5) -- (yMAX);
\foreach \y in {-2,-1,1, 2,3, 4,5}
\draw[shift={(0,\y)},color=black] (2pt,0pt) -- (-2pt,0pt) node[left] {\footnotesize $\y$};
\draw[color=black] (0pt,-10pt) node[left] {\footnotesize $0$};
% Achsenbeschriftung
\draw (xMAX) node[anchor=north east] {$y$};
\draw (yMAX) node[anchor=east] {$\zeta(y)$};
% Para-Plot: 
%\clip(-2,-4) rectangle (2.8,5.8);
\draw[color=red, smooth,samples=100,domain=0:4]plot(\x,{2*\x^2-8*\x+6});
\draw[color=red] (3.8,4) node[anchor=west] {$G_{\zeta}$};
\draw[color=orange, fill=orange] (2,-2) circle (1.5pt);
\draw[color=blue, fill=blue] (3,0) circle (1.5pt);
\draw[color=blue, fill=blue] (1,0) circle (1.5pt);
\draw[color=orange] (2,-2) node[anchor=north] {\footnotesize Scheitelpunkt};
\draw[color=blue] (1.1,0) node[anchor=south west] {\footnotesize Nullstellen};
\end{tikzpicture}
}%

\end{MExample}

\end{MXContent}


\begin{MXContent}{Hyperbeln}{Hyperbeln}{STD}\MLabel{sec:hyperbel}
\MDeclareSiteUXID{VBKM06_EinfacheFunktionen_Hyperbeln}

Wir betrachten Funktionen, die als Abbildungsvorschrift einen \highlight{reziproken Zusammenhang} besitzen. Darunter versteht man das Vorkommen von \highlight{Kehrwerten} in der Abbildungsvorschrift. Zu beachten ist bei der Bestimmung des \modsemph{größtmöglichen Definitionsbereichs} solcher Funktionen, dass der Nenner nicht $0$ werden darf.  

Beispiele reziproker Funktionen sind im Folgenden zusammengestellt; diese ergeben sich als Kehrwerte der Monome und werden auch als Funktionen vom \highlight{hyperbolischen Typ} bezeichnet:
\[
 \function{f_1}{\R\setminus\{0\}}{\R}{x}{\frac{1}{x} \MDFPSpace,}
\]
\[
 \function{f_2}{\R\setminus\{0\}}{\R}{x}{\frac{1}{x^2} \MDFPSpace,}
\]
\[
 \function{f_3}{\R\setminus\{0\}}{\R}{x}{\frac{1}{x^3} \MDFPSpace,}
\]
usw. Ihre Graphen sehen so aus:

%GRAPHEN: hyp2
\MTikzAuto{%
\begin{tikzpicture}
% Koordinatensystem
% x-koordinate
\node (xMAX) at (4.8,0){};
\draw[->,color=black] (-3.5,0) -- (xMAX);
\foreach \x in {-3,-2,-1,1,2,3,4}
\draw[shift={(\x,0)},color=black] (0pt,2pt) -- (0pt,-2pt) node[below] {\footnotesize $\x$};
% y- koordinate
\node (yMAX) at (0,4.8){};
\draw[->,color=black] (0,-3.5) -- (yMAX);
\foreach \y in {-3,-2,-1,1,2,3,4}
\draw[shift={(0,\y)},color=black] (2pt,0pt) -- (-2pt,0pt) node[left] {\footnotesize $\y$};
\draw[color=black] (0pt,-10pt) node[right] {\footnotesize $0$};
% Achsenbeschriftung
\draw (xMAX) node[anchor=north east] {$x$};
\draw (yMAX) node[anchor=east] {$f_1(x),f_2(x),f_3(x)$};
% Graph: 
\clip(-3.5,-3.5) rectangle (4.5,4.5);
%1/x
\draw[color = cyan, smooth,samples=100,domain=0.02:4.5]plot(\x,{1/(\x)});
\draw[color = cyan, smooth,samples=100,domain=-4.5:-0.02]plot(\x,{1/(\x)});
\draw[color = cyan](-2,-0.5) node[below]{$G_{f_1}$};
%1/x^2
\draw[color = orange, smooth,samples=100,domain=0.3:4.5]plot(\x,{1/(\x)^2});
\draw[color = orange, smooth,samples=100,domain=-4.5:-0.3]plot(\x,{1/(\x)^2});
\draw[color = orange](-1,1) node[left]{$G_{f_2}$};
%1/x^3
\draw[color = blue, smooth,samples=100,domain=0.6:4.5]plot(\x,{1/(\x)^3});
\draw[color = blue, smooth,samples=100,domain=-4.5:-0.6]plot(\x,{1/(\x)^3});
\draw[color = blue](-0.9,-1.5) node[left]{$G_{f_3}$};
\end{tikzpicture}
}%

Insbesondere der Graph der Funktion
\[
 \function{f_1}{\R\setminus\{0\}}{\R}{x}{\frac{1}{x} \MDFPSpace,}
\]
wird als \highlight{Hyperbel} bezeichnet.

Allgemein kann man für den Kehrwert eines beliebigen Monoms vom Grad $n\in\N$ also eine entsprechende Funktion hyperbolischen Typs angeben:

\[
 \function{f_n}{\R\setminus\{0\}}{\R}{x}{\frac{1}{x^n} \MDFPeriod}
\]
\begin{MExercise}
Wie lautet die Wertemenge $W_{f_n}$ der Funktion $f_n$ für gerade $n\in\N$ bzw.~für ungerade $n\in\N$ ?
\begin{MHint}{\iSolution}
Stets gilt $\frac{1}{x^n}\neq0$, da ein Quotient nur Null werden kann, wenn der Zähler Null wird. Folglich kommt $0\in\R$ nie in der Wertemenge vor. Da stets $x^n\geq0$ für gerade $n\in\N$ gilt, folgt $\frac{1}{x^n}>0$ für gerade $n\in\N$. Für ungerade $n\in\N$ kann allerdings auch $\frac{1}{x^n}<0$ sein. Es ergibt sich
\[
 W_{f_n}=\MCases{\R\setminus\{0\} & \text{falls}\ n\ \text{ungerade} \\ (0\MIntvlSep \infty) & \text{falls}\ n\ \text{gerade} \MDFPeriod}
\]
Dies ist auch aus den Graphen der Funktionen vom hyperbolischen Typ ersichtlich.
\end{MHint}
\end{MExercise}



Weitere Beispiele für Funktionen vom hyperbolischen Typ haben wir bereits in Beispiel \MNRef{bsp:anwendungen} und Aufgabe \MNRef{ex:anwendungen} in Abschnitt \MNRef{sec:anwendungen} betrachtet. 


\end{MXContent}

\begin{MXContent}{Gebrochenrationale Funktionen}{Gebrochenrational}{STD}\MLabel{sec:gebrochen}
\MDeclareSiteUXID{VBKM06_EinfacheFunktionen_GebrochenRational}
Allgemeine gebrochenrationale Funktionen besitzen Abbildungsvorschriften, die aus dem Quotienten zweier Polynome bestehen. Hier einige Beispiele mit ihren Graphen. Natürlich müssen auch bei diesen Funktionen diejenigen Zahlen aus dem Definitionsbereich ausgeschlossen werden, für die der Nenner in der Abbildungsvorschrift gleich Null wird.

\begin{MExample}\MLabel{einf_bsp_gebr_rat}
\[
 \function{f}{\R}{\R}{x}{\frac{8}{x^2+1} \MDFPSpace,}
\]
\[
 \function{g}{\R\setminus\{-4,\frac{3}{2}\}}{\R}{x}{\frac{-18x+3}{2x^2+5x-12} \MDFPSpace,}
\]
\[
 \function{h}{\R\setminus\{-1\}}{\R}{x}{\frac{x^3-x^2+x}{x+1} \MDFPeriod}
\] 

%GRAPHEN: rat1
\MTikzAuto{%
\begin{tikzpicture}[scale = 0.5]
% Koordinatensystem
% x-koordinate
\node (xMAX) at (10,0){};
\draw[->,color=black] (-9,0) -- (xMAX);
\foreach \x in {-8,-6,-4,-2,2,4,6,8}
\draw[shift={(\x,0)},color=black] (0pt,2pt) -- (0pt,-2pt) node[below] {\footnotesize $\x$};
% y- koordinate
\node (yMAX) at (0,9){};
\draw[->,color=black] (0,-2) -- (yMAX);
\foreach \y in {2,4,6,8}
\draw[shift={(0,\y)},color=black] (2pt,0pt) -- (-2pt,0pt) node[left] {\footnotesize $\y$};
\draw[color=black] (0pt,-10pt) node[right] {\footnotesize $0$};
% Achsenbeschriftung
\draw (xMAX) node[anchor=north east] {$x$};
\draw (yMAX) node[anchor=east] {$f(x)$};
% Graph: 
\clip(-9,-1) rectangle (9.5,8.5);
\draw[color=red,smooth,samples=100,domain=-9:9]plot(\x,{8/((\x)^2+1)}) node[above] {$G_f$};
\end{tikzpicture}
}%

%GRAPHEN: rat2
\MTikzAuto{%
\begin{tikzpicture}[scale = 0.5]
% Koordinatensystem
% x-koordinate
\node (xMAX) at (10,0){};
\draw[->,color=black] (-9,0) -- (xMAX);
\foreach \x in {-8,-6,-4,-2,2,4,6,8}
\draw[shift={(\x,0)},color=black] (0pt,2pt) -- (0pt,-2pt) node[below] {\footnotesize $\x$};
% y- koordinate
\node (yMAX) at (0,9){};
\draw[->,color=black] (0,-7) -- (yMAX);
\foreach \y in {-6,-4,-2,2,4,6,8}
\draw[shift={(0,\y)},color=black] (2pt,0pt) -- (-2pt,0pt) node[left] {\footnotesize $\y$};
\draw[color=black] (0pt,-10pt) node[right] {\footnotesize $0$};
% Achsenbeschriftung
\draw (xMAX) node[anchor=north east] {$x$};
\draw (yMAX) node[anchor=east] {$g(x)$};
% Graph: 
\clip(-9,-7) rectangle (9.5,8.5);
\draw[color=red,smooth,samples=100,domain=-9:-4.8]
		plot(\x,{(-18*(\x)+3)/(2*(\x)^2+5*(\x)-12)});	
\draw[color=red,smooth,samples=100,domain=-3.1:1.3]
		plot(\x,{(-18*(\x)+3)/(2*(\x)^2+5*(\x)-12)}) ;
\draw[color=red,smooth,samples=100,domain=1.8:9.5]
		plot(\x,{(-18*(\x)+3)/(2*(\x)^2+5*(\x)-12)});
\draw[color=red] (1.1,6) node[right] {$G_g$};
%Definitionslücken:		
\draw[color = blue, fill = white](-4,0) circle(3pt);		
\draw[color = blue, fill = white](1.5,0) circle(3pt);
\end{tikzpicture}
}%
\hspace{0.5cm}
%GRAPHEN: rat3
\MTikzAuto{%
\begin{tikzpicture}[y = 0.2cm, x = 0.5cm]
% Koordinatensystem
% x-koordinate
\node (xMAX) at (6,0){};
\draw[->,color=black] (-4,0) -- (xMAX);
\foreach \x in {-3,-2,-1,1,2,3,4,5}
\draw[shift={(\x,0)},color=black] (0pt,2pt) -- (0pt,-2pt) node[below] {\footnotesize $\x$};
% y- koordinate
\node (yMAX) at (0,20){};
\draw[->,color=black] (0,-10) -- (yMAX);
\foreach \y in {-5,5,10,15}
\draw[shift={(0,\y)},color=black] (2pt,0pt) -- (-2pt,0pt) node[left] {\footnotesize $\y$};
\draw[color=black] (0pt,-10pt) node[right] {\footnotesize $0$};
% Achsenbeschriftung
\draw (xMAX) node[anchor=north east] {$x$};
\draw (yMAX) node[anchor=east] {$h(x)$};
% Graph: 
\clip(-4,-10) rectangle (6,19);
\draw[color=red,smooth,samples=100,domain=-3.1:-1.2]
		plot(\x,{((\x)^3-(\x)^2+(\x))/((\x)+1)});
\draw[color=red,smooth,samples=100,domain=-0.8:5.3]
		plot(\x,{((\x)^3-(\x)^2+(\x))/((\x)+1)});
\draw[color=red] (4.5,15) node[left] {$G_h$};
%Definitionslücken:		
\draw[color = blue, fill = white](-1,0) circle(1.5pt);		
\end{tikzpicture}
}%

\end{MExample}


\begin{MExercise}
Gegeben ist die Funktion
\[
 \function{\psi}{D_\psi}{\R}{x}{\frac{-42x}{x^2-\pi} \MDFPeriod}
\]
Bestimmen Sie den größtmöglichen Definitionsbereich $D_\psi\subset\R$ für $\psi$.
\begin{MHint}{\iSolution}
Die Nullstellen des Nenners ergeben sich durch
\[
 x^2-\pi=0\quad\Leftrightarrow\quad x^2=\pi\quad\Leftrightarrow\quad x=\pm\sqrt{\pi} \MDFPeriod
\]
Damit ist $D_\psi=\R\setminus\{-\sqrt{\pi},\sqrt{\pi}\}$.
\end{MHint}
\end{MExercise}




\begin{MExercise}
Bestimmen Sie für die gebrochenrationalen Funktionen im einführenden Beispiel \MNRef{einf_bsp_gebr_rat} jeweils den Zähler- sowie den Nennergrad und berechnen Sie die Nullstellen des Zählers und des Nenners.
\begin{MHint}{\iSolution}
Die Funktion $f$ hat den Zählergrad $0$ und den Nennergrad $2$. Es gibt keine Zählernullstelle ($8\neq0$) und keine Nennernullstelle ($x^2+1=0$ hat keine reelle Lösung).\\
Die Funktion $g$ hat den Zählergrad $1$ und den Nennergrad $2$. Die Zählernullstelle liegt bei $x=\frac{1}{6}$ ($-18x+3=0\Leftrightarrow x=\frac{3}{18}$) und die Nennernullstellen $x_{1}=-4$, $x_{2} = \frac{3}{2}$ erhält man durch Lösen der quadratischen Gleichung $2x^2+5x-12=0$, zum Beispiel mit Hilfe der Mitternachtsformel.\\
Die Funktion $h$ hat den Zählergrad $3$ und den Nennergrad $1$. Die Nennernullstelle liegt einfach bei $x=-1$ ($x+1=0\Leftrightarrow x=-1$). Für die Zählernullstellen muss die Gleichung $x^3-x^2+x=0$ gelöst werden. Nach Ausklammern von $x$ erhält man $x(x^2-x+1)=0$ und folgert, dass eine Nullstelle bei $x=0$ liegt. Schließlich muss noch die quadratische Gleichung $x^2-x+1=0$ mit der Mitternachtsformel gelöst werden. Es ergibt sich hier allerdings eine negative Diskriminante von $\Delta=1^2-4\cdot1\cdot1=-
3$, so dass keine weitere reelle Lösung -- und damit keine weitere Zählernullstelle -- existiert.
\end{MHint}
\end{MExercise}

Die \highlight{Nullstellen} einer gebrochenrationalen Funktion ergeben sich als die Zählernullstellen. So hat zum Beispiel die Funktion
\[
 \function{j}{\R\setminus\{-1\MElSetSep 3\}}{\R}{x}{\frac{x-1}{x^2-2x-3}}
\]
die einzige Nullstelle bei $x=1$.
Die Nennernullstellen gebrochenrationaler Funktionen, welche aus dem Definitionsbereich ausgeschlossen werden, müssen oft noch genauer untersucht werden. Vor allem ist von Interesse, wie die Graphen der Funktionen \modsemph{in der Nähe} der Definitionslücken verlaufen. Die Nennernullstellen gebrochenrationaler Funktionen bezeichnet man auch als \highlight{Polstellen}. Die folgenden Beispiele zeigen, dass es verschiedene Typen von Polstellen gibt.

\begin{MExample}
\[
 \function{f_1}{\R\setminus\{2\}}{\R}{x}{\frac{3}{x-2}}
\]
\[
 \function{f_2}{\R\setminus\{-3\}}{\R}{x}{\frac{2}{(x+3)^2}}
\]
\[
 \function{f_3}{\R\setminus\{1\}}{\R}{x}{\frac{x^2-1}{x-1}}
\]

%GRAPHEN: rat4
\MTikzAuto{%
\begin{tikzpicture}
% Koordinatensystem
% x-koordinate
\node (xMAX) at (5.8,0){};
\draw[->,color=black] (-3,0) -- (xMAX);
\foreach \x in {-2,-1,1,2,3,4,5}
\draw[shift={(\x,0)},color=black] (0pt,2pt) -- (0pt,-2pt) node[below] {\footnotesize $\x$};
% y- koordinate
\node (yMAX) at (0,5.8){};
\draw[->,color=black] (0,-4) -- (yMAX);
\foreach \y in {-3,-2,-1,1,2,3,4,5}
\draw[shift={(0,\y)},color=black] (2pt,0pt) -- (-2pt,0pt) node[left] {\footnotesize $\y$};
\draw[color=black] (0pt,-10pt) node[right] {\footnotesize $0$};
% Achsenbeschriftung
\draw (xMAX) node[anchor=north east] {$x$};
\draw (yMAX) node[anchor=east] {$f_1(x)$};
% Graph: 
\clip(-3,-4) rectangle (5.5,5.5);
\draw[color=red,smooth,samples=100,domain=-3:1.3]
		plot(\x,{3/((\x)-2)}) ;
\draw[color=red,samples=100,domain=2.4:6]
		plot(\x,{3/((\x)-2)}) ;		
\draw[color=red] (2.7,5) node[right] {$G_{f_1}$};
%Definitionslücken:		
\draw[color = blue, fill = white](2,0) circle(1.5pt);
\draw[color = blue, dotted](2,-4)--(2,5.5);
\end{tikzpicture}
}%

% GRAPHEN: rat5
\MTikzAuto{%
\begin{tikzpicture}
% Koordinatensystem
% x-koordinate
\node (xMAX) at (1.8,0){};
\draw[->,color=black] (-6,0) -- (xMAX);
\foreach \x in {-5,-4,-3,-2,-1,1}
\draw[shift={(\x,0)},color=black] (0pt,2pt) -- (0pt,-2pt) node[below] {\footnotesize $\x$};
% y- koordinate
\node (yMAX) at (0,5.8){};
\draw[->,color=black] (0,-1) -- (yMAX);
\foreach \y in {1,2,3,4,5}
\draw[shift={(0,\y)},color=black] (2pt,0pt) -- (-2pt,0pt) node[left] {\footnotesize $\y$};
\draw[color=black] (0pt,-10pt) node[right] {\footnotesize $0$};
% Achsenbeschriftung
\draw (xMAX) node[anchor=north east] {$x$};
\draw (yMAX) node[anchor=east] {$f_2(x)$};
% Graph: 
\clip(-6,-1) rectangle (1.5,5.5);
\draw[color=red,smooth,samples=100,domain=-6:-3.5]
		plot(\x,{2/(((\x)+3)^2)});
\draw[color=red,smooth,samples=100,domain=-2.5:1.5]
		plot(\x,{2/(((\x)+3)^2)});
\draw[color=red] (1,0.2) node[above] {$G_{f_2}$};
%Definitionslücken:		
\draw[color = blue, fill = white](-3,0) circle(1.5pt);
\draw[color = blue, dotted](-3,-1)--(-3,5.5);
\end{tikzpicture}
}%
\hspace{.5cm}
% GRAPHEN: rat6
\MTikzAuto{%
\begin{tikzpicture}
% Koordinatensystem
% x-koordinate
\node (xMAX) at (3.8,0){};
\draw[->,color=black] (-3,0) -- (xMAX);
\foreach \x in {-2,-1,1,2,3}
\draw[shift={(\x,0)},color=black] (0pt,2pt) -- (0pt,-2pt) node[below] {\footnotesize $\x$};
% y- koordinate
\node (yMAX) at (0,5.8){};
\draw[->,color=black] (0,-2) -- (yMAX);
\foreach \y in {-1,1,2,3,4,5}
\draw[shift={(0,\y)},color=black] (2pt,0pt) -- (-2pt,0pt) node[left] {\footnotesize $\y$};
\draw[color=black] (0pt,-10pt) node[right] {\footnotesize $0$};
% Achsenbeschriftung
\draw (xMAX) node[anchor=north east] {$x$};
\draw (yMAX) node[anchor=east] {$f_3(x)$};
% Graph: 
\clip(-3,-2) rectangle (3.5,5.5);
\draw[color=red, smooth,samples=100,domain=-3:3.5]
		plot(\x,{(\x)+1}) node[above, left=2pt] {$G_{f_3}$};		
%hebbare Definitionslücke:		
\draw[color = blue, fill = white](1,0) circle(1.5pt);
\draw[color = blue, dotted](1,-1)--(1,5);
\draw[color = black, fill = white](1,2) circle(1pt);
\end{tikzpicture}
}%
\end{MExample}

Die Stellen $x=2$ und $x=-3$ sind sogenannte \modsemph{echte Polstellen} der Funktionen $f_1$ und $f_2$, die Stelle $x=1$ ist eine sogenannte \modsemph{hebbare Definitionslücke} der Funktion $f_3$. Anhand der Graphen wird der Unterschied zwischen diesen Typen von Polstellen deutlich. Bei echten Polstellen wächst oder fällt der Graph in der Nähe der Polstelle unbeschränkt, und bei hebbaren Definitionslücken mündet er von links und rechts in das "`Loch'' im Graphen ein. 

Anhand der Abbildungsvorschriften der drei Funktionen kommt dieser Unterschied folgendermaßen zum Ausdruck: Die Werte $x=2$ und $x=-3$ sind \modsemph{Nennernullstellen}, aber keine \modsemph{Zählernullstellen} der Funktionen $f_1$ bzw.~$f_2$. Tatsächlich besitzen $f_1$ und $f_2$ gar keine Zählernullstellen. In einem solchen Fall sind die Nennernullstellen immer echte Polstellen.

\begin{MExercise}
Ist die Nennernullstelle der Funktion
\[
 \function{q}{\R\setminus\{\frac{1}{2}\}}{\R}{x}{\frac{x^4-1}{2x-1}}
\]
eine echte Polstelle? Wenn ja, warum?
\begin{MHint}{\iSolution}
Die Stelle $x=\frac{1}{2}$ ist eine Nullstelle des Nenners:
\[
 2x-1=0\quad\Leftrightarrow\quad 2x=1\quad\Leftrightarrow\quad x=\frac{1}{2} \MDFPeriod
\]
Aber für den Zähler gilt:
\[
 x^4-1=0\quad\Leftrightarrow\quad x^4=1\quad\Leftrightarrow\quad x=\pm 1 \MDFPSpace,
\]
womit die Zählernullstellen bei $x=-1$ und $x=1$ liegen und $x=\frac{1}{2}$ keine Zählernullstelle ist. Damit ist $x=\frac{1}{2}$ eine echte Polstelle.
\end{MHint}
\end{MExercise}

Ein weiterer Unterschied wird zwischen den beiden Polstellen von $f_1$ und $f_2$ deutlich. Bei der Polstelle $x=2$ von $f_1$ findet ein \modsemph{Vorzeichenwechsel} der Funktion statt. Der Graph von $f_1$ fällt links der Polstelle unbeschränkt ins Negative und wächst rechts der Polstelle (von rechts kommend) unbeschränkt ins Positive.\\
Der Graph von $f_2$ wächst auf beiden Seiten der Polstelle $x=-3$ (bei Annäherung an diese) ins Positive, es findet also \modsemph{kein Vorzeichenwechsel} statt. 

In der Abbildungsvorschrift von $f_3$ hingegen kann man den Term, der dafür verantwortlich ist, dass man die Polstelle $x=1$ nicht einsetzen darf, herauskürzen. Dies ist bei gebrochenrationalen Funktionen, die eine hebbare Definitionslücke als Polstelle aufweisen, immer so.  

\begin{MExercise}
Bestimmen Sie alle Polstellen/Definitionslücken von
\[
 \function{\gamma}{D_\gamma}{\R}{x}{\frac{3x+6}{x^2-x-6}}
\]
sowie deren Typ. Geben Sie den größtmöglichen Definitionsbereich $D_\gamma\subset\R$ an.
\begin{MHint}{\iSolution}
Die Nennernullstellen sind die Lösungen der quadratischen Gleichung $x^2-x-6=0$, also
\[
 x_{1,2}=\frac{-(-1)\pm\sqrt{(-1)^2-4\cdot(-6)\cdot1}}{2}=\frac{1\pm 5}{2}=\MCases{3\\-2} \MDFPeriod
\]
Folglich ist der größtmögliche Definitionsbereich
\[
 D_\gamma=\R\setminus\{-2\MElSetSep 3\} \MDFPeriod
\]
Die Zählernullstelle ergibt sich durch $3x+6=0$, liegt also ebenfalls bei $x=-2$. Wir können damit die Abbildungsvorschrift von $\gamma$ also für $x\in D_\gamma$ folgendermaßen umformen:
\[
 \gamma(x)=\frac{3x+6}{x^2-x-6}=\frac{3(x+2)}{(x-3)(x+2)}=\frac{3}{x-3} \MDFPeriod
\]
Damit lässt sich die Funktion auch als
\[
 \function{\gamma}{\R\setminus\{-2\MElSetSep 3\}}{\R}{x}{\frac{3}{x-3}}
\]
schreiben und es liegt bei $x=-2$ eine stetig hebbare Definitionslücke und bei $x=3$ eine echte Polstelle mit Vorzeichenwechsel vor.
\end{MHint}
\end{MExercise}
\end{MXContent}

\begin{MXContent}{Asymptoten}{Asymptoten}{STD}\MLabel{sec:asymptoten}
\MDeclareSiteUXID{VBKM06_EinfacheFunktionen_Asymptoten}
Wir wollen in diesem Abschnitt untersuchen, wie sich gebrochenrationale Funktionen im Unendlichen verhalten, falls der \modsemph{Zählergrad kleiner oder gleich dem Nennergrad} ist. Ein Beispiel ist die Funktion
\[
 \function{f}{\R\setminus\{-\pi\}}{\R}{x}{\frac{x}{x+\pi} \MDFPeriod}
\]
In $f$ ist der Zählergrad $1$ und der Nennergrad $1$.
Beispiele hierfür haben wir im vorhergehenden Abschnitt \MNRef{sec:gebrochen} betrachtet. 

\begin{MExample}\MLabel{ex1:polydiv}
Wir betrachten die Funktion 
\[
 \function{g}{\R\setminus\{1\}}{\R}{x}{1+\frac{1}{x-1}}
\]
und stellen fest, dass ihre Abbildungsvorschrift in der Form einer Summe aus einem Polynom (vom Grad $0$) und einem gebrochenrationalen Term vorliegt.
Durch Bilden des Hauptnenners ist es nun einfach, $g(x)$ auf eine gebrochenrationale Form zu bringen, in der der Zählergrad und der Nennergrad gleich sind:
\[
 g(x)=1+\frac{1}{x-1}=\frac{x-1}{x-1}+\frac{1}{x-1}=\frac{x-1+1}{x-1}=\frac{x}{x-1} \MDFPeriod
\]
Wir können $g$ also auch schreiben als
\[
 \function{g}{\R\setminus\{1\}}{\R}{x}{\frac{x}{x-1}}
\]

und betrachten den zugehörigen Graphen:

% GRAPHEN: rat7
\MTikzAuto{%
\begin{tikzpicture}
% Koordinatensystem
% x-koordinate
\node (xMAX) at (7.8,0){};
\draw[->,color=black] (-5.5,0) -- (xMAX);
\foreach \x in {-5,-4,-3,-2,-1,1,2,3,4,5,6,7}
\draw[shift={(\x,0)},color=black] (0pt,2pt) -- (0pt,-2pt) node[below] {\footnotesize $\x$};
% y- koordinate
\node (yMAX) at (0,5.8){};
\draw[->,color=black] (0,-4.5) -- (yMAX);
\foreach \y in {-4,-3,-2,-1,1,2,3,4,5}
\draw[shift={(0,\y)},color=black] (2pt,0pt) -- (-2pt,0pt) node[left] {\footnotesize $\y$};
\draw[color=black] (0pt,-10pt) node[right] {\footnotesize $0$};
% Achsenbeschriftung
\draw (xMAX) node[anchor=north east] {$x$};
\draw (yMAX) node[anchor=east] {$g(x)$};
% Graph: 
\clip(-5,-4.5) rectangle (7,5.5);
\draw[color=red,smooth,samples=100,domain=-5:0.9]
		plot(\x,{\x/(\x-1)});
\draw[color=red,smooth,samples=100,domain=1.1:7]
		plot(\x,{\x/(\x-1)});
\draw[color=red] (5,1.5) node[above] {$G_{g}$};
%Definitionslücke:		
\draw[color = blue, fill = white](1,0) circle(1.5pt);
%Asymptote:
\draw[color=blue, dotted] (-5.5,1) -- (7.5,1);
\end{tikzpicture}
}%

Neben der Polstelle und Definitionslücke bei $x=1$ erkennen wir, dass der Wert $y=1$ eine besondere Rolle spielt. Dieser wird offenbar von der Funktion $g$ nie angenommen. Für die Wertemenge von $g$ gilt $W_g=\R\setminus\{1\}$. Stattdessen \modsemph{nähert} sich $g$ für "`sehr große'' und "`sehr kleine'' Werte der Veränderlichen $x$ immer stärker dem Wert $1$ an, \modsemph{ohne} diesen jemals für eine reelle Zahl $x$ zu erreichen.



Dies erkennt man in der Abbildungsvorschrift $g(x)=1+\frac{1}{x-1}$ folgendermaßen. Für "`sehr große'' ($50$, $100$, $1000$, usw.) oder "`sehr kleine'' ($-50$, $-100$, $-1000$, usw.) Werte für $x$ nähert sich der gebrochenrationale Anteil $\frac{1}{x-1}$
immer mehr der $0$ an, da $x$ dort im Nenner vorkommt. Tendenziell bleibt also für solche Werte nur noch der polynomielle Anteil $1$ aus der Abbildungsvorschrift übrig. Dieser Anteil kann nun durch eine -- in diesem Fall konstante -- Funktion beschrieben werden, die als \highlight{Asymptote} $g_{As}$ der Funktion $g$ bezeichnet wird:
\[
 \function{g_{As}}{\R}{\R}{x}{1 \MDFPeriod}
\]
Da es sich in diesem Fall um eine konstante Funktion handelt, wird diese auch als \modsemph{waagrechte Asymptote} bezeichnet. 
\end{MExample}
\begin{MExercise}\MLabel{auf2:polydiv}
Bestimmen Sie die Asymptote der Funktion 
\[
 \function{i}{\R\setminus\{-2\}}{\R}{x}{3-\frac{6}{x+2}}
\]
sowie die Asymptote der Hyperbel aus Abschnitt \MNRef{sec:hyperbel}. 
\begin{MHint}{\iSolution}
Es gilt 
\[
 i(x)=3-\frac{6}{x+2}
\]
mit gebrochenrationalem Anteil $\frac{6}{x+2}$. Folglich hat die waagrechte Asymptote von $i$ die Abbildungsvorschrift $i_{As}(x)=3$. Die Hyperbel
\[
 \function{f}{\R\setminus\{0\}}{\R}{x}{\frac{1}{x}}
\]
besitzt auch eine Asymptote. Wir können die Abbildungsvorschrift schreiben als
\[
 f(x)=0+\frac{1}{x} \MDFPSpace,
\]
womit für die Asymptote gilt $f_{As}(x)=0$. Hier ist die Asymptote also diejenige Funktion, die konstant $0$ ist: die Nullfunktion bzw. die Querachse des Koordinatensystems. 
\end{MHint}
\end{MExercise}



\begin{MInfo}
Eine gebrochenrationale Funktion $f$ mit Zählerpolynom $p(x)$ vom Grad $z\geq0$ und Nennerpolynom $q(x)$ vom Grad $n\geq0$ der Form
\[
 \function{f}{\R\setminus\{\text{Nennernullstellen}\}}{\R}{x}{f(x)=\frac{p(x)}{q(x)}}
\]
hat als Asymptote eine konstante Funktion (bzw. eine waagrechte Asymptote) falls $z\leq n$ gilt. Insbesondere ist die Nullfunktion die Asymptote im Fall $z<n$.  
\end{MInfo}

\end{MXContent}



\MSubsection{Potenzfunktionen}
\MLabel{VBKM06_Potenz}
\begin{MIntro}
\MDeclareSiteUXID{VBKM06_Potenzfunktionen_Intro}
In den Abschnitten \MNRef{sec:monome} und \MNRef{sec:hyperbel} haben wir Monome und Funktionen vom hyperbolischen Typ kennengelernt. Zusammenfassend kann man diese als folgende Klasse von Funktionen aufschreiben:
\[
 \function{f}{D_f}{\R}{x}{x^k} \MDFPSpace,
\]
wobei $k\in\Z\setminus\{0\}$ und $D_f=\R$ falls $k\in\N$ sowie $D_f=\R\setminus\{0\}$ falls $k\in\Z$ mit $k<0$. In diesem Abschnitt wollen wir beliebige rationale Werte im Exponenten der Abbildungsvorschrift zulassen. Man erhält dann die sogenannten \modsemph{Potenzfunktionen}, die wiederum die Monome und Funktionen vom hyperbolischen Typ als Spezialfälle enthalten. Wir stellen auch deren fundamentale Eigenschaften und einige Anwendungen zusammen.  
\end{MIntro}



\begin{MXContent}{Wurzelfunktionen}{Wurzelfunktionen}{STD}\MLabel{sec:wurzel}
\MDeclareSiteUXID{VBKM06_Potenzfunktionen_Wurzel}
\begin{MExample}\MLabel{bsp1:wurzel}
Untersucht man einen Körper, der sich im freien Fall im homogenen Gravitationsfeld der Erde befindet, so kann man folgenden Zusammenhang zwischen seiner Fallzeit und seinem zurückgelegten Weg feststellen:

\renewcommand{\arraystretch}{1.5}
\begin{tabular}{c|c|c|c|c|c}
Fallzeit $t$ in Sekunden & $0$ & $\sqrt{\frac{2}{g}}$ & $\sqrt{\frac{2}{g}}\cdot \MZahl{1}{5}$ & $\sqrt{\frac{2}{g}}\cdot 2$ & $\sqrt{\frac{2}{g}}\cdot 3$ \\\hline
zurückgelegter Weg $s$ in Metern & $0$ & $1$ & $\MZahl{2}{25}$ & $4$ & $9$
\end{tabular}

Dabei ist $g\approx \MZahl{9}{81}\frac{\MEinheit[]{m}}{\MEinheit[]{s}^2}$ die physikalische Konstante der Fallbeschleunigung. Trägt man nun diese Werte in einem Diagramm mit $t$ auf der Hochachse und $s$ auf der Querachse auf, erhält man:

% GRAPHEN: wurz-mess
\MTikzAuto{%
\begin{tikzpicture}[y = 2cm]
% Koordinatensystem
% x-koordinate
\node (xMAX) at (10,0){};
\draw[->,color=black] (-0.5,0) -- (xMAX);
\foreach \x in {0,1,2.25, 4, 9}
		\draw[shift={(\x,0)},color=black] 
		  (0pt,2pt) -- (0pt,-2pt) node[below] {\footnotesize $\x$};
% y- koordinate
\node (yMAX) at (0,1.8){};
\draw[->,color=black] (0,0) -- (yMAX);

\draw[shift={(0,0)},color=black] (2pt,0pt) -- (-2pt,0pt) node[above left] {\tiny $0$};
\draw[shift={(0,0.45)},color=black] (2pt,0pt) -- (-2pt,0pt) node[left] {\tiny $\sqrt{\frac{2}{g}}$};

\foreach \y in {1.5, 2, 3}
\draw[shift={(0,0.45*\y)},color=black] (2pt,0pt) -- (-2pt,0pt) node[left] {\tiny $\sqrt{\frac{2}{g}}\cdot\y$};

% Achsenbeschriftung
\draw (xMAX) node[anchor=north east] {$s$};
\draw (yMAX) node[anchor=east] {$t(s)$};	
% Punkte
\draw[fill = blue](0,0)circle(1.5pt);
\draw[fill = blue](1,0.45)circle(1.5pt);
\draw[fill = blue](2.25,0.68)circle(1.5pt);
\draw[fill = blue](4,0.9)circle(1.5pt);
\draw[fill = blue](9,1.35)circle(1.5pt);
\end{tikzpicture}
}%

Dies legt nahe, dass man den Zusammenhang zwischen $t$ und $s$, mit $s$ als Veränderlicher, mathematisch durch die Funktion
\[
 \function{t}{[0\MIntvlSep \infty)}{\R}{s}{\sqrt{\frac{2}{g}}\cdot \sqrt{s}}
\]
beschreiben kann, also eine Funktion, in deren Abbildungsvorschrift die Wurzel (genauer gesagt die Quadratwurzel) der Veränderlichen vorkommt. Ihr Graph beinhaltet dann die obigen gemessenen Punkte:

% GRAPHEN: wurz1
\MTikzAuto{%
\begin{tikzpicture}[y = 2cm]
% Koordinatensystem
% x-koordinate
\node (xMAX) at (10,0){};
\draw[->,color=black] (-0.5,0) -- (xMAX);
\foreach \x in {0,1,2.25, 4, 9}
		\draw[shift={(\x,0)},color=black] 
		  (0pt,2pt) -- (0pt,-2pt) node[below] {\footnotesize $\x$};
% y- koordinate
\node (yMAX) at (0,1.8){};
\draw[->,color=black] (0,0) -- (yMAX);
\draw[shift={(0,0)},color=black] (2pt,0pt) -- (-2pt,0pt) node[above left] {\tiny $0$};
\draw[shift={(0,0.45)},color=black] (2pt,0pt) -- (-2pt,0pt) node[left] {\tiny $\sqrt{\frac{2}{g}}$};

\foreach \y in {1.5, 2, 3}
\draw[shift={(0,0.45*\y)},color=black] (2pt,0pt) -- (-2pt,0pt) node[left] {\tiny $\sqrt{\frac{2}{g}}\cdot\y$};
% Achsenbeschriftung
\draw (xMAX) node[anchor=north east] {$s$};
\draw (yMAX) node[anchor=east] {$t(s)$};
%Graph
\draw[smooth,samples=100,domain=0.01:9.5,color=blue] plot(\x,{sqrt(2.0*(\x)/9.81)});	
% Punkte
\draw[fill = blue](0,0)circle(1.5pt);
\draw[fill = blue](1,0.45)circle(1.5pt);
\draw[fill = blue](2.25,0.68)circle(1.5pt);
\draw[fill = blue](4,0.9)circle(1.5pt);
\draw[fill = blue](9,1.35)circle(1.5pt);
\end{tikzpicture}
}%
\end{MExample}

Dieses Beispiel zeigt, dass Funktionen mit Abbildungsvorschriften, die Wurzeln der Veränderlichen enthalten, natürlicherweise in Anwendungen der Mathematik auftauchen. 

Für natürliche Zahlen $n\in\N$, $n>1$ bezeichnet man die Funktionen
\[
 \function{f_n}{D_{f_n}}{\R}{x}{\sqrt[n]{x}=x^{\frac{1}{n}}}
\]
als die Klasse der \highlight{Wurzelfunktionen}. Diese beinhalten offenbar die Quadratwurzel $f_2(x)=\sqrt{x}$, die dritte Wurzel $f_3(x)=\sqrt[3]{x}$, die vierte Wurzel $f_4(x)=\sqrt[4]{x}$, usw.~als Abbildungsvorschriften von Funktionen (vgl. \MSRef{VBKM01_Potenzgesetze}{Potenzgesetze}).

\begin{MExercise}
Benutzen Sie die Potenzrechengesetze, um die Abbildungsvorschrift der Wurzelfunktionen ohne Wurzelzeichen und stattdessen mit Hilfe von Exponenten aufzuschreiben.
\begin{MHint}{\iSolution}
Nach den Potenzrechengesetzen gilt
\[
 f_n(x)=\sqrt[n]{x}=x^{\frac{1}{n}}
\]
für alle natürlichen Zahlen $n$. Also zum Beispiel
\[
 f_2(x)=\sqrt{x}=x^{\frac{1}{2}}\MDFPSpace, \MDFPaSpace f_3(x)=\sqrt[3]{x}=x^{\frac{1}{3}}\MDFPSpace, \MDFPaSpace f_4(x)=\sqrt[4]{x}=x^{\frac{1}{4}}\MDFPSpace, \MDFPaSpace \dots
\]
\end{MHint}
\end{MExercise}

\begin{MExercise}
Welche Funktion $f_n$ ergäbe sich für $n=1$?
\begin{MHint}{\iSolution}
Für $n=1$ gilt nach den Potenzrechengesetzen
\[
 f_1(x)=\sqrt[1]{x}=x^{\frac{1}{1}}=x=\Id(x) \MDFPeriod
\]
Es ergibt sich also die Identität. Diese schließt man für gewöhnlich aus der Klasse der Wurzelfunktionen aus. 
\end{MHint}
\end{MExercise}

Von großem Interesse ist nun der größtmögliche Definitionsbereich $D_{f_n}$, der für diese Wurzelfunktionen möglich ist. Denn offenbar kommt es auf den Wurzelexponenten $n$ an, welche Werte man für $x$ in die Abbildungsvorschriften einsetzen darf, um reelle Werte als Ergebnisse zu erhalten. So erkennen wir, dass bei der Quadratwurzel $\sqrt{\ }$ nur nicht-negative Werte ein reelles Ergebnis liefern. Betrachten wir allerdings die Kubikwurzel $\sqrt[3]{\ }$, so erhalten wir in diesem Fall, dass alle reellen Zahlen eingesetzt wieder reelle Zahlen als Ergebnis liefern, so etwa $\sqrt[3]{-27}=-3$. Allgemein gilt:

\begin{MInfo}
Die Wurzelfunktionen  
\[
 \function{f_n}{D_{f_n}}{\R}{x}{\sqrt[n]{x}}
\]
mit $n\in\N$ und $n>1$ besitzen den größtmöglichen Definitionsbereich $D_{f_n}=[0\MIntvlSep \infty)$ falls $n$ gerade ist und $D_{f_n}=\R$ falls $n$ ungerade ist.
\end{MInfo}

Damit erhält man folgendes Aussehen für die Graphen der ersten vier Wurzelfunktionen $f_2, f_3, f_4, f_5$:

% GRAPHEN: wurz2
\MTikzAuto{%
\begin{tikzpicture}
% Koordinatensystem
% x-koordinate
\node (xMAX) at (8.8,0){};
\draw[->,color=black] (-8,0) -- (xMAX);
\foreach \x in {-7,-6,-5,-4,-3,-2,-1,1,2,3,4,5,6,7,8}
		\draw[shift={(\x,0)},color=black] 
		  (0pt,2pt) -- (0pt,-2pt) node[below] {\footnotesize $\x$};
% y- koordinate
\node (yMAX) at (0,3.8){};
\draw[->,color=black] (0,-3) -- (yMAX);
\foreach \y in {-2,-1,1,2,3}
\draw[shift={(0,\y)},color=black] (2pt,0pt) -- (-2pt,0pt) node[left] {\footnotesize $\y$};
% Achsenbeschriftung
\draw (xMAX) node[anchor=north east] {$x$};
\draw (yMAX) node[anchor=east] {$f_2(x),f_3(x),f_4(x),f_5(x)$};
%Beschriftung
\draw[color = orange] (8.2,2.9) node[right] {$G_{\sqrt{x}}$};
\draw[color = cyan] (8.2,2.3) node[right] {$G_{\sqrt[3]{x}}$};
\draw[color = red] (8.2,1.8) node[right] {$G_{\sqrt[4]{x}}$};
\draw[color = blue] (8.2,1.3) node[right] {$G_{\sqrt[5]{x}}$};
%Graph
\clip(-8,-3) rectangle (8.2,3.5);
% f_2
\draw [rotate around={90:(0,0)}, 
      color = orange,
			smooth,samples=100,
			domain=0:3] 
			plot(\x,{-(\x)^2});	
%f_3
\draw [rotate around={90:(0,0)}, 
      color = cyan,
			smooth,samples=100,
			domain=-2.2:2.2] 
			plot(\x,{(-\x)^3});
%f_4
\draw [rotate around={90:(0,0)}, 
      color = red,
			smooth,samples=100,
			domain=0:1.8] 
			plot(\x,{-(\x)^4});
%f_3
\draw [rotate around={90:(0,0)}, 
      color = blue,
			smooth,samples=100,
			domain=-1.6:1.6] 
			plot(\x,{(-\x)^5});
\end{tikzpicture}
}%

Aus dem Verlauf der Graphen erkennt man, dass alle Wurzelfunktionen \highlight{streng monoton wachsend} sind. 

\begin{MExercise}
Bestimme für die Wurzelfunktionen
\[
 \function{f_n}{D_{f_n}}{\R}{x}{\sqrt[n]{x}}
\]
mit $n\in\N$, $n>1$, den Wertebereich $W_{f_n}$, in Abhängigkeit davon ob $n$ gerade oder ungerade ist.
\begin{MHint}{\iSolution}
Für die geradzahligen Wurzelfunktionen ergeben sich offenbar als Werte nur nicht-negative reelle Zahlen,
denn nach den Potenzrechengesetzen kann $\sqrt{x}$, $\sqrt[4]{x}$, $\sqrt[6]{x}$, usw.~für $x\geq 0$ nie negativ werden.
Die ungeradzahligen Wurzelfunktionen können auch alle negativen reellen Zahlen als Werte liefern. Tatsächlich gilt
$\sqrt[3]{x}<0$, $\sqrt[5]{x}<0$, usw.~genau dann wenn $x<0$ ist. Zusammenfassend gilt dann also
unter Berücksichtigung der strengen Monotonie der Wurzelfunktionen $W_{f_n}=\R$ falls $n$ ungerade sowie $W_{f_n}=[0\MIntvlSep \infty)$ falls $n$ gerade ist.
\end{MHint}


\end{MExercise}

\end{MXContent}



\MSubsection{Exponentialfunktion und Logarithmus}
\MLabel{VBKM06_exponential}

\begin{MIntro}
\MDeclareSiteUXID{VBKM06_Exponentialfunktionen_Intro}


Bei \MEntry{Exponentialfunktionen}{Exponentialfunktion} stellt die Veränderliche im Gegensatz zu den Potenzfunktionen
nicht die \modstextbf{Basis} des Exponentialausdrucks dar, sondern sie bildet den \modstextbf{Exponenten}. Dementsprechend werden wir Zuordnungsvorschriften betrachten wie z.B.:
$$x \longmapsto 2^x\; \text{oder}\; x \longmapsto 10^x \MDFPeriod$$

\ \\ \ \\
Exponentialfunktionen spielen in vielen unterschiedlichen Bereichen eine wichtige Rolle, so etwa bei der Beschreibung biologischer
Wachstumsprozesse - diverse Modelle zur Bevölkerungsentwicklung eingeschlossen -, bei Prozessen des radioaktiven Zerfalls
oder bei einer bestimmten Form der Zinseszinsberechnung. Betrachten wir ein Beispiel:
\begin{MExample}
\MLabel{M06_bsp_bak_pop}
Eine Bakterienkultur enthält zu Versuchsbeginn $500$ Bakterien und verdoppelt ihre Population alle $13$ Minuten. Wir möchten gerne
wissen, wieviele Bakterien nach $1$ Stunde und $15$ Minuten (also nach $75$ Minuten) in der Kultur vorhanden sind?

In einem ersten Anlauf können wir eine einfache Wertetabelle erstellen, die uns die Bakterienpopulation zu Beginn ($t = 0$ min),
nach $t = 13$ min, nach $t = 26$ min usw., also zu Vielfachen der $13$-Minuten-Verdopplungszeitspanne, angibt:
\begin{center}
 \begin{tabular}{|c|r|r|r|r|r|r|r|r|c|}
 \hline
 Zeit $t$ in min & 0 & 13 & 26 & 39 & 52 & 65 & 78 & 91 & usw. \\ \hline
 Anzahl Bakterien & 500 & 1\,000 & 2\,000 & 4\,000 & 8\,000 & 16\,000 & 32\,000 & 64\,000 & usw. \\ 
 \hline
 \end{tabular}
\end{center}
Aus der Tabelle können wir abschätzen, dass die Antwort auf unsere Frage zwischen
$16\,000$ und $32\,000$, wahrscheinlich näher bei $32\,000$, liegen wird. Doch wie sieht es mit einer präzis(er)en Antwort aus?
Dazu müssen wir den \modstextbf{funktionalen Zusammenhang} zwischen allgemeinen $t$-Werten und Bakterienanzahl kennen. In der
unten stehenden Abbildung ist auch der Graph einer Funktion $p$ wiedergegeben; dieser Funktionsgraph füllt sozusagen die
Lücken zwischen den isolierten Punkten, die den Wertepaaren aus der Tabelle entsprechen und die ebenfalls eingezeichnet
sind. Die zugehörige Abbildungsvorschrift ordnet jedem reellen Zeitpunkt eine Populationsgröße zu. Wie wir sehen werden,
handelt es sich bei der Funktion um eine \modstextbf{Exponentialfunktion}.
\begin{center}
\MUGraphicsSolo{bak_pop.png}{scale=1}{width:400px}
\end{center}
Aus der graphischen Darstellung können wir die gesuchte Anzahl an Bakterien schon etwas genauer ablesen. Aber für die
exakte Angabe benötigen wir die Abbildungsvorschrift, die hinter dem Graphen aus der Abbildung steht und die wir hier
zunächst nur angeben:
$$p: [0\MIntvlSep \infty) \longrightarrow (0\MIntvlSep  \infty) \;\text{mit}\; t \longmapsto p(t) = 500 \cdot 2^{(t/13)} \MDFPeriod$$
(In Aufgabe \MRef{M06_bsp_bak_pop_exercise} werden wir diesen funktionalen Zusammenhang begründen.)\\
Damit erhalten wir für $t = 75$ (gemessen in Minuten) den Funktionswert
$$p(75) = 500 \cdot 2^{(75/13)} \approx 500 \cdot \MZahl{54}{539545} \approx 27\;270 \MDFPeriod$$
Also leben nach $75$ Minuten ca. $27\,270$ Bakterien in der fraglichen Population.
\end{MExample}
\end{MIntro}


\begin{MContent}
\MLabel{M06_allgem_exp_fkt}
\MDeclareSiteUXID{VBKM06_Exponentialfunktionen_Allgemein}

Im vorangegangen \MSRef{M06_bsp_bak_pop}{Beispiel} tritt eine \modstextbf{Exponentialfunktion} zur Basis $a = 2$ auf, die Veränderliche - im
Beispiel $t$ - erscheint im Exponenten. Wir wollen nun die allgemeine Abbildungvorschrift für Exponentialfunktionen
zu einer beliebigen Basis $a$ angeben; dabei setzen wir allerdings $a > 0$ voraus:
$$
 \function{f}{\R}{(0;\infty)}{x}{f(x)=f_{0}\cdot a^{\lambda x}}
$$

Dabei bezeichnen $f_0$ und $\lambda$ sogenannte Parameter der Exponentialfunktion, auf die wir weiter
unten eingehen werden.

Der Definitionsbereich aller Exponentialfunktionen wird von allen reellen Zahlen gebildet, $D_f = \R$, wohingegen
der Wertebereich nur aus den positiven reellen Zahlen besteht ($W_f = (0\MIntvlSep  \infty)$), da jedwede Potenz einer
postiven Zahl nur positiv sein kann.

\begin{MExercise}
 Warum setzt man bei den Exponentialfunktionen voraus, dass die Basis $a$ größer Null sein soll? \begin{MHint}{\iSolution}
  Eine Exponentialfunktion soll nicht nur für ausgewählte, spezielle oder isolierte Werte der Veränderlichen $x$
  definiert sein, sondern, wenn möglich, für alle reellen Zahlen. Würde man negative Basiswerte $a < 0$ zulassen, so
  würden sofort Probleme beim Wurzelziehen - siehe $a^{(1/2)} = \sqrt{a}, a^{1/4}, a^{1/12}$ etc. - auftreten.
  Zum Beispiel sind Quadratwurzeln aus negativen Zahlen nicht definiert, vergleiche auch Abschnitt \MRef{VBKM06_Potenz}.
 \end{MHint}
\end{MExercise}

Einige generelle Eigenschaften von Exponentialfunktionen können wir im folgenden Bild erkennen, in dem
Exponentialfunktionen $g: \R \rightarrow (0\MIntvlSep  \infty)$, $x \longmapsto g(x) = a^x$ für verschiedene Werte von $a$ gegenübergestellt
sind:
\begin{center}
\MUGraphicsSolo{expo_vgl.png}{scale=1}{width:400px}
\end{center}
\begin{itemize}
 \item Alle diese Exponentialfunktionen gehen durch den Punkt $(x = 0, y = 1)$: Dies gilt,
 da $g(x = 0) = a^0$ und $a^0 = 1$ für jede Zahl $a$.
 \item Ist $a > 1$, so steigt der Graph von $g$ von links nach rechts (also für wachsende $x$-Werte) an;
 man sagt auch, dass die Funktion $g$ \modstextbf{streng monoton wachsend} ist. Je größer der Wert für $a$ ist, desto
 schneller wächst $g$ für \modstextbf{positive} $x$-Werte. Geht
 man von rechts nach links (also zu immer größeren negativen $x$-Werten), so bildet die negative $x$-Achse eine Asymptote des Graphen.
 \item Ist $a < 1$, so fällt der Graph von $g$ von links nach rechts (also für wachsende $x$-Werte) ab;
 man sagt auch, dass die Funktion $g$ \modstextbf{streng monoton fallend} ist. Je größer der Wert für $a$ ist, desto
 langsamer fällt $g$ für \modstextbf{negative} $x$-Werte. Geht
 man von links nach rechts (also zu immer größeren positiven $x$-Werten), so bildet die positive $x$-Achse eine Asymptote.
\end{itemize}
Und was hat es nun noch mit den Parametern $f_0$ und $\lambda$ auf sich? Der Parameter $f_0$ ist schnell
erklärt: Setzt man den Wert $x = 0$ für die Veränderliche in die allgemeinen Exponentialfunktionen $f$ ein,
$$f(x = 0) = f_0 \cdot a^{\lambda \cdot 0} = f_0 \cdot a^0 = f_0 \cdot 1 = f_0 \MDFPSpace,$$
so erkennt man, dass $f_0$ eine Art \modstextbf{Start- oder Anfangswert} darstellt
(zumindest falls man die Veränderliche $x$ zeitlich interpretiert); der exponentielle Verlauf $a^{\lambda x}$
wird generell mit dem Faktor $f_0$ multipliziert und dementsprechend gewichtet, d.h. gestreckt (für $|f_0| > 1$)
bzw. gestaucht (für $|f_0| < 1$).

Der Parameter $\lambda$ im Exponenten heißt \MEntry{Wachstumsrate}{Wachstumsrate}; er bestimmt, wie stark
die Exponentialfunktion - bei gleichbleibender Basis - wächst (für $\lambda > 0$) oder fällt (für
$\lambda < 0$). 

\begin{MExercise}
 \MLabel{M06_bsp_bak_pop_exercise}
 Begründen Sie die Form der Exponentialfunktion $f(t) = 500 \cdot 2^{(t/13)}$, die im Beispiel
 \MRef{M06_bsp_bak_pop} auftritt! \begin{MHint}{\iSolution}
  In jeder Verdopplungszeitspanne von $13$ Minuten verdoppelt sich - wie der Name schon sagt - die Bakterienpopulation.
  Jeweils bezogen auf den Ausgangswert ($500$ Bakterien) hat sich also die Anzahl an Bakterien nach einer Zeitspanne von $13$
  Minuten verdoppelt, nach zwei solchen Zeitspannen vervierfacht, nach $3 \cdot 13$ Minuten verachtfacht (immer - wie erwähnt -
  im Vergleich zum Anfangswert) usw. Daran erkennen wir, dass bei dem Wachstumsprozess $2$er-Potenzen involviert sind;
  dementsprechend wählen wir als Basis für den funktionalen Zusammenhang $a = 2$.
  
  Diese Überlegung legt auch den Exponenten der gesuchten Exponentialfunktion fest: Unsere Zeitmessung muss sich auf die
  $13$-Minuten-Zeitspanne beziehen, der Exponent ist daher $\Mtfrac{t}{13}$. Nach $13$ Minuten ergibt sich somit für den 
  Exponenten $\Mtfrac{13}{13} = 1$. Der Wachstumsfaktor ergibt sich zu $2^{(13/13)}=2$. Nach zwei Zeitspannen (gleich $26$ Minuten) ist der Exponent $\Mtfrac{26}{13} = 2$ 
  und damit der Wachstumsfaktor insgesamt $2^{(26/13)} = 2^2 = 4$ usw.
  
  Schließlich müssen wir $2^{(t/13)}$ noch mit dem richtigen Anfangswert ($500$ Bakterien) gewichten;
  dies geschieht mit Hilfe des Faktors $500$.
 \end{MHint}
\end{MExercise}
\end{MContent}

\begin{MXContent}{Eulersche Funktion}{Eulersche Funktion}{STD}
\MLabel{M06_e_fkt}
\MDeclareSiteUXID{VBKM06_Exponentialfunktionen_efkt}
 Es gibt eine ganz besondere Exponentialfunktion, manchmal auch als \textit{die} Exponentialfunktion bezeichnet, um die
 wir uns jetzt kümmern wollen. In der Tat lassen sich, wie wir sehen werden, alle anderen Exponentialfunktionen auf diese
 besondere Exponentialfunktion zurückführen. Sie besitzt als Basis die \MEntry{eulersche Zahl}{Eulersche Zahl} $e$. Ihr (ungefährer) Wert beträgt
 $$\MEU = \MZahl{2}{718281828459045235} \dots \MDFPeriod$$
 
 
 Betrachten wir also - zunächst ohne irgendwelche zusätzlichen Parameter - den Graphen \modstextbf{der} Exponentialfunktion,
 $$
 \begin{array}{rrcl}
 g: & \R & \longrightarrow & (0\MIntvlSep  \infty) \\ & x & \longmapsto & g(x) = \MEU^x \MDFPSpace,
 \end{array}
 $$
 wegen der Basis $\MEU$ auch \MEntry{$\MEU$-Funktion}{e-Funktion} oder \MEntry{natürliche Exponentialfunktion}{Exponentialfunktion (natürlich)} genannt:
 \begin{center}
 \MUGraphicsSolo{e_fkt.png}{scale=1}{width:400px}
 \end{center}
 Wenig überraschend zeigt auch die $\MEU$-Funktion das bereits in \MRef{M06_allgem_exp_fkt}
 diskutierte Verhalten der Exponentialfunktionen $x \Mmapsto a^x$ $(a > 1)$;
 schließlich haben wir für die Basis ja auch nur einen speziellen Wert, nämlich $a = \MEU$, gewählt.
 Insbesondere halten wir nochmals fest, dass die $\MEU$-Funktion \modstextbf{streng monoton wachsend} ist, sich für große
 negative $x$-Werte an die negative $x$-Achse anschmiegt und für $x = 0$ den Wert $1$ annimmt.
 \begin{MExercise}
  \MLabel{M06_aufg_e_fkt_2}
  Wie sieht der Graph der Funktion $h: \R \rightarrow (0\MIntvlSep  \infty)$, $x \Mmapsto h(x) = \MEU^{- x}$ aus, und welche generellen
  Eigenschaften besitzt diese Funktion?
  \begin{MHint}{\iSolution}
   \begin{center}
    \begin{tabular}{ccc}
     \MUGraphicsSolo{e_fkt_2.png}{scale=1}{width:300px} &
     \hspace*{1.5cm} &
     \begin{minipage}[b]{6cm}
      Die Funktion $h$ ist streng monoton fallend, für große positive $x$-Werte schmiegt sich der Graph von $h$ an die
      positive $x$-Achse an und für $x = 0$ nimmt $h$ den Wert $h(x = 0) = 1$ an.
     \end{minipage}
    \end{tabular}
   \end{center}
  \end{MHint}
 \end{MExercise}
 


 Eingangs dieses Unterabschnitts haben wir behauptet, dass sich die weiter oben besprochenen Exponentialfunktionen
 auf die $\MEU$-Funktion zurückführen lassen. Dies gelingt mit Hilfe der Identität
 $$a^x = \MEU^{x \cdot\ln(a)} \MDFPSpace,$$
 die für beliebige reelle $a > 0$ und beliebige reelle $x$ gilt. Dabei bezeichnet $\ln$ den
 \MSRef{VBKM06_LNDEF}{natürlichen Logarithmus}, dessen Funktionsgestalt uns im folgenden Abschnitt
 \MRef{M06_Logarithmus} noch ausgiebig beschäftigen wird. 
 \begin{MExercise}
  Begründen Sie die Identität $a^x = \MEU^{x \cdot \ln(a)}$. \begin{MHint}{\iSolution}
   Nach dem Potenzgesetz $(b^r)^s = b^{r \cdot s}$ kann die rechte Seite der in Frage stehenden Identität als
   $\MEU^{x \cdot\ln(a)} = (\MEU^{\ln(a)})^x$ umgeschrieben werden. Da $\ln$ die Umkehrung zu \textit{$\MEU$ hoch}
   ist, folgt $\MEU^{\ln(a)} = a$. Damit folgt $(\MEU^{\ln(a)})^x = a^x$, was in der Tat die linke Seite der Identität darstellt.
  \end{MHint}
 \end{MExercise}
 
 
 \modstextbf{Allgemeine $\MEU$-Funktionen} enthalten die bereits in Unterabschnitt \MRef{M06_allgem_exp_fkt} eingeführten
 Parameter $f_0$ und $\lambda$; ihre funktionale Gestalt sieht also folgendermaßen aus:
 $$
 \begin{array}{rrcl}
  f: & \R & \longrightarrow & (0\MIntvlSep  \infty) \\ & x & \longmapsto & f(x) = f_0 \cdot \MEU^{\lambda x} \MDFPeriod
 \end{array}
 $$
 Wiederum repräsentiert $f_0$ die Möglichkeit von $1$ verschiedener Start- oder Anfangswerte, und der Faktor
 $\lambda$ im Exponenten gestattet unterschiedlich starke (positive oder negative) Wachstumsraten. Wir wollen
 dies abschließend an einem Beispiel verdeutlichen:
 \begin{MExample}
 Bei einer Versuchsreihe mit radioaktiven Jodatomen ($^{131}\mathrm{I}$) ergeben sich im Mittel folgende Daten:
  \begin{center}
   \begin{tabular}{|c|r|r|r|r|r|}
    \hline
    Anzahl Jodatome & 10000 & 5000 & 2500 & 1250 & usw. \\ \hline
    Anzahl Tage seit Beginn & 0 & \MZahl{8}{04} & \MZahl{16}{08} & \MZahl{24}{12} & usw.\\ \hline
   \end{tabular}
  \end{center}
  Mit anderen Worten: Alle $\MZahl{8}{04}$ Tage halbiert sich die Anzahl der Jodatome aufgrund radioaktiven Zerfalls; man
  spricht daher in diesem Zusammenhang davon, dass die \modstextbf{Halbwertszeit} $h$ von Jod-$131$ $h = \MZahl{8}{04}$ Tage beträgt.
  
  Der radioaktive Zerfall folgt einem \modstextbf{Exponentialgesetz}:
  $$N(t) = N_0 \cdot \MEU^{\lambda t} \MDFPeriod$$
  Unsere \modstextbf{Exponentialfunktion} heißt hier $N$; sie gibt die Anzahl der noch vorhandenen Jodatome an. $N_0$
  steht dementsprechend für die Anzahl der Jodatome zu Beginn, also $N_0 = 10000$. Die Veränderliche im vorliegenden
  Beispiel ist die Zeit $t$ (gemessen in Tagen). Von dem Parameter $\lambda$ erwarten wir, dass er negativ ist,
  da es um die Beschreibung eines Zerfallsprozesses, also eines Prozesses mit negativem Wachstum, geht. Wir wollen
  $\lambda$ in der Folge aus den Messdaten bestimmen:
  
  Nach $h = \MZahl{8}{04}$ Tagen sind nur noch $5000$ Jodatome vorhanden, d.h. $N(t = \MZahl{8}{04}) = 5000 = \Mtfrac{N_0}{2}$.
  Verwenden wir das Exponentialgesetz für den radioaktiven Zerfall, so erhalten wir:
  $$\Mdfrac{N_0}{2} = N_0 \cdot \MEU^{\lambda \cdot h}  \MDFPeriod$$
  Wir können $N_0$ auf beiden Seiten der Gleichung kürzen und anschließend logarithmieren (siehe Abschnitt
  \MRef{M06_Logarithmus}):
  $$\ln \left( \Mdfrac12 \right ) = \ln(\MEU^{\lambda \cdot h})  \MDFPeriod$$
  Die linke Seite formen wir gemäß den Rechenregeln für den Logarithmus um (siehe Abschnitt \MRef{M06_Logarithmus}),
  $\ln(1/2) = \ln(1) - \ln(2) = 0 - \ln(2) = - \ln(2)$. Für die rechte Seite beachten
  wir, dass Logarithmieren die Umkehrung zum Exponentieren darstellt, $\ln(\MEU^{\lambda \cdot h}) = \lambda \cdot h$;
  damit folgt:
  $$\begin{array}{crcl} & - \ln(2) & = & \lambda \cdot h \\[.5ex]
  \Leftrightarrow & \lambda & = & - \Mdfrac{\ln(2)}{h}  \MDFPeriod \end{array}$$
  Mit $h = \MZahl{8}{04}$ Tage für Jod-$131$ folgt für den Parameter $\lambda$ im vorliegenden Fall
  $$\lambda \approx - \MZahl{0}{0862} \; \Mdfrac{1}{\text{Tage}}  \MDFPeriod$$
  
  Andere radioaktive Substanzen besitzen andere Halbwertszeiten - Plutonium-239 z.B. weist eine Halbwertszeit
  von ungefähr $24000$ Jahren auf - und führen folglich auf andere Werte für den Parameter $\lambda$ im
  Exponentialgesetz für den radioaktiven Zerfall.
 \end{MExample}
\end{MXContent}



\begin{MXContent}{Logarithmus}{Logarithmus}{STD}
\MLabel{M06_Logarithmus}
\MDeclareSiteUXID{VBKM06_Logarithmusfunktion}

In Abschnitt \MRef{M06_e_fkt} haben wir beim Studium der $\MEU$-Funktion,
 $$
 \begin{array}{rrcl} g : & \R & \longrightarrow & (0\MIntvlSep  \infty) \\
 & x & \longmapsto & g(x) = \MEU^x  \MDFPSpace,
 \end{array}
 $$
 insbesondere auf eine sehr wichtige Eigenschaft der natürlichen Exponentialfunktion hingewiesen, nämlich dass diese
 Funktion streng monoton wachsend ist. Spiegelt man den Graph der Funktion an der Winkelhalbierenden zwischen dem ersten und dritten Quadranten,
 so erhält man den Graphen der natürlichen Logarithmusfunktion - und versieht sie mit einem eigenen Symbol, nämlich $\ln$:
 
\begin{MInfo}
\MLabel{VBKM06_LNDEF}
Die über die Gleichung $\MEU^{\ln(x)}=x$ erklärte Funktion
$$
\function{\ln}{(0\MIntvlSep \infty)}{\R}{x}{\ln(x)}
$$
heißt die \MEntry{natürliche Logarithmusfunktion}{Logarithmusfunktion (natürlich)}.
\end{MInfo}

Die Gleichung ist dabei so zu lesen, dass $\ln(x)=a$ derjenige Wert $a$ ist mit $\MEU^a=x$.
Diese Konstruktion wird im folgenden Bild dargestellt: 
 \begin{center}
 \MUGraphicsSolo{ln_fkt.png}{scale=1}{}
 \end{center}

Folgende Eigenschaften der natürlichen Logarithmusfunktion können wir dem Graphen entnehmen:
 \begin{itemize}
  \item Die Funktion $\ln$ ist streng monoton wachsend.
  \item Nähert man sich von rechts auf der $x$-Achse dem Nullpunkt, so nimmt $\ln(x)$ immer größere negative
  Werte an: Wir halten fest, dass sich der Graph von $\ln$ an die negative Hochachse ($y$-Achse) anschmiegt.
  \item An der Stelle $x = 1$ besitzt die natürliche Logarithmusfunktion den Wert $0$, $\ln(1) = 0$.
 \end{itemize}
 
Neben der natürlichen Logarithmusfunktion gibt es noch andere Logarithmusfunktionen, die jeweils zu einem bestimmten Exponenten gehören:

\begin{MInfo}
Ist $b>0$ ein beliebiger Exponent, so nennt man die über die Gleichung $b^{\log_b(x)}=x$ (sprich: $\log_b(x)=a$ ist derjenige Exponent $a$ mit $b^a=x$) erklärte Funktion
$$
\function{\log_b}{(0\MIntvlSep \infty)}{\R}{x}{\log_b(x)}
$$
die \MEntry{allgemeine Logarithmusfunktion}{Logarithmusfunktion (allgemein)} zur Basis $b$.
\end{MInfo}

Die Logarithmusfunktion kann man in der Regel nicht direkt ausrechnen. Da sie als die Umkehrfunktion zur Exponentialfunktion definiert ist, versucht man in der Regel, ihre Eingabe als Potenz zu schreiben
und den Exponenten abzulesen.

\begin{MExample}
Typische Berechnungen für den natürlichen Logarithmus sind
$$
\ln(\MEU^5) \;=\; 5\MDFPSpace, \MDFPaSpace \ln(\sqrt{\MEU}) \;=\; \ln(\MEU^{\frac12}) \;=\; \frac12
$$
sowie für den allgemeinen Logarithmus
$$
\log_5(25) \;=\; \log_5(5^2) \;=\; 2 \MDFPSpace, \MDFPaSpace \log_3(81) \;=\; \log_3(3^4) \;=\; 4 \MDFPeriod
$$
\end{MExample}

Dabei muss man auf die Basis des Logarithmus achten, beispielsweise ist
$$
\log_2(64) \;=\; \log_2(2^6) \;=\; 6 \MDFPSpace,\MDFPSpace \text{ aber }\MDFPSpace
\log_4(64) \;=\; \log_4(4^3) \;=\; 3 \MDFPeriod
$$

\begin{MExercise}
Berechnen Sie diese Logarithmen:
\begin{MExerciseItems}
\item{\MEquationItem{$\displaystyle\ln(\sqrt[3]{\MEU})$}{\MLSimplifyQuestion{10}{1/3}{5}{}{5}{1}{LNA1}}. \begin{MHint}{\iSolution}Es ist $\ln(\sqrt[3]{\MEU})=\ln(\MEU^{\frac13})=\frac13$.\end{MHint}}
\item{\MEquationItem{$\displaystyle\log_2(256)$}{\MLSimplifyQuestion{10}{8}{5}{}{5}{1}{LNA2}}. \begin{MHint}{\iSolution}Es ist $\log_2(256)=\log_2(2^8)=8$.\end{MHint}}
\item{\MEquationItem{$\displaystyle\log_9(3)$}{\MLSimplifyQuestion{10}{1/2}{5}{}{5}{1}{LNA3}}. \begin{MHint}{\iSolution}Es ist $\log_9(3)=\log_9(9^{\frac12})=\frac12$.\end{MHint}}
\end{MExerciseItems}
\end{MExercise}

In der Mathematik und den Naturwissenschaften werden folgende Logarithmen häufig eingesetzt und erhalten deshalb besondere Symbole:
\begin{itemize}
\item{Logarithmus zur Basis $10$: $\log_{10}(x)=\lg(x)$ oder manchmal auch nur $\log(x)$, dieser Logarithmus gehört zu den Zehnerpotenzen und wird beispielsweise zur Berechnung von pH-Werten in der Chemie eingesetzt.}
\item{Logarithmus zur Basis $2$: $\log_2(x)=\ld(x)$, dieser Logarithmus ist in der Informatik wichtig.}
\item{Logarithmus zur Basis $\MEU$: $\log_\MEU(x)=\ln(\MEU)$, der natürliche Logarithmus ist für praktische Rechnungen meist ungeeignet (es sei denn,
der Ausdruck ist eine $\MEU$-Potenz). Er wird als natürlich bezeichnet, weil die Exponentialfunktion zur Basis $\MEU$ aus mathematischer Sicht einfacher ist
als die allgemeinen Exponentialfunktionen (z.B. weil $\MEU^x$ seine eigene Ableitung ist, $b^x$ aber nicht $b\not=\MEU$).}
\end{itemize}

Für die Logarithmusfunktion gibt es zahlreiche Rechenregeln, die im folgenden Abschnitt erklärt werden.

\end{MXContent}



\begin{MXContent}{Logarithmengesetze}{Logarithmengesetze}{STD}
\MDeclareSiteUXID{VBKM06_Logarithmusgesetze}
 Für das Rechnen mit Logarithmen gelten gewisse Gesetze, die sich aus den \MSRef{VBKM01_Potenzgesetze}{Potenzgesetzen} herleiten lassen:
 
 
 \begin{MInfo}
  \MLabel{M06_log_ges}
  Die folgenden Rechenregeln bezeichnet man als \MEntry{Logarithmengesetze}{Logarithmengesetze}:
  $$\begin{array}{rcll}
   \log (u \cdot v) & = & \log(u) + \log(v) & (u, v >0) \MDFPSpace, \\[.5ex]
   \log \left( \Mdfrac{u}{v} \right) & = & \log (u) - \log (v) & (u, v >0) \MDFPSpace, \\[.5ex]
   \log (u^x) & = & x \cdot \log(u) & (u>0, x \in \R) \MDFPeriod
  \end{array}$$
 \end{MInfo}
 
 Diese Gesetze sind neben dem natürlichen auch für alle anderen Logarithmen richtig und eignen sich dazu, einen gegebenen Term so umzuformen,
 dass Potenzen alleine in den Logarithmen stehen:
 
 \begin{MExample}
 Den Wert $\ld(4^5)$ kann man beispielsweise mit Hilfe der Logarithmengesetze so ausrechnen:
 $$
 \ld(8^5) \;=\; \log_2(8^5) \;=\;5\cdot \log_2(8) \;=\; 5\cdot \log_2(2^3) \;=\; 5\cdot 3 \;=\; 15 \MDFPeriod
 $$
 Produkte in Logarithmen kann man in Summen außerhalb der Logarithmen zerlegen:
 $$
 \lg\left({100\cdot \sqrt{10}\cdot \frac1{10}}\right) \;=\; \lg(100)+\lg(\sqrt{10})-\lg(10) \;=\; 2+\frac12-1 \;=\; \frac32 \MDFPeriod
 $$
 \end{MExample}

 
 
 
 Wichtig bei der Zerlegungsregel $\log(u\cdot v)=\log(u)+\log(v)$ ist, dass sie Produkte in Summen umwandelt. Der umgekehrte Weg ist beim Logarithmus nicht möglich, den Logarithmus einer Summe
 kann man nicht weiter umformen.
 
\end{MXContent}



\MSubsection{Trigonometrische Funktionen}
\MLabel{VBKM06_trigonometrisch}

\begin{MIntro}
\MDeclareSiteUXID{VBKM06_TrigonometrischeFunktionen_Intro}
 Die \MEntry{Trigonometrie}{Trigonometrie} ist der griechischen Wortherkunft nach die \modstextbf{Maßlehre der Dreiecke}.
 Eine zentrale Rolle nehmen dabei die sogenannten \MEntry{trigonometrischen Funktionen}{Trigonometrische Funktionen}, wie \modstextbf{Sinus-}, \modstextbf{Kosinus-} und \modstextbf{Tangensfunktion}, ein.
 
 Das Anwendungsfeld von Sinus, Kosinus, Tangens \& Co. ist jedoch nicht auf
 \glqq simple{\grqq} Dreiecksberechnungen beschränkt. Vielmehr entfalten die trigonometrischen Funktionen
 ihr eigentliches Potenzial erst in den mannigfachen Anwendungsbereichen, deren wichtigste vielleicht
 in der Beschreibung von Schwingungs-- und Wellenvorgängen durch trigonometrische Funktionen in Physik und Technik liegen. Aber auch in vielen anderen
 Gebieten, der Landvermessung etwa oder der Astronomie, kommen sie zum Tragen.
 
\end{MIntro}

\begin{MXContent}{Die Sinusfunktion}{Die Sinusfunktion}{STD}
\MLabel{M06_Sinusfkt}
\MDeclareSiteUXID{VBKM06_TrigonometrischeFunktionen_Sinus}
 In Modul \MRef{VBKM05} wurden die trigonometrischen Funktionen elementar im \MRef{M05_Trigonometrie} über rechtwinklige Dreiecke 
 durch
$$\begin{array}{rcccccl}
  \sin(\alpha) & = & \Mdfrac{\text{Gegenkathete}}{\text{Hypotenuse}} \MDFPeriod & & \\[2ex]
 \end{array}$$
 
 sowie am \MSRef{VBKM05_Trigonometrie_Einheitskreis}{Einheitskreis} erklärt.
 Ausgehend von dieser Definition von $\sin(\alpha)$ gelangt man zur \MEntry{Sinusfunktion}{Sinusfunktion}, indem man den
 Winkel $\alpha$ zur Veränderlichen einer Funktion mit Namen $\sin$ macht. Man kann sich dies an Hand einer Familie von
 rechtwinkligen Dreiecken $A B C$ verdeutlichen, die dem \MEntry{Einheitskreis}{Einheitskreis}, das ist ein
 Kreis mit Radius $r = 1$, auf bestimmte Art und Weise einbeschrieben sind:
 \begin{center}
  \MUGraphicsSolo{sin_fkt_im_kreis.png}{scale=0.83}{width:400px}
 \end{center}

 
 Beginnen wir mit dem Winkel $\alpha = 0^\circ$, also einem zur Strecke entarteten Dreieck, so ist die Länge der
 Strecke $\overline{B C}$ gleich $0$. Lassen wir nun den Punkt $B$ \modstextbf{entgegen
 dem Uhrzeigersinn} um den Kreis wandern, so wächst $\alpha$ - und auch $\sin(\alpha)$ - zunächst an, bis
 für $\alpha = 90^\circ$ ein \modstextbf{maximaler Wert} ($\sin(90^\circ) = 1$) erreicht wird, bevor $\alpha$ weiter zu-,
 aber $\sin(\alpha)$ wieder abnimmt. Für $\alpha = 180^\circ$ ist das Dreieck $A B C$ wieder zur Strecke
 degeneriert, und $\sin(180^\circ) = 0$. Wird $\alpha$
 noch größer, \glqq klappt{\grqq} das Dreieck \glqq nach unten{\grqq}, die Stecke $\overline{B C}$ ist parallel zur
 negativen Hochachse ($y$-Achse) ausgerichtet, ihre Länge daher \modstextbf{negativ}. Für $\alpha = 270^\circ$ tritt der
 \modstextbf{maximal negative Wert} auf, bevor er sich wieder $0$ nähert. Bei $\alpha = 360^\circ$ beginnt das Spiel von Neuem.
 \begin{center}
  \MUGraphicsSolo{sin_fkt.png}{scale=1}{width:400px}
 \end{center}
 Das voranstehende Schaubild gibt den Graphen der Sinusfunktion,
 $$
 \begin{array}{rrcl}
  \sin : & \R & \longrightarrow & [- 1; + 1] \\ & \alpha & \longmapsto & \sin(\alpha) \MDFPSpace,
 \end{array}
 $$
 wieder. Allerdings haben wir auf der Querachse ($\alpha$-Achse) den Winkel $\alpha$ nicht - wie in der bisherigen
 Diskussion - im Gradmaß aufgetragen, sondern wir haben das in diesem Zusammenhang üblichere
 \MSRef{VBKM05_Def_Bogenmass}{Bogenmaß} verwendet.
 
 Halten wir einige der wichtigsten Eigenschaften der Sinusfunktion fest:
 \begin{itemize}
  \item{Die Sinusfunktion ist auf ganz $\R$ definiert, $D_{\sin} = \R$; der Wertebereich besteht dagegen nur aus dem
   Intervall von $- 1$ bis $+ 1$, diese beiden Endpunkte eingeschlossen: $W_{\sin} = [- 1\MIntvlSep  + 1]$}
  \item{Nach gewissen Abständen wiederholt der Graph der Sinusfunktion exakt sein Aussehen; man spricht in diesem Zusammenhang
   von der Periodizität der Sinusfunktion. Die \MEntry{Periode}{Periode (Sinus)} beträgt $360^\circ$ bzw. $2 \pi$.
   Formelmäßig kann man diesen Sachverhalt als
   $$\sin(\alpha) = \sin(\alpha + 2 \pi)$$
   ausdrücken.}
 \end{itemize}
 
 Schon die Betrachtung des Graphen der einfachen Sinusfunktion legt die Verwendung dieser Funktion für die Beschreibung von
 Wellenvorgängen nahe. Um jedoch die gesamte Leistungsfähigkeit der Sinusfunktion ausschöpfen zu
 können, werden zuvor noch einige zusätzliche Parameter eingeführt.
 So können die \glqq Ausschläge{\grqq} der Sinusfunktion mit einem sogenannten \modstextbf{Amplitudenfaktor} $A$ verstärkt
 oder abgemildert, die \glqq Schnelligkeit{\grqq} oder \glqq Dichte{\grqq} der Auf- und Abbewegungen durch einen
 \modstextbf{frequenzartigen} Faktor $a$ beeinflusst und der gesamte Verlauf des Graphen kann mit einer \modstextbf{Verschiebekonstanten}
 $b$ nach rechts oder links verrückt werden. Die \modstextbf{allgemeine Sinusfunktion} besitzt daher folgende Gestalt:
 $$\begin{array}{rrcl}
  f : & \R & \longrightarrow & [- A\MIntvlSep  + A] \\
  & x & \longmapsto & f(x) = A \, \sin (a x + b) \MDFPeriod 
 \end{array}$$
 \begin{MExample}
  \MLabel{M06_Fadenpendel}
  Beim Fadenpendel schwingt eine kleine schwere Masse im Gravitationsfeld der Erde an einem langen dünnen Faden, der z.B. fest
  an der Decke eines (hohen) Raumes verankert ist. Unter gewissen idealisierenden Annahmen und für kleine Werte des Auslenkwinkels
  $\Mvarphi$ aus der \modstextbf{Ruhelage} (der \modstextbf{Lotrechten}) hängt $\Mvarphi$ von der Veränderlichen $t$, der Zeit, über
  eine allgemeine Sinusfunktion ab:
  $$\Mvarphi (t) = A \, \sin(\Mdfrac{2 \pi}{T} t + b) \MDFPeriod$$
  Dabei bezeichnet $T$ die sogenannte \modstextbf{Schwingungsdauer} des Pendels, also diejenige Zeitspanne, die für eine
  vollständige Schwingung vom Pendel benötigt wird.
  
  
 \end{MExample}
\end{MXContent}

\begin{MXContent}{Kosinus und Tangens}{Kosinus, Tangens und Kotangens}{STD}\MLabel{sec:cos}
\MDeclareSiteUXID{VBKM06_TrigonometrischeFunktionen_CosTan}
 Im Grunde genommen müssen wir für Kosinus- und Tangensfunktion die zur Sinusfunktion analogen Überlegungen
 angehen, die wir aus dem vorigen Unterabschnitt \MRef{M06_Sinusfkt} kennen. Da wir schon etwas Übung besitzen, können wir die
 Diskussion etwas straffen. Beginnen wir mit der \MEntry{Kosinusfunktion}{Kosinusfunktion} und betrachten erneut unsere dem Einheitskreis
 einbeschriebenen Dreiecke:
 \begin{center}
  \MUGraphicsSolo{cos_fkt_im_kreis.png}{scale=1}{width:400px}
 \end{center}
 Wiederum besitzen alle Hypotenusen dieser so konstruierten rechtwinkligen Dreiecke die Länge $1$, sodass die Kosinus
 der Winkel $\alpha$ im Bild als Längen der Strecken $\overline{A C}$ auftreten. Bewegen wir wie zuvor den Punkt $B$ 
 \modstextbf{im Gegenuhrzeigersinn} gleichmäßig um den Kreis und variieren so den Winkel $\alpha$, erhalten wir letztlich die
 \modstextbf{Kosinusfunktion}:
 $$
 \begin{array}{rrcl}
  \cos : & \R & \longrightarrow & [- 1; + 1] \\ & \alpha & \longmapsto & \cos(\alpha)
 \end{array} \MDFPeriod
 $$
 \begin{center}
  \MUGraphicsSolo{cos_fkt.png}{scale=1}{width:400px}
 \end{center}
 Das Schaubild gibt neben dem Graphen der Kosinus- (durchgezogene Linie) nochmals denjenigen der Sinusfunktion (gepunktete Linie)
 zu Vergleichszwecken wieder; wir erkennen eine sehr enge Verwandtschaft, die wir noch thematisieren werden.
 
 Welche wichtigen Eigenschaften besitzt die Kosinusfunktion? 
 \begin{itemize}
  \item Die \modstextbf{Kosinusfunktion} ist ebenfalls eine \modstextbf{periodische Funktion}. Die \modstextbf{Periode} ist wieder $2 \pi$ bzw. $360^\circ$.
  \item Der Definitionsbereich der Kosinusfunktion ist ganz $\R$, $D_{\cos} = \R$, der Wertebereich das Intervall von $- 1$ bis
  $+ 1$, die Endpunkte inbegriffen, $W_{\cos} = [- 1\MIntvlSep  + 1]$.
  \item Aus dem obigen Bild der Graphen von $\cos(\alpha)$ und $\sin(\alpha)$ ergibt sich unmittelbar, dass
  $$\cos(\alpha) = \sin \left( \alpha + \Mdfrac{\pi}{2} \right)$$
  für alle reellen Werte von $\alpha$ gilt. Ebenso richtig, aber etwas schwieriger einzusehen, ist
  $$\cos(\alpha) = - \sin \left( \alpha - \Mdfrac{\pi}{2} \right) \MDFPeriod$$
 \end{itemize}
 \begin{MExercise}
  \MLabel{M06_cos_diskussion}
  An welchen Stellen nimmt die Kosinusfunktion ihren maximalen Wert $1$ an, wo ihren maximal negativen Wert $- 1$? An welchen
  Punkten besitzt sie Nullstellen (d.h. wo ist der Funktionswert gleich $0$)?
  \begin{MHint}{\iSolution}
   Es gilt $\cos(0) = 1$; aufgrund der Periodizität mit Periode $2 \pi$ trifft dies auch für $\pm 2 \pi, \pm 2 \cdot 2 \pi,
   \pm 3 \cdot 2 \pi, \dots$ zu. Also nimmt die Kosinusfunktion den maximalen Wert $1$ für alle ganzzahligen Vielfachen von $2 \pi$
   (bzw. für alle geradzahligen Vielfachen von $\pi$) an; man kann dies auch so schreiben:
   $$\cos(\alpha) = 1 \Leftrightarrow \alpha \in \{ 2 k \cdot \pi\MCondSetSep k \in \Z \} \MDFPeriod $$
   Den Wert $- 1$ erreicht die Kosinusfunktion an den Stellen $\dots, - 3 \pi, - \pi, \pi, 3 \pi, 5 \pi, \dots$, also für
   ungeradzahlige Vielfache von $\pi$:
   $$\cos(\alpha) = - 1 \Leftrightarrow \alpha \in \{ (2 k + 1) \cdot \pi\MCondSetSep k \in \Z \} \MDFPeriod $$
   Nullstellen treten für $\dots, - \Mtfrac{3}{2} \pi, - \Mtfrac{1}{2} \pi, \Mtfrac{1}{2} \pi, \Mtfrac32 \pi, \dots$, d.h. für
   halbganzzahlige Vielfache von $\pi$ auf:
   $$\cos(\alpha) = 0 \Leftrightarrow \alpha \in \{ \Mtfrac{2k+1}{2} \cdot \pi; k \in \Z \} \MDFPeriod$$
  \end{MHint}
 \end{MExercise}
 Wie im Falle des Sinus gibt es auch für den Kosinus eine \modstextbf{allgemeine Kosinusfunktion}, in deren Definition
 zusätzliche Freiheiten in Form von Parametern auftauchen (\modstextbf{Amplitudenfaktor} $B$, \modstextbf{Frequenzfaktor} $c$ sowie
 \modstextbf{Verschiebekonstante} $d$); auf diese Art und Weise eröffnet sich wiederum die Möglichkeit, den Funktionsverlauf an
 unterschiedliche Situationen (in Anwendungsbeispielen) anzupassen:
 $$
 \begin{array}{rrcl}
  g : & \R & \longrightarrow & [- A; + A] \\
  & x & \longmapsto & g(x) = B \, \cos (c x + d)  
 \end{array} \MDFPeriod
 $$
 \begin{MExercise}
  In Beispiel \MRef{M06_Fadenpendel} haben wir das Fadenpendel andiskutiert. Insbesondere kann man den zeitlichen Verlauf
  der Pendelauslenkung $\Mvarphi$ unter den Voraussetzungen bestimmen, dass die Schwingungsdauer $T$ gerade $\pi$ Sekunden beträgt,
  und dass zum Zeitpunkt $t = 0$ das Pendel bei einer Auslenkung von $30^\circ$ losgelassen wird:
  $$\Mvarphi (t) = \Mdfrac{\pi}{6} \cdot\sin \left(2 t + \Mdfrac{\pi}{2}\right) \MDFPeriod$$
  Kann man diese Situation auch mit Hilfe der (allgemeinen) Kosinusfunktion (anstelle der Sinusfunktion) beschreiben,
  und wenn ja, wie sieht dann $\Mvarphi (t)$ aus?  \begin{MHint}{\iSolution}
   Die Antwort auf die erste Frage lautet: Ja, es ist möglich, die Kosinusfunktion zur Beschreibung des vorliegenden Sachverhaltes
   heranzuziehen (wie wir sogleich sehen werden).
   
   Im Prinzip könnten wir mit der oben wiedergegebenen allgemeinen Kosinusfunktion $g$ starten und mit Überlegungen, die analog
   zu denjenigen in Beispiel \MRef{M06_Fadenpendel} verlaufen, die Parameter $B$, $c$ und $d$ im vorliegenden Fall bestimmen.
   Einfacher ist es jedoch, sich auf den Zusammenhang $\cos(\alpha) = \sin(\alpha + \Mdfrac{\pi}{2})$
   zwischen Kosinus- und Sinusfunktion zu besinnen. Denn dann folgt sofort
   $$\sin(2 t + \Mdfrac{\pi}{2}) = \cos(2 t) \MDFPSpace,$$
   und damit:
   $$\Mvarphi (t) = \Mdfrac{\pi}{6}\cdot\cos(2 t) \MDFPeriod$$
   
  \end{MHint}
 \end{MExercise}
 
 Der Tangens ist gegeben als das Verhältnis von Sinus zu Kosinus. Damit ist sofort klar, dass die \MEntry{Tangensfunktion}{Tangensfunktion}
 nicht auf allen reellen Zahlen definiert sein kann, denn schließlich besitzt die Kosinusfunktion unendliche viele
 Nullstellen, wie man z.B. in Aufgabe \MRef{M06_cos_diskussion} sehen kann.
 In Aufgabe \MRef{M06_cos_diskussion} wird auch die Lage der Nullstellen von $\cos$ bestimmt ($\cos(\alpha) = 0
 \Leftrightarrow \alpha \in \{ \Mtfrac{2k+1}{2} \cdot \pi; k \in \Z \}$); demzufolge ist der Definitionsbereich der
 Tangensfunktion $D_{\tan} = \R \setminus \{ \Mtfrac{2k+1}{2} \cdot \pi; k \in \Z \}$.
 
 Und wie sieht es mit dem Wertebereich aus? Bei den
 $\cos$-Nullstellen wird die Tangensfunktion gegen positiv bzw.
 negativ unendliche Werte streben und Polstellen haben und bei den $\sin$-Nullstellen wird
 $\sin/\cos$ Null. 
 Dazwischen sind alle reellen Werte möglich, daher ist $W_{\tan} = \R$. Insgesamt ergibt sich für den Graphen
 der Tangensfunktion
 $$
 \begin{array}{rrcl}
  \tan : & \R \setminus \{ \Mtfrac{2k+1}{2} \cdot \pi; k \in \Z \} & \longrightarrow & \R \\
  & \alpha & \longmapsto & \tan(\alpha)
 \end{array}
 $$
 folgendes Bild:
 \begin{center}
  \MUGraphicsSolo{tan_fkt.png}{scale=1}{width:400px}
 \end{center}
 Die Tangensfunktion verläuft zudem periodisch, allerdings mit der Periode $\pi$ bzw. $180^\circ$.
 
 \begin{MExercise}
 Der sogenannte Kotangens (Abkürzung $\cot$) ist definiert durch
 $\cot (\alpha) = \frac1{\tan(\alpha)}= \Mdfrac{\cos(\alpha)}{\sin(\alpha)}$.
 
  Geben Sie Definitions- und Wertebereich der Kotangensfunktion an!
  \begin{MHint}{\iSolution}
   Die Polstellen des Kotangens liegen dort, wo der Sinus $0$ wird, und das ist genau dann der Fall, wenn $\alpha$ ein ganzzahliges Vielfaches
   von $\pi$ ist. Daher müssen wir bei der Definition der Kotangensfunktion genau diese Punkte ausschließen:
   $$D_{\cot} = \R \setminus \{ k \cdot \pi \MCondSetSep k \in \Z \} \MDFPeriod$$
   Zur Bestimmung des Wertebereichs können Betrachtungen durchgeführt werden, die denjenigen beim Tangens stark ähneln; man findet
   $W_{\cot} = \R$.
   \begin{center}
    \MUGraphicsSolo{cot_fkt.png}{scale=1}{width:400px}
   \end{center}
  \end{MHint}
 \end{MExercise}
\end{MXContent}


\MSubsection{Eigenschaften und Konstruktion elementarer Funktionen}
\MLabel{VBKM06_eigenschaften}

\begin{MIntro}
\MDeclareSiteUXID{VBKM06_Eigenschaften_Intro}
In diesem Abschnitt werden wir eine weitere Eigenschaft elementarer Funktionen betrachten, die in den vorhergehenden Abschnitten noch nicht behandelt wurde.
Dies ist die Symmetrie von Funktionen. Weiterhin werden Möglichkeiten untersucht, wie aus den nun bekannten elementaren Funktionen neue konstruiert werden können. Dazu führt man unter anderem Summen, Produkte und Verkettungen von Funktionen ein. 
\end{MIntro}


\begin{MXContent}{Symmetrie}{Symmetrie}{STD}
\MDeclareSiteUXID{VBKM06_Eigenschaften_Symmetrie}

\begin{MInfo}
Eine Funktion $f\colon \R\lto\R$ heißt \highlight{gerade} oder \highlight{achsensymmetrisch}, falls für alle $x\in\R$
\[
 f(x)=f(-x)
\]
gilt. Analog heißt die Funktion \highlight{ungerade} oder \highlight{punktsymmetrisch}, falls für alle $x\in\R$
\[
 f(x)=-f(-x)
\]
gilt.
\end{MInfo}

Diese beiden Symmetriebedingungen für Funktionen sagen also etwas über das Aussehen ihrer Graphen aus. Bei geraden Funktionen ändert sich der Graph bei Spiegelung an der Hochachse nicht, und bei ungeraden Funktionen ändert sich der Graph bei Spiegelung am Ursprung nicht. Wir listen einige Beispiele auf.

\begin{MExample}
\begin{itemize}
 \item Die Funktionen
 \[
  \function{f_1}{\R}{\R}{x}{x^2}
 \]
 und
 \[
  \function{f_2}{\R}{\R}{x}{|x|\MDFPSpace,}
 \]
 also die Standardparabel (vgl.~Abschnitt \MNRef{sec:monome}) und die Betragsfunktion (vgl.~Abschnitt \MNRef{VBKM06_sec:betrag}), sind Beispiele für gerade Funktionen. Es gilt $f_1(-x)=(-x)^2=x^2=f_1(x)$ und $f_2(-x)=|-x|=|x|=f_2(x)$ für alle $x\in\R$. Die Graphen weisen die Spiegelsymmetrie an der Hochachse auf:
 
%GRAPH: sym1
\MTikzAuto{%
\begin{tikzpicture} 
%Koordinatensystem
\node (xMAX) at (2.8,0){};
\node (yMAX) at (0,4.8){};
\draw[->,color=black] (-2.5,0) -- (xMAX);
\foreach \x in {-2,-1,1,2}
\draw[shift={(\x,0)},color=black] (0pt,2pt) -- (0pt,-2pt) node[below] {\footnotesize $\x$};
\draw[->,color=black] (0,-0.5) -- (yMAX);
\foreach \y in {1,2,3,4}
\draw[shift={(0,\y)},color=black] (2pt,0pt) -- (-2pt,0pt) node[left] {\footnotesize $\y$};
\draw[color=black] (0pt,-10pt) node[right] {\footnotesize $0$};
%Achsenbeschriftung
\draw (xMAX) node[anchor=north east] {$x$};
\draw (yMAX) node[anchor=east] {$f_1(x),f_2(x)$};
% Graphen
\draw[color=red,smooth,samples=50,domain=-2.1:2.1] plot(\x,{(\x)^(2.0)});
\draw[color=red] (2.1,4.41) node[anchor=south] {$G_{f_1}$};
\draw[color=blue] (0,0) -- (2,2);
\draw[color=blue] (0,0) -- (-2,2);
\draw[color=blue] (-2.1,2.1) node[anchor=south] {$G_{f_2}$};
\end{tikzpicture}  
}%

 \item Die Funktion
 \[
  \function{g}{\R}{\R}{x}{x^3 \MDFPSpace,}
 \]

 also die kubische Parabel (vgl.~Abschnitt \MNRef{sec:monome}), ist ein Beispiel für eine ungerade Funktion. Es gilt $g(-x)=(-x)^3=-x^3=-g(x)$ für alle $x\in\R$. Der Graph ist punktsymmetrisch bezüglich des Ursprungs:

%GRAPH: sym2
\MTikzAuto{%
\begin{tikzpicture} 
%Koordinatensystem
\node (xMAX) at (2.8,0){};
\node (yMAX) at (0,4.8){};
\draw[->,color=black] (-2.5,0) -- (xMAX);
\foreach \x in {-2,-1,1,2}
\draw[shift={(\x,0)},color=black] (0pt,2pt) -- (0pt,-2pt) node[below] {\footnotesize $\x$};
\draw[->,color=black] (0,-3.5) -- (yMAX);
\foreach \y in {-3,-2,-1,1,2,3,4}
\draw[shift={(0,\y)},color=black] (2pt,0pt) -- (-2pt,0pt) node[left] {\footnotesize $\y$};
\draw[color=black] (0pt,-10pt) node[right] {\footnotesize $0$};
%Achsenbeschriftung
\draw (xMAX) node[anchor=north east] {$x$};
\draw (yMAX) node[anchor=east] {$g$};
\clip(-2.5,-3.5) rectangle (2.5,4.5);
% Graphen
\draw[color = red, smooth,samples=50,domain=-1.9:1.9] plot(\x,{(\x)^(3.0)}); \draw[color = red](-1.4,-3)node[left]{$G_{g}$};
%\draw[color=blue] (0,1) -- (2,1);
%\draw[color=blue] (0,-1) -- (-2,-1);
%\draw[color=blue, fill=blue] (0,0) circle (1.5pt);
%\draw[color=blue, fill=white] (0,1) circle (1.5pt);
%\draw[color=blue, fill=white] (0,-1) circle (1.5pt);
%\draw[color=blue] (2,1) node[anchor=south] {$G_{g_2}$};
\end{tikzpicture} 
}%

 

\end{itemize}
\end{MExample}

Natürlich sind die Symmetrieeigenschaften von Funktionen auch benutzbar, wenn der Definitionsbereich der Funktion nicht die gesamten reellen Zahlen umfasst. Es muss dann aber eine Definitionsmenge vorliegen, die die $0$ \modsemph{in der Mitte des Intervalls} enthält. Ein Beispiel dafür ist die Tangens-Funktion in der Aufgabe unten. 

\begin{MExercise}
Geben Sie von den folgenden Funktionen jeweils an, ob diese gerade, ungerade oder nicht-symmetrisch sind.
\begin{itemize}
 \item[a)] \[\function{f}{\R}{\R}{x}{\MEU^x}\]
 \item[b)] \[\function{g}{\R}{\R}{y}{\sin(y)}\]
 \item[c)] \[\function{h}{(-\frac{\pi}{2}\MIntvlSep \frac{\pi}{2})}{\R}{\alpha}{\tan(\alpha)}\]
 \item[d)] \[\function{i}{\R}{\R}{u}{\cos(u)}\]
 \item[e)] \[\function{j}{\R}{\R}{x}{42}\]
\end{itemize}
\begin{MHint}{\iSolution}
a) nicht-symmetrisch, b) ungerade, c) ungerade, d) gerade, e) gerade 
\end{MHint}

\end{MExercise}

\end{MXContent}



\begin{MXContent}{Summen, Produkte, Verkettungen}{Summen, Produkte, Verkettungen}{STD}\MLabel{Verkettung}
\MDeclareSiteUXID{VBKM06_SummeProduktVerkettung}
In diesem Abschnitt wollen wir nun das große Sortiment an elementaren Funktionen, die wir uns in diesem Modul erarbeitet haben, nutzen, um neue komplexere Funktionen aus den elementaren zu konsturieren. An verschiedenen Stellen im Verlauf dieses Moduls haben wir bereits Funktionen untersucht, deren Abbildungsvorschriften durch Summen- oder Produktbildung aus einfacheren Abbildungsvorschriften zusammengesetzt sind. Man kann natürlich auch Differenzen und unter bestimmten Umständen Quotienten von Abbildungsvorschriften bilden. Das folgende Beispiel stellt nochmal einige solche zusammengesetzte Funktionen zusammen.

\begin{MExample}
\begin{itemize}
 \item Die Funktion
 \[
  \function{f}{\R}{\R}{x}{x+\sin(x)}
 \]
 ist die \modsemph{Summe} aus der Identität (vgl.~Abschnitt \MNRef{sec:linear}) und der Sinusfunktion (vgl.~Abschnitt \MNRef{VBKM06_trigonometrisch}). Sie besitzt den folgenden Graphen:



 %GRAPH: summ1
\MTikzAuto{%
\begin{tikzpicture}[scale=0.5]
%Koordinatensystem
% x-Achse
\node (xMAX) at (8,0){};
\draw[->,color=black] (-7.5,0) -- (xMAX);
\foreach \x in {-7,-6,-5,-4,-3,-2,-1,1,2,3,4,5,6,7}
\draw[shift={(\x,0)},color=black] (0pt,2pt) -- (0pt,-2pt) node[below] {\footnotesize $\x$};
% y-Achse
\node (yMAX) at (0,7.8){};
\draw[->,color=black] (0,-7.5) -- (yMAX);
\foreach \y in {-7,-6,-5,-4,-3,-2,-1,1,2,3,4,5,6,7}
\draw[shift={(0,\y)},color=black] (2pt,0pt) -- (-2pt,0pt) node[left] {\footnotesize $\y$};
\draw[color=black] (0pt,-10pt) node[right] {\footnotesize $0$};
%Achsenbeschriftung
\draw (xMAX) node[anchor=north east] {$x$};
\draw (yMAX) node[anchor=east] {$f(x)$};
%Graph
\draw[color=red,samples=100,domain=-6.2:6.2] plot(\x,{\x + sin(\x r)}); 
\draw[color=red] (2,2) node[anchor=north] {$G_f$};
\end{tikzpicture} 
}%

 \item Die Funktion
 \[
  \function{g}{[1\MIntvlSep 2]}{\R}{y}{y^2-\ln(y)}
 \]
 ist die \modsemph{Differenz} aus der Standardparabel (vgl.~Abschnitt \MNRef{sec:monome}) und der natürlichen Logarithmusfunktion (vgl.~Abschnitt \MNRef{M06_Logarithmus}). Sie besitzt den folgenden Graphen:
  %GRAPH: summ2
\MTikzAuto{%
\begin{tikzpicture}
%Koordinatensystem
% x-Achse
\node (xMAX) at (3.8,0){};
\draw[->,color=black] (-0.5,0) -- (xMAX);
\foreach \x in {1,2,3}
\draw[shift={(\x,0)},color=black] (0pt,2pt) -- (0pt,-2pt) node[below] {\footnotesize $\x$};
% y-Achse
\node (yMAX) at (0,4.8){};
\draw[->,color=black] (0,-0.5) -- (yMAX);
\foreach \y in {1,2,3,4}
\draw[shift={(0,\y)},color=black] (2pt,0pt) -- (-2pt,0pt) node[left] {\footnotesize $\y$};
\draw[color=black] (0pt,-10pt) node[right] {\footnotesize $0$};
%Achsenbeschriftung
\draw (xMAX) node[anchor=north east] {$y$};
\draw (yMAX) node[anchor=east] {$g(y)$};
%Graph
\draw[color=red,samples=50,domain=1:2] plot(\x,{\x*\x - ln(\x)}); 
\draw[color=red,fill=red] (1,1) circle (1pt);
\draw[color=red,fill=red] (2,3.3) circle (1pt);
\draw[color=red] (.8,1) node[anchor=south] {$G_g$};
\end{tikzpicture} 
}%

 \item Die Funktion 
 \[
  \function{h}{(0\MIntvlSep \infty)}{\R}{x}{\MEU^x\frac{1}{x}}
 \]
 ist das \modsemph{Produkt} aus der natürlichen Exponentialfunktion mit Abbildungsvorschrift $\MEU^x$ (vgl.~Abschnitt \MNRef{M06_e_fkt}) und der Hyperbel mit Abbildungvorschrift $\frac{1}{x}$ (vgl.~Abschnitt \MNRef{sec:hyperbel}). Sie besitzt folgenden Graphen:
 
   %GRAPH: summ3
\MTikzAuto{%
\begin{tikzpicture}
%Koordinatensystem
% x-Achse
\node (xMAX) at (3.8,0){};
\draw[->,color=black] (-0.5,0) -- (xMAX);
\foreach \x in {1,2,3}
\draw[shift={(\x,0)},color=black] (0pt,2pt) -- (0pt,-2pt) node[below] {\footnotesize $\x$};
% y-Achse
\node (yMAX) at (0,7.8){};
\draw[->,color=black] (0,-0.5) -- (yMAX);
\foreach \y in {1,2,3,4,5,6,7}
\draw[shift={(0,\y)},color=black] (2pt,0pt) -- (-2pt,0pt) node[left] {\footnotesize $\y$};
\draw[color=black] (0pt,-10pt) node[right] {\footnotesize $0$};
%Achsenbeschriftung
\draw (xMAX) node[anchor=north east] {$x$};
\draw (yMAX) node[anchor=east] {$h(x)$};
%Graph
\draw[color=red,samples=100,domain=0.2:3] plot(\x,{exp(\x)/\x}); 
\draw[color=red] (3.1,6) node[anchor=north] {$G_h$};
\end{tikzpicture} 
}%

 \item Die Funktion
 \[
  \function{\Mvarphi}{\R}{\R}{z}{\frac{cos(z)}{z^2+1}}
 \]
 ist der \modsemph{Quotient} aus der Cosinusfunktion (vgl.~Abschnitt \MNRef{sec:cos}) und dem Polynom zweiten Grades (vgl.~Abschnitt \MNRef{sec:polynome}) mit der Abbildungsvorschrift $z^2+1$. Sie besitzt folgenden Graphen:
 
  %GRAPH: summ4
\MTikzAuto{%
\begin{tikzpicture}[y=3cm]
%Koordinatensystem
% x-Achse
\node (xMAX) at (8,0){};
\draw[->,color=black] (-7.5,0) -- (xMAX);
\foreach \x in {-7,-6,-5,-4,-3,-2,-1,1,2,3,4,5,6,7}
\draw[shift={(\x,0)},color=black] (0pt,2pt) -- (0pt,-2pt) node[below] {\footnotesize $\x$};
% y-Achse
\node (yMAX) at (0,1.8){};
\draw[->,color=black] (0,-1.5) -- (yMAX);
\foreach \y in {-1,1}
\draw[shift={(0,\y)},color=black] (2pt,0pt) -- (-2pt,0pt) node[left] {\footnotesize $\y$};
\draw[color=black] (0pt,-10pt) node[right] {\footnotesize $0$};
%Achsenbeschriftung
\draw (xMAX) node[anchor=north east] {$z$};
\draw (yMAX) node[anchor=east] {$\Mvarphi(z)$};
%Graph
\draw[color=red,samples=100,domain=-7:7] plot(\x,{cos(\x r)/(\x*\x + 1)}); 
\draw[color=red] (1,0.7) node[anchor=south] {$G_{\Mvarphi}$};
\end{tikzpicture}
}%
\end{itemize}
 
\end{MExample}


\begin{MExercise}
Finden Sie weitere Beispiele in diesem Modul für bereits behandelte elementare Funktionen, die mittels Summen-, Differenz-, Produkt- oder Quotientenbildung aus einfacheren elementaren Funktionen hervorgehen. 

\begin{MHint}{\iSolution}
Zum Beispiel:
\begin{itemize}
 \item Die Funktionen vom hyperbolischen Typ (vgl.~Abschnitt \MNRef{sec:hyperbel}) sind alle Quotienten aus der konstanten Funktion $1$ und einem Monom.
 \item Die Monome (vgl.~Abschnitt \MNRef{sec:monome}) sind alle mehrfache Produkte aus der Identität $\Id(x)=x$.
 \item Die linearen Funktionen (vgl.~Abschnitt \MNRef{sec:linear}) sind Produkte aus konstanten Funktionen, die die Steigung beschreiben, und der Identität.
 \item Alle Polynome (vgl.~Abschnitt \MNRef{sec:polynome}) sind Summen und Differenzen von Funktionen, die ihrerseits Produkte aus konstanten Funktionen und Monomen sind.
\end{itemize}
 
\end{MHint}

\end{MExercise}


Zuletzt gibt es noch eine weitere Art, elementare Funktionen zu verknüpfen um neue Funktionen zu erhalten. Dies ist die sogenannte \highlight{Verkettung} oder \highlight{Komposition} von Funktionen. 

Wir betrachten dazu einige Beispiele.

\begin{MExample}
\begin{itemize}
 \item Die Funktionen 
 \[
  \function{f}{\R}{\R}{x}{f(x)=x^2+1}
 \]
 und
 \[
  \function{g}{\R}{\R}{x}{g(x)=\MEU^x}
 \]
 lassen sich auf zweierlei Art verketten. Wir können die Funktion $f\circ g\colon\R\lto\R$ oder die Funktion $g\circ f\colon \R\lto\R$ bilden. Wir erhalten 
 \[
  (f\circ g)(x)=f(g(x))=f(\MEU^x)=(\MEU^x)^2+1=\MEU^{2x}+1 \MDFPSpace,
 \]
 also
 \[
  \function{f\circ g}{\R}{\R}{x}{\MEU^{2x}+1}
 \]
 und
 \[
  (g\circ f)(x)=g(f(x))=g(x^2+1)=\MEU^{x^2+1} \MDFPSpace,
 \]
 also
 \[
  \function{g\circ f}{\R}{\R}{x}{\MEU^{x^2+1} \MDFPeriod}
 \]
 Anhand der Graphen sehen wir, dass dies zwei völlig unterschiedliche Funktionen sind. Es kommt also auf die Reihenfolge der Verkettung an.
 
   %GRAPH: kett1
\MTikzAuto{%
\begin{tikzpicture}[x=2cm,y=0.5cm]
%Koordinatensystem
% x-Achse
\node (xMAX) at (1.8,0){};
\draw[->,color=black] (-1.5,0) -- (xMAX);
\foreach \x in {-1,1}
\draw[shift={(\x,0)},color=black] (0pt,2pt) -- (0pt,-2pt) node[below] {\footnotesize $\x$};
% y-Achse
\node (yMAX) at (0,8.8){};
\draw[->,color=black] (0,-0.5) -- (yMAX);
\foreach \y in {1,2,3,4,5,6,7,8}
\draw[shift={(0,\y)},color=black] (2pt,0pt) -- (-2pt,0pt) node[left] {\footnotesize $\y$};
\draw[color=black] (0pt,-10pt) node[right] {\footnotesize $0$};
%Achsenbeschriftung
\draw (xMAX) node[anchor=north east] {$x$};
\draw (yMAX) node[anchor=east] {$(f\circ g)(x),(g\circ f)(x)$};
%Graph
\draw[color=red,samples=50,domain=-1:1] plot(\x,{exp(\x*\x + 1)}); 
\draw[color=blue,samples=50,domain=-1:1] plot(\x,{exp(2*\x) + 1}); 
\draw[color=red] (-0.7,4) node[anchor=east] {$G_{g\circ f}$};
\draw[color=blue] (-1,1) node[anchor=east] {$G_{f\circ g}$};
\end{tikzpicture}
}%

 \item Bei zwei Funktionen wie 
 \[
  \function{h}{\R}{\R}{x}{\sin(x)}
 \]
 und 
 \[
  \function{w}{[0\MIntvlSep \infty)}{\R}{x}{\sqrt{x}}
 \]
 ist allerdings auf die Definitionsbereiche bei der Verkettung zu achten. Denn wollen wir etwa die verkettete Funktion $w\circ h$ betrachten, so gilt
 \[
  (w\circ h)(x)=w(h(x))=w(\sin(x))=\sqrt{\sin(x)} \MDFPeriod
 \]
 Da die Funktionswerte des Sinus aber auch negativ werden können, man aber in die Quadratwurzelfunktion nur nicht-negative Werte einsetzen darf, muss also der Definitionsbereich der Sinusfunktion entsprechend eingeschränkt werden, so dass sich nicht-negative Werte ergeben, zum Beispiel mittels $x\in[0\MIntvlSep \pi]=D_{w\circ h}$. Wir erhalten also
 \[
  \function{w\circ h}{[0\MIntvlSep \pi]}{\R}{x}{\sqrt{\sin(x)} \MDFPeriod}
 \]
\end{itemize}
\end{MExample}


\begin{MExercise}
Gegeben sind die Funktionen
\[
 \function{f}{\R}{\R}{x}{2x-3 \MDFPSpace}
\]
\[
 \function{g}{\R\setminus\{0\}}{\R}{x}{\frac{1}{x}}
\]

und
\[
 \function{h}{\R}{\R}{x}{\sin(x) \MDFPeriod}
\]
Bestimmen Sie die Verkettungen $f\circ g$, $g\circ f$, $h\circ f$, $h\circ g$, $f\circ f$ und $g\circ g$. Schränken Sie dazu eventuell die Definitionsbereiche so ein, dass die Verkettung zulässig ist. Benutzen Sie jedoch für die verketteten Funktion stets die größtmöglichen Definitionsbereiche.

\begin{MHint}{\iSolution}
\[
 \function{f\circ g}{\R\setminus\{0\}}{\R}{x}{\frac{2}{x}-3}
\]
\[
 \function{g\circ f}{\R\setminus\{\frac{3}{2}\}}{\R}{x}{\frac{1}{2x-3}}
\]
\[
 \function{h\circ f}{\R}{\R}{x}{\sin(2x-3)}
\]
\[
 \function{h\circ g}{\R\setminus\{0\}}{\R}{x}{\sin(\frac{1}{x})}
\] 
\[
 \function{f\circ f}{\R}{\R}{x}{4x-9}
\]
\[
 \function{g\circ g}{\R}{\R}{x}{x}
\]
\end{MHint}

\end{MExercise}


\end{MXContent}


\MSubsection{Abschlusstest}
\MLabel{M06_Abschlusstest}

\begin{MTest}{Abschlusstest zu Modul \arabic{section}}
\MDeclareSiteUXID{VBKM06_Abschlusstest}

\begin{MExercise}
Bestimmen Sie für die beiden Funktionen
\[
 \function{f}{D_f}{\R}{x}{\frac{9x^2-\sin(x)+42}{x^2-2}}
\]
und
\[
 \function{g}{D_g}{\R}{y}{\frac{\ln(y)}{y^2+1}}
\]
jeweils den größtmöglichen Definitionsbereich $D_f$ bzw.~$D_g$.
% Intervallquestion
\end{MExercise}

\begin{MExercise}
Bestimmen Sie für die Funktion
\[
 \function{i}{\R}{\R}{x}{x^2-4x+4+\pi}
\]
die Wertemenge $W_i$.
\end{MExercise}

\begin{MExercise}
Bestimmen Sie in der Exponentialfunktion
\[
 \function{c}{\R}{\R}{x}{A\cdot \MEU^{\lambda x}-1}
\]
die Parameter $A,\lambda\in\R$, so dass $c(0)=1$ und $c(4)=0$ gilt.
\ \\ \ \\
Antwort: \MEquationItem{$A$}{\MLParsedQuestion{10}{2}{3}{EL1}}, \MEquationItem{$\lambda$}{\MLParsedQuestion{15}{-ln(2)/4}{3}{EL2}}.
\ \\
\MInputHint{Einfache Logarithmen können Sie stehen lassen, z.B. kann $\ln(100)$ als \texttt{ln(100)} eingegeben werden auch wenn der exakte Wert von $ln(100)$ nicht bekannt ist.}
\end{MExercise}

\begin{MExercise}
Bestimmen Sie die Verkettung $h=f\circ g\colon\R\to\R$ (Erläuterung: $h(x)=(f\circ g)(x)=f(g(x))$) der Funktionen
\[
 \function{f}{\R}{\R}{x}{C\cdot\sin(x)}
\]
und
\[
 \function{g}{\R}{\R}{x}{B\cdot x+\pi \MDFPeriod}
\]
Antwort: \MEquationItem{$h(x)$}{\MLFunctionQuestion{20}{C*sin(B*x+pi)}{10}{x,B,C}{4}{EL3}}.
\ \\ \ \\
Bestimmem Sie die Parameter, so dass die durch $h$ beschriebene Sinusschwingung diesen Graph besitzt:
 \begin{center}
 \MUGraphics{sinusfrage.png}{width=0.4\linewidth}{Eine Sinusschwingung.}{width:300px}
 %\MCopyrightNotice{\MCCLicense}{NONE}{MINT}{Funktionsplot erstellt und exportiert mit Maple15}{VBKM06_Abbildung_Sinusfrage}
 \end{center}
\ \\ \ \\
Antwort: \MEquationItem{$h(x)$}{\MLFunctionQuestion{20}{3*sin(1/2*pi*x+pi)}{10}{x}{4}{EL4}}.
\end{MExercise}


\begin{MExercise}
Bestimmen Sie die Umkehrfunktion $f=u^{-1}$ von
\[
 \function{u}{(0\MIntvlSep \infty)}{\R}{y}{-\log_2(y) \MDFPeriod}
\]
Die Funktion $f=u^{-1}$ besitzt:
\begin{MExerciseItems}
\item{Den Definitionsbereich \MEquationItem{$D_f$}{\MLIntervalQuestion{20}{(-infty,infty)}{3}{TEID1}}.}
\item{Den Wertebereich \MEquationItem{$W_f$}{\MLIntervalQuestion{20}{(0,infty)}{3}{TEID2}}.}
\item{Die Funktionsvorschrift \MEquationItem{$f(y)=u^{-1}(y)$}{\MLSimplifyQuestion{13}{2^(-y)}{5}{y}{5}{512}{TEID3}}.}
\end{MExerciseItems}
\MInputHint{Geben Sie die Bereiche als Intervalle in der Form \texttt{(a;b)} an, auch \texttt{unendlich} ist als Grenze möglich.}
\end{MExercise}

\begin{MExercise}
Kreuzen Sie an, ob die Aussagen wahr oder falsch sind:
\ \\ \ \\

Die Funktion
\[
 \function{f}{[0\MIntvlSep 3)}{\R}{x}{2x+1}
\]

% Safari kann Checkboxes in lists nicht anzeigen
\begin{tabular}{lll}
\MLCheckbox{0}{EF1} & \ \ & ... kann man kürzer auch als $f(x)=2x+1$ schreiben.\\
\MLCheckbox{1}{EF2} & \ \ & ... ist eine affin-lineare Funktion.\\
\MLCheckbox{0}{EF3} & \ \ & ... hat die Wertemenge $\R$.\\
\MLCheckbox{1}{EF4} & \ \ & ... hat die Steigung $2$.\\
\MLCheckbox{1}{EF5} & \ \ & ... kann nur Werte größer oder gleich $1$ und kleiner $7$ annehmen.\\
\MLCheckbox{1}{EF6} & \ \ & ... hat als Graph ein Stück einer Gerade.\\
\MLCheckbox{1}{EF7} & \ \ & ... hat bei $x=0$ den Wert $1$.\\
\MLCheckbox{0}{EF8} & \ \ & ... hat die Definitionsmenge $\R$.
\end{tabular}
\end{MExercise}

\begin{MExercise}
Bestimmen Sie diese Logarithmen:
\begin{MExerciseItems}
\item{\MEquationItem{$\ln(\MEU^5\cdot\frac1{\sqrt{\MEU}})$}{\MLSimplifyQuestion{10}{9/2}{5}{}{5}{1}{TLG0}}.}
\item{\MEquationItem{$\log_{10}(\MZahl{0}{01})$}{\MLSimplifyQuestion{10}{-2}{5}{}{5}{1}{TLG1}}.}
\item{\MEquationItem{$\log_2(\sqrt{2\cdot 4\cdot 16\cdot 256\cdot 1024})$}{\MLSimplifyQuestion{10}{25/2}{5}{}{5}{1}{TLG2}}.}
\end{MExerciseItems}
\end{MExercise}
\end{MTest}


\newpage
\MPrintIndex

\end{document}
