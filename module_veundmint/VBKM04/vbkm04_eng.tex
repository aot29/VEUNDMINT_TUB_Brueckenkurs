
% ENCODING: ASCII (important as must be compatible both with utf8 and latin1 based authors)
% MINTMOD Version P0.1.0, needs to be consistent with preprocesser object in tex2x and MPragma-Version at the end of this file

% Parameter aus Konvertierungsprozess (PDF und HTML-Erzeugung wenn vom Konverter aus gestartet) werden hier eingefuegt, Preambleincludes werden am Schluss angehaengt

\newif\ifttm                % gesetzt falls Uebersetzung in HTML stattfindet, sonst uebersetzung in PDF

% Wahl der Notationsvariante ist im PDF immer std, in der HTML-Uebersetzung wird vom Konverter die Auswahl modifiziert
\newif\ifvariantstd
\newif\ifvariantunotation
\variantstdtrue % this string was added by tex2x VEUNDMINT preprocessor
 % Diese Zeile wird vom Konverter erkannt und ggf. modifiziert, daher nicht veraendern!


\def\MOutputDVI{1}
\def\MOutputPDF{2}
\def\MOutputHTML{3}
\newcounter{MOutput}

\ifttm
%\usepackage{german}
\usepackage{array}
\usepackage{amsmath}
\usepackage{amssymb}
\usepackage{amsthm}
\else
\documentclass[oneside]{scrbook}
\usepackage{etex}
\usepackage[utf8]{inputenc}
\usepackage{textcomp}
\usepackage[ngerman,english]{babel}
\usepackage[pdftex]{color}
\usepackage{xcolor}
\usepackage{graphicx}
\usepackage[all]{xy}
\usepackage{fancyhdr}
\usepackage{verbatim}
\usepackage{array}
\usepackage{float}
\usepackage{makeidx}
\usepackage{amsmath}
\usepackage{amstext}
\usepackage{amssymb}
\usepackage{amsthm}
\usepackage[ngerman,english]{varioref}
\usepackage{framed}
\usepackage{supertabular}
\usepackage{longtable}
%\usepackage{maxpage}
\usepackage{tikz}
\usepackage{tikzscale}
\usepackage{tikz-3dplot}
\usepackage{bibgerm}
\usepackage{chemarrow}
\usepackage{polynom}
%\usepackage{draftwatermark}
\usepackage{pdflscape}
\usetikzlibrary{calc}
\usetikzlibrary{through}
\usetikzlibrary{shapes.geometric}
\usetikzlibrary{arrows}
\usetikzlibrary{intersections}
\usetikzlibrary{decorations.pathmorphing}
\usetikzlibrary{external}
\usetikzlibrary{patterns}
\usetikzlibrary{fadings}
\usepackage[colorlinks=true,linkcolor=blue]{hyperref}
\usepackage[all]{hypcap}
%\usepackage[colorlinks=true,linkcolor=blue,bookmarksopen=true]{hyperref}
\usepackage{ifpdf}

\usepackage{movie15}

\setcounter{tocdepth}{2} % In Inhaltsverzeichnis bis subsection
\setcounter{secnumdepth}{3} % Nummeriert bis subsubsection

\setlength{\LTpost}{0pt} % Fuer longtable
\setlength{\parindent}{0pt}
\setlength{\parskip}{8pt}
%\setlength{\parskip}{9pt plus 2pt minus 1pt}
\setlength{\abovecaptionskip}{-0.25ex}
\setlength{\belowcaptionskip}{-0.25ex}
\fi

\ifttm
\newcommand{\MDebugMessage}[1]{\special{html:<!-- debugprint;;}#1\special{html:; //-->}}
\else
%\newcommand{\MDebugMessage}[1]{\immediate\write\mintlog{#1}}
\newcommand{\MDebugMessage}[1]{}
\fi

\def\MPageHeaderDef{%
\pagestyle{fancy}%
\fancyhead[r]{(C) VE\&MINT-Project}
\fancyfoot[c]{\thepage\\--- CCL BY-SA 3.0 ---}
}


\ifttm%
\def\MRelax{}%
\else%
\def\MRelax{\relax}%
\fi%

%--------------------------- Uebernahme von speziellen XML-Versionen einiger LaTeX-Kommandos aus xmlbefehle.tex vom alten Kasseler Konverter ---------------

\newcommand{\MSep}{\left\|{\phantom{\frac1g}}\right.}

\newcommand{\ML}{L}

\newcommand{\MGGT}{\mathrm{gcd}}

\ifttm
% Verhindert dass die subsection-nummer doppelt in der toccaption auftaucht (sollte ggf. in toccaption gefixt werden so dass diese Ueberschreibung nicht notwendig ist)
\renewcommand{\thesubsection}{}
% Kommandos die ttm nicht kennt
\newcommand{\binomial}[2]{{#1 \choose #2}} %  Binomialkoeffizienten
\newcommand{\eur}{\begin{html}&euro;\end{html}}
\newcommand{\square}{\begin{html}&square;\end{html}}
\newcommand{\glqq}{"'}  \newcommand{\grqq}{"'}
\newcommand{\nRightarrow}{\special{html: &nrArr; }}
\newcommand{\nmid}{\special{html: &nmid; }}
\newcommand{\nparallel}{\begin{html}&nparallel;\end{html}}
\newcommand{\mapstoo}{\begin{html}<mo>&map;</mo>\end{html}}

% Schnitt und Vereinigungssymbole von Mengen haben zu kleine Abstaende; korrigiert:
\newcommand{\ccup}{\,\!\cup\,\!}
\newcommand{\ccap}{\,\!\cap\,\!}


% Umsetzung von mathbb im HTML
\renewcommand{\mathbb}[1]{\begin{html}<mo>&#1opf;</mo>\end{html}}
\fi

%---------------------- Strukturierung ----------------------------------------------------------------------------------------------------------------------

%---------------------- Kapselung des sectioning findet auf drei Ebenen statt:
% 1. Die LateX-Befehl
% 2. Die D-Versionen der Befehle, die nur die Grade der Abschnitte umhaengen falls notwendig
% 3. Die M-Versionen der Befehle, die zusaetzliche Formatierungen vornehmen, Skripten starten und das HTML codieren
% Im Modultext duerfen nur die M-Befehle verwendet werden!

\ifttm

  \def\Dsubsubsubsection#1{\subsubsubsection{#1}}
  \def\Dsubsubsection#1{\subsubsection{#1}\addtocounter{subsubsection}{1}} % ttm-Fehler korrigieren
  \def\Dsubsection#1{\subsection{#1}}
  \def\Dsection#1{\section{#1}} % Im HTML wird nur der Sektionstitel gegeben
  \def\Dchapter#1{\chapter{#1}}
  \def\Dsubsubsubsectionx#1{\subsubsubsection*{#1}}
  \def\Dsubsubsectionx#1{\subsubsection*{#1}}
  \def\Dsubsectionx#1{\subsection*{#1}}
  \def\Dsectionx#1{\section*{#1}}
  \def\Dchapterx#1{\chapter*{#1}}

\else

  \def\Dsubsubsubsection#1{\subsubsection{#1}}
  \def\Dsubsubsection#1{\subsection{#1}}
  \def\Dsubsection#1{\section{#1}}
  \def\Dsection#1{\chapter{#1}}
  \def\Dchapter#1{\title{#1}}
  \def\Dsubsubsubsectionx#1{\subsubsection*{#1}}
  \def\Dsubsubsectionx#1{\subsection*{#1}}
  \def\Dsubsectionx#1{\section*{#1}}
  \def\Dsectionx#1{\chapter*{#1}}

\fi

\newcommand{\MStdPoints}{4}
\newcommand{\MSetPoints}[1]{\renewcommand{\MStdPoints}{#1}}

% Befehl zum Abbruch der Erstellung (nur PDF)
\newcommand{\MAbort}[1]{\err{#1}}

% Prefix vor Dateieinbindungen, wird in der Baumdatei mit \renewcommand modifiziert
% und auf das Verzeichnisprefix gesetzt, in dem das gerade bearbeitete tex-Dokument liegt.
% Im HTML wird es auf das Verzeichnis der HTML-Datei gesetzt.
% Das Prefix muss mit / enden !
\newcommand{\MDPrefix}{.}

% MRegisterFile notiert eine Datei zur Einbindung in den HTML-Baum. Grafiken mit MGraphics werden automatisch eingebunden.
% Mit MLastFile erhaelt man eine Markierung fuer die zuletzt registrierte Datei.
% Diese Markierung wird im postprocessing durch den physikalischen Dateinamen ersetzt, aber nur den Namen (d.h. \MMaterial gehoert noch davor, vgl Definition von MGraphics)
% Parameter: Pfad/Name der Datei bzw. des Ordners, relativ zur Position des Modul-Tex-Dokuments.
\ifttm
\newcommand{\MRegisterFile}[1]{\addtocounter{MFileNumber}{1}\special{html:<!-- registerfile;;}#1\special{html:;;}\MDPrefix\special{html:;;}\arabic{MFileNumber}\special{html:; //-->}}
\else
\newcommand{\MRegisterFile}[1]{\addtocounter{MFileNumber}{1}}
\fi

% Testen welcher Uebersetzer hier am Werk ist

\ifttm
\setcounter{MOutput}{3}
\else
\ifx\pdfoutput\undefined
  \pdffalse
  \setcounter{MOutput}{\MOutputDVI}
  \message{Verarbeitung mit latex, Ausgabe in dvi.}
\else
  \setcounter{MOutput}{\MOutputPDF}
  \message{Verarbeitung mit pdflatex, Ausgabe in pdf.}
  \ifnum \pdfoutput=0
    \pdffalse
  \setcounter{MOutput}{\MOutputDVI}
  \message{Verarbeitung mit pdflatex, Ausgabe in dvi.}
  \else
    \ifnum\pdfoutput=1
    \pdftrue
  \setcounter{MOutput}{\MOutputPDF}
  \message{Verarbeitung mit pdflatex, Ausgabe in pdf.}
    \fi
  \fi
\fi
\fi

\ifnum\value{MOutput}=\MOutputPDF
\DeclareGraphicsExtensions{.pdf,.png,.jpg}
\fi

\ifnum\value{MOutput}=\MOutputDVI
\DeclareGraphicsExtensions{.eps,.png,.jpg}
\fi

\ifnum\value{MOutput}=\MOutputHTML
% Wird vom Konverter leider nicht erkannt und daher in split.pm hardcodiert!
\DeclareGraphicsExtensions{.png,.jpg,.gif}
\fi

% Umdefinition der hyperref-Nummerierung im PDF-Modus
\ifttm
\else
\renewcommand{\theHfigure}{\arabic{chapter}.\arabic{section}.\arabic{figure}}
\fi

% Makro, um in der HTML-Ausgabe die zuerst zu oeffnende Datei zu kennzeichnen
\ifttm
\newcommand{\MGlobalStart}{\special{html:<!-- mglobalstarttag -->}}
\else
\newcommand{\MGlobalStart}{}
\fi

% Makro, um bei scormlogin ein pullen des Benutzers bei Aufruf der Seite zu erzwingen (typischerweise auf der Einstiegsseite)
\ifttm
\newcommand{\MPullSite}{\special{html:<!-- pullsite //-->}}
\else
\newcommand{\MPullSite}{}
\fi

% Makro, um in der HTML-Ausgabe die Kapiteluebersicht zu kennzeichnen
\ifttm
\newcommand{\MGlobalChapterTag}{\special{html:<!-- mglobalchaptertag -->}}
\else
\newcommand{\MGlobalChapterTag}{}
\fi

% Makro, um in der HTML-Ausgabe die Konfiguration zu kennzeichnen
\ifttm
\newcommand{\MGlobalConfTag}{\special{html:<!-- mglobalconfigtag -->}}
\else
\newcommand{\MGlobalConfTag}{}
\fi

% Makro, um in der HTML-Ausgabe die Standortbeschreibung zu kennzeichnen
\ifttm
\newcommand{\MGlobalLocationTag}{\special{html:<!-- mgloballocationtag -->}}
\else
\newcommand{\MGlobalLocationTag}{}
\fi

% Makro, um in der HTML-Ausgabe die persoenlichen Daten zu kennzeichnen
\ifttm
\newcommand{\MGlobalDataTag}{\special{html:<!-- mglobaldatatag -->}}
\else
\newcommand{\MGlobalDataTag}{}
\fi

% Makro, um in der HTML-Ausgabe die Suchseite zu kennzeichnen
\ifttm
\newcommand{\MGlobalSearchTag}{\special{html:<!-- mglobalsearchtag -->}}
\else
\newcommand{\MGlobalSearchTag}{}
\fi

% Makro, um in der HTML-Ausgabe die Favoritenseite zu kennzeichnen
\ifttm
\newcommand{\MGlobalFavoTag}{\special{html:<!-- mglobalfavoritestag -->}}
\else
\newcommand{\MGlobalFavoTag}{}
\fi

% Makro, um in der HTML-Ausgabe die Eingangstestseite zu kennzeichnen
\ifttm
\newcommand{\MGlobalSTestTag}{\special{html:<!-- mglobalstesttag -->}}
\else
\newcommand{\MGlobalSTestTag}{}
\fi

% Makro, um in der PDF-Ausgabe ein Wasserzeichen zu definieren
\ifttm
\newcommand{\MWatermarkSettings}{\relax}
\else
\newcommand{\MWatermarkSettings}{%
% \SetWatermarkText{(c) MINT-Kolleg Baden-Württemberg 2014}
% \SetWatermarkLightness{0.85}
% \SetWatermarkScale{1.5}
}
\fi

\ifttm
\newcommand{\MBinom}[2]{\left({\begin{array}{c} #1 \\ #2 \end{array}}\right)}
\else
\newcommand{\MBinom}[2]{\binom{#1}{#2}}
\fi

\ifttm
\newcommand{\DeclareMathOperator}[2]{\def#1{\mathrm{#2}}}
\newcommand{\operatorname}[1]{\mathrm{#1}}
\fi

%----------------- Makros fuer die gemischte HTML/PDF-Konvertierung ------------------------------

\newcommand{\MTestName}{\relax} % wird durch Test-Umgebung gesetzt

% Fuer experimentelle Kursinhalte, die im Release-Umsetzungsvorgang eine Fehlermeldung
% produzieren sollen aber sonst normal umgesetzt werden
\newenvironment{MExperimental}{%
}{%
}

% Wird von ttm nicht richtig umgesetzt!!
\newenvironment{MExerciseItems}{%
\renewcommand\theenumi{\alph{enumi}}%
\begin{enumerate}%
}{%
\end{enumerate}%
}


\definecolor{infoshadecolor}{rgb}{0.75,0.75,0.75}
\definecolor{exmpshadecolor}{rgb}{0.875,0.875,0.875}
\definecolor{expeshadecolor}{rgb}{0.95,0.95,0.95}
\definecolor{framecolor}{rgb}{0.2,0.2,0.2}

% Bei PDF-Uebersetzung wird hinter den Start jeder Satz/Info-aehnlichen Umgebung eine leere mbox gesetzt, damit
% fuehrende Listen oder enums nicht den Zeilenumbruch kaputtmachen
%\ifttm
\def\MTB{}
%\else
%\def\MTB{\mbox{}}
%\fi


\ifttm
\newcommand{\MRelates}{\special{html:<mi>&wedgeq;</mi>}}
\else
\def\MRelates{\stackrel{\scriptscriptstyle\wedge}{=}}
\fi

\def\MInch{\text{''}}
\def\Mdd{\textit{''}}

\ifttm
\def\MNL{ \newline }
\newenvironment{MArray}[1]{\begin{array}{#1}}{\end{array}}
\else
\def\MNL{ \\ }
\newenvironment{MArray}[1]{\begin{array}{#1}}{\end{array}}
\fi

\newcommand{\MBox}[1]{$\mathrm{#1}$}
\newcommand{\MMBox}[1]{\mathrm{#1}}


\ifttm%
\newcommand{\Mtfrac}[2]{{\textstyle \frac{#1}{#2}}}
\newcommand{\Mdfrac}[2]{{\displaystyle \frac{#1}{#2}}}
\newcommand{\Mmeasuredangle}{\special{html:<mi>&angmsd;</mi>}}
\else%
\newcommand{\Mtfrac}[2]{\tfrac{#1}{#2}}
\newcommand{\Mdfrac}[2]{\dfrac{#1}{#2}}
\newcommand{\Mmeasuredangle}{\measuredangle}
\relax
\fi

% Matrizen und Vektoren

% Inhalt wird in der Form a & b \\ c & d erwartet
% Vorsicht: MVector = Komponentenspalte, MVec = Variablensymbol
\ifttm%
\newcommand{\MVector}[1]{\left({\begin{array}{c}#1\end{array}}\right)}
\else%
\newcommand{\MVector}[1]{\begin{pmatrix}#1\end{pmatrix}}
\fi



\newcommand{\MVec}[1]{\vec{#1}}
\newcommand{\MDVec}[1]{\overrightarrow{#1}}

%----------------- Umgebungen fuer Definitionen und Saetze ----------------------------------------

% Fuegt einen Tabellen-Zeilenumbruch ein im PDF, aber nicht im HTML
\newcommand{\TSkip}{\ifttm \else&\ \\\fi}

\newenvironment{infoshaded}{%
\def\FrameCommand{\fboxsep=\FrameSep \fcolorbox{framecolor}{infoshadecolor}}%
\MakeFramed {\advance\hsize-\width \FrameRestore}}%
{\endMakeFramed}

\newenvironment{expeshaded}{%
\def\FrameCommand{\fboxsep=\FrameSep \fcolorbox{framecolor}{expeshadecolor}}%
\MakeFramed {\advance\hsize-\width \FrameRestore}}%
{\endMakeFramed}

\newenvironment{exmpshaded}{%
\def\FrameCommand{\fboxsep=\FrameSep \fcolorbox{framecolor}{exmpshadecolor}}%
\MakeFramed {\advance\hsize-\width \FrameRestore}}%
{\endMakeFramed}

\def\STDCOLOR{black}

\ifttm%
\else%
\newtheoremstyle{MSatzStyle}
  {1cm}                   %Space above
  {1cm}                   %Space below
  {\normalfont\itshape}   %Body font
  {}                      %Indent amount (empty = no indent,
                          %\parindent = para indent)
  {\normalfont\bfseries}  %Thm head font
  {}                      %Punctuation after thm head
  {\newline}              %Space after thm head: " " = normal interword
                          %space; \newline = linebreak
  {\thmname{#1}\thmnumber{#2}\thmnote{(#3)} }
                          %Thm head spec (can be left empty, meaning
                          %`normal')
                          %
\newtheoremstyle{MDefStyle}
  {1cm}                   %Space above
  {1cm}                   %Space below
  {\normalfont}           %Body font
  {}                      %Indent amount (empty = no indent,
                          %\parindent = para indent)
  {\normalfont\bfseries}  %Thm head font
  {}                      %Punctuation after thm head
  {\newline}              %Space after thm head: " " = normal interword
                          %space; \newline = linebreak
  {\thmname{#1}\thmnumber{#2}\thmnote{(#3)}}
                          %Thm head spec (can be left empty, meaning
                          %`normal')
\fi%

\newcommand{\MInfoText}{Info}

\newcounter{MHintCounter}
\newcounter{MCodeEditCounter}

\newcounter{MLastIndex}  % Enthaelt die dritte Stelle (Indexnummer) des letzten angelegten Objekts
\newcounter{MLastType}   % Enthaelt den Typ des letzten angelegten Objekts (mithilfe der unten definierten Konstanten). Die Entscheidung, wie der Typ dargstellt wird, wird in split.pm beim Postprocessing getroffen.
\newcounter{MLastTypeEq} % =1 falls das Label in einer Matheumgebung (equation, eqnarray usw.) steht, =2 falls das Label in einer table-Umgebung steht

% Da ttm keine Zahlmakros verarbeiten kann, werden diese Nummern in den Zuweisungen hardcodiert!
\def\MTypeSection{1}          %# Zaehler ist section
\def\MTypeSubsection{2}       %# Zaehler ist subsection
\def\MTypeSubsubsection{3}    %# Zaehler ist subsubsection
\def\MTypeInfo{4}             %# Eine Infobox, Separatzaehler fuer die Chemie (auch wenn es dort nicht nummeriert wird) ist MInfoCounter
\def\MTypeExercise{5}         %# Eine Aufgabe, Separatzaehler fuer die Chemie ist MExerciseCounter
\def\MTypeExample{6}          %# Eine Beispielbox, Separatzaehler fuer die Chemie ist MExampleCounter
\def\MTypeExperiment{7}       %# Eine Versuchsbox, Separatzaehler fuer die Chemie ist MExperimentCounter
\def\MTypeGraphics{8}         %# Eine Graphik, Separatzaehler fuer alle FB ist MGraphicsCounter
\def\MTypeTable{9}            %# Eine Tabellennummer, hat keinen Zaehler da durch table gezaehlt wird
\def\MTypeEquation{10}        %# Eine Gleichungsnummer, hat keinen Zaehler da durch equation/eqnarray gezaehlt wird
\def\MTypeTheorem{11}         % Ein theorem oder xtheorem, Separatzaehler fuer die Chemie ist MTheoremCounter
\def\MTypeVideo{12}           %# Ein Video,Separatzaehler fuer alle FB ist MVideoCounter
\def\MTypeEntry{13}           %# Ein Eintrag fuer die Stichwortliste, wird nicht gezaehlt sondern erhaelt im preparsing ein unique-label

% Zaehler fuer das Labelsystem sind prefixcounter, jeder Zaehler wird VOR dem gezaehlten Objekt inkrementiert und zaehlt daher das aktuelle Objekt
\newcounter{MInfoCounter}
\newcounter{MExerciseCounter}
\newcounter{MExampleCounter}
\newcounter{MExperimentCounter}
\newcounter{MGraphicsCounter}
\newcounter{MTableCounter}
\newcounter{MEquationCounter}  % Nur im HTML, sonst durch "equation"-counter von latex realisiert
\newcounter{MTheoremCounter}
\newcounter{MObjectCounter}   % Gemeinsamer Zaehler fuer Objekte (ausser Grafiken/Tabellen) in Mathe/Info/Physik
\newcounter{MVideoCounter}
\newcounter{MEntryCounter}

\newcounter{MTestSite} % 1 = Subsubsection ist eine Pruefungsseite, 0 = ist eine normale Seite (inkl. Hilfeseite)

\def\MCell{$\phantom{a}$}

\newenvironment{MExportExercise}{\begin{MExercise}}{\end{MExercise}} % wird von mconvert abgefangen

\def\MGenerateExNumber{%
\ifnum\value{MSepNumbers}=0%
\arabic{section}.\arabic{subsection}.\arabic{MObjectCounter}\setcounter{MLastIndex}{\value{MObjectCounter}}%
\else%
\arabic{section}.\arabic{subsection}.\arabic{MExerciseCounter}\setcounter{MLastIndex}{\value{MExerciseCounter}}%
\fi%
}%

\def\MGenerateExmpNumber{%
\ifnum\value{MSepNumbers}=0%
\arabic{section}.\arabic{subsection}.\arabic{MObjectCounter}\setcounter{MLastIndex}{\value{MObjectCounter}}%
\else%
\arabic{section}.\arabic{subsection}.\arabic{MExerciseCounter}\setcounter{MLastIndex}{\value{MExampleCounter}}%
\fi%
}%

\def\MGenerateInfoNumber{%
\ifnum\value{MSepNumbers}=0%
\arabic{section}.\arabic{subsection}.\arabic{MObjectCounter}\setcounter{MLastIndex}{\value{MObjectCounter}}%
\else%
\arabic{section}.\arabic{subsection}.\arabic{MExerciseCounter}\setcounter{MLastIndex}{\value{MInfoCounter}}%
\fi%
}%

\def\MGenerateSiteNumber{%
\arabic{section}.\arabic{subsection}.\arabic{subsubsection}%
}%

% Funktionalitaet fuer Auswahlaufgaben

\newcounter{MExerciseCollectionCounter} % = 0 falls nicht in collection-Umgebung, ansonsten Schachtelungstiefe
\newcounter{MExerciseCollectionTextCounter} % wird von MExercise-Umgebung inkrementiert und von MExerciseCollection-Umgebung auf Null gesetzt

\ifttm
% MExerciseCollection gruppiert Aufgaben, die dynamisch aus der Datenbank gezogen werden und nicht direkt in der HTML-Seite stehen
% Parameter: #1 = ID der Collection, muss eindeutig fuer alle IN DER DB VORHANDENEN collections sein unabhaengig vom Kurs
%            #2 = Optionsargument (im Moment: 1 = Iterative Auswahl, 2 = Zufallsbasierte Auswahl)
\newenvironment{MExerciseCollection}[2]{%
\addtocounter{MExerciseCollectionCounter}{1}
\setcounter{MExerciseCollectionTextCounter}{0}
\special{html:<!-- mexercisecollectionstart;;}#1\special{html:;;}#2\special{html:;; //-->}%
}{%
\special{html:<!-- mexercisecollectionstop //-->}%
\addtocounter{MExerciseCollectionCounter}{-1}
}
\else
\newenvironment{MExerciseCollection}[2]{%
\addtocounter{MExerciseCollectionCounter}{1}
\setcounter{MExerciseCollectionTextCounter}{0}
}{%
\addtocounter{MExerciseCollectionCounter}{-1}
}
\fi

% Bei Uebersetzung nach PDF werden die theorem-Umgebungen verwendet, bei Uebersetzung in HTML ein manuelles Makro
\ifttm%

  \newenvironment{MHint}[1]{  \special{html:<button name="Name_MHint}\arabic{MHintCounter}\special{html:" class="hintbutton_closed btn btn-default" id="MHint}\arabic{MHintCounter}\special{html:_button" %
  type="button" onclick="toggle_hint('MHint}\arabic{MHintCounter}\special{html:');">}#1\special{html:</button>}
  \special{html:<div class="hint well" style="display:none" id="MHint}\arabic{MHintCounter}\special{html:"> }}{\begin{html}</div>\end{html}\addtocounter{MHintCounter}{1}}

  \newenvironment{MCOSHZusatz}{  \special{html:<button name="Name_MHint}\arabic{MHintCounter}\special{html:" class="chintbutton_closed" id="MHint}\arabic{MHintCounter}\special{html:_button" %
  type="button" onclick="toggle_hint('MHint}\arabic{MHintCounter}\special{html:');">}\iCoshAdd{}\special{html:</button>}
  \special{html:<div class="hintc" style="display:none" id="MHint}\arabic{MHintCounter}\special{html:">
  <div class="coshwarn">}\iCoshWarn{}\special{html:</div><br />}
  \addtocounter{MHintCounter}{1}}{\begin{html}</div>\end{html}}


  \newenvironment{MDefinition}{\begin{definition}\setcounter{MLastIndex}{\value{definition}}\ \\}{\end{definition}}


  \newenvironment{MExercise}{
  \renewcommand{\MStdPoints}{4}
  \addtocounter{MExerciseCounter}{1}
  \addtocounter{MObjectCounter}{1}
  \setcounter{MLastType}{5}

\ifnum\value{MExerciseCollectionCounter}=0\else\addtocounter{MExerciseCollectionTextCounter}{1}\special{html:<!-- mexercisetextstart;;}\arabic{MExerciseCollectionTextCounter}\special{html:;; //-->}\fi
  \special{html:<div class="aufgabe"
  id="ADIV_}\MGenerateExNumber\special{html:"><div class="panel
  panel-warning"><div class="panel-heading"><h5 class="title">}%
  \textbf{\iExercise{} \MGenerateExNumber
  } \begin{html}</h5></div><div class="panel-body">\end{html}}{\special{html:</div></div></div><!--
  mfeedbackbutton;Exercise;}\arabic{MTestSite}\special{html:;}\MGenerateExNumber\special{html:; //-->} \ifnum\value{MExerciseCollectionCounter}=0\else\special{html:<!-- mexercisetextstop //-->}\fi
  }
  % Stellt eine Kombination aus Aufgabe, Loesungstext und Eingabefeld bereit,
  % bei der Aufgabentext und Musterloesung sowie die zugehoerigen Feldelemente
  % extern bezogen und div-aktualisiert werden, das Eingabefeld aber immer das gleiche ist.
  \newenvironment{MFetchExercise}{
  \addtocounter{MExerciseCounter}{1}
  \addtocounter{MObjectCounter}{1}
  \setcounter{MLastType}{5}

  \special{html:<div class="aufgabe" id="ADIV_}\MGenerateExNumber\special{html:">}%
  \textbf{\iExercise{} \MGenerateExNumber
  } \ \\%
  \special{html:</div><div class="exfetch_text" id="ADIVTEXT_}\MGenerateExNumber\special{html:">}%
  \special{html:</div><div class="exfetch_sol" id="ADIVSOL_}\MGenerateExNumber\special{html:">}%
  \special{html:</div><div class="exfetch_input" id="ADIVINPUT_}\MGenerateExNumber\special{html:">}%
  }{
  \special{html:</div>}
  }

  \newenvironment{MExample}{
  \addtocounter{MExampleCounter}{1}
  \addtocounter{MObjectCounter}{1}
  \setcounter{MLastType}{6}
  \begin{html}
  <div class="exmp">
  <div class="exmprahmen panel panel-default">
  <div class="panel-heading"><h5 class="panel-title">
  \end{html}\textbf{\iExample{}
  \ifnum\value{MSepNumbers}=0
  \arabic{section}.\arabic{subsection}.\arabic{MObjectCounter}\setcounter{MLastIndex}{\value{MObjectCounter}}
  \else
  \arabic{section}.\arabic{subsection}.\arabic{MExampleCounter}\setcounter{MLastIndex}{\value{MExampleCounter}}
  \fi
  } \begin{html}</h5></div><div class="panel-body">\end{html}}
  {\begin{html}</div></div></div>\end{html}
  \special{html:<!-- mfeedbackbutton;Example;}\arabic{MTestSite}\special{html:;}\MGenerateExmpNumber\special{html:; //-->}
  }

  \newenvironment{MExperiment}{
  \addtocounter{MExperimentCounter}{1}
  \addtocounter{MObjectCounter}{1}
  \setcounter{MLastType}{7}
  \begin{html}
  <div class="expe">
  <div class="experahmen">
  \end{html}\textbf{\iExperiment{}
  \ifnum\value{MSepNumbers}=0
  \arabic{section}.\arabic{subsection}.\arabic{MObjectCounter}\setcounter{MLastIndex}{\value{MObjectCounter}}
  \else
%  \arabic{MExperimentCounter}\setcounter{MLastIndex}{\value{MExperimentCounter}}
  \arabic{section}.\arabic{subsection}.\arabic{MExperimentCounter}\setcounter{MLastIndex}{\value{MExperimentCounter}}
  \fi
  } \ \\}{\begin{html}</div>
  </div>
  \end{html}}

  \newenvironment{MChemInfo}{
  \setcounter{MLastType}{4}
  \begin{html}
  <div class="info">
  <div class="inforahmen">
  \end{html}}{\begin{html}</div>
  </div>
  \end{html}}

  \newenvironment{MXInfo}[1]{
  \addtocounter{MInfoCounter}{1}
  \addtocounter{MObjectCounter}{1}
  \setcounter{MLastType}{4}
  \begin{html}
  <div class="info">
  <div class="inforahmen panel panel-info">
  <div class="panel-heading"><h5 class="panel-title">
  \end{html}\textbf{#1
  \ifnum\value{MInfoNumbers}=0
  \else
    \ifnum\value{MSepNumbers}=0
    \arabic{section}.\arabic{subsection}.\arabic{MObjectCounter}\setcounter{MLastIndex}{\value{MObjectCounter}}
    \else
    \arabic{MInfoCounter}\setcounter{MLastIndex}{\value{MInfoCounter}}
    \fi
  \fi
  } \begin{html}</h5></div><div class="panel-body">\end{html}}
  {\begin{html}</div></div></div>\end{html}
  \special{html:<!-- mfeedbackbutton;Info;}\arabic{MTestSite}\special{html:;}\MGenerateInfoNumber\special{html:; //-->}
  }

  \newenvironment{MInfo}
  {\ifnum\value{MInfoNumbers}=0\begin{MChemInfo}\else\begin{MXInfo}{Info}\ \\ \fi}
  {\ifnum\value{MInfoNumbers}=0\end{MChemInfo}\else\end{MXInfo}\fi}

\else%

  \theoremstyle{MSatzStyle}
  \newtheorem{thm}{Satz}[section]
  \newtheorem{thmc}{Satz}
  \theoremstyle{MDefStyle}
  \newtheorem{defn}[thm]{Definition}
  \newtheorem{exmp}[thm]{\iExample{} }
  \newtheorem{info}[thm]{\MInfoText}
  \theoremstyle{MDefStyle}
  \newtheorem{defnc}{Definition}
  \theoremstyle{MDefStyle}
  \newtheorem{exmpc}{\iExample{} } [section]
  \theoremstyle{MDefStyle}
  \newtheorem{infoc}{\MInfoText}
  \theoremstyle{MDefStyle}
  \newtheorem{exrc}{\iExercise{} } [section]
  \theoremstyle{MDefStyle}
  \newtheorem{verc}{Versuch} [section]

  \newenvironment{MFetchExercise}{}{} % kann im PDF nicht dargestellt werden

  \newenvironment{MExercise}{\begin{exrc}\renewcommand{\MStdPoints}{1}\MTB}{\end{exrc}}
  \newenvironment{MHint}[1]{\ \\ \underline{#1:}\\}{}
  \newenvironment{MCOSHZusatz}{\ \\ \underline{\iCoshAdd{}:}\\}{}
%  \newenvironment{MDefinition}{\ifnum\value{MInfoNumbers}=0\begin{defnc}\else\begin{defn}\fi\MTB}{\ifnum\value{MInfoNumbers}=0\end{defnc}\else\end{defn}\fi}
%  \newenvironment{MExample}{\begin{exmp}}{\ \linebreak[1] \ \ \ \ $\phantom{a}$ \ \hfill $\blacklozenge$\end{exmp}}
  \newenvironment{MExample}{
    \ifnum\value{MInfoNumbers}=0\begin{exmpc}\else\begin{exmp}\fi
    \MTB
    \begin{exmpshaded}
    \ \newline
}{
    \end{exmpshaded}
    \ifnum\value{MInfoNumbers}=0\end{exmpc}\else\end{exmp}\fi
}
  \newenvironment{MChemInfo}{\begin{infoshaded}}{\end{infoshaded}}
%  \newenvironment{MXInfo}{\begin{infoshaded}}{\end{infoshaded}}

  \newenvironment{MInfo}{\ifnum\value{MInfoNumbers}=0\begin{MChemInfo}\else\renewcommand{\MInfoText}{Info}\begin{info}\begin{infoshaded}
  \MTB
   \ \newline
    \fi
  }{\ifnum\value{MInfoNumbers}=0\end{MChemInfo}\else\end{infoshaded}\end{info}\fi}

  \newenvironment{MXInfo}[1]{
    \renewcommand{\MInfoText}{#1}
    \ifnum\value{MInfoNumbers}=0\begin{infoc}\else\begin{info}\fi%
    \MTB
    \begin{infoshaded}
    \ \newline
  }{\end{infoshaded}\ifnum\value{MInfoNumbers}=0\end{infoc}\else\end{info}\fi}

  \newenvironment{MExperiment}{
    \renewcommand{\MInfoText}{\iExperiment{}}
    \ifnum\value{MInfoNumbers}=0\begin{verc}\else\begin{info}\fi
    \MTB
    \begin{expeshaded}
    \ \newline
  }{
    \end{expeshaded}
    \ifnum\value{MInfoNumbers}=0\end{verc}\else\end{info}\fi
  }
\fi%

% MHint sollte nicht direkt fuer Loesungen benutzt werden wegen solutionselect
\newenvironment{MSolution}{\begin{MHint}{Solution}}{\end{MHint}}

\newcounter{MCodeCounter}

\ifttm
\newenvironment{MCode}{\special{html:<!-- mcodestart -->}\ttfamily\color{blue}}{\special{html:<!-- mcodestop -->}}
\else
\newenvironment{MCode}{\begin{flushleft}\ttfamily\addtocounter{MCodeCounter}{1}}{\addtocounter{MCodeCounter}{-1}\end{flushleft}}
% Ohne color-Statement da inkompatible mit framed/shaded-Boxen aus dem framed-package
\fi

%----------------- Sonderdefinitionen fuer Symbole, die der Konverter nicht kann ----------------------------------------------

\ifttm%
\newcommand{\MUnderset}[2]{\underbrace{#2}_{#1}}%
\else%
\newcommand{\MUnderset}[2]{\underset{#1}{#2}}%
\fi%

\ifttm
\newcommand{\MThinspace}{\special{html:<mi>&#x2009;</mi>}}
\else
\newcommand{\MThinspace}{\,}
\fi

\ifttm
\newcommand{\glq}{\begin{html}&sbquo;\end{html}}
\newcommand{\grq}{\begin{html}&lsquo;\end{html}}
\newcommand{\glqq}{\begin{html}&bdquo;\end{html}}
\newcommand{\grqq}{\begin{html}&ldquo;\end{html}}
\fi

% Backsim (PhysikBK)
\ifttm
\newcommand{\backsim}{\begin{html}<mi>&backsim;</mi>\end{html}}
\fi
% Cancel (PhysikBK)
\ifttm
\newcommand{\cancel}[1]{\begin{html}<menclose notation="horizontalstrike"><mi>#1</mi></menclose>\end{html}}
\fi


\ifttm
\newcommand{\MNdash}{\begin{html}&ndash;\end{html}}
\else
\newcommand{\MNdash}{--}
\fi

%\ifttm\def\MIU{\special{html:<mi>&#8520;</mi>}}\else\def\MIU{\mathrm{i}}\fi
\def\MIU{\mathrm{i}}
\def\MEU{e} % TU9-Onlinekurs: italic-e
%\def\MEU{\mathrm{e}} % Alte Onlinemodule: roman-e
\def\MD{d} % Kursives d in Integralen im TU9-Onlinekurs
%\def\MD{\mathrm{d}} % roman-d in den alten Onlinemodulen
\def\MDB{\|}

%zusaetzlicher Leerraum vor "\MD"
\ifttm%
\def\MDSpace{\special{html:<mi>&#x2009;</mi>}}
\else%
\def\MDSpace{\,}
\fi%
\newcommand{\MDwSp}{\MDSpace\MD}%

\ifttm
\def\Mdq{\dq}
\else
\def\Mdq{\dq}
\fi

\def\MSpan#1{\left<{#1}\right>}
\def\MSetminus{\setminus}
\def\MIM{I}

\ifttm
\newcommand{\ld}{\text{ld}}
\newcommand{\lg}{\text{lg}}
\else
\DeclareMathOperator{\ld}{ld}
%\newcommand{\lg}{\text{lg}} % in latex schon definiert
\fi


\def\Mmapsto{\ifttm\special{html:<mi>&mapsto;</mi>}\else\mapsto\fi}
\def\Mvarphi{\ifttm\phi\else\varphi\fi}
\def\Mphi{\ifttm\varphi\else\phi\fi}
\ifttm%
\newcommand{\MEumu}{\special{html:<mi>&#x3BC;</mi>}}%
\else%
\newcommand{\MEumu}{\textrm{\textmu}}%
\fi
\def\Mvarepsilon{\ifttm\epsilon\else\varepsilon\fi}
\def\Mepsilon{\ifttm\varepsilon\else\epsilon\fi}
\def\Mvarkappa{\ifttm\kappa\else\varkappa\fi}
\def\Mkappa{\ifttm\varkappa\else\kappa\fi}
\def\Mcomplement{\ifttm\special{html:<mi>&comp;</mi>}\else\complement\fi}
\def\MWW{\mathrm{WW}}
\def\Mmod{\ifttm\special{html:<mi>&nbsp;mod&nbsp;</mi>}\else\mod\fi}

\ifttm%
\def\mod{\text{\;mod\;}}%
\def\MNEquiv{\special{html:<mi>&NotCongruent;</mi>}}%
\def\MNSubseteq{\special{html:<mi>&NotSubsetEqual;</mi>}}%
\def\MEmptyset{\special{html:<mi>&empty;</mi>}}%
\def\MVDots{\special{html:<mi>&#x22EE;</mi>}}%
\def\MHDots{\special{html:<mi>&#x2026;</mi>}}%
\def\Mddag{\special{html:<mi>&#x1202;</mi>}}%
\def\sphericalangle{\special{html:<mi>&measuredangle;</mi>}}%
\def\nparallel{\special{html:<mi>&nparallel;</mi>}}%
\def\MProofEnd{\special{html:<mi>&#x25FB;</mi>}}%
\newenvironment{MProof}[1]{\underline{#1}:\MCR\MCR}{\hfill $\MProofEnd$}%
\else%
\def\MNEquiv{\not\equiv}%
\def\MNSubseteq{\not\subseteq}%
\def\MEmptyset{\emptyset}%
\def\MVDots{\vdots}%
\def\MHDots{\hdots}%
\def\Mddag{\ddag}%
\newenvironment{MProof}[1]{\begin{proof}[#1]}{\end{proof}}%
\fi%



% Spaces zum Auffuellen von Tabellenbreiten, die nur im HTML wirken
\ifttm%
\def\MTSP{\:}%
\else%
\def\MTSP{}%
\fi%

\DeclareMathOperator{\arsinh}{arsinh}
\DeclareMathOperator{\arcosh}{arcosh}
\DeclareMathOperator{\artanh}{artanh}
\DeclareMathOperator{\arcoth}{arcoth}


\newcommand{\MMathSet}[1]{\mathbb{#1}}
\def\N{\MMathSet{N}}
\def\Z{\MMathSet{Z}}
\def\Q{\MMathSet{Q}}
\def\R{\MMathSet{R}}
\def\C{\MMathSet{C}}

\newcounter{MForLoopCounter}
\newcommand{\MForLoop}[2]{\setcounter{MForLoopCounter}{#1}\ifnum\value{MForLoopCounter}=0{}\else{{#2}\addtocounter{MForLoopCounter}{-1}\MForLoop{\value{MForLoopCounter}}{#2}}\fi}

\newcounter{MSiteCounter}
\newcounter{MFieldCounter} % Kombination section.subsection.site.field ist eindeutig in allen Modulen, field alleine nicht

\newcounter{MiniMarkerCounter}

\ifttm
\newenvironment{MMiniPageP}[1]{\begin{minipage}{#1\linewidth}\special{html:<!-- minimarker;;}\arabic{MiniMarkerCounter}\special{html:;;#1; //-->}}{\end{minipage}\addtocounter{MiniMarkerCounter}{1}}
\else
\newenvironment{MMiniPageP}[1]{\begin{minipage}{#1\linewidth}}{\end{minipage}\addtocounter{MiniMarkerCounter}{1}}
\fi

\newcounter{AlignCounter}

\newcommand{\MStartJustify}{\ifttm\special{html:<!-- startalign;;}\arabic{AlignCounter}\special{html:;;justify; //-->}\fi}
\newcommand{\MStopJustify}{\ifttm\special{html:<!-- stopalign;;}\arabic{AlignCounter}\special{html:; //-->}\fi\addtocounter{AlignCounter}{1}}

\newenvironment{MJTabular}[1]{
\MStartJustify
\begin{tabular}{#1}
}{
\end{tabular}
\MStopJustify
}

\newcommand{\MImageLeft}[2]{
\begin{center}
\begin{tabular}{lc}
\MStartJustify
\begin{MMiniPageP}{0.65}
#1
\end{MMiniPageP}
\MStopJustify
&
\begin{MMiniPageP}{0.3}
#2
\end{MMiniPageP}
\end{tabular}
\end{center}
}

\newcommand{\MImageHalf}[2]{
\begin{center}
\begin{tabular}{lc}
\MStartJustify
\begin{MMiniPageP}{0.45}
#1
\end{MMiniPageP}
\MStopJustify
&
\begin{MMiniPageP}{0.45}
#2
\end{MMiniPageP}
\end{tabular}
\end{center}
}

\newcommand{\MBigImageLeft}[2]{
\begin{center}
\begin{tabular}{lc}
\MStartJustify
\begin{MMiniPageP}{0.25}
#1
\end{MMiniPageP}
\MStopJustify
&
\begin{MMiniPageP}{0.7}
#2
\end{MMiniPageP}
\end{tabular}
\end{center}
}

\ifttm
\def\No{\mathbb{N}_0}
\else
\def\No{\ensuremath{\N_0}}
\fi
\def\MT{\textrm{\tiny T}}
\newcommand{\MTranspose}[1]{{#1}^{\MT}}
\ifttm
\newcommand{\MRe}{\mathsf{Re}}
\newcommand{\MIm}{\mathsf{Im}}
\else
\DeclareMathOperator{\MRe}{Re}
\DeclareMathOperator{\MIm}{Im}
\fi

\newcommand{\Mid}{\mathrm{id}}
\newcommand{\MFeinheit}{\mathrm{feinh}}

\ifttm
\newcommand{\Msubstack}[1]{\begin{array}{c}{#1}\end{array}}
\else
\newcommand{\Msubstack}[1]{\substack{#1}}
\fi

% Typen von Fragefeldern:
% 1 = Alphanumerisch, case-sensitive-Vergleich
% 2 = Ja/Nein-Checkbox, Loesung ist 0 oder 1   (OPTION = Image-id fuer Rueckmeldung)
% 3 = Reelle Zahlen Geparset
% 4 = Funktionen Geparset (mit Stuetzstellen zur ueberpruefung)

% Dieser Befehl erstellt ein interaktives Aufgabenfeld. Parameter:
% - #1 Laenge in Zeichen
% - #2 Loesungstext (alphanumerisch, case sensitive)
% - #3 AufgabenID (alphanumerisch, case sensitive)
% - #4 Typ (Kennnummer)
% - #5 String fuer Optionen (ggf. mit Semikolon getrennte Einzelstrings)
% - #6 Anzahl Punkte
% - #7 uxid (kann z.B. Loesungsstring sein)
% ACHTUNG: Die langen Zeilen bitte so lassen, Zeilenumbrueche im tex werden in div's umgesetzt
\ifttm
\newcommand{\MQuestionID}[7]{
\special{html:<!-- mdeclareuxid;;}UX#7\special{html:;;}\arabic{section}\special{html:;;}#3\special{html:;; //-->}%
\special{html:<!-- mdeclarepoints;;}\arabic{section}\special{html:;;}#3\special{html:;;}#6\special{html:;;}\arabic{MTestSite}\special{html:;;}\arabic{chapter}%
\special{html:;; //--><!-- onloadstart //-->CreateQuestionObj("}#7\special{html:",}\arabic{MFieldCounter}\special{html:,"}#2%
\special{html:","}#3\special{html:",}#4\special{html:,"}#5\special{html:",}#6\special{html:,}\arabic{MTestSite}\special{html:,}\arabic{section}%
\special{html:);<!-- onloadstop //-->}%
\special{html:<input mfieldtype="}#4\special{html:" name="Name_}#3\special{html:" id="}#3\special{html:" type="text" size="}#1\special{html:" maxlength="}#1%
\special{html:" }
\ifnum\value{MGroupActive}=0\special{html:onfocus="handlerFocus(}\arabic{MFieldCounter}\special{html:);" onblur="handlerBlur(}\arabic{MFieldCounter}\special{html:);" onkeyup="handlerChange(}\arabic{MFieldCounter}\special{html:,0);" onpaste="handlerChange(}\arabic{MFieldCounter}\special{html:,0);" oninput="handlerChange(}\arabic{MFieldCounter}\special{html:,0);" onpropertychange="handlerChange(}\arabic{MFieldCounter}\special{html:,0);"/>}%
\special{html:<span><span class="glyphicon glyphicon-question-sign" id="}QM#3\special{html:"/></span>} % needs to be nested spans here, as they get garbled-up somehow
\else%
\special{html:onblur="handlerBlur(}\arabic{MFieldCounter}%
\special{html:);" onfocus="handlerFocus(}\arabic{MFieldCounter}\special{html:);" onkeyup="handlerChange(}\arabic{MFieldCounter}\special{html:,1);" onpaste="handlerChange(}\arabic{MFieldCounter}\special{html:,1);" oninput="handlerChange(}\arabic{MFieldCounter}\special{html:,1);" onpropertychange="handlerChange(}\arabic{MFieldCounter}\special{html:,1);"/>}%
\special{html:<span><span class="glyphicon glyphicon-question-sign" id="}QM#3\special{html:"/></span>} % needs to be nested spans here, as they get garbled-up somehow
\fi
}
\else
\newcommand{\MQuestionID}[7]{
\ifnum\value{QBoxFlag}=1\fbox{$\phantom{\MForLoop{#1}{b}}$}
\else
$\phantom{\MForLoop{#1}{b}}$
\fi%
}
\fi%

% ACHTUNG: Die langen Zeilen bitte so lassen, Zeilenumbrueche im tex werden in div's umgesetzt
% QuestionCheckbox macht ausserhalb einer QuestionGroup keinen Sinn!
% #1 = solution (1 oder 0), ggf. mit ::smc abgetrennt auszuschliessende single-choice-boxen (UXIDs durch , getrennt), #2 = id, #3 = points, #4 = uxid
\ifttm
\newcommand{\MQuestionCheckbox}[4]{
\special{html:<!-- mdeclareuxid;;}UX#4\special{html:;;}\arabic{section}\special{html:;;}#2\special{html:;; //-->}%
\ifnum\value{MGroupActive}=0\MDebugMessage{ERROR: Checkbox Nr. \arabic{MFieldCounter}\ ist nicht in einer Kontrollgruppe, es wird niemals eine Loesung angezeigt!}\fi
\special{html: %
<!-- mdeclarepoints;;}\arabic{section}\special{html:;;}#2\special{html:;;}#3\special{html:;;}\arabic{MTestSite}\special{html:;;}\arabic{chapter}%
\special{html:;; //--><!-- onloadstart //-->CreateQuestionObj("}#4\special{html:",}\arabic{MFieldCounter}\special{html:,"}#1\special{html:","}#2\special{html:",2,"IMG}#2%
\special{html:",}#3\special{html:,}\arabic{MTestSite}\special{html:,}\arabic{section}\special{html:);<!-- onloadstop //-->}%
\special{html:<input mtristate="1" cval="0" mfieldtype="2" type="checkbox" name="Name_}#2\special{html:" id="}#2\special{html:" onchange="handlerChange(}\arabic{MFieldCounter}\special{html:,1);"/>}
\special{html:<span><span class="glyphicon glyphicon-question-sign" id="}IMG#2\special{html:"/></span>} % needs to be nested spans here, as they get garbled-up somehow
}
\else%
\newcommand{\MQuestionCheckbox}[4]{
\ifnum\value{QBoxFlag}=1\fbox{$\phantom{X}$}\else$\phantom{X}$\fi%
}
\fi%

\def\MGenerateID{QFELD_\arabic{section}.\arabic{subsection}.\arabic{MSiteCounter}.QF\arabic{MFieldCounter}}

% #1 = 0/1 ggf. mit ::smc abgetrennt auszuschliessende single-choice-boxen (UXIDs durch , getrennt ohne UX), #2 = uxid ohne UX
\newcommand{\MCheckbox}[2]{
\MQuestionCheckbox{#1}{\MGenerateID}{\MStdPoints}{#2}
\addtocounter{MFieldCounter}{1}
}

% #1 = 0/1 ggf. mit ::smc abgetrennt auszuschliessende single-choice-boxen (UXIDs durch , getrennt ohne UX), #2 = uxid ohne UX
\newcommand{\MLCheckbox}[2]{
\MQuestionCheckbox{#1}{\MGenerateID}{\MStdPoints}{#2}
\addtocounter{MFieldCounter}{1}
}

% Erster Parameter: Zeichenlaenge der Eingabebox, zweiter Parameter: Loesungstext
\newcommand{\MQuestion}[2]{
\MQuestionID{#1}{#2}{\MGenerateID}{1}{0}{\MStdPoints}{#2}
\addtocounter{MFieldCounter}{1}
}

% Erster Parameter: Zeichenlaenge der Eingabebox, zweiter Parameter: Loesungstext
\newcommand{\MLQuestion}[3]{
\MQuestionID{#1}{#2}{\MGenerateID}{1}{0}{\MStdPoints}{#3}
\addtocounter{MFieldCounter}{1}
}

% Parameter: Laenge des Feldes, Loesung (wird auch geparsed), Stellen Genauigkeit hinter dem Komma, weitere Stellen werden mathematisch gerundet vor Vergleich
\newcommand{\MParsedQuestion}[3]{
\MQuestionID{#1}{#2}{\MGenerateID}{3}{#3}{\MStdPoints}{#2}
\addtocounter{MFieldCounter}{1}
}

% Parameter: Laenge des Feldes, Loesung (wird auch geparsed), Stellen Genauigkeit hinter dem Komma, weitere Stellen werden mathematisch gerundet vor Vergleich
\newcommand{\MLParsedQuestion}[4]{
\MQuestionID{#1}{#2}{\MGenerateID}{3}{#3}{\MStdPoints}{#4}
\addtocounter{MFieldCounter}{1}
}

% Parameter: Laenge des Feldes, Loesungsfunktion, Anzahl Stuetzstellen, Funktionsvariablen durch Kommata getrennt (nicht case-sensitive), Anzahl Nachkommastellen im Vergleich
\newcommand{\MFunctionQuestion}[5]{
\MQuestionID{#1}{#2}{\MGenerateID}{4}{#3;#4;#5;0}{\MStdPoints}{#2}
\addtocounter{MFieldCounter}{1}
}

% Parameter: Laenge des Feldes, Loesungsfunktion, Anzahl Stuetzstellen, Funktionsvariablen durch Kommata getrennt (nicht case-sensitive), Anzahl Nachkommastellen im Vergleich, UXID
\newcommand{\MLFunctionQuestion}[6]{
\MQuestionID{#1}{#2}{\MGenerateID}{4}{#3;#4;#5;0}{\MStdPoints}{#6}
\addtocounter{MFieldCounter}{1}
}

% Parameter: Laenge des Feldes, Loesungsintervall, Genauigkeit der Zahlenwertpruefung
\newcommand{\MIntervalQuestion}[3]{
\MQuestionID{#1}{#2}{\MGenerateID}{6}{#3}{\MStdPoints}{#2}
\addtocounter{MFieldCounter}{1}
}

% Parameter: Laenge des Feldes, Loesungsintervall, Genauigkeit der Zahlenwertpruefung, UXID
\newcommand{\MLIntervalQuestion}[4]{
\MQuestionID{#1}{#2}{\MGenerateID}{6}{#3}{\MStdPoints}{#4}
\addtocounter{MFieldCounter}{1}
}

% Parameter: Laenge des Feldes, Loesungsfunktion, Anzahl Stuetzstellen, Funktionsvariable (nicht case-sensitive), Anzahl Nachkommastellen im Vergleich, Vereinfachungsbedingung
% Vereinfachungsbedingung ist eine der Folgenden:
% 0 = Keine Vereinfachungsbedingung
% 1 = Keine Klammern (runde oder eckige) mehr im vereinfachten Ausdruck
% 2 = Faktordarstellung (Term hat Produkte als letzte Operation, Summen als vorgeschaltete Operation)
% 3 = Summendarstellung (Term hat Summen als letzte Operation, Produkte als vorgeschaltete Operation)
% Flag 512: Besondere Stuetzstellen (nur >1 und nur schwach rational), sonst symmetrisch um Nullpunkt und ganze Zahlen inkl. Null werden getroffen
\newcommand{\MSimplifyQuestion}[6]{
\MQuestionID{#1}{#2}{\MGenerateID}{4}{#3;#4;#5;#6}{\MStdPoints}{#2}
\addtocounter{MFieldCounter}{1}
}

\newcommand{\MLSimplifyQuestion}[7]{
\MQuestionID{#1}{#2}{\MGenerateID}{4}{#3;#4;#5;#6}{\MStdPoints}{#7}
\addtocounter{MFieldCounter}{1}
}

% Parameter: Laenge des Feldes, Loesung (optionaler Ausdruck), Anzahl Stuetzstellen, Funktionsvariable (nicht case-sensitive), Anzahl Nachkommastellen im Vergleich, Spezialtyp (string-id)
\newcommand{\MLSpecialQuestion}[7]{
\MQuestionID{#1}{#2}{\MGenerateID}{7}{#3;#4;#5;#6}{\MStdPoints}{#7}
\addtocounter{MFieldCounter}{1}
}

\newcounter{MGroupStart}
\newcounter{MGroupEnd}
\newcounter{MGroupActive}

\newenvironment{MQuestionGroup}{
\setcounter{MGroupStart}{\value{MFieldCounter}}
\setcounter{MGroupActive}{1}
}{
\setcounter{MGroupActive}{0}
\setcounter{MGroupEnd}{\value{MFieldCounter}}
\addtocounter{MGroupEnd}{-1}
}

\newcommand{\MGroupButton}[1]{
\ifttm
\special{html:<button name="Name_Group}\arabic{MGroupStart}\special{html:to}\arabic{MGroupEnd}\special{html:" class="groupbutton" id="Group}\arabic{MGroupStart}\special{html:to}\arabic{MGroupEnd}\special{html:" %
type="button" onclick="group_button(}\arabic{MGroupStart}\special{html:,}\arabic{MGroupEnd}\special{html:);">}#1\special{html:</button>}
\else
\phantom{#1}
\fi
}

%----------------- Makros fuer die modularisierte Darstellung ------------------------------------

\def\MyText#1{#1}

% is used internally by the conversion package, should not be used by original tex documents
\def\MOrgLabel#1{\relax}

\ifttm

% Ein MLabel wird im html codiert durch das tag <!-- mmlabel;;Labelbezeichner;;SubjectArea;;chapter;;section;;subsection;;Index;;Objekttyp; //-->
\def\MLabel#1{%
\ifnum\value{MLastType}=8%
\ifnum\value{MCaptionOn}=0%
\MDebugMessage{ERROR: Grafik \arabic{MGraphicsCounter} hat separates label: #1 (Grafiklabels sollten nur in der Caption stehen)}%
\fi
\fi
\ifnum\value{MLastType}=12%
\ifnum\value{MCaptionOn}=0%
\MDebugMessage{ERROR: Video \arabic{MVideoCounter} hat separates label: #1 (Videolabels sollten nur in der Caption stehen}%
\fi
\fi
\ifnum\value{MLastType}=10\setcounter{MLastIndex}{\value{equation}}\fi
\label{#1}\begin{html}<!-- mmlabel;;#1;;\end{html}\arabic{MSubjectArea}\special{html:;;}\arabic{chapter}\special{html:;;}\arabic{section}\special{html:;;}\arabic{subsection}\special{html:;;}\arabic{MLastIndex}\special{html:;;}\arabic{MLastType}\special{html:; //-->}}%

\else

% Sonderbehandlung im PDF fuer Abbildungen in separater aux-Datei, da MGraphics die figure-Umgebung nicht verwendet
\def\MLabel#1{%
\ifnum\value{MLastType}=8%
\ifnum\value{MCaptionOn}=0%
\MDebugMessage{ERROR: Grafik \arabic{MGraphicsCounter} hat separates label: #1 (Grafiklabels sollten nur in der Caption stehen}%
\fi
\fi
\ifnum\value{MLastType}=12%
\ifnum\value{MCaptionOn}=0%
\MDebugMessage{ERROR: Video \arabic{MVideoCounter} hat separates label: #1 (Videolabels sollten nur in der Caption stehen}%
\fi
\fi
\label{#1}%
}%

\fi

% Gibt Begriff des referenzierten Objekts mit aus, aber nur im HTML, daher nur in Ausnahmefaellen (z.B. Copyrightliste) sinnvoll
\def\MCRef#1{\ifttm\special{html:<!-- mmref;;}#1\special{html:;;1; //-->}\else\vref{#1}\fi}


\def\MRef#1{\ifttm\special{html:<!-- mmref;;}#1\special{html:;;0; //-->}\else\vref{#1}\fi}
\def\MERef#1{\ifttm\special{html:<!-- mmref;;}#1\special{html:;;0; //-->}\else\eqref{#1}\fi}
\def\MNRef#1{\ifttm\special{html:<!-- mmref;;}#1\special{html:;;0; //-->}\else\ref{#1}\fi}
\def\MSRef#1#2{\ifttm\special{html:<!-- msref;;}#1\special{html:;;}#2\special{html:; //-->}\else \if#2\empty \ref{#1} \else \hyperref[#1]{#2}\fi\fi}

\def\MRefRange#1#2{\ifttm\MRef{#1} bis
\MRef{#2}\else\vrefrange[\unskip]{#1}{#2}\fi}

\def\MRefTwo#1#2{\ifttm\MRef{#1} und \MRef{#2}\else%
\let\vRefTLRsav=\reftextlabelrange\let\vRefTPRsav=\reftextpagerange%
\def\reftextlabelrange##1##2{\ref{##1} und~\ref{##2}}%
\def\reftextpagerange##1##2{auf den Seiten~\pageref{#1} und~\pageref{#2}}%
\vrefrange[\unskip]{#1}{#2}%
\let\reftextlabelrange=\vRefTLRsav\let\reftextpagerange=\vRefTPRsav\fi}

% MSectionChapter definiert falls notwendig das Kapitel vor der section. Das ist notwendig, wenn nur ein Einzelmodul uebersetzt wird.
% MChaptersGiven ist ein Counter, der von mconvert.pl vordefiniert wird.
\ifttm
\newcommand{\MSectionChapter}{\ifnum\value{MChaptersGiven}=0{\Dchapter{Modul}}\else{}\fi}
\else
\newcommand{\MSectionChapter}{\ifnum\value{chapter}=0{\Dchapter{Modul}}\else{}\fi}
\fi


\def\MChapter#1{\ifnum\value{MSSEnd}>0{\MSubsectionEndMacros}\addtocounter{MSSEnd}{-1}\fi\Dchapter{#1}}
\def\MSubject#1{\MChapter{#1}} % Schluesselwort HELPSECTION ist reserviert fuer Hilfesektion

\newcommand{\MSectionID}{UNKNOWNID}

\ifttm
\newcommand{\MSetSectionID}[1]{\renewcommand{\MSectionID}{#1}}
\else
\newcommand{\MSetSectionID}[1]{\renewcommand{\MSectionID}{#1}\tikzsetexternalprefix{#1}}
\fi


\newcommand{\MSection}[1]{\MSetSectionID{MODULID}\ifnum\value{MSSEnd}>0{\MSubsectionEndMacros}\addtocounter{MSSEnd}{-1}\fi\MSectionChapter\Dsection{#1}\MSectionStartMacros{#1}\setcounter{MLastIndex}{-1}\setcounter{MLastType}{1}} % Sections werden ueber das section-Feld im mmlabel-Tag identifiziert, nicht ueber das Indexfeld

\def\MSubsection#1{\ifnum\value{MSSEnd}>0{\MSubsectionEndMacros}\addtocounter{MSSEnd}{-1}\fi\ifttm\else\clearpage\fi\Dsubsection{#1}\MSubsectionStartMacros\setcounter{MLastIndex}{-1}\setcounter{MLastType}{2}\addtocounter{MSSEnd}{1}}% Subsections werden ueber das subsection-Feld im mmlabel-Tag identifiziert, nicht ueber das Indexfeld
\def\MSubsectionx#1{\Dsubsectionx{#1}} % Nur zur Verwendung in MSectionStart gedacht
\def\MSubsubsection#1{\Dsubsubsection{#1}\setcounter{MLastIndex}{\value{subsubsection}}\setcounter{MLastType}{3}\ifttm\special{html:<!-- sectioninfo;;}\arabic{section}\special{html:;;}\arabic{subsection}\special{html:;;}\arabic{subsubsection}\special{html:;;1;;}\arabic{MTestSite}\special{html:; //-->}\fi}
\def\MSubsubsectionx#1{\Dsubsubsectionx{#1}\ifttm\special{html:<!-- sectioninfo;;}\arabic{section}\special{html:;;}\arabic{subsection}\special{html:;;}\arabic{subsubsection}\special{html:;;0;;}\arabic{MTestSite}\special{html:; //-->}\else\addcontentsline{toc}{subsection}{#1}\fi}

\ifttm
\def\MSubsubsubsectionx#1{\ \newline\textbf{#1}\special{html:<br />}}
\else
\def\MSubsubsubsectionx#1{\ \newline
\textbf{#1}\ \\
}
\fi


% Dieses Skript wird zu Beginn jedes Modulabschnitts (=Webseite) ausgefuehrt und initialisiert den Aufgabenfeldzaehler
\newcommand{\MPageScripts}{
\setcounter{MFieldCounter}{1}
\addtocounter{MSiteCounter}{1}
\setcounter{MHintCounter}{1}
\setcounter{MCodeEditCounter}{1}
\setcounter{MGroupActive}{0}
\DoQBoxes
% Feldvariablen werden im HTML-Header in conv.pl eingestellt
}

% Dieses Skript wird zum Ende jedes Modulabschnitts (=Webseite) ausgefuehrt
\ifttm
\newcommand{\MEndScripts}{\special{html:<br /><!-- mfeedbackbutton;Seite;}\arabic{MTestSite}\special{html:;}\MGenerateSiteNumber\special{html:; //-->}
}
\else
\newcommand{\MEndScripts}{\relax}
\fi


\newcounter{QBoxFlag}
\newcommand{\DoQBoxes}{\setcounter{QBoxFlag}{1}}
\newcommand{\NoQBoxes}{\setcounter{QBoxFlag}{0}}

\newcounter{MXCTest}
\newcounter{MXCounter}
\newcounter{MSCounter}



\ifttm

% Struktur des sectioninfo-Tags: <!-- sectioninfo;;section;;subsection;;subsubsection;;nr_ausgeben;;testpage; //-->

%Fuegt eine zusaetzliche html-Seite an hinter ALLEN bisherigen und zukuenftigen content-Seiten ausserhalb der vor-zurueck-Schleife (d.h. nur durch Button oder MIntLink erreichbar!)
% #1 = Titel des Modulabschnitts, #2 = Kurztitel fuer die Buttons, #3 = Buttonkennung (STD = default nehmen, NONE = Ohne Button in der Navigation)
\newenvironment{MSContent}[3]{\special{html:<div class="xcontent}\arabic{MSCounter}\special{html:"><!-- scontent;-;}\arabic{MSCounter};-;#1;-;#2;-;#3\special{html: //-->}\MPageScripts\MSubsubsectionx{#1}}{\MEndScripts\special{html:<!-- endscontent;;}\arabic{MSCounter}\special{html: //--></div>}\addtocounter{MSCounter}{1}}

% Fuegt eine zusaetzliche html-Seite ein hinter den bereits vorhandenen content-Seiten (oder als erste Seite) innerhalb der vor-zurueck-Schleife der Navigation
% #1 = Titel des Modulabschnitts, #2 = Kurztitel fuer die Buttons, #3 = Buttonkennung (STD = Defaultbutton, NONE = Ohne Button in der Navigation)
\newenvironment{MXContent}[3]{\special{html:<div class="xcontent}\arabic{MXCounter}\special{html:"><!-- xcontent;-;}\arabic{MXCounter};-;#1;-;#2;-;#3\special{html: //-->}\MPageScripts\MSubsubsection{#1}}{\MEndScripts\special{html:<!-- endxcontent;;}\arabic{MXCounter}\special{html: //--></div>}\addtocounter{MXCounter}{1}}

% Fuegt eine zusaetzliche html-Seite ein die keine subsubsection-Nummer bekommt, nur zur internen Verwendung in mintmod.tex gedacht!
% #1 = Titel des Modulabschnitts, #2 = Kurztitel fuer die Buttons, #3 = Buttonkennung (STD = Defaultbutton, NONE = Ohne Button in der Navigation)
% \newenvironment{MUContent}[3]{\special{html:<div class="xcontent}\arabic{MXCounter}\special{html:"><!-- xcontent;-;}\arabic{MXCounter};-;#1;-;#2;-;#3\special{html: //-->}\MPageScripts\MSubsubsectionx{#1}}{\MEndScripts\special{html:<!-- endxcontent;;}\arabic{MXCounter}\special{html: //--></div>}\addtocounter{MXCounter}{1}}

\newcommand{\MDeclareSiteUXID}[1]{\special{html:<!-- mdeclaresiteuxid;;}#1\special{html:;;}\arabic{chapter}\special{html:;;}\arabic{section}\special{html:;; //-->}}

\else

%\newcommand{\MSubsubsection}[1]{\refstepcounter{subsubsection} \addcontentsline{toc}{subsubsection}{\thesubsubsection. #1}}


% Fuegt eine zusaetzliche html-Seite an hinter den bereits vorhandenen content-Seiten
% #1 = Titel des Modulabschnitts, #2 = Kurztitel fuer die Buttons, #3 = Iconkennung (im PDF wirkungslos)
%\newenvironment{MUContent}[3]{\ifnum\value{MXCTest}>0{\MDebugMessage{ERROR: Geschachtelter SContent}}\fi\MPageScripts\MSubsubsectionx{#1}\addtocounter{MXCTest}{1}}{\addtocounter{MXCounter}{1}\addtocounter{MXCTest}{-1}}
\newenvironment{MXContent}[3]{\ifnum\value{MXCTest}>0{\MDebugMessage{ERROR: Geschachtelter SContent}}\fi\MPageScripts\MSubsubsection{#1}\addtocounter{MXCTest}{1}}{\addtocounter{MXCounter}{1}\addtocounter{MXCTest}{-1}}
\newenvironment{MSContent}[3]{\ifnum\value{MXCTest}>0{\MDebugMessage{ERROR: Geschachtelter XContent}}\fi\MPageScripts\MSubsubsectionx{#1}\addtocounter{MXCTest}{1}}{\addtocounter{MSCounter}{1}\addtocounter{MXCTest}{-1}}

\newcommand{\MDeclareSiteUXID}[1]{\relax}

\fi

% GHEADER und GFOOTER werden von split.pm gefunden, aber nur, wenn nicht HELPSITE oder TESTSITE
\ifttm
\newenvironment{MSectionStart}{\special{html:<div class="xcontent0">}\MSubsubsectionx{\iModuleOverview{}}}{\setcounter{MSSEnd}{0}\special{html:</div>}}
% Darf nicht als XContent nummeriert werden, darf nicht als XContent gelabelt werden, wird aber in eine xcontent-div gesetzt fuer Python-parsing
\else
\newenvironment{MSectionStart}{\MSubsectionx{\iModuleOverview{}}}{\setcounter{MSSEnd}{0}}
\fi

\ifttm
\newenvironment{MIntro}{\begin{MXContent}{\iIntroduction{}}{\iIntroduction{}}{genetisch}\begin{html}<div
class="intro">\end{html}}{\begin{html}</div>\end{html}\end{MXContent}}
\else
\newenvironment{MIntro}{\begin{MXContent}{\iIntroduction{}}{\iIntroduction{}}{genetisch}}{\end{MXContent}}
\fi

\newenvironment{MContent}{\begin{MXContent}{\iContents{}}{\iContents{}}{beweis}}{\end{MXContent}}
\newenvironment{MExercises}{\ifttm\else\clearpage\fi\begin{MXContent}{\iExercises{}}{\iExercises{}}{aufgb}\special{html:<!-- declareexcsymb //-->}}{\end{MXContent}}

% #1 = Lesbare Testbezeichnung
\newenvironment{MTest}[1]{%
\renewcommand{\MTestName}{#1}
\ifttm\else\clearpage\fi%
\addtocounter{MTestSite}{1}%
\begin{MXContent}{#1}{#1}{STD} % {aufgb}%
\special{html:<!-- declaretestsymb //-->}
\begin{MQuestionGroup}%
\MInTestHeader
}%
{%
\end{MQuestionGroup}%
\ \\ \ \\%
\MInTestFooter
\end{MXContent}\addtocounter{MTestSite}{-1}%
}

\newenvironment{MExtra}{\ifttm\else\clearpage\fi\begin{MXContent}{Zusätzliche Inhalte}{Zusatz}{weiterfhrg}}{\end{MXContent}}

\makeindex

\ifttm
\def\MPrintIndex{
\ifnum\value{MSSEnd}>0{\MSubsectionEndMacros}\addtocounter{MSSEnd}{-1}\fi
\renewcommand{\indexname}{Index}
\special{html:<p><!-- printindex //--></p>}
}
\else
\def\MPrintIndex{
\ifnum\value{MSSEnd}>0{\MSubsectionEndMacros}\addtocounter{MSSEnd}{-1}\fi
\renewcommand{\indexname}{Index}
\addcontentsline{toc}{section}{Index}
\printindex
}
\fi


% Konstanten fuer die Modulfaecher

\def\MINTMathematics{1}
\def\MINTInformatics{2}
\def\MINTChemistry{3}
\def\MINTPhysics{4}
\def\MINTEngineering{5}

\newcounter{MSubjectArea}
\newcounter{MInfoNumbers} % Gibt an, ob die Infoboxen nummeriert werden sollen
\newcounter{MSepNumbers} % Gibt an, ob Beispiele und Experimente separat nummeriert werden sollen
\newcommand{\MSetSubject}[1]{
 % ttm kapiert setcounter mit Parametern nicht, also per if abragen und einsetzen
\ifnum#1=1\setcounter{MSubjectArea}{1}\setcounter{MInfoNumbers}{1}\setcounter{MSepNumbers}{0}\fi
\ifnum#1=2\setcounter{MSubjectArea}{2}\setcounter{MInfoNumbers}{1}\setcounter{MSepNumbers}{0}\fi
\ifnum#1=3\setcounter{MSubjectArea}{3}\setcounter{MInfoNumbers}{0}\setcounter{MSepNumbers}{1}\fi
\ifnum#1=4\setcounter{MSubjectArea}{4}\setcounter{MInfoNumbers}{0}\setcounter{MSepNumbers}{0}\fi
\ifnum#1=5\setcounter{MSubjectArea}{5}\setcounter{MInfoNumbers}{1}\setcounter{MSepNumbers}{0}\fi
% Separate Nummerntechnik fuer unsere Chemiker: alles dreistellig
\ifnum#1=3
  \ifttm
  \renewcommand{\theequation}{\arabic{section}.\arabic{subsection}.\arabic{equation}}
  \renewcommand{\thetable}{\arabic{section}.\arabic{subsection}.\arabic{table}}
  \renewcommand{\thefigure}{\arabic{section}.\arabic{subsection}.\arabic{figure}}
  \else
  \renewcommand{\theequation}{\arabic{chapter}.\arabic{section}.\arabic{equation}}
  \renewcommand{\thetable}{\arabic{chapter}.\arabic{section}.\arabic{table}}
  \renewcommand{\thefigure}{\arabic{chapter}.\arabic{section}.\arabic{figure}}
  \fi
\else
  \ifttm
  \renewcommand{\theequation}{\arabic{section}.\arabic{subsection}.\arabic{equation}}
  \renewcommand{\thetable}{\arabic{table}}
  \renewcommand{\thefigure}{\arabic{figure}}
  \else
  \renewcommand{\theequation}{\arabic{chapter}.\arabic{section}.\arabic{equation}}
  \renewcommand{\thetable}{\arabic{table}}
  \renewcommand{\thefigure}{\arabic{figure}}
  \fi
\fi
}

% Fuer tikz Autogenerierung
\newcounter{MTIKZAutofilenumber}

% Spezielle Counter fuer die Bentz-Module
\newcounter{mycounter}
\newcounter{chemapplet}
\newcounter{physapplet}

\newcounter{MSSEnd} % Ist 1 falls ein MSubsection aktiv ist, der einen MSubsectionEndMacro-Aufruf verursacht
\newcounter{MFileNumber}
\def\MLastFile{\special{html:[[!-- mfileref;;}\arabic{MFileNumber}\special{html:; //--]]}}

% Vollstaendiger Pfad ist \MMaterial / \MLastFilePath / \MLastFileName    ==   \MMaterial / \MLastFile

% Wird nur bei kompletter Baum-Erstellung ausgefuehrt!
% #1 = Lesbare Modulbezeichnung
\newcommand{\MSectionStartMacros}[1]{
\setcounter{MTestSite}{0}
\setcounter{MCaptionOn}{0}
\setcounter{MLastTypeEq}{0}
\setcounter{MSSEnd}{0}
\setcounter{MFileNumber}{0} % Preinkrekement-Counter
\setcounter{MTIKZAutofilenumber}{0}
\setcounter{mycounter}{1}
\setcounter{physapplet}{1}
\setcounter{chemapplet}{0}
\ifttm
\special{html:<!-- mdeclaresection;;}\arabic{chapter}\special{html:;;}\arabic{section}\special{html:;;}#1\special{html:;; //-->}%
\else
\setcounter{thmc}{0}
\setcounter{exmpc}{0}
\setcounter{verc}{0}
\setcounter{infoc}{0}
\fi
\setcounter{MiniMarkerCounter}{1}
\setcounter{AlignCounter}{1}
\setcounter{MXCTest}{0}
\setcounter{MCodeCounter}{0}
\setcounter{MEntryCounter}{0}
}

% Wird immer ausgefuehrt
\newcommand{\MSubsectionStartMacros}{
\ifttm\else\MPageHeaderDef\fi
\MWatermarkSettings
\setcounter{MXCounter}{0}
\setcounter{MSCounter}{0}
\setcounter{MSiteCounter}{1}
\setcounter{MExerciseCollectionCounter}{0}
% Zaehler fuer das Labelsystem zuruecksetzen (prefix-Zaehler)
\setcounter{MInfoCounter}{0}
\setcounter{MExerciseCounter}{0}
\setcounter{MExampleCounter}{0}
\setcounter{MExperimentCounter}{0}
\setcounter{MGraphicsCounter}{0}
\setcounter{MTableCounter}{0}
\setcounter{MTheoremCounter}{0}
\setcounter{MObjectCounter}{0}
\setcounter{MEquationCounter}{0}
\setcounter{MVideoCounter}{0}
\setcounter{equation}{0}
\setcounter{figure}{0}
}

\newcommand{\MSubsectionEndMacros}{
% Bei Chemiemodulen das PSE einhaengen, es soll als SContent am Ende erscheinen
\special{html:<!-- subsectionend //-->}
\ifnum\value{MSubjectArea}=3{\MIncludePSE}\fi
}


\ifttm
%\newcommand{\MEmbed}[1]{\MRegisterFile{#1}\begin{html}<embed src="\end{html}\MMaterial/\MLastFile\begin{html}" width="192" height="189"></embed>\end{html}}
\newcommand{\MEmbed}[1]{\MRegisterFile{#1}\begin{html}<embed src="\end{html}\MMaterial/\MLastFile\begin{html}"></embed>\end{html}}
\fi

%----------------- Makros fuer die Textdarstellung -----------------------------------------------

\ifttm
% MUGraphics bindet eine Grafik ein:
% Parameter 1: Dateiname der Grafik, relativ zur Position des Modul-Tex-Dokuments
% Parameter 2: Skalierungsoptionen fuer PDF (fuer includegraphics)
% Parameter 3: Titel fuer die Grafik, wird unter die Grafik mit der Grafiknummer gesetzt und kann MLabel bzw. MCopyrightLabel enthalten
% Parameter 4: Skalierungsoptionen fuer HTML (css-styles)

% ERSATZ: <img alt="My Image" src="data:image/png;base64,iVBORwA<MoreBase64SringHere>" />


\newcommand{\MUGraphics}[4]{\MRegisterFile{#1}\begin{html}
<div class="imagecenter">
<center>
<div>
%<img src="\end{html}\MMaterial/\MLastFile\begin{html}" style="#4" alt="\end{html}\MMaterial/\MLastFile\begin{html}"/>
<img class="mintmodimage" src="../../images/\end{html}#1\begin{html}"/>
</div>
<div class="bildtext">
\end{html}
\addtocounter{MGraphicsCounter}{1}
\setcounter{MLastIndex}{\value{MGraphicsCounter}}
\setcounter{MLastType}{8}
\addtocounter{MCaptionOn}{1}
\ifnum\value{MSepNumbers}=0
\textbf{Abbildung \arabic{MGraphicsCounter}:} #3
\else
\textbf{Abbildung \arabic{section}.\arabic{subsection}.\arabic{MGraphicsCounter}:} #3
\fi
\addtocounter{MCaptionOn}{-1}
\begin{html}
</div>
</center>
</div>
<br />
\end{html}%
\special{html:<!-- mfeedbackbutton;Abbildung;}\arabic{MGraphicsCounter}\special{html:;}\arabic{section}.\arabic{subsection}.\arabic{MGraphicsCounter}\special{html:; //-->}%
}

% MVideo bindet ein Video als Einzeldatei ein:
% Parameter 1: Dateiname des Videos, relativ zur Position des Modul-Tex-Dokuments, ohne die Endung ".mp4"
% Parameter 2: Titel fuer das Video (kann MLabel oder MCopyrightLabel enthalten), wird unter das Video mit der Videonummer gesetzt
\newcommand{\MVideo}[2]{\MRegisterFile{#1.mp4}\begin{html}
<div class="imagecenter">
<center>
<div>
<video width="95\%" controls="controls"><source src="\end{html}\MMaterial/#1.mp4\begin{html}" type="video/mp4">Ihr Browser kann keine MP4-Videos abspielen!</video>
</div>
<div class="bildtext">
\end{html}
\addtocounter{MVideoCounter}{1}
\setcounter{MLastIndex}{\value{MVideoCounter}}
\setcounter{MLastType}{12}
\addtocounter{MCaptionOn}{1}
\ifnum\value{MSepNumbers}=0
\textbf{Video \arabic{MVideoCounter}:} #2
\else
\textbf{Video \arabic{section}.\arabic{subsection}.\arabic{MVideoCounter}:} #2
\fi
\addtocounter{MCaptionOn}{-1}
\begin{html}
</div>
</center>
</div>
<br />
\end{html}}

\newcommand{\MDVideo}[2]{\MRegisterFile{#1.mp4}\MRegisterFile{#1.ogv}\begin{html}
<div class="imagecenter">
<center>
<div>
<video width="70\%" controls><source src="\end{html}\MMaterial/#1.mp4\begin{html}" type="video/mp4"><source src="\end{html}\MMaterial/#1.ogv\begin{html}" type="video/ogg">Ihr Browser kann keine MP4-Videos abspielen!</video>
</div>
<br />
#2
</center>
</div>
<br />
\end{html}
}

\newcommand{\MGraphics}[3]{\MUGraphics{#1}{#2}{#3}{}}

\else

\newcommand{\MVideo}[2]{%
% Kein Video im PDF darstellbar, trotzdem so tun als ob da eines waere
\begin{center}
(\iVideoWarn{})
\end{center}
\addtocounter{MVideoCounter}{1}
\setcounter{MLastIndex}{\value{MVideoCounter}}
\setcounter{MLastType}{12}
\addtocounter{MCaptionOn}{1}
\ifnum\value{MSepNumbers}=0
\textbf{Video \arabic{MVideoCounter}:} #2
\else
\textbf{Video \arabic{section}.\arabic{subsection}.\arabic{MVideoCounter}:} #2
\fi
\addtocounter{MCaptionOn}{-1}
}


% MGraphics bindet eine Grafik ein:
% Parameter 1: Dateiname der Grafik, relativ zur Position des Modul-Tex-Dokuments
% Parameter 2: Skalierungsoptionen fuer PDF (fuer includegraphics)
% Parameter 3: Titel fuer die Grafik, wird unter die Grafik mit der Grafiknummer gesetzt
\newcommand{\MGraphics}[3]{%
\MRegisterFile{#1}%
\ %
\begin{figure}[H]%
\centering{%
\includegraphics[#2]{\MDPrefix/#1}%
\addtocounter{MCaptionOn}{1}%
\caption{#3}%
\addtocounter{MCaptionOn}{-1}%
}%
\end{figure}%
\addtocounter{MGraphicsCounter}{1}\setcounter{MLastIndex}{\value{MGraphicsCounter}}\setcounter{MLastType}{8}\ %
%\ \\Abbildung \ifnum\value{MSepNumbers}=0\else\arabic{chapter}.\arabic{section}.\fi\arabic{MGraphicsCounter}: #3%
}

\newcommand{\MUGraphics}[4]{\MGraphics{#1}{#2}{#3}}


\fi

\newcounter{MCaptionOn} % = 1 falls eine Grafikcaption aktiv ist, = 0 sonst


% MGraphicsSolo bindet eine Grafik pur ein ohne Titel
% Parameter 1: Dateiname der Grafik, relativ zur Position des Modul-Tex-Dokuments
% Parameter 2: Skalierungsoptionen (wirken nur im PDF)
\newcommand{\MGraphicsSolo}[2]{\MUGraphicsSolo{#1}{#2}{}}

% MUGraphicsSolo bindet eine Grafik pur ein ohne Titel, aber mit HTML-Skalierung
% Parameter 1: Dateiname der Grafik, relativ zur Position des Modul-Tex-Dokuments
% Parameter 2: Skalierungsoptionen (wirken nur im PDF)
% Parameter 3: Skalierungsoptionen (wirken nur im HTML), als style-format: "width=???, height=???"
\ifttm
\newcommand{\MUGraphicsSolo}[3]{\MRegisterFile{#1}
% unecessarily convoluted. Use deploy script for images, not tex converter.
%\begin{html}
%<img src="\end{html}\MMaterial/\MLastFile\begin{html}" style="\end{html}#3\begin{html}" alt="\end{html}\MMaterial/\MLastFile\begin{html}"/>
\begin{html}<img class="mintmodimage" src="../../images/\end{html}#1\begin{html}"/>\end{html}
%\end{html}%
\special{html:<!-- mfeedbackbutton;Abbildung;}#1\special{html:;}\MMaterial/\MLastFile\special{html:; //-->}%
}
\else
\newcommand{\MUGraphicsSolo}[3]{\MRegisterFile{#1}\includegraphics[#2]{\MDPrefix/#1}}
\fi

% Externer Link mit URL
% Erster Parameter: Vollstaendige(!) URL des Links
% Zweiter Parameter: Text fuer den Link
\newcommand{\MExtLink}[2]{\ifttm\special{html:<a target="_new" href="}#1\special{html:">}#2\special{html:</a>}\else\href{#1}{#2}\fi} % ohne MINTERLINK!


% Interner Link, die verlinkte Datei muss im gleichen Verzeichnis liegen wie die Modul-Texdatei
% Erster Parameter: Dateiname
% Zweiter Parameter: Text fuer den Link
\newcommand{\MIntLink}[2]{\ifttm\MRegisterFile{#1}\special{html:<a class="MINTERLINK" target="_new" href="}\MMaterial/\MLastFile\special{html:">}#2\special{html:</a>}\else{\href{#1}{#2}}\fi}


\ifttm
\def\MMaterial{:localmaterial:}
\else
\def\MMaterial{\MDPrefix}
\fi

\ifttm
\def\MNoFile#1{:directmaterial:#1}
\else
\def\MNoFile#1{#1}
\fi

\newcommand{\MChem}[1]{$\mathrm{#1}$}

\newcommand{\MApplet}[3]{
% Bindet ein Java-Applet ein, die Parameter sind:
% (wird nur im HTML, aber nicht im PDF erstellt)
% #1 Dateiname des Applets (muss mit ".class" enden)
% #2 = Breite in Pixeln
% #3 = Hoehe in Pixeln
\ifttm
\MRegisterFile{#1}
\begin{html}
<applet code="\end{html}\MMaterial/\MLastFile\begin{html}" width="#2" height="#3" alt="[Java-Applet kann nicht gestartet werden]"></applet>
\end{html}
\fi
}

\newcommand{\MScriptPage}[2]{
% Bindet eine JavaScript-Datei ein, die eine eigene Seite bekommt
% (wird nur im HTML, aber nicht im PDF erstellt)
% #1 Dateiname des Programms (sollte mit ".js" enden)
% #2 = Kurztitel der Seite
\ifttm
\begin{MSContent}{#2}{#2}{puzzle}
\MRegisterFile{#1}
\begin{html}
<script src="\MMaterial/\MLastFile" type="text/javascript"></script>
\end{html}
\end{MSContent}
\fi
}

% Bindet in der Haupttexdatei ein MINT-Modul ein. Parameter 1 ist das Verzeichnis (relativ zur Haupttexdatei), Parameter 2 ist der Dateinahme ohne Pfad.
\newcommand{\IncludeModule}[2]{
\renewcommand{\MDPrefix}{#1}
\input{#1/#2}
\ifnum\value{MSSEnd}>0{\MSubsectionEndMacros}\addtocounter{MSSEnd}{-1}\fi
}

% Der ttm-Konverter setzt keine Makros im \input um, also muss hier getrickst werden:
% Das MDPrefix muss in den einzelnen Modulen manuell eingesetzt werden
\newcommand{\MInputFile}[1]{
\ifttm
\input{#1}
\else
\input{#1}
\fi
}


\newcommand{\MCases}[1]{\left\lbrace{\begin{array}{rl} #1 \end{array}}\right.}

\ifttm
\newenvironment{MCaseEnv}{\left\lbrace\begin{array}{rl}}{\end{array}\right.}
\else
\newenvironment{MCaseEnv}{\left\lbrace\begin{array}{rl}}{\end{array}\right.}
\fi

\def\MSkip{\ifttm\MCR\fi}

\ifttm
\def\MCR{\special{html:<br />}}
\else
\def\MCR{\ \\}
\fi


% Pragmas - Sind Schluesselwoerter, die dem Preprocessing sowie dem Konverter uebergeben werden und bestimmte
%           Aktionen ausloesen. Im Output (PDF und HTML) tauchen sie nicht auf.
\newcommand{\MPragma}[1]{%
\ifttm%
\special{html:<!-- mpragma;-;}#1\special{html:;; -->}%
\else%
% MPragmas werden vom Preprozessor direkt im LaTeX gefunden
\fi%
}

% Ersatz der Befehle textsubscript und textsuperscript, die ttm nicht kennt
\ifttm%
\newcommand{\MTextsubscript}[1]{\special{html:<sub>}#1\special{html:</sub>}}%
\newcommand{\MTextsuperscript}[1]{\special{html:<sup>}#1\special{html:</sup>}}%
\else%
\newcommand{\MTextsubscript}[1]{\textsubscript{#1}}%
\newcommand{\MTextsuperscript}[1]{\textsuperscript{#1}}%
\fi

%------------------ Einbindung von dia-Diagrammen ----------------------------------------------
% Beim preprocessing wird aus jeder dia-Datei eine tex-Datei und eine pdf-Datei erzeugt,
% diese werden hier jeweils im PDF und HTML eingebunden
% Parameter: Dateiname der mit dia erstellten Datei (OHNE die Endung .dia)
\ifttm%
\newcommand{\MDia}[1]{%
\MGraphicsSolo{#1minthtml.png}{}%
}
\else%
\newcommand{\MDia}[1]{%
\MGraphicsSolo{#1mintpdf.png}{scale=0.1667}%
}
\fi%

% subsup funktioniert im Ausdruck $D={\R}^+_0$, also \R geklammert und sup zuerst
% \ifttm
% \def\MSubsup#1#2#3{\special{html:<msubsup>} #1 #2 #3\special{html:</msubsup>}}
% \else
% \def\MSubsup#1#2#3{{#1}^{#3}_{#2}}
% \fi

%\input{local.tex}

% \ifttm
% \else
% \newwrite\mintlog
% \immediate\openout\mintlog=mintlog.txt
% \fi

% ----------------------- tikz autogenerator -------------------------------------------------------------------

\newcommand{\Mtikzexternalize}{\tikzexternalize}% wird bei Konvertierung ueber mconvert ggf. ausgehebelt!

\ifttm
\else
\tikzset%
{
  % Defines a custom style which generates pdf and converts to (low and hi-res quality) png and svg, then deletes the pdf
  % Important: DO NOT directly convert from pdf to hires-png or from svg to png with GraphViz convert as it has some problems and memory leaks
  png export/.style=%
  {
    external/system call/.add={}{;
      pdf2svg "\image.pdf" "\image.svg" ;
      convert -density 112.5 -transparent white "\image.pdf" "\image.png";
      inkscape --export-png="\image.4x.png" --export-dpi=450 --export-background-opacity=0 --without-gui "\image.svg";
      rm "\image.pdf"; rm "\image.log"; rm "\image.dpth"; rm "\image.idx"
    },
    external/force remake,
  }
}
\tikzset{png export}
\tikzsetexternalprefix{}
% PNGs bei externer Erzeugung in "richtiger" Groesse einbinden
\pgfkeys{/pgf/images/include external/.code={\includegraphics[scale=0.64]{#1}}}
\fi

% Spezielle Umgebung fuer Autogenerierung, Bildernamen sind nur innerhalb eines Moduls (einer MSection) eindeutig)

\newcommand{\MTIKZautofilename}{tikzautofile}

\ifttm
% HTML-Version: Vom Autogenerator erzeugte png-Datei einbinden, tikz selbst nicht ausfuehren (sprich: #1 schlucken)
\newcommand{\MTikzAuto}[1]{%
\addtocounter{MTIKZAutofilenumber}{1}
\renewcommand{\MTIKZautofilename}{mtikzauto_\arabic{MTIKZAutofilenumber}}
\MUGraphicsSolo{\MSectionID\MTIKZautofilename.4x.png}{scale=1}{\special{html:[[!-- svgstyle;}\MSectionID\MTIKZautofilename\special{html: //--]]}} % Styleinfos werden aus original-png, nicht 4x-png geholt!
%\MRegisterFile{\MSectionID\MTIKZautofilename.png} % not used right now
%\MRegisterFile{\MSectionID\MTIKZautofilename.svg}
}
\else%
% PDF-Version: Falls Autogenerator aktiv wird Datei automatisch benannt und exportiert
\newcommand{\MTikzAuto}[1]{%
\addtocounter{MTIKZAutofilenumber}{1}%
\renewcommand{\MTIKZautofilename}{mtikzauto_\arabic{MTIKZAutofilenumber}}
\tikzsetnextfilename{\MTIKZautofilename}%
#1%
}
\fi

% In einer reinen LaTeX-Uebersetzung kapselt der Preambelinclude-Befehl nur input,
% in einer konvertergesteuerten PDF/HTML-Uebersetzung wird er dagegen entfernt und
% die Preambeln an mintmod angehaengt, die Ersetzung wird von mconvert.pl vorgenommen.

\newcommand{\MPreambleInclude}[1]{\input{#1}}

% Globale Watermarksettings (werden auch nochmal zu Beginn jedes subsection gesetzt,
% muessen hier aber auch global ausgefuehrt wegen Einfuehrungsseiten und Inhaltsverzeichnis

\MWatermarkSettings

% ---------------------------------- Spezialbefehle fuer AD ------------------------------------------

%Abkuerzung fuer \longrightarrow:
\newcommand{\lto}{\ensuremath{\longrightarrow}}

%Makro fuer Funktionen:
\newcommand{\exfunction}[5]
{\begin{array}{rrcl}
 #1 \colon  & #2 &\lto & #3 \\[.05cm]
  & #4 &\longmapsto  & #5
\end{array}}

\newcommand{\function}[5]{%
#1:\;\left\lbrace{\begin{array}{rcl}
 #2 &\lto & #3 \\
 #4 &\longmapsto  & #5 \end{array}}\right.}


%Die Identitaet:
\DeclareMathOperator{\Id}{Id}

%Die Signumfunktion:
\DeclareMathOperator{\sgn}{sgn}

%Zwei Betonungskommandos (koennen angepasst werden):
\newcommand{\highlight}[1]{#1}
\newcommand{\modstextbf}[1]{#1}
\newcommand{\modsemph}[1]{#1}


\ifttm%
\newcommand{\MModstartBox}{\special{html:<!-- modstartbox //-->}}
\else%
\newcommand{\MModstartBox}{%
\relax
}
\fi

% ---------------------------------- Spezialbefehle fuer JL ------------------------------------------


\def\jccolorfkt{green!50!black} %Farbe des Funktionsgraphen
\def\jccolorfktarea{green!25!white} %Farbe der Fläche unter dem Graphen
\def\jccolorfktareahell{green!12!white} %helle Einfärbung der Fläche unter dem Graphen
\def\jccolorfktwert{green!50!black} %Farbe einzelner Punkte des Graphen

\newcommand{\MPfadBilder}{Bilder}

\ifttm%
\newcommand{\jMD}{\,\MD}%
\else%
\newcommand{\jMD}{\;\MD}%
\fi%

\def\MFormelZoomHint{\MInputHint{%
\iFormZoomHint{}}}

\def\jHTMLHinweisBedienung{\MInputHint{%
\iUsageHint{}}}

\def\jHTMLHinweisEingabeText{\MInputHint{%
\iTextInputHint{}}}

\def\jHTMLHinweisEingabeTerm{\MInputHint{%
\iTermInputHint{}}}

\def\jHTMLHinweisEingabeIntervalle{\MInputHint{%
\iIntervalInputHint{}}}

\def\jHTMLHinweisEingabeFunktionen{\MInputHint{%
\iFunctionInputHint{}}}

\def\jHTMLHinweisEingabeFunktionenSinCos{\MInputHint{%
\iSincosInputHint{}}}

\def\jHTMLHinweisEingabeFunktionenExp{\MInputHint{%
iExpInputHint{}}}

% ---------------------------------- Spezialbefehle fuer Fachbereich Physik --------------------------

\newcommand{\E}{{e}}
\newcommand{\ME}[1]{\cdot 10^{#1}}
\newcommand{\MU}[1]{\;\mathrm{#1}}
\newcommand{\MPG}[3]{%
  \ifnum#2=0%
    #1\ \mathrm{#3}%
  \else%
    #1\cdot 10^{#2}\ \mathrm{#3}%
  \fi}%
%

\newcommand{\MMul}{\MExponentensymbXYZl} % Nur eine Abkuerzung


% ---------------------------------- Stichwortfunktionialitaet ---------------------------------------

% mpreindexentry wird durch Auswahlroutine in conv.pl durch mindexentry substitutiert
\ifttm%
\def\MIndex#1{\index{#1}\special{html:<!-- mpreindexentry;;}#1\special{html:;;}\arabic{MSubjectArea}\special{html:;;}%
\arabic{chapter}\special{html:;;}\arabic{section}\special{html:;;}\arabic{subsection}\special{html:;;}\arabic{MEntryCounter}\special{html:; //-->}%
\setcounter{MLastIndex}{\value{MEntryCounter}}%
\addtocounter{MEntryCounter}{1}%
}%
% Copyrightliste wird als tex-Datei im preprocessing von conv.pl erzeugt und unter converter/tex/entrycollection.tex abgelegt
% Der input-Befehl funktioniert nur, wenn die aufrufende tex-Datei auf der obersten Ebene liegt (d.h. selbst kein input/include ist, insbesondere keine Moduldatei)
\def\MEntryList{} % \input funktioniert nicht, weil ttm (und damit das \input) ausgefuehrt wird, bevor Datei da ist
\else%
\def\MIndex#1{\index{#1}}
\def\MEntryList{\MAbort{Stichwortliste nur im HTML realisierbar}}%
\fi%

\def\MEntry#1#2{\textbf{#1}\MIndex{#2}} % Idee: MLastType auf neuen Entry-Typ und dann ein MLabel vergeben mit autogen-Nummer

% ---------------------------------- Befehle fuer Tests ----------------------------------------------

% MEquationItem stellt eine Eingabezeile der Form Vorgabe = Antwortfeld her, der zweite Parameter kann z.B. MSimplifyQuestion-Befehl sein
\ifttm
\newcommand{\MEquationItem}[2]{{#1}$\,=\,${#2}}%
\else%
\newcommand{\MEquationItem}[2]{{#1}$\;\;=\,${#2}}%
\fi

\ifttm
\newcommand{\MInputHint}[1]{%
\ifnum%
\if\value{MTestSite}>0%
\else%
\begin{html}<span class="text-info">\end{html}{#1}\begin{html}</span>\end{html}
\fi%
\fi%
}
\else
\newcommand{\MInputHint}[1]{\relax}
\fi

\ifttm
\newcommand{\MInTestHeader}{%
\iTest{}}
\else
\newcommand{\MInTestHeader}{%
\relax
}
\fi

\ifttm
\newcommand{\MInTestFooter}{%
\special{html:<button name="Name_TESTFINISH" id="TESTFINISH" type="button" onclick="finish_button('}\MTestName\special{html:');" class="testsbutton">}\iTestSubmit{}\special{html:</button>}%
\special{html:
&nbsp;&nbsp;&nbsp;&nbsp;&nbsp;&nbsp;&nbsp;&nbsp;
<button name="Name_TESTRESET" id="TESTRESET" type="button" onclick="reset_button();" class="testsbutton">}\iTestReset{}\special{html:</button>
<br />
<br />
<div class="xreply">
<p name="Name_TESTEVAL" id="TESTEVAL">}
\iTestEval{}
\special{html:<br />
</p>
</div>}
}
\else
\newcommand{\MInTestFooter}{%
\relax
}
\fi


% ---------------------------------- Notationsmakros -------------------------------------------------------------

% Notationsmakros die nicht von der Kursvariante abhaengig sind

\ifttm%
\def\MTextSF#1{\textrm{#1}}% im HTML zerreisst ein font-Wechsel auf sans-serif die Zeilenhoehe
\else%
\def\MTextSF#1{\textsf{#1}}%
\fi%

\newcommand{\MZahltrennzeichen}[1]{\renewcommand{\MZXYZhltrennzeichen}{#1}}
\newcommand{\MGrad}{{}^{\circ}} % Fuer Winkel im Gradmass

\ifttm
\newcommand{\MZahl}[3][\MZXYZhltrennzeichen]{\edef\MZXYZtemp{\noexpand\special{html:<mn>#2#1#3</mn>}}\MZXYZtemp}
\else
\newcommand{\MZahl}[3][\MZXYZhltrennzeichen]{{}#2{#1}#3}
\fi

\newcommand{\MEinheitenabstand}[1]{\renewcommand{\MEinheitenabstXYZnd}{#1}}
\ifttm
\newcommand{\MEinheit}[2][\MEinheitenabstXYZnd]{{}#1\edef\MEINHtemp{\noexpand\special{html:<mi mathvariant="normal">#2</mi>}}\MEINHtemp}
\else
\newcommand{\MEinheit}[2][\MEinheitenabstXYZnd]{{}#1 \mathrm{#2}}
\fi

\newcommand{\MExponentensymbol}[1]{\renewcommand{\MExponentensymbXYZl}{#1}}
\newcommand{\MExponent}[2][\MExponentensymbXYZl]{{}#1{} 10^{#2}}

%Punkte in 2 und 3 Dimensionen
\newcommand{\MPointTwo}[3][]{#1(#2\MCoordPointSep #3{}#1)}
\newcommand{\MPointThree}[4][]{#1(#2\MCoordPointSep #3\MCoordPointSep #4{}#1)}
\newcommand{\MPointTwoAS}[2]{\left(#1\MCoordPointSep #2\right)}
\newcommand{\MPointThreeAS}[3]{\left(#1\MCoordPointSep #2\MCoordPointSep #3\right)}

% Masseinheit, Standardabstand: \,
\newcommand{\MEinheitenabstXYZnd}{\MThinspace}

% Horizontaler Leerraum zwischen herausgestellter Formel und Interpunktion
\ifttm
\newcommand{\MDFPSpace}{\,}
\newcommand{\MDFPaSpace}{\,\,}
\newcommand{\MBlank}{\ }
\else
\newcommand{\MDFPSpace}{\;}
\newcommand{\MDFPaSpace}{\;\;}
\newcommand{\MBlank}{\ }
\fi

% Satzende in herausgestellter Formel mit horizontalem Leerraum
\newcommand{\MDFPeriod}{\MDFPSpace .}

% Separation von Aufzaehlung und Bedingung in Menge
\newcommand{\MCondSetSep}{\,:\,} %oder '\mid'

% Konverter kennt mathopen nicht
\ifttm
\def\mathopen#1{}
\fi

% -----------------------------------START Rouletteaufgaben ------------------------------------------------------------

\ifttm
% #1 = Dateiname, #2 = eindeutige ID fuer das Roulette im Kurs
\newcommand{\MDirectRouletteExercises}[2]{
\begin{MExercise}
\texttt{Im HTML erscheinen hier Aufgaben aus einer Aufgabenliste...}
\end{MExercise}
}
\else
\newcommand{\MDirectRouletteExercises}[2]{\relax} % wird durch mconvert.pl gefunden und ersetzt
\fi


% ---------------------------------- START Makros, die von der Kursvariante abhaengen ----------------------------------

\ifvariantunotation
  % unotation = An Universitaeten uebliche Notation
  \def\MVariant{unotation}

  % Trennzeichen fuer Dezimalzahlen
  \newcommand{\MZXYZhltrennzeichen}{.}

  % Exponent zur Basis 10 in der Exponentialschreibweise,
  % Standardmalzeichen: \times
  \newcommand{\MExponentensymbXYZl}{\times}

  % Begrenzungszeichen fuer offene Intervalle
  \newcommand{\MoIl}[1][]{\mbox{}#1(\mathopen{}} % bzw. ']'
  \newcommand{\MoIr}[1][]{#1)\mbox{}} % bzw. '['

  % Zahlen-Separation im IntervaLL
  \newcommand{\MIntvlSep}{,} %oder ';'

  % Separation von Elementen in Mengen
  \newcommand{\MElSetSep}{,} %oder ';'

  % Separation von Koordinaten in Punkten
  \newcommand{\MCoordPointSep}{,} %oder ';' oder '|', '\MThinspace|\MThinspace'

\else
  % An dieser Stelle wird angenommen, dass std-Variante aktiv ist
  % std = beschlossene Notation im TU9-Onlinekurs
  \def\MVariant{std}

  % Trennzeichen fuer Dezimalzahlen
  \newcommand{\MZXYZhltrennzeichen}{.}

  % Exponent zur Basis 10 in der Exponentialschreibweise,
  % Standardmalzeichen: \times
  \newcommand{\MExponentensymbXYZl}{\times}

  % Begrenzungszeichen fuer offene Intervalle
  \newcommand{\MoIl}[1][]{\mbox{}#1]\mathopen{}} % bzw. '('
  \newcommand{\MoIr}[1][]{#1[\mbox{}} % bzw. ')'

  % Zahlen-Separation im IntervaLL
  \newcommand{\MIntvlSep}{;} %oder ','

  % Separation von Elementen in Mengen
  \newcommand{\MElSetSep}{;} %oder ','

  % Separation von Koordinaten in Punkten
  \newcommand{\MCoordPointSep}{;} %oder '|', '\MThinspace|\MThinspace'

\fi



% ---------------------------------- ENDE Makros, die von der Kursvariante abhaengen ----------------------------------


% diese Kommandos setzen Mathemodus vorraus
\newcommand{\MGeoAbstand}[2]{[\overline{{#1}{#2}}]}
\newcommand{\MGeoGerade}[2]{{#1}{#2}}
\newcommand{\MGeoStrecke}[2]{\overline{{#1}{#2}}}
\newcommand{\MGeoDreieck}[3]{{#1}{#2}{#3}}

%
\ifttm
\newcommand{\MOhm}{\special{html:<mn>&#x3A9;</mn>}}
\else
\newcommand{\MOhm}{\Omega} %\varOmega
\fi


\def\PERCTAG{\MAbort{PERCTAG ist zur internen verwendung in mconvert.pl reserviert, dieses Makro darf sonst nicht benutzt werden.}}

% Im Gegensatz zu einfachen html-Umgebungen werden MDirectHTML-Umgebungen von mconvert.pl am ganzen ttm-Prozess vorbeigeschleust und aus dem PDF komplett ausgeschnitten
\ifttm%
\newenvironment{MDirectHTML}{\begin{html}}{\end{html}}%
\else%
\newenvironment{MDirectHTML}{\begin{html}}{\end{html}}%
\fi

% Im Gegensatz zu einfachen Mathe-Umgebungen werden MDirectMath-Umgebungen von mconvert.pl am ganzen ttm-Prozess vorbeigeschleust, ueber MathJax realisiert, und im PDF als $$ ... $$ gesetzt
\ifttm%
\newenvironment{MDirectMath}{\begin{html}}{\end{html}}%
\else%
\newenvironment{MDirectMath}{\begin{equation*}}{\end{equation*}}% Vorsicht, auch \[ und \] werden in amsmath durch equation* redefiniert
\fi

% ---------------------------------- Location Management ---------------------------------------------

% #1 = buttonname (muss in files/images liegen und Format 48x48 haben), #2 = Vollstaendiger Einrichtungsname, #3 = Kuerzel der Einrichtung,  #4 = Name der include-texdatei
\ifttm
\newcommand{\MLocationSite}[3]{\special{html:<!-- mlocation;;}#1\special{html:;;}#2\special{html:;;}#3\special{html:;; //-->}}
\else
\newcommand{\MLocationSite}[3]{\relax}
\fi

% ---------------------------------- Copyright Management --------------------------------------------

\newcommand{\MCCLicense}{%
{\color{green}\textbf{CC BY-SA 3.0}}
}

\newcommand{\MCopyrightLabel}[1]{ (\MSRef{L_COPYRIGHTCOLLECTION}{Lizenz})\MLabel{#1}}
\newcommand{\MSilentCopyrightLabel}[1]{\MLabel{#1}} % label is added to tex code but not visible

% Copyrightliste wird als tex-Datei im preprocessing erzeugt und unter converter/tex/copyrightcollection.tex abgelegt
% Der input-Befehl funktioniert nur, wenn die aufrufende tex-Datei auf der obersten Ebene liegt (d.h. selbst kein input/include ist, insbesondere keine Moduldatei)
\newcommand{\MCopyrightCollection}{\input{copyrightcollection.tex}}

% MCopyrightNotice fuegt eine Copyrightnotiz ein, der parser ersetzt diese durch CopyrightNoticePOST im preparsing, diese Definition wird nur fuer reine pdflatex-Uebersetzungen gebraucht
% Parameter: #1: Kurze Lizenzbeschreibung (typischerweise \MCCLicense)
%            #2: Link zum Original (http://...) oder NONE falls das Bild selbst ein Original ist, oder TIKZ falls das Bild aus einer tikz-Umgebung stammt
%            #3: Link zum Autor (http://...) oder MINT falls Original im MINT-Kolleg erstellt oder NONE falls Autor unbekannt
%            #4: Bemerkung (z.B. dass Datei mit Maple exportiert wurde)
%            #5: Labelstring fuer existierendes Label auf das copyrighted Objekt, mit MCopyrightLabel erzeugt
%            Keines der Felder darf leer sein!
\newcommand{\MCopyrightNotice}[5]{\MCopyrightNoticePOST{#1}{#2}{#3}{#4}{#5}}

\ifttm%
\newcommand{\MCopyrightNoticePOST}[5]{\relax}%
\else%
\newcommand{\MCopyrightNoticePOST}[5]{\relax}%
\fi%

% ---------------------------------- Meldungen fuer den Benutzer des Konverters ----------------------
\MPragma{mintmodversion;P0.1.0}
\MPragma{usercomment;This is file mintmod.tex version P0.1.0}


% ----------------------------------- Spezialelemente fuer Konfigurationsseite, werden nicht von mintscripts.js verwaltet --

% #1 = DOM-id der Box
\ifttm\newcommand{\MConfigbox}[1]{\special{html:<input cfieldtype="2" type="checkbox" name="Name_}#1\special{html:" id="}#1\special{html:" onchange="confHandlerChange('}#1\special{html:');"/>}}\fi % darf im PDF nicht aufgerufen werden!

\MPragma{MathSkip}
\MSetSubject{\MINTMathematics}

\Mtikzexternalize

\begin{document}

\MSection{System of Linear Equations}
\MLabel{VBKM04}
\MSetSectionID{lgs}

\begin{MSectionStart}
\MDeclareSiteUXID{VBKM04_START}


This module deals with solution methods for systems of linear equations and consists of:

\begin{itemize}
\item{Section~\MNRef{M04_LGS}: \MSRef{M04_LGS}{What are Systems of Linear Equations?},}
\item{Section~\MNRef{M04_2_Unbekannte}: \MSRef{M04_2_Unbekannte}{Systems of Linear Equations in two Variables}}
\item{Section~\MNRef{M04_3_Unbekannte}: \MSRef{M04_3_Unbekannte}{Systems of Linear Equations in three Variables}}
\item{Section~\MNRef{M04_freier_Parameter}: \MSRef{M04_freier_Parameter}{More General Systems}}
\item{Section~\MNRef{M04_Ausgangstest}: \MSRef{M04_Ausgangstest}{Final Test}}
\end{itemize}

\MModstartBox
\end{MSectionStart}


\MSubsection{What are Systems of Linear Equations?}
\MLabel{M04_LGS}

\begin{MIntro}
\MDeclareSiteUXID{VBKM04_LGSIntro}


A problem in which several variables occur at the same time!? And on top of that, 
a whole slew of equations is involved!? Problems of this kind do not only occur in science and 
engineering but also in other academic fields and in every day life! And they all have to be solved!

But calm down now: it won't get difficult! However, it is true that in very different fields  
you will often encounter situations and problems which can be mathematically modelled by
several equations in several variables. Here, we consider a simple first example.

\begin{MExample}
\MLabel{M04_einfuehrendes_Bsp}
A young group of stuntmen want to pimp up their breakneck cycling stunt by purchasing new rims which create 
garish light effects for their unicycles and bicycles. For a total of $10$ unicycles and bicycles
$13$~rims are required. How many unicycles and bicycles do the group have? 

The first step is to translate the information given in the description of the problem into mathematical equations. Let $x$ denote the desired number of unicycles and $y$ denote the number of 
bicycles. Then the first information given in the problem reads
\begin{eqnarray*}
\mbox{equation}\MBlank (1) & : & x + y = 10
\end{eqnarray*}
since the group have $10$ cycles in total. Moreover, a unicycle has one rim and a bicycle has 
two rims. Since $13$ rims are to be purchased in total, it is also known that
\begin{eqnarray*}
\mbox{equation}\MBlank (2) & : & x + 2 y = 13.
\end{eqnarray*}
Thus, from the problem description two equations arise relating the two variables 
$x$ (number of unicycles) and $y$ (number of bicycles).
\end{MExample}

Of course, sooner or later you want to know how many unicycles and bicycles the group 
of stuntmen really has. In the given example you can guess the values of $x$ and $y$ by
a little trial and error. Here we are actually interested in the \textbf{methods}
for solving problems like the example above \textbf{systematically}.
\end{MIntro}

\begin{MContent}
\MDeclareSiteUXID{VBKM04_LGSContent}

Before we can really start, let us clarify the terminology. 

\begin{MInfo}
Several equations relating a specific number of variables \textbf{at the same time} 
form a so called \MEntry{system of equations}{system of equations}. If the variables 
in every equation occur only linearly, i.e.\ at most to the power of $1$, and
are only multiplied by (constant) numbers, the system is called
a \MEntry{system of linear equations}{system of equations (linear)}, or 
\MEntry{LS}{LS} (linear system).
\end{MInfo}

The two equations in the first example~\MRef{M04_einfuehrendes_Bsp} form a system of linear
equations in the variables $x$ and $y$. In contrast, the three equations 
$$x + y + z = 3,\;\mbox{}\;x + y - z = 1,\;\mbox{and}\; x \cdot y + z = 2$$
do form a system of equations in the variables $x$, $y$, and $z$, but the 
system is \textbf{not} linear since in the third equation the term $x \cdot y$ occurs, which 
is \textbf{bilinear} in the variables $x$ and $y$ and hence violates the condition
of \textbf{linearity}.

By the way, in a system of equations the number of equations need not be equal to the 
number of variables; we will return to this later on.

\begin{MInfo}
If the number of equations in a system equals the number of variables, 
the system of equations is called \MEntry{quadratic}{system of equations (quadratic)}.
\end{MInfo}
\begin{MExercise}
Which of the following systems is a system of linear equations?

\begin{MQuestionGroup}
\begin{tabular}[t]{ll}
\MLCheckbox{1}{M04C10} & $x + y - 3 z = 0$, $2 x - 3 = y$, and $\MZahl{1}{5} x - z = 22 + y$, \\
\MLCheckbox{0}{M04C11} & $\sin(x) + \cos(y) = 1$ and $x - y = 0$, \\
\MLCheckbox{0}{M04C12} & $2 z - 3 y + 4 x = 5$ and $z + y - x^2 = 25$.
\end{tabular}
\end{MQuestionGroup}
\MGroupButton{Eingabe kontrollieren}
\end{MExercise}
Systems of linear equations are distinguished from general systems of equations
by their relative simplicity. Nevertheless, they play an important role in as 
diverse fields as medical science, e.g.\ in association with computed tomography, engineering, e.g. when it is described how sound propagates in complex 
designed spaces, or physics, e.g. concerning the question of which wave 
lengths excited atoms can emit. It is, without doubt, worth dealing with
systems of linear equations intensely.

For systems of equations generally, the question focuses on which 
values the variables have to take such that all equations of the system are
simultaneously satisfied. Such a set of values for the variables is called
a \MEntry{solution of a system of equations}{solution of a system of equations}.

Before we solve systems of equations a detail should be noted: depending on
the problem, it may not be useful to accept all variable values. In the first
example~\MRef{M04_einfuehrendes_Bsp} the variables $x$ and $y$ are the numbers of 
unicycles and bicycles the group of stuntmen owns. Such numbers can only be non-negative
integers, i.e. elements of $\N_0$. Hence, in this case the number range for the 
solutions has to be restricted to $\N_0$ in advance (namely, for both $x$ and $y$).

\begin{MInfo}
The possible number range for the solutions of a system of equations is called 
\MEntry{basic set}{basic set}. The \MEntry{domain}{domain} is the subset of the basic set
for which all the terms of the equations of the system are \textbf{defined}. For systems
of linear equations basic set and domain coincide. Finally, the 
\MEntry{solution set}{solution set} is the subset of the domain which merges 
the \textbf{solutions} of the system. The solution set is denoted by $\ML$.

\end{MInfo}
If the basic set is not explicitly specified -- and cannot be derived from the problem 
description -- we will assume implicitly that the basic set is $\R$, the number range 
of the real numbers.

\end{MContent}



\MSubsection{LS in two Variables}
\MLabel{M04_2_Unbekannte}

\begin{MIntro}
\MDeclareSiteUXID{VBKM04_ZweiUnbekannte}
At first we will restrict ourselves to systems of linear equations in \textbf{two}
variables. 

\begin{MInfo}
\MLabel{M04_2x2_system}
Generally, a system of linear equations (LS) consisting of two equations 
in the variables $x$ and $y$ has the following structure:
\begin{eqnarray*}
	a_{11} \cdot x + a_{12} \cdot y &= b_{1} \MDFPSpace, \\ 
	a_{21} \cdot x + a_{22} \cdot y &= b_2 \MDFPeriod
\end{eqnarray*}
Here, $a_{11}, a_{12}, a_{21}$, and $a_{22}$ are the so called coefficients of the system
of linear equations, which are, as $b_1$ and $b_2$ on the right-hand sides of the 
equations, often real numbers given by the problem description.
If the right-hand sides $b_1$ and $b_2$ both are equal to $0$ ($b_1 = 0 = b_2$), 
the system of linear equations is called \textbf{homogeneous}, and \textbf{inhomogeneous}
otherwise.


\end{MInfo}

Because of their linearity, each of the two equations of the system in info box \MRef{M04_2x2_system}
can be interpreted as the equation of a line in the $x$-$y$-plane. If, for example, the first equation
is solved for $y$,
$$y = - \Mdfrac{a_{11}}{a_{12}} x + \Mdfrac{b_1}{a_{12}} \MDFPSpace ,$$
one immediately sees from this explicate form that it describes a line with the slope 
$m = - a_{11}/a_{12}$ and the $y$-intercept $y_0=b_1/a_{12}$.

Note that solving the equation for $y$ is only allowed if $a_{12} \neq 0$. For $a_{12} = 0$,
the first equation reads $a_{11} \cdot x = b_1$; for $a_{11} \neq 0$ this is equivalent 
to $x = (b_1/a_{11})$, i.e.\ $x$ is a constant. This equation also describes a line, namely 
a line parallel to the $y$-axis with $y$-intercept $(b_1/a_{11})$.

And what about the case in which both $a_{12} = 0$ and $a_{11} = 0$? Then, we also have $b_1 = 0$, since 
otherwise the first equation would result in a contradiction. But for $a_{11} = a_{12} = b_1 = 0$,
the first equation is (for all values of $x$ and $y$) always identically satisfied ($0 = 0$),
and hence useless.

The case of the second equation in info box \MRef{M04_2x2_system} is similar:
$$y = - \Mdfrac{a_{21}}{a_{22}} x + \Mdfrac{b_2}{a_{22}} \MDFPeriod$$
Altogether, one obtains two lines representing the two linear equations. The question
for the solvability and for the solution of the system of linear equations, namely 
\textbf{the question for the simultaneous validity of the two equations} reads as \textbf{the question for
the existence and the position of the intersection point of the two lines}. 
To this, let us investigate a specific example.

\begin{MExample}
The system of linear equations in the first example~\MRef{M04_einfuehrendes_Bsp} reads:
$$\left. \begin{array}{rcl} x + y & = & 10 \\[.5ex] x + 2 y & = & 13 \end{array} \right\} \Leftrightarrow
\left\{ \begin{array}{rcl} y & = & - x + 10 \\[.5ex] y & = & - \Mtfrac12 x + \Mtfrac{13}{2} \end{array} \right. \MDFPeriod$$
(Here, the general coefficients and right-hand sides of the system \MRef{M04_2x2_system}
have the specific values $a_{11} = 1, a_{12} = 1, a_{21} = 1, a_{22} = 2, b_1 = 10$, and $b_2 = 13$.)

The equations describe two lines with the slopes $m_{1} = - 1$ and $m_{2} = - \Mtfrac12$, respectively,
and the $y$-intercepts $y_{0,1} = 10$ and $y_{0,2} = \Mtfrac{13}{2}$, respectively.

\begin{center}
\MTikzAuto{%
\begin{tikzpicture}[x=0.4cm, y=0.5cm] 
%Koordinatensystem
\node (xMAX) at (16.5,0){};
\node (yMAX) at (0,11){};
\draw[->,color=black] (-6.2,0) -- (xMAX);
\foreach \x in {-6, -4, -2, 2, 4, 6, 8, 10, 12, 14, 16}
\draw[shift={(\x,0)},color=black] (0pt,2pt) -- (0pt,-2pt) node[below] {\footnotesize $\x$};
\draw[->,color=black] (0,-1.0) -- (yMAX);
\foreach \y in {2,4,6,8,10}
\draw[shift={(0,\y)},color=black] (2pt,0pt) -- (-2pt,0pt) node[left] {\footnotesize $\y$};
\draw[color=black] (0pt,-8.5pt) node[right] {\footnotesize $0$};
\draw[color=black] (-2.0pt,7pt) node[left] {\footnotesize $0$};
%Achsenbeschriftung
\draw (xMAX) node[anchor=south] {$x$};
\draw (yMAX) node[anchor=west] {$y$};
%Beschriftung
\clip(-6,-0.5) rectangle (16.5,10.5);
\draw[color=red, thick] (15,-1.0) -- (-6,9.5);
\draw[color=blue, thick] (11,-1.0) -- (-0.5,10.5);
\draw[color=black, thick] (7,3) circle (1.5pt);
\draw[color=red] (2.35,4.7) node[anchor=east] {$y=-x/2+13/2$};
\draw[color=blue] (2.7,8) node[anchor=west] {$y=-x+10$};
\draw[color=black] (7,3) node[anchor=south west] {$\MPointTwo{7}{3}$};
\end{tikzpicture}
}%
%%\MUGraphicsSolo{eindeutige_lsg.png}{scale=1}{width:450px}
\end{center}
The figure shows that the two lines do indeed intersect, and one reads off the coordinates 
of the intersection point as $\MPointTwo{x = 7}{y = 3}$. Accordingly, the system of 
linear equations considered here has a unique solution. The solution set consists 
of exactly one pair of numbers, namely ${\ML} = \{ \MPointTwo{x = 7}{y = 3} \}$.
\end{MExample}

This intuitive approach is very well suited to discussing all cases which can generally occur: two
lines in the $x$-$y$-plane can intersect each other -- and then the intersection point is 
necessarily \textbf{unique} --, or the two lines are parallel and thus do not have any
intersection point, or the two lines coincide -- so to speak -- they intersect in 
an infinite number of points. There are no other cases.

Accordingly, the corresponding system of linear equations has one of the following solution sets: 

\begin{MInfo}
An \MEntry{inhomogeneous system of linear equations}{system of linear equations (inhomogeneous)} 
has either one unique solution, no solution, or an infinite number of solutions.

An \MEntry{homogeneous system of linear equations}{system of linear equations (homogeneous)}  
\textbf{has always at last one solution}, namely the so called 
\MEntry{trivial solution}{trivial solution (LS)} $x = 0$, $y = 0$.
Moreover, such a homogeneous system \textbf{can} also have \textbf{an infinite number of solutions}.
\end{MInfo}

This will be illustrated by two further examples starting directly with the 
systems of linear equations:
\begin{MExample}
\MLabel{M04_2x2_nicht_eindeutige_lsg}
In both cases the \textbf{basic set} is the set of the real numbers $\R$.
\begin{center}
\begin{tabular}{l|l}
\begin{minipage}{7.5cm}
$$\left. \begin{array}{rcl} x + y & = & 2 \\[.5ex] 2 x + 2 y & = & 1 \end{array} \right\} \Leftrightarrow
\left\{ \begin{array}{rcl} y & = & - x + 2 \\[.5ex] y & = & - x + \Mtfrac12 \end{array} \right. \MDFPeriod $$
\end{minipage} &
\begin{minipage}{7.5cm}
$$\left. \begin{array}{rcl} x + y & = & 2 \\[.5ex] 2 x + 2 y & = & 4 \end{array} \right\} \Leftrightarrow
\left\{ \begin{array}{rcl} y & = & - x + 2 \\[.5ex] y & = & - x + 2 \end{array} \right. \MDFPeriod $$
\end{minipage} \\[1cm]
\MTikzAuto{%
\begin{tikzpicture}[x=1.0cm, y=1.4cm] 
%Koordinatensystem
\node (xMAX) at (3.5,0){};
\node (yMAX) at (0,2.5){};
\draw[->,color=black] (-2.5,0) -- (xMAX);
\foreach \x in {-2, -1, 1, 2, 3}
\draw[shift={(\x,0)},color=black] (0pt,2pt) -- (0pt,-2pt) node[below] {\footnotesize $\x$};
\draw[->,color=black] (0,-0.5) -- (yMAX);
\foreach \y in {1,2}
\draw[shift={(0,\y)},color=black] (2pt,0pt) -- (-2pt,0pt) node[left] {\footnotesize $\y$};
\draw[color=black] (0pt,-8.5pt) node[right] {\footnotesize $0$};
\draw[color=black] (-2.0pt,7pt) node[left] {\footnotesize $0$};
%Achsenbeschriftung
\draw (xMAX) node[anchor=south] {$x$};
\draw (yMAX) node[anchor=west] {$y$};
%Beschriftung und Graphen
\clip(-2.8,-0.6) rectangle (3.6,2.7);
\draw[color=blue, thick] (1,-0.5) -- (-2,2.5);
\draw[color=blue, thick] (2.5,-0.5) -- (-0.5,2.5);
\draw[color=black] (-0.2,0.5) node[anchor=east] {$y=-x+1/2$};
\draw[color=black] (0.7,1.5) node[anchor=west] {$y=-x+2$};
%%\draw[color=black] (7,3) node[anchor=south west] {$\MPointTwo{7}{3}$};
\end{tikzpicture}
}
%%\MUGraphicsSolo{keine_lsg.png}{scale=1}{width:350px}
&
\MTikzAuto{%
\begin{tikzpicture}[x=1.0cm, y=1.4cm] 
%Koordinatensystem
\node (xMAX) at (4,0){};
\node (yMAX) at (0,2.5){};
\draw[->,color=black] (-1.5,0) -- (xMAX);
\foreach \x in {-1, 1, 2, 3}
\draw[shift={(\x,0)},color=black] (0pt,2pt) -- (0pt,-2pt) node[below] {\footnotesize $\x$};
\draw[->,color=black] (0,-0.5) -- (yMAX);
\foreach \y in {1,2}
\draw[shift={(0,\y)},color=black] (2pt,0pt) -- (-2pt,0pt) node[left] {\footnotesize $\y$};
\draw[color=black] (0pt,-8.5pt) node[right] {\footnotesize $0$};
\draw[color=black] (-2.0pt,7pt) node[left] {\footnotesize $0$};
%Achsenbeschriftung
\draw (xMAX) node[anchor=south] {$x$};
\draw (yMAX) node[anchor=west] {$y$};
%Beschriftung und Graphen
\clip(-1.6,-0.6) rectangle (4,2.7);
\draw[color=red, thick] (2.5,-0.52) -- (-0.5,2.48);
\draw[color=blue, thick] (2.5,-0.48) -- (-0.5,2.52);
\draw[color=red] (1.35,0.45) node[anchor=east] {$2x+2y=4$};
\draw[color=blue] (0.7,1.5) node[anchor=west] {$x+y=2$};
%%\draw[color=black] (7,3) node[anchor=south west] {$\MPointTwo{7}{3}$};
\end{tikzpicture}
}
%%\MUGraphicsSolo{unendl_lsg.png}{scale=1}{width:350px}
\\[.5cm]
\begin{minipage}[t]{7.5cm}
The two lines have the same slope $m = - 1$ but the $y$-intercepts differ ($y_0 = 2$
or $y_0 = 1/2$, respectively). The two lines are parallel. Thus the system of linear equations 
has \textbf{no} solution: 
$${\ML} = \MEmptyset\MDFPeriod $$
\end{minipage} &
\begin{minipage}[t]{7.5cm}
The two lines have both the same slope $m = - 1$ and the same
$y$-intercept $y_0 = 2$. The two lines coincide. The system of linear equations 
has an infinite number of solutions which can be written, for example, as follows:
$${\ML} = \{ \MPointTwo{t}{ - t + 2} \MCondSetSep  t \in \R \}\MDFPeriod $$
\end{minipage}
\end{tabular}
\end{center}
\end{MExample}
For the example on the right, other parametrisations of the solution set 
are possible and allowed. It basically just depends on how to describe the points of 
the (congruent) lines appropriately. The above description of the solution set $\ML$
simply used the equation of the line itself and the control variable was denoted by 
$t$ instead of $x$.

And what about the above mentioned restrictions due to the basic set? 
Let us look at the following example.

\begin{MExample}
At a local fair, an exceptionally clever stallholder promises almost dreamlike prizes for an absurdly low initial wager if one, yes, only one of the passers-by is able to unravel the 
following little mystery:

\textit{I rolled a 
die twice. If I subtract twice the first number I rolled from six times
the second number, I get the number $3$. If I add $6$ to four times the first number, I get the twelve times the second number. Which two numbers did I roll?}

Let the first number he rolled be denoted by $x$ and the second by $y$. 
Then the statements of the stallholder can be translated very quickly into a set of equations:
$$\left. \begin{array}{rcl} 6 y - 2 x & = & 3 \\[.5ex] 4 x + 6 & = & 12 y \end{array} \right\} \Leftrightarrow
\left\{ \begin{array}{rcl} y & = & \Mtfrac13 x + \Mtfrac12 \\[.5ex]
y & = & \Mtfrac13 x + \Mtfrac12 \end{array} \right. \MDFPeriod $$
One realises that the corresponding system of linear equations -- interpreted geometrically -- 
results in two congruent lines. At first glance, it thus seems to have an infinite number
of solutions.

But now the basic set has to be taken into account: since both $x$ and $y$ represent numbers on a die,
the two variables can only take values from the set 
$\{ 1\MElSetSep 2\MElSetSep 3\MElSetSep 4\MElSetSep 5\MElSetSep 6 \}$. If one looks at 
the line $y = \Mtfrac13 x + \Mtfrac12$ in the $x$-$y$-plane, one sees that no possible pair
of numbers thrown on a die fall onto this line. Hence, the solution set in this case is indeed the 
empty set ${\ML} = \MEmptyset$.
\begin{center}
\MTikzAuto{%
\begin{tikzpicture}[x=1.0cm, y=1.0cm] 
%Koordinatensystem
\node (xMAX) at (10.5,0){};
\node (yMAX) at (0,6.5){};
\draw[->,color=black] (-1.5,0) -- (xMAX);
\foreach \x in {-1, 0, 1, 2, 3, 4, 5, 6, 7, 8, 9, 10}
\draw[shift={(\x,0)},color=black] (0pt,2pt) -- (0pt,-2pt) node[below right] {\scriptsize $\x$};
\draw[->,color=black] (0,-0.5) -- (yMAX);
\foreach \y in {0, 1, 2, 3, 4, 5, 6}
\draw[shift={(0,\y)},color=black] (2pt,0pt) -- (-2pt,0pt) node[above left] {\scriptsize $\y$};
%%\draw[color=black] (0pt,-8.5pt) node[right] {\footnotesize $0$};
%%\draw[color=black] (-2.0pt,7pt) node[left] {\footnotesize $0$};
%Achsenbeschriftung
\draw (xMAX) node[anchor=south] {$x$};
\draw (yMAX) node[anchor=west] {$y$};
%Beschriftung und Graphen
\clip(-1.4,-0.4) rectangle (10.5,6.5);
\draw[help lines, gray, dashed] (-2,-1) grid (11,7); % was dotted
\fill[red!50!white, opacity=0.50] (1,1) rectangle (6,6);
\draw[color=brown] (1,1) rectangle (6,6);
\foreach \i in {1, 2, 3, 4, 5, 6}
  \foreach \j in {1, 2, 3, 4, 5, 6}
  {
    \fill[color=blue] (\i,\j) circle (2.0pt);
    \draw[color=black] (\i,\j) circle (2.0pt);
  }
\draw[color=black, thick] (-1.5,0.0) -- (10.5,4.0);
\draw[color=black] (6.3,2.35) node[anchor=west] {\Large $y=\frac{1}{3}x+\frac{1}{2}$};
%%\draw[color=black] (7,3) node[anchor=south west] {$\MPointTwo{7}{3}$};
\end{tikzpicture}
}%
%%\MUGraphicsSolo{betrueger.png}{scale=1}{width:450px}
\end{center}
\end{MExample}
\end{MIntro}

\begin{MXContent}{Substitution Method and Comparison Method}{Substitution and Comparison Method}{STD}
\MDeclareSiteUXID{VBKM04_Einsetzmethode}
\MLabel{M04_einsetz_gleichsetz}
Until now we studied the \MEntry{solvability}{solvability} and the \textbf{graphical solution}
of systems of linear equations of the form \MRef{M04_2x2_system}.
Next, we need to study these systems algebraically.
To this end, we investigate a further example.

\begin{MExample}
For the renovation of their house, the Müller family had to take out two mortgages with a total amount of
 $50{,}000$ Euro. The interest they have to pay annually is in total  $3{,}700$ Euro. The interest rate for 
one mortgage agreement is $5\%$ annually, and  $8\%$ annually for the other. What are the
amounts of the individual mortgages?

Let the amounts of the individual mortgages be denoted by $x$ and $y$. As we know from the 
problem description, the sum of the two amounts is $50{,}000$ Euro. Hence, the first 
equation reads:
\begin{eqnarray*}
\mbox{equation}\MBlank (1) & \colon x + y = 50{,}000 \; \; \; \mbox{(Euro )}\MDFPeriod 
\end{eqnarray*}

The interest burden from the mortgage agreement at a rate of $5\%$ is $\MZahl{0}{05} \cdot x$, 
and the interest burden from the other one at a rate of $8\%$ is $\MZahl{0}{08} \cdot y$. From
the problem description we know that the two values add up to $3{,}700$. This results in
a second equation:
\begin{eqnarray*}
\mbox{equation}\MBlank (2) & \colon \MZahl{0}{05} x + \MZahl{0}{08} y = 3{,}700 \; \; \; \mbox{(Euro )}\MDFPeriod 
\end{eqnarray*}
Again, one ends up with a system of linear equations as in example~\MRef{M04_2x2_system}.

To solve the system algebraically, the first equation is solved for $y$. This results in an  
equation~$(1')$, which is equivalent to equation $(1)$:
\begin{eqnarray*}
\mbox{equation}\MBlank (1') & \colon y = 50{,}000 - x\MDFPeriod 
\end{eqnarray*}
This equation for $y$ can be substituted into equation $(2)$. In the resulting 
equation only the variable  
$x$ occurs, so it can be solved for $x$:
\begin{eqnarray*}
& & \MZahl{0}{05} x + \MZahl{0}{08} (50{,}000 - x) = 3{,}700 \\
\Leftrightarrow & & \MZahl{0}{05} x + 4{,}000 - \MZahl{0}{08} x = 3{,}700 \\
\Leftrightarrow & & \MZahl{0}{03} x = 300 \\
\Leftrightarrow & & x = 10{,}000\MDFPeriod 
\end{eqnarray*}
Substituting the solution for $x$ into equation~$(1')$ results in
\begin{eqnarray*}
& & y = 50{,}000 - 10{,}000 \\
\Leftrightarrow & & y = 40{,}000\MDFPeriod 
\end{eqnarray*}
Hence, the mortgage amounts are $10{,}000$ Euro (mortgage with interest of $5\%$ annually) and
$40{,}000$ Euro (mortgage with interest of $8\%$ annually).
\end{MExample}
The previous example illustrates the characteristics of the so called 
\textbf{substitution method}:
\begin{MInfo}
\MLabel{M04_einsetzmethode}
In the \MEntry{substitution method}{substitution method}, as a first step 
one of the two linear equations is solved for one of the variables -- or a multiple 
of one of the variables. As a second step the solution is 
\textbf{substituted} into the other linear equation. Only three cases can occur:
\begin{itemize}
\item[(i)]{In the resulting equation (after collecting like terms) the other variable
still occurs. Solving the resulting equation for this other variable results in the first
part of the solution. The second part is obtained, for example, by substituting the solution
of the first part into the equation from the first step. The solution is unique. (If this
solution does not belong to the basic set, it has to be excluded.)}
\item[(ii)]{In the resulting equation (after collecting like terms) the other variable
does not occur any more and the equation is a contraction. Then the system of linear equations
has no solution.}
\item[(iii)]{In the resulting equation (after collecting like terms) the other variable
does not occur any more and the equation is always true. Then the system of linear 
equations has an infinite number of solutions (unless the definition of the basic set 
results in some restrictions).}
\end{itemize}
\end{MInfo}
This approach involves certain details. It is not defined which of the linear equations 
is to be solved for which of the variables -- or multiples of the variable. As long as 
the applied transformations are equivalent any of the possible ways will result in the same 
solution. Preferring a specific way is partly a matter of taste and partly a 
matter of skills: a clever choice can simplify some intermediate calculations.

Concerning the \textbf{substitution method}, cases (ii) and (iii) mentioned above 
shall be illustrated again by means of the systems of linear equations 
in example~\MRef{M04_2x2_nicht_eindeutige_lsg}:

\begin{MExample}
For both systems of linear equations the basic set is $\R$.

\begin{center}
\begin{tabular}{c|c}
\begin{minipage}[t]{7.5cm}
$$\begin{array}{rcrcl} \mbox{equation}\MBlank (1): & & x + y & = & 2 \\[.5ex]
\mbox{equation}\MBlank (2): & & 2 x + 2 y & = & 1 \end{array}\MDFPeriod $$
Solving equation $(1)$ for $x$ results in $x = 2 - y$.
Substituting this equation into equation $(2)$ results in:
\begin{eqnarray*}
& & 2 (2 - y) + 2y = 1 \\
& \Leftrightarrow & 4 - 2y + 2y = 1 \\
& \Leftrightarrow & 4 = 1\MDFPeriod 
\end{eqnarray*}
This is a contraction. The LS has \textbf{no} solution.
\end{minipage} &
\begin{minipage}[t]{7.5cm}
$$\begin{array}{rcrcl} \mbox{equation}\MBlank (1): & & x + y & = & 2 \\[.5ex]
\mbox{equation}\MBlank (2): & & 2 x + 2 y & = & 4 \end{array}\MDFPeriod $$
Solving equation $(1)$ for $y$ results in $y = 2 - x$.
Substituting this equation into equation $(2)$ results in:
\begin{eqnarray*}
& & 2x + 2 (2 - x) = 4 \\
& \Leftrightarrow & 2x + 4 - 2x = 4 \\
& \Leftrightarrow & 4 = 4\MDFPeriod 
\end{eqnarray*}
This is always true. The LS has \textbf{an infinite number} of solutions.
\end{minipage}
\end{tabular}
\end{center}
\end{MExample}
The substitution method is not the only approach for solving systems of linear 
equations. In the following section another method is discussed, which is closely related 
to the graphical solution of a LS.

\begin{MInfo}
\MLabel{M04_gleichsetzmethode}
In the  \MEntry{comparison method}{comparison method}, as a first step \textbf{both} 
linear equations are solved for one of the variables -- or a multiple of one of 
the variables. As a second step the two resulting equations will be \textbf{equated}. 
Then the three cases discussed for the substitution method can occur. 
\end{MInfo}
This approach involves certain details as well. For example, it is not
defined for which variable the linear equations are to be solved.

For illustration, the first example is solved again, this time by means of
the comparison method:

\begin{MExample}
The system of linear equations in the first example reads: 

\begin{eqnarray*}
x + y & =  10 \MDFPSpace, \\ x + 2 y & =  13\MDFPeriod 
\end{eqnarray*}
Both equations are solved for $x$
\begin{eqnarray*}
x & =  10 - y\MDFPSpace,  \\ x & =  13 - 2y \MDFPSpace,
\end{eqnarray*}
and the right-hand-sides of the two equations are equated
$$10 - y = 13 - 2y \MDFPSpace$$
which results in $y = 3$. The solution for $y$ can be substituted into one of
the equations solved for $x$ which results in $x = 7$.

\end{MExample}
\begin{MExercise}
Find the solution set of the following system of linear equations
\begin{eqnarray*}
7 x + 2 y &= & 14 \MDFPSpace, \\ 3 x - 5 y & = & 6
\end{eqnarray*}
using the comparison method.
\ \\
\begin{MHint}{Solution}
For example, both equations are solved for $x$: For this, the first equation is multiplied
by $(1/7)$ and solved for $x$:
$$x = \Mtfrac{14}{7} - \Mtfrac27 y \Leftrightarrow x = 2 - \Mtfrac27 y \; : \MBlank\mbox{equation}\MBlank (1') \MDFPSpace .$$
In contrast, the second equation is multiplied by $\Mtfrac13$ before solving it
for $x$:
$$x = \Mtfrac63 + \Mtfrac53 y \Leftrightarrow x = 2 + \Mtfrac53 y \; : \MBlank\mbox{equation}\MBlank (2') \MDFPeriod $$
Equating the two right-hand sides of equation~$(1')$ and $(2')$ results in
$$2 - \Mtfrac27 y = 2 + \Mtfrac53 y \Leftrightarrow 0 = (\Mtfrac27 + \Mtfrac53) y \Leftrightarrow y = 0 \MDFPeriod $$
With this solution for $y$, for example, the first equation results in
$$7 x + 2 \cdot 0 = 14 \Leftrightarrow 7 x = 14 \Leftrightarrow x = 2 \MDFPeriod $$
Hence, the given system of linear equations is solvable uniquely and the 
solution set is  ${\ML} = \{ \MPointTwo{x = 2}{y = 0} \}$.

Alternatively, before equating them, the two equations could have been solved for $y$ 
(or a multiple of $x$ or a multiple of $y$). The solution is always the same.
\end{MHint}
\end{MExercise}
\end{MXContent}

\begin{MXContent}{Addition Method}{Addition Method}{STD}
\MLabel{M04_addition}
\MDeclareSiteUXID{VBKM04_Additionsmethode}
We will now discuss another, third method for solving systems of linear equations 
algebraically. But this method will develop its full potential only for larger systems, i.e.\
systems of many equations in many variables since it can be systematised very well.
Here, we will discuss the general approach. First, let us see an example.

\begin{MExample}
Find the solution of the system of linear equations
$$\begin{array}{rcrcl} \mbox{equation}\MBlank (1): & & 2 x + y & = & 9 \MDFPSpace, \\
\mbox{equation}\MBlank (2): & & 3 x - 11 y & = & 1 \MDFPSpace, \end{array}$$
where the basic set is the range of the real numbers $\R$.

This time, the approach is as follows: The first equation is multiplied by the factor $11$ and this 
results in the equation (1') that is equivalent to equation~$(1)$: 
$$\begin{array}{crclcl} & (2 x + y) \cdot 11 & = & 9 \cdot 11 & & \\
\Leftrightarrow & 22 x + 11 y & = & 99 & & : \MBlank\mbox{equation}\MBlank (1') 
\MDFPeriod \end{array}$$
Subsequently, the new equation $(1')$ is added to equation $(2)$, i.e.\ the \textbf{sum}
of the left-hand sides of $(1')$ and $(2)$ is equated to the \textbf{sum} of
the right-hand sides of $(1')$ and $(2)$. In doing so, the variable $y$ is cancelled
out. This was the reason for selecting the factor $11$ in the previous step. 
$$3 x - 11 y + 22 x + 11 y = 1 + 99 \Leftrightarrow 25 x = 100 \Leftrightarrow x = 4 \MDFPeriod $$
The get the solution for $y$ the just obtained solution for $x$ can be substituted, e.g.\
into equation~$(1)$:
$$2 \cdot 4 + y = 9 \Leftrightarrow 8 + y = 9 \Leftrightarrow y = 1 \MDFPeriod$$
Thus, this system of linear equations has a unique solution ${\ML} =
\{ \MPointTwo{x = 4}{y = 1} \}$.
\end{MExample}
As for the other methods,the approach here is not uniquely defined: 
for example, equation~$(1)$ could have been multiplied by $3$ and equation~$(2)$ by $(- 2)$
$$\begin{array}{rclcrclcl} (2x + y) \cdot 3 & = & 9 \cdot 3 & \Leftrightarrow & 6 x + 3 y & = & 27
& & : \MBlank\mbox{equation}\MBlank ({1'}') \MDFPSpace, \\ 
(3 x - 11 y) \cdot (- 2) & = & 1 \cdot (- 2) & \Leftrightarrow & - 6 x + 22 y & = & - 2
& & : \MBlank\mbox{equation}\MBlank ({2'}') \MDFPeriod\end{array}$$
In the subsequent \textbf{addition} of equation~$({1'}')$ and equation~$({2'}')$ the variable $x$ could have been
eliminated:
$$6 x + 3 y - 6 x + 22 y = 27 - 2 \Leftrightarrow 25 y = 25 \Leftrightarrow y = 1 \MDFPeriod$$
To get the solution for $x$, the solution for $y$ then could have been substituted, e.g.\ 
into equation~$(2)$
$$3 x - 11 \cdot 1 = 1 \Leftrightarrow 3 x = 12 \Leftrightarrow x = 4 \MDFPeriod$$
\begin{MInfo}
In the addition method, one of the linear equations is transformed by multiplying it by 
an arbitrary factor such that in the subsequent \textbf{addition} of the other equation
(at least) one variable is eliminated. (Sometimes it is easier to multiply 
\textbf{both} equations by arbitrary factors before \textbf{adding} them.)
As for the substitution method in info box~\MRef{M04_einsetzmethode} 
(or the comparison method in info box~\MRef{M04_gleichsetzmethode}), three cases can occur, 
resulting in a solution set $\ML$ containing exactly one element, no element, or an infinite number of
elements.
\end{MInfo}
\end{MXContent}

\begin{MExercises}
\MDeclareSiteUXID{VBKM04_Additionsmethode_Exercises}
\begin{MExercise}
Solve the following systems of linear equations using the substitution method:
\begin{MExerciseItems}
\item{$3 x + y = 4$ and $- x + 2 y = 1$,}
\item{$- x + 4 y = 5$ and $2 x - 8 y = - 10$.}
\end{MExerciseItems}

\begin{MHint}{Solution}
\begin{MExerciseItems}
\item{For example, solving the first equation ($3 x + y = 4$) for $y$ results in
$y = 4 - 3 x$. Then this equation can be substituted into the second equation ($- x + 2 y = 1$):
$- x + 2 (4 - 3 x) = 1 \Leftrightarrow - x + 8 - 6 x = 1 \Leftrightarrow - 7 x = - 7
\Leftrightarrow x = 1$. With this value of $x$, for example, the second equation results in
$3 \cdot 1 + y = 4 \Leftrightarrow y = 1$. So, the solution set is 
${\ML} = \{ \MPointTwo{1}{1} \}$.

Of course, one could start differently: For example, one could solve the first equation for $x$
and substitute the solution for $x$ into the second equation, to get a solution for $y$; or one could 
start generally with the second equation and solve this in the first step for $x$ or for $y$. Hence,
there are some details in the approach.}
\item{Solving, for example, the first equation ($- x + 4 y = 5$) for $x$ results in $x = 4 y - 5$. 
Then, this equation can be substituted into the second equation ($2 x - 8 y = - 10$):
 $2 ( 4 y - 5) - 8 y = - 10 \Leftrightarrow 8 y - 10 - 8y = - 10 \Leftrightarrow 0 = 0$. 
So, this is not a new statement, in other words: the second equation does not contain new information. Thus, in
this case the solution set $\ML$ contains an infinite number of solution pairs $\MPointTwo{x}{y}$
which can be parametrised by a real number $t$. If one chooses, for example, $y = t$, the
solution set reads ${\ML} = \{ \MPointTwo{4 t - 5}{t}  \MCondSetSep  t \in \R \}$. This
solution set can be visualised as a line in two-dimensional space:
\begin{center}
\MTikzAuto{%
\begin{tikzpicture}[x=1.0cm, y=2.0cm] 
%Koordinatensystem
\node (xMAX) at (3.4,0){};
\node (yMAX) at (0,3.4){};
\draw[->,color=black] (-8.2,0) -- (xMAX);
\foreach \x in {-8, -7, -6, -5, -4, -3, -2, -1, 0, 1, 2, 3}
\draw[shift={(\x,0)},color=black] (0pt,2pt) -- (0pt,-2pt) node[below right] {\scriptsize $\x$};
\draw[->,color=black] (0,-0.5) -- (yMAX);
\foreach \y in {0, 0.5, 1, 1.5, 2, 2.5, 3}
\draw[shift={(0,\y)},color=black] (2pt,0pt) -- (-2pt,0pt) node[above left] {\scriptsize $\y$};
%Achsenbeschriftung
%%\draw (xMAX) node[anchor=south] {$x$};
%%\draw (yMAX) node[anchor=west] {$y$};
%Beschriftung und Graphen
\clip(-8.2,-0.6) rectangle (3.4,3.4);
\draw[help lines, gray, dashed, xstep=0.5, ystep=0.5] (-9,-1) grid (4,4);  % was dotted
%%\fill[red!50!white, opacity=0.50] (1,1) rectangle (6,6);
\draw[color=black, thick] (-9.0,-1.0) -- (7.0,3.0);
\draw[color=black] (0.1,3.0) node[anchor=south west] {\Large $y=t$};
\draw[color=black] (3.3,0.25) node[anchor=east] {\Large $x=4t-5$};
%%\draw[color=black] (7,3) node[anchor=south west] {$\MPointTwo{7}{3}$};
\end{tikzpicture}
}%
%%\MUGraphicsSolo{unendl_Lsgsm_Aufg_1b.png}{scale=1}{width:450px}
\end{center}
Accordingly, other parametrisations of the solution set are possible, for 
example, by choosing $x \in \R$ as free parameter and characterising the 
above line by its slope and its $y$-intercept, i.e.\ 
${\ML} = \{ \MPointTwo{x}{ \Mtfrac14 x + \Mtfrac54 } \MCondSetSep x \in \R \}$.}
\end{MExerciseItems}
\end{MHint}
\end{MExercise}

\begin{MExercise}
Solve the following systems of linear equations using the addition method:
\begin{MExerciseItems}
\item{$2 x + 4 y = 1$ and $x + 2 y = 3$,}
\item{$- 7 x + 11 y = 40$ and $2 x + 5 y = 13$.}
\end{MExerciseItems}

\begin{MHint}{Solution}
\begin{MExerciseItems}
\item{Multiplying the second equation ($x + 2 y = 3$) by $(- 2)$ results in equation $(2')$:
$- 2 x - 4 y = - 6$. Then the last equation is added to the first equation ($2 x + 4 y = 1$):
$2 x + 4 y - 2 x - 4 y = 1 - 6 \Leftrightarrow 0 = - 5$. This is a contradiction! Hence, the 
solution set for this system of equations is the empty set ${\ML} = \MEmptyset$.}
\item{Multiplying the first equation ($- 7 x + 11 y = 40$) by $2$ results in the 
equation $(1')$: $- 14 x + 22 y = 80$. Multiplying the second equation by $7$ results in
the equation $(2')$: $14 x + 35 y = 91$. Subsequently, adding equation $(1')$ and 
equation $(2')$ results in 
$- 14 x + 22 y + 14 x + 35 y = 80 + 91 \Leftrightarrow 57 y = 171 \Leftrightarrow y = 3$.
Substituting this value of $y$, for example, in the second equation results in 
$2 x + 5 \cdot 3 = 13 \Leftrightarrow
2 x = 13 - 15 \Leftrightarrow 2 x = - 2 \Leftrightarrow x = - 1$. Hence, the solution set $\ML$ 
reads ${\ML} = \{ \MPointTwo{- 1}{ 3} \}$.}
\end{MExerciseItems}
\end{MHint}
\end{MExercise}

\begin{MExercise}
Solve the following system of linear equations graphically: $2 x = 2$ and $x + 3 y = 4$.
\begin{MHint}{Solution}
The first equation ($2 x = 2$) is equivalent to $x = 1$: This equation describes a line 
parallel to the $y$-axis through the point $\MPointTwo{1}{0}$ on the $x$-axis. The
second equation ($x + 3 y = 4$) can be transformed into $y = - \Mtfrac{1}{3} x + \Mtfrac{4}{3}$. 
This equation also describes a line, this time its slope is $- \Mtfrac{1}{3}$ and its 
$y$-intercept is $\Mtfrac{4}{3}$. The graphs are as in the following figure.
\begin{center}
\MTikzAuto{%
\begin{tikzpicture}[x=1.8cm, y=1.8cm] 
%Koordinatensystem
\node (xMAX) at (2.8,0){};
\node (yMAX) at (0,2.8){};
\draw[->,color=black] (-1.0,0) -- (xMAX);
\foreach \x in {0, 1, 2}
\draw[shift={(\x,0)},color=black] (0pt,2pt) -- (0pt,-2pt) node[above right] {\scriptsize $\x$};
\draw[->,color=black] (0,-0.2) -- (yMAX);
\foreach \y in {1, 2}
\draw[shift={(0,\y)},color=black] (2pt,0pt) -- (-2pt,0pt) node[above right] {\scriptsize $\y$};
%Achsenbeschriftung
\draw (xMAX) node[anchor=south east] {$x$};
\draw (yMAX) node[anchor=north east] {$y$};
%Beschriftung und Graphen
\clip(-1.0,-0.2) rectangle (2.8,2.4);
\draw[help lines, gray, dashed, xstep=0.5, ystep=0.5] (-1,-0.5) grid (3,2.5);  % was dotted
\fill[color=black] (1,1) circle (2.0pt);
%%\fill[red!50!white, opacity=0.50] (1,1) rectangle (6,6);
\draw[color=blue, thick] (1.0,-0.2) -- (1.0,2.8);
\draw[color=blue, thick] (-2.0,2.0) -- (4.0,0.0);
\draw[color=black] (1.0,1.75) node[anchor=west] {line 1: $x=1$};
\draw[color=black] (1.0,1.0) node[anchor=south west] {intersection point};
\draw[color=black] (2.0,0.7) node[anchor=north east] {line 2: $y=-\frac{1}{3}x+\frac{4}{3}$};
\end{tikzpicture}
}%
%%\MUGraphicsSolo{Aufgabe_Schnittpunkt_Geraden_1.png}{scale=1}{width:450px}
\end{center}
>From this figure one reads off the coordinates of the intersection point as
$\MPointTwo{x = 1}{y = 1}$. Hence, the solution set is ${\ML} = \{ \MPointTwo{1}{1} \}$.
\end{MHint}
\end{MExercise}
\end{MExercises}


\MSubsection{LS in three Variables}
\MLabel{M04_3_Unbekannte}

\begin{MIntro}
\MDeclareSiteUXID{VBKM04_DreiUnbekannte_Intro}
In the following section we will slightly increase the level of difficulty and 
discuss slightly more complex systems.

\begin{MExample}
\MLabel{M04_einfuehrendes_Bsp_2}
While playing, three children find a wallet with $30$ Euro in it. The first child says: 
``If I keep the money for myself, I will have twice as much money as you both!''  whereupon the 
second child proudly boasts: ``And if I simply pocket the found money, I will have three 
times as much money as you both!'' The third child can only smile smugly: ``And if I take the 
money, I will be five times as rich as you two!'' How much money did the children
own before thet found the wallet?

Let the amounts of money (in Euro) the three children owned before the trove denoted by 
$x$, $y$, and $z$, respectively. The statement of the first child can be translated into
an algebraic equation as follows:
$$x + 30 = 2 (y + z) \Leftrightarrow x - 2 y - 2 z = - 30 \; : \MBlank\mbox{equation}\MBlank (1) \MDFPeriod$$
Likewise, the statement of the second child can be translated into
$$y + 30 = 3 (x + z) \Leftrightarrow - 3 x + y - 3 z = - 30 \; : \MBlank\mbox{equation}\MBlank (2) \MDFPeriod$$
And finally, the statement of the third child is translated into
$$z + 30 = 5 (x + y) \Leftrightarrow - 5 x - 5 y + z = - 30 \; : \MBlank\mbox{equation}\MBlank (3) \MDFPeriod$$
So there arises a system of three linear equations in tree variables denoted here by $x$, $y$, and $z$. 
\end{MExample}
The reader, who is interested in the solution of this little riddle, will find it below 
worked out in detail using both the \textbf{substitution method} 
(see example~\MRef{M04_einf_Bsp_2_rech}) and the addition method (see example~\MRef{M04_einf_Bsp_2_rech_2}).
\begin{MInfo}
\MLabel{M04_3x3_system}
A system of three linear equations in the three variables $x$, $y$, and $z$ has
the following form:
\begin{eqnarray*}
a_{11} \cdot x + a_{12} \cdot y + a_{13} \cdot z & = b_1 \MDFPSpace, \\
a_{21} \cdot x + a_{22} \cdot y + a_{23} \cdot z & = b_2 \MDFPSpace, \\
a_{31} \cdot x + a_{32} \cdot y + a_{33} \cdot z & = b_3 \MDFPeriod
\end{eqnarray*}
Here, $a_{11}$, $a_{12}$, $a_{13}$, $a_{21}$, $a_{22}$, $a_{23}$, $a_{31}$, $a_{32}$, and $a_{33}$
are the \textbf{coefficients} and $b_1$, $b_2$, and $b_3$ the right-hand sides
of the \textbf{system of linear equations}.

Again, the \textbf{system of linear equations} is called \textbf{homogeneous} if 
the right-hand sides $b_1$, $b_2$, and $b_3$ are zero ($b_1 = 0$, $b_2 = 0$, $b_3 = 0$). 
Otherwise, the system is called \textbf{inhomogeneous}.
\end{MInfo}
\end{MIntro}

\begin{MXContent}{Solvability and Comparison Method, graphical Interpretation}{Solvability}{STD}
\MDeclareSiteUXID{VBKM04_LoesbarkeitUndLoesungen3}
As described in section~\MRef{M04_2_Unbekannte} for systems of two linear equations in two variables,
the question for solvability and the solution of the system can be traced back very clearly
to the question for existence and position of the intersection point of two lines. And of course, 
one should think about whether for systems of three linear equations a similar graphical 
interpretation can be found.

If the previous two-dimensional space  (for the variables $x$ and $y$) 
is supplemented by another dimension or variable, namely $z$, then using a linear equation in this
three variables 
$$a_{11} \cdot x + a_{12} \cdot y + a_{13} \cdot z = b_1 \MDFPSpace  $$
a \MEntry{plane}{equation of a plane} is represented \textbf{in general form}. This equation is 
similar to the equation of a line we already investigated. For $a_{13} \neq 0$, this equation
can be solved for $z$:
$$z = \Mtfrac{b_1}{a_{13}} - \Mtfrac{a_{11}}{a_{13}} \cdot x - \Mtfrac{a_{12}}{a_{13}} \cdot y \MDFPSpace , $$
which is the \textbf{explicit form} of the equation of the very same plane. The last equation assigns
every pair $\MPointTwo{x}{y}$, i.e.\ every point in the $x$-$y$-plane, a value $z$, i.e.\ a height 
in three-dimensional space. Thus, a surface above the $x$-$y$-plane is created, which is a plane due to the 
linearity of the equation.

Now, not only the first equation of the system \MRef{M04_3x3_system} has to be satisfied but 
\textbf{simultaneously} also the second and third equation which can be graphically interpreted as planes 
as well. If we are now interested in the solution of a system of three linear equations, 
we have to investigate -- in the graphical interpretation -- the \textbf{intersection behaviour}
 of three planes. On this point, let us first investigate an example.

\begin{MExample}
\MLabel{M04_3x3_bsp_gleichsetz}
Find the solution set of the following system of linear equations
$$\begin{array}{lcrcl} \mbox{equation}\MBlank (1): & & x + y - z & = & 0 \MDFPSpace, \\
\mbox{equation}\MBlank (2): & & x + y + z & = & 6 \MDFPSpace, \\
\mbox{equation}\MBlank (3): & & 2 x - y + z & = & 4\MDFPeriod \end{array} $$
The basic set is the set of real numbers $\R$.


Each of the three equations can be solved for $z$ easily:
$$\begin{array}{lcrcl} \mbox{equation}\MBlank (1'): & & z & = & x + y \MDFPSpace,\\
\mbox{equation}\MBlank (2'): & & z & = & 6 - x - y \MDFPSpace,\\
\mbox{equation}\MBlank (3'): & & z & = & 4 - 2 x + y \MDFPeriod \end{array}$$
\textbf{Equating} the right-hand sides of equation~$(1')$ and equation~$(2')$ 
corresponds graphically to the determination of the 
\MEntry{intersection line}{intersection line} of the two planes described by these equations:
$$x + y = 6 - x - y \Leftrightarrow 2 x + 2 y = 6 \Leftrightarrow y = 3 - x \; : \MBlank\mbox{equation}\MBlank (A) \MDFPeriod $$
Substituting this relation into equation~$(1')$ or equation~$(2')$ results 
in an equation for the $z$-coordinate of the intersection line; in this case
we have $z=3$. The following figure shows this intersection line described by 
equation $(A)$ as the intersection of the two non-parallel planes described 
by equation $(1')$ and equation $(2')$.

\begin{center}
\MTikzAuto{%
\tdplotsetmaincoords{60}{60}
\begin{tikzpicture}[tdplot_main_coords]
      \draw[thick,->] (-5,0,0) -- (6,0,0) node[anchor=north east]{\Large $x$};
      \draw[thick,->] (0,-5,0) -- (0,6,0) node[anchor=north west]{\Large $y$};
      \draw[thick,->] (0,0,-5) -- (0,0,5) node[anchor=south]{\Large $z$};
% Koordinaten-Box in der xy-Ebene
\pgfmathsetmacro{\ax}{-4}
\pgfmathsetmacro{\ay}{-4}
\pgfmathsetmacro{\az}{0}
\tdplottransformmainscreen{\ax}{\ay}{\az}
\pgfpathmoveto{\pgfpoint{\tdplotresx cm}{\tdplotresy cm}}
%
\pgfmathsetmacro{\ax}{-4}
\pgfmathsetmacro{\ay}{4}
\pgfmathsetmacro{\az}{0}
\tdplottransformmainscreen{\ax}{\ay}{\az}
\pgfpathlineto{\pgfpoint{\tdplotresx cm}{\tdplotresy cm}}
%
\pgfmathsetmacro{\ax}{4}
\pgfmathsetmacro{\ay}{4}
\pgfmathsetmacro{\az}{0}
\tdplottransformmainscreen{\ax}{\ay}{\az}
\pgfpathlineto{\pgfpoint{\tdplotresx cm}{\tdplotresy cm}}
%
\pgfmathsetmacro{\ax}{4}
\pgfmathsetmacro{\ay}{-4}
\pgfmathsetmacro{\az}{0}
\tdplottransformmainscreen{\ax}{\ay}{\az}
\pgfpathlineto{\pgfpoint{\tdplotresx cm}{\tdplotresy cm}}
%
\pgfpathclose
\pgfsetstrokecolor{black}
\pgfusepath{stroke}
% Beschriftungen
\draw[color=black,tdplot_main_coords] (-4.0,-4.0,0.0) node[anchor=north east] {$\MPointTwo{-4}{-4}$};
\draw[color=black,tdplot_main_coords] (-4.0,4.0,0.0) node[anchor=south] {$\MPointTwo{-4}{4}$};
\draw[color=black,tdplot_main_coords] (4.0,-4.0,0.0) node[anchor=north] {$\MPointTwo{4}{-4}$};
\draw[color=black,tdplot_main_coords] (4.0,4.0,0.0) node[anchor=north west] {$\MPointTwo{4}{4}$};
\draw[color=black,tdplot_main_coords] (0.0,0.0,-4.0) -- (0.0,0.1,-4.0) node[anchor=west] {$-16$};
\draw[color=black,tdplot_main_coords] (0.0,0.0,-2.0) -- (0.0,0.1,-2.0) node[anchor=west] {$-8$};
\draw[color=black,tdplot_main_coords] (0.0,0.0,2.0) -- (0.0,0.1,2.0) node[anchor=west] {$8$};
\draw[color=black,tdplot_main_coords] (0.0,0.0,4.0) -- (0.0,0.1,4.0) node[anchor=west] {$16$};
% Ebene z=x+y, Teil 1
\pgfmathsetmacro{\ax}{4}
\pgfmathsetmacro{\ay}{-1}
\pgfmathsetmacro{\az}{0.75} %3
\tdplottransformmainscreen{\ax}{\ay}{\az}
\pgfpathmoveto{\pgfpoint{\tdplotresx cm}{\tdplotresy cm}}
%
\pgfmathsetmacro{\ax}{4}
\pgfmathsetmacro{\ay}{4}
\pgfmathsetmacro{\az}{-0.5} % -2
\tdplottransformmainscreen{\ax}{\ay}{\az}
\pgfpathlineto{\pgfpoint{\tdplotresx cm}{\tdplotresy cm}}
%
\pgfmathsetmacro{\ax}{-1}
\pgfmathsetmacro{\ay}{4}
\pgfmathsetmacro{\az}{0.75} % 3
\tdplottransformmainscreen{\ax}{\ay}{\az}
\pgfpathlineto{\pgfpoint{\tdplotresx cm}{\tdplotresy cm}}
%
\pgfpathclose
\pgfsetstrokecolor{green}
\pgfsetfillcolor{green}
\pgfsetfillopacity{0.5}
\pgfusepath{stroke,fill}
% Ebene z=6-x-y
\pgfmathsetmacro{\ax}{-4}
\pgfmathsetmacro{\ay}{-4}
\pgfmathsetmacro{\az}{-2} % -8
\tdplottransformmainscreen{\ax}{\ay}{\az}
\pgfpathmoveto{\pgfpoint{\tdplotresx cm}{\tdplotresy cm}}
%
\pgfmathsetmacro{\ax}{-4}
\pgfmathsetmacro{\ay}{4}
\pgfmathsetmacro{\az}{0}
\tdplottransformmainscreen{\ax}{\ay}{\az}
\pgfpathlineto{\pgfpoint{\tdplotresx cm}{\tdplotresy cm}}
%
\pgfmathsetmacro{\ax}{4}
\pgfmathsetmacro{\ay}{4}
\pgfmathsetmacro{\az}{2} % 8
\tdplottransformmainscreen{\ax}{\ay}{\az}
\pgfpathlineto{\pgfpoint{\tdplotresx cm}{\tdplotresy cm}}
%
\pgfmathsetmacro{\ax}{4}
\pgfmathsetmacro{\ay}{-4}
\pgfmathsetmacro{\az}{0}
\tdplottransformmainscreen{\ax}{\ay}{\az}
\pgfpathlineto{\pgfpoint{\tdplotresx cm}{\tdplotresy cm}}
%
\pgfpathclose
\pgfsetstrokecolor{red}
\pgfsetfillcolor{red}
\pgfsetfillopacity{0.5}
\pgfusepath{stroke,fill}
% Ebene z=x+y, Teil 2
\pgfmathsetmacro{\ax}{-4}
\pgfmathsetmacro{\ay}{-4}
\pgfmathsetmacro{\az}{3.5} % 14
\tdplottransformmainscreen{\ax}{\ay}{\az}
\pgfpathmoveto{\pgfpoint{\tdplotresx cm}{\tdplotresy cm}}
%
\pgfmathsetmacro{\ax}{-4}
\pgfmathsetmacro{\ay}{4}
\pgfmathsetmacro{\az}{1.5} %6
\tdplottransformmainscreen{\ax}{\ay}{\az}
\pgfpathlineto{\pgfpoint{\tdplotresx cm}{\tdplotresy cm}}
%
\pgfmathsetmacro{\ax}{-1}
\pgfmathsetmacro{\ay}{4}
\pgfmathsetmacro{\az}{0.75} % 3
\tdplottransformmainscreen{\ax}{\ay}{\az}
\pgfpathlineto{\pgfpoint{\tdplotresx cm}{\tdplotresy cm}}
%
\pgfmathsetmacro{\ax}{4}
\pgfmathsetmacro{\ay}{-1}
\pgfmathsetmacro{\az}{0.75} % 3
\tdplottransformmainscreen{\ax}{\ay}{\az}
\pgfpathlineto{\pgfpoint{\tdplotresx cm}{\tdplotresy cm}}
%
\pgfmathsetmacro{\ax}{4}
\pgfmathsetmacro{\ay}{-4}
\pgfmathsetmacro{\az}{1.5} % 6
\tdplottransformmainscreen{\ax}{\ay}{\az}
\pgfpathlineto{\pgfpoint{\tdplotresx cm}{\tdplotresy cm}}
%
\pgfpathclose
\pgfsetstrokecolor{green}
\pgfsetfillcolor{green}
\pgfsetfillopacity{0.5}
\pgfusepath{stroke,fill}
% Gerade y=3-x
\pgfmathsetmacro{\ax}{4}
\pgfmathsetmacro{\ay}{-1}
\pgfmathsetmacro{\az}{0.75} % 3
\tdplottransformmainscreen{\ax}{\ay}{\az}
\pgfpathmoveto{\pgfpoint{\tdplotresx cm}{\tdplotresy cm}}
%
\pgfmathsetmacro{\ax}{-1}
\pgfmathsetmacro{\ay}{4}
\pgfmathsetmacro{\az}{0.75} % 3
\tdplottransformmainscreen{\ax}{\ay}{\az}
\pgfpathlineto{\pgfpoint{\tdplotresx cm}{\tdplotresy cm}}
%
\pgfsetlinewidth{1mm}
\pgfsetstrokecolor{yellow}
\pgfusepath{stroke}
% Legende
\pgfsetfillopacity{1.0}
\pgfsetlinewidth{0.4pt}
\fill[red, opacity=0.50, tdplot_main_coords] (-5.0,-5.0,-4.5) circle (1.5mm);
\draw[red, tdplot_main_coords] (-5.0,-5.0,-4.5) circle (1.5mm);
\fill[green, opacity=0.50, tdplot_main_coords] (-5.0,-5.0,-5.0) circle (1.5mm);
\draw[green, tdplot_main_coords] (-5.0,-5.0,-5.0) circle (1.5mm);
\fill[yellow, tdplot_main_coords] (-5.0,-5.0,-5.5) circle (1.0mm);
\draw[color=black] (-5.0,-5.0,-4.5) node[anchor=west] {\ eq. $(1')$: $z=x+y$};
\draw[color=black] (-5.0,-5.0,-5.0) node[anchor=west] {\ eq. $(2')$: $z=6-x-y$};
\draw[color=black] (-5.0,-5.0,-5.5) node[anchor=west] {\ eq. $(A)$: $y=3-x$ ($z=3$)};
\end{tikzpicture}
}%
\end{center}
The totally analogous statement holds if the right-hand sides of equation~$(2')$ and 
equation~$(3')$ are \textbf{equated}. Then, one obtains for the \textbf{intersection line}
of the planes~$(2)$ and $(3)$:
$$6 - x - y = 4 - 2 x + y \Leftrightarrow x - 2 y = - 2 \Leftrightarrow y = 1 + \Mtfrac12 x \; :
\MBlank\mbox{equation}\MBlank (B) \MDFPeriod $$
Substituting this relation into equation~$(2')$ or equation~$(3')$ results 
in an equation for the $z$-coordinate of the intersection line; in this case
we have $z=5-\frac{3}{2}x$.
The following figure shows the intersection line described by equation~$(B)$ as 
the intersection of the two non-parallel planes described by equation~$(2')$
and equation~$(3')$.

\begin{center}
\MTikzAuto{%
\tdplotsetmaincoords{60}{60}
\begin{tikzpicture}[tdplot_main_coords]
      \draw[thick,->] (-5,0,0) -- (6,0,0) node[anchor=north east]{\Large $x$};
      \draw[thick,->] (0,-5,0) -- (0,6,0) node[anchor=north west]{\Large $y$};
      \draw[thick,->] (0,0,-5) -- (0,0,5) node[anchor=south]{\Large $z$};
% Koordinaten-Box in der xy-Ebene
\pgfmathsetmacro{\ax}{-4}
\pgfmathsetmacro{\ay}{-4}
\pgfmathsetmacro{\az}{0}
\tdplottransformmainscreen{\ax}{\ay}{\az}
\pgfpathmoveto{\pgfpoint{\tdplotresx cm}{\tdplotresy cm}}
%
\pgfmathsetmacro{\ax}{-4}
\pgfmathsetmacro{\ay}{4}
\pgfmathsetmacro{\az}{0}
\tdplottransformmainscreen{\ax}{\ay}{\az}
\pgfpathlineto{\pgfpoint{\tdplotresx cm}{\tdplotresy cm}}
%
\pgfmathsetmacro{\ax}{4}
\pgfmathsetmacro{\ay}{4}
\pgfmathsetmacro{\az}{0}
\tdplottransformmainscreen{\ax}{\ay}{\az}
\pgfpathlineto{\pgfpoint{\tdplotresx cm}{\tdplotresy cm}}
%
\pgfmathsetmacro{\ax}{4}
\pgfmathsetmacro{\ay}{-4}
\pgfmathsetmacro{\az}{0}
\tdplottransformmainscreen{\ax}{\ay}{\az}
\pgfpathlineto{\pgfpoint{\tdplotresx cm}{\tdplotresy cm}}
%
\pgfpathclose
\pgfsetstrokecolor{black}
\pgfusepath{stroke}
% Beschriftungen
\draw[color=black,tdplot_main_coords] (-4.0,-4.0,0.0) node[anchor=north east] {$\MPointTwo{-4}{-4}$};
\draw[color=black,tdplot_main_coords] (-4.0,4.0,0.0) node[anchor=south] {$\MPointTwo{-4}{4}$};
\draw[color=black,tdplot_main_coords] (4.0,-4.0,0.0) node[anchor=north] {$\MPointTwo{4}{-4}$};
\draw[color=black,tdplot_main_coords] (4.0,4.0,0.0) node[anchor=north west] {$\MPointTwo{4}{4}$};
\draw[color=black,tdplot_main_coords] (0.0,0.0,-4.0) -- (0.0,0.1,-4.0) node[anchor=west] {$-16$};
\draw[color=black,tdplot_main_coords] (0.0,0.0,-2.0) -- (0.0,0.1,-2.0) node[anchor=west] {$-8$};
\draw[color=black,tdplot_main_coords] (0.0,0.0,2.0) -- (0.0,0.1,2.0) node[anchor=west] {$8$};
\draw[color=black,tdplot_main_coords] (0.0,0.0,4.0) -- (0.0,0.1,4.0) node[anchor=west] {$16$};
% Ebene z=6-x-y, Teil 1
\pgfmathsetmacro{\ax}{-4}
\pgfmathsetmacro{\ay}{-1}
\pgfmathsetmacro{\az}{2.75} % 11
\tdplottransformmainscreen{\ax}{\ay}{\az}
\pgfpathmoveto{\pgfpoint{\tdplotresx cm}{\tdplotresy cm}}
%
\pgfmathsetmacro{\ax}{4}
\pgfmathsetmacro{\ay}{3}
\pgfmathsetmacro{\az}{-0.25} % -1
\tdplottransformmainscreen{\ax}{\ay}{\az}
\pgfpathlineto{\pgfpoint{\tdplotresx cm}{\tdplotresy cm}}
%
\pgfmathsetmacro{\ax}{4}
\pgfmathsetmacro{\ay}{4}
\pgfmathsetmacro{\az}{-0.5} % -2
\tdplottransformmainscreen{\ax}{\ay}{\az}
\pgfpathlineto{\pgfpoint{\tdplotresx cm}{\tdplotresy cm}}
%
\pgfmathsetmacro{\ax}{-4}
\pgfmathsetmacro{\ay}{4}
\pgfmathsetmacro{\az}{1.5} % 6
\tdplottransformmainscreen{\ax}{\ay}{\az}
\pgfpathlineto{\pgfpoint{\tdplotresx cm}{\tdplotresy cm}}
%
\pgfpathclose
\pgfsetstrokecolor{green}
\pgfsetfillcolor{green}
\pgfsetfillopacity{0.5}
\pgfusepath{stroke,fill}
% Ebene z=4-2x+y
\pgfmathsetmacro{\ax}{-4}
\pgfmathsetmacro{\ay}{-4}
\pgfmathsetmacro{\az}{2} % 8
\tdplottransformmainscreen{\ax}{\ay}{\az}
\pgfpathmoveto{\pgfpoint{\tdplotresx cm}{\tdplotresy cm}}
%
\pgfmathsetmacro{\ax}{-4}
\pgfmathsetmacro{\ay}{4}
\pgfmathsetmacro{\az}{4} % 16
\tdplottransformmainscreen{\ax}{\ay}{\az}
\pgfpathlineto{\pgfpoint{\tdplotresx cm}{\tdplotresy cm}}
%
\pgfmathsetmacro{\ax}{4}
\pgfmathsetmacro{\ay}{4}
\pgfmathsetmacro{\az}{0}
\tdplottransformmainscreen{\ax}{\ay}{\az}
\pgfpathlineto{\pgfpoint{\tdplotresx cm}{\tdplotresy cm}}
%
\pgfmathsetmacro{\ax}{4}
\pgfmathsetmacro{\ay}{-4}
\pgfmathsetmacro{\az}{-2} % -8
\tdplottransformmainscreen{\ax}{\ay}{\az}
\pgfpathlineto{\pgfpoint{\tdplotresx cm}{\tdplotresy cm}}
%
\pgfpathclose
\pgfsetstrokecolor{blue}
\pgfsetfillcolor{blue}
\pgfsetfillopacity{0.5}
\pgfusepath{stroke,fill}
% Ebene z=6-x-y, Teil 2
\pgfmathsetmacro{\ax}{-4}
\pgfmathsetmacro{\ay}{-1}
\pgfmathsetmacro{\az}{2.75} % 11
\tdplottransformmainscreen{\ax}{\ay}{\az}
\pgfpathmoveto{\pgfpoint{\tdplotresx cm}{\tdplotresy cm}}
%
\pgfmathsetmacro{\ax}{-4}
\pgfmathsetmacro{\ay}{-4}
\pgfmathsetmacro{\az}{3.5} % 14
\tdplottransformmainscreen{\ax}{\ay}{\az}
\pgfpathlineto{\pgfpoint{\tdplotresx cm}{\tdplotresy cm}}
%
\pgfmathsetmacro{\ax}{4}
\pgfmathsetmacro{\ay}{-4}
\pgfmathsetmacro{\az}{1.5} % 6
\tdplottransformmainscreen{\ax}{\ay}{\az}
\pgfpathlineto{\pgfpoint{\tdplotresx cm}{\tdplotresy cm}}
%
\pgfmathsetmacro{\ax}{4}
\pgfmathsetmacro{\ay}{3}
\pgfmathsetmacro{\az}{-0.25} % -1
\tdplottransformmainscreen{\ax}{\ay}{\az}
\pgfpathlineto{\pgfpoint{\tdplotresx cm}{\tdplotresy cm}}
%
\pgfpathclose
\pgfsetstrokecolor{green}
\pgfsetfillcolor{green}
\pgfsetfillopacity{0.5}
\pgfusepath{stroke,fill}
% Gerade y=1+(1/2)x
\pgfmathsetmacro{\ax}{-4}
\pgfmathsetmacro{\ay}{-1}
\pgfmathsetmacro{\az}{2.75} % 11
\tdplottransformmainscreen{\ax}{\ay}{\az}
\pgfpathmoveto{\pgfpoint{\tdplotresx cm}{\tdplotresy cm}}
%
\pgfmathsetmacro{\ax}{4}
\pgfmathsetmacro{\ay}{3}
\pgfmathsetmacro{\az}{-0.25} % -1
\tdplottransformmainscreen{\ax}{\ay}{\az}
\pgfpathlineto{\pgfpoint{\tdplotresx cm}{\tdplotresy cm}}
%
\pgfsetlinewidth{1mm}
\pgfsetstrokecolor{cyan}
\pgfusepath{stroke}
% Legende
\pgfsetfillopacity{1.0}
\pgfsetlinewidth{0.4pt}
%
\fill[green, opacity=0.50, tdplot_main_coords] (-5.0,-5.0,-4.2) circle (1.5mm);
\draw[green, tdplot_main_coords] (-5.0,-5.0,-4.2) circle (1.5mm);
\fill[blue, opacity=0.50, tdplot_main_coords] (-5.0,-5.0,-4.8) circle (1.5mm);
\draw[blue, tdplot_main_coords] (-5.0,-5.0,-4.8) circle (1.5mm);
\fill[cyan, tdplot_main_coords] (-5.0,-5.0,-5.4) circle (1mm);
%
\draw[color=black] (-5.0,-5.0,-4.2) node[anchor=west] {\ eq. $(2')$: $z=6-x-y$};
\draw[color=black] (-5.0,-5.0,-4.8) node[anchor=west] {\ eq. $(3')$: $z=4-2x+y$};
\draw[color=black] (-5.0,-5.0,-5.4) node[anchor=west] {\ eq. $(B)$: $y=1+\frac{1}{2}x$};
\draw[color=black] (-5.0,-5.0,-6.0) node[anchor=west] {\ \hspace{1.3cm} ($z=5-\frac{3}{2}x$)};
\end{tikzpicture}
}%
\end{center}
Since all equations in the initial linear system have to hold simultaneously, 
the two equations for the two intersection lines of the three planes also have to hold 
simultaneously. Graphically, this is the case at the intersection point of the intersection 
lines. This intersection point is found by \textbf{equating} the right-hand sides of 
equation~$(A)$ and equation~$(B)$:
$$3 - x = 1 + \Mtfrac12 x \Leftrightarrow \Mtfrac32 x = 2 \Leftrightarrow x = \Mtfrac43 \MDFPeriod $$
The value of $y$ can be calculated by inserting the value of $x$, for example, 
in equation~$(A)$:
$$y = 3 - \Mtfrac43 \Leftrightarrow y = \Mtfrac53 \MDFPeriod $$
The following figure shows the intersection lines described by equation~$(A)$ and 
equation~$(B)$ in the $x$-$y$-plane (view from above) and their intersection point.


\begin{center}
\MTikzAuto{%
\begin{tikzpicture}[x=1.0cm, y=1.0cm] 
%Koordinatensystem
\node (xMAX) at (5.0,0){};
\node (yMAX) at (0,5.0){};
\draw[->,color=black] (-5.0,0) -- (xMAX);
\foreach \x in {-4, -3, -2, -1, 0, 1, 2, 3, 4}
\draw[shift={(\x,0)},color=black] (0pt,2pt) -- (0pt,-2pt) node[above right] {\scriptsize $\x$};
\draw[->,color=black] (0,-2.5) -- (yMAX);
\foreach \y in {-2, -1, 1, 2, 3, 4}
\draw[shift={(0,\y)},color=black] (2pt,0pt) -- (-2pt,0pt) node[above right] {\scriptsize $\y$};
%Achsenbeschriftung
\draw (xMAX) node[anchor=south east] {$x$};
\draw (yMAX) node[anchor=north east] {$y$};
%Beschriftung und Graphen
\clip(-4.1,-2.4) rectangle (4.1,4.1);
\draw[help lines, gray, dashed] (-4,-2) grid (4,4); % was dotted
%\fill[color=black] (1,1) circle (2.0pt);
%%\fill[red!50!white, opacity=0.50] (1,1) rectangle (6,6);
\draw[color=yellow, very thick] (-4.0,7.0) -- (4.0,-1.0);
\draw[color=cyan, very thick] (-4.0,-1.0) -- (4.0,3.0);
\fill[color=black] (1.3333333,1.6666666) circle (2.0pt);
\draw[color=black] (1.45,1.85) node[anchor=south] 
{$\MPointTwo{\frac{4}{3}}{\frac{5}{3}}$};
\draw[color=black] (4.0,-1.5) node[anchor=east] {eq. $(A)$: $y=3-x$};
\draw[color=black] (-4.0,-1.5) node[anchor=west] {eq. $(B)$: $y=1+\frac{1}{2}x$};
%\draw[color=black] (2.0,0.7) node[anchor=north east] {Gerade 2: $y=-\frac{1}{3}x+\frac{4}{3}$};
\end{tikzpicture}
}%
\end{center}
Finally, the value of $z$ results by inserting the values of $x$ and $y$, for example, 
in equation~$(1')$:
$$z = \Mtfrac43 + \Mtfrac53 \Leftrightarrow z = \Mtfrac93 \Leftrightarrow z = 3 \MDFPeriod $$
Thus, the given system of linear equations has a unique solution. The solution set is
${\ML} = \{ \MPointThree{x = \Mtfrac43}{y = \Mtfrac53}{z = 3} \}$.
\end{MExample}
The graphical interpretation is now used to describe the solution sets of systems of three linear 
equations that can occur: 
\begin{itemize}
\item{If (at least) \textbf{two} of the three \textbf{planes are parallel} to each other 
(without being congruent), the system has \textbf{no solution}: Planes that are parallel 
(without being congruent) do not intersect and hence the equations describing the planes 
cannot hold simultaneously.}
\item{If \textbf{two} of the three \textbf{planes are congruent}, the intersection set with
the third (non-parallel) plane is an \textbf{intersection line}. All points on this intersection
line are solutions of the system. Hence, the \textbf{solution set} is infinite.}
\item{If the \textbf{three planes are congruent}, all points of the (congruent) planes are 
solutions of the system. Again, the \textbf{solution set} is infinite.}
\item{A unique solution can only exist in this last case: The three (non-parallel and non-congruent)
planes have \textbf{three intersection lines} (intersection of plane~$(1)$ and plane~$(2)$, 
intersection of plane~$(2)$ and plane~$(3)$, and intersection of plane~$(1)$ and plane~$(3)$):
\begin{itemize}
\item{If \textbf{two of the three intersection lines are parallel}, the system has \textbf{no solution}.}
\item{If \textbf{two of the three intersection lines are congruent}, the system has 
\textbf{an infinite number of solutions}.}
\item{If \textbf{the three intersection lines intersect in one point}, the \textbf{solution is unique}
and the \textbf{solution set} consists of \textbf{exactly one element}.}
\end{itemize}}
\end{itemize}
Despite this illustrative graphical interpretation, the case analysis for the solution set is 
rather complex. Therefore, algebraic methods for investigating the solvability 
of systems of linear equations and for finding their solution sets will be all the more important, 
in particular, if the systems will get larger
and the graphical interpretation gets more complex or even impossible.
The \MEntry{addition method}{addition method}, which will be discussed again below,
is one of these suitable methods.

By the way, in the above example~\MRef{M04_3x3_bsp_gleichsetz} it is not necessary to 
find the third intersection line and to check whether this third intersection line intersects
the two other lines in their intersection point: This is automatically true since 
equating the right-hand sides of equation~$(1')$ and equation~$(2')$  (first intersection
line/equation~$(A)$) and equating the right-hand sides of equation~$(2')$ and equation~$(3')$ 
(second intersection line/equation~$(B)$) guaranties the validity of the equation of 
the third intersection line (right-hand side of $(1')$ = right-hand side of $(3')$):
$$x + y = 4 - 2 x + y \Leftrightarrow 3 x = 4 \; : \MBlank\mbox{equation}\MBlank (C) \MDFPeriod $$

In the example, the \MSRef{M04_gleichsetzmethode}{comparison method}
was used since it relates very closely to the geometric interpretation. \textbf{Equating}
explicit equations of planes or lines, respectively, corresponds precisely to the determination
of intersection lines and points.
 
\begin{MInfo}
In the comparison method, as a first step the three linear equations are solved for 
one of the variables -- or for a multiple of one of the variables. Then, the resulting 
new equations are \textbf{equated} in pairs. It is sufficient to equate two pairs.  
Altogether, a \textbf{system of two linear equations} in only two variables results,
which can be subsequently be investigated using methods described in 
section~\MRef{M04_2_Unbekannte}.
\end{MInfo}

\begin{MExercise}
Find the solution set of the following system of linear equations
\begin{eqnarray*}
- x + z & = & 2 \MDFPSpace, \\ - x + y + 2 z & = & 1 \MDFPSpace, \\ y + z & = & 11 \MDFPeriod
\end{eqnarray*}
Use the comparison method and be clever!

\begin{MHint}{Solution}
The first equation does not depend an the variable $y$. Hence, it would
be clever to eliminate $y$ from the second and third equations as well. 
For this, the second and third equations are solved for $y$:
$$y = 1 + x - 2 z \; \MBlank\mbox{and}\MBlank \; y = 11 - z \MDFPeriod $$
Then the right-hand sides are equated:
$$1 + x - 2 z = 11 - z \Leftrightarrow - x + z = - 10 \MDFPeriod $$
Next, this equation has to be used together with the first equation of the initial system. 
Obviously, the left-hand sides of these two equations (the combination $- x + z$ of 
the variables $x$ and $z$) are identical whereas the right-hand sides 
(the values $2$ and $-10$) differ. This is a contraction and hence the system
of linear equations has no solution: ${\ML} = \MEmptyset$.

Other clever approaches exist.
\end{MHint}
\end{MExercise}
\end{MXContent}

\begin{MXContent}{Substitution Method}{Substitution Method}{STD}
\MDeclareSiteUXID{VBKM04_Einsetzmethode3}
The \textbf{substitution method} was already discussed in 
section~\MRef{M04_einsetz_gleichsetz} for systems of two linear equations. 
For systems of three linear equations like the system in example~\MRef{M04_3x3_system},
the approach is basically the same.

\begin{MExample}
\MLabel{M04_einf_Bsp_2_rech}
Let us return to the first example~\MRef{M04_einfuehrendes_Bsp_2} of this section. The 
system of linear equations for the riddle about the three tempted children reads:
$$\begin{array}{lcrcl} \mbox{equation}\MBlank (1): & & x - 2 y - 2 z & = & - 30 \MDFPSpace, \\
\mbox{equation}\MBlank (2): & & - 3 x + y - 3 z & = & - 30 \MDFPSpace, \\
\mbox{equation}\MBlank (3): & & - 5 x - 5 y + z & = & - 30 \MDFPeriod \end{array}$$
For example, one can start by solving equation~$(1)$ for $x$:
$$x = 2 y + 2 z - 30 \; : \MBlank\mbox{equation}\MBlank (1') \MDFPeriod$$
\textbf{Substituting} this equation in equation~$(2)$ and equation~$(3)$ eliminates
the variable $x$ from these equations:
$$- 3 (2 y + 2 z - 30) + y - 3 z = - 30 \Leftrightarrow - 5 y - 9 z = - 120 \; : \MBlank\mbox{equation}\MBlank (2') \MDFPSpace, $$
$$- 5 (2 y + 2 z - 30) - 5 y + z = - 30 \Leftrightarrow - 15 y - 9 z = - 180 \; : \MBlank\mbox{equation}\MBlank (3') \MDFPeriod $$
This step reduces the initial system to a system of \textbf{two} linear equations in the \textbf{two} variables
$y$ and $z$ which can be solved using the methods from the previous section~\MRef{M04_2_Unbekannte}.
For example, equation~$(2')$ can be solved for $y$:
 $$y = 24 - \Mtfrac95 z \MDFPeriod$$
This relation can be \textbf{substituted} into equation~$(3')$:
$$- 15 (24 - \Mtfrac95 z) - 9 z = - 180 \Leftrightarrow 360 - 27 z + 9 z = 180 \Leftrightarrow 18 z = 180
\Leftrightarrow z = 10 \MDFPeriod$$
So, the value of $y$ is
$$y = 24 - \Mtfrac95 \cdot 10 = 24 - 9 \cdot 2 = 6 $$
and finally the value of $x$ is (using, for example, equation~$(1')$):
$$x = 2 \cdot 6 + 2 \cdot 10 - 30 = 12 + 20 - 30 = 2 \MDFPeriod$$
Hence, the system of linear equations has a unique solution. The solution set consists of exactly
one element, namely ${\ML} = \{ \MPointThree{x = 2}{y = 6}{z = 10} \}$.
\end{MExample}

Since the right-hand sides of the three equations are all equal to $- 30$,
the one and other reader may ask himself whether it wouldn't be more practical
to equate all the right-hand sides in pairs and to continue with the resulting equations.

But this approach is not helpful and -- if you are not careful -- possibly even wrong.
In any case, the number of variables would not decrease by this approach. But this is exactly 
what the substitution method and the comparison method are for: In both approaches (and
in the addition method as well), as a first and second step one variable is eliminated such 
that the initial system reduces to a system of two linear equations (and hence to a
simpler problem).

By the way, the approach for the solution of this reduced problem does not depend on the 
initial approach. In other words, it is allowed and can be even clever to start with one 
method, e.g. the substitution method, to reduce the system of three linear equations to 
a system of two linear equations and to solve this simpler system using another 
method, e.g. the substitution method. In this sense, the methods can be mixed.


\begin{MInfo}
In the substitution method, as a first step one of the three linear equations is solved 
for one of the variables -- or for a multiple of one of the variables. As a second step
the resulting relation is \textbf{substituted} into one of the two other linear equations.
It results a \textbf{system of only two linear equations} in the (remaining) \textbf{two}
variables. This system can be solved using one of the methods described in 
section~\MRef{M04_2_Unbekannte}.
\end{MInfo}
\end{MXContent}

\begin{MXContent}{Addition Method}{Addition method}{STD}
\MLabel{M04_3x3_addition}
\MDeclareSiteUXID{VBKM04_Additionsmethode3}
The idea of the addition method, which we already discussed a little before (see
section~\MRef{M04_addition}), is to \textbf{add} equations of the system such that
the number of variables occurring in the system is reduced. For this, one 
of the equations often has to be multiplied by a cleverly chosen factor before 
\textbf{adding} these equations.

The addition method for a system of three linear equations in three variables 
shall be presented in a form that can be easily applied to larger systems. 
To illustrate the approach, we discuss again the system in the first
example~\MRef{M04_einfuehrendes_Bsp_2}, i.e.\ the example of the 
three little crooks.

\begin{MExample}
\MLabel{M04_einf_Bsp_2_rech_2}
The system of linear equations to be solved reads then:
$$\begin{array}{lcrcl} \mbox{equation}\MBlank (1): & & x - 2 y - 2 z & = & - 30 \MDFPSpace, \\
\mbox{equation}\MBlank (2): & & - 3 x + y - 3 z & = & - 30 \MDFPSpace, \\
\mbox{equation}\MBlank (3): & & - 5 x - 5 y + z & = & - 30 \MDFPeriod \end{array}$$
Equation~$(1)$ is left unchanged in the following. 
But equation~$(2)$ is to be replaced by a new equation resulting from the 
\textbf{addition} of equation~$(2)$ and equation~$(1)$ multiplied by a factor of 
$3$ -- shortly noted as $(2) + 3 \cdot (1)$:
$$(-3 x + y - 3 z) + 3 \cdot (x - 2 y - 2 z) = - 30 + 3 \cdot (- 30) \Leftrightarrow
- 5 y - 9 z = - 120 \; : \MBlank\mbox{equation}\MBlank (2') \MDFPeriod $$
Likewise, equation~$(3)$ will be replaced by $(3) + 5 \cdot (1)$, i.e.\
by the \textbf{sum} of equation~$(3)$ and equation~$(1)$ multiplied by a factor of $5$:
$$(- 5 x - 5 y + z) + 5 \cdot (x - 2 y - 2 z) = - 30 + 5 \cdot (- 30) \Leftrightarrow
-15 y - 9 z = - 180 \; : \MBlank\mbox{equation}\MBlank (3') \MDFPeriod $$
The system now reads as follows:
$$\begin{array}{lcrcl} \mbox{equation}\MBlank (1): & & x - 2 y - 2 z & = & - 30 \MDFPSpace, \\
\mbox{equation}\MBlank (2'): & & - 5 y - 9 z & = & - 120 \MDFPSpace, \\
\mbox{equation}\MBlank (3'): & & - 15 y - 9 z & = & - 180  \MDFPeriod \end{array}$$
Equation~$(2')$ and equation~ $(3')$ do not depend on the variable $x$ anymore --
that was the intention and the reason for choosing the factors $3$ and $5$ above, respectively.

The subsystem that consists of the two equations~$(2')$ and $(3')$ in the two variables 
$y$ and $z$ could now be solved using one of the other methods, e.g.\ the substitution
method. But here, it should be solved completely using the addition method. For this, 
equation~$(2')$ and equation~$(1)$ will be left unchanged in the following. In contrast,
equation~$(3')$ has to be replaced, namely by the sum $(3') + (- 3) \cdot (2')$:
$$(- 15 y - 9 z) + (- 3) \cdot (- 5 y - 9 z) = - 180 + (- 3) \cdot (- 120) \Leftrightarrow
18 z = 180 \; : \MBlank\mbox{equation}\MBlank ({3'}') \MDFPeriod $$
Thus, the system has changed again,
$$\begin{array}{lcrcl} \mbox{equation}\MBlank (1): & & x - 2 y - 2 z & = & - 30 \MDFPSpace, \\
\mbox{equation}\MBlank (2'): & & - 5 y - 9 z & = & - 120 \MDFPSpace, \\
\mbox{equation}\MBlank ({3'}'): & & 18 z & = &  180 \MDFPSpace, \end{array}$$
and now has -- at least concerning the left-hand side -- a kind of \textbf{triangular form}.

Solving for the variables is now very simple: The last equation (equation~$({3'}')$)
only depends on a single variable, namely $z$, and hence can be solved for $z$ immediately: 
$z = 10$. 

This value of $z$ is then inserted in the equation in the line above (equation~$(2')$)
that immediately provides the value of $y$: 
$- 5 y - 9 \cdot 10 = - 120 \Leftrightarrow - 5 y = - 30 \Leftrightarrow y = 6$. 

Finally, inserting the values  of $y$ and $z$ in the first equation (equation~$(1)$)
immediately provides the solution for the remaining variable, in the example this is the 
variable $x$: $x - 2 \cdot 6 - 2 \cdot 10 = - 30 \Leftrightarrow x = 2$.
\end{MExample}

An attentive reader may ask themselves whether -- and if so, why -- one is allowed 
to replace an equation in a system of equations by another equation. In the example
above this occurs three times, e.g. if equation~$(2)$ is replaced by a combination
of equation~$(2)$ and three times equation~$(1)$, i.e.\ equation~$(2')$.

Finding the solution of a system of linear equations requires that all equations 
of the system \textbf{hold simultaneously}, i.e.\ in example~\MRef{M04_einf_Bsp_2_rech_2}
it is required that equation~$(1)$ \textbf{and} equation~$(2)$ hold which clearly implies
that also 
$$\mbox{equation}\MBlank (2) + 3 \cdot \MBlank\mbox{equation}\MBlank (1) 
\Leftrightarrow \MBlank\mbox{equation}\MBlank (2') \MDFPeriod $$
holds. If now equation~$(1)$ \textbf{and} equation~$(2')$ hold simultaneously, 
then immediately follows that equation~$(1)$ and
$$\mbox{equation}\MBlank (2') + (-1) \cdot \MBlank\mbox{equation}\MBlank (1) 
\Leftrightarrow 3\cdot\MBlank\mbox{equation}\MBlank (2) 
\Leftrightarrow \MBlank\mbox{equation}\MBlank (2)  $$
hold simultaneously as well. Hence, one is allowed to replace equation~$(2)$
by equation~$(2')$ in the systems of equations.

Importantly, one can see here: if the \textbf{two}
equations -- equation~$(1)$ and equation~$(2)$ -- were both replaced by equation~(2'),
information would be lost and a mistake would be made.
(The requirement of \textbf{only} $(2')$ instead of $(1)$ \textbf{and}
$(2)$ is much weaker.) This is the reason why in the ``new'' systems
some equations are left unchanged: equation~$(1)$ \textbf{and} equation~$(2)$ are
in the corresponding systems equivalent to equation~$(1)$ \text{and} equation~$(2')$.
The same is true for the other replacement in the example above -- and 
generally for such transformations of systems of linear equations by means 
of the addition method.


\begin{MInfo}
In the \MEntry{addition method}{addition method}, pairs of linear equations of the system 
are added while multiplying (at least) one  of the equations by a clever chosen factor 
(or clever chosen factors) such that in the resulting equations (at least) one variable 
is eliminated. It has to be ensured that in the solution process no information is lost, i.e.\
the number of (information relevant) equations is fixed. For this, it is most clever to bring the 
system into \textbf{triangular from}. Then, the solution can be found very easily.
\end{MInfo}
%\begin{MInfo}
%Die systematische Verallgemeinerung der vorstehend beschriebenen Vorgehensweise bei der Additionsmethode auf Systeme
%aus $n$ linearen Gleichungen - wobei $n$ eine natürliche Zahl bezeichnet ($n \in \N$) - führt auf das sogenannte
%\MEntry{Gaußsche Eliminationsverfahren}{Gaußsches Eliminationsverfahren}. Dieses \textbf{Gaußsche Eliminationsverfahren}
%stellt also einen Algorithmus zur Lösung beliebig großer Linearer Gleichungssysteme dar.
%\end{MInfo}
\end{MXContent}

\begin{MExercises}
\MDeclareSiteUXID{VBKM04_Additionsmethode3_Exercises}
\begin{MExercise}
Find the solution set of the following system of linear equations
\begin{eqnarray*}
2 x - y + 5 z & = & 1 \MDFPSpace, \\ 11 x + 8 z & = & 2 \MDFPSpace, \\ - 4 x + y - 3 z & = & - 1
\end{eqnarray*}
using 
\begin{MExerciseItems}
\item{the substitution method,}
\item{the addition method.}
\end{MExerciseItems}

\begin{MHint}{Solution}
\begin{MExerciseItems}
\item{For example, the first equation ($2 x - y + 5 z = 1$) is solved for $y$ resulting
in $y = 2 x + 5 z - 1$, which is then substituted into the third equation ($- 4 x + y - 3 z = - 1$):
$$- 4 x + (2 x + 5 z - 1) - 3 z = - 1 \Leftrightarrow - 2 x + 2 z = 0 \Leftrightarrow z = x \MDFPeriod$$
The last result, which is already solved for the variable $z$, is substituted into the second
equation ($11 x + 8 z = 2$) that does not depend on $y$:
$$11 x + 8 x = 2 \Leftrightarrow 19 x = 2 \Leftrightarrow x = \Mtfrac{2}{19} \MDFPeriod$$
Then also
$$z = \Mtfrac{2}{19} \MDFPSpace ,$$
and for $y$ one obtains from the first equation:
$$y = 2 \cdot \Mtfrac{2}{19} + 5 \cdot \Mtfrac{2}{19} - 1 = \Mtfrac{4 + 10 - 19}{19} = \Mtfrac{-5}{19} \MDFPeriod$$
Hence, the system of linear equations has a unique solution and the solution set is
 ${\ML} = \{ \MPointThree{x = \Mtfrac{2}{19}}{
y = - \Mtfrac{5}{19}}{ z = \Mtfrac{2}{19}} \}$.\\
However, other approaches are equally possible.}
\item{By adding the first and the third equation the variable $y$ is eliminated:
$$(2 x - y + 5 z) + (- 4 x + y - 3 z) = 1 + (- 1) \Leftrightarrow - 2 x + 2 z = 0 \MDFPeriod$$
Multiplying the last equation by $(- 4)$ results in
$$8 x - 8 z = 0 \MDFPSpace ,$$
and adding the second equation ($11 x  + 8 z = 2$) to this result the variable $z$ is eliminated:
$$(8 x - 8 z) + (11 x + 8 z) = 0 + 2 \Leftrightarrow 19 x = 2 \Leftrightarrow x = \Mtfrac{2}{19} \MDFPeriod$$
As in the first part of this exercise, the variables $z$ and $y$ can be determined subsequently; 
of course, the solution set is again ${\ML} = \{ \MPointThree{x = \Mtfrac{2}{19}}{y = - \Mtfrac{5}{19}}{z = \Mtfrac{2}{19}} \}$.\\
However, other approaches are equally possible.}
\end{MExerciseItems}
\end{MHint}
\end{MExercise}

\begin{MExercise}
Consider the following circuit:
\begin{center}
\MTikzAuto{%
\begin{tikzpicture}[x=0.75cm, y=0.75cm] 
%Koordinatensystem
\node (xMAX) at (8.6,0){};
\node (yMAX) at (0,5.0){};
\draw[->,color=black] (0.0,0.0) -- (0.0,2.5);
\draw[color=black] (0.0,2.5) -- (0.0,5.0) -- (7.3,5.0);
\draw[color=black] (0.0,0.0) -- (1.2,0.0);
\draw[color=black] (1.2,-0.8) -- (1.2,0.8);
\fill[color=black, rounded corners=1.0pt] (1.5,-0.4) rectangle (1.6,0.4);
\draw[->,color=black] (1.6,0.0) -- (8.6,0.0) -- (8.6,5.0) -- (7.3,5.0);
\draw[->,color=black] (6.0,0.0) -- (6.0,4.25);
\draw[color=black] (6.0,4.25) -- (6.0,5.0);
%Achsenbeschriftung
%%\draw (xMAX) node[anchor=south] {$x$};
%%\draw (yMAX) node[anchor=west] {$y$};
%Beschriftung und Graphen
%%\clip(-8.2,-0.6) rectangle (3.4,3.4);
\fill[black] (2.0,4.5) rectangle (4.0,5.5);
\fill[black] (5.5,1.5) rectangle (6.5,3.5);
\fill[black] (8.1,1.5) rectangle (9.1,3.5);
\draw[color=black] (0.0,2.5) node[anchor=east] {$I_1$};
\draw[color=black] (6.0,4.25) node[anchor=west] {$I_2$};
\draw[color=black] (7.3,5.0) node[anchor=south] {$I_3$};
\draw[color=black] (3.0,4.5) node[anchor=north] {$R_1$};
\draw[color=black] (5.5,2.5) node[anchor=east] {$R_2$};
\draw[color=black] (9.1,2.5) node[anchor=west] {$R_3$};
\draw[color=black] (1.4,0.6) node[anchor=south west] {$U$};
\end{tikzpicture}
}%
%%\MUGraphicsSolo{schaltung.png}{scale=1.1}{width:450px}
\end{center}
It consists of a source providing a voltage of $U = \MZahl{5}{5}\MEinheit{V}$ 
and three resistors $R_1 = 1\MEinheit{}\MOhm$, $R_2 = 2\MEinheit{}\MOhm$, 
and $R_3 = 3\MEinheit{}\MOhm$.
Find the currents $I_1$, $I_2$, and $I_3$ in the loops.

\textbf{Hints:} The relations between voltages, resistances,
and currents in such circuits are described by the so called \textbf{Kirchhoff's rules}
which in this example result in the following three equations:
\begin{eqnarray*}
I_1 - I_2 - I_3 & = & 0\quad:\quad\mbox{equation}\MBlank (1) \MDFPSpace, \\ R_1 I_1 + R_2 I_2 & = & U\quad:\quad\mbox{equation}\MBlank (2)\MDFPSpace,  \\ R_2 I_2 - R_3 I_3 & = & 0\quad:\quad\mbox{equation}\MBlank (3) \MDFPeriod
\end{eqnarray*}
Additionally, the relation between the physical units Volt ($\MEinheit[]{V}$) (voltage),
Amp\`ere ($\MEinheit[]{A}$) (current) und Ohm ($\MOhm$) 
(resistance) is used: $1\MEinheit{}\MOhm = (1\MEinheit{V}) / (1\MEinheit{A})$.

\begin{MHint}{Solution}
For example, equation~($1$) is solved for $I_1$
$$I_1 = I_2 + I_3 \MDFPSpace ,$$
which is substituted into equation~($2$):
$$R_1 (I_2 + I_3) + R_2 I_2 = U \Leftrightarrow (R_1 + R_2) I_2 + R_1 I_3 = U \; : \MBlank\mbox{equation}\MBlank (2') \MDFPeriod$$
This last equation~($2'$) and equation~($3$) only depend on
the variables $I_2$ and $I_3$. They form a system of two linear equations in two variables, 
which is now solved: e.g.\ equation~($3$) can be solved for~$I_3$:
$$I_3 = \Mdfrac{R_2}{R_3} I_2 \; : \MBlank\mbox{equation}\MBlank (3') \MDFPSpace ,$$
which is then substituted into equation~$(2')$:
\begin{eqnarray*}
& & (R_1 + R_2) I_2 + R_1 \cdot \Mdfrac{R_2}{R_3} I_2 = U \\[.5ex]
& \Leftrightarrow & \left( R_1 + R_2 + R_1 \cdot \Mdfrac{R_2}{R_3} \right) I_2 = U \\[.5ex]
& \Leftrightarrow & \left( \Mdfrac{R_1 R_3}{R_3} + \Mdfrac{R_2 R_3}{R_3} + \Mdfrac{R_1 R_2}{R_3} \right) I_2 = U \\[.5ex]
& \Leftrightarrow & \Mdfrac{R_1 R_2 + R_1 R_3 + R_2 R_3}{R_3} I_2 = U \\[.5ex]
& \Leftrightarrow & I_2 = \Mdfrac{R_3}{R_1 R_2 + R_1 R_3 + R_2 R_3} U \MDFPeriod
\end{eqnarray*}
Hence, for $I_3$ it follows, using equation~$(3')$,
$$I_3 = \Mdfrac{R_2}{R_3} \cdot \Mdfrac{R_3}{R_1 R_2 + R_1 R_3 + R_2 R_3} U = \Mdfrac{R_2}{R_1 R_2 + R_1 R_3 + R_2 R_3} U$$
and for $I_1$
\begin{eqnarray*}
  I_1 = I_2 + I_3 &=& \Mdfrac{R_3}{R_1 R_2 + R_1 R_3 + R_2 R_3} \cdot U + \Mdfrac{R_2}{R_1 R_2 + R_1 R_3 + R_2 R_3} \cdot U \\
  &=& \Mdfrac{R_2 + R_3}{R_1 R_2 + R_1 R_3 + R_2 R_3} \cdot U \MDFPeriod
\end{eqnarray*}
Now, the values of the resistances of the resistors ($R_1 = 1\MEinheit{}\MOhm$, 
$R_2 = 2\MEinheit{}\MOhm$, and $R_3 = 3\MEinheit{}\MOhm$) and the voltage 
($U = \MZahl{5}{5}\MEinheit{V}$) can be inserted. Using
$1\MEinheit{V} = 1\MEinheit{}\MOhm \cdot 1\MEinheit{A}$ this results for $I_1$ in:
$$I_1 = \Mdfrac{(2 + 3)\MEinheit{}\MOhm}{(1 \cdot 2 + 1 \cdot 3 + 2 \cdot 3)\MEinheit{}\MOhm^2} \cdot \MZahl{5}{5} \cdot \MOhm \cdot \MEinheit[]{A}
= \MZahl{2}{5}\MEinheit{A} \MDFPeriod$$
Likewise, for $I_2$ and $I_3$ this results in: 
$I_2 = \MZahl{1}{5}\MEinheit{A}$ and $I_3 = 1\MEinheit{A}.$\\
However, other approaches are equally possible.
\end{MHint}
\end{MExercise}

\begin{MExercise}
Find the solution set of the following system of equations using the addition method:
\begin{eqnarray*}
x + 2 z & = & 5 \MDFPSpace, \\ 3 x + y - 2 z & = & - 1  \MDFPSpace,\\ - x - 2 y + 4 z & = & 7 \MDFPeriod
\end{eqnarray*}

\begin{MHint}{Solution}
Since the first equation ($x + 2 z = 5$) does not depend on the variable $y$ at all, it is reasonable
to eliminate~$y$ from the second and third equation as well. For this, the second equation
($3 x + y - 2 z = - 1$) is multiplied by $2$ and added to the third equation ($- x - 2 y + 4 z = 7$):
\begin{eqnarray*}
& & (- x - 2 y + 4 z) + 2 \cdot (3 x + y - 2 z) = 7 + 2 \cdot (- 1) \\
& \Leftrightarrow & - x - 2 y + 4 z + 6 x + 2 y - 4 z = 7 - 2 \\
& \Leftrightarrow & 5 x = 5 \\
& \Leftrightarrow & x = 1 \MDFPeriod
\end{eqnarray*}
At the same time the variable $z$ is also eliminated. Inserting
the value for $x$ in the first equations results in
$$1 + 2 z = 5 \Leftrightarrow 2 z = 4 \Leftrightarrow z = 2 \MDFPeriod$$
Using the second equation the value of $y$ is
$$3 \cdot 1 + y - 2 \cdot 2 = - 1 \Leftrightarrow 3 + y - 4 = - 1 \Leftrightarrow y = 0 \MDFPeriod$$
Hence, the solution set is ${\ML} = \{ \MPointThree{x = 1}{y = 0}{z = 2} \}$.
\end{MHint}
\end{MExercise}
\end{MExercises}



\MSubsection{More general Systems}
\MLabel{M04_freier_Parameter}

\begin{MIntro}
\MDeclareSiteUXID{VBKM04_AllgemeineLGS_Intro}

At the end of this module two further details concerning systems of linear equations
will be discussed briefly.

Firstly, in systems of linear equations free parameters can occur. These are variable 
quantities -- so called tuning parameters -- which possibly affect the behaviour of the systems, in
particular, the solution sets strongly. Sometimes it is advantageous not to fix all coefficients
and right-hand sides of the equations, but to keep them variable to investigate what happens for 
different values. Sometimes not all coefficients and right-hand sides are known from the 
problem description. 

Secondly, in a system of linear equations either the number of linear equations or the number of 
variables can dominate.

\end{MIntro}

\begin{MXContent}{Systems with a Free Parameter}{Systems with a Free Parameter}{STD}
\MDeclareSiteUXID{VBKM04_FreieParameter}

We start with an example that is very easy but demonstrates the essential point
concerning free parameters in systems of linear equations.

\begin{MExample}
Find the solution set of the following system of linear equations
$$\begin{array}{rcrcl} \mbox{equation}\MBlank (1) : & & x - 2 y & = & 3 \MDFPSpace,\\
\MBlank\mbox{equation}\MBlank (2) : & & - 2 x + 4 y & = & \alpha \end{array}$$
depending on the parameter $\alpha$.

Multiplying equation~$(1)$ by the factor $2$ and adding equation~$(2)$ results in
$$2 \cdot (x - 2 y) + (- 2 x + 4 y) = 2 \cdot 3 + \alpha \Leftrightarrow 2 x - 4 y - 2 x + 4 y = 6 + \alpha
\Leftrightarrow 0 = 6 + \alpha \MDFPeriod$$

Now, two cases have to be distinguished:

\textbf{Case A} ($\alpha \neq - 6$): If the given free parameter $\alpha$ is not equal to $- 6$, the 
last equation is a contraction. In this case the system of linear equations has 
\textbf{no solution}, i.e.\ ${\ML} = \MEmptyset$.

\textbf{Case B} ($\alpha = - 6$): If the given free parameter $\alpha$ is equal to $- 6$, the last 
equation is always satisfied ($0 = 0$). In fact, the two initial equations in this case are multiples of
each other such that only one of them indeed carries information. Accordingly, \textbf{the solution set 
is infinite}: ${\ML}
= \{ \MPointTwo{x = 3 + 2 t}{y = t}  \MCondSetSep t \in \R \}$.
\end{MExample}

The example shows that the solution set may strongly depend on the value of the free
parameter. 

Such a free parameter can occur not only in one of the right-hand sides of the system of
linear equations, but also in the left-hand sides, multiple times or in a function both on the 
left-hand sides and on the right-hand sides. Also several parameters can occur in a 
system at the same time.

Let us now consider a slightly more complex example.

\begin{MExample}
Find the solution set of the following system of linear equations
\begin{eqnarray*}
x + y + \alpha z & = & 1 \MDFPSpace, \\ x + \alpha y + z & = & 1 \MDFPSpace, \\ \alpha x + y + z & = & 1
\end{eqnarray*}
depending on the value of the parameter $\alpha$.

For this, e.g.\ the first equation is solved for the variable $x$ 
$$x = 1 - y - \alpha z \; : \MBlank\mbox{equation}\MBlank (1') \MDFPSpace ,$$
and the result is \textbf{substituted} into the second and third equation:
$$\begin{array}{rclcrclcl} (1 - y - \alpha z) + \alpha y + z & = & 1 & \Leftrightarrow &
-(1 - \alpha) y + (1 - \alpha) z & = & 0 & & : \MBlank\mbox{equation}\MBlank (2') \MDFPSpace, \\
\alpha (1 - y - \alpha z) + y + z & = & 1 & \Leftrightarrow &
(1 - \alpha) y + (1 - \alpha^2) z & = & 1 - \alpha & & : \MBlank\mbox{equation}\MBlank (3') \MDFPeriod \end{array}$$
This results in a system of \textbf{two} linear equations in \textbf{two} variables
$y$ and $z$. It can immediately be seen that for the value $\alpha = 1$ something happens.
Hence, a case analysis is required.

\textbf{Case 1} ($\alpha = 1$): 
In this case the two equations~$(2')$ and $(3')$ are satisfied identically ($0 = 0$) and provide
no further information. The only relation between the variables $x, y$, and $z$ is
equation~$(1')$ or equation~$(1)$, respectively, that reads for $\alpha = 1$ as follows:
$$x + y + z = 1 \; : \MBlank\mbox{equation}\MBlank (\hat{1}) \MDFPeriod$$
Hence, the solution set has an infinite number of elements. The set can be described using 
\textbf{two free parameters}, e.g.\
$${\ML} = \{ \MPointThree{s}{t}{1 - s- t} \MCondSetSep  s, t \in \R \} \MDFPeriod $$
Geometrically, the solution set is exactly the plane described by equation~$(\hat{1})$.

\textbf{Case 2:} ($\alpha \neq 1$): In this case both equation~$(2')$ and equation~$(3')$
can be divided by $(1 - \alpha)$. Using the third binomial formula ($(1 - \alpha^2)
= (1 - \alpha)(1 + \alpha)$) results in 
$$\begin{array}{rclcl} - y + z & = & 0 & & : \MBlank\mbox{equation}\MBlank ({2'}') \MDFPSpace, \\
y + (1 + \alpha) z & = & 1 & & : \MBlank\mbox{equation}\MBlank ({3'}') \MDFPeriod \end{array}$$
According to equation~$({2'}')$ one has $y = z$. This is substituted into equation~$({3'}')$:
$$z + (1 + \alpha) z = 1 \Leftrightarrow (2 + \alpha) z = 1 \; : \MBlank\mbox{equation}\MBlank (\star)\MDFPeriod $$
Again, one has to take care and an additional case analysis is required since $\alpha = - 2$
and $\alpha \neq - 2$ have different consequences:

\textbf{Case 2a} ($\alpha = - 2$): In this (sub)case equation~$(\star)$ reads $0 = 1$.
This is a contradiction and the initial system of equations has no solution, i.e.\
${\ML}= \MEmptyset$.

\textbf{Case 2b:} $\alpha \neq - 2$: In this (sub)case the last equation can be solved 
easily for $z$:
$$z = \Mdfrac{1}{2 + \alpha} \MDFPeriod $$
So, $y$ ($y = z$) and $x$ ($x = 1 - y - \alpha z$) are determined. The initial system
of linear equations has a unique solution, namely 
${\ML} = \{ \MPointThree{x = \Mtfrac{1}{2 + \alpha}}{ y = \Mtfrac{1}{2 + \alpha}}{
z = \Mtfrac{1}{2 + \alpha} } \}$.
\end{MExample}

The previous example indicates the relevance of a clear and precise case analysis.  
Depending on the value of $\alpha$ ($\alpha = 1$, $\alpha = - 2$, or $\alpha \in 
\R \setminus \{- 2\MElSetSep 1 \}$) the solution set is completely different! 
In the first case it is an infinite set, in the second the empty set, and in the 
third the solution set consists of exactly one element!
 
By the way, the exceptionality of the case $\alpha = 1$ could have seen directly
from the initial system of linear equations: For $\alpha = 1$, the same equation
occurs three times, namely $x + y + z = 1$, i.e.\ two of the three equations in the initial system 
do not contribute any information and thus, they are unnecessary. 
For $\alpha = 1$, only the equation $x + y + z = 1$ relates the three variables.
\end{MXContent}


\begin{MExercises}
\MDeclareSiteUXID{VBKM04_FreieParameter_Exercises}
\begin{MExercise}
Find the $y$-intercept $b$ and the slope $m$ of a line described 
by the equation $y=m x+b$ which is defined by two points. The 
first point at  $x_1=\alpha$ lies on the line described by the 
equation $y_1(x)=-2(1+x)$. The second point at $x_2=\beta$ lies on
the line described by the equation $y_2(x)=x-1$. The following figure
illustrates the situation.

\begin{center}
\MTikzAuto{%
\begin{tikzpicture}[x=1.0cm, y=1.0cm] 
%Koordinatensystem
\node (xMAX) at (5.0,0){};
\node (yMAX) at (0,5.0){};
\draw[->,color=black] (-4.5,0) -- (xMAX);
\foreach \x in {-4, -3, -2, -1, 0, 1, 2, 3, 4}
\draw[shift={(\x,0)},color=black] (0pt,2pt) -- (0pt,-2pt) node[above right] {\scriptsize $\x$};
\draw[->,color=black] (0,-2.5) -- (yMAX);
\foreach \y in {-2, -1, 1, 2, 3, 4}
\draw[shift={(0,\y)},color=black] (2pt,0pt) -- (-2pt,0pt) node[above right] {\scriptsize $\y$};
%Achsenbeschriftung
\draw (xMAX) node[anchor=south east] {$x$};
\draw (yMAX) node[anchor=north east] {$y$};
%Beschriftung und Graphen
\clip(-4.5,-2.5) rectangle (4.5,4.5);
\draw[help lines, gray, dashed] (-5,-5) grid (5,5); % was dotted
%\fill[color=black] (1,1) circle (2.0pt);
%%\fill[red!50!white, opacity=0.50] (1,1) rectangle (6,6);
\def\alx{-1.5}
\def\bex{2.5}
%%\pgfmathparse{(\bex*\bex-\alx)/(\bex-\alx)}\let\mrsl=\pgfmathresult
\pgfmathparse{(-2-2*\alx-\bex+1)/(\alx-\bex)}\let\mrsl=\pgfmathresult
\pgfmathparse{-1+\bex*(1-\mrsl)}\let\brsl=\pgfmathresult
%%\draw[color=red, line width=1pt] (-4.0,-4.0) -- (4.0,4.0);
\draw[smooth,samples=3,domain=-4:4, line width=1pt,color=blue] plot(\x,{-1+\x});
%%\draw[smooth,samples=17,domain=-4:4, line width=1pt,color=blue] plot(\x,{\x*\x});
\draw[smooth,samples=3,domain=-4:4, line width=1pt,color=red] plot(\x,{-2*(1+\x)});
\draw[smooth,samples=3,domain=-4.5:4.5, line width=1pt,color=black] plot(\x,{\mrsl*\x+\brsl});
\draw[color=red, very thick] (\alx,2pt) -- (\alx,-2pt) node[anchor=north] {$\alpha$};
\draw[color=red, very thick] (\alx,{-2*(1+\alx)}) circle (1.5pt);
\draw[color=blue, very thick] (\bex,2pt) -- (\bex,-2pt) node[anchor=north] {$\beta$};
\draw[color=blue, very thick] (\bex,{-1+\bex}) circle (1.5pt);
\draw[color=red] (-4.0,-0.8) node[anchor=west] {$y_1(x) = -2(1+x)$};
\draw[color=blue] (-4.0,-1.35) node[anchor=west] {$y_2(x) = -1+x$};
\draw[color=black] (-4.0,-1.8) node[anchor=west] {$y = m x+b$};
\end{tikzpicture}
}
\end{center}
\begin{MExerciseItems}
\item{Find the system of equations for the parameters $b$ and $m$.
\par
The first equation reads
\MEquationItem{$m\alpha+b$}{\MLSimplifyQuestion{15}{-2*(1+alpha)}{10}{alpha}{10}{2}{LGSPar1Gl1}};\\ % 513
the second equation reads \MEquationItem{$m\beta+b$}{\MLSimplifyQuestion{15}{-1+beta}{10}{beta}{10}{2}{LGSPar1Gl2}}. \\ % 513
\MInputHint{The constants $\alpha$ and $\beta$ have to remain in the solution; 
for these \texttt{alpha} and \texttt{beta} can be entered.}
}
\item{Solve this system of equations for $b$ and $m$. For which values of 
$\alpha$ and $\beta$ does the system have a unique solution, no solution,
or an infinite number of solutions?
\par
For $\alpha=-2$ and $\beta=2$ one obtains, for example, the 
solution \MEquationItem{$m$}{\MLParsedQuestion{6}{-1/4}{4}{LGSPar1B1m}}
and \MEquationItem{$b$}{\MLParsedQuestion{6}{5/2}{4}{LGSPar1B1b}},
the case $\alpha=2$ and $\beta=-2$ results in the solution
\MEquationItem{$m$}{\MLParsedQuestion{6}{-3/4}{4}{LGSPar1B2m}}
and \MEquationItem{$b$}{\MLParsedQuestion{6}{-9/2}{4}{LGSPar1B2b}}.
\par
The LS has an infinite number of solutions if 
\MEquationItem{$\alpha$}{\MLParsedQuestion{6}{-1/3}{4}{LGSPar1InfAlp}}
and \MEquationItem{$\beta$}{\MLParsedQuestion{6}{-1/3}{4}{LGSPar1InfBet}}.\\
The corresponding solutions can be parameterised by $m=r$ and
\MEquationItem{$b$}{\MLSimplifyQuestion{10}{r/3-4/3}{10}{r}{10}{2}{LGSPar1Infb}}, $r\in\R$. % 513
}
\item{What is the graphical interpretation of the last two
cases, i.e.\ no solution and an infinite number of solutions?}
\end{MExerciseItems}
%%
\begin{MHint}{Solution}
\begin{MExerciseItems}
\item{%
>From the condition 
\[
  y(x_1=\alpha)=y_1(x_1=\alpha) \MBlank\mbox{and}\MBlank
  y(x_2=\beta)=y_2(x_2=\beta)
\]
the LS for $m$ and $b$ results in:
\[
  \begin{array}{rcll}
     m\alpha +b &=& -2(1+\alpha) & \mbox{equation}\MBlank (1)\MDFPSpace, \\[0.5ex]
     m\beta +b &=& -1+\beta & \mbox{equation}\MBlank (2) \MDFPeriod
  \end{array}
\]
}
\item{%
If equation~$(2)$ is replaced by the difference of equation~$(2)$
and equation~$(1)$, one obtains
\[
  \begin{array}{rcll}
     m\alpha +b &=& -2(1+\alpha) & \mbox{equation}\MBlank (1)\MDFPSpace, \\[0.5ex]
     m(\beta-\alpha) &=& 1+2\alpha+\beta & \mbox{equation}\MBlank (2') \MDFPeriod
  \end{array}
\]
With this, the LS has already triangular form. 
%
\medskip\par\noindent
%
A. For $\beta-\alpha \ne 0 \MBlank \Leftrightarrow \MBlank
\beta\ne\alpha$ equation $(2')$ can be divided by $(\beta-\alpha)$,
which for the slope $m$ results in
\[
  m=\frac{1+2\alpha+\beta}{\beta-\alpha}\MDFPeriod
\]
For example, this result can be substituted into equation~$(1)$ resolved 
for $b$:
\[
  b=-2(1+\alpha)-m\alpha=
  -2(1+\alpha)-\frac{1+2\alpha+\beta}{\beta-\alpha}\alpha
%%  =-\frac{\alpha(1+2\alpha+\beta)+2(1+\alpha)(\beta-\alpha)}{\beta-\alpha}
  \MDFPeriod
\]
The obtained values of $m$ and $b$ represent a unique solution of the considered LS.
\par
For $\alpha=-2$, $\beta=2$, the solution is $m=-1/4, b=5/2$,
and for $\alpha=2$, $\beta=-2$ the solution is $m=-3/4, b=-9/2$.
%
\medskip\par\noindent
%
B. In this case $\beta-\alpha = 0 \MBlank \Leftrightarrow \MBlank
\beta=\alpha$. So, the left-hand side of equation~$(2')$ is zero. 
Thus, the following additional case analysis is required:
%
\smallskip\par\noindent
%
Ba. If the right-hand side of equation is non-zero, i.e.\ 
$(2')$ $1+2\alpha+\beta\ne 0$,
the LS has no solution, i.e.\ $\ML=\MEmptyset$. For
$\beta=\alpha$ one obtains $\beta=\alpha\ne -\frac{1}{3}$.
%
\smallskip\par\noindent
Bb. If the right-hand side of equation~$(2')$ is zero, the LS only
consists of equation~$(1)$ with $\beta=\alpha=-\frac{1}{3}$. In this
case $m=\lambda$, $\lambda\in\R$ can be chosen arbitrary, and $b$ results
directly from equation $(1)$: $b =\frac{1}{3}\lambda-\frac{4}{3}$. 
Thus, the solution set of the LS is 
\[
  \ML = \left\{\MPointTwoAS{m=\lambda}{b=\frac{1}{3}\lambda-\frac{4}{3}} 
  \MCondSetSep \lambda\in\R\right\} \MDFPeriod
\]
}
\item{%
The right-hand sides of equation~$(1)$ and equation~$(2)$ contain
the $y$-values of the initial lines at $x_1=\alpha$ and $x_2=\beta$, i.e.
$y_1(\alpha)$ and $y_2(\beta)$, such that the right-hand side of 
equation~$(2')$ is the difference $y_2(\beta)-y_1(\alpha)$. In case B, i.e.
for $\beta=\alpha$, one then obtains $y_2(\alpha)-y_1(\alpha)$. So, the LS
has no solution (case Ba) if $y_2(\alpha)-y_1(\alpha)\ne 0$
$\Leftrightarrow$ $y_2(\alpha)\ne y_1(\alpha)$, i.e.\ the two lines
have different $y$-values. On the other hand an infinite number of solutions
exist (case Bb) if the two initial equations intersect at $x= \alpha$.
}
\end{MExerciseItems}
\end{MHint}
\end{MExercise}

\begin{MExercise}
Find the solution set of the following LS depending on the parameter 
$t\in\R$.
\[
  \begin{array}{rcll}
      x -y +t z&=& t & 
	  \mbox{equation}\MBlank (1)\MDFPSpace, \\[0.5ex]
      t x + (1-t)y +(1+t^2)z&=& -1+t & 
	  \mbox{equation}\MBlank (2)\MDFPSpace, \\[0.5ex]
      (1-t)x+(-2+t)y+(-1+t-t^2)z&=& t^2 & 
	  \mbox{equation}\MBlank (3) \MDFPeriod
  \end{array}
\]
\par
The LS only has solutions for the following values of the parameter:
$t\in\mbox{}$\MLParsedQuestion{12}{-1,1}{4}{LGSPar2PSolSet}. \\
\MInputHint{Set can be entered in the form \texttt{$\lbrace$a;b;c;$\ldots\rbrace$}. The empty set can be entered as $\lbrace\rbrace$.}
\par
For the smallest value of the parameter $t$ the solution is\\
\MEquationItem{$x$}{\MLSimplifyQuestion{10}{-4}{10}{r}{10}{513}{LGSPar2LVx}},
\MEquationItem{$y$}{\MLSimplifyQuestion{10}{-3-r}{10}{r}{10}{513}{LGSPar1LVy}},
$z=r$, $r\in\R$.\\
For the greatest value of the parameter $t$ the solution is\\
\MEquationItem{$x$}{\MLSimplifyQuestion{10}{-2*r}{10}{r}{10}{513}{LGSPar2HVx}},
\MEquationItem{$y$}{\MLSimplifyQuestion{10}{-1-r}{10}{r}{10}{513}{LGSPar1HVy}},
$z=r$, $r\in\R$.
\end{MExercise}

\begin{MHint}{Solution}
As a first step, equation~$(3)$ is replaced by the sum of 
equation~$(2)$ and equation~$(3)$:
\[
  \begin{array}{rcll}
      x -y +t z&=& t & 
	  \mbox{equation}\MBlank (1)\MDFPSpace, \\[0.5ex]
      t x + (1-t)y +(1+t^2)z&=& -1+t & 
	  \mbox{equation}\MBlank (2)\MDFPSpace, \\[0.5ex]
      x-y+t z&=& -1+t+t^2 & 
	  \mbox{equation}\MBlank (3') \MDFPeriod
  \end{array}
\]
Since the left-hand side of equation~$(1)$ and equation~$(3')$ coincide,
it is useful to replace equation~$(3')$ 
by the sum of the negative of equation~$(1)$ and equation~$(3')$:
\[
  \begin{array}{rcll}
      x -y +t z&=& t & 
	  \mbox{equation}\MBlank (1)\MDFPSpace, \\[0.5ex]
      t x + (1-t)y +(1+t^2)z&=& -1+t & 
	  \mbox{equation}\MBlank (2)\MDFPSpace, \\[0.5ex]
      0 &=& -1+t^2 & 
	  \mbox{equation}\MBlank ({3'}') \MDFPeriod
  \end{array}
\]
subsequently, replacing equation~$(2)$ by the sum of $(-t)$-times equation~$(1)$ and equation~$(2)$ results in a LS in triangular form:
\[
  \begin{array}{rcll}
      x -y +t z&=& t & 
	  \mbox{equation}\MBlank (1)\MDFPSpace, \\[0.5ex]
      y + z&=& -1+t-t^2 & 
	  \mbox{equation}\MBlank (2')\MDFPSpace, \\[0.5ex]
      0 &=& -1+t^2 & 
	  \mbox{equation}\MBlank ({3'}') \MDFPeriod
  \end{array}
\]
Finally, adding equation~$(2')$ to equation~$(1)$ results in
a further simplification,
\[
  \begin{array}{rcll}
      x +(1+t)z&=& -1+2t-t^2 & 
	  \mbox{equation}\MBlank (1')\MDFPSpace, \\[0.5ex]
      y + z&=& -1+t-t^2 & 
	  \mbox{equation}\MBlank (2')\MDFPSpace, \\[0.5ex]
      0 &=& -1+t^2 & 
	  \mbox{equation}\MBlank ({3'}') \MDFPeriod
  \end{array}
\]
This equivalent LS only has a solution if equation~$({3'}')$  is satisfied
as well, i.e.\ $0=-1+t^2$ $\Leftrightarrow$ $t^2=1$. In all other cases the 
solution set is the empty set, i.e.\
\[
  \ML=\MEmptyset \MBlank\mbox{für}\MBlank t\in\R\MSetminus\{-1\MElSetSep 1\}
  \MDFPeriod
\]
If equation~$({3'}')$ is now satisfied, i.e.\ $t=1$ or $t=-1$, 
then $z$ can be chosen arbitrarily, i.e.\ $z=\lambda$, $\lambda\in\R$.
Solving equation~$(1')$ and equation~$(2')$ for $x$ and $y$, respectively, 
for $t=1$ results in the expressions $x=-2z$ and $y=-1-z$, so the solution set is
\[
  \ML = \{\MPointThree{x=-2\lambda}{y=-1-\lambda}{z=\lambda} 
  \MCondSetSep \lambda\in\R\} \MBlank\mbox{for}\MBlank t=1
  \MDFPeriod
\]
Accordingly, for $t=-1$ one obtains $x=-4$ and $y=-3-z$, so the solution set is
\[
  \ML = \{\MPointThree{x=-4}{y=-3-\lambda}{z=\lambda} 
  \MCondSetSep \lambda\in\R\} \MBlank\mbox{for}\MBlank t=-1
  \MDFPeriod
\]
\end{MHint}
\end{MExercises}


\MSubsection{Final Test}
\MLabel{M04_Ausgangstest}

\begin{MTest}{Final Test Modul 4}
\MDeclareSiteUXID{VBKM04_Abschlusstest}

\begin{MExercise}
Find the solution set of the following system of linear equations:
\begin{eqnarray*}
- x + 2 y & = & - 5 \MDFPSpace,  \\ 3 x + y & = & 1 \MDFPeriod
\end{eqnarray*}
The solution set 
\begin{tabular}[t]{ll}
\MLCheckbox{0}{M04C13} & is empty,\\
\MLCheckbox{1}{M04C14} & contains exactly one element: $x =$ \MLParsedQuestion{5}{1}{5}{ATXY1} , $y =$ \MLParsedQuestion{5}{-2}{5}{ATXY2} ,\\
\MLCheckbox{0}{M04C15} & contains an infinite number of solution pairs $\MPointTwo{x}{y}$.
\end{tabular}
\end{MExercise}

\begin{MExercise}
Find the two-digit number such that its digit sum is 6 and exchanging the tens and 
the units digit results in a number which is 18 less.
% 
Answer: \MLParsedQuestion{5}{42}{5}{ATX42}.
\end{MExercise}

\begin{MExercise}
Find the value of the real parameter $\alpha$ for which the 
following system of linear equations 
\begin{eqnarray*}
2 x + y & = & 3 \MDFPSpace, \\ 4 x + 2 y & = & \alpha
\end{eqnarray*}
has an infinite number of solutions.

Answer: $\alpha = $ \MLParsedQuestion{5}{6}{5}{ATX}.
\end{MExercise}

\begin{MExercise}
The following figure shows two lines in two-dimensional space.
\begin{center}
\MTikzAuto{%
\begin{tikzpicture}[x=1.0cm, y=1.0cm] 
%Koordinatensystem
\node (xMAX) at (5.0,0){};
\node (yMAX) at (0,4.7){};
\clip(-4.1,-2.8) rectangle (5.1,4.7);
\draw[help lines, gray, dashed, xstep=1.0, ystep=1.0] (-4,-3) grid (5,5);
\draw[->,color=black] (-4.0,0) -- (xMAX);
\foreach \x in {-3, -2, -1, 0, 1, 2, 3, 4}
\draw[shift={(\x,0)},color=black] (0pt,2pt) -- (0pt,-2pt) node[below right] {\scriptsize $\x$};
\draw[->,color=black] (0,-2.8) -- (yMAX);
\foreach \y in {-2, -1, 0, 1, 2, 3, 4}
\draw[shift={(0,\y)},color=black] (2pt,0pt) -- (-2pt,0pt) node[above left] {\scriptsize $\y$};
%Achsenbeschriftung
\draw (xMAX) node[anchor=south east] {$x$};
\draw (yMAX) node[anchor=north west] {$y$};
%Beschriftung und Graphen
%%\fill[red!50!white, opacity=0.50] (1,1) rectangle (6,6);
\draw[color=blue, thick] (-2.0,-3.0) -- (5.0,4.0);
\draw[color=blue, thick] (3.0,-4.0) -- (-2.0,6.0);
\fill[color=red] (1,0) circle (2.0pt);
\draw[color=black] (1,0) circle (2.0pt);
\draw[color=red] (0.60,0) node[anchor=south] {\scriptsize $P$};
\draw[color=blue] (3.0,2.0) node[anchor=north west] {\scriptsize line 1};
\draw[color=blue] (-0.5,3.0) node[anchor=north east] {\scriptsize line 2};
\end{tikzpicture}
}%
%%\MUGraphicsSolo{LGS_schneidende_Geraden.png}{width=0.5\linewidth}{width:450px}
\end{center}
Find the two equations describing the lines.

Line 1: $y = $ \MLFunctionQuestion{15}{-1+x}{2}{x}{5}{XAFG1},

Line 2: $y = $ \MLFunctionQuestion{15}{2-2*x}{2}{x}{5}{XAFG2}.

What is the number of solutions of the corresponding system 
of linear equations?

The system of linear equations has
\begin{tabular}[t]{ll}
\MLCheckbox{0}{M04C16} & no solution,\\
\MLCheckbox{1}{M04C17} & exactly one solution, or\\
\MLCheckbox{0}{M04C18} & an infinite number of solutions.
\end{tabular}
\end{MExercise}

\begin{MExercise}
Find the solution set of the following system of linear equations consisting
of three equations in three variables:
\begin{eqnarray*}
x + 2 z & = & 3 \MDFPSpace, \\ - x + y + z & = & 1 \MDFPSpace, \\ 2 y + 3 z & = & 5 \MDFPeriod
\end{eqnarray*}
The solution set
\begin{tabular}[t]{ll}
\MLCheckbox{0}{M04C19} & is empty,\\
\MLCheckbox{1}{M04C20} & contains exactly one solution: $x =$ \MLParsedQuestion{5}{1}{5}{ATXY4} , $y =$ \MLParsedQuestion{5}{1}{5}{ATXY5}
, $z = $ \MLParsedQuestion{5}{1}{5}{ATXY6} ,\\
\MLCheckbox{0}{M04C21} & contains an infinite number of solutions $\MPointThree{x}{y}{z}$.
\end{tabular}
\end{MExercise}

\end{MTest}


\newpage
\MPrintIndex

\end{document}
