% MINTMOD Version P0.1.0, needs to be consistent with preprocesser object in tex2x and MPragma-Version at the end of this file

% Parameter aus Konvertierungsprozess (PDF und HTML-Erzeugung wenn vom Konverter aus gestartet) werden hier eingefuegt, Preambleincludes werden am Schluss angehaengt

\newif\ifttm                % gesetzt falls Uebersetzung in HTML stattfindet, sonst uebersetzung in PDF

% Wahl der Notationsvariante ist im PDF immer std, in der HTML-Uebersetzung wird vom Konverter die Auswahl modifiziert
\newif\ifvariantstd
\newif\ifvariantunotation
\variantstdtrue % Diese Zeile wird vom Konverter erkannt und ggf. modifiziert, daher nicht veraendern!


\def\MOutputDVI{1}
\def\MOutputPDF{2}
\def\MOutputHTML{3}
\newcounter{MOutput}

\ifttm
\usepackage{german}
\usepackage{array}
\usepackage{amsmath}
\usepackage{amssymb}
\usepackage{amsthm}
\else
\documentclass[ngerman,oneside]{scrbook}
\usepackage{etex}
\usepackage[latin1]{inputenc}
\usepackage{textcomp}
\usepackage[ngerman]{babel}
\usepackage[pdftex]{color}
\usepackage{xcolor}
\usepackage{graphicx}
\usepackage[all]{xy}
\usepackage{fancyhdr}
\usepackage{verbatim}
\usepackage{array}
\usepackage{float}
\usepackage{makeidx}
\usepackage{amsmath}
\usepackage{amstext}
\usepackage{amssymb}
\usepackage{amsthm}
\usepackage[ngerman]{varioref}
\usepackage{framed}
\usepackage{supertabular}
\usepackage{longtable}
\usepackage{maxpage}
\usepackage{tikz}
\usepackage{tikzscale}
\usepackage{tikz-3dplot}
\usepackage{bibgerm}
\usepackage{chemarrow}
\usepackage{polynom}
%\usepackage{draftwatermark}
\usepackage{pdflscape}
\usetikzlibrary{calc}
\usetikzlibrary{through}
\usetikzlibrary{shapes.geometric}
\usetikzlibrary{arrows}
\usetikzlibrary{intersections}
\usetikzlibrary{decorations.pathmorphing}
\usetikzlibrary{external}
\usetikzlibrary{patterns}
\usetikzlibrary{fadings}
\usepackage[colorlinks=true,linkcolor=blue]{hyperref} 
\usepackage[all]{hypcap}
%\usepackage[colorlinks=true,linkcolor=blue,bookmarksopen=true]{hyperref} 
\usepackage{ifpdf}

\usepackage{movie15}

\setcounter{tocdepth}{2} % In Inhaltsverzeichnis bis subsection
\setcounter{secnumdepth}{3} % Nummeriert bis subsubsection

\setlength{\LTpost}{0pt} % Fuer longtable
\setlength{\parindent}{0pt}
\setlength{\parskip}{8pt}
%\setlength{\parskip}{9pt plus 2pt minus 1pt}
\setlength{\abovecaptionskip}{-0.25ex}
\setlength{\belowcaptionskip}{-0.25ex}
\fi

\ifttm
\newcommand{\MDebugMessage}[1]{\special{html:<!-- debugprint;;}#1\special{html:; //-->}}
\else
%\newcommand{\MDebugMessage}[1]{\immediate\write\mintlog{#1}}
\newcommand{\MDebugMessage}[1]{}
\fi

\def\MPageHeaderDef{%
\pagestyle{fancy}%
\fancyhead[r]{(C) VE\&MINT-Projekt}
\fancyfoot[c]{\thepage\\--- CCL BY-SA 3.0 ---}
}


\ifttm%
\def\MRelax{}%
\else%
\def\MRelax{\relax}%
\fi%

%--------------------------- Uebernahme von speziellen XML-Versionen einiger LaTeX-Kommandos aus xmlbefehle.tex vom alten Kasseler Konverter ---------------

\newcommand{\MSep}{\left\|{\phantom{\frac1g}}\right.}

\newcommand{\ML}{L}

\newcommand{\MGGT}{\mathrm{ggT}}


\ifttm
% Verhindert dass die subsection-nummer doppelt in der toccaption auftaucht (sollte ggf. in toccaption gefixt werden so dass diese Ueberschreibung nicht notwendig ist)
\renewcommand{\thesubsection}{}
% Kommandos die ttm nicht kennt
\newcommand{\binomial}[2]{{#1 \choose #2}} %  Binomialkoeffizienten
\newcommand{\eur}{\begin{html}&euro;\end{html}}
\newcommand{\square}{\begin{html}&square;\end{html}}
\newcommand{\glqq}{"'}  \newcommand{\grqq}{"'}
\newcommand{\nRightarrow}{\special{html: &nrArr; }}
\newcommand{\nmid}{\special{html: &nmid; }}
\newcommand{\nparallel}{\begin{html}&nparallel;\end{html}}
\newcommand{\mapstoo}{\begin{html}<mo>&map;</mo>\end{html}}

% Schnitt und Vereinigungssymbole von Mengen haben zu kleine Abstaende; korrigiert:
\newcommand{\ccup}{\,\!\cup\,\!}
\newcommand{\ccap}{\,\!\cap\,\!}


% Umsetzung von mathbb im HTML
\renewcommand{\mathbb}[1]{\begin{html}<mo>&#1opf;</mo>\end{html}}
\fi

%---------------------- Strukturierung ----------------------------------------------------------------------------------------------------------------------

%---------------------- Kapselung des sectioning findet auf drei Ebenen statt:
% 1. Die LateX-Befehl
% 2. Die D-Versionen der Befehle, die nur die Grade der Abschnitte umhaengen falls notwendig
% 3. Die M-Versionen der Befehle, die zusaetzliche Formatierungen vornehmen, Skripten starten und das HTML codieren
% Im Modultext duerfen nur die M-Befehle verwendet werden!

\ifttm

  \def\Dsubsubsubsection#1{\subsubsubsection{#1}}
  \def\Dsubsubsection#1{\subsubsection{#1}\addtocounter{subsubsection}{1}} % ttm-Fehler korrigieren
  \def\Dsubsection#1{\subsection{#1}}
  \def\Dsection#1{\section{#1}} % Im HTML wird nur der Sektionstitel gegeben
  \def\Dchapter#1{\chapter{#1}}
  \def\Dsubsubsubsectionx#1{\subsubsubsection*{#1}}
  \def\Dsubsubsectionx#1{\subsubsection*{#1}}
  \def\Dsubsectionx#1{\subsection*{#1}}
  \def\Dsectionx#1{\section*{#1}}
  \def\Dchapterx#1{\chapter*{#1}}

\else

  \def\Dsubsubsubsection#1{\subsubsection{#1}}
  \def\Dsubsubsection#1{\subsection{#1}}
  \def\Dsubsection#1{\section{#1}}
  \def\Dsection#1{\chapter{#1}}
  \def\Dchapter#1{\title{#1}}
  \def\Dsubsubsubsectionx#1{\subsubsection*{#1}}
  \def\Dsubsubsectionx#1{\subsection*{#1}}
  \def\Dsubsectionx#1{\section*{#1}}
  \def\Dsectionx#1{\chapter*{#1}}

\fi

\newcommand{\MStdPoints}{4}
\newcommand{\MSetPoints}[1]{\renewcommand{\MStdPoints}{#1}}

% Befehl zum Abbruch der Erstellung (nur PDF)
\newcommand{\MAbort}[1]{\err{#1}}

% Prefix vor Dateieinbindungen, wird in der Baumdatei mit \renewcommand modifiziert
% und auf das Verzeichnisprefix gesetzt, in dem das gerade bearbeitete tex-Dokument liegt.
% Im HTML wird es auf das Verzeichnis der HTML-Datei gesetzt.
% Das Prefix muss mit / enden !
\newcommand{\MDPrefix}{.}

% MRegisterFile notiert eine Datei zur Einbindung in den HTML-Baum. Grafiken mit MGraphics werden automatisch eingebunden.
% Mit MLastFile erhaelt man eine Markierung fuer die zuletzt registrierte Datei.
% Diese Markierung wird im postprocessing durch den physikalischen Dateinamen ersetzt, aber nur den Namen (d.h. \MMaterial gehoert noch davor, vgl Definition von MGraphics)
% Parameter: Pfad/Name der Datei bzw. des Ordners, relativ zur Position des Modul-Tex-Dokuments.
\ifttm
\newcommand{\MRegisterFile}[1]{\addtocounter{MFileNumber}{1}\special{html:<!-- registerfile;;}#1\special{html:;;}\MDPrefix\special{html:;;}\arabic{MFileNumber}\special{html:; //-->}}
\else
\newcommand{\MRegisterFile}[1]{\addtocounter{MFileNumber}{1}}
\fi

% Testen welcher Uebersetzer hier am Werk ist

\ifttm
\setcounter{MOutput}{3}
\else
\ifx\pdfoutput\undefined
  \pdffalse
  \setcounter{MOutput}{\MOutputDVI}
  \message{Verarbeitung mit latex, Ausgabe in dvi.}
\else
  \setcounter{MOutput}{\MOutputPDF}
  \message{Verarbeitung mit pdflatex, Ausgabe in pdf.}
  \ifnum \pdfoutput=0
    \pdffalse
  \setcounter{MOutput}{\MOutputDVI}
  \message{Verarbeitung mit pdflatex, Ausgabe in dvi.}
  \else
    \ifnum\pdfoutput=1
    \pdftrue
  \setcounter{MOutput}{\MOutputPDF}
  \message{Verarbeitung mit pdflatex, Ausgabe in pdf.}
    \fi
  \fi
\fi
\fi

\ifnum\value{MOutput}=\MOutputPDF
\DeclareGraphicsExtensions{.pdf,.png,.jpg}
\fi

\ifnum\value{MOutput}=\MOutputDVI
\DeclareGraphicsExtensions{.eps,.png,.jpg}
\fi

\ifnum\value{MOutput}=\MOutputHTML
% Wird vom Konverter leider nicht erkannt und daher in split.pm hardcodiert!
\DeclareGraphicsExtensions{.png,.jpg,.gif}
\fi

% Umdefinition der hyperref-Nummerierung im PDF-Modus
\ifttm
\else
\renewcommand{\theHfigure}{\arabic{chapter}.\arabic{section}.\arabic{figure}}
\fi

% Makro, um in der HTML-Ausgabe die zuerst zu oeffnende Datei zu kennzeichnen
\ifttm
\newcommand{\MGlobalStart}{\special{html:<!-- mglobalstarttag -->}}
\else
\newcommand{\MGlobalStart}{}
\fi

% Makro, um bei scormlogin ein pullen des Benutzers bei Aufruf der Seite zu erzwingen (typischerweise auf der Einstiegsseite)
\ifttm
\newcommand{\MPullSite}{\special{html:<!-- pullsite //-->}}
\else
\newcommand{\MPullSite}{}
\fi

% Makro, um in der HTML-Ausgabe die Kapiteluebersicht zu kennzeichnen
\ifttm
\newcommand{\MGlobalChapterTag}{\special{html:<!-- mglobalchaptertag -->}}
\else
\newcommand{\MGlobalChapterTag}{}
\fi

% Makro, um in der HTML-Ausgabe die Konfiguration zu kennzeichnen
\ifttm
\newcommand{\MGlobalConfTag}{\special{html:<!-- mglobalconfigtag -->}}
\else
\newcommand{\MGlobalConfTag}{}
\fi

% Makro, um in der HTML-Ausgabe die Standortbeschreibung zu kennzeichnen
\ifttm
\newcommand{\MGlobalLocationTag}{\special{html:<!-- mgloballocationtag -->}}
\else
\newcommand{\MGlobalLocationTag}{}
\fi

% Makro, um in der HTML-Ausgabe die persoenlichen Daten zu kennzeichnen
\ifttm
\newcommand{\MGlobalDataTag}{\special{html:<!-- mglobaldatatag -->}}
\else
\newcommand{\MGlobalDataTag}{}
\fi

% Makro, um in der HTML-Ausgabe die Suchseite zu kennzeichnen
\ifttm
\newcommand{\MGlobalSearchTag}{\special{html:<!-- mglobalsearchtag -->}}
\else
\newcommand{\MGlobalSearchTag}{}
\fi

% Makro, um in der HTML-Ausgabe die Favoritenseite zu kennzeichnen
\ifttm
\newcommand{\MGlobalFavoTag}{\special{html:<!-- mglobalfavoritestag -->}}
\else
\newcommand{\MGlobalFavoTag}{}
\fi

% Makro, um in der HTML-Ausgabe die Eingangstestseite zu kennzeichnen
\ifttm
\newcommand{\MGlobalSTestTag}{\special{html:<!-- mglobalstesttag -->}}
\else
\newcommand{\MGlobalSTestTag}{}
\fi

% Makro, um in der PDF-Ausgabe ein Wasserzeichen zu definieren
\ifttm
\newcommand{\MWatermarkSettings}{\relax}
\else
\newcommand{\MWatermarkSettings}{%
% \SetWatermarkText{(c) MINT-Kolleg Baden-W�rttemberg 2014}
% \SetWatermarkLightness{0.85}
% \SetWatermarkScale{1.5}
}
\fi

\ifttm
\newcommand{\MBinom}[2]{\left({\begin{array}{c} #1 \\ #2 \end{array}}\right)}
\else
\newcommand{\MBinom}[2]{\binom{#1}{#2}}
\fi

\ifttm
\newcommand{\DeclareMathOperator}[2]{\def#1{\mathrm{#2}}}
\newcommand{\operatorname}[1]{\mathrm{#1}}
\fi

%----------------- Makros fuer die gemischte HTML/PDF-Konvertierung ------------------------------

\newcommand{\MTestName}{\relax} % wird durch Test-Umgebung gesetzt

% Fuer experimentelle Kursinhalte, die im Release-Umsetzungsvorgang eine Fehlermeldung
% produzieren sollen aber sonst normal umgesetzt werden
\newenvironment{MExperimental}{%
}{%
}

% Wird von ttm nicht richtig umgesetzt!!
\newenvironment{MExerciseItems}{%
\renewcommand\theenumi{\alph{enumi}}%
\begin{enumerate}%
}{%
\end{enumerate}%
}


\definecolor{infoshadecolor}{rgb}{0.75,0.75,0.75}
\definecolor{exmpshadecolor}{rgb}{0.875,0.875,0.875}
\definecolor{expeshadecolor}{rgb}{0.95,0.95,0.95}
\definecolor{framecolor}{rgb}{0.2,0.2,0.2}

% Bei PDF-Uebersetzung wird hinter den Start jeder Satz/Info-aehnlichen Umgebung eine leere mbox gesetzt, damit
% fuehrende Listen oder enums nicht den Zeilenumbruch kaputtmachen
%\ifttm
\def\MTB{}
%\else
%\def\MTB{\mbox{}}
%\fi


\ifttm
\newcommand{\MRelates}{\special{html:<mi>&wedgeq;</mi>}}
\else
\def\MRelates{\stackrel{\scriptscriptstyle\wedge}{=}}
\fi

\def\MInch{\text{''}}
\def\Mdd{\textit{''}}

\ifttm
\def\MNL{ \newline }
\newenvironment{MArray}[1]{\begin{array}{#1}}{\end{array}}
\else
\def\MNL{ \\ }
\newenvironment{MArray}[1]{\begin{array}{#1}}{\end{array}}
\fi

\newcommand{\MBox}[1]{$\mathrm{#1}$}
\newcommand{\MMBox}[1]{\mathrm{#1}}


\ifttm%
\newcommand{\Mtfrac}[2]{{\textstyle \frac{#1}{#2}}}
\newcommand{\Mdfrac}[2]{{\displaystyle \frac{#1}{#2}}}
\newcommand{\Mmeasuredangle}{\special{html:<mi>&angmsd;</mi>}}
\else%
\newcommand{\Mtfrac}[2]{\tfrac{#1}{#2}}
\newcommand{\Mdfrac}[2]{\dfrac{#1}{#2}}
\newcommand{\Mmeasuredangle}{\measuredangle}
\relax
\fi

% Matrizen und Vektoren

% Inhalt wird in der Form a & b \\ c & d erwartet
% Vorsicht: MVector = Komponentenspalte, MVec = Variablensymbol
\ifttm%
\newcommand{\MVector}[1]{\left({\begin{array}{c}#1\end{array}}\right)}
\else%
\newcommand{\MVector}[1]{\begin{pmatrix}#1\end{pmatrix}}
\fi



\newcommand{\MVec}[1]{\vec{#1}}
\newcommand{\MDVec}[1]{\overrightarrow{#1}}

%----------------- Umgebungen fuer Definitionen und Saetze ----------------------------------------

% Fuegt einen Tabellen-Zeilenumbruch ein im PDF, aber nicht im HTML
\newcommand{\TSkip}{\ifttm \else&\ \\\fi}

\newenvironment{infoshaded}{%
\def\FrameCommand{\fboxsep=\FrameSep \fcolorbox{framecolor}{infoshadecolor}}%
\MakeFramed {\advance\hsize-\width \FrameRestore}}%
{\endMakeFramed}

\newenvironment{expeshaded}{%
\def\FrameCommand{\fboxsep=\FrameSep \fcolorbox{framecolor}{expeshadecolor}}%
\MakeFramed {\advance\hsize-\width \FrameRestore}}%
{\endMakeFramed}

\newenvironment{exmpshaded}{%
\def\FrameCommand{\fboxsep=\FrameSep \fcolorbox{framecolor}{exmpshadecolor}}%
\MakeFramed {\advance\hsize-\width \FrameRestore}}%
{\endMakeFramed}

\def\STDCOLOR{black}

\ifttm%
\else%
\newtheoremstyle{MSatzStyle}
  {1cm}                   %Space above
  {1cm}                   %Space below
  {\normalfont\itshape}   %Body font
  {}                      %Indent amount (empty = no indent,
                          %\parindent = para indent)
  {\normalfont\bfseries}  %Thm head font
  {}                      %Punctuation after thm head
  {\newline}              %Space after thm head: " " = normal interword
                          %space; \newline = linebreak
  {\thmname{#1}\thmnumber{ #2}\thmnote{ (#3)}}
                          %Thm head spec (can be left empty, meaning
                          %`normal')
                          %
\newtheoremstyle{MDefStyle}
  {1cm}                   %Space above
  {1cm}                   %Space below
  {\normalfont}           %Body font
  {}                      %Indent amount (empty = no indent,
                          %\parindent = para indent)
  {\normalfont\bfseries}  %Thm head font
  {}                      %Punctuation after thm head
  {\newline}              %Space after thm head: " " = normal interword
                          %space; \newline = linebreak
  {\thmname{#1}\thmnumber{ #2}\thmnote{ (#3)}}
                          %Thm head spec (can be left empty, meaning
                          %`normal')
\fi%

\newcommand{\MInfoText}{Info}

\newcounter{MHintCounter}
\newcounter{MCodeEditCounter}

\newcounter{MLastIndex}  % Enthaelt die dritte Stelle (Indexnummer) des letzten angelegten Objekts
\newcounter{MLastType}   % Enthaelt den Typ des letzten angelegten Objekts (mithilfe der unten definierten Konstanten). Die Entscheidung, wie der Typ dargstellt wird, wird in split.pm beim Postprocessing getroffen.
\newcounter{MLastTypeEq} % =1 falls das Label in einer Matheumgebung (equation, eqnarray usw.) steht, =2 falls das Label in einer table-Umgebung steht

% Da ttm keine Zahlmakros verarbeiten kann, werden diese Nummern in den Zuweisungen hardcodiert!
\def\MTypeSection{1}          %# Zaehler ist section
\def\MTypeSubsection{2}       %# Zaehler ist subsection
\def\MTypeSubsubsection{3}    %# Zaehler ist subsubsection
\def\MTypeInfo{4}             %# Eine Infobox, Separatzaehler fuer die Chemie (auch wenn es dort nicht nummeriert wird) ist MInfoCounter
\def\MTypeExercise{5}         %# Eine Aufgabe, Separatzaehler fuer die Chemie ist MExerciseCounter
\def\MTypeExample{6}          %# Eine Beispielbox, Separatzaehler fuer die Chemie ist MExampleCounter
\def\MTypeExperiment{7}       %# Eine Versuchsbox, Separatzaehler fuer die Chemie ist MExperimentCounter
\def\MTypeGraphics{8}         %# Eine Graphik, Separatzaehler fuer alle FB ist MGraphicsCounter
\def\MTypeTable{9}            %# Eine Tabellennummer, hat keinen Zaehler da durch table gezaehlt wird
\def\MTypeEquation{10}        %# Eine Gleichungsnummer, hat keinen Zaehler da durch equation/eqnarray gezaehlt wird
\def\MTypeTheorem{11}         % Ein theorem oder xtheorem, Separatzaehler fuer die Chemie ist MTheoremCounter
\def\MTypeVideo{12}           %# Ein Video,Separatzaehler fuer alle FB ist MVideoCounter
\def\MTypeEntry{13}           %# Ein Eintrag fuer die Stichwortliste, wird nicht gezaehlt sondern erhaelt im preparsing ein unique-label 

% Zaehler fuer das Labelsystem sind prefixcounter, jeder Zaehler wird VOR dem gezaehlten Objekt inkrementiert und zaehlt daher das aktuelle Objekt
\newcounter{MInfoCounter}
\newcounter{MExerciseCounter}
\newcounter{MExampleCounter}
\newcounter{MExperimentCounter}
\newcounter{MGraphicsCounter}
\newcounter{MTableCounter}
\newcounter{MEquationCounter}  % Nur im HTML, sonst durch "equation"-counter von latex realisiert
\newcounter{MTheoremCounter}
\newcounter{MObjectCounter}   % Gemeinsamer Zaehler fuer Objekte (ausser Grafiken/Tabellen) in Mathe/Info/Physik
\newcounter{MVideoCounter}
\newcounter{MEntryCounter}

\newcounter{MTestSite} % 1 = Subsubsection ist eine Pruefungsseite, 0 = ist eine normale Seite (inkl. Hilfeseite)

\def\MCell{$\phantom{a}$}

\newenvironment{MExportExercise}{\begin{MExercise}}{\end{MExercise}} % wird von mconvert abgefangen

\def\MGenerateExNumber{%
\ifnum\value{MSepNumbers}=0%
\arabic{section}.\arabic{subsection}.\arabic{MObjectCounter}\setcounter{MLastIndex}{\value{MObjectCounter}}%
\else%
\arabic{section}.\arabic{subsection}.\arabic{MExerciseCounter}\setcounter{MLastIndex}{\value{MExerciseCounter}}%
\fi%
}%

\def\MGenerateExmpNumber{%
\ifnum\value{MSepNumbers}=0%
\arabic{section}.\arabic{subsection}.\arabic{MObjectCounter}\setcounter{MLastIndex}{\value{MObjectCounter}}%
\else%
\arabic{section}.\arabic{subsection}.\arabic{MExerciseCounter}\setcounter{MLastIndex}{\value{MExampleCounter}}%
\fi%
}%

\def\MGenerateInfoNumber{%
\ifnum\value{MSepNumbers}=0%
\arabic{section}.\arabic{subsection}.\arabic{MObjectCounter}\setcounter{MLastIndex}{\value{MObjectCounter}}%
\else%
\arabic{section}.\arabic{subsection}.\arabic{MExerciseCounter}\setcounter{MLastIndex}{\value{MInfoCounter}}%
\fi%
}%

\def\MGenerateSiteNumber{%
\arabic{section}.\arabic{subsection}.\arabic{subsubsection}%
}%

% Funktionalitaet fuer Auswahlaufgaben

\newcounter{MExerciseCollectionCounter} % = 0 falls nicht in collection-Umgebung, ansonsten Schachtelungstiefe
\newcounter{MExerciseCollectionTextCounter} % wird von MExercise-Umgebung inkrementiert und von MExerciseCollection-Umgebung auf Null gesetzt

\ifttm
% MExerciseCollection gruppiert Aufgaben, die dynamisch aus der Datenbank gezogen werden und nicht direkt in der HTML-Seite stehen
% Parameter: #1 = ID der Collection, muss eindeutig fuer alle IN DER DB VORHANDENEN collections sein unabhaengig vom Kurs
%            #2 = Optionsargument (im Moment: 1 = Iterative Auswahl, 2 = Zufallsbasierte Auswahl)
\newenvironment{MExerciseCollection}[2]{%
\addtocounter{MExerciseCollectionCounter}{1}
\setcounter{MExerciseCollectionTextCounter}{0}
\special{html:<!-- mexercisecollectionstart;;}#1\special{html:;;}#2\special{html:;; //-->}%
}{%
\special{html:<!-- mexercisecollectionstop //-->}%
\addtocounter{MExerciseCollectionCounter}{-1}
}
\else
\newenvironment{MExerciseCollection}[2]{%
\addtocounter{MExerciseCollectionCounter}{1}
\setcounter{MExerciseCollectionTextCounter}{0}
}{%
\addtocounter{MExerciseCollectionCounter}{-1}
}
\fi

% Bei Uebersetzung nach PDF werden die theorem-Umgebungen verwendet, bei Uebersetzung in HTML ein manuelles Makro
\ifttm%

  \newenvironment{MHint}[1]{  \special{html:<button name="Name_MHint}\arabic{MHintCounter}\special{html:" class="hintbutton_closed" id="MHint}\arabic{MHintCounter}\special{html:_button" %
  type="button" onclick="toggle_hint('MHint}\arabic{MHintCounter}\special{html:');">}#1\special{html:</button>}
  \special{html:<div class="hint" style="display:none" id="MHint}\arabic{MHintCounter}\special{html:"> }}{\begin{html}</div>\end{html}\addtocounter{MHintCounter}{1}}

  \newenvironment{MCOSHZusatz}{  \special{html:<button name="Name_MHint}\arabic{MHintCounter}\special{html:" class="chintbutton_closed" id="MHint}\arabic{MHintCounter}\special{html:_button" %
  type="button" onclick="toggle_hint('MHint}\arabic{MHintCounter}\special{html:');">}Weiterf�hrende Inhalte\special{html:</button>}
  \special{html:<div class="hintc" style="display:none" id="MHint}\arabic{MHintCounter}\special{html:">
  <div class="coshwarn">Diese Inhalte gehen �ber das Kursniveau hinaus und werden in den Aufgaben und Tests nicht abgefragt.</div><br />}
  \addtocounter{MHintCounter}{1}}{\begin{html}</div>\end{html}}

  
  \newenvironment{MDefinition}{\begin{definition}\setcounter{MLastIndex}{\value{definition}}\ \\}{\end{definition}}

  
  \newenvironment{MExercise}{
  \renewcommand{\MStdPoints}{4}
  \addtocounter{MExerciseCounter}{1}
  \addtocounter{MObjectCounter}{1}
  \setcounter{MLastType}{5}

  \ifnum\value{MExerciseCollectionCounter}=0\else\addtocounter{MExerciseCollectionTextCounter}{1}\special{html:<!-- mexercisetextstart;;}\arabic{MExerciseCollectionTextCounter}\special{html:;; //-->}\fi
  \special{html:<div class="aufgabe" id="ADIV_}\MGenerateExNumber\special{html:">}%
  \textbf{Aufgabe \MGenerateExNumber
  } \ \\}{
  \special{html:</div><!-- mfeedbackbutton;Aufgabe;}\arabic{MTestSite}\special{html:;}\MGenerateExNumber\special{html:; //-->}
  \ifnum\value{MExerciseCollectionCounter}=0\else\special{html:<!-- mexercisetextstop //-->}\fi
  }

  % Stellt eine Kombination aus Aufgabe, Loesungstext und Eingabefeld bereit,
  % bei der Aufgabentext und Musterloesung sowie die zugehoerigen Feldelemente
  % extern bezogen und div-aktualisiert werden, das Eingabefeld aber immer das gleiche ist.
  \newenvironment{MFetchExercise}{
  \addtocounter{MExerciseCounter}{1}
  \addtocounter{MObjectCounter}{1}
  \setcounter{MLastType}{5}

  \special{html:<div class="aufgabe" id="ADIV_}\MGenerateExNumber\special{html:">}%
  \textbf{Aufgabe \MGenerateExNumber
  } \ \\%
  \special{html:</div><div class="exfetch_text" id="ADIVTEXT_}\MGenerateExNumber\special{html:">}%
  \special{html:</div><div class="exfetch_sol" id="ADIVSOL_}\MGenerateExNumber\special{html:">}%
  \special{html:</div><div class="exfetch_input" id="ADIVINPUT_}\MGenerateExNumber\special{html:">}%
  }{
  \special{html:</div>}
  }

  \newenvironment{MExample}{
  \addtocounter{MExampleCounter}{1}
  \addtocounter{MObjectCounter}{1}
  \setcounter{MLastType}{6}
  \begin{html}
  <div class="exmp">
  <div class="exmprahmen">
  \end{html}\textbf{Beispiel
  \ifnum\value{MSepNumbers}=0
  \arabic{section}.\arabic{subsection}.\arabic{MObjectCounter}\setcounter{MLastIndex}{\value{MObjectCounter}}
  \else
  \arabic{section}.\arabic{subsection}.\arabic{MExampleCounter}\setcounter{MLastIndex}{\value{MExampleCounter}}
  \fi
  } \ \\}{\begin{html}</div>
  </div>
  \end{html}
  \special{html:<!-- mfeedbackbutton;Beispiel;}\arabic{MTestSite}\special{html:;}\MGenerateExmpNumber\special{html:; //-->}
  }

  \newenvironment{MExperiment}{
  \addtocounter{MExperimentCounter}{1}
  \addtocounter{MObjectCounter}{1}
  \setcounter{MLastType}{7}
  \begin{html}
  <div class="expe">
  <div class="experahmen">
  \end{html}\textbf{Versuch
  \ifnum\value{MSepNumbers}=0
  \arabic{section}.\arabic{subsection}.\arabic{MObjectCounter}\setcounter{MLastIndex}{\value{MObjectCounter}}
  \else
%  \arabic{MExperimentCounter}\setcounter{MLastIndex}{\value{MExperimentCounter}}
  \arabic{section}.\arabic{subsection}.\arabic{MExperimentCounter}\setcounter{MLastIndex}{\value{MExperimentCounter}}
  \fi
  } \ \\}{\begin{html}</div>
  </div>
  \end{html}}

  \newenvironment{MChemInfo}{
  \setcounter{MLastType}{4}
  \begin{html}
  <div class="info">
  <div class="inforahmen">
  \end{html}}{\begin{html}</div>
  </div>
  \end{html}}

  \newenvironment{MXInfo}[1]{
  \addtocounter{MInfoCounter}{1}
  \addtocounter{MObjectCounter}{1}
  \setcounter{MLastType}{4}
  \begin{html}
  <div class="info">
  <div class="inforahmen">
  \end{html}\textbf{#1
  \ifnum\value{MInfoNumbers}=0
  \else
    \ifnum\value{MSepNumbers}=0
    \arabic{section}.\arabic{subsection}.\arabic{MObjectCounter}\setcounter{MLastIndex}{\value{MObjectCounter}}
    \else
    \arabic{MInfoCounter}\setcounter{MLastIndex}{\value{MInfoCounter}}
    \fi
  \fi
  } \ \\}{\begin{html}</div>
  </div>
  \end{html}
  \special{html:<!-- mfeedbackbutton;Info;}\arabic{MTestSite}\special{html:;}\MGenerateInfoNumber\special{html:; //-->}
  }

  \newenvironment{MInfo}{\ifnum\value{MInfoNumbers}=0\begin{MChemInfo}\else\begin{MXInfo}{Info}\ \\ \fi}{\ifnum\value{MInfoNumbers}=0\end{MChemInfo}\else\end{MXInfo}\fi}

\else%

  \theoremstyle{MSatzStyle}
  \newtheorem{thm}{Satz}[section]
  \newtheorem{thmc}{Satz}
  \theoremstyle{MDefStyle}
  \newtheorem{defn}[thm]{Definition}
  \newtheorem{exmp}[thm]{Beispiel}
  \newtheorem{info}[thm]{\MInfoText}
  \theoremstyle{MDefStyle}
  \newtheorem{defnc}{Definition}
  \theoremstyle{MDefStyle}
  \newtheorem{exmpc}{Beispiel}[section]
  \theoremstyle{MDefStyle}
  \newtheorem{infoc}{\MInfoText}
  \theoremstyle{MDefStyle}
  \newtheorem{exrc}{Aufgabe}[section]
  \theoremstyle{MDefStyle}
  \newtheorem{verc}{Versuch}[section]
  
  \newenvironment{MFetchExercise}{}{} % kann im PDF nicht dargestellt werden
  
  \newenvironment{MExercise}{\begin{exrc}\renewcommand{\MStdPoints}{1}\MTB}{\end{exrc}}
  \newenvironment{MHint}[1]{\ \\ \underline{#1:}\\}{}
  \newenvironment{MCOSHZusatz}{\ \\ \underline{Weiterf�hrende Inhalte:}\\}{}
  \newenvironment{MDefinition}{\ifnum\value{MInfoNumbers}=0\begin{defnc}\else\begin{defn}\fi\MTB}{\ifnum\value{MInfoNumbers}=0\end{defnc}\else\end{defn}\fi}
%  \newenvironment{MExample}{\begin{exmp}}{\ \linebreak[1] \ \ \ \ $\phantom{a}$ \ \hfill $\blacklozenge$\end{exmp}}
  \newenvironment{MExample}{
    \ifnum\value{MInfoNumbers}=0\begin{exmpc}\else\begin{exmp}\fi
    \MTB
    \begin{exmpshaded}
    \ \newline
}{
    \end{exmpshaded}
    \ifnum\value{MInfoNumbers}=0\end{exmpc}\else\end{exmp}\fi
}
  \newenvironment{MChemInfo}{\begin{infoshaded}}{\end{infoshaded}}

  \newenvironment{MInfo}{\ifnum\value{MInfoNumbers}=0\begin{MChemInfo}\else\renewcommand{\MInfoText}{Info}\begin{info}\begin{infoshaded}
  \MTB
   \ \newline
    \fi
  }{\ifnum\value{MInfoNumbers}=0\end{MChemInfo}\else\end{infoshaded}\end{info}\fi}

  \newenvironment{MXInfo}[1]{
    \renewcommand{\MInfoText}{#1}
    \ifnum\value{MInfoNumbers}=0\begin{infoc}\else\begin{info}\fi%
    \MTB
    \begin{infoshaded}
    \ \newline
  }{\end{infoshaded}\ifnum\value{MInfoNumbers}=0\end{infoc}\else\end{info}\fi}

  \newenvironment{MExperiment}{
    \renewcommand{\MInfoText}{Versuch}
    \ifnum\value{MInfoNumbers}=0\begin{verc}\else\begin{info}\fi
    \MTB
    \begin{expeshaded}
    \ \newline
  }{
    \end{expeshaded}
    \ifnum\value{MInfoNumbers}=0\end{verc}\else\end{info}\fi
  }
\fi%

% MHint sollte nicht direkt fuer Loesungen benutzt werden wegen solutionselect
\newenvironment{MSolution}{\begin{MHint}{L"osung}}{\end{MHint}}

\newcounter{MCodeCounter}

\ifttm
\newenvironment{MCode}{\special{html:<!-- mcodestart -->}\ttfamily\color{blue}}{\special{html:<!-- mcodestop -->}}
\else
\newenvironment{MCode}{\begin{flushleft}\ttfamily\addtocounter{MCodeCounter}{1}}{\addtocounter{MCodeCounter}{-1}\end{flushleft}}
% Ohne color-Statement da inkompatible mit framed/shaded-Boxen aus dem framed-package
\fi

%----------------- Sonderdefinitionen fuer Symbole, die der Konverter nicht kann ----------------------------------------------

\ifttm%
\newcommand{\MUnderset}[2]{\underbrace{#2}_{#1}}%
\else%
\newcommand{\MUnderset}[2]{\underset{#1}{#2}}%
\fi%

\ifttm
\newcommand{\MThinspace}{\special{html:<mi>&#x2009;</mi>}}
\else
\newcommand{\MThinspace}{\,}
\fi

\ifttm
\newcommand{\glq}{\begin{html}&sbquo;\end{html}}
\newcommand{\grq}{\begin{html}&lsquo;\end{html}}
\newcommand{\glqq}{\begin{html}&bdquo;\end{html}}
\newcommand{\grqq}{\begin{html}&ldquo;\end{html}}
\fi

\ifttm
\newcommand{\MNdash}{\begin{html}&ndash;\end{html}}
\else
\newcommand{\MNdash}{--}
\fi

%\ifttm\def\MIU{\special{html:<mi>&#8520;</mi>}}\else\def\MIU{\mathrm{i}}\fi
\def\MIU{\mathrm{i}}
\def\MEU{e} % TU9-Onlinekurs: italic-e
%\def\MEU{\mathrm{e}} % Alte Onlinemodule: roman-e
\def\MD{d} % Kursives d in Integralen im TU9-Onlinekurs
%\def\MD{\mathrm{d}} % roman-d in den alten Onlinemodulen
\def\MDB{\|}

%zusaetzlicher Leerraum vor "\MD"
\ifttm%
\def\MDSpace{\special{html:<mi>&#x2009;</mi>}}
\else%
\def\MDSpace{\,}
\fi%
\newcommand{\MDwSp}{\MDSpace\MD}%

\ifttm
\def\Mdq{\dq}
\else
\def\Mdq{\dq}
\fi

\def\MSpan#1{\left<{#1}\right>}
\def\MSetminus{\setminus}
\def\MIM{I}

\ifttm
\newcommand{\ld}{\text{ld}}
\newcommand{\lg}{\text{lg}}
\else
\DeclareMathOperator{\ld}{ld}
%\newcommand{\lg}{\text{lg}} % in latex schon definiert
\fi


\def\Mmapsto{\ifttm\special{html:<mi>&mapsto;</mi>}\else\mapsto\fi} 
\def\Mvarphi{\ifttm\phi\else\varphi\fi}
\def\Mphi{\ifttm\varphi\else\phi\fi}
\ifttm%
\newcommand{\MEumu}{\special{html:<mi>&#x3BC;</mi>}}%
\else%
\newcommand{\MEumu}{\textrm{\textmu}}%
\fi
\def\Mvarepsilon{\ifttm\epsilon\else\varepsilon\fi}
\def\Mepsilon{\ifttm\varepsilon\else\epsilon\fi}
\def\Mvarkappa{\ifttm\kappa\else\varkappa\fi}
\def\Mkappa{\ifttm\varkappa\else\kappa\fi}
\def\Mcomplement{\ifttm\special{html:<mi>&comp;</mi>}\else\complement\fi} 
\def\MWW{\mathrm{WW}}
\def\Mmod{\ifttm\special{html:<mi>&nbsp;mod&nbsp;</mi>}\else\mod\fi} 

\ifttm%
\def\mod{\text{\;mod\;}}%
\def\MNEquiv{\special{html:<mi>&NotCongruent;</mi>}}% 
\def\MNSubseteq{\special{html:<mi>&NotSubsetEqual;</mi>}}%
\def\MEmptyset{\special{html:<mi>&empty;</mi>}}%
\def\MVDots{\special{html:<mi>&#x22EE;</mi>}}%
\def\MHDots{\special{html:<mi>&#x2026;</mi>}}%
\def\Mddag{\special{html:<mi>&#x1202;</mi>}}%
\def\sphericalangle{\special{html:<mi>&measuredangle;</mi>}}%
\def\nparallel{\special{html:<mi>&nparallel;</mi>}}%
\def\MProofEnd{\special{html:<mi>&#x25FB;</mi>}}%
\newenvironment{MProof}[1]{\underline{#1}:\MCR\MCR}{\hfill $\MProofEnd$}%
\else%
\def\MNEquiv{\not\equiv}%
\def\MNSubseteq{\not\subseteq}%
\def\MEmptyset{\emptyset}%
\def\MVDots{\vdots}%
\def\MHDots{\hdots}%
\def\Mddag{\ddag}%
\newenvironment{MProof}[1]{\begin{proof}[#1]}{\end{proof}}%
\fi%



% Spaces zum Auffuellen von Tabellenbreiten, die nur im HTML wirken
\ifttm%
\def\MTSP{\:}%
\else%
\def\MTSP{}%
\fi%

\DeclareMathOperator{\arsinh}{arsinh}
\DeclareMathOperator{\arcosh}{arcosh}
\DeclareMathOperator{\artanh}{artanh}
\DeclareMathOperator{\arcoth}{arcoth}


\newcommand{\MMathSet}[1]{\mathbb{#1}}
\def\N{\MMathSet{N}}
\def\Z{\MMathSet{Z}}
\def\Q{\MMathSet{Q}}
\def\R{\MMathSet{R}}
\def\C{\MMathSet{C}}

\newcounter{MForLoopCounter}
\newcommand{\MForLoop}[2]{\setcounter{MForLoopCounter}{#1}\ifnum\value{MForLoopCounter}=0{}\else{{#2}\addtocounter{MForLoopCounter}{-1}\MForLoop{\value{MForLoopCounter}}{#2}}\fi}

\newcounter{MSiteCounter}
\newcounter{MFieldCounter} % Kombination section.subsection.site.field ist eindeutig in allen Modulen, field alleine nicht

\newcounter{MiniMarkerCounter}

\ifttm
\newenvironment{MMiniPageP}[1]{\begin{minipage}{#1\linewidth}\special{html:<!-- minimarker;;}\arabic{MiniMarkerCounter}\special{html:;;#1; //-->}}{\end{minipage}\addtocounter{MiniMarkerCounter}{1}}
\else
\newenvironment{MMiniPageP}[1]{\begin{minipage}{#1\linewidth}}{\end{minipage}\addtocounter{MiniMarkerCounter}{1}}
\fi

\newcounter{AlignCounter}

\newcommand{\MStartJustify}{\ifttm\special{html:<!-- startalign;;}\arabic{AlignCounter}\special{html:;;justify; //-->}\fi}
\newcommand{\MStopJustify}{\ifttm\special{html:<!-- stopalign;;}\arabic{AlignCounter}\special{html:; //-->}\fi\addtocounter{AlignCounter}{1}}

\newenvironment{MJTabular}[1]{
\MStartJustify
\begin{tabular}{#1}
}{
\end{tabular}
\MStopJustify
}

\newcommand{\MImageLeft}[2]{
\begin{center}
\begin{tabular}{lc}
\MStartJustify
\begin{MMiniPageP}{0.65}
#1
\end{MMiniPageP}
\MStopJustify
&
\begin{MMiniPageP}{0.3}
#2  
\end{MMiniPageP}
\end{tabular}
\end{center}
}

\newcommand{\MImageHalf}[2]{
\begin{center}
\begin{tabular}{lc}
\MStartJustify
\begin{MMiniPageP}{0.45}
#1
\end{MMiniPageP}
\MStopJustify
&
\begin{MMiniPageP}{0.45}
#2  
\end{MMiniPageP}
\end{tabular}
\end{center}
}

\newcommand{\MBigImageLeft}[2]{
\begin{center}
\begin{tabular}{lc}
\MStartJustify
\begin{MMiniPageP}{0.25}
#1
\end{MMiniPageP}
\MStopJustify
&
\begin{MMiniPageP}{0.7}
#2  
\end{MMiniPageP}
\end{tabular}
\end{center}
}

\ifttm
\def\No{\mathbb{N}_0}
\else
\def\No{\ensuremath{\N_0}}
\fi
\def\MT{\textrm{\tiny T}}
\newcommand{\MTranspose}[1]{{#1}^{\MT}}
\ifttm
\newcommand{\MRe}{\mathsf{Re}}
\newcommand{\MIm}{\mathsf{Im}}
\else
\DeclareMathOperator{\MRe}{Re}
\DeclareMathOperator{\MIm}{Im}
\fi

\newcommand{\Mid}{\mathrm{id}}
\newcommand{\MFeinheit}{\mathrm{feinh}}

\ifttm
\newcommand{\Msubstack}[1]{\begin{array}{c}{#1}\end{array}}
\else
\newcommand{\Msubstack}[1]{\substack{#1}}
\fi

% Typen von Fragefeldern:
% 1 = Alphanumerisch, case-sensitive-Vergleich
% 2 = Ja/Nein-Checkbox, Loesung ist 0 oder 1   (OPTION = Image-id fuer Rueckmeldung)
% 3 = Reelle Zahlen Geparset
% 4 = Funktionen Geparset (mit Stuetzstellen zur ueberpruefung)

% Dieser Befehl erstellt ein interaktives Aufgabenfeld. Parameter:
% - #1 Laenge in Zeichen
% - #2 Loesungstext (alphanumerisch, case sensitive)
% - #3 AufgabenID (alphanumerisch, case sensitive)
% - #4 Typ (Kennnummer)
% - #5 String fuer Optionen (ggf. mit Semikolon getrennte Einzelstrings)
% - #6 Anzahl Punkte
% - #7 uxid (kann z.B. Loesungsstring sein)
% ACHTUNG: Die langen Zeilen bitte so lassen, Zeilenumbrueche im tex werden in div's umgesetzt
\newcommand{\MQuestionID}[7]{
\ifttm
\special{html:<!-- mdeclareuxid;;}UX#7\special{html:;;}\arabic{section}\special{html:;;}#3\special{html:;; //-->}%
\special{html:<!-- mdeclarepoints;;}\arabic{section}\special{html:;;}#3\special{html:;;}#6\special{html:;;}\arabic{MTestSite}\special{html:;;}\arabic{chapter}%
\special{html:;; //--><!-- onloadstart //-->CreateQuestionObj("}#7\special{html:",}\arabic{MFieldCounter}\special{html:,"}#2%
\special{html:","}#3\special{html:",}#4\special{html:,"}#5\special{html:",}#6\special{html:,}\arabic{MTestSite}\special{html:,}\arabic{section}%
\special{html:);<!-- onloadstop //-->}%
\special{html:<input mfieldtype="}#4\special{html:" name="Name_}#3\special{html:" id="}#3\special{html:" type="text" size="}#1\special{html:" maxlength="}#1%
\special{html:" }\ifnum\value{MGroupActive}=0\special{html:onfocus="handlerFocus(}\arabic{MFieldCounter}%
\special{html:);" onblur="handlerBlur(}\arabic{MFieldCounter}\special{html:);" onkeyup="handlerChange(}\arabic{MFieldCounter}\special{html:,0);" onpaste="handlerChange(}\arabic{MFieldCounter}\special{html:,0);" oninput="handlerChange(}\arabic{MFieldCounter}\special{html:,0);" onpropertychange="handlerChange(}\arabic{MFieldCounter}\special{html:,0);"/>}%
\special{html:<img src="images/questionmark.gif" width="20" height="20" border="0" align="absmiddle" id="}QM#3\special{html:"/>}
\else%
\special{html:onblur="handlerBlur(}\arabic{MFieldCounter}%
\special{html:);" onfocus="handlerFocus(}\arabic{MFieldCounter}\special{html:);" onkeyup="handlerChange(}\arabic{MFieldCounter}\special{html:,1);" onpaste="handlerChange(}\arabic{MFieldCounter}\special{html:,1);" oninput="handlerChange(}\arabic{MFieldCounter}\special{html:,1);" onpropertychange="handlerChange(}\arabic{MFieldCounter}\special{html:,1);"/>}%
\special{html:<img src="images/questionmark.gif" width="20" height="20" border="0" align="absmiddle" id="}QM#3\special{html:"/>}\fi%
\else%
\ifnum\value{QBoxFlag}=1\fbox{$\phantom{\MForLoop{#1}{b}}$}\else$\phantom{\MForLoop{#1}{b}}$\fi%
\fi%
}

% ACHTUNG: Die langen Zeilen bitte so lassen, Zeilenumbrueche im tex werden in div's umgesetzt
% QuestionCheckbox macht ausserhalb einer QuestionGroup keinen Sinn!
% #1 = solution (1 oder 0), ggf. mit ::smc abgetrennt auszuschliessende single-choice-boxen (UXIDs durch , getrennt), #2 = id, #3 = points, #4 = uxid
\newcommand{\MQuestionCheckbox}[4]{
\ifttm
\special{html:<!-- mdeclareuxid;;}UX#4\special{html:;;}\arabic{section}\special{html:;;}#2\special{html:;; //-->}%
\ifnum\value{MGroupActive}=0\MDebugMessage{ERROR: Checkbox Nr. \arabic{MFieldCounter}\ ist nicht in einer Kontrollgruppe, es wird niemals eine Loesung angezeigt!}\fi
\special{html: %
<!-- mdeclarepoints;;}\arabic{section}\special{html:;;}#2\special{html:;;}#3\special{html:;;}\arabic{MTestSite}\special{html:;;}\arabic{chapter}%
\special{html:;; //--><!-- onloadstart //-->CreateQuestionObj("}#4\special{html:",}\arabic{MFieldCounter}\special{html:,"}#1\special{html:","}#2\special{html:",2,"IMG}#2%
\special{html:",}#3\special{html:,}\arabic{MTestSite}\special{html:,}\arabic{section}\special{html:);<!-- onloadstop //-->}%
\special{html:<input mfieldtype="2" type="checkbox" name="Name_}#2\special{html:" id="}#2\special{html:" onchange="handlerChange(}\arabic{MFieldCounter}\special{html:,1);"/><img src="images/questionmark.gif" name="}Name_IMG#2%
\special{html:" width="20" height="20" border="0" align="absmiddle" id="}IMG#2\special{html:"/> }%
\else%
\ifnum\value{QBoxFlag}=1\fbox{$\phantom{X}$}\else$\phantom{X}$\fi%
\fi%
}

\def\MGenerateID{QFELD_\arabic{section}.\arabic{subsection}.\arabic{MSiteCounter}.QF\arabic{MFieldCounter}}

% #1 = 0/1 ggf. mit ::smc abgetrennt auszuschliessende single-choice-boxen (UXIDs durch , getrennt ohne UX), #2 = uxid ohne UX
\newcommand{\MCheckbox}[2]{
\MQuestionCheckbox{#1}{\MGenerateID}{\MStdPoints}{#2}
\addtocounter{MFieldCounter}{1}
}

% Erster Parameter: Zeichenlaenge der Eingabebox, zweiter Parameter: Loesungstext
\newcommand{\MQuestion}[2]{
\MQuestionID{#1}{#2}{\MGenerateID}{1}{0}{\MStdPoints}{#2}
\addtocounter{MFieldCounter}{1}
}

% Erster Parameter: Zeichenlaenge der Eingabebox, zweiter Parameter: Loesungstext
\newcommand{\MLQuestion}[3]{
\MQuestionID{#1}{#2}{\MGenerateID}{1}{0}{\MStdPoints}{#3}
\addtocounter{MFieldCounter}{1}
}

% Parameter: Laenge des Feldes, Loesung (wird auch geparsed), Stellen Genauigkeit hinter dem Komma, weitere Stellen werden mathematisch gerundet vor Vergleich
\newcommand{\MParsedQuestion}[3]{
\MQuestionID{#1}{#2}{\MGenerateID}{3}{#3}{\MStdPoints}{#2}
\addtocounter{MFieldCounter}{1}
}

% Parameter: Laenge des Feldes, Loesung (wird auch geparsed), Stellen Genauigkeit hinter dem Komma, weitere Stellen werden mathematisch gerundet vor Vergleich
\newcommand{\MLParsedQuestion}[4]{
\MQuestionID{#1}{#2}{\MGenerateID}{3}{#3}{\MStdPoints}{#4}
\addtocounter{MFieldCounter}{1}
}

% Parameter: Laenge des Feldes, Loesungsfunktion, Anzahl Stuetzstellen, Funktionsvariablen durch Kommata getrennt (nicht case-sensitive), Anzahl Nachkommastellen im Vergleich
\newcommand{\MFunctionQuestion}[5]{
\MQuestionID{#1}{#2}{\MGenerateID}{4}{#3;#4;#5;0}{\MStdPoints}{#2}
\addtocounter{MFieldCounter}{1}
}

% Parameter: Laenge des Feldes, Loesungsfunktion, Anzahl Stuetzstellen, Funktionsvariablen durch Kommata getrennt (nicht case-sensitive), Anzahl Nachkommastellen im Vergleich, UXID
\newcommand{\MLFunctionQuestion}[6]{
\MQuestionID{#1}{#2}{\MGenerateID}{4}{#3;#4;#5;0}{\MStdPoints}{#6}
\addtocounter{MFieldCounter}{1}
}

% Parameter: Laenge des Feldes, Loesungsintervall, Genauigkeit der Zahlenwertpruefung
\newcommand{\MIntervalQuestion}[3]{
\MQuestionID{#1}{#2}{\MGenerateID}{6}{#3}{\MStdPoints}{#2}
\addtocounter{MFieldCounter}{1}
}

% Parameter: Laenge des Feldes, Loesungsintervall, Genauigkeit der Zahlenwertpruefung, UXID
\newcommand{\MLIntervalQuestion}[4]{
\MQuestionID{#1}{#2}{\MGenerateID}{6}{#3}{\MStdPoints}{#4}
\addtocounter{MFieldCounter}{1}
}

% Parameter: Laenge des Feldes, Loesungsfunktion, Anzahl Stuetzstellen, Funktionsvariable (nicht case-sensitive), Anzahl Nachkommastellen im Vergleich, Vereinfachungsbedingung
% Vereinfachungsbedingung ist eine der Folgenden:
% 0 = Keine Vereinfachungsbedingung
% 1 = Keine Klammern (runde oder eckige) mehr im vereinfachten Ausdruck
% 2 = Faktordarstellung (Term hat Produkte als letzte Operation, Summen als vorgeschaltete Operation)
% 3 = Summendarstellung (Term hat Summen als letzte Operation, Produkte als vorgeschaltete Operation)
% Flag 512: Besondere Stuetzstellen (nur >1 und nur schwach rational), sonst symmetrisch um Nullpunkt und ganze Zahlen inkl. Null werden getroffen
\newcommand{\MSimplifyQuestion}[6]{
\MQuestionID{#1}{#2}{\MGenerateID}{4}{#3;#4;#5;#6}{\MStdPoints}{#2}
\addtocounter{MFieldCounter}{1}
}

\newcommand{\MLSimplifyQuestion}[7]{
\MQuestionID{#1}{#2}{\MGenerateID}{4}{#3;#4;#5;#6}{\MStdPoints}{#7}
\addtocounter{MFieldCounter}{1}
}

% Parameter: Laenge des Feldes, Loesung (optionaler Ausdruck), Anzahl Stuetzstellen, Funktionsvariable (nicht case-sensitive), Anzahl Nachkommastellen im Vergleich, Spezialtyp (string-id)
\newcommand{\MLSpecialQuestion}[7]{
\MQuestionID{#1}{#2}{\MGenerateID}{7}{#3;#4;#5;#6}{\MStdPoints}{#7}
\addtocounter{MFieldCounter}{1}
}

\newcounter{MGroupStart}
\newcounter{MGroupEnd}
\newcounter{MGroupActive}

\newenvironment{MQuestionGroup}{
\setcounter{MGroupStart}{\value{MFieldCounter}}
\setcounter{MGroupActive}{1}
}{
\setcounter{MGroupActive}{0}
\setcounter{MGroupEnd}{\value{MFieldCounter}}
\addtocounter{MGroupEnd}{-1}
}

\newcommand{\MGroupButton}[1]{
\ifttm
\special{html:<button name="Name_Group}\arabic{MGroupStart}\special{html:to}\arabic{MGroupEnd}\special{html:" id="Group}\arabic{MGroupStart}\special{html:to}\arabic{MGroupEnd}\special{html:" %
type="button" onclick="group_button(}\arabic{MGroupStart}\special{html:,}\arabic{MGroupEnd}\special{html:);">}#1\special{html:</button>}
\else
\phantom{#1}
\fi
}

%----------------- Makros fuer die modularisierte Darstellung ------------------------------------

\def\MyText#1{#1}

% is used internally by the conversion package, should not be used by original tex documents
\def\MOrgLabel#1{\relax}

\ifttm

% Ein MLabel wird im html codiert durch das tag <!-- mmlabel;;Labelbezeichner;;SubjectArea;;chapter;;section;;subsection;;Index;;Objekttyp; //-->
\def\MLabel#1{%
\ifnum\value{MLastType}=8%
\ifnum\value{MCaptionOn}=0%
\MDebugMessage{ERROR: Grafik \arabic{MGraphicsCounter} hat separates label: #1 (Grafiklabels sollten nur in der Caption stehen)}%
\fi
\fi
\ifnum\value{MLastType}=12%
\ifnum\value{MCaptionOn}=0%
\MDebugMessage{ERROR: Video \arabic{MVideoCounter} hat separates label: #1 (Videolabels sollten nur in der Caption stehen}%
\fi
\fi
\ifnum\value{MLastType}=10\setcounter{MLastIndex}{\value{equation}}\fi
\label{#1}\begin{html}<!-- mmlabel;;#1;;\end{html}\arabic{MSubjectArea}\special{html:;;}\arabic{chapter}\special{html:;;}\arabic{section}\special{html:;;}\arabic{subsection}\special{html:;;}\arabic{MLastIndex}\special{html:;;}\arabic{MLastType}\special{html:; //-->}}%

\else

% Sonderbehandlung im PDF fuer Abbildungen in separater aux-Datei, da MGraphics die figure-Umgebung nicht verwendet
\def\MLabel#1{%
\ifnum\value{MLastType}=8%
\ifnum\value{MCaptionOn}=0%
\MDebugMessage{ERROR: Grafik \arabic{MGraphicsCounter} hat separates label: #1 (Grafiklabels sollten nur in der Caption stehen}%
\fi
\fi
\ifnum\value{MLastType}=12%
\ifnum\value{MCaptionOn}=0%
\MDebugMessage{ERROR: Video \arabic{MVideoCounter} hat separates label: #1 (Videolabels sollten nur in der Caption stehen}%
\fi
\fi
\label{#1}%
}%

\fi

% Gibt Begriff des referenzierten Objekts mit aus, aber nur im HTML, daher nur in Ausnahmefaellen (z.B. Copyrightliste) sinnvoll
\def\MCRef#1{\ifttm\special{html:<!-- mmref;;}#1\special{html:;;1; //-->}\else\vref{#1}\fi}


\def\MRef#1{\ifttm\special{html:<!-- mmref;;}#1\special{html:;;0; //-->}\else\vref{#1}\fi}
\def\MERef#1{\ifttm\special{html:<!-- mmref;;}#1\special{html:;;0; //-->}\else\eqref{#1}\fi}
\def\MNRef#1{\ifttm\special{html:<!-- mmref;;}#1\special{html:;;0; //-->}\else\ref{#1}\fi}
\def\MSRef#1#2{\ifttm\special{html:<!-- msref;;}#1\special{html:;;}#2\special{html:; //-->}\else \if#2\empty \ref{#1} \else \hyperref[#1]{#2}\fi\fi} 

\def\MRefRange#1#2{\ifttm\MRef{#1} bis 
\MRef{#2}\else\vrefrange[\unskip]{#1}{#2}\fi}

\def\MRefTwo#1#2{\ifttm\MRef{#1} und \MRef{#2}\else%
\let\vRefTLRsav=\reftextlabelrange\let\vRefTPRsav=\reftextpagerange%
\def\reftextlabelrange##1##2{\ref{##1} und~\ref{##2}}%
\def\reftextpagerange##1##2{auf den Seiten~\pageref{#1} und~\pageref{#2}}%
\vrefrange[\unskip]{#1}{#2}%
\let\reftextlabelrange=\vRefTLRsav\let\reftextpagerange=\vRefTPRsav\fi}

% MSectionChapter definiert falls notwendig das Kapitel vor der section. Das ist notwendig, wenn nur ein Einzelmodul uebersetzt wird.
% MChaptersGiven ist ein Counter, der von mconvert.pl vordefiniert wird.
\ifttm
\newcommand{\MSectionChapter}{\ifnum\value{MChaptersGiven}=0{\Dchapter{Modul}}\else{}\fi}
\else
\newcommand{\MSectionChapter}{\ifnum\value{chapter}=0{\Dchapter{Modul}}\else{}\fi}
\fi


\def\MChapter#1{\ifnum\value{MSSEnd}>0{\MSubsectionEndMacros}\addtocounter{MSSEnd}{-1}\fi\Dchapter{#1}}
\def\MSubject#1{\MChapter{#1}} % Schluesselwort HELPSECTION ist reserviert fuer Hilfesektion

\newcommand{\MSectionID}{UNKNOWNID}

\ifttm
\newcommand{\MSetSectionID}[1]{\renewcommand{\MSectionID}{#1}}
\else
\newcommand{\MSetSectionID}[1]{\renewcommand{\MSectionID}{#1}\tikzsetexternalprefix{#1}}
\fi


\newcommand{\MSection}[1]{\MSetSectionID{MODULID}\ifnum\value{MSSEnd}>0{\MSubsectionEndMacros}\addtocounter{MSSEnd}{-1}\fi\MSectionChapter\Dsection{#1}\MSectionStartMacros{#1}\setcounter{MLastIndex}{-1}\setcounter{MLastType}{1}} % Sections werden ueber das section-Feld im mmlabel-Tag identifiziert, nicht ueber das Indexfeld

\def\MSubsection#1{\ifnum\value{MSSEnd}>0{\MSubsectionEndMacros}\addtocounter{MSSEnd}{-1}\fi\ifttm\else\clearpage\fi\Dsubsection{#1}\MSubsectionStartMacros\setcounter{MLastIndex}{-1}\setcounter{MLastType}{2}\addtocounter{MSSEnd}{1}}% Subsections werden ueber das subsection-Feld im mmlabel-Tag identifiziert, nicht ueber das Indexfeld
\def\MSubsectionx#1{\Dsubsectionx{#1}} % Nur zur Verwendung in MSectionStart gedacht
\def\MSubsubsection#1{\Dsubsubsection{#1}\setcounter{MLastIndex}{\value{subsubsection}}\setcounter{MLastType}{3}\ifttm\special{html:<!-- sectioninfo;;}\arabic{section}\special{html:;;}\arabic{subsection}\special{html:;;}\arabic{subsubsection}\special{html:;;1;;}\arabic{MTestSite}\special{html:; //-->}\fi}
\def\MSubsubsectionx#1{\Dsubsubsectionx{#1}\ifttm\special{html:<!-- sectioninfo;;}\arabic{section}\special{html:;;}\arabic{subsection}\special{html:;;}\arabic{subsubsection}\special{html:;;0;;}\arabic{MTestSite}\special{html:; //-->}\else\addcontentsline{toc}{subsection}{#1}\fi}

\ifttm
\def\MSubsubsubsectionx#1{\ \newline\textbf{#1}\special{html:<br />}}
\else
\def\MSubsubsubsectionx#1{\ \newline
\textbf{#1}\ \\
}
\fi


% Dieses Skript wird zu Beginn jedes Modulabschnitts (=Webseite) ausgefuehrt und initialisiert den Aufgabenfeldzaehler
\newcommand{\MPageScripts}{
\setcounter{MFieldCounter}{1}
\addtocounter{MSiteCounter}{1}
\setcounter{MHintCounter}{1}
\setcounter{MCodeEditCounter}{1}
\setcounter{MGroupActive}{0}
\DoQBoxes
% Feldvariablen werden im HTML-Header in conv.pl eingestellt
}

% Dieses Skript wird zum Ende jedes Modulabschnitts (=Webseite) ausgefuehrt
\ifttm
\newcommand{\MEndScripts}{\special{html:<br /><!-- mfeedbackbutton;Seite;}\arabic{MTestSite}\special{html:;}\MGenerateSiteNumber\special{html:; //-->}
}
\else
\newcommand{\MEndScripts}{\relax}
\fi


\newcounter{QBoxFlag}
\newcommand{\DoQBoxes}{\setcounter{QBoxFlag}{1}}
\newcommand{\NoQBoxes}{\setcounter{QBoxFlag}{0}}

\newcounter{MXCTest}
\newcounter{MXCounter}
\newcounter{MSCounter}



\ifttm

% Struktur des sectioninfo-Tags: <!-- sectioninfo;;section;;subsection;;subsubsection;;nr_ausgeben;;testpage; //-->

%Fuegt eine zusaetzliche html-Seite an hinter ALLEN bisherigen und zukuenftigen content-Seiten ausserhalb der vor-zurueck-Schleife (d.h. nur durch Button oder MIntLink erreichbar!)
% #1 = Titel des Modulabschnitts, #2 = Kurztitel fuer die Buttons, #3 = Buttonkennung (STD = default nehmen, NONE = Ohne Button in der Navigation)
\newenvironment{MSContent}[3]{\special{html:<div class="xcontent}\arabic{MSCounter}\special{html:"><!-- scontent;-;}\arabic{MSCounter};-;#1;-;#2;-;#3\special{html: //-->}\MPageScripts\MSubsubsectionx{#1}}{\MEndScripts\special{html:<!-- endscontent;;}\arabic{MSCounter}\special{html: //--></div>}\addtocounter{MSCounter}{1}}

% Fuegt eine zusaetzliche html-Seite ein hinter den bereits vorhandenen content-Seiten (oder als erste Seite) innerhalb der vor-zurueck-Schleife der Navigation
% #1 = Titel des Modulabschnitts, #2 = Kurztitel fuer die Buttons, #3 = Buttonkennung (STD = Defaultbutton, NONE = Ohne Button in der Navigation)
\newenvironment{MXContent}[3]{\special{html:<div class="xcontent}\arabic{MXCounter}\special{html:"><!-- xcontent;-;}\arabic{MXCounter};-;#1;-;#2;-;#3\special{html: //-->}\MPageScripts\MSubsubsection{#1}}{\MEndScripts\special{html:<!-- endxcontent;;}\arabic{MXCounter}\special{html: //--></div>}\addtocounter{MXCounter}{1}}

% Fuegt eine zusaetzliche html-Seite ein die keine subsubsection-Nummer bekommt, nur zur internen Verwendung in mintmod.tex gedacht!
% #1 = Titel des Modulabschnitts, #2 = Kurztitel fuer die Buttons, #3 = Buttonkennung (STD = Defaultbutton, NONE = Ohne Button in der Navigation)
% \newenvironment{MUContent}[3]{\special{html:<div class="xcontent}\arabic{MXCounter}\special{html:"><!-- xcontent;-;}\arabic{MXCounter};-;#1;-;#2;-;#3\special{html: //-->}\MPageScripts\MSubsubsectionx{#1}}{\MEndScripts\special{html:<!-- endxcontent;;}\arabic{MXCounter}\special{html: //--></div>}\addtocounter{MXCounter}{1}}

\newcommand{\MDeclareSiteUXID}[1]{\special{html:<!-- mdeclaresiteuxid;;}#1\special{html:;;}\arabic{chapter}\special{html:;;}\arabic{section}\special{html:;; //-->}}

\else

%\newcommand{\MSubsubsection}[1]{\refstepcounter{subsubsection} \addcontentsline{toc}{subsubsection}{\thesubsubsection. #1}}


% Fuegt eine zusaetzliche html-Seite an hinter den bereits vorhandenen content-Seiten
% #1 = Titel des Modulabschnitts, #2 = Kurztitel fuer die Buttons, #3 = Iconkennung (im PDF wirkungslos)
%\newenvironment{MUContent}[3]{\ifnum\value{MXCTest}>0{\MDebugMessage{ERROR: Geschachtelter SContent}}\fi\MPageScripts\MSubsubsectionx{#1}\addtocounter{MXCTest}{1}}{\addtocounter{MXCounter}{1}\addtocounter{MXCTest}{-1}}
\newenvironment{MXContent}[3]{\ifnum\value{MXCTest}>0{\MDebugMessage{ERROR: Geschachtelter SContent}}\fi\MPageScripts\MSubsubsection{#1}\addtocounter{MXCTest}{1}}{\addtocounter{MXCounter}{1}\addtocounter{MXCTest}{-1}}
\newenvironment{MSContent}[3]{\ifnum\value{MXCTest}>0{\MDebugMessage{ERROR: Geschachtelter XContent}}\fi\MPageScripts\MSubsubsectionx{#1}\addtocounter{MXCTest}{1}}{\addtocounter{MSCounter}{1}\addtocounter{MXCTest}{-1}}

\newcommand{\MDeclareSiteUXID}[1]{\relax}

\fi 

% GHEADER und GFOOTER werden von split.pm gefunden, aber nur, wenn nicht HELPSITE oder TESTSITE
\ifttm
\newenvironment{MSectionStart}{\special{html:<div class="xcontent0">}\MSubsubsectionx{Modul\"ubersicht}}{\setcounter{MSSEnd}{0}\special{html:</div>}}
% Darf nicht als XContent nummeriert werden, darf nicht als XContent gelabelt werden, wird aber in eine xcontent-div gesetzt fuer Python-parsing
\else
\newenvironment{MSectionStart}{\MSubsectionx{Modul\"ubersicht}}{\setcounter{MSSEnd}{0}}
\fi

\newenvironment{MIntro}{\begin{MXContent}{Einf\"uhrung}{Einf\"uhrung}{genetisch}}{\end{MXContent}}
\newenvironment{MContent}{\begin{MXContent}{Inhalt}{Inhalt}{beweis}}{\end{MXContent}}
\newenvironment{MExercises}{\ifttm\else\clearpage\fi\begin{MXContent}{Aufgaben}{Aufgaben}{aufgb}\special{html:<!-- declareexcsymb //-->}}{\end{MXContent}}

% #1 = Lesbare Testbezeichnung
\newenvironment{MTest}[1]{%
\renewcommand{\MTestName}{#1}
\ifttm\else\clearpage\fi%
\addtocounter{MTestSite}{1}%
\begin{MXContent}{#1}{#1}{STD} % {aufgb}%
\special{html:<!-- declaretestsymb //-->}
\begin{MQuestionGroup}%
\MInTestHeader
}%
{%
\end{MQuestionGroup}%
\ \\ \ \\%
\MInTestFooter
\end{MXContent}\addtocounter{MTestSite}{-1}%
}

\newenvironment{MExtra}{\ifttm\else\clearpage\fi\begin{MXContent}{Zus\"atzliche Inhalte}{Zusatz}{weiterfhrg}}{\end{MXContent}}

\makeindex

\ifttm
\def\MPrintIndex{
\ifnum\value{MSSEnd}>0{\MSubsectionEndMacros}\addtocounter{MSSEnd}{-1}\fi
\renewcommand{\indexname}{Stichwortverzeichnis}
\special{html:<p><!-- printindex //--></p>}
}
\else
\def\MPrintIndex{
\ifnum\value{MSSEnd}>0{\MSubsectionEndMacros}\addtocounter{MSSEnd}{-1}\fi
\renewcommand{\indexname}{Stichwortverzeichnis}
\addcontentsline{toc}{section}{Stichwortverzeichnis}
\printindex
}
\fi


% Konstanten fuer die Modulfaecher

\def\MINTMathematics{1}
\def\MINTInformatics{2}
\def\MINTChemistry{3}
\def\MINTPhysics{4}
\def\MINTEngineering{5}

\newcounter{MSubjectArea}
\newcounter{MInfoNumbers} % Gibt an, ob die Infoboxen nummeriert werden sollen
\newcounter{MSepNumbers} % Gibt an, ob Beispiele und Experimente separat nummeriert werden sollen
\newcommand{\MSetSubject}[1]{
 % ttm kapiert setcounter mit Parametern nicht, also per if abragen und einsetzen
\ifnum#1=1\setcounter{MSubjectArea}{1}\setcounter{MInfoNumbers}{1}\setcounter{MSepNumbers}{0}\fi
\ifnum#1=2\setcounter{MSubjectArea}{2}\setcounter{MInfoNumbers}{1}\setcounter{MSepNumbers}{0}\fi
\ifnum#1=3\setcounter{MSubjectArea}{3}\setcounter{MInfoNumbers}{0}\setcounter{MSepNumbers}{1}\fi
\ifnum#1=4\setcounter{MSubjectArea}{4}\setcounter{MInfoNumbers}{0}\setcounter{MSepNumbers}{0}\fi
\ifnum#1=5\setcounter{MSubjectArea}{5}\setcounter{MInfoNumbers}{1}\setcounter{MSepNumbers}{0}\fi
% Separate Nummerntechnik fuer unsere Chemiker: alles dreistellig
\ifnum#1=3
  \ifttm
  \renewcommand{\theequation}{\arabic{section}.\arabic{subsection}.\arabic{equation}}
  \renewcommand{\thetable}{\arabic{section}.\arabic{subsection}.\arabic{table}} 
  \renewcommand{\thefigure}{\arabic{section}.\arabic{subsection}.\arabic{figure}} 
  \else
  \renewcommand{\theequation}{\arabic{chapter}.\arabic{section}.\arabic{equation}}
  \renewcommand{\thetable}{\arabic{chapter}.\arabic{section}.\arabic{table}}
  \renewcommand{\thefigure}{\arabic{chapter}.\arabic{section}.\arabic{figure}}
  \fi
\else
  \ifttm
  \renewcommand{\theequation}{\arabic{section}.\arabic{subsection}.\arabic{equation}}
  \renewcommand{\thetable}{\arabic{table}}
  \renewcommand{\thefigure}{\arabic{figure}}
  \else
  \renewcommand{\theequation}{\arabic{chapter}.\arabic{section}.\arabic{equation}}
  \renewcommand{\thetable}{\arabic{table}}
  \renewcommand{\thefigure}{\arabic{figure}}
  \fi
\fi
}

% Fuer tikz Autogenerierung
\newcounter{MTIKZAutofilenumber}

% Spezielle Counter fuer die Bentz-Module
\newcounter{mycounter}
\newcounter{chemapplet}
\newcounter{physapplet}

\newcounter{MSSEnd} % Ist 1 falls ein MSubsection aktiv ist, der einen MSubsectionEndMacro-Aufruf verursacht
\newcounter{MFileNumber}
\def\MLastFile{\special{html:[[!-- mfileref;;}\arabic{MFileNumber}\special{html:; //--]]}}

% Vollstaendiger Pfad ist \MMaterial / \MLastFilePath / \MLastFileName    ==   \MMaterial / \MLastFile

% Wird nur bei kompletter Baum-Erstellung ausgefuehrt!
% #1 = Lesbare Modulbezeichnung
\newcommand{\MSectionStartMacros}[1]{
\setcounter{MTestSite}{0}
\setcounter{MCaptionOn}{0}
\setcounter{MLastTypeEq}{0}
\setcounter{MSSEnd}{0}
\setcounter{MFileNumber}{0} % Preinkrekement-Counter
\setcounter{MTIKZAutofilenumber}{0}
\setcounter{mycounter}{1}
\setcounter{physapplet}{1}
\setcounter{chemapplet}{0}
\ifttm
\special{html:<!-- mdeclaresection;;}\arabic{chapter}\special{html:;;}\arabic{section}\special{html:;;}#1\special{html:;; //-->}%
\else
\setcounter{thmc}{0}
\setcounter{exmpc}{0}
\setcounter{verc}{0}
\setcounter{infoc}{0}
\fi
\setcounter{MiniMarkerCounter}{1}
\setcounter{AlignCounter}{1}
\setcounter{MXCTest}{0}
\setcounter{MCodeCounter}{0}
\setcounter{MEntryCounter}{0}
}

% Wird immer ausgefuehrt
\newcommand{\MSubsectionStartMacros}{
\ifttm\else\MPageHeaderDef\fi
\MWatermarkSettings
\setcounter{MXCounter}{0}
\setcounter{MSCounter}{0}
\setcounter{MSiteCounter}{1}
\setcounter{MExerciseCollectionCounter}{0}
% Zaehler fuer das Labelsystem zuruecksetzen (prefix-Zaehler)
\setcounter{MInfoCounter}{0}
\setcounter{MExerciseCounter}{0}
\setcounter{MExampleCounter}{0}
\setcounter{MExperimentCounter}{0}
\setcounter{MGraphicsCounter}{0}
\setcounter{MTableCounter}{0}
\setcounter{MTheoremCounter}{0}
\setcounter{MObjectCounter}{0}
\setcounter{MEquationCounter}{0}
\setcounter{MVideoCounter}{0}
\setcounter{equation}{0}
\setcounter{figure}{0}
}

\newcommand{\MSubsectionEndMacros}{
% Bei Chemiemodulen das PSE einhaengen, es soll als SContent am Ende erscheinen
\special{html:<!-- subsectionend //-->}
\ifnum\value{MSubjectArea}=3{\MIncludePSE}\fi
}


\ifttm
%\newcommand{\MEmbed}[1]{\MRegisterFile{#1}\begin{html}<embed src="\end{html}\MMaterial/\MLastFile\begin{html}" width="192" height="189"></embed>\end{html}}
\newcommand{\MEmbed}[1]{\MRegisterFile{#1}\begin{html}<embed src="\end{html}\MMaterial/\MLastFile\begin{html}"></embed>\end{html}}
\fi

%----------------- Makros fuer die Textdarstellung -----------------------------------------------

\ifttm
% MUGraphics bindet eine Grafik ein:
% Parameter 1: Dateiname der Grafik, relativ zur Position des Modul-Tex-Dokuments
% Parameter 2: Skalierungsoptionen fuer PDF (fuer includegraphics)
% Parameter 3: Titel fuer die Grafik, wird unter die Grafik mit der Grafiknummer gesetzt und kann MLabel bzw. MCopyrightLabel enthalten
% Parameter 4: Skalierungsoptionen fuer HTML (css-styles)

% ERSATZ: <img alt="My Image" src="data:image/png;base64,iVBORwA<MoreBase64SringHere>" />


\newcommand{\MUGraphics}[4]{\MRegisterFile{#1}\begin{html}
<div class="imagecenter">
<center>
<div>
<img src="\end{html}\MMaterial/\MLastFile\begin{html}" style="#4" alt="\end{html}\MMaterial/\MLastFile\begin{html}"/>
</div>
<div class="bildtext">
\end{html}
\addtocounter{MGraphicsCounter}{1}
\setcounter{MLastIndex}{\value{MGraphicsCounter}}
\setcounter{MLastType}{8}
\addtocounter{MCaptionOn}{1}
\ifnum\value{MSepNumbers}=0
\textbf{Abbildung \arabic{MGraphicsCounter}:} #3
\else
\textbf{Abbildung \arabic{section}.\arabic{subsection}.\arabic{MGraphicsCounter}:} #3
\fi
\addtocounter{MCaptionOn}{-1}
\begin{html}
</div>
</center>
</div>
<br />
\end{html}%
\special{html:<!-- mfeedbackbutton;Abbildung;}\arabic{MGraphicsCounter}\special{html:;}\arabic{section}.\arabic{subsection}.\arabic{MGraphicsCounter}\special{html:; //-->}%
}

% MVideo bindet ein Video als Einzeldatei ein:
% Parameter 1: Dateiname des Videos, relativ zur Position des Modul-Tex-Dokuments, ohne die Endung ".mp4"
% Parameter 2: Titel fuer das Video (kann MLabel oder MCopyrightLabel enthalten), wird unter das Video mit der Videonummer gesetzt
\newcommand{\MVideo}[2]{\MRegisterFile{#1.mp4}\begin{html}
<div class="imagecenter">
<center>
<div>
<video width="95\%" controls="controls"><source src="\end{html}\MMaterial/#1.mp4\begin{html}" type="video/mp4">Ihr Browser kann keine MP4-Videos abspielen!</video>
</div>
<div class="bildtext">
\end{html}
\addtocounter{MVideoCounter}{1}
\setcounter{MLastIndex}{\value{MVideoCounter}}
\setcounter{MLastType}{12}
\addtocounter{MCaptionOn}{1}
\ifnum\value{MSepNumbers}=0
\textbf{Video \arabic{MVideoCounter}:} #2
\else
\textbf{Video \arabic{section}.\arabic{subsection}.\arabic{MVideoCounter}:} #2
\fi
\addtocounter{MCaptionOn}{-1}
\begin{html}
</div>
</center>
</div>
<br />
\end{html}}

\newcommand{\MDVideo}[2]{\MRegisterFile{#1.mp4}\MRegisterFile{#1.ogv}\begin{html}
<div class="imagecenter">
<center>
<div>
<video width="70\%" controls><source src="\end{html}\MMaterial/#1.mp4\begin{html}" type="video/mp4"><source src="\end{html}\MMaterial/#1.ogv\begin{html}" type="video/ogg">Ihr Browser kann keine MP4-Videos abspielen!</video>
</div>
<br />
#2
</center>
</div>
<br />
\end{html}
}

\newcommand{\MGraphics}[3]{\MUGraphics{#1}{#2}{#3}{}}

\else

\newcommand{\MVideo}[2]{%
% Kein Video im PDF darstellbar, trotzdem so tun als ob da eines waere
\begin{center}
(Video nicht darstellbar)
\end{center}
\addtocounter{MVideoCounter}{1}
\setcounter{MLastIndex}{\value{MVideoCounter}}
\setcounter{MLastType}{12}
\addtocounter{MCaptionOn}{1}
\ifnum\value{MSepNumbers}=0
\textbf{Video \arabic{MVideoCounter}:} #2
\else
\textbf{Video \arabic{section}.\arabic{subsection}.\arabic{MVideoCounter}:} #2
\fi
\addtocounter{MCaptionOn}{-1}
}


% MGraphics bindet eine Grafik ein:
% Parameter 1: Dateiname der Grafik, relativ zur Position des Modul-Tex-Dokuments
% Parameter 2: Skalierungsoptionen fuer PDF (fuer includegraphics)
% Parameter 3: Titel fuer die Grafik, wird unter die Grafik mit der Grafiknummer gesetzt
\newcommand{\MGraphics}[3]{%
\MRegisterFile{#1}%
\ %
\begin{figure}[H]%
\centering{%
\includegraphics[#2]{\MDPrefix/#1}%
\addtocounter{MCaptionOn}{1}%
\caption{#3}%
\addtocounter{MCaptionOn}{-1}%
}%
\end{figure}%
\addtocounter{MGraphicsCounter}{1}\setcounter{MLastIndex}{\value{MGraphicsCounter}}\setcounter{MLastType}{8}\ %
%\ \\Abbildung \ifnum\value{MSepNumbers}=0\else\arabic{chapter}.\arabic{section}.\fi\arabic{MGraphicsCounter}: #3%
}

\newcommand{\MUGraphics}[4]{\MGraphics{#1}{#2}{#3}}


\fi

\newcounter{MCaptionOn} % = 1 falls eine Grafikcaption aktiv ist, = 0 sonst


% MGraphicsSolo bindet eine Grafik pur ein ohne Titel
% Parameter 1: Dateiname der Grafik, relativ zur Position des Modul-Tex-Dokuments
% Parameter 2: Skalierungsoptionen (wirken nur im PDF)
\newcommand{\MGraphicsSolo}[2]{\MUGraphicsSolo{#1}{#2}{}}

% MUGraphicsSolo bindet eine Grafik pur ein ohne Titel, aber mit HTML-Skalierung
% Parameter 1: Dateiname der Grafik, relativ zur Position des Modul-Tex-Dokuments
% Parameter 2: Skalierungsoptionen (wirken nur im PDF)
% Parameter 3: Skalierungsoptionen (wirken nur im HTML), als style-format: "width=???, height=???"
\ifttm
\newcommand{\MUGraphicsSolo}[3]{\MRegisterFile{#1}\begin{html}
<img src="\end{html}\MMaterial/\MLastFile\begin{html}" style="\end{html}#3\begin{html}" alt="\end{html}\MMaterial/\MLastFile\begin{html}"/>
\end{html}%
\special{html:<!-- mfeedbackbutton;Abbildung;}#1\special{html:;}\MMaterial/\MLastFile\special{html:; //-->}%
}
\else
\newcommand{\MUGraphicsSolo}[3]{\MRegisterFile{#1}\includegraphics[#2]{\MDPrefix/#1}}
\fi

% Externer Link mit URL
% Erster Parameter: Vollstaendige(!) URL des Links
% Zweiter Parameter: Text fuer den Link
\newcommand{\MExtLink}[2]{\ifttm\special{html:<a target="_new" href="}#1\special{html:">}#2\special{html:</a>}\else\href{#1}{#2}\fi} % ohne MINTERLINK!


% Interner Link, die verlinkte Datei muss im gleichen Verzeichnis liegen wie die Modul-Texdatei
% Erster Parameter: Dateiname
% Zweiter Parameter: Text fuer den Link
\newcommand{\MIntLink}[2]{\ifttm\MRegisterFile{#1}\special{html:<a class="MINTERLINK" target="_new" href="}\MMaterial/\MLastFile\special{html:">}#2\special{html:</a>}\else{\href{#1}{#2}}\fi}


\ifttm
\def\MMaterial{:localmaterial:}
\else
\def\MMaterial{\MDPrefix}
\fi

\ifttm
\def\MNoFile#1{:directmaterial:#1}
\else
\def\MNoFile#1{#1}
\fi

\newcommand{\MChem}[1]{$\mathrm{#1}$}

\newcommand{\MApplet}[3]{
% Bindet ein Java-Applet ein, die Parameter sind:
% (wird nur im HTML, aber nicht im PDF erstellt)
% #1 Dateiname des Applets (muss mit ".class" enden)
% #2 = Breite in Pixeln
% #3 = Hoehe in Pixeln
\ifttm
\MRegisterFile{#1}
\begin{html}
<applet code="\end{html}\MMaterial/\MLastFile\begin{html}" width="#2" height="#3" alt="[Java-Applet kann nicht gestartet werden]"></applet>
\end{html}
\fi
}

\newcommand{\MScriptPage}[2]{
% Bindet eine JavaScript-Datei ein, die eine eigene Seite bekommt
% (wird nur im HTML, aber nicht im PDF erstellt)
% #1 Dateiname des Programms (sollte mit ".js" enden)
% #2 = Kurztitel der Seite
\ifttm
\begin{MSContent}{#2}{#2}{puzzle}
\MRegisterFile{#1}
\begin{html}
<script src="\MMaterial/\MLastFile" type="text/javascript"></script>
\end{html}
\end{MSContent}
\fi
}

\newcommand{\MIncludePSE}{
% Bindet bei Chemie-Modulen das PSE ein
% (wird nur im HTML, aber nicht im PDF erstellt)
\ifttm
\special{html:<!-- includepse //-->}
\begin{MSContent}{Periodensystem der Elemente}{PSE}{table}
\MRegisterFile{../files/pse.js}
\MRegisterFile{../files/radio.png}
\begin{html}
<script src="\MMaterial/../files/pse.js" type="text/javascript"></script>
<p id="divid"><br /><br />
<script language="javascript" type="text/javascript">
    startpse("divid","\MMaterial/../files"); 
</script>
</p>
<br />
<br />
<br />
<p>Die Farben der Elementsymbole geben an: <font style="color:Red">gasf&ouml;rmig </font> <font style="color:Blue">fl&uuml;ssig </font> fest</p>
<p>Die Elemente der Gruppe 1 A, 2 A, 3 A usw. geh&ouml;ren zu den Hauptgruppenelementen.</p>
<p>Die Elemente der Gruppe 1 B, 2 B, 3 B usw. geh&ouml;ren zu den Nebengruppenelementen.</p>
<p>() kennzeichnet die Masse des stabilsten Isotops</p>
\end{html}
\end{MSContent}
\fi
}

\newcommand{\MAppletArchive}[4]{
% Bindet ein Java-Applet ein, die Parameter sind:
% (wird nur im HTML, aber nicht im PDF erstellt)
% #1 Dateiname der Klasse mit Appletaufruf (muss mit ".class" enden)
% #2 Dateiname des Archivs (muss mit ".jar" enden)
% #3 = Breite in Pixeln
% #4 = Hoehe in Pixeln
\ifttm
\MRegisterFile{#2}
\begin{html}
<applet code="#1" archive="\end{html}\MMaterial/\MLastFile\begin{html}" codebase="." width="#3" height="#4" alt="[Java-Archiv kann nicht gestartet werden]"></applet>
\end{html}
\fi
}

% Bindet in der Haupttexdatei ein MINT-Modul ein. Parameter 1 ist das Verzeichnis (relativ zur Haupttexdatei), Parameter 2 ist der Dateinahme ohne Pfad.
\newcommand{\IncludeModule}[2]{
\renewcommand{\MDPrefix}{#1}
\input{#1/#2}
\ifnum\value{MSSEnd}>0{\MSubsectionEndMacros}\addtocounter{MSSEnd}{-1}\fi
}

% Der ttm-Konverter setzt keine Makros im \input um, also muss hier getrickst werden:
% Das MDPrefix muss in den einzelnen Modulen manuell eingesetzt werden
\newcommand{\MInputFile}[1]{
\ifttm
\input{#1}
\else
\input{#1}
\fi
}


\newcommand{\MCases}[1]{\left\lbrace{\begin{array}{rl} #1 \end{array}}\right.}

\ifttm
\newenvironment{MCaseEnv}{\left\lbrace\begin{array}{rl}}{\end{array}\right.}
\else
\newenvironment{MCaseEnv}{\left\lbrace\begin{array}{rl}}{\end{array}\right.}
\fi

\def\MSkip{\ifttm\MCR\fi}

\ifttm
\def\MCR{\special{html:<br />}}
\else
\def\MCR{\ \\}
\fi


% Pragmas - Sind Schluesselwoerter, die dem Preprocessing sowie dem Konverter uebergeben werden und bestimmte
%           Aktionen ausloesen. Im Output (PDF und HTML) tauchen sie nicht auf.
\newcommand{\MPragma}[1]{%
\ifttm%
\special{html:<!-- mpragma;-;}#1\special{html:;; -->}%
\else%
% MPragmas werden vom Preprozessor direkt im LaTeX gefunden
\fi%
}

% Ersatz der Befehle textsubscript und textsuperscript, die ttm nicht kennt
\ifttm%
\newcommand{\MTextsubscript}[1]{\special{html:<sub>}#1\special{html:</sub>}}%
\newcommand{\MTextsuperscript}[1]{\special{html:<sup>}#1\special{html:</sup>}}%
\else%
\newcommand{\MTextsubscript}[1]{\textsubscript{#1}}%
\newcommand{\MTextsuperscript}[1]{\textsuperscript{#1}}%
\fi

%------------------ Einbindung von dia-Diagrammen ----------------------------------------------
% Beim preprocessing wird aus jeder dia-Datei eine tex-Datei und eine pdf-Datei erzeugt,
% diese werden hier jeweils im PDF und HTML eingebunden
% Parameter: Dateiname der mit dia erstellten Datei (OHNE die Endung .dia)
\ifttm%
\newcommand{\MDia}[1]{%
\MGraphicsSolo{#1minthtml.png}{}%
}
\else%
\newcommand{\MDia}[1]{%
\MGraphicsSolo{#1mintpdf.png}{scale=0.1667}%
}
\fi%

% subsup funktioniert im Ausdruck $D={\R}^+_0$, also \R geklammert und sup zuerst
% \ifttm
% \def\MSubsup#1#2#3{\special{html:<msubsup>} #1 #2 #3\special{html:</msubsup>}}
% \else
% \def\MSubsup#1#2#3{{#1}^{#3}_{#2}}
% \fi

%\input{local.tex}

% \ifttm
% \else
% \newwrite\mintlog
% \immediate\openout\mintlog=mintlog.txt
% \fi

% ----------------------- tikz autogenerator -------------------------------------------------------------------

\newcommand{\Mtikzexternalize}{\tikzexternalize}% wird bei Konvertierung ueber mconvert ggf. ausgehebelt!

\ifttm
\else
\tikzset%
{
  % Defines a custom style which generates pdf and converts to (low and hi-res quality) png and svg, then deletes the pdf
  % Important: DO NOT directly convert from pdf to hires-png or from svg to png with GraphViz convert as it has some problems and memory leaks
  png export/.style=%
  {
    external/system call/.add={}{; 
      pdf2svg "\image.pdf" "\image.svg" ; 
      convert -density 112.5 -transparent white "\image.pdf" "\image.png"; 
      inkscape --export-png="\image.4x.png" --export-dpi=450 --export-background-opacity=0 --without-gui "\image.svg"; 
      rm "\image.pdf"; rm "\image.log"; rm "\image.dpth"; rm "\image.idx"
    },
    external/force remake,
  }
}
\tikzset{png export}
\tikzsetexternalprefix{}
% PNGs bei externer Erzeugung in "richtiger" Groesse einbinden
\pgfkeys{/pgf/images/include external/.code={\includegraphics[scale=0.64]{#1}}}
\fi

% Spezielle Umgebung fuer Autogenerierung, Bildernamen sind nur innerhalb eines Moduls (einer MSection) eindeutig)

\newcommand{\MTIKZautofilename}{tikzautofile}

\ifttm
% HTML-Version: Vom Autogenerator erzeugte png-Datei einbinden, tikz selbst nicht ausfuehren (sprich: #1 schlucken)
\newcommand{\MTikzAuto}[1]{%
\addtocounter{MTIKZAutofilenumber}{1}
\renewcommand{\MTIKZautofilename}{mtikzauto_\arabic{MTIKZAutofilenumber}}
\MUGraphicsSolo{\MSectionID\MTIKZautofilename.4x.png}{scale=1}{\special{html:[[!-- svgstyle;}\MSectionID\MTIKZautofilename\special{html: //--]]}} % Styleinfos werden aus original-png, nicht 4x-png geholt!
%\MRegisterFile{\MSectionID\MTIKZautofilename.png} % not used right now
%\MRegisterFile{\MSectionID\MTIKZautofilename.svg}
}
\else%
% PDF-Version: Falls Autogenerator aktiv wird Datei automatisch benannt und exportiert
\newcommand{\MTikzAuto}[1]{%
\addtocounter{MTIKZAutofilenumber}{1}%
\renewcommand{\MTIKZautofilename}{mtikzauto_\arabic{MTIKZAutofilenumber}}
\tikzsetnextfilename{\MTIKZautofilename}%
#1%
}
\fi

% In einer reinen LaTeX-Uebersetzung kapselt der Preambelinclude-Befehl nur input,
% in einer konvertergesteuerten PDF/HTML-Uebersetzung wird er dagegen entfernt und
% die Preambeln an mintmod angehaengt, die Ersetzung wird von mconvert.pl vorgenommen.

\newcommand{\MPreambleInclude}[1]{\input{#1}}

% Globale Watermarksettings (werden auch nochmal zu Beginn jedes subsection gesetzt,
% muessen hier aber auch global ausgefuehrt wegen Einfuehrungsseiten und Inhaltsverzeichnis

\MWatermarkSettings
% ---------------------------------- Parametrisierte Aufgaben ----------------------------------------

\ifttm
\newenvironment{MPExercise}{%
\begin{MExercise}%
}{%
\special{html:<button name="Name_MPEX}\arabic{MExerciseCounter}\special{html:" id="MPEX}\arabic{MExerciseCounter}%
\special{html:" type="button" onclick="reroll('}\arabic{MExerciseCounter}\special{html:');">Neue Aufgabe erzeugen</button>}%
\end{MExercise}%
}
\else
\newenvironment{MPExercise}{%
\begin{MExercise}%
}{%
\end{MExercise}%
}
\fi

% Parameter: Name, Min, Max, PDF-Standard. Name in Deklaration OHNE backslash, im Code MIT Backslash
\ifttm
\newcommand{\MGlobalInteger}[4]{\special{html:%
<!-- onloadstart //-->%
MVAR.push(createGlobalInteger("}#1\special{html:",}#2\special{html:,}#3\special{html:,}#4\special{html:)); %
<!-- onloadstop //-->%
<!-- viewmodelstart //-->%
ob}#1\special{html:: ko.observable(rerollMVar("}#1\special{html:")),%
<!-- viewmodelstop //-->%
}%
}%
\else%
\newcommand{\MGlobalInteger}[4]{\newcounter{mvc_#1}\setcounter{mvc_#1}{#4}}
\fi

% Parameter: Name, Min, Max, PDF-Standard. Name in Deklaration OHNE backslash, im Code MIT Backslash, Wert ist Wurzel von value
\ifttm
\newcommand{\MGlobalSqrt}[4]{\special{html:%
<!-- onloadstart //-->%
MVAR.push(createGlobalSqrt("}#1\special{html:",}#2\special{html:,}#3\special{html:,}#4\special{html:)); %
<!-- onloadstop //-->%
<!-- viewmodelstart //-->%
ob}#1\special{html:: ko.observable(rerollMVar("}#1\special{html:")),%
<!-- viewmodelstop //-->%
}%
}%
\else%
\newcommand{\MGlobalSqrt}[4]{\newcounter{mvc_#1}\setcounter{mvc_#1}{#4}}% Funktioniert nicht als Wurzel !!!
\fi

% Parameter: Name, Min, Max, PDF-Standard zaehler, PDF-Standard nenner. Name in Deklaration OHNE backslash, im Code MIT Backslash
\ifttm
\newcommand{\MGlobalFraction}[5]{\special{html:%
<!-- onloadstart //-->%
MVAR.push(createGlobalFraction("}#1\special{html:",}#2\special{html:,}#3\special{html:,}#4\special{html:,}#5\special{html:)); %
<!-- onloadstop //-->%
<!-- viewmodelstart //-->%
ob}#1\special{html:: ko.observable(rerollMVar("}#1\special{html:")),%
<!-- viewmodelstop //-->%
}%
}%
\else%
\newcommand{\MGlobalFraction}[5]{\newcounter{mvc_#1}\setcounter{mvc_#1}{#4}} % Funktioniert nicht als Bruch !!!
\fi

% MVar darf im HTML nur in MEvalMathDisplay-Umgebungen genutzt werden oder in Strings die an den Parser uebergeben werden
\ifttm%
\newcommand{\MVar}[1]{\special{html:[var_}#1\special{html:]}}%
\else%
\newcommand{\MVar}[1]{\arabic{mvc_#1}}%
\fi

\ifttm%
\newcommand{\MRerollButton}[2]{\special{html:<button type="button" onclick="rerollMVar('}#1\special{html:');">}#2\special{html:</button>}}%
\else%
\newcommand{\MRerollButton}[2]{\relax}% Keine sinnvolle Entsprechung im PDF
\fi

% MEvalMathDisplay fuer HTML wird in mconvert.pl im preprocessing realisiert
% PDF: eine equation*-Umgebung (ueber amsmath)
% HTML: Eine Mathjax-Tex-Umgebung, deren Auswertung mit knockout-obervablen gekoppelt ist
% PDF-Version hier nur fuer pdflatex-only-Uebersetzung gegeben

\ifttm\else\newenvironment{MEvalMathDisplay}{\begin{equation*}}{\end{equation*}}\fi

% ---------------------------------- Spezialbefehle fuer AD ------------------------------------------

%Abk�rzung f�r \longrightarrow:
\newcommand{\lto}{\ensuremath{\longrightarrow}}

%Makro f�r Funktionen:
\newcommand{\exfunction}[5]
{\begin{array}{rrcl}
 #1 \colon  & #2 &\lto & #3 \\[.05cm]  
  & #4 &\longmapsto  & #5 
\end{array}}

\newcommand{\function}[5]{%
#1:\;\left\lbrace{\begin{array}{rcl}
 #2 &\lto & #3 \\
 #4 &\longmapsto  & #5 \end{array}}\right.}


%Die Identit�t:
\DeclareMathOperator{\Id}{Id}

%Die Signumfunktion:
\DeclareMathOperator{\sgn}{sgn}

%Zwei Betonungskommandos (k�nnen angepasst werden):
\newcommand{\highlight}[1]{#1}
\newcommand{\modstextbf}[1]{#1}
\newcommand{\modsemph}[1]{#1}


% ---------------------------------- Spezialbefehle fuer JL ------------------------------------------


\def\jccolorfkt{green!50!black} %Farbe des Funktionsgraphen
\def\jccolorfktarea{green!25!white} %Farbe der Fl"ache unter dem Graphen
\def\jccolorfktareahell{green!12!white} %helle Einf"arbung der Fl"ache unter dem Graphen
\def\jccolorfktwert{green!50!black} %Farbe einzelner Punkte des Graphen

\newcommand{\MPfadBilder}{Bilder}

\ifttm%
\newcommand{\jMD}{\,\MD}%
\else%
\newcommand{\jMD}{\;\MD}%
\fi%

\def\jHTMLHinweisBedienung{\MInputHint{%
Mit Hilfe der Symbole am oberen Rand des Fensters
k"onnen Sie durch die einzelnen Abschnitte navigieren.}}

\def\jHTMLHinweisEingabeText{\MInputHint{%
Geben Sie jeweils ein Wort oder Zeichen als Antwort ein.}}

\def\jHTMLHinweisEingabeTerm{\MInputHint{%
Klammern Sie Ihre Terme, um eine eindeutige Eingabe zu erhalten. 
Beispiel: Der Term $\frac{3x+1}{x-2}$ soll in der Form
\texttt{(3*x+1)/((x+2)^2}$ eingegeben werden (wobei auch Leerzeichen 
eingegeben werden k"onnen, damit eine Formel besser lesbar ist).}}

\def\jHTMLHinweisEingabeIntervalle{\MInputHint{%
Intervalle werden links mit einer "offnenden Klammer und rechts mit einer 
schlie"senden Klammer angegeben. Eine runde Klammer wird verwendet, wenn der 
Rand nicht dazu geh"ort, eine eckige, wenn er dazu geh"ort. 
Als Trennzeichen wird ein Komma oder ein Semikolon akzeptiert.
Beispiele: $(a, b)$ offenes Intervall,
$[a; b)$ links abgeschlossenes, rechts offenes Intervall von $a$ bis $b$. 
Die Eingabe $]a;b[$ f"ur ein offenes Intervall wird nicht akzeptiert.
F"ur $\infty$ kann \texttt{infty} oder \texttt{unendlich} geschrieben werden.}}

\def\jHTMLHinweisEingabeFunktionen{\MInputHint{%
Schreiben Sie Malpunkte (geschrieben als \texttt{*}) aus und setzen Sie Klammern um Argumente f�r Funktionen.
Beispiele: Polynom: \texttt{3*x + 0.1}, Sinusfunktion: \texttt{sin(x)}, 
Verkettung von cos und Wurzel: \texttt{cos(sqrt(3*x))}.}}

\def\jHTMLHinweisEingabeFunktionenSinCos{\MInputHint{%
Die Sinusfunktion $\sin x$ wird in der Form \texttt{sin(x)} angegeben, %
$\cos\left(\sqrt{3 x}\right)$ durch \texttt{cos(sqrt(3*x))}.}}

\def\jHTMLHinweisEingabeFunktionenExp{\MInputHint{%
Die Exponentialfunktion $\MEU^{3x^4 + 5}$ wird als
\texttt{exp(3 * x^4 + 5)} angegeben, %
$\ln\left(\sqrt{x} + 3.2\right)$ durch \texttt{ln(sqrt(x) + 3.2)}.}}

% ---------------------------------- Spezialbefehle fuer Fachbereich Physik --------------------------

\newcommand{\E}{{e}}
\newcommand{\ME}[1]{\cdot 10^{#1}}
\newcommand{\MU}[1]{\;\mathrm{#1}}
\newcommand{\MPG}[3]{%
  \ifnum#2=0%
    #1\ \mathrm{#3}%
  \else%
    #1\cdot 10^{#2}\ \mathrm{#3}%
  \fi}%
%

\newcommand{\MMul}{\MExponentensymbXYZl} % Nur eine Abkuerzung


% ---------------------------------- Stichwortfunktionialitaet ---------------------------------------

% mpreindexentry wird durch Auswahlroutine in conv.pl durch mindexentry substitutiert
\ifttm%
\def\MIndex#1{\index{#1}\special{html:<!-- mpreindexentry;;}#1\special{html:;;}\arabic{MSubjectArea}\special{html:;;}%
\arabic{chapter}\special{html:;;}\arabic{section}\special{html:;;}\arabic{subsection}\special{html:;;}\arabic{MEntryCounter}\special{html:; //-->}%
\setcounter{MLastIndex}{\value{MEntryCounter}}%
\addtocounter{MEntryCounter}{1}%
}%
% Copyrightliste wird als tex-Datei im preprocessing von conv.pl erzeugt und unter converter/tex/entrycollection.tex abgelegt
% Der input-Befehl funktioniert nur, wenn die aufrufende tex-Datei auf der obersten Ebene liegt (d.h. selbst kein input/include ist, insbesondere keine Moduldatei)
\def\MEntryList{} % \input funktioniert nicht, weil ttm (und damit das \input) ausgefuehrt wird, bevor Datei da ist
\else%
\def\MIndex#1{\index{#1}}
\def\MEntryList{\MAbort{Stichwortliste nur im HTML realisierbar}}%
\fi%

\def\MEntry#1#2{\textbf{#1}\MIndex{#2}} % Idee: MLastType auf neuen Entry-Typ und dann ein MLabel vergeben mit autogen-Nummer

% ---------------------------------- Befehle fuer Tests ----------------------------------------------

% MEquationItem stellt eine Eingabezeile der Form Vorgabe = Antwortfeld her, der zweite Parameter kann z.B. MSimplifyQuestion-Befehl sein
\ifttm
\newcommand{\MEquationItem}[2]{{#1}$\,=\,${#2}}%
\else%
\newcommand{\MEquationItem}[2]{{#1}$\;\;=\,${#2}}%
\fi

\ifttm
\newcommand{\MInputHint}[1]{%
\ifnum%
\if\value{MTestSite}>0%
\else%
{\color{blue}#1}%
\fi%
\fi%
}
\else
\newcommand{\MInputHint}[1]{\relax}
\fi

\ifttm
\newcommand{\MInTestHeader}{%
Dies ist ein einreichbarer Test:
\begin{itemize}
\item{Im Gegensatz zu den offenen Aufgaben werden beim Eingeben keine Hinweise zur Formulierung der mathematischen Ausdr�cke gegeben.}
\item{Der Test kann jederzeit neu gestartet oder verlassen werden.}
\item{Der Test kann durch die Buttons am Ende der Seite beendet und abgeschickt, oder zur�ckgesetzt werden.}
\item{Der Test kann mehrfach probiert werden. F�r die Statistik z�hlt die zuletzt abgeschickte Version.}
\end{itemize}
}
\else
\newcommand{\MInTestHeader}{%
\relax
}
\fi

\ifttm
\newcommand{\MInTestFooter}{%
\special{html:<button name="Name_TESTFINISH" id="TESTFINISH" type="button" onclick="finish_button('}\MTestName\special{html:');">Test auswerten</button>}%
\begin{html}
&nbsp;&nbsp;&nbsp;&nbsp;&nbsp;&nbsp;&nbsp;&nbsp;
<button name="Name_TESTRESET" id="TESTRESET" type="button" onclick="reset_button();">Test zur�cksetzen</button>
<br />
<br />
<div class="xreply">
<p name="Name_TESTEVAL" id="TESTEVAL">
Hier erscheint die Testauswertung!
<br />
</p>
</div>
\end{html}
}
\else
\newcommand{\MInTestFooter}{%
\relax
}
\fi


% ---------------------------------- Notationsmakros -------------------------------------------------------------

% Notationsmakros die nicht von der Kursvariante abhaengig sind

\newcommand{\MZahltrennzeichen}[1]{\renewcommand{\MZXYZhltrennzeichen}{#1}}

\ifttm
\newcommand{\MZahl}[3][\MZXYZhltrennzeichen]{\edef\MZXYZtemp{\noexpand\special{html:<mn>#2#1#3</mn>}}\MZXYZtemp}
\else
\newcommand{\MZahl}[3][\MZXYZhltrennzeichen]{{}#2{#1}#3}
\fi

\newcommand{\MEinheitenabstand}[1]{\renewcommand{\MEinheitenabstXYZnd}{#1}}
\ifttm
\newcommand{\MEinheit}[2][\MEinheitenabstXYZnd]{{}#1\edef\MEINHtemp{\noexpand\special{html:<mi mathvariant="normal">#2</mi>}}\MEINHtemp} 
\else
\newcommand{\MEinheit}[2][\MEinheitenabstXYZnd]{{}#1 \mathrm{#2}} 
\fi

\newcommand{\MExponentensymbol}[1]{\renewcommand{\MExponentensymbXYZl}{#1}}
\newcommand{\MExponent}[2][\MExponentensymbXYZl]{{}#1{} 10^{#2}} 

%Punkte in 2 und 3 Dimensionen
\newcommand{\MPointTwo}[3][]{#1(#2\MCoordPointSep #3{}#1)}
\newcommand{\MPointThree}[4][]{#1(#2\MCoordPointSep #3\MCoordPointSep #4{}#1)}
\newcommand{\MPointTwoAS}[2]{\left(#1\MCoordPointSep #2\right)}
\newcommand{\MPointThreeAS}[3]{\left(#1\MCoordPointSep #2\MCoordPointSep #3\right)}

% Masseinheit, Standardabstand: \,
\newcommand{\MEinheitenabstXYZnd}{\MThinspace} 

% Horizontaler Leerraum zwischen herausgestellter Formel und Interpunktion
\ifttm
\newcommand{\MDFPSpace}{\,}
\newcommand{\MDFPaSpace}{\,\,}
\newcommand{\MBlank}{\ }
\else
\newcommand{\MDFPSpace}{\;}
\newcommand{\MDFPaSpace}{\;\;}
\newcommand{\MBlank}{\ }
\fi

% Satzende in herausgestellter Formel mit horizontalem Leerraum
\newcommand{\MDFPeriod}{\MDFPSpace .}

% Separation von Aufzaehlung und Bedingung in Menge
\newcommand{\MCondSetSep}{\,:\,} %oder '\mid'

% Konverter kennt mathopen nicht
\ifttm
\def\mathopen#1{}
\fi

% -----------------------------------START Rouletteaufgaben ------------------------------------------------------------

\ifttm
% #1 = Dateiname, #2 = eindeutige ID fuer das Roulette im Kurs
\newcommand{\MDirectRouletteExercises}[2]{
\begin{MExercise}
\texttt{Im HTML erscheinen hier Aufgaben aus einer Aufgabenliste...}
\end{MExercise}
}
\else
\newcommand{\MDirectRouletteExercises}[2]{\relax} % wird durch mconvert.pl gefunden und ersetzt
\fi


% ---------------------------------- START Makros, die von der Kursvariante abhaengen ----------------------------------

\ifvariantunotation
  % unotation = An Universitaeten uebliche Notation
  \def\MVariant{unotation}

  % Trennzeichen fuer Dezimalzahlen
  \newcommand{\MZXYZhltrennzeichen}{.}

  % Exponent zur Basis 10 in der Exponentialschreibweise, 
  % Standardmalzeichen: \times
  \newcommand{\MExponentensymbXYZl}{\times} 

  % Begrenzungszeichen fuer offene Intervalle
  \newcommand{\MoIl}[1][]{\mbox{}#1(\mathopen{}} % bzw. ']'
  \newcommand{\MoIr}[1][]{#1)\mbox{}} % bzw. '['

  % Zahlen-Separation im IntervaLL
  \newcommand{\MIntvlSep}{,} %oder ';'

  % Separation von Elementen in Mengen
  \newcommand{\MElSetSep}{,} %oder ';'

  % Separation von Koordinaten in Punkten
  \newcommand{\MCoordPointSep}{,} %oder ';' oder '|', '\MThinspace|\MThinspace'

\else
  % An dieser Stelle wird angenommen, dass std-Variante aktiv ist
  % std = beschlossene Notation im TU9-Onlinekurs 
  \def\MVariant{std}

  % Trennzeichen fuer Dezimalzahlen
  \newcommand{\MZXYZhltrennzeichen}{,}

  % Exponent zur Basis 10 in der Exponentialschreibweise, 
  % Standardmalzeichen: \times
  \newcommand{\MExponentensymbXYZl}{\times} 

  % Begrenzungszeichen fuer offene Intervalle
  \newcommand{\MoIl}[1][]{\mbox{}#1]\mathopen{}} % bzw. '('
  \newcommand{\MoIr}[1][]{#1[\mbox{}} % bzw. ')'

  % Zahlen-Separation im IntervaLL
  \newcommand{\MIntvlSep}{;} %oder ','
  
  % Separation von Elementen in Mengen
  \newcommand{\MElSetSep}{;} %oder ','

  % Separation von Koordinaten in Punkten
  \newcommand{\MCoordPointSep}{;} %oder '|', '\MThinspace|\MThinspace'

\fi



% ---------------------------------- ENDE Makros, die von der Kursvariante abhaengen ----------------------------------


% diese Kommandos setzen Mathemodus vorraus
\newcommand{\MGeoAbstand}[2]{[\overline{{#1}{#2}}]}
\newcommand{\MGeoGerade}[2]{{#1}{#2}}
\newcommand{\MGeoStrecke}[2]{\overline{{#1}{#2}}}
\newcommand{\MGeoDreieck}[3]{{#1}{#2}{#3}}

%
\ifttm
\newcommand{\MOhm}{\special{html:<mn>&#x3A9;</mn>}}
\else
\newcommand{\MOhm}{\Omega} %\varOmega
\fi


\def\PERCTAG{\MAbort{PERCTAG ist zur internen verwendung in mconvert.pl reserviert, dieses Makro darf sonst nicht benutzt werden.}}

% Im Gegensatz zu einfachen html-Umgebungen werden MDirectHTML-Umgebungen von mconvert.pl am ganzen ttm-Prozess vorbeigeschleust und aus dem PDF komplett ausgeschnitten
\ifttm%
\newenvironment{MDirectHTML}{\begin{html}}{\end{html}}%
\else%
\newenvironment{MDirectHTML}{\begin{html}}{\end{html}}%
\fi

% Im Gegensatz zu einfachen Mathe-Umgebungen werden MDirectMath-Umgebungen von mconvert.pl am ganzen ttm-Prozess vorbeigeschleust, ueber MathJax realisiert, und im PDF als $$ ... $$ gesetzt
\ifttm%
\newenvironment{MDirectMath}{\begin{html}}{\end{html}}%
\else%
\newenvironment{MDirectMath}{\begin{equation*}}{\end{equation*}}% Vorsicht, auch \[ und \] werden in amsmath durch equation* redefiniert
\fi

% ---------------------------------- Location Management ---------------------------------------------

% #1 = buttonname (muss in files/images liegen und Format 48x48 haben), #2 = Vollstaendiger Einrichtungsname, #3 = Kuerzel der Einrichtung,  #4 = Name der include-texdatei
\ifttm
\newcommand{\MLocationSite}[3]{\special{html:<!-- mlocation;;}#1\special{html:;;}#2\special{html:;;}#3\special{html:;; //-->}}
\else
\newcommand{\MLocationSite}[3]{\relax}
\fi

% ---------------------------------- Copyright Management --------------------------------------------

\newcommand{\MCCLicense}{%
{\color{green}\textbf{CC BY-SA 3.0}}
}

\newcommand{\MCopyrightLabel}[1]{ (\MSRef{L_COPYRIGHTCOLLECTION}{Lizenz})\MLabel{#1}}

% Copyrightliste wird als tex-Datei im preprocessing erzeugt und unter converter/tex/copyrightcollection.tex abgelegt
% Der input-Befehl funktioniert nur, wenn die aufrufende tex-Datei auf der obersten Ebene liegt (d.h. selbst kein input/include ist, insbesondere keine Moduldatei)
\newcommand{\MCopyrightCollection}{\input{copyrightcollection.tex}}

% MCopyrightNotice fuegt eine Copyrightnotiz ein, der parser ersetzt diese durch CopyrightNoticePOST im preparsing, diese Definition wird nur fuer reine pdflatex-Uebersetzungen gebraucht
% Parameter: #1: Kurze Lizenzbeschreibung (typischerweise \MCCLicense)
%            #2: Link zum Original (http://...) oder NONE falls das Bild selbst ein Original ist, oder TIKZ falls das Bild aus einer tikz-Umgebung stammt
%            #3: Link zum Autor (http://...) oder MINT falls Original im MINT-Kolleg erstellt oder NONE falls Autor unbekannt
%            #4: Bemerkung (z.B. dass Datei mit Maple exportiert wurde)
%            #5: Labelstring fuer existierendes Label auf das copyrighted Objekt, mit MCopyrightLabel erzeugt
%            Keines der Felder darf leer sein!
\newcommand{\MCopyrightNotice}[5]{\MCopyrightNoticePOST{#1}{#2}{#3}{#4}{#5}}

\ifttm%
\newcommand{\MCopyrightNoticePOST}[5]{\relax}%
\else%
\newcommand{\MCopyrightNoticePOST}[5]{\relax}%
\fi%

% ---------------------------------- Meldungen fuer den Benutzer des Konverters ----------------------
\MPragma{mintmodversion;P0.1.0}
\MPragma{usercomment;This is file mintmod.tex version P0.1.0}


% ----------------------------------- Spezialelemente fuer Konfigurationsseite, werden nicht von mintscripts.js verwaltet --

% #1 = DOM-id der Box
\ifttm\newcommand{\MConfigbox}[1]{\special{html:<input cfieldtype="2" type="checkbox" name="Name_}#1\special{html:" id="}#1\special{html:" onchange="confHandlerChange('}#1\special{html:');"/>}}\fi % darf im PDF nicht aufgerufen werden!


\MPragma{MathSkip}
\MSetSubject{\MINTMathematics}

\Mtikzexternalize

\begin{document}

\MSection{Lineare Gleichungssysteme}
\MLabel{VBKM04}
\MSetSectionID{lgs}

\begin{MSectionStart}
\MDeclareSiteUXID{VBKM04_START}

\MModstartBox
\end{MSectionStart}


\MSubsection{Was sind Lineare Gleichungssysteme?}
\MLabel{M04_LGS}

\begin{MIntro}
\MDeclareSiteUXID{VBKM04_LGSIntro}
Ein Problem mit mehreren Unbekannten gleichzeitig!? Und eine ganze Reihe von Gleichungen dazu!? Problemstellungen dieser Art kommen nicht nur im naturwissenschaftlich-technischen Bereich vor, sondern auch in anderen wissenschaftlichen Disziplinen und im Alltag! Und sie müssen gelöst werden!

Zur Beruhigung vorneweg: Schwierig wird es nicht! Dagegen stimmt es, dass sich in den unterschiedlichsten
Gebieten häufig Situationen und Aufgaben finden, die in der mathematischen Modellierung auf mehrere Gleichungen
in mehreren Unbekannten führen.
Hierzu wird ein erstes einfaches Beispiel betrachtet:
\begin{MExample}
\MLabel{M04_einfuehrendes_Bsp}
Eine junge Artistengruppe möchte ihre halsbrecherische Radnummer zusätzlich aufmotzen, indem sie für ihre Ein- und Zweiräder
neue Felgen mit grellbunten Lichteffekten zukauft. Für die insgesamt $10$ Räder benötigt sie $13$ Felgen.
Wieviele Ein- und wieviele Zweiräder besitzt die Gruppe?

In einem ersten Schritt gilt es, die in der Aufgabenstellung enthaltenen Informationen, wenn möglich, in mathematische
Gleichungen zu übersetzen. Bezeichnet man die gesuchte Anzahl der Einräder mit $x$, diejenige der Zweiräder mit $y$,
so kann man als erste Information aus der Problembeschreibung herauslesen, dass
\begin{eqnarray*}
\mbox{Gleichung}\MBlank (1) & : & x + y = 10
\end{eqnarray*}
gelten muss, da die Gruppe insgesamt $10$ Räder ihr Eigentum nennt. Außerdem hat ein Einrad eine Felge, ein Zweirad dagegen
zwei Felgen. Weil alles in allem $13$ Felgen angeschafft werden sollen, weiß man auch, dass
\begin{eqnarray*}
\mbox{Gleichung}\MBlank (2) & : & x + 2 y = 13
\end{eqnarray*}
ist. Aus der vorliegenden Problemstellung ergeben sich also zwei Gleichungen, die die zwei unbekannten Größen
$x$ (Anzahl der Einräder) und $y$ (Anzahl der Zweiräder) in Beziehung setzen.
\end{MExample}
Früher oder später will man natürlich wissen, über wieviele Ein- bzw. Zweiräder die Artistengruppe tatsächlich verfügt. Im
gegebenen Beispiel kann man die Werte für $x$ und $y$ durch ein wenig Probieren erraten. Aber
eigentlich interessiert man sich für \textbf{systematische Methoden}, um Fragestellungen wie die obige
gezielt zu beantworten.

\end{MIntro}

\begin{MContent}
\MDeclareSiteUXID{VBKM04_LGSContent}
Bevor man richtig loslegen kann, muss man den Sprachgebrauch noch ein bisschen schärfen.
\begin{MInfo}
Mehrere Gleichungen, die auf eine bestimmte Anzahl Unbekannter \textbf{gleichzeitig} zutreffen, bilden ein sogenanntes
\MEntry{Gleichungssystem}{Gleichungssystem}. Kommen in jeder einzelnen Gleichung eines solchen Systems die Unbekannten
in jedem Term nur linear, d.h. höchstens zur Potenz $1$ und ausschließlich multipliziert mit (konstanten)
Zahlen vor, so spricht man von einem \MEntry{Linearen Gleichungssystem}{Gleichungssystem (linear)}, oder kurz \MEntry{LGS}{LGS}.
\end{MInfo}
Die beiden Gleichungen aus dem einführenden Beispiel \MRef{M04_einfuehrendes_Bsp} stellen ein Lineares Gleichungssystem
für zwei Unbekannte $x$ und $y$ dar. Dagegen bilden die drei Gleichungen
$$x + y + z = 3\;\mbox{und}\;x + y - z = 1\;\mbox{und}\; x \cdot y + z = 2$$
in den Unbekannten $x, y$ und $z$ zwar ein Gleichungssystem, jedoch \textbf{kein} lineares, da in der dritten Gleichung
der Term $x \cdot y$ auftritt, der \textbf{bilinear} in $x$ und $y$ ist und daher der Bedingung der \textbf{Linearität}
widerspricht.\\
Übrigens muss bei einem Gleichungssystem die Anzahl der Gleichungen nicht gleich
der Anzahl der Unbekannten sein; darauf wird man später noch zurückkommen.
\begin{MInfo}
Ist die Anzahl der Gleichungen in einem Gleichungssystem gleich der Anzahl der Unbekannten,
so bezeichnet man das Gleichungssystem als \MEntry{quadratisch}{Gleichungssystem (quadratisch)}.
\end{MInfo}
\begin{MExercise}
Bei welchen der folgenden Gleichungssysteme handelt es sich um Lineare Gleichungssysteme?

\begin{MQuestionGroup}
\begin{tabular}[t]{ll}
\MLCheckbox{1}{M04C10} & $x + y - 3 z = 0$ und $2 x - 3 = y$ und $\MZahl{1}{5} x - z = 22 + y$, \\
\MLCheckbox{0}{M04C11} & $\sin(x) + \cos(y) = 1$ und $x - y = 0$, \\
\MLCheckbox{0}{M04C12} & $2 z - 3 y + 4 x = 5$ und $z + y - x^2 = 25$.
\end{tabular}
\end{MQuestionGroup}
\MGroupButton{Eingabe kontrollieren}
\end{MExercise}
Lineare Gleichungssysteme zeichnen sich gegenüber allgemeinen Gleichungssystemen
durch eine meist deutlich größere Einfachheit aus. Nichtsdestotrotz spielen sie in den verschiedensten Bereichen
eine zentral wichtige Rolle, so in der Medizin z.B. im Zusammenhang mit der Computertomographie, in der Technik etwa
bei der Beschreibung, wie sich Schall in komplex gestalteten Räumen ausbreitet, oder in der Physik beispielsweise bei der
Frage, welche Wellenlängen angeregte Atome aussenden können. Daher ist es zweifelsohne lohnenswert, sich intensiv
mit Linearen Gleichungssystemen auseinanderzusetzen.

Im Vordergrund steht bei Gleichungssystemen generell die Frage,
welche Zahlenwerte man für die Unbekannten wählen muss,
damit alle Gleichungen des Systems simultan erfüllt sind. Ein solcher Satz von Zahlenwerten für die Unbekannten
wird auf den Begriff der \MEntry{Lösung eines Gleichungssystems}{Lösung eines Gleichungssystems} führen.

Zuvor sollte jedoch eine Feinheit beachtet werden: Abhängig von der Problemstellung ist es unter Umständen nicht sinnvoll,
alle möglichen Zahlenwerte für die Unbekannten zuzulassen. Im Eingangsbeispiel \MRef{M04_einfuehrendes_Bsp} repräsentieren
die Unbekannten $x$ und $y$ die Stückzahlen der Ein- bzw. Zweiräder im Besitz der Artistengruppe. Solche Stückzahlen
können nur ganze nichtnegative Zahlen, also Elemente von $\N_0$, sein. Daher muss man in diesem Fall die Menge der Zahlen,
aus denen die Lösungen stammen können, von vornherein auf $\N_0$ einschränken (und zwar sowohl für $x$ als auch $y$).
\begin{MInfo}
Diejenige Zahlenmenge, aus der die Lösungen eines Gleichungssystems überhaupt nur stammen können, nennt man
die \MEntry{Grundmenge}{Grundmenge} des Gleichungssystems. Die \MEntry{Definitionsmenge}{Definitionsmenge} ist diejenige Teilmenge%%%
der Grundmenge, für die alle Terme in den Gleichungen des Systems \textbf{definiert} sind.
Für Lineare Gleichungssysteme fallen Grundmenge und Definitionsmenge zusammen. Als \MEntry{Lösungsmenge}{Lösungsmenge}
schließlich bezeichnet man diejenige Teilmenge der Definitionsmenge, die die \textbf{Lösungen} des Systems zusammenfasst.
Diese Lösungsmenge wird mit $\ML$ bezeichnet.
\end{MInfo}
Ist keine weitere Aussage über die  Grundmenge getroffen - und lässt sich auch keine
Aussage aus der Problembeschreibung ableiten -, so wird stillschweigend davon ausgegangen,
dass die Grundmenge gleich $\R$, also gleich der Menge der reellen Zahlen, ist.
\end{MContent}



\MSubsection{LGS mit zwei Unbekannten}
\MLabel{M04_2_Unbekannte}

\begin{MIntro}
\MDeclareSiteUXID{VBKM04_ZweiUnbekannte}
Man beschränkt sich zunächst auf Lineare Gleichungssysteme in \textbf{zwei}
Unbekannten.
\begin{MInfo}
\MLabel{M04_2x2_system}
Allgemein hat ein Lineares Gleichungssystem (LGS), bestehend aus zwei
Gleichungen in den Unbekannten $x$ und $y$, folgende Gestalt:
\begin{eqnarray*}
	a_{11} \cdot x + a_{12} \cdot y &=& b_{1} \MDFPSpace, \\ 
	a_{21} \cdot x + a_{22} \cdot y &=& b_2 \MDFPeriod
\end{eqnarray*}
Dabei sind $a_{11}, a_{12}, a_{21}$ und $a_{22}$ die sogenannten Koeffizienten des
Linearen Gleichungssystems,
die ebenso wie die rechten Seiten $b_1$ und $b_2$ der Gleichungen meist aus den reellen Zahlen stammen und
aufgrund der Problemstellung (weitgehend) vorgegeben sind.

Sind die rechten Seiten $b_1$ und $b_2$ beide gleich $0$ ($b_1 = 0 = b_2$), so spricht man von einem
\textbf{homogenen}, andernfalls von einem \textbf{inhomogenen} Linearen Gleichungssystem.

\end{MInfo}

Aufgrund der Linearität kann jede der beiden Gleichungen des Systems in Infobox \MRef{M04_2x2_system} für sich als Gleichung
einer Geraden in der $x$-$y$-Ebene interpretiert werden: Löst man z.B. die erste Gleichung nach $y$ auf,
$$y = - \Mdfrac{a_{11}}{a_{12}} x + \Mdfrac{b_1}{a_{12}} \MDFPSpace ,$$
so kann man aus dieser expliziten Form direkt ablesen, dass eine Gerade mit der Steigung $m = - \Mdfrac{a_{11}}{a_{12}}$
und dem $y$-Achsenabschnitt $y_0 = \Mdfrac{b_1}{a_{12}}$ beschrieben wird.

Am Rande wird festgehalten, dass das eben erwähnte Freistellen nach $y$ natürlich
nur funktioniert, falls $a_{12} \neq 0$ ist. Ist $a_{12} = 0$,
so lautet die erste Gleichung $a_{11} \cdot x = b_1$; diese ist für $a_{11} \neq 0$ äquivalent zu
$x = \Mdfrac{b_1}{a_{11}}$, was bedeutet, dass $x$ einen konstanten Wert annimmt; dies stellt ebenfalls eine Gerade dar,
nämlich eine Gerade parallel zur $y$-Achse im Abstand $\Mdfrac{b_1}{a_{11}}$.\\
Und was, wenn sowohl $a_{12} = 0$ als auch
$a_{11} = 0$ gilt? Nun, dann muss ebenfalls $b_1 = 0$ sein, da ansonsten die erste Gleichung von vornherein
einen Widerspruch ergeben würde. Für $a_{11} = a_{12} = b_1 = 0$ ist aber die erste Gleichung (für alle
Werte von $x$ und $y$) immer identisch erfüllt ($0 = 0$) und somit wertlos.

Im Fall der zweiten Gleichung in Infobox \MRef{M04_2x2_system} geht man ganz entsprechend vor:
$$y = - \Mdfrac{a_{21}}{a_{22}} x + \Mdfrac{b_2}{a_{22}} \MDFPeriod$$
Insgesamt erhält man zwei Geraden als Repräsentanten der beiden linearen Gleichungen, und die Frage nach Lösbarkeit und Lösung
des Linearen Gleichungssystems, also die \textbf{Frage nach der
gleichzeitigen Gültigkeit beider Gleichungen}, lässt sich als \textbf{Frage nach Existenz und %%%
Lage des Schnittpunkts der beiden Geraden} lesen. Dazu schaut man sich ein konkretes Beispiel an:%%%

\begin{MExample}
Das Lineare Gleichungssystem aus dem einführenden Beispiel \MRef{M04_einfuehrendes_Bsp} lautet:
$$\left. \begin{array}{rcl} x + y & = & 10 \\[.5ex] x + 2 y & = & 13 \end{array} \right\} \Leftrightarrow
\left\{ \begin{array}{rcl} y & = & - x + 10 \\[.5ex] y & = & - \Mtfrac12 x + \Mtfrac{13}{2} \end{array} \right. \MDFPeriod$$
(Hier nehmen die allgemeinen Koeffizienten und rechten Seiten des Systems \MRef{M04_2x2_system} somit die
Werte $a_{11} = 1, a_{12} = 1, a_{21} = 1, a_{22} = 2, b_1 = 10$ und $b_2 = 13$ an.)\\
Es werden zwei Geraden mit den Steigungen $m_{1} = - 1$ bzw. $m_{2} = - \Mtfrac12$ und den $y$-Achsenabschnitten
$y_{0,1} = 10$ bzw. $y_{0,2} = \Mtfrac{13}{2}$ beschrieben:
\begin{center}
\MTikzAuto{%
\begin{tikzpicture}[x=0.4cm, y=0.5cm] 
%Koordinatensystem
\node (xMAX) at (16.5,0){};
\node (yMAX) at (0,11){};
\draw[->,color=black] (-6.2,0) -- (xMAX);
\foreach \x in {-6, -4, -2, 2, 4, 6, 8, 10, 12, 14, 16}
\draw[shift={(\x,0)},color=black] (0pt,2pt) -- (0pt,-2pt) node[below] {\footnotesize $\x$};
\draw[->,color=black] (0,-1.0) -- (yMAX);
\foreach \y in {2,4,6,8,10}
\draw[shift={(0,\y)},color=black] (2pt,0pt) -- (-2pt,0pt) node[left] {\footnotesize $\y$};
\draw[color=black] (0pt,-8.5pt) node[right] {\footnotesize $0$};
\draw[color=black] (-2.0pt,7pt) node[left] {\footnotesize $0$};
%Achsenbeschriftung
\draw (xMAX) node[anchor=south] {$x$};
\draw (yMAX) node[anchor=west] {$y$};
%Beschriftung
\clip(-6,-0.5) rectangle (16.5,10.5);
\draw[color=red, thick] (15,-1.0) -- (-6,9.5);
\draw[color=blue, thick] (11,-1.0) -- (-0.5,10.5);
\draw[color=black, thick] (7,3) circle (1.5pt);
\draw[color=red] (2.35,4.7) node[anchor=east] {$y=-x/2+13/2$};
\draw[color=blue] (2.7,8) node[anchor=west] {$y=-x+10$};
\draw[color=black] (7,3) node[anchor=south west] {$\MPointTwo{7}{3}$};
\end{tikzpicture}
}%
%%\MUGraphicsSolo{eindeutige_lsg.png}{scale=1}{width:450px}
\end{center}
Man erkennt aus dem Schaubild, dass sich die beiden Geraden in der Tat schneiden, und liest
die Koordinaten des Schnittpunktes zu $\MPointTwo{x = 7}{y = 3}$ ab. Dementsprechend besitzt
das hier betrachtete Lineare Gleichungssystem eine eindeutige Lösung; die
Lösungsmenge $\ML$ enthält genau ein Zahlenpaar, ${\ML} = \{ \MPointTwo{x = 7}{y = 3} \}$.
Die Artistengruppe aus Beispiel \MRef{M04_einfuehrendes_Bsp} hat also $7$ Einräder und $3$ Fahrräder.%%%
\end{MExample}
Diese anschauliche Betrachtungsweise eignet sich hervorragend, alle Fälle zu
diskutieren, die überhaupt nur auftreten können: Denn entweder schneiden sich zwei Geraden in der $x$-$y$-Ebene -
und dann ist der Schnittpunkt zwangsläufig \textbf{eindeutig} -, oder aber zwei solche Geraden verlaufen parallel -
und besitzen somit keinen Schnittpunkt -, oder aber die beiden Geraden sind deckungsgleich -
und schneiden sich daher sozusagen in unendlich vielen Punkten. Andere Möglichkeiten sind nicht denkbar.
Demzufolge kann man im Hinblick auf die Mächtigkeit der Lösungsmenge des
zugehörigen linearen Gleichungssystems Folgendes festhalten:
\begin{MInfo}
Ein \MEntry{inhomogenes Lineares Gleichungssystem}{Lineares Gleichungssystem (inhomogen)} besitzt entweder \textbf{keine}, \textbf{eine eindeutige} oder aber \textbf{unendlich viele Lösungen}. %%%

Ein \MEntry{homogenes Lineares Gleichungssystem}{Lineares Gleichungssystem (homogen)} weist \textbf{immer
eine Lösung} auf, nämlich die sogenannte \MEntry{triviale Lösung}{Triviale Lösung (LGS)} $x = 0$ und $y = 0$.
Darüber hinaus \textbf{kann} ein solches homogenes System auch \textbf{unendlich viele Lösungen} besitzen.
\end{MInfo}
Das Gesagte soll an zwei weiteren Beispielen, bei denen man direkt mit den
Linearen Gleichungssystemen startet, verdeutlicht werden:
\begin{MExample}
\MLabel{M04_2x2_nicht_eindeutige_lsg}
In beiden Fällen wählt man als \textbf{Grundmenge} die Menge der reellen Zahlen $\R$.
\begin{center}
\begin{tabular}{l|l}
\begin{minipage}{7.5cm}
$$\left. \begin{array}{rcl} x + y & = & 2 \\[.5ex] 2 x + 2 y & = & 1 \end{array} \right\} \Leftrightarrow
\left\{ \begin{array}{rcl} y & = & - x + 2 \\[.5ex] y & = & - x + \Mtfrac12 \end{array} \right. \MDFPeriod $$
\end{minipage} &
\begin{minipage}{7.5cm}
$$\left. \begin{array}{rcl} x + y & = & 2 \\[.5ex] 2 x + 2 y & = & 4 \end{array} \right\} \Leftrightarrow
\left\{ \begin{array}{rcl} y & = & - x + 2 \\[.5ex] y & = & - x + 2 \end{array} \right. \MDFPeriod $$
\end{minipage} \\[1cm]
\MTikzAuto{%
\begin{tikzpicture}[x=1.0cm, y=1.4cm] 
%Koordinatensystem
\node (xMAX) at (3.5,0){};
\node (yMAX) at (0,2.5){};
\draw[->,color=black] (-2.5,0) -- (xMAX);
\foreach \x in {-2, -1, 1, 2, 3}
\draw[shift={(\x,0)},color=black] (0pt,2pt) -- (0pt,-2pt) node[below] {\footnotesize $\x$};
\draw[->,color=black] (0,-0.5) -- (yMAX);
\foreach \y in {1,2}
\draw[shift={(0,\y)},color=black] (2pt,0pt) -- (-2pt,0pt) node[left] {\footnotesize $\y$};
\draw[color=black] (0pt,-8.5pt) node[right] {\footnotesize $0$};
\draw[color=black] (-2.0pt,7pt) node[left] {\footnotesize $0$};
%Achsenbeschriftung
\draw (xMAX) node[anchor=south] {$x$};
\draw (yMAX) node[anchor=west] {$y$};
%Beschriftung und Graphen
\clip(-2.8,-0.6) rectangle (3.6,2.7);
\draw[color=blue, thick] (1,-0.5) -- (-2,2.5);
\draw[color=blue, thick] (2.5,-0.5) -- (-0.5,2.5);
\draw[color=black] (-0.2,0.5) node[anchor=east] {$y=-x+1/2$};
\draw[color=black] (0.7,1.5) node[anchor=west] {$y=-x+2$};
%%\draw[color=black] (7,3) node[anchor=south west] {$\MPointTwo{7}{3}$};
\end{tikzpicture}
}
%%\MUGraphicsSolo{keine_lsg.png}{scale=1}{width:350px}
&
\MTikzAuto{%
\begin{tikzpicture}[x=1.0cm, y=1.4cm] 
%Koordinatensystem
\node (xMAX) at (4,0){};
\node (yMAX) at (0,2.5){};
\draw[->,color=black] (-1.5,0) -- (xMAX);
\foreach \x in {-1, 1, 2, 3}
\draw[shift={(\x,0)},color=black] (0pt,2pt) -- (0pt,-2pt) node[below] {\footnotesize $\x$};
\draw[->,color=black] (0,-0.5) -- (yMAX);
\foreach \y in {1,2}
\draw[shift={(0,\y)},color=black] (2pt,0pt) -- (-2pt,0pt) node[left] {\footnotesize $\y$};
\draw[color=black] (0pt,-8.5pt) node[right] {\footnotesize $0$};
\draw[color=black] (-2.0pt,7pt) node[left] {\footnotesize $0$};
%Achsenbeschriftung
\draw (xMAX) node[anchor=south] {$x$};
\draw (yMAX) node[anchor=west] {$y$};
%Beschriftung und Graphen
\clip(-1.6,-0.6) rectangle (4,2.7);
\draw[color=red, thick] (2.5,-0.52) -- (-0.5,2.48);
\draw[color=blue, thick] (2.5,-0.48) -- (-0.5,2.52);
\draw[color=red] (1.35,0.45) node[anchor=east] {$2x+2y=4$};
\draw[color=blue] (0.7,1.5) node[anchor=west] {$x+y=2$};
%%\draw[color=black] (7,3) node[anchor=south west] {$\MPointTwo{7}{3}$};
\end{tikzpicture}
}
%%\MUGraphicsSolo{unendl_lsg.png}{scale=1}{width:350px}
\\[.5cm]
\begin{minipage}[t]{7.5cm}
Die beiden Geraden besitzen dieselbe Steigung $m = - 1$, aber verschiedene $y$-Achsenabschnitte ($y_0 = 2$
bzw. $y_0 = \Mdfrac12$); sie verlaufen parallel; das Lineare Gleichungssystem besitzt \textbf{keine}
Lösung: $${\ML} = \MEmptyset\MDFPeriod $$
\end{minipage} &
\begin{minipage}[t]{7.5cm}
Die beiden Geraden besitzen sowohl dieselbe Steigung $m = - 1$ als auch denselben
$y$-Achsenabschnitt $y_0 = 2$; sie sind deckungsgleich; das Lineare Gleichungssystem besitzt
unendlich viele Lösungen, die z.B. wie folgt angegeben werden können:
$${\ML} = \{ \MPointTwo{t}{ - t + 2} \MCondSetSep  t \in \R \}\MDFPeriod $$
\end{minipage}
\end{tabular}
\end{center}
\end{MExample}
Im Falle des Beispiels in der rechten Spalte sind andere Parametrisierungen der Lösungsmenge möglich und
erlaubt. Es kommt im Grunde nur darauf an, die Punkte der (deckungsgleichen) Geraden geeignet zu beschreiben.
Bei der obigen Angabe von $\ML$ wurde einfach die Geradengleichung selbst verwendet und die
Laufvariable $t$ statt $x$ genannt.

Und was hat es mit den oben erwähnten möglichen Einschränkungen wegen der Grundmenge auf sich? Auch hierzu ein Beispiel:
\begin{MExample}
Auf einem Volksfest verspricht ein besonders pfiffiger Standbesitzer geradezu traumhafte Preise und das gegen einen
lächerlich geringen Spieleinsatz, wenn, ja wenn einer der Passanten ihm nur folgendes kleine Rätsel löst:
{\textit Ich habe mit einem Würfel zweimal gewürfelt. Ziehe ich vom Sechsfachen der zweiten Augenzahl das
Zweifache der ersten ab, so erhalte ich die Zahl $3$. Addiere ich andererseits zum Vierfachen der ersten
Augenzahl die Zahl $6$, so bekomme ich das Zwölffache der zweiten Augenzahl. Welche beiden Zahlen habe ich gewürfelt?}

Bezeichnet man die Augenzahl des ersten Würfelwurfs mit $x$, diejenige des zweiten mit $y$, so kann man die
Aussagen des Standbesitzers sehr schnell in Gleichungen übersetzen:
$$\left. \begin{array}{rcl} 6 y - 2 x & = & 3 \\[.5ex] 4 x + 6 & = & 12 y \end{array} \right\} \Leftrightarrow
\left\{ \begin{array}{rcl} y & = & \Mtfrac13 x + \Mtfrac12 \\[.5ex]
y & = & \Mtfrac13 x + \Mtfrac12 \end{array} \right. \MDFPeriod $$
Man stellt fest, dass das entstehende Lineare Gleichungssystem - anschaulich interpretiert - auf zwei
deckungsgleiche Geraden führt. Vordergründig scheint es daher unendlich viele Lösungen zu geben.

Hier kommt jetzt allerdings die Grundmenge ins Spiel: Da sowohl $x$ als auch $y$ Augenzahlen
eines Würfels repräsentieren, können beide Unbekannte jeweils nur einen Wert aus der Menge $\{ 1\MElSetSep 2\MElSetSep 3\MElSetSep 4\MElSetSep 5\MElSetSep 6 \}$
annehmen. Betrachtet man die Gerade $y = \Mtfrac13 x + \Mtfrac12$ in der $x$-$y$-Ebene,
\begin{center}
\MTikzAuto{%
\begin{tikzpicture}[x=1.0cm, y=1.0cm] 
%Koordinatensystem
\node (xMAX) at (10.5,0){};
\node (yMAX) at (0,6.5){};
\draw[->,color=black] (-1.5,0) -- (xMAX);
\foreach \x in {-1, 0, 1, 2, 3, 4, 5, 6, 7, 8, 9, 10}
\draw[shift={(\x,0)},color=black] (0pt,2pt) -- (0pt,-2pt) node[below right] {\scriptsize $\x$};
\draw[->,color=black] (0,-0.5) -- (yMAX);
\foreach \y in {0, 1, 2, 3, 4, 5, 6}
\draw[shift={(0,\y)},color=black] (2pt,0pt) -- (-2pt,0pt) node[above left] {\scriptsize $\y$};
%%\draw[color=black] (0pt,-8.5pt) node[right] {\footnotesize $0$};
%%\draw[color=black] (-2.0pt,7pt) node[left] {\footnotesize $0$};
%Achsenbeschriftung
\draw (xMAX) node[anchor=south] {$x$};
\draw (yMAX) node[anchor=west] {$y$};
%Beschriftung und Graphen
\clip(-1.4,-0.4) rectangle (10.5,6.5);
\draw[help lines, gray, dashed] (-2,-1) grid (11,7); % was dotted
\fill[red!50!white, opacity=0.50] (1,1) rectangle (6,6);
\draw[color=brown] (1,1) rectangle (6,6);
\foreach \i in {1, 2, 3, 4, 5, 6}
  \foreach \j in {1, 2, 3, 4, 5, 6}
  {
    \fill[color=blue] (\i,\j) circle (2.0pt);
    \draw[color=black] (\i,\j) circle (2.0pt);
  }
\draw[color=black, thick] (-1.5,0.0) -- (10.5,4.0);
\draw[color=black] (6.3,2.35) node[anchor=west] {\Large $y=\frac{1}{3}x+\frac{1}{2}$};
%%\draw[color=black] (7,3) node[anchor=south west] {$\MPointTwo{7}{3}$};
\end{tikzpicture}
}%
%%\MUGraphicsSolo{betrueger.png}{scale=1}{width:450px}
\end{center}
so erkennt man, dass
kein mögliches Augenzahlpaar auf dieser Geraden liegt; daher ist die Lösungsmenge hier tatsächlich leer,
${\ML} = \MEmptyset$.
\end{MExample}
\end{MIntro}

\begin{MXContent}{Die Einsetzmethode und die Gleichsetzmethode}{Einsetz- und Gleichsetzmethode}{STD}
\MDeclareSiteUXID{VBKM04_Einsetzmethode}
\MLabel{M04_einsetz_gleichsetz}
Bisher wurden Fragen der \MEntry{Lösbarkeit}{Lösbarkeit} und der \textbf{graphischen Lösung} von
Linearen Gleichungssystemen der Gestalt aus \MRef{M04_2x2_system} untersucht.%%%
Eine rechnerische Behandlung solcher Systeme steht noch aus, was jetzt nachgeholt werden soll.
Dazu betrachtet man ein weiteres Beispiel:
\begin{MExample}
Familie Müller hat für die Renovierung ihres Hauses zwei Kredite in einer Gesamthöhe von $50\,000$ Euro
aufnehmen müssen, für die sie pro Jahr zusammen $3\,700$ Euro allein an Zinsen zu bezahlen hat. Für den
einen Kreditvertrag fallen $5\%$, für den anderen $8\%$ jährliche Zinsen an. Über welche Beträge
laufen die einzelnen Kredite?

Man bezeichnet die gesuchten Kredithöhen der beiden Verträge mit $x$ und $y$. Die
Summe dieser beiden Beträge beläuft sich laut Aufgabentext auf $50\,000$ Euro, also lautet die erste Gleichung:
\begin{eqnarray*}
\mbox{Gleichung}\MBlank (1) & : & x + y = 50\,000 \; \; \; \mbox{(in Euro)}\MDFPeriod 
\end{eqnarray*}
Die Zinslast aus dem mit $5\%$ verzinsten Vertrag beträgt $\MZahl{0}{05} \cdot x$, die aus dem anderen Vertrag  mit $8\%$ Zinsen
$\MZahl{0}{08} \cdot y$. Beide Lasten summieren sich gemäß Aufgabetext auf $3\,700$ Euro; dies liefert eine zweite Gleichung:
\begin{eqnarray*}
\mbox{Gleichung}\MBlank (2) & : & \MZahl{0}{05} x + \MZahl{0}{08} y = 3\,700 \; \; \; \mbox{(in Euro)}\MDFPeriod 
\end{eqnarray*}
Wiederum landet man bei einem Linearen Gleichungssystem vom Typ \MRef{M04_2x2_system}.

Für die rechnerische Lösung stellt man Gleichung~$(1)$ nach $y$ um; es entsteht eine zu $(1)$ äquivalente Gleichung~$(1')$:
\begin{eqnarray*}
\mbox{Gleichung}\MBlank (1') & : & y = 50\,000 - x\MDFPeriod 
\end{eqnarray*}
Diesen Ausdruck für $y$ kann man nun in Gleichung~$(2)$ für $y$ \textbf{einsetzen}, sodass die resultierende Gleichung nur noch
$x$ als Unbekannte enthält und dementsprechend aufgelöst werden kann:
\begin{eqnarray*}
& & \MZahl{0}{05} x + \MZahl{0}{08} (50\,000 - x) = 3\,700 \\
\Leftrightarrow & & \MZahl{0}{05} x + 4\,000 - \MZahl{0}{08} x = 3\,700 \\
\Leftrightarrow & & \MZahl{0}{03} x = 300 \\
\Leftrightarrow & & x = 10\,000\MDFPeriod 
\end{eqnarray*}
Setzt man dieses Ergebnis für $x$ in Gleichung~$(1')$ ein, so folgt:
\begin{eqnarray*}
& & y = 50\,000 - 10\,000 \\
\Leftrightarrow & & y = 40\,000\MDFPeriod 
\end{eqnarray*}
Die gesuchten Kreditvolumina betragen daher $10\,000$ Euro (Vertrag mit $5\%$ Verzinsung) und $40\,000$ Euro
(Vertrag mit $8\%$ Verzinsung).
\end{MExample}
Das voranstehende Beispiel demonstriert auf charakteristische Art und Weise das Vorgehen bei der sogenannten
\textbf{Einsetzmethode}:
\begin{MInfo}
\MLabel{M04_einsetzmethode}
Bei der \MEntry{Einsetzmethode}{Einsetzmethode} wird eine der beiden linearen Gleichungen in einem ersten Schritt
nach einer der Unbekannten - oder nach einem Vielfachen einer der Unbekannten - freigestellt; dieses Ergebnis wird im
zweiten Schritt in die andere lineare Gleichung \textbf{eingesetzt}. Es können nun drei Fälle auftreten:
\begin{itemize}
\item[(i)]{Die resultierende Gleichung enthält (nach dem Zusammenfassen gleichartiger Terme)
noch die andere der beiden Unbekannten. Das Auflösen der resultierenden
Gleichung nach dieser anderen Unbekannten liefert den ersten Teil des Ergebnisses; den zweiten Teil erhält man zum Beispiel,
indem man das erste Teilergebnis in die Gleichung aus dem ersten Schritt einsetzt. Die Lösung ist eindeutig. (Gehört
diese Lösung allerdings nicht zur Grundmenge, so muss sie ausgeschlossen werden.)}
\item[(ii)]{Die resultierende Gleichung enthält die andere der beiden Unbekannten nicht mehr und stellt einen Widerspruch
in sich dar. Dann besitzt das Lineare Gleichungssystem keine Lösung.}
\item[(iii)]{Die resultierende Gleichung enthält die andere der beiden Unbekannten nicht mehr und ist automatisch immer
gültig. Dann besitzt das Lineare Gleichungssystem unendlich viele Lösungen (wenn sich durch die zulässige Grundmenge keine Einschränkungen ergeben).}
\end{itemize}
\end{MInfo}
Bei dieser Vorgehensweise bestehen gewisse Freiheiten. Es ist nicht festgelegt, welche der linearen Gleichungen
des Systems nach welcher Unbekannten - oder Vielfachen davon - aufgelöst werden soll; solange es sich generell
um Äquivalenzumformungen handelt, führt jeder der möglichen Wege zum selben Resultat. Die Bevorzugung eines
bestimmten Lösungsweges ist zum Teil eine Frage des Geschmacks und zum Teil eine Frage der Geschicklichkeit:
Einige Zwischenrechnungen können sich vereinfachen, wenn eine clevere Wahl getroffen wird.

Im Zusammenhang mit der \textbf{Einsetzmethode} sollen die oben angesprochenen Fälle (ii) und (iii) noch an
den Linearen Gleichungssystemen aus Beispiel \MRef{M04_2x2_nicht_eindeutige_lsg} illustriert werden:
\begin{MExample}
Als Grundmenge bei beiden Linearen Gleichungssystemen legt man wiederum $\R$ fest.
\begin{center}
\begin{tabular}{c|c}
\begin{minipage}[t]{7.5cm}
$$\begin{array}{rcrcl} \mbox{Gleichung}\MBlank (1): & & x + y & = & 2 \\[.5ex]
\mbox{Gleichung}\MBlank (2): & & 2 x + 2 y & = & 1 \end{array}\MDFPeriod $$
Freistellen der Gleichung $(1)$ nach $x$ liefert $x = 2 - y$.
Dies in Gleichung $(2)$ eingesetzt ergibt:
\begin{eqnarray*}
& & 2 (2 - y) + 2y = 1 \\
& \Leftrightarrow & 4 - 2y + 2y = 1 \\
& \Leftrightarrow & 4 = 1\MDFPeriod 
\end{eqnarray*}
Dies ist ein Widerspruch; das LGS besitzt \textbf{keine} Lösung.
\end{minipage} &
\begin{minipage}[t]{7.5cm}
$$\begin{array}{rcrcl} \mbox{Gleichung}\MBlank (1): & & x + y & = & 2 \\[.5ex]
\mbox{Gleichung}\MBlank (2): & & 2 x + 2 y & = & 4 \end{array}\MDFPeriod $$
Freistellen der Gleichung $(1)$ nach $y$ liefert $y = 2 - x$.
Dies in Gleichung $(2)$ eingesetzt ergibt:
\begin{eqnarray*}
& & 2x + 2 (2 - x) = 4 \\
& \Leftrightarrow & 2x + 4 - 2x = 4 \\
& \Leftrightarrow & 4 = 4\MDFPeriod 
\end{eqnarray*}
Dies ist immer wahr; das LGS besitzt \textbf{unendlich viele} Lösungen.
\end{minipage}
\end{tabular}
\end{center}
\end{MExample}
Die Einsetzmethode ist nicht das einzige Verfahren, um Lineare Gleichungssysteme
rechnerisch zu lösen. Nachfolgend betrachtet man eine weitere Methode, die sehr eng mit der graphischen Lösung eines LGS verwandt ist.
\begin{MInfo}
\MLabel{M04_gleichsetzmethode}
Bei der \MEntry{Gleichsetzmethode}{Gleichsetzmethode} werden \textbf{beide} linearen Gleichungen in einem ersten
Schritt nach einer der Unbekannten - oder nach einem Vielfachen einer der Unbekannten - freigestellt. Die beiden
resultierenden neuen Gleichungen werden dann im zweiten Schritt \textbf{gleichgesetzt}.
Es können dann abermals die drei im Zusammenhang mit der Einsetzmethode diskutierten Fälle auftreten.
\end{MInfo}
Auch dieses Verfahren beinhaltet gewisse Freiheiten; so ist zum Beispiel nicht vorgeschrieben, nach welcher Unbekannten
die linearen Gleichungen freigestellt werden sollen.

Zur Demonstration wird das Eingangsbeispiel nochmals gelöst, jetzt mit Hilfe der Gleichsetzmethode:
\begin{MExample}
Das Lineare Gleichungssystem des einführenden Beispiels lautet:
\begin{eqnarray*}
x + y & = & 10 \MDFPSpace, \\ x + 2 y & = & 13\MDFPeriod 
\end{eqnarray*}
Man löst beide Gleichungen nach $x$ auf,
\begin{eqnarray*}
x & = & 10 - y\MDFPSpace,  \\ x & = & 13 - 2y \MDFPSpace,
\end{eqnarray*}
und setzt die rechten Seiten der beiden Gleichungen gleich,
$$10 - y = 13 - 2y \MDFPSpace,$$
was auf $y = 3$ führt. Dieses Ergebnis kann man in eine der beiden nach $x$ aufgelösten Gleichungen einsetzen,
um $x = 7$ zu erhalten.
\end{MExample}
\begin{MExercise}
Bestimmen Sie die Lösungsmenge für das Lineare Gleichungssystem
\begin{eqnarray*}
7 x + 2 y &= & 14 \MDFPSpace, \\ 3 x - 5 y & = & 6
\end{eqnarray*}
mit Hilfe der Gleichsetzmethode.
\ \\
\begin{MHint}{Lösung}
Man löst beispielsweise beide Gleichungen nach $x$ auf: dazu multipliziert man die erste Gleichung mit
$\Mdfrac17$ und stellt nach $x$ um,
$$x = \Mtfrac{14}{7} - \Mtfrac27 y \Leftrightarrow x = 2 - \Mtfrac27 y \; : \MBlank\mbox{Gleichung}\MBlank (1') \MDFPSpace ;$$
die zweite Gleichung multipliziert man dagegen mit $\Mtfrac13$, bevor man nach $x$ freistellt,
$$x = \Mtfrac63 + \Mtfrac53 y \Leftrightarrow x = 2 + \Mtfrac53 y \; : \MBlank\mbox{Gleichung}\MBlank (2') \MDFPeriod $$
Gleichsetzen der beiden rechten Seiten der Gleichungen~$(1')$ und $(2')$ liefert:
$$2 - \Mtfrac27 y = 2 + \Mtfrac53 y \Leftrightarrow 0 = (\Mtfrac27 + \Mtfrac53) y \Leftrightarrow y = 0 \MDFPeriod $$
Mit diesem Ergebnis für $y$ ergibt z.B. die erste Gleichung:
$$7 x + 2 \cdot 0 = 14 \Leftrightarrow 7 x = 14 \Leftrightarrow x = 2 \MDFPeriod $$
Also ist das gegebene Lineare Gleichungssystem eindeutig lösbar und ${\ML} = \{ \MPointTwo{x = 2}{y = 0} \}$.

Alternativ hätte man die beiden Gleichungen vor dem Gleichsetzen auch nach $y$ auflösen können (oder nach einem
Vielfachen von $x$ oder nach einem Vielfachen von $y$). Das Endresultat ist jedesmal dasselbe.
\end{MHint}
\end{MExercise}
\end{MXContent}

\begin{MXContent}{Die Additionsmethode}{Additionsmethode}{STD}
\MLabel{M04_addition}
\MDeclareSiteUXID{VBKM04_Additionsmethode}
Es soll noch ein weiteres, drittes, Verfahren zur rechnerischen Lösung von Linearen Gleichungssystemen
vorgestellt werden, das sein eigentliches Potential aber erst bei größeren Systemen, d.h. vielen linearen Gleichungen in
vielen Unbekannten entwickeln wird, da es sich sehr gut systematisieren lässt. Hier soll es um die prinzipielle
Vorgehensweise gehen. Zu Beginn wird ein Beispiel betrachtet:
\begin{MExample}
Man sucht die Lösungsmenge des Linearen Gleichungssystems
$$\begin{array}{rcrcl} \mbox{Gleichung}\MBlank (1): & & 2 x + y & = & 9 \MDFPSpace, \\
\mbox{Gleichung}\MBlank (2): & & 3 x - 11 y & = & 1 \MDFPSpace, \end{array}$$
wobei als Grundmenge die Menge der reellen Zahlen $\R$ gewählt wird.

Diesmal wird zur Lösung folgender Weg eingeschlagen: Man multipliziert Gleichung~$(1)$ mit dem Faktor $11$ durch und
erhält eine zu Gleichung~$(1)$ äquivalente Gleichung:
$$\begin{array}{crclcl} & (2 x + y) \cdot 11 & = & 9 \cdot 11 & & \\
\Leftrightarrow & 22 x + 11 y & = & 99 & & : \MBlank\mbox{Gleichung}\MBlank (1') 
\MDFPeriod \end{array}$$
Anschließend \textbf{addiert} man die neue Gleichung $(1')$ zu Gleichung $(2)$ hinzu, d.h. man setzt die
\textbf{Summe} der linken Seiten von $(2)$ und $(1')$ gleich der \textbf{Summe} der rechten Seiten
von $(2)$ und $(1')$. Dabei fällt die Unbekannte~$y$
heraus; dies ist übrigens der Grund für die Wahl des Faktors $11$ im vorherigen Schritt:
$$3 x - 11 y + 22 x + 11 y = 1 + 99 \Leftrightarrow 25 x = 100 \Leftrightarrow x = 4 \MDFPeriod $$
Um den Lösungswert für $y$ zu bekommen, kann man das gerade erzielte Resultat für $x$ z.B. in Gleichung~$(1)$
einsetzen:
$$2 \cdot 4 + y = 9 \Leftrightarrow 8 + y = 9 \Leftrightarrow y = 1 \MDFPeriod$$
Das Lineare Gleichungssystem des vorliegenden Beispiels besitzt also eine eindeutige Lösung, ${\ML} =
\{ \MPointTwo{x = 4}{y = 1} \}$.
\end{MExample}
Auch bei diesem Verfahren ist das Vorgehen nicht eindeutig festgelegt: So hätte man z.B. auch Gleichung~$(1)$ mit $3$
und Gleichung~$(2)$ mit $(- 2)$ durchmultiplizieren können,
$$\begin{array}{rclcrclcl} (2x + y) \cdot 3 & = & 9 \cdot 3 & \Leftrightarrow & 6 x + 3 y & = & 27
& & : \MBlank\mbox{Gleichung}\MBlank ({1''}) \MDFPSpace, \\ 
(3 x - 11 y) \cdot (- 2) & = & 1 \cdot (- 2) & \Leftrightarrow & - 6 x + 22 y & = & - 2
& & : \MBlank\mbox{Gleichung}\MBlank ({2''}) \MDFPSpace , \end{array}$$
um bei der anschließenden \textbf{Addition} der Gleichungen~$({1''})$ und $({2''})$ die Variable $x$ zu eliminieren:
$$6 x + 3 y - 6 x + 22 y = 27 - 2 \Leftrightarrow 25 y = 25 \Leftrightarrow y = 1 \MDFPeriod$$
Das Ergebnis für $y$ hätte man dann z.B. in Gleichung~$(2)$ einsetzen können, um $x$ zu bestimmen:
$$3 x - 11 \cdot 1 = 1 \Leftrightarrow 3 x = 12 \Leftrightarrow x = 4 \MDFPeriod$$
\begin{MInfo}
Bei der \MEntry{Additionsmethode}{Additionsmethode} wird eine der linearen Gleichungen durch geschickte Multiplikation
mit einem geeigneten Faktor so umgeformt, dass bei der anschließenden \textbf{Addition} der anderen Gleichung (zumindest)
eine Unbekannte herausfällt. (Manchmal ist es einfacher, \textbf{beide} Gleichungen vor der \textbf{Addition} mit
passend gewählten Faktoren zu multiplizieren.) Wie im Fall der Einsetzmethode
\MRef{M04_einsetzmethode} (oder der Gleichsetzmethode \MRef{M04_gleichsetzmethode}) können%%%
anschließend drei Fälle auftreten, die auf eine Lösungsmenge $\ML$ mit genau einem Element, keinem Element oder
unendlich vielen Elementen führen.
\end{MInfo}
\end{MXContent}

\begin{MExercises}
\MDeclareSiteUXID{VBKM04_Additionsmethode_Exercises}
\begin{MExercise}
Lösen Sie die folgenden Linearen Gleichungssysteme mit Hilfe der Einsetzmethode:
\begin{MExerciseItems}
\item{$3 x + y = 4$ und $- x + 2 y = 1$,}
\item{$- x + 4 y = 5$ und $2 x - 8 y = - 10$.}
\end{MExerciseItems}

\begin{MHint}{Lösung}
\begin{MExerciseItems}
\item{Auflösen z.B. der 1.~Gleichung ($3 x + y = 4$) nach $y$ liefert $y = 4 - 3 x$. Dies kann man dann in die 2.~Gleichung
($- x + 2 y = 1$) einsetzen: $- x + 2 (4 - 3 x) = 1 \Leftrightarrow - x + 8 - 6 x = 1 \Leftrightarrow - 7 x = - 7
\Leftrightarrow x = 1$. Mit diesem Ergebnis für $x$ liefert in der Folge z.B. die 1.~Gleichung $3 \cdot 1 + y = 4
\Leftrightarrow y = 1$. Die Lösungsmenge $\ML$ lautet hier also ${\ML} = \{ \MPointTwo{1}{1} \}$.

Selbstverständlich könnte man auch anders beginnen: Man könnte z.B. die 1.~Gleichung nach $x$ auflösen und
das Ergebnis für $x$ dann in die 2.~Gleichung einsetzen, um $y$ zu bestimmen; oder man könnte generell mit
der 2.~Gleichung beginnen und diese im 1.~Schritt nach $x$ oder nach $y$ auflösen.
Es bestehen also einige Freiheiten in der Vorgehensweise.}
\item{Auflösen z.B. der 1.~Gleichung ($- x + 4 y = 5$) nach $x$ liefert $x = 4 y - 5$. Dies kann man dann in die 2.~Gleichung
($2 x - 8 y = - 10$) einsetzen: $2 ( 4 y - 5) - 8 y = - 10 \Leftrightarrow 8 y - 10 - 8y = - 10 \Leftrightarrow 0 = 0$.
Es entsteht also keine neue Aussage; mit anderen Worten: Die 2.~Gleichung enthält keine neue Information. Damit enthält
die Lösungsmenge $\ML$ in diesem Fall unendlich viele Lösungspaare $\MPointTwo{x}{y}$, die sich durch eine reelle Zahl $t$
parametrisieren lassen. Wählt man z.B. $y = t$, so lautet die Lösungsmenge ${\ML} = \{ \MPointTwo{4 t - 5}{t}  \MCondSetSep  t \in \R \}$.
Diese Lösungsmenge lässt sich als Gerade im zweidimensionalen Raum veranschaulichen:
\begin{center}
\MTikzAuto{%
\begin{tikzpicture}[x=1.0cm, y=2.0cm] 
%Koordinatensystem
\node (xMAX) at (3.4,0){};
\node (yMAX) at (0,3.4){};
\draw[->,color=black] (-8.2,0) -- (xMAX);
\foreach \x in {-8, -7, -6, -5, -4, -3, -2, -1, 0, 1, 2, 3}
\draw[shift={(\x,0)},color=black] (0pt,2pt) -- (0pt,-2pt) node[below right] {\scriptsize $\x$};
\draw[->,color=black] (0,-0.5) -- (yMAX);
\foreach \y in {0, 0.5, 1, 1.5, 2, 2.5, 3}
\draw[shift={(0,\y)},color=black] (2pt,0pt) -- (-2pt,0pt) node[above left] {\scriptsize $\y$};
%Achsenbeschriftung
%%\draw (xMAX) node[anchor=south] {$x$};
%%\draw (yMAX) node[anchor=west] {$y$};
%Beschriftung und Graphen
\clip(-8.2,-0.6) rectangle (3.4,3.4);
\draw[help lines, gray, dashed, xstep=0.5, ystep=0.5] (-9,-1) grid (4,4);  % was dotted
%%\fill[red!50!white, opacity=0.50] (1,1) rectangle (6,6);
\draw[color=black, thick] (-9.0,-1.0) -- (7.0,3.0);
\draw[color=black] (0.1,3.0) node[anchor=south west] {\Large $y=t$};
\draw[color=black] (3.3,0.25) node[anchor=east] {\Large $x=4t-5$};
%%\draw[color=black] (7,3) node[anchor=south west] {$\MPointTwo{7}{3}$};
\end{tikzpicture}
}%
%%\MUGraphicsSolo{unendl_Lsgsm_Aufg_1b.png}{scale=1}{width:450px}
\end{center}
Dementsprechend sind andere Parametrisierungen der Lösungsmenge möglich, z.B. indem man als freien Parameter $x \in \R$
wählt und die obige Gerade mit Hilfe ihrer Steigung und ihres $y$-Achsenabschnittes charakterisiert, also
${\ML} = \{ \MPointTwo{x}{ \Mtfrac14 x + \Mtfrac54 } \MCondSetSep x \in \R \}$.}
\end{MExerciseItems}
\end{MHint}
\end{MExercise}

\begin{MExercise}
Lösen Sie die folgenden Linearen Gleichungssysteme mit Hilfe der Additionsmethode:
\begin{MExerciseItems}
\item{$2 x + 4 y = 1$ und $x + 2 y = 3$,}
\item{$- 7 x + 11 y = 40$ und $2 x + 5 y = 13$.}
\end{MExerciseItems}

\begin{MHint}{Lösung}
\begin{MExerciseItems}
\item{Multipliziert man z.B. die 2.~Gleichung ($x + 2 y = 3$) mit $(- 2)$, so entsteht die Gleichung $(2')$:
$- 2 x - 4 y = - 6$. Die letzte Gleichung addiert man dann zur 1.~Gleichung ($2 x + 4 y = 1$):
$2 x + 4 y - 2 x - 4 y = 1 - 6 \Leftrightarrow 0 = - 5$. Dies ist ein Widerspruch! Somit ist die Lösungsmenge
$\ML$ für dieses Lineare Gleichungssystem leer: ${\ML} = \MEmptyset$.}
\item{Multiplikation der 1.~Gleichung ($- 7 x + 11 y = 40$) mit $2$ führt auf die Gleichung $(1')$:
$- 14 x + 22 y = 80$; Multiplikation der 2.~Gleichung mit $7$ auf die Gleichung $(2')$:
$14 x + 35 y = 91$. Anschließende Addition der Gleichungen $(1')$ und $(2')$ liefert:
$- 14 x + 22 y + 14 x + 35 y = 80 + 91 \Leftrightarrow 57 y = 171 \Leftrightarrow y = 3$. Setzt man dieses
Ergebnis für $y$ z.B. in die 2.~Gleichung ein, so entsteht: $2 x + 5 \cdot 3 = 13 \Leftrightarrow
2 x = 13 - 15 \Leftrightarrow 2 x = - 2 \Leftrightarrow x = - 1$. Damit lautet die Lösungsmenge $\ML$ hier:
${\ML} = \{ \MPointTwo{- 1}{ 3} \}$.}
\end{MExerciseItems}
\end{MHint}
\end{MExercise}

\begin{MExercise}
Lösen Sie das folgende Lineare Gleichungssystem graphisch: $2 x = 2$ und $x + 3 y = 4$.

\begin{MHint}{Lösung}
Die 1.~Gleichung ($2 x = 2$) ist äquivalent zu $x = 1$: Diese Gleichung beschreibt eine Gerade parallel zur
$y$-Achse durch den Punkt $\MPointTwo{1}{0}$ auf der $x$-Achse. Die 2.~Gleichung ($x + 3 y = 4$) kann umgeformt werden zu
$y = - \Mtfrac{1}{3} x + \Mtfrac{4}{3}$; dies beschreibt ebenfalls eine Gerade, diesmal mit der Steigung $- \Mtfrac{1}{3}$ und 
dem $y$-Achsenabschnitt $\Mtfrac{4}{3}$. Es ergibt sich also folgendes Bild:
\begin{center}
\MTikzAuto{%
\begin{tikzpicture}[x=1.8cm, y=1.8cm] 
%Koordinatensystem
\node (xMAX) at (2.8,0){};
\node (yMAX) at (0,2.8){};
\draw[->,color=black] (-1.0,0) -- (xMAX);
\foreach \x in {0, 1, 2}
\draw[shift={(\x,0)},color=black] (0pt,2pt) -- (0pt,-2pt) node[above right] {\scriptsize $\x$};
\draw[->,color=black] (0,-0.2) -- (yMAX);
\foreach \y in {1, 2}
\draw[shift={(0,\y)},color=black] (2pt,0pt) -- (-2pt,0pt) node[above right] {\scriptsize $\y$};
%Achsenbeschriftung
\draw (xMAX) node[anchor=south east] {$x$};
\draw (yMAX) node[anchor=north east] {$y$};
%Beschriftung und Graphen
\clip(-1.0,-0.2) rectangle (2.8,2.4);
\draw[help lines, gray, dashed, xstep=0.5, ystep=0.5] (-1,-0.5) grid (3,2.5);  % was dotted
\fill[color=black] (1,1) circle (2.0pt);
%%\fill[red!50!white, opacity=0.50] (1,1) rectangle (6,6);
\draw[color=blue, thick] (1.0,-0.2) -- (1.0,2.8);
\draw[color=blue, thick] (-2.0,2.0) -- (4.0,0.0);
\draw[color=black] (1.0,1.75) node[anchor=west] {Gerade 1: $x=1$};
\draw[color=black] (1.0,1.0) node[anchor=south west] {Schnittpunkt};
\draw[color=black] (2.0,0.7) node[anchor=north east] {Gerade 2: $y=-\frac{1}{3}x+\frac{4}{3}$};
\end{tikzpicture}
}%
%%\MUGraphicsSolo{Aufgabe_Schnittpunkt_Geraden_1.png}{scale=1}{width:450px}
\end{center}
Aus diesem Bild liest man die Koordinaten des Schnittpunktes zu $\MPointTwo{x = 1}{y = 1}$ ab; daher: ${\ML} = \{ \MPointTwo{1}{1} \}$.
\end{MHint}
\end{MExercise}

\end{MExercises}


\MSubsection{LGS mit drei Unbekannten}
\MLabel{M04_3_Unbekannte}

\begin{MIntro}
\MDeclareSiteUXID{VBKM04_DreiUnbekannte_Intro}
In der Folge wird der Schwierigkeitsgrad ein wenig gesteigert, indem man zur Behandlung etwas komplizierterer
Systeme übergeht:
\begin{MExample}
\MLabel{M04_einfuehrendes_Bsp_2}
Drei Kinder finden beim Spielen ein Portemonnaie mit $30$ Euro. Da meint das erste Kind: \glqq Wenn ich
das Geld für mich alleine behalte, so besitze ich doppelt so viel wie ihr beide zusammen!{\grqq} Woraufhin das
zweite Kind mit vor Stolz geschwellter Brust prahlt: \glqq Und wenn ich das gefundene Geld einfach einstecke, habe
ich dreimal so viel wie ihr beide!{\grqq} Das dritte Kind kann da nur überlegen schmunzeln: \glqq Und wenn ich's
nehme, so bin ich fünfmal so reich wie ihr beiden Gauner!{\grqq} Wieviel Geld besitzen die Kinder vor dem Fund?

Bezeichnet man die Geldsummen (in Euro) des ersten, zweiten bzw. dritten Kindes vor dem Portemonnaiefund
mit $x$, $y$ bzw. $z$, so kann man die Aussage des ersten Kindes wie folgt in eine mathematische Gleichung
übersetzen:
$$x + 30 = 2 (y + z) \Leftrightarrow x - 2 y - 2 z = - 30 \; : \MBlank\mbox{Gleichung}\MBlank (1) \MDFPeriod$$
Analog wird die Aussage des zweiten Kindes übertragen,
$$y + 30 = 3 (x + z) \Leftrightarrow - 3 x + y - 3 z = - 30 \; : \MBlank\mbox{Gleichung}\MBlank (2) \MDFPSpace ,$$
und schließlich diejenige des dritten Kindes:
$$z + 30 = 5 (x + y) \Leftrightarrow - 5 x - 5 y + z = - 30 \; : \MBlank\mbox{Gleichung}\MBlank (3) \MDFPeriod$$
Es entsteht also ein System aus drei linearen Gleichungen in drei Unbekannten, die hier $x, y$ und $z$ heißen.
\end{MExample}
Wen die Lösung dieses kleinen Rätsels interessiert, der wird sie weiter unten sowohl im Rahmen der
\textbf{Einsetzmethode} (siehe \MRef{M04_einf_Bsp_2_rech}) als auch im Rahmen der Additionsmethode
(siehe \MRef{M04_einf_Bsp_2_rech_2}) ausführlich vorgerechnet finden.
\begin{MInfo}
\MLabel{M04_3x3_system}
Ein Lineares Gleichungssystem, bestehend aus drei Gleichungen in den drei
Unbekannten $x, y$ und $z$, besitzt folgende allgemeine Gestalt:
\begin{eqnarray*}
a_{11} \cdot x + a_{12} \cdot y + a_{13} \cdot z & = & b_1 \MDFPSpace, \\
a_{21} \cdot x + a_{22} \cdot y + a_{23} \cdot z & = & b_2 \MDFPSpace, \\
a_{31} \cdot x + a_{32} \cdot y + a_{33} \cdot z & = & b_3 \MDFPeriod
\end{eqnarray*}
Dabei bezeichnen $a_{11}, a_{12}, a_{13}, a_{21}, a_{22}, a_{23}, a_{31}, a_{32}$ und $a_{33}$ die
\textbf{Koeffizienten} sowie $b_1, b_2$ und $b_3$ die rechten Seiten des \textbf{Linearen Gleichungssystems}.

Wiederum nennt man das \textbf{Lineare Gleichungssystem} \textbf{homogen}, falls die rechten Seiten $b_1, b_2$
und $b_3$ verschwinden ($b_1 = 0$, $b_2 = 0$, $b_3 = 0$). Andernfalls heißt das System \textbf{inhomogen}.
\end{MInfo}
\end{MIntro}

\begin{MXContent}{Lösbarkeit und Gleichsetzmethode, Graphische Interpretation}{Lösbarkeit}{STD}
\MDeclareSiteUXID{VBKM04_LoesbarkeitUndLoesungen3}
Im Falle von Systemen aus zwei linearen Gleichungen in zwei Unbekannten kann man - wie in \MRef{M04_2_Unbekannte}
gesehen - die Frage nach Lösbarkeit und Lösung des Systems sehr schön anschaulich in die Frage nach Existenz
und Lage des Schnittpunkts zweier Geraden übersetzen. Und natürlich sollte man sich überlegen, ob für Systeme aus
drei linearen Gleichungen in drei Unbekannten eine ähnlich bildhafte Interpretation gegeben werden kann.

Erweitert man den bisherigen Raum ($x$ und $y$) um eine weitere Dimension oder Variable, nämlich $z$, dann kann man mit einer linearen Gleichung dieser drei Variablen,
$$a_{11} \cdot x + a_{12} \cdot y + a_{13} \cdot z = b_1 \MDFPSpace , $$
eine \MEntry{Ebene}{Ebenengleichung} \textbf{in Koordinatenform} darstellen, ganz analog zu den Geradengleichungen, die bisher untersucht wurden. Für $a_{13} \neq 0$ kann diese
Gleichung nach $z$ aufgelöst werden,
$$z = \Mtfrac{b_1}{a_{13}} - \Mtfrac{a_{11}}{a_{13}} \cdot x - \Mtfrac{a_{12}}{a_{13}} \cdot y \MDFPSpace ,$$
sodass die \textbf{explizite Form} der Gleichung ebenjener Ebene resultiert. Die letzte Gleichung besagt, dass jedem Paar
$\MPointTwo{x}{y}$, also jedem Punkt der $x$-$y$-Koordinatenebene, gemäß der Vorschrift der rechten Seite ein Wert $z$, also
gewissermaßen eine Höhe, zugeordnet wird; dadurch entsteht eine Fläche über der $x$-$y$-Ebene, die aufgrund der Linearität
der Gleichung selbst eine Ebene ist.

Nun muss aber nicht nur die erste Gleichung im System \MRef{M04_3x3_system} gelten, sondern es müssen \textbf{gleichzeitig}
auch die zweite und die dritte Gleichung erfüllt sein, die bildlich ebenfalls als Ebenen interpretiert werden können. Wonach
also bei der Suche nach Lösungen eines Systems aus drei linearen Gleichungen in drei Unbekannten gefahndet wird, ist - in der
anschaulichen Sprache - das \textbf{Schnittverhalten} dreier Ebenen. Hierzu wird zunächst ein Beispiel betrachtet:
\begin{MExample}
\MLabel{M04_3x3_bsp_gleichsetz}
Es wird nach der Lösungsmenge des Linearen Gleichungssystems
$$\begin{array}{lcrcl} \mbox{Gleichung}\MBlank (1): & & x + y - z & = & 0 \MDFPSpace, \\
\mbox{Gleichung}\MBlank (2): & & x + y + z & = & 6 \MDFPSpace, \\
\mbox{Gleichung}\MBlank (3): & & 2 x - y + z & = & 4 \end{array}$$
gesucht; als Grundmenge wählt man die Menge der reellen Zahlen $\R$.

Jede der drei linearen Gleichungen lässt sich problemlos nach $z$ umstellen:
$$\begin{array}{lcrcl} \mbox{Gleichung}\MBlank (1'): & & z & = & x + y \MDFPSpace,\\
\mbox{Gleichung}\MBlank (2'): & & z & = & 6 - x - y \MDFPSpace,\\
\mbox{Gleichung}\MBlank (3'): & & z & = & 4 - 2 x + y \MDFPeriod \end{array}$$
\textbf{Setzt} man jetzt die rechten Seiten der Gleichungen~$(1')$ und $(2')$ \textbf{gleich}, so bedeutet dies -
anschaulich gesprochen -, dass man die \MEntry{Schnittgerade}{Schnittgerade} der beiden durch diese Gleichungen
beschriebenen Ebenen bestimmt:
$$x + y = 6 - x - y \Leftrightarrow 2 x + 2 y = 6 \Leftrightarrow y = 3 - x \; : \MBlank\mbox{Gleichung}\MBlank (A) \MDFPeriod $$
Durch Einsetzen dieser Beziehung in Gleichung~$(1')$ oder $(2')$ erhält man
die zugehörigen $z$-Koordinaten der Schnittgeraden; hier ergibt sich $z=3$.
Die folgende perspektivische Skizze zeigt diese Schnittgerade, Gleichung $(A)$, 
als den Schnitt der durch die Gleichungen $(1')$ und $(2')$ beschriebenen,
nicht parallelen Ebenen:
\begin{center}
\MTikzAuto{%
\tdplotsetmaincoords{60}{60}
\begin{tikzpicture}[tdplot_main_coords]
      \draw[thick,->] (-5,0,0) -- (6,0,0) node[anchor=north east]{\Large $x$};
      \draw[thick,->] (0,-5,0) -- (0,6,0) node[anchor=north west]{\Large $y$};
      \draw[thick,->] (0,0,-5) -- (0,0,5) node[anchor=south]{\Large $z$};
% Koordinaten-Box in der xy-Ebene
\pgfmathsetmacro{\ax}{-4}
\pgfmathsetmacro{\ay}{-4}
\pgfmathsetmacro{\az}{0}
\tdplottransformmainscreen{\ax}{\ay}{\az}
\pgfpathmoveto{\pgfpoint{\tdplotresx cm}{\tdplotresy cm}}
%
\pgfmathsetmacro{\ax}{-4}
\pgfmathsetmacro{\ay}{4}
\pgfmathsetmacro{\az}{0}
\tdplottransformmainscreen{\ax}{\ay}{\az}
\pgfpathlineto{\pgfpoint{\tdplotresx cm}{\tdplotresy cm}}
%
\pgfmathsetmacro{\ax}{4}
\pgfmathsetmacro{\ay}{4}
\pgfmathsetmacro{\az}{0}
\tdplottransformmainscreen{\ax}{\ay}{\az}
\pgfpathlineto{\pgfpoint{\tdplotresx cm}{\tdplotresy cm}}
%
\pgfmathsetmacro{\ax}{4}
\pgfmathsetmacro{\ay}{-4}
\pgfmathsetmacro{\az}{0}
\tdplottransformmainscreen{\ax}{\ay}{\az}
\pgfpathlineto{\pgfpoint{\tdplotresx cm}{\tdplotresy cm}}
%
\pgfpathclose
\pgfsetstrokecolor{black}
\pgfusepath{stroke}
% Beschriftungen
\draw[color=black,tdplot_main_coords] (-4.0,-4.0,0.0) node[anchor=north east] {$\MPointTwo{-4}{-4}$};
\draw[color=black,tdplot_main_coords] (-4.0,4.0,0.0) node[anchor=south] {$\MPointTwo{-4}{4}$};
\draw[color=black,tdplot_main_coords] (4.0,-4.0,0.0) node[anchor=north] {$\MPointTwo{4}{-4}$};
\draw[color=black,tdplot_main_coords] (4.0,4.0,0.0) node[anchor=north west] {$\MPointTwo{4}{4}$};
\draw[color=black,tdplot_main_coords] (0.0,0.0,-4.0) -- (0.0,0.1,-4.0) node[anchor=west] {$-16$};
\draw[color=black,tdplot_main_coords] (0.0,0.0,-2.0) -- (0.0,0.1,-2.0) node[anchor=west] {$-8$};
\draw[color=black,tdplot_main_coords] (0.0,0.0,2.0) -- (0.0,0.1,2.0) node[anchor=west] {$8$};
\draw[color=black,tdplot_main_coords] (0.0,0.0,4.0) -- (0.0,0.1,4.0) node[anchor=west] {$16$};
% Ebene z=x+y, Teil 1
\pgfmathsetmacro{\ax}{4}
\pgfmathsetmacro{\ay}{-1}
\pgfmathsetmacro{\az}{0.75} %3
\tdplottransformmainscreen{\ax}{\ay}{\az}
\pgfpathmoveto{\pgfpoint{\tdplotresx cm}{\tdplotresy cm}}
%
\pgfmathsetmacro{\ax}{4}
\pgfmathsetmacro{\ay}{4}
\pgfmathsetmacro{\az}{-0.5} % -2
\tdplottransformmainscreen{\ax}{\ay}{\az}
\pgfpathlineto{\pgfpoint{\tdplotresx cm}{\tdplotresy cm}}
%
\pgfmathsetmacro{\ax}{-1}
\pgfmathsetmacro{\ay}{4}
\pgfmathsetmacro{\az}{0.75} % 3
\tdplottransformmainscreen{\ax}{\ay}{\az}
\pgfpathlineto{\pgfpoint{\tdplotresx cm}{\tdplotresy cm}}
%
\pgfpathclose
\pgfsetstrokecolor{green}
\pgfsetfillcolor{green}
\pgfsetfillopacity{0.5}
\pgfusepath{stroke,fill}
% Ebene z=6-x-y
\pgfmathsetmacro{\ax}{-4}
\pgfmathsetmacro{\ay}{-4}
\pgfmathsetmacro{\az}{-2} % -8
\tdplottransformmainscreen{\ax}{\ay}{\az}
\pgfpathmoveto{\pgfpoint{\tdplotresx cm}{\tdplotresy cm}}
%
\pgfmathsetmacro{\ax}{-4}
\pgfmathsetmacro{\ay}{4}
\pgfmathsetmacro{\az}{0}
\tdplottransformmainscreen{\ax}{\ay}{\az}
\pgfpathlineto{\pgfpoint{\tdplotresx cm}{\tdplotresy cm}}
%
\pgfmathsetmacro{\ax}{4}
\pgfmathsetmacro{\ay}{4}
\pgfmathsetmacro{\az}{2} % 8
\tdplottransformmainscreen{\ax}{\ay}{\az}
\pgfpathlineto{\pgfpoint{\tdplotresx cm}{\tdplotresy cm}}
%
\pgfmathsetmacro{\ax}{4}
\pgfmathsetmacro{\ay}{-4}
\pgfmathsetmacro{\az}{0}
\tdplottransformmainscreen{\ax}{\ay}{\az}
\pgfpathlineto{\pgfpoint{\tdplotresx cm}{\tdplotresy cm}}
%
\pgfpathclose
\pgfsetstrokecolor{red}
\pgfsetfillcolor{red}
\pgfsetfillopacity{0.5}
\pgfusepath{stroke,fill}
% Ebene z=x+y, Teil 2
\pgfmathsetmacro{\ax}{-4}
\pgfmathsetmacro{\ay}{-4}
\pgfmathsetmacro{\az}{3.5} % 14
\tdplottransformmainscreen{\ax}{\ay}{\az}
\pgfpathmoveto{\pgfpoint{\tdplotresx cm}{\tdplotresy cm}}
%
\pgfmathsetmacro{\ax}{-4}
\pgfmathsetmacro{\ay}{4}
\pgfmathsetmacro{\az}{1.5} %6
\tdplottransformmainscreen{\ax}{\ay}{\az}
\pgfpathlineto{\pgfpoint{\tdplotresx cm}{\tdplotresy cm}}
%
\pgfmathsetmacro{\ax}{-1}
\pgfmathsetmacro{\ay}{4}
\pgfmathsetmacro{\az}{0.75} % 3
\tdplottransformmainscreen{\ax}{\ay}{\az}
\pgfpathlineto{\pgfpoint{\tdplotresx cm}{\tdplotresy cm}}
%
\pgfmathsetmacro{\ax}{4}
\pgfmathsetmacro{\ay}{-1}
\pgfmathsetmacro{\az}{0.75} % 3
\tdplottransformmainscreen{\ax}{\ay}{\az}
\pgfpathlineto{\pgfpoint{\tdplotresx cm}{\tdplotresy cm}}
%
\pgfmathsetmacro{\ax}{4}
\pgfmathsetmacro{\ay}{-4}
\pgfmathsetmacro{\az}{1.5} % 6
\tdplottransformmainscreen{\ax}{\ay}{\az}
\pgfpathlineto{\pgfpoint{\tdplotresx cm}{\tdplotresy cm}}
%
\pgfpathclose
\pgfsetstrokecolor{green}
\pgfsetfillcolor{green}
\pgfsetfillopacity{0.5}
\pgfusepath{stroke,fill}
% Gerade y=3-x
\pgfmathsetmacro{\ax}{4}
\pgfmathsetmacro{\ay}{-1}
\pgfmathsetmacro{\az}{0.75} % 3
\tdplottransformmainscreen{\ax}{\ay}{\az}
\pgfpathmoveto{\pgfpoint{\tdplotresx cm}{\tdplotresy cm}}
%
\pgfmathsetmacro{\ax}{-1}
\pgfmathsetmacro{\ay}{4}
\pgfmathsetmacro{\az}{0.75} % 3
\tdplottransformmainscreen{\ax}{\ay}{\az}
\pgfpathlineto{\pgfpoint{\tdplotresx cm}{\tdplotresy cm}}
%
\pgfsetlinewidth{1mm}
\pgfsetstrokecolor{yellow}
\pgfusepath{stroke}
% Legende
\pgfsetfillopacity{1.0}
\pgfsetlinewidth{0.4pt}
\fill[red, opacity=0.50, tdplot_main_coords] (-5.0,-5.0,-4.5) circle (1.5mm);
\draw[red, tdplot_main_coords] (-5.0,-5.0,-4.5) circle (1.5mm);
\fill[green, opacity=0.50, tdplot_main_coords] (-5.0,-5.0,-5.0) circle (1.5mm);
\draw[green, tdplot_main_coords] (-5.0,-5.0,-5.0) circle (1.5mm);
\fill[yellow, tdplot_main_coords] (-5.0,-5.0,-5.5) circle (1.0mm);
\draw[color=black] (-5.0,-5.0,-4.5) node[anchor=west] {\ Gl. $(1')$: $z=x+y$};
\draw[color=black] (-5.0,-5.0,-5.0) node[anchor=west] {\ Gl. $(2')$: $z=6-x-y$};
\draw[color=black] (-5.0,-5.0,-5.5) node[anchor=west] {\ Gl. $(A)$: $y=3-x$ ($z=3$)};
\end{tikzpicture}
}%
\end{center}
Die vollkommen analoge Aussage gilt, falls man die rechten Seiten von Gleichung~$(2')$ und $(3')$ \textbf{gleichsetzt};
man erhält dann die \textbf{Schnittgerade} der Ebenen~$(2)$ und $(3)$:
$$6 - x - y = 4 - 2 x + y \Leftrightarrow x - 2 y = - 2 \Leftrightarrow y = 1 + \Mtfrac12 x \; :
\MBlank\mbox{Gleichung}\MBlank (B) \MDFPeriod $$
Deren $z$-Koordinaten ergeben sich entsprechend durch Einsetzen dieser 
Beziehung in Gleichung $(2')$ oder $(3')$ zu $z=5-\frac{3}{2}x$.
Eine diesem Sachverhalt entsprechende perspektivische Skizze aus derselben
Blickrichtung zeigt diese Schnittgerade, Gleichung $(B)$, als den Schnitt
der durch die Gleichungen $(2')$ und $(3')$ beschriebenen,
nicht parallelen Ebenen:
\begin{center}
\MTikzAuto{%
\tdplotsetmaincoords{60}{60}
\begin{tikzpicture}[tdplot_main_coords]
      \draw[thick,->] (-5,0,0) -- (6,0,0) node[anchor=north east]{\Large $x$};
      \draw[thick,->] (0,-5,0) -- (0,6,0) node[anchor=north west]{\Large $y$};
      \draw[thick,->] (0,0,-5) -- (0,0,5) node[anchor=south]{\Large $z$};
% Koordinaten-Box in der xy-Ebene
\pgfmathsetmacro{\ax}{-4}
\pgfmathsetmacro{\ay}{-4}
\pgfmathsetmacro{\az}{0}
\tdplottransformmainscreen{\ax}{\ay}{\az}
\pgfpathmoveto{\pgfpoint{\tdplotresx cm}{\tdplotresy cm}}
%
\pgfmathsetmacro{\ax}{-4}
\pgfmathsetmacro{\ay}{4}
\pgfmathsetmacro{\az}{0}
\tdplottransformmainscreen{\ax}{\ay}{\az}
\pgfpathlineto{\pgfpoint{\tdplotresx cm}{\tdplotresy cm}}
%
\pgfmathsetmacro{\ax}{4}
\pgfmathsetmacro{\ay}{4}
\pgfmathsetmacro{\az}{0}
\tdplottransformmainscreen{\ax}{\ay}{\az}
\pgfpathlineto{\pgfpoint{\tdplotresx cm}{\tdplotresy cm}}
%
\pgfmathsetmacro{\ax}{4}
\pgfmathsetmacro{\ay}{-4}
\pgfmathsetmacro{\az}{0}
\tdplottransformmainscreen{\ax}{\ay}{\az}
\pgfpathlineto{\pgfpoint{\tdplotresx cm}{\tdplotresy cm}}
%
\pgfpathclose
\pgfsetstrokecolor{black}
\pgfusepath{stroke}
% Beschriftungen
\draw[color=black,tdplot_main_coords] (-4.0,-4.0,0.0) node[anchor=north east] {$\MPointTwo{-4}{-4}$};
\draw[color=black,tdplot_main_coords] (-4.0,4.0,0.0) node[anchor=south] {$\MPointTwo{-4}{4}$};
\draw[color=black,tdplot_main_coords] (4.0,-4.0,0.0) node[anchor=north] {$\MPointTwo{4}{-4}$};
\draw[color=black,tdplot_main_coords] (4.0,4.0,0.0) node[anchor=north west] {$\MPointTwo{4}{4}$};
\draw[color=black,tdplot_main_coords] (0.0,0.0,-4.0) -- (0.0,0.1,-4.0) node[anchor=west] {$-16$};
\draw[color=black,tdplot_main_coords] (0.0,0.0,-2.0) -- (0.0,0.1,-2.0) node[anchor=west] {$-8$};
\draw[color=black,tdplot_main_coords] (0.0,0.0,2.0) -- (0.0,0.1,2.0) node[anchor=west] {$8$};
\draw[color=black,tdplot_main_coords] (0.0,0.0,4.0) -- (0.0,0.1,4.0) node[anchor=west] {$16$};
% Ebene z=6-x-y, Teil 1
\pgfmathsetmacro{\ax}{-4}
\pgfmathsetmacro{\ay}{-1}
\pgfmathsetmacro{\az}{2.75} % 11
\tdplottransformmainscreen{\ax}{\ay}{\az}
\pgfpathmoveto{\pgfpoint{\tdplotresx cm}{\tdplotresy cm}}
%
\pgfmathsetmacro{\ax}{4}
\pgfmathsetmacro{\ay}{3}
\pgfmathsetmacro{\az}{-0.25} % -1
\tdplottransformmainscreen{\ax}{\ay}{\az}
\pgfpathlineto{\pgfpoint{\tdplotresx cm}{\tdplotresy cm}}
%
\pgfmathsetmacro{\ax}{4}
\pgfmathsetmacro{\ay}{4}
\pgfmathsetmacro{\az}{-0.5} % -2
\tdplottransformmainscreen{\ax}{\ay}{\az}
\pgfpathlineto{\pgfpoint{\tdplotresx cm}{\tdplotresy cm}}
%
\pgfmathsetmacro{\ax}{-4}
\pgfmathsetmacro{\ay}{4}
\pgfmathsetmacro{\az}{1.5} % 6
\tdplottransformmainscreen{\ax}{\ay}{\az}
\pgfpathlineto{\pgfpoint{\tdplotresx cm}{\tdplotresy cm}}
%
\pgfpathclose
\pgfsetstrokecolor{green}
\pgfsetfillcolor{green}
\pgfsetfillopacity{0.5}
\pgfusepath{stroke,fill}
% Ebene z=4-2x+y
\pgfmathsetmacro{\ax}{-4}
\pgfmathsetmacro{\ay}{-4}
\pgfmathsetmacro{\az}{2} % 8
\tdplottransformmainscreen{\ax}{\ay}{\az}
\pgfpathmoveto{\pgfpoint{\tdplotresx cm}{\tdplotresy cm}}
%
\pgfmathsetmacro{\ax}{-4}
\pgfmathsetmacro{\ay}{4}
\pgfmathsetmacro{\az}{4} % 16
\tdplottransformmainscreen{\ax}{\ay}{\az}
\pgfpathlineto{\pgfpoint{\tdplotresx cm}{\tdplotresy cm}}
%
\pgfmathsetmacro{\ax}{4}
\pgfmathsetmacro{\ay}{4}
\pgfmathsetmacro{\az}{0}
\tdplottransformmainscreen{\ax}{\ay}{\az}
\pgfpathlineto{\pgfpoint{\tdplotresx cm}{\tdplotresy cm}}
%
\pgfmathsetmacro{\ax}{4}
\pgfmathsetmacro{\ay}{-4}
\pgfmathsetmacro{\az}{-2} % -8
\tdplottransformmainscreen{\ax}{\ay}{\az}
\pgfpathlineto{\pgfpoint{\tdplotresx cm}{\tdplotresy cm}}
%
\pgfpathclose
\pgfsetstrokecolor{blue}
\pgfsetfillcolor{blue}
\pgfsetfillopacity{0.5}
\pgfusepath{stroke,fill}
% Ebene z=6-x-y, Teil 2
\pgfmathsetmacro{\ax}{-4}
\pgfmathsetmacro{\ay}{-1}
\pgfmathsetmacro{\az}{2.75} % 11
\tdplottransformmainscreen{\ax}{\ay}{\az}
\pgfpathmoveto{\pgfpoint{\tdplotresx cm}{\tdplotresy cm}}
%
\pgfmathsetmacro{\ax}{-4}
\pgfmathsetmacro{\ay}{-4}
\pgfmathsetmacro{\az}{3.5} % 14
\tdplottransformmainscreen{\ax}{\ay}{\az}
\pgfpathlineto{\pgfpoint{\tdplotresx cm}{\tdplotresy cm}}
%
\pgfmathsetmacro{\ax}{4}
\pgfmathsetmacro{\ay}{-4}
\pgfmathsetmacro{\az}{1.5} % 6
\tdplottransformmainscreen{\ax}{\ay}{\az}
\pgfpathlineto{\pgfpoint{\tdplotresx cm}{\tdplotresy cm}}
%
\pgfmathsetmacro{\ax}{4}
\pgfmathsetmacro{\ay}{3}
\pgfmathsetmacro{\az}{-0.25} % -1
\tdplottransformmainscreen{\ax}{\ay}{\az}
\pgfpathlineto{\pgfpoint{\tdplotresx cm}{\tdplotresy cm}}
%
\pgfpathclose
\pgfsetstrokecolor{green}
\pgfsetfillcolor{green}
\pgfsetfillopacity{0.5}
\pgfusepath{stroke,fill}
% Gerade y=1+(1/2)x
\pgfmathsetmacro{\ax}{-4}
\pgfmathsetmacro{\ay}{-1}
\pgfmathsetmacro{\az}{2.75} % 11
\tdplottransformmainscreen{\ax}{\ay}{\az}
\pgfpathmoveto{\pgfpoint{\tdplotresx cm}{\tdplotresy cm}}
%
\pgfmathsetmacro{\ax}{4}
\pgfmathsetmacro{\ay}{3}
\pgfmathsetmacro{\az}{-0.25} % -1
\tdplottransformmainscreen{\ax}{\ay}{\az}
\pgfpathlineto{\pgfpoint{\tdplotresx cm}{\tdplotresy cm}}
%
\pgfsetlinewidth{1mm}
\pgfsetstrokecolor{cyan}
\pgfusepath{stroke}
% Legende
\pgfsetfillopacity{1.0}
\pgfsetlinewidth{0.4pt}
%
\fill[green, opacity=0.50, tdplot_main_coords] (-5.0,-5.0,-4.2) circle (1.5mm);
\draw[green, tdplot_main_coords] (-5.0,-5.0,-4.2) circle (1.5mm);
\fill[blue, opacity=0.50, tdplot_main_coords] (-5.0,-5.0,-4.8) circle (1.5mm);
\draw[blue, tdplot_main_coords] (-5.0,-5.0,-4.8) circle (1.5mm);
\fill[cyan, tdplot_main_coords] (-5.0,-5.0,-5.4) circle (1mm);
%
\draw[color=black] (-5.0,-5.0,-4.2) node[anchor=west] {\ Gl. $(2')$: $z=6-x-y$};
\draw[color=black] (-5.0,-5.0,-4.8) node[anchor=west] {\ Gl. $(3')$: $z=4-2x+y$};
\draw[color=black] (-5.0,-5.0,-5.4) node[anchor=west] {\ Gl. $(B)$: $y=1+\frac{1}{2}x$};
\draw[color=black] (-5.0,-5.0,-6.0) node[anchor=west] {\ \hspace{1.3cm} ($z=5-\frac{3}{2}x$)};
\end{tikzpicture}
}%
\end{center}
Da im Ausgangssystem alle drei Gleichungen simultan gelten sollen, müssen auch die zwei gerade hergeleiteten Geradengleichungen
gleichzeitig erfüllt sein. Im bildlichen Kontext suchen man daher den Schnittpunkt dieser beiden Geraden;
diesen Schnittpunkt bekommt man, indem man die rechten Seiten der Gleichungen~$(A)$ und $(B)$ \textbf{gleichsetzt}:
$$3 - x = 1 + \Mtfrac12 x \Leftrightarrow \Mtfrac32 x = 2 \Leftrightarrow x = \Mtfrac43 \MDFPeriod $$
Den Lösungswert für $y$ kann man durch Einsetzen des Ergebnisses für $x$ z.B. in Gleichung~$(A)$ berechnen:
$$y = 3 - \Mtfrac43 \Leftrightarrow y = \Mtfrac53 \MDFPeriod $$
Die folgende Skizze zeigt die Schnittgeraden, Gleichungen $(A)$ und $(B)$,
in der $x$-$y$-Ebene, Projektion - Ansicht von oben -, und deren Schnittpunkt:
\begin{center}
\MTikzAuto{%
\begin{tikzpicture}[x=1.0cm, y=1.0cm] 
%Koordinatensystem
\node (xMAX) at (5.0,0){};
\node (yMAX) at (0,5.0){};
\draw[->,color=black] (-5.0,0) -- (xMAX);
\foreach \x in {-4, -3, -2, -1, 0, 1, 2, 3, 4}
\draw[shift={(\x,0)},color=black] (0pt,2pt) -- (0pt,-2pt) node[above right] {\scriptsize $\x$};
\draw[->,color=black] (0,-2.5) -- (yMAX);
\foreach \y in {-2, -1, 1, 2, 3, 4}
\draw[shift={(0,\y)},color=black] (2pt,0pt) -- (-2pt,0pt) node[above right] {\scriptsize $\y$};
%Achsenbeschriftung
\draw (xMAX) node[anchor=south east] {$x$};
\draw (yMAX) node[anchor=north east] {$y$};
%Beschriftung und Graphen
\clip(-4.1,-2.4) rectangle (4.1,4.1);
\draw[help lines, gray, dashed] (-4,-2) grid (4,4); % was dotted
%\fill[color=black] (1,1) circle (2.0pt);
%%\fill[red!50!white, opacity=0.50] (1,1) rectangle (6,6);
\draw[color=yellow, very thick] (-4.0,7.0) -- (4.0,-1.0);
\draw[color=cyan, very thick] (-4.0,-1.0) -- (4.0,3.0);
\fill[color=black] (1.3333333,1.6666666) circle (2.0pt);
\draw[color=black] (1.45,1.85) node[anchor=south] 
{$\MPointTwo{\frac{4}{3}}{\frac{5}{3}}$};
\draw[color=black] (4.0,-1.5) node[anchor=east] {Gl. $(A)$: $y=3-x$};
\draw[color=black] (-4.0,-1.5) node[anchor=west] {Gl. $(B)$: $y=1+\frac{1}{2}x$};
%\draw[color=black] (2.0,0.7) node[anchor=north east] {Gerade 2: $y=-\frac{1}{3}x+\frac{4}{3}$};
\end{tikzpicture}
}%
\end{center}
Der Wert für $z$ schließlich resultiert, indem man die Werte für $x$ und $y$ z.B. in Gleichung~$(1')$ einsetzt:
$$z = \Mtfrac43 + \Mtfrac53 \Leftrightarrow z = \Mtfrac93 \Leftrightarrow z = 3 \MDFPeriod $$
Das gegebene Lineare Gleichungssystem ist also eindeutig lösbar; für die Lösungsmenge erhält man
${\ML} = \{ \MPointThree{x = \Mtfrac43}{y = \Mtfrac53}{z = 3} \}$.
\end{MExample}
Um die Frage der Lösbarkeit eines Systems aus drei linearen Gleichungen in
drei Unbekannten allgemeiner und etwas genauer unter die Lupe zu nehmen,
soll noch für einen Moment in der bildhaften Welt verweilt werden:
\begin{itemize}
\item{Liegen (mindestens) \textbf{zwei} der drei \textbf{Ebenen parallel} zueinander (ohne deckungsgleich zu sein),
so besitzt das System \textbf{keine Lösung}: Parallele (nicht deckungsgleiche) Ebenen schneiden sich nicht;
daher ist es unmöglich, die zu den Ebenen gehörenden Gleichungen simultan zu erfüllen.}
\item{Sind \textbf{zwei} der drei \textbf{Ebenen deckungsgleich}, so wird die Schnittmenge mit der dritten (nicht
parallelen und nicht deckungsgleichen) Ebene durch eine \textbf{Schnittgerade} gebildet; alle Punkte dieser
Schnittgeraden repräsentieren Lösungen des Systems, die \textbf{Lösungsmenge} ist daher \textbf{unendlich mächtig}.}
\item{Sind alle \textbf{drei Ebenen deckungsgleich}, so sind alle Punkte der (deckungsgleichen) Ebene(n) Lösungen des Systems;
wiederum ist die \textbf{Lösungsmenge unendlich mächtig}.}
\item{Eine eindeutige Lösung kann höchstens in diesem letzten Fall eintreten: Die drei (nicht parallelen und nicht
deckungsgleichen) Ebenen führen auf \textbf{drei Schnittgeraden} (Ebene~$(1)$ mit Ebene~$(2)$, Ebene~$(2)$ mit Ebene~$(3)$,
Ebene~$(1)$ mit Ebene~$(3)$):
\begin{itemize}
\item{Liegen \textbf{zwei Schnittgeraden parallel} zueinander, so besitzt das System \textbf{keine Lösung}.}
\item{Sind \textbf{zwei Schnittgeraden deckungsgleich}, so weist das System \textbf{unendlich viele Lö\-sun\-gen} auf.}
\item{\textbf{Schneiden sich die Schnittgeraden in einem Punkt}, so ist die \textbf{Lösung eindeutig}
und die \textbf{Lösungsmenge} besteht \textbf{aus genau einem Element}.}
\end{itemize}}
\end{itemize}
Man sieht, dass trotz der anschaulichen Interpretation das genaue Auseinanderdividieren der einzelnen Fälle doch einigermaßen
kompliziert ist. Um so wichtiger wird es sein, gerade wenn die Systeme noch größer werden und die Anschauung schwieriger
bzw. unmöglich wird, rechnerische Methoden zur Hand zu haben, um die Lösbarkeit Linearer Gleichungssysteme zu
untersuchen und deren Lösungsmengen zu bestimmen. Das \MEntry{Additionsverfahren}{Additionsverfahren}, dessen
Diskussion weiter unten erneut aufgegriffen wird, wird ein solch geeignetes Verfahren sein.

Übrigens ist es im obigen Beispiel \MRef{M04_3x3_bsp_gleichsetz} nicht nötig, die dritte Schnittgerade zu bestimmen und
zu überprüfen, dass diese dritte Schnittgerade die beiden anderen Geraden in deren Schnittpunkt schneidet: Dies ist
automatisch gewährleistet, da durch das Gleichsetzen der rechten Seite von Gleichung~$(1')$ mit derjenigen von
Gleichung~$(2')$ (erste Schnittgerade / Gleichung~$(A)$) und durch das Gleichsetzen der rechten Seite von
Gleichung~$(2')$ mit derjenigen von Gleichung~$(3')$ (zweite Schnittgerade / Gleichung~$(B)$) die Gültigkeit der
Gleichung für die dritte Schnittgerade (rechte Seite von $(1')$ = rechte Seite von $(3')$),
$$x + y = 4 - 2 x + y \Leftrightarrow 3 x = 4 \; : \MBlank\mbox{Gleichung}\MBlank (C) \MDFPSpace ,$$
garantiert ist.

Im Beispiel wurde als rechnerisches Verfahren die \MSRef{M04_gleichsetzmethode}{Gleichsetzmethode} verwendet, da sie
sehr eng mit der anschaulich geometrischen Interpretation zusammenhängt. Durch das \textbf{Gleichsetzen} expliziter
Ebenen- bzw. Geradengleichungen werden ja genau die Schnittgeraden bzw. -punkte bestimmt.
\begin{MInfo}
Bei der Bei der \MEntry{Gleichsetzmethode}{Gleichsetzmethode} werden die drei linearen Gleichungen in einem ersten
Schritt nach einer der Unbekannten - oder nach einem Vielfachen einer der Unbekannten - umgestellt. Die
resultierenden neuen Gleichungen werden dann im zweiten Schritt paarweise \textbf{gleichgesetzt}, wobei es genügt,
dies für zwei Paare durchzuführen. Insgesamt entsteht ein \textbf{Lineares
Gleichungssystem}, das nur noch aus \textbf{zwei} Gleichungen in (den verbliebenen) \textbf{zwei} Unbekannten
aufgebaut ist und das anschließend mit den Methoden des Abschnitts \MRef{M04_2_Unbekannte} bearbeitet werden kann.
\end{MInfo}

\begin{MExercise}
Bestimmen Sie die Lösungsmenge des Linearen Gleichungssystems
\begin{eqnarray*}
- x + z & = & 2 \MDFPSpace, \\ - x + y + 2 z & = & 1 \MDFPSpace, \\ y + z & = & 11 \MDFPeriod
\end{eqnarray*}
Verwenden Sie die Gleichsetzmethode und gehen Sie geschickt vor!

\begin{MHint}{Lösung}
Die erste Gleichung hängt von vornherein nicht von der Unbekannten $y$ ab; daher bietet es sich an, aus der
zweiten und dritten Gleichung zunächst ebenfalls $y$ zu eliminieren. Man löst dazu die zweite und die dritte
Gleichung jeweils nach $y$ auf,
$$y = 1 + x - 2 z \; \MBlank\mbox{und}\MBlank \; y = 11 - z \MDFPSpace ,$$
und setzt anschließend die rechten Seiten gleich:
$$1 + x - 2 z = 11 - z \Leftrightarrow - x + z = - 10 \MDFPeriod $$
Als Nächstes muss man diese letzte Gleichung und die erste Gleichung des ursprünglichen Systems
weiterverarbeiten. Beim Betrachten dieser beiden Gleichungen fällt auf, dass die linken Seiten (die
Kombination $- x + z$ der Unbekannten $x$ und $z$) identisch sind, wohingegen die rechten Seiten
(die Zahlenwerte $2$ und $-10$) offensichtlich nicht übereinstimmen. Daher stößt man auf einen
Widerspruch und das gegebene Lineare Gleichungssystem besitzt keine Lösung, ${\ML} = \MEmptyset$.\\
Andere, ebenso geschickte, Lösungswege sind möglich.
\end{MHint}
\end{MExercise}
\end{MXContent}

\begin{MXContent}{Die Einsetzmethode}{Einsetzmethode}{STD}
\MDeclareSiteUXID{VBKM04_Einsetzmethode3}
Die \textbf{Einsetzmethode} wurde bereits im Rahmen von Systemen, bestehend aus zwei Gleichungen
in zwei Unbekannten, behandelt, siehe \MRef{M04_einsetz_gleichsetz}. Auch im vorliegenden Fall eines
Systems vom Typ \MRef{M04_3x3_system} ändert sich an der Vorgehensweise nichts Prinzipielles:
\begin{MExample}
\MLabel{M04_einf_Bsp_2_rech}
Das einleitende Beispiel \MRef{M04_einfuehrendes_Bsp_2} dieses Abschnitts wird wieder aufgegriffen; das Lineare
Gleichungssystem im Falle des Rätsels um die drei in Versuchung geführten Kinder lautet:
$$\begin{array}{lcrcl} \mbox{Gleichung}\MBlank (1): & & x - 2 y - 2 z & = & - 30 \MDFPSpace, \\
\mbox{Gleichung}\MBlank (2): & & - 3 x + y - 3 z & = & - 30 \MDFPSpace, \\
\mbox{Gleichung}\MBlank (3): & & - 5 x - 5 y + z & = & - 30 \MDFPeriod \end{array}$$
Man kann mit der Lösung z.B. so beginnen, dass man Gleichung~$(1)$ nach $x$ freistellt,
$$x = 2 y + 2 z - 30 \; : \MBlank\mbox{Gleichung}\MBlank (1') \MDFPeriod$$
\textbf{Setzt} man diese Gleichung in die Gleichungen~$(2)$ und $(3)$ \textbf{ein}, so wird aus den
letztgenannten Gleichungen die Unbekannte $x$ eliminiert:
$$- 3 (2 y + 2 z - 30) + y - 3 z = - 30 \Leftrightarrow - 5 y - 9 z = - 120 \; : \MBlank\mbox{Gleichung}\MBlank (2') \MDFPSpace, $$
$$- 5 (2 y + 2 z - 30) - 5 y + z = - 30 \Leftrightarrow - 15 y - 9 z = - 180 \; : \MBlank\mbox{Gleichung}\MBlank (3') \MDFPeriod $$
Man erhält also in diesem Zwischenschritt ein System aus \textbf{zwei} linearen Gleichungen in den \textbf{zwei}
Unbekannten $y$ und $z$, das mit den Methoden des vorigen Abschnitts \MRef{M04_2_Unbekannte} weiterbehandelt wird.\\
So kann man z.B. Gleichung~$(2')$ nach $y$ auflösen,
$$y = 24 - \Mtfrac95 z \MDFPSpace ,$$
und diesen Ausdruck in Gleichung~$(3')$ \textbf{einsetzen}:
$$- 15 (24 - \Mtfrac95 z) - 9 z = - 180 \Leftrightarrow 360 - 27 z + 9 z = 180 \Leftrightarrow 18 z = 180
\Leftrightarrow z = 10 \MDFPeriod$$
Damit ergibt sich $y$ mit Gleichung~$(2')$ zu %%%
$$y = 24 - \Mtfrac95 \cdot 10 = 24 - 9 \cdot 2 = 6 $$
und schließlich $x$ (mit Hilfe von beispielsweise Gleichung~$(1')$):
$$x = 2 \cdot 6 + 2 \cdot 10 - 30 = 12 + 20 - 30 = 2 \MDFPeriod$$
Das Lineare Gleichungssystem besitzt daher eine eindeutige Lösung; die Lösungsmenge $\ML$ enthält genau
ein Element, ${\ML} = \{ \MPointThree{x = 2}{y = 6}{z = 10} \}$. Damit hatte das erste Kind vor dem Portemonnaie-Fund 2, das zweite Kind 6 und das dritte Kind 10 Euro.%%%
\end{MExample}
Vielleicht fragt sich der eine oder andere im Zusammenhang mir dem voranstehenden Beispiel, ob es in Anbetracht
der identischen rechten Seiten - alle gleich $- 30$ - nicht praktischer wäre, einfach die linken Seiten paarweise
gleichzusetzen und mit den resultierenden Gleichungen weiterzuarbeiten.

Ein solches Vorgehen ist allerdings nicht
zielführend und - wenn man nicht genau aufpasst, was man tut - unter Umständen sogar falsch. Jedenfalls würde es
keine Reduktion in der Anzahl der Unbekannten nach sich ziehen. Und das ist ja gerade die Intention, die sowohl
hinter der Einsetz- als auch hinter der Gleichsetzmethode steckt: In beiden Verfahren (und
auch in der Additionsmethode) geht es darum, in einem ersten und zweiten Schritt eine der Unbekannten
zu eliminieren, sodass nur noch ein System aus zwei linearen Gleichungen in zwei Unbekannten (und damit ein
einfacheres Problem) verbleibt.

Übrigens, wie man dann dieses reduzierte Problem löst, kann unabhängig von der Art und Weise,
wie man begonnen hat, gewählt werden.
Mit anderen Worten, es ist durchaus zulässig und - vom rechentechnischen Standpunkt aus gesehen - eventuell
sogar clever, beim Lösen eines Linearen Gleichungssystems
mit drei Gleichungen in drei Unbekannten mit einer Methode, z.B. der Einsetzmethode, zu beginnen,
um die Problemstellung auf ein
System aus zwei Gleichungen in zwei Unbekannten sozusagen \glqq herunterzubrechen{\grqq},
und dieses einfachere System dann mit einer anderen Methode, z.B. der
Gleichsetzmethode, weiterzubehandeln. In diesem Sinne können die Verfahren also gemischt werden.
\begin{MInfo}
Bei der Einsetzmethode wird eine der drei linearen Gleichungen in einem ersten Schritt
nach einer der Unbekannten - oder nach einem Vielfachen einer der Unbekannten - freigestellt; dieses Ergebnis wird im
zweiten Schritt in die beiden anderen linearen Gleichungen \textbf{eingesetzt}. Es entsteht ein \textbf{Lineares
Gleichungssystem}, das nur noch aus \textbf{zwei} Gleichungen in (den verbliebenen) \textbf{zwei} Unbekannten
aufgebaut ist und das anschließend mit den Methoden des Abschnitts \MRef{M04_2_Unbekannte} bearbeitet werden kann.
\end{MInfo}
\end{MXContent}

\begin{MXContent}{Die Additionsmethode}{Additionsmethode}{STD}
\MLabel{M04_3x3_addition}
\MDeclareSiteUXID{VBKM04_Additionsmethode3}
Der Grundgedanke der Additionsmethode, der auch schon früher andiskutiert wurde (siehe \MRef{M04_addition}),
besteht darin, Gleichungen des Systems so zu \textbf{addieren}, dass in der resultierenden Gleichung
nur noch eine reduzierte Anzahl an Unbekannten auftritt. Dazu ist es meist erforderlich, vor der \textbf{Addition}
eine der Gleichungen mit einem geschickt gewählten Faktor zu multiplizieren.

Die Additionsmethode soll für Systeme aus drei Gleichungen in drei Unbekannten
gleich in einer Form vorgestellt werden, die sich leicht auf größere Systeme übertragen lässt. Dazu bedient man sich zur
Illustration der Vorgehensweise noch einmal des einführenden Beispiels \MRef{M04_einfuehrendes_Bsp_2}, also
des Beispiels der möglicherweise diebischen Kinder:
\begin{MExample}
\MLabel{M04_einf_Bsp_2_rech_2}
Das zu einer erneuten Lösung auffordernde System linearer Gleichungen lautet also:
$$\begin{array}{lcrcl} \mbox{Gleichung}\MBlank (1): & & x - 2 y - 2 z & = & - 30 \MDFPSpace, \\
\mbox{Gleichung}\MBlank (2): & & - 3 x + y - 3 z & = & - 30 \MDFPSpace, \\
\mbox{Gleichung}\MBlank (3): & & - 5 x - 5 y + z & = & - 30 \MDFPeriod \end{array}$$
Gleichung~$(1)$ wird im Folgenden unverändert beibehalten. Gleichung~$(2)$ jedoch soll durch eine
neue Gleichung ersetzt werden, die durch \textbf{Addition} von Gleichung~$(2)$ mit der mit dem Faktor $3$
durchmultiplizierten Gleichung~$(1)$ - kurz notiert als $(2) + 3 \cdot (1)$ - gewonnen wird:
$$(-3 x + y - 3 z) + 3 \cdot (x - 2 y - 2 z) = - 30 + 3 \cdot (- 30) \Leftrightarrow
- 5 y - 9 z = - 120 \; : \MBlank\mbox{Gleichung}\MBlank (2') \MDFPeriod $$
Ähnlich geht man mit Gleichung~$(3)$ vor: Sie wird durch $(3) + 5 \cdot (1)$, also durch die
\textbf{Summe} aus Gleichung~$(3)$ und der mit dem Faktor $5$ durchmultiplizierten Gleichung~$(1)$ ersetzt:
$$(- 5 x - 5 y + z) + 5 \cdot (x - 2 y - 2 z) = - 30 + 5 \cdot (- 30) \Leftrightarrow
-15 y - 9 z = - 180 \; : \MBlank\mbox{Gleichung}\MBlank (3') \MDFPeriod $$
Das System sieht jetzt folgendermaßen aus:
$$\begin{array}{lcrcl} \mbox{Gleichung}\MBlank (1): & & x - 2 y - 2 z & = & - 30 \MDFPSpace, \\
\mbox{Gleichung}\MBlank (2'): & & - 5 y - 9 z & = & - 120 \MDFPSpace, \\
\mbox{Gleichung}\MBlank (3'): & & - 15 y - 9 z & = & - 180  \MDFPeriod \end{array}$$
Aus den Gleichungen~$(2')$ und $(3')$ ist die Abhängigkeit von der Unbekannten $x$ verschwunden - das war das
Ziel und der Grund für die Faktoren $3$ bzw. $5$ bei den obigen Summationen.

Man könnte nun das Untersystem, bestehend aus den zwei Gleichungen~$(2')$ und $(3')$ in den zwei Unbekannten
$y$ und $z$ mit einer anderen Methode, z.B. der Einsetzmethode, weiterbehandeln. Doch es soll lieber vollständig
innerhalb der Additionsmethode gearbeitet werden: Dazu nimmt man im Folgenden Gleichung~$(2')$ - ebenso wie Gleichung~$(1)$ -
von Änderungen aus; Gleichung~$(3')$ soll aber nochmals ersetzt werden und zwar durch die Summe
$(3') + (- 3) \cdot (2')$:
$$(- 15 y - 9 z) + (- 3) \cdot (- 5 y - 9 z) = - 180 + (- 3) \cdot (- 120) \Leftrightarrow
18 z = 180 \; : \MBlank\mbox{Gleichung}\MBlank ({3'}') \MDFPeriod $$
Das System hat ein weiteres Mal sein Aussehen gewandelt,
$$\begin{array}{lcrcl} \mbox{Gleichung}\MBlank (1): & & x - 2 y - 2 z & = & - 30 \MDFPSpace, \\
\mbox{Gleichung}\MBlank (2'): & & - 5 y - 9 z & = & - 120 \MDFPSpace, \\
\mbox{Gleichung}\MBlank ({3'}'): & & 18 z & = &  180 \MDFPSpace, \end{array}$$
und besitzt nun - zumindest was die linken Seiten anbelangt - eine Art \textbf{Dreiecksform}.%%%

Die Bestimmung der Unbekannten ist nun äußerst einfach: Die unterste Gleichung (Gleichung~$({3'}')$) hängt nur noch
von einer einzigen Unbekannten, nämlich $z$, ab und kann daher sofort aufgelöst werden, $z = 10$.\\
Mit diesem Resultat für $z$ geht man in die darüber stehende Gleichung (Gleichung~$(2')$), die dann unmittelbar
das Ergebnis für die nächste Unbekannte, hier $y$, liefert: $- 5 y - 9 \cdot 10 = - 120 \Leftrightarrow - 5 y =
- 30 \Leftrightarrow y = 6$.\\
Schließlich ergeben $y$ und $z$ in die oberste Gleichung (Gleichung~$(1)$) eingesetzt sofort die Lösung
für die verbleibende Unbekannte, im vorliegenden Beispiel $x$: $x - 2 \cdot 6 - 2 \cdot 10 = - 30 \Leftrightarrow
x = 2$.\\
Also ist die Lösungsmenge gegeben durch ${\ML} = \{ \MPointThree{x = 2}{y = 6}{z = 10} \}$.%%%
\end{MExample}
Die aufmerksame Leserin, der aufmerksame Leser mag sich beim Studium des obigen Beispiels unter Umständen die Frage
stellen, ob - und wenn ja, warum - es zulässig ist, in einem System eine Gleichung durch eine andere zu ersetzen.%%%
Im Beispiel geschieht dies an drei Punkten, etwa wenn an die Stelle der Gleichung~$(2)$ die Kombination Gleichung~$(2)$
plus $3$ mal Gleichung~$(1)$, also Gleichung~$(2')$, tritt.

Wenn man nach der Lösung eines Gleichungssystems forscht, verlangt man die \textbf{gleichzeitige Gül\-tig\-keit}
aller Gleichungen des Systems. Damit insistiert man - um im Beispiel \MRef{M04_einf_Bsp_2_rech_2} zu sprechen - mit
der Forderung, dass Gleichung~$(1)$ \textbf{und} Gleichung~$(2)$ gelten sollen, klarerweise auch auf der Gültigkeit von
$$\mbox{Gleichung}\MBlank (2) + 3 \cdot \MBlank\mbox{Gleichung}\MBlank (1) \Leftrightarrow \MBlank\mbox{Gleichung}\MBlank (2') \MDFPeriod $$
Gelten nun stattdessen simultan Gleichung~$(1)$ \textbf{und} Gleichung~$(2')$, 
dann folgt umgekehrt auch die Gültigkeit von Gleichung~$(2)$ mit
$$\mbox{Gleichung}\MBlank (2') + (-1) \cdot \MBlank\mbox{Gleichung}\MBlank (1) \Leftrightarrow 3\cdot\MBlank\mbox{Gleichung}\MBlank (2) \Leftrightarrow \MBlank\mbox{Gleichung}\MBlank (2) \MDFPeriod $$
Also ist es zulässig, im System Gleichung~$(2)$ durch Gleichung~$(2')$ zu ersetzen.

Zugleich erkennt man hier einen weiteren wichtigen Punkt: Würde man die \textbf{eine} Gleichung~$(2')$ an die
Stelle der \textbf{beiden} Gleichungen - Gleichung~$(1)$ und Gleichung~$(2)$ - treten lassen, so würde man an
Information einbüßen und sogar einen Fehler begehen. (Die Forderung von \textbf{nur} $(2')$ statt $(1)$ \textbf{und}
$(2)$ ist eine wesentlich schwächere.) Dies ist der Grund dafür, warum man in die \glqq neuen{\grqq} Systeme einige
Gleichungen ungeändert übernimmt: Gleichung~$(1)$ \textbf{und} Gleichung~$(2)$ sind in den jeweiligen Systemen äquivalent
zu Gleichung~$(1)$ und Gleichung~$(2')$. Analoges trifft natürlich auch auf die anderen Ersetzungen in obigem
Beispiel zu - und allgemeiner bei derartigen Umformungen Linearer Gleichungssysteme innerhalb der Additionsmethode.
\begin{MInfo}
Bei der \MEntry{Additionsmethode}{Additionsmethode} werden Paare linearer Gleichungen des Systems - gegebenenfalls
nach der Multiplikation (mindestens) einer der beiden Gleichungen mit einem geschickt gewählten Faktor (bzw. mit
geschickt gewählten Faktoren) - mit dem Ziel addiert, dass in den resultierenden Gleichungen (zumindest) eine Unbekannte
herausfällt. Dabei ist darauf zu achten, dass bei der fortschreitenden Lösungsfindung keine Information verloren geht,
sprich, dass die Anzahl der (informationsrelevanten) Gleichungen stets erhalten bleibt. Am geschicktesten bringt man
in diesem Zuge das Gleichungssystem auf eine \textbf{Dreiecksform}. Dann ist das anschließende Auffinden der Lösung
besonders einfach.
\end{MInfo}
%\begin{MInfo}
%Die systematische Verallgemeinerung der vorstehend beschriebenen Vorgehensweise bei der Additionsmethode auf Systeme
%aus $n$ linearen Gleichungen - wobei $n$ eine natürliche Zahl bezeichnet ($n \in \N$) - führt auf das sogenannte
%\MEntry{Gaußsche Eliminationsverfahren}{Gaußsches Eliminationsverfahren}. Dieses \textbf{Gaußsche Eliminationsverfahren}
%stellt also einen Algorithmus zur Lösung beliebig großer Linearer Gleichungssysteme dar.
%\end{MInfo}
\end{MXContent}

\begin{MExercises}
\MDeclareSiteUXID{VBKM04_Additionsmethode3_Exercises}
\begin{MExercise}
Geben Sie die Lösungsmenge für das Lineare Gleichungssystem
\begin{eqnarray*}
2 x - y + 5 z & = & 1 \MDFPSpace, \\ 11 x + 8 z & = & 2 \MDFPSpace, \\ - 4 x + y - 3 z & = & - 1
\end{eqnarray*}
an. Verwenden Sie zum Lösen
\begin{MExerciseItems}
\item{die Einsetzmethode,}
\item{die Additionsmethode.}
\end{MExerciseItems}

\begin{MHint}{Lösung}
\begin{MExerciseItems}
\item{Man stellt z.B. die erste Gleichung ($2 x - y + 5 z = 1$) nach $y$ frei, $y = 2 x + 5 z - 1$,
und setzt dieses Ergebnis in die dritte Gleichung ($- 4 x + y - 3 z = - 1$) ein:
$$- 4 x + (2 x + 5 z - 1) - 3 z = - 1 \Leftrightarrow - 2 x + 2 z = 0 \Leftrightarrow z = x \MDFPeriod$$
Das letzte Ergebnis, das schon nach der Unbekannten $z$ aufgelöst wurde, setzt man in die zweite
Gleichung ein ($11 x + 8 z = 2$), die von vornherein unabhängig von $y$ ist:
$$11 x + 8 x = 2 \Leftrightarrow 19 x = 2 \Leftrightarrow x = \Mtfrac{2}{19} \MDFPeriod$$
Damit gilt auch
$$z = \Mtfrac{2}{19} \MDFPSpace ,$$
und für $y$ erhält man - gemäß der Gleichung aus der ersten Umformung -:
$$y = 2 \cdot \Mtfrac{2}{19} + 5 \cdot \Mtfrac{2}{19} - 1 = \Mtfrac{4 + 10 - 19}{19} = \Mtfrac{-5}{19} \MDFPeriod$$
Also ist das Lineare Gleichungssystem eindeutig lösbar mit ${\ML} = \{ \MPointThree{x = \Mtfrac{2}{19}}{
y = - \Mtfrac{5}{19}}{ z = \Mtfrac{2}{19}} \}$.\\
Alternative Lösungswege sind ebenso gut möglich.}
\item{Addition der ersten und der dritten Gleichung eliminiert die Unbekannte $y$:
$$(2 x - y + 5 z) + (- 4 x + y - 3 z) = 1 + (- 1) \Leftrightarrow - 2 x + 2 z = 0 \MDFPeriod$$
Multipliziert man die letzte Gleichung mit $(- 4)$, so entsteht
$$8 x - 8 z = 0 \MDFPSpace ,$$
und die anschließende Addition der zweiten Gleichung ($11 x  + 8 z = 2$) sorgt dafür, dass die Abhängigkeit
von $z$ herausfällt:
$$(8 x - 8 z) + (11 x + 8 z) = 0 + 2 \Leftrightarrow 19 x = 2 \Leftrightarrow x = \Mtfrac{2}{19} \MDFPeriod$$
Wie im ersten Aufgabenteil können in der Folge $z$ und $y$ bestimmt werden; natürlich bekommt man
dasselbe Ergebnis ${\ML} = \{ \MPointThree{x = \Mtfrac{2}{19}}{y = - \Mtfrac{5}{19}}{z = \Mtfrac{2}{19}} \}$.\\
Alternative Lösungswege sind ebenso gut möglich.}
\end{MExerciseItems}
\end{MHint}
\end{MExercise}

\begin{MExercise}
Die folgende einfache Schaltung soll betrachtet werden:
\begin{center}
\MTikzAuto{%
\begin{tikzpicture}[x=0.75cm, y=0.75cm] 
%Koordinatensystem
\node (xMAX) at (8.6,0){};
\node (yMAX) at (0,5.0){};
\draw[->,color=black] (0.0,0.0) -- (0.0,2.5);
\draw[color=black] (0.0,2.5) -- (0.0,5.0) -- (7.3,5.0);
\draw[color=black] (0.0,0.0) -- (1.2,0.0);
\draw[color=black] (1.2,-0.8) -- (1.2,0.8);
\fill[color=black, rounded corners=1.0pt] (1.5,-0.4) rectangle (1.6,0.4);
\draw[->,color=black] (1.6,0.0) -- (8.6,0.0) -- (8.6,5.0) -- (7.3,5.0);
\draw[->,color=black] (6.0,0.0) -- (6.0,4.25);
\draw[color=black] (6.0,4.25) -- (6.0,5.0);
%Achsenbeschriftung
%%\draw (xMAX) node[anchor=south] {$x$};
%%\draw (yMAX) node[anchor=west] {$y$};
%Beschriftung und Graphen
%%\clip(-8.2,-0.6) rectangle (3.4,3.4);
\fill[black] (2.0,4.5) rectangle (4.0,5.5);
\fill[black] (5.5,1.5) rectangle (6.5,3.5);
\fill[black] (8.1,1.5) rectangle (9.1,3.5);
\draw[color=black] (0.0,2.5) node[anchor=east] {$I_1$};
\draw[color=black] (6.0,4.25) node[anchor=west] {$I_2$};
\draw[color=black] (7.3,5.0) node[anchor=south] {$I_3$};
\draw[color=black] (3.0,4.5) node[anchor=north] {$R_1$};
\draw[color=black] (5.5,2.5) node[anchor=east] {$R_2$};
\draw[color=black] (9.1,2.5) node[anchor=west] {$R_3$};
\draw[color=black] (1.4,0.6) node[anchor=south west] {$U$};
\end{tikzpicture}
}%
%%\MUGraphicsSolo{schaltung.png}{scale=1.1}{width:450px}
\end{center}
Sie ist aus einer Spannungsquelle, die eine Spannung $U = \MZahl{5}{5}\MEinheit{V}$ liefern soll, sowie aus drei Widerständen
$R_1 = 1\MEinheit{}\MOhm$, $R_2 = 2\MEinheit{}\MOhm$ und $R_3 = 3\MEinheit{}\MOhm$ aufgebaut. Gefragt ist nach den in den einzelnen
Zweigen fließenden Strömen $I_1$, $I_2$ und $I_3$.\\
\textbf{Hinweise:} Die Zusammenhänge zwischen den interessierenden Größen, sprich den Spannungen, den 
Widerständen und den Stromstärken, werden für solche Schaltungen von den sogenannten \textbf{Kirchhoffschen Regeln}
geliefert, die im vorliegenden Beispiel drei Gleichungen bereitstellen:
\begin{eqnarray*}
I_1 - I_2 - I_3 & = & 0\quad:\quad\mbox{Gleichung}\MBlank (1) \MDFPSpace, \\ R_1 I_1 + R_2 I_2 & = & U\quad:\quad\mbox{Gleichung}\MBlank (2)\MDFPSpace,  \\ R_2 I_2 - R_3 I_3 & = & 0\quad:\quad\mbox{Gleichung}\MBlank (3) \MDFPeriod
\end{eqnarray*}
Außerdem wird die Beziehung zwischen den physikalischen Einheiten Volt ($\MEinheit[]{V}$) (für die Spannung),
Amp\`ere ($\MEinheit[]{A}$) (für die Stromstärke) und Ohm ($\MOhm$) (für den Widerstand) benötigt: $1\MEinheit{}\MOhm = (1\MEinheit{V}) / (1\MEinheit{A})$.

\begin{MHint}{Lösung}
Man löst z.B. die erste Gleichung nach $I_1$ auf,
$$I_1 = I_2 + I_3 \MDFPSpace ,$$
und setzt das Ergebnis in die zweite Gleichung ein:
$$R_1 (I_2 + I_3) + R_2 I_2 = U \Leftrightarrow (R_1 + R_2) I_2 + R_1 I_3 = U \; : \MBlank\mbox{Gleichung}\MBlank (2') \MDFPeriod$$
Die letzte Gleichung und die dritte Gleichung im ursprünglichen System hängen nur noch von den Unbekannten
$I_2$ und $I_3$ ab. Sie bilden ein System aus zwei linearen Gleichungen in zwei Unbekannten, das man jetzt
weiter löst: Dazu stellt man z.B. die dritte Gleichung des ursprünglichen Systems nach $I_3$ frei,
$$I_3 = \Mdfrac{R_2}{R_3} I_2 \; : \MBlank\mbox{Gleichung}\MBlank (3') \MDFPSpace ,$$
und setzt das Resultat in Gleichung~$(2')$ ein:
\begin{eqnarray*}
& & (R_1 + R_2) I_2 + R_1 \cdot \Mdfrac{R_2}{R_3} I_2 = U \\[.5ex]
& \Leftrightarrow & \left( R_1 + R_2 + R_1 \cdot \Mdfrac{R_2}{R_3} \right) I_2 = U \\[.5ex]
& \Leftrightarrow & \left( \Mdfrac{R_1 R_3}{R_3} + \Mdfrac{R_2 R_3}{R_3} + \Mdfrac{R_1 R_2}{R_3} \right) I_2 = U \\[.5ex]
& \Leftrightarrow & \Mdfrac{R_1 R_2 + R_1 R_3 + R_2 R_3}{R_3} I_2 = U \\[.5ex]
& \Leftrightarrow & I_2 = \Mdfrac{R_3}{R_1 R_2 + R_1 R_3 + R_2 R_3} U \MDFPeriod
\end{eqnarray*}
Damit folgt für $I_3$ wegen Gleichung~$(3')$
$$I_3 = \Mdfrac{R_2}{R_3} \cdot \Mdfrac{R_3}{R_1 R_2 + R_1 R_3 + R_2 R_3} U = \Mdfrac{R_2}{R_1 R_2 + R_1 R_3 + R_2 R_3} U$$
und für $I_1$ schließlich:
\begin{eqnarray*}
  I_1 = I_2 + I_3 &=& \Mdfrac{R_3}{R_1 R_2 + R_1 R_3 + R_2 R_3} \cdot U + \Mdfrac{R_2}{R_1 R_2 + R_1 R_3 + R_2 R_3} \cdot U \\
  &=& \Mdfrac{R_2 + R_3}{R_1 R_2 + R_1 R_3 + R_2 R_3} \cdot U \MDFPeriod
\end{eqnarray*}
Nun kann man die in der Aufgabenstellung vorgegebenen Werte für die Widerstände
($R_1 = 1\MEinheit{}\MOhm$, $R_2 = 2\MEinheit{}\MOhm$ und $R_3 = 3\MEinheit{}\MOhm$)
und die Spannung ($U = \MZahl{5}{5}\MEinheit{V}$) einsetzen; für $I_1$ erhält man unter Verwendung von $1\MEinheit{V} = 1\MEinheit{}\MOhm \cdot 1\MEinheit{A}$:
$$I_1 = \Mdfrac{(2 + 3)\MEinheit{}\MOhm}{(1 \cdot 2 + 1 \cdot 3 + 2 \cdot 3)\MEinheit{}\MOhm^2} \cdot \MZahl{5}{5} \cdot \MOhm \cdot \MEinheit[]{A}
= \MZahl{2}{5}\MEinheit{A} \MDFPeriod$$
Analog findet man für $I_2$ und $I_3$: $I_2 = \MZahl{1}{5}\MEinheit{A}$ und $I_3 = 1\MEinheit{A}.$\\
Alternative Lösungswege sind ebenso gut möglich.
\end{MHint}
\end{MExercise}

\begin{MExercise}
Lösen Sie das folgende Lineare Gleichungssystem mit Hilfe der Additionsmethode:
\begin{eqnarray*}
x + 2 z & = & 5 \MDFPSpace, \\ 3 x + y - 2 z & = & - 1  \MDFPSpace,\\ - x - 2 y + 4 z & = & 7 \MDFPeriod
\end{eqnarray*}

\begin{MHint}{Lösung}
Da die erste Gleichung ($x + 2 z = 5$) von vornherein nicht von der Unbekannten $y$ abhängt, bietet es sich
an, aus der zweiten und dritten Gleichung $y$ ebenfalls zu eliminieren. Dazu multipliziert man die zweite
Gleichung ($3 x + y - 2 z = - 1$) mit $2$ und addiert die resultierende Gleichung zur dritten Gleichung
($- x - 2 y + 4 z = 7$) hinzu:
\begin{eqnarray*}
& & (- x - 2 y + 4 z) + 2 \cdot (3 x + y - 2 z) = 7 + 2 \cdot (- 1) \\
& \Leftrightarrow & - x - 2 y + 4 z + 6 x + 2 y - 4 z = 7 - 2 \\
& \Leftrightarrow & 5 x = 5 \\
& \Leftrightarrow & x = 1 \MDFPeriod
\end{eqnarray*}
Zufälligerweise fällt gleichzeitig auch die Unbekannte $z$ heraus.
Mit dem Ergebnis für $x$ liefert die erste Gleichung
$$1 + 2 z = 5 \Leftrightarrow 2 z = 4 \Leftrightarrow z = 2 \MDFPeriod$$
Den Wert für $y$ erhält man sodann, indem man z.B. die zweite Gleichung verwendet:
$$3 \cdot 1 + y - 2 \cdot 2 = - 1 \Leftrightarrow 3 + y - 4 = - 1 \Leftrightarrow y = 0 \MDFPeriod$$
Also lautet die Lösungsmenge ${\ML} = \{ \MPointThree{x = 1}{y = 0}{z = 2} \}$.
\end{MHint}
\end{MExercise}
\end{MExercises}



\MSubsection{Allgemeinere Systeme}
\MLabel{M04_freier_Parameter}

\begin{MIntro}
\MDeclareSiteUXID{VBKM04_AllgemeineLGS_Intro}
Zu Systemen linearer Gleichungen könnte man noch eine ganze Menge mehr sagen. Um aber die Fülle an
Informationen nicht allzu übermächtig werden zu lassen, beschränkt man sich zum Abschluss dieses Moduls
auf zwei weitere, kleinere Punkte.

Zum einen können Gleichungssysteme sogenannte freie Parameter enthalten, das sind variable
Größen - oder mit anderen Worten eine Art Stellschrauben -, die das Verhalten der Systeme und insbesondere
der Lösungsmengen unter Umständen stark beeinflussen. Manchmal ist es von Vorteil, nicht alle Koeffizienten und nicht
alle rechten Seiten der Gleichungen als feste, konkrete Zahlenwerte vorzugeben, sondern sie variabel
zu lassen, um zu studieren, was in verschiedenen Situationen passieren kann; manchmal kennt man auch nicht
alle Koeffizienten oder rechten Seiten aufgrund der Aufgaben- oder Problemstellung.

Zum anderen ist es möglich, dass die Anzahl der linearen Gleichungen und
 die Anzahl der Unbekannten in einem Gleichungssystem unterschiedlich groß ist.%%%
\end{MIntro}

\begin{MXContent}{Systeme mit freiem Parameter}{Systeme mit freiem Parameter}{STD}
\MDeclareSiteUXID{VBKM04_FreieParameter}
Am Anfang steht ein Beispiel, das zugegebenermaßen sehr einfach ist, aber dennoch auf einen,
wenn nicht \textbf{den}, entscheidenden Punkt im Zusammenhang mit freien Parametern in Systemen
linearer Gleichungen hinführen wird:
\begin{MExample}
Es soll nach der Lösungsmenge für das Lineare Gleichungssystem
$$\begin{array}{rcrcl} \mbox{Gleichung}\MBlank (1) : & & x - 2 y & = & 3 \MDFPSpace,\\
\MBlank\mbox{Gleichung}\MBlank (2) : & & - 2 x + 4 y & = & \alpha \end{array}$$
in Abhängigkeit von dem Parameter $\alpha$ gesucht werden.

Dazu multipliziert man Gleichung~$(1)$ mit dem Faktor $2$ durch und addiert Gleichung~$(2)$:
$$2 \cdot (x - 2 y) + (- 2 x + 4 y) = 2 \cdot 3 + \alpha \Leftrightarrow 2 x - 4 y - 2 x + 4 y = 6 + \alpha
\Leftrightarrow 0 = 6 + \alpha \MDFPeriod$$

Nun müssen zwei Fälle unterschieden werden:

\textbf{Fall A}: $\alpha \neq - 6$: Ist der vorgegebene freie Parameter $\alpha$ ungleich $- 6$, so stößt man
auf einen Widerspruch. In diesem Fall besitzt das Lineare Gleichungssystem \textbf{keine Lösung}, ${\ML}
= \MEmptyset$.

\textbf{Fall B}: $\alpha = - 6$: Ist der vorgegebene freie Parameter $\alpha$ gleich $-6$, so ist die Gleichung
identisch erfüllt ($0 = 0$). In der Tat sind in diesem Fall die beiden Ausgangsgleichungen gerade Vielfache
voneinander, sodass nur eine von ihnen tatsächlich Information trägt. Dementsprechend ist jetzt die
\textbf{Lösungsmenge unendlich mächtig} und kann beispielsweise folgendermaßen angegeben werden: ${\ML}
= \{ \MPointTwo{x = 3 + 2 t}{y = t}  \MCondSetSep t \in \R \}$.
\end{MExample}
Das Beispiel zeigt, dass das Aussehen der Lösungsmenge sehr stark von der Wahl des freien Parameters abhängen kann.

Ein solcher freier Parameter kann aber nicht nur auf einer der rechten Seiten des Linearen Gleichungssystems
auftreten, sondern auch auf den linken Seiten, ja er kann auch mehrfach, in funktionalen Abwandlungen, sowohl
links als auch rechts usw. vorkommen. Auch mehrere freie Parameter sind denkbar. Ein etwas komplizierteres Beispiel ist gegeben durch:
\begin{MExample}
Im Folgenden soll die Lösungsmenge des Linearen Gleichungssystems
\begin{eqnarray*}
x + y + \alpha z & = & 1 \MDFPSpace, \\ x + \alpha y + z & = & 1 \MDFPSpace, \\ \alpha x + y + z & = & 1
\end{eqnarray*}
studiert werden und zwar in Abhängigkeit von dem Wert des Parameters $\alpha$.

Dazu löst man beispielsweise die erste Gleichung nach der Unbekannten $x$ auf,
$$x = 1 - y - \alpha z \; : \MBlank\mbox{Gleichung}\MBlank (1') \MDFPSpace ,$$
und \textbf{setzt} das Ergebnis für $x$ in die zweite und in die dritte Gleichung \textbf{ein}:
$$\begin{array}{rclcrclcl} (1 - y - \alpha z) + \alpha y + z & = & 1 & \Leftrightarrow &
-(1 - \alpha) y + (1 - \alpha) z & = & 0 & & : \MBlank\mbox{Gleichung}\MBlank (2') \MDFPSpace, \\
\alpha (1 - y - \alpha z) + y + z & = & 1 & \Leftrightarrow &
(1 - \alpha) y + (1 - \alpha^2) z & = & 1 - \alpha & & : \MBlank\mbox{Gleichung}\MBlank (3') \MDFPeriod \end{array}$$
Es entsteht ein System aus \textbf{zwei} linearen Gleichungen in den \textbf{zwei} Unbekannten $y$ und $z$.
Diesem System sieht man sofort an, dass für den Wert $\alpha = 1$ etwas Besonderes passiert.
Man führt daher eine Fallunterscheidung durch:

\textbf{Fall 1:} $\alpha = 1$: In diesem Fall sind die beiden Gleichungen~$(2')$ und $(3')$ identisch
erfüllt ($0 = 0$) und liefern keine weiteren Informationen; als einzige Beziehung zwischen
den Unbekannten $x, y$ und $z$ verbleibt Gleichung~$(1')$ bzw. Gleichung~$(1)$, die für $\alpha = 1$
$$x + y + z = 1 \; : \MBlank\mbox{Gleichung}\MBlank (\hat{1})$$
lautet. Daher besitzt die Lösungsmenge hier unendlich viele Elemente. Sie kann mit Hilfe \textbf{zweier freier
Parameter} angegeben werden, z.B.
$${\ML} = \{ \MPointThree{s}{t}{1 - s- t} \MCondSetSep  s, t \in \R \} \MDFPeriod $$
Anschaulich gesprochen entspricht die Lösungsmenge gerade der durch Gleichung~$(\hat{1})$ beschriebenen Ebene.

\textbf{Fall 2:} $\alpha \neq 1$: In diesem Fall kann man sowohl Gleichung~$(2')$ als auch Gleichung~$(3')$
durch $(1 - \alpha)$ dividieren; man erhält unter Verwendung der 3.~Binomischen Formel ($(1 - \alpha^2)
= (1 - \alpha)(1 + \alpha)$):
$$\begin{array}{rclcl} - y + z & = & 0 & & : \MBlank\mbox{Gleichung}\MBlank ({2'}') \MDFPSpace, \\
y + (1 + \alpha) z & = & 1 & & : \MBlank\mbox{Gleichung}\MBlank ({3'}') \MDFPeriod \end{array}$$
Gleichung~$({2'}')$ besagt, dass $y = z$ gilt; dies setzt man in Gleichung~$({3'}')$ ein:
$$z + (1 + \alpha) z = 1 \Leftrightarrow (2 + \alpha) z = 1 \; : \MBlank\mbox{Gleichung}\MBlank (\star)\MDFPeriod $$
Jetzt muss man nochmals vorsichtig sein und eine weitere Fallunterscheidung durchführen, da $\alpha = - 2$
und $\alpha \neq - 2$ unterschiedliche Konsequenzen nach sich ziehen:

\textbf{Fall 2a:} $\alpha = - 2$: In diesem (Unter-)Fall lautet Gleichung $(\star)$: $0 = 1$. Somit tritt ein
Widerspruch auf und das Lineare Gleichungssystem, von dem ausgegangen wurde, hat keine Lösung, ${\ML}
= \MEmptyset$.

\textbf{Fall 2b:} $\alpha \neq - 2$: In diesem (Unter-)Fall kann man die letzte Gleichung
problemlos nach $z$ auflösen,
$$z = \Mdfrac{1}{2 + \alpha} \MDFPeriod $$

Damit liegen auch $y$ ($y = z$) und $x$ ($x = 1 - y - \alpha z$) fest; das ursprüngliche Lineare Gleichungssystem
besitzt eine eindeutige Lösung, ${\ML} = \{ \MPointThree{x = \Mtfrac{1}{2 + \alpha}}{ y = \Mtfrac{1}{2 + \alpha}}{
z = \Mtfrac{1}{2 + \alpha} } \}$.
\end{MExample}
Im Vorhergehenden wurde das Beispiel auch deshalb so ausführlich ausbuchstabiert, um eindringlich auf die
Wichtigkeit einer sauberen und genauen Fallunterscheidung hinzuweisen. Je nachdem, ob $\alpha = 1$, $\alpha = - 2$
oder $\alpha \in \R \setminus \{- 2\MElSetSep 1 \}$ gilt, sieht die Lösungsmenge völlig verschieden aus! Im ersten Fall
ist sie unendlich mächtig, im zweiten leer und im dritten besteht sie aus genau einem Element!

Übrigens hätte man die Besonderheit des Falles $\alpha = 1$ auch direkt an dem ursprünglichen System ablesen
können: Für $\alpha = 1$ tritt dreimal dieselbe Gleichung, nämlich $x + y + z = 1$, auf; d.h. zwei der drei Gleichungen
im Ausgangssystem enthalten keinerlei neue Information und sind daher überflüssig; nur die Ebenengleichung
$x + y + z = 1$ stellt für $\alpha = 1$ eine Beziehung zwischen den drei Unbekannten her.
\end{MXContent}


\begin{MExercises}
\MDeclareSiteUXID{VBKM04_FreieParameter_Exercises}
\begin{MExercise}
Es sollen der Achsenabschnitt $b$ und die Steigung $m$ einer Geraden
mit der Darstellung $y=m x+b$ bestimmt werden, welche durch zwei Punkte
festgelegt wird. Der erste Punkt an der Stelle $x_1=\alpha$ liegt auf
der Geraden, welche durch die Gleichung $y_1(x)=-2(1+x)$ beschrieben wird; 
der zweite Punkt an der Stelle $x_2=\beta$ liegt auf der Geraden,
die durch $y_2(x)=x-1$ beschrieben wird. Die Situation wird durch die
nachfolgende Graphik verdeutlicht.
\begin{center}
\MTikzAuto{%
\begin{tikzpicture}[x=1.0cm, y=1.0cm] 
%Koordinatensystem
\node (xMAX) at (5.0,0){};
\node (yMAX) at (0,5.0){};
\draw[->,color=black] (-4.5,0) -- (xMAX);
\foreach \x in {-4, -3, -2, -1, 0, 1, 2, 3, 4}
\draw[shift={(\x,0)},color=black] (0pt,2pt) -- (0pt,-2pt) node[above right] {\scriptsize $\x$};
\draw[->,color=black] (0,-2.5) -- (yMAX);
\foreach \y in {-2, -1, 1, 2, 3, 4}
\draw[shift={(0,\y)},color=black] (2pt,0pt) -- (-2pt,0pt) node[above right] {\scriptsize $\y$};
%Achsenbeschriftung
\draw (xMAX) node[anchor=south east] {$x$};
\draw (yMAX) node[anchor=north east] {$y$};
%Beschriftung und Graphen
\clip(-4.5,-2.5) rectangle (4.5,4.5);
\draw[help lines, gray, dashed] (-5,-5) grid (5,5); % was dotted
%\fill[color=black] (1,1) circle (2.0pt);
%%\fill[red!50!white, opacity=0.50] (1,1) rectangle (6,6);
\def\alx{-1.5}
\def\bex{2.5}
%%\pgfmathparse{(\bex*\bex-\alx)/(\bex-\alx)}\let\mrsl=\pgfmathresult
\pgfmathparse{(-2-2*\alx-\bex+1)/(\alx-\bex)}\let\mrsl=\pgfmathresult
\pgfmathparse{-1+\bex*(1-\mrsl)}\let\brsl=\pgfmathresult
%%\draw[color=red, line width=1pt] (-4.0,-4.0) -- (4.0,4.0);
\draw[smooth,samples=3,domain=-4:4, line width=1pt,color=blue] plot(\x,{-1+\x});
%%\draw[smooth,samples=17,domain=-4:4, line width=1pt,color=blue] plot(\x,{\x*\x});
\draw[smooth,samples=3,domain=-4:4, line width=1pt,color=red] plot(\x,{-2*(1+\x)});
\draw[smooth,samples=3,domain=-4.5:4.5, line width=1pt,color=black] plot(\x,{\mrsl*\x+\brsl});
\draw[color=red, very thick] (\alx,2pt) -- (\alx,-2pt) node[anchor=north] {$\alpha$};
\draw[color=red, very thick] (\alx,{-2*(1+\alx)}) circle (1.5pt);
\draw[color=blue, very thick] (\bex,2pt) -- (\bex,-2pt) node[anchor=north] {$\beta$};
\draw[color=blue, very thick] (\bex,{-1+\bex}) circle (1.5pt);
\draw[color=red] (-4.0,-0.8) node[anchor=west] {$y_1(x) = -2(1+x)$};
\draw[color=blue] (-4.0,-1.35) node[anchor=west] {$y_2(x) = -1+x$};
\draw[color=black] (-4.0,-1.8) node[anchor=west] {$y = m x+b$};
\end{tikzpicture}
}
\end{center}
\begin{MExerciseItems}
\item{Bestimmen Sie das Gleichungssystem für die Geradenparameter $b$ und $m$.
\par
Die erste Gleichung lautet \MEquationItem{$m\alpha+b$}{\MLSimplifyQuestion{15}{-2*(1+alpha)}{10}{alpha}{10}{2}{LGSPar1Gl1}};\\ % 513
die zweite Gleichung lautet \MEquationItem{$m\beta+b$}{\MLSimplifyQuestion{15}{-1+beta}{10}{beta}{10}{2}{LGSPar1Gl2}}. \\ % 513
\MInputHint{Die Konstanten $\alpha$ und $\beta$ müssen in der Lösung stehen bleiben; 
für diese kann man \texttt{alpha} und \texttt{beta} eingeben.}
}
\item{Lösen Sie dieses Gleichungssystem für $b$ und $m$. Für welche $\alpha$
und $\beta$ ergibt sich eine eindeutige, keine bzw. unendlich
viele Lösungen?
\par
Z.B. erhält man für den Fall $\alpha=-2$ und $\beta=2$ die
Lösung \MEquationItem{$m$}{\MLParsedQuestion{6}{-1/4}{4}{LGSPar1B1m}}
und \MEquationItem{$b$}{\MLParsedQuestion{6}{5/2}{4}{LGSPar1B1b}},
für den Fall $\alpha=2$ und $\beta=-2$ ergibt sich die
Lösung \MEquationItem{$m$}{\MLParsedQuestion{6}{-3/4}{4}{LGSPar1B2m}}
und \MEquationItem{$b$}{\MLParsedQuestion{6}{-9/2}{4}{LGSPar1B2b}}.
\par
Das LGS besitzt unendlich viele Lösungen, falls
\MEquationItem{$\alpha$}{\MLParsedQuestion{6}{-1/3}{4}{LGSPar1InfAlp}}
und \MEquationItem{$\beta$}{\MLParsedQuestion{6}{-1/3}{4}{LGSPar1InfBet}} ist.\\
Die zugehörigen Lösungen lassen sich parametrisieren mit $m=r$ und 
\MEquationItem{$b$}{\MLSimplifyQuestion{10}{r/3-4/3}{10}{r}{10}{2}{LGSPar1Infb}}, $r\in\R$. % 513
}
\item{Was bedeuten die letzten beiden Fälle, d.h. keine bzw. unendlich
viele Lösungen, anschaulich?}
\end{MExerciseItems}
%%
\begin{MHint}{Lösung}
\begin{MExerciseItems}
\item{%
Das LGS für $m$ und $b$ ergibt sich aus den Bedingungen 
\[
  y(x_1=\alpha)=y_1(x_1=\alpha) \MBlank\mbox{und}\MBlank
  y(x_2=\beta)=y_2(x_2=\beta)
\]
zu
\[
  \begin{array}{rcll}
     m\alpha +b &=& -2(1+\alpha) & \mbox{Gl.}\MBlank (1)\MDFPSpace, \\[0.5ex]
     m\beta +b &=& -1+\beta & \mbox{Gl.}\MBlank (2) \MDFPeriod
  \end{array}
\]
}
\item{%
Ersetzt man Gl. (2) durch die Differenz von Gl. $(2)$ und Gl. $(1)$,
dann erhält man
\[
  \begin{array}{rcll}
     m\alpha +b &=& -2(1+\alpha) & \mbox{Gl.}\MBlank (1)\MDFPSpace, \\[0.5ex]
     m(\beta-\alpha) &=& 1+2\alpha+\beta & \mbox{Gl.}\MBlank (2') \MDFPeriod
  \end{array}
\]
Das LGS liegt damit bereits in Dreiecksform vor.
%
\medskip\par\noindent
%
A. Für den Fall $\beta-\alpha \ne 0 \MBlank \Leftrightarrow \MBlank
\beta\ne\alpha$ kann Gl. $(2')$ durch $(\beta-\alpha)$ dividiert werden,
womit man die Steigung $m$ zu
\[
  m=\frac{1+2\alpha+\beta}{\beta-\alpha}
\]
erhält. Dieses Ergebnis kann man z.B. in die nach $b$ aufgelöste Gl. (1)
einsetzen:
\[
  b=-2(1+\alpha)-m\alpha=
  -2(1+\alpha)-\frac{1+2\alpha+\beta}{\beta-\alpha}\alpha
%%  =-\frac{\alpha(1+2\alpha+\beta)+2(1+\alpha)(\beta-\alpha)}{\beta-\alpha}
  \MDFPeriod
\]
Die so gefundenen Werte für $m$ und $b$ stellen eine eindeutige Lösung
des betrachteten LGS dar.
\par
Für das Beispiel $\alpha=-2, \beta=2$ erhält man die Lösung $m=-1/4, b=3/2$%%% 
und für $\alpha=2, \beta=-2$ das Ergebnis $m=-3/4, b=-9/2$.
%
\medskip\par\noindent
%
B. Gilt nun $\beta-\alpha = 0 \MBlank \Leftrightarrow \MBlank %%%
\beta=\alpha$. Damit verschwindet die linke Seite von Gl. $(2')$. Dies macht
die folgende weitere Fallunterscheidung nötig:
%
\smallskip\par\noindent
%
B(i). Gilt nun für die rechte Seite von Gl. $(2')$ $1+2\alpha+\beta\ne 0$,%%%
dann besitzt das LGS keine Lösung, $\ML=\MEmptyset$. Setzt man noch
$\beta=\alpha$ ein, dann erhält man $\beta=\alpha\ne -\frac{1}{3}$.
%
\smallskip\par\noindent
B(ii). Verschwindet die rechte Seite von Gl. $(2')$, dann besteht das LGS%%%
faktisch nur noch aus Gl. (1) mit $\beta=\alpha=-\frac{1}{3}$.
In diesem Fall kann man $m=\lambda$, 
$\lambda\in\R$, beliebig vorgeben, und $b$ ergibt sich direkt aus Gl. (1)
zu $b =\frac{1}{3}\lambda-\frac{4}{3}$. Die Lösungsmenge stellt sich damit
wie folgt dar:
\[
  \ML = \left\{\MPointTwoAS{m=\lambda}{b=\frac{1}{3}\lambda-\frac{4}{3}} 
  \MCondSetSep \lambda\in\R\right\} \MDFPeriod
\]
}
\item{%
Die rechten Seiten der Gln. $(1)$ und $(2)$ enthalten die $y$-Werte
der Ausgangsgeraden an den Stellen $\alpha$ und $\beta$, d.h.
$y_1(\alpha)$ und $y_2(\beta)$, sodass in Gl. $(2')$ auf der rechten
Seite die Differenz $y_2(\beta)-y_1(\alpha)$ steht. Im Fall 2.,
$\beta=\alpha$, erhält man dann $y_2(\alpha)-y_1(\alpha)$. Das LGS
hat also keine Lösung (Fall 2a), wenn dann $y_2(\alpha)-y_1(\alpha)\ne 0$
$\Leftrightarrow$ $y_2(\alpha)\ne y_1(\alpha)$ gilt,
die beiden Geraden also verschiedene $y$-Werte besitzen. Dahingegen
existieren unendlich viele Lösungen (Fall 2b), wenn sich die beiden
Ausgangsgeraden an der Stelle $\alpha$ schneiden.
}
\end{MExerciseItems}
\end{MHint}
\end{MExercise}

\begin{MExercise}
Bestimmen Sie für das folgende parameterabhängige LGS die Lösungsmenge 
für alle $t\in\R$:
\[
  \begin{array}{rcll}
      x -y +t z&=& t & 
	  \mbox{Gl.}\MBlank (1)\MDFPSpace, \\[0.5ex]
      t x + (1-t)y +(1+t^2)z&=& -1+t & 
	  \mbox{Gl.}\MBlank (2)\MDFPSpace, \\[0.5ex]
      (1-t)x+(-2+t)y+(-1+t-t^2)z&=& t^2 & 
	  \mbox{Gl.}\MBlank (3) \MDFPeriod
  \end{array}
\]
\par
Das LGS besitzt Lösungen nur für folgende Werte des Parameters:
$t\in\mbox{}$\MLParsedQuestion{12}{-1,1}{4}{LGSPar2PSolSet}. \\
\MInputHint{Mengen können in der Form \texttt{$\lbrace$a;b;c;$\ldots\rbrace$} eingegeben werden. Die leere Menge kann als $\lbrace\rbrace$ eingegeben werden.}
\par
Für den kleinsten dieser Parameterwerte kann die Lösung angegeben werden durch:\\
\MEquationItem{$x$}{\MLSimplifyQuestion{10}{-4}{10}{r}{10}{513}{LGSPar2LVx}},
\MEquationItem{$y$}{\MLSimplifyQuestion{10}{-3-r}{10}{r}{10}{513}{LGSPar1LVy}},
$z=r$, $r\in\R$.\\
Für den größten dieser Parameterwerte kann die Lösung angegeben werden durch:\\
\MEquationItem{$x$}{\MLSimplifyQuestion{10}{-2*r}{10}{r}{10}{513}{LGSPar2HVx}},
\MEquationItem{$y$}{\MLSimplifyQuestion{10}{-1-r}{10}{r}{10}{513}{LGSPar1HVy}},
$z=r$, $r\in\R$.
\end{MExercise}

\begin{MHint}{Lösung}
In einem ersten Schritt kann man Gl. $(2)$ zu Gl. $(3)$ addieren und 
damit Gl. $(3)$ ersetzen:
\[
  \begin{array}{rcll}
      x -y +t z&=& t & 
	  \mbox{Gl.}\MBlank (1)\MDFPSpace, \\[0.5ex]
      t x + (1-t)y +(1+t^2)z&=& -1+t & 
	  \mbox{Gl.}\MBlank (2)\MDFPSpace, \\[0.5ex]
      x-y+t z&=& -1+t+t^2 & 
	  \mbox{Gl.}\MBlank (3') \MDFPeriod
  \end{array}
\]
Da die linken Seiten der Gln. $(1)$ und $(3')$ übereinstimmen, bietet es
sich an, das Negative von Gl. $(1)$ zu Gl. $(3')$ zu addieren und damit
Gl. $(3')$ zu ersetzen:
\[
  \begin{array}{rcll}
      x -y +t z&=& t & 
	  \mbox{Gl.}\MBlank (1)\MDFPSpace, \\[0.5ex]
      t x + (1-t)y +(1+t^2)z&=& -1+t & 
	  \mbox{Gl.}\MBlank (2)\MDFPSpace, \\[0.5ex]
      0 &=& -1+t^2 & 
	  \mbox{Gl.}\MBlank ({3'}') \MDFPeriod
  \end{array}
\]
Nachfolgend erhält man für das LGS eine Dreiecksform, wenn man das
$(-t)$-fache von Gl. $(1)$ zu Gl. $(2)$ hinzuaddiert und damit Gl. $(2)$
ersetzt:
\[
  \begin{array}{rcll}
      x -y +t z&=& t & 
	  \mbox{Gl.}\MBlank (1)\MDFPSpace, \\[0.5ex]
      y + z&=& -1+t-t^2 & 
	  \mbox{Gl.}\MBlank (2')\MDFPSpace, \\[0.5ex]
      0 &=& -1+t^2 & 
	  \mbox{Gl.}\MBlank ({3'}') \MDFPeriod
  \end{array}
\]
Schließlich führt die Addition von Gl. $(2')$ zu Gl. $(1)$ zur einer
weiteren Vereinfachung,
\[
  \begin{array}{rcll}
      x +(1+t)z&=& -1+2t-t^2 & 
	  \mbox{Gl.}\MBlank (1')\MDFPSpace, \\[0.5ex]
      y + z&=& -1+t-t^2 & 
	  \mbox{Gl.}\MBlank (2')\MDFPSpace, \\[0.5ex]
      0 &=& -1+t^2 & 
	  \mbox{Gl.}\MBlank ({3''}) \MDFPeriod
  \end{array}
\]
Dieses äquivalente LGS besitzt nur dann eine Lösung, wenn auch Gl. $({3''})$ 
erfüllt ist, also $0=-1+t^2$ $\Leftrightarrow$ $t^2=1$ gilt.
In allen anderen Fällen ist die Lösungsmenge leer, d.h.
\[
  \ML=\MEmptyset \MBlank\mbox{für}\MBlank t\in\R\MSetminus\{-1\MElSetSep 1\}
  \MDFPeriod
\]
Ist nun Gl. $({3''})$ erfüllt, also $t=1$ oder $t=-1$, dann kann $z$
frei gewählt werden, d.h. $z=\lambda$, $\lambda\in\R$.
Die Umstellung der Gln. $(1')$ und $(2')$ nach $x$ bzw. $y$ liefert
im Fall $t=1$ die Ausdrücke $x=-2z$ und $y=-1-z$, also
\[
  \ML = \{\MPointThree{x=-2\lambda}{y=-1-\lambda}{z=\lambda} 
  \MCondSetSep \lambda\in\R\} \MBlank\mbox{für}\MBlank t=1
  \MDFPeriod
\]
Entsprechend ergeben sich im Fall $t=-1$ die Ausdrücke 
$x=-4$ und $y=-3-z$, also
\[
  \ML = \{\MPointThree{x=-4}{y=-3-\lambda}{z=\lambda} 
  \MCondSetSep \lambda\in\R\} \MBlank\mbox{für}\MBlank t=-1
  \MDFPeriod
\]
\end{MHint}
\end{MExercises}


\MSubsection{Abschlusstest}
\MLabel{M04_Ausgangstest}

\begin{MTest}{Abschlusstest Modul \arabic{section}}
\MDeclareSiteUXID{VBKM04_Abschlusstest}

\begin{MExercise}
Bestimmen Sie die Lösungsmenge für folgendes Lineares Gleichungssystem:
\begin{eqnarray*}
- x + 2 y & = & - 5 \MDFPSpace,  \\ 3 x + y & = & 1 \MDFPeriod
\end{eqnarray*}
Die Lösungsmenge
\begin{tabular}[t]{ll}
\MLCheckbox{0}{M04C13} & ist leer,\\
\MLCheckbox{1}{M04C14} & enthält genau eine Lösung: $x =$ \MLParsedQuestion{5}{1}{5}{ATXY1} , $y =$ \MLParsedQuestion{5}{-2}{5}{ATXY2} ,\\
\MLCheckbox{0}{M04C15} & enthält unendlich viele Lösungspaare $\MPointTwo{x}{y}$.
\end{tabular}
\end{MExercise}

\begin{MExercise}
Geben Sie diejenige zweistellige Zahl an, die bei Vertauschen von Einer- und Zehnerziffer auf eine um 18 kleinere Zahl
führt und deren Quersumme 6 ist.
% 
Antwort: \MLParsedQuestion{5}{42}{5}{ATX42}.
\end{MExercise}

\begin{MExercise}
Für welchen Wert des reellen Parameters $\alpha$ besitzt das Lineare Gleichungssystem
\begin{eqnarray*}
2 x + y & = & 3 \MDFPSpace, \\ 4 x + 2 y & = & \alpha
\end{eqnarray*}
unendlich viele Lösungen?

Antwort: $\alpha = $ \MLParsedQuestion{5}{6}{5}{ATX}.
\end{MExercise}

\begin{MExercise}
Die Abbildung zeigt zwei Geraden im zweidimensionalen Raum.
\begin{center}
\MTikzAuto{%
\begin{tikzpicture}[x=1.0cm, y=1.0cm] 
%Koordinatensystem
\node (xMAX) at (5.0,0){};
\node (yMAX) at (0,4.7){};
\clip(-4.1,-2.8) rectangle (5.1,4.7);
\draw[help lines, gray, dashed, xstep=1.0, ystep=1.0] (-4,-3) grid (5,5);
\draw[->,color=black] (-4.0,0) -- (xMAX);
\foreach \x in {-3, -2, -1, 0, 1, 2, 3, 4}
\draw[shift={(\x,0)},color=black] (0pt,2pt) -- (0pt,-2pt) node[below right] {\scriptsize $\x$};
\draw[->,color=black] (0,-2.8) -- (yMAX);
\foreach \y in {-2, -1, 0, 1, 2, 3, 4}
\draw[shift={(0,\y)},color=black] (2pt,0pt) -- (-2pt,0pt) node[above left] {\scriptsize $\y$};
%Achsenbeschriftung
\draw (xMAX) node[anchor=south east] {$x$};
\draw (yMAX) node[anchor=north west] {$y$};
%Beschriftung und Graphen
%%\fill[red!50!white, opacity=0.50] (1,1) rectangle (6,6);
\draw[color=blue, thick] (-2.0,-3.0) -- (5.0,4.0);
\draw[color=blue, thick] (3.0,-4.0) -- (-2.0,6.0);
\fill[color=red] (1,0) circle (2.0pt);
\draw[color=black] (1,0) circle (2.0pt);
\draw[color=red] (0.60,0) node[anchor=south] {\scriptsize $P$};
\draw[color=blue] (3.0,2.0) node[anchor=north west] {\scriptsize Gerade 1};
\draw[color=blue] (-0.5,3.0) node[anchor=north east] {\scriptsize Gerade 2};
\end{tikzpicture}
}%
%%\MUGraphicsSolo{LGS_schneidende_Geraden.png}{width=0.5\linewidth}{width:450px}
\end{center}
Stellen Sie die beiden Geradengleichungen auf:

Gerade 1: $y = $ \MLFunctionQuestion{15}{-1+x}{2}{x}{5}{XAFG1},

Gerade 2: $y = $ \MLFunctionQuestion{15}{2-2*x}{2}{x}{5}{XAFG2}.

Wieviele Lösungen besitzt das zugehörige Lineare Gleichungssystem?

Es besitzt
\begin{tabular}[t]{ll}
\MLCheckbox{0}{M04C16} & keine Lösung,\\
\MLCheckbox{1}{M04C17} & genau eine Lösung oder\\
\MLCheckbox{0}{M04C18} & unendlich viele Lösungen.
\end{tabular}
\end{MExercise}

\begin{MExercise}
Geben Sie die Lösungsmenge für folgendes Lineare Gleichungssystem, bestehend aus drei Gleichungen mit drei Unbekannten, an:
\begin{eqnarray*}
x + 2 z & = & 3 \MDFPSpace, \\ - x + y + z & = & 1 \MDFPSpace, \\ 2 y + 3 z & = & 5 \MDFPeriod
\end{eqnarray*}
Die Lösungsmenge
\begin{tabular}[t]{ll}
\MLCheckbox{0}{M04C19} & ist leer,\\
\MLCheckbox{1}{M04C20} & enthält genau eine Lösung: $x =$ \MLParsedQuestion{5}{1}{5}{ATXY4} , $y =$ \MLParsedQuestion{5}{1}{5}{ATXY5}
, $z = $ \MLParsedQuestion{5}{1}{5}{ATXY6} ,\\
\MLCheckbox{0}{M04C21} & enthält unendlich viele Lösungen $\MPointThree{x}{y}{z}$.
\end{tabular}
\end{MExercise}

\end{MTest}


\newpage
\MPrintIndex

\end{document}
