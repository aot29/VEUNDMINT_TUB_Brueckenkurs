\ifttm
\MSetSubject{\MINTPhysics}
\MSubject{HELPSECTION}
\MSection{Welcome}

% Beim Übersetzen aufgefallen:
% - Der Text ist so nur für die in Karlsruhe gehostete Version korrekt. Für unsere müssen wir noch einiges anpassen, insbesondere die Behauptung, wo das gehostet ist und die Kontakt-Email.
% - Es gibt zwei Abschnitte "Suitable browsers" mit unterschiedlichem Focus und nicht ganz konsistenten Informationen.
% - Der juristische Kram scheint mir dilettantisch und potenziell problematisch. Wenn man z.B. im Absatz zum Haftungsausschluss was über mögliche Farbverfälschungen erzählt (wen interessiert das?), muss man damit rechnen, dass ein Richter den Haftungsausschluss auch nur für so einen Kleinkram als wirksam ansieht, aber nicht für Todesfälle, weil jemand eine falsche Formel in die Flugzeugsteuerungssoftware kopiert hat.
% - Wenn der juristische Kram nicht so dilettantisch wäre, müsste man hier bei der Übersetzung sehr aufpassen. Das ist was für Juristen. Evt. kann und sollte man schreiben, dass die Übersetzung nur zur schnellen Information dient und der deutsche Text maßgeblich ist.
% Hans Adler


\begin{MSectionStart}
\MDeclareSiteUXID{VBKM_FIRSTPAGE}
\MGlobalStart
\begin{html}

<div class="jumbotron">
<h1 class="start" >Preparatory Online Course in Mathematics</h1>
<p>
Welcome to the Preparatory Online Course in Mathematics for Refugees (VE&MINT project/TU Berlin).
</p>
<p>
This is a preliminary version of the first chapters, where a few things are still broken.
</p>
<p>
The purpose of this course is to help you brush up on school-level mathematics to prepare you for studying a technical subject at university level.
Course data is saved automatically in your browser. Explicit registration for the course is not required.
</p>
<p>
<center><button class="stdbutton btn btn-primary btn-lg" type="button" onclick="opensite(':directmaterial:sectionx1.1.0.html');">Start course</button></center>
<br />
</p>
</div>
\end{html}

%\textbf{Start with an assessment test}\\
%In \MSRef{L_TEST01START}{this mode}, you can do an online test on the course materials. Completion takes about an hour.
%Based on the test result you get a recommendation on how to use the online course.
%You can also use it to gather objective information that can serve as a basis for telephonic advice
%or for a recommended preparatory course in a physical setting.

\textbf{Start chapters in order}\\
In \MSRef{VBKM01}{this mode}, you can complete the units in order. Every unit features a short introductory test
as well as a final test with which you can verify the successful completion. 

\textbf{Search for a keyword}\\
The \MSRef{L_SEARCHSITE}{keyword search page} provides links to course unit contents relevant to each keyword.
You can use it to selectively close isolated gaps in your knowledge.

\textbf{Check settings}\\
On the \MSRef{L_CONFIG}{Settings} page, you can define which course data is saved and how it is saved.
The standard setting is that course data (in particular the solutions entered and scores) are saved in your
browser and solution data is only sent to our servers in anonymised form for statistical purposes.

\textbf{Direct entry to a specific unit}
You can navigate to individual learning units and their sections at any time to work on them:
\begin{itemize}
\item{Chapter 01: \MSRef{VBKM01}{Elementary arithmetic}}
\item{Chapter 02: \MSRef{VBKM02}{Equations in a single variable}}
\item{Chapter 03: \MSRef{VBKM03}{Inequalities in a single variable}}
\item{Chapter 04: \MSRef{VBKM04}{Systems of linear equations}}
%\item{Modul 05: \MSRef{VBKM05}{Elementare Geometrie}}
%\item{Modul 06: \MSRef{VBKM06}{Elementare Funktionen}}
%\item{Modul 07: \MSRef{VBKM07}{Differentialrechnung}}
%\item{Modul 08: \MSRef{VBKM08}{Integralrechnung}}
%\item{Modul 09: \MSRef{VBKM09}{Orientierung im zweidimensionalen Koordinatensystem}}
%\item{Modul 10: \MSRef{VBKM10}{Grundlagen der anschaulichen Vektorgeometrie}}
\end{itemize}
You are also free to work on tests and exercises in arbitrary order and repeatedly.
Just click the relevant entries in the table of contents in the left border area.

\special{html:<center><a href="http://creativecommons.org/licenses/by-sa/3.0/de/deed.en" target="_new"><img src="../images/cclbysa.png" alt="Bild CCL-BY-SA"></a></center>}\\
The courses contents are provided under Creative Commons License CC BY-SA 3.0 and can be
copied or used in modified form so long as the origin (the present course) is cited.
For detailed information on licence and reuse, see \MSRef{L_COPYRIGHTCOLLECTION}{here}.
\ \\ \ \\
This course was developed within the scope of the \MExtLink{http://www.ve-und-mint.de}{VE\&MINT} project.
\ \\ \ \\
\begin{tabular}{lllll}
Course version: & \MSignatureMain (\MSignatureVersion) & \ \ &
Course date: & \MSignatureDate\\
Local version: & \MSignatureLocalization & \ \ & 
Course variant: & \MSignatureVariant\\
\end{tabular}

\begin{html}
<br />
<img src="../images/logo_tuberlin.png" style="height:68px">&nbsp;
<img src="../images/logo_unidarmstadt.png" style="height:68px">&nbsp;
<img src="../images/logo_unihannover.jpg" style="height:68px">&nbsp;
<img src="../images/logo_kit.png" style="height:68px">&nbsp;
<img src="../images/logo_unikassel.png" style="height:68px">&nbsp;
<img src="../images/logo_unipaderborn.png" style="height:68px">&nbsp;
<img src="../images/logo_unistuttgart.jpg" style="height:68px">
\end{html}
\ \\
\begin{html}
<p>
<h3  class="start">Suitable browsers</h3>
The following browsers can be used for the course: Firefox, Internet Explorer, Chrome, Safari, Opera.<br />
Some other browsers have difficulties rendering our unit pages correctly.
<br />
We recommend using only the fully updated latest versions of these browsers.
In particular, the course cannot be completed with obsolete browsers such as Internet Explorer 8 or earlier.
<br />
<br />
<script type="text/javascript">
document.write("Ihr Browsertyp: " + navigator.appName + ", Browserkennung: " + navigator.userAgent);
</script>
<br />
Our pages use active content including <a class="start" href="http://www.mathjax.org" target="_new">MathJax</a>. This requires JavaScript support to be activated.
<br />
<a href="http://www.mathjax.org"><img title="Powered by MathJax" src="../images/mj_logo.png" border="0" alt="Powered by MathJax" /></a><br />
<br />
<script type="text/javascript">
document.write("JavaScript is activated.");
</script>
<noscript>
JavaScript is not activated!
</noscript>
</p>
<br/>
<p >
In case of technical problems and difficulties, please contact the <a class="start" href="mailto:brueckenkurs@innocampus.tu-berlin.de">staff</a>.<br/>
<br/>
<br/>
</p>
\end{html}
\end{MSectionStart}

\MSubsection{Information and Imprint}

\begin{MXContent}{Course Information}{Course Information}{STD}
\MDeclareSiteUXID{VBKM_COURSEINFORMATION}

This preparatory online course serves those interested in studying a technical subject at university level as a preparation and for
assessing their competence in mathematics. 
In addition to extensive instructional material, the course features diagnostic self-assessment tests as well as
numerous instructional videos and interactive exercises.
The course was developed within the scope of the VE\&MINT project.
\ \\ \ \\
VE\&MINT is a project cooperation of MINT-Kolleg Baden-Württemberg
with VEMINT-Konsortium, Leibniz Universität Hannover and Technische Universität Berlin which seeks to
offer a preparatory mathematical online course that is freely accessible nationwide based on a free licence,
and more generally to promote exchange of instructional material as well as software between its sites.
\ \\ \ \\
For further information on the online course and the project, see \MExtLink{http://www.ve-und-mint.de}{www.ve-und-mint.de}.
\end{MXContent}

\begin{MXContent}{List of Authors}{List of Authors}{STD}
\MDeclareSiteUXID{VBKM_AUTHORS}
The following institutions participated in the preparation of this course:

% MAIL bei stuttgartern noch inkorrekt
\MExtLink{http://www.mint-kolleg.de}{MINT-Kolleg Baden-Württemberg} with sites \MExtLink{http://www.kit.edu}{KIT} and \MExtLink{http://www.uni-stuttgart.de}{Universität Stuttgart}:\\
\ \\
Direction: Dr. Claudia Goll (KA/S), Dr. Tobias Bentz (KA), Dr. Norbert Röhrl (S)\\ \ \\
\ \\ \ \\
Technical and content contact: \MExtLink{mailto:daniel.haase@kit.edu}{Dr. Daniel Haase}
\ \\ \ \\
Lecturers/authors:\\
\begin{tabular}{lll}
Akkar, Zineb&Dr.& \\
App, Andreas&Dr.& \\
Beer, Julia&Dipl. Geoökol.& \\
Dege, Christopher&M.A.& \\
Deißler, Juliane&Dipl.-Math.& \\
Dirmeier, Alexander&Dr.& \\
Feiler, Simon&Dr.& \\
Gulino, Harriet& \\
Haase, Daniel&Dr.& \\
Hägele, Constanze&Dr.& \\
Hankele, Vera&Dr.&\\
Hardy, Edme H.&PD Dr.& \\
Häußling, Rainer&PD Dr.& \\
Heidbüchel, Jörg&Dr.&\\
Helfrich-Schkarbanenko, Andreas&Dr.& \\
Herold, Heike&Dipl.-Ing.&\\
Hoffmann, Heiko&Dr.&\\
Karl, Inge&Dipl.-Ing.& \\
Kempf, Sonja&Dipl.-Inf.& \\
Kleb, Joachim&Dr.& \\
Koß, Rainer&Dipl.-Math.& \\
Liedtke, Jürgen&Dr.&\\
Lilli, Markus&Dr.& \\
Merkt, Domnic&Dr.& \\
Nese, Chandrasekhar&Dr.& \\
Pintschovius, Ursel&Dipl.-Biol.& \\
Pohl, Tanja&Dipl.-Ing.& \\
Rapedius, Kevin&Dr.& \\
Rutka, Vita&Dr.& \\
Schulz, Monika&Dr.& \\
Schüpp-Niewa, Barbara&Dr.& \\
Sternal, Oliver&Dr.& \\
Stroh, Tilo&Dr.& \\
Vettin, Laura&Dipl.-Math.& \\
Walliser, Nils-Ole&Dr.& \\
Weyreter, Gunther&Dr.-Ing.& \\
Ziebarth, Eva&Dr.&
\end{tabular}
\ \\ \ \\
\MExtLink{http://www.vemint.de}{VEMINT-Konsortium} with sites Kassel, Paderborn and Darmstadt:\\
\begin{itemize}
\item{Prof. Dr. Wolfram Koepf (KS)}
\item{Prof. Dr. Rolf Biehler (PB)}
\item{Prof. Dr. Regina Bruder (DA)}
\end{itemize}
% Dozenten/Autoren:\\
% \begin{tabular}{lll}
% ? & ? &?
% \end{tabular}
\ \\ \ \\
\MExtLink{http://www.innocampus.tu-berlin.de}{innoCampus} at \MExtLink{http://www.tu-berlin.de}{TU Berlin}:\\
\begin{tabular}{lll}
Born, Stefan & Dr.& \MExtLink{mailto:born@math.tu-berlin.de}{eMail}\\
Zorn, Erhard & Dipl. Phys. & \MExtLink{mailto:erhard@math.tu-berlin.de}{eMail}
\end{tabular}
\ \\ \ \\
\MExtLink{http://www.uni-hannover.de}{Leibniz Universität Hannover} represented by
\begin{itemize}
\item{Institut für Didaktik der Mathematik und Physik (Prof. Dr. Reinhard Hochmuth),}
\item{Institut für Algebraische Geometrie (Dr. Anne Frühbis-Krüger).}
\end{itemize}
\end{MXContent}

\begin{MXContent}{Imprint}{Imprint}{info}
\MDeclareSiteUXID{VBKM_IMPRESSUM}
\MLabel{L_COPYRIGHTCOLLECTION}

All course contents and materials are published under the open licence \MExtLink{http://creativecommons.org/licenses/by-sa/3.0/de/deed.en}{CC BY-SA 3.0}
and can be reused or adaptes so long as the author is named.
Authors' rights and copyrights of MINT-Kolleg Baden-Württemberg and
the VEMINT sites shall remain unaffected.
\ \\ \ \\
Reuse of course materials requires provision of the following information: Author's name (VE\&MINT-Projekt), applicable licence along with a URI/URL,
as well as an indication that the contents were changed (if applicable). Example:

\begin{center}
\begin{html}
<img src="../../images/tbeispiel.png" alt="Example image CC BY-SA"><br />
\end{html}
Prepared using materials of VE\&MINT project by J. Doe, lizence \MExtLink{http://creativecommons.org/licenses/by-sa/3.0/de/deed.en}{CC BY-SA 3.0 de}.
\end{center}
\ \\

The online units make use of the following materials, which themselves are under free licences:
\begin{itemize}
\item{Icons based on material from the \MExtLink{http://openiconlibrary.sourceforge.net}{Open Icon Library}.}
\item{Graphics that were created with \MExtLink{http://www.gimp.org}{GIMP}.}
\item{Function plots that were generated with \MExtLink{http://pgf.sourceforge.net}{PGF/TikZ} or exported from a CAS.}
\item{Videos that were created by \MExtLink{http://www.zml.kit.edu}{Zentrum für mediales Lernen} at \MExtLink{http://www.kit.edu}{KIT}.}
\end{itemize}

This online course was developed within the scope of the \MExtLink{http://www.ve-und-mint.de}{VE\&MINT} project by the partner universities that cooperate in the project.
You are viewing the version that is offered nationwide by \MExtLink{http://www.kit.edu}{Karlsruher Institut für Technologie}.
Some partner university offer versions of the course that are adjusted for their degree programmes.
\ \\ \ \\
Responsible for the content of this course: \MExtLink{http://www.mint-kolleg.de}{MINT-Kolleg Baden-Württemberg}, Dr. Claudia Goll. 
\ \\ \ \\
Please observe the notes concerning \MSRef{L_HAS}{Exclusion of liability}.
\ \\ \ \\
Licences for individual media contents of this course:\ \\
\MCopyrightCollection



\end{MXContent}

\begin{MXContent}{Exclusion of liability}{Exclusion of liability}{STD}
\MLabel{L_HAS}
\MDeclareSiteUXID{VBKM_LEGAL}

\MSubsubsubsectionx{Functionality and warranty}
The materials and software components used in this course are subject to a free licence but do not claim to be free from defects with regards to content or technology.
Even htough the contents were checked carefully, we can guarantee neither correctness or freedom from technical defects of the contents and software, nor availability of our servers.
This is particularly true as regards client-side software components (such as HTML5 and JS code) as well as the contents obtained from our servers.
%Unübersetzt, da an dieser Stelle unsinnig:
%Für auftretende clientseitige Probleme wie z.B. fehlerhaftes Browserverhalten oder optisch verfälschende Wiedergabe unserer Inhalte können wir aufgrund der Vielzahl der Browser und der häufigen
%Sicherheitsupdates für die gängigen Betriebssysteme ebenfalls keine Gewährleistung geben. -- Hans Adler

\MSubsubsubsectionx{Content and information exchange}
According to German law, the service provider is responsible for its own contents on these pages but is not obligated to survey transmitted or saved foreign information
or to search for circumstances that indicate illegal activity.
Obligations to remove or block the use of information according to general law remain unaffected.
However, liability in this connection is possible only from the the moment the service provider obtained knowledge of a concrete infringement.
If we get knowledge of any relevant infringements we will remove these contents immediately.

\MSubsubsubsectionx{Liability for Links}
Our service contains links to external websites by third parties, on whose contents we have no influence.
Therefore we cannot take responsibility for these foreign contents.
Responsible for the contens of the pages linked to is always the respective provider or operator.
The pages linked to were checked for possible infringements at the time of linking. Illegal contents were not in evidence at the time of linking.
Howerver, permanently controlling the contents of linked pages is unreasonable when there are no concrete indications of infringement.
If we get knowledge of any such links we will remove them immediately.

\MSubsubsubsectionx{Copyright / authors' right}
The contents developed by the VE\&MINT project as well as the software used on this site is subject to German authors' right.
Copying, adapting and sharing is permitted according to version 3.0 of the \MExtLink{http://de.wikipedia.org/wiki/Creative_Commons}{Creative Commons License}.
Contents contributed by or adopted from third parties are also subject to this licence. Respective copyrights and authors' rights remain unaffected.
Should you become aware of a copyright violation, please let us know. If we get aware of infringements we will remove any such contents promptly.
\end{MXContent}



\MSubsection{Display of formulas}

\begin{MXContent}{Display of formulas in the units}{Display of formulas}{STD}
\MDeclareSiteUXID{VBKM_DISPLAYINFO}

\begin{html}
<p>
Formulas in the units are displayed using the <a href="http://www.mathjax.org" target="_new">MathJax</a> library. This library offers two representation modes:
<ul>
<li> <b> MathML:</b> This mode represents the formula symbols as MathML tags suitable to the text.</li><br />
<li> <b>HTML-CSS:</b> In this mode, the formula symbols are included as pictures. This representation is more optically pleasant, but
display of the formulas in the browser is slower. We recommend this representation, especially
if your browser's MathML support is not adequate. This is the case especially in old browser versions.</li>
</ul>
You can configure the representation of formulas by right-clicking a formula and selecting the items Settings and Math Renderer in teh MathJax menu that opens.

<br />
<br />
Our online units assume an active internet connection. Missing formula symbols on unit pages indicate that your connection was interrupted or that
your browser prevents the loading of the symbols. In this case, proceed as described in the next section.
</p>

<h3 class="start">Suitable browsers </h3>
<p>
You can use the following browsers: Firefox, Internet Explorer, Opera.<br />
We recommend using <a href="http://www.mozilla.org/de/firefox/new/" target="_new">Firefox</a> (from version 3.6).<br /><br />
Internet Explorer may prevent the loading of special formula elements,<br />
in which case the use is prompted to permit loading of unsecure contents.<br /><br />

These messages can be turned off as follows:<br />

<ul>
 <li>
      Open Internet Options and click the "Security" tab.<br /><br />
      <img src="../images/iew1.png"> <br clear="all"><br clear="all">
 </li>

 <li> Click Trusted Sites (green check mark).<br /><br />
 </li>

 <li>
      Click "Sites" to add the ILIAS base address to the list:<br /><br />
      Base address of KIT's ILIAS: <a href="https://ilias.rz.uni-karlsruhe.de" target="_new">https://ilias.rz.uni-karlsruhe.de</a><br />
      <img src="../images/iew2.png"> <br clear="all"><br clear="all">
 </li>
 <li>
      Click "Close" and "OK" at the bottom of the dialogue window, close Internet Explorer and start die online unit again.
 </li>
</ul>

<br />
<br />

<p>
In case of technical difficulties and questions, please contact the <a class="start" href="mailto:brueckenkurs@innocampus.tu-berlin.de">staff</a>.<br/>
<br />
<b r/>
</p>
\end{html}

\end{MXContent}


\fi
