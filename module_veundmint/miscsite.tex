

\ifttm
\MSetSubject{\MINTPhysics}
\MSubject{Sonstiges}
\MSection{Benutzereinstellungen}

\begin{MSectionStart}
\MDeclareSiteUXID{VBKM_MISCENTRY}
Auf den folgenden Seiten finden sich
\begin{itemize}
\item{die \MSRef{L_CHAPTER}{Kapitelübersicht} des Kurses,}
\item{die \MSRef{L_CONFIG}{Persönlichen Einstellungen} für den Onlinekurs.}
\end{itemize}
\end{MSectionStart}

\MSubsection{Kapitelübersicht}

\begin{MXContent}{Kapitelübersicht}{Kapitel}{STD}
\MLabel{L_CHAPTER}
\MDeclareSiteUXID{VBKM_MISCCHAPTERS}
\MGlobalChapterTag
Einzelne Lernmodule und deren Abschnitte können jederzeit manuell angesteuert und bearbeitet werden:
\begin{itemize}
\item{Modul 01: \MSRef{VBKM01}{Elementares Rechnen}}
\item{Modul 02: \MSRef{VBKM02}{Gleichungen in einer Unbekannten}}
\item{Modul 03: \MSRef{VBKM03}{Ungleichungen in einer Unbekannten}}
\item{Modul 04: \MSRef{VBKM04}{Lineare Gleichungssysteme}}
\item{Modul 05: \MSRef{VBKM05}{Elementare Geometrie}}
\item{Modul 06: \MSRef{VBKM06}{Elementare Funktionen}}
\item{Modul 07: \MSRef{VBKM07}{Differentialrechnung}}
\item{Modul 08: \MSRef{VBKM08}{Integralrechnung}}
\item{Modul 09: \MSRef{VBKM09}{Orientierung im zweidimensionalen Koordinatensystem}}
\item{Modul 10: \MSRef{VBKM10}{Grundlagen der anschaulichen Vektorgeometrie}}
\end{itemize}
Auch die Tests und die Übungsaufgaben können in beliebiger Reihenfolge und auch mehrfach bearbeitet werden.
Dazu sind einfach die Einträge im Inhaltsverzeichnis im linken Randbereich anzuklicken.
\end{MXContent}

\MSubsection{Meine Kursdaten}

\begin{MXContent}{Meine Kursdaten}{Kursdaten}{STD}
\MLabel{L_CDATA}
\MGlobalDataTag
\MDeclareSiteUXID{VBKM_MISCCOURSEDATA}

Die persönlichen Daten können auf der \MSRef{L_CONFIG}{Einstellungsseite} eingerichtet werden.

Der Bearbeitungsstand der Kursmodule wird aufgrund der bearbeiteten Übungsaufgaben sowie der Ergebnisse
der Abschlusstests gemessen. Der Kurs gilt als erfolgreich absolviert, wenn sämtliche Abschlusstests mit mindestens 90\% der Punkte bestanden wurden:

\begin{html}
<div>
<p id="CDATAS">
Kann nicht auf Browserdaten zugreifen!
</p>
</div>
\end{html}

% \begin{html}
%  <div class="progress">
%   <div class="progress-bar progress-bar-striped active" role="progressbar" aria-valuenow="40" aria-valuemin="0" aria-valuemax="100" style="width:40\%">
%     40\%
%   </div>
% </div>
% \end{html}

\end{MXContent}


\MSubsection{Einstellungen}

\begin{MXContent}{Anmeldung zum Kurs}{Anmeldung zum Kurs}{STD}
\MLabel{L_CONFIG}
\MGlobalConfTag
\MDeclareSiteUXID{VBKM_MISCSETTINGS}

Falls Sie schon einen Account für den Brückenkurs besitzen,
können Sie sich auf der \MSRef{L_LOGIN}{Login-Seite} einloggen.
\begin{html}
<div class="show_scorm" style="display:none">
<br />
<b>Sie sind über das Lernsystem angemeldet und registriert.</b>
</div>
\end{html}
\ \\

\begin{html}
<div class="show_noscorm" style="display:none">
<b>Einen neuen Account registrieren:</b><br />
Nach erfolgreicher Registrierung werden die Daten auf unseren Servern abgelegt
und Ihr Bearbeitungsstand ist auch auf anderen Rechnern und Geräten verfügbar:
<div class="xreply">
<p id="LOGINFIELD">
Kann nicht auf Browserdaten zugreifen!
</p>
Persönliche Angaben (notwendig um eine Teilnahmebestätigungen zu erstellen):&nbsp;&nbsp;<button type="button" id="updatepdatabutton" style="visibility:hidden" class="criticalbutton" onclick="userupdate_click();">Persönliche Daten aktualisieren</button>
<table>
  <tr><td align=left>Vorname:</td><td align=left><input id="USER_VNAME" type="text" size="40" oninput="userupdate_check();" onkeyup="userupdate_check();" onpaste="userupdate_check();" onpropertychange="usercheck();"></input></td></tr>
  <tr><td align=left>Nachname:</td><td align=left><input id="USER_SNAME" type="text" size="40" oninput="userupdate_check();" onkeyup="userupdate_check();" onpaste="userupdate_check();" onpropertychange="usercheck();"></input></td></tr>
  <tr><td align=left>eMail:</td><td align=left><input id="USER_EMAIL" type="text" size="40" oninput="userupdate_check();" onkeyup="userupdate_check();" onpaste="userupdate_check();" onpropertychange="usercheck();"></input></td></tr>
</table>
<br />
Freiwillige Angaben zum angestrebten Studiengang (falls geplant bzw. bekannt):
<table>
  <tr><td align=left>Studiengang:</td><td align=left><input id="USER_SGANG" type="text" size="40" oninput="userupdate_check();" onkeyup="userupdate_check();" onpaste="userupdate_check();" onpropertychange="usercheck();"></input></td></tr>
  <tr><td align=left>Universität / Hochschule:</td><td align=left><input id="USER_UNI" type="text" size="40" oninput="userupdate_check();" onkeyup="userupdate_check();" onpaste="userupdate_check();" onpropertychange="usercheck();"></input></td></tr>
</table>
<br />
<br />
<div id="USERNAMEFIELD">
Wählen Sie einen Loginnamen für die Kursbearbeitung:<br />
Loginname: <input id="USER_UNAME" type="text" size="18" oninput="usercheck();" onkeyup="usercheck();" onpaste="usercheck();" onpropertychange="usercheck();"></input><img id="checkuserimg" src="images/questionmark.png">
<br />
<div class="usercreatereply"><p id="ulreply_p"> </p></div>
<br /><br />
</div>

Nach der Registrierung werden Ihre Daten auf unseren Kursservern hinterlegt, ebenso Ihre eingegebenen Lösungen und Testergebnisse. Diese Daten werden für anonymisierte
statistische Erhebungen genutzt und um die Qualtität der Inhalte und der Aufgaben zu verbessern. Bitte beachten Sie, dass Teilnahmebestätigungen für den Kurs nur ausgestellt
werden können, wenn die freiwilligen Felder vollständig ausgefüllt sind.
</div>
</div>
\end{html}

% \textbf{Kursdaten zurücksetzen:}
% 
% % <button type="button" id="CREATEBUTTON" style="height:60px;width:250px;background:#D0E0E0" onclick="usercreatelocal_click(1);">Obige Benutzerdaten im Browser registrieren</button><br />
% Die in diesem Browser und auf unseren Servern gespeicherten Daten zu diesem Kurs werden zurückgesetzt,
% der Kurs kann dadurch von vorne bearbeitet werden. Benutzername, Passwort und persönliche Daten bleiben erhalten.
% 
% \begin{html}
% <button type="button" class="criticalbutton" id="RESETBUTTON1" onclick="userreset_click();">Kursdaten zurücksetzen</button>
% \end{html}

\begin{html}
<div class="show_scorm" style="display:none">
<b>Ihre Benutzerdaten können Sie im Lernsystem einsehen und verändern.</b>
</div>
<div class="show_noscorm" style="display:none">
<b>Account löschen</b><br />
Die in diesem Browser und auf unseren Servern gespeicherten Daten inklusive des Accounts werden gelöscht.
Beachten Sie, dass ein einmal registrierter Accountname auch nach Löschung des Accounts nicht mehr für eine Registrierung
zur Verfügung steht.
<br /><br />
<button type="button" class="criticalbutton" id="RESETBUTTON2" onclick="userdelete_click();">Account löschen</button>
</div>
<br /><br />
<div class="xreply">
<p id="CHECKIS">
Kann nicht auf Browserdaten zugreifen!
</p>
</div>
\end{html}


\end{MXContent}

\begin{MXContent}{Login}{Login}{STD}
\MLabel{L_LOGIN}
\MGlobalLoginTag
\MDeclareSiteUXID{VBKM_MISCLOGIN}
\begin{html}
<div class="show_noscorm" style="display:none">
<div id="ONLYLOGINFIELD"></div>
\end{html}

Zum Einloggen sind der Benutzername und das Passwort der Registrierung erforderlich.

\begin{html}
</div>
<div class="show_scorm" style="display:none">
<b>Sie können sich nur über das Lernsystem an- und abmelden.</b>
</div>
\end{html}

\end{MXContent}

\MSubsection{Logout}

\begin{MXContent}{Logout}{Logout}{STD}
\MLabel{L_LOGOUT}
\MGlobalLogoutTag
\MDeclareSiteUXID{VBKM_MISCLOGOUT}

\begin{html}
<script type="text/javascript">
requestLogout = 1;
</script>
<p id="LOGINFIELD">
Kann nicht auf Browserdaten zugreifen!
</p>
<div class="show_scorm" style="display:none">
<b>Der Kurs ist geschlossen, ggf. müssen Sie ihn noch über das Lernsystem verlassen oder das Fenster schließen.</b>
</div>
<div class="show_noscorm" style="display:none">
<b>Der Kurs ist geschlossen, ggf. müssen Sie noch das Browserfenster oder den Tab schließen.</b>
</div>
\end{html}


\end{MXContent}


\MSubsection{Suchen}

\begin{MXContent}{Suchen}{Suchen}{STD}
\MLabel{L_SEARCHSITE}
\MDeclareSiteUXID{VBKM_MISCSEARCH}
\MGlobalSearchTag

Mit Strg-F sucht der Browser nach Schlagwörtern in der Liste. Folgende Stichwörter werden im Onlinekurs erklärt:

\special{html:<!-- msearchtable //-->}

\end{MXContent}

\begin{MXContent}{Favoriten}{Favoriten}{STD}
\MLabel{L_FAVORITESSITE}
\MGlobalFavoTag
\MDeclareSiteUXID{VBKM_MISCFAVORITES}

\special{html:<div id="FAVORITELISTLONG"></div>}

\end{MXContent}

\fi

