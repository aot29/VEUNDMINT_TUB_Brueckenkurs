\ifttm
\MSetSubject{\MINTPhysics}
\MSubject{Sonstiges}
\MSection{Benutzereinstellungen}

\begin{MSectionStart}
Auf den folgenden Seiten finden sich
\begin{itemize}
\item{die \MSRef{L_CHAPTER}{Kapitel�bersicht} des Kurses,}
\item{die pers�nlichen \MSRef{L_CDATA}{Kursdaten},}
\item{die \MSRef{L_CONFIG}{Einstellungen} f�r den Onlinekurs.}
\end{itemize}
\end{MSectionStart}

\MSubsection{Kapitel�bersicht}

\begin{MXContent}{Kapitel�bersicht}{Kapitel}{STD}
\MLabel{L_CHAPTER}
\MGlobalChapterTag
Einzelne Lernmodule und deren Abschnitte k�nnen jederzeit manuell angesteuert und bearbeitet werden:
\begin{itemize}
\item{Modul 01: \MSRef{VBKM01}{Elementares Rechnen}}
\item{Modul 02: \MSRef{VBKM02}{Gleichungen in einer Unbekannten}}
\item{Modul 03: \MSRef{VBKM03}{Ungleichungen in einer Unbekannten}}
\item{Modul 04: \MSRef{VBKM04}{Lineare Gleichungssysteme}}
\item{Modul 05: \MSRef{VBKM05}{Elementare Geometrie}}
\item{Modul 06: \MSRef{VBKM06}{Elementare Funktionen}}
\item{Modul 07: \MSRef{VBKM07}{Differentialrechnung}}
\item{Modul 08: \MSRef{VBKM08}{Integralrechnung}}
\item{Modul 09: \MSRef{VBKM09}{Orientierung im zweidimensionalen Koordinatensystem}}
\item{Modul 10: \MSRef{VBKM10}{Grundlagen der anschaulichen Vektorgeometrie}}
\end{itemize}
Auch die Tests und die �bungsaufgaben k�nnen in beliebiger Reihenfolge und auch mehrfach bearbeitet werden.
Dazu sind einfach die Eintr�ge im Inhaltsverzeichnis im linken Randbereich anzuklicken.
\end{MXContent}

\MSubsection{Meine Kursdaten}

\begin{MXContent}{Meine Kursdaten}{Kursdaten}{STD}
\MLabel{L_CDATA}
\MGlobalDataTag

Die pers�nlichen Daten k�nnen auf der \MSRef{L_CONFIG}{Einstellungsseite} eingerichtet werden.

Der Bearbeitungsstand der Kursmodule wird aufgrund der bearbeiteten �bungsaufgaben sowie der Ergebnisse
der Abschlusstests gemessen. Der Kurs gilt als erfolgreich absolviert, wenn s�mtliche Abschlusstests mit mindestens 90\% der Punkte bestanden wurden:

\begin{html}
<div>
<p id="CDATAS">
Kann nicht auf Browserdaten zugreifen!
</p>
</div>
\end{html}

% \begin{html}
%  <div class="progress">
%   <div class="progress-bar progress-bar-striped active" role="progressbar" aria-valuenow="40" aria-valuemin="0" aria-valuemax="100" style="width:40\%">
%     40\%
%   </div>
% </div>
% \end{html}

\end{MXContent}


\MSubsection{Einstellungen}

\begin{MXContent}{Einstellungen}{Einstellungen}{STD}
\MLabel{L_CONFIG}
\MGlobalConfTag

\begin{html}
<div class="xreply">
<p id="LOGINFIELD">
Kann nicht auf Browserdaten zugreifen!
</p>
<div id="USERNAMEFIELD">
W�hlen Sie einen Loginnamen f�r die Kursbearbeitung:<br />
Loginname: <input id="USER_UNAME" type="text" size="18" oninput="usercheck();" onkeyup="usercheck();" onpaste="usercheck();" onpropertychange="usercheck();"></input><img id="checkuserimg" src="images/questionmark.gif">
<br />
<div class="usercreatereply"><p id="ulreply_p"> </p></div>
<br /><br />
Freiwillige Angaben:
<table>
  <tr><td align=left>Vorname:</td><td align=left><input id="USER_VNAME" type="text" size="40"></input></td></tr>
  <tr><td align=left>Nachname:</td><td align=left><input id="USER_SNAME" type="text" size="40"></input></td></tr>
  <tr><td align=left>eMail:</td><td align=left><input id="USER_EMAIL" type="text" size="40"></input></td></tr>
</table>
<button type="button" id="CREATEBUTTON" style="height:60px;width:250px;background:#D0E0E0" onclick="usercreatelocal_click(1);">Obige Benutzerdaten im Browser registrieren</button><br />
</div>

<button type="button" id="RESETBUTTON" style="height:60px;width:250px;background:#FFD0C0" onclick="userreset_click();">Alle Benutzerdaten l�schen</button>
</div>

\end{html}

% <button type="button" id="LOGINBUTTON" style="height:60px;width:250px;background:#D0E0E0" onclick="userlogin_click();">Benutzer wechseln</button><br />


Daten werden wie folgt kommuniziert oder gespeichert:

\begin{itemize}
\item{\MConfigbox{CF_LOCAL} Kursdaten werden im Browser gespeichert.}
\item{\MConfigbox{CF_USAGE} Nutzungsverhalten wird anonym kommuniziert.}
\item{\MConfigbox{CF_TESTS} L�sungen werden anonym kommuniziert.}
\end{itemize}
\begin{html}
<div class="xreply">
<p id="CHECKIS">
Kann nicht auf Browserdaten zugreifen!
</p>
</div>
\end{html}

\begin{MInfo}
\MLabel{L_LS}
\textbf{Information zur Datenspeicherung}
\ \\ \ \\
Bei Aktivierung der Datenspeicherung werden die Kursdaten (insbesondere eingegebene L�sungen) in Ihrem Browser gespeichert.
Die Daten liegen auf Ihrem Rechner vor und werden nicht �ber das Internet versendet. Die Daten sind daher nicht sichtbar, wenn Sie sich
von einem anderen Browser oder Rechner anmelden. Technisch werden die Daten im lokalen Speicherbereich des Webbrowsers abgelegt.
\end{MInfo}


% \ \\ \ \\
% Ihre lokal gespeicherten Daten:
% 
% \begin{html}
% <textarea name="Name_OBJARRAYS" id="OBJARRAYS" rows="40" cols="80" style="background-color:#F0F0F0"></textarea>
% <script>
% var e = document.getElementById("OBJARRAYS");
% e.value = "?";
% e.readOnly = true;
% </script>
% \end{html}
% 
% \ \\ \ \\
% Das intersiteobj:
% 
% \begin{html}
% <textarea name="Name_OBJOUT" id="OBJOUT" rows="40" cols="80" style="background-color:#F0F0F0"></textarea>
% <script>
% var mys = JSON.stringify(intersiteobj);
% console.log("Decoded intersiteobj");
% var e = document.getElementById("OBJOUT");
% e.value = mys;
% e.readOnly = true;
% </script>
% \end{html}


\end{MXContent}

% \MSubsection{Lesezeichen}
% 
% \begin{MSContent}{Lesezeichen}{Lesezeichen}{STD}
% \MLabel{L_FAVORITES}
% \MGlobalFavoTag
% 
% ?
% 
% \end{MSContent}

\MSubsection{Suchen}

\begin{MXContent}{Suchen}{Suchen}{STD}
\MLabel{L_SEARCHSITE}
\MGlobalSearchTag

Mit Strg-F sucht der Browser nach Schlagw�rtern in der Liste. Folgende Stichw�rter werden im Onlinekurs erkl�rt:

\special{html:<!-- msearchtable //-->}

\end{MXContent}

\begin{MXContent}{Favoriten}{Favoriten}{STD}
\MLabel{L_FAVORITESSITE}
\MGlobalFavoTag

\special{html:<div id="FAVORITELISTLONG"></div>}

\end{MXContent}

\fi
