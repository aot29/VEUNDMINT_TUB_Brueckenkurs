\ifttm
\MSetSubject{\MINTPhysics}
\MSubject{Sonstiges}
\MSection{Benutzereinstellungen}

\begin{MSectionStart}
\MDeclareSiteUXID{VBKM_MISCENTRY}
Auf den folgenden Seiten finden sich
\begin{itemize}
\item{die \MSRef{L_CHAPTER}{Kapitel�bersicht} des Kurses,}
\item{die pers�nlichen \MSRef{L_CDATA}{Kursdaten},}
\item{die \MSRef{L_CONFIG}{Einstellungen} f�r den Onlinekurs.}
\end{itemize}
\end{MSectionStart}

\MSubsection{Kapitel�bersicht}

\begin{MXContent}{Kapitel�bersicht}{Kapitel}{STD}
\MLabel{L_CHAPTER}
\MDeclareSiteUXID{VBKM_MISCCHAPTERS}
\MGlobalChapterTag
Einzelne Lernmodule und deren Abschnitte k�nnen jederzeit manuell angesteuert und bearbeitet werden:
\begin{itemize}
\item{Modul 01: \MSRef{VBKM01}{Elementares Rechnen}}
\item{Modul 02: \MSRef{VBKM02}{Gleichungen in einer Unbekannten}}
\item{Modul 03: \MSRef{VBKM03}{Ungleichungen in einer Unbekannten}}
\item{Modul 04: \MSRef{VBKM04}{Lineare Gleichungssysteme}}
\item{Modul 05: \MSRef{VBKM05}{Elementare Geometrie}}
\item{Modul 06: \MSRef{VBKM06}{Elementare Funktionen}}
\item{Modul 07: \MSRef{VBKM07}{Differentialrechnung}}
\item{Modul 08: \MSRef{VBKM08}{Integralrechnung}}
\item{Modul 09: \MSRef{VBKM09}{Orientierung im zweidimensionalen Koordinatensystem}}
\item{Modul 10: \MSRef{VBKM10}{Grundlagen der anschaulichen Vektorgeometrie}}
\end{itemize}
Auch die Tests und die �bungsaufgaben k�nnen in beliebiger Reihenfolge und auch mehrfach bearbeitet werden.
Dazu sind einfach die Eintr�ge im Inhaltsverzeichnis im linken Randbereich anzuklicken.
\end{MXContent}

\MSubsection{Meine Kursdaten}

\begin{MXContent}{Meine Kursdaten}{Kursdaten}{STD}
\MLabel{L_CDATA}
\MGlobalDataTag
\MDeclareSiteUXID{VBKM_MISCCOURSEDATA}

Die pers�nlichen Daten k�nnen auf der \MSRef{L_CONFIG}{Einstellungsseite} eingerichtet werden.

Der Bearbeitungsstand der Kursmodule wird aufgrund der bearbeiteten �bungsaufgaben sowie der Ergebnisse
der Abschlusstests gemessen. Der Kurs gilt als erfolgreich absolviert, wenn s�mtliche Abschlusstests mit mindestens 90\% der Punkte bestanden wurden:

\begin{html}
<div>
<p id="CDATAS">
Kann nicht auf Browserdaten zugreifen!
</p>
</div>
\end{html}

% \begin{html}
%  <div class="progress">
%   <div class="progress-bar progress-bar-striped active" role="progressbar" aria-valuenow="40" aria-valuemin="0" aria-valuemax="100" style="width:40\%">
%     40\%
%   </div>
% </div>
% \end{html}

\end{MXContent}


\MSubsection{Einstellungen}

\begin{MXContent}{Anmeldung zum Kurs}{Anmeldung zum Kurs}{STD}
\MLabel{L_CONFIG}
\MGlobalConfTag
\MDeclareSiteUXID{VBKM_MISCSETTINGS}

Falls Sie schon einen Account f�r den Br�ckenkurs besitzen,
k�nnen Sie sich auf der \MSRef{L_LOGIN}{Login-Seite} einloggen.
\ \\

\textbf{Einen neuen Account registrieren:}\\
Nach erfolgreicher Registrierung werden die Daten auf unseren Servern abgelegt
und Ihr Bearbeitungsstand ist auch auf anderen Rechnern und Ger�ten verf�gbar:


\begin{html}
<div class="xreply">
<p id="LOGINFIELD">
Kann nicht auf Browserdaten zugreifen!
</p>
Pers�nliche Angaben (notwendig um eine Teilnahmebest�tigungen zu erstellen):&nbsp;&nbsp;<button type="button" id="updatepdatabutton" style="visibility:hidden" class="criticalbutton" onclick="userupdate_click();">Pers�nliche Daten aktualisieren</button>
<table>
  <tr><td align=left>Vorname:</td><td align=left><input id="USER_VNAME" type="text" size="40" oninput="userupdate_check();" onkeyup="userupdate_check();" onpaste="userupdate_check();" onpropertychange="usercheck();"></input></td></tr>
  <tr><td align=left>Nachname:</td><td align=left><input id="USER_SNAME" type="text" size="40" oninput="userupdate_check();" onkeyup="userupdate_check();" onpaste="userupdate_check();" onpropertychange="usercheck();"></input></td></tr>
  <tr><td align=left>eMail:</td><td align=left><input id="USER_EMAIL" type="text" size="40" oninput="userupdate_check();" onkeyup="userupdate_check();" onpaste="userupdate_check();" onpropertychange="usercheck();"></input></td></tr>
</table>
<br />
Freiwillige Angaben zum angestrebten Studiengang (falls geplant bzw. bekannt):
<table>
  <tr><td align=left>Studiengang:</td><td align=left><input id="USER_SGANG" type="text" size="40" oninput="userupdate_check();" onkeyup="userupdate_check();" onpaste="userupdate_check();" onpropertychange="usercheck();"></input></td></tr>
  <tr><td align=left>Universit�t / Hochschule:</td><td align=left><input id="USER_UNI" type="text" size="40" oninput="userupdate_check();" onkeyup="userupdate_check();" onpaste="userupdate_check();" onpropertychange="usercheck();"></input></td></tr>
</table>
<br />
<br />
<div id="USERNAMEFIELD">
W�hlen Sie einen Loginnamen f�r die Kursbearbeitung:<br />
Loginname: <input id="USER_UNAME" type="text" size="18" oninput="usercheck();" onkeyup="usercheck();" onpaste="usercheck();" onpropertychange="usercheck();"></input><img id="checkuserimg" src="images/questionmark.png">
<br />
<div class="usercreatereply"><p id="ulreply_p"> </p></div>
<br /><br />
</div>

Nach der Registrierung werden Ihre Daten auf unseren Kursservern hinterlegt, ebenso Ihre eingegebenen L�sungen und Testergebnisse. Diese Daten werden f�r anonymisierte
statistische Erhebungen genutzt und um die Qualtit�t der Inhalte und der Aufgaben zu verbessern. Bitte beachten Sie, dass Teilnahmebest�tigungen f�r den Kurs nur ausgestellt
werden k�nnen, wenn die freiwilligen Felder vollst�ndig ausgef�llt sind.
</div>

\end{html}


\textbf{Kursdaten l�schen:}

% <button type="button" id="CREATEBUTTON" style="height:60px;width:250px;background:#D0E0E0" onclick="usercreatelocal_click(1);">Obige Benutzerdaten im Browser registrieren</button><br />

Die in diesem Browser und auf unseren Servern gespeicherten Daten werden gel�scht,
der Kurs kann dadurch von vorne bearbeitet werden. Benutzername, Passwort und pers�nliche Daten bleiben erhalten.

\begin{html}
<button type="button" class="criticalbutton" id="RESETBUTTON1" onclick="userreset_click();">Alle Benutzerdaten l�schen</button>
\end{html}

\textbf{Account l�schen:}

Die in diesem Browser und auf unseren Servern gespeicherten Daten inklusive des Accounts werden gel�scht.
Beachten Sie, dass ein einmal registrierter Accountname auch nach L�sung des Accounts nicht mehr f�r eine Registrierung
zur Verf�gung steht.

\begin{html}
<button type="button" class="criticalbutton" id="RESETBUTTON2" onclick="userdelete_click();">Account l�schen</button>
\end{html}

\begin{html}
<div class="xreply">
<p id="CHECKIS">
Kann nicht auf Browserdaten zugreifen!
</p>
</div>
\end{html}

\end{MXContent}

\begin{MXContent}{Login}{Login}{STD}
\MLabel{L_LOGIN}
\MGlobalLoginTag
\MDeclareSiteUXID{VBKM_MISCLOGIN}
\begin{html}
<button type="button" id="LOGINBUTTON" style="height:60px;width:250px;background:#D0E0E0" onclick="userlogin_click();">Einloggen in den Kurs</button><br />
\end{html}

Zum Einloggen sind der Benutzername und das Passwort der Registrierung erforderlich.

\end{MXContent}


\begin{MXContent}{Logout}{Logout}{STD}
\MLabel{L_LOGOUT}
\MGlobalLogoutTag
\MDeclareSiteUXID{VBKM_MISCLOGOUT}

\begin{html}
<script type="text/javascript">
requestLogout = 1;
</script>
<p id="LOGINFIELD">
Kann nicht auf Browserdaten zugreifen!
</p>
\end{html}


Das Browserfenster bzw. der Tab kann nun geschlossen, oder der Kurs weiter bearbeitet werden.

\end{MXContent}


\MSubsection{Suchen}

\begin{MXContent}{Suchen}{Suchen}{STD}
\MLabel{L_SEARCHSITE}
\MDeclareSiteUXID{VBKM_MISCSEARCH}
\MGlobalSearchTag

Mit Strg-F sucht der Browser nach Schlagw�rtern in der Liste. Folgende Stichw�rter werden im Onlinekurs erkl�rt:

\special{html:<!-- msearchtable //-->}

\end{MXContent}

\begin{MXContent}{Favoriten}{Favoriten}{STD}
\MLabel{L_FAVORITESSITE}
\MGlobalFavoTag
\MDeclareSiteUXID{VBKM_MISCFAVORITES}

\special{html:<div id="FAVORITELISTLONG"></div>}

\end{MXContent}

\fi
