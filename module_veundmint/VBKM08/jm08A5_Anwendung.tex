%LaTeX-2e-File, Liedtke, 20140828.
%Inhalt: Anwendung der Integralrechnung.
%zuletzt bearbeitet: 20141002.

%Inhalt zum Abschnitt: Anwendung der Integralrechnung.
%\MSubsection{Anwendung zur Fl"achenberechnung}
%\begin{MContent}

%\begin{MXContent}{Geometrische Anwendung in der Fl"achenberechnung}%
%{Geometrische Anwendung in der Fl"achenberechnung}{STD}


\MPragma{MathSkip}

\begin{MXContent}{Fl"achenberechnung}{Fl"achenberechnung}{STD}
Eine erste Anwendung der Integrationsrechnung ist die Berechnung von Fl�cheninhalten, deren R�nder von mathematischen Funktionen beschrieben werden k�nnen.
Zur Veranschaulichung ist in der folgenden Abbildung (linkes Bild) die Funktion $f(x) = \frac{1}{2} x^3$ auf dem Intervall $[-2,2]$ dargestellt. Unser Ziel ist
die Berechnung des Fl�cheninhalts, der vom Graphen der Funktion und der $x$-Achse eingeschlossen wird. Unsere bisherigen Untersuchungen ergeben, dass das
Integral �ber diese ungerade Funktion in den Grenzen von -2 bis 2 genau Null ergeben wird, da die linke und rechte Teilfl�che gleich gro� sind, aber bei
der Integration unterschiedliche Vorzeichen erhalten. Das Integral entspricht hier also nicht dem Wert des Fl�cheninhalts.
Spiegeln wir jedoch die "`negative"' Fl�che an der $x$-Achse, geben der Funktion also ein positives Vorzeichen (rechtes Bild), dann k�nnen wir den Fl�cheninhalt
richtig bestimmen. Mathematisch bedeutet das, dass wir nicht das Integral der Funktion $f$ berechnen, sondern das Integral des Betrags $\left|f\right|$.

\ifttm
%\MUGraphics{\MPfadBilder/BildFlaechePolynomBetragxdrei.png}{scale=0.3}{Zur Berechnung von Fl"achen}{}
\MUGraphics{\MPfadBilder/jb08A5_Flaechenberechnung.png}{scale=0.3}{Zur Berechnung von Fl"achen}{}
\else
%LaTeX-File, Liedtke, 20141002.
%Bild Integral der Funktion $x^3/2$ von $-a$ bis $a$.
%Quelle: BildFlaechePolynomxdrei.tex (Modul Integrationstechniken, Liedtke)
\begin{center}
\begin{small}
\renewcommand{\jTikZScale}{0.8}
\tikzsetnextfilename{jb08A5_Flaechenberechnung}
\begin{tikzpicture}[scale=\jTikZScale,line width=1.5pt]
\begin{scope}[xshift=-8.5cm]
%\clip(-3.7,-4.7) rectangle (3.7, 4.7);
%Koordinatenachsen:
\draw[->] (-3.6, 0) -- (3.8, 0) node[below left]{$x$}; %x-Achse
\draw[->] (0, -4.6) -- (0, 4.8) node[below left]{$y$}; %y-Achse
%Achsenbeschriftung:
\foreach \x in {1, 2, 3} \draw (\x, 0) -- ++(0, -0.1) node[below] {$\x$}; 
\foreach \x in {1, 2, 3} \draw (-\x, 0) -- ++(0, 0.1) node[above] {$-\x$}; 
\foreach \y in {1, 2, 3, 4} \draw (0, \y) -- ++(-0.1, 0) node[left] {$\y$};
\foreach \y in {1, 2, 3, 4} \draw (0, -\y) -- ++(-0.1, 0) node[left] {$-\y$};
\node[below left] at (0, 0) {$0$};
%Funktionsgraph:
\draw[domain=-2:2,samples=120,color=\jccolorfkt, fill=\jccolorfktarea] %
  plot (\x, {1/2*pow(\x,3)}) -- (2, 4) -- (2, 0) -- (-2, 0) -- (-2, -4); 
%Beschriftung:
\draw[color=\jccolorfkt] (1.4, 0.7) -- +(0.4, 0);
\draw[color=\jccolorfkt] (1.6, 0.5) -- +(0, 0.4);
\draw[color=\jccolorfkt] (-1.4, -0.7) -- +(-0.4, 0);
\end{scope}
\begin{scope}{xshift=8.5cm}
%Quelle: BildFlaechePolynomBetragxdrei.tex (Modul Integrationstechniken, Liedtke)
%\clip(-3.7,-4.7) rectangle (3.7, 4.7);
%Koordinatenachsen:
\draw[->] (-3.6, 0) -- (3.8, 0) node[below left]{$x$}; %x-Achse
\draw[->] (0, -4.6) -- (0, 4.8) node[below left]{$y$}; %y-Achse
%Achsenbeschriftung:
\foreach \x in {1, 2, 3} \draw (\x, 0) -- ++(0, -0.1) node[below] {$\x$}; 
\foreach \x in {1, 2, 3} \draw (-\x, 0) -- ++(0, -0.1);
\node[below] at (-3, -0.1) {$-3$};
%\draw (-1, 0) -- ++(0, -0.1) node[below,fill=white] {$-1$}; 
\foreach \y in {1, 2, 3, 4} \draw (0, \y) -- ++(-0.1, 0) node[left] {$\y$};
\foreach \y in {1, 2, 3, 4} \draw (0, -\y) -- ++(-0.1, 0) node[left] {$-\y$};
\node[below left] at (0, 0) {$0$};
%Funktionsgraph:
\draw[domain=-2:2,samples=200,color=\jccolorfkt, fill=\jccolorfktarea] %
  plot (\x, {1/2*abs(pow(\x,3))}) -- (2, 4) -- (2, 0) -- (-2, 0) -- (-2, 4); 
\draw[domain=-2:0,samples=120,color=green!50!black,dashed,fill=\jccolorfktareahell] %
 (-2, 0) -- (-2, -4) -- plot (\x, {1/2*pow(\x,3)}) ; 
%Beschriftung:
\draw[color=\jccolorfkt] (1.4, 0.7) -- +(0.4, 0);
\draw[color=\jccolorfkt] (1.6, 0.5) -- +(0, 0.4);
\draw[color=\jccolorfkt] (-1.4, 0.7) -- +(-0.4, 0);
\draw[color=\jccolorfkt] (-1.6, 0.5) -- +(0, 0.4);
\draw[color=green!75!white] (-1.4, -0.7) -- +(-0.4, 0);
\end{scope}
\end{tikzpicture}
\end{small}
\end{center}
\fi

Durch die Bildung des Betrags der Funktion ben�tigen wir eine Aufteilung des Integrals in die Bereiche mit positivem und negativem Vorzeichen.
F"ur die Berechnung hei�t dies, dass wir das Integrationsintervall in Abschnitte zu unterteilen, in denen die Funktionswerte dasselbe Vorzeichen haben.

\begin{MXInfo}{Fl"achenberechnung} 
Gegeben ist eine Funktionen $f: [a, b] \rightarrow \R$ auf einem 
Intervall $[a, b]$. 
Weiter seien $x_1$ bis $x_m$ die Nullstellen von $f$ mit 
$x_1 < x_2 < \ldots < x_m$. 
Es werden $x_0 := a$ und $x_{m+1} := b$ gesetzt.

Dann
ist der Fl"acheninhalt zwischen dem Graphen von $f$ und der $x$-Achse gleich
\[
\int_{a}^{b} |f(x)| \jMD x %
= \sum_{k=0}^{m} \left|\int_{x_k}^{x_{k+1}} f(x) \jMD x\right|. %% 
\]
\end{MXInfo}

Sehen wir uns dies am oben dargestellten Beispiel etwas genauer an.

\begin{MExample}
Wir berechnen den Fl�cheninhalt $A$, den die Funktion $f(x) = \frac{1}{2} x^3$ im Bereich $[-2,2]$ mit der $x$-Ache einschlie�t. Die einzige Nullstelle der gegebenen Funktion finden wir bei $x_0 = 0$. Wir teilen den Integrationsbereich also in die beiden Teilintervalle $[-2,0]$ und $[0,2]$ auf und berechnen mit
%
\begin{eqnarray*}
A &=& \int_{-2}^{2} \left|\frac{1}{2} x^3\right| \jMD x = \left|\int_{-2}^{0} \frac{1}{2} x^3 \jMD x\right| + \left|\int_{0}^{2} \frac{1}{2} x^3 \jMD x\right| \\
  &=& \left|\left[\frac{1}{8} x^4\right]_{-2}^{0}\right| + \left|\left[\frac{1}{8} x^4\right]_{0}^{2}\right| \\
  &=& \left|0 - 2\right| + \left|2 - 0\right| \\
  &=& 4
\end{eqnarray*}
%
den Fl�cheninhalt zwischen Kurve und $x$-Achse zu $A = 4$.
\end{MExample}

Wir k�nnen nicht nur Fl�cheninhalte zwischen einer Kurve und der $x$-Achse bestimmen, sondern auch den Inhalt einer Fl�che, die von zwei Kurven eingeschlossen wird, wie in der folgenden Abildung veranschaulicht.

\ifttm
%\MUGraphics{\MPfadBilder/BildFlaechePolynomBetragxdrei.png}{scale=0.3}{Zur Berechnung von Fl"achen}{}
\MUGraphics{\MPfadBilder/jb08A5_FlaecheZwischenGraphen.png}{scale=0.3}%
%{Zur Berechnung von Fl"achen}{}
{Beispiel einer Fl"ache zwischen dem gr"un gezeichneten Graphen 
von $f$ und dem rot gezeichneten Graphen von $g$: Zur Berechnung des 
Fl"acheninhalts wird die Differenz der Funktionen $f - g$ betrachtet.}{0.5}
%\begin{center}
%\MUGraphicsSolo{\MPfadBilder/BildFlaecheB1.png}{scale=0.5}{}
%\MUGraphicsSolo{\MPfadBilder/BildFlaecheB2.png}{scale=0.5}{}
%\MUGraphicsSolo{\MPfadBilder/BildFlaecheB3.png}{scale=0.5}{}
%
%Beispiel einer Fl"ache zwischen dem gr"un gezeichneten Graphen 
%von $f$ und dem rot gezeichneten Graphen von $g$: Zur Berechnung des 
%Fl"acheninhalts wird die Differenz der Funktionen $f - g$ betrachtet.
%\end{center}
\else
%Quelle: BildFlaecheB1, MatheBildIntABspFlaecheB1.tex (Liedtke).
%Quelle: BildFlaecheB2, MatheBildIntABspFlaecheB2.tex (Liedtke).
%Quelle: BildFlaecheB3, MatheBildIntABspFlaecheB3.tex (Liedtke).
\begin{center}
\begin{small}
\renewcommand{\jTikZScale}{0.6}
\tikzsetnextfilename{jm08A5_FlaecheZwischenGraphen}
\begin{tikzpicture}[scale=\jTikZScale,line width=1.5pt,
 declare function={
  Fktf(\x) = -1/2 * (\x - 4) * (\x - 4) + 7/2; 
%  Fktg(\x) = 1/2 * (\x - 5) + 3; %  Fktg(\x) = 1/2 * \x + 1/2;
  Fktg(\x) = 1/2 * (\x + 1);
}
] 
\begin{scope}[xshift=-8.5cm]
%\clip(-0.7,-0.7) rectangle (6.2, 4.4);
%Koordinatenachsen:
\draw[->] (-0.6, 0) -- (6.1, 0) node[below left]{$x$}; %x-Achse
\draw[->] (0, -0.6) -- (0, 5.1) node[below left]{$y$}; %y-Achse
%Achsenbeschriftung:
\foreach \x in {1, 2, 3, 4, 5} \draw (\x, 0) -- ++(0, -0.1)%
 node[below] {$\x$}; 
\foreach \y in {1, 2, 3, 4} \draw (0, \y) -- ++(-0.1, 0) node[left] {$\y$};
\node[below left] at (0, 0) {$0$};
%Fl"ache unter $f$:
\draw[domain=2:5,samples=200,color=\jccolorfkt,fill=\jccolorfktarea] %
 plot (\x, {Fktf(\x)}) -- (5,0) -- (2,0) -- (2, {Fktf(2)}); 
%Fl"ache unter $g$:
%\draw[domain=2:5,samples=2,color=red!50!black,fill=red!50!white] %
% plot (\x, {Fktg(\x)}) -- (5,0) -- (2,0) -- (2, {Fktf(2)}); 
%Funktionsgraphen:
\draw[domain=2:5,samples=200,color=\jccolorfkt] %
 plot (\x, {Fktf(\x)}); 
\draw[domain=2:5,samples=2,color=red!50!black] %
 plot (\x, {Fktg(\x)}); 
%Schnittpunkte:
\draw[color=black] (2.0, 1.5) circle[radius=1pt];
\draw[color=black] (5.0, 3.0) circle[radius=1pt];
\end{scope}
%
\begin{scope}[xshift=0cm]
%\clip(-0.7,-0.7) rectangle (6.2, 4.4);
%Koordinatenachsen:
\draw[->] (-0.6, 0) -- (6.1, 0) node[below left]{$x$}; %x-Achse
\draw[->] (0, -0.6) -- (0, 5.1) node[below left]{$y$}; %y-Achse
%Achsenbeschriftung:
\foreach \x in {1, 2, 3, 4, 5} \draw (\x, 0) -- ++(0, -0.1)%
 node[below] {$\x$}; 
\foreach \y in {1, 2, 3, 4} \draw (0, \y) -- ++(-0.1, 0) node[left] {$\y$};
\node[below left] at (0, 0) {$0$};
%Fl"ache unter $f$:
%\draw[domain=2:5,samples=200,color=green!50!black,fill=green!50!white] %
% plot (\x, {Fktf(\x)}) -- (5,0) -- (2,0) -- (2, {Fktf(2)}); 
%Fl"ache unter $g$:
\draw[domain=2:5,samples=2,color=red!50!black,fill=red!50!white] %
 plot (\x, {Fktg(\x)}) -- (5,0) -- (2,0) -- (2, {Fktf(2)}); 
%Funktionsgraphen:
\draw[domain=2:5,samples=200,color=\jccolorfkt] %
 plot (\x, {Fktf(\x)}); 
\draw[domain=2:5,samples=2,color=red!50!black] %
 plot (\x, {Fktg(\x)}); 
%Schnittpunkte:
\draw[color=black] (2.0, 1.5) circle[radius=1pt];
\draw[color=black] (5.0, 3.0) circle[radius=1pt];
\end{scope}
%
\begin{scope}[xshift=8.5cm]
%\clip(-0.7,-0.7) rectangle (6.2, 4.4);
%Koordinatenachsen:
\draw[->] (-0.6, 0) -- (6.1, 0) node[below left]{$x$}; %x-Achse
\draw[->] (0, -0.6) -- (0, 5.1) node[below left]{$y$}; %y-Achse
%Achsenbeschriftung:
\foreach \x in {1, 2, 3, 4, 5} \draw (\x, 0) -- ++(0, -0.1)%
 node[below] {$\x$}; 
\foreach \y in {1, 2, 3, 4} \draw (0, \y) -- ++(-0.1, 0) node[left] {$\y$};
\node[below left] at (0, 0) {$0$};
%Fl"ache zwischen $f$ und $g$:
\draw[domain=2:5,samples=200,color=\jccolorfkt,fill=\jccolorfktarea] %
 plot (\x, {Fktf(\x)}) -- (5, {Fktg(5)}) -- (2, {Fktg(2)}); 
%Fl"ache unter $g$:
%\draw[domain=2:5,samples=2,color=red!50!black,fill=green!25!white]%
% plot (\x, {Fktg(\x)}) -- (5,0) -- (2,0) -- (2, {Fktf(2)}); 
\draw[domain=2:5,samples=2,color=red!50!black,dashed,% dotted,%
pattern color=red!50!black,pattern=crosshatch dots] %
 plot (\x, {Fktg(\x)}) -- (5,0) -- (2,0) -- (2, {Fktf(2)}); 
%Funktionsgraphen:
\draw[domain=2:5,samples=200,color=\jccolorfkt] %
 plot (\x, {Fktf(\x)}); 
\draw[domain=2:5,samples=2,color=red!50!black] %
 plot (\x, {Fktg(\x)}); 
%Schnittpunkte:
\draw[color=black] (2.0, 1.5) circle[radius=1pt];
\draw[color=black] (5.0, 3.0) circle[radius=1pt];
\end{scope}
\end{tikzpicture}
\end{small}
\end{center}
\fi

Dieses Prizip wollen wir uns ebenfalls zuerst formal und danach an einem Beispiel ansehen.

\begin{MXInfo}{Fl"achenberechnung zwischen den Graphen zweier Funktionen} 
Gegeben sind zwei Funktionen $f, g: [a, b] \rightarrow \R$ auf einem 
Intervall $[a, b]$. Weiter seien $x_1$ bis $x_m$ die Nullstellen von $f-g$ mit 
$x_1 < x_2 < \ldots < x_m$. Es werden $x_0 := a$ und $x_{m+1} := b$ gesetzt.

Dann kann der Fl"acheninhalt zwischen dem Graphen von $f$ und dem von $g$ durch
%
\[
\int_{a}^{b} |f(x) - g(x)| \jMD x %
= \sum_{k=0}^{m} \left|\int_{x_k}^{x_{k+1}} (f(x) - g(x)) \jMD x\right|. %% 
\]
%
berechnet werden.
\end{MXInfo}

Sehen wir uns dies an einem Beispiel an.

\begin{MExample}
Wir berechnen den Inhalt $I_A$ der Fl"ache zwischen den Graphen von 
$f(x) = x^2$ und $g(x) := 8 - \frac{1}{4} x^4$ f"ur $x \in [-2, 2]$.
Zun"achst untersuchen wir die Differenz $f - g$ der Funktionen auf
ihre Nullstellen hat. Mit
%
\begin{eqnarray*}
f(x) - g(x) & = & \frac{1}{4} x^4 + x^2 - 8 \\
 & = & \frac{1}{4} \left( x^4 + 4 x^2 - 32 \right) \\
 & = & \frac{1}{4} \left( x^4 + 4 x^2 + 2^2 - 2^2 - 32 \right) \\
 & = & \frac{1}{4} \left( \left( x^2 + 2 \right)^2 - 36 \right) \\
\end{eqnarray*}
%
k�nnen wir die reellen Nullstellen von $f-g$ berechnen:
\begin{eqnarray*}
& \left( x^2 + 2 \right)^2 - 36 = 0 \\
\Leftrightarrow & \left( x^2 + 2 \right)^2 = 36 \\
\Leftrightarrow & x^2 + 2 = 6 \\
\Leftrightarrow & x^2 = 4 \\
\Leftrightarrow & x = \pm 2 \\
\end{eqnarray*}

In unserer Rechnung haben nach dem Ziehen der ersten Wurzel auf eine n�here Betrachtung des Falls $x^2 + 2 = -6$ verzichtet, da wir
aus der daraus folgenden Gleichung $x^2 = -8$ keine reellen Nullstellen erhalten. Die reellen Nullstellen von $f-g$ sind $-2$ und $2$.
Dies sind gleichzeitig auch die Randstellen des Intervalls $[-2, 2]$. Eine Aufteilung des Integrals in verschiedene Bereiche ist also
nicht n�tig. Damit erhalten wir den Fl"acheninhalt $I_A$ zu
%
\begin{eqnarray*}
I_A = \int_{-2}^{2} \left|f(x) - g(x)\right| \jMD x %
 & = & \left|\int_{-2}^{2} \left(x^2 - \left(8 - \frac{1}{4} x^4\right)\right) \MD x\right| \\
% & = & 2 \left|\int_{0}^{2} \left(x^2 - \left(8 - \frac{1}{4} x^4\right)\right) \MD x\right| \\
 & = & 2 \left|\left[\frac{1}{20} x^5 + \frac{1}{3} x^3 - 8 x\right]_{0}^{2}\right| \\
% & = & 2 \left|\left(\frac{32}{20} + \frac{8}{3} - 16\right) - 0\right| \\
% & = & \left|2 \cdot 8 \frac{3 + 5}{15} - 32\right| \\
 & = & 23 + \frac{7}{15} \\
 & = & \frac{352}{15}. %%
\end{eqnarray*}
%
\end{MExample}

\end{MXContent}


\begin{MXContent}{Physikalische Anwendungen}{Physikalische Anwendungen}{STD}

Die Geschwindigkeit $v$ beschreibt die momentane "Anderungsrate des Ortes 
zur Zeit $t$. Es gilt also $v = \frac{\text{d}s}{\text{d}t}$, wenn wir $v = v(t)$ und $s = s(t)$
als Fuktionen der Zeit auffassen.
Der aktuelle Aufenthaltsort $s(T)$ ergibt sich durch die Umkehrung der Ableitung, also durch die Integration der
Geschwindigkeit �ber die Zeit. Mit dem Anfangswert $s(t = 0) = s_0$ zur Zeit $t = 0$ erhalten wir damit
%
\begin{eqnarray*}
\int_{0}^{T}\frac{\text{d}s}{\text{d}t} \jMD t &=& \int_{0}^{T} v \jMD t \\
\left[s(t)\right]_{0}^{T} &=& \int_{0}^{T} v \jMD t \\
s(T) - s(0) &=& \int_{0}^{T} v \jMD t \\
s(T) &=& s_0 + \int_0^T v(t) \jMD t.
\end{eqnarray*}
%
Ein weiteres Beispiel aus der Physik, das Ihnen bekannt sein k�nnte, ist die Bestimmung der Arbeit als Produkt aus Kraft und Weg: $W = F \cdot s$. Ist die Kraft jedoch wegabh�ngig, gilt dieses Gesetz nicht mehr in seiner einfachen Form. Um die Arbeit, die z.B. beim Verschieben eines massiven K�rpers entlang eines Weges verrichtet wird, zu bestimmen, m�ssen wir die aufgewendete Kraft dann entlang des Weges vom Anfangspunkt $s_1$ bis zum Endpunkt $s_2$ integrieren:

%
\begin{eqnarray*}
W = \int_{s_1}^{s_2}F(s) \jMD s.
\end{eqnarray*}
%
Dies sollen nur zwei Beispiele aus dem physikalisch-technischen Bereich sein. Sie werden im Verlauf Ihres Studiums einer ganzen Reihe weiterer Anwendungen der Integration begegnen.
\end{MXContent}

%\end{MContent}

%Dateiende.