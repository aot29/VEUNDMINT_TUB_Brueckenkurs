%LaTeX-2e-File, Liedtke, 20140828.
%Inhalt: Einf"uhrung des bestimmten Integrals.
%zuletzt bearbeitet: 20140929.

\MPragma{MathSkip}


%Inhalt zum Abschnitt: Integral
\begin{MXContent}{Integral}{Integral}{STD}

%Hier wird die Idee zu einer {\glqq}globalen Kenngr"o"se{\grqq} in der 
%Form des nach Riemann benannten Integrals vorgestellt.

%Das nach Riemann benannte Integral ordnet einer Funktion eine Zahl zu, die 
Das Integral einer Funktion ist eine Zahl, die sich aus einer Summenbildung 
-- man sagt auch aus einer Bilanzierung -- von gewichteten Funktionswerten 
ergibt. 
%F"ur das nach Riemann benannte Integral wird die Gewichtung "uber die 
%L"ange des Definitionsbereichs vorgenommen. 
In dem nach Riemann benannten Integral werden der Funktionsverlauf durch eine 
Treppenfunktion angen"ahert, und die Funktionswerte, gewichtet 
mit der jeweiligen Intervalll"ange, aufsummiert.

\MUGraphics{\MPfadBilder/BildIntegralObersumme.png}{scale=0.5}%
{Zur Definition des Integrals: Treppenfunktion zur Unterteilung in acht 
Teilintervalle}{}

\ifttm
\MUGraphics{\MPfadBilder/jb08A2_Integraldefinition.png}{scale=0.5}%
{Zur Definition des Integrals: Treppenfunktion zur Unterteilung in acht 
Teilintervalle}{}
\else
\begin{center}
%Bild: {\MPfadBilder/jb08A2_Integraldefinition.tex}
%LaTeX-File, Liedtke, 20140925.
%VBKM-Modul 7 Differentialrechnung: Bild zur Monotonie.
%Bildname: jb08A2_Integraldefinition.tex.
%Erstellt: 20140925, Liedtke.
\begin{small}
\renewcommand{\jTikZScale}{1.0}
\tikzsetnextfilename{jb08A2_Integraldefinition}
\begin{tikzpicture}[line width=1.5pt,scale=\jTikZScale, %
declare function={
  x0 = 0;
  x1 = 8;
  fkt(\x) = 4/(\x + 1);
}
] %[every node/.style={fill=white}] 
%,every node/.style={fill=white}] 
%Koordinatenachsen:
\draw[->] (-0.6, 0) -- (9, 0) node[below left]{$x$}; %x-Achse
\draw[->] (0, -0.6) -- (0, 5) node[below left]{$y$}; %y-Achse
%Achsenbeschriftung:
\foreach \x in {1, 2, 3, 4, 5, 6, 7, 8} \draw (\x, 0) -- ++(0, -0.1) %
 node[below] {$\x$}; 
\foreach \y in {1, 2, 3, 4} \draw (0, \y) -- ++(-0.1, 0) %
 node[below left] {$\y$};
%\node[below left] at (0, 0) {$0$};
%Treppenfunktion:
\begin{scope}[line width=1pt,color=blue]
\foreach \x in {0, 1, 2, 3, 4, 5, 6, 7} \draw (\x, {fkt(\x)}) -- ++(1, 0) %
 -- ++(0, {(-1)*fkt(\x)}); %
\end{scope}
%Funktion:
\draw[domain=0.0:8.0,samples=120,color=\jccolorfkt] %
 plot (\x, {fkt(\x)});
%Beschriftung:
\end{tikzpicture}
\end{small}
%end of file
\end{center}
\fi
%Bildende.


\begin{MXInfo}{Integral} 
Gegeben ist eine Funktion $f: [a, b] \rightarrow \R$ auf einem reellen Intervall
$[a, b]$. 
Das \MEntry{bestimmte Integral}{Integral!bestimmtes} von $f$ ist der Grenzwert
der Summen
\begin{equation}
S_n := \sum_{k=1}^n f(z_k) \cdot \Delta(x_k) %
\qquad \text{mit } \Delta(x_k) := x_k - x_{k-1} %%
\end{equation}
f"ur $n$ gegen unendlich, wenn $\max_{1 \leq k \leq n} \Delta(x_k)$ gegen null 
strebt.
Dabei ist $a = x_0 < x_1 < x_2 < \ldots < x_n = b$ und $z_k$ eine Zahl 
zwischen $x_{k-1}$ und $x_k$ f"ur $k \in \N$ mit $1 \leq k \leq n$. 
Die Summen $S_n$ werden auch Riemannsummen zur Zerlegung $(x_0, \ldots, x_n)$ 
und den Zwischenstellen $z_k$ genannt.

Wenn dieser Grenzwert, das Integral, existiert, wird daf"ur
\begin{equation}
\int_a^b f(x) \jMD x = \lim_{n \rightarrow \infty} S_n %%
\end{equation}
geschrieben. Es hei"st $a$ Untergrenze, $b$ Obergrenze des Integrals "uber dem
Integranden $f(x)$ mit der Integrationsvariablen $x$.
\end{MXInfo}

Als Beispiel wird das Integral von $f: [0, b] \rightarrow \R, x \mapsto x$ 
berechnet, 
wobei die Berechnung des Grenzwertes im Vordergrund steht.

\begin{MExample}
Es soll das Integral von $f: [0, b] \rightarrow \R, x \mapsto x$ berechnet werden.
Dazu werden zum Intervall $[0, b]$ Teilintervalle $[x_{k-1}, x_k]$ mit 
$x_0 := 0$ und $x_k := x_{k-1} + \frac{b}{n}$ betrachtet. Dann ist 
$x_k = \frac{b \cdot k}{n}$ und damit 
$a = x_0 < x_1 < x_2 < \ldots < x_n = b$.

Mit $\Delta(x_k) = x_k - x_{k-1} = \frac{b}{n}$ ergibt sich
\[
S_n = \sum_{k=1}^{n} \frac{b \cdot k}{n} \cdot \frac{b}{n} %
 = \frac{b^2}{n^2} \sum_{k=1}^{n} k %
 = \frac{b^2}{n^2} \frac{n (n+1)}{2} %
 = \frac{b^2}{2} \cdot \frac{n+1}{n} %
 = \frac{b^2}{2} \cdot \left(1 + \frac{1}{n}\right) %
\]
Mit $\displaystyle\lim_{n \rightarrow \infty} \frac{1}{n} = 0$ ergibt sich f"ur 
das Integral
\[
\int_0^b x \jMD x = \lim_{n \rightarrow \infty} S_n =  \frac{1}{2} b^2, %%
\]
wenn auch alle anderen Zerlegungen des Intervalls denselben Grenzwert haben,
was tats"achlich der Fall ist, in diesem Beispiel aber nicht weiter ausgef"uhrt 
werden soll. 
\end{MExample}

Aus allgemeinen "Uberlegung zu stetigen Funktionen ergibt sich, dass jedenfalls 
f"ur solche Funktionen das Integral stets existiert.

Damit ist eine sehr gro"se Klasse von Funktionen integrierbar: alle Polynome,
rationale Funktionen, trigonometrische und Exponential- und Logarithmusfunktionen
sowie deren Verkn"upfungen.

Somit ist es f"ur praktische Rechnungen von besonderem Interesse, m"oglichst 
einfach Integrale berechnen zu k"onnen. Ein wichtiges Ergebnis dazu ist der 
Hauptsatz der Differential- und Integralrechnung, der im n"achsten Abschnitt 
erl"autert wird.

Eine Absch"atzung des Integrals ergibt sich, wenn man einerseits Riemannsummen 
$U_n$ mit {\glqq}Zwischenstellen{\grqq} $z_k$ berechnet, f"ur die $f(z_k)$ ein
minimaler Wert der Funktion $f$ auf $[x_{k-1}, x_k]$ ist, und andererseits 
solche Riemannsummen $O_n$ betrachtet, f"ur die $f(z_k)$ einen maximalen Wert 
hat. (Sofern das Minimum bzw. Maximum nicht existiert, betrachtet man das 
Infimum bzw. das Supremum.)

Aufgrund dieser Konstruktion ist $U_n \leq O_n$. 
Dementsprechend hei"sen die Riemannsummen $U_n$ Untersummen und $O_n$ Obersummen. 
Wenn die Grenzwerte f"ur aller Unterteilungen 
existieren und gleich sind, wenn also das Integral existiert, dann gilt die 
Ungleichung
\begin{equation}
U_n \leq \int_a^b f(x) \jMD x \leq O_n
\end{equation}
Eine solche Ungleichung wird auch Absch"atzung genannt, da ein gesuchter Wert 
mit einem anderen, m"oglichst einfach zu berechnenden Wert verglichen wird, 
der zudem m"oglichst {\glqq}nahe{\grqq} am gesuchten Wert liegt.

\begin{MExample}
Im vorherigen Beispiel wurde zu $f(x) = x$ f"ur jedes $n \in \N$ und 
$x_k = \frac{b \cdot k}{n}$, also 
$\Delta(x_k) = x_k - x_{k-1} = \frac{b}{n}$ 
eine Obersumme
\[
O_n = \sum_{k=1}^n \sup_{[x_{k-1}, x_k]} f(x) \cdot \Delta(x_k) %
 = \sum_{k=1}^n \frac{b \cdot k}{n} \cdot \Delta(x_k) %%
\]
konstruiert.

Eine Untersumme $U_n$ ist f"ur diese Einteilung durch
\[
U_n = \sum_{k=1}^n \inf_{[x_{k-1}, x_k]} f(x) \cdot \Delta(x_k) %
 = \sum_{k=1}^n \frac{b \cdot (k-1)}{n} \cdot \Delta(x_k) %%
\]
gegeben.
\end{MExample}

\end{MXContent}


%\begin{MXContent}{Rechenregeln f"ur Integrale}{Rechenregeln f"ur Integrale}{STD}
\begin{MXContent}{Rechenregeln}{Rechenregeln}{STD}

Sei $f: [a, b] \rightarrow \R$ eine integrierbare Funktion.

F"ur jede Zahl $z$ zwischen $a$ und $b$ gilt
\begin{equation}
\int_a^b f(x) \jMD x = \int_a^z f(x) \jMD x + \int_z^b f(x) \jMD x %%
\end{equation}

Indem 
\begin{equation}
\int_\beta^\alpha f(x) \jMD x := -\int_\alpha^\beta f(x) \jMD x %%
\end{equation}
gesetzt wird, gilt obige Regel f"ur alle reellen Zahlen $z$, f"ur die die 
beiden rechts stehenden Integrale existieren.

Die Rechenregel ist praktisch, um Funktionen mit Betr"agen oder andere 
abschnittsweise definierte Funktionen zu integrieren.

\begin{MExample}
Das Integral der Funktion $f: [-4, 6] \rightarrow \R, x \mapsto |x|$ ist
\begin{eqnarray*}
\int_{-4}^{6} |x| \jMD x % 
 = \int_{-4}^{0} -x \jMD x + \int_{0}^{6} x \jMD x %
 & = & \left[-\frac{1}{2} x^2\right]_{-4}^{0} % 
   + \left[\frac{1}{2} x^2\right]_{0}^{6} \\
 & = & (0 - (-8)) + (18 - 0) %
 = 26 %%
\end{eqnarray*}
\end{MExample}

Die Integration "uber die Summe zweier Funktionen kann ebenfalls in zwei 
Integrale zerlegt werden: 
Seien $f$ und $g$ auf $[a, b]$ integrierbare Funktionen und $r$ eine reelle Zahl.

Dann gilt
\begin{equation}
\int_a^b (f(x) + g(x)) \jMD x = \int_a^b f(x) \jMD x + \int_a^b g(x) \jMD x %%
\end{equation}

F"ur Vielfache einer Funktion gilt
\begin{equation}
\int_a^b r \cdot f(x) \jMD x = r \cdot \int_a^b f(x) \jMD x %%
\end{equation}
\end{MXContent}


\begin{MXContent}{Eigenschaften des Integrals}{Eigenschaften des Integrals}{STD}

F"ur ungerade Funktionen $f: [-c, c] \rightarrow \R$ ist das Integral null. 
Als Beispiel wird hier der Graph der Funktion $f$ auf $[-2; 2]$ mit 
$f(x) = x^3$ gezeigt:
\MUGraphics{\MPfadBilder/BildFlaechePolynomxdrei.png}{scale=0.4}%
{Ungerade Funktion auf einem Intervall $[-2, 2]$.}{}

Denn die 
beiden Teilfl"achen zwischen dem Graphen von $f$ und der positiven $x$-Achse 
beziehungsweise der negativen $x$-Achse gehen durch eine Punktspiegelung 
ineinander "uber. Positive Funktionswerte gehen in negative "uber und umgekehrt.
In der Riemannsumme aufsummiert ergibt sich im Grenzwert null.

\begin{equation}
\int_{-c}^c f(x) \jMD x = 0 %%
\end{equation}

Im Fall einer geraden Funktion $g: [-c, c] \rightarrow \R$ ist der Graph 
symmetrisch
bez"uglich der $y$-Achse. Die Fl"ache zwischen dem Graphen von $f$ und der 
$x$-Achse ist hier symmetrisch bez"uglich der $y$-Achse. Die Teilfl"ache links 
davon ist also das Spiegelbild der rechts liegenden Fl"ache. Beide zusammen 
ergeben die Gesamtfl"ache.

\begin{equation}
\int_{-c}^c g(x) \jMD x = 2 \cdot \int_{0}^c g(x) \jMD x %%
\end{equation}

Vergleiche von Funktionen "ubertragen sich auf deren Integral:
Wenn $f(x) \leq g(x)$ f"ur alle $x$-Werte ist, gilt 

\begin{equation}
\int_a^b f(x) \jMD x \leq \int_a^b f(x) \jMD x %%
\end{equation}

Insbesondere ist
\begin{equation}
\left|\int_a^b f(x) \jMD x\right| \leq \int_a^b |f(x)| \jMD x %%
\end{equation}

Weiter gilt f"ur eine stetige Funktion $f$ dann
\begin{equation}
(b-a) \cdot \min f \leq \int_a^b f(x) \jMD x \leq (b-a) \cdot \max f %%
\end{equation}
%wenn das Minimum $\min f$ und das Maximum $\max f$ existiert (andernfalls 
%muss das Infimum bzw. das Supremum der beschr"ankten Funktion $f$ betrachtet
%werden).

\end{MXContent}


%Dateiende.

