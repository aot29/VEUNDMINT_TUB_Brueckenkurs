% MINTMOD Version P0.1.0, needs to be consistent with preprocesser object in tex2x and MPragma-Version at the end of this file

% Parameter aus Konvertierungsprozess (PDF und HTML-Erzeugung wenn vom Konverter aus gestartet) werden hier eingefuegt, Preambleincludes werden am Schluss angehaengt

\newif\ifttm                % gesetzt falls Uebersetzung in HTML stattfindet, sonst uebersetzung in PDF

% Wahl der Notationsvariante ist im PDF immer std, in der HTML-Uebersetzung wird vom Konverter die Auswahl modifiziert
\newif\ifvariantstd
\newif\ifvariantunotation
\variantstdtrue % Diese Zeile wird vom Konverter erkannt und ggf. modifiziert, daher nicht veraendern!


\def\MOutputDVI{1}
\def\MOutputPDF{2}
\def\MOutputHTML{3}
\newcounter{MOutput}

\ifttm
\usepackage{german}
\usepackage{array}
\usepackage{amsmath}
\usepackage{amssymb}
\usepackage{amsthm}
\else
\documentclass[ngerman,oneside]{scrbook}
\usepackage{etex}
\usepackage[latin1]{inputenc}
\usepackage{textcomp}
\usepackage[ngerman]{babel}
\usepackage[pdftex]{color}
\usepackage{xcolor}
\usepackage{graphicx}
\usepackage[all]{xy}
\usepackage{fancyhdr}
\usepackage{verbatim}
\usepackage{array}
\usepackage{float}
\usepackage{makeidx}
\usepackage{amsmath}
\usepackage{amstext}
\usepackage{amssymb}
\usepackage{amsthm}
\usepackage[ngerman]{varioref}
\usepackage{framed}
\usepackage{supertabular}
\usepackage{longtable}
\usepackage{maxpage}
\usepackage{tikz}
\usepackage{tikzscale}
\usepackage{tikz-3dplot}
\usepackage{bibgerm}
\usepackage{chemarrow}
\usepackage{polynom}
%\usepackage{draftwatermark}
\usepackage{pdflscape}
\usetikzlibrary{calc}
\usetikzlibrary{through}
\usetikzlibrary{shapes.geometric}
\usetikzlibrary{arrows}
\usetikzlibrary{intersections}
\usetikzlibrary{decorations.pathmorphing}
\usetikzlibrary{external}
\usetikzlibrary{patterns}
\usetikzlibrary{fadings}
\usepackage[colorlinks=true,linkcolor=blue]{hyperref} 
\usepackage[all]{hypcap}
%\usepackage[colorlinks=true,linkcolor=blue,bookmarksopen=true]{hyperref} 
\usepackage{ifpdf}

\usepackage{movie15}

\setcounter{tocdepth}{2} % In Inhaltsverzeichnis bis subsection
\setcounter{secnumdepth}{3} % Nummeriert bis subsubsection

\setlength{\LTpost}{0pt} % Fuer longtable
\setlength{\parindent}{0pt}
\setlength{\parskip}{8pt}
%\setlength{\parskip}{9pt plus 2pt minus 1pt}
\setlength{\abovecaptionskip}{-0.25ex}
\setlength{\belowcaptionskip}{-0.25ex}
\fi

\ifttm
\newcommand{\MDebugMessage}[1]{\special{html:<!-- debugprint;;}#1\special{html:; //-->}}
\else
%\newcommand{\MDebugMessage}[1]{\immediate\write\mintlog{#1}}
\newcommand{\MDebugMessage}[1]{}
\fi

\def\MPageHeaderDef{%
\pagestyle{fancy}%
\fancyhead[r]{(C) VE\&MINT-Projekt}
\fancyfoot[c]{\thepage\\--- CCL BY-SA 3.0 ---}
}


\ifttm%
\def\MRelax{}%
\else%
\def\MRelax{\relax}%
\fi%

%--------------------------- Uebernahme von speziellen XML-Versionen einiger LaTeX-Kommandos aus xmlbefehle.tex vom alten Kasseler Konverter ---------------

\newcommand{\MSep}{\left\|{\phantom{\frac1g}}\right.}

\newcommand{\ML}{L}

\newcommand{\MGGT}{\mathrm{ggT}}


\ifttm
% Verhindert dass die subsection-nummer doppelt in der toccaption auftaucht (sollte ggf. in toccaption gefixt werden so dass diese Ueberschreibung nicht notwendig ist)
\renewcommand{\thesubsection}{}
% Kommandos die ttm nicht kennt
\newcommand{\binomial}[2]{{#1 \choose #2}} %  Binomialkoeffizienten
\newcommand{\eur}{\begin{html}&euro;\end{html}}
\newcommand{\square}{\begin{html}&square;\end{html}}
\newcommand{\glqq}{"'}  \newcommand{\grqq}{"'}
\newcommand{\nRightarrow}{\special{html: &nrArr; }}
\newcommand{\nmid}{\special{html: &nmid; }}
\newcommand{\nparallel}{\begin{html}&nparallel;\end{html}}
\newcommand{\mapstoo}{\begin{html}<mo>&map;</mo>\end{html}}

% Schnitt und Vereinigungssymbole von Mengen haben zu kleine Abstaende; korrigiert:
\newcommand{\ccup}{\,\!\cup\,\!}
\newcommand{\ccap}{\,\!\cap\,\!}


% Umsetzung von mathbb im HTML
\renewcommand{\mathbb}[1]{\begin{html}<mo>&#1opf;</mo>\end{html}}
\fi

%---------------------- Strukturierung ----------------------------------------------------------------------------------------------------------------------

%---------------------- Kapselung des sectioning findet auf drei Ebenen statt:
% 1. Die LateX-Befehl
% 2. Die D-Versionen der Befehle, die nur die Grade der Abschnitte umhaengen falls notwendig
% 3. Die M-Versionen der Befehle, die zusaetzliche Formatierungen vornehmen, Skripten starten und das HTML codieren
% Im Modultext duerfen nur die M-Befehle verwendet werden!

\ifttm

  \def\Dsubsubsubsection#1{\subsubsubsection{#1}}
  \def\Dsubsubsection#1{\subsubsection{#1}\addtocounter{subsubsection}{1}} % ttm-Fehler korrigieren
  \def\Dsubsection#1{\subsection{#1}}
  \def\Dsection#1{\section{#1}} % Im HTML wird nur der Sektionstitel gegeben
  \def\Dchapter#1{\chapter{#1}}
  \def\Dsubsubsubsectionx#1{\subsubsubsection*{#1}}
  \def\Dsubsubsectionx#1{\subsubsection*{#1}}
  \def\Dsubsectionx#1{\subsection*{#1}}
  \def\Dsectionx#1{\section*{#1}}
  \def\Dchapterx#1{\chapter*{#1}}

\else

  \def\Dsubsubsubsection#1{\subsubsection{#1}}
  \def\Dsubsubsection#1{\subsection{#1}}
  \def\Dsubsection#1{\section{#1}}
  \def\Dsection#1{\chapter{#1}}
  \def\Dchapter#1{\title{#1}}
  \def\Dsubsubsubsectionx#1{\subsubsection*{#1}}
  \def\Dsubsubsectionx#1{\subsection*{#1}}
  \def\Dsubsectionx#1{\section*{#1}}
  \def\Dsectionx#1{\chapter*{#1}}

\fi

\newcommand{\MStdPoints}{4}
\newcommand{\MSetPoints}[1]{\renewcommand{\MStdPoints}{#1}}

% Befehl zum Abbruch der Erstellung (nur PDF)
\newcommand{\MAbort}[1]{\err{#1}}

% Prefix vor Dateieinbindungen, wird in der Baumdatei mit \renewcommand modifiziert
% und auf das Verzeichnisprefix gesetzt, in dem das gerade bearbeitete tex-Dokument liegt.
% Im HTML wird es auf das Verzeichnis der HTML-Datei gesetzt.
% Das Prefix muss mit / enden !
\newcommand{\MDPrefix}{.}

% MRegisterFile notiert eine Datei zur Einbindung in den HTML-Baum. Grafiken mit MGraphics werden automatisch eingebunden.
% Mit MLastFile erhaelt man eine Markierung fuer die zuletzt registrierte Datei.
% Diese Markierung wird im postprocessing durch den physikalischen Dateinamen ersetzt, aber nur den Namen (d.h. \MMaterial gehoert noch davor, vgl Definition von MGraphics)
% Parameter: Pfad/Name der Datei bzw. des Ordners, relativ zur Position des Modul-Tex-Dokuments.
\ifttm
\newcommand{\MRegisterFile}[1]{\addtocounter{MFileNumber}{1}\special{html:<!-- registerfile;;}#1\special{html:;;}\MDPrefix\special{html:;;}\arabic{MFileNumber}\special{html:; //-->}}
\else
\newcommand{\MRegisterFile}[1]{\addtocounter{MFileNumber}{1}}
\fi

% Testen welcher Uebersetzer hier am Werk ist

\ifttm
\setcounter{MOutput}{3}
\else
\ifx\pdfoutput\undefined
  \pdffalse
  \setcounter{MOutput}{\MOutputDVI}
  \message{Verarbeitung mit latex, Ausgabe in dvi.}
\else
  \setcounter{MOutput}{\MOutputPDF}
  \message{Verarbeitung mit pdflatex, Ausgabe in pdf.}
  \ifnum \pdfoutput=0
    \pdffalse
  \setcounter{MOutput}{\MOutputDVI}
  \message{Verarbeitung mit pdflatex, Ausgabe in dvi.}
  \else
    \ifnum\pdfoutput=1
    \pdftrue
  \setcounter{MOutput}{\MOutputPDF}
  \message{Verarbeitung mit pdflatex, Ausgabe in pdf.}
    \fi
  \fi
\fi
\fi

\ifnum\value{MOutput}=\MOutputPDF
\DeclareGraphicsExtensions{.pdf,.png,.jpg}
\fi

\ifnum\value{MOutput}=\MOutputDVI
\DeclareGraphicsExtensions{.eps,.png,.jpg}
\fi

\ifnum\value{MOutput}=\MOutputHTML
% Wird vom Konverter leider nicht erkannt und daher in split.pm hardcodiert!
\DeclareGraphicsExtensions{.png,.jpg,.gif}
\fi

% Umdefinition der hyperref-Nummerierung im PDF-Modus
\ifttm
\else
\renewcommand{\theHfigure}{\arabic{chapter}.\arabic{section}.\arabic{figure}}
\fi

% Makro, um in der HTML-Ausgabe die zuerst zu oeffnende Datei zu kennzeichnen
\ifttm
\newcommand{\MGlobalStart}{\special{html:<!-- mglobalstarttag -->}}
\else
\newcommand{\MGlobalStart}{}
\fi

% Makro, um bei scormlogin ein pullen des Benutzers bei Aufruf der Seite zu erzwingen (typischerweise auf der Einstiegsseite)
\ifttm
\newcommand{\MPullSite}{\special{html:<!-- pullsite //-->}}
\else
\newcommand{\MPullSite}{}
\fi

% Makro, um in der HTML-Ausgabe die Kapiteluebersicht zu kennzeichnen
\ifttm
\newcommand{\MGlobalChapterTag}{\special{html:<!-- mglobalchaptertag -->}}
\else
\newcommand{\MGlobalChapterTag}{}
\fi

% Makro, um in der HTML-Ausgabe die Konfiguration zu kennzeichnen
\ifttm
\newcommand{\MGlobalConfTag}{\special{html:<!-- mglobalconfigtag -->}}
\else
\newcommand{\MGlobalConfTag}{}
\fi

% Makro, um in der HTML-Ausgabe die Standortbeschreibung zu kennzeichnen
\ifttm
\newcommand{\MGlobalLocationTag}{\special{html:<!-- mgloballocationtag -->}}
\else
\newcommand{\MGlobalLocationTag}{}
\fi

% Makro, um in der HTML-Ausgabe die persoenlichen Daten zu kennzeichnen
\ifttm
\newcommand{\MGlobalDataTag}{\special{html:<!-- mglobaldatatag -->}}
\else
\newcommand{\MGlobalDataTag}{}
\fi

% Makro, um in der HTML-Ausgabe die Suchseite zu kennzeichnen
\ifttm
\newcommand{\MGlobalSearchTag}{\special{html:<!-- mglobalsearchtag -->}}
\else
\newcommand{\MGlobalSearchTag}{}
\fi

% Makro, um in der HTML-Ausgabe die Favoritenseite zu kennzeichnen
\ifttm
\newcommand{\MGlobalFavoTag}{\special{html:<!-- mglobalfavoritestag -->}}
\else
\newcommand{\MGlobalFavoTag}{}
\fi

% Makro, um in der HTML-Ausgabe die Eingangstestseite zu kennzeichnen
\ifttm
\newcommand{\MGlobalSTestTag}{\special{html:<!-- mglobalstesttag -->}}
\else
\newcommand{\MGlobalSTestTag}{}
\fi

% Makro, um in der PDF-Ausgabe ein Wasserzeichen zu definieren
\ifttm
\newcommand{\MWatermarkSettings}{\relax}
\else
\newcommand{\MWatermarkSettings}{%
% \SetWatermarkText{(c) MINT-Kolleg Baden-W�rttemberg 2014}
% \SetWatermarkLightness{0.85}
% \SetWatermarkScale{1.5}
}
\fi

\ifttm
\newcommand{\MBinom}[2]{\left({\begin{array}{c} #1 \\ #2 \end{array}}\right)}
\else
\newcommand{\MBinom}[2]{\binom{#1}{#2}}
\fi

\ifttm
\newcommand{\DeclareMathOperator}[2]{\def#1{\mathrm{#2}}}
\newcommand{\operatorname}[1]{\mathrm{#1}}
\fi

%----------------- Makros fuer die gemischte HTML/PDF-Konvertierung ------------------------------

\newcommand{\MTestName}{\relax} % wird durch Test-Umgebung gesetzt

% Fuer experimentelle Kursinhalte, die im Release-Umsetzungsvorgang eine Fehlermeldung
% produzieren sollen aber sonst normal umgesetzt werden
\newenvironment{MExperimental}{%
}{%
}

% Wird von ttm nicht richtig umgesetzt!!
\newenvironment{MExerciseItems}{%
\renewcommand\theenumi{\alph{enumi}}%
\begin{enumerate}%
}{%
\end{enumerate}%
}


\definecolor{infoshadecolor}{rgb}{0.75,0.75,0.75}
\definecolor{exmpshadecolor}{rgb}{0.875,0.875,0.875}
\definecolor{expeshadecolor}{rgb}{0.95,0.95,0.95}
\definecolor{framecolor}{rgb}{0.2,0.2,0.2}

% Bei PDF-Uebersetzung wird hinter den Start jeder Satz/Info-aehnlichen Umgebung eine leere mbox gesetzt, damit
% fuehrende Listen oder enums nicht den Zeilenumbruch kaputtmachen
%\ifttm
\def\MTB{}
%\else
%\def\MTB{\mbox{}}
%\fi


\ifttm
\newcommand{\MRelates}{\special{html:<mi>&wedgeq;</mi>}}
\else
\def\MRelates{\stackrel{\scriptscriptstyle\wedge}{=}}
\fi

\def\MInch{\text{''}}
\def\Mdd{\textit{''}}

\ifttm
\def\MNL{ \newline }
\newenvironment{MArray}[1]{\begin{array}{#1}}{\end{array}}
\else
\def\MNL{ \\ }
\newenvironment{MArray}[1]{\begin{array}{#1}}{\end{array}}
\fi

\newcommand{\MBox}[1]{$\mathrm{#1}$}
\newcommand{\MMBox}[1]{\mathrm{#1}}


\ifttm%
\newcommand{\Mtfrac}[2]{{\textstyle \frac{#1}{#2}}}
\newcommand{\Mdfrac}[2]{{\displaystyle \frac{#1}{#2}}}
\newcommand{\Mmeasuredangle}{\special{html:<mi>&angmsd;</mi>}}
\else%
\newcommand{\Mtfrac}[2]{\tfrac{#1}{#2}}
\newcommand{\Mdfrac}[2]{\dfrac{#1}{#2}}
\newcommand{\Mmeasuredangle}{\measuredangle}
\relax
\fi

% Matrizen und Vektoren

% Inhalt wird in der Form a & b \\ c & d erwartet
% Vorsicht: MVector = Komponentenspalte, MVec = Variablensymbol
\ifttm%
\newcommand{\MVector}[1]{\left({\begin{array}{c}#1\end{array}}\right)}
\else%
\newcommand{\MVector}[1]{\begin{pmatrix}#1\end{pmatrix}}
\fi



\newcommand{\MVec}[1]{\vec{#1}}
\newcommand{\MDVec}[1]{\overrightarrow{#1}}

%----------------- Umgebungen fuer Definitionen und Saetze ----------------------------------------

% Fuegt einen Tabellen-Zeilenumbruch ein im PDF, aber nicht im HTML
\newcommand{\TSkip}{\ifttm \else&\ \\\fi}

\newenvironment{infoshaded}{%
\def\FrameCommand{\fboxsep=\FrameSep \fcolorbox{framecolor}{infoshadecolor}}%
\MakeFramed {\advance\hsize-\width \FrameRestore}}%
{\endMakeFramed}

\newenvironment{expeshaded}{%
\def\FrameCommand{\fboxsep=\FrameSep \fcolorbox{framecolor}{expeshadecolor}}%
\MakeFramed {\advance\hsize-\width \FrameRestore}}%
{\endMakeFramed}

\newenvironment{exmpshaded}{%
\def\FrameCommand{\fboxsep=\FrameSep \fcolorbox{framecolor}{exmpshadecolor}}%
\MakeFramed {\advance\hsize-\width \FrameRestore}}%
{\endMakeFramed}

\def\STDCOLOR{black}

\ifttm%
\else%
\newtheoremstyle{MSatzStyle}
  {1cm}                   %Space above
  {1cm}                   %Space below
  {\normalfont\itshape}   %Body font
  {}                      %Indent amount (empty = no indent,
                          %\parindent = para indent)
  {\normalfont\bfseries}  %Thm head font
  {}                      %Punctuation after thm head
  {\newline}              %Space after thm head: " " = normal interword
                          %space; \newline = linebreak
  {\thmname{#1}\thmnumber{ #2}\thmnote{ (#3)}}
                          %Thm head spec (can be left empty, meaning
                          %`normal')
                          %
\newtheoremstyle{MDefStyle}
  {1cm}                   %Space above
  {1cm}                   %Space below
  {\normalfont}           %Body font
  {}                      %Indent amount (empty = no indent,
                          %\parindent = para indent)
  {\normalfont\bfseries}  %Thm head font
  {}                      %Punctuation after thm head
  {\newline}              %Space after thm head: " " = normal interword
                          %space; \newline = linebreak
  {\thmname{#1}\thmnumber{ #2}\thmnote{ (#3)}}
                          %Thm head spec (can be left empty, meaning
                          %`normal')
\fi%

\newcommand{\MInfoText}{Info}

\newcounter{MHintCounter}
\newcounter{MCodeEditCounter}

\newcounter{MLastIndex}  % Enthaelt die dritte Stelle (Indexnummer) des letzten angelegten Objekts
\newcounter{MLastType}   % Enthaelt den Typ des letzten angelegten Objekts (mithilfe der unten definierten Konstanten). Die Entscheidung, wie der Typ dargstellt wird, wird in split.pm beim Postprocessing getroffen.
\newcounter{MLastTypeEq} % =1 falls das Label in einer Matheumgebung (equation, eqnarray usw.) steht, =2 falls das Label in einer table-Umgebung steht

% Da ttm keine Zahlmakros verarbeiten kann, werden diese Nummern in den Zuweisungen hardcodiert!
\def\MTypeSection{1}          %# Zaehler ist section
\def\MTypeSubsection{2}       %# Zaehler ist subsection
\def\MTypeSubsubsection{3}    %# Zaehler ist subsubsection
\def\MTypeInfo{4}             %# Eine Infobox, Separatzaehler fuer die Chemie (auch wenn es dort nicht nummeriert wird) ist MInfoCounter
\def\MTypeExercise{5}         %# Eine Aufgabe, Separatzaehler fuer die Chemie ist MExerciseCounter
\def\MTypeExample{6}          %# Eine Beispielbox, Separatzaehler fuer die Chemie ist MExampleCounter
\def\MTypeExperiment{7}       %# Eine Versuchsbox, Separatzaehler fuer die Chemie ist MExperimentCounter
\def\MTypeGraphics{8}         %# Eine Graphik, Separatzaehler fuer alle FB ist MGraphicsCounter
\def\MTypeTable{9}            %# Eine Tabellennummer, hat keinen Zaehler da durch table gezaehlt wird
\def\MTypeEquation{10}        %# Eine Gleichungsnummer, hat keinen Zaehler da durch equation/eqnarray gezaehlt wird
\def\MTypeTheorem{11}         % Ein theorem oder xtheorem, Separatzaehler fuer die Chemie ist MTheoremCounter
\def\MTypeVideo{12}           %# Ein Video,Separatzaehler fuer alle FB ist MVideoCounter
\def\MTypeEntry{13}           %# Ein Eintrag fuer die Stichwortliste, wird nicht gezaehlt sondern erhaelt im preparsing ein unique-label 

% Zaehler fuer das Labelsystem sind prefixcounter, jeder Zaehler wird VOR dem gezaehlten Objekt inkrementiert und zaehlt daher das aktuelle Objekt
\newcounter{MInfoCounter}
\newcounter{MExerciseCounter}
\newcounter{MExampleCounter}
\newcounter{MExperimentCounter}
\newcounter{MGraphicsCounter}
\newcounter{MTableCounter}
\newcounter{MEquationCounter}  % Nur im HTML, sonst durch "equation"-counter von latex realisiert
\newcounter{MTheoremCounter}
\newcounter{MObjectCounter}   % Gemeinsamer Zaehler fuer Objekte (ausser Grafiken/Tabellen) in Mathe/Info/Physik
\newcounter{MVideoCounter}
\newcounter{MEntryCounter}

\newcounter{MTestSite} % 1 = Subsubsection ist eine Pruefungsseite, 0 = ist eine normale Seite (inkl. Hilfeseite)

\def\MCell{$\phantom{a}$}

\newenvironment{MExportExercise}{\begin{MExercise}}{\end{MExercise}} % wird von mconvert abgefangen

\def\MGenerateExNumber{%
\ifnum\value{MSepNumbers}=0%
\arabic{section}.\arabic{subsection}.\arabic{MObjectCounter}\setcounter{MLastIndex}{\value{MObjectCounter}}%
\else%
\arabic{section}.\arabic{subsection}.\arabic{MExerciseCounter}\setcounter{MLastIndex}{\value{MExerciseCounter}}%
\fi%
}%

\def\MGenerateExmpNumber{%
\ifnum\value{MSepNumbers}=0%
\arabic{section}.\arabic{subsection}.\arabic{MObjectCounter}\setcounter{MLastIndex}{\value{MObjectCounter}}%
\else%
\arabic{section}.\arabic{subsection}.\arabic{MExerciseCounter}\setcounter{MLastIndex}{\value{MExampleCounter}}%
\fi%
}%

\def\MGenerateInfoNumber{%
\ifnum\value{MSepNumbers}=0%
\arabic{section}.\arabic{subsection}.\arabic{MObjectCounter}\setcounter{MLastIndex}{\value{MObjectCounter}}%
\else%
\arabic{section}.\arabic{subsection}.\arabic{MExerciseCounter}\setcounter{MLastIndex}{\value{MInfoCounter}}%
\fi%
}%

\def\MGenerateSiteNumber{%
\arabic{section}.\arabic{subsection}.\arabic{subsubsection}%
}%

% Funktionalitaet fuer Auswahlaufgaben

\newcounter{MExerciseCollectionCounter} % = 0 falls nicht in collection-Umgebung, ansonsten Schachtelungstiefe
\newcounter{MExerciseCollectionTextCounter} % wird von MExercise-Umgebung inkrementiert und von MExerciseCollection-Umgebung auf Null gesetzt

\ifttm
% MExerciseCollection gruppiert Aufgaben, die dynamisch aus der Datenbank gezogen werden und nicht direkt in der HTML-Seite stehen
% Parameter: #1 = ID der Collection, muss eindeutig fuer alle IN DER DB VORHANDENEN collections sein unabhaengig vom Kurs
%            #2 = Optionsargument (im Moment: 1 = Iterative Auswahl, 2 = Zufallsbasierte Auswahl)
\newenvironment{MExerciseCollection}[2]{%
\addtocounter{MExerciseCollectionCounter}{1}
\setcounter{MExerciseCollectionTextCounter}{0}
\special{html:<!-- mexercisecollectionstart;;}#1\special{html:;;}#2\special{html:;; //-->}%
}{%
\special{html:<!-- mexercisecollectionstop //-->}%
\addtocounter{MExerciseCollectionCounter}{-1}
}
\else
\newenvironment{MExerciseCollection}[2]{%
\addtocounter{MExerciseCollectionCounter}{1}
\setcounter{MExerciseCollectionTextCounter}{0}
}{%
\addtocounter{MExerciseCollectionCounter}{-1}
}
\fi

% Bei Uebersetzung nach PDF werden die theorem-Umgebungen verwendet, bei Uebersetzung in HTML ein manuelles Makro
\ifttm%

  \newenvironment{MHint}[1]{  \special{html:<button name="Name_MHint}\arabic{MHintCounter}\special{html:" class="hintbutton_closed" id="MHint}\arabic{MHintCounter}\special{html:_button" %
  type="button" onclick="toggle_hint('MHint}\arabic{MHintCounter}\special{html:');">}#1\special{html:</button>}
  \special{html:<div class="hint" style="display:none" id="MHint}\arabic{MHintCounter}\special{html:"> }}{\begin{html}</div>\end{html}\addtocounter{MHintCounter}{1}}

  \newenvironment{MCOSHZusatz}{  \special{html:<button name="Name_MHint}\arabic{MHintCounter}\special{html:" class="chintbutton_closed" id="MHint}\arabic{MHintCounter}\special{html:_button" %
  type="button" onclick="toggle_hint('MHint}\arabic{MHintCounter}\special{html:');">}Weiterf�hrende Inhalte\special{html:</button>}
  \special{html:<div class="hintc" style="display:none" id="MHint}\arabic{MHintCounter}\special{html:">
  <div class="coshwarn">Diese Inhalte gehen �ber das Kursniveau hinaus und werden in den Aufgaben und Tests nicht abgefragt.</div><br />}
  \addtocounter{MHintCounter}{1}}{\begin{html}</div>\end{html}}

  
  \newenvironment{MDefinition}{\begin{definition}\setcounter{MLastIndex}{\value{definition}}\ \\}{\end{definition}}

  
  \newenvironment{MExercise}{
  \renewcommand{\MStdPoints}{4}
  \addtocounter{MExerciseCounter}{1}
  \addtocounter{MObjectCounter}{1}
  \setcounter{MLastType}{5}

  \ifnum\value{MExerciseCollectionCounter}=0\else\addtocounter{MExerciseCollectionTextCounter}{1}\special{html:<!-- mexercisetextstart;;}\arabic{MExerciseCollectionTextCounter}\special{html:;; //-->}\fi
  \special{html:<div class="aufgabe" id="ADIV_}\MGenerateExNumber\special{html:">}%
  \textbf{Aufgabe \MGenerateExNumber
  } \ \\}{
  \special{html:</div><!-- mfeedbackbutton;Aufgabe;}\arabic{MTestSite}\special{html:;}\MGenerateExNumber\special{html:; //-->}
  \ifnum\value{MExerciseCollectionCounter}=0\else\special{html:<!-- mexercisetextstop //-->}\fi
  }

  % Stellt eine Kombination aus Aufgabe, Loesungstext und Eingabefeld bereit,
  % bei der Aufgabentext und Musterloesung sowie die zugehoerigen Feldelemente
  % extern bezogen und div-aktualisiert werden, das Eingabefeld aber immer das gleiche ist.
  \newenvironment{MFetchExercise}{
  \addtocounter{MExerciseCounter}{1}
  \addtocounter{MObjectCounter}{1}
  \setcounter{MLastType}{5}

  \special{html:<div class="aufgabe" id="ADIV_}\MGenerateExNumber\special{html:">}%
  \textbf{Aufgabe \MGenerateExNumber
  } \ \\%
  \special{html:</div><div class="exfetch_text" id="ADIVTEXT_}\MGenerateExNumber\special{html:">}%
  \special{html:</div><div class="exfetch_sol" id="ADIVSOL_}\MGenerateExNumber\special{html:">}%
  \special{html:</div><div class="exfetch_input" id="ADIVINPUT_}\MGenerateExNumber\special{html:">}%
  }{
  \special{html:</div>}
  }

  \newenvironment{MExample}{
  \addtocounter{MExampleCounter}{1}
  \addtocounter{MObjectCounter}{1}
  \setcounter{MLastType}{6}
  \begin{html}
  <div class="exmp">
  <div class="exmprahmen">
  \end{html}\textbf{Beispiel
  \ifnum\value{MSepNumbers}=0
  \arabic{section}.\arabic{subsection}.\arabic{MObjectCounter}\setcounter{MLastIndex}{\value{MObjectCounter}}
  \else
  \arabic{section}.\arabic{subsection}.\arabic{MExampleCounter}\setcounter{MLastIndex}{\value{MExampleCounter}}
  \fi
  } \ \\}{\begin{html}</div>
  </div>
  \end{html}
  \special{html:<!-- mfeedbackbutton;Beispiel;}\arabic{MTestSite}\special{html:;}\MGenerateExmpNumber\special{html:; //-->}
  }

  \newenvironment{MExperiment}{
  \addtocounter{MExperimentCounter}{1}
  \addtocounter{MObjectCounter}{1}
  \setcounter{MLastType}{7}
  \begin{html}
  <div class="expe">
  <div class="experahmen">
  \end{html}\textbf{Versuch
  \ifnum\value{MSepNumbers}=0
  \arabic{section}.\arabic{subsection}.\arabic{MObjectCounter}\setcounter{MLastIndex}{\value{MObjectCounter}}
  \else
%  \arabic{MExperimentCounter}\setcounter{MLastIndex}{\value{MExperimentCounter}}
  \arabic{section}.\arabic{subsection}.\arabic{MExperimentCounter}\setcounter{MLastIndex}{\value{MExperimentCounter}}
  \fi
  } \ \\}{\begin{html}</div>
  </div>
  \end{html}}

  \newenvironment{MChemInfo}{
  \setcounter{MLastType}{4}
  \begin{html}
  <div class="info">
  <div class="inforahmen">
  \end{html}}{\begin{html}</div>
  </div>
  \end{html}}

  \newenvironment{MXInfo}[1]{
  \addtocounter{MInfoCounter}{1}
  \addtocounter{MObjectCounter}{1}
  \setcounter{MLastType}{4}
  \begin{html}
  <div class="info">
  <div class="inforahmen">
  \end{html}\textbf{#1
  \ifnum\value{MInfoNumbers}=0
  \else
    \ifnum\value{MSepNumbers}=0
    \arabic{section}.\arabic{subsection}.\arabic{MObjectCounter}\setcounter{MLastIndex}{\value{MObjectCounter}}
    \else
    \arabic{MInfoCounter}\setcounter{MLastIndex}{\value{MInfoCounter}}
    \fi
  \fi
  } \ \\}{\begin{html}</div>
  </div>
  \end{html}
  \special{html:<!-- mfeedbackbutton;Info;}\arabic{MTestSite}\special{html:;}\MGenerateInfoNumber\special{html:; //-->}
  }

  \newenvironment{MInfo}{\ifnum\value{MInfoNumbers}=0\begin{MChemInfo}\else\begin{MXInfo}{Info}\ \\ \fi}{\ifnum\value{MInfoNumbers}=0\end{MChemInfo}\else\end{MXInfo}\fi}

\else%

  \theoremstyle{MSatzStyle}
  \newtheorem{thm}{Satz}[section]
  \newtheorem{thmc}{Satz}
  \theoremstyle{MDefStyle}
  \newtheorem{defn}[thm]{Definition}
  \newtheorem{exmp}[thm]{Beispiel}
  \newtheorem{info}[thm]{\MInfoText}
  \theoremstyle{MDefStyle}
  \newtheorem{defnc}{Definition}
  \theoremstyle{MDefStyle}
  \newtheorem{exmpc}{Beispiel}[section]
  \theoremstyle{MDefStyle}
  \newtheorem{infoc}{\MInfoText}
  \theoremstyle{MDefStyle}
  \newtheorem{exrc}{Aufgabe}[section]
  \theoremstyle{MDefStyle}
  \newtheorem{verc}{Versuch}[section]
  
  \newenvironment{MFetchExercise}{}{} % kann im PDF nicht dargestellt werden
  
  \newenvironment{MExercise}{\begin{exrc}\renewcommand{\MStdPoints}{1}\MTB}{\end{exrc}}
  \newenvironment{MHint}[1]{\ \\ \underline{#1:}\\}{}
  \newenvironment{MCOSHZusatz}{\ \\ \underline{Weiterf�hrende Inhalte:}\\}{}
  \newenvironment{MDefinition}{\ifnum\value{MInfoNumbers}=0\begin{defnc}\else\begin{defn}\fi\MTB}{\ifnum\value{MInfoNumbers}=0\end{defnc}\else\end{defn}\fi}
%  \newenvironment{MExample}{\begin{exmp}}{\ \linebreak[1] \ \ \ \ $\phantom{a}$ \ \hfill $\blacklozenge$\end{exmp}}
  \newenvironment{MExample}{
    \ifnum\value{MInfoNumbers}=0\begin{exmpc}\else\begin{exmp}\fi
    \MTB
    \begin{exmpshaded}
    \ \newline
}{
    \end{exmpshaded}
    \ifnum\value{MInfoNumbers}=0\end{exmpc}\else\end{exmp}\fi
}
  \newenvironment{MChemInfo}{\begin{infoshaded}}{\end{infoshaded}}

  \newenvironment{MInfo}{\ifnum\value{MInfoNumbers}=0\begin{MChemInfo}\else\renewcommand{\MInfoText}{Info}\begin{info}\begin{infoshaded}
  \MTB
   \ \newline
    \fi
  }{\ifnum\value{MInfoNumbers}=0\end{MChemInfo}\else\end{infoshaded}\end{info}\fi}

  \newenvironment{MXInfo}[1]{
    \renewcommand{\MInfoText}{#1}
    \ifnum\value{MInfoNumbers}=0\begin{infoc}\else\begin{info}\fi%
    \MTB
    \begin{infoshaded}
    \ \newline
  }{\end{infoshaded}\ifnum\value{MInfoNumbers}=0\end{infoc}\else\end{info}\fi}

  \newenvironment{MExperiment}{
    \renewcommand{\MInfoText}{Versuch}
    \ifnum\value{MInfoNumbers}=0\begin{verc}\else\begin{info}\fi
    \MTB
    \begin{expeshaded}
    \ \newline
  }{
    \end{expeshaded}
    \ifnum\value{MInfoNumbers}=0\end{verc}\else\end{info}\fi
  }
\fi%

% MHint sollte nicht direkt fuer Loesungen benutzt werden wegen solutionselect
\newenvironment{MSolution}{\begin{MHint}{L"osung}}{\end{MHint}}

\newcounter{MCodeCounter}

\ifttm
\newenvironment{MCode}{\special{html:<!-- mcodestart -->}\ttfamily\color{blue}}{\special{html:<!-- mcodestop -->}}
\else
\newenvironment{MCode}{\begin{flushleft}\ttfamily\addtocounter{MCodeCounter}{1}}{\addtocounter{MCodeCounter}{-1}\end{flushleft}}
% Ohne color-Statement da inkompatible mit framed/shaded-Boxen aus dem framed-package
\fi

%----------------- Sonderdefinitionen fuer Symbole, die der Konverter nicht kann ----------------------------------------------

\ifttm%
\newcommand{\MUnderset}[2]{\underbrace{#2}_{#1}}%
\else%
\newcommand{\MUnderset}[2]{\underset{#1}{#2}}%
\fi%

\ifttm
\newcommand{\MThinspace}{\special{html:<mi>&#x2009;</mi>}}
\else
\newcommand{\MThinspace}{\,}
\fi

\ifttm
\newcommand{\glq}{\begin{html}&sbquo;\end{html}}
\newcommand{\grq}{\begin{html}&lsquo;\end{html}}
\newcommand{\glqq}{\begin{html}&bdquo;\end{html}}
\newcommand{\grqq}{\begin{html}&ldquo;\end{html}}
\fi

\ifttm
\newcommand{\MNdash}{\begin{html}&ndash;\end{html}}
\else
\newcommand{\MNdash}{--}
\fi

%\ifttm\def\MIU{\special{html:<mi>&#8520;</mi>}}\else\def\MIU{\mathrm{i}}\fi
\def\MIU{\mathrm{i}}
\def\MEU{e} % TU9-Onlinekurs: italic-e
%\def\MEU{\mathrm{e}} % Alte Onlinemodule: roman-e
\def\MD{d} % Kursives d in Integralen im TU9-Onlinekurs
%\def\MD{\mathrm{d}} % roman-d in den alten Onlinemodulen
\def\MDB{\|}

%zusaetzlicher Leerraum vor "\MD"
\ifttm%
\def\MDSpace{\special{html:<mi>&#x2009;</mi>}}
\else%
\def\MDSpace{\,}
\fi%
\newcommand{\MDwSp}{\MDSpace\MD}%

\ifttm
\def\Mdq{\dq}
\else
\def\Mdq{\dq}
\fi

\def\MSpan#1{\left<{#1}\right>}
\def\MSetminus{\setminus}
\def\MIM{I}

\ifttm
\newcommand{\ld}{\text{ld}}
\newcommand{\lg}{\text{lg}}
\else
\DeclareMathOperator{\ld}{ld}
%\newcommand{\lg}{\text{lg}} % in latex schon definiert
\fi


\def\Mmapsto{\ifttm\special{html:<mi>&mapsto;</mi>}\else\mapsto\fi} 
\def\Mvarphi{\ifttm\phi\else\varphi\fi}
\def\Mphi{\ifttm\varphi\else\phi\fi}
\ifttm%
\newcommand{\MEumu}{\special{html:<mi>&#x3BC;</mi>}}%
\else%
\newcommand{\MEumu}{\textrm{\textmu}}%
\fi
\def\Mvarepsilon{\ifttm\epsilon\else\varepsilon\fi}
\def\Mepsilon{\ifttm\varepsilon\else\epsilon\fi}
\def\Mvarkappa{\ifttm\kappa\else\varkappa\fi}
\def\Mkappa{\ifttm\varkappa\else\kappa\fi}
\def\Mcomplement{\ifttm\special{html:<mi>&comp;</mi>}\else\complement\fi} 
\def\MWW{\mathrm{WW}}
\def\Mmod{\ifttm\special{html:<mi>&nbsp;mod&nbsp;</mi>}\else\mod\fi} 

\ifttm%
\def\mod{\text{\;mod\;}}%
\def\MNEquiv{\special{html:<mi>&NotCongruent;</mi>}}% 
\def\MNSubseteq{\special{html:<mi>&NotSubsetEqual;</mi>}}%
\def\MEmptyset{\special{html:<mi>&empty;</mi>}}%
\def\MVDots{\special{html:<mi>&#x22EE;</mi>}}%
\def\MHDots{\special{html:<mi>&#x2026;</mi>}}%
\def\Mddag{\special{html:<mi>&#x1202;</mi>}}%
\def\sphericalangle{\special{html:<mi>&measuredangle;</mi>}}%
\def\nparallel{\special{html:<mi>&nparallel;</mi>}}%
\def\MProofEnd{\special{html:<mi>&#x25FB;</mi>}}%
\newenvironment{MProof}[1]{\underline{#1}:\MCR\MCR}{\hfill $\MProofEnd$}%
\else%
\def\MNEquiv{\not\equiv}%
\def\MNSubseteq{\not\subseteq}%
\def\MEmptyset{\emptyset}%
\def\MVDots{\vdots}%
\def\MHDots{\hdots}%
\def\Mddag{\ddag}%
\newenvironment{MProof}[1]{\begin{proof}[#1]}{\end{proof}}%
\fi%



% Spaces zum Auffuellen von Tabellenbreiten, die nur im HTML wirken
\ifttm%
\def\MTSP{\:}%
\else%
\def\MTSP{}%
\fi%

\DeclareMathOperator{\arsinh}{arsinh}
\DeclareMathOperator{\arcosh}{arcosh}
\DeclareMathOperator{\artanh}{artanh}
\DeclareMathOperator{\arcoth}{arcoth}


\newcommand{\MMathSet}[1]{\mathbb{#1}}
\def\N{\MMathSet{N}}
\def\Z{\MMathSet{Z}}
\def\Q{\MMathSet{Q}}
\def\R{\MMathSet{R}}
\def\C{\MMathSet{C}}

\newcounter{MForLoopCounter}
\newcommand{\MForLoop}[2]{\setcounter{MForLoopCounter}{#1}\ifnum\value{MForLoopCounter}=0{}\else{{#2}\addtocounter{MForLoopCounter}{-1}\MForLoop{\value{MForLoopCounter}}{#2}}\fi}

\newcounter{MSiteCounter}
\newcounter{MFieldCounter} % Kombination section.subsection.site.field ist eindeutig in allen Modulen, field alleine nicht

\newcounter{MiniMarkerCounter}

\ifttm
\newenvironment{MMiniPageP}[1]{\begin{minipage}{#1\linewidth}\special{html:<!-- minimarker;;}\arabic{MiniMarkerCounter}\special{html:;;#1; //-->}}{\end{minipage}\addtocounter{MiniMarkerCounter}{1}}
\else
\newenvironment{MMiniPageP}[1]{\begin{minipage}{#1\linewidth}}{\end{minipage}\addtocounter{MiniMarkerCounter}{1}}
\fi

\newcounter{AlignCounter}

\newcommand{\MStartJustify}{\ifttm\special{html:<!-- startalign;;}\arabic{AlignCounter}\special{html:;;justify; //-->}\fi}
\newcommand{\MStopJustify}{\ifttm\special{html:<!-- stopalign;;}\arabic{AlignCounter}\special{html:; //-->}\fi\addtocounter{AlignCounter}{1}}

\newenvironment{MJTabular}[1]{
\MStartJustify
\begin{tabular}{#1}
}{
\end{tabular}
\MStopJustify
}

\newcommand{\MImageLeft}[2]{
\begin{center}
\begin{tabular}{lc}
\MStartJustify
\begin{MMiniPageP}{0.65}
#1
\end{MMiniPageP}
\MStopJustify
&
\begin{MMiniPageP}{0.3}
#2  
\end{MMiniPageP}
\end{tabular}
\end{center}
}

\newcommand{\MImageHalf}[2]{
\begin{center}
\begin{tabular}{lc}
\MStartJustify
\begin{MMiniPageP}{0.45}
#1
\end{MMiniPageP}
\MStopJustify
&
\begin{MMiniPageP}{0.45}
#2  
\end{MMiniPageP}
\end{tabular}
\end{center}
}

\newcommand{\MBigImageLeft}[2]{
\begin{center}
\begin{tabular}{lc}
\MStartJustify
\begin{MMiniPageP}{0.25}
#1
\end{MMiniPageP}
\MStopJustify
&
\begin{MMiniPageP}{0.7}
#2  
\end{MMiniPageP}
\end{tabular}
\end{center}
}

\ifttm
\def\No{\mathbb{N}_0}
\else
\def\No{\ensuremath{\N_0}}
\fi
\def\MT{\textrm{\tiny T}}
\newcommand{\MTranspose}[1]{{#1}^{\MT}}
\ifttm
\newcommand{\MRe}{\mathsf{Re}}
\newcommand{\MIm}{\mathsf{Im}}
\else
\DeclareMathOperator{\MRe}{Re}
\DeclareMathOperator{\MIm}{Im}
\fi

\newcommand{\Mid}{\mathrm{id}}
\newcommand{\MFeinheit}{\mathrm{feinh}}

\ifttm
\newcommand{\Msubstack}[1]{\begin{array}{c}{#1}\end{array}}
\else
\newcommand{\Msubstack}[1]{\substack{#1}}
\fi

% Typen von Fragefeldern:
% 1 = Alphanumerisch, case-sensitive-Vergleich
% 2 = Ja/Nein-Checkbox, Loesung ist 0 oder 1   (OPTION = Image-id fuer Rueckmeldung)
% 3 = Reelle Zahlen Geparset
% 4 = Funktionen Geparset (mit Stuetzstellen zur ueberpruefung)

% Dieser Befehl erstellt ein interaktives Aufgabenfeld. Parameter:
% - #1 Laenge in Zeichen
% - #2 Loesungstext (alphanumerisch, case sensitive)
% - #3 AufgabenID (alphanumerisch, case sensitive)
% - #4 Typ (Kennnummer)
% - #5 String fuer Optionen (ggf. mit Semikolon getrennte Einzelstrings)
% - #6 Anzahl Punkte
% - #7 uxid (kann z.B. Loesungsstring sein)
% ACHTUNG: Die langen Zeilen bitte so lassen, Zeilenumbrueche im tex werden in div's umgesetzt
\newcommand{\MQuestionID}[7]{
\ifttm
\special{html:<!-- mdeclareuxid;;}UX#7\special{html:;;}\arabic{section}\special{html:;;}#3\special{html:;; //-->}%
\special{html:<!-- mdeclarepoints;;}\arabic{section}\special{html:;;}#3\special{html:;;}#6\special{html:;;}\arabic{MTestSite}\special{html:;;}\arabic{chapter}%
\special{html:;; //--><!-- onloadstart //-->CreateQuestionObj("}#7\special{html:",}\arabic{MFieldCounter}\special{html:,"}#2%
\special{html:","}#3\special{html:",}#4\special{html:,"}#5\special{html:",}#6\special{html:,}\arabic{MTestSite}\special{html:,}\arabic{section}%
\special{html:);<!-- onloadstop //-->}%
\special{html:<input mfieldtype="}#4\special{html:" name="Name_}#3\special{html:" id="}#3\special{html:" type="text" size="}#1\special{html:" maxlength="}#1%
\special{html:" }\ifnum\value{MGroupActive}=0\special{html:onfocus="handlerFocus(}\arabic{MFieldCounter}%
\special{html:);" onblur="handlerBlur(}\arabic{MFieldCounter}\special{html:);" onkeyup="handlerChange(}\arabic{MFieldCounter}\special{html:,0);" onpaste="handlerChange(}\arabic{MFieldCounter}\special{html:,0);" oninput="handlerChange(}\arabic{MFieldCounter}\special{html:,0);" onpropertychange="handlerChange(}\arabic{MFieldCounter}\special{html:,0);"/>}%
\special{html:<img src="images/questionmark.gif" width="20" height="20" border="0" align="absmiddle" id="}QM#3\special{html:"/>}
\else%
\special{html:onblur="handlerBlur(}\arabic{MFieldCounter}%
\special{html:);" onfocus="handlerFocus(}\arabic{MFieldCounter}\special{html:);" onkeyup="handlerChange(}\arabic{MFieldCounter}\special{html:,1);" onpaste="handlerChange(}\arabic{MFieldCounter}\special{html:,1);" oninput="handlerChange(}\arabic{MFieldCounter}\special{html:,1);" onpropertychange="handlerChange(}\arabic{MFieldCounter}\special{html:,1);"/>}%
\special{html:<img src="images/questionmark.gif" width="20" height="20" border="0" align="absmiddle" id="}QM#3\special{html:"/>}\fi%
\else%
\ifnum\value{QBoxFlag}=1\fbox{$\phantom{\MForLoop{#1}{b}}$}\else$\phantom{\MForLoop{#1}{b}}$\fi%
\fi%
}

% ACHTUNG: Die langen Zeilen bitte so lassen, Zeilenumbrueche im tex werden in div's umgesetzt
% QuestionCheckbox macht ausserhalb einer QuestionGroup keinen Sinn!
% #1 = solution (1 oder 0), ggf. mit ::smc abgetrennt auszuschliessende single-choice-boxen (UXIDs durch , getrennt), #2 = id, #3 = points, #4 = uxid
\newcommand{\MQuestionCheckbox}[4]{
\ifttm
\special{html:<!-- mdeclareuxid;;}UX#4\special{html:;;}\arabic{section}\special{html:;;}#2\special{html:;; //-->}%
\ifnum\value{MGroupActive}=0\MDebugMessage{ERROR: Checkbox Nr. \arabic{MFieldCounter}\ ist nicht in einer Kontrollgruppe, es wird niemals eine Loesung angezeigt!}\fi
\special{html: %
<!-- mdeclarepoints;;}\arabic{section}\special{html:;;}#2\special{html:;;}#3\special{html:;;}\arabic{MTestSite}\special{html:;;}\arabic{chapter}%
\special{html:;; //--><!-- onloadstart //-->CreateQuestionObj("}#4\special{html:",}\arabic{MFieldCounter}\special{html:,"}#1\special{html:","}#2\special{html:",2,"IMG}#2%
\special{html:",}#3\special{html:,}\arabic{MTestSite}\special{html:,}\arabic{section}\special{html:);<!-- onloadstop //-->}%
\special{html:<input mfieldtype="2" type="checkbox" name="Name_}#2\special{html:" id="}#2\special{html:" onchange="handlerChange(}\arabic{MFieldCounter}\special{html:,1);"/><img src="images/questionmark.gif" name="}Name_IMG#2%
\special{html:" width="20" height="20" border="0" align="absmiddle" id="}IMG#2\special{html:"/> }%
\else%
\ifnum\value{QBoxFlag}=1\fbox{$\phantom{X}$}\else$\phantom{X}$\fi%
\fi%
}

\def\MGenerateID{QFELD_\arabic{section}.\arabic{subsection}.\arabic{MSiteCounter}.QF\arabic{MFieldCounter}}

% #1 = 0/1 ggf. mit ::smc abgetrennt auszuschliessende single-choice-boxen (UXIDs durch , getrennt ohne UX), #2 = uxid ohne UX
\newcommand{\MCheckbox}[2]{
\MQuestionCheckbox{#1}{\MGenerateID}{\MStdPoints}{#2}
\addtocounter{MFieldCounter}{1}
}

% Erster Parameter: Zeichenlaenge der Eingabebox, zweiter Parameter: Loesungstext
\newcommand{\MQuestion}[2]{
\MQuestionID{#1}{#2}{\MGenerateID}{1}{0}{\MStdPoints}{#2}
\addtocounter{MFieldCounter}{1}
}

% Erster Parameter: Zeichenlaenge der Eingabebox, zweiter Parameter: Loesungstext
\newcommand{\MLQuestion}[3]{
\MQuestionID{#1}{#2}{\MGenerateID}{1}{0}{\MStdPoints}{#3}
\addtocounter{MFieldCounter}{1}
}

% Parameter: Laenge des Feldes, Loesung (wird auch geparsed), Stellen Genauigkeit hinter dem Komma, weitere Stellen werden mathematisch gerundet vor Vergleich
\newcommand{\MParsedQuestion}[3]{
\MQuestionID{#1}{#2}{\MGenerateID}{3}{#3}{\MStdPoints}{#2}
\addtocounter{MFieldCounter}{1}
}

% Parameter: Laenge des Feldes, Loesung (wird auch geparsed), Stellen Genauigkeit hinter dem Komma, weitere Stellen werden mathematisch gerundet vor Vergleich
\newcommand{\MLParsedQuestion}[4]{
\MQuestionID{#1}{#2}{\MGenerateID}{3}{#3}{\MStdPoints}{#4}
\addtocounter{MFieldCounter}{1}
}

% Parameter: Laenge des Feldes, Loesungsfunktion, Anzahl Stuetzstellen, Funktionsvariablen durch Kommata getrennt (nicht case-sensitive), Anzahl Nachkommastellen im Vergleich
\newcommand{\MFunctionQuestion}[5]{
\MQuestionID{#1}{#2}{\MGenerateID}{4}{#3;#4;#5;0}{\MStdPoints}{#2}
\addtocounter{MFieldCounter}{1}
}

% Parameter: Laenge des Feldes, Loesungsfunktion, Anzahl Stuetzstellen, Funktionsvariablen durch Kommata getrennt (nicht case-sensitive), Anzahl Nachkommastellen im Vergleich, UXID
\newcommand{\MLFunctionQuestion}[6]{
\MQuestionID{#1}{#2}{\MGenerateID}{4}{#3;#4;#5;0}{\MStdPoints}{#6}
\addtocounter{MFieldCounter}{1}
}

% Parameter: Laenge des Feldes, Loesungsintervall, Genauigkeit der Zahlenwertpruefung
\newcommand{\MIntervalQuestion}[3]{
\MQuestionID{#1}{#2}{\MGenerateID}{6}{#3}{\MStdPoints}{#2}
\addtocounter{MFieldCounter}{1}
}

% Parameter: Laenge des Feldes, Loesungsintervall, Genauigkeit der Zahlenwertpruefung, UXID
\newcommand{\MLIntervalQuestion}[4]{
\MQuestionID{#1}{#2}{\MGenerateID}{6}{#3}{\MStdPoints}{#4}
\addtocounter{MFieldCounter}{1}
}

% Parameter: Laenge des Feldes, Loesungsfunktion, Anzahl Stuetzstellen, Funktionsvariable (nicht case-sensitive), Anzahl Nachkommastellen im Vergleich, Vereinfachungsbedingung
% Vereinfachungsbedingung ist eine der Folgenden:
% 0 = Keine Vereinfachungsbedingung
% 1 = Keine Klammern (runde oder eckige) mehr im vereinfachten Ausdruck
% 2 = Faktordarstellung (Term hat Produkte als letzte Operation, Summen als vorgeschaltete Operation)
% 3 = Summendarstellung (Term hat Summen als letzte Operation, Produkte als vorgeschaltete Operation)
% Flag 512: Besondere Stuetzstellen (nur >1 und nur schwach rational), sonst symmetrisch um Nullpunkt und ganze Zahlen inkl. Null werden getroffen
\newcommand{\MSimplifyQuestion}[6]{
\MQuestionID{#1}{#2}{\MGenerateID}{4}{#3;#4;#5;#6}{\MStdPoints}{#2}
\addtocounter{MFieldCounter}{1}
}

\newcommand{\MLSimplifyQuestion}[7]{
\MQuestionID{#1}{#2}{\MGenerateID}{4}{#3;#4;#5;#6}{\MStdPoints}{#7}
\addtocounter{MFieldCounter}{1}
}

% Parameter: Laenge des Feldes, Loesung (optionaler Ausdruck), Anzahl Stuetzstellen, Funktionsvariable (nicht case-sensitive), Anzahl Nachkommastellen im Vergleich, Spezialtyp (string-id)
\newcommand{\MLSpecialQuestion}[7]{
\MQuestionID{#1}{#2}{\MGenerateID}{7}{#3;#4;#5;#6}{\MStdPoints}{#7}
\addtocounter{MFieldCounter}{1}
}

\newcounter{MGroupStart}
\newcounter{MGroupEnd}
\newcounter{MGroupActive}

\newenvironment{MQuestionGroup}{
\setcounter{MGroupStart}{\value{MFieldCounter}}
\setcounter{MGroupActive}{1}
}{
\setcounter{MGroupActive}{0}
\setcounter{MGroupEnd}{\value{MFieldCounter}}
\addtocounter{MGroupEnd}{-1}
}

\newcommand{\MGroupButton}[1]{
\ifttm
\special{html:<button name="Name_Group}\arabic{MGroupStart}\special{html:to}\arabic{MGroupEnd}\special{html:" id="Group}\arabic{MGroupStart}\special{html:to}\arabic{MGroupEnd}\special{html:" %
type="button" onclick="group_button(}\arabic{MGroupStart}\special{html:,}\arabic{MGroupEnd}\special{html:);">}#1\special{html:</button>}
\else
\phantom{#1}
\fi
}

%----------------- Makros fuer die modularisierte Darstellung ------------------------------------

\def\MyText#1{#1}

% is used internally by the conversion package, should not be used by original tex documents
\def\MOrgLabel#1{\relax}

\ifttm

% Ein MLabel wird im html codiert durch das tag <!-- mmlabel;;Labelbezeichner;;SubjectArea;;chapter;;section;;subsection;;Index;;Objekttyp; //-->
\def\MLabel#1{%
\ifnum\value{MLastType}=8%
\ifnum\value{MCaptionOn}=0%
\MDebugMessage{ERROR: Grafik \arabic{MGraphicsCounter} hat separates label: #1 (Grafiklabels sollten nur in der Caption stehen)}%
\fi
\fi
\ifnum\value{MLastType}=12%
\ifnum\value{MCaptionOn}=0%
\MDebugMessage{ERROR: Video \arabic{MVideoCounter} hat separates label: #1 (Videolabels sollten nur in der Caption stehen}%
\fi
\fi
\ifnum\value{MLastType}=10\setcounter{MLastIndex}{\value{equation}}\fi
\label{#1}\begin{html}<!-- mmlabel;;#1;;\end{html}\arabic{MSubjectArea}\special{html:;;}\arabic{chapter}\special{html:;;}\arabic{section}\special{html:;;}\arabic{subsection}\special{html:;;}\arabic{MLastIndex}\special{html:;;}\arabic{MLastType}\special{html:; //-->}}%

\else

% Sonderbehandlung im PDF fuer Abbildungen in separater aux-Datei, da MGraphics die figure-Umgebung nicht verwendet
\def\MLabel#1{%
\ifnum\value{MLastType}=8%
\ifnum\value{MCaptionOn}=0%
\MDebugMessage{ERROR: Grafik \arabic{MGraphicsCounter} hat separates label: #1 (Grafiklabels sollten nur in der Caption stehen}%
\fi
\fi
\ifnum\value{MLastType}=12%
\ifnum\value{MCaptionOn}=0%
\MDebugMessage{ERROR: Video \arabic{MVideoCounter} hat separates label: #1 (Videolabels sollten nur in der Caption stehen}%
\fi
\fi
\label{#1}%
}%

\fi

% Gibt Begriff des referenzierten Objekts mit aus, aber nur im HTML, daher nur in Ausnahmefaellen (z.B. Copyrightliste) sinnvoll
\def\MCRef#1{\ifttm\special{html:<!-- mmref;;}#1\special{html:;;1; //-->}\else\vref{#1}\fi}


\def\MRef#1{\ifttm\special{html:<!-- mmref;;}#1\special{html:;;0; //-->}\else\vref{#1}\fi}
\def\MERef#1{\ifttm\special{html:<!-- mmref;;}#1\special{html:;;0; //-->}\else\eqref{#1}\fi}
\def\MNRef#1{\ifttm\special{html:<!-- mmref;;}#1\special{html:;;0; //-->}\else\ref{#1}\fi}
\def\MSRef#1#2{\ifttm\special{html:<!-- msref;;}#1\special{html:;;}#2\special{html:; //-->}\else \if#2\empty \ref{#1} \else \hyperref[#1]{#2}\fi\fi} 

\def\MRefRange#1#2{\ifttm\MRef{#1} bis 
\MRef{#2}\else\vrefrange[\unskip]{#1}{#2}\fi}

\def\MRefTwo#1#2{\ifttm\MRef{#1} und \MRef{#2}\else%
\let\vRefTLRsav=\reftextlabelrange\let\vRefTPRsav=\reftextpagerange%
\def\reftextlabelrange##1##2{\ref{##1} und~\ref{##2}}%
\def\reftextpagerange##1##2{auf den Seiten~\pageref{#1} und~\pageref{#2}}%
\vrefrange[\unskip]{#1}{#2}%
\let\reftextlabelrange=\vRefTLRsav\let\reftextpagerange=\vRefTPRsav\fi}

% MSectionChapter definiert falls notwendig das Kapitel vor der section. Das ist notwendig, wenn nur ein Einzelmodul uebersetzt wird.
% MChaptersGiven ist ein Counter, der von mconvert.pl vordefiniert wird.
\ifttm
\newcommand{\MSectionChapter}{\ifnum\value{MChaptersGiven}=0{\Dchapter{Modul}}\else{}\fi}
\else
\newcommand{\MSectionChapter}{\ifnum\value{chapter}=0{\Dchapter{Modul}}\else{}\fi}
\fi


\def\MChapter#1{\ifnum\value{MSSEnd}>0{\MSubsectionEndMacros}\addtocounter{MSSEnd}{-1}\fi\Dchapter{#1}}
\def\MSubject#1{\MChapter{#1}} % Schluesselwort HELPSECTION ist reserviert fuer Hilfesektion

\newcommand{\MSectionID}{UNKNOWNID}

\ifttm
\newcommand{\MSetSectionID}[1]{\renewcommand{\MSectionID}{#1}}
\else
\newcommand{\MSetSectionID}[1]{\renewcommand{\MSectionID}{#1}\tikzsetexternalprefix{#1}}
\fi


\newcommand{\MSection}[1]{\MSetSectionID{MODULID}\ifnum\value{MSSEnd}>0{\MSubsectionEndMacros}\addtocounter{MSSEnd}{-1}\fi\MSectionChapter\Dsection{#1}\MSectionStartMacros{#1}\setcounter{MLastIndex}{-1}\setcounter{MLastType}{1}} % Sections werden ueber das section-Feld im mmlabel-Tag identifiziert, nicht ueber das Indexfeld

\def\MSubsection#1{\ifnum\value{MSSEnd}>0{\MSubsectionEndMacros}\addtocounter{MSSEnd}{-1}\fi\ifttm\else\clearpage\fi\Dsubsection{#1}\MSubsectionStartMacros\setcounter{MLastIndex}{-1}\setcounter{MLastType}{2}\addtocounter{MSSEnd}{1}}% Subsections werden ueber das subsection-Feld im mmlabel-Tag identifiziert, nicht ueber das Indexfeld
\def\MSubsectionx#1{\Dsubsectionx{#1}} % Nur zur Verwendung in MSectionStart gedacht
\def\MSubsubsection#1{\Dsubsubsection{#1}\setcounter{MLastIndex}{\value{subsubsection}}\setcounter{MLastType}{3}\ifttm\special{html:<!-- sectioninfo;;}\arabic{section}\special{html:;;}\arabic{subsection}\special{html:;;}\arabic{subsubsection}\special{html:;;1;;}\arabic{MTestSite}\special{html:; //-->}\fi}
\def\MSubsubsectionx#1{\Dsubsubsectionx{#1}\ifttm\special{html:<!-- sectioninfo;;}\arabic{section}\special{html:;;}\arabic{subsection}\special{html:;;}\arabic{subsubsection}\special{html:;;0;;}\arabic{MTestSite}\special{html:; //-->}\else\addcontentsline{toc}{subsection}{#1}\fi}

\ifttm
\def\MSubsubsubsectionx#1{\ \newline\textbf{#1}\special{html:<br />}}
\else
\def\MSubsubsubsectionx#1{\ \newline
\textbf{#1}\ \\
}
\fi


% Dieses Skript wird zu Beginn jedes Modulabschnitts (=Webseite) ausgefuehrt und initialisiert den Aufgabenfeldzaehler
\newcommand{\MPageScripts}{
\setcounter{MFieldCounter}{1}
\addtocounter{MSiteCounter}{1}
\setcounter{MHintCounter}{1}
\setcounter{MCodeEditCounter}{1}
\setcounter{MGroupActive}{0}
\DoQBoxes
% Feldvariablen werden im HTML-Header in conv.pl eingestellt
}

% Dieses Skript wird zum Ende jedes Modulabschnitts (=Webseite) ausgefuehrt
\ifttm
\newcommand{\MEndScripts}{\special{html:<br /><!-- mfeedbackbutton;Seite;}\arabic{MTestSite}\special{html:;}\MGenerateSiteNumber\special{html:; //-->}
}
\else
\newcommand{\MEndScripts}{\relax}
\fi


\newcounter{QBoxFlag}
\newcommand{\DoQBoxes}{\setcounter{QBoxFlag}{1}}
\newcommand{\NoQBoxes}{\setcounter{QBoxFlag}{0}}

\newcounter{MXCTest}
\newcounter{MXCounter}
\newcounter{MSCounter}



\ifttm

% Struktur des sectioninfo-Tags: <!-- sectioninfo;;section;;subsection;;subsubsection;;nr_ausgeben;;testpage; //-->

%Fuegt eine zusaetzliche html-Seite an hinter ALLEN bisherigen und zukuenftigen content-Seiten ausserhalb der vor-zurueck-Schleife (d.h. nur durch Button oder MIntLink erreichbar!)
% #1 = Titel des Modulabschnitts, #2 = Kurztitel fuer die Buttons, #3 = Buttonkennung (STD = default nehmen, NONE = Ohne Button in der Navigation)
\newenvironment{MSContent}[3]{\special{html:<div class="xcontent}\arabic{MSCounter}\special{html:"><!-- scontent;-;}\arabic{MSCounter};-;#1;-;#2;-;#3\special{html: //-->}\MPageScripts\MSubsubsectionx{#1}}{\MEndScripts\special{html:<!-- endscontent;;}\arabic{MSCounter}\special{html: //--></div>}\addtocounter{MSCounter}{1}}

% Fuegt eine zusaetzliche html-Seite ein hinter den bereits vorhandenen content-Seiten (oder als erste Seite) innerhalb der vor-zurueck-Schleife der Navigation
% #1 = Titel des Modulabschnitts, #2 = Kurztitel fuer die Buttons, #3 = Buttonkennung (STD = Defaultbutton, NONE = Ohne Button in der Navigation)
\newenvironment{MXContent}[3]{\special{html:<div class="xcontent}\arabic{MXCounter}\special{html:"><!-- xcontent;-;}\arabic{MXCounter};-;#1;-;#2;-;#3\special{html: //-->}\MPageScripts\MSubsubsection{#1}}{\MEndScripts\special{html:<!-- endxcontent;;}\arabic{MXCounter}\special{html: //--></div>}\addtocounter{MXCounter}{1}}

% Fuegt eine zusaetzliche html-Seite ein die keine subsubsection-Nummer bekommt, nur zur internen Verwendung in mintmod.tex gedacht!
% #1 = Titel des Modulabschnitts, #2 = Kurztitel fuer die Buttons, #3 = Buttonkennung (STD = Defaultbutton, NONE = Ohne Button in der Navigation)
% \newenvironment{MUContent}[3]{\special{html:<div class="xcontent}\arabic{MXCounter}\special{html:"><!-- xcontent;-;}\arabic{MXCounter};-;#1;-;#2;-;#3\special{html: //-->}\MPageScripts\MSubsubsectionx{#1}}{\MEndScripts\special{html:<!-- endxcontent;;}\arabic{MXCounter}\special{html: //--></div>}\addtocounter{MXCounter}{1}}

\newcommand{\MDeclareSiteUXID}[1]{\special{html:<!-- mdeclaresiteuxid;;}#1\special{html:;;}\arabic{chapter}\special{html:;;}\arabic{section}\special{html:;; //-->}}

\else

%\newcommand{\MSubsubsection}[1]{\refstepcounter{subsubsection} \addcontentsline{toc}{subsubsection}{\thesubsubsection. #1}}


% Fuegt eine zusaetzliche html-Seite an hinter den bereits vorhandenen content-Seiten
% #1 = Titel des Modulabschnitts, #2 = Kurztitel fuer die Buttons, #3 = Iconkennung (im PDF wirkungslos)
%\newenvironment{MUContent}[3]{\ifnum\value{MXCTest}>0{\MDebugMessage{ERROR: Geschachtelter SContent}}\fi\MPageScripts\MSubsubsectionx{#1}\addtocounter{MXCTest}{1}}{\addtocounter{MXCounter}{1}\addtocounter{MXCTest}{-1}}
\newenvironment{MXContent}[3]{\ifnum\value{MXCTest}>0{\MDebugMessage{ERROR: Geschachtelter SContent}}\fi\MPageScripts\MSubsubsection{#1}\addtocounter{MXCTest}{1}}{\addtocounter{MXCounter}{1}\addtocounter{MXCTest}{-1}}
\newenvironment{MSContent}[3]{\ifnum\value{MXCTest}>0{\MDebugMessage{ERROR: Geschachtelter XContent}}\fi\MPageScripts\MSubsubsectionx{#1}\addtocounter{MXCTest}{1}}{\addtocounter{MSCounter}{1}\addtocounter{MXCTest}{-1}}

\newcommand{\MDeclareSiteUXID}[1]{\relax}

\fi 

% GHEADER und GFOOTER werden von split.pm gefunden, aber nur, wenn nicht HELPSITE oder TESTSITE
\ifttm
\newenvironment{MSectionStart}{\special{html:<div class="xcontent0">}\MSubsubsectionx{Modul\"ubersicht}}{\setcounter{MSSEnd}{0}\special{html:</div>}}
% Darf nicht als XContent nummeriert werden, darf nicht als XContent gelabelt werden, wird aber in eine xcontent-div gesetzt fuer Python-parsing
\else
\newenvironment{MSectionStart}{\MSubsectionx{Modul\"ubersicht}}{\setcounter{MSSEnd}{0}}
\fi

\newenvironment{MIntro}{\begin{MXContent}{Einf\"uhrung}{Einf\"uhrung}{genetisch}}{\end{MXContent}}
\newenvironment{MContent}{\begin{MXContent}{Inhalt}{Inhalt}{beweis}}{\end{MXContent}}
\newenvironment{MExercises}{\ifttm\else\clearpage\fi\begin{MXContent}{Aufgaben}{Aufgaben}{aufgb}\special{html:<!-- declareexcsymb //-->}}{\end{MXContent}}

% #1 = Lesbare Testbezeichnung
\newenvironment{MTest}[1]{%
\renewcommand{\MTestName}{#1}
\ifttm\else\clearpage\fi%
\addtocounter{MTestSite}{1}%
\begin{MXContent}{#1}{#1}{STD} % {aufgb}%
\special{html:<!-- declaretestsymb //-->}
\begin{MQuestionGroup}%
\MInTestHeader
}%
{%
\end{MQuestionGroup}%
\ \\ \ \\%
\MInTestFooter
\end{MXContent}\addtocounter{MTestSite}{-1}%
}

\newenvironment{MExtra}{\ifttm\else\clearpage\fi\begin{MXContent}{Zus\"atzliche Inhalte}{Zusatz}{weiterfhrg}}{\end{MXContent}}

\makeindex

\ifttm
\def\MPrintIndex{
\ifnum\value{MSSEnd}>0{\MSubsectionEndMacros}\addtocounter{MSSEnd}{-1}\fi
\renewcommand{\indexname}{Stichwortverzeichnis}
\special{html:<p><!-- printindex //--></p>}
}
\else
\def\MPrintIndex{
\ifnum\value{MSSEnd}>0{\MSubsectionEndMacros}\addtocounter{MSSEnd}{-1}\fi
\renewcommand{\indexname}{Stichwortverzeichnis}
\addcontentsline{toc}{section}{Stichwortverzeichnis}
\printindex
}
\fi


% Konstanten fuer die Modulfaecher

\def\MINTMathematics{1}
\def\MINTInformatics{2}
\def\MINTChemistry{3}
\def\MINTPhysics{4}
\def\MINTEngineering{5}

\newcounter{MSubjectArea}
\newcounter{MInfoNumbers} % Gibt an, ob die Infoboxen nummeriert werden sollen
\newcounter{MSepNumbers} % Gibt an, ob Beispiele und Experimente separat nummeriert werden sollen
\newcommand{\MSetSubject}[1]{
 % ttm kapiert setcounter mit Parametern nicht, also per if abragen und einsetzen
\ifnum#1=1\setcounter{MSubjectArea}{1}\setcounter{MInfoNumbers}{1}\setcounter{MSepNumbers}{0}\fi
\ifnum#1=2\setcounter{MSubjectArea}{2}\setcounter{MInfoNumbers}{1}\setcounter{MSepNumbers}{0}\fi
\ifnum#1=3\setcounter{MSubjectArea}{3}\setcounter{MInfoNumbers}{0}\setcounter{MSepNumbers}{1}\fi
\ifnum#1=4\setcounter{MSubjectArea}{4}\setcounter{MInfoNumbers}{0}\setcounter{MSepNumbers}{0}\fi
\ifnum#1=5\setcounter{MSubjectArea}{5}\setcounter{MInfoNumbers}{1}\setcounter{MSepNumbers}{0}\fi
% Separate Nummerntechnik fuer unsere Chemiker: alles dreistellig
\ifnum#1=3
  \ifttm
  \renewcommand{\theequation}{\arabic{section}.\arabic{subsection}.\arabic{equation}}
  \renewcommand{\thetable}{\arabic{section}.\arabic{subsection}.\arabic{table}} 
  \renewcommand{\thefigure}{\arabic{section}.\arabic{subsection}.\arabic{figure}} 
  \else
  \renewcommand{\theequation}{\arabic{chapter}.\arabic{section}.\arabic{equation}}
  \renewcommand{\thetable}{\arabic{chapter}.\arabic{section}.\arabic{table}}
  \renewcommand{\thefigure}{\arabic{chapter}.\arabic{section}.\arabic{figure}}
  \fi
\else
  \ifttm
  \renewcommand{\theequation}{\arabic{section}.\arabic{subsection}.\arabic{equation}}
  \renewcommand{\thetable}{\arabic{table}}
  \renewcommand{\thefigure}{\arabic{figure}}
  \else
  \renewcommand{\theequation}{\arabic{chapter}.\arabic{section}.\arabic{equation}}
  \renewcommand{\thetable}{\arabic{table}}
  \renewcommand{\thefigure}{\arabic{figure}}
  \fi
\fi
}

% Fuer tikz Autogenerierung
\newcounter{MTIKZAutofilenumber}

% Spezielle Counter fuer die Bentz-Module
\newcounter{mycounter}
\newcounter{chemapplet}
\newcounter{physapplet}

\newcounter{MSSEnd} % Ist 1 falls ein MSubsection aktiv ist, der einen MSubsectionEndMacro-Aufruf verursacht
\newcounter{MFileNumber}
\def\MLastFile{\special{html:[[!-- mfileref;;}\arabic{MFileNumber}\special{html:; //--]]}}

% Vollstaendiger Pfad ist \MMaterial / \MLastFilePath / \MLastFileName    ==   \MMaterial / \MLastFile

% Wird nur bei kompletter Baum-Erstellung ausgefuehrt!
% #1 = Lesbare Modulbezeichnung
\newcommand{\MSectionStartMacros}[1]{
\setcounter{MTestSite}{0}
\setcounter{MCaptionOn}{0}
\setcounter{MLastTypeEq}{0}
\setcounter{MSSEnd}{0}
\setcounter{MFileNumber}{0} % Preinkrekement-Counter
\setcounter{MTIKZAutofilenumber}{0}
\setcounter{mycounter}{1}
\setcounter{physapplet}{1}
\setcounter{chemapplet}{0}
\ifttm
\special{html:<!-- mdeclaresection;;}\arabic{chapter}\special{html:;;}\arabic{section}\special{html:;;}#1\special{html:;; //-->}%
\else
\setcounter{thmc}{0}
\setcounter{exmpc}{0}
\setcounter{verc}{0}
\setcounter{infoc}{0}
\fi
\setcounter{MiniMarkerCounter}{1}
\setcounter{AlignCounter}{1}
\setcounter{MXCTest}{0}
\setcounter{MCodeCounter}{0}
\setcounter{MEntryCounter}{0}
}

% Wird immer ausgefuehrt
\newcommand{\MSubsectionStartMacros}{
\ifttm\else\MPageHeaderDef\fi
\MWatermarkSettings
\setcounter{MXCounter}{0}
\setcounter{MSCounter}{0}
\setcounter{MSiteCounter}{1}
\setcounter{MExerciseCollectionCounter}{0}
% Zaehler fuer das Labelsystem zuruecksetzen (prefix-Zaehler)
\setcounter{MInfoCounter}{0}
\setcounter{MExerciseCounter}{0}
\setcounter{MExampleCounter}{0}
\setcounter{MExperimentCounter}{0}
\setcounter{MGraphicsCounter}{0}
\setcounter{MTableCounter}{0}
\setcounter{MTheoremCounter}{0}
\setcounter{MObjectCounter}{0}
\setcounter{MEquationCounter}{0}
\setcounter{MVideoCounter}{0}
\setcounter{equation}{0}
\setcounter{figure}{0}
}

\newcommand{\MSubsectionEndMacros}{
% Bei Chemiemodulen das PSE einhaengen, es soll als SContent am Ende erscheinen
\special{html:<!-- subsectionend //-->}
\ifnum\value{MSubjectArea}=3{\MIncludePSE}\fi
}


\ifttm
%\newcommand{\MEmbed}[1]{\MRegisterFile{#1}\begin{html}<embed src="\end{html}\MMaterial/\MLastFile\begin{html}" width="192" height="189"></embed>\end{html}}
\newcommand{\MEmbed}[1]{\MRegisterFile{#1}\begin{html}<embed src="\end{html}\MMaterial/\MLastFile\begin{html}"></embed>\end{html}}
\fi

%----------------- Makros fuer die Textdarstellung -----------------------------------------------

\ifttm
% MUGraphics bindet eine Grafik ein:
% Parameter 1: Dateiname der Grafik, relativ zur Position des Modul-Tex-Dokuments
% Parameter 2: Skalierungsoptionen fuer PDF (fuer includegraphics)
% Parameter 3: Titel fuer die Grafik, wird unter die Grafik mit der Grafiknummer gesetzt und kann MLabel bzw. MCopyrightLabel enthalten
% Parameter 4: Skalierungsoptionen fuer HTML (css-styles)

% ERSATZ: <img alt="My Image" src="data:image/png;base64,iVBORwA<MoreBase64SringHere>" />


\newcommand{\MUGraphics}[4]{\MRegisterFile{#1}\begin{html}
<div class="imagecenter">
<center>
<div>
<img src="\end{html}\MMaterial/\MLastFile\begin{html}" style="#4" alt="\end{html}\MMaterial/\MLastFile\begin{html}"/>
</div>
<div class="bildtext">
\end{html}
\addtocounter{MGraphicsCounter}{1}
\setcounter{MLastIndex}{\value{MGraphicsCounter}}
\setcounter{MLastType}{8}
\addtocounter{MCaptionOn}{1}
\ifnum\value{MSepNumbers}=0
\textbf{Abbildung \arabic{MGraphicsCounter}:} #3
\else
\textbf{Abbildung \arabic{section}.\arabic{subsection}.\arabic{MGraphicsCounter}:} #3
\fi
\addtocounter{MCaptionOn}{-1}
\begin{html}
</div>
</center>
</div>
<br />
\end{html}%
\special{html:<!-- mfeedbackbutton;Abbildung;}\arabic{MGraphicsCounter}\special{html:;}\arabic{section}.\arabic{subsection}.\arabic{MGraphicsCounter}\special{html:; //-->}%
}

% MVideo bindet ein Video als Einzeldatei ein:
% Parameter 1: Dateiname des Videos, relativ zur Position des Modul-Tex-Dokuments, ohne die Endung ".mp4"
% Parameter 2: Titel fuer das Video (kann MLabel oder MCopyrightLabel enthalten), wird unter das Video mit der Videonummer gesetzt
\newcommand{\MVideo}[2]{\MRegisterFile{#1.mp4}\begin{html}
<div class="imagecenter">
<center>
<div>
<video width="95\%" controls="controls"><source src="\end{html}\MMaterial/#1.mp4\begin{html}" type="video/mp4">Ihr Browser kann keine MP4-Videos abspielen!</video>
</div>
<div class="bildtext">
\end{html}
\addtocounter{MVideoCounter}{1}
\setcounter{MLastIndex}{\value{MVideoCounter}}
\setcounter{MLastType}{12}
\addtocounter{MCaptionOn}{1}
\ifnum\value{MSepNumbers}=0
\textbf{Video \arabic{MVideoCounter}:} #2
\else
\textbf{Video \arabic{section}.\arabic{subsection}.\arabic{MVideoCounter}:} #2
\fi
\addtocounter{MCaptionOn}{-1}
\begin{html}
</div>
</center>
</div>
<br />
\end{html}}

\newcommand{\MDVideo}[2]{\MRegisterFile{#1.mp4}\MRegisterFile{#1.ogv}\begin{html}
<div class="imagecenter">
<center>
<div>
<video width="70\%" controls><source src="\end{html}\MMaterial/#1.mp4\begin{html}" type="video/mp4"><source src="\end{html}\MMaterial/#1.ogv\begin{html}" type="video/ogg">Ihr Browser kann keine MP4-Videos abspielen!</video>
</div>
<br />
#2
</center>
</div>
<br />
\end{html}
}

\newcommand{\MGraphics}[3]{\MUGraphics{#1}{#2}{#3}{}}

\else

\newcommand{\MVideo}[2]{%
% Kein Video im PDF darstellbar, trotzdem so tun als ob da eines waere
\begin{center}
(Video nicht darstellbar)
\end{center}
\addtocounter{MVideoCounter}{1}
\setcounter{MLastIndex}{\value{MVideoCounter}}
\setcounter{MLastType}{12}
\addtocounter{MCaptionOn}{1}
\ifnum\value{MSepNumbers}=0
\textbf{Video \arabic{MVideoCounter}:} #2
\else
\textbf{Video \arabic{section}.\arabic{subsection}.\arabic{MVideoCounter}:} #2
\fi
\addtocounter{MCaptionOn}{-1}
}


% MGraphics bindet eine Grafik ein:
% Parameter 1: Dateiname der Grafik, relativ zur Position des Modul-Tex-Dokuments
% Parameter 2: Skalierungsoptionen fuer PDF (fuer includegraphics)
% Parameter 3: Titel fuer die Grafik, wird unter die Grafik mit der Grafiknummer gesetzt
\newcommand{\MGraphics}[3]{%
\MRegisterFile{#1}%
\ %
\begin{figure}[H]%
\centering{%
\includegraphics[#2]{\MDPrefix/#1}%
\addtocounter{MCaptionOn}{1}%
\caption{#3}%
\addtocounter{MCaptionOn}{-1}%
}%
\end{figure}%
\addtocounter{MGraphicsCounter}{1}\setcounter{MLastIndex}{\value{MGraphicsCounter}}\setcounter{MLastType}{8}\ %
%\ \\Abbildung \ifnum\value{MSepNumbers}=0\else\arabic{chapter}.\arabic{section}.\fi\arabic{MGraphicsCounter}: #3%
}

\newcommand{\MUGraphics}[4]{\MGraphics{#1}{#2}{#3}}


\fi

\newcounter{MCaptionOn} % = 1 falls eine Grafikcaption aktiv ist, = 0 sonst


% MGraphicsSolo bindet eine Grafik pur ein ohne Titel
% Parameter 1: Dateiname der Grafik, relativ zur Position des Modul-Tex-Dokuments
% Parameter 2: Skalierungsoptionen (wirken nur im PDF)
\newcommand{\MGraphicsSolo}[2]{\MUGraphicsSolo{#1}{#2}{}}

% MUGraphicsSolo bindet eine Grafik pur ein ohne Titel, aber mit HTML-Skalierung
% Parameter 1: Dateiname der Grafik, relativ zur Position des Modul-Tex-Dokuments
% Parameter 2: Skalierungsoptionen (wirken nur im PDF)
% Parameter 3: Skalierungsoptionen (wirken nur im HTML), als style-format: "width=???, height=???"
\ifttm
\newcommand{\MUGraphicsSolo}[3]{\MRegisterFile{#1}\begin{html}
<img src="\end{html}\MMaterial/\MLastFile\begin{html}" style="\end{html}#3\begin{html}" alt="\end{html}\MMaterial/\MLastFile\begin{html}"/>
\end{html}%
\special{html:<!-- mfeedbackbutton;Abbildung;}#1\special{html:;}\MMaterial/\MLastFile\special{html:; //-->}%
}
\else
\newcommand{\MUGraphicsSolo}[3]{\MRegisterFile{#1}\includegraphics[#2]{\MDPrefix/#1}}
\fi

% Externer Link mit URL
% Erster Parameter: Vollstaendige(!) URL des Links
% Zweiter Parameter: Text fuer den Link
\newcommand{\MExtLink}[2]{\ifttm\special{html:<a target="_new" href="}#1\special{html:">}#2\special{html:</a>}\else\href{#1}{#2}\fi} % ohne MINTERLINK!


% Interner Link, die verlinkte Datei muss im gleichen Verzeichnis liegen wie die Modul-Texdatei
% Erster Parameter: Dateiname
% Zweiter Parameter: Text fuer den Link
\newcommand{\MIntLink}[2]{\ifttm\MRegisterFile{#1}\special{html:<a class="MINTERLINK" target="_new" href="}\MMaterial/\MLastFile\special{html:">}#2\special{html:</a>}\else{\href{#1}{#2}}\fi}


\ifttm
\def\MMaterial{:localmaterial:}
\else
\def\MMaterial{\MDPrefix}
\fi

\ifttm
\def\MNoFile#1{:directmaterial:#1}
\else
\def\MNoFile#1{#1}
\fi

\newcommand{\MChem}[1]{$\mathrm{#1}$}

\newcommand{\MApplet}[3]{
% Bindet ein Java-Applet ein, die Parameter sind:
% (wird nur im HTML, aber nicht im PDF erstellt)
% #1 Dateiname des Applets (muss mit ".class" enden)
% #2 = Breite in Pixeln
% #3 = Hoehe in Pixeln
\ifttm
\MRegisterFile{#1}
\begin{html}
<applet code="\end{html}\MMaterial/\MLastFile\begin{html}" width="#2" height="#3" alt="[Java-Applet kann nicht gestartet werden]"></applet>
\end{html}
\fi
}

\newcommand{\MScriptPage}[2]{
% Bindet eine JavaScript-Datei ein, die eine eigene Seite bekommt
% (wird nur im HTML, aber nicht im PDF erstellt)
% #1 Dateiname des Programms (sollte mit ".js" enden)
% #2 = Kurztitel der Seite
\ifttm
\begin{MSContent}{#2}{#2}{puzzle}
\MRegisterFile{#1}
\begin{html}
<script src="\MMaterial/\MLastFile" type="text/javascript"></script>
\end{html}
\end{MSContent}
\fi
}

\newcommand{\MIncludePSE}{
% Bindet bei Chemie-Modulen das PSE ein
% (wird nur im HTML, aber nicht im PDF erstellt)
\ifttm
\special{html:<!-- includepse //-->}
\begin{MSContent}{Periodensystem der Elemente}{PSE}{table}
\MRegisterFile{../files/pse.js}
\MRegisterFile{../files/radio.png}
\begin{html}
<script src="\MMaterial/../files/pse.js" type="text/javascript"></script>
<p id="divid"><br /><br />
<script language="javascript" type="text/javascript">
    startpse("divid","\MMaterial/../files"); 
</script>
</p>
<br />
<br />
<br />
<p>Die Farben der Elementsymbole geben an: <font style="color:Red">gasf&ouml;rmig </font> <font style="color:Blue">fl&uuml;ssig </font> fest</p>
<p>Die Elemente der Gruppe 1 A, 2 A, 3 A usw. geh&ouml;ren zu den Hauptgruppenelementen.</p>
<p>Die Elemente der Gruppe 1 B, 2 B, 3 B usw. geh&ouml;ren zu den Nebengruppenelementen.</p>
<p>() kennzeichnet die Masse des stabilsten Isotops</p>
\end{html}
\end{MSContent}
\fi
}

\newcommand{\MAppletArchive}[4]{
% Bindet ein Java-Applet ein, die Parameter sind:
% (wird nur im HTML, aber nicht im PDF erstellt)
% #1 Dateiname der Klasse mit Appletaufruf (muss mit ".class" enden)
% #2 Dateiname des Archivs (muss mit ".jar" enden)
% #3 = Breite in Pixeln
% #4 = Hoehe in Pixeln
\ifttm
\MRegisterFile{#2}
\begin{html}
<applet code="#1" archive="\end{html}\MMaterial/\MLastFile\begin{html}" codebase="." width="#3" height="#4" alt="[Java-Archiv kann nicht gestartet werden]"></applet>
\end{html}
\fi
}

% Bindet in der Haupttexdatei ein MINT-Modul ein. Parameter 1 ist das Verzeichnis (relativ zur Haupttexdatei), Parameter 2 ist der Dateinahme ohne Pfad.
\newcommand{\IncludeModule}[2]{
\renewcommand{\MDPrefix}{#1}
\input{#1/#2}
\ifnum\value{MSSEnd}>0{\MSubsectionEndMacros}\addtocounter{MSSEnd}{-1}\fi
}

% Der ttm-Konverter setzt keine Makros im \input um, also muss hier getrickst werden:
% Das MDPrefix muss in den einzelnen Modulen manuell eingesetzt werden
\newcommand{\MInputFile}[1]{
\ifttm
\input{#1}
\else
\input{#1}
\fi
}


\newcommand{\MCases}[1]{\left\lbrace{\begin{array}{rl} #1 \end{array}}\right.}

\ifttm
\newenvironment{MCaseEnv}{\left\lbrace\begin{array}{rl}}{\end{array}\right.}
\else
\newenvironment{MCaseEnv}{\left\lbrace\begin{array}{rl}}{\end{array}\right.}
\fi

\def\MSkip{\ifttm\MCR\fi}

\ifttm
\def\MCR{\special{html:<br />}}
\else
\def\MCR{\ \\}
\fi


% Pragmas - Sind Schluesselwoerter, die dem Preprocessing sowie dem Konverter uebergeben werden und bestimmte
%           Aktionen ausloesen. Im Output (PDF und HTML) tauchen sie nicht auf.
\newcommand{\MPragma}[1]{%
\ifttm%
\special{html:<!-- mpragma;-;}#1\special{html:;; -->}%
\else%
% MPragmas werden vom Preprozessor direkt im LaTeX gefunden
\fi%
}

% Ersatz der Befehle textsubscript und textsuperscript, die ttm nicht kennt
\ifttm%
\newcommand{\MTextsubscript}[1]{\special{html:<sub>}#1\special{html:</sub>}}%
\newcommand{\MTextsuperscript}[1]{\special{html:<sup>}#1\special{html:</sup>}}%
\else%
\newcommand{\MTextsubscript}[1]{\textsubscript{#1}}%
\newcommand{\MTextsuperscript}[1]{\textsuperscript{#1}}%
\fi

%------------------ Einbindung von dia-Diagrammen ----------------------------------------------
% Beim preprocessing wird aus jeder dia-Datei eine tex-Datei und eine pdf-Datei erzeugt,
% diese werden hier jeweils im PDF und HTML eingebunden
% Parameter: Dateiname der mit dia erstellten Datei (OHNE die Endung .dia)
\ifttm%
\newcommand{\MDia}[1]{%
\MGraphicsSolo{#1minthtml.png}{}%
}
\else%
\newcommand{\MDia}[1]{%
\MGraphicsSolo{#1mintpdf.png}{scale=0.1667}%
}
\fi%

% subsup funktioniert im Ausdruck $D={\R}^+_0$, also \R geklammert und sup zuerst
% \ifttm
% \def\MSubsup#1#2#3{\special{html:<msubsup>} #1 #2 #3\special{html:</msubsup>}}
% \else
% \def\MSubsup#1#2#3{{#1}^{#3}_{#2}}
% \fi

%\input{local.tex}

% \ifttm
% \else
% \newwrite\mintlog
% \immediate\openout\mintlog=mintlog.txt
% \fi

% ----------------------- tikz autogenerator -------------------------------------------------------------------

\newcommand{\Mtikzexternalize}{\tikzexternalize}% wird bei Konvertierung ueber mconvert ggf. ausgehebelt!

\ifttm
\else
\tikzset%
{
  % Defines a custom style which generates pdf and converts to (low and hi-res quality) png and svg, then deletes the pdf
  % Important: DO NOT directly convert from pdf to hires-png or from svg to png with GraphViz convert as it has some problems and memory leaks
  png export/.style=%
  {
    external/system call/.add={}{; 
      pdf2svg "\image.pdf" "\image.svg" ; 
      convert -density 112.5 -transparent white "\image.pdf" "\image.png"; 
      inkscape --export-png="\image.4x.png" --export-dpi=450 --export-background-opacity=0 --without-gui "\image.svg"; 
      rm "\image.pdf"; rm "\image.log"; rm "\image.dpth"; rm "\image.idx"
    },
    external/force remake,
  }
}
\tikzset{png export}
\tikzsetexternalprefix{}
% PNGs bei externer Erzeugung in "richtiger" Groesse einbinden
\pgfkeys{/pgf/images/include external/.code={\includegraphics[scale=0.64]{#1}}}
\fi

% Spezielle Umgebung fuer Autogenerierung, Bildernamen sind nur innerhalb eines Moduls (einer MSection) eindeutig)

\newcommand{\MTIKZautofilename}{tikzautofile}

\ifttm
% HTML-Version: Vom Autogenerator erzeugte png-Datei einbinden, tikz selbst nicht ausfuehren (sprich: #1 schlucken)
\newcommand{\MTikzAuto}[1]{%
\addtocounter{MTIKZAutofilenumber}{1}
\renewcommand{\MTIKZautofilename}{mtikzauto_\arabic{MTIKZAutofilenumber}}
\MUGraphicsSolo{\MSectionID\MTIKZautofilename.4x.png}{scale=1}{\special{html:[[!-- svgstyle;}\MSectionID\MTIKZautofilename\special{html: //--]]}} % Styleinfos werden aus original-png, nicht 4x-png geholt!
%\MRegisterFile{\MSectionID\MTIKZautofilename.png} % not used right now
%\MRegisterFile{\MSectionID\MTIKZautofilename.svg}
}
\else%
% PDF-Version: Falls Autogenerator aktiv wird Datei automatisch benannt und exportiert
\newcommand{\MTikzAuto}[1]{%
\addtocounter{MTIKZAutofilenumber}{1}%
\renewcommand{\MTIKZautofilename}{mtikzauto_\arabic{MTIKZAutofilenumber}}
\tikzsetnextfilename{\MTIKZautofilename}%
#1%
}
\fi

% In einer reinen LaTeX-Uebersetzung kapselt der Preambelinclude-Befehl nur input,
% in einer konvertergesteuerten PDF/HTML-Uebersetzung wird er dagegen entfernt und
% die Preambeln an mintmod angehaengt, die Ersetzung wird von mconvert.pl vorgenommen.

\newcommand{\MPreambleInclude}[1]{\input{#1}}

% Globale Watermarksettings (werden auch nochmal zu Beginn jedes subsection gesetzt,
% muessen hier aber auch global ausgefuehrt wegen Einfuehrungsseiten und Inhaltsverzeichnis

\MWatermarkSettings
% ---------------------------------- Parametrisierte Aufgaben ----------------------------------------

\ifttm
\newenvironment{MPExercise}{%
\begin{MExercise}%
}{%
\special{html:<button name="Name_MPEX}\arabic{MExerciseCounter}\special{html:" id="MPEX}\arabic{MExerciseCounter}%
\special{html:" type="button" onclick="reroll('}\arabic{MExerciseCounter}\special{html:');">Neue Aufgabe erzeugen</button>}%
\end{MExercise}%
}
\else
\newenvironment{MPExercise}{%
\begin{MExercise}%
}{%
\end{MExercise}%
}
\fi

% Parameter: Name, Min, Max, PDF-Standard. Name in Deklaration OHNE backslash, im Code MIT Backslash
\ifttm
\newcommand{\MGlobalInteger}[4]{\special{html:%
<!-- onloadstart //-->%
MVAR.push(createGlobalInteger("}#1\special{html:",}#2\special{html:,}#3\special{html:,}#4\special{html:)); %
<!-- onloadstop //-->%
<!-- viewmodelstart //-->%
ob}#1\special{html:: ko.observable(rerollMVar("}#1\special{html:")),%
<!-- viewmodelstop //-->%
}%
}%
\else%
\newcommand{\MGlobalInteger}[4]{\newcounter{mvc_#1}\setcounter{mvc_#1}{#4}}
\fi

% Parameter: Name, Min, Max, PDF-Standard. Name in Deklaration OHNE backslash, im Code MIT Backslash, Wert ist Wurzel von value
\ifttm
\newcommand{\MGlobalSqrt}[4]{\special{html:%
<!-- onloadstart //-->%
MVAR.push(createGlobalSqrt("}#1\special{html:",}#2\special{html:,}#3\special{html:,}#4\special{html:)); %
<!-- onloadstop //-->%
<!-- viewmodelstart //-->%
ob}#1\special{html:: ko.observable(rerollMVar("}#1\special{html:")),%
<!-- viewmodelstop //-->%
}%
}%
\else%
\newcommand{\MGlobalSqrt}[4]{\newcounter{mvc_#1}\setcounter{mvc_#1}{#4}}% Funktioniert nicht als Wurzel !!!
\fi

% Parameter: Name, Min, Max, PDF-Standard zaehler, PDF-Standard nenner. Name in Deklaration OHNE backslash, im Code MIT Backslash
\ifttm
\newcommand{\MGlobalFraction}[5]{\special{html:%
<!-- onloadstart //-->%
MVAR.push(createGlobalFraction("}#1\special{html:",}#2\special{html:,}#3\special{html:,}#4\special{html:,}#5\special{html:)); %
<!-- onloadstop //-->%
<!-- viewmodelstart //-->%
ob}#1\special{html:: ko.observable(rerollMVar("}#1\special{html:")),%
<!-- viewmodelstop //-->%
}%
}%
\else%
\newcommand{\MGlobalFraction}[5]{\newcounter{mvc_#1}\setcounter{mvc_#1}{#4}} % Funktioniert nicht als Bruch !!!
\fi

% MVar darf im HTML nur in MEvalMathDisplay-Umgebungen genutzt werden oder in Strings die an den Parser uebergeben werden
\ifttm%
\newcommand{\MVar}[1]{\special{html:[var_}#1\special{html:]}}%
\else%
\newcommand{\MVar}[1]{\arabic{mvc_#1}}%
\fi

\ifttm%
\newcommand{\MRerollButton}[2]{\special{html:<button type="button" onclick="rerollMVar('}#1\special{html:');">}#2\special{html:</button>}}%
\else%
\newcommand{\MRerollButton}[2]{\relax}% Keine sinnvolle Entsprechung im PDF
\fi

% MEvalMathDisplay fuer HTML wird in mconvert.pl im preprocessing realisiert
% PDF: eine equation*-Umgebung (ueber amsmath)
% HTML: Eine Mathjax-Tex-Umgebung, deren Auswertung mit knockout-obervablen gekoppelt ist
% PDF-Version hier nur fuer pdflatex-only-Uebersetzung gegeben

\ifttm\else\newenvironment{MEvalMathDisplay}{\begin{equation*}}{\end{equation*}}\fi

% ---------------------------------- Spezialbefehle fuer AD ------------------------------------------

%Abk�rzung f�r \longrightarrow:
\newcommand{\lto}{\ensuremath{\longrightarrow}}

%Makro f�r Funktionen:
\newcommand{\exfunction}[5]
{\begin{array}{rrcl}
 #1 \colon  & #2 &\lto & #3 \\[.05cm]  
  & #4 &\longmapsto  & #5 
\end{array}}

\newcommand{\function}[5]{%
#1:\;\left\lbrace{\begin{array}{rcl}
 #2 &\lto & #3 \\
 #4 &\longmapsto  & #5 \end{array}}\right.}


%Die Identit�t:
\DeclareMathOperator{\Id}{Id}

%Die Signumfunktion:
\DeclareMathOperator{\sgn}{sgn}

%Zwei Betonungskommandos (k�nnen angepasst werden):
\newcommand{\highlight}[1]{#1}
\newcommand{\modstextbf}[1]{#1}
\newcommand{\modsemph}[1]{#1}


% ---------------------------------- Spezialbefehle fuer JL ------------------------------------------


\def\jccolorfkt{green!50!black} %Farbe des Funktionsgraphen
\def\jccolorfktarea{green!25!white} %Farbe der Fl"ache unter dem Graphen
\def\jccolorfktareahell{green!12!white} %helle Einf"arbung der Fl"ache unter dem Graphen
\def\jccolorfktwert{green!50!black} %Farbe einzelner Punkte des Graphen

\newcommand{\MPfadBilder}{Bilder}

\ifttm%
\newcommand{\jMD}{\,\MD}%
\else%
\newcommand{\jMD}{\;\MD}%
\fi%

\def\jHTMLHinweisBedienung{\MInputHint{%
Mit Hilfe der Symbole am oberen Rand des Fensters
k"onnen Sie durch die einzelnen Abschnitte navigieren.}}

\def\jHTMLHinweisEingabeText{\MInputHint{%
Geben Sie jeweils ein Wort oder Zeichen als Antwort ein.}}

\def\jHTMLHinweisEingabeTerm{\MInputHint{%
Klammern Sie Ihre Terme, um eine eindeutige Eingabe zu erhalten. 
Beispiel: Der Term $\frac{3x+1}{x-2}$ soll in der Form
\texttt{(3*x+1)/((x+2)^2}$ eingegeben werden (wobei auch Leerzeichen 
eingegeben werden k"onnen, damit eine Formel besser lesbar ist).}}

\def\jHTMLHinweisEingabeIntervalle{\MInputHint{%
Intervalle werden links mit einer "offnenden Klammer und rechts mit einer 
schlie"senden Klammer angegeben. Eine runde Klammer wird verwendet, wenn der 
Rand nicht dazu geh"ort, eine eckige, wenn er dazu geh"ort. 
Als Trennzeichen wird ein Komma oder ein Semikolon akzeptiert.
Beispiele: $(a, b)$ offenes Intervall,
$[a; b)$ links abgeschlossenes, rechts offenes Intervall von $a$ bis $b$. 
Die Eingabe $]a;b[$ f"ur ein offenes Intervall wird nicht akzeptiert.
F"ur $\infty$ kann \texttt{infty} oder \texttt{unendlich} geschrieben werden.}}

\def\jHTMLHinweisEingabeFunktionen{\MInputHint{%
Schreiben Sie Malpunkte (geschrieben als \texttt{*}) aus und setzen Sie Klammern um Argumente f�r Funktionen.
Beispiele: Polynom: \texttt{3*x + 0.1}, Sinusfunktion: \texttt{sin(x)}, 
Verkettung von cos und Wurzel: \texttt{cos(sqrt(3*x))}.}}

\def\jHTMLHinweisEingabeFunktionenSinCos{\MInputHint{%
Die Sinusfunktion $\sin x$ wird in der Form \texttt{sin(x)} angegeben, %
$\cos\left(\sqrt{3 x}\right)$ durch \texttt{cos(sqrt(3*x))}.}}

\def\jHTMLHinweisEingabeFunktionenExp{\MInputHint{%
Die Exponentialfunktion $\MEU^{3x^4 + 5}$ wird als
\texttt{exp(3 * x^4 + 5)} angegeben, %
$\ln\left(\sqrt{x} + 3.2\right)$ durch \texttt{ln(sqrt(x) + 3.2)}.}}

% ---------------------------------- Spezialbefehle fuer Fachbereich Physik --------------------------

\newcommand{\E}{{e}}
\newcommand{\ME}[1]{\cdot 10^{#1}}
\newcommand{\MU}[1]{\;\mathrm{#1}}
\newcommand{\MPG}[3]{%
  \ifnum#2=0%
    #1\ \mathrm{#3}%
  \else%
    #1\cdot 10^{#2}\ \mathrm{#3}%
  \fi}%
%

\newcommand{\MMul}{\MExponentensymbXYZl} % Nur eine Abkuerzung


% ---------------------------------- Stichwortfunktionialitaet ---------------------------------------

% mpreindexentry wird durch Auswahlroutine in conv.pl durch mindexentry substitutiert
\ifttm%
\def\MIndex#1{\index{#1}\special{html:<!-- mpreindexentry;;}#1\special{html:;;}\arabic{MSubjectArea}\special{html:;;}%
\arabic{chapter}\special{html:;;}\arabic{section}\special{html:;;}\arabic{subsection}\special{html:;;}\arabic{MEntryCounter}\special{html:; //-->}%
\setcounter{MLastIndex}{\value{MEntryCounter}}%
\addtocounter{MEntryCounter}{1}%
}%
% Copyrightliste wird als tex-Datei im preprocessing von conv.pl erzeugt und unter converter/tex/entrycollection.tex abgelegt
% Der input-Befehl funktioniert nur, wenn die aufrufende tex-Datei auf der obersten Ebene liegt (d.h. selbst kein input/include ist, insbesondere keine Moduldatei)
\def\MEntryList{} % \input funktioniert nicht, weil ttm (und damit das \input) ausgefuehrt wird, bevor Datei da ist
\else%
\def\MIndex#1{\index{#1}}
\def\MEntryList{\MAbort{Stichwortliste nur im HTML realisierbar}}%
\fi%

\def\MEntry#1#2{\textbf{#1}\MIndex{#2}} % Idee: MLastType auf neuen Entry-Typ und dann ein MLabel vergeben mit autogen-Nummer

% ---------------------------------- Befehle fuer Tests ----------------------------------------------

% MEquationItem stellt eine Eingabezeile der Form Vorgabe = Antwortfeld her, der zweite Parameter kann z.B. MSimplifyQuestion-Befehl sein
\ifttm
\newcommand{\MEquationItem}[2]{{#1}$\,=\,${#2}}%
\else%
\newcommand{\MEquationItem}[2]{{#1}$\;\;=\,${#2}}%
\fi

\ifttm
\newcommand{\MInputHint}[1]{%
\ifnum%
\if\value{MTestSite}>0%
\else%
{\color{blue}#1}%
\fi%
\fi%
}
\else
\newcommand{\MInputHint}[1]{\relax}
\fi

\ifttm
\newcommand{\MInTestHeader}{%
Dies ist ein einreichbarer Test:
\begin{itemize}
\item{Im Gegensatz zu den offenen Aufgaben werden beim Eingeben keine Hinweise zur Formulierung der mathematischen Ausdr�cke gegeben.}
\item{Der Test kann jederzeit neu gestartet oder verlassen werden.}
\item{Der Test kann durch die Buttons am Ende der Seite beendet und abgeschickt, oder zur�ckgesetzt werden.}
\item{Der Test kann mehrfach probiert werden. F�r die Statistik z�hlt die zuletzt abgeschickte Version.}
\end{itemize}
}
\else
\newcommand{\MInTestHeader}{%
\relax
}
\fi

\ifttm
\newcommand{\MInTestFooter}{%
\special{html:<button name="Name_TESTFINISH" id="TESTFINISH" type="button" onclick="finish_button('}\MTestName\special{html:');">Test auswerten</button>}%
\begin{html}
&nbsp;&nbsp;&nbsp;&nbsp;&nbsp;&nbsp;&nbsp;&nbsp;
<button name="Name_TESTRESET" id="TESTRESET" type="button" onclick="reset_button();">Test zur�cksetzen</button>
<br />
<br />
<div class="xreply">
<p name="Name_TESTEVAL" id="TESTEVAL">
Hier erscheint die Testauswertung!
<br />
</p>
</div>
\end{html}
}
\else
\newcommand{\MInTestFooter}{%
\relax
}
\fi


% ---------------------------------- Notationsmakros -------------------------------------------------------------

% Notationsmakros die nicht von der Kursvariante abhaengig sind

\newcommand{\MZahltrennzeichen}[1]{\renewcommand{\MZXYZhltrennzeichen}{#1}}

\ifttm
\newcommand{\MZahl}[3][\MZXYZhltrennzeichen]{\edef\MZXYZtemp{\noexpand\special{html:<mn>#2#1#3</mn>}}\MZXYZtemp}
\else
\newcommand{\MZahl}[3][\MZXYZhltrennzeichen]{{}#2{#1}#3}
\fi

\newcommand{\MEinheitenabstand}[1]{\renewcommand{\MEinheitenabstXYZnd}{#1}}
\ifttm
\newcommand{\MEinheit}[2][\MEinheitenabstXYZnd]{{}#1\edef\MEINHtemp{\noexpand\special{html:<mi mathvariant="normal">#2</mi>}}\MEINHtemp} 
\else
\newcommand{\MEinheit}[2][\MEinheitenabstXYZnd]{{}#1 \mathrm{#2}} 
\fi

\newcommand{\MExponentensymbol}[1]{\renewcommand{\MExponentensymbXYZl}{#1}}
\newcommand{\MExponent}[2][\MExponentensymbXYZl]{{}#1{} 10^{#2}} 

%Punkte in 2 und 3 Dimensionen
\newcommand{\MPointTwo}[3][]{#1(#2\MCoordPointSep #3{}#1)}
\newcommand{\MPointThree}[4][]{#1(#2\MCoordPointSep #3\MCoordPointSep #4{}#1)}
\newcommand{\MPointTwoAS}[2]{\left(#1\MCoordPointSep #2\right)}
\newcommand{\MPointThreeAS}[3]{\left(#1\MCoordPointSep #2\MCoordPointSep #3\right)}

% Masseinheit, Standardabstand: \,
\newcommand{\MEinheitenabstXYZnd}{\MThinspace} 

% Horizontaler Leerraum zwischen herausgestellter Formel und Interpunktion
\ifttm
\newcommand{\MDFPSpace}{\,}
\newcommand{\MDFPaSpace}{\,\,}
\newcommand{\MBlank}{\ }
\else
\newcommand{\MDFPSpace}{\;}
\newcommand{\MDFPaSpace}{\;\;}
\newcommand{\MBlank}{\ }
\fi

% Satzende in herausgestellter Formel mit horizontalem Leerraum
\newcommand{\MDFPeriod}{\MDFPSpace .}

% Separation von Aufzaehlung und Bedingung in Menge
\newcommand{\MCondSetSep}{\,:\,} %oder '\mid'

% Konverter kennt mathopen nicht
\ifttm
\def\mathopen#1{}
\fi

% -----------------------------------START Rouletteaufgaben ------------------------------------------------------------

\ifttm
% #1 = Dateiname, #2 = eindeutige ID fuer das Roulette im Kurs
\newcommand{\MDirectRouletteExercises}[2]{
\begin{MExercise}
\texttt{Im HTML erscheinen hier Aufgaben aus einer Aufgabenliste...}
\end{MExercise}
}
\else
\newcommand{\MDirectRouletteExercises}[2]{\relax} % wird durch mconvert.pl gefunden und ersetzt
\fi


% ---------------------------------- START Makros, die von der Kursvariante abhaengen ----------------------------------

\ifvariantunotation
  % unotation = An Universitaeten uebliche Notation
  \def\MVariant{unotation}

  % Trennzeichen fuer Dezimalzahlen
  \newcommand{\MZXYZhltrennzeichen}{.}

  % Exponent zur Basis 10 in der Exponentialschreibweise, 
  % Standardmalzeichen: \times
  \newcommand{\MExponentensymbXYZl}{\times} 

  % Begrenzungszeichen fuer offene Intervalle
  \newcommand{\MoIl}[1][]{\mbox{}#1(\mathopen{}} % bzw. ']'
  \newcommand{\MoIr}[1][]{#1)\mbox{}} % bzw. '['

  % Zahlen-Separation im IntervaLL
  \newcommand{\MIntvlSep}{,} %oder ';'

  % Separation von Elementen in Mengen
  \newcommand{\MElSetSep}{,} %oder ';'

  % Separation von Koordinaten in Punkten
  \newcommand{\MCoordPointSep}{,} %oder ';' oder '|', '\MThinspace|\MThinspace'

\else
  % An dieser Stelle wird angenommen, dass std-Variante aktiv ist
  % std = beschlossene Notation im TU9-Onlinekurs 
  \def\MVariant{std}

  % Trennzeichen fuer Dezimalzahlen
  \newcommand{\MZXYZhltrennzeichen}{,}

  % Exponent zur Basis 10 in der Exponentialschreibweise, 
  % Standardmalzeichen: \times
  \newcommand{\MExponentensymbXYZl}{\times} 

  % Begrenzungszeichen fuer offene Intervalle
  \newcommand{\MoIl}[1][]{\mbox{}#1]\mathopen{}} % bzw. '('
  \newcommand{\MoIr}[1][]{#1[\mbox{}} % bzw. ')'

  % Zahlen-Separation im IntervaLL
  \newcommand{\MIntvlSep}{;} %oder ','
  
  % Separation von Elementen in Mengen
  \newcommand{\MElSetSep}{;} %oder ','

  % Separation von Koordinaten in Punkten
  \newcommand{\MCoordPointSep}{;} %oder '|', '\MThinspace|\MThinspace'

\fi



% ---------------------------------- ENDE Makros, die von der Kursvariante abhaengen ----------------------------------


% diese Kommandos setzen Mathemodus vorraus
\newcommand{\MGeoAbstand}[2]{[\overline{{#1}{#2}}]}
\newcommand{\MGeoGerade}[2]{{#1}{#2}}
\newcommand{\MGeoStrecke}[2]{\overline{{#1}{#2}}}
\newcommand{\MGeoDreieck}[3]{{#1}{#2}{#3}}

%
\ifttm
\newcommand{\MOhm}{\special{html:<mn>&#x3A9;</mn>}}
\else
\newcommand{\MOhm}{\Omega} %\varOmega
\fi


\def\PERCTAG{\MAbort{PERCTAG ist zur internen verwendung in mconvert.pl reserviert, dieses Makro darf sonst nicht benutzt werden.}}

% Im Gegensatz zu einfachen html-Umgebungen werden MDirectHTML-Umgebungen von mconvert.pl am ganzen ttm-Prozess vorbeigeschleust und aus dem PDF komplett ausgeschnitten
\ifttm%
\newenvironment{MDirectHTML}{\begin{html}}{\end{html}}%
\else%
\newenvironment{MDirectHTML}{\begin{html}}{\end{html}}%
\fi

% Im Gegensatz zu einfachen Mathe-Umgebungen werden MDirectMath-Umgebungen von mconvert.pl am ganzen ttm-Prozess vorbeigeschleust, ueber MathJax realisiert, und im PDF als $$ ... $$ gesetzt
\ifttm%
\newenvironment{MDirectMath}{\begin{html}}{\end{html}}%
\else%
\newenvironment{MDirectMath}{\begin{equation*}}{\end{equation*}}% Vorsicht, auch \[ und \] werden in amsmath durch equation* redefiniert
\fi

% ---------------------------------- Location Management ---------------------------------------------

% #1 = buttonname (muss in files/images liegen und Format 48x48 haben), #2 = Vollstaendiger Einrichtungsname, #3 = Kuerzel der Einrichtung,  #4 = Name der include-texdatei
\ifttm
\newcommand{\MLocationSite}[3]{\special{html:<!-- mlocation;;}#1\special{html:;;}#2\special{html:;;}#3\special{html:;; //-->}}
\else
\newcommand{\MLocationSite}[3]{\relax}
\fi

% ---------------------------------- Copyright Management --------------------------------------------

\newcommand{\MCCLicense}{%
{\color{green}\textbf{CC BY-SA 3.0}}
}

\newcommand{\MCopyrightLabel}[1]{ (\MSRef{L_COPYRIGHTCOLLECTION}{Lizenz})\MLabel{#1}}

% Copyrightliste wird als tex-Datei im preprocessing erzeugt und unter converter/tex/copyrightcollection.tex abgelegt
% Der input-Befehl funktioniert nur, wenn die aufrufende tex-Datei auf der obersten Ebene liegt (d.h. selbst kein input/include ist, insbesondere keine Moduldatei)
\newcommand{\MCopyrightCollection}{\input{copyrightcollection.tex}}

% MCopyrightNotice fuegt eine Copyrightnotiz ein, der parser ersetzt diese durch CopyrightNoticePOST im preparsing, diese Definition wird nur fuer reine pdflatex-Uebersetzungen gebraucht
% Parameter: #1: Kurze Lizenzbeschreibung (typischerweise \MCCLicense)
%            #2: Link zum Original (http://...) oder NONE falls das Bild selbst ein Original ist, oder TIKZ falls das Bild aus einer tikz-Umgebung stammt
%            #3: Link zum Autor (http://...) oder MINT falls Original im MINT-Kolleg erstellt oder NONE falls Autor unbekannt
%            #4: Bemerkung (z.B. dass Datei mit Maple exportiert wurde)
%            #5: Labelstring fuer existierendes Label auf das copyrighted Objekt, mit MCopyrightLabel erzeugt
%            Keines der Felder darf leer sein!
\newcommand{\MCopyrightNotice}[5]{\MCopyrightNoticePOST{#1}{#2}{#3}{#4}{#5}}

\ifttm%
\newcommand{\MCopyrightNoticePOST}[5]{\relax}%
\else%
\newcommand{\MCopyrightNoticePOST}[5]{\relax}%
\fi%

% ---------------------------------- Meldungen fuer den Benutzer des Konverters ----------------------
\MPragma{mintmodversion;P0.1.0}
\MPragma{usercomment;This is file mintmod.tex version P0.1.0}


% ----------------------------------- Spezialelemente fuer Konfigurationsseite, werden nicht von mintscripts.js verwaltet --

% #1 = DOM-id der Box
\ifttm\newcommand{\MConfigbox}[1]{\special{html:<input cfieldtype="2" type="checkbox" name="Name_}#1\special{html:" id="}#1\special{html:" onchange="confHandlerChange('}#1\special{html:');"/>}}\fi % darf im PDF nicht aufgerufen werden!


\Mtikzexternalize
\MPragma{MathSkip}

\begin{document}

\MSection{Integral Calculus}\MLabel{VBKM08}
\MSetSectionID{VBKM08}
\begin{MSectionStart}
\MDeclareSiteUXID{VBKM08_START}

\MModstartBox
\end{MSectionStart}

%%%Abschnitt
\MSubsection{Antiderivatives}
\MLabel{M08A_Stammfunktionen}

\begin{MIntro}
\MDeclareSiteUXID{VBKM08_Stammfunktionen_Intro}

In the previous module we studied derivatives of functions. 
As for every other arithmetic operation,the question of finding an inverse operation arises. For example, subtraction is the inverse operation of addition, and division is the inverse operation of 
multiplication. The question of the inverse operation of differentiation leads
to the introduction of integral calculus the definition of an
antiderivative. The relation between derivative and antiderivative can be explained 
very easily. If a derivative $f'$ can be assigned to a function $f$, and the derivative 
$f'$ is also considered as a function, then the function $f$ could be assigned to this function 
$f'$ by inverting the operation of ``differentiation''. Thus, in this 
chapter the question is: for a given function $f$, can we find another function 
with $f$ as its derivative? 

The applications of integral calculus are as varied as the applications of 
differential calculus. In physics, for example, the force $F$ acting on a mass $m$ may be investigated. From the well-known relation $F = m a$ ($a$ 
being the acceleration of the object), the acceleration $a = F/m$ can be calculated 
from the force. If the acceleration is interpreted as the rate of velocity change, i.e.
$a = \frac{\MD v}{\MD t}$, then the velocity can be determined subsequently from the inverse 
of the derivative -- from integral calculus. Similar relations can be found 
in many fields of science and engineering, and also in economics. Integral 
calculus is used for the calculation of areas, centres of mass, bending
properties of beams or the solutions of so called differential equations,
which are used so frequently in science and engineering.
\end{MIntro}

\begin{MXContent}{Antiderivatives}{Antiderivatives}{STD}
\MDeclareSiteUXID{VBKM08_Stammfunktionen_Content}

In the context of this course we will discuss integral calculus for functions on 
``connected domains'' as it is of particular significance for many practical applications. 
In mathematical terms, the domains of the functions will be intervals. As the inverses of derivatives, antiderivatives will be also defined on intervals.

\begin{MXInfo}{Antiderivative} 
Let an interval $D \subseteq \R$ and a function $f: D \rightarrow \R$ be given.
If there exists a differentiable function $F: D  \rightarrow \R$ that has 
$f$ as its derivative, i.e. $F'(x) = f(x)$ for all $x \in D$, then $F$ is called an 
\MEntry{antiderivative}{antiderivative} of~$f$.
\end{MXInfo}

Let us first consider a few examples.

\begin{MExample}
The function $F$ with $F(x) = -\cos(x)$ has the derivative
\[
F'(x) = -(-\sin(x)) = \sin(x) \MDFPeriod %%
\]
Thus, $F$ is an antiderivative of $f$ with $f(x) = \sin(x)$.
%\[
%\int \sin(x) \MDwSp x = -\cos(x) %%
%\]
%eine Stammfunktion von $f(x) = \sin(x)$.
\end{MExample}

%\begin{MExample}
%Die Funktion $G(x) = \MEU^{3x + 7}$ hat die Ableitung 
%$G'(x) = 3 \cdot \MEU^{3 x + 7}$.
%Deshalb ist
%\[
%\int 3 \cdot \MEU^{3 x + 7} \MDwSp x = \MEU^{3 x + 7} %%
%\]
%eine Stammfunktion von $f(x) = 3 \MEU^{3 x + 7}$.
%\end{MExample}

\begin{MExample}
The function $G$ with $G(x) = \frac{1}{3} \MEU^{3x + 7}$ has the derivative 
\[
G'(x) = \frac{1}{3} \cdot 3 \cdot \MEU^{3 x + 7} \MDFPeriod
\]
Hence, $G$ is an antiderivative of $g$ with $g(x) = \MEU^{3 x + 7}$.
\end{MExample}

Next we will consider another very simple example, which illustrates an important point to note when calculating antiderivatives.

\begin{MExample}
Let a constant function $H$ with the function value $H(x) = 18$ be given on an interval. 
Then the function $H$ has the derivative 
\[
H'(x) = 0 \MDFPeriod
\]
Hence, $H$ is an antiderivative of $h$ with $h(x) = 0$.
\end{MExample}

The last example is a little surprising because the derivative of a constant function 
is the zero function. Thus, \emph{every} constant function $F$ is an antiderivative of
$f$ with $f(x) = 0$ on an interval, i.e. $F(x)$ is equal to any number $C$ for every 
value of $x$. However, the antiderivative $F(x)$ cannot be any other function 
than a \emph{constant} one if $f$ is defined on an interval.

\begin{MXInfo}{All Derivatives of the Zero Function}
The function $F$ is an antiderivative of $f$ with $f(x) = 0$ on an interval 
if and only if $F$ is a constant function, i.e. if a real number $C$ exists such 
that $F(x) = C$ for all values of $x$ in the interval.
\end{MXInfo}

If the functions $F$ and $G$ have the same derivative, i.e. $f = F' = G'$, then 
we have $G'(x) - F'(x) = 0$. Taking the antiderivative on an interval on both sides of the 
equation results in the relation $G(x) - F(x) = C$. Thus, we have $G(x) = F(x) + C$. 
Therefore, if $F$ is an antiderivative of $f$, then $G$ with $G(x) = F(x) + C$ is also an
antiderivative of $f$. 

\begin{MXInfo}{Statement on Antiderivatives}
If $F$ and $G$ are antiderivatives of $f: D \rightarrow \R$ on an interval $D$, then there exists
a real number $C$ such that
\[
F(x) = G(x) + C %
\qquad \text{for all } x \in D \MDFPeriod %%
\]
This is also written as
\[
\int f(x) \MDwSp x = F(x) + C,  %%
\]
to express how all antiderivatives of $f$ look.
\end{MXInfo}

The set of all antiderivatives is also called 
\MEntry{indefinite integral}{integral (indefinite)} and is written 
according to the statement above as
\[
\int f(x) \MDwSp x = F(x) + C\MDFPSpace , %%
\]
where $F$ is any antiderivative of $f$.

This notation of the indefinite integral emphasises that it is a function 
$F$ with $F' = f$ that is calculated for a given function $f$. How this expression is 
used to calculate the (definite) integral of a continuous function $f$ is 
described by the 
\MEntry{fundamental theorem of calculus}{fundamental theorem (calculus)}
discussed in the next section in Info Box~\MRef{HSDDIR}.


How do we know the value of this constant $C$? If we only look for an antiderivative
of $f$ with $f(x) = 0$ on an interval without knowing any other conditions, then the constant
$C$ is indefinite. $C$ is only definite if an additional function value $y_0 = F(x_0)$ of $F$ at a point 
$x_0$ is given.

\begin{MExample}
For example, for $f$ with $f(x) = 2x + 5$, we have
\[
\int \left( 2 x + 5 \right) \MD x = x^2 + 5x + C \MDFPeriod %%
\]
If we look for the antiderivative $F$ of $f$ with $F(0) = 6$, then we set 
$6 = F(0) = 0^2 + 5 \cdot 0 + C = C$ and hence, $C = 6$. Thus, the 
antiderivative is in this case $F(x) = x^2 + 5 x + 6$.
\end{MExample}

If the relation between the derivative $f = F'$ and the antiderivative $F$
is written in the way discussed above for the types of functions considered so far, then one obtains
the following table:

\begin{MXInfo}{A Small Table of Antiderivatives}%
\MIndex{antiderivatives (table)}%
The functions $f$ are considered on an interval. The antiderivatives of these 
functions are given as an indefinite integral:
\ifttm
\[
\begin{array}{ll}
\text{Function } f & \text{Antiderivatives } F \\
\hline
f(x) = 0                    & F(x) = \int 0 \MDwSp x = C \\
f(x) = x^n                  & F(x) = \int x^n \MDwSp x = \frac{1}{n+1} \cdot x^{n+1} + C \\
f(x) = \sin(x)              & F(x) = \int \sin(x) \MDwSp x = -\cos(x) + C \\
f(x) = \sin(k x)            & F(x) = \int \sin(k x) \MDwSp x = -\frac{1}{k}\,\cos(k x) + C \\
f(x) = \cos(x)              & F(x) = \int \cos(x) \MDwSp x = \sin(x) + C \\
f(x) = \cos(k x)            & F(x) = \int \cos(k x) \MDwSp x = \frac{1}{k}\,\sin(k x) + C \\
f(x) = \MEU^x               & F(x) = \int \MEU^x \MDwSp x = \MEU^x + C \\
f(x) = \MEU^{k x}           & F(x) = \int \MEU^{k x} \MDwSp x = \frac{1}{k}\,\MEU^{k x} + C \\
f(x) = x^{-1} = \frac{1}{x} & F(x) = \int \frac{1}{x} \MDwSp x = \ln|x| + C  %
\text{ for } x > 0 \text{ or } x < 0 %%
\end{array}
\]
\else
\[
\begin{array}{ll}
\text{Function } f & \mbox{Antiderivatives } F \\
\hline
f(x) = 0                    & F(x) = \int 0 \MDwSp x = C \\[1ex]
f(x) = x^n                  & F(x) = \int x^n \MDwSp x = \frac{1}{n+1} \cdot x^{n+1} + C \\[1ex]
f(x) = \sin(x)              & F(x) = \int \sin(x) \MDwSp x = -\cos(x) + C \\[1ex]
f(x) = \sin(k x)            & F(x) = \int \sin(k x) \MDwSp x = -\frac{1}{k}\,\cos(k x) + C \\[1ex]
f(x) = \cos(x)              & F(x) = \int \cos(x) \MDwSp x = \sin(x) + C \\[1ex]
f(x) = \cos(k x)            & F(x) = \int \cos(k x) \MDwSp x = \frac{1}{k}\,\sin(k x) + C \\[1ex]
f(x) = \MEU^x               & F(x) = \int \MEU^x \MDwSp x = \MEU^x + C \\[1ex]
f(x) = \MEU^{k x}           & F(x) = \int \MEU^{k x} \MDwSp x = \frac{1}{k}\,\MEU^{k x} + C \\[1ex]
f(x) = x^{-1} = \frac{1}{x} & F(x) = \int \frac{1}{x} \MDwSp x = \ln|x| + C %
\text{ for } x > 0 \text{ or } x < 0 %%
\end{array}
\]
\fi
Here, $k$ and $C$ denote arbitrary real numbers with $k \neq 0$, and $n$ is an integer with $n \neq -1$.
\end{MXInfo}

The next example shows how the table is used.

\begin{MExample}
Find the indefinite integral of the function $f$ with $f(x) = 10 x^2 - 6 = 10 x^2 - 6 x^0$.

>From the table above we read off the antiderivatives of $g$ with $g(x) = x$ and $h$ with 
$h(x) = x^0 = 1$: The function $G$ with $G(x) = \frac{1}{1+1} \cdot x^{1+1} = \frac{1}{2} \cdot x^2$
is an antiderivative of $g$, and the function $H$ with $H(x) = \frac{1}{0+1} \cdot x^{0+1} = x$ is an antiderivative
of $h$. Thus, the function $F: \R \to \R$ with 
\[
F(x) = 10 \cdot \frac{1}{2} x^2 - 6 \cdot x = 5 x^2 - 6 x %%
\]
is an antiderivative of $f$. We see that
\[
\int \left(10 x - 6 \right) \MD x = 5 x^2 - 6 x + C
\]
describes the set of antiderivatives of $f: \R \to \R$ with $f(x) = 10 x - 6$, where
$C$ is an arbitrary real number.

The notation using the constant $C$ expresses that, for example, $G: \R \to \R$
with $G(x) := 5 x^2 - 6 x - 7$ is also an antiderivative of $f$, where $C = -7$, 
since $G'(x) = 5 \cdot 2 x - 6 = f(x)$  for all $x \in \R$.
\end{MExample}

In table books the antiderivatives are generally listed neglecting the constants. 
However, for calculations it is necessary to state that several functions differing 
by a constant can exist. In solving problems of applied mathematics, the constant $C$ is often 
determined by additional conditions, such as a given function value 
of the antiderivative.
%\begin{MExample}
%Die Funktion $F(x) = 5 x^2 - 6 x$ hat die Ableitung $F'(x) = 10 x - 6$. 
%Somit ist 
%\[
%\int (10 x - 6) = 5 x^2 - 6 x
%\]
%eine Stammfunktion von $f(x) = 10 x - 6$.
%
%Die Ableitung der Funktion $G(x) = 5 x^2 - 6 x + 3$ ist 
%$G'(x) = 10 x - 6 = F'(x)$, also dieselbe wie von $F$. Damit ist auch $G$ eine 
%Stammfunktion von $f(x) = 10 x - 6$.
%\end{MExample}
%\end{MXContent}

%Die Bestimmung einer Stammfunktion ist nicht immer einfach, mitunter eine 
%Kunst. Eine Kontrolle lohnt sich also und ergibt sich definitionsgem"a"s 
%einfach dadurch, dass die Ableitung berechnet und mit der gegebenen Funktion
%verglichen wird, in der Praxis meistens eine Anwendung der Rechentechnik der 
%Ableitungsregeln.

\begin{MXInfo}{Practical Note}
It is very easy to check whether the antiderivative of a given function $f$ was 
found correctly. Take the derivative of the found antiderivative and compare
it to the initially given function $f$. If the functions coincide, then the 
calculation was correct. If the result does not coincide with the function $f$, then 
the antiderivative has to be checked again.
\end{MXInfo}
\end{MXContent}


%%%Uebungen zum Abschnitt:
\begin{MExercises}
\MDeclareSiteUXID{VBKM08_Stammfunktionen_Exercises}

\begin{MExercise}
%Loesungshinweise erstellt (jgl).
Specify an antiderivative:
\begin{MExerciseItems}
\item{\MEquationItem{$\displaystyle\int \left(12 x^2 - 4 x^7\right) \MD x $}
{\MLSimplifyQuestion{30}{4*x^3 - 1/2*x^8}{10}{x}{4}{32}{LSTFX1}}.
\begin{MHint}{Solution}{%
We have
\[
\int \left(12 x^2 - 4 x^7\right) \MD x %
= 12 \cdot \frac{1}{3} \cdot x^3 - 4 \cdot \frac{1}{8} \cdot x^8 %
= 4 \cdot x^3 - \frac{1}{2} \cdot x^8 + C\MDFPeriod %
\]
}
\end{MHint}}
%
\item{\MEquationItem{$\displaystyle\int \left(\sin(x) + \cos(x)\right) \MD x$}
{\MLSimplifyQuestion{30}{sin(x)-cos(x)}{10}{x}{4}{32}{LSTFX2}}.
\begin{MHint}{Solution}{%
We have
\[
\int \left(\sin(x) + \cos(x)\right) \MD x %
= -\cos(x) + \sin(x) + C %
= \sin(x) - \cos(x) + C\MDFPeriod %
\]
}\end{MHint}}
%Teilaufgabe ersetzt, da arctan nicht (mehr) in der Liste von Stammfunktionen
%vermerkt werden soll (jgl):
%\item{\MEquationItem{$\displaystyle\int \frac{1}{x^2 + 1} \MDwSp x$}{\MLSimplifyQuestion{30}{arctan(x)}{10}{x}{4}{32}{LSTFX3}}. \begin{MHint}{L�sung}{%
%% Die Stammfunktion kann aus der obigen Tabelle entnommen werden.
%}\end{MHint}}
\item{\MEquationItem{$\displaystyle\int \frac{1}{6 \sqrt{x}} \MDwSp x$}
{\MLSimplifyQuestion{30}{1/3*sqrt(x)}{10}{x}{4}{544}{LSTFX3}}. 
\begin{MHint}{Solution}{%
We have
$\frac{1}{6 \sqrt{x}} = \frac{1}{6} \cdot \frac{1}{\sqrt{x}} %
= \frac{1}{6} \cdot x^{-\frac{1}{2}}$. Hence,
\[
\int \frac{1}{6 \sqrt{x}} \MDwSp x %
= \frac{1}{6} \cdot \int x^{-\frac{1}{2}} \MDwSp x %
= \frac{1}{6} \cdot \frac{1}{-\frac{1}{2} + 1} \cdot x^{-\frac{1}{2} + 1} + C %
= \frac{1}{6} \cdot 2 \cdot x^{\frac{1}{2}} + C %
= \frac{1}{3} \cdot \sqrt{x} + C\MDFPeriod %
\]
}
\end{MHint}}
\end{MExerciseItems}
\end{MExercise}

\begin{MExercise}
%Loesungshinweise erstellt (jgl).
Find an antiderivative:
\begin{MExerciseItems}
\item{\MEquationItem{$\displaystyle\int \MEU^{x+2} \MDwSp x$}{\MLSimplifyQuestion{30}{exp(x+2)}{10}{x}{4}{32}{LSTNF1}}.}
\begin{MHint}{Solution}
We have
$\displaystyle \int \MEU^{x+2} \MDwSp x = %
\int \MEU^x \cdot \MEU^2 \MDwSp x = %
\MEU^2 \cdot \int \MEU^x \MDwSp x = %
\MEU^2 \cdot \MEU^x + C = %
\MEU^x \cdot \MEU^2 + C = %
\MEU^{x + 2} + C$.
\end{MHint}
%
\item{\MEquationItem{$\displaystyle\int 3 x\cdot \sqrt[4]{x} \MDwSp x$}{\MLSimplifyQuestion{30}{4/3*x^(9/4)}{1}{x}{10}{32}{LSTNF2}}.}
\begin{MHint}{Solution}
We have
$\displaystyle \int 3 x \cdot \sqrt[4]{x} \MDwSp x = %
3 \cdot \int x \cdot x^{\frac{1}{4}} \MDwSp x = %
3 \cdot \int x^{\frac{5}{4}} \MDwSp x = %
3 \cdot \frac{4}{9} \cdot x^{\frac{9}{4}} + C = %
\frac{4}{3} \cdot x^{\frac{9}{4}} + C$.
\end{MHint}
\end{MExerciseItems}
\jHTMLHinweisEingabeFunktionenExp
\end{MExercise}

%Aufgabe herausgenommen (jgl), da diese in Stellungnahmen als zu schwierig
%eingestuft wurde.
%\begin{MExercise}
%Beschreiben Sie den Betrag $|x|$ einer reellen Zahl $x$ mittels einer
%Fallunterscheidung und bestimmen Sie damit $\displaystyle\int |x| \MDwSp x$.
%
%Antwort: $\displaystyle\int |x| \MDwSp x$ %
%= \MLSimplifyQuestion{60}{1/2*x*abs(x)}{10}{x}{4}{32}{LSTNF3}
%\ifttm
%\ \\ \ \\
%\else\relax\fi
%\MInputHint{Die Stammfunktion kann entweder �ber eine Fallunterscheidung, oder wieder mit Hilfe der Betragsfunktion notiert werden.
%Fallunterscheidungen mit zwei F�llen k�nnen mit dem Textst�ck \texttt{falls(bedingung,fall1,fall2)} eingegeben werden, beispielsweise tippt man
%eine Fallunterscheidung der Form
%$$
%|x| \;=\; \left\lbrace{\begin{array}{ll} \phantom{-}x & \text{falls$\;x>0$} \\ -x & \text{falls$\;x\leq 0$}\end{array}}\right.
%$$
%als \texttt{falls(x>0,x,-x)} ein.
%}
%\end{MExercise}

%Neue Aufgabe (jgl) als Vorbereitung auf den Abschlusstest:
\begin{MExercise} %Stammfunktionen:
Decide whether the following statements are true for real-valued functions.
%Hierbei sind in der letzten Teilaufgabe irgendwelche Funktionen $f$ und $g$ 
%gegeben, die jeweils Stammfunktionen besitzen, und es ist $F$ eine 
%Stammfunktion von $f$ und $G$ eine Stammfunktion von $g$.

\ifttm
\begin{MQuestionGroup}
\begin{tabular}{|l|l|}
\hline
%richtig: & falsch: & Aussage: \\
 True? & Statement: \\
 \MLCheckbox{0}{M08Ex1101a} & % \MLCheckbox{1}{M08Ex1101b} &
$F$ with $F(x) = -\frac{\cos(\pi x) + 2}{\pi}$ is an antiderivative 
of $f$ with $f(x) = \sin(\pi x) + 2$. \\
%
 \MLCheckbox{1}{M08Ex1102a} & % \MLCheckbox{0}{M08Ex1102b} &
$F$ with $F(x) = -\frac{\cos(\pi x) + 2}{\pi}$ is an antiderivative
of $f$ with $f(x) = \sin(\pi x)$. \\
% 
 \MLCheckbox{0}{M08Ex1103a} & % \MLCheckbox{1}{M08Ex1103b} &
$F$ with $F(x) = -7$ is an antiderivative of $f$ with $f(x) = -7x$ %
for $x \in \R$. \\
% 
 \MLCheckbox{1}{M08Ex1104a} & % \MLCheckbox{0}{M08Ex1104b} &
$F$ with $F(x) = (\sin(x))^2$ is an antiderivative 
of $f$ with $f(x) = 2 \sin(x) \cos(x)$. \\
%
 \MLCheckbox{1}{M08Ex1105a} & % \MLCheckbox{0}{M08Ex1105b} &
If $F$ is an antiderivative of $f$, and $G$ is an antiderivative of $g$, 
then $F + G$ is an antiderivative of $f + g$. \\
\hline
\end{tabular}
\end{MQuestionGroup}
\MGroupButton{Check entries}
%
\else
%
\begin{tabular}[t]{ccp{140mm}}
 True: & False: & Statement: \\
 \MLCheckbox{0}{M08Ex1101a} & \MLCheckbox{1}{M08Ex1101b} &
$F$ with $F(x) = -\frac{\cos(\pi x) + 2}{\pi}$ is an antiderivative 
of $f$ with $f(x) = \sin(\pi x) + 2$. \\
%
 \MLCheckbox{1}{M08Ex1102a} & \MLCheckbox{0}{M08Ex1102b} &
$F$ with $F(x) = -\frac{\cos(\pi x) + 2}{\pi}$ is an antiderivative 
of $f$ with $f(x) = \sin(\pi x)$. \\
% 
 \MLCheckbox{0}{M08Ex1103a} & \MLCheckbox{1}{M08Ex1103b} &
$F$ with $F(x) = -7$ is an antiderivative of $f$ with $f(x) = -7x$
for $x \in \R$. \\
% 
 \MLCheckbox{1}{M08Ex1104a} & \MLCheckbox{0}{M08Ex1104b} &
$F$ with $F(x) = (\sin(x))^2$ is an antiderivative  
of $f$ with $f(x) = 2 \sin(x) \cos(x)$. \\
%
 \MLCheckbox{1}{M08Ex1105a} & \MLCheckbox{0}{M08Ex1105b} &
If $F$ is an antiderivative of $f$, and $G$ is an antiderivative of $g$, 
then $F + G$ is an antiderivative of $f + g$. %%
\end{tabular}
\fi
\begin{MHint}{Solution}
\begin{itemize}
\item The derivative of $F$ with $F(x) =  -\frac{\cos(\pi x) + 2}{\pi}$ is
$F'(x) = \sin(\pi x) \neq \sin(\pi x) + 2 = f(x)$. Therefore $F$ is not 
an antiderivative of $f$.
\item The derivative of $F$ with $F(x) =  -\frac{\cos(\pi x) + 2}{\pi}$ is
$F'(x) = \sin(\pi x) = f(x)$. $F$ is an antiderivative of~$f$.
\item The derivative of $F$ with $F(x) =  -7$ is
$F'(x) = 0 \neq -7x = f(x)$ (for $x \neq 0$). Hence, $F$ is not an
antiderivative of $f$.
\item The derivative of $F$ with $F(x) =  (\sin(x))^2$ is, using the chain rule, 
$F'(x) = 2 \cdot \sin(x) \cdot \cos(x) = f(x)$. Thus, $F$ 
is an antiderivative of $f$.
\item If $F$ is an antiderivative of $f$, and $G$ is an antiderivative of $g$, 
then $F$ and $G$ are differentiable, where $F' = f$ and $G' = g$. 
Thus, $F + G$ is differentiable, and we have 
$(F + G)'(x) = F'(x) + G'(x) = f(x) + g(x) = (f + g)(x)$, i.e.
$F + G$ is an antiderivative of $f + g$. %%
\end{itemize}
\end{MHint}
\end{MExercise}


%Alle Teilaufgaben der Aufgabe ersetzt, da die Aufgabe als zu umfangreich 
%bezeichnet wurde beziehungsweise da arctan nicht (mehr) in der Liste der 
%Stammfunktionen aufgefuehrt werden soll (jgl):
%\begin{MExercise}
%Bestimmen Sie eine Stammfunktion zu
%\begin{MExerciseItems}
%\item $f(x) := \frac{1+x+x^2+\sqrt{x}}{x}$,
%\item $g(x) := 4 + \left(\frac{4 \cos^2 x}{(2 \sin x)^2}\right)^{-1}$,
%\item $h(x) := \frac{2}{4 + (2x)^2}$,
%\end{MExerciseItems}
%nachdem Sie die Funktionsterme vereinfacht haben:
%
%\begin{MExerciseItems}
%\item{Mit der Vereinfachung \MEquationItem{$f(x)$}{\MLSimplifyQuestion{30}{1/x+1+x+1/sqrt(x)}{10}{x}{4}{512}{ISF1}}\\ 
%ergibt sich \MEquationItem{$F(x)$}{\MLSimplifyQuestion{30}{ln(x)+x+1/2*x^2+2*sqrt(x)}{10}{x}{4}{544}{ISF1b}} f�r $x>0$.\\
%\begin{MHint}{L�sung}
%Die Funktion l�sst sich zu $f(x)=\frac1x+1+x+x^{-\frac12}$ vereinfachen, was auf 
%$$
%F(x) \;=\; \ln(x)+x+\frac12x^2+2x^{\frac12} \;=\;\ln(x)+x+\frac12x^2+2\sqrt{x}
%$$
%f�hrt (bis auf eine Konstante).
%\end{MHint}
%}
%\item{Mit der Vereinfachung \MEquationItem{$g(x)$}{\MLSimplifyQuestion{30}{4 + (tan(x))^2}{10}{x}{4}{512}{ISF2}}\\
%ergibt sich \MEquationItem{$G(x)$}{\MLSimplifyQuestion{30}{3*x + tan(x)}{10}{x}{4}{544}{ISF2b}}.\\
%\begin{MHint}{L�sung}
%Die Funktion l�sst sich zu $g(x)=4+\tan(x)^2$ vereinfachen, was auf 
%$$
%G(x) \;=\; 3x+\tan(x)
%$$
%f�hrt (bis auf eine Konstante).
%\end{MHint}
%}
%\item{Mit der Vereinfachung \MEquationItem{$h(x)$}{\MLSimplifyQuestion{30}{1/2*1/(1 + x^2)}{10}{x}{4}{0}{ISF3}}\\
%ergibt sich \MEquationItem{$H(x)$}{\MLSimplifyQuestion{30}{1/2*arctan(x)}{10}{x}{4}{32}{ISF3b}}.\\
%\begin{MHint}{L�sung}
%Die Funktion l�sst sich zu $h(x)=\frac12\cdot \frac1{1+x^2}$ vereinfachen, was auf 
%$$
%H(x) \;=\; \frac12\cdot \arctan(x)
%$$
%f�hrt (bis auf eine Konstante).
%\end{MHint}
%}
%\end{MExerciseItems}
%\MInputHint{Schreiben Sie beispielsweise $\cos^2(x)$ als \texttt{cos(x)^2} und den Arkustangens als \texttt{arctan(x)}.}
%\end{MExercise}


%Neue Aufgabe als Ersatz fuer die vorherige Version (siehe oben, jgl):
\begin{MExercise}
Find an antiderivative of 
\begin{MExerciseItems}
\item $\displaystyle f(x) := \frac{8 x^3 - 6 x^2}{x^4}$
%\item $\displaystyle g(x) := \frac{5 x}{x + 2 x^2}$,
%\item $\displaystyle g(x) := \frac{6 \sqrt{x} + 1}{3 \sqrt{x^5}}$,
\item $\displaystyle g(x) := \frac{18 x^2}{3 \sqrt{x^5}}$
\item $\displaystyle h(x) := \frac{x + 2 \sqrt{x}}{4x}$
\end{MExerciseItems}
for $x > 0$, after rewriting the terms as reduced sums of fractions:

\begin{MExerciseItems}
\item{With the simplification \MEquationItem{$f(x)$}{\MLSimplifyQuestion{30}{8/x-6/x^2}{10}{x}{4}{512}{ISF1}}\\ 
we have the antiderivative \MEquationItem{$F(x)$}{\MLSimplifyQuestion{30}{8*ln(x)+6/x}{10}{x}{4}{544}{ISF1b}} for $x>0$.\\
\begin{MHint}{Solution}
The function can be rewritten as %
$f(x) = \frac{8}{x} - \frac{6}{x^2} = \frac{8}{x} - 6 x^{-2}$
which results in the antiderivative 
\[
F(x) \;=\; 8 \cdot \ln(x) + \frac{6}{x} %
\]
for $x > 0$ (up to a constant).
\end{MHint}
}
\item{With the simplification  \MEquationItem{$g(x)$}{\MLSimplifyQuestion{30}{6/sqrt(x)}{10}{x}{4}{512}{ISF2}}\\
we have the antiderivative \MEquationItem{$G(x)$}{\MLSimplifyQuestion{30}{12*sqrt(x)}{10}{x}{4}{544}{ISF2b}} for $x > 0$.\\
\begin{MHint}{Solution}
The function can be rewritten as  %
$g(x) = \frac{6}{\sqrt{x}} = 6 \cdot x^{-\frac{1}{2}}$ %
which results in the antiderivative
\[
G(x) \;=\; 6 \cdot 2 \cdot \sqrt{x} = 12 \cdot \sqrt{x} %%
\]
for $x > 0$ (up to a constant).
\end{MHint}
}
\item{With the simplification \MEquationItem{$h(x)$}{\MLSimplifyQuestion{30}{1/4+1/(2*sqrt(x))}{10}{x}{4}{512}{ISF3}}\\
we have the antiderivative \MEquationItem{$H(x)$}{\MLSimplifyQuestion{30}{x/4+sqrt(x)}{10}{x}{4}{544}{ISF3b}} for $x > 0$.\\
\begin{MHint}{Solution}
The function can be rewritten as %
$h(x)=\frac{1}{4} + \frac{1}{2 \cdot \sqrt{x}}$ 
which results in the antiderivative 
\[
H(x) \;=\; \frac{1}{4} \cdot x + \sqrt{x} %%
\]
for $x > 0$ (up to a constant).
\end{MHint}
}
\end{MExerciseItems}
\MInputHint{Enter, for example, $\sqrt{x}$ as \texttt{sqrt(x)}.}
\end{MExercise}


%Aufgabe durch die nachfolgende neue Aufgabe ersetzt (jgl):
%\begin{MExercise}
%Finden Sie zwei Polynome $p(x)$ und $q(x)$, sodass $F$ und $G$ mit 
%$F(x) = p(x) + x \cdot \ln(x)$ bzw. $G(x) = q(x) + x \cdot \ln(x)$
%Stammfunktionen von $f(x) = \ln(x)$ f"ur $x > 0$ sind.
%Dann ist $F(x) - G(x)$ ein Polynom vom Grad \MLParsedQuestion{4}{0}{4}{IG21}, 
%und es gilt $F'(x) = $\MLSimplifyQuestion{10}{ln(x)}{1}{x}{20}{0}{IGYY1}.
%%Loesungshinweis erstellt (jgl):
%\begin{MHint}{L"osung}
%Es ist $H$ mit $H(x) = r(x) + x \cdot \ln(x)$ eine Stammfunktion von 
%$f$ mit $f(x) = \ln(x)$, wenn $H'(x) = f(x)$ f"ur alle $x > 0$ und damit
%\[
%\ln(x) = f(x) = F'(x) = r'(x) + \ln(x) + x \cdot \frac{1}{x} %
% = r'(x) + 1 + \ln(x) %%
%\]
%gilt, woraus
%$0 = r'(x) + 1$ und damit $r'(x) = -1$ folgt. Somit ist $r(x) = -x + C$ f"ur 
%eine Zahl $C$. Beispielsweise sind $F$ und $G$ mit
%$F(x) = -x + 0 + x \cdot \ln(x)$ sowie
%$G(x) = -x + -14 + x \cdot \ln(x)$ Stammfunktionen von $f$.
%
%Es ist $F(x) - G(x) = 0 - (-14) = 14$, sodass $F - G$ ein Polynom vom Grad $0$
%ist. Weiter ist $F'(x) = -1 + \ln(x) + 1 = \ln(x)$ f"ur alle $x > 0$.
%\end{MHint}
%\end{MExercise}

%Aufgabe neu erstellt (jgl):
\begin{MExercise}
Consider a function $f$ with $f(x) = \frac{1}{x}$ for $x > 0$.
Moreover, the functions $F_1$ and $F_2$ with
$F_1(x) = \ln(7x)$ or $F_2(x) = \ln(x + 7)$ for $x > 0$ are given.
Calculate the derivatives of $F_1$ and $F_2$, and state whether $F_1$ and $F_2$ are antiderivatives of $f$:

We have
 \MEquationItem{$F_1'(x)$}{\MLSimplifyQuestion{30}{1/x}{10}{x}{4}{512}{IEx11ln}}
and 
 \MEquationItem{$F_2'(x)$}{\MLSimplifyQuestion{30}{1/(x+7)}{10}{x}{4}{512}{IEx12ln}}. 

Check the correct answer(s).
\par
\begin{MQuestionGroup}
\MLCheckbox{1}{M08Ex1120a} $F_1$ is an antiderivative of $f$.\ \\
\MLCheckbox{0}{M08Ex1120b} $F_2$ is an antiderivative of $f$.\ \\
\end{MQuestionGroup}
\par
\MGroupButton{Check entries}

\begin{MHint}{Solution}
The derivative of $F_1$ with $F_1(x) = \ln(7 x)$ for $x > 0$ is
$F_1'(x) = \frac{1}{7 x} \cdot 7 = \frac{1}{x} = f(x)$. Thus, $F_1$ is an antiderivative
of $f$.

The function $F_2$ with $F_2(x) = \ln(x + 7)$ for $x > 0$ has the derivative
$F_2'(x) = \frac{1}{x + 7} \cdot 1 = \frac{1}{x + 7} \neq \frac{1}{x}$ for 
all $x > 0$. Hence, $F_2$ is not an antiderivative of $f$.
%
%Optional koennte auf eine moegliche Umformung des Funktionsterms der ersten
%Funktion hingewiesen werden, um zu verdeutlichen, weshalb die Funktion dieselbe
%Ableitung wie die natuerliche Logarithmusfunktion hat (jgl):
%Zur Berechnung der Ableitung von $F_1$ soll noch darauf hingewiesen werden,
%dass der Funktionsterm mit den Rechenregeln zum Logarithmus zu 
%$F_1(x) = \ln(7 \cdot x) = \ln(7) + \ln(x)$ umgeformt werden kann, woraus sich
%dann ebenfalls die Ableitung $F_1'(x) = \frac{1}{x}$ ergibt.
\end{MHint}
\end{MExercise}

%Letztes Aufgabenelement darf kein MSimplifyQuestion sein, weil die Eingabebox 
%sonst unter dem Bildschirmrand landet (jgl, 2014).
\begin{MExercise}
%Aufgabentext ueberarbeitet (jgl):
Assume that $F$ is an antiderivative of $f$ with $f(x)=1 + x^2$, and $F$
has the function value $F(0) = 1$. $F(3)$ equals \MLParsedQuestion{10}{13}{10}{IG22}.
%Loesungshinweis erstellt (jgl):
\begin{MHint}{Solution}
By assumption, $F$ is an antiderivative of $f$ with $f(x) = 1 + x^2$. Thus, we have 
 $F'(x) = f(x) = 1 + x^2 = x^0 + x^2$, from which
\[
F(x) = \frac{1}{0 + 1} \cdot x^{0+1} + \frac{1}{2 + 1} \cdot x^{2 + 1} + C %
= x + \frac{1}{3} \cdot x^3 + C %%
\]
follows, for a number $C$. Moreover, we have $F(0) = 1$ which tells us
$1 = F(0) = 0 + \frac{1}{3} \cdot 0^3 + C = C$. This results in
$F(x) = \frac{1}{3} \cdot x^3 + x + 1$. Substituting $x = 3$ gives 
the required value $F(3) = \frac{1}{3} \cdot 3^3 + 3 + 1 = 13$.
\end{MHint}
\end{MExercise}

\end{MExercises}




%%%Abschnitt
\MSubsection{Definite Integral}\MLabel{M08A_Integral}

\begin{MIntro}
\MDeclareSiteUXID{VBKM08_Integral_Intro}

The derivative $f'(x_0)$ of a differentiable function $f$ describes 
how the function values change ''in the vicinity of'' a point $x_0$. 
If, for example, the derivative is positive, then the function $f$ is 
monotonically increasing. Geometrically, this means that the slope of a tangent 
line to the graph at the point $x_0$ is positive. The derivative provides 
a local view of the function at every point $x_0$. This way, a lot of 
detailed information can be collected.

Conversely, a ``global characteristic'' is obtained if a ``summary'' 
of the function is generated by summing up the weighted function values. In mathematics, 
this sum is called the integral or integral value of the function. Geometrically, 
this concept provides a way to calculate the area under the graph of a function. 
It was Bernhard Riemann who specified this approach and who gave his name to the 
so called \MEntry{Riemann integral}{integral (Riemann)}.
\end{MIntro}


\begin{MXContent}{Integral}{Integral}{STD}
\MDeclareSiteUXID{VBKM08_Integral_Content}

%Hier wird die Idee zu einer {\glqq}globalen Kenngr"o"se{\grqq} in der 
%Form des nach Riemann benannten Integrals vorgestellt.

%Das nach Riemann benannte Integral ordnet einer Funktion eine Zahl zu, die 
%Das Integral einer Funktion ist eine Zahl, die sich aus einer Summenbildung
%von gewichteten Funktionswerten ergibt. 
%F"ur das nach Riemann benannte Integral wird die Gewichtung "uber die 
%L"ange des Definitionsbereichs vorgenommen. 
The integral of a function $f$ with $f(x) \geq 0$ can be interpreted 
as the ``area under the graph'' of the function. In the so called
Riemann integral, the graph of the function is approximated by a step function,
and the values of this step function are summed up, weighted by the corresponding
interval length, i.e. the ``width of a step''. This approach is illustrated in the 
figure below.
%
\begin{center}
\MTikzAuto{%
\ifttm\else\begin{small}\fi
\begin{tikzpicture}[line width=1.5pt,scale=1.0, %
declare function={
  x0 = 1;
  x1 = 2;
  x2 = 4;
  x3 = 7;
  x4 = 8;
  IntX(\n) = array({1, 2, 4, 7, 8},\n);
  ZwWert(\n) = array({1, 1.5, 3, 5.5, 7.5},\n);
  IntAbs(\n) = array({0, 1, 2, 3, 1},\n);
  fkt(\x) = 2 + (\x - 2)*(\x - 3)*(\x - 7)/20;
}
] %[every node/.style={fill=white}] 
%,every node/.style={fill=white}] 
%Rechteck, farbig:
\begin{scope}[color=blue!20!white]
\foreach \n in {0, 1, 2, 3} 
\fill ({IntX(\n)}, 0.0) %
   rectangle ({IntX(\n) + IntAbs(\n+1)}, {fkt(ZwWert(\n+1))}); 
\end{scope}
%Koordinatenachsen:
\draw[->] (-0.6, 0) -- (9, 0) node[below left]{$x$}; %x-Achse
\draw[->] (0, -0.6) -- (0, 5) node[below left]{$y$}; %y-Achse
%Achsenbeschriftung:
\foreach \x/\xtext in {1/{1}, 2/{2}, 3/{}, 4/{4}, 5/{}, 6/{}, 7/{7}, 8/{8}} %
  \draw (\x, 0) -- ++(0, -0.1) %
   node[below] {$\xtext$}; 
\foreach \y in {1, 2, 3, 4} \draw (0, \y) -- ++(-0.1, 0) %
 node[left] {$\y$};
%\node[below left] at (0, 0) {$0$};
%Treppenfunktion:
\begin{scope}[line width=1pt,color=blue, fill=blue!50!white]
\foreach \n in {0, 1, 2, 3} 
\draw ({IntX(\n)}, {fkt(ZwWert(\n+1))}) -- ++({IntAbs(\n+1)}, 0); 
%\foreach \n/\zwert in {0/{1.5}, 1/{3}, 2/{5.5}, 3/{7.5}} 
%\node[above] at ({ZwWert(\n+1)}, {fkt(ZwWert(\n+1))}) {$f(\zwert)$}; %
\node[above left] at ({ZwWert(1)}, {fkt(ZwWert(1))}) {$f(1.5)$}; %
\node[above] at ({ZwWert(2)}, {fkt(ZwWert(2))}) {$f(3)$}; %
\node[above] at ({ZwWert(3)}, {fkt(ZwWert(3))}) {$f(5.5)$}; %
\node[above left] at ({ZwWert(4)}, {fkt(ZwWert(4))}) {$f(7.5)$}; %
\end{scope}
%Funktion:
\draw[domain=1.0:8.0,samples=120,color=\jccolorfkt] %
 plot (\x, {fkt(\x)});
%Beschriftung extern:
%\node at (-3,-1.5) {Zur Definition des Integrals: Funktion angen�hert durch %
% eine Treppenfunktion,};
%\node at (-3,-2) {unterteilt in vier Teilintervalle.};
\end{tikzpicture}
}

Definition of the Riemann integral. The function is approximated by a step function that is divided here into 
four subintervals. 
\ifttm\else\end{small}\fi
\end{center}
%
We can see here that the area under the graph of the function is initially approximated 
by rectangles. The length of the (horizontal) side of these rectangles is determined by 
the length of an interval on the $x$-axis, while the length of the second (vertical) 
side is determined by a function value $f(z_k)$ at the point $z_k$ in the 
corresponding $x$-interval. Then the areas of these rectangles are calculated, and finally 
summed up. The smaller the intervals on the $x$-axis, the more the sum calculated this way 
approaches the ``real'' value  of the area under the graph (the correct integral value of the function).

Formally, a sum $S_n$ of the form
%
\[
S_n = \sum_{k=0}^{n-1} f(z_k) \cdot \Delta(x_k) \qquad \text{with } %
 \Delta(x_k) = x_{k+1} - x_{k} %%
\]
%
is calculated. In the example considered here, the interval $[0\MIntvlSep 8]$ is divided into
four parts, where $x_0 = 1$, $x_1 = 2$, $x_2 = 4$, and $x_3 = 7$. If the areas of these four parts 
are summed according to the formula above, this results in
%
\begin{eqnarray*}
S_4 &=& f(z_0) \cdot (x_1-x_0) + f(z_1) \cdot (x_2-x_1) + f(z_2) \cdot (x_3-x_2) + f(z_3) \cdot (x_4-x_3)\\
 &=& f(z_0) \cdot (2-1) + f(z_1) \cdot (4-2) + f(z_2) \cdot (7-4) + f(z_3) \cdot (8-7)\\
 &=& f(z_0) \cdot 1 + f(z_1) \cdot 2 + f(z_2) \cdot 3 + f(z_3) \cdot 1 \MDFPeriod
\end{eqnarray*}
%
To get precise values for the area, it is obviously not sufficient to subdivide the interval into just a few subintervals. In general, it will be necessary to decrease the maximal length of the 
subintervals $x_{k+1} - x_k$ gradually, which in turn requires more summands $f(z_k) \cdot (x_{k+1} - x_k)$ 
to be calculated and added. Hence, the limit is considered as the maximal length of 
the subintervals tends to zero.

In principle, the approach discussed above can also be applied to functions with negative 
function values. We will explain in Section~\MRef{M08A_Anwendung} how the area is calculated then.
However, note that in the definition of the integral a few aspects have to be observed that go
beyond the scope of this course. Therefore, we refer to advanced textbooks for details concerning 
the assumptions in the definitions below.

\begin{MXInfo}{Integral}
Let a function $f: [a\MIntvlSep b] \rightarrow \R$ on a real interval $[a\MIntvlSep b]$
be given. If the number of subintervals is increased such that $x_{k+1} - x_k$
approaches $0$, then 
%
\begin{equation}\label{eq:integral}
\int_a^b f(x) \MDwSp x = \lim_{n \rightarrow \infty} S_n %
 = \lim_{n \rightarrow \infty} \sum_{k=0}^{n-1} f(z_k) \cdot (x_{k+1} - x_{k}) %
\quad \text{with } x_{k} \leq z_k \leq x_{k+1} %%
\end{equation}
%
is called the \MEntry{definite integral}{integral (definite)} of $f$ with the 
lower limit of integration $a$ and the upper limit of integration $b$ (if the limit
exists and does not depend on the respective partition). The function $f$ is then 
called \MEntry{integrable}{integrable}. In this context the function $f$ is also 
called the \MEntry{integrand}{integrand}.
\end{MXInfo}
%
%\begin{MXInfo}{Integral}
%Gegeben ist eine Funktion $f: [a, b] \rightarrow \R$ auf einem reellen Intervall
%$[a, b]$. 
%Das \MEntry{bestimmte Integral}{Integral!bestimmtes} von $f$ ist der Grenzwert der Summe
%
%\begin{equation}
%S_n := \sum_{k=0}^{n-1} f(x_k) \cdot \Delta(x_k) \qquad \text{mit } \Delta(x_k) := x_{k+1} - x_{k} %%
%\end{equation}
%
%f"ur $n$ gegen unendlich, wenn $\max_{1 \leq k \leq n} \Delta(x_k)$ gegen null 
%strebt.
%Dabei ist $a = x_0 < x_1 < x_2 < \ldots < x_n = b$ und $z_k$ eine Zahl 
%zwischen $x_{k-1}$ und $x_k$ f"ur $k \in \N$ mit $1 \leq k \leq n$. 
%Die Summen $S_n$ werden auch Riemannsummen zur Zerlegung $(x_0, \ldots, x_n)$ 
%und den Zwischenstellen $z_k$ genannt.

%Wenn dieser Grenzwert, das Integral, existiert, wird daf"ur
%\begin{equation}
%\int_a^b f(x) \MDwSp x = \lim_{n \rightarrow \infty} S_n %%
%\end{equation}
%geschrieben. Es hei"st $a$ Untergrenze, $b$ Obergrenze des Integrals "uber dem
%Integranden $f(x)$ mit der Integrationsvariablen $x$.
%\end{MXInfo}

In principle, this approach can also result in an indefinite value, 
i.e.\ the integral does not exist. However, advanced considerations show that the integral 
exists for every continuous function.
%
As an example, we calculate the integral of $f: [0\MIntvlSep 1] \rightarrow \R,\, x \Mmapsto x$,
where we focus on the calculation of the limit.

\begin{MExample}
Calculate the integral of $f: [0\MIntvlSep 1] \rightarrow \R,\, x \Mmapsto x$.

For this purpose, the interval $[0\MIntvlSep 1]$ is subdivided into subintervals $[x_{k}\MIntvlSep x_{k+1}]$
of equal length with $x_0 := 0$ and $x_k := x_{k-1} + \frac{1}{n}$. Thus, the length of a subinterval
is $\Delta(x_k) = x_{k+1} - x_{k} = \frac{1}{n}$.

Investigating the length of the interval for $n$ tending to infinity shows
that $\Delta(x_k)$ is getting smaller and approaches zero. Thus, the assumption
for the calculation of a definite integral is satisfied. 

Furthermore, for the values of $x_k$, we obtain from the length of an interval 
$x_k = \frac{k}{n}$. If we set $z_k = x_k$ for the intermediate points, we obtain 
$f(z_k) = f(x_k) = x_k = \frac{k}{n}$. 

Substituting these terms into the formula 
(\ref{eq:integral}) gives the equation
\begin{eqnarray*}
S_n &=& \sum_{k=0}^{n-1} f(x_k)\cdot\Delta(x_k) = \sum_{k=0}^{n-1} x_k \cdot \frac{1}{n} = \sum_{k=0}^{n-1} \frac{k}{n} \cdot \frac{1}{n} = \frac{1}{n^2} \cdot \sum_{k=0}^{n-1} k = \frac{1}{n^2} \cdot \sum_{k=1}^{n-1} k\\
 &=& \frac{1}{n^2} \frac{n (n-1)}{2} = \frac{1}{2} \cdot \frac{n-1}{n} = \frac{1}{2} \cdot \left(1 - \frac{1}{n}\right) \MDFPSpace , %
\end{eqnarray*}
where we used the formula $\sum_{k=1}^{n-1} k = \frac{1}{2}\,n(n-1)$ of C.\,F.~Gauss. 
And with $\displaystyle\lim_{n \rightarrow \infty} \frac{1}{n} = 0$, we find for the integral
\[
\int_0^1 x \MDwSp x = \lim_{n \rightarrow \infty} S_n = \frac{1}{2} \MDFPeriod %%
\]
\end{MExample}

The large class of integrable functions includes all polynomials, rational functions, 
trigonometric functions, exponential functions, logarithmic functions and their 
compositions.

%==========================
% Anfang Hauptsatz

To simplify the calculations, rules for integrating functions are required that are as easy as 
possible. An important result is provided by the so called 
\MEntry{fundamental theorem of calculus}{fundamental theorem (integral)}. It describes 
the relation between the antiderivative of a continuous function and its integral. 

\begin{MXInfo}{Fundamental Theorem of Calculus}
\MLabel{HSDDIR}
Let a continuous function $f: [a\MIntvlSep b] \rightarrow \R$ on a real interval
$[a\MIntvlSep b]$ be given. Then the function $f$ has an antiderivative, and for 
every antiderivative $F$ of $f$, we have
\[
\int_{a}^{b} f(x) \MDwSp x \;=\;  \left[{F(x)}\right]_a^b\;=\; F(b) - F(a) \MDFPeriod %% 
\]
\end{MXInfo}

As a simple example, we will calculate the definite integral of the function $f$ with
$f(x) = x^2$ between $a = 1$ and $b = 2$. Using the rules for the determination of  
antiderivatives and the fundamental theorem of calculus this problem can be solved very easily. 

\begin{MExample}
According to the table of antiderivatives given in the first section of this module, the 
function $f: [1\MIntvlSep 2] \rightarrow \R$ with $f(x) := x^2$ has the antiderivative $F$
with $F(x) = \frac{1}{3} x^{3} + C$ for a real number $C$. From the fundamental theorem, we have
\[
\int_1^2 x^2 \MDwSp x %
 = \left[\frac{1}{3} x^{3} \,+\, C \right]_1^2 %
 = \left(\frac{1}{3}\, 2^{3} \,+\,C \right) - \left(\frac{1}{3}\, 1^{3} \,+\, C \right) %
 = \frac{7}{3} \MDFPeriod %%
\]
The calculation shows that the constant is cancelled after substituting the 
lower and upper limits of integration such that, in practice, it can already be
``suppressed'' in taking the antiderivative for the definite integral, so 
for the calculation of the definite integral one can choose $C = 0$.
%
%Beispielsweise ist f"ur $n = 1$, also $f(x) = x$ dann
%$F$ mit $F(x) = \frac{1}{2} x^2$ eine Stammfunktion, sodass
%\[
%\int_a^b x \MDwSp x = \left[\frac{1}{2} x^2\right]_a^b %
% = \frac{1}{2} b^2 - \frac{1}{2} a^2 %%
%\]
%gilt.
\end{MExample}

The equation in the fundamental theorem also applies to every 
intermediate value $z \in [a\MIntvlSep b]$ such that all
function values $F(z)$ can be calculated from
\[
F(z) - F(a) = \int_a^z f(x) \MDwSp x \MDFPSpace , %%
\]
if the derivative $F' = f$ and a function value, for example the value $F(a)$, are known. 
One also says that the antiderivative $F$ is reconstructed from the derivative $F' = f$.

Application examples for the reconstruction of a function $F$ from its derivative 
$F' = f$ are discussed at the end of Section~\MRef{M08A_Anwendung}.

%In der Situation des Hauptsatzes ist f"ur jede Zahl $z$ zwischen $a$ und $b$ dann
%\[
%\int_{a}^{z} f(x) \MDwSp x = F(z) - F(a). %% 
%\]
%Folglich ergibt sich $F(z)$ gem"a"s
%\begin{equation}
%F(z) = F(a) + \int_{a}^{z} f(x) \MDwSp x = F(a) + \int_{a}^{z} F'(x) \MDwSp x. %% 
%\end{equation}
%aus dem Funktionswert und dem Integral "uber die Ableitung. Das hei"st, dass 
%die Funktion aus der Integration "uber die Ableitung, also die "Anderungsrate
%wiedergewonnen werden kann.

%Dies ist f"ur viele Anwendungen ein bemerkenswertes Hilfsmittel aus der 
%Mathematik. Beispielsweise werden in naturwissenschaftlichen oder technischen 
%Vorg"angen oft Ver"anderungen wie die Geschwindigkeit gemessen. Die Bewegung 
%kann dann daraus mittels Integration rekonstruiert werden.

% Ende Hauptsatz
%==========================

%Eine Absch"atzung des Integrals ergibt sich, wenn man einerseits Riemannsummen 
%$U_n$ mit {\glqq}Zwischenstellen{\grqq} $z_k$ berechnet, f"ur die $f(z_k)$ ein
%minimaler Wert der Funktion $f$ auf $[x_{k-1}, x_k]$ ist, und andererseits 
%solche Riemannsummen $O_n$ betrachtet, f"ur die $f(z_k)$ einen maximalen Wert 
%hat. (Sofern das Minimum bzw. Maximum nicht existiert, betrachtet man das 
%Infimum bzw. das Supremum.)

%Aufgrund dieser Konstruktion ist $U_n \leq O_n$. 
%Dementsprechend hei"sen die Riemannsummen $U_n$ Untersummen und $O_n$ Obersummen. 
%Wenn die Grenzwerte f"ur aller Unterteilungen 
%existieren und gleich sind, wenn also das Integral existiert, dann gilt die 
%Ungleichung
%\begin{equation}
%U_n \leq \int_a^b f(x) \MDwSp x \leq O_n
%\end{equation}
%Eine solche Ungleichung wird auch Absch"atzung genannt, da ein gesuchter Wert 
%mit einem anderen, m"oglichst einfach zu berechnenden Wert verglichen wird, 
%der zudem m"oglichst {\glqq}nahe{\grqq} am gesuchten Wert liegt.

%\begin{MExample}
%Im vorherigen Beispiel wurde zu $f(x) = x$ f"ur jedes $n \in \N$ und 
%$x_k = \frac{b \cdot k}{n}$, also 
%$\Delta(x_k) = x_k - x_{k-1} = \frac{b}{n}$ 
%eine Obersumme
%\[
%O_n = \sum_{k=1}^n \sup_{[x_{k-1}, x_k]} f(x) \cdot \Delta(x_k) %
% = \sum_{k=1}^n \frac{b \cdot k}{n} \cdot \Delta(x_k) %%
%\]
%konstruiert.

%Eine Untersumme $U_n$ ist f"ur diese Einteilung durch
%\[
%U_n = \sum_{k=1}^n \inf_{[x_{k-1}, x_k]} f(x) \cdot \Delta(x_k) %
% = \sum_{k=1}^n \frac{b \cdot (k-1)}{n} \cdot \Delta(x_k) %%
%\]
%gegeben.
%\end{MExample}

\end{MXContent}

%\begin{MXContent}{Rechenregeln f"ur Integrale}{Rechenregeln f"ur Integrale}{STD}
\begin{MXContent}{Calculation Rules}{Calculation Rules}{STD}
\MDeclareSiteUXID{VBKM08_Rechenregeln}

\begin{MXInfo}{Partition of the Interval of Integration}
Let $f: [a\MIntvlSep b] \rightarrow \R$ be an integrable function. Then for every 
number $z$ between $a$ and $b$, we have
%
$$
\int_a^b f(x) \MDwSp x = \int_a^z f(x) \MDwSp x + \int_z^b f(x) \MDwSp x \MDFPeriod%%
$$
\end{MXInfo}

With the definition 
%
$$
\int_d^c f(x) \MDwSp x := -\int_c^d f(x) \MDwSp x %%
$$
%
the rule above applies to all real numbers $z$ for which the two integrals on the 
right-hand side of the equation exist, even if $z$ does not lie between $a$ and $b$. 
Before we demonstrate this calculation with an example, we will examine 
the definition above in more detail. 


\begin{MXInfo}{Exchanging the Limits of Integration}
Let $f: [a\MIntvlSep b] \rightarrow \R$ be an integrable function. The integral 
of the function $f$ from $a$ to $b$ is calculated according to the rule
\[
\int_b^a f(x) \MDwSp x = -\int_a^b f(x) \MDwSp x \MDFPeriod %%
\]
\end{MXInfo}
The calculation rule described above is convenient when integrating functions 
that involve absolute values, or piecewise-defined functions.


\begin{MExample}
The integral of the function $f: [-4\MIntvlSep 6] \rightarrow \R, x \Mmapsto |x|$ is
\begin{eqnarray*}
\int_{-4}^{6} |x| \MDwSp x % 
 & = & \int_{-4}^{0} (-x) \MDwSp x + \int_{0}^{6} x \MDwSp x \\
 & = & \left[-\frac{1}{2} x^2\right]_{-4}^{0} + \left[\frac{1}{2} x^2\right]_{0}^{6} \\
 & = & (0 - (-8)) + (18 - 0) \\
 & = & 26 \MDFPeriod
\end{eqnarray*}
\end{MExample}

The integration over a sum of two functions can also be split up into two 
integrals:

\begin{MXInfo}{Sum and Constant Multiple Rule}
\MIndex{sum rule (integral)}
Let $f$ and $g$ be integrable functions on $[a\MIntvlSep b]$, and let $r$ be a real number. 
Then
\begin{equation}
\int_a^b (f(x) + g(x)) \MDwSp x = \int_a^b f(x) \MDwSp x + \int_a^b g(x) \MDwSp x \MDFPeriod %%
\end{equation}

For constant multiples of a function, we have
\begin{equation}
\int_a^b r \cdot f(x) \MDwSp x = r \cdot \int_a^b f(x) \MDwSp x \MDFPeriod %%
\end{equation}
\end{MXInfo}

\begin{MCOSHZusatz}
%Formulierungen im Zusatz ueberarbeitet und Aufgaben erstellt (jgl):

There is also a calculation rule for the integration of a product of two functions, which results from the product rule for the derivative.

\begin{MXInfo}{Integration by Parts}
Let $u$ and $v$ be differentiable functions on $[a\MIntvlSep b]$ with the 
continuous derivatives $u'$ and $v'$. For the integral of the function 
$u \cdot v'$, we have
%
$$
\int_a^b u(x) \cdot v'(x) \MDwSp x %
 = \left[u(x)\cdot v(x)\right]_a^b - \int_a^b u'(x) \cdot v(x) \MDwSp x \MDFPSpace, %%
$$
%
where $u'$ is the derivative of $u$ and $v$ is an antiderivative of $v'$. 
This calculation rule is called \MEntry{integration by parts or partial integration}{integration (by parts)}.
\end{MXInfo}

This rule is also illustrated by an example.

\begin{MExample}
Calculate the integral
%
$$
\int_{0}^{\pi} x \sin(x) \MDwSp x
$$
%
by means of integration by parts. For this purpose, we choose the functions 
$u$ and $v'$ such that
%
\begin{eqnarray*}
u(x) = x \quad \text{and} \quad v'(x) = \sin(x) \MDFPeriod %%
\end{eqnarray*}
%
Thus, we have
%
\begin{eqnarray*}
u'(x) = 1 \quad\text{and}\quad v(x) = -\cos x \MDFPeriod
\end{eqnarray*}
%
The required integral can now be calculated using integration by parts:
%
\begin{eqnarray*}
\int_{0}^{\pi} x \sin x \MDwSp x %
& = & \left[x \cdot (-\cos(x))\right]_0^\pi %
      - \int_{0}^{\pi} 1 \cdot (-\cos x) \MDwSp x \\
& = & \pi \cdot (-\cos(\pi)) - 0 + \int_{0}^{\pi} \cos(x) \MDwSp x \\
& = & \pi \cdot (-(-1)) + \left[\sin(x)\right]_0^\pi = \pi \MDFPeriod
\end{eqnarray*}
%
The assignments of the functions $u$ and $v'$ have to be appropriate. This becomes 
obvious if in this example the assignments of $u$ and  $v'$ are exchanged. 
Readers are invited to calculate this integral with exchanged assignments 
of $u$ and $v'$!
\end{MExample}

In the following two exercises we practice using the rule of integration by parts. 

\begin{MExercise}
Calculate the integral $\displaystyle I = \int_{1}^{4} x \cdot \MEU^{x} \MDwSp x$: 
$I = $\MLParsedQuestion{10}{3*e^4}{4}{M08ZEx01}. 

\begin{MHint}{Solution}
The integrand $f$ with $f(x) = x \cdot \MEU^{x}$ is a product of a polynomial 
$u$ with $u(x) = x$ and an exponential function. The derivative of $u$ is $u'(x) = 1$, i.e. 
a constant function. Moreover, the antiderivative of $v'$ with $v'(x) = \MEU^{x}$ is $v$ with $v(x) = \MEU^{x}$. 
Thus, integration by parts results in
\begin{eqnarray*}
 \int_{1}^{4} x \cdot \MEU^{x} \MDwSp x %
& = & \left[ x \cdot \MEU^{x} \right]_{1}^{4} - \int_{1}^{4} 1 \cdot \MEU^{x} \MDwSp x \\
& = & \left[ x \cdot \MEU^{x} \right]_{1}^{4} - \left[ \MEU^{x} \right]_{1}^{4} \\
& = & \left[ x \cdot \MEU^{x} - \MEU^{x} \right]_{1}^{4} \\
& = & \left[ (x - 1) \cdot \MEU^{x} \right]_{1}^{4} \\
& = & 3 \MEU^{4} \MDFPeriod %%
\end{eqnarray*}
\end{MHint}
\end{MExercise}

\begin{MExercise}
Calculate the integral $I = \displaystyle \int_{1}^{8} x \cdot \ln(x) \MDwSp x$: 
$I = $\MLParsedQuestion{25}{96*ln(2)-16+1/4}{4}{M08ZEx02}. 

\begin{MHint}{Solution}
The integrand $f$ with $f(x) = x \cdot \ln(x) = \ln(x) \cdot x$ for
$1 \leq x \leq 8$ is a product of a polynomial and a logarithmic function. The 
derivative of the logarithmic function $u$ with $u(x) = \ln(x)$ is $u'(x) = \frac{1}{x}$.
Thus, the function $u'$ is a ``simple'' rational function. Moreover, the antiderivative of 
the polynomial $v'$ with $v'(x) = x$ is known, namely $v$ with $v(x) = \frac{1}{2} \cdot x^2$. 

The product $u' \cdot v$ with $u'(x) \cdot v(x) = \frac{1}{x} \cdot \frac{1}{2} x^2 = \frac{1}{2} x$  for 
$1 \leq x \leq 8$ is a continuous function for which an antiderivative is known. Thus, 
the required integral can be calculated by means of integration by parts:
\begin{eqnarray*}
\int_{1}^{8} x \cdot \ln(x) \MDwSp x %
 = \int_{1}^{8} \ln(x) \cdot x \MDwSp x %
& = & \left[ \ln(x) \cdot \frac{1}{2} x^2 \right]_{1}^{8} %
      - \int_{1}^{8} \frac{1}{x} \cdot \frac{1}{2} x^2 \MDwSp x \\
& = & \left[ \ln(x) \cdot \frac{1}{2} x^2 \right]_{1}^{8} %
      - \int_{1}^{8} \frac{1}{2} x \MDwSp x \\
%& = & \left[ \ln(x) \cdot \frac{1}{2} x^2 \right]_{1}^{8} %
%     - \left[ \frac{1}{2} \cdot \frac{1}{2} x^2 \right]_{1}^{8} \\
& = & \left[ \frac{1}{2} x^2 \cdot \ln(x) - \frac{1}{4} x^2 \right]_{1}^{8} \\
& = & \left(32 \ln(8) - 16\right) - \left(1 \cdot \ln(1) - \frac{1}{4}\right) \\
& = & 96 \ln(2) - 16 + \frac{1}{4} \MDFPSpace , %%
\end{eqnarray*}
where $\ln(8) = \ln(2^3) = 3 \cdot \ln(2)$ and $\ln(1) = 0$ was used.
\end{MHint}
\end{MExercise}
\end{MCOSHZusatz}

\end{MXContent}


\begin{MXContent}{Properties of the Integral}{Properties of the Integral}{STD}
\MDeclareSiteUXID{VBKM08_Eigenschaften_Integral}

For odd functions $f: [-c\MIntvlSep c] \rightarrow \R$, the integral is zero. 
This shall be explained by means of the example of the function $f$ on $[-2\MIntvlSep 2]$
with $f(x) = x^3$ shown in the figure below. 
%\ifttm
%\MUGraphics{BildFlaechePolynomxdrei.png}{scale=0.4}{Ungerade Funktion $f(x) = x^3$ auf einem Intervall $[-2\MIntvlSep 2]$.}{}
%\else
\begin{center}
\MTikzAuto{%
\ifttm\else\begin{small}\fi
\begin{tikzpicture}[scale=0.8,line width=1.5pt]
%\clip(-3.7,-4.7) rectangle (3.7, 4.7);
%Koordinatenachsen:
\draw[->] (-3.6, 0) -- (3.8, 0) node[below left]{$x$}; %x-Achse
\draw[->] (0, -4.7) -- (0, 4.6) node[left]{$y$}; %y-Achse
%Achsenbeschriftung:
\foreach \x in {1, 2, 3} \draw (\x, 0) -- ++(0, -0.1) node[below] {$\x$}; 
\foreach \x in {1, 2, 3} \draw (-\x, 0) -- ++(0, 0.1) node[above] {$-\x$}; 
\foreach \y in {1, 2, 3, 4} \draw (0, \y) -- ++(-0.1, 0) node[left] {$\y$};
\foreach \y in {1, 2, 3, 4} \draw (0, -\y) -- ++(-0.1, 0) node[left] {$-\y$};
\node[below left] at (0, 0) {$0$};
%Funktionsgraph:
\draw[domain=-2:2,samples=200,color=\jccolorfkt, fill=\jccolorfktarea] %
  plot (\x, {1/2*pow(\x,3)}) -- (2, 4) -- (2, 0) -- (-2, 0) -- (-2, -4); 
%Beschriftung:
\draw[color=\jccolorfkt] (1.4, 0.7) -- +(0.4, 0);
\draw[color=\jccolorfkt] (1.6, 0.5) -- +(0, 0.4);
\draw[color=\jccolorfkt] (-1.4, -0.7) -- +(-0.4, 0);
\end{tikzpicture}
\ifttm\else\end{small}\fi
}
\end{center}
%\fi
The graph of $f$ is subdivided into two parts, namely in a part between $-2$ and $0$
and a part between $0$ and $2$, and the two regions bounded by the graph and 
the $x$-axis are investigated. The two regions can be transferred into each other
by a point reflection (central inversion). Both regions are equal in size. However, 
forming the Riemann sum of both regions, one finds that the area of the region below the $x$-axis 
takes a negative value. Thus, if the integrals for the two regions are added to calculate the 
integral over the interval from $-2$ to $2$, the area of the region above the positive $x$-axis is positive. 
The area of the region below the negative $x$-axis is equal in size but has a negative sign. Thus, the sum 
of the two areas equals zero. Thus, for odd functions $f$, we have the following rule:
$$
\int_{-c}^c f(x) \MDwSp x = 0 \MDFPeriod
$$
For an even function $g: [-c\MIntvlSep c] \rightarrow \R$, the graph is symmetric 
with respect to the $y$-axis.
\begin{center}
\MTikzAuto{%
\ifttm\else\begin{small}\fi
\begin{tikzpicture}[scale=0.8,line width=1.5pt]
%\clip(-3.7,-4.7) rectangle (3.7, 4.7);
%Koordinatenachsen:
\draw[->] (-3.6, 0) -- (3.8, 0) node[below left]{$x$}; %x-Achse
\draw[->] (0, -0.7) -- (0, 4.6) node[left]{$y$}; %y-Achse
%Achsenbeschriftung:
\foreach \x in {1, 2, 3} \draw (\x, 0) -- ++(0, -0.1) node[below] {$\x$}; 
\foreach \x in {1, 2, 3} \draw (-\x, 0) -- ++(0, -0.1) node[below] {$\x$};
%
\foreach \y in {1, 2, 3, 4} \draw (0, \y) -- ++(-0.1, 0) node[left] {$\y$};
\node[below left] at (0, 0) {$0$};
%Funktionsgraph:
\draw[domain=-2:2,samples=200,color=\jccolorfkt, fill=\jccolorfktarea] %
  plot (\x, {abs(pow(\x,2))}) -- (2, 4) -- (2, 0) -- (-2, 0) -- (-2, 4); 
%Beschriftung:
\draw[color=\jccolorfkt] (1.4, 0.7) -- +(0.4, 0);
\draw[color=\jccolorfkt] (1.6, 0.5) -- +(0, 0.4);
\draw[color=\jccolorfkt] (-1.4, 0.7) -- +(-0.4, 0);
\draw[color=\jccolorfkt] (-1.6, 0.5) -- +(0, 0.4);
\end{tikzpicture}
\ifttm\else\end{small}\fi
}
\end{center}
The region between the graph of $g$ and the $x$-axis is here symmetric with respect to 
the $y$-axis. Thus, the region to the left of the $y$-axis is the mirror image of 
the region to the right. The sum of the areas of the two regions is
$$
\int_{-c}^c g(x) \MDwSp x = 2 \cdot \int_{0}^c g(x) \MDwSp x \MDFPeriod
$$

This rule for the integral applies to every integrable function $g$ that is even, 
even if the function takes negative values. Due to the calculation rule above, it is then 
sufficient to calculate the integral for non-negative values of $x$ with the lower limit 
$0$ and the upper limit $c$.

Often, the calculation of an integral is easier if the integrand 
is first transformed into a known form. Examples of possible transformations 
shall be considered below. In the first example, power functions 
are investigated.

\begin{MExample}
Calculate the integral
\[
\int_{1}^{4} (x - 2) \cdot \sqrt{x} \MDwSp x \MDFPeriod %%
\]
First, the integrand is transformed to simplify the calculation:
\[
 (x - 2) \cdot \sqrt{x} = x \sqrt{x} - 2 \sqrt{x} %
 = x^{\frac{3}{2}} - 2 x^{\frac{1}{2}} \MDFPeriod %%
\]
Now the integral can be calculated more easily:
\begin{eqnarray*}
\int_{1}^{4} (x - 2) \cdot \sqrt{x} \MDwSp x %%
& = &
\int_{1}^{4} \left(x^{\frac{3}{2}} - 2 x^{\frac{1}{2}}\right) \MD x %%
  =  
\left[\frac{2}{5} x^{\frac{5}{2}} - \frac{4}{3} x^{\frac{3}{2}}\right]_{1}^{4} \\
& = &
\left(\frac{2}{5} \left(\sqrt{4}\right)^5 - \frac{4}{3} \left(\sqrt{4}\right)^3 \right) %
 - \left(\frac{2}{5} \cdot 1 - \frac{4}{3} \cdot 1 \right) \\
& = &
\left(\frac{64}{5} - \frac{32}{3}\right) - \left(\frac{2}{5} - \frac{4}{3}\right) \\
& = & \frac{62}{5} - \frac{28}{3} \\
& = & 3 + \frac{1}{15} \MDFPeriod %%
\end{eqnarray*}
\end{MExample}

%Im n"achsten Beispiel wird die Umformung eines Integranden mit trigonometrischen 
%Funktionen durchgef�hrt.

The next example demonstrates a transformation of an integrand involving exponential functions.


\begin{MExample}
Calculate the integral 
\[
\int_{-2}^{3} \frac{8 \MEU^{3 + x} - 12 \MEU^{2 x}}{2 \MEU^{x}} \MDwSp x \MDFPeriod %%
\]
According to the calculation rule for exponential functions one obtains
\[
\frac{8 \MEU^{3 + x} - 12 \MEU^{2 x}}{2 \MEU^{x}} %
= \frac{8 \MEU^{3 + x}}{2 \MEU^{x}} - \frac{12 \MEU^{2 x}}{2 \MEU^{x}} %
= 4 \MEU^{3 + x - x} - 6 \MEU^{2 x - x} %
= 4 \MEU^{3} - 6 \MEU^{x} \MDFPSpace, %%
\]
such that the integral can finally be calculated easily:
\begin{eqnarray*}
\int_{-2}^{3} \frac{8 \MEU^{3 + x} - 12 \MEU^{2 x}}{2 \MEU^{x}} \MDwSp x %
= 
\int_{-2}^{3} \left( 4 \MEU^{3} - 6 \MEU^{x} \right) \MD x %
& = & 
\left[4 \MEU^{3} \cdot x - 6 \MEU^{x} \right]_{-2}^{3} \\
& = & 
\left(4 \MEU^{3} \cdot 3 - 6 \MEU^{3} \right) %
 - \left(4 \MEU^{3} \cdot (-2) - 6 \MEU^{-2} \right) \\
& = &
14 \MEU^{3} + \frac{6}{\MEU^{2}} \MDFPeriod %%
\end{eqnarray*}
\end{MExample}

Consider a rational function. If the degree of the numerator polynomial is greater 
or equal to the degree of the denominator polynomial, a 
polynomial long division is carried out first (see Module~\MRef{VBKM06}). Depending on the 
situation, further transformations (e.g. partial fraction decomposition) may be 
appropriate. These can be found in advanced textbooks and formularies. In 
the following example, a polynomial long division is carried out to integrate 
a rational function.

\begin{MExample}
Calculate the integral
\[
\int_{-1}^{1} \frac{4 x^2 - x + 4}{x^2 + 1} \MDwSp x \MDFPeriod%%
\]
First, we transform the integrand using polynomial long division:
\[
4 x^2 - x + 4 = (x^2 + 1) \cdot 4 - x %%
\]
so
\[
\frac{4 x^2 - x + 4}{x^2 + 1} = 4 - \frac{x}{x^2 + 1} \MDFPeriod %%
\]
Thus, we have
\begin{eqnarray*}
\int_{-1}^{1} \frac{4 x^2 - x + 4}{x^2 + 1} \MDwSp x %%
& = & 
\int_{-1}^{1} \left(4 - \frac{x}{x^2 + 1}\right) \MD x \\
& = & 
\int_{-1}^{1} 4 \MDwSp x - \int_{-1}^{1} \frac{x}{x^2 + 1} \MDwSp x %%
= \left[4 x\right]_{-1}^{1} - 0 %
= 8 \MDFPeriod
\end{eqnarray*}
The integrand in the second integral is an odd function and centrally symmetric on the interval 
$[-1\MIntvlSep 1]$, so the second integral equals zero. 

Here, a specific example was given to provide a first impression of the integration 
of rational functions. In advanced mathematics lectures and in the literature this 
approach is described in general terms.
\end{MExample}

\end{MXContent}


%%%Uebungen zum Abschnitt zum bestimmten Integral:
\begin{MExercises}
\MDeclareSiteUXID{VBKM08_BestimmtesIntegral_Exercises}

%Einfuehrende Erlaeuterung zur ersten Aufgabe erstellt (jgl).
The first exercise picks up the idea (from the definition of the integral) of calculating the value of the
integral based on an appropriate partition. For this purpose, in this first exercise more general shapes are used besides the standard rectangles, such as 
triangles.

\begin{MExercise}
%Aufgabentext ueberarbeitet (jgl).
Calculate the integral $\displaystyle \int_{-3}^{4} f(x) \MDwSp x$ of the function 
$f:[-3\MIntvlSep 4] \rightarrow \R$ with the graph shown in the figure below applying methods 
of the elementary geometry, i.e. by subdividing the ``area below the graph'' into elementary 
geometrical shapes such as triangles or rectangles that either lie above or below the $x$-axis. 
In this case, you can subsequently calculate the single areas using the formulas for triangles 
and rectangles.

\begin{center}
\MTikzAuto{%
\ifttm\else\begin{small}\fi
\begin{tikzpicture}[line width=1.5pt,scale=0.8, %
declare function={
  u0 = -3;
  u1 = 0;
  u2 = 2;
  u3 = 4;
  fkt1(\x) = \x + 1; % $-3 \leq x \leq 0$
  fkt2(\x) = -\x + 1 - abs(-\x + 1) + 1; % $0 \leq x \leq 2$
  fkt3(\x) = (\x - 3 + abs(\x - 3))/2 - 1; % $2 \leq x \leq 4$
%Auch erlaubt:
%  fkt3(\z) = (\z - 3 + abs(\z - 3))/2 - 1; % $2 \leq x \leq 4$
}
] %[every node/.style={fill=white}] 
%Koordinatenachsen:
\draw[->] ({u0-1}, 0) -- ({u3+1}, 0) node[below left]{$x$}; %x-Achse
%Berechnung der Werte: Minimalstelle ist u0, Maximalstelle ist u1:
\draw[->] (0, {fkt1(u0)-1}) -- (0, {fkt1(u1)+1}) node[below left]{$y$}; %y-Achse
%Achsenbeschriftung:
\foreach \x in {-3, -2, -1, 1, 2, 3, 4} \draw (\x, 0) -- ++(0, -0.1) %
 node[below] {$\x$}; 
\foreach \y in {-2, -1, 1} \draw (0, \y) -- ++(-0.1, 0) %
 node[left] {$\y$};
\node[below left] at (0, 0) {$0$};
%Funktion:
\draw[domain=u0:u1,samples=120,color=\jccolorfkt] %
 plot (\x, {fkt1(\x)});
\draw[domain=u1:u2,samples=120,color=\jccolorfkt] %
 plot (\x, {fkt2(\x)});
\draw[domain=u2:u3,samples=120,color=\jccolorfkt] %
 plot (\x, {fkt3(\x)});
%Beschriftung:
\node[right] at (1.1, {fkt2(1)}) {$f(x)$};
%
\end{tikzpicture}
\ifttm\else\end{small}\fi
}
\end{center}
Tthe value of the integral is the sum of the areas of regions lying above the 
$x$-axis minus the sum of the areas of regions lying below the $x$-axis. In this sense, 
the value of the integral can be considered as the sum of signed areas.

The value of the integral 
$\displaystyle \int_{-3}^{4} f(x) \MDwSp x$
is \MLParsedQuestion{10}{-2}{4}{IG23}.

\begin{MHint}{Solution}
The area is subdivided by vertical lines at $x_0 = -3$, $x_1 = -1$, $x_2 = 0$, $x_3 = 1$,
$x_4 = \frac{3}{2}$, $x_5 = 2$, $x_6 = 3$, and $x_7 = 4$ into regions that are bounded each 
by the graph of $f$, the $x$-axis and the lines at $x_{k-1}$ and $x_k$ for $1 \leq k \leq 7$.

Then the corresponding signed areas are summed up resulting in the value of the integral:
\[
\int_{-3}^{4} f(x) \MDwSp x %
= -\frac{2 \cdot 2}{2} + \frac{1 \cdot 1}{2} + 1 \cdot 1 %
 + \frac{\frac{1}{2} \cdot 1}{2}
 - \frac{\frac{1}{2} \cdot 1}{2}
 - 1 \cdot 1 - \frac{1 \cdot 1}{2} %
= -2\MDFPeriod %%
\]
Of course, the area can also be subdivided into other regions. If, for example, 
the area is subdivided by vertical lines at $z_0 = -3$, $z_1 = -1$, $z_2 = \frac{3}{2}$, and $z_3 = 4$
into three regions, the area of the region $B_2$ between the lines at $z_1$ and $z_2$ equals 
the area of the region $B_3$ between the lines at $z_2$ and $z_3$. However, the signs of the areas of $B_2$ and $B_3$
are opposite since $B_2$ lies above the $x$-axis and $B_3$ below.
\begin{center}
\MTikzAuto{%
\ifttm\else\begin{small}\fi
\begin{tikzpicture}[line width=1.5pt,scale=0.8, %
declare function={
  u0 = -3;
  u1 = 0;
  u2 = 2;
  u3 = 4;
  fkt1(\x) = \x + 1; % $-3 \leq x \leq 0$
  fkt2(\x) = -\x + 1 - abs(-\x + 1) + 1; % $0 \leq x \leq 2$
  fkt3(\x) = (\x - 3 + abs(\x - 3))/2 - 1; % $2 \leq x \leq 4$
%Auch erlaubt:
%  fkt3(\z) = (\z - 3 + abs(\z - 3))/2 - 1; % $2 \leq x \leq 4$
}
] %[every node/.style={fill=white}] 
%Koordinatenachsen:
\draw[->] ({u0-1}, 0) -- ({u3+1}, 0) node[below left]{$x$}; %x-Achse
%Berechnung der Werte: Minimalstelle ist u0, Maximalstelle ist u1:
\draw[->] (0, {fkt1(u0)-1}) -- (0, {fkt1(u1)+1}) node[below left]{$y$}; %y-Achse
%Achsenbeschriftung:
\foreach \x in {-3, -2, -1, 1, 2, 3, 4} \draw (\x, 0) -- ++(0, -0.1);
\node[above] at (-3, 0) {$-3$}; 
\node[below] at (1, 0) {$1$}; 
\node[below] at (4, 0) {$4$}; 
\foreach \y in {-2, -1, 1} \draw (0, \y) -- ++(-0.1, 0) %
 node[left] {$y$};
\node[below left] at (0, 0) {$0$};
%Begrenzung:
\draw[style=dotted] (-3,-2) -- (-3,0);
%Funktion:
\draw[domain=u0:u1,samples=120,color=\jccolorfkt] %
 plot (\x, {fkt1(\x)});
\draw[domain=u1:u2,samples=120,color=\jccolorfkt] %
 plot (\x, {fkt2(\x)});
\draw[domain=u2:u3,samples=120,color=\jccolorfkt] %
 plot (\x, {fkt3(\x)});
%Beschriftung:
\node[right] at (1.1, {fkt2(1)}) {$f(x)$};
\node at (-2.5, -0.75) {$B_1$};
\node at (0.5, 0.5) {$B_2$};
\node at (2.6, -0.5) {$B_3$};
%
\end{tikzpicture}
\ifttm\else\end{small}\fi
}
\end{center}
Thus, the value of the integral equals the negative area of the region $B_1$ between 
the lines $z_0$ and $z_1$, which lies below the $x$-axis.
\end{MHint}
\end{MExercise}


\begin{MExercise}
Calculate the following integrals:
\begin{MExerciseItems}
\item{\MEquationItem{$\displaystyle \int_{0}^{5} 3 \MDwSp x$}{\MLParsedQuestion{10}{15}{4}{IG24}}}
\item{\MEquationItem{$\displaystyle \int_{0}^{5} -4 \MDwSp x$}{\MLParsedQuestion{10}{-20}{4}{IG25}}}
\item{\MEquationItem{$\displaystyle \int_{0}^{4} 2 x \MDwSp x$}{\MLParsedQuestion{10}{16}{4}{IG26}}}
\item{\MEquationItem{$\displaystyle \int_{0}^{4} \left(4 - x\right) \MD x$}{\MLParsedQuestion{10}{8}{4}{IG27}}}
\end{MExerciseItems}
%Loesungshinweise erstellt (jgl):
\begin{MHint}{Solution}
>Using the fundamental theorem of calculus, we obtain
\begin{enumerate}
\item $\displaystyle \int_{0}^{5} 3 \MDwSp x %
= \left[ 3 x \right]_0^5 = 15\MDFPSpace$
%
\item $\displaystyle \int_{0}^{5} -4 \MDwSp x %
= \left[ -4 x \right]_0^5 = -20\MDFPSpace$
%
\item $\displaystyle \int_{0}^{4} 2 x \MDwSp x %
= \left[ x^2 \right]_0^4 = 16\MDFPSpace$
%
\item $\displaystyle \int_{0}^{4} \left(4 - x\right) \MD x %
= \left[ 4 x - \frac{1}{2} \cdot x^2 \right]_0^4 = 8\MDFPSpace$
%
\end{enumerate}
\end{MHint}
\end{MExercise}


\begin{MExercise}
The value of the integral
$\displaystyle \int_{-\pi}^{\pi} \left(5 x^3 - 4 \sin(x)\right) \MD x$
is \MLParsedQuestion{10}{0}{4}{IG28}.
%Loesungshinweise erstellt (jgl):
\begin{MHint}{Solution}
>From the fact that the integrand is odd and the interval of integration is 
symmetric with respect to the origin $(0,0)$, we can deduce that the value of the 
integral is $0$. Alternatively, we can calculate the integral using the fundamental theorem 
of calculus:
\[
\int_{-\pi}^{\pi} \left(5 x^3 - 4 \sin(x)\right) \MD x %
= \left[ \frac{5}{4} \cdot x^4 + 4 \cos(x) \right]_{-\pi}^{\pi} %
= \left[ \frac{5}{4} \cdot x^4 + 4 \cos(x) \right]_{-\pi}^{\pi} %
= 0\MDFPeriod %%
\]
\end{MHint}
\end{MExercise}


\begin{MExercise}
Calculate a real number $z$ such that the value of the integral 
\[
\int_{0}^{2} \left(x^2 + z\cdot x+1\right) \MD x %
\]
equals zero. %
The required value of $z$ 
is \MEquationItem{$z$}{\MLParsedQuestion{10}{-7/3}{4}{ICONSTFIND}}.

\begin{MHint}{Solution}
If we consider $z$ as an unknown constant, we have
$$
\int_{0}^{2} \left(x^2 + z\cdot x+1\right) \MD x %
\;=\; \left[{\frac13x^3+\frac12 z x^2+x}\right]_0^2 %
\;=\; \frac83+2z+2 \MDFPeriod %%
$$
Hence, $z=-\frac{14}{6} = -\frac{7}{3}$ is the required value.
\end{MHint}
\end{MExercise}

\begin{MExercise}
Calculate the following integrals:
\begin{MExerciseItems}
\item{\MEquationItem{$\displaystyle \int_{-3}^{2} \left(1 + 6 x^2 - 4 x\right) \MD x$}{\MLParsedQuestion{30}{85}{4}{IG29}}}
\item{\MEquationItem{$\displaystyle \int_{1}^{9} \frac{5}{\sqrt{4 x}} \MDwSp x$}{\MLParsedQuestion{30}{10}{4}{IG30}}}
\end{MExerciseItems}

\begin{MHint}{Solution}
The integrand $f$ with $f(x) = 1 + 6 x^2 - 4x = 6 x^2 - 4 x + 1$ is a polynomial. 
Thus, $F$ with $F(x) = 2 x^3 - 2 x^2 + x$ is an antiderivative of $f$. From the 
fundamental theorem, we have
\[
 \int_{0}^{1} \left(1 + 6 x^2 - 4 x\right) \MD x %
 = \left[ 2 x^3 - 2 x^2 + x \right]_{-3}^{2} %
 = 2 (8 - 4) + 2 - (2 (-27 - 9) - 3) = 85 \MDFPeriod %%
\]
In the second part of the exercise, the integrand $f(x) = \frac{5}{\sqrt{4 x}}= \frac{5}{2} x^{-1/2}$
is a product of a root function and a constant factor. Thus, $F$ with 
$F(x) = 5 x^{1/2} = 5 \sqrt{x}$ is an antiderivative of $f$. From the fundamental theorem,
we have
\[
 \int_{1}^{9} \frac{5}{\sqrt{4 x}} \MDwSp x %
 = \left[ 5 \sqrt{x} \right]_{1}^{9} %
 = 5 (3 - 1) = 10 \MDFPeriod %%
\]
\end{MHint}
\end{MExercise}

\begin{MExercise}
The value of the integral
$\displaystyle \int_{-24}^{-6} \frac{1}{2 x} \MDwSp x$
is \MLParsedQuestion{10}{-ln(2)}{4}{ILN1}.

\begin{MHint}{Solution}
The integrand $f$ with $f(x) = \frac{1}{2 x} = \frac{1}{2} \cdot \frac{1}{x}$ 
for $x < 0$ has an antiderivative $F$ with $F(x) = \frac{1}{2} \ln|x|$.
>From the fundamental theorem, we have
\[
 \int_{-24}^{-6} \frac{1}{2 x} \MDwSp x %
 = \left[ \frac{1}{2} \ln|x| \right]_{-24}^{-6} %
 = \frac{1}{2} \left(\ln|-6| - \ln|-24|\right) %
 = \frac{1}{2} \ln\left(\frac{6}{24}\right) %
 = \frac{1}{2} \ln\left(2^{-2}\right) %
 = -\ln(2) \MDFPeriod %
\]
\end{MHint}
\end{MExercise}

\begin{MExercise}
Calculate the following integrals
\begin{MExerciseItems}
\item{\MEquationItem{$\displaystyle \int_{0}^{3} (2 x - 1) \MDwSp x$}{\MLParsedQuestion{30}{6}{4}{IG31}}}
\item{\MEquationItem{$\displaystyle \int_{-3}^{0} (1 - 2 x) \MDwSp x$}{\MLParsedQuestion{30}{12}{4}{IG32}}}
\end{MExerciseItems}

\begin{MHint}{Solution}
The integrand $f$ with $f(x) = 2 x - 1$ is a polynomial. $F$ with $F(x) = x^2 - x$ is an antiderivative of $f$. From the fundamental theorem, we have
\[
 \int_{0}^{3} (2 x - 1) \MDwSp x %
 = \left[ x^2 - x \right]_{0}^{3} %
 = 9 - 3 - 0%
 = 6 \MDFPeriod %
\]
In the second part of the exercise, the integrand $f$ with $f(x) = 1 - 2 x$ is 
also a polynomial. Thus, $F$ with $F(x) = x - x^2$ is an antiderivative of $f$. From the fundamental 
theorem, we have
\[
 \int_{-3}^{0} (1 - 2 x) \MDwSp x %
 = \left[ x - x^2 \right]_{-3}^{0} %
 = 0 - (-3 - 9)%
 = 12 \MDFPeriod %
\]
\end{MHint}
\end{MExercise}

\begin{MExercise}
Calculate the integral

\MEquationItem{$\displaystyle \int_{\pi}^{3 \pi} \left(\frac{3 \pi}{x^2} - 4 \sin(x)\right) \MD x$}{\MLParsedQuestion{30}{2}{4}{IG33}}.
% \item{\MEquationItem{$\displaystyle \int_{-3}^{3} \left(|x| - |x + 1|\right) \MD x$}{\MLParsedQuestion{30}{-1}{4}{IG34}}.}

\begin{MHint}{Solution}
The integrand $f$ with $f(x) = \frac{3 \pi}{x^2} - 4 \sin(x)$ has an antiderivative $F$ with
$F(x) = -\frac{3 \pi}{x} + 4 \cos(x)$. From the fundamental 
theorem, we have
\[
 \int_{\pi}^{3 \pi} \left(\frac{3 \pi}{x^2} - 4 \sin(x)\right) \MD x %
 = \left[ -\frac{3 \pi}{x} + 4 \cos(x) \right]_{\pi}^{3 \pi} %
 = ( -1 - 4 ) - ( -3 - 4 )%
 = 2 \MDFPeriod %
\]
Remark: The integral over a period $2 \pi$ of the two periodic functions $\sin$ and $\cos$ equals zero. 
However, for other periodic functions, as for example $f(x) = \sin(x) + \frac{1}{2}$, 
the value of the integral over a period can differ from zero. 
\end{MHint}
\end{MExercise}

\end{MExercises}




%%%Abschnitt
\MSubsection{Applications}\MLabel{M08A_Anwendung}

\begin{MIntro}
\MDeclareSiteUXID{VBKM08_Anwendungen_Intro}
The integral calculus has many varied applications, in particular in science 
and engineering. Here, the calculation of areas of regions with boundaries described by 
mathematical functions shall be studied first. This 
application is not purely mathematical, but is used 
in the determination of centres of mass, rotational properties of rigid bodies or 
the bending properties of beams or girders. At the end of this section 
a few more physical applications are considered.
\end{MIntro}

\begin{MXContent}{Calculation of Areas}{Calculation of Areas}{STD}
\MDeclareSiteUXID{VBKM08_Anwendungen_Flaechenberechnung}

A first application of the integral calculus is the calculation 
of \MEntry{areas}{area (integral)} of regions with boundaries that can be described by 
mathematical functions. For illustration purposes, the figure below (left part) shows 
the function $f(x) = \frac{1}{2} x^3$ on the interval $[-2\MIntvlSep 2]$.
The goal is to calculate the area that is bounded by the graph of the 
function and the $x$-axis. From our previous investigations, we know 
that the integral over this odd function in the limits from $-2$ to 
$2$ equals zero since the area of the left region equals the area of the right region 
but during integration they are assigned different signs. Thus, the value of the 
integral does not match the actual value of the area. However, if 
the ``negative'' area is reflected across the $x$-axis, i.e. 
the function is assigned a positive sign (right part of the figure). Now 
the area is determined correctly by the integral. In mathematical terms, 
it is not the integral of the function $f$ that is calculated but the 
integral of its absolute value $\left|f\right|$.

\begin{center}
\MTikzAuto{%
\ifttm\else\begin{small}\fi
\begin{tikzpicture}[scale=0.8,line width=1.5pt]
\begin{scope}[xshift=-8.5cm]
%\clip(-3.7,-4.7) rectangle (3.7, 4.7);
%Koordinatenachsen:
\draw[->] (-3.6, 0) -- (3.8, 0) node[below left]{$x$}; %x-Achse
\draw[->] (0, -4.6) -- (0, 4.8) node[below left]{$y$}; %y-Achse
%Achsenbeschriftung:
\foreach \x in {1, 2, 3} \draw (\x, 0) -- ++(0, -0.1) node[below] {$\x$}; 
\foreach \x in {1, 2, 3} \draw (-\x, 0) -- ++(0, 0.1) node[above] {$-\x$}; 
\foreach \y in {1, 2, 3, 4} \draw (0, \y) -- ++(-0.1, 0) node[left] {$\y$};
\foreach \y in {1, 2, 3, 4} \draw (0, -\y) -- ++(-0.1, 0) node[left] {$-\y$};
\node[below left] at (0, 0) {$0$};
%Funktionsgraph:
\draw[domain=-2:2,samples=120,color=\jccolorfkt, fill=\jccolorfktarea] %
  plot (\x, {1/2*pow(\x,3)}) -- (2, 4) -- (2, 0) -- (-2, 0) -- (-2, -4); 
%Beschriftung:
\draw[color=\jccolorfkt] (1.4, 0.7) -- +(0.4, 0);
\draw[color=\jccolorfkt] (1.6, 0.5) -- +(0, 0.4);
\draw[color=\jccolorfkt] (-1.4, -0.7) -- +(-0.4, 0);
\end{scope}
\begin{scope}{xshift=8.5cm}
%Quelle: BildFlaechePolynomBetragxdrei.tex (Modul Integrationstechniken, Liedtke)
%\clip(-3.7,-4.7) rectangle (3.7, 4.7);
%Koordinatenachsen:
\draw[->] (-3.6, 0) -- (3.8, 0) node[below left]{$x$}; %x-Achse
\draw[->] (0, -4.6) -- (0, 4.8) node[below left]{$y$}; %y-Achse
%Achsenbeschriftung:
\foreach \x in {1, 2, 3} \draw (\x, 0) -- ++(0, -0.1) node[below] {$\x$}; 
\foreach \x in {1, 2, 3} \draw (-\x, 0) -- ++(0, -0.1);
\node[below] at (-3, -0.1) {$-3$};
%\draw (-1, 0) -- ++(0, -0.1) node[below,fill=white] {$-1$}; 
\foreach \y in {1, 2, 3, 4} \draw (0, \y) -- ++(-0.1, 0) node[left] {$\y$};
\foreach \y in {1, 2, 3, 4} \draw (0, -\y) -- ++(-0.1, 0) node[left] {$-\y$};
\node[below left] at (0, 0) {$0$};
%Funktionsgraph:
\draw[domain=-2:2,samples=200,color=\jccolorfkt, fill=\jccolorfktarea] %
  plot (\x, {1/2*abs(pow(\x,3))}) -- (2, 4) -- (2, 0) -- (-2, 0) -- (-2, 4); 
\draw[domain=-2:0,samples=120,color=green!50!black,dashed,fill=\jccolorfktareahell] %
 (-2, 0) -- (-2, -4) -- plot (\x, {1/2*pow(\x,3)}) ; 
%Beschriftung:
\draw[color=\jccolorfkt] (1.4, 0.7) -- +(0.4, 0);
\draw[color=\jccolorfkt] (1.6, 0.5) -- +(0, 0.4);
\draw[color=\jccolorfkt] (-1.4, 0.7) -- +(-0.4, 0);
\draw[color=\jccolorfkt] (-1.6, 0.5) -- +(0, 0.4);
\draw[color=green!75!white] (-1.4, -0.7) -- +(-0.4, 0);
\end{scope}
\end{tikzpicture}
\ifttm\else\end{small}\fi
}
\end{center}

Since the integral is now taken over the absolute value of the function, a 
partition of the integral depending on the sign of the function is required: 
the interval of integration is subdivided into subintervals on which the 
function values have the same sign. For continuous functions, these subintervals are 
determined by the zeros of the function.

\begin{MXInfo}{Calculation of Areas} 
Let a continuous function $f: [a\MIntvlSep  b] \rightarrow \R$ on an 
interval $[a\MIntvlSep  b]$ be given. Moreover, let $x_1$, \ldots, $x_m$ be 
the zeros of $f$ with $x_1 < x_2 < \ldots < x_m$. We set $x_0 := a$ and $x_{m+1} := b$.

Then, the area bounded by the graph of $f$ and the $x$-axis equals
\[
\int_{a}^{b} |f(x)| \MDwSp x %
= \sum_{k=0}^{m} \left|\int_{x_k}^{x_{k+1}} f(x) \MDwSp x\right| \MDFPeriod %% 
\]
\end{MXInfo}

This shall be explained in more detail for the example above.

\begin{MExample}
We have to calculate the area $I_A$ of the region that is bounded by the continuous function $f$ with 
$f(x) = \frac{1}{2} x^3$ and the $x$-axis on the interval $[-2\MIntvlSep 2]$. The only zero 
of the given function is at $x_0 = 0$. The interval of integration is subdivided into the 
two subintervals $[-2\MIntvlSep 0]$ and $[0\MIntvlSep 2]$. Thus, for the area of the region between the 
graph of the function and the $x$-axis we calculate
%
\begin{eqnarray*}
I_A  =  \int_{-2}^{2} \left|\frac{1}{2} x^3\right| \MD x %
 & = & \left|\int_{-2}^{0} \frac{1}{2} x^3 \MDwSp x\right| + \left|\int_{0}^{2} \frac{1}{2} x^3 \MDwSp x\right| \\
 & = & \left|\left[\frac{1}{8} x^4\right]_{-2}^{0}\right| + \left|\left[\frac{1}{8} x^4\right]_{0}^{2}\right| \\
  & = & \left|0 - 2\right| + \left|2 - 0\right| \\
  & = & 4 \MDFPSpace ,
\end{eqnarray*}
%
to obtain the value $I_A = 4$.
\end{MExample}

However, not only areas of regions between a graph of a function and the $x$-axis can be calculated but also 
areas of regions that are bounded by two graphs of functions as illustrated in the figure below. The right-hand part 
of this figure indicates the required area that is calculated as the difference of the area in the left part of 
the figure and the area in the middle.

\begin{center}
\MTikzAuto{%
%{Beispiel einer Fl"ache zwischen dem gr"un gezeichneten Graphen 
%von $f$ und dem rot gezeichneten Graphen von $g$: Zur Berechnung des 
%Fl"acheninhalts wird die Differenz der Funktionen $f - g$ betrachtet.}{0.5}
\ifttm\else\begin{small}\fi
\begin{tikzpicture}[scale=0.6,line width=1.5pt,
 declare function={
  Fktf(\x) = -1/2 * (\x - 4) * (\x - 4) + 7/2; 
%  Fktg(\x) = 1/2 * (\x - 5) + 3; %  Fktg(\x) = 1/2 * \x + 1/2;
  Fktg(\x) = 1/2 * (\x + 1);
}
] 
\begin{scope}[xshift=-8.5cm]
%\clip(-0.7,-0.7) rectangle (6.2, 4.4);
%Koordinatenachsen:
\draw[->] (-0.6, 0) -- (6.1, 0) node[below left]{$x$}; %x-Achse
\draw[->] (0, -0.6) -- (0, 5.1) node[below left]{$y$}; %y-Achse
%Achsenbeschriftung:
\foreach \x in {1, 2, 3, 4, 5} \draw (\x, 0) -- ++(0, -0.1)%
 node[below] {$\x$}; 
\foreach \y in {1, 2, 3, 4} \draw (0, \y) -- ++(-0.1, 0) node[left] {$\y$};
\node[below left] at (0, 0) {$0$};
%Fl"ache unter $f$:
\draw[domain=2:5,samples=200,color=\jccolorfkt,fill=\jccolorfktarea] %
 plot (\x, {Fktf(\x)}) -- (5,0) -- (2,0) -- (2, {Fktf(2)}); 
%Fl"ache unter $g$:
%\draw[domain=2:5,samples=2,color=red!50!black,fill=red!50!white] %
% plot (\x, {Fktg(\x)}) -- (5,0) -- (2,0) -- (2, {Fktf(2)}); 
%Funktionsgraphen:
\draw[domain=2:5,samples=200,color=\jccolorfkt] %
 plot (\x, {Fktf(\x)}); 
\draw[domain=2:5,samples=2,color=red!50!black] %
 plot (\x, {Fktg(\x)}); 
%Schnittpunkte:
\draw[color=black] (2.0, 1.5) circle[radius=1pt];
\draw[color=black] (5.0, 3.0) circle[radius=1pt];
\end{scope}
%
\begin{scope}[xshift=0cm]
%\clip(-0.7,-0.7) rectangle (6.2, 4.4);
%Koordinatenachsen:
\draw[->] (-0.6, 0) -- (6.1, 0) node[below left]{$x$}; %x-Achse
\draw[->] (0, -0.6) -- (0, 5.1) node[below left]{$y$}; %y-Achse
%Achsenbeschriftung:
\foreach \x in {1, 2, 3, 4, 5} \draw (\x, 0) -- ++(0, -0.1)%
 node[below] {$\x$}; 
\foreach \y in {1, 2, 3, 4} \draw (0, \y) -- ++(-0.1, 0) node[left] {$\y$};
\node[below left] at (0, 0) {$0$};
%Fl"ache unter $f$:
%\draw[domain=2:5,samples=200,color=green!50!black,fill=green!50!white] %
% plot (\x, {Fktf(\x)}) -- (5,0) -- (2,0) -- (2, {Fktf(2)}); 
%Fl"ache unter $g$:
\draw[domain=2:5,samples=2,color=red!50!black,fill=red!50!white] %
 plot (\x, {Fktg(\x)}) -- (5,0) -- (2,0) -- (2, {Fktf(2)}); 
%Funktionsgraphen:
\draw[domain=2:5,samples=200,color=\jccolorfkt] %
 plot (\x, {Fktf(\x)}); 
\draw[domain=2:5,samples=2,color=red!50!black] %
 plot (\x, {Fktg(\x)}); 
%Schnittpunkte:
\draw[color=black] (2.0, 1.5) circle[radius=1pt];
\draw[color=black] (5.0, 3.0) circle[radius=1pt];
\end{scope}
%
\begin{scope}[xshift=8.5cm]
%\clip(-0.7,-0.7) rectangle (6.2, 4.4);
%Koordinatenachsen:
\draw[->] (-0.6, 0) -- (6.1, 0) node[below left]{$x$}; %x-Achse
\draw[->] (0, -0.6) -- (0, 5.1) node[below left]{$y$}; %y-Achse
%Achsenbeschriftung:
\foreach \x in {1, 2, 3, 4, 5} \draw (\x, 0) -- ++(0, -0.1)%
 node[below] {$\x$}; 
\foreach \y in {1, 2, 3, 4} \draw (0, \y) -- ++(-0.1, 0) node[left] {$\y$};
\node[below left] at (0, 0) {$0$};
%Fl"ache zwischen $f$ und $g$:
\draw[domain=2:5,samples=200,color=\jccolorfkt,fill=\jccolorfktarea] %
 plot (\x, {Fktf(\x)}) -- (5, {Fktg(5)}) -- (2, {Fktg(2)}); 
%Fl"ache unter $g$:
%\draw[domain=2:5,samples=2,color=red!50!black,fill=green!25!white]%
% plot (\x, {Fktg(\x)}) -- (5,0) -- (2,0) -- (2, {Fktf(2)}); 
\draw[domain=2:5,samples=2,color=red!50!black,dashed,% dotted,%
pattern color=red!50!black,pattern=crosshatch dots] %
 plot (\x, {Fktg(\x)}) -- (5,0) -- (2,0) -- (2, {Fktf(2)}); 
%Funktionsgraphen:
\draw[domain=2:5,samples=200,color=\jccolorfkt] %
 plot (\x, {Fktf(\x)}); 
\draw[domain=2:5,samples=2,color=red!50!black] %
 plot (\x, {Fktg(\x)}); 
%Schnittpunkte:
\draw[color=black] (2.0, 1.5) circle[radius=1pt];
\draw[color=black] (5.0, 3.0) circle[radius=1pt];
\end{scope}
\end{tikzpicture}
\ifttm\else\end{small}\fi
}
\end{center}

Again, this principle shall first be presented formally, then explained 
by means of an example.

\begin{MXInfo}{Calculation of Areas of Regions between the Graphs of two Functions} 
Let two continuous functions $f, g: [a\MIntvlSep  b] \rightarrow \R$ on an interval
$[a\MIntvlSep  b]$ be given. Moreover, let $x_1$, \ldots, $x_m$ be the zeros of 
$f - g$ with $x_1 < x_2 < \ldots < x_m$. We set $x_0 := a$ and $x_{m+1} := b$.

Then the area of the region between the graphs of $f$ and $g$ can be calculated from
%
\[
\int_{a}^{b} |f(x) - g(x)| \MDwSp x %
= \sum_{k=0}^{m} \left|\int_{x_k}^{x_{k+1}} (f(x) - g(x)) \MDwSp x\right| \MDFPeriod %% 
\]
%
\end{MXInfo}

Let us now consider an example.

\begin{MExample}
Calculate the area $I_A$ of the region between the graphs of $f$ and $g$ with $f(x) = x^2$ and 
$g(x) = 8 - \frac{1}{4} x^4$ for $x \in [-2\MIntvlSep  2]$.

First, we find the zeros of the function $f - g$. From
%
\begin{eqnarray*}
f(x) - g(x) & = & \frac{1}{4} x^4 + x^2 - 8 \\
 & = & \frac{1}{4} \left( x^4 + 4 x^2 - 32 \right) \\
 & = & \frac{1}{4} \left( x^4 + 4 x^2 + 2^2 - 2^2 - 32 \right) \\
 & = & \frac{1}{4} \left( \left( x^2 + 2 \right)^2 - 36 \right) %%
\end{eqnarray*}
%
the real zeros of $f - g$ can be calculated:
\begin{eqnarray*}
& \left( x^2 + 2 \right)^2 - 36 = 0 \\
\Leftrightarrow & \left( x^2 + 2 \right)^2 = 36 \\
\Leftrightarrow & x^2 + 2 = 6 \\
\Leftrightarrow & x^2 = 4 \\
\Leftrightarrow & x = \pm 2 \MDFPeriod %%
\end{eqnarray*}
Alternatively, from the third binomial formula, we have:
\begin{eqnarray*}
 0 & = & \left( x^2 + 2 \right)^2 - 36 %
 = \left( x^2 + 2 \right)^2 - 6^2 \\
& = & \left(x^2 + 2 - 6 \right) \cdot \left(x^2 + 2 + 6 \right) %
 = \left(x^2 - 4 \right) \cdot \left(x^2 + 2 + 6 \right) %
 = (x - 2) \cdot (x + 2) \cdot \left(x^2 + 8 \right) \MDFPeriod %%
\end{eqnarray*}
%
In the first calculation, after taking the first root we did not 
consider the case $x^2 + 2 = -6$ any further since the zeros obtained from
the resulting equation $x^2 = -8$ are not real. Thus, the real zeros of 
$f-g$ are $-2$ and $2$. These are also the boundary points 
of the interval of integration. Thus, a partition of the integral into 
different parts is not necessary. On the interval of integration, the function values 
of $f$ are smaller than the function values of $g$. For the area of the region, we obtain
\begin{eqnarray*}
I_A &=& \int_{-2}^{2} \left|f(x) - g(x)\right| \MD x \\
 & = & \int_{-2}^{2} \left({g(x) - f(x)}\right) \MD x \\
 & = & \int_{-2}^{2} \left({-\frac{1}{4} x^4 - x^2 + 8 }\right) \MD x \\
 & = & 2 \int_{0}^{2} \left({-\frac{1}{4} x^4 - x^2 + 8 }\right) \MD x %
  \MDFPSpace, \qquad \text{since the integrand is even} \\
 & = & \left[{-\frac{1}{20} x^5 - \frac{1}{3} x^3 + 8 x}\right]_{0}^{2} \\
 & = & 2\cdot \left({-\frac{32}{20}-\frac{8}{3}+16}\right) \\
 & = & \frac{352}{15} \MDFPeriod %%
\end{eqnarray*}
\end{MExample}

\end{MXContent}

\begin{MXContent}{Applications in the Sciences}{Applications in the Sciences}{STD}
\MDeclareSiteUXID{VBKM08_Anwendungen_Naturwissenschaften}

The velocity $v$ describes the instantaneous rate of change of position at the time $t$. 
Thus, we have $v = \frac{\MD s}{\MD t}$ if $v = v(t)$ and $s = s(t)$ are considered as functions 
of $t$. The current position $s(T)$ results from the inversion of the derivative, i.e. by 
integration of the velocity over the time $t$. With the initial value  $s(t = 0) = s_0$
at the time $t = 0$, we have
%
\begin{eqnarray*}
\int_{0}^{T}\frac{\MD s}{\MD t} \MDwSp t &=& \int_{0}^{T} v \MDwSp t \\
\Leftrightarrow\;\;\left[s(t)\right]_{0}^{T} &=& \int_{0}^{T} v \MDwSp t \\
\Leftrightarrow\;\;s(T) - s(0) &=& \int_{0}^{T} v \MDwSp t \\
\Leftrightarrow\;\;s(T) &=& s_0 + \int_0^T v(t) \MDwSp t \MDFPeriod
\end{eqnarray*}
%
In mathematical terms, this situation can be described as follows: if the \emph{derivative} $f'$
of a function $f$ and a single function value $f(x_0)$ are known, then the function can 
by calculated by means of the integral. In this context one says that the function values are 
reconstructed from the derivative.

If, for example, a population of bacteria increases approximately according to 
%gem"a"s $B'$ mit $B'(t) = \MEU^{\MZahl{0}{6} t}$ f"ur $t \geq 0$ vermehrt 
$B'$ with $B'(t) = \MZahl{0}{6} t$ for $t \geq 0$ and initially the population consists of
$B(0) = 100$ bacteria, then the number $B$ of bacteria in the population at time  
$T$ is described by 
\[
%B(T) - B(0) = \int_{0}^{T} \MEU^{\MZahl{0}{6} t} \MDwSp t %
B(T) - B(0) = \int_{0}^{T} \MZahl{0}{6} t \MDwSp t %
\]
and hence by
\[
%B(T) = B(0) + \int_{0}^{T} \MEU^{\MZahl{0}{6} t} \MDwSp t %
B(T) = B(0) + \int_{0}^{T} \MZahl{0}{6} t \MDwSp t %
%= 100 + \int_{0}^{T} \MZahl{0}{6} t \MDwSp t %
= 100 + \MZahl{0}{6} \int_{0}^{T} t \MDwSp t %
= 100 + \MZahl{0}{3} \left(T^2 - 0^2\right) %
= 100 + \MZahl{0}{3} T^2 \MDFPeriod %
\]
Thus, the fundamental theorem of calculus provides an important tool for reconstructing 
a function if its derivative is known (and continuous). However, in practical applications 
the functions will often be more sophisticated, for example consisting of combinations 
of exponential functions.

A further example from physics, which may be familiar, is the determination 
of the work as a product of force and displacement: $W = F_s \cdot s$. Here, $F_s$ is the projection 
of the force onto the direction of the travelled path. However, if the force depends on the path, then this
law does not apply in its simple form. For example, to calculate the work done by moving  
a massive body along a path, the force has to be integrated along the path from the initial point 
$s_1$ to the end point $s_2$:
%
\begin{eqnarray*}
W = \int_{s_1}^{s_2}F_s(s) \MDwSp s \MDFPeriod
\end{eqnarray*}
%
These are only three examples from the sciences of how the notion of an 
integral is useful. Depending on the subject of your study you will encounter a whole 
series of further applications of integration.
\end{MXContent}



%%%Uebungen zum Abschnitt:
\begin{MExercises}
\MDeclareSiteUXID{VBKM08_Anwendungen_Exercises}

\begin{MExercise}
Calculate the area $I_A$ of the region $A$ that is bounded by the graph of the 
function 
$f: [-2 \pi\MIntvlSep  2 \pi] \rightarrow \R, x \Mmapsto 3 \sin\left(x\right)$ 
and the $x$-axis.

Answer: \MEquationItem{$I_A$}{\MLParsedQuestion{30}{24}{4}{IG35}}.

\begin{MHint}{Solution}
The function $f$ with $f(x) = 3 \sin\left(x\right)$ has, on the interval  
$[-2 \pi\MIntvlSep  2 \pi]$, the zeros $-2 \pi$, $-\pi$, $0$, $\pi$, and 
$2 \pi$. Since the graph of $f$ is centrally symmetric with respect to the origin, 
for the area we obtain:
\begin{eqnarray*}
\int_{-2 \pi}^{2 \pi} |f(x)| \MDwSp x %
 & = & 3 \cdot \int_{-2 \pi}^{2 \pi}  \left|\sin\left(x\right)\right| \MD x \\
 & = & 3 \cdot 2 \cdot \int_{0}^{2 \pi}  \left|\sin\left(x\right)\right| \MD x,
 \qquad \text{since the function\ $|\sin|$\ is even,} \\
 & = & 6 \cdot \left( \int_{0}^{\pi}  \sin\left(x\right) \MD x %
         +\int_{\pi}^{2\pi}  \left(-\sin\left(x\right)\right) \MD x\right) \\
 & = & 6 \cdot \left( \left[ -\cos\left(x\right)\right]_0^\pi %
         + \left[ \cos\left(x\right)\right]_{\pi}^{2\pi}\right) \\
 & = & 6 \cdot \left(\left(-(-1)+1\right)+\left(1-(-1)\right)\right) \\
 & = & 24 \MDFPeriod %
\end{eqnarray*}
Of course, the integral can also be calculated without noting that the graph of $f$ is 
centrally symmetric with respect to the origin.
\end{MHint}
\end{MExercise}


\begin{MExercise}
Calculate the area $I_A$ of the region $A$ bounded by the graphs of the functions
$f: [1\MIntvlSep  3] \rightarrow \R, x \Mmapsto 3 - (x - 2)^2$ and 
$g: [1\MIntvlSep  3] \rightarrow \R, x \Mmapsto 2 \cdot (x - 2)^4$. Draw the graphs of the 
functions before calculating the area.

Answer: \MEquationItem{$I_A$}{\MLParsedQuestion{30}{22+8/15}{4}{IG36}}.

\begin{MHint}{Solution}
To calculate the area $I_A$ of the region between the graphs of the functions 
$f$ and $g$, the difference $f - g$ with $f(x) - g(x) = 3 - (x - 2)^2 - 2 \cdot ( x - 2)^4$
on the interval $[1\MIntvlSep  3]$ is considered.

\begin{center}
\MTikzAuto{%
\ifttm\else\begin{small}\fi
\begin{tikzpicture}[line width=1.5pt,scale=1.0,
declare function={
  u1 = 1;
  u2 = 3;
  fkt1(\x) = 3 - ((\x - 2) * (\x - 2)); % $1 \leq x \leq 3$
  fkt2(\x) = 2 * pow({\x - 2}, 4); % $1 \leq x \leq 3$
}
] %[every node/.style={fill=white}] 
%Koordinatenachsen:
\draw[->] (-0.5, 0) -- (3.5, 0) node[below left]{$x$}; %x-Achse
\draw[->] (0, -0.5) -- (0, 3.5) node[below left]{$y$}; %y-Achse
%Achsenbeschriftung:
\foreach \x in {1, 2, 3} \draw (\x, 0) -- ++(0, -0.1) %
 node[below] {$\x$}; 
\foreach \y in {1, 2, 3} \draw (0, \y) -- ++(-0.1, 0) %
 node[left] {$\y$};
\node[below left] at (0, 0) {$0$};
%Funktion:
\draw[domain=u1:u2,samples=120,color=\jccolorfkt, fill=\jccolorfktarea] %
 plot (\x, {fkt1(\x)});
\draw[domain=u1:u2,samples=120,color=blue, fill=\jccolorfktarea] %\jccolorfkt] %
 plot (\x, {fkt2(\x)+0.014});
%Beschriftung:
\node[right] at (2.75, {fkt1(2.7)}) {$f(x)$};
\node[right] at (2.75, {fkt2(2.7)}) {$g(x)$};
%
\end{tikzpicture}
\ifttm\else\end{small}\fi
}
\end{center}
>From the drawing of the graphs of the functions we see that the difference 
$f(x) - g(x)$ is greater or equal to zero for $x \in [1\MIntvlSep  3]$. This can also be seen 
by calculation: According to the assumption, we have $1 \leq x \leq 3$ and thus $-1 \leq x - 2 \leq 1$. 
Hence, $(x - 2)^2 \leq 1$ and thus $-(x - 2)^2 \geq -1$ such that 
\[
f(x) - g(x) \geq 3 - 1 - 2 \cdot 1 = 0 %%
\]
for all $1 \leq x \leq 3$. 

Thus, for the calculation of the area, we have to evaluate the integral 
$\displaystyle \int_1^3 \left(f(x) - g(x)\right) \MD x$. For this purpose, 
the terms can be multiplied out and integrated according to the sum rule. 
Another way is to consider the terms of the functions in more detail:
in this situation we have two terms, namely $(x - 2)^2$
and $(x - 2)^4$, that result from shifting the known terms $z^2$
and $z^4$ according to $z = x - 2$. An antiderivative of $h$ with $h(z) = z^2$
is $H$ with $H(z) = \frac{1}{3} \cdot z^3$. If we now consider $F$ with 
$F(x) = 3 x - \frac{1}{3} \cdot (x - 2)^3$ accordingly, then we have (from the 
chain rule) $F'(x) = 3 - \frac{1}{3} \cdot 3 \cdot (x - 2)^2 \cdot 1 = f(x)$. Here, 
the last factor results from the derivative of the inner function $u$ with 
$u(x) = x - 2$. Therefore, $F$ is an antiderivative of $f$. Likewise, 
it can be checked that $G$ with $G(x) = \frac{2}{5} \cdot (x - 2)^5$ is a 
antiderivative of $g$.

Thus, for the area $I_A$ of the region between the graphs of the functions, we have 
\begin{eqnarray*}
I_A & = & \left|\int_1^3 \left(3 - (x - 2)^2 - 2 \cdot (x - 2)^4\right) \MD x \right| \\
 & = & \left|\left[ 3 x - \frac{1}{3} (x - 2)^3 - \frac{2}{5} (x - 2)^5 \right]_{1}^{3} \right| \\
 & = & \left| 27 - \frac{1}{3} - \frac{2}{5} %
      - \left(3 + \frac{1}{3} + \frac{2}{5}\right) \right| \\
 & = & 22 + \frac{8}{15} \MDFPeriod %%
\end{eqnarray*}
\end{MHint}
\end{MExercise}

%\begin{MExercise}
%Berechnen Sie den Inhalt $I_A$ der Fl"ache $A$, die durch die Graphen der 
%Funktionen
%$f: [0\MIntvlSep  1] \rightarrow \R, x \Mmapsto \sin\left(\pi x\right)$ und 
%$g: [0\MIntvlSep  1] \rightarrow \R, x \Mmapsto 12 x^2 - 12 x$ 
%eingeschlossen wird.
%Zeichnen Sie zun"achst die Graphen der Funktionen, bevor Sie den 
%Fl"acheninhalt berechnen.
%
%Antwort: \MEquationItem{$I_A$}{\MLParsedQuestion{30}{30}{2-2/pi}{IG36}}.
%
%\begin{MHint}{L�sung}
%Man betrachtet $f(x) - g(x) = 12 x - 12 x^2 + \sin\left(\pi x\right)$ auf 
%dem Intervall
%$[0\MIntvlSep  1]$. Die Funktion $f - g$ wird nur an den R"andern des 
%Intervalls $[0\MIntvlSep  1]$ gleich Null. Dies kann man anhand einer 
%Zeichnung der Funktionsgraphen sehen.
%\begin{center}
%\MTikzAuto{%
%\ifttm\else\begin{small}\fi
%\begin{tikzpicture}[line width=1.5pt,scale=2.0,
%declare function={
%  u1 = 0;
%  u2 = 1;
%  fkt1(\x) = sin(pi * \x r); % $0 \leq x \leq 1$
%  fkt2(\x) = 12 * \x * (\x - 1); % $0 \leq x \leq 1$
%}
%] %[every node/.style={fill=white}] 
%%Koordinatenachsen:
%\draw[->] (-0.5, 0) -- (1.5, 0) node[below left]{$x$}; %x-Achse
%\draw[->] (0, -3.5) -- (0, 1.5) node[below left]{$y$}; %y-Achse
%%Achsenbeschriftung:
%%\foreach \x in {1} \draw (\x, 0) -- ++(0, -0.1) %
%% node[below right] {$\x$}; 
%\node[below right] at (1, 0) {$1$}; 
%\foreach \y in {-2, -1, 1} \draw (0, \y) -- ++(-0.1, 0) %
% node[left] {$\y$};
%\node[below left] at (0, 0) {$0$};
%%Funktion:
%\draw[domain=u1:u2,samples=120,color=\jccolorfkt] %
% plot (\x, {fkt1(\x)});
%\draw[domain=u1:u2,samples=120,color=blue] %\jccolorfkt] %
% plot (\x, {fkt2(\x)});
%%Beschriftung:
%\node[right] at (0.75, {fkt1(0.7)}) {$f(x) = \sin(\pi x) \geq 0$};
%\node[right] at (0.75, {fkt2(0.7)}) {$g(x) = 12 x (1 - x) \leq 0$};
%%
%\end{tikzpicture}
%\ifttm\else\end{small}\fi
%}
%\end{center}
%
%Berechnet man also $f - g$, dann zieht man auf dem gesamten offenen Intervall 
%$\MoIl 0\MIntvlSep  1\MoIr$ eine negative Zahl von einer positiven Zahl ab, 
%erh�lt also durchgehend positive Werte f�r $f - g$. Nur an den R�ndern rechnet 
%man $0 - 0$ und erh�lt dort auch f�r die Differenz der Funktionen den Wert Null.
%
%Damit ergibt sich f"ur den Fl"acheninhalt $I_A$ zwischen den Kurven
%\begin{eqnarray*}
%I_A & = & \left|\int_0^1 \left(12 x^2 - 12 x + \sin\left(\pi x\right)\right) \MD x \right| \\
% & = & \left|\left[ 4 x^3 - 6 x^2 - \frac{1}{\pi} \cos\left(\pi x\right) \right]_{0}^{1} \right| \\
% & = & \left| 4 - 6 + \frac{1}{\pi} - \left( 0 - 0 - \frac{1}{\pi}\right) \right| \\
% & = & \left|-2 + \frac{2}{\pi} \right| = 2 - \frac{2}{\pi} \MDFPeriod %%
%\end{eqnarray*}
%Anmerkung:
%Dass die Funktionen $f$ und $g$ nur an den R"andern gemeinsame Funktionswerte 
%haben, kann man auch rechnerisch mit einer einfachen �berlegung verstehen:
%
%Auf dem Intervall $\MoIl 0\MIntvlSep  1\MoIr$ ist die Funktion $g$ mit
%$g\left(x\right) = 12x^2 - 12x = 12x\left(x-1\right)$
%immer kleiner als Null, da $x-1$ in diesem Intervall kleiner als Null ist. 
%Nimmt man nun die Randwerte des Intervalls hinzu, so sieht man, dass die 
%Funktion $g$ dort ihre Nullstellen hat.
%
%Untersucht man hingegen die Funktion $f$ mit 
%$f\left(x\right) = \sin\left(\pi x\right)$ 
%auf demselben Intervall, dann findet man, dass diese Funktion dort immer 
%positiv ist und ebenfalls an den R�ndern ihre Nullstellen besitzt.
%\end{MHint}
%\end{MExercise}

In the next exercise, a physical problem will be formulated in mathematical terms, 
where the description involves a simplification. This shall exemplify that 
the mathematical notation can, in principle, also be used in applications. 
In practise, shorter or simpler formulations may occur. For example, domain and range are not 
given explicitly if they can be deduced from the context.

\begin{MExercise}
Calculate the work $W$ done by a force on a small spherical homogeneous body $k$
with mass $m$ in lifting it \emph{against} the gravitational force 
$F: [r_1\MIntvlSep  \infty\MoIr \rightarrow \R, r \Mmapsto F(r) := -\gamma \cdot \frac{m \cdot M}{r^2}$ 
from the surface of spherical homogeneous body $K$ with radius $r_1 = 1$ and mass $M = 2$ 
to a height of $r_2 = 4$ (all lengths are measured with respect to the centre of the body $K$).
Here, the mass $m$ and the gravitational constant $\gamma$ are assumed to be given, and the 
smaller body $k$ is assumed to be point-like in comparison to the body $K$.

Answer: \MEquationItem{$W$}{\MLSimplifyQuestion{30}{3/2*gamma*m}{10}{gamma,m}{4}{0}{ILGAMMA}}.
\ \\
\MInputHint{The constants $m$ and $\gamma$ have to occur in the solution, enter $\gamma$ as \texttt{gamma}.}
\ \\
\begin{MHint}{Solution}
The force $F_s$ that acts along the path on the small body $k$ with mass $m$ to lift it 
from the surface of the body $K$ is directed \emph{against} the gravitational force $F$.
Thus, we have $F_s = -F$.

Hence, the work $W$ done by the force in lifting the small body $k$ from $r_1 = 1$ to 
$r_2 = 4$ is 
\begin{eqnarray*}
W = \int_1^4 F_s(r) \MDwSp r %
 = - \int_1^4 F(r) \MDwSp r %
 & = & -\int_1^4 -\gamma \cdot \frac{2 \cdot m}{r^2} \MDwSp r \\
 & = & \int_1^4 \gamma \cdot \frac{2 \cdot m}{r^2} \MDwSp r \\
 & = & \left[ -\gamma \cdot \frac{2 \cdot m}{r} \right]_{1}^{4} \\
 & = & -\gamma \cdot 2 \cdot m \left(\frac{1}{4} - \frac{1}{1}\right) \\
 & = & \gamma \cdot \frac{3 m}{2} \MDFPeriod %%
\end{eqnarray*}
\end{MHint}
%
\end{MExercise}

\end{MExercises}


\MSubsection{Final Test}
\MLabel{VBKM08_Abschlusstest}


\begin{MTest}{Final Test Module \arabic{section}}
\MDeclareSiteUXID{VBKM08_Abschlusstest}
\begin{MExercise} %Stammfunktionen:
Find an antiderivative for each of the following functions:
\begin{MExerciseItems}
\item $\displaystyle\int 3 \sqrt{x} \MDwSp x = $%
\MLSimplifyQuestion{26}{2*x^(3/2)}{1}{x}{4}{32}{SIMPLE18}
%\item $\displaystyle\int \left(2 x - \frac{1}{x+\pi}\right) \MD x = $%
%\MSimplifyQuestion{26}{x^2-ln(x+pi)}{1}{x}{20}{32}
\item $\displaystyle\int \left(2 x - \MEU^{x+\pi}\right) \MD x = $%
\MLSimplifyQuestion{26}{x^2-exp(x+pi)}{10}{x}{4}{32}{SIMPLE19}
\end{MExerciseItems}
\jHTMLHinweisEingabeFunktionenExp
\end{MExercise}

%Die Aufgaben in den Abschlusstests sollen sich am "geforderten Mindestniveau"
%orientieren. Deshalb wurde die Aufgabe durch die nachfolgende neue Aufgabe
%ersetzt:
%\begin{MExercise} %Integralbegriff und Eigenschaften
%Die Funktion $f(x) := \frac{1}{x+1}$ ist f"ur $x \geq 0$ 
%streng monoton \MLQuestion{20}{fallend}{IG3}. Somit gilt
%\ifttm
%
%\begin{center}
%$\;\displaystyle\sum_{k=0}^7 \frac{1}{k+1} \cdot (k+1 - k) \;$%
%\MLQuestion{5}{>=}{IG5}%
%$\;\displaystyle\int_0^8 \frac{1}{x+1} \MDwSp x.$
%\end{enter}
%
%\else
%\[
%\sum_{k=0}^7 \frac{1}{k+1} \cdot (k+1 - k) \;\; \MLQuestion{5}{>=}{IG6} \;\; %
%\int_0^8 \frac{1}{x+1} \MDwSp x. %%
%\]
%\fi
%\MInputHint{Erg"anzen Sie den Text zu einer richtigen Aussage. Vergleiche 
%werden mit \texttt{=}, \texttt{<=} oder \texttt{>=} geschrieben.}
%\end{MExercise}

%Neue Aufgabe:
\begin{MExercise} %Integralbegriff und Eigenschaften
Calculate the integrals
\ifttm

\begin{center}
$\displaystyle\int_{1}^{\MEU} \frac{1}{2 x} \MDwSp x = $%
\MLParsedQuestion{10}{1/2}{4}{IGEx1237}
and
$\displaystyle\int_{5}^{8} \frac{6}{x-4} \MDwSp x = $%
\MLParsedQuestion{10}{12*ln(2)}{4}{IGEx1238}
\end{center}

\else
\[
\int_{1}^{\MEU} \frac{1}{2 x} \MDwSp x = %
\MLParsedQuestion{10}{1/2}{4}{IGEx1237}
\qquad \text{and} \qquad
\int_{5}^{8} \frac{6}{x-4} \MDwSp x = %
\MLParsedQuestion{10}{12*ln(2)}{4}{IGEx1238}
\]
\fi
\end{MExercise}


\begin{MExercise} %Berechnung bestimmter Integrale
Calculate the integrals
\ifttm

\begin{center}
$\displaystyle\int_{0}^{3} x \cdot \sqrt{x+1} \MDwSp x = $%
\MLParsedQuestion{10}{116/15}{4}{IGEx1239} and 
$\displaystyle\int_{\pi}^{\frac{3\pi}{4}} 5 \sin(4 x - 3 \pi) \MDwSp x = $%
\MLParsedQuestion{10}{-5/2}{4}{IGEx1240}
\end{center}

\else
\[
\int_{0}^{3} x \cdot \sqrt{x+1} \MDwSp x = %
\MLParsedQuestion{10}{116/15}{4}{IGEx1239}
\qquad \text{and} \qquad
\int_{\pi}^{\frac{3\pi}{4}} 5 \sin(4 x - 3 \pi) \MDwSp x = %
\MLParsedQuestion{10}{-5/2}{4}{IGEx1240}
\]
\fi
\end{MExercise}


\begin{MExercise} %Eigenschaften
Fill in the boxes.
\ifttm

\begin{center}
$\displaystyle 2 \int_{a}^{4} |x^3| \MDwSp x %
 = \int_{-4}^{4} |x^3| \MDwSp x \;$\MLQuestion{5}{>=}{IG9}%
$\;\left| \int_{-4}^{4} x^3 \MDwSp x \right|$ %%
\end{center}
for $a = $\MLParsedQuestion{5}{0}{4}{IG10}.

\else
\[
2 \int_{\MLParsedQuestion{4}{0}{4}{IG44}}^{4} |x^3| \MDwSp x %
 = \int_{-4}^{4} |x^3| \MDwSp x %
\;\; \MLQuestion{5}{>=}{IG11} \;\; %
\left| \int_{-4}^{4} x^3 \MDwSp x \right| %%
\]
\fi
\MInputHint{Fill in the boxes such that the statement is correct. Enter comparing symbols as 
\texttt{=}, \texttt{<=} or \texttt{>=}.}
\end{MExercise}


\begin{MExercise} %Anwendung: Fl"achenberechung
Calculate the area $I_A$ of the region $A$ that is bounded by the graphs of the two functions
%$f: [-3; 2] \rightarrow \R, x \Mmapsto x^2$ und 
%$g: [-3; 2] \rightarrow \R, x \Mmapsto 6 - x$ eingeschlossen wird.
$f$ and $g$ on $[-3\MIntvlSep  2]$ with $f(x) = x^2$ and $g(x) = 6 - x$.

Answer: $I_A = $\MLParsedQuestion{10}{125/6}{4}{IG45}
\end{MExercise}

\begin{MExercise} %Stammfunktionen: Gemeinsame Aufgabe mit OMB+
Let an antiderivative $F$ of the function $f$ and an 
antiderivative $G$ of the function $g$ be given. Moreover, the function 
$\Mid$ with $\Mid(x) = x$ is given.

Which of the following statements are always true (provided the corresponding 
combinations/compositions are possible)?

\ifttm
\begin{tabular}{|l|l|}
\hline
% richtig: & falsch: & Aussage: \\
 True? & Statement: \\
 \MLCheckbox{0}{M08C01a} & % \MLCheckbox{1}{M08C01b} &
$\Mid \cdot F$ is an antiderivative of  $\Mid \cdot f$ \\
%
 \MLCheckbox{0}{M08C02a} & % \MLCheckbox{1}{M08C02b} &
$F \circ G$ is an antiderivative of  $f \circ g$ \\
%
 \MLCheckbox{1}{M08C03a} & % \MLCheckbox{0}{M08C03b} &
$F - G$ is an antiderivative of  $f - g$\\
%
 \MLCheckbox{0}{M08C04a} & % \MLCheckbox{1}{M08C04b} &
$F / G$ is an antiderivative of  $f / g$ \\
%
 \MLCheckbox{0}{M08C05a} & % \MLCheckbox{1}{M08C05b} &
$F \cdot G$ is an antiderivative of  $f \cdot g$ \\
%
 \MLCheckbox{1}{M08C06a} & % \MLCheckbox{0}{M08C06b} &
$-20 \cdot F$ is an antiderivative of  $-20 \cdot f$ \\
\hline
\end{tabular}
\else
\begin{tabular}[t]{ccl}
%\begin{tabular}[t]{lcc}
%Aussage: & richtig: & falsch: \\
 True: & False: & Statement: \\
 \MLCheckbox{0}{M08C01a} & \MLCheckbox{1}{M08C01b} &
$\Mid \cdot F$ is an antiderivative of  $\Mid \cdot f$ \\
% & \MLCheckbox{0}{M08C01a} & \MLCheckbox{1}{M08C01b} \\
%
 \MLCheckbox{0}{M08C02a} & \MLCheckbox{1}{M08C02b} &
$F \circ G$ is an antiderivative of  $f \circ g$ \\
% & \MLCheckbox{0}{M08C02a} & \MLCheckbox{1}{M08C02b} \\
%
 \MLCheckbox{1}{M08C03a} & \MLCheckbox{0}{M08C03b} &
$F - G$ is an antiderivative of  $f - g$ \\
% & \MLCheckbox{1}{M08C03a} & \MLCheckbox{0}{M08C03b} \\
%
 \MLCheckbox{0}{M08C04a} & \MLCheckbox{1}{M08C04b} &
$F / G$ is an antiderivative of  $f / g$ \\
% & \MLCheckbox{0}{M08C04a} & \MLCheckbox{1}{M08C04b} \\
%
 \MLCheckbox{0}{M08C05a} & \MLCheckbox{1}{M08C05b} &
$F \cdot G$ is an antiderivative of  $f \cdot g$ \\
% & \MLCheckbox{0}{M08C05a} & \MLCheckbox{1}{M08C05b} \\
%
 \MLCheckbox{1}{M08C06a} & \MLCheckbox{0}{M08C06b} &
$-20 \cdot F$ is an antiderivative of  $-20 \cdot f$ %
% & \MLCheckbox{1}{M08C06a} & \MLCheckbox{0}{M08C06b} %%
%
\end{tabular}
\fi
\end{MExercise}
\end{MTest}

\clearpage
\MPrintIndex

\end{document}

%Dateiende.


