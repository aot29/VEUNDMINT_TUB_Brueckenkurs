%LaTeX-2e-File, Liedtke, 20140828.
%Inhalt: Hauptsatz der Differential- und Integralrechnung.
%zuletzt bearbeitet: 20140929.


\MPragma{MathSkip}


%Inhalt zum Abschnitt: Hauptsatz
\begin{MContent}
%\begin{MXContent}{Hauptsatz}{Hauptsatz}{STD}


Der \MEntry{Hauptsatz}{Hauptsatz} der Differential- und Integralrechnung 
beschreibt einen Zusammenhang zwischen den Stammfunktionen einer Funktion und 
deren Integral.

\begin{MXInfo}{Hauptsatz der Differential- und Integralrechnung} 
Gegeben ist eine stetige Funktion $f: [a, b] \rightarrow \R$ auf einem reellen 
Intervall $[a, b]$. 

Dann besitzt $f$ eine Stammfunktion, und f"ur jede Stammfunktion $F$ von $f$ 
gilt f"ur das Integral von $f$ dann
\[
\int_{a}^{b} f(x) \jMD x = F(b) - F(a). %% 
\]
\end{MXInfo}

Als einfaches Beispiel wird das bestimmte Integral der Funktion $f$ mit 
$f(x) = x^n$ f"ur $a \leq x \leq b$ und einem gegebenen $n \in \N$ berechnet.
Mit den Rechenregeln f"ur Integrale k"onnen damit dann alle Polynome integriert
werden.

\begin{MExample}
Die Funktion $f: [a, b] \rightarrow \R$ mit $f(x) := x^n$ f"ur $a \leq x \leq b$ 
hat nach der Tabelle aus dem ersten Abschnitt eine Stammfunktion $F$ mit 
$F(x) = \frac{1}{n+1} x^{n+1}$
Damit ist
\[
\int_a^b x^n \jMD x = \left[\frac{1}{n+1} x^{n+1}\right]_a^b %
 = \frac{1}{n+1} b^{n+1} - \frac{1}{n+1} a^{n+1} %%
\]
Beispielsweise ist f"ur $n = 1$, also $f(x) = x$ dann
$F$ mit $F(x) = \frac{1}{2} x^2$ eine Stammfunktion, sodass
\[
\int_a^b x \jMD x = \left[\frac{1}{2} x^2\right]_a^b %
 = \frac{1}{2} b^2 - \frac{1}{2} a^2 %%
\]
gilt. 
\end{MExample}

In der Situation des Hauptsatzes ist f"ur jede Zahl $z$ zwischen $a$ und $b$ dann
\[
\int_{a}^{z} f(x) \MD x = F(z) - F(a). %% 
\]
Folglich ergibt sich $F(z)$ gem"a"s
\begin{equation}
F(z) = F(a) + \int_{a}^{z} f(x) \jMD x = F(a) + \int_{a}^{z} F'(x) \jMD x. %% 
\end{equation}
aus dem Funktionswert und dem Integral "uber die Ableitung. Das hei"st, dass 
die Funktion aus der Integration "uber die Ableitung, also die "Anderungsrate
wiedergewonnen werden kann.

Dies ist f"ur viele Anwendungen ein bemerkenswertes Hilfsmittel aus der 
Mathematik. Beispielsweise werden in naturwissenschaftlichen oder technischen 
Vorg"angen oft Ver"anderungen wie die Geschwindigkeit gemessen. Die Bewegung 
kann dann daraus mittels Integration rekonstruiert werden.

%\end{MXContent}

\end{MContent}

%Dateiende.

