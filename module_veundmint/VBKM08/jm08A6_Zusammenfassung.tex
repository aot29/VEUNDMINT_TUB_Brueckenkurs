%LaTeX-2e-File, Liedtke, 20140828.
%Inhalt: Einf"uhrung in die Integralrechnung: Abschnitt Zusammenfassung.
%zuletzt bearbeitet: 20140929.

\MPragma{MathSkip}

%\MSubsubsection{Stammfunktionen}
\begin{MXContent}{Stammfunktionen}{Stammfunktionen}{STD}
Gegeben ist eine Funktion $f: D \rightarrow \R$. 
Wenn eine Funktion $F: D  \rightarrow \R$ differenzierbar ist und $F'(x) = f(x)$ 
f"ur alle $x \in D$ gilt, dann hei"st $F$ eine Stammfunktion von $f$.
\end{MXContent}


%\MSubsubsection{Integral}
\begin{MXContent}{Integral}{Integral}{STD}
Das bestimmte Integral einer beschr"ankten Funktion $f: [a, b] \rightarrow \R$ 
ist Grenzwert der Summen
\[
\sum_{k=1}^n f(z_k) \cdot \Delta(x_k) %%
\]
f"ur $n$ gegen unendlich, wenn $\max_{1 \leq k \leq n} \Delta(x_k)$ gegen null 
strebt.

Das Integral hat folgende Eigenschaften:
\begin{enumerate}
\item Summen und Vielfache von Funktionen: 
Wenn $u$ und $v$ integrierbar sind und $r \in \R$ ist, gilt
\[
\int_{a}^b \left( u(x) + v(x) \right) \MD x %
= \int_a^b u(x) \MD x + \int_a^b v(x) \jMD x %% 
\]
und
\[
\int_{a}^b r \cdot u(x) \jMD x %
= r \cdot \int_a^b u(x) \jMD x %% 
\]
%
\item Absch"atzung:
\[
\left|\int_{a}^b f(x) \MD x\right| \leq \int_a^b |f(x)| \MD x %% 
\]
und f"ur $f(x) \leq g(x)$ gilt
\[
\int_{a}^b f(x) \jMD x \leq \int_a^b g(x) \jMD x %% 
\]
\end{enumerate}
\end{MXContent}


%\MSubsubsection{Hauptsatz}
\begin{MXContent}{Hauptsatz}{Hauptsatz}{STD}
Gegeben ist eine stetige Funktion $f: [a, b] \rightarrow \R$ auf einem 
reellen Intervall $[a, b]$. 
Dann besitzt $f$ eine Stammfunktion, und f"ur jede Stammfunktion $F$ von $f$ 
gilt
\[
\int_{a}^b f(x) \jMD x = F(b) - F(a). %% 
\]
\end{MXContent}


%\MSubsubsection{Rechenregeln und Integrationsmethoden}
\begin{MXContent}{Integrationsmethoden}{Integrationsmethoden}{STD}

\begin{enumerate}

\item Umformung des Integranden:
%
Je nach Integrand kann dieser in eine bekannte Form des Integranden 
zur"uckgef"uhrt werden: F"ur rationale Funktionen ist eine Polynomdivision
ein erster Schritt zur Integration. F"ur trigonometrische Funktionen k"onnen 
hier die Additionstheoreme hilfreich sein. F"ur Wurzelfunktionen sind die 
Potenzgesetze und f"ur Exponential- und Logarithmusfunktionen deren
Rechenregeln zu nennen. Insbesondere ist $b^x = \MEU^{x \cdot \ln(b)}$ 
f"ur $b > 0$.

\item Integration von Verkettungen mit linearen Funktionen:
Wenn die Verkettung von $f$ und $u(x) = m x + b$ existiert und 
integrierbar ist, gilt:
\[
\int_{a}^{b} f(u(x)) \cdot g'(x) \jMD x %
= \int_{a}^{b} f(m x + b) \cdot g'(x) \jMD x %
= \int_{a}^{b} f(m x + b) \cdot m \jMD x %
= \int_{u(a)}^{u(b)} f(x) \MD x. %% 
\]
%\item Partielle Integration:
%Wenn $u$ und die Ableitung $v'$ stetig sind, gilt
%\[
%\int_{a}^b u(x) \cdot v'(x) \MD x %
%= \left[u(x) \cdot v(x)\right]_a^b - \int_a^b u'(x) \cdot v(x) \MD x. %% 
%\]
\end{enumerate}
\end{MXContent}


%\MSubsubsection{Anwendung der Integralrechnung}
\begin{MXContent}{Anwendung}{Anwendung}{STD}
Gegeben sind stetige reelle Funktionen $f$ und $g$ auf einem Intervall $[a, b]$. 
Weiter seien $x_1$ bis $x_m$ alle Nullstellen von $f - g$ mit 
$x_1 < x_2 < \ldots < x_m$. 
Es wird $x_0 = a$ und $x_{m+1} = b$ gesetzt.

Dann ist der Fl"acheninhalt zwischen dem Graphen der stetigen Funktionen $f$ 
und $g$ gleich
\[
\int_{a}^b |f(x) - g(x)| \jMD x %
= \sum_{k=0}^{m} \left|\int_{x_k}^{x_{k+1}} f(x) - g(x) \jMD x\right|. %% 
\]
Die Berechnung des Fl"acheninhalts zwischen dem Graphen von $f$ und der 
$x$-Achse ergibt sich, indem $g(x) = 0$ gew"ahlt wird.
\end{MXContent}

%Dateiende.

