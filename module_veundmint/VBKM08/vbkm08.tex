% MINTMOD Version P0.1.0, needs to be consistent with preprocesser object in tex2x and MPragma-Version at the end of this file

% Parameter aus Konvertierungsprozess (PDF und HTML-Erzeugung wenn vom Konverter aus gestartet) werden hier eingefuegt, Preambleincludes werden am Schluss angehaengt

\newif\ifttm                % gesetzt falls Uebersetzung in HTML stattfindet, sonst uebersetzung in PDF

% Wahl der Notationsvariante ist im PDF immer std, in der HTML-Uebersetzung wird vom Konverter die Auswahl modifiziert
\newif\ifvariantstd
\newif\ifvariantunotation
\variantstdtrue % Diese Zeile wird vom Konverter erkannt und ggf. modifiziert, daher nicht veraendern!


\def\MOutputDVI{1}
\def\MOutputPDF{2}
\def\MOutputHTML{3}
\newcounter{MOutput}

\ifttm
\usepackage{german}
\usepackage{array}
\usepackage{amsmath}
\usepackage{amssymb}
\usepackage{amsthm}
\else
\documentclass[ngerman,oneside]{scrbook}
\usepackage{etex}
\usepackage[latin1]{inputenc}
\usepackage{textcomp}
\usepackage[ngerman]{babel}
\usepackage[pdftex]{color}
\usepackage{xcolor}
\usepackage{graphicx}
\usepackage[all]{xy}
\usepackage{fancyhdr}
\usepackage{verbatim}
\usepackage{array}
\usepackage{float}
\usepackage{makeidx}
\usepackage{amsmath}
\usepackage{amstext}
\usepackage{amssymb}
\usepackage{amsthm}
\usepackage[ngerman]{varioref}
\usepackage{framed}
\usepackage{supertabular}
\usepackage{longtable}
\usepackage{maxpage}
\usepackage{tikz}
\usepackage{tikzscale}
\usepackage{tikz-3dplot}
\usepackage{bibgerm}
\usepackage{chemarrow}
\usepackage{polynom}
%\usepackage{draftwatermark}
\usepackage{pdflscape}
\usetikzlibrary{calc}
\usetikzlibrary{through}
\usetikzlibrary{shapes.geometric}
\usetikzlibrary{arrows}
\usetikzlibrary{intersections}
\usetikzlibrary{decorations.pathmorphing}
\usetikzlibrary{external}
\usetikzlibrary{patterns}
\usetikzlibrary{fadings}
\usepackage[colorlinks=true,linkcolor=blue]{hyperref} 
\usepackage[all]{hypcap}
%\usepackage[colorlinks=true,linkcolor=blue,bookmarksopen=true]{hyperref} 
\usepackage{ifpdf}

\usepackage{movie15}

\setcounter{tocdepth}{2} % In Inhaltsverzeichnis bis subsection
\setcounter{secnumdepth}{3} % Nummeriert bis subsubsection

\setlength{\LTpost}{0pt} % Fuer longtable
\setlength{\parindent}{0pt}
\setlength{\parskip}{8pt}
%\setlength{\parskip}{9pt plus 2pt minus 1pt}
\setlength{\abovecaptionskip}{-0.25ex}
\setlength{\belowcaptionskip}{-0.25ex}
\fi

\ifttm
\newcommand{\MDebugMessage}[1]{\special{html:<!-- debugprint;;}#1\special{html:; //-->}}
\else
%\newcommand{\MDebugMessage}[1]{\immediate\write\mintlog{#1}}
\newcommand{\MDebugMessage}[1]{}
\fi

\def\MPageHeaderDef{%
\pagestyle{fancy}%
\fancyhead[r]{(C) VE\&MINT-Projekt}
\fancyfoot[c]{\thepage\\--- CCL BY-SA 3.0 ---}
}


\ifttm%
\def\MRelax{}%
\else%
\def\MRelax{\relax}%
\fi%

%--------------------------- Uebernahme von speziellen XML-Versionen einiger LaTeX-Kommandos aus xmlbefehle.tex vom alten Kasseler Konverter ---------------

\newcommand{\MSep}{\left\|{\phantom{\frac1g}}\right.}

\newcommand{\ML}{L}

\newcommand{\MGGT}{\mathrm{ggT}}


\ifttm
% Verhindert dass die subsection-nummer doppelt in der toccaption auftaucht (sollte ggf. in toccaption gefixt werden so dass diese Ueberschreibung nicht notwendig ist)
\renewcommand{\thesubsection}{}
% Kommandos die ttm nicht kennt
\newcommand{\binomial}[2]{{#1 \choose #2}} %  Binomialkoeffizienten
\newcommand{\eur}{\begin{html}&euro;\end{html}}
\newcommand{\square}{\begin{html}&square;\end{html}}
\newcommand{\glqq}{"'}  \newcommand{\grqq}{"'}
\newcommand{\nRightarrow}{\special{html: &nrArr; }}
\newcommand{\nmid}{\special{html: &nmid; }}
\newcommand{\nparallel}{\begin{html}&nparallel;\end{html}}
\newcommand{\mapstoo}{\begin{html}<mo>&map;</mo>\end{html}}

% Schnitt und Vereinigungssymbole von Mengen haben zu kleine Abstaende; korrigiert:
\newcommand{\ccup}{\,\!\cup\,\!}
\newcommand{\ccap}{\,\!\cap\,\!}


% Umsetzung von mathbb im HTML
\renewcommand{\mathbb}[1]{\begin{html}<mo>&#1opf;</mo>\end{html}}
\fi

%---------------------- Strukturierung ----------------------------------------------------------------------------------------------------------------------

%---------------------- Kapselung des sectioning findet auf drei Ebenen statt:
% 1. Die LateX-Befehl
% 2. Die D-Versionen der Befehle, die nur die Grade der Abschnitte umhaengen falls notwendig
% 3. Die M-Versionen der Befehle, die zusaetzliche Formatierungen vornehmen, Skripten starten und das HTML codieren
% Im Modultext duerfen nur die M-Befehle verwendet werden!

\ifttm

  \def\Dsubsubsubsection#1{\subsubsubsection{#1}}
  \def\Dsubsubsection#1{\subsubsection{#1}\addtocounter{subsubsection}{1}} % ttm-Fehler korrigieren
  \def\Dsubsection#1{\subsection{#1}}
  \def\Dsection#1{\section{#1}} % Im HTML wird nur der Sektionstitel gegeben
  \def\Dchapter#1{\chapter{#1}}
  \def\Dsubsubsubsectionx#1{\subsubsubsection*{#1}}
  \def\Dsubsubsectionx#1{\subsubsection*{#1}}
  \def\Dsubsectionx#1{\subsection*{#1}}
  \def\Dsectionx#1{\section*{#1}}
  \def\Dchapterx#1{\chapter*{#1}}

\else

  \def\Dsubsubsubsection#1{\subsubsection{#1}}
  \def\Dsubsubsection#1{\subsection{#1}}
  \def\Dsubsection#1{\section{#1}}
  \def\Dsection#1{\chapter{#1}}
  \def\Dchapter#1{\title{#1}}
  \def\Dsubsubsubsectionx#1{\subsubsection*{#1}}
  \def\Dsubsubsectionx#1{\subsection*{#1}}
  \def\Dsubsectionx#1{\section*{#1}}
  \def\Dsectionx#1{\chapter*{#1}}

\fi

\newcommand{\MStdPoints}{4}
\newcommand{\MSetPoints}[1]{\renewcommand{\MStdPoints}{#1}}

% Befehl zum Abbruch der Erstellung (nur PDF)
\newcommand{\MAbort}[1]{\err{#1}}

% Prefix vor Dateieinbindungen, wird in der Baumdatei mit \renewcommand modifiziert
% und auf das Verzeichnisprefix gesetzt, in dem das gerade bearbeitete tex-Dokument liegt.
% Im HTML wird es auf das Verzeichnis der HTML-Datei gesetzt.
% Das Prefix muss mit / enden !
\newcommand{\MDPrefix}{.}

% MRegisterFile notiert eine Datei zur Einbindung in den HTML-Baum. Grafiken mit MGraphics werden automatisch eingebunden.
% Mit MLastFile erhaelt man eine Markierung fuer die zuletzt registrierte Datei.
% Diese Markierung wird im postprocessing durch den physikalischen Dateinamen ersetzt, aber nur den Namen (d.h. \MMaterial gehoert noch davor, vgl Definition von MGraphics)
% Parameter: Pfad/Name der Datei bzw. des Ordners, relativ zur Position des Modul-Tex-Dokuments.
\ifttm
\newcommand{\MRegisterFile}[1]{\addtocounter{MFileNumber}{1}\special{html:<!-- registerfile;;}#1\special{html:;;}\MDPrefix\special{html:;;}\arabic{MFileNumber}\special{html:; //-->}}
\else
\newcommand{\MRegisterFile}[1]{\addtocounter{MFileNumber}{1}}
\fi

% Testen welcher Uebersetzer hier am Werk ist

\ifttm
\setcounter{MOutput}{3}
\else
\ifx\pdfoutput\undefined
  \pdffalse
  \setcounter{MOutput}{\MOutputDVI}
  \message{Verarbeitung mit latex, Ausgabe in dvi.}
\else
  \setcounter{MOutput}{\MOutputPDF}
  \message{Verarbeitung mit pdflatex, Ausgabe in pdf.}
  \ifnum \pdfoutput=0
    \pdffalse
  \setcounter{MOutput}{\MOutputDVI}
  \message{Verarbeitung mit pdflatex, Ausgabe in dvi.}
  \else
    \ifnum\pdfoutput=1
    \pdftrue
  \setcounter{MOutput}{\MOutputPDF}
  \message{Verarbeitung mit pdflatex, Ausgabe in pdf.}
    \fi
  \fi
\fi
\fi

\ifnum\value{MOutput}=\MOutputPDF
\DeclareGraphicsExtensions{.pdf,.png,.jpg}
\fi

\ifnum\value{MOutput}=\MOutputDVI
\DeclareGraphicsExtensions{.eps,.png,.jpg}
\fi

\ifnum\value{MOutput}=\MOutputHTML
% Wird vom Konverter leider nicht erkannt und daher in split.pm hardcodiert!
\DeclareGraphicsExtensions{.png,.jpg,.gif}
\fi

% Umdefinition der hyperref-Nummerierung im PDF-Modus
\ifttm
\else
\renewcommand{\theHfigure}{\arabic{chapter}.\arabic{section}.\arabic{figure}}
\fi

% Makro, um in der HTML-Ausgabe die zuerst zu oeffnende Datei zu kennzeichnen
\ifttm
\newcommand{\MGlobalStart}{\special{html:<!-- mglobalstarttag -->}}
\else
\newcommand{\MGlobalStart}{}
\fi

% Makro, um bei scormlogin ein pullen des Benutzers bei Aufruf der Seite zu erzwingen (typischerweise auf der Einstiegsseite)
\ifttm
\newcommand{\MPullSite}{\special{html:<!-- pullsite //-->}}
\else
\newcommand{\MPullSite}{}
\fi

% Makro, um in der HTML-Ausgabe die Kapiteluebersicht zu kennzeichnen
\ifttm
\newcommand{\MGlobalChapterTag}{\special{html:<!-- mglobalchaptertag -->}}
\else
\newcommand{\MGlobalChapterTag}{}
\fi

% Makro, um in der HTML-Ausgabe die Konfiguration zu kennzeichnen
\ifttm
\newcommand{\MGlobalConfTag}{\special{html:<!-- mglobalconfigtag -->}}
\else
\newcommand{\MGlobalConfTag}{}
\fi

% Makro, um in der HTML-Ausgabe die Standortbeschreibung zu kennzeichnen
\ifttm
\newcommand{\MGlobalLocationTag}{\special{html:<!-- mgloballocationtag -->}}
\else
\newcommand{\MGlobalLocationTag}{}
\fi

% Makro, um in der HTML-Ausgabe die persoenlichen Daten zu kennzeichnen
\ifttm
\newcommand{\MGlobalDataTag}{\special{html:<!-- mglobaldatatag -->}}
\else
\newcommand{\MGlobalDataTag}{}
\fi

% Makro, um in der HTML-Ausgabe die Suchseite zu kennzeichnen
\ifttm
\newcommand{\MGlobalSearchTag}{\special{html:<!-- mglobalsearchtag -->}}
\else
\newcommand{\MGlobalSearchTag}{}
\fi

% Makro, um in der HTML-Ausgabe die Favoritenseite zu kennzeichnen
\ifttm
\newcommand{\MGlobalFavoTag}{\special{html:<!-- mglobalfavoritestag -->}}
\else
\newcommand{\MGlobalFavoTag}{}
\fi

% Makro, um in der HTML-Ausgabe die Eingangstestseite zu kennzeichnen
\ifttm
\newcommand{\MGlobalSTestTag}{\special{html:<!-- mglobalstesttag -->}}
\else
\newcommand{\MGlobalSTestTag}{}
\fi

% Makro, um in der PDF-Ausgabe ein Wasserzeichen zu definieren
\ifttm
\newcommand{\MWatermarkSettings}{\relax}
\else
\newcommand{\MWatermarkSettings}{%
% \SetWatermarkText{(c) MINT-Kolleg Baden-W�rttemberg 2014}
% \SetWatermarkLightness{0.85}
% \SetWatermarkScale{1.5}
}
\fi

\ifttm
\newcommand{\MBinom}[2]{\left({\begin{array}{c} #1 \\ #2 \end{array}}\right)}
\else
\newcommand{\MBinom}[2]{\binom{#1}{#2}}
\fi

\ifttm
\newcommand{\DeclareMathOperator}[2]{\def#1{\mathrm{#2}}}
\newcommand{\operatorname}[1]{\mathrm{#1}}
\fi

%----------------- Makros fuer die gemischte HTML/PDF-Konvertierung ------------------------------

\newcommand{\MTestName}{\relax} % wird durch Test-Umgebung gesetzt

% Fuer experimentelle Kursinhalte, die im Release-Umsetzungsvorgang eine Fehlermeldung
% produzieren sollen aber sonst normal umgesetzt werden
\newenvironment{MExperimental}{%
}{%
}

% Wird von ttm nicht richtig umgesetzt!!
\newenvironment{MExerciseItems}{%
\renewcommand\theenumi{\alph{enumi}}%
\begin{enumerate}%
}{%
\end{enumerate}%
}


\definecolor{infoshadecolor}{rgb}{0.75,0.75,0.75}
\definecolor{exmpshadecolor}{rgb}{0.875,0.875,0.875}
\definecolor{expeshadecolor}{rgb}{0.95,0.95,0.95}
\definecolor{framecolor}{rgb}{0.2,0.2,0.2}

% Bei PDF-Uebersetzung wird hinter den Start jeder Satz/Info-aehnlichen Umgebung eine leere mbox gesetzt, damit
% fuehrende Listen oder enums nicht den Zeilenumbruch kaputtmachen
%\ifttm
\def\MTB{}
%\else
%\def\MTB{\mbox{}}
%\fi


\ifttm
\newcommand{\MRelates}{\special{html:<mi>&wedgeq;</mi>}}
\else
\def\MRelates{\stackrel{\scriptscriptstyle\wedge}{=}}
\fi

\def\MInch{\text{''}}
\def\Mdd{\textit{''}}

\ifttm
\def\MNL{ \newline }
\newenvironment{MArray}[1]{\begin{array}{#1}}{\end{array}}
\else
\def\MNL{ \\ }
\newenvironment{MArray}[1]{\begin{array}{#1}}{\end{array}}
\fi

\newcommand{\MBox}[1]{$\mathrm{#1}$}
\newcommand{\MMBox}[1]{\mathrm{#1}}


\ifttm%
\newcommand{\Mtfrac}[2]{{\textstyle \frac{#1}{#2}}}
\newcommand{\Mdfrac}[2]{{\displaystyle \frac{#1}{#2}}}
\newcommand{\Mmeasuredangle}{\special{html:<mi>&angmsd;</mi>}}
\else%
\newcommand{\Mtfrac}[2]{\tfrac{#1}{#2}}
\newcommand{\Mdfrac}[2]{\dfrac{#1}{#2}}
\newcommand{\Mmeasuredangle}{\measuredangle}
\relax
\fi

% Matrizen und Vektoren

% Inhalt wird in der Form a & b \\ c & d erwartet
% Vorsicht: MVector = Komponentenspalte, MVec = Variablensymbol
\ifttm%
\newcommand{\MVector}[1]{\left({\begin{array}{c}#1\end{array}}\right)}
\else%
\newcommand{\MVector}[1]{\begin{pmatrix}#1\end{pmatrix}}
\fi



\newcommand{\MVec}[1]{\vec{#1}}
\newcommand{\MDVec}[1]{\overrightarrow{#1}}

%----------------- Umgebungen fuer Definitionen und Saetze ----------------------------------------

% Fuegt einen Tabellen-Zeilenumbruch ein im PDF, aber nicht im HTML
\newcommand{\TSkip}{\ifttm \else&\ \\\fi}

\newenvironment{infoshaded}{%
\def\FrameCommand{\fboxsep=\FrameSep \fcolorbox{framecolor}{infoshadecolor}}%
\MakeFramed {\advance\hsize-\width \FrameRestore}}%
{\endMakeFramed}

\newenvironment{expeshaded}{%
\def\FrameCommand{\fboxsep=\FrameSep \fcolorbox{framecolor}{expeshadecolor}}%
\MakeFramed {\advance\hsize-\width \FrameRestore}}%
{\endMakeFramed}

\newenvironment{exmpshaded}{%
\def\FrameCommand{\fboxsep=\FrameSep \fcolorbox{framecolor}{exmpshadecolor}}%
\MakeFramed {\advance\hsize-\width \FrameRestore}}%
{\endMakeFramed}

\def\STDCOLOR{black}

\ifttm%
\else%
\newtheoremstyle{MSatzStyle}
  {1cm}                   %Space above
  {1cm}                   %Space below
  {\normalfont\itshape}   %Body font
  {}                      %Indent amount (empty = no indent,
                          %\parindent = para indent)
  {\normalfont\bfseries}  %Thm head font
  {}                      %Punctuation after thm head
  {\newline}              %Space after thm head: " " = normal interword
                          %space; \newline = linebreak
  {\thmname{#1}\thmnumber{ #2}\thmnote{ (#3)}}
                          %Thm head spec (can be left empty, meaning
                          %`normal')
                          %
\newtheoremstyle{MDefStyle}
  {1cm}                   %Space above
  {1cm}                   %Space below
  {\normalfont}           %Body font
  {}                      %Indent amount (empty = no indent,
                          %\parindent = para indent)
  {\normalfont\bfseries}  %Thm head font
  {}                      %Punctuation after thm head
  {\newline}              %Space after thm head: " " = normal interword
                          %space; \newline = linebreak
  {\thmname{#1}\thmnumber{ #2}\thmnote{ (#3)}}
                          %Thm head spec (can be left empty, meaning
                          %`normal')
\fi%

\newcommand{\MInfoText}{Info}

\newcounter{MHintCounter}
\newcounter{MCodeEditCounter}

\newcounter{MLastIndex}  % Enthaelt die dritte Stelle (Indexnummer) des letzten angelegten Objekts
\newcounter{MLastType}   % Enthaelt den Typ des letzten angelegten Objekts (mithilfe der unten definierten Konstanten). Die Entscheidung, wie der Typ dargstellt wird, wird in split.pm beim Postprocessing getroffen.
\newcounter{MLastTypeEq} % =1 falls das Label in einer Matheumgebung (equation, eqnarray usw.) steht, =2 falls das Label in einer table-Umgebung steht

% Da ttm keine Zahlmakros verarbeiten kann, werden diese Nummern in den Zuweisungen hardcodiert!
\def\MTypeSection{1}          %# Zaehler ist section
\def\MTypeSubsection{2}       %# Zaehler ist subsection
\def\MTypeSubsubsection{3}    %# Zaehler ist subsubsection
\def\MTypeInfo{4}             %# Eine Infobox, Separatzaehler fuer die Chemie (auch wenn es dort nicht nummeriert wird) ist MInfoCounter
\def\MTypeExercise{5}         %# Eine Aufgabe, Separatzaehler fuer die Chemie ist MExerciseCounter
\def\MTypeExample{6}          %# Eine Beispielbox, Separatzaehler fuer die Chemie ist MExampleCounter
\def\MTypeExperiment{7}       %# Eine Versuchsbox, Separatzaehler fuer die Chemie ist MExperimentCounter
\def\MTypeGraphics{8}         %# Eine Graphik, Separatzaehler fuer alle FB ist MGraphicsCounter
\def\MTypeTable{9}            %# Eine Tabellennummer, hat keinen Zaehler da durch table gezaehlt wird
\def\MTypeEquation{10}        %# Eine Gleichungsnummer, hat keinen Zaehler da durch equation/eqnarray gezaehlt wird
\def\MTypeTheorem{11}         % Ein theorem oder xtheorem, Separatzaehler fuer die Chemie ist MTheoremCounter
\def\MTypeVideo{12}           %# Ein Video,Separatzaehler fuer alle FB ist MVideoCounter
\def\MTypeEntry{13}           %# Ein Eintrag fuer die Stichwortliste, wird nicht gezaehlt sondern erhaelt im preparsing ein unique-label 

% Zaehler fuer das Labelsystem sind prefixcounter, jeder Zaehler wird VOR dem gezaehlten Objekt inkrementiert und zaehlt daher das aktuelle Objekt
\newcounter{MInfoCounter}
\newcounter{MExerciseCounter}
\newcounter{MExampleCounter}
\newcounter{MExperimentCounter}
\newcounter{MGraphicsCounter}
\newcounter{MTableCounter}
\newcounter{MEquationCounter}  % Nur im HTML, sonst durch "equation"-counter von latex realisiert
\newcounter{MTheoremCounter}
\newcounter{MObjectCounter}   % Gemeinsamer Zaehler fuer Objekte (ausser Grafiken/Tabellen) in Mathe/Info/Physik
\newcounter{MVideoCounter}
\newcounter{MEntryCounter}

\newcounter{MTestSite} % 1 = Subsubsection ist eine Pruefungsseite, 0 = ist eine normale Seite (inkl. Hilfeseite)

\def\MCell{$\phantom{a}$}

\newenvironment{MExportExercise}{\begin{MExercise}}{\end{MExercise}} % wird von mconvert abgefangen

\def\MGenerateExNumber{%
\ifnum\value{MSepNumbers}=0%
\arabic{section}.\arabic{subsection}.\arabic{MObjectCounter}\setcounter{MLastIndex}{\value{MObjectCounter}}%
\else%
\arabic{section}.\arabic{subsection}.\arabic{MExerciseCounter}\setcounter{MLastIndex}{\value{MExerciseCounter}}%
\fi%
}%

\def\MGenerateExmpNumber{%
\ifnum\value{MSepNumbers}=0%
\arabic{section}.\arabic{subsection}.\arabic{MObjectCounter}\setcounter{MLastIndex}{\value{MObjectCounter}}%
\else%
\arabic{section}.\arabic{subsection}.\arabic{MExerciseCounter}\setcounter{MLastIndex}{\value{MExampleCounter}}%
\fi%
}%

\def\MGenerateInfoNumber{%
\ifnum\value{MSepNumbers}=0%
\arabic{section}.\arabic{subsection}.\arabic{MObjectCounter}\setcounter{MLastIndex}{\value{MObjectCounter}}%
\else%
\arabic{section}.\arabic{subsection}.\arabic{MExerciseCounter}\setcounter{MLastIndex}{\value{MInfoCounter}}%
\fi%
}%

\def\MGenerateSiteNumber{%
\arabic{section}.\arabic{subsection}.\arabic{subsubsection}%
}%

% Funktionalitaet fuer Auswahlaufgaben

\newcounter{MExerciseCollectionCounter} % = 0 falls nicht in collection-Umgebung, ansonsten Schachtelungstiefe
\newcounter{MExerciseCollectionTextCounter} % wird von MExercise-Umgebung inkrementiert und von MExerciseCollection-Umgebung auf Null gesetzt

\ifttm
% MExerciseCollection gruppiert Aufgaben, die dynamisch aus der Datenbank gezogen werden und nicht direkt in der HTML-Seite stehen
% Parameter: #1 = ID der Collection, muss eindeutig fuer alle IN DER DB VORHANDENEN collections sein unabhaengig vom Kurs
%            #2 = Optionsargument (im Moment: 1 = Iterative Auswahl, 2 = Zufallsbasierte Auswahl)
\newenvironment{MExerciseCollection}[2]{%
\addtocounter{MExerciseCollectionCounter}{1}
\setcounter{MExerciseCollectionTextCounter}{0}
\special{html:<!-- mexercisecollectionstart;;}#1\special{html:;;}#2\special{html:;; //-->}%
}{%
\special{html:<!-- mexercisecollectionstop //-->}%
\addtocounter{MExerciseCollectionCounter}{-1}
}
\else
\newenvironment{MExerciseCollection}[2]{%
\addtocounter{MExerciseCollectionCounter}{1}
\setcounter{MExerciseCollectionTextCounter}{0}
}{%
\addtocounter{MExerciseCollectionCounter}{-1}
}
\fi

% Bei Uebersetzung nach PDF werden die theorem-Umgebungen verwendet, bei Uebersetzung in HTML ein manuelles Makro
\ifttm%

  \newenvironment{MHint}[1]{  \special{html:<button name="Name_MHint}\arabic{MHintCounter}\special{html:" class="hintbutton_closed" id="MHint}\arabic{MHintCounter}\special{html:_button" %
  type="button" onclick="toggle_hint('MHint}\arabic{MHintCounter}\special{html:');">}#1\special{html:</button>}
  \special{html:<div class="hint" style="display:none" id="MHint}\arabic{MHintCounter}\special{html:"> }}{\begin{html}</div>\end{html}\addtocounter{MHintCounter}{1}}

  \newenvironment{MCOSHZusatz}{  \special{html:<button name="Name_MHint}\arabic{MHintCounter}\special{html:" class="chintbutton_closed" id="MHint}\arabic{MHintCounter}\special{html:_button" %
  type="button" onclick="toggle_hint('MHint}\arabic{MHintCounter}\special{html:');">}Weiterf�hrende Inhalte\special{html:</button>}
  \special{html:<div class="hintc" style="display:none" id="MHint}\arabic{MHintCounter}\special{html:">
  <div class="coshwarn">Diese Inhalte gehen �ber das Kursniveau hinaus und werden in den Aufgaben und Tests nicht abgefragt.</div><br />}
  \addtocounter{MHintCounter}{1}}{\begin{html}</div>\end{html}}

  
  \newenvironment{MDefinition}{\begin{definition}\setcounter{MLastIndex}{\value{definition}}\ \\}{\end{definition}}

  
  \newenvironment{MExercise}{
  \renewcommand{\MStdPoints}{4}
  \addtocounter{MExerciseCounter}{1}
  \addtocounter{MObjectCounter}{1}
  \setcounter{MLastType}{5}

  \ifnum\value{MExerciseCollectionCounter}=0\else\addtocounter{MExerciseCollectionTextCounter}{1}\special{html:<!-- mexercisetextstart;;}\arabic{MExerciseCollectionTextCounter}\special{html:;; //-->}\fi
  \special{html:<div class="aufgabe" id="ADIV_}\MGenerateExNumber\special{html:">}%
  \textbf{Aufgabe \MGenerateExNumber
  } \ \\}{
  \special{html:</div><!-- mfeedbackbutton;Aufgabe;}\arabic{MTestSite}\special{html:;}\MGenerateExNumber\special{html:; //-->}
  \ifnum\value{MExerciseCollectionCounter}=0\else\special{html:<!-- mexercisetextstop //-->}\fi
  }

  % Stellt eine Kombination aus Aufgabe, Loesungstext und Eingabefeld bereit,
  % bei der Aufgabentext und Musterloesung sowie die zugehoerigen Feldelemente
  % extern bezogen und div-aktualisiert werden, das Eingabefeld aber immer das gleiche ist.
  \newenvironment{MFetchExercise}{
  \addtocounter{MExerciseCounter}{1}
  \addtocounter{MObjectCounter}{1}
  \setcounter{MLastType}{5}

  \special{html:<div class="aufgabe" id="ADIV_}\MGenerateExNumber\special{html:">}%
  \textbf{Aufgabe \MGenerateExNumber
  } \ \\%
  \special{html:</div><div class="exfetch_text" id="ADIVTEXT_}\MGenerateExNumber\special{html:">}%
  \special{html:</div><div class="exfetch_sol" id="ADIVSOL_}\MGenerateExNumber\special{html:">}%
  \special{html:</div><div class="exfetch_input" id="ADIVINPUT_}\MGenerateExNumber\special{html:">}%
  }{
  \special{html:</div>}
  }

  \newenvironment{MExample}{
  \addtocounter{MExampleCounter}{1}
  \addtocounter{MObjectCounter}{1}
  \setcounter{MLastType}{6}
  \begin{html}
  <div class="exmp">
  <div class="exmprahmen">
  \end{html}\textbf{Beispiel
  \ifnum\value{MSepNumbers}=0
  \arabic{section}.\arabic{subsection}.\arabic{MObjectCounter}\setcounter{MLastIndex}{\value{MObjectCounter}}
  \else
  \arabic{section}.\arabic{subsection}.\arabic{MExampleCounter}\setcounter{MLastIndex}{\value{MExampleCounter}}
  \fi
  } \ \\}{\begin{html}</div>
  </div>
  \end{html}
  \special{html:<!-- mfeedbackbutton;Beispiel;}\arabic{MTestSite}\special{html:;}\MGenerateExmpNumber\special{html:; //-->}
  }

  \newenvironment{MExperiment}{
  \addtocounter{MExperimentCounter}{1}
  \addtocounter{MObjectCounter}{1}
  \setcounter{MLastType}{7}
  \begin{html}
  <div class="expe">
  <div class="experahmen">
  \end{html}\textbf{Versuch
  \ifnum\value{MSepNumbers}=0
  \arabic{section}.\arabic{subsection}.\arabic{MObjectCounter}\setcounter{MLastIndex}{\value{MObjectCounter}}
  \else
%  \arabic{MExperimentCounter}\setcounter{MLastIndex}{\value{MExperimentCounter}}
  \arabic{section}.\arabic{subsection}.\arabic{MExperimentCounter}\setcounter{MLastIndex}{\value{MExperimentCounter}}
  \fi
  } \ \\}{\begin{html}</div>
  </div>
  \end{html}}

  \newenvironment{MChemInfo}{
  \setcounter{MLastType}{4}
  \begin{html}
  <div class="info">
  <div class="inforahmen">
  \end{html}}{\begin{html}</div>
  </div>
  \end{html}}

  \newenvironment{MXInfo}[1]{
  \addtocounter{MInfoCounter}{1}
  \addtocounter{MObjectCounter}{1}
  \setcounter{MLastType}{4}
  \begin{html}
  <div class="info">
  <div class="inforahmen">
  \end{html}\textbf{#1
  \ifnum\value{MInfoNumbers}=0
  \else
    \ifnum\value{MSepNumbers}=0
    \arabic{section}.\arabic{subsection}.\arabic{MObjectCounter}\setcounter{MLastIndex}{\value{MObjectCounter}}
    \else
    \arabic{MInfoCounter}\setcounter{MLastIndex}{\value{MInfoCounter}}
    \fi
  \fi
  } \ \\}{\begin{html}</div>
  </div>
  \end{html}
  \special{html:<!-- mfeedbackbutton;Info;}\arabic{MTestSite}\special{html:;}\MGenerateInfoNumber\special{html:; //-->}
  }

  \newenvironment{MInfo}{\ifnum\value{MInfoNumbers}=0\begin{MChemInfo}\else\begin{MXInfo}{Info}\ \\ \fi}{\ifnum\value{MInfoNumbers}=0\end{MChemInfo}\else\end{MXInfo}\fi}

\else%

  \theoremstyle{MSatzStyle}
  \newtheorem{thm}{Satz}[section]
  \newtheorem{thmc}{Satz}
  \theoremstyle{MDefStyle}
  \newtheorem{defn}[thm]{Definition}
  \newtheorem{exmp}[thm]{Beispiel}
  \newtheorem{info}[thm]{\MInfoText}
  \theoremstyle{MDefStyle}
  \newtheorem{defnc}{Definition}
  \theoremstyle{MDefStyle}
  \newtheorem{exmpc}{Beispiel}[section]
  \theoremstyle{MDefStyle}
  \newtheorem{infoc}{\MInfoText}
  \theoremstyle{MDefStyle}
  \newtheorem{exrc}{Aufgabe}[section]
  \theoremstyle{MDefStyle}
  \newtheorem{verc}{Versuch}[section]
  
  \newenvironment{MFetchExercise}{}{} % kann im PDF nicht dargestellt werden
  
  \newenvironment{MExercise}{\begin{exrc}\renewcommand{\MStdPoints}{1}\MTB}{\end{exrc}}
  \newenvironment{MHint}[1]{\ \\ \underline{#1:}\\}{}
  \newenvironment{MCOSHZusatz}{\ \\ \underline{Weiterf�hrende Inhalte:}\\}{}
  \newenvironment{MDefinition}{\ifnum\value{MInfoNumbers}=0\begin{defnc}\else\begin{defn}\fi\MTB}{\ifnum\value{MInfoNumbers}=0\end{defnc}\else\end{defn}\fi}
%  \newenvironment{MExample}{\begin{exmp}}{\ \linebreak[1] \ \ \ \ $\phantom{a}$ \ \hfill $\blacklozenge$\end{exmp}}
  \newenvironment{MExample}{
    \ifnum\value{MInfoNumbers}=0\begin{exmpc}\else\begin{exmp}\fi
    \MTB
    \begin{exmpshaded}
    \ \newline
}{
    \end{exmpshaded}
    \ifnum\value{MInfoNumbers}=0\end{exmpc}\else\end{exmp}\fi
}
  \newenvironment{MChemInfo}{\begin{infoshaded}}{\end{infoshaded}}

  \newenvironment{MInfo}{\ifnum\value{MInfoNumbers}=0\begin{MChemInfo}\else\renewcommand{\MInfoText}{Info}\begin{info}\begin{infoshaded}
  \MTB
   \ \newline
    \fi
  }{\ifnum\value{MInfoNumbers}=0\end{MChemInfo}\else\end{infoshaded}\end{info}\fi}

  \newenvironment{MXInfo}[1]{
    \renewcommand{\MInfoText}{#1}
    \ifnum\value{MInfoNumbers}=0\begin{infoc}\else\begin{info}\fi%
    \MTB
    \begin{infoshaded}
    \ \newline
  }{\end{infoshaded}\ifnum\value{MInfoNumbers}=0\end{infoc}\else\end{info}\fi}

  \newenvironment{MExperiment}{
    \renewcommand{\MInfoText}{Versuch}
    \ifnum\value{MInfoNumbers}=0\begin{verc}\else\begin{info}\fi
    \MTB
    \begin{expeshaded}
    \ \newline
  }{
    \end{expeshaded}
    \ifnum\value{MInfoNumbers}=0\end{verc}\else\end{info}\fi
  }
\fi%

% MHint sollte nicht direkt fuer Loesungen benutzt werden wegen solutionselect
\newenvironment{MSolution}{\begin{MHint}{L"osung}}{\end{MHint}}

\newcounter{MCodeCounter}

\ifttm
\newenvironment{MCode}{\special{html:<!-- mcodestart -->}\ttfamily\color{blue}}{\special{html:<!-- mcodestop -->}}
\else
\newenvironment{MCode}{\begin{flushleft}\ttfamily\addtocounter{MCodeCounter}{1}}{\addtocounter{MCodeCounter}{-1}\end{flushleft}}
% Ohne color-Statement da inkompatible mit framed/shaded-Boxen aus dem framed-package
\fi

%----------------- Sonderdefinitionen fuer Symbole, die der Konverter nicht kann ----------------------------------------------

\ifttm%
\newcommand{\MUnderset}[2]{\underbrace{#2}_{#1}}%
\else%
\newcommand{\MUnderset}[2]{\underset{#1}{#2}}%
\fi%

\ifttm
\newcommand{\MThinspace}{\special{html:<mi>&#x2009;</mi>}}
\else
\newcommand{\MThinspace}{\,}
\fi

\ifttm
\newcommand{\glq}{\begin{html}&sbquo;\end{html}}
\newcommand{\grq}{\begin{html}&lsquo;\end{html}}
\newcommand{\glqq}{\begin{html}&bdquo;\end{html}}
\newcommand{\grqq}{\begin{html}&ldquo;\end{html}}
\fi

\ifttm
\newcommand{\MNdash}{\begin{html}&ndash;\end{html}}
\else
\newcommand{\MNdash}{--}
\fi

%\ifttm\def\MIU{\special{html:<mi>&#8520;</mi>}}\else\def\MIU{\mathrm{i}}\fi
\def\MIU{\mathrm{i}}
\def\MEU{e} % TU9-Onlinekurs: italic-e
%\def\MEU{\mathrm{e}} % Alte Onlinemodule: roman-e
\def\MD{d} % Kursives d in Integralen im TU9-Onlinekurs
%\def\MD{\mathrm{d}} % roman-d in den alten Onlinemodulen
\def\MDB{\|}

%zusaetzlicher Leerraum vor "\MD"
\ifttm%
\def\MDSpace{\special{html:<mi>&#x2009;</mi>}}
\else%
\def\MDSpace{\,}
\fi%
\newcommand{\MDwSp}{\MDSpace\MD}%

\ifttm
\def\Mdq{\dq}
\else
\def\Mdq{\dq}
\fi

\def\MSpan#1{\left<{#1}\right>}
\def\MSetminus{\setminus}
\def\MIM{I}

\ifttm
\newcommand{\ld}{\text{ld}}
\newcommand{\lg}{\text{lg}}
\else
\DeclareMathOperator{\ld}{ld}
%\newcommand{\lg}{\text{lg}} % in latex schon definiert
\fi


\def\Mmapsto{\ifttm\special{html:<mi>&mapsto;</mi>}\else\mapsto\fi} 
\def\Mvarphi{\ifttm\phi\else\varphi\fi}
\def\Mphi{\ifttm\varphi\else\phi\fi}
\ifttm%
\newcommand{\MEumu}{\special{html:<mi>&#x3BC;</mi>}}%
\else%
\newcommand{\MEumu}{\textrm{\textmu}}%
\fi
\def\Mvarepsilon{\ifttm\epsilon\else\varepsilon\fi}
\def\Mepsilon{\ifttm\varepsilon\else\epsilon\fi}
\def\Mvarkappa{\ifttm\kappa\else\varkappa\fi}
\def\Mkappa{\ifttm\varkappa\else\kappa\fi}
\def\Mcomplement{\ifttm\special{html:<mi>&comp;</mi>}\else\complement\fi} 
\def\MWW{\mathrm{WW}}
\def\Mmod{\ifttm\special{html:<mi>&nbsp;mod&nbsp;</mi>}\else\mod\fi} 

\ifttm%
\def\mod{\text{\;mod\;}}%
\def\MNEquiv{\special{html:<mi>&NotCongruent;</mi>}}% 
\def\MNSubseteq{\special{html:<mi>&NotSubsetEqual;</mi>}}%
\def\MEmptyset{\special{html:<mi>&empty;</mi>}}%
\def\MVDots{\special{html:<mi>&#x22EE;</mi>}}%
\def\MHDots{\special{html:<mi>&#x2026;</mi>}}%
\def\Mddag{\special{html:<mi>&#x1202;</mi>}}%
\def\sphericalangle{\special{html:<mi>&measuredangle;</mi>}}%
\def\nparallel{\special{html:<mi>&nparallel;</mi>}}%
\def\MProofEnd{\special{html:<mi>&#x25FB;</mi>}}%
\newenvironment{MProof}[1]{\underline{#1}:\MCR\MCR}{\hfill $\MProofEnd$}%
\else%
\def\MNEquiv{\not\equiv}%
\def\MNSubseteq{\not\subseteq}%
\def\MEmptyset{\emptyset}%
\def\MVDots{\vdots}%
\def\MHDots{\hdots}%
\def\Mddag{\ddag}%
\newenvironment{MProof}[1]{\begin{proof}[#1]}{\end{proof}}%
\fi%



% Spaces zum Auffuellen von Tabellenbreiten, die nur im HTML wirken
\ifttm%
\def\MTSP{\:}%
\else%
\def\MTSP{}%
\fi%

\DeclareMathOperator{\arsinh}{arsinh}
\DeclareMathOperator{\arcosh}{arcosh}
\DeclareMathOperator{\artanh}{artanh}
\DeclareMathOperator{\arcoth}{arcoth}


\newcommand{\MMathSet}[1]{\mathbb{#1}}
\def\N{\MMathSet{N}}
\def\Z{\MMathSet{Z}}
\def\Q{\MMathSet{Q}}
\def\R{\MMathSet{R}}
\def\C{\MMathSet{C}}

\newcounter{MForLoopCounter}
\newcommand{\MForLoop}[2]{\setcounter{MForLoopCounter}{#1}\ifnum\value{MForLoopCounter}=0{}\else{{#2}\addtocounter{MForLoopCounter}{-1}\MForLoop{\value{MForLoopCounter}}{#2}}\fi}

\newcounter{MSiteCounter}
\newcounter{MFieldCounter} % Kombination section.subsection.site.field ist eindeutig in allen Modulen, field alleine nicht

\newcounter{MiniMarkerCounter}

\ifttm
\newenvironment{MMiniPageP}[1]{\begin{minipage}{#1\linewidth}\special{html:<!-- minimarker;;}\arabic{MiniMarkerCounter}\special{html:;;#1; //-->}}{\end{minipage}\addtocounter{MiniMarkerCounter}{1}}
\else
\newenvironment{MMiniPageP}[1]{\begin{minipage}{#1\linewidth}}{\end{minipage}\addtocounter{MiniMarkerCounter}{1}}
\fi

\newcounter{AlignCounter}

\newcommand{\MStartJustify}{\ifttm\special{html:<!-- startalign;;}\arabic{AlignCounter}\special{html:;;justify; //-->}\fi}
\newcommand{\MStopJustify}{\ifttm\special{html:<!-- stopalign;;}\arabic{AlignCounter}\special{html:; //-->}\fi\addtocounter{AlignCounter}{1}}

\newenvironment{MJTabular}[1]{
\MStartJustify
\begin{tabular}{#1}
}{
\end{tabular}
\MStopJustify
}

\newcommand{\MImageLeft}[2]{
\begin{center}
\begin{tabular}{lc}
\MStartJustify
\begin{MMiniPageP}{0.65}
#1
\end{MMiniPageP}
\MStopJustify
&
\begin{MMiniPageP}{0.3}
#2  
\end{MMiniPageP}
\end{tabular}
\end{center}
}

\newcommand{\MImageHalf}[2]{
\begin{center}
\begin{tabular}{lc}
\MStartJustify
\begin{MMiniPageP}{0.45}
#1
\end{MMiniPageP}
\MStopJustify
&
\begin{MMiniPageP}{0.45}
#2  
\end{MMiniPageP}
\end{tabular}
\end{center}
}

\newcommand{\MBigImageLeft}[2]{
\begin{center}
\begin{tabular}{lc}
\MStartJustify
\begin{MMiniPageP}{0.25}
#1
\end{MMiniPageP}
\MStopJustify
&
\begin{MMiniPageP}{0.7}
#2  
\end{MMiniPageP}
\end{tabular}
\end{center}
}

\ifttm
\def\No{\mathbb{N}_0}
\else
\def\No{\ensuremath{\N_0}}
\fi
\def\MT{\textrm{\tiny T}}
\newcommand{\MTranspose}[1]{{#1}^{\MT}}
\ifttm
\newcommand{\MRe}{\mathsf{Re}}
\newcommand{\MIm}{\mathsf{Im}}
\else
\DeclareMathOperator{\MRe}{Re}
\DeclareMathOperator{\MIm}{Im}
\fi

\newcommand{\Mid}{\mathrm{id}}
\newcommand{\MFeinheit}{\mathrm{feinh}}

\ifttm
\newcommand{\Msubstack}[1]{\begin{array}{c}{#1}\end{array}}
\else
\newcommand{\Msubstack}[1]{\substack{#1}}
\fi

% Typen von Fragefeldern:
% 1 = Alphanumerisch, case-sensitive-Vergleich
% 2 = Ja/Nein-Checkbox, Loesung ist 0 oder 1   (OPTION = Image-id fuer Rueckmeldung)
% 3 = Reelle Zahlen Geparset
% 4 = Funktionen Geparset (mit Stuetzstellen zur ueberpruefung)

% Dieser Befehl erstellt ein interaktives Aufgabenfeld. Parameter:
% - #1 Laenge in Zeichen
% - #2 Loesungstext (alphanumerisch, case sensitive)
% - #3 AufgabenID (alphanumerisch, case sensitive)
% - #4 Typ (Kennnummer)
% - #5 String fuer Optionen (ggf. mit Semikolon getrennte Einzelstrings)
% - #6 Anzahl Punkte
% - #7 uxid (kann z.B. Loesungsstring sein)
% ACHTUNG: Die langen Zeilen bitte so lassen, Zeilenumbrueche im tex werden in div's umgesetzt
\newcommand{\MQuestionID}[7]{
\ifttm
\special{html:<!-- mdeclareuxid;;}UX#7\special{html:;;}\arabic{section}\special{html:;;}#3\special{html:;; //-->}%
\special{html:<!-- mdeclarepoints;;}\arabic{section}\special{html:;;}#3\special{html:;;}#6\special{html:;;}\arabic{MTestSite}\special{html:;;}\arabic{chapter}%
\special{html:;; //--><!-- onloadstart //-->CreateQuestionObj("}#7\special{html:",}\arabic{MFieldCounter}\special{html:,"}#2%
\special{html:","}#3\special{html:",}#4\special{html:,"}#5\special{html:",}#6\special{html:,}\arabic{MTestSite}\special{html:,}\arabic{section}%
\special{html:);<!-- onloadstop //-->}%
\special{html:<input mfieldtype="}#4\special{html:" name="Name_}#3\special{html:" id="}#3\special{html:" type="text" size="}#1\special{html:" maxlength="}#1%
\special{html:" }\ifnum\value{MGroupActive}=0\special{html:onfocus="handlerFocus(}\arabic{MFieldCounter}%
\special{html:);" onblur="handlerBlur(}\arabic{MFieldCounter}\special{html:);" onkeyup="handlerChange(}\arabic{MFieldCounter}\special{html:,0);" onpaste="handlerChange(}\arabic{MFieldCounter}\special{html:,0);" oninput="handlerChange(}\arabic{MFieldCounter}\special{html:,0);" onpropertychange="handlerChange(}\arabic{MFieldCounter}\special{html:,0);"/>}%
\special{html:<img src="images/questionmark.gif" width="20" height="20" border="0" align="absmiddle" id="}QM#3\special{html:"/>}
\else%
\special{html:onblur="handlerBlur(}\arabic{MFieldCounter}%
\special{html:);" onfocus="handlerFocus(}\arabic{MFieldCounter}\special{html:);" onkeyup="handlerChange(}\arabic{MFieldCounter}\special{html:,1);" onpaste="handlerChange(}\arabic{MFieldCounter}\special{html:,1);" oninput="handlerChange(}\arabic{MFieldCounter}\special{html:,1);" onpropertychange="handlerChange(}\arabic{MFieldCounter}\special{html:,1);"/>}%
\special{html:<img src="images/questionmark.gif" width="20" height="20" border="0" align="absmiddle" id="}QM#3\special{html:"/>}\fi%
\else%
\ifnum\value{QBoxFlag}=1\fbox{$\phantom{\MForLoop{#1}{b}}$}\else$\phantom{\MForLoop{#1}{b}}$\fi%
\fi%
}

% ACHTUNG: Die langen Zeilen bitte so lassen, Zeilenumbrueche im tex werden in div's umgesetzt
% QuestionCheckbox macht ausserhalb einer QuestionGroup keinen Sinn!
% #1 = solution (1 oder 0), ggf. mit ::smc abgetrennt auszuschliessende single-choice-boxen (UXIDs durch , getrennt), #2 = id, #3 = points, #4 = uxid
\newcommand{\MQuestionCheckbox}[4]{
\ifttm
\special{html:<!-- mdeclareuxid;;}UX#4\special{html:;;}\arabic{section}\special{html:;;}#2\special{html:;; //-->}%
\ifnum\value{MGroupActive}=0\MDebugMessage{ERROR: Checkbox Nr. \arabic{MFieldCounter}\ ist nicht in einer Kontrollgruppe, es wird niemals eine Loesung angezeigt!}\fi
\special{html: %
<!-- mdeclarepoints;;}\arabic{section}\special{html:;;}#2\special{html:;;}#3\special{html:;;}\arabic{MTestSite}\special{html:;;}\arabic{chapter}%
\special{html:;; //--><!-- onloadstart //-->CreateQuestionObj("}#4\special{html:",}\arabic{MFieldCounter}\special{html:,"}#1\special{html:","}#2\special{html:",2,"IMG}#2%
\special{html:",}#3\special{html:,}\arabic{MTestSite}\special{html:,}\arabic{section}\special{html:);<!-- onloadstop //-->}%
\special{html:<input mfieldtype="2" type="checkbox" name="Name_}#2\special{html:" id="}#2\special{html:" onchange="handlerChange(}\arabic{MFieldCounter}\special{html:,1);"/><img src="images/questionmark.gif" name="}Name_IMG#2%
\special{html:" width="20" height="20" border="0" align="absmiddle" id="}IMG#2\special{html:"/> }%
\else%
\ifnum\value{QBoxFlag}=1\fbox{$\phantom{X}$}\else$\phantom{X}$\fi%
\fi%
}

\def\MGenerateID{QFELD_\arabic{section}.\arabic{subsection}.\arabic{MSiteCounter}.QF\arabic{MFieldCounter}}

% #1 = 0/1 ggf. mit ::smc abgetrennt auszuschliessende single-choice-boxen (UXIDs durch , getrennt ohne UX), #2 = uxid ohne UX
\newcommand{\MCheckbox}[2]{
\MQuestionCheckbox{#1}{\MGenerateID}{\MStdPoints}{#2}
\addtocounter{MFieldCounter}{1}
}

% Erster Parameter: Zeichenlaenge der Eingabebox, zweiter Parameter: Loesungstext
\newcommand{\MQuestion}[2]{
\MQuestionID{#1}{#2}{\MGenerateID}{1}{0}{\MStdPoints}{#2}
\addtocounter{MFieldCounter}{1}
}

% Erster Parameter: Zeichenlaenge der Eingabebox, zweiter Parameter: Loesungstext
\newcommand{\MLQuestion}[3]{
\MQuestionID{#1}{#2}{\MGenerateID}{1}{0}{\MStdPoints}{#3}
\addtocounter{MFieldCounter}{1}
}

% Parameter: Laenge des Feldes, Loesung (wird auch geparsed), Stellen Genauigkeit hinter dem Komma, weitere Stellen werden mathematisch gerundet vor Vergleich
\newcommand{\MParsedQuestion}[3]{
\MQuestionID{#1}{#2}{\MGenerateID}{3}{#3}{\MStdPoints}{#2}
\addtocounter{MFieldCounter}{1}
}

% Parameter: Laenge des Feldes, Loesung (wird auch geparsed), Stellen Genauigkeit hinter dem Komma, weitere Stellen werden mathematisch gerundet vor Vergleich
\newcommand{\MLParsedQuestion}[4]{
\MQuestionID{#1}{#2}{\MGenerateID}{3}{#3}{\MStdPoints}{#4}
\addtocounter{MFieldCounter}{1}
}

% Parameter: Laenge des Feldes, Loesungsfunktion, Anzahl Stuetzstellen, Funktionsvariablen durch Kommata getrennt (nicht case-sensitive), Anzahl Nachkommastellen im Vergleich
\newcommand{\MFunctionQuestion}[5]{
\MQuestionID{#1}{#2}{\MGenerateID}{4}{#3;#4;#5;0}{\MStdPoints}{#2}
\addtocounter{MFieldCounter}{1}
}

% Parameter: Laenge des Feldes, Loesungsfunktion, Anzahl Stuetzstellen, Funktionsvariablen durch Kommata getrennt (nicht case-sensitive), Anzahl Nachkommastellen im Vergleich, UXID
\newcommand{\MLFunctionQuestion}[6]{
\MQuestionID{#1}{#2}{\MGenerateID}{4}{#3;#4;#5;0}{\MStdPoints}{#6}
\addtocounter{MFieldCounter}{1}
}

% Parameter: Laenge des Feldes, Loesungsintervall, Genauigkeit der Zahlenwertpruefung
\newcommand{\MIntervalQuestion}[3]{
\MQuestionID{#1}{#2}{\MGenerateID}{6}{#3}{\MStdPoints}{#2}
\addtocounter{MFieldCounter}{1}
}

% Parameter: Laenge des Feldes, Loesungsintervall, Genauigkeit der Zahlenwertpruefung, UXID
\newcommand{\MLIntervalQuestion}[4]{
\MQuestionID{#1}{#2}{\MGenerateID}{6}{#3}{\MStdPoints}{#4}
\addtocounter{MFieldCounter}{1}
}

% Parameter: Laenge des Feldes, Loesungsfunktion, Anzahl Stuetzstellen, Funktionsvariable (nicht case-sensitive), Anzahl Nachkommastellen im Vergleich, Vereinfachungsbedingung
% Vereinfachungsbedingung ist eine der Folgenden:
% 0 = Keine Vereinfachungsbedingung
% 1 = Keine Klammern (runde oder eckige) mehr im vereinfachten Ausdruck
% 2 = Faktordarstellung (Term hat Produkte als letzte Operation, Summen als vorgeschaltete Operation)
% 3 = Summendarstellung (Term hat Summen als letzte Operation, Produkte als vorgeschaltete Operation)
% Flag 512: Besondere Stuetzstellen (nur >1 und nur schwach rational), sonst symmetrisch um Nullpunkt und ganze Zahlen inkl. Null werden getroffen
\newcommand{\MSimplifyQuestion}[6]{
\MQuestionID{#1}{#2}{\MGenerateID}{4}{#3;#4;#5;#6}{\MStdPoints}{#2}
\addtocounter{MFieldCounter}{1}
}

\newcommand{\MLSimplifyQuestion}[7]{
\MQuestionID{#1}{#2}{\MGenerateID}{4}{#3;#4;#5;#6}{\MStdPoints}{#7}
\addtocounter{MFieldCounter}{1}
}

% Parameter: Laenge des Feldes, Loesung (optionaler Ausdruck), Anzahl Stuetzstellen, Funktionsvariable (nicht case-sensitive), Anzahl Nachkommastellen im Vergleich, Spezialtyp (string-id)
\newcommand{\MLSpecialQuestion}[7]{
\MQuestionID{#1}{#2}{\MGenerateID}{7}{#3;#4;#5;#6}{\MStdPoints}{#7}
\addtocounter{MFieldCounter}{1}
}

\newcounter{MGroupStart}
\newcounter{MGroupEnd}
\newcounter{MGroupActive}

\newenvironment{MQuestionGroup}{
\setcounter{MGroupStart}{\value{MFieldCounter}}
\setcounter{MGroupActive}{1}
}{
\setcounter{MGroupActive}{0}
\setcounter{MGroupEnd}{\value{MFieldCounter}}
\addtocounter{MGroupEnd}{-1}
}

\newcommand{\MGroupButton}[1]{
\ifttm
\special{html:<button name="Name_Group}\arabic{MGroupStart}\special{html:to}\arabic{MGroupEnd}\special{html:" id="Group}\arabic{MGroupStart}\special{html:to}\arabic{MGroupEnd}\special{html:" %
type="button" onclick="group_button(}\arabic{MGroupStart}\special{html:,}\arabic{MGroupEnd}\special{html:);">}#1\special{html:</button>}
\else
\phantom{#1}
\fi
}

%----------------- Makros fuer die modularisierte Darstellung ------------------------------------

\def\MyText#1{#1}

% is used internally by the conversion package, should not be used by original tex documents
\def\MOrgLabel#1{\relax}

\ifttm

% Ein MLabel wird im html codiert durch das tag <!-- mmlabel;;Labelbezeichner;;SubjectArea;;chapter;;section;;subsection;;Index;;Objekttyp; //-->
\def\MLabel#1{%
\ifnum\value{MLastType}=8%
\ifnum\value{MCaptionOn}=0%
\MDebugMessage{ERROR: Grafik \arabic{MGraphicsCounter} hat separates label: #1 (Grafiklabels sollten nur in der Caption stehen)}%
\fi
\fi
\ifnum\value{MLastType}=12%
\ifnum\value{MCaptionOn}=0%
\MDebugMessage{ERROR: Video \arabic{MVideoCounter} hat separates label: #1 (Videolabels sollten nur in der Caption stehen}%
\fi
\fi
\ifnum\value{MLastType}=10\setcounter{MLastIndex}{\value{equation}}\fi
\label{#1}\begin{html}<!-- mmlabel;;#1;;\end{html}\arabic{MSubjectArea}\special{html:;;}\arabic{chapter}\special{html:;;}\arabic{section}\special{html:;;}\arabic{subsection}\special{html:;;}\arabic{MLastIndex}\special{html:;;}\arabic{MLastType}\special{html:; //-->}}%

\else

% Sonderbehandlung im PDF fuer Abbildungen in separater aux-Datei, da MGraphics die figure-Umgebung nicht verwendet
\def\MLabel#1{%
\ifnum\value{MLastType}=8%
\ifnum\value{MCaptionOn}=0%
\MDebugMessage{ERROR: Grafik \arabic{MGraphicsCounter} hat separates label: #1 (Grafiklabels sollten nur in der Caption stehen}%
\fi
\fi
\ifnum\value{MLastType}=12%
\ifnum\value{MCaptionOn}=0%
\MDebugMessage{ERROR: Video \arabic{MVideoCounter} hat separates label: #1 (Videolabels sollten nur in der Caption stehen}%
\fi
\fi
\label{#1}%
}%

\fi

% Gibt Begriff des referenzierten Objekts mit aus, aber nur im HTML, daher nur in Ausnahmefaellen (z.B. Copyrightliste) sinnvoll
\def\MCRef#1{\ifttm\special{html:<!-- mmref;;}#1\special{html:;;1; //-->}\else\vref{#1}\fi}


\def\MRef#1{\ifttm\special{html:<!-- mmref;;}#1\special{html:;;0; //-->}\else\vref{#1}\fi}
\def\MERef#1{\ifttm\special{html:<!-- mmref;;}#1\special{html:;;0; //-->}\else\eqref{#1}\fi}
\def\MNRef#1{\ifttm\special{html:<!-- mmref;;}#1\special{html:;;0; //-->}\else\ref{#1}\fi}
\def\MSRef#1#2{\ifttm\special{html:<!-- msref;;}#1\special{html:;;}#2\special{html:; //-->}\else \if#2\empty \ref{#1} \else \hyperref[#1]{#2}\fi\fi} 

\def\MRefRange#1#2{\ifttm\MRef{#1} bis 
\MRef{#2}\else\vrefrange[\unskip]{#1}{#2}\fi}

\def\MRefTwo#1#2{\ifttm\MRef{#1} und \MRef{#2}\else%
\let\vRefTLRsav=\reftextlabelrange\let\vRefTPRsav=\reftextpagerange%
\def\reftextlabelrange##1##2{\ref{##1} und~\ref{##2}}%
\def\reftextpagerange##1##2{auf den Seiten~\pageref{#1} und~\pageref{#2}}%
\vrefrange[\unskip]{#1}{#2}%
\let\reftextlabelrange=\vRefTLRsav\let\reftextpagerange=\vRefTPRsav\fi}

% MSectionChapter definiert falls notwendig das Kapitel vor der section. Das ist notwendig, wenn nur ein Einzelmodul uebersetzt wird.
% MChaptersGiven ist ein Counter, der von mconvert.pl vordefiniert wird.
\ifttm
\newcommand{\MSectionChapter}{\ifnum\value{MChaptersGiven}=0{\Dchapter{Modul}}\else{}\fi}
\else
\newcommand{\MSectionChapter}{\ifnum\value{chapter}=0{\Dchapter{Modul}}\else{}\fi}
\fi


\def\MChapter#1{\ifnum\value{MSSEnd}>0{\MSubsectionEndMacros}\addtocounter{MSSEnd}{-1}\fi\Dchapter{#1}}
\def\MSubject#1{\MChapter{#1}} % Schluesselwort HELPSECTION ist reserviert fuer Hilfesektion

\newcommand{\MSectionID}{UNKNOWNID}

\ifttm
\newcommand{\MSetSectionID}[1]{\renewcommand{\MSectionID}{#1}}
\else
\newcommand{\MSetSectionID}[1]{\renewcommand{\MSectionID}{#1}\tikzsetexternalprefix{#1}}
\fi


\newcommand{\MSection}[1]{\MSetSectionID{MODULID}\ifnum\value{MSSEnd}>0{\MSubsectionEndMacros}\addtocounter{MSSEnd}{-1}\fi\MSectionChapter\Dsection{#1}\MSectionStartMacros{#1}\setcounter{MLastIndex}{-1}\setcounter{MLastType}{1}} % Sections werden ueber das section-Feld im mmlabel-Tag identifiziert, nicht ueber das Indexfeld

\def\MSubsection#1{\ifnum\value{MSSEnd}>0{\MSubsectionEndMacros}\addtocounter{MSSEnd}{-1}\fi\ifttm\else\clearpage\fi\Dsubsection{#1}\MSubsectionStartMacros\setcounter{MLastIndex}{-1}\setcounter{MLastType}{2}\addtocounter{MSSEnd}{1}}% Subsections werden ueber das subsection-Feld im mmlabel-Tag identifiziert, nicht ueber das Indexfeld
\def\MSubsectionx#1{\Dsubsectionx{#1}} % Nur zur Verwendung in MSectionStart gedacht
\def\MSubsubsection#1{\Dsubsubsection{#1}\setcounter{MLastIndex}{\value{subsubsection}}\setcounter{MLastType}{3}\ifttm\special{html:<!-- sectioninfo;;}\arabic{section}\special{html:;;}\arabic{subsection}\special{html:;;}\arabic{subsubsection}\special{html:;;1;;}\arabic{MTestSite}\special{html:; //-->}\fi}
\def\MSubsubsectionx#1{\Dsubsubsectionx{#1}\ifttm\special{html:<!-- sectioninfo;;}\arabic{section}\special{html:;;}\arabic{subsection}\special{html:;;}\arabic{subsubsection}\special{html:;;0;;}\arabic{MTestSite}\special{html:; //-->}\else\addcontentsline{toc}{subsection}{#1}\fi}

\ifttm
\def\MSubsubsubsectionx#1{\ \newline\textbf{#1}\special{html:<br />}}
\else
\def\MSubsubsubsectionx#1{\ \newline
\textbf{#1}\ \\
}
\fi


% Dieses Skript wird zu Beginn jedes Modulabschnitts (=Webseite) ausgefuehrt und initialisiert den Aufgabenfeldzaehler
\newcommand{\MPageScripts}{
\setcounter{MFieldCounter}{1}
\addtocounter{MSiteCounter}{1}
\setcounter{MHintCounter}{1}
\setcounter{MCodeEditCounter}{1}
\setcounter{MGroupActive}{0}
\DoQBoxes
% Feldvariablen werden im HTML-Header in conv.pl eingestellt
}

% Dieses Skript wird zum Ende jedes Modulabschnitts (=Webseite) ausgefuehrt
\ifttm
\newcommand{\MEndScripts}{\special{html:<br /><!-- mfeedbackbutton;Seite;}\arabic{MTestSite}\special{html:;}\MGenerateSiteNumber\special{html:; //-->}
}
\else
\newcommand{\MEndScripts}{\relax}
\fi


\newcounter{QBoxFlag}
\newcommand{\DoQBoxes}{\setcounter{QBoxFlag}{1}}
\newcommand{\NoQBoxes}{\setcounter{QBoxFlag}{0}}

\newcounter{MXCTest}
\newcounter{MXCounter}
\newcounter{MSCounter}



\ifttm

% Struktur des sectioninfo-Tags: <!-- sectioninfo;;section;;subsection;;subsubsection;;nr_ausgeben;;testpage; //-->

%Fuegt eine zusaetzliche html-Seite an hinter ALLEN bisherigen und zukuenftigen content-Seiten ausserhalb der vor-zurueck-Schleife (d.h. nur durch Button oder MIntLink erreichbar!)
% #1 = Titel des Modulabschnitts, #2 = Kurztitel fuer die Buttons, #3 = Buttonkennung (STD = default nehmen, NONE = Ohne Button in der Navigation)
\newenvironment{MSContent}[3]{\special{html:<div class="xcontent}\arabic{MSCounter}\special{html:"><!-- scontent;-;}\arabic{MSCounter};-;#1;-;#2;-;#3\special{html: //-->}\MPageScripts\MSubsubsectionx{#1}}{\MEndScripts\special{html:<!-- endscontent;;}\arabic{MSCounter}\special{html: //--></div>}\addtocounter{MSCounter}{1}}

% Fuegt eine zusaetzliche html-Seite ein hinter den bereits vorhandenen content-Seiten (oder als erste Seite) innerhalb der vor-zurueck-Schleife der Navigation
% #1 = Titel des Modulabschnitts, #2 = Kurztitel fuer die Buttons, #3 = Buttonkennung (STD = Defaultbutton, NONE = Ohne Button in der Navigation)
\newenvironment{MXContent}[3]{\special{html:<div class="xcontent}\arabic{MXCounter}\special{html:"><!-- xcontent;-;}\arabic{MXCounter};-;#1;-;#2;-;#3\special{html: //-->}\MPageScripts\MSubsubsection{#1}}{\MEndScripts\special{html:<!-- endxcontent;;}\arabic{MXCounter}\special{html: //--></div>}\addtocounter{MXCounter}{1}}

% Fuegt eine zusaetzliche html-Seite ein die keine subsubsection-Nummer bekommt, nur zur internen Verwendung in mintmod.tex gedacht!
% #1 = Titel des Modulabschnitts, #2 = Kurztitel fuer die Buttons, #3 = Buttonkennung (STD = Defaultbutton, NONE = Ohne Button in der Navigation)
% \newenvironment{MUContent}[3]{\special{html:<div class="xcontent}\arabic{MXCounter}\special{html:"><!-- xcontent;-;}\arabic{MXCounter};-;#1;-;#2;-;#3\special{html: //-->}\MPageScripts\MSubsubsectionx{#1}}{\MEndScripts\special{html:<!-- endxcontent;;}\arabic{MXCounter}\special{html: //--></div>}\addtocounter{MXCounter}{1}}

\newcommand{\MDeclareSiteUXID}[1]{\special{html:<!-- mdeclaresiteuxid;;}#1\special{html:;;}\arabic{chapter}\special{html:;;}\arabic{section}\special{html:;; //-->}}

\else

%\newcommand{\MSubsubsection}[1]{\refstepcounter{subsubsection} \addcontentsline{toc}{subsubsection}{\thesubsubsection. #1}}


% Fuegt eine zusaetzliche html-Seite an hinter den bereits vorhandenen content-Seiten
% #1 = Titel des Modulabschnitts, #2 = Kurztitel fuer die Buttons, #3 = Iconkennung (im PDF wirkungslos)
%\newenvironment{MUContent}[3]{\ifnum\value{MXCTest}>0{\MDebugMessage{ERROR: Geschachtelter SContent}}\fi\MPageScripts\MSubsubsectionx{#1}\addtocounter{MXCTest}{1}}{\addtocounter{MXCounter}{1}\addtocounter{MXCTest}{-1}}
\newenvironment{MXContent}[3]{\ifnum\value{MXCTest}>0{\MDebugMessage{ERROR: Geschachtelter SContent}}\fi\MPageScripts\MSubsubsection{#1}\addtocounter{MXCTest}{1}}{\addtocounter{MXCounter}{1}\addtocounter{MXCTest}{-1}}
\newenvironment{MSContent}[3]{\ifnum\value{MXCTest}>0{\MDebugMessage{ERROR: Geschachtelter XContent}}\fi\MPageScripts\MSubsubsectionx{#1}\addtocounter{MXCTest}{1}}{\addtocounter{MSCounter}{1}\addtocounter{MXCTest}{-1}}

\newcommand{\MDeclareSiteUXID}[1]{\relax}

\fi 

% GHEADER und GFOOTER werden von split.pm gefunden, aber nur, wenn nicht HELPSITE oder TESTSITE
\ifttm
\newenvironment{MSectionStart}{\special{html:<div class="xcontent0">}\MSubsubsectionx{Modul\"ubersicht}}{\setcounter{MSSEnd}{0}\special{html:</div>}}
% Darf nicht als XContent nummeriert werden, darf nicht als XContent gelabelt werden, wird aber in eine xcontent-div gesetzt fuer Python-parsing
\else
\newenvironment{MSectionStart}{\MSubsectionx{Modul\"ubersicht}}{\setcounter{MSSEnd}{0}}
\fi

\newenvironment{MIntro}{\begin{MXContent}{Einf\"uhrung}{Einf\"uhrung}{genetisch}}{\end{MXContent}}
\newenvironment{MContent}{\begin{MXContent}{Inhalt}{Inhalt}{beweis}}{\end{MXContent}}
\newenvironment{MExercises}{\ifttm\else\clearpage\fi\begin{MXContent}{Aufgaben}{Aufgaben}{aufgb}\special{html:<!-- declareexcsymb //-->}}{\end{MXContent}}

% #1 = Lesbare Testbezeichnung
\newenvironment{MTest}[1]{%
\renewcommand{\MTestName}{#1}
\ifttm\else\clearpage\fi%
\addtocounter{MTestSite}{1}%
\begin{MXContent}{#1}{#1}{STD} % {aufgb}%
\special{html:<!-- declaretestsymb //-->}
\begin{MQuestionGroup}%
\MInTestHeader
}%
{%
\end{MQuestionGroup}%
\ \\ \ \\%
\MInTestFooter
\end{MXContent}\addtocounter{MTestSite}{-1}%
}

\newenvironment{MExtra}{\ifttm\else\clearpage\fi\begin{MXContent}{Zus\"atzliche Inhalte}{Zusatz}{weiterfhrg}}{\end{MXContent}}

\makeindex

\ifttm
\def\MPrintIndex{
\ifnum\value{MSSEnd}>0{\MSubsectionEndMacros}\addtocounter{MSSEnd}{-1}\fi
\renewcommand{\indexname}{Stichwortverzeichnis}
\special{html:<p><!-- printindex //--></p>}
}
\else
\def\MPrintIndex{
\ifnum\value{MSSEnd}>0{\MSubsectionEndMacros}\addtocounter{MSSEnd}{-1}\fi
\renewcommand{\indexname}{Stichwortverzeichnis}
\addcontentsline{toc}{section}{Stichwortverzeichnis}
\printindex
}
\fi


% Konstanten fuer die Modulfaecher

\def\MINTMathematics{1}
\def\MINTInformatics{2}
\def\MINTChemistry{3}
\def\MINTPhysics{4}
\def\MINTEngineering{5}

\newcounter{MSubjectArea}
\newcounter{MInfoNumbers} % Gibt an, ob die Infoboxen nummeriert werden sollen
\newcounter{MSepNumbers} % Gibt an, ob Beispiele und Experimente separat nummeriert werden sollen
\newcommand{\MSetSubject}[1]{
 % ttm kapiert setcounter mit Parametern nicht, also per if abragen und einsetzen
\ifnum#1=1\setcounter{MSubjectArea}{1}\setcounter{MInfoNumbers}{1}\setcounter{MSepNumbers}{0}\fi
\ifnum#1=2\setcounter{MSubjectArea}{2}\setcounter{MInfoNumbers}{1}\setcounter{MSepNumbers}{0}\fi
\ifnum#1=3\setcounter{MSubjectArea}{3}\setcounter{MInfoNumbers}{0}\setcounter{MSepNumbers}{1}\fi
\ifnum#1=4\setcounter{MSubjectArea}{4}\setcounter{MInfoNumbers}{0}\setcounter{MSepNumbers}{0}\fi
\ifnum#1=5\setcounter{MSubjectArea}{5}\setcounter{MInfoNumbers}{1}\setcounter{MSepNumbers}{0}\fi
% Separate Nummerntechnik fuer unsere Chemiker: alles dreistellig
\ifnum#1=3
  \ifttm
  \renewcommand{\theequation}{\arabic{section}.\arabic{subsection}.\arabic{equation}}
  \renewcommand{\thetable}{\arabic{section}.\arabic{subsection}.\arabic{table}} 
  \renewcommand{\thefigure}{\arabic{section}.\arabic{subsection}.\arabic{figure}} 
  \else
  \renewcommand{\theequation}{\arabic{chapter}.\arabic{section}.\arabic{equation}}
  \renewcommand{\thetable}{\arabic{chapter}.\arabic{section}.\arabic{table}}
  \renewcommand{\thefigure}{\arabic{chapter}.\arabic{section}.\arabic{figure}}
  \fi
\else
  \ifttm
  \renewcommand{\theequation}{\arabic{section}.\arabic{subsection}.\arabic{equation}}
  \renewcommand{\thetable}{\arabic{table}}
  \renewcommand{\thefigure}{\arabic{figure}}
  \else
  \renewcommand{\theequation}{\arabic{chapter}.\arabic{section}.\arabic{equation}}
  \renewcommand{\thetable}{\arabic{table}}
  \renewcommand{\thefigure}{\arabic{figure}}
  \fi
\fi
}

% Fuer tikz Autogenerierung
\newcounter{MTIKZAutofilenumber}

% Spezielle Counter fuer die Bentz-Module
\newcounter{mycounter}
\newcounter{chemapplet}
\newcounter{physapplet}

\newcounter{MSSEnd} % Ist 1 falls ein MSubsection aktiv ist, der einen MSubsectionEndMacro-Aufruf verursacht
\newcounter{MFileNumber}
\def\MLastFile{\special{html:[[!-- mfileref;;}\arabic{MFileNumber}\special{html:; //--]]}}

% Vollstaendiger Pfad ist \MMaterial / \MLastFilePath / \MLastFileName    ==   \MMaterial / \MLastFile

% Wird nur bei kompletter Baum-Erstellung ausgefuehrt!
% #1 = Lesbare Modulbezeichnung
\newcommand{\MSectionStartMacros}[1]{
\setcounter{MTestSite}{0}
\setcounter{MCaptionOn}{0}
\setcounter{MLastTypeEq}{0}
\setcounter{MSSEnd}{0}
\setcounter{MFileNumber}{0} % Preinkrekement-Counter
\setcounter{MTIKZAutofilenumber}{0}
\setcounter{mycounter}{1}
\setcounter{physapplet}{1}
\setcounter{chemapplet}{0}
\ifttm
\special{html:<!-- mdeclaresection;;}\arabic{chapter}\special{html:;;}\arabic{section}\special{html:;;}#1\special{html:;; //-->}%
\else
\setcounter{thmc}{0}
\setcounter{exmpc}{0}
\setcounter{verc}{0}
\setcounter{infoc}{0}
\fi
\setcounter{MiniMarkerCounter}{1}
\setcounter{AlignCounter}{1}
\setcounter{MXCTest}{0}
\setcounter{MCodeCounter}{0}
\setcounter{MEntryCounter}{0}
}

% Wird immer ausgefuehrt
\newcommand{\MSubsectionStartMacros}{
\ifttm\else\MPageHeaderDef\fi
\MWatermarkSettings
\setcounter{MXCounter}{0}
\setcounter{MSCounter}{0}
\setcounter{MSiteCounter}{1}
\setcounter{MExerciseCollectionCounter}{0}
% Zaehler fuer das Labelsystem zuruecksetzen (prefix-Zaehler)
\setcounter{MInfoCounter}{0}
\setcounter{MExerciseCounter}{0}
\setcounter{MExampleCounter}{0}
\setcounter{MExperimentCounter}{0}
\setcounter{MGraphicsCounter}{0}
\setcounter{MTableCounter}{0}
\setcounter{MTheoremCounter}{0}
\setcounter{MObjectCounter}{0}
\setcounter{MEquationCounter}{0}
\setcounter{MVideoCounter}{0}
\setcounter{equation}{0}
\setcounter{figure}{0}
}

\newcommand{\MSubsectionEndMacros}{
% Bei Chemiemodulen das PSE einhaengen, es soll als SContent am Ende erscheinen
\special{html:<!-- subsectionend //-->}
\ifnum\value{MSubjectArea}=3{\MIncludePSE}\fi
}


\ifttm
%\newcommand{\MEmbed}[1]{\MRegisterFile{#1}\begin{html}<embed src="\end{html}\MMaterial/\MLastFile\begin{html}" width="192" height="189"></embed>\end{html}}
\newcommand{\MEmbed}[1]{\MRegisterFile{#1}\begin{html}<embed src="\end{html}\MMaterial/\MLastFile\begin{html}"></embed>\end{html}}
\fi

%----------------- Makros fuer die Textdarstellung -----------------------------------------------

\ifttm
% MUGraphics bindet eine Grafik ein:
% Parameter 1: Dateiname der Grafik, relativ zur Position des Modul-Tex-Dokuments
% Parameter 2: Skalierungsoptionen fuer PDF (fuer includegraphics)
% Parameter 3: Titel fuer die Grafik, wird unter die Grafik mit der Grafiknummer gesetzt und kann MLabel bzw. MCopyrightLabel enthalten
% Parameter 4: Skalierungsoptionen fuer HTML (css-styles)

% ERSATZ: <img alt="My Image" src="data:image/png;base64,iVBORwA<MoreBase64SringHere>" />


\newcommand{\MUGraphics}[4]{\MRegisterFile{#1}\begin{html}
<div class="imagecenter">
<center>
<div>
<img src="\end{html}\MMaterial/\MLastFile\begin{html}" style="#4" alt="\end{html}\MMaterial/\MLastFile\begin{html}"/>
</div>
<div class="bildtext">
\end{html}
\addtocounter{MGraphicsCounter}{1}
\setcounter{MLastIndex}{\value{MGraphicsCounter}}
\setcounter{MLastType}{8}
\addtocounter{MCaptionOn}{1}
\ifnum\value{MSepNumbers}=0
\textbf{Abbildung \arabic{MGraphicsCounter}:} #3
\else
\textbf{Abbildung \arabic{section}.\arabic{subsection}.\arabic{MGraphicsCounter}:} #3
\fi
\addtocounter{MCaptionOn}{-1}
\begin{html}
</div>
</center>
</div>
<br />
\end{html}%
\special{html:<!-- mfeedbackbutton;Abbildung;}\arabic{MGraphicsCounter}\special{html:;}\arabic{section}.\arabic{subsection}.\arabic{MGraphicsCounter}\special{html:; //-->}%
}

% MVideo bindet ein Video als Einzeldatei ein:
% Parameter 1: Dateiname des Videos, relativ zur Position des Modul-Tex-Dokuments, ohne die Endung ".mp4"
% Parameter 2: Titel fuer das Video (kann MLabel oder MCopyrightLabel enthalten), wird unter das Video mit der Videonummer gesetzt
\newcommand{\MVideo}[2]{\MRegisterFile{#1.mp4}\begin{html}
<div class="imagecenter">
<center>
<div>
<video width="95\%" controls="controls"><source src="\end{html}\MMaterial/#1.mp4\begin{html}" type="video/mp4">Ihr Browser kann keine MP4-Videos abspielen!</video>
</div>
<div class="bildtext">
\end{html}
\addtocounter{MVideoCounter}{1}
\setcounter{MLastIndex}{\value{MVideoCounter}}
\setcounter{MLastType}{12}
\addtocounter{MCaptionOn}{1}
\ifnum\value{MSepNumbers}=0
\textbf{Video \arabic{MVideoCounter}:} #2
\else
\textbf{Video \arabic{section}.\arabic{subsection}.\arabic{MVideoCounter}:} #2
\fi
\addtocounter{MCaptionOn}{-1}
\begin{html}
</div>
</center>
</div>
<br />
\end{html}}

\newcommand{\MDVideo}[2]{\MRegisterFile{#1.mp4}\MRegisterFile{#1.ogv}\begin{html}
<div class="imagecenter">
<center>
<div>
<video width="70\%" controls><source src="\end{html}\MMaterial/#1.mp4\begin{html}" type="video/mp4"><source src="\end{html}\MMaterial/#1.ogv\begin{html}" type="video/ogg">Ihr Browser kann keine MP4-Videos abspielen!</video>
</div>
<br />
#2
</center>
</div>
<br />
\end{html}
}

\newcommand{\MGraphics}[3]{\MUGraphics{#1}{#2}{#3}{}}

\else

\newcommand{\MVideo}[2]{%
% Kein Video im PDF darstellbar, trotzdem so tun als ob da eines waere
\begin{center}
(Video nicht darstellbar)
\end{center}
\addtocounter{MVideoCounter}{1}
\setcounter{MLastIndex}{\value{MVideoCounter}}
\setcounter{MLastType}{12}
\addtocounter{MCaptionOn}{1}
\ifnum\value{MSepNumbers}=0
\textbf{Video \arabic{MVideoCounter}:} #2
\else
\textbf{Video \arabic{section}.\arabic{subsection}.\arabic{MVideoCounter}:} #2
\fi
\addtocounter{MCaptionOn}{-1}
}


% MGraphics bindet eine Grafik ein:
% Parameter 1: Dateiname der Grafik, relativ zur Position des Modul-Tex-Dokuments
% Parameter 2: Skalierungsoptionen fuer PDF (fuer includegraphics)
% Parameter 3: Titel fuer die Grafik, wird unter die Grafik mit der Grafiknummer gesetzt
\newcommand{\MGraphics}[3]{%
\MRegisterFile{#1}%
\ %
\begin{figure}[H]%
\centering{%
\includegraphics[#2]{\MDPrefix/#1}%
\addtocounter{MCaptionOn}{1}%
\caption{#3}%
\addtocounter{MCaptionOn}{-1}%
}%
\end{figure}%
\addtocounter{MGraphicsCounter}{1}\setcounter{MLastIndex}{\value{MGraphicsCounter}}\setcounter{MLastType}{8}\ %
%\ \\Abbildung \ifnum\value{MSepNumbers}=0\else\arabic{chapter}.\arabic{section}.\fi\arabic{MGraphicsCounter}: #3%
}

\newcommand{\MUGraphics}[4]{\MGraphics{#1}{#2}{#3}}


\fi

\newcounter{MCaptionOn} % = 1 falls eine Grafikcaption aktiv ist, = 0 sonst


% MGraphicsSolo bindet eine Grafik pur ein ohne Titel
% Parameter 1: Dateiname der Grafik, relativ zur Position des Modul-Tex-Dokuments
% Parameter 2: Skalierungsoptionen (wirken nur im PDF)
\newcommand{\MGraphicsSolo}[2]{\MUGraphicsSolo{#1}{#2}{}}

% MUGraphicsSolo bindet eine Grafik pur ein ohne Titel, aber mit HTML-Skalierung
% Parameter 1: Dateiname der Grafik, relativ zur Position des Modul-Tex-Dokuments
% Parameter 2: Skalierungsoptionen (wirken nur im PDF)
% Parameter 3: Skalierungsoptionen (wirken nur im HTML), als style-format: "width=???, height=???"
\ifttm
\newcommand{\MUGraphicsSolo}[3]{\MRegisterFile{#1}\begin{html}
<img src="\end{html}\MMaterial/\MLastFile\begin{html}" style="\end{html}#3\begin{html}" alt="\end{html}\MMaterial/\MLastFile\begin{html}"/>
\end{html}%
\special{html:<!-- mfeedbackbutton;Abbildung;}#1\special{html:;}\MMaterial/\MLastFile\special{html:; //-->}%
}
\else
\newcommand{\MUGraphicsSolo}[3]{\MRegisterFile{#1}\includegraphics[#2]{\MDPrefix/#1}}
\fi

% Externer Link mit URL
% Erster Parameter: Vollstaendige(!) URL des Links
% Zweiter Parameter: Text fuer den Link
\newcommand{\MExtLink}[2]{\ifttm\special{html:<a target="_new" href="}#1\special{html:">}#2\special{html:</a>}\else\href{#1}{#2}\fi} % ohne MINTERLINK!


% Interner Link, die verlinkte Datei muss im gleichen Verzeichnis liegen wie die Modul-Texdatei
% Erster Parameter: Dateiname
% Zweiter Parameter: Text fuer den Link
\newcommand{\MIntLink}[2]{\ifttm\MRegisterFile{#1}\special{html:<a class="MINTERLINK" target="_new" href="}\MMaterial/\MLastFile\special{html:">}#2\special{html:</a>}\else{\href{#1}{#2}}\fi}


\ifttm
\def\MMaterial{:localmaterial:}
\else
\def\MMaterial{\MDPrefix}
\fi

\ifttm
\def\MNoFile#1{:directmaterial:#1}
\else
\def\MNoFile#1{#1}
\fi

\newcommand{\MChem}[1]{$\mathrm{#1}$}

\newcommand{\MApplet}[3]{
% Bindet ein Java-Applet ein, die Parameter sind:
% (wird nur im HTML, aber nicht im PDF erstellt)
% #1 Dateiname des Applets (muss mit ".class" enden)
% #2 = Breite in Pixeln
% #3 = Hoehe in Pixeln
\ifttm
\MRegisterFile{#1}
\begin{html}
<applet code="\end{html}\MMaterial/\MLastFile\begin{html}" width="#2" height="#3" alt="[Java-Applet kann nicht gestartet werden]"></applet>
\end{html}
\fi
}

\newcommand{\MScriptPage}[2]{
% Bindet eine JavaScript-Datei ein, die eine eigene Seite bekommt
% (wird nur im HTML, aber nicht im PDF erstellt)
% #1 Dateiname des Programms (sollte mit ".js" enden)
% #2 = Kurztitel der Seite
\ifttm
\begin{MSContent}{#2}{#2}{puzzle}
\MRegisterFile{#1}
\begin{html}
<script src="\MMaterial/\MLastFile" type="text/javascript"></script>
\end{html}
\end{MSContent}
\fi
}

\newcommand{\MIncludePSE}{
% Bindet bei Chemie-Modulen das PSE ein
% (wird nur im HTML, aber nicht im PDF erstellt)
\ifttm
\special{html:<!-- includepse //-->}
\begin{MSContent}{Periodensystem der Elemente}{PSE}{table}
\MRegisterFile{../files/pse.js}
\MRegisterFile{../files/radio.png}
\begin{html}
<script src="\MMaterial/../files/pse.js" type="text/javascript"></script>
<p id="divid"><br /><br />
<script language="javascript" type="text/javascript">
    startpse("divid","\MMaterial/../files"); 
</script>
</p>
<br />
<br />
<br />
<p>Die Farben der Elementsymbole geben an: <font style="color:Red">gasf&ouml;rmig </font> <font style="color:Blue">fl&uuml;ssig </font> fest</p>
<p>Die Elemente der Gruppe 1 A, 2 A, 3 A usw. geh&ouml;ren zu den Hauptgruppenelementen.</p>
<p>Die Elemente der Gruppe 1 B, 2 B, 3 B usw. geh&ouml;ren zu den Nebengruppenelementen.</p>
<p>() kennzeichnet die Masse des stabilsten Isotops</p>
\end{html}
\end{MSContent}
\fi
}

\newcommand{\MAppletArchive}[4]{
% Bindet ein Java-Applet ein, die Parameter sind:
% (wird nur im HTML, aber nicht im PDF erstellt)
% #1 Dateiname der Klasse mit Appletaufruf (muss mit ".class" enden)
% #2 Dateiname des Archivs (muss mit ".jar" enden)
% #3 = Breite in Pixeln
% #4 = Hoehe in Pixeln
\ifttm
\MRegisterFile{#2}
\begin{html}
<applet code="#1" archive="\end{html}\MMaterial/\MLastFile\begin{html}" codebase="." width="#3" height="#4" alt="[Java-Archiv kann nicht gestartet werden]"></applet>
\end{html}
\fi
}

% Bindet in der Haupttexdatei ein MINT-Modul ein. Parameter 1 ist das Verzeichnis (relativ zur Haupttexdatei), Parameter 2 ist der Dateinahme ohne Pfad.
\newcommand{\IncludeModule}[2]{
\renewcommand{\MDPrefix}{#1}
\input{#1/#2}
\ifnum\value{MSSEnd}>0{\MSubsectionEndMacros}\addtocounter{MSSEnd}{-1}\fi
}

% Der ttm-Konverter setzt keine Makros im \input um, also muss hier getrickst werden:
% Das MDPrefix muss in den einzelnen Modulen manuell eingesetzt werden
\newcommand{\MInputFile}[1]{
\ifttm
\input{#1}
\else
\input{#1}
\fi
}


\newcommand{\MCases}[1]{\left\lbrace{\begin{array}{rl} #1 \end{array}}\right.}

\ifttm
\newenvironment{MCaseEnv}{\left\lbrace\begin{array}{rl}}{\end{array}\right.}
\else
\newenvironment{MCaseEnv}{\left\lbrace\begin{array}{rl}}{\end{array}\right.}
\fi

\def\MSkip{\ifttm\MCR\fi}

\ifttm
\def\MCR{\special{html:<br />}}
\else
\def\MCR{\ \\}
\fi


% Pragmas - Sind Schluesselwoerter, die dem Preprocessing sowie dem Konverter uebergeben werden und bestimmte
%           Aktionen ausloesen. Im Output (PDF und HTML) tauchen sie nicht auf.
\newcommand{\MPragma}[1]{%
\ifttm%
\special{html:<!-- mpragma;-;}#1\special{html:;; -->}%
\else%
% MPragmas werden vom Preprozessor direkt im LaTeX gefunden
\fi%
}

% Ersatz der Befehle textsubscript und textsuperscript, die ttm nicht kennt
\ifttm%
\newcommand{\MTextsubscript}[1]{\special{html:<sub>}#1\special{html:</sub>}}%
\newcommand{\MTextsuperscript}[1]{\special{html:<sup>}#1\special{html:</sup>}}%
\else%
\newcommand{\MTextsubscript}[1]{\textsubscript{#1}}%
\newcommand{\MTextsuperscript}[1]{\textsuperscript{#1}}%
\fi

%------------------ Einbindung von dia-Diagrammen ----------------------------------------------
% Beim preprocessing wird aus jeder dia-Datei eine tex-Datei und eine pdf-Datei erzeugt,
% diese werden hier jeweils im PDF und HTML eingebunden
% Parameter: Dateiname der mit dia erstellten Datei (OHNE die Endung .dia)
\ifttm%
\newcommand{\MDia}[1]{%
\MGraphicsSolo{#1minthtml.png}{}%
}
\else%
\newcommand{\MDia}[1]{%
\MGraphicsSolo{#1mintpdf.png}{scale=0.1667}%
}
\fi%

% subsup funktioniert im Ausdruck $D={\R}^+_0$, also \R geklammert und sup zuerst
% \ifttm
% \def\MSubsup#1#2#3{\special{html:<msubsup>} #1 #2 #3\special{html:</msubsup>}}
% \else
% \def\MSubsup#1#2#3{{#1}^{#3}_{#2}}
% \fi

%\input{local.tex}

% \ifttm
% \else
% \newwrite\mintlog
% \immediate\openout\mintlog=mintlog.txt
% \fi

% ----------------------- tikz autogenerator -------------------------------------------------------------------

\newcommand{\Mtikzexternalize}{\tikzexternalize}% wird bei Konvertierung ueber mconvert ggf. ausgehebelt!

\ifttm
\else
\tikzset%
{
  % Defines a custom style which generates pdf and converts to (low and hi-res quality) png and svg, then deletes the pdf
  % Important: DO NOT directly convert from pdf to hires-png or from svg to png with GraphViz convert as it has some problems and memory leaks
  png export/.style=%
  {
    external/system call/.add={}{; 
      pdf2svg "\image.pdf" "\image.svg" ; 
      convert -density 112.5 -transparent white "\image.pdf" "\image.png"; 
      inkscape --export-png="\image.4x.png" --export-dpi=450 --export-background-opacity=0 --without-gui "\image.svg"; 
      rm "\image.pdf"; rm "\image.log"; rm "\image.dpth"; rm "\image.idx"
    },
    external/force remake,
  }
}
\tikzset{png export}
\tikzsetexternalprefix{}
% PNGs bei externer Erzeugung in "richtiger" Groesse einbinden
\pgfkeys{/pgf/images/include external/.code={\includegraphics[scale=0.64]{#1}}}
\fi

% Spezielle Umgebung fuer Autogenerierung, Bildernamen sind nur innerhalb eines Moduls (einer MSection) eindeutig)

\newcommand{\MTIKZautofilename}{tikzautofile}

\ifttm
% HTML-Version: Vom Autogenerator erzeugte png-Datei einbinden, tikz selbst nicht ausfuehren (sprich: #1 schlucken)
\newcommand{\MTikzAuto}[1]{%
\addtocounter{MTIKZAutofilenumber}{1}
\renewcommand{\MTIKZautofilename}{mtikzauto_\arabic{MTIKZAutofilenumber}}
\MUGraphicsSolo{\MSectionID\MTIKZautofilename.4x.png}{scale=1}{\special{html:[[!-- svgstyle;}\MSectionID\MTIKZautofilename\special{html: //--]]}} % Styleinfos werden aus original-png, nicht 4x-png geholt!
%\MRegisterFile{\MSectionID\MTIKZautofilename.png} % not used right now
%\MRegisterFile{\MSectionID\MTIKZautofilename.svg}
}
\else%
% PDF-Version: Falls Autogenerator aktiv wird Datei automatisch benannt und exportiert
\newcommand{\MTikzAuto}[1]{%
\addtocounter{MTIKZAutofilenumber}{1}%
\renewcommand{\MTIKZautofilename}{mtikzauto_\arabic{MTIKZAutofilenumber}}
\tikzsetnextfilename{\MTIKZautofilename}%
#1%
}
\fi

% In einer reinen LaTeX-Uebersetzung kapselt der Preambelinclude-Befehl nur input,
% in einer konvertergesteuerten PDF/HTML-Uebersetzung wird er dagegen entfernt und
% die Preambeln an mintmod angehaengt, die Ersetzung wird von mconvert.pl vorgenommen.

\newcommand{\MPreambleInclude}[1]{\input{#1}}

% Globale Watermarksettings (werden auch nochmal zu Beginn jedes subsection gesetzt,
% muessen hier aber auch global ausgefuehrt wegen Einfuehrungsseiten und Inhaltsverzeichnis

\MWatermarkSettings
% ---------------------------------- Parametrisierte Aufgaben ----------------------------------------

\ifttm
\newenvironment{MPExercise}{%
\begin{MExercise}%
}{%
\special{html:<button name="Name_MPEX}\arabic{MExerciseCounter}\special{html:" id="MPEX}\arabic{MExerciseCounter}%
\special{html:" type="button" onclick="reroll('}\arabic{MExerciseCounter}\special{html:');">Neue Aufgabe erzeugen</button>}%
\end{MExercise}%
}
\else
\newenvironment{MPExercise}{%
\begin{MExercise}%
}{%
\end{MExercise}%
}
\fi

% Parameter: Name, Min, Max, PDF-Standard. Name in Deklaration OHNE backslash, im Code MIT Backslash
\ifttm
\newcommand{\MGlobalInteger}[4]{\special{html:%
<!-- onloadstart //-->%
MVAR.push(createGlobalInteger("}#1\special{html:",}#2\special{html:,}#3\special{html:,}#4\special{html:)); %
<!-- onloadstop //-->%
<!-- viewmodelstart //-->%
ob}#1\special{html:: ko.observable(rerollMVar("}#1\special{html:")),%
<!-- viewmodelstop //-->%
}%
}%
\else%
\newcommand{\MGlobalInteger}[4]{\newcounter{mvc_#1}\setcounter{mvc_#1}{#4}}
\fi

% Parameter: Name, Min, Max, PDF-Standard. Name in Deklaration OHNE backslash, im Code MIT Backslash, Wert ist Wurzel von value
\ifttm
\newcommand{\MGlobalSqrt}[4]{\special{html:%
<!-- onloadstart //-->%
MVAR.push(createGlobalSqrt("}#1\special{html:",}#2\special{html:,}#3\special{html:,}#4\special{html:)); %
<!-- onloadstop //-->%
<!-- viewmodelstart //-->%
ob}#1\special{html:: ko.observable(rerollMVar("}#1\special{html:")),%
<!-- viewmodelstop //-->%
}%
}%
\else%
\newcommand{\MGlobalSqrt}[4]{\newcounter{mvc_#1}\setcounter{mvc_#1}{#4}}% Funktioniert nicht als Wurzel !!!
\fi

% Parameter: Name, Min, Max, PDF-Standard zaehler, PDF-Standard nenner. Name in Deklaration OHNE backslash, im Code MIT Backslash
\ifttm
\newcommand{\MGlobalFraction}[5]{\special{html:%
<!-- onloadstart //-->%
MVAR.push(createGlobalFraction("}#1\special{html:",}#2\special{html:,}#3\special{html:,}#4\special{html:,}#5\special{html:)); %
<!-- onloadstop //-->%
<!-- viewmodelstart //-->%
ob}#1\special{html:: ko.observable(rerollMVar("}#1\special{html:")),%
<!-- viewmodelstop //-->%
}%
}%
\else%
\newcommand{\MGlobalFraction}[5]{\newcounter{mvc_#1}\setcounter{mvc_#1}{#4}} % Funktioniert nicht als Bruch !!!
\fi

% MVar darf im HTML nur in MEvalMathDisplay-Umgebungen genutzt werden oder in Strings die an den Parser uebergeben werden
\ifttm%
\newcommand{\MVar}[1]{\special{html:[var_}#1\special{html:]}}%
\else%
\newcommand{\MVar}[1]{\arabic{mvc_#1}}%
\fi

\ifttm%
\newcommand{\MRerollButton}[2]{\special{html:<button type="button" onclick="rerollMVar('}#1\special{html:');">}#2\special{html:</button>}}%
\else%
\newcommand{\MRerollButton}[2]{\relax}% Keine sinnvolle Entsprechung im PDF
\fi

% MEvalMathDisplay fuer HTML wird in mconvert.pl im preprocessing realisiert
% PDF: eine equation*-Umgebung (ueber amsmath)
% HTML: Eine Mathjax-Tex-Umgebung, deren Auswertung mit knockout-obervablen gekoppelt ist
% PDF-Version hier nur fuer pdflatex-only-Uebersetzung gegeben

\ifttm\else\newenvironment{MEvalMathDisplay}{\begin{equation*}}{\end{equation*}}\fi

% ---------------------------------- Spezialbefehle fuer AD ------------------------------------------

%Abk�rzung f�r \longrightarrow:
\newcommand{\lto}{\ensuremath{\longrightarrow}}

%Makro f�r Funktionen:
\newcommand{\exfunction}[5]
{\begin{array}{rrcl}
 #1 \colon  & #2 &\lto & #3 \\[.05cm]  
  & #4 &\longmapsto  & #5 
\end{array}}

\newcommand{\function}[5]{%
#1:\;\left\lbrace{\begin{array}{rcl}
 #2 &\lto & #3 \\
 #4 &\longmapsto  & #5 \end{array}}\right.}


%Die Identit�t:
\DeclareMathOperator{\Id}{Id}

%Die Signumfunktion:
\DeclareMathOperator{\sgn}{sgn}

%Zwei Betonungskommandos (k�nnen angepasst werden):
\newcommand{\highlight}[1]{#1}
\newcommand{\modstextbf}[1]{#1}
\newcommand{\modsemph}[1]{#1}


% ---------------------------------- Spezialbefehle fuer JL ------------------------------------------


\def\jccolorfkt{green!50!black} %Farbe des Funktionsgraphen
\def\jccolorfktarea{green!25!white} %Farbe der Fl"ache unter dem Graphen
\def\jccolorfktareahell{green!12!white} %helle Einf"arbung der Fl"ache unter dem Graphen
\def\jccolorfktwert{green!50!black} %Farbe einzelner Punkte des Graphen

\newcommand{\MPfadBilder}{Bilder}

\ifttm%
\newcommand{\jMD}{\,\MD}%
\else%
\newcommand{\jMD}{\;\MD}%
\fi%

\def\jHTMLHinweisBedienung{\MInputHint{%
Mit Hilfe der Symbole am oberen Rand des Fensters
k"onnen Sie durch die einzelnen Abschnitte navigieren.}}

\def\jHTMLHinweisEingabeText{\MInputHint{%
Geben Sie jeweils ein Wort oder Zeichen als Antwort ein.}}

\def\jHTMLHinweisEingabeTerm{\MInputHint{%
Klammern Sie Ihre Terme, um eine eindeutige Eingabe zu erhalten. 
Beispiel: Der Term $\frac{3x+1}{x-2}$ soll in der Form
\texttt{(3*x+1)/((x+2)^2}$ eingegeben werden (wobei auch Leerzeichen 
eingegeben werden k"onnen, damit eine Formel besser lesbar ist).}}

\def\jHTMLHinweisEingabeIntervalle{\MInputHint{%
Intervalle werden links mit einer "offnenden Klammer und rechts mit einer 
schlie"senden Klammer angegeben. Eine runde Klammer wird verwendet, wenn der 
Rand nicht dazu geh"ort, eine eckige, wenn er dazu geh"ort. 
Als Trennzeichen wird ein Komma oder ein Semikolon akzeptiert.
Beispiele: $(a, b)$ offenes Intervall,
$[a; b)$ links abgeschlossenes, rechts offenes Intervall von $a$ bis $b$. 
Die Eingabe $]a;b[$ f"ur ein offenes Intervall wird nicht akzeptiert.
F"ur $\infty$ kann \texttt{infty} oder \texttt{unendlich} geschrieben werden.}}

\def\jHTMLHinweisEingabeFunktionen{\MInputHint{%
Schreiben Sie Malpunkte (geschrieben als \texttt{*}) aus und setzen Sie Klammern um Argumente f�r Funktionen.
Beispiele: Polynom: \texttt{3*x + 0.1}, Sinusfunktion: \texttt{sin(x)}, 
Verkettung von cos und Wurzel: \texttt{cos(sqrt(3*x))}.}}

\def\jHTMLHinweisEingabeFunktionenSinCos{\MInputHint{%
Die Sinusfunktion $\sin x$ wird in der Form \texttt{sin(x)} angegeben, %
$\cos\left(\sqrt{3 x}\right)$ durch \texttt{cos(sqrt(3*x))}.}}

\def\jHTMLHinweisEingabeFunktionenExp{\MInputHint{%
Die Exponentialfunktion $\MEU^{3x^4 + 5}$ wird als
\texttt{exp(3 * x^4 + 5)} angegeben, %
$\ln\left(\sqrt{x} + 3.2\right)$ durch \texttt{ln(sqrt(x) + 3.2)}.}}

% ---------------------------------- Spezialbefehle fuer Fachbereich Physik --------------------------

\newcommand{\E}{{e}}
\newcommand{\ME}[1]{\cdot 10^{#1}}
\newcommand{\MU}[1]{\;\mathrm{#1}}
\newcommand{\MPG}[3]{%
  \ifnum#2=0%
    #1\ \mathrm{#3}%
  \else%
    #1\cdot 10^{#2}\ \mathrm{#3}%
  \fi}%
%

\newcommand{\MMul}{\MExponentensymbXYZl} % Nur eine Abkuerzung


% ---------------------------------- Stichwortfunktionialitaet ---------------------------------------

% mpreindexentry wird durch Auswahlroutine in conv.pl durch mindexentry substitutiert
\ifttm%
\def\MIndex#1{\index{#1}\special{html:<!-- mpreindexentry;;}#1\special{html:;;}\arabic{MSubjectArea}\special{html:;;}%
\arabic{chapter}\special{html:;;}\arabic{section}\special{html:;;}\arabic{subsection}\special{html:;;}\arabic{MEntryCounter}\special{html:; //-->}%
\setcounter{MLastIndex}{\value{MEntryCounter}}%
\addtocounter{MEntryCounter}{1}%
}%
% Copyrightliste wird als tex-Datei im preprocessing von conv.pl erzeugt und unter converter/tex/entrycollection.tex abgelegt
% Der input-Befehl funktioniert nur, wenn die aufrufende tex-Datei auf der obersten Ebene liegt (d.h. selbst kein input/include ist, insbesondere keine Moduldatei)
\def\MEntryList{} % \input funktioniert nicht, weil ttm (und damit das \input) ausgefuehrt wird, bevor Datei da ist
\else%
\def\MIndex#1{\index{#1}}
\def\MEntryList{\MAbort{Stichwortliste nur im HTML realisierbar}}%
\fi%

\def\MEntry#1#2{\textbf{#1}\MIndex{#2}} % Idee: MLastType auf neuen Entry-Typ und dann ein MLabel vergeben mit autogen-Nummer

% ---------------------------------- Befehle fuer Tests ----------------------------------------------

% MEquationItem stellt eine Eingabezeile der Form Vorgabe = Antwortfeld her, der zweite Parameter kann z.B. MSimplifyQuestion-Befehl sein
\ifttm
\newcommand{\MEquationItem}[2]{{#1}$\,=\,${#2}}%
\else%
\newcommand{\MEquationItem}[2]{{#1}$\;\;=\,${#2}}%
\fi

\ifttm
\newcommand{\MInputHint}[1]{%
\ifnum%
\if\value{MTestSite}>0%
\else%
{\color{blue}#1}%
\fi%
\fi%
}
\else
\newcommand{\MInputHint}[1]{\relax}
\fi

\ifttm
\newcommand{\MInTestHeader}{%
Dies ist ein einreichbarer Test:
\begin{itemize}
\item{Im Gegensatz zu den offenen Aufgaben werden beim Eingeben keine Hinweise zur Formulierung der mathematischen Ausdr�cke gegeben.}
\item{Der Test kann jederzeit neu gestartet oder verlassen werden.}
\item{Der Test kann durch die Buttons am Ende der Seite beendet und abgeschickt, oder zur�ckgesetzt werden.}
\item{Der Test kann mehrfach probiert werden. F�r die Statistik z�hlt die zuletzt abgeschickte Version.}
\end{itemize}
}
\else
\newcommand{\MInTestHeader}{%
\relax
}
\fi

\ifttm
\newcommand{\MInTestFooter}{%
\special{html:<button name="Name_TESTFINISH" id="TESTFINISH" type="button" onclick="finish_button('}\MTestName\special{html:');">Test auswerten</button>}%
\begin{html}
&nbsp;&nbsp;&nbsp;&nbsp;&nbsp;&nbsp;&nbsp;&nbsp;
<button name="Name_TESTRESET" id="TESTRESET" type="button" onclick="reset_button();">Test zur�cksetzen</button>
<br />
<br />
<div class="xreply">
<p name="Name_TESTEVAL" id="TESTEVAL">
Hier erscheint die Testauswertung!
<br />
</p>
</div>
\end{html}
}
\else
\newcommand{\MInTestFooter}{%
\relax
}
\fi


% ---------------------------------- Notationsmakros -------------------------------------------------------------

% Notationsmakros die nicht von der Kursvariante abhaengig sind

\newcommand{\MZahltrennzeichen}[1]{\renewcommand{\MZXYZhltrennzeichen}{#1}}

\ifttm
\newcommand{\MZahl}[3][\MZXYZhltrennzeichen]{\edef\MZXYZtemp{\noexpand\special{html:<mn>#2#1#3</mn>}}\MZXYZtemp}
\else
\newcommand{\MZahl}[3][\MZXYZhltrennzeichen]{{}#2{#1}#3}
\fi

\newcommand{\MEinheitenabstand}[1]{\renewcommand{\MEinheitenabstXYZnd}{#1}}
\ifttm
\newcommand{\MEinheit}[2][\MEinheitenabstXYZnd]{{}#1\edef\MEINHtemp{\noexpand\special{html:<mi mathvariant="normal">#2</mi>}}\MEINHtemp} 
\else
\newcommand{\MEinheit}[2][\MEinheitenabstXYZnd]{{}#1 \mathrm{#2}} 
\fi

\newcommand{\MExponentensymbol}[1]{\renewcommand{\MExponentensymbXYZl}{#1}}
\newcommand{\MExponent}[2][\MExponentensymbXYZl]{{}#1{} 10^{#2}} 

%Punkte in 2 und 3 Dimensionen
\newcommand{\MPointTwo}[3][]{#1(#2\MCoordPointSep #3{}#1)}
\newcommand{\MPointThree}[4][]{#1(#2\MCoordPointSep #3\MCoordPointSep #4{}#1)}
\newcommand{\MPointTwoAS}[2]{\left(#1\MCoordPointSep #2\right)}
\newcommand{\MPointThreeAS}[3]{\left(#1\MCoordPointSep #2\MCoordPointSep #3\right)}

% Masseinheit, Standardabstand: \,
\newcommand{\MEinheitenabstXYZnd}{\MThinspace} 

% Horizontaler Leerraum zwischen herausgestellter Formel und Interpunktion
\ifttm
\newcommand{\MDFPSpace}{\,}
\newcommand{\MDFPaSpace}{\,\,}
\newcommand{\MBlank}{\ }
\else
\newcommand{\MDFPSpace}{\;}
\newcommand{\MDFPaSpace}{\;\;}
\newcommand{\MBlank}{\ }
\fi

% Satzende in herausgestellter Formel mit horizontalem Leerraum
\newcommand{\MDFPeriod}{\MDFPSpace .}

% Separation von Aufzaehlung und Bedingung in Menge
\newcommand{\MCondSetSep}{\,:\,} %oder '\mid'

% Konverter kennt mathopen nicht
\ifttm
\def\mathopen#1{}
\fi

% -----------------------------------START Rouletteaufgaben ------------------------------------------------------------

\ifttm
% #1 = Dateiname, #2 = eindeutige ID fuer das Roulette im Kurs
\newcommand{\MDirectRouletteExercises}[2]{
\begin{MExercise}
\texttt{Im HTML erscheinen hier Aufgaben aus einer Aufgabenliste...}
\end{MExercise}
}
\else
\newcommand{\MDirectRouletteExercises}[2]{\relax} % wird durch mconvert.pl gefunden und ersetzt
\fi


% ---------------------------------- START Makros, die von der Kursvariante abhaengen ----------------------------------

\ifvariantunotation
  % unotation = An Universitaeten uebliche Notation
  \def\MVariant{unotation}

  % Trennzeichen fuer Dezimalzahlen
  \newcommand{\MZXYZhltrennzeichen}{.}

  % Exponent zur Basis 10 in der Exponentialschreibweise, 
  % Standardmalzeichen: \times
  \newcommand{\MExponentensymbXYZl}{\times} 

  % Begrenzungszeichen fuer offene Intervalle
  \newcommand{\MoIl}[1][]{\mbox{}#1(\mathopen{}} % bzw. ']'
  \newcommand{\MoIr}[1][]{#1)\mbox{}} % bzw. '['

  % Zahlen-Separation im IntervaLL
  \newcommand{\MIntvlSep}{,} %oder ';'

  % Separation von Elementen in Mengen
  \newcommand{\MElSetSep}{,} %oder ';'

  % Separation von Koordinaten in Punkten
  \newcommand{\MCoordPointSep}{,} %oder ';' oder '|', '\MThinspace|\MThinspace'

\else
  % An dieser Stelle wird angenommen, dass std-Variante aktiv ist
  % std = beschlossene Notation im TU9-Onlinekurs 
  \def\MVariant{std}

  % Trennzeichen fuer Dezimalzahlen
  \newcommand{\MZXYZhltrennzeichen}{,}

  % Exponent zur Basis 10 in der Exponentialschreibweise, 
  % Standardmalzeichen: \times
  \newcommand{\MExponentensymbXYZl}{\times} 

  % Begrenzungszeichen fuer offene Intervalle
  \newcommand{\MoIl}[1][]{\mbox{}#1]\mathopen{}} % bzw. '('
  \newcommand{\MoIr}[1][]{#1[\mbox{}} % bzw. ')'

  % Zahlen-Separation im IntervaLL
  \newcommand{\MIntvlSep}{;} %oder ','
  
  % Separation von Elementen in Mengen
  \newcommand{\MElSetSep}{;} %oder ','

  % Separation von Koordinaten in Punkten
  \newcommand{\MCoordPointSep}{;} %oder '|', '\MThinspace|\MThinspace'

\fi



% ---------------------------------- ENDE Makros, die von der Kursvariante abhaengen ----------------------------------


% diese Kommandos setzen Mathemodus vorraus
\newcommand{\MGeoAbstand}[2]{[\overline{{#1}{#2}}]}
\newcommand{\MGeoGerade}[2]{{#1}{#2}}
\newcommand{\MGeoStrecke}[2]{\overline{{#1}{#2}}}
\newcommand{\MGeoDreieck}[3]{{#1}{#2}{#3}}

%
\ifttm
\newcommand{\MOhm}{\special{html:<mn>&#x3A9;</mn>}}
\else
\newcommand{\MOhm}{\Omega} %\varOmega
\fi


\def\PERCTAG{\MAbort{PERCTAG ist zur internen verwendung in mconvert.pl reserviert, dieses Makro darf sonst nicht benutzt werden.}}

% Im Gegensatz zu einfachen html-Umgebungen werden MDirectHTML-Umgebungen von mconvert.pl am ganzen ttm-Prozess vorbeigeschleust und aus dem PDF komplett ausgeschnitten
\ifttm%
\newenvironment{MDirectHTML}{\begin{html}}{\end{html}}%
\else%
\newenvironment{MDirectHTML}{\begin{html}}{\end{html}}%
\fi

% Im Gegensatz zu einfachen Mathe-Umgebungen werden MDirectMath-Umgebungen von mconvert.pl am ganzen ttm-Prozess vorbeigeschleust, ueber MathJax realisiert, und im PDF als $$ ... $$ gesetzt
\ifttm%
\newenvironment{MDirectMath}{\begin{html}}{\end{html}}%
\else%
\newenvironment{MDirectMath}{\begin{equation*}}{\end{equation*}}% Vorsicht, auch \[ und \] werden in amsmath durch equation* redefiniert
\fi

% ---------------------------------- Location Management ---------------------------------------------

% #1 = buttonname (muss in files/images liegen und Format 48x48 haben), #2 = Vollstaendiger Einrichtungsname, #3 = Kuerzel der Einrichtung,  #4 = Name der include-texdatei
\ifttm
\newcommand{\MLocationSite}[3]{\special{html:<!-- mlocation;;}#1\special{html:;;}#2\special{html:;;}#3\special{html:;; //-->}}
\else
\newcommand{\MLocationSite}[3]{\relax}
\fi

% ---------------------------------- Copyright Management --------------------------------------------

\newcommand{\MCCLicense}{%
{\color{green}\textbf{CC BY-SA 3.0}}
}

\newcommand{\MCopyrightLabel}[1]{ (\MSRef{L_COPYRIGHTCOLLECTION}{Lizenz})\MLabel{#1}}

% Copyrightliste wird als tex-Datei im preprocessing erzeugt und unter converter/tex/copyrightcollection.tex abgelegt
% Der input-Befehl funktioniert nur, wenn die aufrufende tex-Datei auf der obersten Ebene liegt (d.h. selbst kein input/include ist, insbesondere keine Moduldatei)
\newcommand{\MCopyrightCollection}{\input{copyrightcollection.tex}}

% MCopyrightNotice fuegt eine Copyrightnotiz ein, der parser ersetzt diese durch CopyrightNoticePOST im preparsing, diese Definition wird nur fuer reine pdflatex-Uebersetzungen gebraucht
% Parameter: #1: Kurze Lizenzbeschreibung (typischerweise \MCCLicense)
%            #2: Link zum Original (http://...) oder NONE falls das Bild selbst ein Original ist, oder TIKZ falls das Bild aus einer tikz-Umgebung stammt
%            #3: Link zum Autor (http://...) oder MINT falls Original im MINT-Kolleg erstellt oder NONE falls Autor unbekannt
%            #4: Bemerkung (z.B. dass Datei mit Maple exportiert wurde)
%            #5: Labelstring fuer existierendes Label auf das copyrighted Objekt, mit MCopyrightLabel erzeugt
%            Keines der Felder darf leer sein!
\newcommand{\MCopyrightNotice}[5]{\MCopyrightNoticePOST{#1}{#2}{#3}{#4}{#5}}

\ifttm%
\newcommand{\MCopyrightNoticePOST}[5]{\relax}%
\else%
\newcommand{\MCopyrightNoticePOST}[5]{\relax}%
\fi%

% ---------------------------------- Meldungen fuer den Benutzer des Konverters ----------------------
\MPragma{mintmodversion;P0.1.0}
\MPragma{usercomment;This is file mintmod.tex version P0.1.0}


% ----------------------------------- Spezialelemente fuer Konfigurationsseite, werden nicht von mintscripts.js verwaltet --

% #1 = DOM-id der Box
\ifttm\newcommand{\MConfigbox}[1]{\special{html:<input cfieldtype="2" type="checkbox" name="Name_}#1\special{html:" id="}#1\special{html:" onchange="confHandlerChange('}#1\special{html:');"/>}}\fi % darf im PDF nicht aufgerufen werden!


\Mtikzexternalize
\MPragma{MathSkip}

\begin{document}

\MSection{Integralrechnung}\MLabel{VBKM08}
\MSetSectionID{VBKM08}
\begin{MSectionStart}
\MDeclareSiteUXID{VBKM08_START}

\MModstartBox
\end{MSectionStart}

%%%Abschnitt
\MSubsection{Stammfunktionen}
\MLabel{M08A_Stammfunktionen}

\begin{MIntro}
\MDeclareSiteUXID{VBKM08_Stammfunktionen_Intro}
Im letzten Kapitel wurden Ableitungen von Funktionen behandelt. Natürlich 
stellt sich, wie bei jeder Rechenoperation, auch hier die Frage nach der 
Umkehrung, so wie die Subtraktion als Umkehrung der Addition aufgefasst werden 
kann oder die Division als Umkehrung der Multiplikation. Die Suche nach der
Umkehrung der Ableitung führt zur Einführung der Integralrechnung und damit zur 
Stammfunktionsbildung. Der Zusammenhang ist sehr einfach erklärt. Kann man 
einer Funktion $f$ eine Ableitung $f'$ zuordnen und fasst auch $f'$ als Funktion 
auf, so kann man auch der Funktion $f'$ eine Funktion $f$ zuordnen, indem man 
die Operation "`Ableitung"' rückgängig macht.
Man dreht in diesem Kapitel also die Fragestellung um: Kann man zu einer 
Funktion $f$ eine andere Funktion finden, deren Ableitung wieder die 
Funktion $f$ ist?

Die Anwendungen der Integralrechnung sind genauso vielfältig wie die 
Anwendungen der Differentialrechnung. Untersucht man zum Beispiel 
in der Physik die 
Kraft $F$, die auf einen Körper mit der Masse $m$ wirkt, dann kann man unter 
Verwendung des bekannten Zusammenhangs $F = m a$, wo $a$ die Beschleunigung 
des Körpers ist, zunächst aus der Kraft die Beschleunigung $a = F/m$ berechnen. 
Interpretiert man die Beschleunigung als Änderungsrate der Geschwindigkeit, 
$a = \frac{\MD v}{\MD t}$, dann kann man anschließend die Geschwindigkeit 
über die Umkehrung der Ableitung -- also durch die Integralrechnung -- bestimmen.
Ähnliche Zusammenhänge lassen sich in vielen Bereichen aus Naturwissenschaften, 
Technik und auch Wirtschaftswissenschaften finden. So benötigt man die 
Integralrechnung zur Bestimmung von Flächen, von Schwerpunkten, 
Biegeeigenschaften von Balken oder zur Lösung sogenannter 
Differentialgleichungen, mit denen viele Probleme im 
naturwissenschaftlich-technischen Umfeld beschrieben werden.
\end{MIntro}

\begin{MXContent}{Stammfunktionen}{Stammfunktionen}{STD}
\MDeclareSiteUXID{VBKM08_Stammfunktionen_Content}

Im Kontext dieses Kurses werden die Fragestellungen der Integralrechnung für 
Funktionen auf {\glqq}zusammenhängenden{\grqq} Definitionsbereichen betrachtet,
wie dies für viele praktische Fragestellungen von besonderer Bedeutung ist.
Mathematisch formuliert, werden die Definitionsbereiche der Funktionen  
Intervalle sein. Dazu passend werden hier Stammfunktionen auf Intervallen 
definiert, mit denen die Frage nach der {\glqq}Umkehrung{\grqq} der Ableitung
betrachtet wird.

\begin{MXInfo}{Stammfunktion} 
Gegeben ist ein Intervall $D \subseteq \R$ und eine 
Funktion $f: D \rightarrow \R$. 
Wenn es eine differenzierbare Funktion $F: D  \rightarrow \R$ gibt, deren 
Ableitung gleich $f$ ist, für die also $F'(x) = f(x)$ für alle $x \in D$ gilt, 
dann heißt $F$ eine \MEntry{Stammfunktion}{Stammfunktion} von $f$.
\end{MXInfo}

Zunächst sollen einige Beispiele betrachtet werden.
\begin{MExample}
Die Funktion $F$ mit $F(x) = -\cos(x)$ hat die Ableitung 
\[
F'(x) = -(-\sin(x)) = \sin(x) \MDFPeriod %%
\]
Somit ist $F$ eine Stammfunktion von $f$ mit $f(x) = \sin(x)$.
%\[
%\int \sin(x) \MDwSp x = -\cos(x) %%
%\]
%eine Stammfunktion von $f(x) = \sin(x)$.
\end{MExample}

%\begin{MExample}
%Die Funktion $G(x) = \MEU^{3x + 7}$ hat die Ableitung 
%$G'(x) = 3 \cdot \MEU^{3 x + 7}$.
%Deshalb ist
%\[
%\int 3 \cdot \MEU^{3 x + 7} \MDwSp x = \MEU^{3 x + 7} %%
%\]
%eine Stammfunktion von $f(x) = 3 \MEU^{3 x + 7}$.
%\end{MExample}

\begin{MExample}
Die Funktion $G$ mit $G(x) = \frac{1}{3} \MEU^{3x + 7}$ hat die Ableitung 
\[
G'(x) = \frac{1}{3} \cdot 3 \cdot \MEU^{3 x + 7} \MDFPeriod
\]
Deshalb ist $G$ eine Stammfunktion von $g$ mit $g(x) = \MEU^{3 x + 7}$.
\end{MExample}

Es wird noch ein ganz einfaches Beispiel betrachtet, an dem ein wichtiger 
Aspekt deutlich wird, wenn eine Stammfunktion gesucht wird.
\begin{MExample}
Es sei $H$ eine konstante Funktion auf einem Intervall mit dem Funktionswert 
$H(x) = 18$ gegeben. Dann hat $H$ die Ableitung 
\[
H'(x) = 0 \MDFPeriod
\]
Deshalb ist $H$ eine Stammfunktion von $h$ mit $h(x) = 0$.
\end{MExample}

Das letzte Beispiel ist wenig überraschend, denn die Ableitung einer 
konstanten Funktion ist die Nullfunktion.
Somit ist \emph{jede} konstante Funktion $F$ eine Stammfunktion von $f$ mit 
$f(x) = 0$ auf einem Intervall, das heißt, es ist $F(x)$ gleich irgendeiner 
Zahl $C$ für jeden $x$-Wert.
Andere Möglichkeiten, als dass es sich um irgendeine \emph{konstante} Funktion 
handelt, gibt es aber nicht, wenn $f$ auf einem Intervall definiert ist.

\begin{MXInfo}{Alle Stammfunktionen der Nullfunktion}
Es ist $F$ genau dann eine Stammfunktion von $f$ mit $f(x) = 0$ auf einem
Intervall, wenn $F$ eine konstante Funktion ist, das heißt, wenn es eine 
reelle Zahl $C$ gibt, sodass $F(x) = C$ für alle $x$-Werte des Intervalls ist.
\end{MXInfo}

Wenn die Funktionen $F$ und $G$ dieselbe Ableitung $f = F' = G'$ haben, dann 
ist $G'(x) - F'(x) = 0$. Bildet man nun auf beiden Seiten der Gleichung die 
Stammfunktion auf einem Intervall, dann erhält man den Zusammenhang 
$G(x) - F(x) = C$. Somit ist $G(x) = F(x) + C$. Hat man also mit $F$ eine 
Stammfunktion von $f$ gefunden, dann ist auch $G$ mit $G(x) = F(x) + C$
eine Stammfunktion von $f$.

\begin{MXInfo}{Aussage über Stammfunktionen}
Wenn $F$ und $G$ Stammfunktionen von $f: D \rightarrow \R$ auf einem Intervall
$D$ sind, dann gibt es eine reelle Zahl $C$, sodass
\[
F(x) = G(x) + C %
\qquad \text{für alle } x \in D %%
\]
gilt. Hierfür schreibt man auch
\[
\int f(x) \MDwSp x = F(x) + C \MDFPSpace, %%
\]
um auszudrücken, wie sämtliche Stammfunktionen von $f$ aussehen.
\end{MXInfo}

Die Gesamtheit aller Stammfunktionen wird auch 
\MEntry{unbestimmtes Integral}{Integral (unbestimmt)} genannt und in der 
oben eingeführten Form 
\[
\int f(x) \MDwSp x = F(x) + C %%
\]
notiert, wobei $F$ irgendeine Stammfunktion von $f$ ist. 

Durch die Schreibweise des unbestimmten Integrals wird betont, dass zur 
gegebenen Funktion $f$ eine Funktion $F$ mit $F' = f$ gesucht wird.
Wie damit das (bestimmte) Integral einer stetigen Funktion $f$ berechnet 
werden kann, beschreibt der 
\MEntry{Hauptsatz der Differential- und Integralrechnung}{Hauptsatz (Integral)},
der im nächsten Abschnitt in \MRef{HSDDIR} erläutert wird. 

Und woher kennt man den Wert dieser Konstanten $C$? 
Ist nur nach einer Stammfunktion von $f$ mit $f(x) = 0$ auf einem Intervall 
gefragt, ohne dass weitere Anforderungen bekannt sind, ist $C$ nicht festgelegt.
Erst wenn zusätzlich ein Funktionswert $y_0 = F(x_0)$ von $F$ an einer 
Stelle $x_0$ vorgegeben wird, ist dann auch $C$ festgelegt.
\begin{MExample}
Beispielsweise ergibt sich für $f$ mit $f(x) = 2x + 5$ dann
\[
\int \left( 2 x + 5 \right) \MD x = x^2 + 5x + C \MDFPeriod %%
\]
Wenn nun diejenige Stammfunktion $F$ von $f$ gesucht wird, für die $F(0) = 6$
gilt, dann muss $6 = F(0) = 0^2 + 5 \cdot 0 + C = C$ und somit $C = 6$ sein.
Damit ist dann $F(x) = x^2 + 5 x + 6$.
\end{MExample}
Notiert man die Beziehung zwischen Ableitung $f = F'$ und Stammfunktionen $F$ in 
der eben besprochenen umgekehrten Sichtweise für die bisher betrachteten 
Funktionsklassen, ergibt sich die folgende Tabelle:

\begin{MXInfo}{Eine kleine Tabelle von Stammfunktionen}%
\MIndex{Stammfunktionen (Tabelle)}%
Es werden Funktionen $f$ auf einem Intervall betrachtet, zu denen die 
Stammfunktionen in der Schreibweise des unbestimmten Integrals angegeben werden:
\ifttm
\[
\begin{array}{ll}
\text{Funktion } f & \text{Stammfunktionen } F \\
\hline
f(x) = 0                    & F(x) = \int 0 \MDwSp x = C \\
f(x) = x^n                  & F(x) = \int x^n \MDwSp x = \frac{1}{n+1} \cdot x^{n+1} + C \\
f(x) = \sin(x)              & F(x) = \int \sin(x) \MDwSp x = -\cos(x) + C \\
f(x) = \sin(k x)            & F(x) = \int \sin(k x) \MDwSp x = -\frac{1}{k}\,\cos(k x) + C \\
f(x) = \cos(x)              & F(x) = \int \cos(x) \MDwSp x = \sin(x) + C \\
f(x) = \cos(k x)            & F(x) = \int \cos(k x) \MDwSp x = \frac{1}{k}\,\sin(k x) + C \\
f(x) = \MEU^x               & F(x) = \int \MEU^x \MDwSp x = \MEU^x + C \\
f(x) = \MEU^{k x}           & F(x) = \int \MEU^{k x} \MDwSp x = \frac{1}{k}\,\MEU^{k x} + C \\
f(x) = x^{-1} = \frac{1}{x} & F(x) = \int \frac{1}{x} \MDwSp x = \ln|x| + C  %
\text{ für } x > 0 \text{ oder } x < 0 %%
\end{array}
\]
\else
\[
\begin{array}{ll}
\text{Funktion } f & \mbox{Stammfunktionen } F \\
\hline
f(x) = 0                    & F(x) = \int 0 \MDwSp x = C \\[1ex]
f(x) = x^n                  & F(x) = \int x^n \MDwSp x = \frac{1}{n+1} \cdot x^{n+1} + C \\[1ex]
f(x) = \sin(x)              & F(x) = \int \sin(x) \MDwSp x = -\cos(x) + C \\[1ex]
f(x) = \sin(k x)            & F(x) = \int \sin(k x) \MDwSp x = -\frac{1}{k}\,\cos(k x) + C \\[1ex]
f(x) = \cos(x)              & F(x) = \int \cos(x) \MDwSp x = \sin(x) + C \\[1ex]
f(x) = \cos(k x)            & F(x) = \int \cos(k x) \MDwSp x = \frac{1}{k}\,\sin(k x) + C \\[1ex]
f(x) = \MEU^x               & F(x) = \int \MEU^x \MDwSp x = \MEU^x + C \\[1ex]
f(x) = \MEU^{k x}           & F(x) = \int \MEU^{k x} \MDwSp x = \frac{1}{k}\,\MEU^{k x} + C \\[1ex]
f(x) = x^{-1} = \frac{1}{x} & F(x) = \int \frac{1}{x} \MDwSp x = \ln|x| + C %
\text{ für } x > 0 \text{ oder } x < 0 %%
\end{array}
\]
\fi
Hier bezeichnen $k$ und $C$ beliebige reelle Zahlen mit $k \neq 0$ und $n$ eine 
ganze Zahl mit $n \neq -1$.
\end{MXInfo}

Das nächste Beispiel zeigt, wie die Tabelle gelesen wird.

\begin{MExample}
Zur Funktion $f$ mit $f(x) = 10 x^2 - 6 = 10 x^2 - 6 x^0$ wird das unbestimmte 
Integral gesucht.
Aus der obigen Tabelle können Stammfunktionen zu $g$ mit $g(x) = x$ und 
$h$ mit $h(x) = x^0 = 1$ abgelesen werden: Es ist
$G$ mit $G(x) = \frac{1}{1+1} \cdot x^{1+1} = \frac{1}{2} \cdot x^2$ eine 
Stammfunktion von $g$ und $H$ mit $H(x) = \frac{1}{0+1} \cdot x^{0+1} = x$ eine
Stammfunktion von $h$. Somit ist die Funktion $F: \R \to \R$ mit 
\[
F(x) = 10 \cdot \frac{1}{2} x^2 - 6 \cdot x = 5 x^2 - 6 x %%
\] 
eine Stammfunktion von $f$. Damit wird durch
\[
\int \left(10 x - 6 \right) \MD x = 5 x^2 - 6 x + C
\]
die Gesamtheit der Stammfunktionen von $f: \R \to \R$ mit $f(x) = 10 x - 6$ 
beschrieben, wobei $C$ für eine beliebige reelle Zahl steht.

Die Schreibweise mit der Konstanten $C$ drückt aus, dass beispielsweise auch 
$G: \R \to \R$ mit $G(x) := 5 x^2 - 6 x - 7$ eine Stammfunktion von $f$ ist, 
wobei $C = -7$ ist, denn es ist $G'(x) = 5 \cdot 2 x - 6 = f(x)$ für alle 
$x \in \R$.
\end{MExample}

In Tabellenwerken wird auf die Angaben der Konstanten oft verzichtet. In einer 
Rechnung ist es allerdings wichtig, anzugeben, dass es mehrere Funktionen geben 
kann. Bei der Lösung anwendungsbezogener Aufgaben wird die Konstante $C$ häufig 
durch die Angabe weiterer Bedingungen wie beispielsweise die Angabe eines 
Funktionswertes einer Stammfunktion festgelegt.

%\begin{MExample}
%Die Funktion $F(x) = 5 x^2 - 6 x$ hat die Ableitung $F'(x) = 10 x - 6$. 
%Somit ist 
%\[
%\int (10 x - 6) = 5 x^2 - 6 x
%\]
%eine Stammfunktion von $f(x) = 10 x - 6$.
%
%Die Ableitung der Funktion $G(x) = 5 x^2 - 6 x + 3$ ist 
%$G'(x) = 10 x - 6 = F'(x)$, also dieselbe wie von $F$. Damit ist auch $G$ eine 
%Stammfunktion von $f(x) = 10 x - 6$.
%\end{MExample}
%\end{MXContent}

%Die Bestimmung einer Stammfunktion ist nicht immer einfach, mitunter eine 
%Kunst. Eine Kontrolle lohnt sich also und ergibt sich definitionsgemäß 
%einfach dadurch, dass die Ableitung berechnet und mit der gegebenen Funktion
%verglichen wird, in der Praxis meistens eine Anwendung der Rechentechnik der 
%Ableitungsregeln.

\begin{MXInfo}{Ein praktischer Hinweis}
Die Überprüfung, ob man eine Stammfunktion einer vorgegebenen Funktion $f$ 
richtig gebildet hat, ist sehr einfach. Man bestimmt die Ableitung der gefundenen 
Stammfunktion und vergleicht diese mit der ursprünglich vorgegebenen Funktion $f$.
Stimmen beide überein, dann war die Rechnung richtig. Stimmt das Ergebnis der 
Probe nicht mit der Funktion $f$ überein, so muss die Stammfunktion noch einmal 
überprüft werden.
\end{MXInfo}
\end{MXContent}


%%%Uebungen zum Abschnitt:
\begin{MExercises}
\MDeclareSiteUXID{VBKM08_Stammfunktionen_Exercises}

\begin{MExercise}
%Loesungshinweise erstellt (jgl).
Geben Sie eine Stammfunktion an:
\begin{MExerciseItems}
\item{\MEquationItem{$\displaystyle\int \left(12 x^2 - 4 x^7\right) \MD x $}
{\MLSimplifyQuestion{30}{4*x^3 - 1/2*x^8}{10}{x}{4}{32}{LSTFX1}}.
\begin{MHint}{\iSolution}{%
Es ist
\[
\int \left(12 x^2 - 4 x^7\right) \MD x %
= 12 \cdot \frac{1}{3} \cdot x^3 - 4 \cdot \frac{1}{8} \cdot x^8 %
= 4 \cdot x^3 - \frac{1}{2} \cdot x^8 + C\MDFPeriod %
\]
}
\end{MHint}}
%
\item{\MEquationItem{$\displaystyle\int \left(\sin(x) + \cos(x)\right) \MD x$}
{\MLSimplifyQuestion{30}{sin(x)-cos(x)}{10}{x}{4}{32}{LSTFX2}}.
\begin{MHint}{\iSolution}{%
Es ist
\[
\int \left(\sin(x) + \cos(x)\right) \MD x %
= -\cos(x) + \sin(x) + C %
= \sin(x) - \cos(x) + C\MDFPeriod %
\]
}\end{MHint}}
%Teilaufgabe ersetzt, da arctan nicht (mehr) in der Liste von Stammfunktionen
%vermerkt werden soll (jgl):
%\item{\MEquationItem{$\displaystyle\int \frac{1}{x^2 + 1} \MDwSp x$}{\MLSimplifyQuestion{30}{arctan(x)}{10}{x}{4}{32}{LSTFX3}}. \begin{MHint}{\iSolution}{%
%% Die Stammfunktion kann aus der obigen Tabelle entnommen werden.
%}\end{MHint}}
\item{\MEquationItem{$\displaystyle\int \frac{1}{6 \sqrt{x}} \MDwSp x$}
{\MLSimplifyQuestion{30}{1/3*sqrt(x)}{10}{x}{4}{544}{LSTFX3}}. 
\begin{MHint}{\iSolution}{%
Es ist 
$\frac{1}{6 \sqrt{x}} = \frac{1}{6} \cdot \frac{1}{\sqrt{x}} %
= \frac{1}{6} \cdot x^{-\frac{1}{2}}$. Somit ist
\[
\int \frac{1}{6 \sqrt{x}} \MDwSp x %
= \frac{1}{6} \cdot \int x^{-\frac{1}{2}} \MDwSp x %
= \frac{1}{6} \cdot \frac{1}{-\frac{1}{2} + 1} \cdot x^{-\frac{1}{2} + 1} + C %
= \frac{1}{6} \cdot 2 \cdot x^{\frac{1}{2}} + C %
= \frac{1}{3} \cdot \sqrt{x} + C\MDFPeriod %
\]
}
\end{MHint}}
\end{MExerciseItems}
\end{MExercise}

\begin{MExercise}
%Loesungshinweise erstellt (jgl).
Bestimmen Sie eine Stammfunktion:
\begin{MExerciseItems}
\item{\MEquationItem{$\displaystyle\int \MEU^{x+2} \MDwSp x$}{\MLSimplifyQuestion{30}{exp(x+2)}{10}{x}{4}{32}{LSTNF1}}.}
\begin{MHint}{Lösung}
Es ist
$\displaystyle \int \MEU^{x+2} \MDwSp x = %
\int \MEU^x \cdot \MEU^2 \MDwSp x = %
\MEU^2 \cdot \int \MEU^x \MDwSp x = %
\MEU^2 \cdot \MEU^x + C = %
\MEU^x \cdot \MEU^2 + C = %
\MEU^{x + 2} + C$.
\end{MHint}
%
\item{\MEquationItem{$\displaystyle\int 3 x\cdot \sqrt[4]{x} \MDwSp x$}{\MLSimplifyQuestion{30}{4/3*x^(9/4)}{1}{x}{10}{32}{LSTNF2}}.}
\begin{MHint}{Lösung}
Es ist
$\displaystyle \int 3 x \cdot \sqrt[4]{x} \MDwSp x = %
3 \cdot \int x \cdot x^{\frac{1}{4}} \MDwSp x = %
3 \cdot \int x^{\frac{5}{4}} \MDwSp x = %
3 \cdot \frac{4}{9} \cdot x^{\frac{9}{4}} + C = %
\frac{4}{3} \cdot x^{\frac{9}{4}} + C$.
\end{MHint}
\end{MExerciseItems}
\jHTMLHinweisEingabeFunktionenExp
\end{MExercise}

%Aufgabe herausgenommen (jgl), da diese in Stellungnahmen als zu schwierig
%eingestuft wurde.
%\begin{MExercise}
%Beschreiben Sie den Betrag $|x|$ einer reellen Zahl $x$ mittels einer
%Fallunterscheidung und bestimmen Sie damit $\displaystyle\int |x| \MDwSp x$.
%
%Antwort: $\displaystyle\int |x| \MDwSp x$ %
%= \MLSimplifyQuestion{60}{1/2*x*abs(x)}{10}{x}{4}{32}{LSTNF3}
%\ifttm
%\ \\ \ \\
%\else\relax\fi
%\MInputHint{Die Stammfunktion kann entweder über eine Fallunterscheidung, oder wieder mit Hilfe der Betragsfunktion notiert werden.
%Fallunterscheidungen mit zwei Fällen können mit dem Textstück \texttt{falls(bedingung,fall1,fall2)} eingegeben werden, beispielsweise tippt man
%eine Fallunterscheidung der Form
%$$
%|x| \;=\; \left\lbrace{\begin{array}{ll} \phantom{-}x & \text{falls$\;x>0$} \\ -x & \text{falls$\;x\leq 0$}\end{array}}\right.
%$$
%als \texttt{falls(x>0,x,-x)} ein.
%}
%\end{MExercise}

%Neue Aufgabe (jgl) als Vorbereitung auf den Abschlusstest:
\begin{MExercise} %Stammfunktionen:
Entscheiden Sie, ob die folgenden Aussagen für reelle Funktionen richtig sind.
%Hierbei sind in der letzten Teilaufgabe irgendwelche Funktionen $f$ und $g$ 
%gegeben, die jeweils Stammfunktionen besitzen, und es ist $F$ eine 
%Stammfunktion von $f$ und $G$ eine Stammfunktion von $g$.

\ifttm
\begin{MQuestionGroup}
\begin{tabular}{|l|l|}
\hline
%richtig: & falsch: & Aussage: \\
 richtig? & Aussage: \\
 \MLCheckbox{0}{M08Ex1101a} & % \MLCheckbox{1}{M08Ex1101b} &
$F$ mit $F(x) = -\frac{\cos(\pi x) + 2}{\pi}$ ist eine Stammfunktion 
von $f$ mit $f(x) = \sin(\pi x) + 2$. \\
%
 \MLCheckbox{1}{M08Ex1102a} & % \MLCheckbox{0}{M08Ex1102b} &
$F$ mit $F(x) = -\frac{\cos(\pi x) + 2}{\pi}$ ist eine Stammfunktion 
von $f$ mit $f(x) = \sin(\pi x)$. \\
% 
 \MLCheckbox{0}{M08Ex1103a} & % \MLCheckbox{1}{M08Ex1103b} &
$F$ mit $F(x) = -7$ ist eine Stammfunktion von $f$ mit $f(x) = -7x$ %
für $x \in \R$. \\
% 
 \MLCheckbox{1}{M08Ex1104a} & % \MLCheckbox{0}{M08Ex1104b} &
$F$ mit $F(x) = (\sin(x))^2$ ist eine Stammfunktion 
von $f$ mit $f(x) = 2 \sin(x) \cos(x)$. \\
%
 \MLCheckbox{1}{M08Ex1105a} & % \MLCheckbox{0}{M08Ex1105b} &
Wenn $F$ eine Stammfunktion von $f$ ist, $G$ eine Stammfunktion von $g$, 
dann ist $F + G$ ist eine Stammfunktion von $f + g$. \\
\hline
\end{tabular}
\end{MQuestionGroup}
\MGroupButton{Eingabe überprüfen}
%
\else
%
\begin{tabular}[t]{ccp{140mm}}
 richtig: & falsch: & Aussage: \\
 \MLCheckbox{0}{M08Ex1101a} & \MLCheckbox{1}{M08Ex1101b} &
$F$ mit $F(x) = -\frac{\cos(\pi x) + 2}{\pi}$ ist eine Stammfunktion 
von $f$ mit $f(x) = \sin(\pi x) + 2$. \\
%
 \MLCheckbox{1}{M08Ex1102a} & \MLCheckbox{0}{M08Ex1102b} &
$F$ mit $F(x) = -\frac{\cos(\pi x) + 2}{\pi}$ ist eine Stammfunktion 
von $f$ mit $f(x) = \sin(\pi x)$. \\
% 
 \MLCheckbox{0}{M08Ex1103a} & \MLCheckbox{1}{M08Ex1103b} &
$F$ mit $F(x) = -7$ ist eine Stammfunktion von $f$ mit $f(x) = -7x$
für $x \in \R$. \\
% 
 \MLCheckbox{1}{M08Ex1104a} & \MLCheckbox{0}{M08Ex1104b} &
$F$ mit $F(x) = (\sin(x))^2$ ist eine Stammfunktion 
von $f$ mit $f(x) = 2 \sin(x) \cos(x)$. \\
%
 \MLCheckbox{1}{M08Ex1105a} & \MLCheckbox{0}{M08Ex1105b} &
Wenn $F$ eine Stammfunktion von $f$ ist, $G$ eine Stammfunktion von $g$, 
dann ist $F + G$ ist eine Stammfunktion von $f + g$. %%
\end{tabular}
\fi
\begin{MHint}{Lösung}
\begin{itemize}
\item Die Ableitung von $F$ mit $F(x) =  -\frac{\cos(\pi x) + 2}{\pi}$ ist
$F'(x) = \sin(\pi x) \neq \sin(\pi x) + 2 = f(x)$, sodass $F$ keine 
Stammfunktion von $f$ ist.
\item Die Ableitung von $F$ mit $F(x) =  -\frac{\cos(\pi x) + 2}{\pi}$ ist
$F'(x) = \sin(\pi x) = f(x)$, sodass $F$ eine Stammfunktion von $f$ ist.
\item Die Ableitung von $F$ mit $F(x) =  -7$ ist 
$F'(x) = 0 \neq -7x = f(x)$ (für $x \neq 0$). Deshalb ist $F$ keine 
Stammfunktion von $f$.
\item Die Ableitung von $F$ mit $F(x) =  (\sin(x))^2$ ergibt sich mit der
Kettenregel zu $F'(x) = 2 \cdot \sin(x) \cdot \cos(x) = f(x)$. Somit ist $F$ 
eine Stammfunktion von $f$.
\item Wenn $F$ eine Stammfunktion von $f$ ist, $G$ eine Stammfunktion von $g$, 
dann sind $F$ und $G$ differenzierbar, wobei $F' = f$ und $G' = g$ gilt. 
Somit ist auch $F + G$ differenzierbar, und es gilt 
$(F + G)'(x) = F'(x) + G'(x) = f(x) + g(x) = (f + g)(x)$. Das heißt, dass 
$F + G$ eine Stammfunktion von $f + g$ ist. %%
\end{itemize}
\end{MHint}
\end{MExercise}


%Alle Teilaufgaben der Aufgabe ersetzt, da die Aufgabe als zu umfangreich 
%bezeichnet wurde beziehungsweise da arctan nicht (mehr) in der Liste der 
%Stammfunktionen aufgefuehrt werden soll (jgl):
%\begin{MExercise}
%Bestimmen Sie eine Stammfunktion zu
%\begin{MExerciseItems}
%\item $f(x) := \frac{1+x+x^2+\sqrt{x}}{x}$,
%\item $g(x) := 4 + \left(\frac{4 \cos^2 x}{(2 \sin x)^2}\right)^{-1}$,
%\item $h(x) := \frac{2}{4 + (2x)^2}$,
%\end{MExerciseItems}
%nachdem Sie die Funktionsterme vereinfacht haben:
%
%\begin{MExerciseItems}
%\item{Mit der Vereinfachung \MEquationItem{$f(x)$}{\MLSimplifyQuestion{30}{1/x+1+x+1/sqrt(x)}{10}{x}{4}{512}{ISF1}}\\ 
%ergibt sich \MEquationItem{$F(x)$}{\MLSimplifyQuestion{30}{ln(x)+x+1/2*x^2+2*sqrt(x)}{10}{x}{4}{544}{ISF1b}} für $x>0$.\\
%\begin{MHint}{\iSolution}
%Die Funktion lässt sich zu $f(x)=\frac1x+1+x+x^{-\frac12}$ vereinfachen, was auf 
%$$
%F(x) \;=\; \ln(x)+x+\frac12x^2+2x^{\frac12} \;=\;\ln(x)+x+\frac12x^2+2\sqrt{x}
%$$
%führt (bis auf eine Konstante).
%\end{MHint}
%}
%\item{Mit der Vereinfachung \MEquationItem{$g(x)$}{\MLSimplifyQuestion{30}{4 + (tan(x))^2}{10}{x}{4}{512}{ISF2}}\\
%ergibt sich \MEquationItem{$G(x)$}{\MLSimplifyQuestion{30}{3*x + tan(x)}{10}{x}{4}{544}{ISF2b}}.\\
%\begin{MHint}{\iSolution}
%Die Funktion lässt sich zu $g(x)=4+\tan(x)^2$ vereinfachen, was auf 
%$$
%G(x) \;=\; 3x+\tan(x)
%$$
%führt (bis auf eine Konstante).
%\end{MHint}
%}
%\item{Mit der Vereinfachung \MEquationItem{$h(x)$}{\MLSimplifyQuestion{30}{1/2*1/(1 + x^2)}{10}{x}{4}{0}{ISF3}}\\
%ergibt sich \MEquationItem{$H(x)$}{\MLSimplifyQuestion{30}{1/2*arctan(x)}{10}{x}{4}{32}{ISF3b}}.\\
%\begin{MHint}{\iSolution}
%Die Funktion lässt sich zu $h(x)=\frac12\cdot \frac1{1+x^2}$ vereinfachen, was auf 
%$$
%H(x) \;=\; \frac12\cdot \arctan(x)
%$$
%führt (bis auf eine Konstante).
%\end{MHint}
%}
%\end{MExerciseItems}
%\MInputHint{Schreiben Sie beispielsweise $\cos^2(x)$ als \texttt{cos(x)^2} und den Arkustangens als \texttt{arctan(x)}.}
%\end{MExercise}


%Neue Aufgabe als Ersatz fuer die vorherige Version (siehe oben, jgl):
\begin{MExercise}
Bestimmen Sie eine Stammfunktion zu
\begin{MExerciseItems}
\item $\displaystyle f(x) := \frac{8 x^3 - 6 x^2}{x^4}$,
%\item $\displaystyle g(x) := \frac{5 x}{x + 2 x^2}$,
%\item $\displaystyle g(x) := \frac{6 \sqrt{x} + 1}{3 \sqrt{x^5}}$,
\item $\displaystyle g(x) := \frac{18 x^2}{3 \sqrt{x^5}}$,
\item $\displaystyle h(x) := \frac{x + 2 \sqrt{x}}{4x}$,
\end{MExerciseItems}
für $x > 0$, nachdem Sie die Funktionsterme als gekürzte Summen von Brüchen
geschrieben haben:

\begin{MExerciseItems}
\item{Mit der Vereinfachung \MEquationItem{$f(x)$}{\MLSimplifyQuestion{30}{8/x-6/x^2}{10}{x}{4}{512}{ISF1}}\\ 
ergibt sich \MEquationItem{$F(x)$}{\MLSimplifyQuestion{30}{8*ln(x)+6/x}{10}{x}{4}{544}{ISF1b}} für $x>0$.\\
\begin{MHint}{\iSolution}
Der Funktionsterm lässt sich zu 
$f(x) = \frac{8}{x} - \frac{6}{x^2} = \frac{8}{x} - 6 x^{-2}$ vereinfachen,
was auf 
\[
F(x) \;=\; 8 \cdot \ln(x) + \frac{6}{x} %
\]
für $x > 0$ führt (bis auf eine Konstante).
\end{MHint}
}
\item{Mit der Vereinfachung \MEquationItem{$g(x)$}{\MLSimplifyQuestion{30}{6/sqrt(x)}{10}{x}{4}{512}{ISF2}}\\
ergibt sich \MEquationItem{$G(x)$}{\MLSimplifyQuestion{30}{12*sqrt(x)}{10}{x}{4}{544}{ISF2b}} für $x > 0$.\\
\begin{MHint}{\iSolution}
Der Funktionsterm lässt sich zu %
$g(x) = \frac{6}{\sqrt{x}} = 6 \cdot x^{-\frac{1}{2}}$ %
vereinfachen, was auf 
\[
G(x) \;=\; 6 \cdot 2 \cdot \sqrt{x} = 12 \cdot \sqrt{x} %%
\]
für $x > 0$ führt (bis auf eine Konstante).
\end{MHint}
}
\item{Mit der Vereinfachung \MEquationItem{$h(x)$}{\MLSimplifyQuestion{30}{1/4+1/(2*sqrt(x))}{10}{x}{4}{512}{ISF3}}\\
ergibt sich \MEquationItem{$H(x)$}{\MLSimplifyQuestion{30}{x/4+sqrt(x)}{10}{x}{4}{544}{ISF3b}} für $x > 0$.\\
\begin{MHint}{\iSolution}
Der Funktionsterm lässt sich zu %
$h(x)=\frac{1}{4} + \frac{1}{2 \cdot \sqrt{x}}$ 
vereinfachen, was auf 
\[
H(x) \;=\; \frac{1}{4} \cdot x + \sqrt{x} %%
\]
für $x > 0$ führt (bis auf eine Konstante).
\end{MHint}
}
\end{MExerciseItems}
\MInputHint{Schreiben Sie beispielsweise $\sqrt{x}$ als \texttt{sqrt(x)}.}
\end{MExercise}


%Aufgabe durch die nachfolgende neue Aufgabe ersetzt (jgl):
%\begin{MExercise}
%Finden Sie zwei Polynome $p(x)$ und $q(x)$, sodass $F$ und $G$ mit 
%$F(x) = p(x) + x \cdot \ln(x)$ bzw. $G(x) = q(x) + x \cdot \ln(x)$
%Stammfunktionen von $f(x) = \ln(x)$ für $x > 0$ sind.
%Dann ist $F(x) - G(x)$ ein Polynom vom Grad \MLParsedQuestion{4}{0}{4}{IG21}, 
%und es gilt $F'(x) = $\MLSimplifyQuestion{10}{ln(x)}{1}{x}{20}{0}{IGYY1}.
%%Loesungshinweis erstellt (jgl):
%\begin{MHint}{Lösung}
%Es ist $H$ mit $H(x) = r(x) + x \cdot \ln(x)$ eine Stammfunktion von 
%$f$ mit $f(x) = \ln(x)$, wenn $H'(x) = f(x)$ für alle $x > 0$ und damit
%\[
%\ln(x) = f(x) = F'(x) = r'(x) + \ln(x) + x \cdot \frac{1}{x} %
% = r'(x) + 1 + \ln(x) %%
%\]
%gilt, woraus
%$0 = r'(x) + 1$ und damit $r'(x) = -1$ folgt. Somit ist $r(x) = -x + C$ für 
%eine Zahl $C$. Beispielsweise sind $F$ und $G$ mit
%$F(x) = -x + 0 + x \cdot \ln(x)$ sowie
%$G(x) = -x + -14 + x \cdot \ln(x)$ Stammfunktionen von $f$.
%
%Es ist $F(x) - G(x) = 0 - (-14) = 14$, sodass $F - G$ ein Polynom vom Grad $0$
%ist. Weiter ist $F'(x) = -1 + \ln(x) + 1 = \ln(x)$ für alle $x > 0$.
%\end{MHint}
%\end{MExercise}

%Aufgabe neu erstellt (jgl):
\begin{MExercise}
Gegeben ist eine Funktion $f$ mit $f(x) = \frac{1}{x}$ für $x > 0$.
Weiter sind Funktionen $F_1$ und $F_2$ mit 
$F_1(x) = \ln(7x)$ bzw. $F_2(x) = \ln(x + 7)$ für $x > 0$ gegeben.
Berechnen Sie die Ableitung von $F_1$ und von $F_2$, und beantworten Sie die 
Frage ob, es sich um Stammfunktionen von $f$ handelt:

Es ist: 
 \MEquationItem{$F_1'(x)$}{\MLSimplifyQuestion{30}{1/x}{10}{x}{4}{512}{IEx11ln}}
und 
 \MEquationItem{$F_2'(x)$}{\MLSimplifyQuestion{30}{1/(x+7)}{10}{x}{4}{512}{IEx12ln}}. 

Kreuzen Sie die richtigen Aussagen an:
\par
\begin{MQuestionGroup}
\MLCheckbox{1}{M08Ex1120a} $F_1$ ist eine Stammfunktion von $f$.\ \\
\MLCheckbox{0}{M08Ex1120b} $F_2$ ist eine Stammfunktion von $f$.\ \\
\end{MQuestionGroup}
\par
\MGroupButton{Eingaben überprüfen}

\begin{MHint}{Lösung}
Die Ableitung von $F_1$ mit $F_1(x) = \ln(7 x)$ für $x > 0$ ist
$F_1'(x) = \frac{1}{7 x} \cdot 7 = \frac{1}{x} = f(x)$. Somit ist $F_1$ eine 
Stammfunktion von $f$.

Die Funktion $F_2$ mit $F_2(x) = \ln(x + 7)$ für $x > 0$ hat die Ableitung
$F_2'(x) = \frac{1}{x + 7} \cdot 1 = \frac{1}{x + 7} \neq \frac{1}{x}$ für 
alle $x > 0$. Deshalb ist $F_2$ keine Stammfunktion von $f$.
%
%Optional koennte auf eine moegliche Umformung des Funktionsterms der ersten
%Funktion hingewiesen werden, um zu verdeutlichen, weshalb die Funktion dieselbe
%Ableitung wie die natuerliche Logarithmusfunktion hat (jgl):
%Zur Berechnung der Ableitung von $F_1$ soll noch darauf hingewiesen werden,
%dass der Funktionsterm mit den Rechenregeln zum Logarithmus zu 
%$F_1(x) = \ln(7 \cdot x) = \ln(7) + \ln(x)$ umgeformt werden kann, woraus sich
%dann ebenfalls die Ableitung $F_1'(x) = \frac{1}{x}$ ergibt.
\end{MHint}
\end{MExercise}

%Letztes Aufgabenelement darf kein MSimplifyQuestion sein, weil die Eingabebox 
%sonst unter dem Bildschirmrand landet (jgl, 2014).
\begin{MExercise}
%Aufgabentext ueberarbeitet (jgl):
Es wird angenommen, dass $F$ eine Stammfunktion von $f$ mit $f(x)=1 + x^2$ ist
und $F$ den Funktionswert $F(0) = 1$ hat. Dann 
ist $F(3)$ = \MLParsedQuestion{10}{13}{10}{IG22}.
%Loesungshinweis erstellt (jgl):
\begin{MHint}{Lösung}
Nach Voraussetzung ist $F$ eine Stammfunktion von $f$ mit $f(x) = 1 + x^2$. 
Somit gilt $F'(x) = f(x) = 1 + x^2 = x^0 + x^2$, woraus 
\[
F(x) = \frac{1}{0 + 1} \cdot x^{0+1} + \frac{1}{2 + 1} \cdot x^{2 + 1} + C %
= x + \frac{1}{3} \cdot x^3 + C %%
\]
für eine Zahl $C$ folgt.
Außerdem soll $F(0) = 1$ gelten, sodass 
$1 = F(0) = 0 + \frac{1}{3} \cdot 0^3 + C = C$ ist. Damit erhält man
$F(x) = \frac{1}{3} \cdot x^3 + x + 1$. Einsetzen von $x = 3$ ergibt den 
gesuchten Wert $F(3) = \frac{1}{3} \cdot 3^3 + 3 + 1 = 13$.
\end{MHint}
\end{MExercise}

\end{MExercises}




%%%Abschnitt
\MSubsection{Bestimmtes Integral}\MLabel{M08A_Integral}

\begin{MIntro}
\MDeclareSiteUXID{VBKM08_Integral_Intro}
Die Ableitung $f'(x_0)$ einer differenzierbaren Funktion $f$ beschreibt, 
wie sich die Funktionswerte {\glqq}in der Nähe{\grqq} der Stelle $x_0$ ändern:
Wenn beispielsweise die Ableitung positiv ist, dann ist $f$ monoton wachsend.
Geometrisch betrachtet drückt sich dies durch eine Tangente an den Graphen 
von $f$ an der Stelle $x_0$ mit positiver Steigung aus.
Die Ableitung bietet eine lokale Sichtweise auf die Funktion an jeder Stelle 
$x_0$. Dadurch gewinnt man sehr viele Detailinformationen.

Umgekehrt erhält man eine {\glqq}globale Kenngröße{\grqq}, wenn man eine 
{\glqq}Bilanz{\grqq} erstellt, indem man die Funktionswerte gewichtet 
aufsummiert. In der Mathematik spricht man vom Integral oder Integralwert
der Funktion.
Geometrisch betrachtet, führt diese Idee auf eine Möglichkeit, den 
Flächeninhalt unter dem Graphen einer Funktion zu berechnen.
Präzisiert wurde dieses Vorgehen von Bernhard Riemann, nach dem dieses 
Integral auch \MEntry{Riemann-Integral}{Integral (Riemann)} genannt wird.
\end{MIntro}

\begin{MXContent}{Integral}{Integral}{STD}
\MDeclareSiteUXID{VBKM08_Integral_Content}

%Hier wird die Idee zu einer {\glqq}globalen Kenngröße{\grqq} in der 
%Form des nach Riemann benannten Integrals vorgestellt.

%Das nach Riemann benannte Integral ordnet einer Funktion eine Zahl zu, die 
%Das Integral einer Funktion ist eine Zahl, die sich aus einer Summenbildung
%von gewichteten Funktionswerten ergibt. 
%Für das nach Riemann benannte Integral wird die Gewichtung über die 
%Länge des Definitionsbereichs vorgenommen. 
Das Integral einer Funktion $f$ mit $f(x) \geq 0$ kann als 
{\glqq}Fläche unter der Kurve{\grqq} interpretiert werden. In dem nach 
Riemann benannten Integral wird der Funktionsverlauf durch 
eine Treppenfunktion angenähert, und deren Funktionswerte werden, gewichtet 
mit der jeweiligen Intervalllänge bzw. {\glqq}Breite einer Treppenstufe{\grqq}, 
aufsummiert. Dies ist in der unten gezeigten Abbildung beispielhaft dargestellt.
%
\begin{center}
\MTikzAuto{%
\ifttm\else\begin{small}\fi
\begin{tikzpicture}[line width=1.5pt,scale=1.0, %
declare function={
  x0 = 1;
  x1 = 2;
  x2 = 4;
  x3 = 7;
  x4 = 8;
  IntX(\n) = array({1, 2, 4, 7, 8},\n);
  ZwWert(\n) = array({1, 1.5, 3, 5.5, 7.5},\n);
  IntAbs(\n) = array({0, 1, 2, 3, 1},\n);
  fkt(\x) = 2 + (\x - 2)*(\x - 3)*(\x - 7)/20;
}
] %[every node/.style={fill=white}] 
%,every node/.style={fill=white}] 
%Rechteck, farbig:
\begin{scope}[color=blue!20!white]
\foreach \n in {0, 1, 2, 3} 
\fill ({IntX(\n)}, 0.0) %
   rectangle ({IntX(\n) + IntAbs(\n+1)}, {fkt(ZwWert(\n+1))}); 
\end{scope}
%Koordinatenachsen:
\draw[->] (-0.6, 0) -- (9, 0) node[below left]{$x$}; %x-Achse
\draw[->] (0, -0.6) -- (0, 5) node[below left]{$y$}; %y-Achse
%Achsenbeschriftung:
\foreach \x/\xtext in {1/{1}, 2/{2}, 3/{}, 4/{4}, 5/{}, 6/{}, 7/{7}, 8/{8}} %
  \draw (\x, 0) -- ++(0, -0.1) %
   node[below] {$\xtext$}; 
\foreach \y in {1, 2, 3, 4} \draw (0, \y) -- ++(-0.1, 0) %
 node[left] {$\y$};
%\node[below left] at (0, 0) {$0$};
%Treppenfunktion:
\begin{scope}[line width=1pt,color=blue, fill=blue!50!white]
\foreach \n in {0, 1, 2, 3} 
\draw ({IntX(\n)}, {fkt(ZwWert(\n+1))}) -- ++({IntAbs(\n+1)}, 0); 
%\foreach \n/\zwert in {0/{1.5}, 1/{3}, 2/{5.5}, 3/{7.5}} 
%\node[above] at ({ZwWert(\n+1)}, {fkt(ZwWert(\n+1))}) {$f(\zwert)$}; %
\node[above left] at ({ZwWert(1)}, {fkt(ZwWert(1))}) {$f(1.5)$}; %
\node[above] at ({ZwWert(2)}, {fkt(ZwWert(2))}) {$f(3)$}; %
\node[above] at ({ZwWert(3)}, {fkt(ZwWert(3))}) {$f(5.5)$}; %
\node[above left] at ({ZwWert(4)}, {fkt(ZwWert(4))}) {$f(7.5)$}; %
\end{scope}
%Funktion:
\draw[domain=1.0:8.0,samples=120,color=\jccolorfkt] %
 plot (\x, {fkt(\x)});
%Beschriftung extern:
%\node at (-3,-1.5) {Zur Definition des Integrals: Funktion angenähert durch %
% eine Treppenfunktion,};
%\node at (-3,-2) {unterteilt in vier Teilintervalle.};
\end{tikzpicture}
}

Zur Definition des Integrals: Funktion angenähert durch %
eine Treppenfunktion,\\
unterteilt in vier Teilintervalle.
\ifttm\else\end{small}\fi
\end{center}
%
Man erkennt, dass die Fläche unter der Kurve zunächst durch Rechtecke
angenähert wird, deren eine (horizontale) Seitenlänge durch ein Intervall 
auf der $x$-Achse bestimmt wird, während die Länge der zweiten (vertikalen) 
Seite durch einen Funktionswert $f(z_k)$ an der Stelle $z_k$ innerhalb des 
dazu gehörenden $x$-Intervalls beschrieben wird. Man bestimmt nun die Flächen 
dieser Rechtecke und summiert diese Teilflächen auf.
Je kleiner die Intervalle auf der $x$-Achse werden, umso mehr nähert sich die 
so berechnete Summe dem {\glqq}wahren{\grqq} Wert der Fläche unter der Kurve, 
also dem Integral der Funktion, an.

Formal heißt das, dass man eine Summe $S_n$ der Form
%
\[
S_n = \sum_{k=0}^{n-1} f(z_k) \cdot \Delta(x_k) \qquad \text{mit } %
 \Delta(x_k) = x_{k+1} - x_{k} %%
\]
%
bestimmt. Im betrachteten Beispiel wird das Intervall $[0\MIntvlSep 8]$ in 
vier Teile eingeteilt. Dabei sind $x_0 = 1$, $x_1 = 2$, $x_2 = 4$ und $x_3 = 7$. 
Wendet man darauf diese Summenformel an, so erhält man
%
\begin{eqnarray*}
S_4 &=& f(z_0) \cdot (x_1-x_0) + f(z_1) \cdot (x_2-x_1) + f(z_2) \cdot (x_3-x_2) + f(z_3) \cdot (x_4-x_3)\\
 &=& f(z_0) \cdot (2-1) + f(z_1) \cdot (4-2) + f(z_2) \cdot (7-4) + f(z_3) \cdot (8-7)\\
 &=& f(z_0) \cdot 1 + f(z_1) \cdot 2 + f(z_2) \cdot 3 + f(z_3) \cdot 1 \MDFPeriod
\end{eqnarray*}
%
Offensichtlich genügt es nicht, nur einige wenige Teilintervalle zu 
betrachten. Im Allgemeinen wird man die maximale Länge der Teilintervalle 
$x_{k+1} - x_k$ immer kleiner wählen müssen und damit für immer mehr 
Teilintervalle die Summanden $f(z_k) \cdot (x_{k+1} - x_k)$ berechnen und 
addieren, um einen möglichst genauen Wert der Fläche zu berechnen. Deshalb
betrachtet man den Grenzwert, dass die maximale Intervalllänge der 
Teilintervalle gegen Null geht.

Prinzipiell kann obiges Vorgehen auch auf Funktionen mit negativen 
Funktionswerten angewandt werden. Wie man dann den Flächeninhalt berechnet,
wird im Abschnitt \MRef{M08A_Anwendung} erläutert.
In jedem Fall sind einige Aspekte in der Definition des Integrals zu beachten, 
die über den Rahmen dieses Kurses hinausgehen. Deshalb wird für die Details 
zu den Voraussetzungen in der nachfolgenden Definition auf die weiterführende 
Literatur verwiesen.

\begin{MXInfo}{Integral}
Gegeben ist eine Funktion $f: [a\MIntvlSep b] \rightarrow \R$ auf einem reellen 
Intervall $[a\MIntvlSep b]$. Zu {\glqq}feiner werdenden{\grqq} Unterteilungen 
des Intervalls, sodass $x_{k+1} - x_k$ gegen $0$ geht, nennt man
%
\begin{equation}
\int_a^b f(x) \MDwSp x = \lim_{n \rightarrow \infty} S_n %
 = \lim_{n \rightarrow \infty} \sum_{k=0}^{n-1} f(z_k) \cdot (x_{k+1} - x_{k}) %
\quad \text{mit } x_{k} \leq z_k \leq x_{k+1} %%
\end{equation}
%
das \MEntry{bestimmte Integral}{Integral (bestimmt)} von $f$ mit der 
Untergrenze $a$ und der Obergrenze $b$ (wenn der Grenzwert existiert und 
unabhängig von der jeweiligen Unterteilung ist), und die Funktion $f$ 
heißt dann \MEntry{integrierbar}{integrierbar}.
Die Funktion $f$ wird in diesem Kontext auch \MEntry{Integrand}{Integrand} 
genannt.
\end{MXInfo}
%
%\begin{MXInfo}{Integral}
%Gegeben ist eine Funktion $f: [a, b] \rightarrow \R$ auf einem reellen Intervall
%$[a, b]$. 
%Das \MEntry{bestimmte Integral}{Integral!bestimmtes} von $f$ ist der Grenzwert der Summe
%
%\begin{equation}
%S_n := \sum_{k=0}^{n-1} f(x_k) \cdot \Delta(x_k) \qquad \text{mit } \Delta(x_k) := x_{k+1} - x_{k} %%
%\end{equation}
%
%für $n$ gegen unendlich, wenn $\max_{1 \leq k \leq n} \Delta(x_k)$ gegen null 
%strebt.
%Dabei ist $a = x_0 < x_1 < x_2 < \ldots < x_n = b$ und $z_k$ eine Zahl 
%zwischen $x_{k-1}$ und $x_k$ für $k \in \N$ mit $1 \leq k \leq n$. 
%Die Summen $S_n$ werden auch Riemannsummen zur Zerlegung $(x_0, \ldots, x_n)$ 
%und den Zwischenstellen $z_k$ genannt.

%Wenn dieser Grenzwert, das Integral, existiert, wird dafür
%\begin{equation}
%\int_a^b f(x) \MDwSp x = \lim_{n \rightarrow \infty} S_n %%
%\end{equation}
%geschrieben. Es heißt $a$ Untergrenze, $b$ Obergrenze des Integrals über dem
%Integranden $f(x)$ mit der Integrationsvariablen $x$.
%\end{MXInfo}
Prinzipiell kann es vorkommen, dass auf diese Weise gar kein bestimmter Wert 
berechnet werden kann, das Integral also nicht exisitert.
Weiterführende Überlegungen zeigen, dass jedenfalls für jede stetige 
Funktion das Integral existiert.
%
Als Beispiel wird das Integral von 
$f: [0\MIntvlSep 1] \rightarrow \R,\, x \Mmapsto x$ 
berechnet, wobei die Berechnung des Grenzwertes im Vordergrund steht.

\begin{MExample}
Es soll das Integral von $f: [0\MIntvlSep 1] \rightarrow \R,\, x \Mmapsto x$ 
berechnet werden.
Dazu teilt man das Intervall $[0\MIntvlSep 1]$ am einfachsten in gleich breite
Teilintervalle $[x_{k}\MIntvlSep x_{k+1}]$ mit 
$x_0 := 0$ und $x_k := x_{k-1} + \frac{1}{n}$ ein. Die Intervalllänge ist also
$\Delta(x_k) = x_{k+1} - x_{k} = \frac{1}{n}$.

Untersucht man die Intervalllänge auf ihr Verhalten für $n$ gegen unendlich, 
dann sieht man, dass $\Delta(x_k)$ immer kleiner wird und gegen Null strebt. 
Die Voraussetzung für die Berechnung eines bestimmten Integrals ist also gegeben.

Für die Werte $x_k$ findet man unter Zuhilfenahme der Intervalllänge außerdem 
$x_k = \frac{k}{n}$. Wählt man $z_k = x_k$ für die Zwischenstellen, ergibt 
sich $f(z_k) = f(x_k) = x_k = \frac{k}{n}$.

Setzt man diese Ergebnisse in die Summenformel ein, dann erhält man unter der 
Verwendung der Formel $\sum_{k=1}^{n-1} k = \frac{1}{2}\,n(n-1)$, die 
nach C. F. Gauß auch als {\glqq}kleiner Gauß{\grqq} bezeichnet wird, die
Gleichung
%
\begin{eqnarray*}
S_n &=& \sum_{k=0}^{n-1} f(x_k)\cdot\Delta(x_k) = \sum_{k=0}^{n-1} x_k \cdot \frac{1}{n} = \sum_{k=0}^{n-1} \frac{k}{n} \cdot \frac{1}{n} = \frac{1}{n^2} \cdot \sum_{k=0}^{n-1} k = \frac{1}{n^2} \cdot \sum_{k=1}^{n-1} k\\
 &=& \frac{1}{n^2} \frac{n (n-1)}{2} = \frac{1}{2} \cdot \frac{n-1}{n} = \frac{1}{2} \cdot \left(1 - \frac{1}{n}\right) \MDFPeriod %
\end{eqnarray*}
%
Und mit $\displaystyle\lim_{n \rightarrow \infty} \frac{1}{n} = 0$ ergibt 
sich für das Integral
\[
\int_0^1 x \MDwSp x = \lim_{n \rightarrow \infty} S_n = \frac{1}{2} \MDFPeriod %%
\]
\end{MExample}

Zur großen Klasse integrierbarer Funktionen gehören alle Polynome,
gebrochenrationalen Funktionen, trigonometrischen Funktionen und Exponential- 
und Logarithmusfunktionen sowie deren Verknüpfungen.

%==========================
% Anfang Hauptsatz

Um Rechnungen möglichst unkompliziert durchführen zu können, sind möglichst 
einfache Regeln zur Integration von Funktionen nötig. Ein wichtiges Ergebnis 
liefert der 
sogenannte \MEntry{Hauptsatz der Differential- und Integralrechnung}{Hauptsatz (Integralrechnung)}.
Er beschreibt einen Zusammenhang zwischen den Stammfunktionen einer stetigen 
Funktion und deren Integral.

\begin{MXInfo}{Hauptsatz der Differential- und Integralrechnung}
\MLabel{HSDDIR}
Gegeben ist eine stetige Funktion $f: [a\MIntvlSep b] \rightarrow \R$ auf 
einem reellen Intervall $[a\MIntvlSep b]$. Dann besitzt $f$ eine Stammfunktion, 
und für jede Stammfunktion $F$ von $f$ gilt
\[
\int_{a}^{b} f(x) \MDwSp x \;=\;  \left[{F(x)}\right]_a^b\;=\; F(b) - F(a) \MDFPeriod %% 
\]
\end{MXInfo}

Als einfaches Beispiel wird das bestimmte Integral der Funktion $f$ mit 
$f(x) = x^2$ zwischen $a = 1$ und $b = 2$ berechnet. Mit den Regeln für die 
Bestimmung von Stammfunktionen und dem Hauptsatz der Differential- und 
Integralrechnung kann man diese Aufgabe sehr leicht lösen.

\begin{MExample}
Die Funktion $f: [1\MIntvlSep 2] \rightarrow \R$ mit $f(x) := x^2$ hat nach 
der Tabelle aus dem ersten Abschnitt eine Stammfunktion $F$ mit 
$F(x) = \frac{1}{3} x^{3} + C$ für eine reelle Zahl $C$.
Mit dem Hauptsatz ergibt sich
\[
\int_1^2 x^2 \MDwSp x %
 = \left[\frac{1}{3} x^{3} \,+\, C \right]_1^2 %
 = \left(\frac{1}{3}\, 2^{3} \,+\,C \right) - \left(\frac{1}{3}\, 1^{3} \,+\, C \right) %
 = \frac{7}{3} \MDFPeriod %%
\]
Wie man sieht, fällt die Konstante nach dem Einsetzen der Grenzen weg, sodass 
man sie in der Praxis bei der Berechnung von bestimmten Integralen bereits bei 
der Bildung der Stammfunktion "`unterschlagen"' kann. 
Das heißt, man kann für die Berechnung des bestimmten Integrals $C = 0$ wählen.
%
%Beispielsweise ist für $n = 1$, also $f(x) = x$ dann
%$F$ mit $F(x) = \frac{1}{2} x^2$ eine Stammfunktion, sodass
%\[
%\int_a^b x \MDwSp x = \left[\frac{1}{2} x^2\right]_a^b %
% = \frac{1}{2} b^2 - \frac{1}{2} a^2 %%
%\]
%gilt.
\end{MExample}

Die Gleichung im Hauptsatz gilt auch für jeden Zwischenwert 
$z \in [a\MIntvlSep b]$, sodass man gemäß
\[
F(z) - F(a) = \int_a^z f(x) \MDwSp x %%
\]
damit alle Funktionswerte $F(z)$ berechnen kann, wenn die Ableitung $F' = f$
und ein Funktionswert, beispielsweise $F(a)$, bekannt sind. Hierfür sagt man 
auch, dass $F$ aus der Ableitung $F' = f$ rekonstruiert wird. 

Anwendungsbeispiele zur Rekonstruktion einer Funktion $F$ aus ihrer Ableitung
$F' = f$ werden am Ende des Abschnitts \MRef{M08A_Anwendung} vorgestellt.

%In der Situation des Hauptsatzes ist für jede Zahl $z$ zwischen $a$ und $b$ dann
%\[
%\int_{a}^{z} f(x) \MDwSp x = F(z) - F(a). %% 
%\]
%Folglich ergibt sich $F(z)$ gemäß
%\begin{equation}
%F(z) = F(a) + \int_{a}^{z} f(x) \MDwSp x = F(a) + \int_{a}^{z} F'(x) \MDwSp x. %% 
%\end{equation}
%aus dem Funktionswert und dem Integral über die Ableitung. Das heißt, dass 
%die Funktion aus der Integration über die Ableitung, also die Änderungsrate
%wiedergewonnen werden kann.

%Dies ist für viele Anwendungen ein bemerkenswertes Hilfsmittel aus der 
%Mathematik. Beispielsweise werden in naturwissenschaftlichen oder technischen 
%Vorgängen oft Veränderungen wie die Geschwindigkeit gemessen. Die Bewegung 
%kann dann daraus mittels Integration rekonstruiert werden.

% Ende Hauptsatz
%==========================

%Eine Abschätzung des Integrals ergibt sich, wenn man einerseits Riemannsummen 
%$U_n$ mit {\glqq}Zwischenstellen{\grqq} $z_k$ berechnet, für die $f(z_k)$ ein
%minimaler Wert der Funktion $f$ auf $[x_{k-1}, x_k]$ ist, und andererseits 
%solche Riemannsummen $O_n$ betrachtet, für die $f(z_k)$ einen maximalen Wert 
%hat. (Sofern das Minimum bzw. Maximum nicht existiert, betrachtet man das 
%Infimum bzw. das Supremum.)

%Aufgrund dieser Konstruktion ist $U_n \leq O_n$. 
%Dementsprechend heißen die Riemannsummen $U_n$ Untersummen und $O_n$ Obersummen. 
%Wenn die Grenzwerte für aller Unterteilungen 
%existieren und gleich sind, wenn also das Integral existiert, dann gilt die 
%Ungleichung
%\begin{equation}
%U_n \leq \int_a^b f(x) \MDwSp x \leq O_n
%\end{equation}
%Eine solche Ungleichung wird auch Abschätzung genannt, da ein gesuchter Wert 
%mit einem anderen, möglichst einfach zu berechnenden Wert verglichen wird, 
%der zudem möglichst {\glqq}nahe{\grqq} am gesuchten Wert liegt.

%\begin{MExample}
%Im vorherigen Beispiel wurde zu $f(x) = x$ für jedes $n \in \N$ und 
%$x_k = \frac{b \cdot k}{n}$, also 
%$\Delta(x_k) = x_k - x_{k-1} = \frac{b}{n}$ 
%eine Obersumme
%\[
%O_n = \sum_{k=1}^n \sup_{[x_{k-1}, x_k]} f(x) \cdot \Delta(x_k) %
% = \sum_{k=1}^n \frac{b \cdot k}{n} \cdot \Delta(x_k) %%
%\]
%konstruiert.

%Eine Untersumme $U_n$ ist für diese Einteilung durch
%\[
%U_n = \sum_{k=1}^n \inf_{[x_{k-1}, x_k]} f(x) \cdot \Delta(x_k) %
% = \sum_{k=1}^n \frac{b \cdot (k-1)}{n} \cdot \Delta(x_k) %%
%\]
%gegeben.
%\end{MExample}

\end{MXContent}


%\begin{MXContent}{Rechenregeln für Integrale}{Rechenregeln für Integrale}{STD}
\begin{MXContent}{Rechenregeln}{Rechenregeln}{STD}
\MDeclareSiteUXID{VBKM08_Rechenregeln}

\begin{MXInfo}{Zerlegung des Integrationsintervalls eines Integrals}
Sei $f: [a\MIntvlSep b] \rightarrow \R$ eine integrierbare Funktion. Dann gilt 
für jede Zahl $z$ zwischen $a$ und $b$
%
$$
\int_a^b f(x) \MDwSp x = \int_a^z f(x) \MDwSp x + \int_z^b f(x) \MDwSp x \MDFPeriod%%
$$
\end{MXInfo}

Mit der Festlegung
%
$$
\int_d^c f(x) \MDwSp x := -\int_c^d f(x) \MDwSp x %%
$$
%
gilt die obige Regel für alle reellen Zahlen $z$, für die die beiden rechts 
stehenden Integrale existieren, auch wenn $z$ nicht zwischen $a$ und $b$ liegt.
Bevor obige Rechenregel an einem Beispiel erläutert wird, wird die genannte
Festlegung noch ausführlich notiert.

\begin{MXInfo}{Vertauschung der Grenzen eines Integrals}
Sei $f: [a\MIntvlSep b] \rightarrow \R$ eine integrierbare Funktion. Dann wird
das Integral der Funktion $f$ von $b$ bis $a$ gemäß 
\[
\int_b^a f(x) \MDwSp x = -\int_a^b f(x) \MDwSp x %%
\]
berechnet.
\end{MXInfo}

Die oben beschriebene Rechenregel ist praktisch, um Funktionen mit Beträgen 
oder andere abschnittsweise definierte Funktionen zu integrieren.

\begin{MExample}
Das Integral der Funktion $f: [-4\MIntvlSep 6] \rightarrow \R, x \Mmapsto |x|$ ist
\begin{eqnarray*}
\int_{-4}^{6} |x| \MDwSp x % 
 & = & \int_{-4}^{0} (-x) \MDwSp x + \int_{0}^{6} x \MDwSp x \\
 & = & \left[-\frac{1}{2} x^2\right]_{-4}^{0} + \left[\frac{1}{2} x^2\right]_{0}^{6} \\
 & = & (0 - (-8)) + (18 - 0) \\
 & = & 26 \MDFPeriod
\end{eqnarray*}
\end{MExample}

Die Integration über die Summe zweier Funktionen kann ebenfalls in zwei 
Integrale zerlegt werden: 

\begin{MXInfo}{Summen- und Faktorregel}
\MIndex{Summenregel (Integral)}
Seien $f$ und $g$ auf $[a\MIntvlSep b]$ integrierbare Funktionen und $r$ eine 
reelle Zahl. Dann gilt
\begin{equation}
\int_a^b (f(x) + g(x)) \MDwSp x = \int_a^b f(x) \MDwSp x + \int_a^b g(x) \MDwSp x \MDFPeriod %%
\end{equation}

Für Vielfache einer Funktion gilt
\begin{equation}
\int_a^b r \cdot f(x) \MDwSp x = r \cdot \int_a^b f(x) \MDwSp x \MDFPeriod %%
\end{equation}
\end{MXInfo}

\begin{MCOSHZusatz}
%Formulierungen im Zusatz ueberarbeitet und Aufgaben erstellt (jgl):
Auch für die Berechnung eines Produktes zweier Funktionen gibt es eine 
Rechenregel. Sie ergibt sich aus der Produktregel der Ableitung.

\begin{MXInfo}{Partielle Integration}
Seien $u$ und $v$ auf $[a\MIntvlSep b]$ differenzierbare Funktionen mit 
stetigen Ableitungen $u'$ beziehungsweise $v'$. 
Für das Integral der Funktion $u \cdot v'$ gilt dann
%
$$
\int_a^b u(x) \cdot v'(x) \MDwSp x %
 = \left[u(x)\cdot v(x)\right]_a^b - \int_a^b u'(x) \cdot v(x) \MDwSp x \MDFPSpace, %%
$$
%
wobei $u'$ die Ableitung von $u$ ist und $v$ eine Stammfunktion von $v'$ ist.
Diese Rechenregel wird \MEntry{partielle Integration}{Integration (partielle)} 
genannt.
\end{MXInfo}

Auch zu dieser Regel wird ein Beispiel betrachtet:

\begin{MExample}
Es wird das Integral
%
$$
\int_{0}^{\pi} x \sin(x) \MDwSp x
$$
%
mit Hilfe der partiellen Integration berechnet. Dazu werden die Funktionen $u$ 
und $v'$ mit
%
\begin{eqnarray*}
u(x) = x \quad \text{und} \quad v'(x) = \sin(x) %%
\end{eqnarray*}
%
gewählt. Damit erhält man 
%
\begin{eqnarray*}
u'(x) = 1 \quad\text{und}\quad v(x) = -\cos x \MDFPeriod
\end{eqnarray*}
%
So kann man das gesuchte Integral mit partieller Integration berechnen:
%
\begin{eqnarray*}
\int_{0}^{\pi} x \sin x \MDwSp x %
& = & \left[x \cdot (-\cos(x))\right]_0^\pi %
      - \int_{0}^{\pi} 1 \cdot (-\cos x) \MDwSp x \\
& = & \pi \cdot (-\cos(\pi)) - 0 + \int_{0}^{\pi} \cos(x) \MDwSp x \\
& = & \pi \cdot (-(-1)) + \left[\sin(x)\right]_0^\pi = \pi \MDFPeriod
\end{eqnarray*}
%
Die Zuordnung der Funktionen $u$ und $v'$ muss zielführend erfolgen. Dies wird 
am obigen Beispiel deutlich, wenn die Rollen von $u$ und $v'$ vertauscht werden. 
Die Leserin bzw. der Leser mag probieren, dieses Beispiel zu lösen, indem $u$ 
und $v'$ anders herum gewählt werden!
\end{MExample}

Anhand der folgenden beiden Übungsaufgaben kann man die Regel zur 
partiellen Integration selbst anwenden lernen.

\begin{MExercise}
Es soll das Integral $\displaystyle I = \int_{1}^{4} x \cdot \MEU^{x} \MDwSp x$
berechnet werden: $I = $\MLParsedQuestion{10}{3*e^4}{4}{M08ZEx01}. 

\begin{MHint}{Lösung}
Der Integrand $f$ mit $f(x) = x \cdot \MEU^{x}$ ist ein Produkt einer 
Polynomfunktion $u$ mit $u(x) = x$ und einer Exponentialfunktion. Die 
Ableitung von $u$ ist $u'(x) = 1$, also eine konstante Funktion. Außerdem 
ist zur Exponentialfunktion eine Stammfunktion bekannt: Zu $v'$ mit 
$v'(x) = \MEU^{x}$ ist $v$ mit $v(x) = \MEU^{x}$ eine Stammfunktion. 
Damit erhält man mit partieller Integration
\begin{eqnarray*}
 \int_{1}^{4} x \cdot \MEU^{x} \MDwSp x %
& = & \left[ x \cdot \MEU^{x} \right]_{1}^{4} - \int_{1}^{4} 1 \cdot \MEU^{x} \MDwSp x \\
& = & \left[ x \cdot \MEU^{x} \right]_{1}^{4} - \left[ \MEU^{x} \right]_{1}^{4} \\
& = & \left[ x \cdot \MEU^{x} - \MEU^{x} \right]_{1}^{4} \\
& = & \left[ (x - 1) \cdot \MEU^{x} \right]_{1}^{4} \\
& = & 3 \MEU^{4} \MDFPeriod %%
\end{eqnarray*}
\end{MHint}
\end{MExercise}

\begin{MExercise}
Es soll das Integral $I = \displaystyle \int_{1}^{8} x \cdot \ln(x) \MDwSp x$
berechnet werden: $I = $\MLParsedQuestion{25}{96*ln(2)-16+1/4}{4}{M08ZEx02}. 

\begin{MHint}{Lösung}
Der Integrand $f$ mit $f(x) = x \cdot \ln(x) = \ln(x) \cdot x$ für 
$1 \leq x \leq 8$ ist ein Produkt einer Polynomfunktion und einer 
Logarithmusfunktion. Die Ableitung der Logarithmusfunktion $u$ mit 
$u(x) = \ln(x)$ ergibt $u'(x) = \frac{1}{x}$.
Somit ist $u'$ eine "`einfache"' rationale Funktion. Weiter ist zur 
Polynomfunktion $v'$ mit $v'(x) = x$ eine Stammfunktion bekannt,
nämlich $v$ mit $v(x) = \frac{1}{2} \cdot x^2$. 

Das Produkt $u' \cdot v$ mit 
$u'(x) \cdot v(x) = \frac{1}{x} \cdot \frac{1}{2} x^2 = \frac{1}{2} x$ 
für $1 \leq x \leq 8$ ist eine stetige Funktion, zu der ebenfalls eine 
Stammfunktion bekannt ist. Damit kann das gesuchte Integral mittels 
partieller Integration gemäß
\begin{eqnarray*}
\int_{1}^{8} x \cdot \ln(x) \MDwSp x %
 = \int_{1}^{8} \ln(x) \cdot x \MDwSp x %
& = & \left[ \ln(x) \cdot \frac{1}{2} x^2 \right]_{1}^{8} %
      - \int_{1}^{8} \frac{1}{x} \cdot \frac{1}{2} x^2 \MDwSp x \\
& = & \left[ \ln(x) \cdot \frac{1}{2} x^2 \right]_{1}^{8} %
      - \int_{1}^{8} \frac{1}{2} x \MDwSp x \\
%& = & \left[ \ln(x) \cdot \frac{1}{2} x^2 \right]_{1}^{8} %
%     - \left[ \frac{1}{2} \cdot \frac{1}{2} x^2 \right]_{1}^{8} \\
& = & \left[ \frac{1}{2} x^2 \cdot \ln(x) - \frac{1}{4} x^2 \right]_{1}^{8} \\
& = & \left(32 \ln(8) - 16\right) - \left(1 \cdot \ln(1) - \frac{1}{4}\right) \\
& = & 96 \ln(2) - 16 + \frac{1}{4} \MDFPeriod %%
\end{eqnarray*}
berechnet werden, wobei $\ln(8) = \ln(2^3) = 3 \cdot \ln(2)$ und $\ln(1) = 0$ 
verwendet wurde.
\end{MHint}
\end{MExercise}
\end{MCOSHZusatz}

\end{MXContent}


\begin{MXContent}{Eigenschaften des Integrals}{Eigenschaften des Integrals}{STD}
\MDeclareSiteUXID{VBKM08_Eigenschaften_Integral}

Für ungerade Funktionen $f: [-c\MIntvlSep c] \rightarrow \R$ ist das Integral 
Null. Dies soll am Beispiel der Funktion $f$ auf $[-2\MIntvlSep 2]$ mit 
$f(x) = x^3$ erläutert werden:
%\ifttm
%\MUGraphics{BildFlaechePolynomxdrei.png}{scale=0.4}{Ungerade Funktion $f(x) = x^3$ auf einem Intervall $[-2\MIntvlSep 2]$.}{}
%\else
\begin{center}
\MTikzAuto{%
\ifttm\else\begin{small}\fi
\begin{tikzpicture}[scale=0.8,line width=1.5pt]
%\clip(-3.7,-4.7) rectangle (3.7, 4.7);
%Koordinatenachsen:
\draw[->] (-3.6, 0) -- (3.8, 0) node[below left]{$x$}; %x-Achse
\draw[->] (0, -4.7) -- (0, 4.6) node[left]{$y$}; %y-Achse
%Achsenbeschriftung:
\foreach \x in {1, 2, 3} \draw (\x, 0) -- ++(0, -0.1) node[below] {$\x$}; 
\foreach \x in {1, 2, 3} \draw (-\x, 0) -- ++(0, 0.1) node[above] {$-\x$}; 
\foreach \y in {1, 2, 3, 4} \draw (0, \y) -- ++(-0.1, 0) node[left] {$\y$};
\foreach \y in {1, 2, 3, 4} \draw (0, -\y) -- ++(-0.1, 0) node[left] {$-\y$};
\node[below left] at (0, 0) {$0$};
%Funktionsgraph:
\draw[domain=-2:2,samples=200,color=\jccolorfkt, fill=\jccolorfktarea] %
  plot (\x, {1/2*pow(\x,3)}) -- (2, 4) -- (2, 0) -- (-2, 0) -- (-2, -4); 
%Beschriftung:
\draw[color=\jccolorfkt] (1.4, 0.7) -- +(0.4, 0);
\draw[color=\jccolorfkt] (1.6, 0.5) -- +(0, 0.4);
\draw[color=\jccolorfkt] (-1.4, -0.7) -- +(-0.4, 0);
\end{tikzpicture}
\ifttm\else\end{small}\fi
}
\end{center}
%\fi

Man teilt den Graphen von $f$ in zwei Teile zwischen $-2$ und $0$ bzw. $0$ und
$2$ ein und untersucht die Teilflächen, die der Graph in beiden Bereichen 
mit der $x$-Achse einschließt. Man kann die beiden Teilflächen durch eine 
Punktspiegelung ineinander überführen. Beide Teilflächen sind gleich groß. 
Bildet man jeweils die Riemann-Summen, dann stellt man fest, dass Flächen,
die unterhalb der $x$-Achse liegen, im Integral einen negativen Wert annehmen. 
Wenn man also die beiden hier abgebildeten Teilflächen addiert, um das Integral 
über den gesamten Bereich von $-2$ bis $2$ zu berechnen, erhält die Fläche 
über der positiven $x$-Achse einen positiven Wert, während die Fläche unterhalb 
der negativen $x$-Achse gleich groß ist, aber einen negativen Wert annimmt. 
Die Summe der beiden Teilflächen ist also Null. Für ungerade Funktionen $f$ gilt 
die Regel:

$$
\int_{-c}^c f(x) \MDwSp x = 0 \MDFPeriod
$$

Im Fall einer geraden Funktion $g: [-c\MIntvlSep c] \rightarrow \R$ ist der Graph 
symmetrisch bezüglich der $y$-Achse. 
\begin{center}
\MTikzAuto{%
\ifttm\else\begin{small}\fi
\begin{tikzpicture}[scale=0.8,line width=1.5pt]
%\clip(-3.7,-4.7) rectangle (3.7, 4.7);
%Koordinatenachsen:
\draw[->] (-3.6, 0) -- (3.8, 0) node[below left]{$x$}; %x-Achse
\draw[->] (0, -0.7) -- (0, 4.6) node[left]{$y$}; %y-Achse
%Achsenbeschriftung:
\foreach \x in {1, 2, 3} \draw (\x, 0) -- ++(0, -0.1) node[below] {$\x$}; 
\foreach \x in {1, 2, 3} \draw (-\x, 0) -- ++(0, -0.1) node[below] {$\x$};
%
\foreach \y in {1, 2, 3, 4} \draw (0, \y) -- ++(-0.1, 0) node[left] {$\y$};
\node[below left] at (0, 0) {$0$};
%Funktionsgraph:
\draw[domain=-2:2,samples=200,color=\jccolorfkt, fill=\jccolorfktarea] %
  plot (\x, {abs(pow(\x,2))}) -- (2, 4) -- (2, 0) -- (-2, 0) -- (-2, 4); 
%Beschriftung:
\draw[color=\jccolorfkt] (1.4, 0.7) -- +(0.4, 0);
\draw[color=\jccolorfkt] (1.6, 0.5) -- +(0, 0.4);
\draw[color=\jccolorfkt] (-1.4, 0.7) -- +(-0.4, 0);
\draw[color=\jccolorfkt] (-1.6, 0.5) -- +(0, 0.4);
\end{tikzpicture}
\ifttm\else\end{small}\fi
}
\end{center}
Die Fläche zwischen dem Graphen von $g$ und der $x$-Achse ist hier 
symmetrisch bezüglich der $y$-Achse. Die Teilfläche links davon ist also 
das Spiegelbild der rechts liegenden Fläche. Beide zusammen ergeben die 
Gesamtfläche

$$
\int_{-c}^c g(x) \MDwSp x = 2 \cdot \int_{0}^c g(x) \MDwSp x \MDFPeriod
$$

Diese Regel für das Integral gilt für jede integrierbare Funktion $g$, die
gerade ist, auch wenn es negative Funktionswerte gibt. Aufgrund der Rechenregel
genügt es dann, das Integral für nichtnegative $x$-Werte mit der Untergrenze 
$0$ und der Obergrenze $c$ zu berechnen.

In sehr vielen Situationen wird die Berechnung eines Integrals einfacher, wenn 
man den Integranden vor der Integration in eine bekannte Form bringt. 
Beispiele zu möglichen Umformungen sollen im Folgenden betrachtet werden. 
Im ersten Beispiel werden Potenzfunktionen untersucht.

\begin{MExample}
Es soll das Integral
\[
\int_{1}^{4} (x - 2) \cdot \sqrt{x} \MDwSp x %%
\]
berechnet werden. Um die Rechnung zu vereinfachen, formt man den Integranden um:
\[
 (x - 2) \cdot \sqrt{x} = x \sqrt{x} - 2 \sqrt{x} %
 = x^{\frac{3}{2}} - 2 x^{\frac{1}{2}} \MDFPeriod %%
\]
Damit kann man das Integral einfacher lösen:
\begin{eqnarray*}
\int_{1}^{4} (x - 2) \cdot \sqrt{x} \MDwSp x %%
& = &
\int_{1}^{4} \left(x^{\frac{3}{2}} - 2 x^{\frac{1}{2}}\right) \MD x %%
  =  
\left[\frac{2}{5} x^{\frac{5}{2}} - \frac{4}{3} x^{\frac{3}{2}}\right]_{1}^{4} \\
& = &
\left(\frac{2}{5} \left(\sqrt{4}\right)^5 - \frac{4}{3} \left(\sqrt{4}\right)^3 \right) %
 - \left(\frac{2}{5} \cdot 1 - \frac{4}{3} \cdot 1 \right) \\
& = &
\left(\frac{64}{5} - \frac{32}{3}\right) - \left(\frac{2}{5} - \frac{4}{3}\right) \\
& = & \frac{62}{5} - \frac{28}{3} \\
& = & 3 + \frac{1}{15} \MDFPeriod %%
\end{eqnarray*}
\end{MExample}

%Im nächsten Beispiel wird die Umformung eines Integranden mit trigonometrischen 
%Funktionen durchgeführt.
Im nächsten Beispiel wird die Umformung eines Integranden mit 
Exponentialfunktionen durchgeführt.

\begin{MExample}
Es wird das Integral
\[
\int_{-2}^{3} \frac{8 \MEU^{3 + x} - 12 \MEU^{2 x}}{2 \MEU^{x}} \MDwSp x %%
\]
berechnet. Mit den Rechenregeln für die Exponentialfunktion erhält man 
\[
\frac{8 \MEU^{3 + x} - 12 \MEU^{2 x}}{2 \MEU^{x}} %
= \frac{8 \MEU^{3 + x}}{2 \MEU^{x}} - \frac{12 \MEU^{2 x}}{2 \MEU^{x}} %
= 4 \MEU^{3 + x - x} - 6 \MEU^{2 x - x} %
= 4 \MEU^{3} - 6 \MEU^{x} \MDFPSpace, %%
\]
sodass das Integral schließlich auf sehr einfache Art und Weise gelöst werden 
kann:
\begin{eqnarray*}
\int_{-2}^{3} \frac{8 \MEU^{3 + x} - 12 \MEU^{2 x}}{2 \MEU^{x}} \MDwSp x %
= 
\int_{-2}^{3} \left( 4 \MEU^{3} - 6 \MEU^{x} \right) \MD x %
& = & 
\left[4 \MEU^{3} \cdot x - 6 \MEU^{x} \right]_{-2}^{3} \\
& = & 
\left(4 \MEU^{3} \cdot 3 - 6 \MEU^{3} \right) %
 - \left(4 \MEU^{3} \cdot (-2) - 6 \MEU^{-2} \right) \\
& = &
14 \MEU^{3} + \frac{6}{\MEU^{2}} \MDFPeriod %%
\end{eqnarray*}
\end{MExample}

Ist bei einer rationalen Funktion der Grad des Zählerpolynoms größer oder gleich 
dem Grad des Nennerpolynoms, dann führt man zunächst eine Polynomdivision 
(siehe Modul \MRef{VBKM06}) durch. Je nach Situation können sich auch noch 
weitere Umformungen (z.B. Partialbruchzerlegung) anbieten, die man in der
weiterführenden Literatur und Formelsammlungen finden kann. Im folgenden Beispiel
wird eine Polynomdivision durchgeführt, um eine rationale Funktion zu integrieren.

\begin{MExample}
Es wird das Integral
\[
\int_{-1}^{1} \frac{4 x^2 - x + 4}{x^2 + 1} \MDwSp x %%
\]
berechnet.
Dazu formt man zunächst den Integranden mittels Polynomdivision 
\[
4 x^2 - x + 4 = (x^2 + 1) \cdot 4 - x %%
\]
zu
\[
\frac{4 x^2 - x + 4}{x^2 + 1} = 4 - \frac{x}{x^2 + 1} %%
\]
um. Damit ist dann
\begin{eqnarray*}
\int_{-1}^{1} \frac{4 x^2 - x + 4}{x^2 + 1} \MDwSp x %%
& = & 
\int_{-1}^{1} \left(4 - \frac{x}{x^2 + 1}\right) \MD x \\
& = & 
\int_{-1}^{1} 4 \MDwSp x - \int_{-1}^{1} \frac{x}{x^2 + 1} \MDwSp x %%
= \left[4 x\right]_{-1}^{1} - 0 %
= 8 \MDFPeriod
\end{eqnarray*}
Denn der Integrand des zweiten Integrals ist eine ungerade Funktion und im 
Integrationsintervall $[-1\MIntvlSep 1]$ punktsymmetrisch, sodass der Wert
des zweiten Integrals Null ist.

Hier wurde eine besondere Situation gewählt, um einen ersten Eindruck zur
Integration rationaler Funktionen zu vermitteln. In weiterführenden 
mathematischen Vorlesungen oder in der Literatur wird dies allgemein beschrieben.
\end{MExample}

\end{MXContent}


%%%Uebungen zum Abschnitt zum bestimmten Integral:
\begin{MExercises}
\MDeclareSiteUXID{VBKM08_BestimmtesIntegral_Exercises}

%Einfuehrende Erlaeuterung zur ersten Aufgabe erstellt (jgl).
In der ersten Aufgabe wird die Idee zur Definition des Integrals aufgegriffen,
mittels geeigneter Zerlegungen den Integralwert zu berechnen, wobei in der 
Aufgabe neben Rechtecken allgemeiner beispielsweise auch Dreiecksflächen 
verwendet werden.

\begin{MExercise}
%Aufgabentext ueberarbeitet (jgl).
Berechnen Sie zu $f:[-3\MIntvlSep 4] \rightarrow \R$ mit dem unten dargestellten 
Graphen das Integral $\displaystyle \int_{-3}^{4} f(x) \MDwSp x$ mit Methoden 
aus der elementaren Geometrie, indem Sie die entsprechende Fläche 
{\glqq}unter dem Graphen der Funktion{\grqq} in elementare geometrische Flächen
wie Dreiecke oder Rechtecke zerlegen, die entweder oberhalb oder unterhalb 
der $x$-Achse liegen.
Die einzelnen Flächeninhalte können Sie in dieser Situation dann mit 
Formeln für Dreiecke oder Rechtecke berechnen.

\begin{center}
\MTikzAuto{%
\ifttm\else\begin{small}\fi
\begin{tikzpicture}[line width=1.5pt,scale=0.8, %
declare function={
  u0 = -3;
  u1 = 0;
  u2 = 2;
  u3 = 4;
  fkt1(\x) = \x + 1; % $-3 \leq x \leq 0$
  fkt2(\x) = -\x + 1 - abs(-\x + 1) + 1; % $0 \leq x \leq 2$
  fkt3(\x) = (\x - 3 + abs(\x - 3))/2 - 1; % $2 \leq x \leq 4$
%Auch erlaubt:
%  fkt3(\z) = (\z - 3 + abs(\z - 3))/2 - 1; % $2 \leq x \leq 4$
}
] %[every node/.style={fill=white}] 
%Koordinatenachsen:
\draw[->] ({u0-1}, 0) -- ({u3+1}, 0) node[below left]{$x$}; %x-Achse
%Berechnung der Werte: Minimalstelle ist u0, Maximalstelle ist u1:
\draw[->] (0, {fkt1(u0)-1}) -- (0, {fkt1(u1)+1}) node[below left]{$y$}; %y-Achse
%Achsenbeschriftung:
\foreach \x in {-3, -2, -1, 1, 2, 3, 4} \draw (\x, 0) -- ++(0, -0.1) %
 node[below] {$\x$}; 
\foreach \y in {-2, -1, 1} \draw (0, \y) -- ++(-0.1, 0) %
 node[left] {$\y$};
\node[below left] at (0, 0) {$0$};
%Funktion:
\draw[domain=u0:u1,samples=120,color=\jccolorfkt] %
 plot (\x, {fkt1(\x)});
\draw[domain=u1:u2,samples=120,color=\jccolorfkt] %
 plot (\x, {fkt2(\x)});
\draw[domain=u2:u3,samples=120,color=\jccolorfkt] %
 plot (\x, {fkt3(\x)});
%Beschriftung:
\node[right] at (1.1, {fkt2(1)}) {$f(x)$};
%
\end{tikzpicture}
\ifttm\else\end{small}\fi
}
\end{center}
Der Integralwert ergibt sich dann als Summe der Teilflächen, die oberhalb der 
$x$-Achse liegen, abzüglich der Summe der Teilflächen, die unterhalb der 
$x$-Achse liegen. In diesem Sinne kann man den Integralwert als die Summe
vorzeichenbehafteter Flächenwerte verstehen.

Der Integralwert von
$\displaystyle \int_{-3}^{4} f(x) \MDwSp x$
ist \MLParsedQuestion{10}{-2}{4}{IG23}.

\begin{MHint}{Lösung}
Die Fläche wird mit Geraden durch $x_0 = -3$, $x_1 = -1$, $x_2 = 0$, $x_3 = 1$,
$x_4 = \frac{3}{2}$, $x_5 = 2$, $x_6 = 3$ und $x_7 = 4$ in Teilflächen zerlegt,
die jeweils durch den Graphen von $f$, die $x$-Achse und durch die Geraden 
$x_{k-1}$ und $x_k$ für $1 \leq k \leq 7$ begrenzt werden.

Es werden die zugehörigen vorzeichenbehafteten Flächeninhalte aufsummiert, 
sodass sich der Integralwert ergibt:
\[
\int_{-3}^{4} f(x) \MDwSp x %
= -\frac{2 \cdot 2}{2} + \frac{1 \cdot 1}{2} + 1 \cdot 1 %
 + \frac{\frac{1}{2} \cdot 1}{2}
 - \frac{\frac{1}{2} \cdot 1}{2}
 - 1 \cdot 1 - \frac{1 \cdot 1}{2} %
= -2\MDFPeriod %%
\]
Die Fläche kann auch in andere Teilflächen zerlegt werden. Wird die Fläche
beispielsweise durch $z_0 = -3$, $z_1 = -1$, $z_2 = \frac{3}{2}$ und $z_3 = 4$ 
in drei Teilflächen zerlegt, hat die Teilfläche $B_2$ zwischen $z_1$ und $z_2$ 
denselben Flächeninhalt wie die Teilfläche $B_3$ zwischen $z_2$ und $z_3$, 
jedoch erhalten die Werte verschiedenes Vorzeichen, da $B_2$ oberhalb der 
$x$-Achse liegt und $B_3$ unterhalb. 
\begin{center}
\MTikzAuto{%
\ifttm\else\begin{small}\fi
\begin{tikzpicture}[line width=1.5pt,scale=0.8, %
declare function={
  u0 = -3;
  u1 = 0;
  u2 = 2;
  u3 = 4;
  fkt1(\x) = \x + 1; % $-3 \leq x \leq 0$
  fkt2(\x) = -\x + 1 - abs(-\x + 1) + 1; % $0 \leq x \leq 2$
  fkt3(\x) = (\x - 3 + abs(\x - 3))/2 - 1; % $2 \leq x \leq 4$
%Auch erlaubt:
%  fkt3(\z) = (\z - 3 + abs(\z - 3))/2 - 1; % $2 \leq x \leq 4$
}
] %[every node/.style={fill=white}] 
%Koordinatenachsen:
\draw[->] ({u0-1}, 0) -- ({u3+1}, 0) node[below left]{$x$}; %x-Achse
%Berechnung der Werte: Minimalstelle ist u0, Maximalstelle ist u1:
\draw[->] (0, {fkt1(u0)-1}) -- (0, {fkt1(u1)+1}) node[below left]{$y$}; %y-Achse
%Achsenbeschriftung:
\foreach \x in {-3, -2, -1, 1, 2, 3, 4} \draw (\x, 0) -- ++(0, -0.1);
\node[above] at (-3, 0) {$-3$}; 
\node[below] at (1, 0) {$1$}; 
\node[below] at (4, 0) {$4$}; 
\foreach \y in {-2, -1, 1} \draw (0, \y) -- ++(-0.1, 0) %
 node[left] {$y$};
\node[below left] at (0, 0) {$0$};
%Begrenzung:
\draw[style=dotted] (-3,-2) -- (-3,0);
%Funktion:
\draw[domain=u0:u1,samples=120,color=\jccolorfkt] %
 plot (\x, {fkt1(\x)});
\draw[domain=u1:u2,samples=120,color=\jccolorfkt] %
 plot (\x, {fkt2(\x)});
\draw[domain=u2:u3,samples=120,color=\jccolorfkt] %
 plot (\x, {fkt3(\x)});
%Beschriftung:
\node[right] at (1.1, {fkt2(1)}) {$f(x)$};
\node at (-2.5, -0.75) {$B_1$};
\node at (0.5, 0.5) {$B_2$};
\node at (2.6, -0.5) {$B_3$};
%
\end{tikzpicture}
\ifttm\else\end{small}\fi
}
\end{center}
Somit ist der Integralwert gleich dem negativen Flächeninhalt der Fläche 
$B_1$ zwischen $z_0$ und $z_1$, die unterhalb der $x$-Achse liegt.
\end{MHint}
\end{MExercise}


\begin{MExercise}
Berechnen Sie die folgenden Integrale:
\begin{MExerciseItems}
\item{\MEquationItem{$\displaystyle \int_{0}^{5} 3 \MDwSp x$}{\MLParsedQuestion{10}{15}{4}{IG24}},}
\item{\MEquationItem{$\displaystyle \int_{0}^{5} -4 \MDwSp x$}{\MLParsedQuestion{10}{-20}{4}{IG25}},}
\item{\MEquationItem{$\displaystyle \int_{0}^{4} 2 x \MDwSp x$}{\MLParsedQuestion{10}{16}{4}{IG26}},}
\item{\MEquationItem{$\displaystyle \int_{0}^{4} \left(4 - x\right) \MD x$}{\MLParsedQuestion{10}{8}{4}{IG27}}.}
\end{MExerciseItems}
%Loesungshinweise erstellt (jgl):
\begin{MHint}{Lösung}
Mit dem Hauptsatz der Differential- und Integralrechnung erhält man
\begin{enumerate}
\item $\displaystyle \int_{0}^{5} 3 \MDwSp x %
= \left[ 3 x \right]_0^5 = 15\MDFPSpace$,
%
\item $\displaystyle \int_{0}^{5} -4 \MDwSp x %
= \left[ -4 x \right]_0^5 = -20\MDFPSpace$,
%
\item $\displaystyle \int_{0}^{4} 2 x \MDwSp x %
= \left[ x^2 \right]_0^4 = 16\MDFPSpace$,
%
\item $\displaystyle \int_{0}^{4} \left(4 - x\right) \MD x %
= \left[ 4 x - \frac{1}{2} \cdot x^2 \right]_0^4 = 8\MDFPSpace$.
%
\end{enumerate}
\end{MHint}
\end{MExercise}


\begin{MExercise}
Der Wert des Integrals
$\displaystyle \int_{-\pi}^{\pi} \left(5 x^3 - 4 \sin(x)\right) \MD x$
ist \MLParsedQuestion{10}{0}{4}{IG28}.
%Loesungshinweise erstellt (jgl):
\begin{MHint}{Lösung}
Der Integrand ist ungerade und das Integrationsintervall ist symmetrisch 
bezüglich $0$, sodass das Integral den Wert $0$ hat.
Oder es wird das Integral mit dem Hauptsatz berechnet:
\[
\int_{-\pi}^{\pi} \left(5 x^3 - 4 \sin(x)\right) \MD x %
= \left[ \frac{5}{4} \cdot x^4 + 4 \cos(x) \right]_{-\pi}^{\pi} %
= \left[ \frac{5}{4} \cdot x^4 + 4 \cos(x) \right]_{-\pi}^{\pi} %
= 0\MDFPeriod %%
\]
\end{MHint}
\end{MExercise}


\begin{MExercise}
Berechnen Sie eine reelle Zahl $z$ so, dass der Integralwert
\[
\int_{0}^{2} \left(x^2 + z\cdot x+1\right) \MD x %
\]
den Wert $0$ ergibt: %
Der Wert für $z$ 
ist \MEquationItem{$z$}{\MLParsedQuestion{10}{-7/3}{4}{ICONSTFIND}}.

\begin{MHint}{\iSolution}
Nimmt man $z$ als unbekannte Konstante, so ist
$$
\int_{0}^{2} \left(x^2 + z\cdot x+1\right) \MD x %
\;=\; \left[{\frac13x^3+\frac12 z x^2+x}\right]_0^2 %
\;=\; \frac83+2z+2 \MDFPeriod %%
$$
Also ist $z=-\frac{14}{6} = -\frac{7}{3}$ der gesuchte Wert.
\end{MHint}
\end{MExercise}

\begin{MExercise}
Berechnen Sie die Integrale:
\begin{MExerciseItems}
\item{\MEquationItem{$\displaystyle \int_{-3}^{2} \left(1 + 6 x^2 - 4 x\right) \MD x$}{\MLParsedQuestion{30}{85}{4}{IG29}},}
\item{\MEquationItem{$\displaystyle \int_{1}^{9} \frac{5}{\sqrt{4 x}} \MDwSp x$}{\MLParsedQuestion{30}{10}{4}{IG30}}.}
\end{MExerciseItems}

\begin{MHint}{\iSolution}
Der Integrand $f$ mit $f(x) = 1 + 6 x^2 - 4x = 6 x^2 - 4 x + 1$ ist ein Polynom,
sodass $F$ mit $F(x) = 2 x^3 - 2 x^2 + x$ eine Stammfunktion ist. Mit dem 
Hauptsatz folgt
\[
 \int_{0}^{1} \left(1 + 6 x^2 - 4 x\right) \MD x %
 = \left[ 2 x^3 - 2 x^2 + x \right]_{-3}^{2} %
 = 2 (8 - 4) + 2 - (2 (-27 - 9) - 3) = 85 \MDFPeriod %%
\]
Beim zweiten Aufgabenteil ist
der Integrand $f(x) = \frac{5}{\sqrt{4 x}}= \frac{5}{2} x^{-1/2}$ ein 
Produkt einer Wurzelfunktion mit einem konstanten Faktor. Damit ist $F$ mit
$F(x) = 5 x^{1/2} = 5 \sqrt{x}$ eine Stammfunktion von $f$. Mit dem Hauptsatz 
folgt
\[
 \int_{1}^{9} \frac{5}{\sqrt{4 x}} \MDwSp x %
 = \left[ 5 \sqrt{x} \right]_{1}^{9} %
 = 5 (3 - 1) = 10 \MDFPeriod %%
\]
\end{MHint}
\end{MExercise}

\begin{MExercise}
Der Wert des Integrals
$\displaystyle \int_{-24}^{-6} \frac{1}{2 x} \MDwSp x$
ist \MLParsedQuestion{10}{-ln(2)}{4}{ILN1}.

\begin{MHint}{\iSolution}
Zum Integranden $f$ mit $f(x) = \frac{1}{2 x} = \frac{1}{2} \cdot \frac{1}{x}$ 
für $x < 0$ ist $F$ mit $F(x) = \frac{1}{2} \ln|x|$ eine Stammfunktion.
Mit dem Hauptsatz folgt
\[
 \int_{-24}^{-6} \frac{1}{2 x} \MDwSp x %
 = \left[ \frac{1}{2} \ln|x| \right]_{-24}^{-6} %
 = \frac{1}{2} \left(\ln|-6| - \ln|-24|\right) %
 = \frac{1}{2} \ln\left(\frac{6}{24}\right) %
 = \frac{1}{2} \ln\left(2^{-2}\right) %
 = -\ln(2) \MDFPeriod %
\]
\end{MHint}
\end{MExercise}

\begin{MExercise}
Berechnen Sie die Integrale
\begin{MExerciseItems}
\item{\MEquationItem{$\displaystyle \int_{0}^{3} (2 x - 1) \MDwSp x$}{\MLParsedQuestion{30}{6}{4}{IG31}},}
\item{\MEquationItem{$\displaystyle \int_{-3}^{0} (1 - 2 x) \MDwSp x$}{\MLParsedQuestion{30}{12}{4}{IG32}}.}
\end{MExerciseItems}

\begin{MHint}{\iSolution}
Der Integrand $f$ mit $f(x) = 2 x - 1$ ist eine Polynomfunktion. Damit ist
$F$ mit $F(x) = x^2 - x$ eine Stammfunktion von $f$. Mit dem Hauptsatz 
folgt
\[
 \int_{0}^{3} (2 x - 1) \MDwSp x %
 = \left[ x^2 - x \right]_{0}^{3} %
 = 9 - 3 - 0%
 = 6 \MDFPeriod %
\]
Im zweiten Aufgabenteil ist der Integrand $f$ mit $f(x) = 1 - 2 x$ ebenfalls 
eine Polynomfunktion. Damit ist $F$ mit $F(x) = x - x^2$ eine Stammfunktion 
von $f$. Mit dem Hauptsatz folgt
\[
 \int_{-3}^{0} (1 - 2 x) \MDwSp x %
 = \left[ x - x^2 \right]_{-3}^{0} %
 = 0 - (-3 - 9)%
 = 12 \MDFPeriod %
\]
\end{MHint}
\end{MExercise}

\begin{MExercise}
Berechnen Sie das Integral

\MEquationItem{$\displaystyle \int_{\pi}^{3 \pi} \left(\frac{3 \pi}{x^2} - 4 \sin(x)\right) \MD x$}{\MLParsedQuestion{30}{2}{4}{IG33}}.
% \item{\MEquationItem{$\displaystyle \int_{-3}^{3} \left(|x| - |x + 1|\right) \MD x$}{\MLParsedQuestion{30}{-1}{4}{IG34}}.}

\begin{MHint}{\iSolution}
Zum Integranden $f$ mit $f(x) = \frac{3 \pi}{x^2} - 4 \sin(x)$ ist $F$ mit
$F(x) = -\frac{3 \pi}{x} + 4 \cos(x)$ eine Stammfunktion. Mit dem 
Hauptsatz folgt
\[
 \int_{\pi}^{3 \pi} \left(\frac{3 \pi}{x^2} - 4 \sin(x)\right) \MD x %
 = \left[ -\frac{3 \pi}{x} + 4 \cos(x) \right]_{\pi}^{3 \pi} %
 = ( -1 - 4 ) - ( -3 - 4 )%
 = 2 \MDFPeriod %
\]
Anmerkung: Die periodischen Funktionen $\sin$ und $\cos$ mit Periode $2 \pi$ 
haben die Eigenschaft, dass das Integral über eine Periode gleich Null ist. 
Für andere periodische Funktionen wie zum Beispiel $f$ mit 
$f(x) = \sin(x) + \frac{1}{2}$ können sich für das Integral auch Werte 
ungleich Null ergeben, wenn das Integrationsintervall eine Periode der 
Funktion umfasst.
\end{MHint}
\end{MExercise}

\end{MExercises}




%%%Abschnitt
\MSubsection{Anwendungen}\MLabel{M08A_Anwendung}

\begin{MIntro}
\MDeclareSiteUXID{VBKM08_Anwendungen_Intro}
Die Integralrechnung findet sehr viele Anwendungen, insbesondere in 
naturwissenschaftlich-technischen Bereichen. Beispielhaft soll hier zunächst 
die Berechnung von Flächen erörtert werden, deren Ränder durch mathematische 
Funktionen beschrieben werden können. Auch dies ist keine rein mathematische 
Anwendung, sondern findet Einsatzmöglichkeiten bei der Bestimmung von 
Schwerpunkten, von Rotationseigenschaften starrer Körper oder den 
Biegeeigenschaften von Balken oder Stahlträgern. Zum Abschluss werden einige 
weitere physikalisch-technische Anwendungen betrachtet.
\end{MIntro}

\begin{MXContent}{Flächenberechnung}{Flächenberechnung}{STD}
\MDeclareSiteUXID{VBKM08_Anwendungen_Flaechenberechnung}
Eine erste Anwendung der Integrationsrechnung ist die Berechnung von 
\MEntry{Flächeninhalten}{Flächeninhalt (Integral)}, deren Ränder von 
mathematischen Funktionen beschrieben werden können.
Zur Veranschaulichung ist in der folgenden Abbildung (linkes Bild) die 
Funktion $f(x) = \frac{1}{2} x^3$ auf dem Intervall $[-2\MIntvlSep 2]$ 
dargestellt. Das Ziel ist die Berechnung des Flächeninhalts, der vom Graphen 
der Funktion und der $x$-Achse eingeschlossen wird. Die bisherigen 
Untersuchungen ergeben, dass das Integral über diese ungerade Funktion in 
den Grenzen von $-2$ bis $2$ genau Null ergeben wird, da die linke und rechte 
Teilfläche gleich groß sind, aber bei der Integration unterschiedliche 
Vorzeichen erhalten. Das Integral entspricht hier also nicht dem Wert des 
Flächeninhalts.
Spiegelt man jedoch die "`negative"' Fläche an der $x$-Achse, gibt man der 
Funktion also ein positives Vorzeichen (rechtes Bild), dann kann man den 
Flächeninhalt richtig bestimmen. Mathematisch bedeutet das, dass man nicht 
das Integral der Funktion $f$ berechnet, sondern das Integral des Betrags 
$\left|f\right|$.

\begin{center}
\MTikzAuto{%
\ifttm\else\begin{small}\fi
\begin{tikzpicture}[scale=0.8,line width=1.5pt]
\begin{scope}[xshift=-8.5cm]
%\clip(-3.7,-4.7) rectangle (3.7, 4.7);
%Koordinatenachsen:
\draw[->] (-3.6, 0) -- (3.8, 0) node[below left]{$x$}; %x-Achse
\draw[->] (0, -4.6) -- (0, 4.8) node[below left]{$y$}; %y-Achse
%Achsenbeschriftung:
\foreach \x in {1, 2, 3} \draw (\x, 0) -- ++(0, -0.1) node[below] {$\x$}; 
\foreach \x in {1, 2, 3} \draw (-\x, 0) -- ++(0, 0.1) node[above] {$-\x$}; 
\foreach \y in {1, 2, 3, 4} \draw (0, \y) -- ++(-0.1, 0) node[left] {$\y$};
\foreach \y in {1, 2, 3, 4} \draw (0, -\y) -- ++(-0.1, 0) node[left] {$-\y$};
\node[below left] at (0, 0) {$0$};
%Funktionsgraph:
\draw[domain=-2:2,samples=120,color=\jccolorfkt, fill=\jccolorfktarea] %
  plot (\x, {1/2*pow(\x,3)}) -- (2, 4) -- (2, 0) -- (-2, 0) -- (-2, -4); 
%Beschriftung:
\draw[color=\jccolorfkt] (1.4, 0.7) -- +(0.4, 0);
\draw[color=\jccolorfkt] (1.6, 0.5) -- +(0, 0.4);
\draw[color=\jccolorfkt] (-1.4, -0.7) -- +(-0.4, 0);
\end{scope}
\begin{scope}{xshift=8.5cm}
%Quelle: BildFlaechePolynomBetragxdrei.tex (Modul Integrationstechniken, Liedtke)
%\clip(-3.7,-4.7) rectangle (3.7, 4.7);
%Koordinatenachsen:
\draw[->] (-3.6, 0) -- (3.8, 0) node[below left]{$x$}; %x-Achse
\draw[->] (0, -4.6) -- (0, 4.8) node[below left]{$y$}; %y-Achse
%Achsenbeschriftung:
\foreach \x in {1, 2, 3} \draw (\x, 0) -- ++(0, -0.1) node[below] {$\x$}; 
\foreach \x in {1, 2, 3} \draw (-\x, 0) -- ++(0, -0.1);
\node[below] at (-3, -0.1) {$-3$};
%\draw (-1, 0) -- ++(0, -0.1) node[below,fill=white] {$-1$}; 
\foreach \y in {1, 2, 3, 4} \draw (0, \y) -- ++(-0.1, 0) node[left] {$\y$};
\foreach \y in {1, 2, 3, 4} \draw (0, -\y) -- ++(-0.1, 0) node[left] {$-\y$};
\node[below left] at (0, 0) {$0$};
%Funktionsgraph:
\draw[domain=-2:2,samples=200,color=\jccolorfkt, fill=\jccolorfktarea] %
  plot (\x, {1/2*abs(pow(\x,3))}) -- (2, 4) -- (2, 0) -- (-2, 0) -- (-2, 4); 
\draw[domain=-2:0,samples=120,color=green!50!black,dashed,fill=\jccolorfktareahell] %
 (-2, 0) -- (-2, -4) -- plot (\x, {1/2*pow(\x,3)}) ; 
%Beschriftung:
\draw[color=\jccolorfkt] (1.4, 0.7) -- +(0.4, 0);
\draw[color=\jccolorfkt] (1.6, 0.5) -- +(0, 0.4);
\draw[color=\jccolorfkt] (-1.4, 0.7) -- +(-0.4, 0);
\draw[color=\jccolorfkt] (-1.6, 0.5) -- +(0, 0.4);
\draw[color=green!75!white] (-1.4, -0.7) -- +(-0.4, 0);
\end{scope}
\end{tikzpicture}
\ifttm\else\end{small}\fi
}
\end{center}

Durch die Bildung des Betrags der Funktion benötigt man eine Aufteilung des 
Integrals in die Bereiche mit positivem und negativem Vorzeichen.
Für die Berechnung heißt dies, dass man das Integrationsintervall in 
Abschnitte zu unterteilt, in denen die Funktionswerte dasselbe Vorzeichen haben.
Für stetige Funktionen ergeben sich diese durch die Nullstellen der Funktion.

\begin{MXInfo}{Flächenberechnung} 
Gegeben ist eine stetige Funktionen $f: [a\MIntvlSep  b] \rightarrow \R$ auf 
einem Intervall $[a\MIntvlSep  b]$. 
Weiter seien $x_1$ bis $x_m$ die Nullstellen von $f$ mit 
$x_1 < x_2 < \ldots < x_m$. 
Es werden $x_0 := a$ und $x_{m+1} := b$ gesetzt.

Dann ist der Flächeninhalt zwischen dem Graphen von $f$ und der $x$-Achse gleich
\[
\int_{a}^{b} |f(x)| \MDwSp x %
= \sum_{k=0}^{m} \left|\int_{x_k}^{x_{k+1}} f(x) \MDwSp x\right| \MDFPeriod %% 
\]
\end{MXInfo}

Dies soll für das oben dargestellte Beispiel etwas genauer erörtert werden.

\begin{MExample}
Zu berechnen ist der Flächeninhalt $I_A$ der Fläche, den die stetige Funktion
$f$ mit $f(x) = \frac{1}{2} x^3$ im Bereich $[-2\MIntvlSep 2]$ mit der $x$-Ache 
einschließt. Die einzige Nullstelle der gegebenen Funktion befindet sich bei 
$x_0 = 0$. Man teilt den Integrationsbereich also in die beiden Teilintervalle
$[-2\MIntvlSep 0]$ und $[0\MIntvlSep 2]$ auf und berechnet mit
%
\begin{eqnarray*}
I_A  =  \int_{-2}^{2} \left|\frac{1}{2} x^3\right| \MD x %
 & = & \left|\int_{-2}^{0} \frac{1}{2} x^3 \MDwSp x\right| + \left|\int_{0}^{2} \frac{1}{2} x^3 \MDwSp x\right| \\
 & = & \left|\left[\frac{1}{8} x^4\right]_{-2}^{0}\right| + \left|\left[\frac{1}{8} x^4\right]_{0}^{2}\right| \\
  & = & \left|0 - 2\right| + \left|2 - 0\right| \\
  & = & 4
\end{eqnarray*}
%
den Flächeninhalt zwischen Kurve und $x$-Achse zu $I_A = 4$.
\end{MExample}

Man kann nicht nur Flächeninhalte zwischen einer Kurve und der $x$-Achse 
bestimmen, sondern auch den Inhalt einer Fläche, die von zwei Kurven 
eingeschlossen wird, wie in der folgenden Abbildung veranschaulicht wird.
Dort zeigt das rechte Bild die gesuchte Fläche, deren Flächeninhalt als
die Differenz der Flächeninhalte aus dem linken und dem mittleren Bild 
berechnet wird.

\begin{center}
\MTikzAuto{%
%{Beispiel einer Fläche zwischen dem grün gezeichneten Graphen 
%von $f$ und dem rot gezeichneten Graphen von $g$: Zur Berechnung des 
%Flächeninhalts wird die Differenz der Funktionen $f - g$ betrachtet.}{0.5}
\ifttm\else\begin{small}\fi
\begin{tikzpicture}[scale=0.6,line width=1.5pt,
 declare function={
  Fktf(\x) = -1/2 * (\x - 4) * (\x - 4) + 7/2; 
%  Fktg(\x) = 1/2 * (\x - 5) + 3; %  Fktg(\x) = 1/2 * \x + 1/2;
  Fktg(\x) = 1/2 * (\x + 1);
}
] 
\begin{scope}[xshift=-8.5cm]
%\clip(-0.7,-0.7) rectangle (6.2, 4.4);
%Koordinatenachsen:
\draw[->] (-0.6, 0) -- (6.1, 0) node[below left]{$x$}; %x-Achse
\draw[->] (0, -0.6) -- (0, 5.1) node[below left]{$y$}; %y-Achse
%Achsenbeschriftung:
\foreach \x in {1, 2, 3, 4, 5} \draw (\x, 0) -- ++(0, -0.1)%
 node[below] {$\x$}; 
\foreach \y in {1, 2, 3, 4} \draw (0, \y) -- ++(-0.1, 0) node[left] {$\y$};
\node[below left] at (0, 0) {$0$};
%Fläche unter $f$:
\draw[domain=2:5,samples=200,color=\jccolorfkt,fill=\jccolorfktarea] %
 plot (\x, {Fktf(\x)}) -- (5,0) -- (2,0) -- (2, {Fktf(2)}); 
%Fläche unter $g$:
%\draw[domain=2:5,samples=2,color=red!50!black,fill=red!50!white] %
% plot (\x, {Fktg(\x)}) -- (5,0) -- (2,0) -- (2, {Fktf(2)}); 
%Funktionsgraphen:
\draw[domain=2:5,samples=200,color=\jccolorfkt] %
 plot (\x, {Fktf(\x)}); 
\draw[domain=2:5,samples=2,color=red!50!black] %
 plot (\x, {Fktg(\x)}); 
%Schnittpunkte:
\draw[color=black] (2.0, 1.5) circle[radius=1pt];
\draw[color=black] (5.0, 3.0) circle[radius=1pt];
\end{scope}
%
\begin{scope}[xshift=0cm]
%\clip(-0.7,-0.7) rectangle (6.2, 4.4);
%Koordinatenachsen:
\draw[->] (-0.6, 0) -- (6.1, 0) node[below left]{$x$}; %x-Achse
\draw[->] (0, -0.6) -- (0, 5.1) node[below left]{$y$}; %y-Achse
%Achsenbeschriftung:
\foreach \x in {1, 2, 3, 4, 5} \draw (\x, 0) -- ++(0, -0.1)%
 node[below] {$\x$}; 
\foreach \y in {1, 2, 3, 4} \draw (0, \y) -- ++(-0.1, 0) node[left] {$\y$};
\node[below left] at (0, 0) {$0$};
%Fläche unter $f$:
%\draw[domain=2:5,samples=200,color=green!50!black,fill=green!50!white] %
% plot (\x, {Fktf(\x)}) -- (5,0) -- (2,0) -- (2, {Fktf(2)}); 
%Fläche unter $g$:
\draw[domain=2:5,samples=2,color=red!50!black,fill=red!50!white] %
 plot (\x, {Fktg(\x)}) -- (5,0) -- (2,0) -- (2, {Fktf(2)}); 
%Funktionsgraphen:
\draw[domain=2:5,samples=200,color=\jccolorfkt] %
 plot (\x, {Fktf(\x)}); 
\draw[domain=2:5,samples=2,color=red!50!black] %
 plot (\x, {Fktg(\x)}); 
%Schnittpunkte:
\draw[color=black] (2.0, 1.5) circle[radius=1pt];
\draw[color=black] (5.0, 3.0) circle[radius=1pt];
\end{scope}
%
\begin{scope}[xshift=8.5cm]
%\clip(-0.7,-0.7) rectangle (6.2, 4.4);
%Koordinatenachsen:
\draw[->] (-0.6, 0) -- (6.1, 0) node[below left]{$x$}; %x-Achse
\draw[->] (0, -0.6) -- (0, 5.1) node[below left]{$y$}; %y-Achse
%Achsenbeschriftung:
\foreach \x in {1, 2, 3, 4, 5} \draw (\x, 0) -- ++(0, -0.1)%
 node[below] {$\x$}; 
\foreach \y in {1, 2, 3, 4} \draw (0, \y) -- ++(-0.1, 0) node[left] {$\y$};
\node[below left] at (0, 0) {$0$};
%Fläche zwischen $f$ und $g$:
\draw[domain=2:5,samples=200,color=\jccolorfkt,fill=\jccolorfktarea] %
 plot (\x, {Fktf(\x)}) -- (5, {Fktg(5)}) -- (2, {Fktg(2)}); 
%Fläche unter $g$:
%\draw[domain=2:5,samples=2,color=red!50!black,fill=green!25!white]%
% plot (\x, {Fktg(\x)}) -- (5,0) -- (2,0) -- (2, {Fktf(2)}); 
\draw[domain=2:5,samples=2,color=red!50!black,dashed,% dotted,%
pattern color=red!50!black,pattern=crosshatch dots] %
 plot (\x, {Fktg(\x)}) -- (5,0) -- (2,0) -- (2, {Fktf(2)}); 
%Funktionsgraphen:
\draw[domain=2:5,samples=200,color=\jccolorfkt] %
 plot (\x, {Fktf(\x)}); 
\draw[domain=2:5,samples=2,color=red!50!black] %
 plot (\x, {Fktg(\x)}); 
%Schnittpunkte:
\draw[color=black] (2.0, 1.5) circle[radius=1pt];
\draw[color=black] (5.0, 3.0) circle[radius=1pt];
\end{scope}
\end{tikzpicture}
\ifttm\else\end{small}\fi
}
\end{center}

Dieses Prinzip soll ebenfalls zuerst formal vorgestellt und danach an einem 
Beispiel erläutert werden.

\begin{MXInfo}{Flächenberechnung zwischen den Graphen zweier Funktionen} 
Gegeben sind zwei stetige Funktionen $f, g: [a\MIntvlSep  b] \rightarrow \R$ 
auf einem Intervall $[a\MIntvlSep  b]$. Weiter seien $x_1$ bis $x_m$ die 
Nullstellen von $f - g$ mit $x_1 < x_2 < \ldots < x_m$. Es werden $x_0 := a$ 
und $x_{m+1} := b$ gesetzt.

Dann kann der Flächeninhalt zwischen dem Graphen von $f$ und dem von $g$ durch
%
\[
\int_{a}^{b} |f(x) - g(x)| \MDwSp x %
= \sum_{k=0}^{m} \left|\int_{x_k}^{x_{k+1}} (f(x) - g(x)) \MDwSp x\right| %% 
\]
%
berechnet werden.
\end{MXInfo}

Dazu soll nun ein Beispiel betrachtet werden.

\begin{MExample}
Berechnet werden soll der Inhalt $I_A$ der Fläche zwischen den Graphen von 
$f$ und $g$ mit $f(x) = x^2$ und $g(x) = 8 - \frac{1}{4} x^4$ für 
$x \in [-2\MIntvlSep  2]$.
Zunächst untersucht man die Differenz $f - g$ der Funktionen auf
ihre Nullstellen. Mit
%
\begin{eqnarray*}
f(x) - g(x) & = & \frac{1}{4} x^4 + x^2 - 8 \\
 & = & \frac{1}{4} \left( x^4 + 4 x^2 - 32 \right) \\
 & = & \frac{1}{4} \left( x^4 + 4 x^2 + 2^2 - 2^2 - 32 \right) \\
 & = & \frac{1}{4} \left( \left( x^2 + 2 \right)^2 - 36 \right) %%
\end{eqnarray*}
%
kann man die reellen Nullstellen von $f - g$ berechnen:
\begin{eqnarray*}
& \left( x^2 + 2 \right)^2 - 36 = 0 \\
\Leftrightarrow & \left( x^2 + 2 \right)^2 = 36 \\
\Leftrightarrow & x^2 + 2 = 6 \\
\Leftrightarrow & x^2 = 4 \\
\Leftrightarrow & x = \pm 2  %%
\end{eqnarray*}
oder mit der dritten binomischen Formel:
\begin{eqnarray*}
 0 & = & \left( x^2 + 2 \right)^2 - 36 %
 = \left( x^2 + 2 \right)^2 - 6^2 \\
& = & \left(x^2 + 2 - 6 \right) \cdot \left(x^2 + 2 + 6 \right) %
 = \left(x^2 - 4 \right) \cdot \left(x^2 + 2 + 6 \right) %
 = (x - 2) \cdot (x + 2) \cdot \left(x^2 + 8 \right) \MDFPeriod %%
\end{eqnarray*}
%
In der ersten Rechnung wurde nach dem Ziehen der ersten Wurzel auf eine nähere 
Betrachtung des Falls $x^2 + 2 = -6$ verzichtet, da man
aus der daraus folgenden Gleichung $x^2 = -8$ keine reellen Nullstellen erhält.
Die reellen Nullstellen von $f-g$ sind $-2$ und $2$.
Dies sind gleichzeitig auch die Randstellen des Intervalls $[-2\MIntvlSep  2]$.
Eine Aufteilung des Integrals in verschiedene Bereiche ist also nicht nötig. 
Die Werte von $f$ sind kleiner als die von $g$ auf diesem Intervall. Damit
erhält man den Flächeninhalt 

\begin{eqnarray*}
I_A &=& \int_{-2}^{2} \left|f(x) - g(x)\right| \MD x \\
 & = & \int_{-2}^{2} \left({g(x) - f(x)}\right) \MD x \\
 & = & \int_{-2}^{2} \left({-\frac{1}{4} x^4 - x^2 + 8 }\right) \MD x \\
 & = & 2 \int_{0}^{2} \left({-\frac{1}{4} x^4 - x^2 + 8 }\right) \MD x %
  \MDFPSpace, \qquad \text{da der Integrand gerade ist} \\
 & = & \left[{-\frac{1}{20} x^5 - \frac{1}{3} x^3 + 8 x}\right]_{0}^{2} \\
 & = & 2\cdot \left({-\frac{32}{20}-\frac{8}{3}+16}\right) \\
 & = & \frac{352}{15} \MDFPeriod %%
\end{eqnarray*}
\end{MExample}

\end{MXContent}


\begin{MXContent}{Naturwissenschaftliche Anwendungen}{Naturwissenschaftliche Anwenungen}{STD}
\MDeclareSiteUXID{VBKM08_Anwendungen_Naturwissenschaften}

Die Geschwindigkeit $v$ beschreibt die momentane Änderungsrate des Ortes 
zur Zeit $t$. Es gilt also $v = \frac{\MD s}{\MD t}$, wenn man $v = v(t)$ 
und $s = s(t)$ als Funktionen der Zeit auffasst.
Der aktuelle Aufenthaltsort $s(T)$ ergibt sich durch die Umkehrung der 
Ableitung, also durch die Integration der Geschwindigkeit über die Zeit. 
Mit dem Anfangswert $s(t = 0) = s_0$ zur Zeit $t = 0$ erhält man damit
%
\begin{eqnarray*}
\int_{0}^{T}\frac{\MD s}{\MD t} \MDwSp t &=& \int_{0}^{T} v \MDwSp t \\
\Leftrightarrow\;\;\left[s(t)\right]_{0}^{T} &=& \int_{0}^{T} v \MDwSp t \\
\Leftrightarrow\;\;s(T) - s(0) &=& \int_{0}^{T} v \MDwSp t \\
\Leftrightarrow\;\;s(T) &=& s_0 + \int_0^T v(t) \MDwSp t \MDFPeriod
\end{eqnarray*}
%
Die Situation kann man mathematisch so zusammenfassen: Wenn die 
\emph{Ableitung} $f'$ einer Funktion $f$ und ein einzelner 
Funktionswert $f(x_0)$ bekannt ist, kann die Funktion mit Hilfe des Integrals 
berechnet werden. Hierfür sagt man auch, dass die Funktionswerte aus der 
Ableitung rekonstruiert werden.

Wenn sich beispielsweise eine Bakterienpopulation näherungsweise 
%gemäß $B'$ mit $B'(t) = \MEU^{\MZahl{0}{6} t}$ für $t \geq 0$ vermehrt 
gemäß $B'$ mit $B'(t) = \MZahl{0}{6} t$ für $t \geq 0$ vermehrt 
und zu Beginn $B(0) = 100$ Bakterien vorhanden sind, wird der Bestand $B$ 
zur Zeit $T$ durch
\[
%B(T) - B(0) = \int_{0}^{T} \MEU^{\MZahl{0}{6} t} \MDwSp t %
B(T) - B(0) = \int_{0}^{T} \MZahl{0}{6} t \MDwSp t %
\]
und damit durch
\[
%B(T) = B(0) + \int_{0}^{T} \MEU^{\MZahl{0}{6} t} \MDwSp t %
B(T) = B(0) + \int_{0}^{T} \MZahl{0}{6} t \MDwSp t %
%= 100 + \int_{0}^{T} \MZahl{0}{6} t \MDwSp t %
= 100 + \MZahl{0}{6} \int_{0}^{T} t \MDwSp t %
= 100 + \MZahl{0}{3} \left(T^2 - 0^2\right) %
= 100 + \MZahl{0}{3} T^2 %
\]
beschrieben. Der Hauptsatz der Differential- und Integralrechnung bietet für 
solche Fragestellungen ein wichtiges Hilfsmittel, eine Funktion zu rekonstruieren,
wenn ihre Ableitung bekannt (und stetig) ist. In praktischen Anwendungen werden
die Funktionen allerdings oft etwas komplizierter sein, zum Beispiel aus 
Verknüpfungen mit Exponentialfunktionen bestehen.

Ein weiteres Beispiel aus der Physik, das der Leserin oder dem Leser bekannt 
sein könnte, ist die Bestimmung der Arbeit als Produkt aus Kraft und Weg:
$W = F_s \cdot s$. Dabei ist $F_s$ die Projektion der Kraft auf die Wegrichtung.
Ist die Kraft jedoch wegabhängig, gilt dieses Gesetz nicht mehr in seiner 
einfachen Form. Um die Arbeit, die z.B. beim Verschieben eines massiven 
Körpers entlang eines Weges verrichtet wird, zu bestimmen, muss man die 
aufgewendete Kraft dann entlang des Weges vom Anfangspunkt $s_1$ bis zum 
Endpunkt $s_2$ integrieren:
%
\begin{eqnarray*}
W = \int_{s_1}^{s_2}F_s(s) \MDwSp s \MDFPeriod
\end{eqnarray*}
%
Dies sollen nur drei Beispiele aus dem naturwissenschaftlich-technischen 
Bereich sein, in denen der Integralbegriff hilfreich ist. 
Im Verlauf des Studiums werden je nach Fachrichtung eine ganze Reihe weiterer 
Anwendungen der Integration auftauchen.
\end{MXContent}



%%%Uebungen zum Abschnitt:
\begin{MExercises}
\MDeclareSiteUXID{VBKM08_Anwendungen_Exercises}

\begin{MExercise}
Berechnen Sie den Inhalt $I_A$ der Fläche $A$, die durch den Graphen der 
Funktion
$f: [-2 \pi\MIntvlSep  2 \pi] \rightarrow \R, x \Mmapsto 3 \sin\left(x\right)$ 
und die $x$-Achse begrenzt wird.

Antwort: \MEquationItem{$I_A$}{\MLParsedQuestion{30}{24}{4}{IG35}}.

\begin{MHint}{\iSolution}
Die Funktion $f$ mit $f(x) = 3 \sin\left(x\right)$ hat auf dem Intervall 
$[-2 \pi\MIntvlSep  2 \pi]$ die Nullstellen $-2 \pi$, $-\pi$, $0$, $\pi$ und 
$2 \pi$. Da der Graph von $f$ punktsymmetrisch zum Nullpunkt ist, ergibt sich 
der Flächeninhalt mit der folgenden Rechnung zu
\begin{eqnarray*}
\int_{-2 \pi}^{2 \pi} |f(x)| \MDwSp x %
 & = & 3 \cdot \int_{-2 \pi}^{2 \pi}  \left|\sin\left(x\right)\right| \MD x \\
 & = & 3 \cdot 2 \cdot \int_{0}^{2 \pi}  \left|\sin\left(x\right)\right| \MD x,
 \qquad \text{da die Funktion\ $|\sin|$\ gerade ist,} \\
 & = & 6 \cdot \left( \int_{0}^{\pi}  \sin\left(x\right) \MD x %
         +\int_{\pi}^{2\pi}  \left(-\sin\left(x\right)\right) \MD x\right) \\
 & = & 6 \cdot \left( \left[ -\cos\left(x\right)\right]_0^\pi %
         + \left[ \cos\left(x\right)\right]_{\pi}^{2\pi}\right) \\
 & = & 6 \cdot \left(\left(-(-1)+1\right)+\left(1-(-1)\right)\right) \\
 & = & 24 \MDFPeriod %
\end{eqnarray*}
Die Berechnung ist natürlich auch ohne die Beobachtung möglich, dass der 
Graph von $f$ punktsymmetrisch zum Nullpunkt ist.
\end{MHint}
\end{MExercise}


\begin{MExercise}
Berechnen Sie den Inhalt $I_A$ der Fläche $A$, die durch die Graphen der 
Funktionen
$f: [1\MIntvlSep  3] \rightarrow \R, x \Mmapsto 3 - (x - 2)^2$ und 
$g: [1\MIntvlSep  3] \rightarrow \R, x \Mmapsto 2 \cdot (x - 2)^4$ 
eingeschlossen wird.
Zeichnen Sie dazu zunächst die Graphen der Funktionen, bevor Sie den 
Flächeninhalt berechnen.

Antwort: \MEquationItem{$I_A$}{\MLParsedQuestion{30}{22+8/15}{4}{IG36}}.

\begin{MHint}{\iSolution}
Zur Berechnung des Flächeninhalts $I_A$ der Fläche zwischen den 
Funktionsgraphen von $f$ und $g$ wird $f - g$ mit 
$f(x) - g(x) = 3 - (x - 2)^2 - 2 \cdot ( x - 2)^4$ auf dem Intervall
$[1\MIntvlSep  3]$ betrachtet. 

\begin{center}
\MTikzAuto{%
\ifttm\else\begin{small}\fi
\begin{tikzpicture}[line width=1.5pt,scale=1.0,
declare function={
  u1 = 1;
  u2 = 3;
  fkt1(\x) = 3 - ((\x - 2) * (\x - 2)); % $1 \leq x \leq 3$
  fkt2(\x) = 2 * pow({\x - 2}, 4); % $1 \leq x \leq 3$
}
] %[every node/.style={fill=white}] 
%Koordinatenachsen:
\draw[->] (-0.5, 0) -- (3.5, 0) node[below left]{$x$}; %x-Achse
\draw[->] (0, -0.5) -- (0, 3.5) node[below left]{$y$}; %y-Achse
%Achsenbeschriftung:
\foreach \x in {1, 2, 3} \draw (\x, 0) -- ++(0, -0.1) %
 node[below] {$\x$}; 
\foreach \y in {1, 2, 3} \draw (0, \y) -- ++(-0.1, 0) %
 node[left] {$\y$};
\node[below left] at (0, 0) {$0$};
%Funktion:
\draw[domain=u1:u2,samples=120,color=\jccolorfkt, fill=\jccolorfktarea] %
 plot (\x, {fkt1(\x)});
\draw[domain=u1:u2,samples=120,color=blue, fill=\jccolorfktarea] %\jccolorfkt] %
 plot (\x, {fkt2(\x)+0.014});
%Beschriftung:
\node[right] at (2.75, {fkt1(2.7)}) {$f(x)$};
\node[right] at (2.75, {fkt2(2.7)}) {$g(x)$};
%
\end{tikzpicture}
\ifttm\else\end{small}\fi
}
\end{center}
Aus den Zeichnungen der Funktionsgraphen ist zu sehen, dass die Differenz 
$f(x) - g(x)$ größer gleich Null für $x \in [1\MIntvlSep  3]$ ist. 
Dies kann auch rechnerisch festgestellt werden:
Nach Voraussetzung ist hier $1 \leq x \leq 3$ und damit $-1 \leq x - 2 \leq 1$. 
Folglich ist hier $(x - 2)^2 \leq 1$ und damit $-(x - 2)^2 \geq -1$, sodass
\[
f(x) - g(x) \geq 3 - 1 - 2 \cdot 1 = 0 %%
\]
für $1 \leq x \leq 3$ gilt. 

Für die Berechnung des Flächeninhalts ist somit das Integral 
$\displaystyle \int_1^3 \left(f(x) - g(x)\right) \MD x$ auszuwerten. Dazu können 
die Funktionsterme ausmultipliziert und mit der Summenregel integriert werden.
Eine andere Möglichkeit besteht darin, die beiden Funktionsterme genauer zu
betrachten:
In der gegebenen Situation des Beispiels liegen mit $(x - 2)^2$ bzw. 
$(x - 2)^4$ Terme vor, die durch Verschiebung aus den bekannten Termen 
$z^2$ bzw. $z^4$ gemäß $z = x - 2$ hervorgehen.
Eine Stammfunktion von $h$ mit $h(z) = z^2$ ist $H$ mit 
$H(z) = \frac{1}{3} \cdot z^3$. Wenn man jetzt entsprechend $F$ mit 
$F(x) = 3 x - \frac{1}{3} \cdot (x - 2)^3$ betrachtet, ergibt sich mit der 
Kettenregel
$F'(x) = 3 - \frac{1}{3} \cdot 3 \cdot (x - 2)^2 \cdot 1 = f(x)$. 
Dabei ergibt sich der letzte Faktor aus der Ableitung der inneren Funktion 
$u$ mit $u(x) = x - 2$.  Deshalb ist $F$ eine Stammfunktion von $f$. 
Entsprechend kann man nachrechnen, dass $G$ mit 
$G(x) = \frac{2}{5} \cdot (x - 2)^5$ eine Stammfunktion von $g$ ist.

Damit ergibt sich für den Flächeninhalt $I_A$ zwischen den Funktionsgraphen
\begin{eqnarray*}
I_A & = & \left|\int_1^3 \left(3 - (x - 2)^2 - 2 \cdot (x - 2)^4\right) \MD x \right| \\
 & = & \left|\left[ 3 x - \frac{1}{3} (x - 2)^3 - \frac{2}{5} (x - 2)^5 \right]_{1}^{3} \right| \\
 & = & \left| 27 - \frac{1}{3} - \frac{2}{5} %
      - \left(3 + \frac{1}{3} + \frac{2}{5}\right) \right| \\
 & = & 22 + \frac{8}{15} \MDFPeriod %%
\end{eqnarray*}
\end{MHint}
\end{MExercise}

%\begin{MExercise}
%Berechnen Sie den Inhalt $I_A$ der Fläche $A$, die durch die Graphen der 
%Funktionen
%$f: [0\MIntvlSep  1] \rightarrow \R, x \Mmapsto \sin\left(\pi x\right)$ und 
%$g: [0\MIntvlSep  1] \rightarrow \R, x \Mmapsto 12 x^2 - 12 x$ 
%eingeschlossen wird.
%Zeichnen Sie zunächst die Graphen der Funktionen, bevor Sie den 
%Flächeninhalt berechnen.
%
%Antwort: \MEquationItem{$I_A$}{\MLParsedQuestion{30}{30}{2-2/pi}{IG36}}.
%
%\begin{MHint}{\iSolution}
%Man betrachtet $f(x) - g(x) = 12 x - 12 x^2 + \sin\left(\pi x\right)$ auf 
%dem Intervall
%$[0\MIntvlSep  1]$. Die Funktion $f - g$ wird nur an den Rändern des 
%Intervalls $[0\MIntvlSep  1]$ gleich Null. Dies kann man anhand einer 
%Zeichnung der Funktionsgraphen sehen.
%\begin{center}
%\MTikzAuto{%
%\ifttm\else\begin{small}\fi
%\begin{tikzpicture}[line width=1.5pt,scale=2.0,
%declare function={
%  u1 = 0;
%  u2 = 1;
%  fkt1(\x) = sin(pi * \x r); % $0 \leq x \leq 1$
%  fkt2(\x) = 12 * \x * (\x - 1); % $0 \leq x \leq 1$
%}
%] %[every node/.style={fill=white}] 
%%Koordinatenachsen:
%\draw[->] (-0.5, 0) -- (1.5, 0) node[below left]{$x$}; %x-Achse
%\draw[->] (0, -3.5) -- (0, 1.5) node[below left]{$y$}; %y-Achse
%%Achsenbeschriftung:
%%\foreach \x in {1} \draw (\x, 0) -- ++(0, -0.1) %
%% node[below right] {$\x$}; 
%\node[below right] at (1, 0) {$1$}; 
%\foreach \y in {-2, -1, 1} \draw (0, \y) -- ++(-0.1, 0) %
% node[left] {$\y$};
%\node[below left] at (0, 0) {$0$};
%%Funktion:
%\draw[domain=u1:u2,samples=120,color=\jccolorfkt] %
% plot (\x, {fkt1(\x)});
%\draw[domain=u1:u2,samples=120,color=blue] %\jccolorfkt] %
% plot (\x, {fkt2(\x)});
%%Beschriftung:
%\node[right] at (0.75, {fkt1(0.7)}) {$f(x) = \sin(\pi x) \geq 0$};
%\node[right] at (0.75, {fkt2(0.7)}) {$g(x) = 12 x (1 - x) \leq 0$};
%%
%\end{tikzpicture}
%\ifttm\else\end{small}\fi
%}
%\end{center}
%
%Berechnet man also $f - g$, dann zieht man auf dem gesamten offenen Intervall 
%$\MoIl 0\MIntvlSep  1\MoIr$ eine negative Zahl von einer positiven Zahl ab, 
%erhält also durchgehend positive Werte für $f - g$. Nur an den Rändern rechnet 
%man $0 - 0$ und erhält dort auch für die Differenz der Funktionen den Wert Null.
%
%Damit ergibt sich für den Flächeninhalt $I_A$ zwischen den Kurven
%\begin{eqnarray*}
%I_A & = & \left|\int_0^1 \left(12 x^2 - 12 x + \sin\left(\pi x\right)\right) \MD x \right| \\
% & = & \left|\left[ 4 x^3 - 6 x^2 - \frac{1}{\pi} \cos\left(\pi x\right) \right]_{0}^{1} \right| \\
% & = & \left| 4 - 6 + \frac{1}{\pi} - \left( 0 - 0 - \frac{1}{\pi}\right) \right| \\
% & = & \left|-2 + \frac{2}{\pi} \right| = 2 - \frac{2}{\pi} \MDFPeriod %%
%\end{eqnarray*}
%Anmerkung:
%Dass die Funktionen $f$ und $g$ nur an den Rändern gemeinsame Funktionswerte 
%haben, kann man auch rechnerisch mit einer einfachen Überlegung verstehen:
%
%Auf dem Intervall $\MoIl 0\MIntvlSep  1\MoIr$ ist die Funktion $g$ mit
%$g\left(x\right) = 12x^2 - 12x = 12x\left(x-1\right)$
%immer kleiner als Null, da $x-1$ in diesem Intervall kleiner als Null ist. 
%Nimmt man nun die Randwerte des Intervalls hinzu, so sieht man, dass die 
%Funktion $g$ dort ihre Nullstellen hat.
%
%Untersucht man hingegen die Funktion $f$ mit 
%$f\left(x\right) = \sin\left(\pi x\right)$ 
%auf demselben Intervall, dann findet man, dass diese Funktion dort immer 
%positiv ist und ebenfalls an den Rändern ihre Nullstellen besitzt.
%\end{MHint}
%\end{MExercise}

In der nächsten Aufgabe wird eine physikalische Fragestellung in der Sprache
der Mathematik formuliert, wobei eine Vereinfachung in der Beschreibung 
vorgenommen wird. Dies soll exemplarisch verdeutlichen, dass die 
mathematischen Schreibweisen wie hier für Funktionen prinzipiell auch in 
Anwendungen eingesetzt werden können. In der Praxis werden oft kürze 
Formulierungen verwendet. So werden beispielsweise Definitionsbereich und 
Zielbereich einer Funktion nicht explizit notiert, wenn sich diese aus dem 
Kontext ergeben.

\begin{MExercise}
Berechnen Sie die Arbeit $W$, die nötig ist, um einen kleinen kugelförmigen 
homogenen Körper $k$, der die Masse $m$ hat, \emph{gegen} die Gravitationskraft
$F: [r_1\MIntvlSep  \infty\MoIr \rightarrow \R, r \Mmapsto F(r) := -\gamma \cdot \frac{m \cdot M}{r^2}$ 
von der Oberfläche eines kugelförmigen homogenen Körpers $K$ mit 
Radius $r_1 = 1$ und Masse $M = 2$ bis zu einer Entfernung $r_2 = 4$ zu 
bewegen (alle Längen beziehen sich auf den Mittelpunkt des Körpers $K$).
Hierbei sind die Masse $m$ und die Gravitationskonstante $\gamma$ 
als gegebene Werte anzunehmen, und der kleine Körper $k$ wird im Vergleich zum 
Körper $K$ als punktförmig angesehen.

Antwort: \MEquationItem{$W$}{\MLSimplifyQuestion{30}{3/2*gamma*m}{10}{gamma,m}{4}{0}{ILGAMMA}}.
\ \\
\MInputHint{Die Konstanten $m$ und $\gamma$ müssen in der Lösung stehen bleiben, 
für $\gamma$ kann man \texttt{gamma} eingeben.}
\ \\
\begin{MHint}{\iSolution}
Die Kraft $F_s$ längs des Weges, durch die der kleine Körper $k$ mit der 
Masse $m$ von der Oberfläche des Körpers $K$ weg bewegt wird, zeigt 
\emph{entgegen} der Gravitationskraft $F$. Somit ist $F_s = -F$.

Die aufzuwendende Arbeit $W$ von $r_1 = 1$ zu $r_2 = 4$ ergibt sich damit zu
\begin{eqnarray*}
W = \int_1^4 F_s(r) \MDwSp r %
 = - \int_1^4 F(r) \MDwSp r %
 & = & -\int_1^4 -\gamma \cdot \frac{2 \cdot m}{r^2} \MDwSp r \\
 & = & \int_1^4 \gamma \cdot \frac{2 \cdot m}{r^2} \MDwSp r \\
 & = & \left[ -\gamma \cdot \frac{2 \cdot m}{r} \right]_{1}^{4} \\
 & = & -\gamma \cdot 2 \cdot m \left(\frac{1}{4} - \frac{1}{1}\right) \\
 & = & \gamma \cdot \frac{3 m}{2} \MDFPeriod %%
\end{eqnarray*}
\end{MHint}
%
\end{MExercise}

\end{MExercises}


\MSubsection{Abschlusstest}
\MLabel{VBKM08_Abschlusstest}


\begin{MTest}{Abschlusstest Kapitel \arabic{section}}
\MDeclareSiteUXID{VBKM08_Abschlusstest}
\begin{MExercise} %Stammfunktionen:
Bestimmen Sie jeweils eine Stammfunktion:
\begin{MExerciseItems}
\item $\displaystyle\int 3 \sqrt{x} \MDwSp x = $%
\MLSimplifyQuestion{26}{2*x^(3/2)}{1}{x}{4}{32}{SIMPLE18}
%\item $\displaystyle\int \left(2 x - \frac{1}{x+\pi}\right) \MD x = $%
%\MSimplifyQuestion{26}{x^2-ln(x+pi)}{1}{x}{20}{32}
\item $\displaystyle\int \left(2 x - \MEU^{x+\pi}\right) \MD x = $%
\MLSimplifyQuestion{26}{x^2-exp(x+pi)}{10}{x}{4}{32}{SIMPLE19}
\end{MExerciseItems}
\jHTMLHinweisEingabeFunktionenExp
\end{MExercise}

%Die Aufgaben in den Abschlusstests sollen sich am "geforderten Mindestniveau"
%orientieren. Deshalb wurde die Aufgabe durch die nachfolgende neue Aufgabe
%ersetzt:
%\begin{MExercise} %Integralbegriff und Eigenschaften
%Die Funktion $f(x) := \frac{1}{x+1}$ ist für $x \geq 0$ 
%streng monoton \MLQuestion{20}{fallend}{IG3}. Somit gilt
%\ifttm
%
%\begin{center}
%$\;\displaystyle\sum_{k=0}^7 \frac{1}{k+1} \cdot (k+1 - k) \;$%
%\MLQuestion{5}{>=}{IG5}%
%$\;\displaystyle\int_0^8 \frac{1}{x+1} \MDwSp x.$
%\end{enter}
%
%\else
%\[
%\sum_{k=0}^7 \frac{1}{k+1} \cdot (k+1 - k) \;\; \MLQuestion{5}{>=}{IG6} \;\; %
%\int_0^8 \frac{1}{x+1} \MDwSp x. %%
%\]
%\fi
%\MInputHint{Ergänzen Sie den Text zu einer richtigen Aussage. Vergleiche 
%werden mit \texttt{=}, \texttt{<=} oder \texttt{>=} geschrieben.}
%\end{MExercise}

%Neue Aufgabe:
\begin{MExercise} %Integralbegriff und Eigenschaften
Berechnen Sie die Integrale
\ifttm

\begin{center}
$\displaystyle\int_{1}^{\MEU} \frac{1}{2 x} \MDwSp x = $%
\MLParsedQuestion{10}{1/2}{4}{IGEx1237}
und
$\displaystyle\int_{5}^{8} \frac{6}{x-4} \MDwSp x = $%
\MLParsedQuestion{10}{12*ln(2)}{4}{IGEx1238}
\end{center}

\else
\[
\int_{1}^{\MEU} \frac{1}{2 x} \MDwSp x = %
\MLParsedQuestion{10}{1/2}{4}{IGEx1237}
\qquad \text{und} \qquad
\int_{5}^{8} \frac{6}{x-4} \MDwSp x = %
\MLParsedQuestion{10}{12*ln(2)}{4}{IGEx1238}
\]
\fi
\end{MExercise}


\begin{MExercise} %Berechnung bestimmter Integrale
Berechnen Sie die Integrale:
\ifttm

\begin{center}
$\displaystyle\int_{0}^{3} x \cdot \sqrt{x+1} \MDwSp x = $%
\MLParsedQuestion{10}{116/15}{4}{IGEx1239} und 
$\displaystyle\int_{\pi}^{\frac{3\pi}{4}} 5 \sin(4 x - 3 \pi) \MDwSp x = $%
\MLParsedQuestion{10}{-5/2}{4}{IGEx1240}
\end{center}

\else
\[
\int_{0}^{3} x \cdot \sqrt{x+1} \MDwSp x = %
\MLParsedQuestion{10}{116/15}{4}{IGEx1239}
\qquad \text{und} \qquad
\int_{\pi}^{\frac{3\pi}{4}} 5 \sin(4 x - 3 \pi) \MDwSp x = %
\MLParsedQuestion{10}{-5/2}{4}{IGEx1240}
\]
\fi
\end{MExercise}


\begin{MExercise} %Eigenschaften
Es gilt:
\ifttm

\begin{center}
$\displaystyle 2 \int_{a}^{4} |x^3| \MDwSp x %
 = \int_{-4}^{4} |x^3| \MDwSp x \;$\MLQuestion{5}{>=}{IG9}%
$\;\left| \int_{-4}^{4} x^3 \MDwSp x \right|$ %%
\end{center}
für $a = $\MLParsedQuestion{5}{0}{4}{IG10}.

\else
\[
2 \int_{\MLParsedQuestion{4}{0}{4}{IG44}}^{4} |x^3| \MDwSp x %
 = \int_{-4}^{4} |x^3| \MDwSp x %
\;\; \MLQuestion{5}{>=}{IG11} \;\; %
\left| \int_{-4}^{4} x^3 \MDwSp x \right| %%
\]
\fi
\MInputHint{Ergänzen Sie den Text zu einer richtigen Aussage. Vergleiche 
werden mit \texttt{=}, \texttt{<=} oder \texttt{>=} geschrieben.}
\end{MExercise}


\begin{MExercise} %Anwendung: Flächenberechung
Berechnen Sie den Inhalt $I_A$ der Fläche $A$, die durch die Graphen der 
beiden Funktionen
%$f: [-3; 2] \rightarrow \R, x \Mmapsto x^2$ und 
%$g: [-3; 2] \rightarrow \R, x \Mmapsto 6 - x$ eingeschlossen wird.
$f$ und $g$ auf $[-3\MIntvlSep  2]$ mit $f(x) = x^2$ und $g(x) = 6 - x$ 
eingeschlossen wird.

Antwort: $I_A = $\MLParsedQuestion{10}{125/6}{4}{IG45}
\end{MExercise}

\begin{MExercise} %Stammfunktionen: Gemeinsame Aufgabe mit OMB+
Es ist eine Stammfunktion $F$ der Funktion $f$ gegeben, und eine 
Stammfunktion $G$ von $g$. Weiter ist eine Funktion $\Mid$ mit 
$\Mid(x) = x$ gegeben.

Welche der folgenden Aussagen gelten stets (wenn die jeweiligen 
Verknüpfungen möglich sind)?

\ifttm
\begin{tabular}{|l|l|}
\hline
% richtig: & falsch: & Aussage: \\
 richtig? & Aussage: \\
 \MLCheckbox{0}{M08C01a} & % \MLCheckbox{1}{M08C01b} &
$\Mid \cdot F$ ist eine Stammfunktion von $\Mid \cdot f$? \\
%
 \MLCheckbox{0}{M08C02a} & % \MLCheckbox{1}{M08C02b} &
$F \circ G$ ist eine Stammfunktion von $f \circ g$? \\
%
 \MLCheckbox{1}{M08C03a} & % \MLCheckbox{0}{M08C03b} &
$F - G$ ist eine Stammfunktion von $f - g$? \\
%
 \MLCheckbox{0}{M08C04a} & % \MLCheckbox{1}{M08C04b} &
$F / G$ ist eine Stammfunktion von $f / g$? \\
%
 \MLCheckbox{0}{M08C05a} & % \MLCheckbox{1}{M08C05b} &
$F \cdot G$ ist eine Stammfunktion von $f \cdot g$? \\
%
 \MLCheckbox{1}{M08C06a} & % \MLCheckbox{0}{M08C06b} &
$-20 \cdot F$ ist eine Stammfunktion von $-20 \cdot f$? \\
\hline
\end{tabular}
\else
\begin{tabular}[t]{ccl}
%\begin{tabular}[t]{lcc}
%Aussage: & richtig: & falsch: \\
 richtig: & falsch: & Aussage: \\
 \MLCheckbox{0}{M08C01a} & \MLCheckbox{1}{M08C01b} &
$\Mid \cdot F$ ist eine Stammfunktion von $\Mid \cdot f$? \\
% & \MLCheckbox{0}{M08C01a} & \MLCheckbox{1}{M08C01b} \\
%
 \MLCheckbox{0}{M08C02a} & \MLCheckbox{1}{M08C02b} &
$F \circ G$ ist eine Stammfunktion von $f \circ g$? \\
% & \MLCheckbox{0}{M08C02a} & \MLCheckbox{1}{M08C02b} \\
%
 \MLCheckbox{1}{M08C03a} & \MLCheckbox{0}{M08C03b} &
$F - G$ ist eine Stammfunktion von $f - g$? \\
% & \MLCheckbox{1}{M08C03a} & \MLCheckbox{0}{M08C03b} \\
%
 \MLCheckbox{0}{M08C04a} & \MLCheckbox{1}{M08C04b} &
$F / G$ ist eine Stammfunktion von $f / g$? \\
% & \MLCheckbox{0}{M08C04a} & \MLCheckbox{1}{M08C04b} \\
%
 \MLCheckbox{0}{M08C05a} & \MLCheckbox{1}{M08C05b} &
$F \cdot G$ ist eine Stammfunktion von $f \cdot g$? \\
% & \MLCheckbox{0}{M08C05a} & \MLCheckbox{1}{M08C05b} \\
%
 \MLCheckbox{1}{M08C06a} & \MLCheckbox{0}{M08C06b} &
$-20 \cdot F$ ist eine Stammfunktion von $-20 \cdot f$? %
% & \MLCheckbox{1}{M08C06a} & \MLCheckbox{0}{M08C06b} %%
%
\end{tabular}
\fi
\end{MExercise}
\end{MTest}

\clearpage
\MPrintIndex

\end{document}

%Dateiende.

