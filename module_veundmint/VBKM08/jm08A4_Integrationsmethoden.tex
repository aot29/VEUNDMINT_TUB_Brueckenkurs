%LaTeX-2e-File, Liedtke, 20140828.
%Inhalt: Rechenregeln und Integrationsmethoden.
%zuletzt bearbeitet: 20140929.

%Inhalt zum Abschnitt: Integrationsmethoden.

\MPragma{MathSkip}

\begin{MXContent}{Elementare Umformungen}{Elementare Umformungen}{STD}

In etlichen Situationen wird die Berechung eines Integrals einfacher, 
wenn als erstes der Integrand in einer m"oglichst bekannten Art formuliert 
wird. Dies wird an einigen Beispielen illustriert.

Im ersten Beispiel werden Potzenfunktionen betrachtet.
\begin{MExample}
Es wird das Integral 
\[
\int_{1}^{4} (x - 2) \cdot \sqrt{x} \jMD x %%
\]
berechnet. Dazu wird zun"achst der Integrand zu
\[
 (x - 2) \cdot \sqrt{x} = x \sqrt{x} - 2 \sqrt{x} %
 = x^{\frac{3}{2}} - 2 x^{\frac{1}{2}} %%
\]
umgeformt. Damit ist dann
\begin{eqnarray*}
\int_{1}^{4} (x - 2) \cdot \sqrt{x} \jMD x %%
& = &
\int_{1}^{4} \left(x^{\frac{3}{2}} - 2 x^{\frac{1}{2}}\right) \jMD x %%
  =  
\left[\frac{2}{5} x^{\frac{5}{2}} - \frac{4}{3} x^{\frac{3}{2}}\right]_{1}^{4} \\
& = &
\left(\frac{2}{5} \left(\sqrt{4}\right)^5 - \frac{4}{3} \left(\sqrt{4}\right)^3 \right) %
 - \left(\frac{2}{5} \cdot 1 - \frac{4}{3} \cdot 1 \right) \\
& = &
\left(\frac{64}{5} - \frac{32}{3}\right) - \left(\frac{2}{5} - \frac{4}{3}\right) \\
& = & \frac{62}{5} - \frac{28}{3} \\
& = & 3 + \frac{1}{15} %%
\end{eqnarray*}
\end{MExample}

Das n"achste Beispiel illustriert die Umformung eines Integranden mit 
trigonometrischen Funktionen.
\begin{MExample}
Es wird das Integral 
\[
\int_{-\frac{\pi}{2}}^{\frac{\pi}{2}} \frac{\sin(x)}{\tan(x)} \jMD x %%
\]
berechnet, indem mit der Definition der Tangensfunktion die Umformung
\[
\frac{\sin(x)}{\tan(x)} %
= \frac{\sin(x)}{\frac{\sin(x)}{\cos(x)}} %
= \cos(x) %%
\]
vorgenommen wird, sodass sich
\[
\int_{-\frac{\pi}{2}}^{\frac{\pi}{2}} \frac{\sin(x)}{\tan(x)} \jMD x %%
= 
\int_{-\frac{\pi}{2}}^{\frac{\pi}{2}} \cos(x) \jMD x %%
= 
\left[\sin(x)\right]_{-\frac{\pi}{2}}^{\frac{\pi}{2}} %%
= 
1 - (-1) = 2. %%
\]
ergibt.
\end{MExample}

F"ur rationale Funktionen wird zun"achst eine Polynomdivision durchgef"uhrt, 
wenn der Grad des Z"ahlerpolynoms gr"o"ser oder gleich dem des Nennerpolynoms 
ist. Je nach Situation werden sich dann noch weitere Umformungen anbieten,
die Sie in der weiterf"uhrenden Literatur unter dem Titel Partialbruchzerlegung
finden. Im folgenden Beispiel wird der erste wichtige vorbereitende Schritt 
mit der Polynomdivision vorgestellt, um rationale Funktionen zu integrieren.
\begin{MExample}
Es wird das Integral 
\[
\int_{-1}^{1} \frac{4 x^2 - x + 4}{x^2 + 1} \jMD x %%
\]
berechnet. Dazu wird zun"achst der Integrand mittels Polynomdivision zu
\[
\frac{4 x^2 - x + 4}{x^2 + 1} = 4 - \frac{x}{x^2 + 1} %%
\]
umgeformt. Damit ist dann
\begin{eqnarray*}
\int_{-1}^{1} \frac{4 x^2 - x + 4}{x^2 + 1} \jMD x %%
& = & 
\int_{-1}^{1} \left(4 - \frac{x}{x^2 + 1}\right) \jMD x \\
& = & 
\int_{-1}^{1} 4 \jMD x - \int_{-1}^{1} \frac{x}{x^2 + 1} \jMD x %%
= \left[4 x\right]_{-1}^{1} - 0 %
= 8.
\end{eqnarray*}
Denn der Integrand des zweiten Integrals ist ungerade und das 
Integrationsintervall $[-1, 1]$ ist bzgl. null symmetrisch, sodass der Wert
des zweiten Integrals null ist.
\end{MExample}

\end{MXContent}


\begin{MXContent}{Verkettete Integranden}{Verkettete Integranden}{STD}

In vielen praktischen Anwendungen ergeben sich verkettete Funktionen.
Beispielsweise wenn eine Umrechnung von einer Temperaturskala in eine andere
erfolgen soll: Die Umrechnung der Temperaturangabe in Kelvin in Grad Celsius 
ist mathematisch gesehen einfach eine Verschiebung des Nullpunktes. 
Wenn $x$ die Temperatur in Kelvin angibt, ist $u(x) = x + \msz[,]{273}{15}$ die 
Temperatur in Grad Celsius. Die Umrechnung von Grad Fahrenheit in 
Grad Celsius erfordert neben einer Verschiebung auch noch eine Streckung  
gem"a"s $u(x) = \frac{5}{9} (x - 32) = \frac{5}{9} x - \frac{160}{9}$, 
wenn $x$ die Temperatur in Grad Fahrenheit und $u(x)$ in Grad Celsius angibt.

Im Folgenden wird eine M"oglichkeit vorgestellt, das Integral von Funktionen
zu berechnen, f"ur die derartige Skalen"anderungen vorgenommen werden sollen, 
mathematisch ausgedr"uckt, bedeutet dies die Berechung des Integrals einer 
Verkettung von Funktionen.

Nach dem Hauptsatz kann das Integral einer stetigen Funktion mit Hilfe einer 
Stammfunktion berechnet werden. Somit ist eine Funktion gesucht, deren 
Ableitung gleich der gegebenen verketteten Funktionen ist.  
Hier gehen wir davon aus, dass es sich um eine Verkettung mit einer linearen 
Funktion $u(x) = m x + b$ handelt.

Deshalb betrachten wir zun"achst noch einmal die Kettenregel der 
Differentialrechnung, um eine Stammfunktion f"ur die verkettete Funktion 
%$h(x) = f(u(x)) = f(m x + b)$ 
zu finden, und damit dann das Integral mit Hilfe des Hauptsatzes zu berechnen.

Die Ableitung von $H(x) = F(u(x)) = F(m x + b)$ ergibt sich aus der Kettenregel 
der Differentialrechnung (siehe auch Modul 7):
\[
H'(x) = F'(u(x)) \cdot u'(x) = F'(m x + b) \cdot m = m \cdot F'(m x + b). %%
\]
Von links nach rechts gelesen lautet die Gleichung
\[
m \cdot F'(m x + b) = H'(x). %%
\]
Mit dem Hauptsatz und den vereinbarten Schreibweisen folgt
\begin{eqnarray*}
\int_a^b m \cdot F'(m x + b) \jMD x %
 & = & \int_a^b H'(x) \MD x \\
 & = & H(x)|_a^b = H(b) - H(a) = F(u(b)) - F(u(a)) \\
 & = & \left[F(z)\right]_{u(a)}^{u(b)} \\
 & = & \int_{u(a)}^{u(b)} F'(z) \jMD z %%
\end{eqnarray*}
Wird die Funktion $F'$ noch mit $f$ bezeichnet, also $f(z) := F'(z)$ gesetzt,
ergibt sich
\begin{eqnarray*}
\int_a^b m \cdot f(m x + b) \jMD x %
 & = & \int_a^b m \cdot F'(m x + b) \jMD x \\
 & = & \int_{u(a)}^{u(b)} F'(z) \jMD x \\
 & = & \int_{u(a)}^{u(b)} f(z) \jMD z %%
\end{eqnarray*}

\begin{MXInfo}{Integration verketteter Funktionen} 
Gegeben sind stetige Funktionen $f: [a, b] \rightarrow \R$ und 
$u: [a, b] \rightarrow \R$ mit $u(x) = m x + d$.

Dann gilt
\[
\int_{a}^{b} f(u(x)) \cdot u'(x) \jMD x %
= \int_{u(a)}^{u(b)} f(z) \jMD z. %% 
\]
und f"ur $u(x) = m x + d$ dann
\begin{equation}
\int_{a}^{b} m \cdot f(m x + d) \jMD x %
  =  \int_{a}^{b} f(m x + d) \cdot m \jMD x %
  =  \int_{u(a)}^{u(b)} f(z) \jMD z. %% 
\end{equation}
Multiplikation mit $\frac{1}{m}$ f"uhrt auf 
\begin{eqnarray*}
\int_{a}^{b} f(m x + d) \jMD x %
 & = & \int_{a}^{b} \frac{1}{m} \cdot m \cdot f(m x + d) \jMD x \\
 & = & \frac{1}{m} \cdot \int_{a}^{b} m \cdot f(m x + d) \jMD x \\
 = \frac{1}{m} \cdot \int_{u(a)}^{u(b)} f(z) \jMD z %%
\end{eqnarray*}
Damit gilt dann
\[
\int_{a}^{b} f(m x + d) \jMD x %
 = \frac{1}{m} \cdot \int_{u(a)}^{u(b)} f(z) \jMD z %%
\]
\end{MXInfo}

Insbesondere ergibt sich f"ur $m = 1$, wenn also nur die $x$-Achse verschoben
wird, dann
\[
\int_{a}^{b} f(x + d) \jMD x = \int_{a+d}^{b+d} f(z) \jMD z.
\]
Auch aus der Interpretation des Integrals als Summe der (orientierten) 
Fl"acheninhalt unter dem Graphen der Funktion ist das Ergebnis verst"andlich, 
dass die Verschiebung der $x$-Achse nur eine entsprechende Verschiebung der 
Integrationsgrenzen bedeutet.

\begin{MExample}
F"ur $h(x) := (3 x - 7)^5$ gilt mit $u(x) := 3 x - 7$ und $u'(x) = 3$ dann
\begin{eqnarray*}
 &  & \int_{2}^{3} (3 x - 7)^5 \jMD x \\
 & = & \int_{-1}^{2} \frac{1}{3} z^5 \jMD z % 
  \qquad \text{mit } u(2) = -1, u(3) = 2 \\
 & = & \left[\frac{1}{3} z^5\right]_{-1}^{2} \\
 & = & \frac{1}{3} \left(2^5 - (-1)^5 \right) \\
 & = & \frac{33}{3} = 11 \\
\end{eqnarray*}
\end{MExample}

%\begin{MExample}
%F"ur $h(x) := \cos(3 x - \frac{\pi}{4})$ gilt
%\begin{eqnarray*}
% &  & \int_{-\frac{\pi}{4}}^{\frac{\pi}{12}} \cos(3 x - \frac{\pi}{4}) \jMD x \\
% & = & \frac{1}{3} \int_{-\pi}^{0} \cos(z) \jMD z \\
% & = & \frac{1}{3} \left[\sin(z)\right]_{-\pi}^{0} \\
% & = & \frac{1}{3} \left[\sin(z)\right]_{-\pi}^{0} \\
%\end{eqnarray*}
%\end{MExample}

In vielen praktischen Anwendungen erfolgt die Integration einer stetigen 
Funktion $f$ anhand des Hauptsatzes, indem eine Stammfunktion zu $f$ gesucht 
wird. Die vorgestellten Beispiele m"ogen einen ersten Eindruck davon 
vermitteln, dass die Berechnung einer Stammfunktion anspruchsvoll sein kann.
In diesem Zusammenhang soll nochmals an die wesentliche Eigenschaft einer 
Stammfunktion erinnert werden, anhand derer eine Kontrolle des Ergebnisses
meist einfach m"oglich ist.

\begin{MXInfo}{Ein praktischer Hinweis zur Erinnerung}
% zur Kunst des Integrierens und zur Technik der Ableitung}
Wenn zur Integration einer Funktion $f$ eine Stammfunktion $F$ auf irgend 
einem Weg bestimmt wurde, dann kann das Ergebnis dadurch kontrolliert werden, 
dass die Ableitung $F'$ berechnet wird. Nur wenn $F' = f$ gilt, ist $F$ eine 
Stammfunktion von $f$.
\end{MXInfo}
\end{MXContent}




%Dateiende.

