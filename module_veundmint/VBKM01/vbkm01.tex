% MINTMOD Version P0.1.0, needs to be consistent with preprocesser object in tex2x and MPragma-Version at the end of this file

% Parameter aus Konvertierungsprozess (PDF und HTML-Erzeugung wenn vom Konverter aus gestartet) werden hier eingefuegt, Preambleincludes werden am Schluss angehaengt

\newif\ifttm                % gesetzt falls Uebersetzung in HTML stattfindet, sonst uebersetzung in PDF

% Wahl der Notationsvariante ist im PDF immer std, in der HTML-Uebersetzung wird vom Konverter die Auswahl modifiziert
\newif\ifvariantstd
\newif\ifvariantunotation
\variantstdtrue % Diese Zeile wird vom Konverter erkannt und ggf. modifiziert, daher nicht veraendern!


\def\MOutputDVI{1}
\def\MOutputPDF{2}
\def\MOutputHTML{3}
\newcounter{MOutput}

\ifttm
\usepackage{german}
\usepackage{array}
\usepackage{amsmath}
\usepackage{amssymb}
\usepackage{amsthm}
\else
\documentclass[ngerman,oneside]{scrbook}
\usepackage{etex}
\usepackage[latin1]{inputenc}
\usepackage{textcomp}
\usepackage[ngerman]{babel}
\usepackage[pdftex]{color}
\usepackage{xcolor}
\usepackage{graphicx}
\usepackage[all]{xy}
\usepackage{fancyhdr}
\usepackage{verbatim}
\usepackage{array}
\usepackage{float}
\usepackage{makeidx}
\usepackage{amsmath}
\usepackage{amstext}
\usepackage{amssymb}
\usepackage{amsthm}
\usepackage[ngerman]{varioref}
\usepackage{framed}
\usepackage{supertabular}
\usepackage{longtable}
\usepackage{maxpage}
\usepackage{tikz}
\usepackage{tikzscale}
\usepackage{tikz-3dplot}
\usepackage{bibgerm}
\usepackage{chemarrow}
\usepackage{polynom}
%\usepackage{draftwatermark}
\usepackage{pdflscape}
\usetikzlibrary{calc}
\usetikzlibrary{through}
\usetikzlibrary{shapes.geometric}
\usetikzlibrary{arrows}
\usetikzlibrary{intersections}
\usetikzlibrary{decorations.pathmorphing}
\usetikzlibrary{external}
\usetikzlibrary{patterns}
\usetikzlibrary{fadings}
\usepackage[colorlinks=true,linkcolor=blue]{hyperref} 
\usepackage[all]{hypcap}
%\usepackage[colorlinks=true,linkcolor=blue,bookmarksopen=true]{hyperref} 
\usepackage{ifpdf}

\usepackage{movie15}

\setcounter{tocdepth}{2} % In Inhaltsverzeichnis bis subsection
\setcounter{secnumdepth}{3} % Nummeriert bis subsubsection

\setlength{\LTpost}{0pt} % Fuer longtable
\setlength{\parindent}{0pt}
\setlength{\parskip}{8pt}
%\setlength{\parskip}{9pt plus 2pt minus 1pt}
\setlength{\abovecaptionskip}{-0.25ex}
\setlength{\belowcaptionskip}{-0.25ex}
\fi

\ifttm
\newcommand{\MDebugMessage}[1]{\special{html:<!-- debugprint;;}#1\special{html:; //-->}}
\else
%\newcommand{\MDebugMessage}[1]{\immediate\write\mintlog{#1}}
\newcommand{\MDebugMessage}[1]{}
\fi

\def\MPageHeaderDef{%
\pagestyle{fancy}%
\fancyhead[r]{(C) VE\&MINT-Projekt}
\fancyfoot[c]{\thepage\\--- CCL BY-SA 3.0 ---}
}


\ifttm%
\def\MRelax{}%
\else%
\def\MRelax{\relax}%
\fi%

%--------------------------- Uebernahme von speziellen XML-Versionen einiger LaTeX-Kommandos aus xmlbefehle.tex vom alten Kasseler Konverter ---------------

\newcommand{\MSep}{\left\|{\phantom{\frac1g}}\right.}

\newcommand{\ML}{L}

\newcommand{\MGGT}{\mathrm{ggT}}


\ifttm
% Verhindert dass die subsection-nummer doppelt in der toccaption auftaucht (sollte ggf. in toccaption gefixt werden so dass diese Ueberschreibung nicht notwendig ist)
\renewcommand{\thesubsection}{}
% Kommandos die ttm nicht kennt
\newcommand{\binomial}[2]{{#1 \choose #2}} %  Binomialkoeffizienten
\newcommand{\eur}{\begin{html}&euro;\end{html}}
\newcommand{\square}{\begin{html}&square;\end{html}}
\newcommand{\glqq}{"'}  \newcommand{\grqq}{"'}
\newcommand{\nRightarrow}{\special{html: &nrArr; }}
\newcommand{\nmid}{\special{html: &nmid; }}
\newcommand{\nparallel}{\begin{html}&nparallel;\end{html}}
\newcommand{\mapstoo}{\begin{html}<mo>&map;</mo>\end{html}}

% Schnitt und Vereinigungssymbole von Mengen haben zu kleine Abstaende; korrigiert:
\newcommand{\ccup}{\,\!\cup\,\!}
\newcommand{\ccap}{\,\!\cap\,\!}


% Umsetzung von mathbb im HTML
\renewcommand{\mathbb}[1]{\begin{html}<mo>&#1opf;</mo>\end{html}}
\fi

%---------------------- Strukturierung ----------------------------------------------------------------------------------------------------------------------

%---------------------- Kapselung des sectioning findet auf drei Ebenen statt:
% 1. Die LateX-Befehl
% 2. Die D-Versionen der Befehle, die nur die Grade der Abschnitte umhaengen falls notwendig
% 3. Die M-Versionen der Befehle, die zusaetzliche Formatierungen vornehmen, Skripten starten und das HTML codieren
% Im Modultext duerfen nur die M-Befehle verwendet werden!

\ifttm

  \def\Dsubsubsubsection#1{\subsubsubsection{#1}}
  \def\Dsubsubsection#1{\subsubsection{#1}\addtocounter{subsubsection}{1}} % ttm-Fehler korrigieren
  \def\Dsubsection#1{\subsection{#1}}
  \def\Dsection#1{\section{#1}} % Im HTML wird nur der Sektionstitel gegeben
  \def\Dchapter#1{\chapter{#1}}
  \def\Dsubsubsubsectionx#1{\subsubsubsection*{#1}}
  \def\Dsubsubsectionx#1{\subsubsection*{#1}}
  \def\Dsubsectionx#1{\subsection*{#1}}
  \def\Dsectionx#1{\section*{#1}}
  \def\Dchapterx#1{\chapter*{#1}}

\else

  \def\Dsubsubsubsection#1{\subsubsection{#1}}
  \def\Dsubsubsection#1{\subsection{#1}}
  \def\Dsubsection#1{\section{#1}}
  \def\Dsection#1{\chapter{#1}}
  \def\Dchapter#1{\title{#1}}
  \def\Dsubsubsubsectionx#1{\subsubsection*{#1}}
  \def\Dsubsubsectionx#1{\subsection*{#1}}
  \def\Dsubsectionx#1{\section*{#1}}
  \def\Dsectionx#1{\chapter*{#1}}

\fi

\newcommand{\MStdPoints}{4}
\newcommand{\MSetPoints}[1]{\renewcommand{\MStdPoints}{#1}}

% Befehl zum Abbruch der Erstellung (nur PDF)
\newcommand{\MAbort}[1]{\err{#1}}

% Prefix vor Dateieinbindungen, wird in der Baumdatei mit \renewcommand modifiziert
% und auf das Verzeichnisprefix gesetzt, in dem das gerade bearbeitete tex-Dokument liegt.
% Im HTML wird es auf das Verzeichnis der HTML-Datei gesetzt.
% Das Prefix muss mit / enden !
\newcommand{\MDPrefix}{.}

% MRegisterFile notiert eine Datei zur Einbindung in den HTML-Baum. Grafiken mit MGraphics werden automatisch eingebunden.
% Mit MLastFile erhaelt man eine Markierung fuer die zuletzt registrierte Datei.
% Diese Markierung wird im postprocessing durch den physikalischen Dateinamen ersetzt, aber nur den Namen (d.h. \MMaterial gehoert noch davor, vgl Definition von MGraphics)
% Parameter: Pfad/Name der Datei bzw. des Ordners, relativ zur Position des Modul-Tex-Dokuments.
\ifttm
\newcommand{\MRegisterFile}[1]{\addtocounter{MFileNumber}{1}\special{html:<!-- registerfile;;}#1\special{html:;;}\MDPrefix\special{html:;;}\arabic{MFileNumber}\special{html:; //-->}}
\else
\newcommand{\MRegisterFile}[1]{\addtocounter{MFileNumber}{1}}
\fi

% Testen welcher Uebersetzer hier am Werk ist

\ifttm
\setcounter{MOutput}{3}
\else
\ifx\pdfoutput\undefined
  \pdffalse
  \setcounter{MOutput}{\MOutputDVI}
  \message{Verarbeitung mit latex, Ausgabe in dvi.}
\else
  \setcounter{MOutput}{\MOutputPDF}
  \message{Verarbeitung mit pdflatex, Ausgabe in pdf.}
  \ifnum \pdfoutput=0
    \pdffalse
  \setcounter{MOutput}{\MOutputDVI}
  \message{Verarbeitung mit pdflatex, Ausgabe in dvi.}
  \else
    \ifnum\pdfoutput=1
    \pdftrue
  \setcounter{MOutput}{\MOutputPDF}
  \message{Verarbeitung mit pdflatex, Ausgabe in pdf.}
    \fi
  \fi
\fi
\fi

\ifnum\value{MOutput}=\MOutputPDF
\DeclareGraphicsExtensions{.pdf,.png,.jpg}
\fi

\ifnum\value{MOutput}=\MOutputDVI
\DeclareGraphicsExtensions{.eps,.png,.jpg}
\fi

\ifnum\value{MOutput}=\MOutputHTML
% Wird vom Konverter leider nicht erkannt und daher in split.pm hardcodiert!
\DeclareGraphicsExtensions{.png,.jpg,.gif}
\fi

% Umdefinition der hyperref-Nummerierung im PDF-Modus
\ifttm
\else
\renewcommand{\theHfigure}{\arabic{chapter}.\arabic{section}.\arabic{figure}}
\fi

% Makro, um in der HTML-Ausgabe die zuerst zu oeffnende Datei zu kennzeichnen
\ifttm
\newcommand{\MGlobalStart}{\special{html:<!-- mglobalstarttag -->}}
\else
\newcommand{\MGlobalStart}{}
\fi

% Makro, um bei scormlogin ein pullen des Benutzers bei Aufruf der Seite zu erzwingen (typischerweise auf der Einstiegsseite)
\ifttm
\newcommand{\MPullSite}{\special{html:<!-- pullsite //-->}}
\else
\newcommand{\MPullSite}{}
\fi

% Makro, um in der HTML-Ausgabe die Kapiteluebersicht zu kennzeichnen
\ifttm
\newcommand{\MGlobalChapterTag}{\special{html:<!-- mglobalchaptertag -->}}
\else
\newcommand{\MGlobalChapterTag}{}
\fi

% Makro, um in der HTML-Ausgabe die Konfiguration zu kennzeichnen
\ifttm
\newcommand{\MGlobalConfTag}{\special{html:<!-- mglobalconfigtag -->}}
\else
\newcommand{\MGlobalConfTag}{}
\fi

% Makro, um in der HTML-Ausgabe die Standortbeschreibung zu kennzeichnen
\ifttm
\newcommand{\MGlobalLocationTag}{\special{html:<!-- mgloballocationtag -->}}
\else
\newcommand{\MGlobalLocationTag}{}
\fi

% Makro, um in der HTML-Ausgabe die persoenlichen Daten zu kennzeichnen
\ifttm
\newcommand{\MGlobalDataTag}{\special{html:<!-- mglobaldatatag -->}}
\else
\newcommand{\MGlobalDataTag}{}
\fi

% Makro, um in der HTML-Ausgabe die Suchseite zu kennzeichnen
\ifttm
\newcommand{\MGlobalSearchTag}{\special{html:<!-- mglobalsearchtag -->}}
\else
\newcommand{\MGlobalSearchTag}{}
\fi

% Makro, um in der HTML-Ausgabe die Favoritenseite zu kennzeichnen
\ifttm
\newcommand{\MGlobalFavoTag}{\special{html:<!-- mglobalfavoritestag -->}}
\else
\newcommand{\MGlobalFavoTag}{}
\fi

% Makro, um in der HTML-Ausgabe die Eingangstestseite zu kennzeichnen
\ifttm
\newcommand{\MGlobalSTestTag}{\special{html:<!-- mglobalstesttag -->}}
\else
\newcommand{\MGlobalSTestTag}{}
\fi

% Makro, um in der PDF-Ausgabe ein Wasserzeichen zu definieren
\ifttm
\newcommand{\MWatermarkSettings}{\relax}
\else
\newcommand{\MWatermarkSettings}{%
% \SetWatermarkText{(c) MINT-Kolleg Baden-W�rttemberg 2014}
% \SetWatermarkLightness{0.85}
% \SetWatermarkScale{1.5}
}
\fi

\ifttm
\newcommand{\MBinom}[2]{\left({\begin{array}{c} #1 \\ #2 \end{array}}\right)}
\else
\newcommand{\MBinom}[2]{\binom{#1}{#2}}
\fi

\ifttm
\newcommand{\DeclareMathOperator}[2]{\def#1{\mathrm{#2}}}
\newcommand{\operatorname}[1]{\mathrm{#1}}
\fi

%----------------- Makros fuer die gemischte HTML/PDF-Konvertierung ------------------------------

\newcommand{\MTestName}{\relax} % wird durch Test-Umgebung gesetzt

% Fuer experimentelle Kursinhalte, die im Release-Umsetzungsvorgang eine Fehlermeldung
% produzieren sollen aber sonst normal umgesetzt werden
\newenvironment{MExperimental}{%
}{%
}

% Wird von ttm nicht richtig umgesetzt!!
\newenvironment{MExerciseItems}{%
\renewcommand\theenumi{\alph{enumi}}%
\begin{enumerate}%
}{%
\end{enumerate}%
}


\definecolor{infoshadecolor}{rgb}{0.75,0.75,0.75}
\definecolor{exmpshadecolor}{rgb}{0.875,0.875,0.875}
\definecolor{expeshadecolor}{rgb}{0.95,0.95,0.95}
\definecolor{framecolor}{rgb}{0.2,0.2,0.2}

% Bei PDF-Uebersetzung wird hinter den Start jeder Satz/Info-aehnlichen Umgebung eine leere mbox gesetzt, damit
% fuehrende Listen oder enums nicht den Zeilenumbruch kaputtmachen
%\ifttm
\def\MTB{}
%\else
%\def\MTB{\mbox{}}
%\fi


\ifttm
\newcommand{\MRelates}{\special{html:<mi>&wedgeq;</mi>}}
\else
\def\MRelates{\stackrel{\scriptscriptstyle\wedge}{=}}
\fi

\def\MInch{\text{''}}
\def\Mdd{\textit{''}}

\ifttm
\def\MNL{ \newline }
\newenvironment{MArray}[1]{\begin{array}{#1}}{\end{array}}
\else
\def\MNL{ \\ }
\newenvironment{MArray}[1]{\begin{array}{#1}}{\end{array}}
\fi

\newcommand{\MBox}[1]{$\mathrm{#1}$}
\newcommand{\MMBox}[1]{\mathrm{#1}}


\ifttm%
\newcommand{\Mtfrac}[2]{{\textstyle \frac{#1}{#2}}}
\newcommand{\Mdfrac}[2]{{\displaystyle \frac{#1}{#2}}}
\newcommand{\Mmeasuredangle}{\special{html:<mi>&angmsd;</mi>}}
\else%
\newcommand{\Mtfrac}[2]{\tfrac{#1}{#2}}
\newcommand{\Mdfrac}[2]{\dfrac{#1}{#2}}
\newcommand{\Mmeasuredangle}{\measuredangle}
\relax
\fi

% Matrizen und Vektoren

% Inhalt wird in der Form a & b \\ c & d erwartet
% Vorsicht: MVector = Komponentenspalte, MVec = Variablensymbol
\ifttm%
\newcommand{\MVector}[1]{\left({\begin{array}{c}#1\end{array}}\right)}
\else%
\newcommand{\MVector}[1]{\begin{pmatrix}#1\end{pmatrix}}
\fi



\newcommand{\MVec}[1]{\vec{#1}}
\newcommand{\MDVec}[1]{\overrightarrow{#1}}

%----------------- Umgebungen fuer Definitionen und Saetze ----------------------------------------

% Fuegt einen Tabellen-Zeilenumbruch ein im PDF, aber nicht im HTML
\newcommand{\TSkip}{\ifttm \else&\ \\\fi}

\newenvironment{infoshaded}{%
\def\FrameCommand{\fboxsep=\FrameSep \fcolorbox{framecolor}{infoshadecolor}}%
\MakeFramed {\advance\hsize-\width \FrameRestore}}%
{\endMakeFramed}

\newenvironment{expeshaded}{%
\def\FrameCommand{\fboxsep=\FrameSep \fcolorbox{framecolor}{expeshadecolor}}%
\MakeFramed {\advance\hsize-\width \FrameRestore}}%
{\endMakeFramed}

\newenvironment{exmpshaded}{%
\def\FrameCommand{\fboxsep=\FrameSep \fcolorbox{framecolor}{exmpshadecolor}}%
\MakeFramed {\advance\hsize-\width \FrameRestore}}%
{\endMakeFramed}

\def\STDCOLOR{black}

\ifttm%
\else%
\newtheoremstyle{MSatzStyle}
  {1cm}                   %Space above
  {1cm}                   %Space below
  {\normalfont\itshape}   %Body font
  {}                      %Indent amount (empty = no indent,
                          %\parindent = para indent)
  {\normalfont\bfseries}  %Thm head font
  {}                      %Punctuation after thm head
  {\newline}              %Space after thm head: " " = normal interword
                          %space; \newline = linebreak
  {\thmname{#1}\thmnumber{ #2}\thmnote{ (#3)}}
                          %Thm head spec (can be left empty, meaning
                          %`normal')
                          %
\newtheoremstyle{MDefStyle}
  {1cm}                   %Space above
  {1cm}                   %Space below
  {\normalfont}           %Body font
  {}                      %Indent amount (empty = no indent,
                          %\parindent = para indent)
  {\normalfont\bfseries}  %Thm head font
  {}                      %Punctuation after thm head
  {\newline}              %Space after thm head: " " = normal interword
                          %space; \newline = linebreak
  {\thmname{#1}\thmnumber{ #2}\thmnote{ (#3)}}
                          %Thm head spec (can be left empty, meaning
                          %`normal')
\fi%

\newcommand{\MInfoText}{Info}

\newcounter{MHintCounter}
\newcounter{MCodeEditCounter}

\newcounter{MLastIndex}  % Enthaelt die dritte Stelle (Indexnummer) des letzten angelegten Objekts
\newcounter{MLastType}   % Enthaelt den Typ des letzten angelegten Objekts (mithilfe der unten definierten Konstanten). Die Entscheidung, wie der Typ dargstellt wird, wird in split.pm beim Postprocessing getroffen.
\newcounter{MLastTypeEq} % =1 falls das Label in einer Matheumgebung (equation, eqnarray usw.) steht, =2 falls das Label in einer table-Umgebung steht

% Da ttm keine Zahlmakros verarbeiten kann, werden diese Nummern in den Zuweisungen hardcodiert!
\def\MTypeSection{1}          %# Zaehler ist section
\def\MTypeSubsection{2}       %# Zaehler ist subsection
\def\MTypeSubsubsection{3}    %# Zaehler ist subsubsection
\def\MTypeInfo{4}             %# Eine Infobox, Separatzaehler fuer die Chemie (auch wenn es dort nicht nummeriert wird) ist MInfoCounter
\def\MTypeExercise{5}         %# Eine Aufgabe, Separatzaehler fuer die Chemie ist MExerciseCounter
\def\MTypeExample{6}          %# Eine Beispielbox, Separatzaehler fuer die Chemie ist MExampleCounter
\def\MTypeExperiment{7}       %# Eine Versuchsbox, Separatzaehler fuer die Chemie ist MExperimentCounter
\def\MTypeGraphics{8}         %# Eine Graphik, Separatzaehler fuer alle FB ist MGraphicsCounter
\def\MTypeTable{9}            %# Eine Tabellennummer, hat keinen Zaehler da durch table gezaehlt wird
\def\MTypeEquation{10}        %# Eine Gleichungsnummer, hat keinen Zaehler da durch equation/eqnarray gezaehlt wird
\def\MTypeTheorem{11}         % Ein theorem oder xtheorem, Separatzaehler fuer die Chemie ist MTheoremCounter
\def\MTypeVideo{12}           %# Ein Video,Separatzaehler fuer alle FB ist MVideoCounter
\def\MTypeEntry{13}           %# Ein Eintrag fuer die Stichwortliste, wird nicht gezaehlt sondern erhaelt im preparsing ein unique-label 

% Zaehler fuer das Labelsystem sind prefixcounter, jeder Zaehler wird VOR dem gezaehlten Objekt inkrementiert und zaehlt daher das aktuelle Objekt
\newcounter{MInfoCounter}
\newcounter{MExerciseCounter}
\newcounter{MExampleCounter}
\newcounter{MExperimentCounter}
\newcounter{MGraphicsCounter}
\newcounter{MTableCounter}
\newcounter{MEquationCounter}  % Nur im HTML, sonst durch "equation"-counter von latex realisiert
\newcounter{MTheoremCounter}
\newcounter{MObjectCounter}   % Gemeinsamer Zaehler fuer Objekte (ausser Grafiken/Tabellen) in Mathe/Info/Physik
\newcounter{MVideoCounter}
\newcounter{MEntryCounter}

\newcounter{MTestSite} % 1 = Subsubsection ist eine Pruefungsseite, 0 = ist eine normale Seite (inkl. Hilfeseite)

\def\MCell{$\phantom{a}$}

\newenvironment{MExportExercise}{\begin{MExercise}}{\end{MExercise}} % wird von mconvert abgefangen

\def\MGenerateExNumber{%
\ifnum\value{MSepNumbers}=0%
\arabic{section}.\arabic{subsection}.\arabic{MObjectCounter}\setcounter{MLastIndex}{\value{MObjectCounter}}%
\else%
\arabic{section}.\arabic{subsection}.\arabic{MExerciseCounter}\setcounter{MLastIndex}{\value{MExerciseCounter}}%
\fi%
}%

\def\MGenerateExmpNumber{%
\ifnum\value{MSepNumbers}=0%
\arabic{section}.\arabic{subsection}.\arabic{MObjectCounter}\setcounter{MLastIndex}{\value{MObjectCounter}}%
\else%
\arabic{section}.\arabic{subsection}.\arabic{MExerciseCounter}\setcounter{MLastIndex}{\value{MExampleCounter}}%
\fi%
}%

\def\MGenerateInfoNumber{%
\ifnum\value{MSepNumbers}=0%
\arabic{section}.\arabic{subsection}.\arabic{MObjectCounter}\setcounter{MLastIndex}{\value{MObjectCounter}}%
\else%
\arabic{section}.\arabic{subsection}.\arabic{MExerciseCounter}\setcounter{MLastIndex}{\value{MInfoCounter}}%
\fi%
}%

\def\MGenerateSiteNumber{%
\arabic{section}.\arabic{subsection}.\arabic{subsubsection}%
}%

% Funktionalitaet fuer Auswahlaufgaben

\newcounter{MExerciseCollectionCounter} % = 0 falls nicht in collection-Umgebung, ansonsten Schachtelungstiefe
\newcounter{MExerciseCollectionTextCounter} % wird von MExercise-Umgebung inkrementiert und von MExerciseCollection-Umgebung auf Null gesetzt

\ifttm
% MExerciseCollection gruppiert Aufgaben, die dynamisch aus der Datenbank gezogen werden und nicht direkt in der HTML-Seite stehen
% Parameter: #1 = ID der Collection, muss eindeutig fuer alle IN DER DB VORHANDENEN collections sein unabhaengig vom Kurs
%            #2 = Optionsargument (im Moment: 1 = Iterative Auswahl, 2 = Zufallsbasierte Auswahl)
\newenvironment{MExerciseCollection}[2]{%
\addtocounter{MExerciseCollectionCounter}{1}
\setcounter{MExerciseCollectionTextCounter}{0}
\special{html:<!-- mexercisecollectionstart;;}#1\special{html:;;}#2\special{html:;; //-->}%
}{%
\special{html:<!-- mexercisecollectionstop //-->}%
\addtocounter{MExerciseCollectionCounter}{-1}
}
\else
\newenvironment{MExerciseCollection}[2]{%
\addtocounter{MExerciseCollectionCounter}{1}
\setcounter{MExerciseCollectionTextCounter}{0}
}{%
\addtocounter{MExerciseCollectionCounter}{-1}
}
\fi

% Bei Uebersetzung nach PDF werden die theorem-Umgebungen verwendet, bei Uebersetzung in HTML ein manuelles Makro
\ifttm%

  \newenvironment{MHint}[1]{  \special{html:<button name="Name_MHint}\arabic{MHintCounter}\special{html:" class="hintbutton_closed" id="MHint}\arabic{MHintCounter}\special{html:_button" %
  type="button" onclick="toggle_hint('MHint}\arabic{MHintCounter}\special{html:');">}#1\special{html:</button>}
  \special{html:<div class="hint" style="display:none" id="MHint}\arabic{MHintCounter}\special{html:"> }}{\begin{html}</div>\end{html}\addtocounter{MHintCounter}{1}}

  \newenvironment{MCOSHZusatz}{  \special{html:<button name="Name_MHint}\arabic{MHintCounter}\special{html:" class="chintbutton_closed" id="MHint}\arabic{MHintCounter}\special{html:_button" %
  type="button" onclick="toggle_hint('MHint}\arabic{MHintCounter}\special{html:');">}Weiterf�hrende Inhalte\special{html:</button>}
  \special{html:<div class="hintc" style="display:none" id="MHint}\arabic{MHintCounter}\special{html:">
  <div class="coshwarn">Diese Inhalte gehen �ber das Kursniveau hinaus und werden in den Aufgaben und Tests nicht abgefragt.</div><br />}
  \addtocounter{MHintCounter}{1}}{\begin{html}</div>\end{html}}

  
  \newenvironment{MDefinition}{\begin{definition}\setcounter{MLastIndex}{\value{definition}}\ \\}{\end{definition}}

  
  \newenvironment{MExercise}{
  \renewcommand{\MStdPoints}{4}
  \addtocounter{MExerciseCounter}{1}
  \addtocounter{MObjectCounter}{1}
  \setcounter{MLastType}{5}

  \ifnum\value{MExerciseCollectionCounter}=0\else\addtocounter{MExerciseCollectionTextCounter}{1}\special{html:<!-- mexercisetextstart;;}\arabic{MExerciseCollectionTextCounter}\special{html:;; //-->}\fi
  \special{html:<div class="aufgabe" id="ADIV_}\MGenerateExNumber\special{html:">}%
  \textbf{Aufgabe \MGenerateExNumber
  } \ \\}{
  \special{html:</div><!-- mfeedbackbutton;Aufgabe;}\arabic{MTestSite}\special{html:;}\MGenerateExNumber\special{html:; //-->}
  \ifnum\value{MExerciseCollectionCounter}=0\else\special{html:<!-- mexercisetextstop //-->}\fi
  }

  % Stellt eine Kombination aus Aufgabe, Loesungstext und Eingabefeld bereit,
  % bei der Aufgabentext und Musterloesung sowie die zugehoerigen Feldelemente
  % extern bezogen und div-aktualisiert werden, das Eingabefeld aber immer das gleiche ist.
  \newenvironment{MFetchExercise}{
  \addtocounter{MExerciseCounter}{1}
  \addtocounter{MObjectCounter}{1}
  \setcounter{MLastType}{5}

  \special{html:<div class="aufgabe" id="ADIV_}\MGenerateExNumber\special{html:">}%
  \textbf{Aufgabe \MGenerateExNumber
  } \ \\%
  \special{html:</div><div class="exfetch_text" id="ADIVTEXT_}\MGenerateExNumber\special{html:">}%
  \special{html:</div><div class="exfetch_sol" id="ADIVSOL_}\MGenerateExNumber\special{html:">}%
  \special{html:</div><div class="exfetch_input" id="ADIVINPUT_}\MGenerateExNumber\special{html:">}%
  }{
  \special{html:</div>}
  }

  \newenvironment{MExample}{
  \addtocounter{MExampleCounter}{1}
  \addtocounter{MObjectCounter}{1}
  \setcounter{MLastType}{6}
  \begin{html}
  <div class="exmp">
  <div class="exmprahmen">
  \end{html}\textbf{Beispiel
  \ifnum\value{MSepNumbers}=0
  \arabic{section}.\arabic{subsection}.\arabic{MObjectCounter}\setcounter{MLastIndex}{\value{MObjectCounter}}
  \else
  \arabic{section}.\arabic{subsection}.\arabic{MExampleCounter}\setcounter{MLastIndex}{\value{MExampleCounter}}
  \fi
  } \ \\}{\begin{html}</div>
  </div>
  \end{html}
  \special{html:<!-- mfeedbackbutton;Beispiel;}\arabic{MTestSite}\special{html:;}\MGenerateExmpNumber\special{html:; //-->}
  }

  \newenvironment{MExperiment}{
  \addtocounter{MExperimentCounter}{1}
  \addtocounter{MObjectCounter}{1}
  \setcounter{MLastType}{7}
  \begin{html}
  <div class="expe">
  <div class="experahmen">
  \end{html}\textbf{Versuch
  \ifnum\value{MSepNumbers}=0
  \arabic{section}.\arabic{subsection}.\arabic{MObjectCounter}\setcounter{MLastIndex}{\value{MObjectCounter}}
  \else
%  \arabic{MExperimentCounter}\setcounter{MLastIndex}{\value{MExperimentCounter}}
  \arabic{section}.\arabic{subsection}.\arabic{MExperimentCounter}\setcounter{MLastIndex}{\value{MExperimentCounter}}
  \fi
  } \ \\}{\begin{html}</div>
  </div>
  \end{html}}

  \newenvironment{MChemInfo}{
  \setcounter{MLastType}{4}
  \begin{html}
  <div class="info">
  <div class="inforahmen">
  \end{html}}{\begin{html}</div>
  </div>
  \end{html}}

  \newenvironment{MXInfo}[1]{
  \addtocounter{MInfoCounter}{1}
  \addtocounter{MObjectCounter}{1}
  \setcounter{MLastType}{4}
  \begin{html}
  <div class="info">
  <div class="inforahmen">
  \end{html}\textbf{#1
  \ifnum\value{MInfoNumbers}=0
  \else
    \ifnum\value{MSepNumbers}=0
    \arabic{section}.\arabic{subsection}.\arabic{MObjectCounter}\setcounter{MLastIndex}{\value{MObjectCounter}}
    \else
    \arabic{MInfoCounter}\setcounter{MLastIndex}{\value{MInfoCounter}}
    \fi
  \fi
  } \ \\}{\begin{html}</div>
  </div>
  \end{html}
  \special{html:<!-- mfeedbackbutton;Info;}\arabic{MTestSite}\special{html:;}\MGenerateInfoNumber\special{html:; //-->}
  }

  \newenvironment{MInfo}{\ifnum\value{MInfoNumbers}=0\begin{MChemInfo}\else\begin{MXInfo}{Info}\ \\ \fi}{\ifnum\value{MInfoNumbers}=0\end{MChemInfo}\else\end{MXInfo}\fi}

\else%

  \theoremstyle{MSatzStyle}
  \newtheorem{thm}{Satz}[section]
  \newtheorem{thmc}{Satz}
  \theoremstyle{MDefStyle}
  \newtheorem{defn}[thm]{Definition}
  \newtheorem{exmp}[thm]{Beispiel}
  \newtheorem{info}[thm]{\MInfoText}
  \theoremstyle{MDefStyle}
  \newtheorem{defnc}{Definition}
  \theoremstyle{MDefStyle}
  \newtheorem{exmpc}{Beispiel}[section]
  \theoremstyle{MDefStyle}
  \newtheorem{infoc}{\MInfoText}
  \theoremstyle{MDefStyle}
  \newtheorem{exrc}{Aufgabe}[section]
  \theoremstyle{MDefStyle}
  \newtheorem{verc}{Versuch}[section]
  
  \newenvironment{MFetchExercise}{}{} % kann im PDF nicht dargestellt werden
  
  \newenvironment{MExercise}{\begin{exrc}\renewcommand{\MStdPoints}{1}\MTB}{\end{exrc}}
  \newenvironment{MHint}[1]{\ \\ \underline{#1:}\\}{}
  \newenvironment{MCOSHZusatz}{\ \\ \underline{Weiterf�hrende Inhalte:}\\}{}
  \newenvironment{MDefinition}{\ifnum\value{MInfoNumbers}=0\begin{defnc}\else\begin{defn}\fi\MTB}{\ifnum\value{MInfoNumbers}=0\end{defnc}\else\end{defn}\fi}
%  \newenvironment{MExample}{\begin{exmp}}{\ \linebreak[1] \ \ \ \ $\phantom{a}$ \ \hfill $\blacklozenge$\end{exmp}}
  \newenvironment{MExample}{
    \ifnum\value{MInfoNumbers}=0\begin{exmpc}\else\begin{exmp}\fi
    \MTB
    \begin{exmpshaded}
    \ \newline
}{
    \end{exmpshaded}
    \ifnum\value{MInfoNumbers}=0\end{exmpc}\else\end{exmp}\fi
}
  \newenvironment{MChemInfo}{\begin{infoshaded}}{\end{infoshaded}}

  \newenvironment{MInfo}{\ifnum\value{MInfoNumbers}=0\begin{MChemInfo}\else\renewcommand{\MInfoText}{Info}\begin{info}\begin{infoshaded}
  \MTB
   \ \newline
    \fi
  }{\ifnum\value{MInfoNumbers}=0\end{MChemInfo}\else\end{infoshaded}\end{info}\fi}

  \newenvironment{MXInfo}[1]{
    \renewcommand{\MInfoText}{#1}
    \ifnum\value{MInfoNumbers}=0\begin{infoc}\else\begin{info}\fi%
    \MTB
    \begin{infoshaded}
    \ \newline
  }{\end{infoshaded}\ifnum\value{MInfoNumbers}=0\end{infoc}\else\end{info}\fi}

  \newenvironment{MExperiment}{
    \renewcommand{\MInfoText}{Versuch}
    \ifnum\value{MInfoNumbers}=0\begin{verc}\else\begin{info}\fi
    \MTB
    \begin{expeshaded}
    \ \newline
  }{
    \end{expeshaded}
    \ifnum\value{MInfoNumbers}=0\end{verc}\else\end{info}\fi
  }
\fi%

% MHint sollte nicht direkt fuer Loesungen benutzt werden wegen solutionselect
\newenvironment{MSolution}{\begin{MHint}{L"osung}}{\end{MHint}}

\newcounter{MCodeCounter}

\ifttm
\newenvironment{MCode}{\special{html:<!-- mcodestart -->}\ttfamily\color{blue}}{\special{html:<!-- mcodestop -->}}
\else
\newenvironment{MCode}{\begin{flushleft}\ttfamily\addtocounter{MCodeCounter}{1}}{\addtocounter{MCodeCounter}{-1}\end{flushleft}}
% Ohne color-Statement da inkompatible mit framed/shaded-Boxen aus dem framed-package
\fi

%----------------- Sonderdefinitionen fuer Symbole, die der Konverter nicht kann ----------------------------------------------

\ifttm%
\newcommand{\MUnderset}[2]{\underbrace{#2}_{#1}}%
\else%
\newcommand{\MUnderset}[2]{\underset{#1}{#2}}%
\fi%

\ifttm
\newcommand{\MThinspace}{\special{html:<mi>&#x2009;</mi>}}
\else
\newcommand{\MThinspace}{\,}
\fi

\ifttm
\newcommand{\glq}{\begin{html}&sbquo;\end{html}}
\newcommand{\grq}{\begin{html}&lsquo;\end{html}}
\newcommand{\glqq}{\begin{html}&bdquo;\end{html}}
\newcommand{\grqq}{\begin{html}&ldquo;\end{html}}
\fi

\ifttm
\newcommand{\MNdash}{\begin{html}&ndash;\end{html}}
\else
\newcommand{\MNdash}{--}
\fi

%\ifttm\def\MIU{\special{html:<mi>&#8520;</mi>}}\else\def\MIU{\mathrm{i}}\fi
\def\MIU{\mathrm{i}}
\def\MEU{e} % TU9-Onlinekurs: italic-e
%\def\MEU{\mathrm{e}} % Alte Onlinemodule: roman-e
\def\MD{d} % Kursives d in Integralen im TU9-Onlinekurs
%\def\MD{\mathrm{d}} % roman-d in den alten Onlinemodulen
\def\MDB{\|}

%zusaetzlicher Leerraum vor "\MD"
\ifttm%
\def\MDSpace{\special{html:<mi>&#x2009;</mi>}}
\else%
\def\MDSpace{\,}
\fi%
\newcommand{\MDwSp}{\MDSpace\MD}%

\ifttm
\def\Mdq{\dq}
\else
\def\Mdq{\dq}
\fi

\def\MSpan#1{\left<{#1}\right>}
\def\MSetminus{\setminus}
\def\MIM{I}

\ifttm
\newcommand{\ld}{\text{ld}}
\newcommand{\lg}{\text{lg}}
\else
\DeclareMathOperator{\ld}{ld}
%\newcommand{\lg}{\text{lg}} % in latex schon definiert
\fi


\def\Mmapsto{\ifttm\special{html:<mi>&mapsto;</mi>}\else\mapsto\fi} 
\def\Mvarphi{\ifttm\phi\else\varphi\fi}
\def\Mphi{\ifttm\varphi\else\phi\fi}
\ifttm%
\newcommand{\MEumu}{\special{html:<mi>&#x3BC;</mi>}}%
\else%
\newcommand{\MEumu}{\textrm{\textmu}}%
\fi
\def\Mvarepsilon{\ifttm\epsilon\else\varepsilon\fi}
\def\Mepsilon{\ifttm\varepsilon\else\epsilon\fi}
\def\Mvarkappa{\ifttm\kappa\else\varkappa\fi}
\def\Mkappa{\ifttm\varkappa\else\kappa\fi}
\def\Mcomplement{\ifttm\special{html:<mi>&comp;</mi>}\else\complement\fi} 
\def\MWW{\mathrm{WW}}
\def\Mmod{\ifttm\special{html:<mi>&nbsp;mod&nbsp;</mi>}\else\mod\fi} 

\ifttm%
\def\mod{\text{\;mod\;}}%
\def\MNEquiv{\special{html:<mi>&NotCongruent;</mi>}}% 
\def\MNSubseteq{\special{html:<mi>&NotSubsetEqual;</mi>}}%
\def\MEmptyset{\special{html:<mi>&empty;</mi>}}%
\def\MVDots{\special{html:<mi>&#x22EE;</mi>}}%
\def\MHDots{\special{html:<mi>&#x2026;</mi>}}%
\def\Mddag{\special{html:<mi>&#x1202;</mi>}}%
\def\sphericalangle{\special{html:<mi>&measuredangle;</mi>}}%
\def\nparallel{\special{html:<mi>&nparallel;</mi>}}%
\def\MProofEnd{\special{html:<mi>&#x25FB;</mi>}}%
\newenvironment{MProof}[1]{\underline{#1}:\MCR\MCR}{\hfill $\MProofEnd$}%
\else%
\def\MNEquiv{\not\equiv}%
\def\MNSubseteq{\not\subseteq}%
\def\MEmptyset{\emptyset}%
\def\MVDots{\vdots}%
\def\MHDots{\hdots}%
\def\Mddag{\ddag}%
\newenvironment{MProof}[1]{\begin{proof}[#1]}{\end{proof}}%
\fi%



% Spaces zum Auffuellen von Tabellenbreiten, die nur im HTML wirken
\ifttm%
\def\MTSP{\:}%
\else%
\def\MTSP{}%
\fi%

\DeclareMathOperator{\arsinh}{arsinh}
\DeclareMathOperator{\arcosh}{arcosh}
\DeclareMathOperator{\artanh}{artanh}
\DeclareMathOperator{\arcoth}{arcoth}


\newcommand{\MMathSet}[1]{\mathbb{#1}}
\def\N{\MMathSet{N}}
\def\Z{\MMathSet{Z}}
\def\Q{\MMathSet{Q}}
\def\R{\MMathSet{R}}
\def\C{\MMathSet{C}}

\newcounter{MForLoopCounter}
\newcommand{\MForLoop}[2]{\setcounter{MForLoopCounter}{#1}\ifnum\value{MForLoopCounter}=0{}\else{{#2}\addtocounter{MForLoopCounter}{-1}\MForLoop{\value{MForLoopCounter}}{#2}}\fi}

\newcounter{MSiteCounter}
\newcounter{MFieldCounter} % Kombination section.subsection.site.field ist eindeutig in allen Modulen, field alleine nicht

\newcounter{MiniMarkerCounter}

\ifttm
\newenvironment{MMiniPageP}[1]{\begin{minipage}{#1\linewidth}\special{html:<!-- minimarker;;}\arabic{MiniMarkerCounter}\special{html:;;#1; //-->}}{\end{minipage}\addtocounter{MiniMarkerCounter}{1}}
\else
\newenvironment{MMiniPageP}[1]{\begin{minipage}{#1\linewidth}}{\end{minipage}\addtocounter{MiniMarkerCounter}{1}}
\fi

\newcounter{AlignCounter}

\newcommand{\MStartJustify}{\ifttm\special{html:<!-- startalign;;}\arabic{AlignCounter}\special{html:;;justify; //-->}\fi}
\newcommand{\MStopJustify}{\ifttm\special{html:<!-- stopalign;;}\arabic{AlignCounter}\special{html:; //-->}\fi\addtocounter{AlignCounter}{1}}

\newenvironment{MJTabular}[1]{
\MStartJustify
\begin{tabular}{#1}
}{
\end{tabular}
\MStopJustify
}

\newcommand{\MImageLeft}[2]{
\begin{center}
\begin{tabular}{lc}
\MStartJustify
\begin{MMiniPageP}{0.65}
#1
\end{MMiniPageP}
\MStopJustify
&
\begin{MMiniPageP}{0.3}
#2  
\end{MMiniPageP}
\end{tabular}
\end{center}
}

\newcommand{\MImageHalf}[2]{
\begin{center}
\begin{tabular}{lc}
\MStartJustify
\begin{MMiniPageP}{0.45}
#1
\end{MMiniPageP}
\MStopJustify
&
\begin{MMiniPageP}{0.45}
#2  
\end{MMiniPageP}
\end{tabular}
\end{center}
}

\newcommand{\MBigImageLeft}[2]{
\begin{center}
\begin{tabular}{lc}
\MStartJustify
\begin{MMiniPageP}{0.25}
#1
\end{MMiniPageP}
\MStopJustify
&
\begin{MMiniPageP}{0.7}
#2  
\end{MMiniPageP}
\end{tabular}
\end{center}
}

\ifttm
\def\No{\mathbb{N}_0}
\else
\def\No{\ensuremath{\N_0}}
\fi
\def\MT{\textrm{\tiny T}}
\newcommand{\MTranspose}[1]{{#1}^{\MT}}
\ifttm
\newcommand{\MRe}{\mathsf{Re}}
\newcommand{\MIm}{\mathsf{Im}}
\else
\DeclareMathOperator{\MRe}{Re}
\DeclareMathOperator{\MIm}{Im}
\fi

\newcommand{\Mid}{\mathrm{id}}
\newcommand{\MFeinheit}{\mathrm{feinh}}

\ifttm
\newcommand{\Msubstack}[1]{\begin{array}{c}{#1}\end{array}}
\else
\newcommand{\Msubstack}[1]{\substack{#1}}
\fi

% Typen von Fragefeldern:
% 1 = Alphanumerisch, case-sensitive-Vergleich
% 2 = Ja/Nein-Checkbox, Loesung ist 0 oder 1   (OPTION = Image-id fuer Rueckmeldung)
% 3 = Reelle Zahlen Geparset
% 4 = Funktionen Geparset (mit Stuetzstellen zur ueberpruefung)

% Dieser Befehl erstellt ein interaktives Aufgabenfeld. Parameter:
% - #1 Laenge in Zeichen
% - #2 Loesungstext (alphanumerisch, case sensitive)
% - #3 AufgabenID (alphanumerisch, case sensitive)
% - #4 Typ (Kennnummer)
% - #5 String fuer Optionen (ggf. mit Semikolon getrennte Einzelstrings)
% - #6 Anzahl Punkte
% - #7 uxid (kann z.B. Loesungsstring sein)
% ACHTUNG: Die langen Zeilen bitte so lassen, Zeilenumbrueche im tex werden in div's umgesetzt
\newcommand{\MQuestionID}[7]{
\ifttm
\special{html:<!-- mdeclareuxid;;}UX#7\special{html:;;}\arabic{section}\special{html:;;}#3\special{html:;; //-->}%
\special{html:<!-- mdeclarepoints;;}\arabic{section}\special{html:;;}#3\special{html:;;}#6\special{html:;;}\arabic{MTestSite}\special{html:;;}\arabic{chapter}%
\special{html:;; //--><!-- onloadstart //-->CreateQuestionObj("}#7\special{html:",}\arabic{MFieldCounter}\special{html:,"}#2%
\special{html:","}#3\special{html:",}#4\special{html:,"}#5\special{html:",}#6\special{html:,}\arabic{MTestSite}\special{html:,}\arabic{section}%
\special{html:);<!-- onloadstop //-->}%
\special{html:<input mfieldtype="}#4\special{html:" name="Name_}#3\special{html:" id="}#3\special{html:" type="text" size="}#1\special{html:" maxlength="}#1%
\special{html:" }\ifnum\value{MGroupActive}=0\special{html:onfocus="handlerFocus(}\arabic{MFieldCounter}%
\special{html:);" onblur="handlerBlur(}\arabic{MFieldCounter}\special{html:);" onkeyup="handlerChange(}\arabic{MFieldCounter}\special{html:,0);" onpaste="handlerChange(}\arabic{MFieldCounter}\special{html:,0);" oninput="handlerChange(}\arabic{MFieldCounter}\special{html:,0);" onpropertychange="handlerChange(}\arabic{MFieldCounter}\special{html:,0);"/>}%
\special{html:<img src="images/questionmark.gif" width="20" height="20" border="0" align="absmiddle" id="}QM#3\special{html:"/>}
\else%
\special{html:onblur="handlerBlur(}\arabic{MFieldCounter}%
\special{html:);" onfocus="handlerFocus(}\arabic{MFieldCounter}\special{html:);" onkeyup="handlerChange(}\arabic{MFieldCounter}\special{html:,1);" onpaste="handlerChange(}\arabic{MFieldCounter}\special{html:,1);" oninput="handlerChange(}\arabic{MFieldCounter}\special{html:,1);" onpropertychange="handlerChange(}\arabic{MFieldCounter}\special{html:,1);"/>}%
\special{html:<img src="images/questionmark.gif" width="20" height="20" border="0" align="absmiddle" id="}QM#3\special{html:"/>}\fi%
\else%
\ifnum\value{QBoxFlag}=1\fbox{$\phantom{\MForLoop{#1}{b}}$}\else$\phantom{\MForLoop{#1}{b}}$\fi%
\fi%
}

% ACHTUNG: Die langen Zeilen bitte so lassen, Zeilenumbrueche im tex werden in div's umgesetzt
% QuestionCheckbox macht ausserhalb einer QuestionGroup keinen Sinn!
% #1 = solution (1 oder 0), ggf. mit ::smc abgetrennt auszuschliessende single-choice-boxen (UXIDs durch , getrennt), #2 = id, #3 = points, #4 = uxid
\newcommand{\MQuestionCheckbox}[4]{
\ifttm
\special{html:<!-- mdeclareuxid;;}UX#4\special{html:;;}\arabic{section}\special{html:;;}#2\special{html:;; //-->}%
\ifnum\value{MGroupActive}=0\MDebugMessage{ERROR: Checkbox Nr. \arabic{MFieldCounter}\ ist nicht in einer Kontrollgruppe, es wird niemals eine Loesung angezeigt!}\fi
\special{html: %
<!-- mdeclarepoints;;}\arabic{section}\special{html:;;}#2\special{html:;;}#3\special{html:;;}\arabic{MTestSite}\special{html:;;}\arabic{chapter}%
\special{html:;; //--><!-- onloadstart //-->CreateQuestionObj("}#4\special{html:",}\arabic{MFieldCounter}\special{html:,"}#1\special{html:","}#2\special{html:",2,"IMG}#2%
\special{html:",}#3\special{html:,}\arabic{MTestSite}\special{html:,}\arabic{section}\special{html:);<!-- onloadstop //-->}%
\special{html:<input mfieldtype="2" type="checkbox" name="Name_}#2\special{html:" id="}#2\special{html:" onchange="handlerChange(}\arabic{MFieldCounter}\special{html:,1);"/><img src="images/questionmark.gif" name="}Name_IMG#2%
\special{html:" width="20" height="20" border="0" align="absmiddle" id="}IMG#2\special{html:"/> }%
\else%
\ifnum\value{QBoxFlag}=1\fbox{$\phantom{X}$}\else$\phantom{X}$\fi%
\fi%
}

\def\MGenerateID{QFELD_\arabic{section}.\arabic{subsection}.\arabic{MSiteCounter}.QF\arabic{MFieldCounter}}

% #1 = 0/1 ggf. mit ::smc abgetrennt auszuschliessende single-choice-boxen (UXIDs durch , getrennt ohne UX), #2 = uxid ohne UX
\newcommand{\MCheckbox}[2]{
\MQuestionCheckbox{#1}{\MGenerateID}{\MStdPoints}{#2}
\addtocounter{MFieldCounter}{1}
}

% Erster Parameter: Zeichenlaenge der Eingabebox, zweiter Parameter: Loesungstext
\newcommand{\MQuestion}[2]{
\MQuestionID{#1}{#2}{\MGenerateID}{1}{0}{\MStdPoints}{#2}
\addtocounter{MFieldCounter}{1}
}

% Erster Parameter: Zeichenlaenge der Eingabebox, zweiter Parameter: Loesungstext
\newcommand{\MLQuestion}[3]{
\MQuestionID{#1}{#2}{\MGenerateID}{1}{0}{\MStdPoints}{#3}
\addtocounter{MFieldCounter}{1}
}

% Parameter: Laenge des Feldes, Loesung (wird auch geparsed), Stellen Genauigkeit hinter dem Komma, weitere Stellen werden mathematisch gerundet vor Vergleich
\newcommand{\MParsedQuestion}[3]{
\MQuestionID{#1}{#2}{\MGenerateID}{3}{#3}{\MStdPoints}{#2}
\addtocounter{MFieldCounter}{1}
}

% Parameter: Laenge des Feldes, Loesung (wird auch geparsed), Stellen Genauigkeit hinter dem Komma, weitere Stellen werden mathematisch gerundet vor Vergleich
\newcommand{\MLParsedQuestion}[4]{
\MQuestionID{#1}{#2}{\MGenerateID}{3}{#3}{\MStdPoints}{#4}
\addtocounter{MFieldCounter}{1}
}

% Parameter: Laenge des Feldes, Loesungsfunktion, Anzahl Stuetzstellen, Funktionsvariablen durch Kommata getrennt (nicht case-sensitive), Anzahl Nachkommastellen im Vergleich
\newcommand{\MFunctionQuestion}[5]{
\MQuestionID{#1}{#2}{\MGenerateID}{4}{#3;#4;#5;0}{\MStdPoints}{#2}
\addtocounter{MFieldCounter}{1}
}

% Parameter: Laenge des Feldes, Loesungsfunktion, Anzahl Stuetzstellen, Funktionsvariablen durch Kommata getrennt (nicht case-sensitive), Anzahl Nachkommastellen im Vergleich, UXID
\newcommand{\MLFunctionQuestion}[6]{
\MQuestionID{#1}{#2}{\MGenerateID}{4}{#3;#4;#5;0}{\MStdPoints}{#6}
\addtocounter{MFieldCounter}{1}
}

% Parameter: Laenge des Feldes, Loesungsintervall, Genauigkeit der Zahlenwertpruefung
\newcommand{\MIntervalQuestion}[3]{
\MQuestionID{#1}{#2}{\MGenerateID}{6}{#3}{\MStdPoints}{#2}
\addtocounter{MFieldCounter}{1}
}

% Parameter: Laenge des Feldes, Loesungsintervall, Genauigkeit der Zahlenwertpruefung, UXID
\newcommand{\MLIntervalQuestion}[4]{
\MQuestionID{#1}{#2}{\MGenerateID}{6}{#3}{\MStdPoints}{#4}
\addtocounter{MFieldCounter}{1}
}

% Parameter: Laenge des Feldes, Loesungsfunktion, Anzahl Stuetzstellen, Funktionsvariable (nicht case-sensitive), Anzahl Nachkommastellen im Vergleich, Vereinfachungsbedingung
% Vereinfachungsbedingung ist eine der Folgenden:
% 0 = Keine Vereinfachungsbedingung
% 1 = Keine Klammern (runde oder eckige) mehr im vereinfachten Ausdruck
% 2 = Faktordarstellung (Term hat Produkte als letzte Operation, Summen als vorgeschaltete Operation)
% 3 = Summendarstellung (Term hat Summen als letzte Operation, Produkte als vorgeschaltete Operation)
% Flag 512: Besondere Stuetzstellen (nur >1 und nur schwach rational), sonst symmetrisch um Nullpunkt und ganze Zahlen inkl. Null werden getroffen
\newcommand{\MSimplifyQuestion}[6]{
\MQuestionID{#1}{#2}{\MGenerateID}{4}{#3;#4;#5;#6}{\MStdPoints}{#2}
\addtocounter{MFieldCounter}{1}
}

\newcommand{\MLSimplifyQuestion}[7]{
\MQuestionID{#1}{#2}{\MGenerateID}{4}{#3;#4;#5;#6}{\MStdPoints}{#7}
\addtocounter{MFieldCounter}{1}
}

% Parameter: Laenge des Feldes, Loesung (optionaler Ausdruck), Anzahl Stuetzstellen, Funktionsvariable (nicht case-sensitive), Anzahl Nachkommastellen im Vergleich, Spezialtyp (string-id)
\newcommand{\MLSpecialQuestion}[7]{
\MQuestionID{#1}{#2}{\MGenerateID}{7}{#3;#4;#5;#6}{\MStdPoints}{#7}
\addtocounter{MFieldCounter}{1}
}

\newcounter{MGroupStart}
\newcounter{MGroupEnd}
\newcounter{MGroupActive}

\newenvironment{MQuestionGroup}{
\setcounter{MGroupStart}{\value{MFieldCounter}}
\setcounter{MGroupActive}{1}
}{
\setcounter{MGroupActive}{0}
\setcounter{MGroupEnd}{\value{MFieldCounter}}
\addtocounter{MGroupEnd}{-1}
}

\newcommand{\MGroupButton}[1]{
\ifttm
\special{html:<button name="Name_Group}\arabic{MGroupStart}\special{html:to}\arabic{MGroupEnd}\special{html:" id="Group}\arabic{MGroupStart}\special{html:to}\arabic{MGroupEnd}\special{html:" %
type="button" onclick="group_button(}\arabic{MGroupStart}\special{html:,}\arabic{MGroupEnd}\special{html:);">}#1\special{html:</button>}
\else
\phantom{#1}
\fi
}

%----------------- Makros fuer die modularisierte Darstellung ------------------------------------

\def\MyText#1{#1}

% is used internally by the conversion package, should not be used by original tex documents
\def\MOrgLabel#1{\relax}

\ifttm

% Ein MLabel wird im html codiert durch das tag <!-- mmlabel;;Labelbezeichner;;SubjectArea;;chapter;;section;;subsection;;Index;;Objekttyp; //-->
\def\MLabel#1{%
\ifnum\value{MLastType}=8%
\ifnum\value{MCaptionOn}=0%
\MDebugMessage{ERROR: Grafik \arabic{MGraphicsCounter} hat separates label: #1 (Grafiklabels sollten nur in der Caption stehen)}%
\fi
\fi
\ifnum\value{MLastType}=12%
\ifnum\value{MCaptionOn}=0%
\MDebugMessage{ERROR: Video \arabic{MVideoCounter} hat separates label: #1 (Videolabels sollten nur in der Caption stehen}%
\fi
\fi
\ifnum\value{MLastType}=10\setcounter{MLastIndex}{\value{equation}}\fi
\label{#1}\begin{html}<!-- mmlabel;;#1;;\end{html}\arabic{MSubjectArea}\special{html:;;}\arabic{chapter}\special{html:;;}\arabic{section}\special{html:;;}\arabic{subsection}\special{html:;;}\arabic{MLastIndex}\special{html:;;}\arabic{MLastType}\special{html:; //-->}}%

\else

% Sonderbehandlung im PDF fuer Abbildungen in separater aux-Datei, da MGraphics die figure-Umgebung nicht verwendet
\def\MLabel#1{%
\ifnum\value{MLastType}=8%
\ifnum\value{MCaptionOn}=0%
\MDebugMessage{ERROR: Grafik \arabic{MGraphicsCounter} hat separates label: #1 (Grafiklabels sollten nur in der Caption stehen}%
\fi
\fi
\ifnum\value{MLastType}=12%
\ifnum\value{MCaptionOn}=0%
\MDebugMessage{ERROR: Video \arabic{MVideoCounter} hat separates label: #1 (Videolabels sollten nur in der Caption stehen}%
\fi
\fi
\label{#1}%
}%

\fi

% Gibt Begriff des referenzierten Objekts mit aus, aber nur im HTML, daher nur in Ausnahmefaellen (z.B. Copyrightliste) sinnvoll
\def\MCRef#1{\ifttm\special{html:<!-- mmref;;}#1\special{html:;;1; //-->}\else\vref{#1}\fi}


\def\MRef#1{\ifttm\special{html:<!-- mmref;;}#1\special{html:;;0; //-->}\else\vref{#1}\fi}
\def\MERef#1{\ifttm\special{html:<!-- mmref;;}#1\special{html:;;0; //-->}\else\eqref{#1}\fi}
\def\MNRef#1{\ifttm\special{html:<!-- mmref;;}#1\special{html:;;0; //-->}\else\ref{#1}\fi}
\def\MSRef#1#2{\ifttm\special{html:<!-- msref;;}#1\special{html:;;}#2\special{html:; //-->}\else \if#2\empty \ref{#1} \else \hyperref[#1]{#2}\fi\fi} 

\def\MRefRange#1#2{\ifttm\MRef{#1} bis 
\MRef{#2}\else\vrefrange[\unskip]{#1}{#2}\fi}

\def\MRefTwo#1#2{\ifttm\MRef{#1} und \MRef{#2}\else%
\let\vRefTLRsav=\reftextlabelrange\let\vRefTPRsav=\reftextpagerange%
\def\reftextlabelrange##1##2{\ref{##1} und~\ref{##2}}%
\def\reftextpagerange##1##2{auf den Seiten~\pageref{#1} und~\pageref{#2}}%
\vrefrange[\unskip]{#1}{#2}%
\let\reftextlabelrange=\vRefTLRsav\let\reftextpagerange=\vRefTPRsav\fi}

% MSectionChapter definiert falls notwendig das Kapitel vor der section. Das ist notwendig, wenn nur ein Einzelmodul uebersetzt wird.
% MChaptersGiven ist ein Counter, der von mconvert.pl vordefiniert wird.
\ifttm
\newcommand{\MSectionChapter}{\ifnum\value{MChaptersGiven}=0{\Dchapter{Modul}}\else{}\fi}
\else
\newcommand{\MSectionChapter}{\ifnum\value{chapter}=0{\Dchapter{Modul}}\else{}\fi}
\fi


\def\MChapter#1{\ifnum\value{MSSEnd}>0{\MSubsectionEndMacros}\addtocounter{MSSEnd}{-1}\fi\Dchapter{#1}}
\def\MSubject#1{\MChapter{#1}} % Schluesselwort HELPSECTION ist reserviert fuer Hilfesektion

\newcommand{\MSectionID}{UNKNOWNID}

\ifttm
\newcommand{\MSetSectionID}[1]{\renewcommand{\MSectionID}{#1}}
\else
\newcommand{\MSetSectionID}[1]{\renewcommand{\MSectionID}{#1}\tikzsetexternalprefix{#1}}
\fi


\newcommand{\MSection}[1]{\MSetSectionID{MODULID}\ifnum\value{MSSEnd}>0{\MSubsectionEndMacros}\addtocounter{MSSEnd}{-1}\fi\MSectionChapter\Dsection{#1}\MSectionStartMacros{#1}\setcounter{MLastIndex}{-1}\setcounter{MLastType}{1}} % Sections werden ueber das section-Feld im mmlabel-Tag identifiziert, nicht ueber das Indexfeld

\def\MSubsection#1{\ifnum\value{MSSEnd}>0{\MSubsectionEndMacros}\addtocounter{MSSEnd}{-1}\fi\ifttm\else\clearpage\fi\Dsubsection{#1}\MSubsectionStartMacros\setcounter{MLastIndex}{-1}\setcounter{MLastType}{2}\addtocounter{MSSEnd}{1}}% Subsections werden ueber das subsection-Feld im mmlabel-Tag identifiziert, nicht ueber das Indexfeld
\def\MSubsectionx#1{\Dsubsectionx{#1}} % Nur zur Verwendung in MSectionStart gedacht
\def\MSubsubsection#1{\Dsubsubsection{#1}\setcounter{MLastIndex}{\value{subsubsection}}\setcounter{MLastType}{3}\ifttm\special{html:<!-- sectioninfo;;}\arabic{section}\special{html:;;}\arabic{subsection}\special{html:;;}\arabic{subsubsection}\special{html:;;1;;}\arabic{MTestSite}\special{html:; //-->}\fi}
\def\MSubsubsectionx#1{\Dsubsubsectionx{#1}\ifttm\special{html:<!-- sectioninfo;;}\arabic{section}\special{html:;;}\arabic{subsection}\special{html:;;}\arabic{subsubsection}\special{html:;;0;;}\arabic{MTestSite}\special{html:; //-->}\else\addcontentsline{toc}{subsection}{#1}\fi}

\ifttm
\def\MSubsubsubsectionx#1{\ \newline\textbf{#1}\special{html:<br />}}
\else
\def\MSubsubsubsectionx#1{\ \newline
\textbf{#1}\ \\
}
\fi


% Dieses Skript wird zu Beginn jedes Modulabschnitts (=Webseite) ausgefuehrt und initialisiert den Aufgabenfeldzaehler
\newcommand{\MPageScripts}{
\setcounter{MFieldCounter}{1}
\addtocounter{MSiteCounter}{1}
\setcounter{MHintCounter}{1}
\setcounter{MCodeEditCounter}{1}
\setcounter{MGroupActive}{0}
\DoQBoxes
% Feldvariablen werden im HTML-Header in conv.pl eingestellt
}

% Dieses Skript wird zum Ende jedes Modulabschnitts (=Webseite) ausgefuehrt
\ifttm
\newcommand{\MEndScripts}{\special{html:<br /><!-- mfeedbackbutton;Seite;}\arabic{MTestSite}\special{html:;}\MGenerateSiteNumber\special{html:; //-->}
}
\else
\newcommand{\MEndScripts}{\relax}
\fi


\newcounter{QBoxFlag}
\newcommand{\DoQBoxes}{\setcounter{QBoxFlag}{1}}
\newcommand{\NoQBoxes}{\setcounter{QBoxFlag}{0}}

\newcounter{MXCTest}
\newcounter{MXCounter}
\newcounter{MSCounter}



\ifttm

% Struktur des sectioninfo-Tags: <!-- sectioninfo;;section;;subsection;;subsubsection;;nr_ausgeben;;testpage; //-->

%Fuegt eine zusaetzliche html-Seite an hinter ALLEN bisherigen und zukuenftigen content-Seiten ausserhalb der vor-zurueck-Schleife (d.h. nur durch Button oder MIntLink erreichbar!)
% #1 = Titel des Modulabschnitts, #2 = Kurztitel fuer die Buttons, #3 = Buttonkennung (STD = default nehmen, NONE = Ohne Button in der Navigation)
\newenvironment{MSContent}[3]{\special{html:<div class="xcontent}\arabic{MSCounter}\special{html:"><!-- scontent;-;}\arabic{MSCounter};-;#1;-;#2;-;#3\special{html: //-->}\MPageScripts\MSubsubsectionx{#1}}{\MEndScripts\special{html:<!-- endscontent;;}\arabic{MSCounter}\special{html: //--></div>}\addtocounter{MSCounter}{1}}

% Fuegt eine zusaetzliche html-Seite ein hinter den bereits vorhandenen content-Seiten (oder als erste Seite) innerhalb der vor-zurueck-Schleife der Navigation
% #1 = Titel des Modulabschnitts, #2 = Kurztitel fuer die Buttons, #3 = Buttonkennung (STD = Defaultbutton, NONE = Ohne Button in der Navigation)
\newenvironment{MXContent}[3]{\special{html:<div class="xcontent}\arabic{MXCounter}\special{html:"><!-- xcontent;-;}\arabic{MXCounter};-;#1;-;#2;-;#3\special{html: //-->}\MPageScripts\MSubsubsection{#1}}{\MEndScripts\special{html:<!-- endxcontent;;}\arabic{MXCounter}\special{html: //--></div>}\addtocounter{MXCounter}{1}}

% Fuegt eine zusaetzliche html-Seite ein die keine subsubsection-Nummer bekommt, nur zur internen Verwendung in mintmod.tex gedacht!
% #1 = Titel des Modulabschnitts, #2 = Kurztitel fuer die Buttons, #3 = Buttonkennung (STD = Defaultbutton, NONE = Ohne Button in der Navigation)
% \newenvironment{MUContent}[3]{\special{html:<div class="xcontent}\arabic{MXCounter}\special{html:"><!-- xcontent;-;}\arabic{MXCounter};-;#1;-;#2;-;#3\special{html: //-->}\MPageScripts\MSubsubsectionx{#1}}{\MEndScripts\special{html:<!-- endxcontent;;}\arabic{MXCounter}\special{html: //--></div>}\addtocounter{MXCounter}{1}}

\newcommand{\MDeclareSiteUXID}[1]{\special{html:<!-- mdeclaresiteuxid;;}#1\special{html:;;}\arabic{chapter}\special{html:;;}\arabic{section}\special{html:;; //-->}}

\else

%\newcommand{\MSubsubsection}[1]{\refstepcounter{subsubsection} \addcontentsline{toc}{subsubsection}{\thesubsubsection. #1}}


% Fuegt eine zusaetzliche html-Seite an hinter den bereits vorhandenen content-Seiten
% #1 = Titel des Modulabschnitts, #2 = Kurztitel fuer die Buttons, #3 = Iconkennung (im PDF wirkungslos)
%\newenvironment{MUContent}[3]{\ifnum\value{MXCTest}>0{\MDebugMessage{ERROR: Geschachtelter SContent}}\fi\MPageScripts\MSubsubsectionx{#1}\addtocounter{MXCTest}{1}}{\addtocounter{MXCounter}{1}\addtocounter{MXCTest}{-1}}
\newenvironment{MXContent}[3]{\ifnum\value{MXCTest}>0{\MDebugMessage{ERROR: Geschachtelter SContent}}\fi\MPageScripts\MSubsubsection{#1}\addtocounter{MXCTest}{1}}{\addtocounter{MXCounter}{1}\addtocounter{MXCTest}{-1}}
\newenvironment{MSContent}[3]{\ifnum\value{MXCTest}>0{\MDebugMessage{ERROR: Geschachtelter XContent}}\fi\MPageScripts\MSubsubsectionx{#1}\addtocounter{MXCTest}{1}}{\addtocounter{MSCounter}{1}\addtocounter{MXCTest}{-1}}

\newcommand{\MDeclareSiteUXID}[1]{\relax}

\fi 

% GHEADER und GFOOTER werden von split.pm gefunden, aber nur, wenn nicht HELPSITE oder TESTSITE
\ifttm
\newenvironment{MSectionStart}{\special{html:<div class="xcontent0">}\MSubsubsectionx{Modul\"ubersicht}}{\setcounter{MSSEnd}{0}\special{html:</div>}}
% Darf nicht als XContent nummeriert werden, darf nicht als XContent gelabelt werden, wird aber in eine xcontent-div gesetzt fuer Python-parsing
\else
\newenvironment{MSectionStart}{\MSubsectionx{Modul\"ubersicht}}{\setcounter{MSSEnd}{0}}
\fi

\newenvironment{MIntro}{\begin{MXContent}{Einf\"uhrung}{Einf\"uhrung}{genetisch}}{\end{MXContent}}
\newenvironment{MContent}{\begin{MXContent}{Inhalt}{Inhalt}{beweis}}{\end{MXContent}}
\newenvironment{MExercises}{\ifttm\else\clearpage\fi\begin{MXContent}{Aufgaben}{Aufgaben}{aufgb}\special{html:<!-- declareexcsymb //-->}}{\end{MXContent}}

% #1 = Lesbare Testbezeichnung
\newenvironment{MTest}[1]{%
\renewcommand{\MTestName}{#1}
\ifttm\else\clearpage\fi%
\addtocounter{MTestSite}{1}%
\begin{MXContent}{#1}{#1}{STD} % {aufgb}%
\special{html:<!-- declaretestsymb //-->}
\begin{MQuestionGroup}%
\MInTestHeader
}%
{%
\end{MQuestionGroup}%
\ \\ \ \\%
\MInTestFooter
\end{MXContent}\addtocounter{MTestSite}{-1}%
}

\newenvironment{MExtra}{\ifttm\else\clearpage\fi\begin{MXContent}{Zus\"atzliche Inhalte}{Zusatz}{weiterfhrg}}{\end{MXContent}}

\makeindex

\ifttm
\def\MPrintIndex{
\ifnum\value{MSSEnd}>0{\MSubsectionEndMacros}\addtocounter{MSSEnd}{-1}\fi
\renewcommand{\indexname}{Stichwortverzeichnis}
\special{html:<p><!-- printindex //--></p>}
}
\else
\def\MPrintIndex{
\ifnum\value{MSSEnd}>0{\MSubsectionEndMacros}\addtocounter{MSSEnd}{-1}\fi
\renewcommand{\indexname}{Stichwortverzeichnis}
\addcontentsline{toc}{section}{Stichwortverzeichnis}
\printindex
}
\fi


% Konstanten fuer die Modulfaecher

\def\MINTMathematics{1}
\def\MINTInformatics{2}
\def\MINTChemistry{3}
\def\MINTPhysics{4}
\def\MINTEngineering{5}

\newcounter{MSubjectArea}
\newcounter{MInfoNumbers} % Gibt an, ob die Infoboxen nummeriert werden sollen
\newcounter{MSepNumbers} % Gibt an, ob Beispiele und Experimente separat nummeriert werden sollen
\newcommand{\MSetSubject}[1]{
 % ttm kapiert setcounter mit Parametern nicht, also per if abragen und einsetzen
\ifnum#1=1\setcounter{MSubjectArea}{1}\setcounter{MInfoNumbers}{1}\setcounter{MSepNumbers}{0}\fi
\ifnum#1=2\setcounter{MSubjectArea}{2}\setcounter{MInfoNumbers}{1}\setcounter{MSepNumbers}{0}\fi
\ifnum#1=3\setcounter{MSubjectArea}{3}\setcounter{MInfoNumbers}{0}\setcounter{MSepNumbers}{1}\fi
\ifnum#1=4\setcounter{MSubjectArea}{4}\setcounter{MInfoNumbers}{0}\setcounter{MSepNumbers}{0}\fi
\ifnum#1=5\setcounter{MSubjectArea}{5}\setcounter{MInfoNumbers}{1}\setcounter{MSepNumbers}{0}\fi
% Separate Nummerntechnik fuer unsere Chemiker: alles dreistellig
\ifnum#1=3
  \ifttm
  \renewcommand{\theequation}{\arabic{section}.\arabic{subsection}.\arabic{equation}}
  \renewcommand{\thetable}{\arabic{section}.\arabic{subsection}.\arabic{table}} 
  \renewcommand{\thefigure}{\arabic{section}.\arabic{subsection}.\arabic{figure}} 
  \else
  \renewcommand{\theequation}{\arabic{chapter}.\arabic{section}.\arabic{equation}}
  \renewcommand{\thetable}{\arabic{chapter}.\arabic{section}.\arabic{table}}
  \renewcommand{\thefigure}{\arabic{chapter}.\arabic{section}.\arabic{figure}}
  \fi
\else
  \ifttm
  \renewcommand{\theequation}{\arabic{section}.\arabic{subsection}.\arabic{equation}}
  \renewcommand{\thetable}{\arabic{table}}
  \renewcommand{\thefigure}{\arabic{figure}}
  \else
  \renewcommand{\theequation}{\arabic{chapter}.\arabic{section}.\arabic{equation}}
  \renewcommand{\thetable}{\arabic{table}}
  \renewcommand{\thefigure}{\arabic{figure}}
  \fi
\fi
}

% Fuer tikz Autogenerierung
\newcounter{MTIKZAutofilenumber}

% Spezielle Counter fuer die Bentz-Module
\newcounter{mycounter}
\newcounter{chemapplet}
\newcounter{physapplet}

\newcounter{MSSEnd} % Ist 1 falls ein MSubsection aktiv ist, der einen MSubsectionEndMacro-Aufruf verursacht
\newcounter{MFileNumber}
\def\MLastFile{\special{html:[[!-- mfileref;;}\arabic{MFileNumber}\special{html:; //--]]}}

% Vollstaendiger Pfad ist \MMaterial / \MLastFilePath / \MLastFileName    ==   \MMaterial / \MLastFile

% Wird nur bei kompletter Baum-Erstellung ausgefuehrt!
% #1 = Lesbare Modulbezeichnung
\newcommand{\MSectionStartMacros}[1]{
\setcounter{MTestSite}{0}
\setcounter{MCaptionOn}{0}
\setcounter{MLastTypeEq}{0}
\setcounter{MSSEnd}{0}
\setcounter{MFileNumber}{0} % Preinkrekement-Counter
\setcounter{MTIKZAutofilenumber}{0}
\setcounter{mycounter}{1}
\setcounter{physapplet}{1}
\setcounter{chemapplet}{0}
\ifttm
\special{html:<!-- mdeclaresection;;}\arabic{chapter}\special{html:;;}\arabic{section}\special{html:;;}#1\special{html:;; //-->}%
\else
\setcounter{thmc}{0}
\setcounter{exmpc}{0}
\setcounter{verc}{0}
\setcounter{infoc}{0}
\fi
\setcounter{MiniMarkerCounter}{1}
\setcounter{AlignCounter}{1}
\setcounter{MXCTest}{0}
\setcounter{MCodeCounter}{0}
\setcounter{MEntryCounter}{0}
}

% Wird immer ausgefuehrt
\newcommand{\MSubsectionStartMacros}{
\ifttm\else\MPageHeaderDef\fi
\MWatermarkSettings
\setcounter{MXCounter}{0}
\setcounter{MSCounter}{0}
\setcounter{MSiteCounter}{1}
\setcounter{MExerciseCollectionCounter}{0}
% Zaehler fuer das Labelsystem zuruecksetzen (prefix-Zaehler)
\setcounter{MInfoCounter}{0}
\setcounter{MExerciseCounter}{0}
\setcounter{MExampleCounter}{0}
\setcounter{MExperimentCounter}{0}
\setcounter{MGraphicsCounter}{0}
\setcounter{MTableCounter}{0}
\setcounter{MTheoremCounter}{0}
\setcounter{MObjectCounter}{0}
\setcounter{MEquationCounter}{0}
\setcounter{MVideoCounter}{0}
\setcounter{equation}{0}
\setcounter{figure}{0}
}

\newcommand{\MSubsectionEndMacros}{
% Bei Chemiemodulen das PSE einhaengen, es soll als SContent am Ende erscheinen
\special{html:<!-- subsectionend //-->}
\ifnum\value{MSubjectArea}=3{\MIncludePSE}\fi
}


\ifttm
%\newcommand{\MEmbed}[1]{\MRegisterFile{#1}\begin{html}<embed src="\end{html}\MMaterial/\MLastFile\begin{html}" width="192" height="189"></embed>\end{html}}
\newcommand{\MEmbed}[1]{\MRegisterFile{#1}\begin{html}<embed src="\end{html}\MMaterial/\MLastFile\begin{html}"></embed>\end{html}}
\fi

%----------------- Makros fuer die Textdarstellung -----------------------------------------------

\ifttm
% MUGraphics bindet eine Grafik ein:
% Parameter 1: Dateiname der Grafik, relativ zur Position des Modul-Tex-Dokuments
% Parameter 2: Skalierungsoptionen fuer PDF (fuer includegraphics)
% Parameter 3: Titel fuer die Grafik, wird unter die Grafik mit der Grafiknummer gesetzt und kann MLabel bzw. MCopyrightLabel enthalten
% Parameter 4: Skalierungsoptionen fuer HTML (css-styles)

% ERSATZ: <img alt="My Image" src="data:image/png;base64,iVBORwA<MoreBase64SringHere>" />


\newcommand{\MUGraphics}[4]{\MRegisterFile{#1}\begin{html}
<div class="imagecenter">
<center>
<div>
<img src="\end{html}\MMaterial/\MLastFile\begin{html}" style="#4" alt="\end{html}\MMaterial/\MLastFile\begin{html}"/>
</div>
<div class="bildtext">
\end{html}
\addtocounter{MGraphicsCounter}{1}
\setcounter{MLastIndex}{\value{MGraphicsCounter}}
\setcounter{MLastType}{8}
\addtocounter{MCaptionOn}{1}
\ifnum\value{MSepNumbers}=0
\textbf{Abbildung \arabic{MGraphicsCounter}:} #3
\else
\textbf{Abbildung \arabic{section}.\arabic{subsection}.\arabic{MGraphicsCounter}:} #3
\fi
\addtocounter{MCaptionOn}{-1}
\begin{html}
</div>
</center>
</div>
<br />
\end{html}%
\special{html:<!-- mfeedbackbutton;Abbildung;}\arabic{MGraphicsCounter}\special{html:;}\arabic{section}.\arabic{subsection}.\arabic{MGraphicsCounter}\special{html:; //-->}%
}

% MVideo bindet ein Video als Einzeldatei ein:
% Parameter 1: Dateiname des Videos, relativ zur Position des Modul-Tex-Dokuments, ohne die Endung ".mp4"
% Parameter 2: Titel fuer das Video (kann MLabel oder MCopyrightLabel enthalten), wird unter das Video mit der Videonummer gesetzt
\newcommand{\MVideo}[2]{\MRegisterFile{#1.mp4}\begin{html}
<div class="imagecenter">
<center>
<div>
<video width="95\%" controls="controls"><source src="\end{html}\MMaterial/#1.mp4\begin{html}" type="video/mp4">Ihr Browser kann keine MP4-Videos abspielen!</video>
</div>
<div class="bildtext">
\end{html}
\addtocounter{MVideoCounter}{1}
\setcounter{MLastIndex}{\value{MVideoCounter}}
\setcounter{MLastType}{12}
\addtocounter{MCaptionOn}{1}
\ifnum\value{MSepNumbers}=0
\textbf{Video \arabic{MVideoCounter}:} #2
\else
\textbf{Video \arabic{section}.\arabic{subsection}.\arabic{MVideoCounter}:} #2
\fi
\addtocounter{MCaptionOn}{-1}
\begin{html}
</div>
</center>
</div>
<br />
\end{html}}

\newcommand{\MDVideo}[2]{\MRegisterFile{#1.mp4}\MRegisterFile{#1.ogv}\begin{html}
<div class="imagecenter">
<center>
<div>
<video width="70\%" controls><source src="\end{html}\MMaterial/#1.mp4\begin{html}" type="video/mp4"><source src="\end{html}\MMaterial/#1.ogv\begin{html}" type="video/ogg">Ihr Browser kann keine MP4-Videos abspielen!</video>
</div>
<br />
#2
</center>
</div>
<br />
\end{html}
}

\newcommand{\MGraphics}[3]{\MUGraphics{#1}{#2}{#3}{}}

\else

\newcommand{\MVideo}[2]{%
% Kein Video im PDF darstellbar, trotzdem so tun als ob da eines waere
\begin{center}
(Video nicht darstellbar)
\end{center}
\addtocounter{MVideoCounter}{1}
\setcounter{MLastIndex}{\value{MVideoCounter}}
\setcounter{MLastType}{12}
\addtocounter{MCaptionOn}{1}
\ifnum\value{MSepNumbers}=0
\textbf{Video \arabic{MVideoCounter}:} #2
\else
\textbf{Video \arabic{section}.\arabic{subsection}.\arabic{MVideoCounter}:} #2
\fi
\addtocounter{MCaptionOn}{-1}
}


% MGraphics bindet eine Grafik ein:
% Parameter 1: Dateiname der Grafik, relativ zur Position des Modul-Tex-Dokuments
% Parameter 2: Skalierungsoptionen fuer PDF (fuer includegraphics)
% Parameter 3: Titel fuer die Grafik, wird unter die Grafik mit der Grafiknummer gesetzt
\newcommand{\MGraphics}[3]{%
\MRegisterFile{#1}%
\ %
\begin{figure}[H]%
\centering{%
\includegraphics[#2]{\MDPrefix/#1}%
\addtocounter{MCaptionOn}{1}%
\caption{#3}%
\addtocounter{MCaptionOn}{-1}%
}%
\end{figure}%
\addtocounter{MGraphicsCounter}{1}\setcounter{MLastIndex}{\value{MGraphicsCounter}}\setcounter{MLastType}{8}\ %
%\ \\Abbildung \ifnum\value{MSepNumbers}=0\else\arabic{chapter}.\arabic{section}.\fi\arabic{MGraphicsCounter}: #3%
}

\newcommand{\MUGraphics}[4]{\MGraphics{#1}{#2}{#3}}


\fi

\newcounter{MCaptionOn} % = 1 falls eine Grafikcaption aktiv ist, = 0 sonst


% MGraphicsSolo bindet eine Grafik pur ein ohne Titel
% Parameter 1: Dateiname der Grafik, relativ zur Position des Modul-Tex-Dokuments
% Parameter 2: Skalierungsoptionen (wirken nur im PDF)
\newcommand{\MGraphicsSolo}[2]{\MUGraphicsSolo{#1}{#2}{}}

% MUGraphicsSolo bindet eine Grafik pur ein ohne Titel, aber mit HTML-Skalierung
% Parameter 1: Dateiname der Grafik, relativ zur Position des Modul-Tex-Dokuments
% Parameter 2: Skalierungsoptionen (wirken nur im PDF)
% Parameter 3: Skalierungsoptionen (wirken nur im HTML), als style-format: "width=???, height=???"
\ifttm
\newcommand{\MUGraphicsSolo}[3]{\MRegisterFile{#1}\begin{html}
<img src="\end{html}\MMaterial/\MLastFile\begin{html}" style="\end{html}#3\begin{html}" alt="\end{html}\MMaterial/\MLastFile\begin{html}"/>
\end{html}%
\special{html:<!-- mfeedbackbutton;Abbildung;}#1\special{html:;}\MMaterial/\MLastFile\special{html:; //-->}%
}
\else
\newcommand{\MUGraphicsSolo}[3]{\MRegisterFile{#1}\includegraphics[#2]{\MDPrefix/#1}}
\fi

% Externer Link mit URL
% Erster Parameter: Vollstaendige(!) URL des Links
% Zweiter Parameter: Text fuer den Link
\newcommand{\MExtLink}[2]{\ifttm\special{html:<a target="_new" href="}#1\special{html:">}#2\special{html:</a>}\else\href{#1}{#2}\fi} % ohne MINTERLINK!


% Interner Link, die verlinkte Datei muss im gleichen Verzeichnis liegen wie die Modul-Texdatei
% Erster Parameter: Dateiname
% Zweiter Parameter: Text fuer den Link
\newcommand{\MIntLink}[2]{\ifttm\MRegisterFile{#1}\special{html:<a class="MINTERLINK" target="_new" href="}\MMaterial/\MLastFile\special{html:">}#2\special{html:</a>}\else{\href{#1}{#2}}\fi}


\ifttm
\def\MMaterial{:localmaterial:}
\else
\def\MMaterial{\MDPrefix}
\fi

\ifttm
\def\MNoFile#1{:directmaterial:#1}
\else
\def\MNoFile#1{#1}
\fi

\newcommand{\MChem}[1]{$\mathrm{#1}$}

\newcommand{\MApplet}[3]{
% Bindet ein Java-Applet ein, die Parameter sind:
% (wird nur im HTML, aber nicht im PDF erstellt)
% #1 Dateiname des Applets (muss mit ".class" enden)
% #2 = Breite in Pixeln
% #3 = Hoehe in Pixeln
\ifttm
\MRegisterFile{#1}
\begin{html}
<applet code="\end{html}\MMaterial/\MLastFile\begin{html}" width="#2" height="#3" alt="[Java-Applet kann nicht gestartet werden]"></applet>
\end{html}
\fi
}

\newcommand{\MScriptPage}[2]{
% Bindet eine JavaScript-Datei ein, die eine eigene Seite bekommt
% (wird nur im HTML, aber nicht im PDF erstellt)
% #1 Dateiname des Programms (sollte mit ".js" enden)
% #2 = Kurztitel der Seite
\ifttm
\begin{MSContent}{#2}{#2}{puzzle}
\MRegisterFile{#1}
\begin{html}
<script src="\MMaterial/\MLastFile" type="text/javascript"></script>
\end{html}
\end{MSContent}
\fi
}

\newcommand{\MIncludePSE}{
% Bindet bei Chemie-Modulen das PSE ein
% (wird nur im HTML, aber nicht im PDF erstellt)
\ifttm
\special{html:<!-- includepse //-->}
\begin{MSContent}{Periodensystem der Elemente}{PSE}{table}
\MRegisterFile{../files/pse.js}
\MRegisterFile{../files/radio.png}
\begin{html}
<script src="\MMaterial/../files/pse.js" type="text/javascript"></script>
<p id="divid"><br /><br />
<script language="javascript" type="text/javascript">
    startpse("divid","\MMaterial/../files"); 
</script>
</p>
<br />
<br />
<br />
<p>Die Farben der Elementsymbole geben an: <font style="color:Red">gasf&ouml;rmig </font> <font style="color:Blue">fl&uuml;ssig </font> fest</p>
<p>Die Elemente der Gruppe 1 A, 2 A, 3 A usw. geh&ouml;ren zu den Hauptgruppenelementen.</p>
<p>Die Elemente der Gruppe 1 B, 2 B, 3 B usw. geh&ouml;ren zu den Nebengruppenelementen.</p>
<p>() kennzeichnet die Masse des stabilsten Isotops</p>
\end{html}
\end{MSContent}
\fi
}

\newcommand{\MAppletArchive}[4]{
% Bindet ein Java-Applet ein, die Parameter sind:
% (wird nur im HTML, aber nicht im PDF erstellt)
% #1 Dateiname der Klasse mit Appletaufruf (muss mit ".class" enden)
% #2 Dateiname des Archivs (muss mit ".jar" enden)
% #3 = Breite in Pixeln
% #4 = Hoehe in Pixeln
\ifttm
\MRegisterFile{#2}
\begin{html}
<applet code="#1" archive="\end{html}\MMaterial/\MLastFile\begin{html}" codebase="." width="#3" height="#4" alt="[Java-Archiv kann nicht gestartet werden]"></applet>
\end{html}
\fi
}

% Bindet in der Haupttexdatei ein MINT-Modul ein. Parameter 1 ist das Verzeichnis (relativ zur Haupttexdatei), Parameter 2 ist der Dateinahme ohne Pfad.
\newcommand{\IncludeModule}[2]{
\renewcommand{\MDPrefix}{#1}
\input{#1/#2}
\ifnum\value{MSSEnd}>0{\MSubsectionEndMacros}\addtocounter{MSSEnd}{-1}\fi
}

% Der ttm-Konverter setzt keine Makros im \input um, also muss hier getrickst werden:
% Das MDPrefix muss in den einzelnen Modulen manuell eingesetzt werden
\newcommand{\MInputFile}[1]{
\ifttm
\input{#1}
\else
\input{#1}
\fi
}


\newcommand{\MCases}[1]{\left\lbrace{\begin{array}{rl} #1 \end{array}}\right.}

\ifttm
\newenvironment{MCaseEnv}{\left\lbrace\begin{array}{rl}}{\end{array}\right.}
\else
\newenvironment{MCaseEnv}{\left\lbrace\begin{array}{rl}}{\end{array}\right.}
\fi

\def\MSkip{\ifttm\MCR\fi}

\ifttm
\def\MCR{\special{html:<br />}}
\else
\def\MCR{\ \\}
\fi


% Pragmas - Sind Schluesselwoerter, die dem Preprocessing sowie dem Konverter uebergeben werden und bestimmte
%           Aktionen ausloesen. Im Output (PDF und HTML) tauchen sie nicht auf.
\newcommand{\MPragma}[1]{%
\ifttm%
\special{html:<!-- mpragma;-;}#1\special{html:;; -->}%
\else%
% MPragmas werden vom Preprozessor direkt im LaTeX gefunden
\fi%
}

% Ersatz der Befehle textsubscript und textsuperscript, die ttm nicht kennt
\ifttm%
\newcommand{\MTextsubscript}[1]{\special{html:<sub>}#1\special{html:</sub>}}%
\newcommand{\MTextsuperscript}[1]{\special{html:<sup>}#1\special{html:</sup>}}%
\else%
\newcommand{\MTextsubscript}[1]{\textsubscript{#1}}%
\newcommand{\MTextsuperscript}[1]{\textsuperscript{#1}}%
\fi

%------------------ Einbindung von dia-Diagrammen ----------------------------------------------
% Beim preprocessing wird aus jeder dia-Datei eine tex-Datei und eine pdf-Datei erzeugt,
% diese werden hier jeweils im PDF und HTML eingebunden
% Parameter: Dateiname der mit dia erstellten Datei (OHNE die Endung .dia)
\ifttm%
\newcommand{\MDia}[1]{%
\MGraphicsSolo{#1minthtml.png}{}%
}
\else%
\newcommand{\MDia}[1]{%
\MGraphicsSolo{#1mintpdf.png}{scale=0.1667}%
}
\fi%

% subsup funktioniert im Ausdruck $D={\R}^+_0$, also \R geklammert und sup zuerst
% \ifttm
% \def\MSubsup#1#2#3{\special{html:<msubsup>} #1 #2 #3\special{html:</msubsup>}}
% \else
% \def\MSubsup#1#2#3{{#1}^{#3}_{#2}}
% \fi

%\input{local.tex}

% \ifttm
% \else
% \newwrite\mintlog
% \immediate\openout\mintlog=mintlog.txt
% \fi

% ----------------------- tikz autogenerator -------------------------------------------------------------------

\newcommand{\Mtikzexternalize}{\tikzexternalize}% wird bei Konvertierung ueber mconvert ggf. ausgehebelt!

\ifttm
\else
\tikzset%
{
  % Defines a custom style which generates pdf and converts to (low and hi-res quality) png and svg, then deletes the pdf
  % Important: DO NOT directly convert from pdf to hires-png or from svg to png with GraphViz convert as it has some problems and memory leaks
  png export/.style=%
  {
    external/system call/.add={}{; 
      pdf2svg "\image.pdf" "\image.svg" ; 
      convert -density 112.5 -transparent white "\image.pdf" "\image.png"; 
      inkscape --export-png="\image.4x.png" --export-dpi=450 --export-background-opacity=0 --without-gui "\image.svg"; 
      rm "\image.pdf"; rm "\image.log"; rm "\image.dpth"; rm "\image.idx"
    },
    external/force remake,
  }
}
\tikzset{png export}
\tikzsetexternalprefix{}
% PNGs bei externer Erzeugung in "richtiger" Groesse einbinden
\pgfkeys{/pgf/images/include external/.code={\includegraphics[scale=0.64]{#1}}}
\fi

% Spezielle Umgebung fuer Autogenerierung, Bildernamen sind nur innerhalb eines Moduls (einer MSection) eindeutig)

\newcommand{\MTIKZautofilename}{tikzautofile}

\ifttm
% HTML-Version: Vom Autogenerator erzeugte png-Datei einbinden, tikz selbst nicht ausfuehren (sprich: #1 schlucken)
\newcommand{\MTikzAuto}[1]{%
\addtocounter{MTIKZAutofilenumber}{1}
\renewcommand{\MTIKZautofilename}{mtikzauto_\arabic{MTIKZAutofilenumber}}
\MUGraphicsSolo{\MSectionID\MTIKZautofilename.4x.png}{scale=1}{\special{html:[[!-- svgstyle;}\MSectionID\MTIKZautofilename\special{html: //--]]}} % Styleinfos werden aus original-png, nicht 4x-png geholt!
%\MRegisterFile{\MSectionID\MTIKZautofilename.png} % not used right now
%\MRegisterFile{\MSectionID\MTIKZautofilename.svg}
}
\else%
% PDF-Version: Falls Autogenerator aktiv wird Datei automatisch benannt und exportiert
\newcommand{\MTikzAuto}[1]{%
\addtocounter{MTIKZAutofilenumber}{1}%
\renewcommand{\MTIKZautofilename}{mtikzauto_\arabic{MTIKZAutofilenumber}}
\tikzsetnextfilename{\MTIKZautofilename}%
#1%
}
\fi

% In einer reinen LaTeX-Uebersetzung kapselt der Preambelinclude-Befehl nur input,
% in einer konvertergesteuerten PDF/HTML-Uebersetzung wird er dagegen entfernt und
% die Preambeln an mintmod angehaengt, die Ersetzung wird von mconvert.pl vorgenommen.

\newcommand{\MPreambleInclude}[1]{\input{#1}}

% Globale Watermarksettings (werden auch nochmal zu Beginn jedes subsection gesetzt,
% muessen hier aber auch global ausgefuehrt wegen Einfuehrungsseiten und Inhaltsverzeichnis

\MWatermarkSettings
% ---------------------------------- Parametrisierte Aufgaben ----------------------------------------

\ifttm
\newenvironment{MPExercise}{%
\begin{MExercise}%
}{%
\special{html:<button name="Name_MPEX}\arabic{MExerciseCounter}\special{html:" id="MPEX}\arabic{MExerciseCounter}%
\special{html:" type="button" onclick="reroll('}\arabic{MExerciseCounter}\special{html:');">Neue Aufgabe erzeugen</button>}%
\end{MExercise}%
}
\else
\newenvironment{MPExercise}{%
\begin{MExercise}%
}{%
\end{MExercise}%
}
\fi

% Parameter: Name, Min, Max, PDF-Standard. Name in Deklaration OHNE backslash, im Code MIT Backslash
\ifttm
\newcommand{\MGlobalInteger}[4]{\special{html:%
<!-- onloadstart //-->%
MVAR.push(createGlobalInteger("}#1\special{html:",}#2\special{html:,}#3\special{html:,}#4\special{html:)); %
<!-- onloadstop //-->%
<!-- viewmodelstart //-->%
ob}#1\special{html:: ko.observable(rerollMVar("}#1\special{html:")),%
<!-- viewmodelstop //-->%
}%
}%
\else%
\newcommand{\MGlobalInteger}[4]{\newcounter{mvc_#1}\setcounter{mvc_#1}{#4}}
\fi

% Parameter: Name, Min, Max, PDF-Standard. Name in Deklaration OHNE backslash, im Code MIT Backslash, Wert ist Wurzel von value
\ifttm
\newcommand{\MGlobalSqrt}[4]{\special{html:%
<!-- onloadstart //-->%
MVAR.push(createGlobalSqrt("}#1\special{html:",}#2\special{html:,}#3\special{html:,}#4\special{html:)); %
<!-- onloadstop //-->%
<!-- viewmodelstart //-->%
ob}#1\special{html:: ko.observable(rerollMVar("}#1\special{html:")),%
<!-- viewmodelstop //-->%
}%
}%
\else%
\newcommand{\MGlobalSqrt}[4]{\newcounter{mvc_#1}\setcounter{mvc_#1}{#4}}% Funktioniert nicht als Wurzel !!!
\fi

% Parameter: Name, Min, Max, PDF-Standard zaehler, PDF-Standard nenner. Name in Deklaration OHNE backslash, im Code MIT Backslash
\ifttm
\newcommand{\MGlobalFraction}[5]{\special{html:%
<!-- onloadstart //-->%
MVAR.push(createGlobalFraction("}#1\special{html:",}#2\special{html:,}#3\special{html:,}#4\special{html:,}#5\special{html:)); %
<!-- onloadstop //-->%
<!-- viewmodelstart //-->%
ob}#1\special{html:: ko.observable(rerollMVar("}#1\special{html:")),%
<!-- viewmodelstop //-->%
}%
}%
\else%
\newcommand{\MGlobalFraction}[5]{\newcounter{mvc_#1}\setcounter{mvc_#1}{#4}} % Funktioniert nicht als Bruch !!!
\fi

% MVar darf im HTML nur in MEvalMathDisplay-Umgebungen genutzt werden oder in Strings die an den Parser uebergeben werden
\ifttm%
\newcommand{\MVar}[1]{\special{html:[var_}#1\special{html:]}}%
\else%
\newcommand{\MVar}[1]{\arabic{mvc_#1}}%
\fi

\ifttm%
\newcommand{\MRerollButton}[2]{\special{html:<button type="button" onclick="rerollMVar('}#1\special{html:');">}#2\special{html:</button>}}%
\else%
\newcommand{\MRerollButton}[2]{\relax}% Keine sinnvolle Entsprechung im PDF
\fi

% MEvalMathDisplay fuer HTML wird in mconvert.pl im preprocessing realisiert
% PDF: eine equation*-Umgebung (ueber amsmath)
% HTML: Eine Mathjax-Tex-Umgebung, deren Auswertung mit knockout-obervablen gekoppelt ist
% PDF-Version hier nur fuer pdflatex-only-Uebersetzung gegeben

\ifttm\else\newenvironment{MEvalMathDisplay}{\begin{equation*}}{\end{equation*}}\fi

% ---------------------------------- Spezialbefehle fuer AD ------------------------------------------

%Abk�rzung f�r \longrightarrow:
\newcommand{\lto}{\ensuremath{\longrightarrow}}

%Makro f�r Funktionen:
\newcommand{\exfunction}[5]
{\begin{array}{rrcl}
 #1 \colon  & #2 &\lto & #3 \\[.05cm]  
  & #4 &\longmapsto  & #5 
\end{array}}

\newcommand{\function}[5]{%
#1:\;\left\lbrace{\begin{array}{rcl}
 #2 &\lto & #3 \\
 #4 &\longmapsto  & #5 \end{array}}\right.}


%Die Identit�t:
\DeclareMathOperator{\Id}{Id}

%Die Signumfunktion:
\DeclareMathOperator{\sgn}{sgn}

%Zwei Betonungskommandos (k�nnen angepasst werden):
\newcommand{\highlight}[1]{#1}
\newcommand{\modstextbf}[1]{#1}
\newcommand{\modsemph}[1]{#1}


% ---------------------------------- Spezialbefehle fuer JL ------------------------------------------


\def\jccolorfkt{green!50!black} %Farbe des Funktionsgraphen
\def\jccolorfktarea{green!25!white} %Farbe der Fl"ache unter dem Graphen
\def\jccolorfktareahell{green!12!white} %helle Einf"arbung der Fl"ache unter dem Graphen
\def\jccolorfktwert{green!50!black} %Farbe einzelner Punkte des Graphen

\newcommand{\MPfadBilder}{Bilder}

\ifttm%
\newcommand{\jMD}{\,\MD}%
\else%
\newcommand{\jMD}{\;\MD}%
\fi%

\def\jHTMLHinweisBedienung{\MInputHint{%
Mit Hilfe der Symbole am oberen Rand des Fensters
k"onnen Sie durch die einzelnen Abschnitte navigieren.}}

\def\jHTMLHinweisEingabeText{\MInputHint{%
Geben Sie jeweils ein Wort oder Zeichen als Antwort ein.}}

\def\jHTMLHinweisEingabeTerm{\MInputHint{%
Klammern Sie Ihre Terme, um eine eindeutige Eingabe zu erhalten. 
Beispiel: Der Term $\frac{3x+1}{x-2}$ soll in der Form
\texttt{(3*x+1)/((x+2)^2}$ eingegeben werden (wobei auch Leerzeichen 
eingegeben werden k"onnen, damit eine Formel besser lesbar ist).}}

\def\jHTMLHinweisEingabeIntervalle{\MInputHint{%
Intervalle werden links mit einer "offnenden Klammer und rechts mit einer 
schlie"senden Klammer angegeben. Eine runde Klammer wird verwendet, wenn der 
Rand nicht dazu geh"ort, eine eckige, wenn er dazu geh"ort. 
Als Trennzeichen wird ein Komma oder ein Semikolon akzeptiert.
Beispiele: $(a, b)$ offenes Intervall,
$[a; b)$ links abgeschlossenes, rechts offenes Intervall von $a$ bis $b$. 
Die Eingabe $]a;b[$ f"ur ein offenes Intervall wird nicht akzeptiert.
F"ur $\infty$ kann \texttt{infty} oder \texttt{unendlich} geschrieben werden.}}

\def\jHTMLHinweisEingabeFunktionen{\MInputHint{%
Schreiben Sie Malpunkte (geschrieben als \texttt{*}) aus und setzen Sie Klammern um Argumente f�r Funktionen.
Beispiele: Polynom: \texttt{3*x + 0.1}, Sinusfunktion: \texttt{sin(x)}, 
Verkettung von cos und Wurzel: \texttt{cos(sqrt(3*x))}.}}

\def\jHTMLHinweisEingabeFunktionenSinCos{\MInputHint{%
Die Sinusfunktion $\sin x$ wird in der Form \texttt{sin(x)} angegeben, %
$\cos\left(\sqrt{3 x}\right)$ durch \texttt{cos(sqrt(3*x))}.}}

\def\jHTMLHinweisEingabeFunktionenExp{\MInputHint{%
Die Exponentialfunktion $\MEU^{3x^4 + 5}$ wird als
\texttt{exp(3 * x^4 + 5)} angegeben, %
$\ln\left(\sqrt{x} + 3.2\right)$ durch \texttt{ln(sqrt(x) + 3.2)}.}}

% ---------------------------------- Spezialbefehle fuer Fachbereich Physik --------------------------

\newcommand{\E}{{e}}
\newcommand{\ME}[1]{\cdot 10^{#1}}
\newcommand{\MU}[1]{\;\mathrm{#1}}
\newcommand{\MPG}[3]{%
  \ifnum#2=0%
    #1\ \mathrm{#3}%
  \else%
    #1\cdot 10^{#2}\ \mathrm{#3}%
  \fi}%
%

\newcommand{\MMul}{\MExponentensymbXYZl} % Nur eine Abkuerzung


% ---------------------------------- Stichwortfunktionialitaet ---------------------------------------

% mpreindexentry wird durch Auswahlroutine in conv.pl durch mindexentry substitutiert
\ifttm%
\def\MIndex#1{\index{#1}\special{html:<!-- mpreindexentry;;}#1\special{html:;;}\arabic{MSubjectArea}\special{html:;;}%
\arabic{chapter}\special{html:;;}\arabic{section}\special{html:;;}\arabic{subsection}\special{html:;;}\arabic{MEntryCounter}\special{html:; //-->}%
\setcounter{MLastIndex}{\value{MEntryCounter}}%
\addtocounter{MEntryCounter}{1}%
}%
% Copyrightliste wird als tex-Datei im preprocessing von conv.pl erzeugt und unter converter/tex/entrycollection.tex abgelegt
% Der input-Befehl funktioniert nur, wenn die aufrufende tex-Datei auf der obersten Ebene liegt (d.h. selbst kein input/include ist, insbesondere keine Moduldatei)
\def\MEntryList{} % \input funktioniert nicht, weil ttm (und damit das \input) ausgefuehrt wird, bevor Datei da ist
\else%
\def\MIndex#1{\index{#1}}
\def\MEntryList{\MAbort{Stichwortliste nur im HTML realisierbar}}%
\fi%

\def\MEntry#1#2{\textbf{#1}\MIndex{#2}} % Idee: MLastType auf neuen Entry-Typ und dann ein MLabel vergeben mit autogen-Nummer

% ---------------------------------- Befehle fuer Tests ----------------------------------------------

% MEquationItem stellt eine Eingabezeile der Form Vorgabe = Antwortfeld her, der zweite Parameter kann z.B. MSimplifyQuestion-Befehl sein
\ifttm
\newcommand{\MEquationItem}[2]{{#1}$\,=\,${#2}}%
\else%
\newcommand{\MEquationItem}[2]{{#1}$\;\;=\,${#2}}%
\fi

\ifttm
\newcommand{\MInputHint}[1]{%
\ifnum%
\if\value{MTestSite}>0%
\else%
{\color{blue}#1}%
\fi%
\fi%
}
\else
\newcommand{\MInputHint}[1]{\relax}
\fi

\ifttm
\newcommand{\MInTestHeader}{%
Dies ist ein einreichbarer Test:
\begin{itemize}
\item{Im Gegensatz zu den offenen Aufgaben werden beim Eingeben keine Hinweise zur Formulierung der mathematischen Ausdr�cke gegeben.}
\item{Der Test kann jederzeit neu gestartet oder verlassen werden.}
\item{Der Test kann durch die Buttons am Ende der Seite beendet und abgeschickt, oder zur�ckgesetzt werden.}
\item{Der Test kann mehrfach probiert werden. F�r die Statistik z�hlt die zuletzt abgeschickte Version.}
\end{itemize}
}
\else
\newcommand{\MInTestHeader}{%
\relax
}
\fi

\ifttm
\newcommand{\MInTestFooter}{%
\special{html:<button name="Name_TESTFINISH" id="TESTFINISH" type="button" onclick="finish_button('}\MTestName\special{html:');">Test auswerten</button>}%
\begin{html}
&nbsp;&nbsp;&nbsp;&nbsp;&nbsp;&nbsp;&nbsp;&nbsp;
<button name="Name_TESTRESET" id="TESTRESET" type="button" onclick="reset_button();">Test zur�cksetzen</button>
<br />
<br />
<div class="xreply">
<p name="Name_TESTEVAL" id="TESTEVAL">
Hier erscheint die Testauswertung!
<br />
</p>
</div>
\end{html}
}
\else
\newcommand{\MInTestFooter}{%
\relax
}
\fi


% ---------------------------------- Notationsmakros -------------------------------------------------------------

% Notationsmakros die nicht von der Kursvariante abhaengig sind

\newcommand{\MZahltrennzeichen}[1]{\renewcommand{\MZXYZhltrennzeichen}{#1}}

\ifttm
\newcommand{\MZahl}[3][\MZXYZhltrennzeichen]{\edef\MZXYZtemp{\noexpand\special{html:<mn>#2#1#3</mn>}}\MZXYZtemp}
\else
\newcommand{\MZahl}[3][\MZXYZhltrennzeichen]{{}#2{#1}#3}
\fi

\newcommand{\MEinheitenabstand}[1]{\renewcommand{\MEinheitenabstXYZnd}{#1}}
\ifttm
\newcommand{\MEinheit}[2][\MEinheitenabstXYZnd]{{}#1\edef\MEINHtemp{\noexpand\special{html:<mi mathvariant="normal">#2</mi>}}\MEINHtemp} 
\else
\newcommand{\MEinheit}[2][\MEinheitenabstXYZnd]{{}#1 \mathrm{#2}} 
\fi

\newcommand{\MExponentensymbol}[1]{\renewcommand{\MExponentensymbXYZl}{#1}}
\newcommand{\MExponent}[2][\MExponentensymbXYZl]{{}#1{} 10^{#2}} 

%Punkte in 2 und 3 Dimensionen
\newcommand{\MPointTwo}[3][]{#1(#2\MCoordPointSep #3{}#1)}
\newcommand{\MPointThree}[4][]{#1(#2\MCoordPointSep #3\MCoordPointSep #4{}#1)}
\newcommand{\MPointTwoAS}[2]{\left(#1\MCoordPointSep #2\right)}
\newcommand{\MPointThreeAS}[3]{\left(#1\MCoordPointSep #2\MCoordPointSep #3\right)}

% Masseinheit, Standardabstand: \,
\newcommand{\MEinheitenabstXYZnd}{\MThinspace} 

% Horizontaler Leerraum zwischen herausgestellter Formel und Interpunktion
\ifttm
\newcommand{\MDFPSpace}{\,}
\newcommand{\MDFPaSpace}{\,\,}
\newcommand{\MBlank}{\ }
\else
\newcommand{\MDFPSpace}{\;}
\newcommand{\MDFPaSpace}{\;\;}
\newcommand{\MBlank}{\ }
\fi

% Satzende in herausgestellter Formel mit horizontalem Leerraum
\newcommand{\MDFPeriod}{\MDFPSpace .}

% Separation von Aufzaehlung und Bedingung in Menge
\newcommand{\MCondSetSep}{\,:\,} %oder '\mid'

% Konverter kennt mathopen nicht
\ifttm
\def\mathopen#1{}
\fi

% -----------------------------------START Rouletteaufgaben ------------------------------------------------------------

\ifttm
% #1 = Dateiname, #2 = eindeutige ID fuer das Roulette im Kurs
\newcommand{\MDirectRouletteExercises}[2]{
\begin{MExercise}
\texttt{Im HTML erscheinen hier Aufgaben aus einer Aufgabenliste...}
\end{MExercise}
}
\else
\newcommand{\MDirectRouletteExercises}[2]{\relax} % wird durch mconvert.pl gefunden und ersetzt
\fi


% ---------------------------------- START Makros, die von der Kursvariante abhaengen ----------------------------------

\ifvariantunotation
  % unotation = An Universitaeten uebliche Notation
  \def\MVariant{unotation}

  % Trennzeichen fuer Dezimalzahlen
  \newcommand{\MZXYZhltrennzeichen}{.}

  % Exponent zur Basis 10 in der Exponentialschreibweise, 
  % Standardmalzeichen: \times
  \newcommand{\MExponentensymbXYZl}{\times} 

  % Begrenzungszeichen fuer offene Intervalle
  \newcommand{\MoIl}[1][]{\mbox{}#1(\mathopen{}} % bzw. ']'
  \newcommand{\MoIr}[1][]{#1)\mbox{}} % bzw. '['

  % Zahlen-Separation im IntervaLL
  \newcommand{\MIntvlSep}{,} %oder ';'

  % Separation von Elementen in Mengen
  \newcommand{\MElSetSep}{,} %oder ';'

  % Separation von Koordinaten in Punkten
  \newcommand{\MCoordPointSep}{,} %oder ';' oder '|', '\MThinspace|\MThinspace'

\else
  % An dieser Stelle wird angenommen, dass std-Variante aktiv ist
  % std = beschlossene Notation im TU9-Onlinekurs 
  \def\MVariant{std}

  % Trennzeichen fuer Dezimalzahlen
  \newcommand{\MZXYZhltrennzeichen}{,}

  % Exponent zur Basis 10 in der Exponentialschreibweise, 
  % Standardmalzeichen: \times
  \newcommand{\MExponentensymbXYZl}{\times} 

  % Begrenzungszeichen fuer offene Intervalle
  \newcommand{\MoIl}[1][]{\mbox{}#1]\mathopen{}} % bzw. '('
  \newcommand{\MoIr}[1][]{#1[\mbox{}} % bzw. ')'

  % Zahlen-Separation im IntervaLL
  \newcommand{\MIntvlSep}{;} %oder ','
  
  % Separation von Elementen in Mengen
  \newcommand{\MElSetSep}{;} %oder ','

  % Separation von Koordinaten in Punkten
  \newcommand{\MCoordPointSep}{;} %oder '|', '\MThinspace|\MThinspace'

\fi



% ---------------------------------- ENDE Makros, die von der Kursvariante abhaengen ----------------------------------


% diese Kommandos setzen Mathemodus vorraus
\newcommand{\MGeoAbstand}[2]{[\overline{{#1}{#2}}]}
\newcommand{\MGeoGerade}[2]{{#1}{#2}}
\newcommand{\MGeoStrecke}[2]{\overline{{#1}{#2}}}
\newcommand{\MGeoDreieck}[3]{{#1}{#2}{#3}}

%
\ifttm
\newcommand{\MOhm}{\special{html:<mn>&#x3A9;</mn>}}
\else
\newcommand{\MOhm}{\Omega} %\varOmega
\fi


\def\PERCTAG{\MAbort{PERCTAG ist zur internen verwendung in mconvert.pl reserviert, dieses Makro darf sonst nicht benutzt werden.}}

% Im Gegensatz zu einfachen html-Umgebungen werden MDirectHTML-Umgebungen von mconvert.pl am ganzen ttm-Prozess vorbeigeschleust und aus dem PDF komplett ausgeschnitten
\ifttm%
\newenvironment{MDirectHTML}{\begin{html}}{\end{html}}%
\else%
\newenvironment{MDirectHTML}{\begin{html}}{\end{html}}%
\fi

% Im Gegensatz zu einfachen Mathe-Umgebungen werden MDirectMath-Umgebungen von mconvert.pl am ganzen ttm-Prozess vorbeigeschleust, ueber MathJax realisiert, und im PDF als $$ ... $$ gesetzt
\ifttm%
\newenvironment{MDirectMath}{\begin{html}}{\end{html}}%
\else%
\newenvironment{MDirectMath}{\begin{equation*}}{\end{equation*}}% Vorsicht, auch \[ und \] werden in amsmath durch equation* redefiniert
\fi

% ---------------------------------- Location Management ---------------------------------------------

% #1 = buttonname (muss in files/images liegen und Format 48x48 haben), #2 = Vollstaendiger Einrichtungsname, #3 = Kuerzel der Einrichtung,  #4 = Name der include-texdatei
\ifttm
\newcommand{\MLocationSite}[3]{\special{html:<!-- mlocation;;}#1\special{html:;;}#2\special{html:;;}#3\special{html:;; //-->}}
\else
\newcommand{\MLocationSite}[3]{\relax}
\fi

% ---------------------------------- Copyright Management --------------------------------------------

\newcommand{\MCCLicense}{%
{\color{green}\textbf{CC BY-SA 3.0}}
}

\newcommand{\MCopyrightLabel}[1]{ (\MSRef{L_COPYRIGHTCOLLECTION}{Lizenz})\MLabel{#1}}

% Copyrightliste wird als tex-Datei im preprocessing erzeugt und unter converter/tex/copyrightcollection.tex abgelegt
% Der input-Befehl funktioniert nur, wenn die aufrufende tex-Datei auf der obersten Ebene liegt (d.h. selbst kein input/include ist, insbesondere keine Moduldatei)
\newcommand{\MCopyrightCollection}{\input{copyrightcollection.tex}}

% MCopyrightNotice fuegt eine Copyrightnotiz ein, der parser ersetzt diese durch CopyrightNoticePOST im preparsing, diese Definition wird nur fuer reine pdflatex-Uebersetzungen gebraucht
% Parameter: #1: Kurze Lizenzbeschreibung (typischerweise \MCCLicense)
%            #2: Link zum Original (http://...) oder NONE falls das Bild selbst ein Original ist, oder TIKZ falls das Bild aus einer tikz-Umgebung stammt
%            #3: Link zum Autor (http://...) oder MINT falls Original im MINT-Kolleg erstellt oder NONE falls Autor unbekannt
%            #4: Bemerkung (z.B. dass Datei mit Maple exportiert wurde)
%            #5: Labelstring fuer existierendes Label auf das copyrighted Objekt, mit MCopyrightLabel erzeugt
%            Keines der Felder darf leer sein!
\newcommand{\MCopyrightNotice}[5]{\MCopyrightNoticePOST{#1}{#2}{#3}{#4}{#5}}

\ifttm%
\newcommand{\MCopyrightNoticePOST}[5]{\relax}%
\else%
\newcommand{\MCopyrightNoticePOST}[5]{\relax}%
\fi%

% ---------------------------------- Meldungen fuer den Benutzer des Konverters ----------------------
\MPragma{mintmodversion;P0.1.0}
\MPragma{usercomment;This is file mintmod.tex version P0.1.0}


% ----------------------------------- Spezialelemente fuer Konfigurationsseite, werden nicht von mintscripts.js verwaltet --

% #1 = DOM-id der Box
\ifttm\newcommand{\MConfigbox}[1]{\special{html:<input cfieldtype="2" type="checkbox" name="Name_}#1\special{html:" id="}#1\special{html:" onchange="confHandlerChange('}#1\special{html:');"/>}}\fi % darf im PDF nicht aufgerufen werden!


\MPragma{MathSkip}

\Mtikzexternalize

\begin{document}

\MSection{Elementares Rechnen}
\MLabel{VBKM01}
\MSetSectionID{VBKM01} % wird fuer tikz-Dateien verwendet



\begin{MSectionStart}
\MDeclareSiteUXID{VBKM01_START}
\MModstartBox

In diesem Modul wird ein Überblick über die mathematischen Grundlagen zum elementaren Rechnen gegeben und die Notation eingeführt und erklärt.
\end{MSectionStart}

\MSubsection{Zahlen, Variablen, Terme}
\MLabel{M01_Zahlenbereiche}

\begin{MIntro}
\MDeclareSiteUXID{VBKM01_ZahlenIntro}

%Einfügung Mengenschreibweise
Die Mathematik ist eine Wissenschaft, in der allgemein abstrakte Strukturen
und deren logische Zusammenhänge untersucht werden. 
Bevor auf die eigentlichen Inhalte dieses Abschnitts näher eingegangen wird,
soll kurz auf den grundlegenden Begriff der \MEntry{Menge}{Menge} Bezug
genommen werden.

\begin{MInfo}
Um Aussagen in kompakter Weise über eine Reihe von strukturell ähnlichen
Objekten treffen zu können, kann man solche Objekte in Mengen zusammenfassen,
die als Behältnis für die Objekte dienen. Seien die Objekte mit $a,b,c,\ldots$
benannt, dann bildet das Symbol $M=\{a\MElSetSep b \MElSetSep c \MElSetSep
\ldots\}$ die Menge $M$, welche die vorigen Objekte als \textbf{Elemente}
enthält. Letzteres schreibt man kurz $a\in M$, $b\in M$, $c\in M$ usw.;
das Zeichen \glqq{}$\in$\grqq{} bedeutet also \glqq{}ist Element von\grqq{}.
(Manchmal ist es schreibtechnisch geboten, die Reihenfolge von Element und
Menge auszutauschen. Zum Erhalten der gleichen Aussage(n) wird dann das
$\in$-Symbol umgedreht, d.h. $M\ni a$, $M\ni b$, $M\ni c$ usw. heißt
dann dasselbe, wobei \glqq{}$\ni$\grqq{} somit \glqq{}enthält als Element\grqq{}
bzw.  \glqq{}beinhaltet\grqq{} bedeutet.)

Neben der aufzählenden Schreibweise von Mengen existieren weitere 
Schreibweisen. Wenn die Elemente z.B. eine Bedingung $B$ erfüllen sollen,
dann schreibt man $T=\{x\MCondSetSep x\MBlank\text{erfüllt}\MBlank B\}$.
Wird dabei $x$ (explizit) aus einer umfassenderen Menge $U$ entnommen,
dann wird dies auch in der Form
$T=\{x\MCondSetSep x\in U \MBlank\text{und}\MBlank 
x\MBlank\text{erfüllt}\MBlank B\}$
oder kurz
$T=\{x\in U\MCondSetSep x\MBlank\text{erfüllt}\MBlank B\}$
geschrieben.
%%Da die Bedingung $B$ eine Einschränkung für die Elemente von $U$ darstellt,
%%ist $T$ dann in natürlicher Weise eine \textbf{Teilmenge} von $U$,
%%notiert in Formelnotation als $T\subset U$.

Aussagen wie \glqq{}$x\in U$\grqq{} oder
\glqq{}$x\MBlank\text{erfüllt}\MBlank B$\grqq{} sind Aussagen im 
mathematischen Sinn, d.h. ihnen kann ein eindeutiger Wahrheitswert
\glqq{}wahr\grqq{} oder \glqq{}falsch\grqq{} zugeordnet werden.
Seien $A_1$ und $A_2$ solche Aussagen. Für den Fall, dass sowohl
$A_1$ als auch $A_2$ gelten soll, also $A_1$ \textbf{und} $A_2$ gelten
sollen, schreibt man auch $A_1 \wedge A_2$. Im Fall, dass nur eine von
beiden Aussagen zu gelten braucht, d.h. $A_1$ \textbf{oder} $A_2$
(oder beide Aussagen) gelten sollen, schreibt man auch $A_1\vee A_2$.

Für zwei Mengen $M$ und $N$ notiert man
\begin{itemize}
\item{%
$M\subseteq N$, d.h. $M$ ist (unechte) \textbf{Teilmenge} von $N$, wenn jedes
Element von $M$ auch in $N$ enthalten ist; gibt es dann mindestens ein Element
in $N$, das nicht in $M$ enthalten ist, wenn also $M$ eine echte Teilmenge von
$N$ ist, so schreibt man (auch) $M\subset N$;
}
\item{%
$M\cup N$ für die \textbf{Vereinigung} der beiden Mengen; diese bezeichnet
jene Menge, die alle Elemente enthält, die in mindestens einer der beiden
Mengen vorkommen;
}
\item{%
$M\cap N$ für den \textbf{Schnitt} der beiden Mengen; dieser bezeichnet
jene Menge, in der alle Elemente enthalten sind, die in beiden Mengen
vorkommen;
}
\item{%
$N\MSetminus M$ für die \textbf{Differenzmenge}, d.h. für diejenige Menge,
welche die Elemente von $N$ enthält, die \textit{nicht} in $M$ vorkommen.
}
\end{itemize}
Die obige Vereinigung ist also charakterisiert durch Elemente, die
$(x\in M) \vee (x\in N)$ erfüllen. Für die Elemente der obigen Schnittmenge
gilt dagegen $(x\in M) \wedge (x\in N)$. Demgegenüber enthält die obige
Differenzmenge solche Elemente, für die $(x\in N) \wedge (x\notin M)$ gilt.
Mit dem Symbol $\notin$-Symbol wird die Verneinung (Negation) der
Element-Aussage beschrieben.
\end{MInfo}
%

Mathematik beinhaltet die Welt der Zahlen:
$$\dots\MElSetSep 0\MElSetSep -3\MElSetSep 4\MElSetSep \Mdfrac{4}{5}\MElSetSep \sqrt 2\MElSetSep \MEU\MElSetSep \pi\MElSetSep \MZahl{12}{3}\MElSetSep 10^{23}\MElSetSep \dots \MDFPeriod$$ 
Wenn man verschiedene Zahlen näher 
betrachtet, so erkennt man jedoch grundlegende Unterschiede. Manche Zahlen 
lassen sich nicht als geschlossener Dezimalbruch darstellen, andere sind schier
unvorstellbar (imaginär), wieder andere kann man an den Fingern abzählen oder 
aber als Lösungen von Gleichungen gewinnen. 

\begin{MInfo}
Die in diesem Kurs verwendeten Zahlenbereiche sind:\ifttm\else\ \\\fi
\begin{tabular}{ll}
$\displaystyle \N=\lbrace 1\MElSetSep 2\MElSetSep 3\MElSetSep\ldots\rbrace$ & die \textbf{Menge der natürlichen Zahlen ohne Null},\\ 
$\displaystyle \N_{0}=\lbrace 0\MElSetSep 1\MElSetSep 2\MElSetSep 3\MElSetSep\ldots\rbrace$ &die \textbf{Menge der natürlichen Zahlen inklusive Null},\\
$\displaystyle \Z=\lbrace \ldots\MElSetSep  -2\MElSetSep -1\MElSetSep 0\MElSetSep 1\MElSetSep 2\MElSetSep \ldots\rbrace$ & die \textbf{Menge der ganzen Zahlen},\\
$\displaystyle \Q$ & die \textbf{Menge der rationalen Zahlen (Brüche)},\\
$\displaystyle \R$ & die \textbf{Menge der reellen Zahlen}.\\ 
\end{tabular}
\end{MInfo}

Diese Zahlenbereiche sind nicht unabhängig voneinander, sondern bilden eine Kette ineinandergeschachtelter Zahlenmengen:
$$\N\subset\N_{0}\subset\Z\subset\Q\subset\R \MDFPeriod$$ 
%%Dabei bedeutet das Symbol $\subset$, dass der links stehende Zahlenbereich eine Teilmenge des rechts stehenden Zahlenbereichs ist.
%%Beispielsweise ist jede ganze Zahl aus $\Z$ immer auch eine rationale Zahl aus $\Q$.
Diese Zahlenbereiche erhält man, indem man sich nacheinander die 
Lösungen folgender Gleichungen anschaut und die Zahlenbereiche so erweitert, dass immer eine Lösung existiert:

\begin{center}
\begin{tabular}{|c|r|r|r|c|}
\hline
Zahlenbereich & lösbare Gleichung & nicht lösbar& Hinzunahme&neuer Bereich\\
\hline 
$\displaystyle \N$ & $x+2=4$ & $x+2=1$ & negativer Zahlen& $\displaystyle \Z$\\
$\displaystyle \Z$ & $4x=20$ &$4x=5$ & von Brüchen&$\displaystyle \Q$\\
$\displaystyle \Q$ & $x^2=4$ &$x^2=2$ & irrationaler Zahlen&$\displaystyle \R$\\
$\displaystyle \R$ & $x^2=2$ &$x^2=-1$ & usw. & \\
\hline
\end{tabular}
\end{center}


Natürliche Zahlen treten immer dann auf, wenn Anzahlen bestimmt oder Dinge nummeriert werden müssen.
Sie spielen in der Kombinatorik eine große Rolle: die Anzahl der Möglichkeiten, aus 49 Kugeln 6 Kugeln zu ziehen,
ist zum Beispiel eine natürliche Zahl. In der Informatik bilden sie die Grundlage für die verschiedenen Zahlensysteme:
das Dualsystem hat die Basis 2, das Dezimalsystem die Basis 10 und das Hexadezimalsystem die Basis 16.
Bestimmte natürliche Zahlen, die Primzahlen, bilden die Grundlage der modernen Verschlüsselungstechniken.

In der Menge der natürlichen Zahlen lässt es sich einfach rechnen, aber man stößt an Grenzen, wenn man zum Beispiel
eine Temperaturangabe von 3$^\circ$C liest (handelt es sich um Plus- oder Minusgrade?) oder eine Gleichung
der Form $x+5=1$ auflösen möchte. Daher muss die Menge der natürlichen Zahlen um die negativen natürlichen Zahlen erweitert werden
und man erhält $\Z$. Die \MEntry{Menge der ganzen Zahlen}{Ganze Zahlen $\Z$} wird mit
$$\Z:=\{\dots\MElSetSep  -4\MElSetSep -3\MElSetSep -2\MElSetSep -1\MElSetSep 0\MElSetSep 1\MElSetSep 2\MElSetSep 3\MElSetSep 4\MElSetSep \dots\}$$
bezeichnet. 
Ganze Zahlen werden immer dann benötigt, wenn das Vorzeichen der natürlichen Zahlen eine Rolle spielt.
In $\Z$ können Zahlen unbeschränkt voneinander subtrahiert werden, d.h. Gleichungssysteme der Form $\displaystyle a+x=b$ sind in $\Z$ immer lösbar ($\displaystyle x=b+(-a)$).

\begin{center}
%-%\MUGraphicsSolo{ganzzahlen.png}{width=10cm}{width:400px}
\MTikzAuto{%
\begin{tikzpicture}[x=1.0cm, y=1.0cm] 
\draw[thick, black] (-5.2,0) -- (6.2,0);
\foreach \x in {-5, -4, ..., 6}
\draw[shift={(\x,0)},color=black] (0pt,6pt) -- (0pt,-6pt) node[below=1.5pt] {\normalsize $\x$};
\end{tikzpicture}
}
\end{center}

Auf den ganzen Zahlen lässt sich eindeutig ein Vergleichssymbol $<$ (\textit{kleiner als} genannt) definieren, die ganzen Zahlen lassen sich damit zu einer Kette anordnen: %%%
$$\dots < -3 < -2 < -1 < 0 < 1 < 2 < 3 < \dots \MDFPeriod$$


Eine rationale Zahl stellt das Verhältnis zweier ganzer Zahlen dar:

\begin{MInfo}
Die Menge der \MEntry{rationalen Zahlen}{Rationale Zahlen $\Q$} wird mit
$$\Q:=\left\{{\Mdfrac{p}{q}\MCondSetSep p,q\in\Z, q\neq 0 }\right\}$$
bezeichnet. Die Elemente $\displaystyle \Mtfrac{p}{q}$ der Menge $\Q$ heißen \MEntry{Brüche}{Bruch}, wobei $p$ der \MEntry{Zähler}{Zähler} des Bruchs und $q$ der von Null
verschiedene \MEntry{Nenner}{Nenner} des Bruchs ist.
\end{MInfo}
Rationale Zahlen spielen immer dann eine Rolle, wenn Angaben \glqq{}genauer\grqq{} werden sollen, also Temperaturen in Bruchteilen von $^\circ$C angegeben, Anteile von Flächen eingefärbt oder Medikamente aus bestimmten Bestandteilen zusammengemischt werden sollen.

Dabei ist zu beachten, dass die Darstellung als Bruch nicht eindeutig ist, man kann die gleiche Zahl durch mehrere Brüche beschreiben.
Beispielsweise ist
$$
2 \;=\; \Mdfrac{4}{2} \;=\; \Mdfrac{1024}{512}
$$
die gleiche rationale Zahl.

Andererseits kann nicht jede Zahl auf dem Zahlenstrahl als Bruch dargestellt werden.
Betrachtet man zum Beispiel ein Quadrat mit der Seitenlänge 1 und will die Länge der Diagonalen $d$ berechnen, so erhält man nach dem Satz von Pythagoras ($c^2=a^2+b^2$): %%%

\vspace*{5mm}

\ifttm
\begin{minipage}{6cm}
\else
\begin{minipage}{0.45\linewidth}
\fi
\begin{center}
%-%\MGraphicsSolo{quadrat.png}{width=3cm}
\MTikzAuto{%
\begin{tikzpicture}[x=4.0cm, y=4.0cm] 
\draw[thick, black] (0,0) -- (1,0) -- (1,1) -- (0,1) -- cycle;
\draw[thick, black] (0,1) -- (1,0);
\draw[color=black] (0.5,0.5) node[anchor=south west] {\normalsize $d$};
\draw[stealth'-,black] (0.0,-0.1) -- (0.36,-0.1);
\draw[-stealth',black] (0.65,-0.1) -- (1.0,-0.1);
\draw (0.5,-0.1) node {\scriptsize $1\MEinheit{cm}$};
\end{tikzpicture}
}
\end{center}
\end{minipage}
\ifttm
\begin{minipage}{7cm}
\else
\begin{minipage}{0.55\linewidth}
\fi
\begin{center}
$$d^2 = 1^2+1^2 = 2, \;\mbox{also formal}\; d=\sqrt{2}.$$
\end{center}
\end{minipage}

\vspace*{5mm}

Eine weitere Zahl, die nicht als Bruch dargestellt werden kann, erhält man durch Abrollen eines Rades mit Durchmesser $1$ auf der Zahlengeraden. Es handelt sich um die Zahl $\pi$ (\textit{Pi} genannt). Man kann zeigen, dass diese beiden Zahlen ($\sqrt{2}$ und $\pi$) nicht in Form eines Bruchs geschrieben werden können. (Der Beweis für $\sqrt{2}$ ist dabei verhältnismäßig einfach.) Sie sind zwei Beispiele aus der Menge der sogenannten \MEntry{irrationalen Zahlen}{Irrationale Zahlen}. %%%

\vspace*{5mm}

\begin{center}
%-%\MGraphicsSolo{kreispi.png}{width=10cm}
\MTikzAuto{%
\begin{tikzpicture}[x=2.0cm, y=2.0cm] 
\draw[-stealth',black] (-0.7,0.0) -- (4.0,0.0) node[right=3pt] {$x$};
\draw[color=black] (0,0.5) circle (0.5) ({pi},0.5) circle (0.5);
\draw[-stealth',black] (1.0,0.7) -- (2.14,0.7);
\draw[color=black] (0,3pt) -- (0,-3pt) node[below=2pt] {\small $0$};
\draw[color=black] ({pi},3pt) -- ({pi},-3pt) node[below=2pt] {\small Pi};
\draw[stealth'-,color=black] (0,0.5) ++(20:0.65) arc (20:60:0.65);
\end{tikzpicture}
}
\end{center}

Eine Zahl ist irrational,  wenn sie nicht rational ist, also nicht als Bruch aufgeschrieben werden kann.
Die irrationalen Zahlen schließen nun die noch vorhandenen Lücken auf der Zahlengeraden, jedem Punkt entspricht genau eine reelle Zahl.
\begin{MInfo}
Die Menge der \MEntry{reellen Zahlen}{Reelle Zahlen $\R$} wird mit $\R$ bezeichnet und setzt sich aus der Menge der rationalen Zahlen und der Menge der irrationalen Zahlen zusammen.
Sie enthält alle auf der Zahlengeraden darstellbaren Zahlen.
\end{MInfo}

Reelle Zahlen dienen als Maßzahlen für Längen, Flächeninhalte, Temperaturen, Massen, etc. 
In diesem Kurs werden die mathematischen Probleme typischerweise mit reellen Zahlen gelöst.

Eine Grundeigenschaft reeller Zahlen ist, dass diese geordnet sind, d.h.
für zwei reelle Zahlen $a,b$ gilt genau eine der drei Beziehungen
$a<b$ (\textit{kleiner als}), $a=b$ (\textit{ist gleich}) oder $a>b$ (\textit{größer als}). Eine weitere definierende Eigenschaft ist %%%
die Vollständigkeit, die - grob gesprochen - die 
\glqq Lückenlosigkeit\grqq{} der Zahlengeraden beschreibt. 

\begin{MInfo}
\MLabel{VBKM01_Intervalle}
Für zwei verschiedene reelle Zahlen betrachtet man insbesondere alle Zahlen,
die auf der Zahlengeraden zwischen diesen beiden Zahlen liegen. Solche
Teilmengen reeller Zahlen bezeichnet man als \MEntry{Intervalle}{Intervall}.
Deren Beschreibung wird so festgelegt, dass man ihnen eine linke 
Intervallgrenze ($a$) und eine rechte Intervallgrenze $(b)$ zuordnet
mit $a<b$. Je nachdem, ob eine oder beide Intervallgrenzen zum
Intervall dazugehören, ergeben sich folgende Fälle:
\begin{itemize}
\item{%
$\{x\in\R \MCondSetSep x\ge a \MBlank\text{und}\MBlank x\le b\}
= [a\MIntvlSep b]$ bezeichnet das \textbf{abgeschlossene} Intervall
zwischen $a$ und $b$, bei dem die Grenzen zum Intervall dazugehören.
}
\item{%
$\{x\in\R \MCondSetSep x> a \MBlank\text{und}\MBlank x< b\}
= \MoIl a\MIntvlSep b\MoIr$ bezeichnet das \textbf{offene} Intervall
zwischen $a$ und $b$, bei dem die Grenzen \textit{nicht} zum Intervall
dazugehören.
}
\item{%
$\{x\in\R \MCondSetSep x\ge a \MBlank\text{und}\MBlank x< b\}
= [a\MIntvlSep b\MoIr$ bezeichnet das \textbf{links abgeschlossene und 
rechts offene} Intervall zwischen $a$ und $b$, bei dem die linke Grenze
zum Intervall dazugehört, die rechte aber nicht.
}
\item{%
$\{x\in\R \MCondSetSep x> a \MBlank\text{und}\MBlank x\le b\}
= \MoIl a\MIntvlSep b]$ bezeichnet das \textbf{links offene und 
rechts abgeschlossene} Intervall zwischen $a$ und $b$, bei dem die rechte 
Grenze zum Intervall dazugehört, die linke aber nicht.
}
\end{itemize}
Die Intervalle der letzten beiden Typen heißen auch \textbf{halboffene}
Intervalle.

Im Falle offener Intervallenden kann man auch \textbf{unbeschränkte}
Intervalle betrachten. In diesen Fällen entfällt die jeweilige
Bedingung in der Mengenbeschreibung:
$\{x\in\R \MCondSetSep x\ge a\} = \left[a\MIntvlSep \infty\MoIr[\right]$,
$\{x\in\R \MCondSetSep x> a\} = \MoIl[\left] a\MIntvlSep \infty\MoIr[\right]$,
$\{x\in\R \MCondSetSep x\le b\} = \MoIl[\left] {-}\infty\MIntvlSep b\right]$,
$\{x\in\R \MCondSetSep x< b\} = \MoIl[\left] {-}\infty\MIntvlSep b\MoIr[\right]$,
$\{x\in\R\} = \R = \MoIl[\left] {-}\infty\MIntvlSep \infty\MoIr[\right]$.

Darüber hinaus sind folgende Bezeichnungen gebräuchlich:
$\R^{+}=\MoIl[\left] 0\MIntvlSep \infty\MoIr[\right]$,  
${\R_{0}}^{+}= \left[ 0\MIntvlSep \infty\MoIr[\right]$,  
$\R^{-}=\MoIl[\left] -\infty \MIntvlSep 0\MoIr[\right]$,  
${\R_{0}}^{-}=\MoIl[\left] -\infty \MIntvlSep 0\right]$.  


Eine letzte Bemerkung zur Schreibweise: Sie werden in der Literatur zwei
verschiedene Schreibweisen für das offene Ende eines Intervalls finden,
entweder mit runden Klammern oder mit nach außen weisenden eckigen KLammern, etwa $[a\MIntvlSep b[=[a\MIntvlSep b)$,
$]a\MIntvlSep b[=(a \MIntvlSep b)$.  Beide Schreibweisen sind korrekt, lassen Sie sich davon nicht verwirren.
\end{MInfo}

\end{MIntro}

\begin{MXContent}{Variablen und Terme}{Variablen und Terme}{STD}
\MDeclareSiteUXID{VBKM01_VariablenTerme}

Die Verwendung von Variablen, Termen und Gleichungen ist notwendig, um Aussagen mit noch unbestimmten Werten zu formalisieren.

\begin{MInfo}
Eine \MEntry{Variable}{Variable} ist ein Symbol (typischerweise ein Buchstabe), das als Platzhalter für einen unbestimmten Wert
eingesetzt wird. Ein \MEntry{Term}{Term} ist ein mathematischer Ausdruck, der Variablen, Rechenoperationen und weitere Symbole enthalten kann,
und der nach Einsetzung von Zahlen für die Variablen einen konkreten Zahlenwert ergibt. Terme können zu Gleichungen bzw. Ungleichungen
kombiniert oder in Funktionsbeschreibungen eingesetzt werden, dazu später mehr.
\end{MInfo}


\begin{MExample}
Die textuelle Frage

\textit{In einer Schulklasse gibt es vier Mädchen mehr als Jungs und insgesamt 20 Kinder, wieviele Mädchen bzw. Jungs sind in der Klasse?}

kann man beispielsweise formalisieren, indem man die Variablen $a$ für die Anzahl der Mädchen und
$b$ für die Anzahl der Jungs in der Schulklasse einführt und damit die beiden Gleichungen
$a=b+4$ und $a+b=20$ aufstellt. Diese kann man durch Einsetzen nun auflösen zu $a=12$ und $b=8$
und daraus den textuellen Antwortsatz

\textit{In der Schulklasse befinden sich $12$ Mädchen und $8$ Jungs}

aufbauen. Dabei ist beispielsweise $b+4$ ein Term, $b$ selbst ist eine Variable und $a+b=20$ ist eine Gleichung mit einem Term auf
der linken und einer Zahl auf der rechten Seite.
\end{MExample}

Generell werden Variablen (und manchmal auch Terme) mit kleinen lateinischen Buchstaben $x$, $y$, $z$, usw. bezeichnet. Oft werden auch griechische Buchstaben %%%
dafür verwendet, zum Beispiel wenn Winkel(variablen) von Zahlen(variablen) unterschieden werden sollen. %%%

\begin{MInfo}
\MLabel{VBKM01_Griechisch}
Hier werden die Buchstaben des griechischen Alphabets in einer Übersicht
gezeigt, die auch die Großbuchstaben enthält, sortiert nach dem griechischen Alphabet:
\ \\ \ \\
\begin{tabular}{|*{4}{cc|@{}c@{}|}{cc|}}
\ifttm\hline\else\firsthline\fi
%Die Grossbuchstaben, die nicht auch vom lateinischen Alphabet genutzt
%werden, druckt pdfLaTeX aufrecht, MathML dagegen kursiv.
%Kursive griechische Grossbuchstaben erhaelt man mit den momentanen
%Einstellungen in pdfLaTeX bzw. mintmod mit "\var...".
$\alpha$,       $A$  & \glqq alpha\grqq         & \MTSP &
$\beta$,        $B$  & \glqq beta\grqq          & \MTSP &
$\gamma$,       \ifttm$\Gamma$\else$\varGamma$\fi   & \glqq gamma\grqq        & \MTSP &
$\delta$,       \ifttm$\Delta$\else$\varDelta$\fi   & \glqq delta\grqq        & \MTSP &
$\Mvarepsilon$, $E$  & \glqq epsilon\grqq       \\
$\zeta$,        $Z$  & \glqq zeta\grqq          & \MTSP &
$\eta$,         $H$  & \glqq eta\grqq           & \MTSP &
$\vartheta$,    \ifttm$\Theta$\else$\varTheta$\fi   & \glqq theta\grqq        & \MTSP &
$\iota$,        $I$  & \glqq iota\grqq          & \MTSP &
$\Mvarkappa$,   $K$  & \glqq kappa\grqq         \\
$\lambda$,      \ifttm$\Lambda$\else$\varLambda$\fi & \glqq lambda\grqq       & \MTSP &
$\mu$,          $M$  & \glqq mü\grqq          & \MTSP &
$\nu$,          $N$  & \glqq nü\grqq          & \MTSP &
$\xi$,          \ifttm$\Xi$\else$\varXi$\fi         & \glqq xi\grqq           & \MTSP &
$o$,            $O$  & \glqq omikron\grqq       \\
$\pi$,          \ifttm$\Pi$\else$\varPi$\fi         & \glqq pi\grqq           & \MTSP &
$\varrho$,      $P$  & \glqq rho\grqq               & \MTSP &
$\sigma$,       \ifttm$\Sigma$\else$\varSigma$\fi   & \glqq sigma\grqq        & \MTSP &
% $\varsigma$,                                  & \glqq sigma\grqq         &
$\tau$,         $T$  & \glqq tau\grqq               & \MTSP &
$\upsilon$,     \ifttm$\Upsilon$\else$\varUpsilon$\fi & \glqq üpsilon\grqq     \\
$\Mvarphi$,     \ifttm$\Phi$\else$\varPhi$\fi       & \glqq phi\grqq          & \MTSP &
$\chi$,         $X$  & \glqq chi\grqq          & \MTSP &
$\psi$,         \ifttm$\Psi$\else$\varPsi$\fi       & \glqq psi\grqq          & \MTSP &
$\omega$,       \ifttm$\Omega$\else$\varOmega$\fi   & \glqq omega\grqq        & \MTSP &
 \special{html:&nbsp;}                         & \special{html:&nbsp;}    \\
 \ifttm\hline\else\lasthline\fi
\end{tabular}
\ \\
\MInputHint{Griechische Buchstaben können in Aufgaben mit Ihrer Bezeichnung eingegeben werden, z.~B.~\texttt{alpha} statt $\alpha$.}
\end{MInfo}

Bei einem Term ist wesentlich, dass er zu einem konkreten Zahlenwert ausgewertet werden kann,
wenn man Zahlen für die im Term auftretenden Variablen einsetzt:

\begin{MExample}
Die folgenden Ausdrücke sind Terme:
\begin{itemize}
\item{$x\cdot (y+z)-1$, für $x=1$, $y=2$ und $z=0$ erhält man beispielsweise den Wert $1$ des Terms.}
\item{$\sin(\alpha)+\cos(\alpha)$, für $\alpha=0^\circ$ und $\beta=0^\circ$ erhält man beispielsweise den Wert $1$ (für die Berechnung von Sinus und Kosinus sei auf das Kapitel \MNRef{VBKM05} verwiesen).}
\item{$1+2+3+4$, es treten keine Variablen auf, trotzdem handelt es sich um einen Term (der immer den Wert $10$ ergibt).}
\item{$\Mtfrac{\alpha+\beta}{1+\gamma}$, beispielsweise erhält man für $\alpha=1$, $\beta=2$ und $\gamma=3$ den Wert $\Mtfrac{3}{4}$ für den Term.
Hier darf man aber nicht $\gamma=-1$ einsetzen.}
\item{$\sin(\pi (x+1))$, beispielsweise ergibt der Term den Wert Null wenn man für $x$ eine ganze Zahl einsetzt.}
\item{$z$, eine Variable für sich allein ist auch ein Term.}
\item{$1+2+3+\cdots+(n-1)+n$ ist ein Term, bei dem die Variable $n$ im Term auftritt und gleichzeitig seine Länge festlegt.}
\end{itemize}
\end{MExample}

\begin{MExample}
Diese Ausdrücke sind keine Terme im Sinne der Mathematik:
\begin{itemize}
\item{$a+b=20$, ist eine Gleichung (Einsetzen von Werten für $a$ und $b$ ergibt keine Zahl, sondern die Gleichung ist eben wahr oder falsch).}
\item{$a\cdot (b+c$ ist nicht richtig geklammert,}
\item{\glqq\textit{Anteil der Mädchen in der Schulklasse}\grqq\ ist kein Term, kann aber durch den Term $\Mtfrac{a}{a+b}$ formalisiert werden,}
\item{$\sin$ ist kein Term sondern ein Funktionsname, dagegen ist $\sin(\alpha)$ ein Term (der bei Einsetzen eines Winkels für $\alpha$ ausgewertet werden kann).}
\end{itemize}
\end{MExample}

\begin{MExercise}
Gegeben sind jeweils ein Term und Zahlenwerte für die im Term auftretenden Variablen. Wie lautet die Auswertung des Terms?
\begin{MExerciseItems}
\item{$\Mtfrac{\alpha+\beta}{\alpha-\beta}$ nimmt den Wert \MLParsedQuestion{10}{5}{3}{ER1} an für $\alpha=6$ und $\beta=4$.}
\item{$y^2+x^2$ nimmt den Wert \MLParsedQuestion{10}{2}{3}{ER2} an für $y=2x+1$ und $x=-1$.}
\item{$1+2+3+\cdots+(n-1)+n$ nimmt den Wert \MLParsedQuestion{10}{21}{3}{ER3} an für $n=6$.}
\end{MExerciseItems}
\begin{MHint}{\iSolution}
Einsetzen der gegebenen Werte für die Variablen ergibt \\ $\Mtfrac{\alpha+\beta}{\alpha-\beta}=\Mtfrac{6+4}{6-4}=\Mtfrac{10}{2}=5$ in Teil a), \\
$y^2+x^2=(-1)^2+(-1)^2=2$ in Teil b) mit $y=2 \cdot (-1)+1=-1$ und \\ $1+2+3+4+5+6=21$ in Teil c). %%%
\end{MHint}
\end{MExercise}

\begin{MExercise}
Formalisieren Sie mit den vorgegebenen Variablen den Anteil der Mädchen, deren Anzahl durch die Variable $a$ gegeben sei,
sowie den der Jungen, deren Anzahl durch die Variable $b$ gegeben sei, an der Gesamtzahl an Kindern:

Anteil der Mädchen ist \MLSimplifyQuestion{15}{a/(a+b)}{5}{a,b}{5}{512}{LSFF1} und
Anteil der Jungen ist \MLSimplifyQuestion{15}{b/(a+b)}{5}{a,b}{5}{512}{LSFF2}.\\
\MInputHint{Brüche können mit dem Strich (über der 7-Taste auf den meisten Tastaturen) eingegeben werden,
dabei sollten Zähler bzw. Nenner geklammert werden wenn Rechenoperationen auftreten. Beispielsweise kann
man den Bruch $\Mtfrac{1+x}{2+y}$ eingeben als \texttt{(1+x)/(2+y)}.}

\begin{MHint}{\iSolution}
Die Gesamtanzahl der Kinder ist $a+b$, der Mädchenanteil ist daher $\Mtfrac{a}{a+b}$ und der Jungenanteil ist $\Mtfrac{b}{a+b}$.
\end{MHint}
\end{MExercise}


Terme können auch ineinander eingesetzt werden:

\begin{MInfo}
Beim \MEntry{Einsetzen}{Einsetzen (Terme)} von Termen wird ein Term anstelle eines Symbols in einem anderen Term eingesetzt,
wobei das ersetzte Symbol ggf. vorher geklammert werden muss, wenn der einzusetzende Term mehrere Ausdrücke enthält.
\end{MInfo}

\begin{MExample}
Setzt man in den Term $x^2+y^2$ beispielsweise den Wert $x=1+2+3$ ein, so entsteht der neue Term $x^2+y^2=(1+2+3)^2+y^2=36+y^2$, und nicht etwa
$1+2+3^2+y^2=12+y^2$.
\end{MExample}

\begin{MExercise}
Welcher Term entsteht, wenn man in $x^2+y^2$ Folgendes einsetzt und soweit wie möglich vereinfacht: %%%
\begin{MExerciseItems}
\item{Den Winkel $\alpha$ sowohl für $x$ als auch für $y$: Dann ist \MEquationItem{$x^2+y^2$}{\MLFunctionQuestion{13}{2*alpha^2}{5}{alpha}{5}{ERX1}}.} 
\item{Die Zahl $2$ für $y$ und der Term $t+1$ für $x$: Dann ist \MEquationItem{$x^2+y^2$}{\MLFunctionQuestion{13}{4+(t+1)^2}{5}{t}{5}{ERX2}}.}
\item{Den Term $z+1$ für $x$ und der Term $z-1$ für $y$: Dann ist \MEquationItem{$x^2+y^2$}{\MLFunctionQuestion{13}{2*z*z+2}{5}{z}{5}{ERX3}}.}
\end{MExerciseItems}
\MInputHint{Der griechische Buchstabe $\alpha$ kann als \texttt{alpha} eingetippt werden.}

\begin{MHint}{\iSolution}
Am sichersten ist es, die Variablen vor der Termeinsetzung zu klammern, wenn der neue Term mehrere Symbole enthält:
\begin{MExerciseItems}
\item{$x^2+y^2=\alpha^2+\alpha^2=2\alpha^2$.}
\item{$x^2+y^2=(x)^2+y^2=(t+1)^2+2^2=t^2+2t+5$.}
\item{$x^2+y^2=(x)^2+(y)^2=(z+1)^2+(z-1)^2=z^2+2z+1+z^2-2z+1=2z^2+2$.}
\end{MExerciseItems}
\end{MHint}
\end{MExercise}

\begin{MExercise}
In dieser Figur habe ein Kästchen auf dem Papier die Seitenlänge $x$. Welchen Flächeninhalt (als Term in der Variablen $x$) besitzt
die Figur?

%%\ifttm%
%%\begin{center}
%%\MUGraphics{tikz01.png}{}{Eine Figur auf kariertem Papier.}{}
%%\end{center}
%%\else%
\begin{center}
\MTikzAuto{%
\begin{tikzpicture}
\filldraw[fill=black!25] (3.5,2.5) circle[radius=2.5];
\filldraw[fill=white] (2.5,3.5) circle[radius=0.5];
\filldraw[fill=white] (4.5,3.5) circle[radius=0.5];
\filldraw[fill=white] (2,1) rectangle ++ (3,1);
\draw[dashed] (0.5,-0.5) grid (6.5,5.5);
\node at (current bounding box.south) [below] {(Ein Kästchen hat die Länge $x$)};
\end{tikzpicture}
}
\par\ifttm\else\penalty+10000\fi
Eine Figur auf kariertem Papier.
\end{center}
%%\fi

Antwort:
\begin{itemize}
\item{Der große Kreis hat insgesamt den Flächeninhalt \MLFunctionQuestion{25}{25/4*pi*x*x}{5}{x}{5}{ERX11},}
\item{je ein kleiner Kreis hat den Flächeninhalt \MLFunctionQuestion{25}{1/4*pi*x*x}{5}{x}{5}{ERX12},}
\item{die Figur insgesamt hat den Flächeninhalt \MLFunctionQuestion{40}{(25/4*pi-1/2*pi-3)*x*x}{5}{x}{5}{ERX13}.}
\end{itemize}
\MInputHint{Die Kreiszahl $\pi$ kann als \texttt{pi} eingegeben werden.}

\begin{MHint}{Tipp zur Flächenberechnung}
In späteren Kapiteln wird die Berechnung von Flächen vorgestellt. Für diese Aufgabe benötigt man davon nur, dass 
ein Rechteck mit Seitenlängen $a$ und $b$ den Flächeninhalt $a\cdot b$ (geschrieben \texttt{a*b}) besitzt,
und dass ein Kreis mit Radius $r$ den Flächeninhalt $\pi r^2$ (geschrieben \texttt{pi*r\^2}) besitzt.
\end{MHint}

\begin{MHint}{\iSolution}
Der große Kreis hat insgesamt den Flächeninhalt $(\MZahl{2}{5})^2\pi x^2=\Mtfrac{25}{4}\pi x^2$ (oder getippt \texttt{25/4*pi*x*x}). %%%
Je ein kleiner Kreis hat den Flächeninhalt $(\MZahl{0}{5})^2\pi x^2=\Mtfrac{1}{4}\pi x^2$ und
die Figur insgesamt hat den Flächeninhalt $(\Mtfrac{25}{4}\pi-\Mtfrac{1}{2}\pi-3)\cdot x^2$. %%%
\end{MHint}
\end{MExercise}

\end{MXContent}

\begin{MXContent}{Terme umformen}{Terme umformen}{STD}
\MLabel{VBKM01_TermeUmformen}
\MDeclareSiteUXID{VBKM01_TermeUmformen}
Verschiedene Operationen in einem Term können den gleichen Wert beschreiben, beispielsweise ist $x+x$ eine von $2x$
verschiedene Symbolanordnung, die aber den gleichen Term beschreibt, d.h. wenn man eine konkrete Zahl für $x$ einsetzt
kommt bei $x+x$ und $2x$ der gleiche Wert heraus.

\begin{MInfo}
Zwischen Termen wird ein Gleichheitszeichen geschrieben, wenn diese stets zum gleichen Wert ausgewertet werden.
\end{MInfo}

Neue Terme entstehen in der Regel durch Umformen vorhandener Terme:

\begin{MInfo}
Eine \MEntry{Umformung}{Umformung} eines Terms entsteht, indem man eine oder mehrere Rechenregeln auf einen Term anwendet:
\begin{itemize}
\item{Zusammenfassen: $a+a+\cdots+a=n\cdot a$ ($n$ ist die Anzahl der Summanden).}
\item{Distributivgesetze (\glqq Ausmultiplizieren\grqq): $(a+b)\cdot c = a c+b c$ und $c\cdot(a+b)=c a+c b$.}
\item{Kommutativgesetz: $a+b=b+a$.}
\item{Assoziativgesetz (\glqq Klammern umsetzen bei gleichen Operationen\grqq): $a+(b+c)=(a+b)+c=a+b+c$, ebenso für die Multiplikation.}
\item{Rechenregeln für Potenzen und spezielle Funktionen.}
\item{Rechenregeln für bestimmte Formen von Termen (z.B. die binomischen Formeln).}
\item{Rechenregeln für Brüche: $\Mtfrac1{\phantom{2}\Mtfrac{a}{b}\phantom{2}}=\Mtfrac{b}{a}$.}
\end{itemize}
\end{MInfo}

Die Regeln werden im Detail in den folgenden Abschnitten vorgestellt.
Ziel dieser Umformungen ist es meist,
den Term einfacher zu machen, einzelne Variablen zu isolieren oder
den Term in eine gewünschte Form zu bringen:

\begin{MExample}
Zulässige Umformungen sind
\begin{itemize}
\item{$a(a+a+a)+a^2+a^2+a^2 = 6a^2$, der Term auf der rechten Seite ist einfacher da er weniger Symbole benötigt.} %%%
\item{$(x+3)^2-9=x^2+6x$ (1. binomische Formel), beide Terme beschreiben eine Parabel. An der linken Seite kann man gut
den Scheitelpunkt $(-3,-9)$ ablesen, an der rechten Seite die beiden Nullstellen ($x_1=0$ und $x_2=-6$).}
\item{$1+3x+3x^2+x^3=(1+x)^3$, an der rechten Seite kann man beispielsweise ablesen, dass die durch den Term beschriebene Funktion
nur die Nullstelle $x_1=-1$ besitzt.}
\item{$\Mtfrac{a+1}{a}=1+\Mtfrac{1}{a}$, an der linken Seite kann man ablesen, dass der Term die Nullstelle $a_1=-1$ besitzt, an der rechten
Seite kann man ablesen, dass der Term für sehr große $a$ gegen den Wert $1$ strebt (weil $\Mtfrac{1}{a}$ dann sehr klein ist).}
\end{itemize}
\end{MExample}


\begin{MExerciseCollection}{TESTCOLLECTION001}{1}

\begin{MExercise}
Formen Sie in eine Summendarstellung um: \MEquationItem{$a\cdot(b+c)+c\cdot(a+b)$}{\MLSimplifyQuestion{20}{a*(b+c)+c*(a+b)}{3}{a,b,c}{3}{1}{TC1}}.
\begin{MHint}{\iSolution}
$$
a\cdot(b+c)+c\cdot(a+b) \;=\; a b + a c + c a + c b \;=\; a b + 2 a c + b c
$$
\end{MHint}
\end{MExercise}

\begin{MExercise}
Formen Sie in eine Summendarstellung um: \MEquationItem{$(x-y)(z-x)+(x-z)(y-z)$}{\MLSimplifyQuestion{20}{(x-y)*(z-x)+(x-z)*(y-z)}{3}{x,y,z}{3}{1}{TC2}}.
\begin{MHint}{\iSolution}
$$
(x-y)(z-x)+(x-z)(y-z) \;=\; x z - x^2 -y z +y x + x y - x z - z y + z^2 \;=\; -x^2 - 2 y z + 2 x y + z^2 %%%
$$
\end{MHint}
\end{MExercise}

\begin{MExercise}
Formen Sie in eine Summendarstellung um: \MEquationItem{$(a+b+2)(a+1)$}{\MLSimplifyQuestion{20}{(a+b+2)*(a+1)}{3}{a,b,c}{3}{1}{TC3}}.
\begin{MHint}{\iSolution}
$$
(a+b+2)(a+1) \;=\; a^2 + b a + 2a + a + b + 2 \;=\; a^2 + a b + 3a + b +2
$$
\end{MHint}
\end{MExercise}

\end{MExerciseCollection}




\end{MXContent}


\MSubsection{Bruchrechnung}
\MLabel{M01_Bruchrechnung}
\begin{MXContent}{Mit Brüchen rechnen}{Mit Brüchen rechnen}{STD}
\MDeclareSiteUXID{VBKM01_Bruchrechnung}

Ein Bruch ist eine rationale Zahl der Form $\displaystyle \Mtfrac{\mbox{Zähler}}{\mbox{Nenner}}$, wobei Zähler und Nenner ganze Zahlen sind und der Nenner $\neq 0$ ist. 
Beispiele hierfür sind:
$$\Mdfrac{1}{2}\MDFPSpace, \MDFPaSpace\Mdfrac{5}{-10}\MDFPSpace, \MDFPaSpace\Mdfrac{-17}{12}\MDFPSpace, \MDFPaSpace\Mdfrac{1}{23}\MDFPSpace, \MDFPaSpace\Mdfrac{4}{6}\MDFPSpace, \MDFPaSpace\Mdfrac{-2}{3}\MDFPSpace, \MDFPaSpace \ldots \MDFPeriod$$
Sehr schnell erkennt man, dass ein und dieselbe rationale Zahl beliebig viele äquivalente Darstellungen haben kann. Zum Beispiel gilt:
$$\Mdfrac{12}{36} = \Mdfrac{1}{3} = \Mdfrac{24}{72} = \Mdfrac{-12}{-36} = \Mdfrac{3}{9} = \Mdfrac{2}{6} = \Mdfrac{120}{360}= \dots \MDFPeriod$$
Die verschiedenen Darstellungen gehen durch \textbf{Kürzen} bzw. \textbf{Erweitern} ineinander über.
\begin{MInfo}
Brüche werden \MEntry{gekürzt}{Kürzen}, indem Zähler und Nenner durch dieselbe ganze Zahl ungleich Null dividiert werden.

Brüche werden \MEntry{erweitert}{Erweitern}, indem Zähler und Nenner mit derselben ganzen Zahl ungleich Null multipliziert werden.
\end{MInfo}

\begin{MExample}
Drei Freunde möchten sich eine Pizza teilen. Tom isst $\displaystyle \Mtfrac{1}{4}$ der Pizza, Tim $\displaystyle \Mtfrac{1}{3}$ der Pizza. Wieviel Pizza ist noch für ihren Freund Sven übrig, der eigentlich immer den meisten Hunger hat?\\
Der Ergebnis wird mithilfe der Bruchrechnung bestimmt: Zunächst müssen zwei Brüche addiert werden, um festzustellen, wieviel Tim und Tom schon von der Pizza gegessen haben: 
$$\Mdfrac{1}{4} + \Mdfrac{1}{3} = \Mdfrac{1\cdot 3}{4\cdot 3}+\Mdfrac{1\cdot 4}{3\cdot 4} = \Mdfrac{3}{12} +\Mdfrac{4}{12} = \Mdfrac{7}{12} \MDFPeriod$$
Hier erkennt man schon die beiden wichtigsten Schritte: zunächst müssen die beiden Brüche durch Erweitern auf den sogenannten \textbf{Hauptnenner} gebracht oder man sagt auch \textbf{gleichnamig} gemacht werden. Wenn die Brüche dann denselben Nenner besitzen, können sie addiert werden, indem ihre Zähler addiert und der gemeinsame Nenner übernommen wird.
Mit dem Ergebnis, dass Tim und Tom $\displaystyle \Mtfrac{7}{12}$ der Pizza gegessen haben, kann durch Subtraktion berechnet werden, wie viel für Sven übrig bleibt: 
$$ 1 - \Mdfrac{7}{12} = \Mdfrac{12}{12} - \Mdfrac{7}{12} = \Mdfrac{5}{12} \MDFPeriod$$
Auch hier werden die Brüche wieder auf den Hauptnenner gebracht und anschließend die Zähler subtrahiert. Die beiden Freunde haben also für den immer hungrigen Sven tatsächlich die meiste Pizza übriggelassen. 
\end{MExample}

In dieser Trainingsaufgabe kann das Kürzen von Zahlen in Zähler und Nenner eingeübt werden:

\MDirectRouletteExercises{fraction_cancellation.rtex}{VBKM01_FRACTIONTRAINING}

Schwieriger wird es, wenn Unbestimmte in Zähler und Nenner auftreten. Diese können genau wie Zahlen (aber nicht mit Zahlen) gekürzt werden, beispielsweise ist
$$
\frac{4x^2y^3+3y^2}{10y^2} \;=\; \frac{4x^2y+3}{10}
$$
nach Kürzung durch den Term $y^2$ Zähler und Nenner.
Die folgende Trainingsaufgabe wurde um Unbestimmte erweitert:

\MDirectRouletteExercises{vfraction_cancellation.rtex}{VBKM01_VFRACTIONTRAINING}

\begin{MInfo}
Der \MEntry{Hauptnenner}{Hauptnenner} von zwei Brüchen ist das kleinste gemeinsame Vielfache (kgV) der beiden Nenner.

Das \MEntry{kleinste gemeinsame Vielfache}{kgV} (kgV) zweier Zahlen $z_1$ und $z_2$ ist die kleinste natürliche Zahl $k$ von der sowohl Zahl $z_1$ als auch Zahl $z_2$ Teiler sind. \\%%%
Der \MEntry{größte gemeinsame Teiler}{ggT} (ggT) zweier Zahlen $z_1$ und $z_2$ ist die größte natürliche Zahl $g$, die Zahl $z_1$ als auch Zahl $z_2$ teilt.%%%
\end{MInfo}

Ist die Bestimmung des kgV bei den folgenden Rechenregeln zu kompliziert, so kann an seiner Stelle auch das einfache Produkt der Nenner benutzt werden:


\begin{MInfo}
Brüche werden \MEntry{addiert/subtrahiert}{Addition (Bruch)}, indem man sie auf den gleichen Nenner bringt und die Zähler anschließend addiert/subtrahiert, d.~h.
$$\Mdfrac{a}{b} \pm \Mdfrac{c}{d} = \Mdfrac{a d\pm b c}{b d} \MDFPSpace, \MDFPaSpace b d\neq 0 \MDFPeriod$$
Üblicherweise werden die Brüche auf den Hauptnenner erweitert. 
\end{MInfo}

Beispielsweise ist das kleinste gemeinsame Vielfache von $6=2\cdot 3$ und $15=3\cdot 5$ die Zahl $2\cdot3\cdot 5=30$, das Produkt ist dagegen $6\cdot 15=90$. Man kann also
$$
\Mdfrac{1}{6}+\Mdfrac{1}{15} \;=\; \Mdfrac{5}{30}+\Mdfrac{2}{30} \;=\; \Mdfrac{7}{30}
$$
aber auch
$$
\Mdfrac{1}{6}+\Mdfrac1{15} \;=\; \Mdfrac{15}{90}+\Mdfrac{6}{90} \;=\; \Mdfrac{21}{90}
$$
rechnen und den letzten Bruch dann noch zu $\Mtfrac{7}{30}$ kürzen.

\begin{MExample}
Das kleinste gemeinsame Vielfache für den Hauptnenner ist die kleinste Zahl,
die von allen beteiligten Nennern geteilt wird. Falls die Zahlen keine gemeinsamen Faktoren haben,
ist es einfach das Produkt der beiden Zahlen:
\begin{eqnarray*}
\Mdfrac{1}{6}+\Mdfrac{1}{10} &=& \Mdfrac{5}{30}+\Mdfrac{3}{30} \;=\; \Mdfrac{8}{30} \;=\; \Mdfrac{4}{15}\MDFPSpace , \ \\
\Mdfrac{1}{6}+\Mdfrac{1}{10} &=& \Mdfrac{10}{60}+\Mdfrac{6}{60} \;=\; \Mdfrac{16}{60} \;=\; \Mdfrac{4}{15}\MDFPaSpace  \text{(auch richtig)}, \ \\
\Mdfrac{4}{15}-\Mdfrac1{2} &=& \Mdfrac{8}{30}-\Mdfrac{15}{30} \;=\; \Mdfrac{8-15}{30} \;=\; -\Mdfrac{7}{30}\MDFPSpace , \ \\
\Mdfrac{1}{3}+\Mdfrac{1}{9} &=& \Mdfrac{3}{9}+\Mdfrac19 \;=\; \Mdfrac49\MDFPSpace ,\ \\
\Mdfrac1{2^2}+\Mdfrac1{2^4} & = & \Mdfrac{2^2}{2^4}+\Mdfrac{1}{2^4} \;=\; \Mdfrac{5}{16}\MDFPSpace ,\ \\
\Mdfrac12+\Mdfrac13+\Mdfrac17 & = & \Mdfrac{21}{42}+\Mdfrac{14}{42}+\Mdfrac{6}{42} \;=\; \Mdfrac{41}{42}\MDFPeriod
\end{eqnarray*}
\end{MExample}

Bei der Bildung von Hauptnennern können auch Terme mit Variablen zum Einsatz kommen.
Da die Bruchumformungen für alle Werte dieser Variablen richtig sein sollen,
müssen diese wie Zahlen ohne gemeinsame Faktoren behandelt werden:

\begin{MExample}
Sind $x$ und $y$ eine Variablen, so gilt
\begin{eqnarray*}
\Mdfrac13+\Mdfrac1x & = & \Mdfrac{x}{3\cdot x}+\Mdfrac{3}{3\cdot x} \; =\; \Mdfrac{3+x}{3\cdot x}\MDFPSpace ,\\
\Mdfrac{1}{x}+\Mdfrac1{y} & = & \Mdfrac{y}{x\cdot y}+\Mdfrac{x}{x\cdot y} \;=\; \Mdfrac{x+y}{x\cdot y}\MDFPSpace ,\ \\
\Mdfrac{1}{(x+1)^2}+\Mdfrac1{x+1} & = & \Mdfrac{1}{(x+1)^2}+\Mdfrac{x+1}{(x+1)^2}\;=\;\Mdfrac{x+2}{(x+1)^2}\MDFPeriod
\end{eqnarray*}
\end{MExample}

\begin{MExercise}
Diese Summen sollen über Hauptnenner (oder das Produkt der Nenner) ausgerechnet werden:
\begin{MExerciseItems}
\item{\MEquationItem{$\Mtfrac12-\Mtfrac18$}{\MLSimplifyQuestion{9}{-1/8+1/2}{4}{}{4}{16}{HN1}}.}%%%
\item{\MEquationItem{$\Mtfrac13+\Mtfrac15+\Mtfrac16$}{\MLSimplifyQuestion{11}{1/3+1/5+1/6}{4}{}{4}{16}{HN2}}.}%%%
\item{\MEquationItem{$\Mtfrac1{2x}+\Mtfrac1{3x}$}{\MLSimplifyQuestion{13}{5/(6*x)}{4}{x}{4}{528}{HN3}}.}%%%
\end{MExerciseItems}
\MInputHint{Bei dieser Aufgabe dürfen keine Rechenoperationen bis auf Multiplikation \texttt{*} und den Divisionsstrich \texttt{/} eingegeben werden.}

\begin{MHint}{\iSolution}
Bilden des Hauptnenners und Zusammenfassen/Kürzen ergibt
\begin{eqnarray*}
\Mdfrac12-\Mdfrac18 & = & \Mdfrac48-\Mdfrac18 \;=\; \Mdfrac38 \MDFPSpace ,\\
\Mdfrac13+\Mdfrac15+\Mdfrac16 & = & \Mdfrac{10}{30}+\Mdfrac6{30}+\Mdfrac{5}{30} \;=\; \Mdfrac{21}{30} \;=\; \Mdfrac{7}{10}\MDFPSpace ,\\
\Mdfrac1{2x}+\Mdfrac1{3x} & = & \Mdfrac{3}{6x}+\Mdfrac{2}{6x} \;=\; \Mdfrac{5}{6x} \MDFPSpace .
\end{eqnarray*}
\end{MHint}
\end{MExercise}

\begin{MExercise}
Bei gleichnamigen Brüchen darf man nur die Zähler addieren bzw. zerlegen, für den Nenner gibt es keine solche Regel.
Berechnen Sie zum Nachweis die folgenden Zahlenwerte, indem Sie den Hauptnenner bilden und soweit wie möglich kürzen:
\begin{MExerciseItems}
\item{\MEquationItem{$\Mdfrac12+\Mdfrac13$}{\MLSpecialQuestion{10}{5/6}{5}{0}{0}{exactfraction}{GLN1}} aber \MEquationItem{$\Mdfrac1{2+3}$}{\MLSpecialQuestion{12}{1/5}{5}{0}{0}{exactfraction}{GLN2}}.}
\item{\MEquationItem{$\Mdfrac{1+2}{5+6}$}{\MLSpecialQuestion{12}{3/11}{5}{0}{0}{exactfraction}{GLN3}} aber \MEquationItem{$\Mdfrac15+\Mdfrac26$}{\MLSpecialQuestion{12}{8/15}{5}{0}{0}{exactfraction}{GLN4}}.}
\end{MExerciseItems}
\begin{MHint}{\iSolution}
Summen von Nennern darf man nicht zusammenfassen, auch nicht bei gleichem Zähler, hier ist
$$
\Mdfrac12+\Mdfrac13\;=\; \Mdfrac{3}{6}+\Mdfrac26 \;=\; \Mdfrac56\;\;\text{aber}\;\;
\Mdfrac{1}{2+3} \;=\; \Mdfrac15\MDFPSpace .
$$
Auch das einfache \glqq Auseinandernehmen\grqq\ von Bestandteilen der Brüche ist nicht erlaubt, hier ist
$$
\Mdfrac{1+2}{5+6} \;=\; \Mdfrac{3}{11}  \;\;\text{aber}\;\; \Mdfrac15+\Mdfrac26 \;=\; \Mdfrac{6}{30}+\Mdfrac{10}{30} \;=\; \Mdfrac{16}{30}\;=\; \Mdfrac{8}{15}\MDFPSpace .
$$
\end{MHint}
\end{MExercise}

\begin{MInfo}
Brüche werden \MEntry{multipliziert}{Multiplikation (Bruch)}, indem Zähler mit Zähler und Nenner mit Nenner multipliziert werden, d.~h. 
$$\Mdfrac{a}{b} \cdot \Mdfrac{c}{d} = \Mdfrac{a\cdot c}{b\cdot d}\MDFPSpace, \MDFPaSpace b d\neq 0 \MDFPeriod$$
\end{MInfo}

Die Division zweier Brüche wird auf die Multiplikation zurückgeführt:

\begin{MInfo}
Brüche werden \MEntry{dividiert}{Division (Bruch)}, indem der erste Bruch mit dem Kehrwert des zweiten Bruchs multipliziert wird, d.h.
$$\Mdfrac{a}{b} : \Mdfrac{c}{d} = \Mdfrac{a}{b} \cdot \Mdfrac{d}{c} = \Mdfrac{a\cdot d}{b\cdot c}\MDFPSpace, \MDFPaSpace b,c,d\neq 0 \MDFPeriod$$
Die Division zweier Brüche kann auch als \textbf{Doppelbruch} geschrieben werden:
$$\Mdfrac{a}{b} : \Mdfrac{c}{d} = \Mdfrac{\Mdfrac{a}{b}}{\Mdfrac{c}{d}} \MDFPeriod$$
\end{MInfo}

\begin{MExample}
Die Multiplikation bzw. Division zweier Brüche sieht unter Berücksichtigung von eventuellem Kürzen folgendermaßen aus:
$$\Mdfrac23\cdot \Mdfrac45 = \Mdfrac{2\cdot 4}{3\cdot 5} = \Mdfrac{8}{15}\MDFPSpace, \MDFPaSpace \Mdfrac23 : \Mdfrac45 = \Mdfrac23\cdot \Mdfrac54 = \Mdfrac{10}{12} = \Mdfrac56 \MDFPeriod$$
\end{MExample}


\end{MXContent}

\begin{MXContent}{Umwandeln von Brüchen}{Umwandeln von Brüchen}{STD}
\MDeclareSiteUXID{VBKM01_USBrueche}

Wird ein Bruch ausdividiert, so erhält man einen \MEntry{Dezimalbruch}{Dezimalbruch} bzw. eine Dezimalzahl, zum Beispiel
$$\Mdfrac{1}{2} = \MZahl{0}{5} \MDFPSpace, \MDFPaSpace \Mdfrac{1}{3} = \MZahl{0}{33333}\ldots= \MZahl{0}{}\bar{3} \MDFPSpace, \MDFPaSpace \Mdfrac{1}{7} = \MZahl{0}{}\overline{142857} \MDFPSpace, \MDFPaSpace \Mdfrac{1}{8} = \MZahl{0}{125} \MDFPeriod$$
Schon an diesen Beispielen zeigt sich, dass die Division entweder aufgehen kann und man
einen \MEntry{endlichen Dezimalbruch}{Dezimalbruch (endlich)} erhält oder aber die Ziffern wiederholen sich
in einer bestimmten Reihenfolge, dann liegt ein \MEntry{unendlicher periodischer Dezimalbruch}{Dezimalbruch (periodisch)} vor. Dabei wird die sich kleinste periodisch wiederholende Ziffernfolge mit einem Strich über den entsprechenden Ziffern gekennzeichnet. %%%

Die Umwandlung von endlichen Dezimalbrüchen in Brüche geschieht mit der Stellentafel. Jeder Dezimalbruch hat die Form
\begin{center}
\ifttm\else\dots\fi\,\begin{tabular}{|c|c|c|c|c|c|c|c|c|c|}
\hline
1&2&3&4&5&\MZXYZhltrennzeichen &6&7&8&9\\
\hline
ZT&T&H&Z&E&\MZXYZhltrennzeichen &z&h&t&zt\\
\hline
\end{tabular}\,\ifttm\else\dots\fi
\end{center}
\vspace*{2mm} %%%

wobei ZT \ldots\ Zehntausender, T \ldots\ Tausender, H \ldots\ Hunderter, Z \ldots\ Zehner, E \ldots\ Einer, z \ldots\ Zehntel, h \ldots\ Hundertstel, t \ldots\ Tausendstel, zt \ldots\ Zehntausendstel u.s.w. beschreiben.\\
Die Umwandlung sieht dann folgendermaßen aus:
\begin{eqnarray*}
\MZahl{4}{375} &=& 4 + \Mdfrac{3}{10}+\Mdfrac{7}{100}+\Mdfrac{5}{1000} \ \\
&=& 4 + \Mdfrac{300+70+5}{1000} \ \\
&=& 4 + \Mdfrac{375}{1000}\ \\
&=& 4 + \Mdfrac{75}{200} \ \\
&=& 4 + \Mdfrac{15}{40} = \Mdfrac{35}{8} \MDFPeriod
\end{eqnarray*}
Aber wie sieht es bei periodischen Dezimalbrüchen aus? Anscheinend müssten hier unendlich viele Brüche aufsummiert werden, was in der Praxis natürlich wenig Sinn macht. Daher bedient man sich bei der \textbf{Umwandlung unendlicher periodischer Dezimalbrüche in Brüche} eines Tricks:

\begin{MInfo}
\MLabel{Mathematik_Grundlagen_UDB}
Die \MEntry{Umwandlung}{Umwandlung (Dezimalbrüche)} periodischer Dezimalbrüche in Brüche geschieht, indem man durch Multiplikation mit einer Zehnerpotenz
die periodischen Nachkommastellen vor das Komma holt. Dies ergibt eine Gleichung der Form $10^k\cdot x=x+n$
für den Dezimalbruch $x$, der zu $x=\Mdfrac{n}{10^k-1}$ (ein gewöhnlicher Bruch) aufgelöst werden kann.
\end{MInfo}

\begin{MExample}
	Die Zahl $\MZahl{0}{}\bar 6$ soll in einen Bruch umgewandelt werden. Hierzu multipliziert man die Zahl
mit $10$ und subtrahiert vom Ergebnis die Ausgangszahl, um die unendliche Periode
zu eliminieren:
\begin{center}
\begin{tabular}{crclcr}
	&$10$ & $\cdot$ & $\MZahl{0}{}\bar 6$ & = & $\MZahl{6}{}\bar 6$\\
	$-$ & $1$ & $\cdot$ & $\MZahl{0}{}\bar 6$ & = & $\MZahl{0}{}\bar 6$\\
\hline
$\Rightarrow$&$9$ & $\cdot$ & $\MZahl{0}{}\bar 6$ & = & $\MZahl{6}{0}$\\
\end{tabular}
\end{center}
Aus der letzten Beziehung folgt nach Division durch $9$ sofort:\quad $\displaystyle \MZahl{0}{}\bar 6 = \Mtfrac{6}{9} = \Mtfrac{2}{3}.$
\end{MExample}

Dieses Vorgehen funktioniert auch, wenn sich nicht alle Ziffern hinter dem Komma periodisch wiederholen:

\begin{MExample}
Der Dezimalbruch $\MZahl{0}{8}\bar{3}=\MZahl{0}{83333}\ldots$ soll in einen Bruch umgewandelt werden:
\begin{center}
\begin{tabular}{crclcr}
&$100$ & $\cdot$ & $\MZahl{0}{8}\bar 3$ & = & $\MZahl{83}{}\bar 3$\\
$-$&$10$ & $\cdot$ & $\MZahl{0}{8}\bar 3$ & = & $\MZahl{8}{}\bar 3$\\
\hline
$\Rightarrow$&$90$ & $\cdot$ & $\MZahl{0}{8}\bar 3$ & = & $\MZahl{75}{0}$\\
\end{tabular}
\end{center}
Division durch $90$ liefert das Ergebnis: $\displaystyle \MZahl{0}{8}\bar 3 = \Mtfrac{75}{90} = \Mtfrac{5}{6}$.
\end{MExample}

Die Vorgehensweise ist also immer dieselbe: durch geeignete Multiplikation mit Zehnerpotenzen und anschließender Subtraktion wird die unendliche Periode entfernt.

\begin{MExercise}
Berechnen Sie mit dem obigen Verfahren einen gewöhnlichen und gekürzten Bruch, der den Wert $\MZahl{0}{45555}\ldots$ darstellt.
\ \\ \ \\
Antwort: \MEquationItem{$\displaystyle \MZahl{0}{4}\overline{5}$}{\MLSpecialQuestion{12}{41/90}{5}{0}{0}{exactfraction}{BRZ3}}.
\ \\
\MInputHint{Geben Sie den Bruch in der Form \texttt{Zähler/Nenner} maximal gekürzt und mit positivem Nenner ein.}
\ \\
\begin{MHint}{\iSolution}
Multiplikation von $x=\MZahl{0}{4}\overline{5}$ mit einer geeigneten Zehnerpotenz ergibt
$$
10x-x=\MZahl{4}{1}\;\Rightarrow\; 9x=\Mdfrac{41}{10}\;\Rightarrow\;x=\Mdfrac{41}{90} \MDFPeriod
$$
Dieser Bruch ist auch schon maximal gekürzt.
\end{MHint}
\end{MExercise}

Beim \MEntry{überschlägigen Rechnen}{Überschlägiges Rechnen} (wenn man also nur ungefähr die Größe oder das Verhältnis einer Zahl zu anderen Zahlen abschätzen möchte ohne
den exakten Wert als Dezimalbruch zu kennen) ist es dagegen hilfreich, statt einer Umwandlung mit dem Hauptnenner (sprich dem kgV aller Nenner) zu multiplizieren: %%%

\begin{MExample}
\MLabel{VBKM01_Bsp_Anordnung}
Die Brüche $\frac23$, $\frac{32}{12}$ und $\frac{12}{15}$ sollen der Größe nach angeordnet werden. Dazu multipliziert man die Brüche mit dem Hauptnenner (hier ist das $60$).
Die Nenner verschwinden und es entstehen die ganzen Zahlen
$$
\frac23\cdot 60\;=\; 2\cdot 20\;=\; 40 \;\; ,\;\;
\frac{32}{12}\cdot 60 \;=\; 32\cdot 5\;=\; 160 \;\; , \;\;
\frac{12}{15}\cdot 60 \;=\; 12\cdot 4 \;=\; 48 \MDFPeriod
$$
Anordnen nach Größe ergibt $40<48<160$. Damit ist dann $\frac23<\frac{12}{15}<\frac{32}{12}$ da die Multiplikation der Brüche mit der gleichen Zahl $60$
die Anordnung der Brüche nicht verändert (im Abschnitt \MNRef{M03_Ungleichungen} über Ungleichungen und wie man diese umformt).
\end{MExample}

\begin{MExercise}
Wie lautet die Anordnung der Brüche $\frac{16}{15}$, $\frac12$, $\frac23$, $\frac2{-3}$, $\frac{60}{90}$ und $\frac43$ der Größe nach?
\ \\
\MLParsedQuestion{7}{2/(-3)}{5}{VBKM01_USR1}\ $<$\ 
\MLParsedQuestion{7}{1/2}{5}{VBKM01_USR2}\ $<$\ 
\MLParsedQuestion{7}{60/90}{5}{VBKM01_USR3}\ $=$\ 
\MLParsedQuestion{7}{2/3}{5}{VBKM01_USR4}\ $<$\ 
\MLParsedQuestion{7}{16/15}{5}{VBKM01_USR5} \ $<$\ 
\MLParsedQuestion{7}{4/3}{5}{VBKM01_USR6} .
\ \\
\begin{MHint}{\iSolution}
Durchmultiplizieren mit dem Hauptnenner $180$ ergibt die Zahlen $192$, $90$, $120$, $-120$, $120$ und $240$, was auf die Anordnung
$$
-120\; < \; 90 \; < \; 120 \; = \; 120\; < \; 192 \; < \; 240
$$
und damit auf
$$
\frac2{-3} \; < \; \frac12 \;<\; \frac{60}{90} \; =\; \frac23\; < \; \frac{16}{15} \; < \; \frac{4}{3}
$$
führt.
\end{MHint}
\end{MExercise}

\end{MXContent}

\begin{MExercises}
\MDeclareSiteUXID{VBKM01_Bruchrechnung_Exercises}

\begin{MExercise}
Kürzen Sie die folgenden Brüche soweit wie möglich:
\begin{MExerciseItems}
\item{\MEquationItem{$\displaystyle \Mdfrac{216}{240}$}{\MLSpecialQuestion{8}{9/10}{5}{0}{0}{exactfraction}{BRF1}}. \begin{MHint}{Lösung}Wegen $\MGGT(216,240)=24$ ist $\displaystyle\Mdfrac{216}{240}=\Mdfrac{216:24}{240:24}=\Mdfrac{9}{10}$.\end{MHint}}
\item{\MEquationItem{$\displaystyle \Mdfrac{36}{72}$}{\MLSpecialQuestion{8}{1/2}{5}{0}{0}{exactfraction}{BRF2}}. \begin{MHint}{Lösung}$36$ teilt $72$, also ist $\Mdfrac{36}{72}=\Mdfrac{1}{2}.$\end{MHint}} 
\item{\MEquationItem{$\displaystyle \Mdfrac{48}{144}$}{\MLSpecialQuestion{8}{1/3}{5}{0}{0}{exactfraction}{BRF3}}. \begin{MHint}{Lösung}$48$ teilt $144$, also ist $\Mdfrac{48}{144}=\Mdfrac{1}{3}.$\end{MHint}}
\item{\MEquationItem{$\displaystyle\Mdfrac{-a+2b}{-4b+2a}$}{\MLSpecialQuestion{8}{-1/2}{5}{0}{0}{exactfraction}{BRF4}} falls $a$ nicht gleich \MLSimplifyQuestion{5}{2*b}{5}{a,b}{5}{1}{BRF4b} ist. \begin{MHint}{Lösung}Kürzen ergibt $\Mdfrac{-a+2b}{-4b+2a}=\Mdfrac{(-1)\cdot (-2b+a)}{2\cdot (-2b+a)}=-\Mdfrac12$. Der Bruch an sich ist nur für $a\not=2b$ definiert, da sonst der Nenner gleich Null wird.\end{MHint}} 
%\item $\displaystyle\Mdfrac{18\cdot u^2\cdot v \cdot w^3}{45\cdot u\cdot v\cdot w^2}$, \quad $u,v,w\neq 0$\\\begin{MHint}{Lösung} $$\Mdfrac{18\cdot u^2 \cdot v\cdot w^3}{45\cdot u\cdot v\cdot w^2}=\Mdfrac{2\cdot u\cdot w}{5}$$\end{MHint} 
\end{MExerciseItems}
\MInputHint{Geben Sie die Brüche in der Form \texttt{Zähler/Nenner} maximal gekürzt und mit positivem Nenner ein. Ziehen Sie Vorzeichen stets vor die Brüche
und verwenden Sie keine Klammern.}
\end{MExercise}

% \begin{MExercise}
% Eine wichtige Anwendung der Bruchrechnung findet sich in der Musik. 
% \begin{enumerate}
% \item Stellen Sie fest, um welche Takte es sich hierbei handelt, wenn die folgenden Noten in einem Takt gegeben sind:
% 
% \textbf{a)} $\displaystyle \Mdfrac18, \Mdfrac18, \Mdfrac12$ \qquad \textbf{b)} $\displaystyle \Mdfrac{1}{16}, \Mdfrac14, \Mdfrac{1}{16}, \Mdfrac14, \Mdfrac18$ \qquad \textbf{c)} $\displaystyle \Mdfrac18, \Mdfrac{1}{16}, \Mdfrac14, \Mdfrac{1}{16}$
% \ \\
% \begin{MHint}{Lösung}
% Die Summe der Notenlängen zeigt an, wieviele ganze Noten ein Takt besitzt:
% \begin{itemize}
% \item{$\displaystyle \Mdfrac18+\Mdfrac18+\Mdfrac12=\Mdfrac{1+1+4}{8}=\Mdfrac68$ oder $\displaystyle \Mdfrac34$,}
% \item{$\displaystyle \Mdfrac{1}{16}+ \Mdfrac14+ \Mdfrac{1}{16}+ \Mdfrac14+ \Mdfrac18=\Mdfrac{1+4+1+4+2}{16}=\Mdfrac{12}{16}=\Mdfrac{6}{8}$ oder $\displaystyle \Mdfrac34$,}
% \item{$\displaystyle \Mdfrac18+ \Mdfrac{1}{16}+ \Mdfrac14+ \Mdfrac{1}{16}=\Mdfrac{2+1+4+1}{16}=\Mdfrac{8}{16}=\Mdfrac12$ oder $\displaystyle \Mdfrac24$.}
% \end{itemize}
% \end{MHint}
% \ \\
% \item Welche Noten fehlen, damit es sich um einen $\displaystyle \Mdfrac34$-Takt handelt?
% \ \\
% \textbf{a)} $\displaystyle \Mdfrac18, \Mdfrac14$ \qquad \textbf{b)} $\displaystyle \Mdfrac{1}{16}, \Mdfrac18, \Mdfrac18$
% \ \\
% \begin{MHint}{Lösung}
% Wir ergänzen die Summe der Noten zum Bruch $\Mdfrac34$:
% \begin{itemize}
% \item{$\displaystyle \Mdfrac18+ \Mdfrac14=\Mdfrac38$, addiert man drei Achtelnoten, so erhält man $\displaystyle \Mdfrac38+\Mdfrac38=\Mdfrac68=\Mdfrac34$,}
% \item{$\displaystyle \Mdfrac{1}{16}+ \Mdfrac18+ \Mdfrac18=\Mdfrac{5}{16}$, addiert man 7 Sechzehntelnoten (oder eine andere Aufteilung die zu $\displaystyle \Mdfrac7{16}$ in der Summe führt), so ist die Gesamtsumme $\displaystyle \Mdfrac5{16}+\Mdfrac7{16}=\Mdfrac{12}{16}=\Mdfrac34$.}
% \end{itemize}
% \end{MHint}
% \end{enumerate} 
% 
% \end{MExercise}

\begin{MExercise}
Berechnen bzw. vereinfachen Sie die folgenden Ausdrücke so weit wie möglich:
\begin{MExerciseItems}
\item{\MEquationItem{$\displaystyle \Mdfrac12-\Mdfrac27+\Mdfrac38+\Mdfrac34$}{\MLSpecialQuestion{10}{75/56}{5}{0}{0}{exactfraction}{BRX1}}. \begin{MHint}{Lösung}Summieren über den Hauptnenner ergibt $\Mdfrac12-\Mdfrac27+\Mdfrac38+\Mdfrac34=\Mdfrac{28}{56}-\Mdfrac{16}{56}+\Mdfrac{21}{56}+\Mdfrac{42}{56}=\Mdfrac{75}{56}$ da $\MGGT(2,7,8,4)=56$ ist.\end{MHint}}
\item{\MEquationItem{$\displaystyle \Mdfrac{3}{13}:\Mdfrac{7}{26}$}{\MLSpecialQuestion{10}{6/7}{5}{0}{0}{exactfraction}{BRX2}}. \begin{MHint}{Lösung}Die Division durch einen Bruch ist das Gleiche wie die Multiplikation
mit seinem Kehrwert: $\Mdfrac{3}{13}:\Mdfrac{7}{26}=\Mdfrac{3}{13}\cdot\Mdfrac{26}{7}=\Mdfrac{3\cdot 26}{13\cdot 7}=\Mdfrac{3\cdot 2}{1\cdot 7}=\Mdfrac67$.\end{MHint}}
\item{\MEquationItem{$\displaystyle \left(\MZahl{1}{}\bar{4}\cdot 3-\Mdfrac12\right)\cdot\Mdfrac{6}{7}$}{\MLSpecialQuestion{10}{23/7}{5}{0}{0}{exactfraction}{BRX3}}. \begin{MHint}{Lösung}Umwandeln des Dezimalausdrucks in einen Bruch ergibt
$\left(\MZahl{1}{}\bar{4}\cdot 3-\Mdfrac12\right)\cdot\Mdfrac{6}{7}=\left(\MZahl{1}{}\bar{4}\cdot 18-3\right)\cdot\Mdfrac{1}{7}=\left(26-3\right)\cdot\Mdfrac{1}{7}=\Mdfrac{23}{7}$.\end{MHint}}
%%\item{\MEquationItem{$\displaystyle \Mdfrac{3a}{3a+6b}+\Mdfrac{2b}{a+2b}$}{\MLSpecialQuestion{10}{1}{5}{0}{0}{exactfraction}{BRX4}}. \begin{MHint}{Lösung}Summieren über den Hauptnenner ergibt
%%$\Mdfrac{3a}{3a+6b}+\Mdfrac{2b}{a+2b}=\Mdfrac{3a}{3a+6b}+\Mdfrac{6b}{3a+6b}=\Mdfrac{3a+6b}{3a+6b}=1$.\end{MHint}}
%%\item{\MEquationItem{$\displaystyle \Mdfrac{18x^2-48x y+32y^2}{12y-9x}\cdot \Mdfrac{18x+24y}{9x^2-16y^2}$}{\MLSpecialQuestion{10}{-4}{5}{0}{0}{exactfraction}{BRX5}}. \begin{MHint}{Lösung}Vereinfachen des Ausdrucks ergibt
%%\begin{eqnarray*}
%%\Mdfrac{18x^2-48x y+32y^2}{12y-9x}\cdot \Mdfrac{18x+24y}{9x^2-16y^2} & =&
%%4\cdot \Mdfrac{9x^2-24x y+16y^2}{4y-3x}\cdot \Mdfrac{3x+4y}{9x^2-16y^2}\\
%%& =& 4\cdot \Mdfrac{(3x-4y)^2}{4y-3x}\cdot \Mdfrac{3x+4y}{9x^2-16y^2}\\
%%& =& -4\cdot \Mdfrac{(3x-4y)^2}{3x-4y}\cdot \Mdfrac{3x+4y}{(3x+4y)(3x-4y)}=-4 \MDFPeriod
%%\end{eqnarray*}\end{MHint}}
\end{MExerciseItems}
\MInputHint{Geben Sie die Brüche maximal gekürzt und mit positivem Nenner ein.}
\end{MExercise}

\begin{MExercise}
Wandeln Sie die folgenden unendlichen periodischen Dezimalbrüche in Brüche um und kürzen Sie soweit wie möglich:
\begin{MExerciseItems}
\item{\MEquationItem{$\displaystyle \MZahl{0}{}\overline{4}$}{\MLSpecialQuestion{12}{4/9}{5}{0}{0}{exactfraction}{BRZ1}}.}
\item{\MEquationItem{$\displaystyle \MZahl{0}{}\overline{23}$}{\MLSpecialQuestion{12}{23/99}{5}{0}{0}{exactfraction}{BRZ2}}.}
\item{\MEquationItem{$\displaystyle \MZahl{0}{12}\overline{34}$}{\MLSpecialQuestion{12}{1222/9900}{9}{0}{0}{exactfraction}{BRZ4}}.}
\item{\MEquationItem{$\displaystyle \MZahl{0}{}\overline{9}$}{\MLSpecialQuestion{12}{1}{5}{0}{0}{exactfraction}{BRZ5}}.}
\end{MExerciseItems}
\MInputHint{Geben Sie die Brüche maximal gekürzt und mit positivem Nenner ein.}
\end{MExercise}
\ \\
\begin{MHint}{Lösung}
Mithilfe des \MSRef{Mathematik_Grundlagen_UDB}{Umformungstricks} für unendliche Dezimalbrüche erhält man diese Lösungen:
\begin{itemize}
\item{$x=\MZahl{0}{}\overline{4}$, also $\displaystyle 10x-x=4\;\Rightarrow\;9x=4\;\Rightarrow\;x=\Mtfrac49$,}
\item{$x=\MZahl{0}{}\overline{23}$, also $\displaystyle 100x-x=23\;\Rightarrow\;99x=23\;\Rightarrow\; x=\Mtfrac{23}{99}$,}
\item{$x=\MZahl{0}{12}\overline{34}$, also $\displaystyle 100x-x=\MZahl{12}{22}\;\Rightarrow\;99x=\Mtfrac{1222}{100}\;\Rightarrow\;x=\Mtfrac{1222}{9900}=\Mtfrac{611}{4950}$,}
\item{$x=\MZahl{0}{}\overline{9}$, also $\displaystyle 10x-x=9\;\Rightarrow\;9x=9\;\Rightarrow\;x=1$.}
 \end{itemize}
 Beim letzten Aufgabenteil ist zu beachten, dass $1=\MZahl{1}{000}\ldots$ und $1=\MZahl{0}{999}\ldots=\MZahl{0}{}\overline{9}$
 zwei verschiedene Dezimalbruchdarstellungen für die gleiche Zahl sind.
\end{MHint}
\end{MExercises}

\MSubsection{Umformen von Termen}
\MLabel{M01_Umformen}


\begin{MIntro}
\MDeclareSiteUXID{VBKM01_UmformenIntro}
Was genau sind Terme? 
\begin{MInfo}
\MEntry{Terme}{Terme} sind Rechenausdrücke, die eine Kombination von Zahlen, Variablen, Klammern und geeigneten Rechenoperationen darstellen.
\end{MInfo}

Terme kann man auf zwei Arten interpretieren:
\begin{itemize}
\item{Als funktionale Ausdrücke: Wenn man für die im Term auftretenden Variablen konkrete Zahlen einsetzt, so ergibt der Term einen Zahlenwert.
Beispielsweise ist $x+x-1$ ein Term und sobald man $x=2$ einsetzt erhält man den Wert $3$. Auch $2x-1$ ist ein Term, dieser Term
kann zu $x+x-1$ umgeformt werden und ergibt daher den gleichen Wert, wenn man $x=2$ einsetzt. Als symbolischer Ausdruck an sich ist $x+x-1$ verschieden von
$2x-1$, als funktionaler Ausdruck sind beide aber gleich: Egal welchen Wert man für $x$ einsetzt, beide Terme ergeben immer das gleiche Endergebnis.
Ein Term kann auch an sich einen Wert darstellen, wenn keine Variablen auftreten. Beispielsweise ist $3\cdot (2+4)$ ein Term mit Wert
$18$.}
\item{Als Auswertungsvorschrift: Ein Term kann als eine Art Anleitung interpretiert werden, wie man aus gegebenen Werten (in den Variablen)
einen neuen Wert berechnet. Beispielsweise kann man den Term $x^2-1$ lesen als \glqq Quadriere den Wert in $x$ und ziehe Eins ab\grqq.
Er ist verschieden von dem Term $(x+1)(x-1)$, auch wenn gleiche Werte herauskommen. Der zweite Term beschreibt die Auswertung als
\glqq Addiere Eins zu $x$ und multipliziere mit dem Wert, der entsteht, wenn man von $x$ Eins abzieht\grqq.
Beide Terme sind mathematisch gleich. Man schreibt $x^2-1=(x-1)(x+1)$, stellt aber zwei verschiedene Möglichkeiten dar,
den Wert auszurechnen. Je nach Problemstellung kann einer der beiden Terme besser geeignet sein, um das Problem zu lösen.}
\end{itemize}

\end{MIntro}


\begin{MXContent}{Termumformungen}{Termumformungen}{STD}
\MDeclareSiteUXID{VBKM01_Termumformungen}

\ \\
Interessant wird der Umgang mit Termen, wenn die Frage der Gleichheit zweier Termausdrücke gestellt wird oder komplizierte Terme vereinfacht werden sollen. 

\begin{MInfo}\MLabel{Mathematik_Grundlagen_BinomischeFormel}
Zwei Terme sind \MEntry{gleich}{Gleichheit (Terme)}, wenn sie durch zulässige Termumformungen ineinander überführt werden können. Komplizierte Terme können durch die Anwendung von Rechengesetzen vereinfacht werden. Hierbei sind folgende Regeln zu beachten:
\begin{enumerate}
\item Es gilt Potenzrechnung vor Punktrechnung vor Strichrechnung.
\item Beim Rechnen mit Klammern gelten die Distributivgesetze:
$$a\cdot(b\pm c) = a\cdot b\pm a\cdot c \MDFPSpace, \MDFPaSpace (a\pm b)\cdot c = a\cdot c \pm b\cdot c \MDFPeriod$$
\item Mit $d\neq 0$ gilt: $\displaystyle (a\pm b):d = \Mtfrac{a}{d}\pm \Mtfrac{b}{d}.$
\item Bei geschachtelten Klammerausdrücken werden zunächst die inneren und dann die äußeren Klammern unter Beachtung der Rechengesetze aufgelöst.
\end{enumerate}
\end{MInfo}

\begin{MExercise}
Lösen Sie die Klammern auf und vereinfachen Sie die Terme soweit möglich:
\begin{MExerciseItems}
\item{\MEquationItem{$\displaystyle (1-a)\cdot(1-b)$}{\MLSimplifyQuestion{20}{(1-a)*(1-b)}{5}{a,b}{5}{1}{PMPF2}}. \begin{MHint}{Lösung}$$(1-a)\cdot(1-b)\;=\; 1-a-b+a b \MDFPeriod $$\end{MHint}}
\item{\MEquationItem{$\displaystyle 5a-(2b-(2a-7b)+4a)-3b$}{\MLSimplifyQuestion{20}{3*a-12*b}{5}{a,b}{5}{1}{PMPF1}}. \begin{MHint}{Lösung}$$5a-(2b-(2a-7b)+4a)-3b \; = \; 5a-2b+2a-7b-4a-3b \;=\; 3a-12b \MDFPeriod  $$\end{MHint}}
\end{MExerciseItems}
\MInputHint{In den Eingaben dürfen keine Klammern mehr auftreten.}
\end{MExercise}

\begin{MInfo}
Die drei \MEntry{binomischen Formeln}{Binomische Formeln} lauten:
$$ (a+b)^2 = a^2+2a b+b^2 \MDFPSpace, \MDFPaSpace (a-b)^2 = a^2-2a b+b^2 \MDFPSpace,\MDFPaSpace (a+b)(a-b) = a^2-b^2 \MDFPeriod $$
\end{MInfo}

% Aufgaben und Fließtext !

Für $a$ und $b$ können dabei sowohl Zahlen als auch ganze Terme auftreten:

\begin{MExample}
Hier ein paar typische Anwendungen der binomischen Formeln:
\begin{itemize}
\item{$(1+2x)^2=1^2+2\cdot1\cdot 2x+(2x)^2=1+4x+4x^2$.}
\item{$(1+2x)(1-2x)=1^2-(2x)^2=1-4x^2$.}
\item{$x^4-1=(x^2+1)(x^2-1)=(x^2+1)(x+1)(x-1)$, daran kann man ablesen, dass $x^4-1$ nur die Nullstellen $x=1$ und $x=-1$ in den reellen Zahlen besitzt.}
\item{$(1+x+y)^2=\left({(1+x)+y}\right)^2= (1+x)^2+2(1+x)y+y^2=1+2x+x^2+2y+2x y+y^2$.}
\end{itemize}
\end{MExample}

\begin{MExercise}
Wenden Sie eine binomische Formel an, um den Term zu vereinfachen:\ \\ \ \\
\MEquationItem{$\displaystyle (-3x+4)(4-3x)$}{\MLSimplifyQuestion{30}{16-24*x+9*x^2}{5}{x}{5}{1}{BMPF4}}.

\begin{MHint}{Lösung}$$(-3x+4)(4-3x)\;=\; (4-3x)(4-3x)\;=\; (4-3x)^2\;=\; 16-24x+9x^2 \MDFPeriod $$\end{MHint}%%%
\end{MExercise}

\begin{MExample}
Die binomischen Formeln können zum geschickten Umformen von quadratischen Ausdrücken verwendet werden.
%%% doppelt gewesen
Dies ist sehr hilfreich, wenn Zahlenquadrate ohne Taschenrechner auszurechnen sind. Dabei zerlegt man
die zu quadrierende Zahl in eine einfache Zahl (meist eine Zehnerpotenz) und den Rest:
\begin{eqnarray*}
103^2 & = & (100+3)^2 = 100^2+2\cdot 100\cdot 3+3^2 = 10609 \MDFPSpace,\\
49^2 & = & (50-1)^2 = 50^2-2\cdot 50\cdot 1 +1^2 = 2401 \MDFPSpace,\\
61^2-59^2 & = & (61-59)(61+59) = 2\cdot 120 = 240 \MDFPeriod
\end{eqnarray*}
\end{MExample}

\begin{MExercise}
Berechnen Sie mit der Technik aus dem vorherigen Beispiel \MEquationItem{$\displaystyle 1005^2$}{\MLSimplifyQuestion{8}{1010025}{5}{x}{5}{1024}{BMPF0}}.

\begin{MHint}{Lösung}
$$1005^2 \;=\; (1000+5)^2\;=\; 1000000+2\cdot 1000\cdot 5+25 \;=\; 1010025 \MDFPeriod $$\end{MHint}
\end{MExercise}

Im folgenden \MSRef{VBKM01_AufgabenUmformen}{Aufgabenabschnitt} können Sie die Umformungstechniken an zahlreichen Aufgaben einüben.

\end{MXContent}

\begin{MExercises}
\MLabel{VBKM01_AufgabenUmformen}
\MDeclareSiteUXID{VBKM01_Termumformungen_Exercises}
\begin{MExercise}
Vereinfachen Sie die folgenden Terme für geeignete Zahlen $a,b,x,y,z$:
\begin{MExerciseItems}
\item{\MEquationItem{$\displaystyle \Mdfrac{3x -6x y^2+4 x y z}{-2x}$}{\MLSimplifyQuestion{20}{-3/2+3*y^2-2*y*z}{5}{x,y,z}{5}{1}{UMPF2}}. \begin{MHint}{Lösung}$$\Mdfrac{3x -6x y^2+4 x y z}{-2x}\;=\; -\Mdfrac32+3y^2-2y z \MDFPeriod  $$\end{MHint}}
\item{\MEquationItem{$\displaystyle (3a-2b)\cdot(4a-6)$}{\MLSimplifyQuestion{26}{12*a^2-18*a-8*a*b+12*b}{5}{a,b}{5}{1}{UMPF3}}. \begin{MHint}{Lösung} $$(3a-2b)\cdot(4a-6)\;=\; 12a^2-18a-8a b+12b\MDFPeriod  $$\end{MHint}}
\item{\MEquationItem{$\displaystyle (2a+3b)^2-(3a-2b)^2$}{\MLSimplifyQuestion{26}{-5*a^2+24*a*b+5*b^2}{5}{a,b}{5}{1}{UMPF4}}. \begin{MHint}{Lösung} \begin{eqnarray*}(2a+3b)^2-(3a-2b)^2&=& (4a^2+12a b+9b^2)-(9a^2-12a b+4b^2)\ \\ &=& -5a^2+24a b+5b^2  \: . \end{eqnarray*}\end{MHint}}
%%\item{\MEquationItem{$\displaystyle \Mdfrac{3a}{3a+6b}+\Mdfrac{2b}{a+2b}$}{\MLSimplifyQuestion{25}{1}{5}{a,b}{5}{1}{UMPF7}}. \begin{MHint}{Lösung}$$\Mdfrac{3a}{3a+6b}+\Mdfrac{2b}{a+2b} \; = \; \Mdfrac{a}{a+2b}+\Mdfrac{2b}{a+2b}\;=\; \Mdfrac{a+2b}{a+2b} \;=\; 1 \MDFPeriod $$\end{MHint}}
\item{\MEquationItem{$\displaystyle \Mdfrac{3a}{3a+6b}+\Mdfrac{2b}{a+2b}$}{\MLSpecialQuestion{10}{1}{5}{0}{0}{exactfraction}{UMPF7}}. \begin{MHint}{Lösung}Summieren über den Hauptnenner ergibt
$\Mdfrac{3a}{3a+6b}+\Mdfrac{2b}{a+2b}=\Mdfrac{3a}{3a+6b}+\Mdfrac{6b}{3a+6b}=\Mdfrac{3a+6b}{3a+6b}=1$.\end{MHint}}
\end{MExerciseItems}
\MInputHint{In den Eingaben dürfen keine Klammern mehr auftreten.}
\end{MExercise}

\begin{MExercise}
Diese Aufgaben erfordern etwas mehr Durchhaltevermögen. Vereinfachen Sie:\ \\
\begin{MExerciseItems}
\item{\MEquationItem{$\displaystyle \Mdfrac12x(4x+3y)+\Mdfrac32(5x^2-6x y)$}{\MLSimplifyQuestion{36}{19/2*x^2-15/2*x*y}{5}{x,y}{5}{1}{UMPF5}}. \begin{MHint}{Lösung} $$\Mdfrac12x(4x+3y)+\Mdfrac32(5x^2-6x y)\;=\; 2x^2+\Mdfrac32x y+\Mdfrac{15}2x^2-9x y \;=\; \Mdfrac{19}{2}x^2-\Mdfrac{15}{2}x y  \MDFPeriod  $$\end{MHint}}
%%\item{\MEquationItem{$\displaystyle \Mdfrac{18x^2-48x y+32y^2}{12y-9x}\cdot \Mdfrac{18x+24y}{9x^2-16y}$}{\MLSimplifyQuestion{25}{-4}{5}{x,y}{5}{1}{UMPF8}}. \begin{MHint}{Lösung}\begin{eqnarray*}
%%    \Mdfrac{18x^2-48x y+32y^2}{12y-9x}\cdot \Mdfrac{18x+24y}{9x^2-16y}  & = &
%%    \Mdfrac{2\cdot (3x-4y)^2}{-3\cdot(3x-4y)}\cdot \Mdfrac{6\cdot(3x+4y)}{(3x+4y)(3x-4y)}\ \\
%%    &=&-\Mdfrac23(3x-4y)\cdot \Mdfrac{6}{3x-4y} \;=\; -4\: .
%%    \end{eqnarray*}\end{MHint}}
\item{\MEquationItem{$\displaystyle \Mdfrac{18x^2-48x y+32y^2}{12y-9x}\cdot \Mdfrac{18x+24y}{9x^2-16y^2}$}{\MLSpecialQuestion{10}{-4}{5}{0}{0}{exactfraction}{UMPF8}}. \begin{MHint}{Lösung}Vereinfachen des Ausdrucks ergibt
\begin{eqnarray*}
\Mdfrac{18x^2-48x y+32y^2}{12y-9x}\cdot \Mdfrac{18x+24y}{9x^2-16y^2} & =&
\Mdfrac{2\cdot (9x^2-24x y+16y^2)}{3\cdot (4y-3x)}\cdot \Mdfrac{6\cdot (3x+4y)}{9x^2-16y^2}\\ %%%
& =& 4\cdot \Mdfrac{(3x-4y)^2}{4y-3x}\cdot \Mdfrac{3x+4y}{9x^2-16y^2}\\
& =& 4\cdot \Mdfrac{(3x-4y)^2}{(-1)\cdot (3x-4y)}\cdot \Mdfrac{3x+4y}{(3x+4y)(3x-4y)}=-4 \MDFPeriod %%%
\end{eqnarray*}\end{MHint}}
\item{\MEquationItem{$\displaystyle (a^2+5a-2)(2a^2-3a-9)-\left(\Mdfrac12a^2+3a-5\right)(a^2-4a+3)$}{\MLSimplifyQuestion{35}{3/2*a^4+6*a^3-25/2*a^2-68*a+33}{5}{a}{5}{1}{UMPF6}}. \begin{MHint}{Lösung}$$ 
(a^2+5a-2)(2a^2-3a-9)-\left(\Mdfrac12a^2+3a-5\right)(a^2-4a+3)\ \ \ \ \ \ \ \ \ \ \ \ 
$$
\begin{eqnarray*}
     & = & 2a^4+10a^3-4a^2-3a^3-15a^2+6a-9a^2-45a+18\\&&\ \ \ \ \ \  -\left({\Mdfrac12a^4+3a^3-5a^2-2a^3-12a^2+20a+\Mdfrac32a^2+9a-15}\right)\ \\ \ \\
     &= & \Mdfrac32a^4+6a^3-\Mdfrac{25}{2}a^2-68a+33 \MDFPeriod
    \end{eqnarray*}\end{MHint}}
\end{MExerciseItems}
% Hier Eingabe des Originals bei b) noch moeglich wenn man das hoch-2-Symbol nimmt
\MInputHint{In den Eingaben dürfen keine Klammern mehr auftreten.}
\end{MExercise}

\begin{MExercise}
Berechnen Sie mit Hilfe einer binomischen Formel:
\begin{MExerciseItems}
\item{\MEquationItem{$\displaystyle 43^2$}{\MLSimplifyQuestion{6}{1849}{5}{x}{5}{1024}{BMPF1}}. \begin{MHint}{Lösung}$$43^2 \;=\; (40+3)^2 \;=\; 40^2+2\cdot 40\cdot 3+3^2\;=\; 1600+240+9 \;=\; 1849 \MDFPeriod $$\end{MHint}}
\item{\MEquationItem{$\displaystyle 97^2$}{\MLSimplifyQuestion{6}{9409}{5}{x}{5}{1024}{BMPF2}}. \begin{MHint}{Lösung}$$97^2\;=\; (100-3)^2 \;=\; 100^2-2\cdot100\cdot 3+3^2 \;=\; 10000-600+9\;=\; 9409 \MDFPeriod $$\end{MHint}}%%%
\item{\MEquationItem{$\displaystyle 41^2-38^2$}{\MLSimplifyQuestion{6}{237}{5}{x}{5}{1024}{BMPF3}}. \begin{MHint}{Lösung}$$41^2-38^2\;=\; (41+38)(41-38)\;=\; 79\cdot 3\;=\; 237 \MDFPeriod $$\end{MHint}}
\end{MExerciseItems}
\end{MExercise}

\begin{MExercise}
Wenden Sie eine binomische Formel an, um das Produkt aufzulösen, und fassen Sie das Ergebnis zusammen:
\begin{MExerciseItems}
\item{\MEquationItem{$\displaystyle (-5x y-2)^2$}{\MLSimplifyQuestion{30}{25*x^2*y^2+20*x*y+4}{5}{x,y}{5}{1}{BMPF5}}. \begin{MHint}{Lösung}$$(-5x y-2)^2 \;=\; (-1)^2\cdot(5x y+2)^2 \;=\; 25x^2y^2+20x y+4 \MDFPeriod $$\end{MHint}}
\item{\MEquationItem{$\displaystyle (-6a b+7b c)(-6a b-7b c)$}{\MLSimplifyQuestion{36}{36*a^2*b^2-49*b^2*c^2}{5}{a,b,c}{5}{1}{BMPF6}}. \begin{MHint}{Lösung}$$ (-6a b+7b c)(-6a b-7b c) \;=\; (-6a b)^2-(7b c)^2 \;=\; 36a^2b^2-49b^2c^2 \MDFPeriod $$\end{MHint}}
\item{\MEquationItem{$\displaystyle (-6a b+7b c)(-6a b+7b c)$}{\MLSimplifyQuestion{36}{36*a^2*b^2-84*a*b^2c+49*b^2*c^2}{5}{a,b,c}{5}{1}{BMPF7}}. \begin{MHint}{Lösung}$$ (-6a b+7b c)(-6a b+7b c) \;=\; (-6a b +7b c)^2 \;=\; 36a^2b^2-84a b^2c+49b^2c^2 \MDFPeriod $$\end{MHint}} 
\item{\MEquationItem{$\displaystyle (x^2+3)(-x^2-3)$}{\MLSimplifyQuestion{20}{-x^4-6*x^2-9}{5}{x}{5}{1}{BMPF8}}. \begin{MHint}{Lösung}$$ (x^2+3)(-x^2-3)\;=\; -(x^2+3)^2\;=\; -x^4-6x^2-9 \MDFPeriod $$\end{MHint}}
\end{MExerciseItems}
\MInputHint{In den Eingaben dürfen keine Klammern mehr auftreten.}
\end{MExercise}
\begin{MExercise}
Faktorisieren Sie die folgenden Terme so weit wie möglich mit Hilfe einer binomischen Formel:
\begin{MExerciseItems}
\item{\MEquationItem{$\displaystyle 4x^2+12x y+9y^2$}{\MLSimplifyQuestion{15}{(2*x+3*y)^2}{5}{x,y}{5}{0}{BMPF9}}. \begin{MHint}{Lösung}$$4x^2+12x y+9y^2 \;=\; (2x+3y)^2   \MDFPeriod $$\end{MHint}}
\item{\MEquationItem{$\displaystyle 64a^2-96a+36$}{\MLSimplifyQuestion{10}{(8*a-6)^2}{5}{a}{5}{0}{BMPF10}}. \begin{MHint}{Lösung}$$64a^2-96a+36\;=\;(8a-6)^2   \MDFPeriod $$\end{MHint}}
% Aufgabentyp faktorisieren ohne Platztrick noch nicht machbar:
%\item{\MEquationItem{$\displaystyle 9x^2-16$}{\MLSimplifyQuestion{20}{(3*x+4)*(3*x-4)}{5}{x}{5}{0}{BMPF11}}. \begin{MHint}{Lösung}$$9x^2-16\;=\; (3x+4)(3x-4)  \MDFPeriod $$\end{MHint}}
%\item{\MEquationItem{$\displaystyle 81-\Mdfrac14y^2$\\\begin{MHint}{Lösung}$$81-\Mdfrac14y^2\;=\;\left({9+\Mdfrac12y}\right)\left({9-\Mdfrac12y}\right)\MDFPeriod $$\end{MHint}   
\item{\MEquationItem{$\displaystyle 25x^2-16y^2+15x+12y$}{\MLSimplifyQuestion{22}{(5*x+4*y)*(5*x-4*y+3)}{5}{x,y}{5}{0}{BMPF15}}. \begin{MHint}{Lösung}\begin{eqnarray*}25x^2-16y^2+15x+12y&=&(5x)^2-(4y)^2+3\cdot(5x+4y)\ \\ &=& (5x+4y)(5x-4y)+3\cdot(5x+4y)\ \\ &=& (5x+4y)(5x-4y+3)  \: .\end{eqnarray*}\end{MHint}}
\end{MExerciseItems}
\end{MExercise}
\MInputHint{Faktorisieren Sie die das Ergebnis so weit, dass es in die Eingabefelder passt.}
\end{MExercises}


\begin{MXContent}{Summen- und Produktdarstellung}{Summen/Produkte}{STD}
\MLabel{M01_SummenProdukte}
\MDeclareSiteUXID{VBKM01_SummenProdukte}

% Beispiel der Einbettung eines GeoGebra-getubeten Applets:
% 
% % Code zwischen begin und end wurde 1:1 aus Einbettungsdownload von tube.geogebra.org uebernommen
% % Wichtig bei GeoGebra: MDirectHTML und nicht \begin{html}...\end{html} nutzen.
% \ifttm
% \begin{MDirectHTML}
% <iframe scrolling="no" src="https://www.geogebratube.org/material/iframe/id/17652/width/600/height/500/border/888888/rc/false/ai/false/sdz/true/smb/false/stb/false/stbh/true/ld/false/sri/true/at/preferjava" width="600px" height="500px" style="border:0px;"> </iframe>
% \end{MDirectHTML}
% \else
% Kein Applet im PDF
% \fi
% Hier ist das Applet zuende.


Mathematische Ausdrücke und Terme kann man auf verschiedene Arten notieren, die jeweils bestimmte Vor- und Nachteile haben. Dabei unterscheidet man im Wesentlichen,
welche mathematischen Operationen zuletzt im Ausdruck ausgeführt werden. Die wichtigsten Typen sind Summen- und Produktdarstellungen.

\begin{MInfo}
Bei einer \MEntry{Produktdarstellung}{Produktdarstellung} ist die Produktoperation die zuletzt ausgeführte Operation. Wegen der Punkt-vor-Strich-Regel erreicht man diese Form nur dadurch,
dass man Klammern um die Faktoren setzt. Aus der Produktdarstellung kann man besonders einfach ablesen, wann der fragliche Term den Wert Null annimmt. Das passiert genau dann, wenn
einer der Faktoren (d.h. meist eine der Klammern) Null wird.
\end{MInfo}

Beispielsweise wird der Term $(x-1)\cdot (x-2)$ zu Null, falls $x=1$ oder $x=2$ eingesetzt wird. Für alle anderen Werte für $x$ ist er nicht Null. 

\begin{MInfo}
Bei einer \MEntry{Summendarstellung}{Summendarstellung} ist Addition bzw. Subtraktion die zuletzt ausgeführte Operation. Wegen der Punkt-vor-Strich-Regel sind Terme ohne Klammern
automatisch in dieser Form. In der Summendarstellung lässt sich das asymptotische Verhalten eines Ausdrucks besonders leicht ablesen.
Mit dem asymptotischen Verhalten einer Funktion wird beschrieben, wie sich
die Funktion verhält, wenn man zu betragsmäßig immer größeren Werten der
Variable $x$ bis an die im Unendlichen liegenden Grenzen des
Definitionsbereichs herangeht.
Bei Polynomen z.B. wird es nur durch den
Term mit dem höchsten Exponenten festgelegt.
\end{MInfo}

Um zwischen beiden Darstellungen zu wechseln, gibt es mehrere Techniken.

\begin{MInfo}
Beim \MEntry{Ausmultiplizieren}{Ausmultiplizieren} werden Faktoren multipliziert, indem jeder Summand eines Faktors mit jedem Summanden des anderen Faktors multipliziert und die Ergebnisse summiert werden.
Liegen mehr als zwei Faktoren vor, so sollten diese schrittweise (immer nur zwei miteinander) ausmultipliziert werden.
\end{MInfo}

\begin{MExample}
Die Funktion $f(x)=(x+3)(x-2)(x+1)$ multipliziert man wie folgt aus:
\begin{eqnarray*}
f(x)
& = & (x+3)\cdot (x-2)\cdot (x+1) \ \\ \ \\
& = & (x^2+3x-2x-6)\cdot (x+1) \ \\ \ \\
& = & (x^2+x-6)\cdot (x+1) \ \\ \ \\
& = & x^3+x^2-6x\: +\: x^2+x-6 \ \\ \ \\
& = & x^3+2x^2-5x-6 \MDFPeriod
\end{eqnarray*}
\end{MExample}

\begin{MExercise}
Multiplizieren Sie diese Terme vollständig aus und fassen Sie zusammen. Geben Sie das asymptotische Verhalten des Gesamtausdrucks an:
\begin{MExerciseItems}
\item{$f(x)\;=\;(3-x)(x+1)$ = \MLSimplifyQuestion{30}{(3-x)*(x+1)}{10}{x}{5}{1}{LSFF3}.
\begin{MHint}{Lösung}$$(3-x)(x+1) \; = \; 3x + 3 -x^2 -x \;=\; 3 + 2x - x^2 $$\end{MHint}\\
%%Das asymptotische Verhalten ist
\begin{center}
%%$\displaystyle \lim_{x\rightarrow +\infty} f(x)\;=\;$
Beschreibung des asymptotischen Verhaltens:\\ 
Wenn $x$ gegen $\infty$ strebt, dann strebt $f(x)$ gegen
\MLSimplifyQuestion{10}{-infty}{3}{infty}{5}{1}{LSFF4}\;. \begin{MHint}{Lösung}%
Die Asymptotik kann, wenn sie sich in eindeutiger Weise ergibt, abkürzend mit dem Symbol \glqq{}lim\grqq{} bezeichnet werden:
$$\lim_{x\rightarrow \infty} f(x) \; = \; -\infty \MDFPeriod$$\end{MHint}\\  % Zusaetzliches Leerzeichen um Loesungen unterschiedlich zu machen
Wenn $x$ gegen $-\infty$ strebt, dann strebt $f(x)$ gegen
%%$\displaystyle \lim_{x\rightarrow -\infty} f(x)\;=\;$
\MLSimplifyQuestion{10}{-infty}{3}{infty}{5}{1}{LSFF5}\;. \\
\begin{MHint}{Lösung}%
In diesem Fall ergibt sich 
$$\lim_{x\rightarrow -\infty} f(x) \; = \; -\infty \MDFPeriod$$\end{MHint}
\end{center}
\MInputHint{Unendliche Grenzwerte $\infty$ und Asymptoten können Sie als \texttt{unendlich} oder \texttt{infinity} eintippen,
entsprechend \texttt{-unendlich} für $-\infty$.
Das asymptotische Verhalten wird in einem späteren Modul erklärt, falls Sie die Symbolik noch nicht beherrschen, können Sie diesen
Aufgabenteil überspringen.}
}
\item{$(x+4)(2-x)(x+2)$ = \MLSimplifyQuestion{30}{(x+4)*(2-x)*(x+2)}{10}{x}{5}{1}{LSFF6}. \begin{MHint}{Lösung}$$(x+4)(2-x)(x+2) \; = \; (x+4)(4-x^2) \;=\; 4x-x^3+16-4x^2 \;=\; 16 + 4x - 4x^2 -x^3 $$\end{MHint}} %%%
\item{$(3-x)(x+1)^2$ = \MLSimplifyQuestion{30}{(3-x)*(x+1)*(x+1)}{10}{x}{5}{1}{LSFF7}. \begin{MHint}{Lösung}$$(3-x)(x+1)^2 \; = \; (3-x)(x^2+2x+1) \;=\; 3x^2+6x+3-x^3-2x^2-x \;=\; 3 + 5x + x^2 -x^3 $$\end{MHint}}
\item{$t\cdot (t+1)\cdot (t^2+t+1)$ = \MLSimplifyQuestion{30}{t*(t+1)*(t*t+t+1)}{10}{t}{5}{1}{LSFF8}. \begin{MHint}{Lösung}$$t\cdot (t+1)\cdot (t^2+t+1) \; = \; (t^2+t)(t^2+t+1) \;=\; t^4+t^3+t^2+t^3+t^2+t \;=\; t^4+2t^3+2t^2+t $$\end{MHint}}
\end{MExerciseItems}
%%% folgende auskommentiert, da alte Loesung, sollte geloescht werden
%\begin{MHint}{\iSolution}
%Ausmultiplizieren ergibt $f(x)=(3-x)(x+1) =-x^2+2x+3$. Wegen des führenden Terms $-x^2$ besitzt diese Funktion als Graph eine
%nach unten geöffnete Parabel mit den Asymptoten $-\infty$ in beide Richtungen:
%$$
%\lim_{x\rightarrow \infty} f(x) \;=\; -\infty \MDFPSpace, \MDFPaSpace \lim_{x\rightarrow -\infty} f(x) \;=\; -\infty\MDFPeriod 
%$$
%In den anderen Aufgabenteilen ergibt sich durch Ausmultiplizieren
%\begin{eqnarray*}
%(x+4)(2-x)(x+2) &=& -x^3-4x^2+4x+16 \MDFPSpace,\ \\
%(3-x)(x+1)^2 &=& (3-x)(x^2+2x+1) \;=\; -x^3+x^2+5x+3 \MDFPSpace, \ \\
%t\cdot (t+1)\cdot (t^2+t+1) & = & t\cdot(t^3+2t^2+2t+1) \;=\; t^4+2t^3+2t^2+t \MDFPeriod
%\end{eqnarray*}
%\end{MHint}
\end{MExercise}

\begin{MExercise}
Dieser Graph gehört zu einem Polynom $g(x)$ zweiten Grades:

\begin{center}
\MTikzAuto{%
\begin{tikzpicture}[x=1.0cm, y=0.2cm, scale=0.80] 
\draw[black] (-4,0) -- (4,0) (0,-27) -- (0,13);
\foreach \x in {-4, -3, -2, -1, 1, 2, 3, 4}
\draw[shift={(\x,0)},color=black] (0pt,0pt) -- (0pt,-3.5pt) node[below=-2.0pt] {\tiny $\x$};
\foreach \x in {-3.5, -2.5, ..., 4.0}
\draw[shift={(\x,0)},color=black] (0pt,0pt) -- (0pt,-1.8pt);
\foreach \y in {-20, -10, 10}
\draw[shift={(0,\y)},color=black] (0pt,0pt) -- (-3.5pt,0pt) node[left=-3.0pt] {\tiny $\y$};
\foreach \y in {-27, -26, ..., 13}
\draw[shift={(0,\y)},color=black] (0pt,0pt) -- (-1.8pt,0pt);
\draw[black] (-0.0pt,-0.0pt) node[anchor=north east] {\tiny $0$};
\clip(-4.0,-27.0) rectangle (4.0,13.0);
%%\draw[help lines, gray!50, xstep=0.5, ystep=1, dashed] (-4,-26.5) grid (4,12.5);
\draw[smooth,samples=27,domain=-4:4, line width=0.8pt,color=red] plot(\x,{-2*\x*\x+2*\x+12});
\end{tikzpicture}
}
\par
Graph von $g(x)$.
\end{center}

\begin{MExerciseItems}
\item{Der Graph besitzt zwei Nullstellen $x_1$ und $x_2$, die daraus entstehenden Faktoren ergeben eingesetzt und ausmultipliziert das Polynom $f(x)=(x-x_1)(x-x_2)$ = \MLSimplifyQuestion{30}{(x+2)*(x-3)}{10}{x}{5}{1}{LSFF9}.} %%%
\item{Dieses Polynom gehört nicht zum Graph, denn an der Stelle $x=0$ besitzt $f(x)$ den Wert \MLParsedQuestion{5}{-6}{5}{ER4}, aber $g(x)$ besitzt
laut Graph den Wert \MLParsedQuestion{5}{12}{5}{ER4b}.
Diesen Unterschied kann man korrigieren, indem man $g(x)=c\cdot f(x)$ setzt mit dem Vorfaktor $c$ = \MLParsedQuestion{5}{-2}{5}{ER5}.}
\item{Zusammensetzen ergibt schließlich $g(x)$ = \MLSimplifyQuestion{30}{(0-2)*(x+2)*(x-3)}{4}{x}{3}{1}{LSFF10} in Produktdarstellung.}
\end{MExerciseItems}
\begin{MHint}{\iSolution}
Der Graph zeigt die Nullstellen $x_1=-2$ und $x_2=3$, die daraus entstehenden Faktoren ergeben ausmultipliziert
das Polynom $f(x)=(x+2)(x-3)=x^2-x-6$. \\
An der Stelle $x=0$ ist $f(0)=-6$ aber $g(0)=12$ laut Graph. Dies kann korrigiert werden, indem man noch den Faktor $-2$ hinzunimmt. \\
Insgesamt erhält man $g(x)=-2x^2+2x+12$.
\end{MHint}
\end{MExercise}


\begin{MExercise}
Multiplizieren Sie vollständig aus: $(a+2b+3c)^2$ = \MLSimplifyQuestion{44}{(a+(2*b)+(3*c))^2}{3}{a,b,c}{3}{1}{LSFF11}.

\begin{MHint}{\iSolution}
Am einfachsten multipliziert man jeden Summanden links mit jedem Summanden rechts und fasst anschließend zusammen:
\begin{eqnarray*}
(a+2b+3c)^2 & = & (a+2b+3c)\cdot(a+2b+3c)  \ \\
&=& a^2+2a b+3a c+2a b+4b^2+6b c+3a c+6b c+9c^2 \ \\
&=& a^2 +4a b+6a c+4b^2+12b c+9c^2 \MDFPeriod 
\end{eqnarray*}
\end{MHint}
\end{MExercise}

\end{MXContent}

\MSubsection{Potenzen und Wurzeln}
\MLabel{M01_Potenzenwurzeln}

\begin{MXContent}{Potenzrechnung und Wurzeln}{Potenzen und Wurzeln}{STD}
\MDeclareSiteUXID{VBKM01_PotenzenWurzeln}

Der folgende Abschnitt beschäftigt sich mit Ausdrücken der Form $\displaystyle a^s$. Hierbei sei $a\in\R$. Aber für welche Zahlen $s$ kann die Potenz sinnvoll definiert werden?

Potenzen mit natürlichem Exponenten werden folgendermaßen definiert:
\begin{MInfo}
Sei $n\in\N$. Die $n$-te \MEntry{Potenz}{Potenz} einer Zahl $a\in\R$ ist das $n$-fache Produkt der Zahl $a$ mit sich selbst:
\ifttm
\begin{eqnarray*}
a^n & =&  a\cdot a\cdot a\cdot\dots\cdot a\: .\\ && \;\;(n\:\text{Faktoren})
\end{eqnarray*}
\else
$$a^n = \underbrace{a\cdot a\cdot a\cdot\dots\cdot a}_{\mbox{$n$ Faktoren}} \MDFPeriod$$
\fi
$a$ wird als \MEntry{Basis}{Basis} und $n$ als \MEntry{Exponent}{Exponent} bezeichnet.\\
\end{MInfo}

Dabei gibt es einige besondere Fälle, die man am besten auswendig können sollte:

\begin{MInfo}
Ist der Exponent Null, so ist der Wert der Potenz gleich Eins, also beispielsweise $4^0=1$ aber auch $0^0=1$.
Ist dagegen die Basis Null und der Exponent $n>0$, so ist $0^n=0$. Ist die Basis $-1$, so ist
$$
(-1)^n \;=\; -1 \;\text{falls Exponent ungerade} \;\; , \;\;
(-1)^n \;=\; 1 \;\text{falls Exponent gerade}\MDFPeriod
$$
\end{MInfo}


\begin{MExample}
 $$3^2 = 3\cdot 3 = 9 \MDFPSpace, \MDFPaSpace (-2)^3 = (-2)\cdot(-2)\cdot(-2) = -8 \MDFPSpace, \MDFPaSpace \left(\Mdfrac12\right)^4 = \Mdfrac12\cdot\Mdfrac12\cdot\Mdfrac12\cdot\Mdfrac12=\Mdfrac{1}{16} \MDFPeriod$$
\end{MExample}
Viele Potenzen können mit obiger Rechenregel berechnet werden -- aber wie sieht es mit $\displaystyle 2^{-2}$ aus?
\begin{MInfo}
Potenzen mit negativen natürlichen Exponenten werden definiert durch die Formel
 $a^{-n} = \Mtfrac{1}{a^n}, n\in\N, a\neq 0.$ 
\end{MInfo}
Folglich berechnet sich $\displaystyle 2^{-2} = \Mtfrac{1}{2^2}=\Mtfrac14$. Analog ergibt sich $\displaystyle (-2)^{-2} = \Mtfrac{1}{(-2)^2} = \Mtfrac14$.


\begin{MExercise}
Welche Zahlenwerte haben diese Potenzen?
\begin{MExerciseItems}
\item{\MEquationItem{$\displaystyle 5^3$}{\MLSimplifyQuestion{8}{5^3}{5}{}{5}{256}{PPP1}}.\\\begin{MHint}{Lösung} $$5^3 =5\cdot5\cdot5=25\cdot 5= 125 \MDFPeriod$$\end{MHint}}
\item{\MEquationItem{$\displaystyle (-1)^{1001}$}{\MLSimplifyQuestion{8}{-1}{5}{}{5}{256}{PPP2}}.\\\begin{MHint}{Lösung} $$(-1)^{1001}=-1 \;\text{da Exponent ungerade}\MDFPeriod$$\end{MHint}}
\item{\MEquationItem{$\displaystyle \left({-\Mdfrac12}\right)^{-3}$}{\MLSimplifyQuestion{8}{-8}{5}{}{5}{256}{PPP2b}}.\\\begin{MHint}{Lösung} $$(-\frac12)^{-3}=\frac1{(-\frac12)^3}=\frac1{-\frac18}=-8\MDFPeriod$$\end{MHint}}
\item{\MEquationItem{$\displaystyle \left((-3)^2\right)^3$}{\MLSimplifyQuestion{8}{9^3}{5}{}{5}{256}{PPP3}}.\\\begin{MHint}{Lösung} $$\left((-3)^2\right)^3=9^3=729 \MDFPeriod$$\end{MHint}}
\end{MExerciseItems}
\end{MExercise}


Aber schon bei einem rationalen Exponenten der Form $\displaystyle \Mtfrac{1}{n}, n\in\N$, muss die Definition wieder erweitert werden, um zum Beispiel $\displaystyle 4^{\Mtfrac12}$ berechnen zu können. Diese Potenz lässt sich auch in Wurzelschreibweise umwandeln und man erhält $\displaystyle 4^{\Mtfrac12} = \sqrt{4} =2$. Allgemein gilt:
\begin{MInfo}
Sei $n\in\N$ und $a\in\R$ mit $a\geq 0.$ Die $n$-te Wurzel besitzt die Potenzschreibweise $\displaystyle \sqrt[n]{a} = a^{\Mtfrac{1}{n}}$.
\end{MInfo} 
Dies führt auf eine Umkehrung der Potenzrechnung, das Wurzelziehen.

\begin{MInfo}
Die $n$-te \MEntry{Wurzel}{Wurzel} einer Zahl $a\in\R, a\geq 0,$ ist diejenige Zahl, deren $n$-te Potenz gleich $a$ ist:
$$a^{\Mtfrac{1}{n}} = \sqrt[n]{a} = b \Longrightarrow b^n = a \MDFPeriod$$
$a$ wird als \MEntry{Wurzelbasis}{Wurzelbasis} oder \MEntry{Radikand}{Radikand} und $n$ als \MEntry{Wurzelexponent}{Exponent} bezeichnet.\\
Es gilt $\displaystyle \sqrt[1]{a} = a$ und es heißen 
$\displaystyle \sqrt[2]{a} = \sqrt{a}$ die \MEntry{Quadratwurzel}{Quadratwurzel} und 
 $\displaystyle \sqrt[3]{a} $ die \MEntry{Kubikwurzel}{Kubikwurzel} von $a$.
\end{MInfo}

\begin{MExample}
$$16^{\Mtfrac12} = \sqrt[2]{16} = \sqrt{16} = 4\MDFPSpace, \MDFPaSpace 27^{\Mtfrac13} = \sqrt[3]{27} = 3 \MDFPeriod$$
\end{MExample}

\begin{MInfo}
Für $n\in\N, a,b\in\R$ mit $a,b\geq 0$ gelten die folgenden \MEntry{Wurzelrechenregeln}{Rechenregeln (Wurzel)}:
\begin{enumerate}
\item Zwei Wurzeln mit gleichem Exponenten werden multipliziert, indem die Wurzel aus dem Produkt der Radikanden gezogen wird, der Wurzelexponent bleibt unverändert:
$$\sqrt[n]{a}\cdot \sqrt[n]{b} = \sqrt[n]{a\cdot b} \MDFPeriod$$
\item Zwei Wurzeln mit gleichem Exponenten werden dividiert, indem die Wurzel aus dem Quotienten der Radikanden gezogen wird, der Wurzelexponent bleibt unverändert:
$$\Mdfrac{\sqrt[n]{a}}{\sqrt[n]{b}} = \sqrt[n]{\Mdfrac{a}{b}}, b\neq 0 \MDFPeriod$$
\end{enumerate}
\end{MInfo}
Aber wie kann die Zahl $\displaystyle \left(\sqrt[10]{4}\right)^5$ berechnet werden?

\begin{MInfo}
Seien $m,n\in\N$ und $a\in\R, a\geq 0.$
\begin{enumerate}
\item Die $m$-te Potenz einer Wurzel bildet man, indem die $m$-te Potenz des Radikanden gebildet wird, der Wurzelexponent bleibt unverändert:
$$\left(\sqrt[n]{a}\right)^m = \sqrt[n]{a^m} = a^{\Mtfrac{m}{n}} \MDFPeriod$$
\item Die \textbf{$m$-te Wurzel aus einer Wurzel} bildet man, indem die Wurzelexponenten multipliziert werden und die Basis gleich bleibt (\MEntry{Radizieren einer Wurzel}{Radizieren}).
$$\sqrt[m]{\sqrt[n]{a}} = \sqrt[m\cdot n]{a} \MDFPeriod$$
\end{enumerate}
\end{MInfo}
Somit erhält man
$$\sqrt[10]{4^5} = (4^5)^{\Mtfrac{1}{10}} = 4^{\Mtfrac{5}{10}} = 4^{\Mtfrac12} = \sqrt[2]{4} = 2 \MDFPeriod$$

\begin{MExample}
Die Berechnung einer allgemeinen Potenz mit rationalem Exponenten sieht dann folgendermaßen aus:
$$\left(\Mdfrac14\right)^{-\Mtfrac23} = \sqrt[3]{\left(\Mdfrac14\right)^{-2}} = \sqrt[3]{\Mdfrac{1}{\left(\Mdfrac14\right)^2}} = \sqrt[3]{\Mdfrac{1}{\Mdfrac{1}{16}}} = \sqrt[3]{16} = \sqrt[3]{2^3\cdot 2}=2\sqrt[3]{2} \MDFPeriod$$ 
\end{MExample}

Die für Potenzen mit reeller Basis und rationalem Exponenten gültigen Rechenregeln werden unter der Bezeichnung Potenzgesetze zusammengefasst.
Die Regeln differieren in Abhängigkeit von der Betrachtung von Potenzen mit derselben Basis bzw. demselben Exponenten.

\begin{MExercise}
Berechnen Sie die folgenden Wurzeln (es kommen hier stets ganze Zahlen heraus):
\begin{MExerciseItems}
\item{\MEquationItem{$\displaystyle \left(\sqrt[5]{3}\right)^5$}{\MLSimplifyQuestion{15}{3}{5}{}{5}{1024}{PPP13}}.\\\begin{MHint}{Lösung} $$\left(\sqrt[5]{3}\right)^5=(3^5)^{\Mtfrac15}=3^{5\cdot \Mdfrac15}=3 \MDFPeriod$$\end{MHint}}
\item{\MEquationItem{$\displaystyle \sqrt[4]{256}$}{\MLSimplifyQuestion{15}{4}{5}{}{5}{1024}{PPP14}}.\\\begin{MHint}{Lösung} $$\sqrt[4]{256}=\left({2^8}\right)^{\Mtfrac14}=2^2=4 \MDFPeriod$$\end{MHint}}
\end{MExerciseItems}
\end{MExercise}

\end{MXContent}

\begin{MXContent}{Rechnen mit Potenzen}{Rechnen mit Potenzen}{STD}
\MDeclareSiteUXID{VBKM01_RPW}

Die folgenden Rechenregeln ermöglichen das Umformen und Vereinfachen von Ausdrücken, die Potenzen oder Wurzeln enthalten:

\begin{MInfo}
\MLabel{VBKM01_Potenzgesetze}
Für $a,b \in \R, a,b>0, p,q\in\Q$ gelten die \MEntry{Potenzgesetze}{Potenzgesetze}:
$$a^p\cdot b^p = (a\cdot b)^p \MDFPSpace, \MDFPaSpace \Mdfrac{a^p}{b^p} = \left(\Mdfrac{a}{b}\right)^p \MDFPSpace,  \MDFPaSpace a^p\cdot a^q = a^{p+q} \MDFPSpace, \MDFPaSpace \Mdfrac{a^p}{a^q} = a^{p-q} \MDFPSpace, \MDFPaSpace (a^p)^q = a^{p\cdot q} \MDFPeriod$$ 
\end{MInfo}

Insbesondere ist zu beachten, dass im Allgemeinen $\displaystyle (a^p)^q \neq a^{p^q}$ ist, d.~h. bei mehrfachem Potenzieren sollten Klammern gesetzt werden.
Zum Beispiel ist $\displaystyle \left(2^3\right)^2 = 8^2 = 64$, aber $\displaystyle 2^{\left(3^2\right)} = 2^9 = 512$.
\ \\ \ \\
\MFormelZoomHint
\ \\
\begin{MExample}
Sind keine Klammern gesetzt, so wird $a^{p^q}$ als $a^{(p^q)}$ interpretiert, also beispielsweise
\begin{eqnarray*}
2^{3^4} & =& 2^{3\cdot3\cdot3\cdot 3} \;=\; 2^{81} \;=\; 2417851639229258349412352\;\; \text{(Exponent zuerst ausgewertet)}\\
(2^3)^4 &=& 8^4 \;=\; 4096 \;\; \text{(Klammer zuerst ausgewertet)}\MDFPeriod
\end{eqnarray*}
Alternativ könnte man auch mit den Potenzgesetzen $(2^3)^4=2^{(3\cdot 4)}=2^{12}=4096$ ausrechnen.
\end{MExample}

\begin{MExercise}
Die folgenden Ausdrücke kann man mit Hilfe der Potenzgesetze vereinfachen:
\begin{MExerciseItems}
\item{\MEquationItem{$\displaystyle 3^3\cdot 3^5\cdot 3^{-1}$}{\MLSimplifyQuestion{15}{3^7}{5}{}{5}{2048}{PPP7}}.\\\begin{MHint}{Lösung} $$3^3\cdot 3^5\cdot 3^{-1}=3^{3+5-1}=3^7 \MDFPeriod$$\end{MHint}}
\item{\MEquationItem{$\displaystyle 4^2\cdot 3^2$}{\MLSimplifyQuestion{15}{12^2}{5}{}{5}{2048}{PPP8}}.\\\begin{MHint}{Lösung} $$4^2\cdot 3^2=(4\cdot 3)^2=12^2 \MDFPeriod$$\end{MHint}}
\end{MExerciseItems}
\end{MExercise}

Beim Vergleichen von Potenzen und Wurzeln ist Vorsicht geboten: Nicht nur die Zahlenwerte, auch die Vorzeichen von Exponent und Basis haben einen Einfluss darauf,
ob der Wert der Potenz groß oder klein ist:

\begin{MExample}
Bei positiver Basis und negativen Exponenten nimmt der Wert der Potenz ab, wenn man die Basis vergrößert:
\begin{eqnarray*}
2^{-1} &=& \frac12\;=\; \MZahl{0}{5}\ \\
3^{-1} &=& \frac13\;=\; \MZahl{0}{}\bar 3\ \\
4^{-1} &=& \frac14\;=\; \MZahl{0}{25}\; \text{usw.}
\end{eqnarray*}

Bei negativer Basis wechselt dagegen das Vorzeichen der Potenz, wenn man den Exponenten erhöht:
\begin{eqnarray*}
(-2)^1 &=& -2\\
(-2)^2 &=& 4\\
(-2)^3 &=& -8\\
(-2)^4 &=& 16 \; \text{usw.}
\end{eqnarray*}

Das Ziehen von Wurzeln (bzw. Potenzieren mit einer positiven Zahl kleiner Eins) verkleinert eine Basis $>1$, aber vergrößert eine Basis $<1$:
\begin{eqnarray*}
\sqrt{2} &=& \MZahl{1}{414}\ldots\;\; < \;\; 2\\
\sqrt{3} &=& \MZahl{1}{732}\ldots\;\; < \;\; 3\\
\sqrt{\MZahl{0}{5}} &=& \MZahl{0}{707}\ldots\;\; > \;\; \MZahl{0}{5}\\
\sqrt{\MZahl{0}{}\bar 3} &=& \MZahl{0}{577}\ldots\;\; > \;\; \MZahl{0}{}\bar 3\;\text{usw.}
\end{eqnarray*}

\end{MExample}

\begin{MExercise}
Ordnen Sie diese Potenzen der Größe nach an unter Beachtung der Vorzeichen von Basen und Exponenten: $2^3$, $2^{-3}$, $3^2$, $(-3)^2$, $(-3)^{-2}$, $3^{\frac12}$, $2^{\frac13}$:
\ \\
\MLParsedQuestion{9}{(-3)^(-2)}{5}{VBKM01_USR7}\ $<$\ 
\MLParsedQuestion{9}{2^(-3)}{5}{VBKM01_USR8}\ $<$\ 
\MLParsedQuestion{9}{2^(1/3)}{5}{VBKM01_USR9}\ $<$\ 
\MLParsedQuestion{9}{3^(1/2)}{5}{VBKM01_USR10}\ $<$\ 
\MLParsedQuestion{9}{2^3}{5}{VBKM01_USR11} \ $<$\ 
\MLParsedQuestion{9}{3^2}{5}{VBKM01_USR12} \ $=$\ 
\MLParsedQuestion{9}{(-3)^2}{5}{VBKM01_USR13}\ .
\ \\
\begin{MHint}{\iSolution}
Hier gibt es mehrere Wege zur richtigen Lösung. Beispielsweise lohnt es sich, zunächst alle Potenzen mit Drei zu potenzieren (analog zur Rechnung in Beispiel \MRef{VBKM01_Bsp_Anordnung}, um die letzten beiden Potenzen einfacher zu ordnen).
Potenzieren mit der Drei ist dabei erlaubt, weil es eine ungerade Zahl ist, die als Exponent das Vorzeichen nicht aufhebt. Potenzieren mit der Zwei würde dagegen die richtige
Anordnung der Zahlen aufheben. Es ergibt sich
\begin{eqnarray*}
(2^3)^3 &=& 2^{3\cdot 3} \;=\; 2^9 \;=\; 512\\
(2^{-3})^3 &=& 2^{(-3)\cdot 3}\;=\; 2^{-9} \;=\; \frac1{512}\;\text{da Exponent negativ}\\
(3^2)^3 &=& 9^3 \;=\; 81\cdot9 \;=\; 729\\
((-3)^2)^3 &=& 9^3 \;=\; 729\;\text{da Exponent gerade}\\
((-3)^{-2})^3 &=& \frac1{((-3)^2)^3} \;=\; \frac1{729}\\
(3^{\frac12})^3 &=& 3^{\frac32}\; >\; 3 \;\text{(da Exponent größer als Eins)}\\
(2^{\frac13})^3 &=& 2\: .
\end{eqnarray*}
Vergleichen dieser Werte führt auf die Anordnung
$$
(-3)^{-2}\;<\;2^{-3}\;<\;2^{\frac13}\;<\;3^{\frac12}\; < \; 2^3\; < \; 3^2 \;=\; (-3)^2\MDFPeriod
$$
\end{MHint}
\end{MExercise}

\end{MXContent}

\begin{MExercises}
\MDeclareSiteUXID{VBKM01_PotenzenWurzeln_Exercises}

\begin{MExercise}
Berechnen Sie die folgenden Potenzen:
\begin{MExerciseItems}
\item{\MEquationItem{$\displaystyle \left(-\Mdfrac{3}{5}\right)^4$}{\MLSpecialQuestion{10}{81/625}{5}{0}{0}{exactfraction}{PPP4}}.\\\begin{MHint}{Lösung} $$\left(-\Mdfrac{3}{5}\right)^4=\Mdfrac{3^4}{5^4}=\Mdfrac{81}{625} \MDFPeriod$$\end{MHint}}
\item{\MEquationItem{$\displaystyle \left(2^{-2}\right)^{-3}$}{\MLSimplifyQuestion{15}{2^6}{5}{}{5}{256}{PPP5}}.\\\begin{MHint}{Lösung} $$\left(2^{-2}\right)^{-3}=2^{(-2)\cdot(-3)}=2^6=64 \MDFPeriod$$\end{MHint}}
\item{\MEquationItem{$\displaystyle\left(-\Mdfrac{1}{2}\right)^0 $}{\MLSimplifyQuestion{15}{1}{5}{}{5}{256}{PPP6}}.\\\begin{MHint}{Lösung} $$\left(-\Mdfrac{1}{2}\right)^0=1 \MDFPeriod$$\end{MHint}}
\end{MExerciseItems}
\end{MExercise}

\begin{MExercise}
Vereinfachen Sie die folgenden Ausdrücke mit den Potenzregeln und durch Kürzen, die Potenzen brauchen Sie dabei nicht auszuwerten:
\begin{MExerciseItems}
\item{\MEquationItem{$\displaystyle \Mdfrac{(-2)^7}{(-2)^5}$}{\MLSimplifyQuestion{15}{4}{5}{}{5}{2048}{PPP9}}.\\\begin{MHint}{Lösung} $$\Mdfrac{(-2)^7}{(-2)^5}=(-2)^{7-5}=(-2)^2 \MDFPeriod$$\end{MHint}}
\item{\MEquationItem{$\displaystyle 6^2\cdot 3^{-2}$}{\MLSimplifyQuestion{15}{4}{5}{}{5}{2048}{PPP10}}.\\\begin{MHint}{Lösung} $$6^2\cdot 3^{-2}=6^2\cdot \left({\Mdfrac13}\right)^{2}=\left({6\cdot\Mdfrac13}\right)^2=2^2 \MDFPeriod$$\end{MHint}}
\item{\MEquationItem{$\displaystyle \Mdfrac{64^3}{8^3}$}{\MLSimplifyQuestion{15}{8^3}{5}{}{5}{2048}{PPP11}}.\\\begin{MHint}{Lösung} $$\Mdfrac{64^3}{8^3}=\left({\Mdfrac{64}{8}}\right)^3=8^3 \MDFPeriod$$\end{MHint}}
\item{\MEquationItem{$\displaystyle \left(\Mdfrac34\right)^{\Mtfrac13}\cdot\left(\Mdfrac34\right)^{\Mtfrac23}$}{\MLSpecialQuestion{15}{3/4}{5}{0}{0}{exactfraction}{PPP12}}.\\
\begin{MHint}{Lösung}
$$
\left(\Mdfrac34\right)^{\Mtfrac13}\cdot\left(\Mdfrac34\right)^{\Mtfrac23} = \left(\Mdfrac34\right)^{\Mtfrac13+\Mtfrac23} \;=\; \Mdfrac34  \MDFPeriod
$$
\end{MHint}}
\end{MExerciseItems}
\end{MExercise}

\begin{MExercise}
Berechnen Sie die folgenden Wurzeln (es kommen hier stets ganze Zahlen heraus):
\begin{MExerciseItems}
\item{\MEquationItem{$\displaystyle \sqrt[3]{3}\cdot \sqrt[3]{9}$}{\MLSimplifyQuestion{15}{3}{5}{}{5}{1024}{PPP15}}.\\\begin{MHint}{Lösung} $$\sqrt[3]{3}\cdot \sqrt[3]{9}=\sqrt[3]{3\cdot 3\cdot 3}=3 \MDFPeriod$$\end{MHint}}
\item{\MEquationItem{$\displaystyle \sqrt[3]{343}$}{\MLSimplifyQuestion{15}{7}{5}{}{5}{1024}{PPP16}}.\\\begin{MHint}{Lösung} $$\sqrt[3]{343}=\sqrt[3]{7^3}=7 \MDFPeriod$$\end{MHint}}
\end{MExerciseItems}
\end{MExercise}

\begin{MExercise}
Vereinfachen Sie die folgenden Ausdrücke und berechnen Sie den Zahlenwert als gekürzten Bruch ohne Potenzausdrücke:
\begin{MExerciseItems}
\item{\MEquationItem{$\displaystyle \Mdfrac{3^3\cdot 6^3}{9\cdot2^3\cdot 4^3}$}{\MLSpecialQuestion{10}{81/64}{5}{0}{0}{exactfraction}{PPP17}}.\\\begin{MHint}{Lösung} $$\Mdfrac{3^3\cdot 6^3}{9\cdot2^3\cdot 4^3}=\Mdfrac{3^3\cdot (2 \cdot 3)^3}{3^2\cdot2^3\cdot 4^3}=\Mdfrac{3^3\cdot 2^3 \cdot 3^2\cdot 3}{3^2\cdot2^3\cdot 4^3}=\Mdfrac{3^4}{4^3}=\Mdfrac{81}{64} \MDFPeriod$$\end{MHint}}
\item{\MEquationItem{$\displaystyle 3^2\cdot 9^{-3}\cdot 27^{6}\cdot 27^{-3}$}{\MLSpecialQuestion{10}{243}{5}{0}{0}{exactfraction}{PPP18}}.\\\begin{MHint}{Lösung} $$3^2\cdot 9^{-3}\cdot 27^{6}\cdot 27^{-3}=3^{2+2\cdot(-3)+3\cdot 6+3\cdot(-3)}=3^{5}=243 \MDFPeriod$$\end{MHint}}
\end{MExerciseItems}
\end{MExercise}

% \begin{MExercise}
% Vereinfachen Sie diese Ausdrücke soweit wie möglich:
% \begin{MExerciseItems}
% \item{\MEquationItem{$\displaystyle \Mdfrac{\sqrt[5]{a^3}}{\sqrt[4]{a^5}}\quad (a>0)$}{\MLSimplifyQuestion{15}{a^{-\Mtfrac{13}{20}}}{5}{a}{5}{2048}{PPP19}}.\\\begin{MHint}{Lösung} $$\Mdfrac{\sqrt[5]{a^3}}{\sqrt[4]{a^5}}=a^{\Mtfrac35-\Mdfrac54}=a^{\Mtfrac{12-25}{20}}=a^{-\Mtfrac{13}{20}} \MDFPeriod$$\end{MHint}
% 
% \MInputHint{Quadratwurzeln können Sie in der Form \texttt{sqrt(x y z)} eingeben.}


\end{MExercises}

\MSubsection{Abschlusstest}
\MLabel{M01_Ausgangstest}

\begin{MTest}{Abschlusstest Kapitel 1}
\MDeclareSiteUXID{VBKM01_Abschlusstest}
\begin{MExercise}
\MSetPoints{1}
Kreuzen Sie an, ob diese mathematischen Ausdrücke jeweils Gleichungen, Ungleichungen, Terme oder Zahlen darstellen (Mehrfachnennung ist möglich):
\ \\
\begin{tabular}{|l|c|c|c|c|}
  \hline
  Mathematischer Ausdruck  & Gleichung & Ungleichung & Term & Zahl \\ \hline
  $1+\Mdfrac12-3(3-\Mdfrac12)$ & \MLCheckbox{0}{TX11} & \MLCheckbox{0}{TX12} &\MLCheckbox{1}{TX13} &\MLCheckbox{1}{TX14} \\ \hline
  $5^x-x^5$                & \MLCheckbox{0}{TX21} & \MLCheckbox{0}{TX22} &\MLCheckbox{1}{TX23} &\MLCheckbox{0}{TX24} \\ \hline
  $x^2<\sqrt{x}$           & \MLCheckbox{0}{TX41} & \MLCheckbox{1}{TX42} &\MLCheckbox{0}{TX43} &\MLCheckbox{0}{TX44} \\ \hline
  $x y z-1$                  & \MLCheckbox{0}{TX31} & \MLCheckbox{0}{TX32} &\MLCheckbox{1}{TX33} &\MLCheckbox{0}{TX34} \\ \hline
  $b^2=4a c$               & \MLCheckbox{1}{TX51} & \MLCheckbox{0}{TX52} &\MLCheckbox{0}{TX53} &\MLCheckbox{0}{TX54} \\ \hline
\end{tabular}
\end{MExercise}

\begin{MExercise}
Vereinfachen Sie den Doppelbruch $\Mdfrac{3+\Mtfrac32}{\Mtfrac1{12}+\Mtfrac14}$ zu einem gekürzten Einfachbruch: \MLParsedQuestion{5}{27/2}{5}{ER6}
\MInputHint{Tippen Sie beispielsweise $\Mdfrac{11}{12}$ als \texttt{11/12} ein.}
\end{MExercise}

\begin{MExercise}
Multiplizieren Sie diesen Term vollständig aus und fassen Sie zusammen:

$(x-1)\cdot(x+1)\cdot(x-2)$ = \MLSimplifyQuestion{40}{(x-1)*(x+1)*(x-2)}{10}{x}{5}{1}{SMPPOLY}.

\MInputHint{Beispielsweise tippen Sie $(x+1)(x+2)$ = \texttt{x^2+3*x+2} oder auch \texttt{x*x+3*x+2}.}
\end{MExercise}

\begin{MExercise}
Wenden Sie jeweils eine binomische Formel an, um den Term umzuformen:
\begin{MExerciseItems}
\item{$(x-3)(x+3)$ = \MLSimplifyQuestion{14}{(x-3)*(x+3)}{10}{x}{5}{1}{VBK1}.}
\item{$(x-1)^2$ = \MLSimplifyQuestion{14}{(x-1)*(x-1)}{10}{x}{5}{1}{VBK2}.}
\item{$(2x+4)^2$ = \MLSimplifyQuestion{14}{(2*x+4)*(2*x+4)}{10}{x}{5}{1}{VBK3}.}
\end{MExerciseItems}
\MInputHint{Beispielsweise tippen Sie $(x+1)^2$ = \texttt{x^2+2*x+1} oder auch \texttt{x*x+2*x+1}.}
\end{MExercise}

\begin{MExercise}
Schreiben Sie diesen Potenz- und Wurzelausdruck als einfache Potenz mit einem rationalen Exponenten:
\ \\ \ \\
\MEquationItem{$\Mdfrac{x^3}{\left({\sqrt{x}}\right)^3}$}{\MLSimplifyQuestion{10}{x^(3/2)}{10}{x}{5}{576}{VBK4}}.\\
\MInputHint{Beispielsweise tippen Sie $\sqrt{x}\cdot x^2$ = \texttt{x^(5/2)} oder auch \texttt{x^(2.5)},\\vergessen Sie die Klammern um den Bruch nicht.}
\end{MExercise}
\end{MTest}


\newpage
\MPrintIndex

\end{document}
