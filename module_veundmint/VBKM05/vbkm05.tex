% MINTMOD Version P0.1.0, needs to be consistent with preprocesser object in tex2x and MPragma-Version at the end of this file

% Parameter aus Konvertierungsprozess (PDF und HTML-Erzeugung wenn vom Konverter aus gestartet) werden hier eingefuegt, Preambleincludes werden am Schluss angehaengt

\newif\ifttm                % gesetzt falls Uebersetzung in HTML stattfindet, sonst uebersetzung in PDF

% Wahl der Notationsvariante ist im PDF immer std, in der HTML-Uebersetzung wird vom Konverter die Auswahl modifiziert
\newif\ifvariantstd
\newif\ifvariantunotation
\variantstdtrue % Diese Zeile wird vom Konverter erkannt und ggf. modifiziert, daher nicht veraendern!


\def\MOutputDVI{1}
\def\MOutputPDF{2}
\def\MOutputHTML{3}
\newcounter{MOutput}

\ifttm
\usepackage{german}
\usepackage{array}
\usepackage{amsmath}
\usepackage{amssymb}
\usepackage{amsthm}
\else
\documentclass[ngerman,oneside]{scrbook}
\usepackage{etex}
\usepackage[latin1]{inputenc}
\usepackage{textcomp}
\usepackage[ngerman]{babel}
\usepackage[pdftex]{color}
\usepackage{xcolor}
\usepackage{graphicx}
\usepackage[all]{xy}
\usepackage{fancyhdr}
\usepackage{verbatim}
\usepackage{array}
\usepackage{float}
\usepackage{makeidx}
\usepackage{amsmath}
\usepackage{amstext}
\usepackage{amssymb}
\usepackage{amsthm}
\usepackage[ngerman]{varioref}
\usepackage{framed}
\usepackage{supertabular}
\usepackage{longtable}
\usepackage{maxpage}
\usepackage{tikz}
\usepackage{tikzscale}
\usepackage{tikz-3dplot}
\usepackage{bibgerm}
\usepackage{chemarrow}
\usepackage{polynom}
%\usepackage{draftwatermark}
\usepackage{pdflscape}
\usetikzlibrary{calc}
\usetikzlibrary{through}
\usetikzlibrary{shapes.geometric}
\usetikzlibrary{arrows}
\usetikzlibrary{intersections}
\usetikzlibrary{decorations.pathmorphing}
\usetikzlibrary{external}
\usetikzlibrary{patterns}
\usetikzlibrary{fadings}
\usepackage[colorlinks=true,linkcolor=blue]{hyperref} 
\usepackage[all]{hypcap}
%\usepackage[colorlinks=true,linkcolor=blue,bookmarksopen=true]{hyperref} 
\usepackage{ifpdf}

\usepackage{movie15}

\setcounter{tocdepth}{2} % In Inhaltsverzeichnis bis subsection
\setcounter{secnumdepth}{3} % Nummeriert bis subsubsection

\setlength{\LTpost}{0pt} % Fuer longtable
\setlength{\parindent}{0pt}
\setlength{\parskip}{8pt}
%\setlength{\parskip}{9pt plus 2pt minus 1pt}
\setlength{\abovecaptionskip}{-0.25ex}
\setlength{\belowcaptionskip}{-0.25ex}
\fi

\ifttm
\newcommand{\MDebugMessage}[1]{\special{html:<!-- debugprint;;}#1\special{html:; //-->}}
\else
%\newcommand{\MDebugMessage}[1]{\immediate\write\mintlog{#1}}
\newcommand{\MDebugMessage}[1]{}
\fi

\def\MPageHeaderDef{%
\pagestyle{fancy}%
\fancyhead[r]{(C) VE\&MINT-Projekt}
\fancyfoot[c]{\thepage\\--- CCL BY-SA 3.0 ---}
}


\ifttm%
\def\MRelax{}%
\else%
\def\MRelax{\relax}%
\fi%

%--------------------------- Uebernahme von speziellen XML-Versionen einiger LaTeX-Kommandos aus xmlbefehle.tex vom alten Kasseler Konverter ---------------

\newcommand{\MSep}{\left\|{\phantom{\frac1g}}\right.}

\newcommand{\ML}{L}

\newcommand{\MGGT}{\mathrm{ggT}}


\ifttm
% Verhindert dass die subsection-nummer doppelt in der toccaption auftaucht (sollte ggf. in toccaption gefixt werden so dass diese Ueberschreibung nicht notwendig ist)
\renewcommand{\thesubsection}{}
% Kommandos die ttm nicht kennt
\newcommand{\binomial}[2]{{#1 \choose #2}} %  Binomialkoeffizienten
\newcommand{\eur}{\begin{html}&euro;\end{html}}
\newcommand{\square}{\begin{html}&square;\end{html}}
\newcommand{\glqq}{"'}  \newcommand{\grqq}{"'}
\newcommand{\nRightarrow}{\special{html: &nrArr; }}
\newcommand{\nmid}{\special{html: &nmid; }}
\newcommand{\nparallel}{\begin{html}&nparallel;\end{html}}
\newcommand{\mapstoo}{\begin{html}<mo>&map;</mo>\end{html}}

% Schnitt und Vereinigungssymbole von Mengen haben zu kleine Abstaende; korrigiert:
\newcommand{\ccup}{\,\!\cup\,\!}
\newcommand{\ccap}{\,\!\cap\,\!}


% Umsetzung von mathbb im HTML
\renewcommand{\mathbb}[1]{\begin{html}<mo>&#1opf;</mo>\end{html}}
\fi

%---------------------- Strukturierung ----------------------------------------------------------------------------------------------------------------------

%---------------------- Kapselung des sectioning findet auf drei Ebenen statt:
% 1. Die LateX-Befehl
% 2. Die D-Versionen der Befehle, die nur die Grade der Abschnitte umhaengen falls notwendig
% 3. Die M-Versionen der Befehle, die zusaetzliche Formatierungen vornehmen, Skripten starten und das HTML codieren
% Im Modultext duerfen nur die M-Befehle verwendet werden!

\ifttm

  \def\Dsubsubsubsection#1{\subsubsubsection{#1}}
  \def\Dsubsubsection#1{\subsubsection{#1}\addtocounter{subsubsection}{1}} % ttm-Fehler korrigieren
  \def\Dsubsection#1{\subsection{#1}}
  \def\Dsection#1{\section{#1}} % Im HTML wird nur der Sektionstitel gegeben
  \def\Dchapter#1{\chapter{#1}}
  \def\Dsubsubsubsectionx#1{\subsubsubsection*{#1}}
  \def\Dsubsubsectionx#1{\subsubsection*{#1}}
  \def\Dsubsectionx#1{\subsection*{#1}}
  \def\Dsectionx#1{\section*{#1}}
  \def\Dchapterx#1{\chapter*{#1}}

\else

  \def\Dsubsubsubsection#1{\subsubsection{#1}}
  \def\Dsubsubsection#1{\subsection{#1}}
  \def\Dsubsection#1{\section{#1}}
  \def\Dsection#1{\chapter{#1}}
  \def\Dchapter#1{\title{#1}}
  \def\Dsubsubsubsectionx#1{\subsubsection*{#1}}
  \def\Dsubsubsectionx#1{\subsection*{#1}}
  \def\Dsubsectionx#1{\section*{#1}}
  \def\Dsectionx#1{\chapter*{#1}}

\fi

\newcommand{\MStdPoints}{4}
\newcommand{\MSetPoints}[1]{\renewcommand{\MStdPoints}{#1}}

% Befehl zum Abbruch der Erstellung (nur PDF)
\newcommand{\MAbort}[1]{\err{#1}}

% Prefix vor Dateieinbindungen, wird in der Baumdatei mit \renewcommand modifiziert
% und auf das Verzeichnisprefix gesetzt, in dem das gerade bearbeitete tex-Dokument liegt.
% Im HTML wird es auf das Verzeichnis der HTML-Datei gesetzt.
% Das Prefix muss mit / enden !
\newcommand{\MDPrefix}{.}

% MRegisterFile notiert eine Datei zur Einbindung in den HTML-Baum. Grafiken mit MGraphics werden automatisch eingebunden.
% Mit MLastFile erhaelt man eine Markierung fuer die zuletzt registrierte Datei.
% Diese Markierung wird im postprocessing durch den physikalischen Dateinamen ersetzt, aber nur den Namen (d.h. \MMaterial gehoert noch davor, vgl Definition von MGraphics)
% Parameter: Pfad/Name der Datei bzw. des Ordners, relativ zur Position des Modul-Tex-Dokuments.
\ifttm
\newcommand{\MRegisterFile}[1]{\addtocounter{MFileNumber}{1}\special{html:<!-- registerfile;;}#1\special{html:;;}\MDPrefix\special{html:;;}\arabic{MFileNumber}\special{html:; //-->}}
\else
\newcommand{\MRegisterFile}[1]{\addtocounter{MFileNumber}{1}}
\fi

% Testen welcher Uebersetzer hier am Werk ist

\ifttm
\setcounter{MOutput}{3}
\else
\ifx\pdfoutput\undefined
  \pdffalse
  \setcounter{MOutput}{\MOutputDVI}
  \message{Verarbeitung mit latex, Ausgabe in dvi.}
\else
  \setcounter{MOutput}{\MOutputPDF}
  \message{Verarbeitung mit pdflatex, Ausgabe in pdf.}
  \ifnum \pdfoutput=0
    \pdffalse
  \setcounter{MOutput}{\MOutputDVI}
  \message{Verarbeitung mit pdflatex, Ausgabe in dvi.}
  \else
    \ifnum\pdfoutput=1
    \pdftrue
  \setcounter{MOutput}{\MOutputPDF}
  \message{Verarbeitung mit pdflatex, Ausgabe in pdf.}
    \fi
  \fi
\fi
\fi

\ifnum\value{MOutput}=\MOutputPDF
\DeclareGraphicsExtensions{.pdf,.png,.jpg}
\fi

\ifnum\value{MOutput}=\MOutputDVI
\DeclareGraphicsExtensions{.eps,.png,.jpg}
\fi

\ifnum\value{MOutput}=\MOutputHTML
% Wird vom Konverter leider nicht erkannt und daher in split.pm hardcodiert!
\DeclareGraphicsExtensions{.png,.jpg,.gif}
\fi

% Umdefinition der hyperref-Nummerierung im PDF-Modus
\ifttm
\else
\renewcommand{\theHfigure}{\arabic{chapter}.\arabic{section}.\arabic{figure}}
\fi

% Makro, um in der HTML-Ausgabe die zuerst zu oeffnende Datei zu kennzeichnen
\ifttm
\newcommand{\MGlobalStart}{\special{html:<!-- mglobalstarttag -->}}
\else
\newcommand{\MGlobalStart}{}
\fi

% Makro, um bei scormlogin ein pullen des Benutzers bei Aufruf der Seite zu erzwingen (typischerweise auf der Einstiegsseite)
\ifttm
\newcommand{\MPullSite}{\special{html:<!-- pullsite //-->}}
\else
\newcommand{\MPullSite}{}
\fi

% Makro, um in der HTML-Ausgabe die Kapiteluebersicht zu kennzeichnen
\ifttm
\newcommand{\MGlobalChapterTag}{\special{html:<!-- mglobalchaptertag -->}}
\else
\newcommand{\MGlobalChapterTag}{}
\fi

% Makro, um in der HTML-Ausgabe die Konfiguration zu kennzeichnen
\ifttm
\newcommand{\MGlobalConfTag}{\special{html:<!-- mglobalconfigtag -->}}
\else
\newcommand{\MGlobalConfTag}{}
\fi

% Makro, um in der HTML-Ausgabe die Standortbeschreibung zu kennzeichnen
\ifttm
\newcommand{\MGlobalLocationTag}{\special{html:<!-- mgloballocationtag -->}}
\else
\newcommand{\MGlobalLocationTag}{}
\fi

% Makro, um in der HTML-Ausgabe die persoenlichen Daten zu kennzeichnen
\ifttm
\newcommand{\MGlobalDataTag}{\special{html:<!-- mglobaldatatag -->}}
\else
\newcommand{\MGlobalDataTag}{}
\fi

% Makro, um in der HTML-Ausgabe die Suchseite zu kennzeichnen
\ifttm
\newcommand{\MGlobalSearchTag}{\special{html:<!-- mglobalsearchtag -->}}
\else
\newcommand{\MGlobalSearchTag}{}
\fi

% Makro, um in der HTML-Ausgabe die Favoritenseite zu kennzeichnen
\ifttm
\newcommand{\MGlobalFavoTag}{\special{html:<!-- mglobalfavoritestag -->}}
\else
\newcommand{\MGlobalFavoTag}{}
\fi

% Makro, um in der HTML-Ausgabe die Eingangstestseite zu kennzeichnen
\ifttm
\newcommand{\MGlobalSTestTag}{\special{html:<!-- mglobalstesttag -->}}
\else
\newcommand{\MGlobalSTestTag}{}
\fi

% Makro, um in der PDF-Ausgabe ein Wasserzeichen zu definieren
\ifttm
\newcommand{\MWatermarkSettings}{\relax}
\else
\newcommand{\MWatermarkSettings}{%
% \SetWatermarkText{(c) MINT-Kolleg Baden-W�rttemberg 2014}
% \SetWatermarkLightness{0.85}
% \SetWatermarkScale{1.5}
}
\fi

\ifttm
\newcommand{\MBinom}[2]{\left({\begin{array}{c} #1 \\ #2 \end{array}}\right)}
\else
\newcommand{\MBinom}[2]{\binom{#1}{#2}}
\fi

\ifttm
\newcommand{\DeclareMathOperator}[2]{\def#1{\mathrm{#2}}}
\newcommand{\operatorname}[1]{\mathrm{#1}}
\fi

%----------------- Makros fuer die gemischte HTML/PDF-Konvertierung ------------------------------

\newcommand{\MTestName}{\relax} % wird durch Test-Umgebung gesetzt

% Fuer experimentelle Kursinhalte, die im Release-Umsetzungsvorgang eine Fehlermeldung
% produzieren sollen aber sonst normal umgesetzt werden
\newenvironment{MExperimental}{%
}{%
}

% Wird von ttm nicht richtig umgesetzt!!
\newenvironment{MExerciseItems}{%
\renewcommand\theenumi{\alph{enumi}}%
\begin{enumerate}%
}{%
\end{enumerate}%
}


\definecolor{infoshadecolor}{rgb}{0.75,0.75,0.75}
\definecolor{exmpshadecolor}{rgb}{0.875,0.875,0.875}
\definecolor{expeshadecolor}{rgb}{0.95,0.95,0.95}
\definecolor{framecolor}{rgb}{0.2,0.2,0.2}

% Bei PDF-Uebersetzung wird hinter den Start jeder Satz/Info-aehnlichen Umgebung eine leere mbox gesetzt, damit
% fuehrende Listen oder enums nicht den Zeilenumbruch kaputtmachen
%\ifttm
\def\MTB{}
%\else
%\def\MTB{\mbox{}}
%\fi


\ifttm
\newcommand{\MRelates}{\special{html:<mi>&wedgeq;</mi>}}
\else
\def\MRelates{\stackrel{\scriptscriptstyle\wedge}{=}}
\fi

\def\MInch{\text{''}}
\def\Mdd{\textit{''}}

\ifttm
\def\MNL{ \newline }
\newenvironment{MArray}[1]{\begin{array}{#1}}{\end{array}}
\else
\def\MNL{ \\ }
\newenvironment{MArray}[1]{\begin{array}{#1}}{\end{array}}
\fi

\newcommand{\MBox}[1]{$\mathrm{#1}$}
\newcommand{\MMBox}[1]{\mathrm{#1}}


\ifttm%
\newcommand{\Mtfrac}[2]{{\textstyle \frac{#1}{#2}}}
\newcommand{\Mdfrac}[2]{{\displaystyle \frac{#1}{#2}}}
\newcommand{\Mmeasuredangle}{\special{html:<mi>&angmsd;</mi>}}
\else%
\newcommand{\Mtfrac}[2]{\tfrac{#1}{#2}}
\newcommand{\Mdfrac}[2]{\dfrac{#1}{#2}}
\newcommand{\Mmeasuredangle}{\measuredangle}
\relax
\fi

% Matrizen und Vektoren

% Inhalt wird in der Form a & b \\ c & d erwartet
% Vorsicht: MVector = Komponentenspalte, MVec = Variablensymbol
\ifttm%
\newcommand{\MVector}[1]{\left({\begin{array}{c}#1\end{array}}\right)}
\else%
\newcommand{\MVector}[1]{\begin{pmatrix}#1\end{pmatrix}}
\fi



\newcommand{\MVec}[1]{\vec{#1}}
\newcommand{\MDVec}[1]{\overrightarrow{#1}}

%----------------- Umgebungen fuer Definitionen und Saetze ----------------------------------------

% Fuegt einen Tabellen-Zeilenumbruch ein im PDF, aber nicht im HTML
\newcommand{\TSkip}{\ifttm \else&\ \\\fi}

\newenvironment{infoshaded}{%
\def\FrameCommand{\fboxsep=\FrameSep \fcolorbox{framecolor}{infoshadecolor}}%
\MakeFramed {\advance\hsize-\width \FrameRestore}}%
{\endMakeFramed}

\newenvironment{expeshaded}{%
\def\FrameCommand{\fboxsep=\FrameSep \fcolorbox{framecolor}{expeshadecolor}}%
\MakeFramed {\advance\hsize-\width \FrameRestore}}%
{\endMakeFramed}

\newenvironment{exmpshaded}{%
\def\FrameCommand{\fboxsep=\FrameSep \fcolorbox{framecolor}{exmpshadecolor}}%
\MakeFramed {\advance\hsize-\width \FrameRestore}}%
{\endMakeFramed}

\def\STDCOLOR{black}

\ifttm%
\else%
\newtheoremstyle{MSatzStyle}
  {1cm}                   %Space above
  {1cm}                   %Space below
  {\normalfont\itshape}   %Body font
  {}                      %Indent amount (empty = no indent,
                          %\parindent = para indent)
  {\normalfont\bfseries}  %Thm head font
  {}                      %Punctuation after thm head
  {\newline}              %Space after thm head: " " = normal interword
                          %space; \newline = linebreak
  {\thmname{#1}\thmnumber{ #2}\thmnote{ (#3)}}
                          %Thm head spec (can be left empty, meaning
                          %`normal')
                          %
\newtheoremstyle{MDefStyle}
  {1cm}                   %Space above
  {1cm}                   %Space below
  {\normalfont}           %Body font
  {}                      %Indent amount (empty = no indent,
                          %\parindent = para indent)
  {\normalfont\bfseries}  %Thm head font
  {}                      %Punctuation after thm head
  {\newline}              %Space after thm head: " " = normal interword
                          %space; \newline = linebreak
  {\thmname{#1}\thmnumber{ #2}\thmnote{ (#3)}}
                          %Thm head spec (can be left empty, meaning
                          %`normal')
\fi%

\newcommand{\MInfoText}{Info}

\newcounter{MHintCounter}
\newcounter{MCodeEditCounter}

\newcounter{MLastIndex}  % Enthaelt die dritte Stelle (Indexnummer) des letzten angelegten Objekts
\newcounter{MLastType}   % Enthaelt den Typ des letzten angelegten Objekts (mithilfe der unten definierten Konstanten). Die Entscheidung, wie der Typ dargstellt wird, wird in split.pm beim Postprocessing getroffen.
\newcounter{MLastTypeEq} % =1 falls das Label in einer Matheumgebung (equation, eqnarray usw.) steht, =2 falls das Label in einer table-Umgebung steht

% Da ttm keine Zahlmakros verarbeiten kann, werden diese Nummern in den Zuweisungen hardcodiert!
\def\MTypeSection{1}          %# Zaehler ist section
\def\MTypeSubsection{2}       %# Zaehler ist subsection
\def\MTypeSubsubsection{3}    %# Zaehler ist subsubsection
\def\MTypeInfo{4}             %# Eine Infobox, Separatzaehler fuer die Chemie (auch wenn es dort nicht nummeriert wird) ist MInfoCounter
\def\MTypeExercise{5}         %# Eine Aufgabe, Separatzaehler fuer die Chemie ist MExerciseCounter
\def\MTypeExample{6}          %# Eine Beispielbox, Separatzaehler fuer die Chemie ist MExampleCounter
\def\MTypeExperiment{7}       %# Eine Versuchsbox, Separatzaehler fuer die Chemie ist MExperimentCounter
\def\MTypeGraphics{8}         %# Eine Graphik, Separatzaehler fuer alle FB ist MGraphicsCounter
\def\MTypeTable{9}            %# Eine Tabellennummer, hat keinen Zaehler da durch table gezaehlt wird
\def\MTypeEquation{10}        %# Eine Gleichungsnummer, hat keinen Zaehler da durch equation/eqnarray gezaehlt wird
\def\MTypeTheorem{11}         % Ein theorem oder xtheorem, Separatzaehler fuer die Chemie ist MTheoremCounter
\def\MTypeVideo{12}           %# Ein Video,Separatzaehler fuer alle FB ist MVideoCounter
\def\MTypeEntry{13}           %# Ein Eintrag fuer die Stichwortliste, wird nicht gezaehlt sondern erhaelt im preparsing ein unique-label 

% Zaehler fuer das Labelsystem sind prefixcounter, jeder Zaehler wird VOR dem gezaehlten Objekt inkrementiert und zaehlt daher das aktuelle Objekt
\newcounter{MInfoCounter}
\newcounter{MExerciseCounter}
\newcounter{MExampleCounter}
\newcounter{MExperimentCounter}
\newcounter{MGraphicsCounter}
\newcounter{MTableCounter}
\newcounter{MEquationCounter}  % Nur im HTML, sonst durch "equation"-counter von latex realisiert
\newcounter{MTheoremCounter}
\newcounter{MObjectCounter}   % Gemeinsamer Zaehler fuer Objekte (ausser Grafiken/Tabellen) in Mathe/Info/Physik
\newcounter{MVideoCounter}
\newcounter{MEntryCounter}

\newcounter{MTestSite} % 1 = Subsubsection ist eine Pruefungsseite, 0 = ist eine normale Seite (inkl. Hilfeseite)

\def\MCell{$\phantom{a}$}

\newenvironment{MExportExercise}{\begin{MExercise}}{\end{MExercise}} % wird von mconvert abgefangen

\def\MGenerateExNumber{%
\ifnum\value{MSepNumbers}=0%
\arabic{section}.\arabic{subsection}.\arabic{MObjectCounter}\setcounter{MLastIndex}{\value{MObjectCounter}}%
\else%
\arabic{section}.\arabic{subsection}.\arabic{MExerciseCounter}\setcounter{MLastIndex}{\value{MExerciseCounter}}%
\fi%
}%

\def\MGenerateExmpNumber{%
\ifnum\value{MSepNumbers}=0%
\arabic{section}.\arabic{subsection}.\arabic{MObjectCounter}\setcounter{MLastIndex}{\value{MObjectCounter}}%
\else%
\arabic{section}.\arabic{subsection}.\arabic{MExerciseCounter}\setcounter{MLastIndex}{\value{MExampleCounter}}%
\fi%
}%

\def\MGenerateInfoNumber{%
\ifnum\value{MSepNumbers}=0%
\arabic{section}.\arabic{subsection}.\arabic{MObjectCounter}\setcounter{MLastIndex}{\value{MObjectCounter}}%
\else%
\arabic{section}.\arabic{subsection}.\arabic{MExerciseCounter}\setcounter{MLastIndex}{\value{MInfoCounter}}%
\fi%
}%

\def\MGenerateSiteNumber{%
\arabic{section}.\arabic{subsection}.\arabic{subsubsection}%
}%

% Funktionalitaet fuer Auswahlaufgaben

\newcounter{MExerciseCollectionCounter} % = 0 falls nicht in collection-Umgebung, ansonsten Schachtelungstiefe
\newcounter{MExerciseCollectionTextCounter} % wird von MExercise-Umgebung inkrementiert und von MExerciseCollection-Umgebung auf Null gesetzt

\ifttm
% MExerciseCollection gruppiert Aufgaben, die dynamisch aus der Datenbank gezogen werden und nicht direkt in der HTML-Seite stehen
% Parameter: #1 = ID der Collection, muss eindeutig fuer alle IN DER DB VORHANDENEN collections sein unabhaengig vom Kurs
%            #2 = Optionsargument (im Moment: 1 = Iterative Auswahl, 2 = Zufallsbasierte Auswahl)
\newenvironment{MExerciseCollection}[2]{%
\addtocounter{MExerciseCollectionCounter}{1}
\setcounter{MExerciseCollectionTextCounter}{0}
\special{html:<!-- mexercisecollectionstart;;}#1\special{html:;;}#2\special{html:;; //-->}%
}{%
\special{html:<!-- mexercisecollectionstop //-->}%
\addtocounter{MExerciseCollectionCounter}{-1}
}
\else
\newenvironment{MExerciseCollection}[2]{%
\addtocounter{MExerciseCollectionCounter}{1}
\setcounter{MExerciseCollectionTextCounter}{0}
}{%
\addtocounter{MExerciseCollectionCounter}{-1}
}
\fi

% Bei Uebersetzung nach PDF werden die theorem-Umgebungen verwendet, bei Uebersetzung in HTML ein manuelles Makro
\ifttm%

  \newenvironment{MHint}[1]{  \special{html:<button name="Name_MHint}\arabic{MHintCounter}\special{html:" class="hintbutton_closed" id="MHint}\arabic{MHintCounter}\special{html:_button" %
  type="button" onclick="toggle_hint('MHint}\arabic{MHintCounter}\special{html:');">}#1\special{html:</button>}
  \special{html:<div class="hint" style="display:none" id="MHint}\arabic{MHintCounter}\special{html:"> }}{\begin{html}</div>\end{html}\addtocounter{MHintCounter}{1}}

  \newenvironment{MCOSHZusatz}{  \special{html:<button name="Name_MHint}\arabic{MHintCounter}\special{html:" class="chintbutton_closed" id="MHint}\arabic{MHintCounter}\special{html:_button" %
  type="button" onclick="toggle_hint('MHint}\arabic{MHintCounter}\special{html:');">}Weiterf�hrende Inhalte\special{html:</button>}
  \special{html:<div class="hintc" style="display:none" id="MHint}\arabic{MHintCounter}\special{html:">
  <div class="coshwarn">Diese Inhalte gehen �ber das Kursniveau hinaus und werden in den Aufgaben und Tests nicht abgefragt.</div><br />}
  \addtocounter{MHintCounter}{1}}{\begin{html}</div>\end{html}}

  
  \newenvironment{MDefinition}{\begin{definition}\setcounter{MLastIndex}{\value{definition}}\ \\}{\end{definition}}

  
  \newenvironment{MExercise}{
  \renewcommand{\MStdPoints}{4}
  \addtocounter{MExerciseCounter}{1}
  \addtocounter{MObjectCounter}{1}
  \setcounter{MLastType}{5}

  \ifnum\value{MExerciseCollectionCounter}=0\else\addtocounter{MExerciseCollectionTextCounter}{1}\special{html:<!-- mexercisetextstart;;}\arabic{MExerciseCollectionTextCounter}\special{html:;; //-->}\fi
  \special{html:<div class="aufgabe" id="ADIV_}\MGenerateExNumber\special{html:">}%
  \textbf{Aufgabe \MGenerateExNumber
  } \ \\}{
  \special{html:</div><!-- mfeedbackbutton;Aufgabe;}\arabic{MTestSite}\special{html:;}\MGenerateExNumber\special{html:; //-->}
  \ifnum\value{MExerciseCollectionCounter}=0\else\special{html:<!-- mexercisetextstop //-->}\fi
  }

  % Stellt eine Kombination aus Aufgabe, Loesungstext und Eingabefeld bereit,
  % bei der Aufgabentext und Musterloesung sowie die zugehoerigen Feldelemente
  % extern bezogen und div-aktualisiert werden, das Eingabefeld aber immer das gleiche ist.
  \newenvironment{MFetchExercise}{
  \addtocounter{MExerciseCounter}{1}
  \addtocounter{MObjectCounter}{1}
  \setcounter{MLastType}{5}

  \special{html:<div class="aufgabe" id="ADIV_}\MGenerateExNumber\special{html:">}%
  \textbf{Aufgabe \MGenerateExNumber
  } \ \\%
  \special{html:</div><div class="exfetch_text" id="ADIVTEXT_}\MGenerateExNumber\special{html:">}%
  \special{html:</div><div class="exfetch_sol" id="ADIVSOL_}\MGenerateExNumber\special{html:">}%
  \special{html:</div><div class="exfetch_input" id="ADIVINPUT_}\MGenerateExNumber\special{html:">}%
  }{
  \special{html:</div>}
  }

  \newenvironment{MExample}{
  \addtocounter{MExampleCounter}{1}
  \addtocounter{MObjectCounter}{1}
  \setcounter{MLastType}{6}
  \begin{html}
  <div class="exmp">
  <div class="exmprahmen">
  \end{html}\textbf{Beispiel
  \ifnum\value{MSepNumbers}=0
  \arabic{section}.\arabic{subsection}.\arabic{MObjectCounter}\setcounter{MLastIndex}{\value{MObjectCounter}}
  \else
  \arabic{section}.\arabic{subsection}.\arabic{MExampleCounter}\setcounter{MLastIndex}{\value{MExampleCounter}}
  \fi
  } \ \\}{\begin{html}</div>
  </div>
  \end{html}
  \special{html:<!-- mfeedbackbutton;Beispiel;}\arabic{MTestSite}\special{html:;}\MGenerateExmpNumber\special{html:; //-->}
  }

  \newenvironment{MExperiment}{
  \addtocounter{MExperimentCounter}{1}
  \addtocounter{MObjectCounter}{1}
  \setcounter{MLastType}{7}
  \begin{html}
  <div class="expe">
  <div class="experahmen">
  \end{html}\textbf{Versuch
  \ifnum\value{MSepNumbers}=0
  \arabic{section}.\arabic{subsection}.\arabic{MObjectCounter}\setcounter{MLastIndex}{\value{MObjectCounter}}
  \else
%  \arabic{MExperimentCounter}\setcounter{MLastIndex}{\value{MExperimentCounter}}
  \arabic{section}.\arabic{subsection}.\arabic{MExperimentCounter}\setcounter{MLastIndex}{\value{MExperimentCounter}}
  \fi
  } \ \\}{\begin{html}</div>
  </div>
  \end{html}}

  \newenvironment{MChemInfo}{
  \setcounter{MLastType}{4}
  \begin{html}
  <div class="info">
  <div class="inforahmen">
  \end{html}}{\begin{html}</div>
  </div>
  \end{html}}

  \newenvironment{MXInfo}[1]{
  \addtocounter{MInfoCounter}{1}
  \addtocounter{MObjectCounter}{1}
  \setcounter{MLastType}{4}
  \begin{html}
  <div class="info">
  <div class="inforahmen">
  \end{html}\textbf{#1
  \ifnum\value{MInfoNumbers}=0
  \else
    \ifnum\value{MSepNumbers}=0
    \arabic{section}.\arabic{subsection}.\arabic{MObjectCounter}\setcounter{MLastIndex}{\value{MObjectCounter}}
    \else
    \arabic{MInfoCounter}\setcounter{MLastIndex}{\value{MInfoCounter}}
    \fi
  \fi
  } \ \\}{\begin{html}</div>
  </div>
  \end{html}
  \special{html:<!-- mfeedbackbutton;Info;}\arabic{MTestSite}\special{html:;}\MGenerateInfoNumber\special{html:; //-->}
  }

  \newenvironment{MInfo}{\ifnum\value{MInfoNumbers}=0\begin{MChemInfo}\else\begin{MXInfo}{Info}\ \\ \fi}{\ifnum\value{MInfoNumbers}=0\end{MChemInfo}\else\end{MXInfo}\fi}

\else%

  \theoremstyle{MSatzStyle}
  \newtheorem{thm}{Satz}[section]
  \newtheorem{thmc}{Satz}
  \theoremstyle{MDefStyle}
  \newtheorem{defn}[thm]{Definition}
  \newtheorem{exmp}[thm]{Beispiel}
  \newtheorem{info}[thm]{\MInfoText}
  \theoremstyle{MDefStyle}
  \newtheorem{defnc}{Definition}
  \theoremstyle{MDefStyle}
  \newtheorem{exmpc}{Beispiel}[section]
  \theoremstyle{MDefStyle}
  \newtheorem{infoc}{\MInfoText}
  \theoremstyle{MDefStyle}
  \newtheorem{exrc}{Aufgabe}[section]
  \theoremstyle{MDefStyle}
  \newtheorem{verc}{Versuch}[section]
  
  \newenvironment{MFetchExercise}{}{} % kann im PDF nicht dargestellt werden
  
  \newenvironment{MExercise}{\begin{exrc}\renewcommand{\MStdPoints}{1}\MTB}{\end{exrc}}
  \newenvironment{MHint}[1]{\ \\ \underline{#1:}\\}{}
  \newenvironment{MCOSHZusatz}{\ \\ \underline{Weiterf�hrende Inhalte:}\\}{}
  \newenvironment{MDefinition}{\ifnum\value{MInfoNumbers}=0\begin{defnc}\else\begin{defn}\fi\MTB}{\ifnum\value{MInfoNumbers}=0\end{defnc}\else\end{defn}\fi}
%  \newenvironment{MExample}{\begin{exmp}}{\ \linebreak[1] \ \ \ \ $\phantom{a}$ \ \hfill $\blacklozenge$\end{exmp}}
  \newenvironment{MExample}{
    \ifnum\value{MInfoNumbers}=0\begin{exmpc}\else\begin{exmp}\fi
    \MTB
    \begin{exmpshaded}
    \ \newline
}{
    \end{exmpshaded}
    \ifnum\value{MInfoNumbers}=0\end{exmpc}\else\end{exmp}\fi
}
  \newenvironment{MChemInfo}{\begin{infoshaded}}{\end{infoshaded}}

  \newenvironment{MInfo}{\ifnum\value{MInfoNumbers}=0\begin{MChemInfo}\else\renewcommand{\MInfoText}{Info}\begin{info}\begin{infoshaded}
  \MTB
   \ \newline
    \fi
  }{\ifnum\value{MInfoNumbers}=0\end{MChemInfo}\else\end{infoshaded}\end{info}\fi}

  \newenvironment{MXInfo}[1]{
    \renewcommand{\MInfoText}{#1}
    \ifnum\value{MInfoNumbers}=0\begin{infoc}\else\begin{info}\fi%
    \MTB
    \begin{infoshaded}
    \ \newline
  }{\end{infoshaded}\ifnum\value{MInfoNumbers}=0\end{infoc}\else\end{info}\fi}

  \newenvironment{MExperiment}{
    \renewcommand{\MInfoText}{Versuch}
    \ifnum\value{MInfoNumbers}=0\begin{verc}\else\begin{info}\fi
    \MTB
    \begin{expeshaded}
    \ \newline
  }{
    \end{expeshaded}
    \ifnum\value{MInfoNumbers}=0\end{verc}\else\end{info}\fi
  }
\fi%

% MHint sollte nicht direkt fuer Loesungen benutzt werden wegen solutionselect
\newenvironment{MSolution}{\begin{MHint}{L"osung}}{\end{MHint}}

\newcounter{MCodeCounter}

\ifttm
\newenvironment{MCode}{\special{html:<!-- mcodestart -->}\ttfamily\color{blue}}{\special{html:<!-- mcodestop -->}}
\else
\newenvironment{MCode}{\begin{flushleft}\ttfamily\addtocounter{MCodeCounter}{1}}{\addtocounter{MCodeCounter}{-1}\end{flushleft}}
% Ohne color-Statement da inkompatible mit framed/shaded-Boxen aus dem framed-package
\fi

%----------------- Sonderdefinitionen fuer Symbole, die der Konverter nicht kann ----------------------------------------------

\ifttm%
\newcommand{\MUnderset}[2]{\underbrace{#2}_{#1}}%
\else%
\newcommand{\MUnderset}[2]{\underset{#1}{#2}}%
\fi%

\ifttm
\newcommand{\MThinspace}{\special{html:<mi>&#x2009;</mi>}}
\else
\newcommand{\MThinspace}{\,}
\fi

\ifttm
\newcommand{\glq}{\begin{html}&sbquo;\end{html}}
\newcommand{\grq}{\begin{html}&lsquo;\end{html}}
\newcommand{\glqq}{\begin{html}&bdquo;\end{html}}
\newcommand{\grqq}{\begin{html}&ldquo;\end{html}}
\fi

\ifttm
\newcommand{\MNdash}{\begin{html}&ndash;\end{html}}
\else
\newcommand{\MNdash}{--}
\fi

%\ifttm\def\MIU{\special{html:<mi>&#8520;</mi>}}\else\def\MIU{\mathrm{i}}\fi
\def\MIU{\mathrm{i}}
\def\MEU{e} % TU9-Onlinekurs: italic-e
%\def\MEU{\mathrm{e}} % Alte Onlinemodule: roman-e
\def\MD{d} % Kursives d in Integralen im TU9-Onlinekurs
%\def\MD{\mathrm{d}} % roman-d in den alten Onlinemodulen
\def\MDB{\|}

%zusaetzlicher Leerraum vor "\MD"
\ifttm%
\def\MDSpace{\special{html:<mi>&#x2009;</mi>}}
\else%
\def\MDSpace{\,}
\fi%
\newcommand{\MDwSp}{\MDSpace\MD}%

\ifttm
\def\Mdq{\dq}
\else
\def\Mdq{\dq}
\fi

\def\MSpan#1{\left<{#1}\right>}
\def\MSetminus{\setminus}
\def\MIM{I}

\ifttm
\newcommand{\ld}{\text{ld}}
\newcommand{\lg}{\text{lg}}
\else
\DeclareMathOperator{\ld}{ld}
%\newcommand{\lg}{\text{lg}} % in latex schon definiert
\fi


\def\Mmapsto{\ifttm\special{html:<mi>&mapsto;</mi>}\else\mapsto\fi} 
\def\Mvarphi{\ifttm\phi\else\varphi\fi}
\def\Mphi{\ifttm\varphi\else\phi\fi}
\ifttm%
\newcommand{\MEumu}{\special{html:<mi>&#x3BC;</mi>}}%
\else%
\newcommand{\MEumu}{\textrm{\textmu}}%
\fi
\def\Mvarepsilon{\ifttm\epsilon\else\varepsilon\fi}
\def\Mepsilon{\ifttm\varepsilon\else\epsilon\fi}
\def\Mvarkappa{\ifttm\kappa\else\varkappa\fi}
\def\Mkappa{\ifttm\varkappa\else\kappa\fi}
\def\Mcomplement{\ifttm\special{html:<mi>&comp;</mi>}\else\complement\fi} 
\def\MWW{\mathrm{WW}}
\def\Mmod{\ifttm\special{html:<mi>&nbsp;mod&nbsp;</mi>}\else\mod\fi} 

\ifttm%
\def\mod{\text{\;mod\;}}%
\def\MNEquiv{\special{html:<mi>&NotCongruent;</mi>}}% 
\def\MNSubseteq{\special{html:<mi>&NotSubsetEqual;</mi>}}%
\def\MEmptyset{\special{html:<mi>&empty;</mi>}}%
\def\MVDots{\special{html:<mi>&#x22EE;</mi>}}%
\def\MHDots{\special{html:<mi>&#x2026;</mi>}}%
\def\Mddag{\special{html:<mi>&#x1202;</mi>}}%
\def\sphericalangle{\special{html:<mi>&measuredangle;</mi>}}%
\def\nparallel{\special{html:<mi>&nparallel;</mi>}}%
\def\MProofEnd{\special{html:<mi>&#x25FB;</mi>}}%
\newenvironment{MProof}[1]{\underline{#1}:\MCR\MCR}{\hfill $\MProofEnd$}%
\else%
\def\MNEquiv{\not\equiv}%
\def\MNSubseteq{\not\subseteq}%
\def\MEmptyset{\emptyset}%
\def\MVDots{\vdots}%
\def\MHDots{\hdots}%
\def\Mddag{\ddag}%
\newenvironment{MProof}[1]{\begin{proof}[#1]}{\end{proof}}%
\fi%



% Spaces zum Auffuellen von Tabellenbreiten, die nur im HTML wirken
\ifttm%
\def\MTSP{\:}%
\else%
\def\MTSP{}%
\fi%

\DeclareMathOperator{\arsinh}{arsinh}
\DeclareMathOperator{\arcosh}{arcosh}
\DeclareMathOperator{\artanh}{artanh}
\DeclareMathOperator{\arcoth}{arcoth}


\newcommand{\MMathSet}[1]{\mathbb{#1}}
\def\N{\MMathSet{N}}
\def\Z{\MMathSet{Z}}
\def\Q{\MMathSet{Q}}
\def\R{\MMathSet{R}}
\def\C{\MMathSet{C}}

\newcounter{MForLoopCounter}
\newcommand{\MForLoop}[2]{\setcounter{MForLoopCounter}{#1}\ifnum\value{MForLoopCounter}=0{}\else{{#2}\addtocounter{MForLoopCounter}{-1}\MForLoop{\value{MForLoopCounter}}{#2}}\fi}

\newcounter{MSiteCounter}
\newcounter{MFieldCounter} % Kombination section.subsection.site.field ist eindeutig in allen Modulen, field alleine nicht

\newcounter{MiniMarkerCounter}

\ifttm
\newenvironment{MMiniPageP}[1]{\begin{minipage}{#1\linewidth}\special{html:<!-- minimarker;;}\arabic{MiniMarkerCounter}\special{html:;;#1; //-->}}{\end{minipage}\addtocounter{MiniMarkerCounter}{1}}
\else
\newenvironment{MMiniPageP}[1]{\begin{minipage}{#1\linewidth}}{\end{minipage}\addtocounter{MiniMarkerCounter}{1}}
\fi

\newcounter{AlignCounter}

\newcommand{\MStartJustify}{\ifttm\special{html:<!-- startalign;;}\arabic{AlignCounter}\special{html:;;justify; //-->}\fi}
\newcommand{\MStopJustify}{\ifttm\special{html:<!-- stopalign;;}\arabic{AlignCounter}\special{html:; //-->}\fi\addtocounter{AlignCounter}{1}}

\newenvironment{MJTabular}[1]{
\MStartJustify
\begin{tabular}{#1}
}{
\end{tabular}
\MStopJustify
}

\newcommand{\MImageLeft}[2]{
\begin{center}
\begin{tabular}{lc}
\MStartJustify
\begin{MMiniPageP}{0.65}
#1
\end{MMiniPageP}
\MStopJustify
&
\begin{MMiniPageP}{0.3}
#2  
\end{MMiniPageP}
\end{tabular}
\end{center}
}

\newcommand{\MImageHalf}[2]{
\begin{center}
\begin{tabular}{lc}
\MStartJustify
\begin{MMiniPageP}{0.45}
#1
\end{MMiniPageP}
\MStopJustify
&
\begin{MMiniPageP}{0.45}
#2  
\end{MMiniPageP}
\end{tabular}
\end{center}
}

\newcommand{\MBigImageLeft}[2]{
\begin{center}
\begin{tabular}{lc}
\MStartJustify
\begin{MMiniPageP}{0.25}
#1
\end{MMiniPageP}
\MStopJustify
&
\begin{MMiniPageP}{0.7}
#2  
\end{MMiniPageP}
\end{tabular}
\end{center}
}

\ifttm
\def\No{\mathbb{N}_0}
\else
\def\No{\ensuremath{\N_0}}
\fi
\def\MT{\textrm{\tiny T}}
\newcommand{\MTranspose}[1]{{#1}^{\MT}}
\ifttm
\newcommand{\MRe}{\mathsf{Re}}
\newcommand{\MIm}{\mathsf{Im}}
\else
\DeclareMathOperator{\MRe}{Re}
\DeclareMathOperator{\MIm}{Im}
\fi

\newcommand{\Mid}{\mathrm{id}}
\newcommand{\MFeinheit}{\mathrm{feinh}}

\ifttm
\newcommand{\Msubstack}[1]{\begin{array}{c}{#1}\end{array}}
\else
\newcommand{\Msubstack}[1]{\substack{#1}}
\fi

% Typen von Fragefeldern:
% 1 = Alphanumerisch, case-sensitive-Vergleich
% 2 = Ja/Nein-Checkbox, Loesung ist 0 oder 1   (OPTION = Image-id fuer Rueckmeldung)
% 3 = Reelle Zahlen Geparset
% 4 = Funktionen Geparset (mit Stuetzstellen zur ueberpruefung)

% Dieser Befehl erstellt ein interaktives Aufgabenfeld. Parameter:
% - #1 Laenge in Zeichen
% - #2 Loesungstext (alphanumerisch, case sensitive)
% - #3 AufgabenID (alphanumerisch, case sensitive)
% - #4 Typ (Kennnummer)
% - #5 String fuer Optionen (ggf. mit Semikolon getrennte Einzelstrings)
% - #6 Anzahl Punkte
% - #7 uxid (kann z.B. Loesungsstring sein)
% ACHTUNG: Die langen Zeilen bitte so lassen, Zeilenumbrueche im tex werden in div's umgesetzt
\newcommand{\MQuestionID}[7]{
\ifttm
\special{html:<!-- mdeclareuxid;;}UX#7\special{html:;;}\arabic{section}\special{html:;;}#3\special{html:;; //-->}%
\special{html:<!-- mdeclarepoints;;}\arabic{section}\special{html:;;}#3\special{html:;;}#6\special{html:;;}\arabic{MTestSite}\special{html:;;}\arabic{chapter}%
\special{html:;; //--><!-- onloadstart //-->CreateQuestionObj("}#7\special{html:",}\arabic{MFieldCounter}\special{html:,"}#2%
\special{html:","}#3\special{html:",}#4\special{html:,"}#5\special{html:",}#6\special{html:,}\arabic{MTestSite}\special{html:,}\arabic{section}%
\special{html:);<!-- onloadstop //-->}%
\special{html:<input mfieldtype="}#4\special{html:" name="Name_}#3\special{html:" id="}#3\special{html:" type="text" size="}#1\special{html:" maxlength="}#1%
\special{html:" }\ifnum\value{MGroupActive}=0\special{html:onfocus="handlerFocus(}\arabic{MFieldCounter}%
\special{html:);" onblur="handlerBlur(}\arabic{MFieldCounter}\special{html:);" onkeyup="handlerChange(}\arabic{MFieldCounter}\special{html:,0);" onpaste="handlerChange(}\arabic{MFieldCounter}\special{html:,0);" oninput="handlerChange(}\arabic{MFieldCounter}\special{html:,0);" onpropertychange="handlerChange(}\arabic{MFieldCounter}\special{html:,0);"/>}%
\special{html:<img src="images/questionmark.gif" width="20" height="20" border="0" align="absmiddle" id="}QM#3\special{html:"/>}
\else%
\special{html:onblur="handlerBlur(}\arabic{MFieldCounter}%
\special{html:);" onfocus="handlerFocus(}\arabic{MFieldCounter}\special{html:);" onkeyup="handlerChange(}\arabic{MFieldCounter}\special{html:,1);" onpaste="handlerChange(}\arabic{MFieldCounter}\special{html:,1);" oninput="handlerChange(}\arabic{MFieldCounter}\special{html:,1);" onpropertychange="handlerChange(}\arabic{MFieldCounter}\special{html:,1);"/>}%
\special{html:<img src="images/questionmark.gif" width="20" height="20" border="0" align="absmiddle" id="}QM#3\special{html:"/>}\fi%
\else%
\ifnum\value{QBoxFlag}=1\fbox{$\phantom{\MForLoop{#1}{b}}$}\else$\phantom{\MForLoop{#1}{b}}$\fi%
\fi%
}

% ACHTUNG: Die langen Zeilen bitte so lassen, Zeilenumbrueche im tex werden in div's umgesetzt
% QuestionCheckbox macht ausserhalb einer QuestionGroup keinen Sinn!
% #1 = solution (1 oder 0), ggf. mit ::smc abgetrennt auszuschliessende single-choice-boxen (UXIDs durch , getrennt), #2 = id, #3 = points, #4 = uxid
\newcommand{\MQuestionCheckbox}[4]{
\ifttm
\special{html:<!-- mdeclareuxid;;}UX#4\special{html:;;}\arabic{section}\special{html:;;}#2\special{html:;; //-->}%
\ifnum\value{MGroupActive}=0\MDebugMessage{ERROR: Checkbox Nr. \arabic{MFieldCounter}\ ist nicht in einer Kontrollgruppe, es wird niemals eine Loesung angezeigt!}\fi
\special{html: %
<!-- mdeclarepoints;;}\arabic{section}\special{html:;;}#2\special{html:;;}#3\special{html:;;}\arabic{MTestSite}\special{html:;;}\arabic{chapter}%
\special{html:;; //--><!-- onloadstart //-->CreateQuestionObj("}#4\special{html:",}\arabic{MFieldCounter}\special{html:,"}#1\special{html:","}#2\special{html:",2,"IMG}#2%
\special{html:",}#3\special{html:,}\arabic{MTestSite}\special{html:,}\arabic{section}\special{html:);<!-- onloadstop //-->}%
\special{html:<input mfieldtype="2" type="checkbox" name="Name_}#2\special{html:" id="}#2\special{html:" onchange="handlerChange(}\arabic{MFieldCounter}\special{html:,1);"/><img src="images/questionmark.gif" name="}Name_IMG#2%
\special{html:" width="20" height="20" border="0" align="absmiddle" id="}IMG#2\special{html:"/> }%
\else%
\ifnum\value{QBoxFlag}=1\fbox{$\phantom{X}$}\else$\phantom{X}$\fi%
\fi%
}

\def\MGenerateID{QFELD_\arabic{section}.\arabic{subsection}.\arabic{MSiteCounter}.QF\arabic{MFieldCounter}}

% #1 = 0/1 ggf. mit ::smc abgetrennt auszuschliessende single-choice-boxen (UXIDs durch , getrennt ohne UX), #2 = uxid ohne UX
\newcommand{\MCheckbox}[2]{
\MQuestionCheckbox{#1}{\MGenerateID}{\MStdPoints}{#2}
\addtocounter{MFieldCounter}{1}
}

% Erster Parameter: Zeichenlaenge der Eingabebox, zweiter Parameter: Loesungstext
\newcommand{\MQuestion}[2]{
\MQuestionID{#1}{#2}{\MGenerateID}{1}{0}{\MStdPoints}{#2}
\addtocounter{MFieldCounter}{1}
}

% Erster Parameter: Zeichenlaenge der Eingabebox, zweiter Parameter: Loesungstext
\newcommand{\MLQuestion}[3]{
\MQuestionID{#1}{#2}{\MGenerateID}{1}{0}{\MStdPoints}{#3}
\addtocounter{MFieldCounter}{1}
}

% Parameter: Laenge des Feldes, Loesung (wird auch geparsed), Stellen Genauigkeit hinter dem Komma, weitere Stellen werden mathematisch gerundet vor Vergleich
\newcommand{\MParsedQuestion}[3]{
\MQuestionID{#1}{#2}{\MGenerateID}{3}{#3}{\MStdPoints}{#2}
\addtocounter{MFieldCounter}{1}
}

% Parameter: Laenge des Feldes, Loesung (wird auch geparsed), Stellen Genauigkeit hinter dem Komma, weitere Stellen werden mathematisch gerundet vor Vergleich
\newcommand{\MLParsedQuestion}[4]{
\MQuestionID{#1}{#2}{\MGenerateID}{3}{#3}{\MStdPoints}{#4}
\addtocounter{MFieldCounter}{1}
}

% Parameter: Laenge des Feldes, Loesungsfunktion, Anzahl Stuetzstellen, Funktionsvariablen durch Kommata getrennt (nicht case-sensitive), Anzahl Nachkommastellen im Vergleich
\newcommand{\MFunctionQuestion}[5]{
\MQuestionID{#1}{#2}{\MGenerateID}{4}{#3;#4;#5;0}{\MStdPoints}{#2}
\addtocounter{MFieldCounter}{1}
}

% Parameter: Laenge des Feldes, Loesungsfunktion, Anzahl Stuetzstellen, Funktionsvariablen durch Kommata getrennt (nicht case-sensitive), Anzahl Nachkommastellen im Vergleich, UXID
\newcommand{\MLFunctionQuestion}[6]{
\MQuestionID{#1}{#2}{\MGenerateID}{4}{#3;#4;#5;0}{\MStdPoints}{#6}
\addtocounter{MFieldCounter}{1}
}

% Parameter: Laenge des Feldes, Loesungsintervall, Genauigkeit der Zahlenwertpruefung
\newcommand{\MIntervalQuestion}[3]{
\MQuestionID{#1}{#2}{\MGenerateID}{6}{#3}{\MStdPoints}{#2}
\addtocounter{MFieldCounter}{1}
}

% Parameter: Laenge des Feldes, Loesungsintervall, Genauigkeit der Zahlenwertpruefung, UXID
\newcommand{\MLIntervalQuestion}[4]{
\MQuestionID{#1}{#2}{\MGenerateID}{6}{#3}{\MStdPoints}{#4}
\addtocounter{MFieldCounter}{1}
}

% Parameter: Laenge des Feldes, Loesungsfunktion, Anzahl Stuetzstellen, Funktionsvariable (nicht case-sensitive), Anzahl Nachkommastellen im Vergleich, Vereinfachungsbedingung
% Vereinfachungsbedingung ist eine der Folgenden:
% 0 = Keine Vereinfachungsbedingung
% 1 = Keine Klammern (runde oder eckige) mehr im vereinfachten Ausdruck
% 2 = Faktordarstellung (Term hat Produkte als letzte Operation, Summen als vorgeschaltete Operation)
% 3 = Summendarstellung (Term hat Summen als letzte Operation, Produkte als vorgeschaltete Operation)
% Flag 512: Besondere Stuetzstellen (nur >1 und nur schwach rational), sonst symmetrisch um Nullpunkt und ganze Zahlen inkl. Null werden getroffen
\newcommand{\MSimplifyQuestion}[6]{
\MQuestionID{#1}{#2}{\MGenerateID}{4}{#3;#4;#5;#6}{\MStdPoints}{#2}
\addtocounter{MFieldCounter}{1}
}

\newcommand{\MLSimplifyQuestion}[7]{
\MQuestionID{#1}{#2}{\MGenerateID}{4}{#3;#4;#5;#6}{\MStdPoints}{#7}
\addtocounter{MFieldCounter}{1}
}

% Parameter: Laenge des Feldes, Loesung (optionaler Ausdruck), Anzahl Stuetzstellen, Funktionsvariable (nicht case-sensitive), Anzahl Nachkommastellen im Vergleich, Spezialtyp (string-id)
\newcommand{\MLSpecialQuestion}[7]{
\MQuestionID{#1}{#2}{\MGenerateID}{7}{#3;#4;#5;#6}{\MStdPoints}{#7}
\addtocounter{MFieldCounter}{1}
}

\newcounter{MGroupStart}
\newcounter{MGroupEnd}
\newcounter{MGroupActive}

\newenvironment{MQuestionGroup}{
\setcounter{MGroupStart}{\value{MFieldCounter}}
\setcounter{MGroupActive}{1}
}{
\setcounter{MGroupActive}{0}
\setcounter{MGroupEnd}{\value{MFieldCounter}}
\addtocounter{MGroupEnd}{-1}
}

\newcommand{\MGroupButton}[1]{
\ifttm
\special{html:<button name="Name_Group}\arabic{MGroupStart}\special{html:to}\arabic{MGroupEnd}\special{html:" id="Group}\arabic{MGroupStart}\special{html:to}\arabic{MGroupEnd}\special{html:" %
type="button" onclick="group_button(}\arabic{MGroupStart}\special{html:,}\arabic{MGroupEnd}\special{html:);">}#1\special{html:</button>}
\else
\phantom{#1}
\fi
}

%----------------- Makros fuer die modularisierte Darstellung ------------------------------------

\def\MyText#1{#1}

% is used internally by the conversion package, should not be used by original tex documents
\def\MOrgLabel#1{\relax}

\ifttm

% Ein MLabel wird im html codiert durch das tag <!-- mmlabel;;Labelbezeichner;;SubjectArea;;chapter;;section;;subsection;;Index;;Objekttyp; //-->
\def\MLabel#1{%
\ifnum\value{MLastType}=8%
\ifnum\value{MCaptionOn}=0%
\MDebugMessage{ERROR: Grafik \arabic{MGraphicsCounter} hat separates label: #1 (Grafiklabels sollten nur in der Caption stehen)}%
\fi
\fi
\ifnum\value{MLastType}=12%
\ifnum\value{MCaptionOn}=0%
\MDebugMessage{ERROR: Video \arabic{MVideoCounter} hat separates label: #1 (Videolabels sollten nur in der Caption stehen}%
\fi
\fi
\ifnum\value{MLastType}=10\setcounter{MLastIndex}{\value{equation}}\fi
\label{#1}\begin{html}<!-- mmlabel;;#1;;\end{html}\arabic{MSubjectArea}\special{html:;;}\arabic{chapter}\special{html:;;}\arabic{section}\special{html:;;}\arabic{subsection}\special{html:;;}\arabic{MLastIndex}\special{html:;;}\arabic{MLastType}\special{html:; //-->}}%

\else

% Sonderbehandlung im PDF fuer Abbildungen in separater aux-Datei, da MGraphics die figure-Umgebung nicht verwendet
\def\MLabel#1{%
\ifnum\value{MLastType}=8%
\ifnum\value{MCaptionOn}=0%
\MDebugMessage{ERROR: Grafik \arabic{MGraphicsCounter} hat separates label: #1 (Grafiklabels sollten nur in der Caption stehen}%
\fi
\fi
\ifnum\value{MLastType}=12%
\ifnum\value{MCaptionOn}=0%
\MDebugMessage{ERROR: Video \arabic{MVideoCounter} hat separates label: #1 (Videolabels sollten nur in der Caption stehen}%
\fi
\fi
\label{#1}%
}%

\fi

% Gibt Begriff des referenzierten Objekts mit aus, aber nur im HTML, daher nur in Ausnahmefaellen (z.B. Copyrightliste) sinnvoll
\def\MCRef#1{\ifttm\special{html:<!-- mmref;;}#1\special{html:;;1; //-->}\else\vref{#1}\fi}


\def\MRef#1{\ifttm\special{html:<!-- mmref;;}#1\special{html:;;0; //-->}\else\vref{#1}\fi}
\def\MERef#1{\ifttm\special{html:<!-- mmref;;}#1\special{html:;;0; //-->}\else\eqref{#1}\fi}
\def\MNRef#1{\ifttm\special{html:<!-- mmref;;}#1\special{html:;;0; //-->}\else\ref{#1}\fi}
\def\MSRef#1#2{\ifttm\special{html:<!-- msref;;}#1\special{html:;;}#2\special{html:; //-->}\else \if#2\empty \ref{#1} \else \hyperref[#1]{#2}\fi\fi} 

\def\MRefRange#1#2{\ifttm\MRef{#1} bis 
\MRef{#2}\else\vrefrange[\unskip]{#1}{#2}\fi}

\def\MRefTwo#1#2{\ifttm\MRef{#1} und \MRef{#2}\else%
\let\vRefTLRsav=\reftextlabelrange\let\vRefTPRsav=\reftextpagerange%
\def\reftextlabelrange##1##2{\ref{##1} und~\ref{##2}}%
\def\reftextpagerange##1##2{auf den Seiten~\pageref{#1} und~\pageref{#2}}%
\vrefrange[\unskip]{#1}{#2}%
\let\reftextlabelrange=\vRefTLRsav\let\reftextpagerange=\vRefTPRsav\fi}

% MSectionChapter definiert falls notwendig das Kapitel vor der section. Das ist notwendig, wenn nur ein Einzelmodul uebersetzt wird.
% MChaptersGiven ist ein Counter, der von mconvert.pl vordefiniert wird.
\ifttm
\newcommand{\MSectionChapter}{\ifnum\value{MChaptersGiven}=0{\Dchapter{Modul}}\else{}\fi}
\else
\newcommand{\MSectionChapter}{\ifnum\value{chapter}=0{\Dchapter{Modul}}\else{}\fi}
\fi


\def\MChapter#1{\ifnum\value{MSSEnd}>0{\MSubsectionEndMacros}\addtocounter{MSSEnd}{-1}\fi\Dchapter{#1}}
\def\MSubject#1{\MChapter{#1}} % Schluesselwort HELPSECTION ist reserviert fuer Hilfesektion

\newcommand{\MSectionID}{UNKNOWNID}

\ifttm
\newcommand{\MSetSectionID}[1]{\renewcommand{\MSectionID}{#1}}
\else
\newcommand{\MSetSectionID}[1]{\renewcommand{\MSectionID}{#1}\tikzsetexternalprefix{#1}}
\fi


\newcommand{\MSection}[1]{\MSetSectionID{MODULID}\ifnum\value{MSSEnd}>0{\MSubsectionEndMacros}\addtocounter{MSSEnd}{-1}\fi\MSectionChapter\Dsection{#1}\MSectionStartMacros{#1}\setcounter{MLastIndex}{-1}\setcounter{MLastType}{1}} % Sections werden ueber das section-Feld im mmlabel-Tag identifiziert, nicht ueber das Indexfeld

\def\MSubsection#1{\ifnum\value{MSSEnd}>0{\MSubsectionEndMacros}\addtocounter{MSSEnd}{-1}\fi\ifttm\else\clearpage\fi\Dsubsection{#1}\MSubsectionStartMacros\setcounter{MLastIndex}{-1}\setcounter{MLastType}{2}\addtocounter{MSSEnd}{1}}% Subsections werden ueber das subsection-Feld im mmlabel-Tag identifiziert, nicht ueber das Indexfeld
\def\MSubsectionx#1{\Dsubsectionx{#1}} % Nur zur Verwendung in MSectionStart gedacht
\def\MSubsubsection#1{\Dsubsubsection{#1}\setcounter{MLastIndex}{\value{subsubsection}}\setcounter{MLastType}{3}\ifttm\special{html:<!-- sectioninfo;;}\arabic{section}\special{html:;;}\arabic{subsection}\special{html:;;}\arabic{subsubsection}\special{html:;;1;;}\arabic{MTestSite}\special{html:; //-->}\fi}
\def\MSubsubsectionx#1{\Dsubsubsectionx{#1}\ifttm\special{html:<!-- sectioninfo;;}\arabic{section}\special{html:;;}\arabic{subsection}\special{html:;;}\arabic{subsubsection}\special{html:;;0;;}\arabic{MTestSite}\special{html:; //-->}\else\addcontentsline{toc}{subsection}{#1}\fi}

\ifttm
\def\MSubsubsubsectionx#1{\ \newline\textbf{#1}\special{html:<br />}}
\else
\def\MSubsubsubsectionx#1{\ \newline
\textbf{#1}\ \\
}
\fi


% Dieses Skript wird zu Beginn jedes Modulabschnitts (=Webseite) ausgefuehrt und initialisiert den Aufgabenfeldzaehler
\newcommand{\MPageScripts}{
\setcounter{MFieldCounter}{1}
\addtocounter{MSiteCounter}{1}
\setcounter{MHintCounter}{1}
\setcounter{MCodeEditCounter}{1}
\setcounter{MGroupActive}{0}
\DoQBoxes
% Feldvariablen werden im HTML-Header in conv.pl eingestellt
}

% Dieses Skript wird zum Ende jedes Modulabschnitts (=Webseite) ausgefuehrt
\ifttm
\newcommand{\MEndScripts}{\special{html:<br /><!-- mfeedbackbutton;Seite;}\arabic{MTestSite}\special{html:;}\MGenerateSiteNumber\special{html:; //-->}
}
\else
\newcommand{\MEndScripts}{\relax}
\fi


\newcounter{QBoxFlag}
\newcommand{\DoQBoxes}{\setcounter{QBoxFlag}{1}}
\newcommand{\NoQBoxes}{\setcounter{QBoxFlag}{0}}

\newcounter{MXCTest}
\newcounter{MXCounter}
\newcounter{MSCounter}



\ifttm

% Struktur des sectioninfo-Tags: <!-- sectioninfo;;section;;subsection;;subsubsection;;nr_ausgeben;;testpage; //-->

%Fuegt eine zusaetzliche html-Seite an hinter ALLEN bisherigen und zukuenftigen content-Seiten ausserhalb der vor-zurueck-Schleife (d.h. nur durch Button oder MIntLink erreichbar!)
% #1 = Titel des Modulabschnitts, #2 = Kurztitel fuer die Buttons, #3 = Buttonkennung (STD = default nehmen, NONE = Ohne Button in der Navigation)
\newenvironment{MSContent}[3]{\special{html:<div class="xcontent}\arabic{MSCounter}\special{html:"><!-- scontent;-;}\arabic{MSCounter};-;#1;-;#2;-;#3\special{html: //-->}\MPageScripts\MSubsubsectionx{#1}}{\MEndScripts\special{html:<!-- endscontent;;}\arabic{MSCounter}\special{html: //--></div>}\addtocounter{MSCounter}{1}}

% Fuegt eine zusaetzliche html-Seite ein hinter den bereits vorhandenen content-Seiten (oder als erste Seite) innerhalb der vor-zurueck-Schleife der Navigation
% #1 = Titel des Modulabschnitts, #2 = Kurztitel fuer die Buttons, #3 = Buttonkennung (STD = Defaultbutton, NONE = Ohne Button in der Navigation)
\newenvironment{MXContent}[3]{\special{html:<div class="xcontent}\arabic{MXCounter}\special{html:"><!-- xcontent;-;}\arabic{MXCounter};-;#1;-;#2;-;#3\special{html: //-->}\MPageScripts\MSubsubsection{#1}}{\MEndScripts\special{html:<!-- endxcontent;;}\arabic{MXCounter}\special{html: //--></div>}\addtocounter{MXCounter}{1}}

% Fuegt eine zusaetzliche html-Seite ein die keine subsubsection-Nummer bekommt, nur zur internen Verwendung in mintmod.tex gedacht!
% #1 = Titel des Modulabschnitts, #2 = Kurztitel fuer die Buttons, #3 = Buttonkennung (STD = Defaultbutton, NONE = Ohne Button in der Navigation)
% \newenvironment{MUContent}[3]{\special{html:<div class="xcontent}\arabic{MXCounter}\special{html:"><!-- xcontent;-;}\arabic{MXCounter};-;#1;-;#2;-;#3\special{html: //-->}\MPageScripts\MSubsubsectionx{#1}}{\MEndScripts\special{html:<!-- endxcontent;;}\arabic{MXCounter}\special{html: //--></div>}\addtocounter{MXCounter}{1}}

\newcommand{\MDeclareSiteUXID}[1]{\special{html:<!-- mdeclaresiteuxid;;}#1\special{html:;;}\arabic{chapter}\special{html:;;}\arabic{section}\special{html:;; //-->}}

\else

%\newcommand{\MSubsubsection}[1]{\refstepcounter{subsubsection} \addcontentsline{toc}{subsubsection}{\thesubsubsection. #1}}


% Fuegt eine zusaetzliche html-Seite an hinter den bereits vorhandenen content-Seiten
% #1 = Titel des Modulabschnitts, #2 = Kurztitel fuer die Buttons, #3 = Iconkennung (im PDF wirkungslos)
%\newenvironment{MUContent}[3]{\ifnum\value{MXCTest}>0{\MDebugMessage{ERROR: Geschachtelter SContent}}\fi\MPageScripts\MSubsubsectionx{#1}\addtocounter{MXCTest}{1}}{\addtocounter{MXCounter}{1}\addtocounter{MXCTest}{-1}}
\newenvironment{MXContent}[3]{\ifnum\value{MXCTest}>0{\MDebugMessage{ERROR: Geschachtelter SContent}}\fi\MPageScripts\MSubsubsection{#1}\addtocounter{MXCTest}{1}}{\addtocounter{MXCounter}{1}\addtocounter{MXCTest}{-1}}
\newenvironment{MSContent}[3]{\ifnum\value{MXCTest}>0{\MDebugMessage{ERROR: Geschachtelter XContent}}\fi\MPageScripts\MSubsubsectionx{#1}\addtocounter{MXCTest}{1}}{\addtocounter{MSCounter}{1}\addtocounter{MXCTest}{-1}}

\newcommand{\MDeclareSiteUXID}[1]{\relax}

\fi 

% GHEADER und GFOOTER werden von split.pm gefunden, aber nur, wenn nicht HELPSITE oder TESTSITE
\ifttm
\newenvironment{MSectionStart}{\special{html:<div class="xcontent0">}\MSubsubsectionx{Modul\"ubersicht}}{\setcounter{MSSEnd}{0}\special{html:</div>}}
% Darf nicht als XContent nummeriert werden, darf nicht als XContent gelabelt werden, wird aber in eine xcontent-div gesetzt fuer Python-parsing
\else
\newenvironment{MSectionStart}{\MSubsectionx{Modul\"ubersicht}}{\setcounter{MSSEnd}{0}}
\fi

\newenvironment{MIntro}{\begin{MXContent}{Einf\"uhrung}{Einf\"uhrung}{genetisch}}{\end{MXContent}}
\newenvironment{MContent}{\begin{MXContent}{Inhalt}{Inhalt}{beweis}}{\end{MXContent}}
\newenvironment{MExercises}{\ifttm\else\clearpage\fi\begin{MXContent}{Aufgaben}{Aufgaben}{aufgb}\special{html:<!-- declareexcsymb //-->}}{\end{MXContent}}

% #1 = Lesbare Testbezeichnung
\newenvironment{MTest}[1]{%
\renewcommand{\MTestName}{#1}
\ifttm\else\clearpage\fi%
\addtocounter{MTestSite}{1}%
\begin{MXContent}{#1}{#1}{STD} % {aufgb}%
\special{html:<!-- declaretestsymb //-->}
\begin{MQuestionGroup}%
\MInTestHeader
}%
{%
\end{MQuestionGroup}%
\ \\ \ \\%
\MInTestFooter
\end{MXContent}\addtocounter{MTestSite}{-1}%
}

\newenvironment{MExtra}{\ifttm\else\clearpage\fi\begin{MXContent}{Zus\"atzliche Inhalte}{Zusatz}{weiterfhrg}}{\end{MXContent}}

\makeindex

\ifttm
\def\MPrintIndex{
\ifnum\value{MSSEnd}>0{\MSubsectionEndMacros}\addtocounter{MSSEnd}{-1}\fi
\renewcommand{\indexname}{Stichwortverzeichnis}
\special{html:<p><!-- printindex //--></p>}
}
\else
\def\MPrintIndex{
\ifnum\value{MSSEnd}>0{\MSubsectionEndMacros}\addtocounter{MSSEnd}{-1}\fi
\renewcommand{\indexname}{Stichwortverzeichnis}
\addcontentsline{toc}{section}{Stichwortverzeichnis}
\printindex
}
\fi


% Konstanten fuer die Modulfaecher

\def\MINTMathematics{1}
\def\MINTInformatics{2}
\def\MINTChemistry{3}
\def\MINTPhysics{4}
\def\MINTEngineering{5}

\newcounter{MSubjectArea}
\newcounter{MInfoNumbers} % Gibt an, ob die Infoboxen nummeriert werden sollen
\newcounter{MSepNumbers} % Gibt an, ob Beispiele und Experimente separat nummeriert werden sollen
\newcommand{\MSetSubject}[1]{
 % ttm kapiert setcounter mit Parametern nicht, also per if abragen und einsetzen
\ifnum#1=1\setcounter{MSubjectArea}{1}\setcounter{MInfoNumbers}{1}\setcounter{MSepNumbers}{0}\fi
\ifnum#1=2\setcounter{MSubjectArea}{2}\setcounter{MInfoNumbers}{1}\setcounter{MSepNumbers}{0}\fi
\ifnum#1=3\setcounter{MSubjectArea}{3}\setcounter{MInfoNumbers}{0}\setcounter{MSepNumbers}{1}\fi
\ifnum#1=4\setcounter{MSubjectArea}{4}\setcounter{MInfoNumbers}{0}\setcounter{MSepNumbers}{0}\fi
\ifnum#1=5\setcounter{MSubjectArea}{5}\setcounter{MInfoNumbers}{1}\setcounter{MSepNumbers}{0}\fi
% Separate Nummerntechnik fuer unsere Chemiker: alles dreistellig
\ifnum#1=3
  \ifttm
  \renewcommand{\theequation}{\arabic{section}.\arabic{subsection}.\arabic{equation}}
  \renewcommand{\thetable}{\arabic{section}.\arabic{subsection}.\arabic{table}} 
  \renewcommand{\thefigure}{\arabic{section}.\arabic{subsection}.\arabic{figure}} 
  \else
  \renewcommand{\theequation}{\arabic{chapter}.\arabic{section}.\arabic{equation}}
  \renewcommand{\thetable}{\arabic{chapter}.\arabic{section}.\arabic{table}}
  \renewcommand{\thefigure}{\arabic{chapter}.\arabic{section}.\arabic{figure}}
  \fi
\else
  \ifttm
  \renewcommand{\theequation}{\arabic{section}.\arabic{subsection}.\arabic{equation}}
  \renewcommand{\thetable}{\arabic{table}}
  \renewcommand{\thefigure}{\arabic{figure}}
  \else
  \renewcommand{\theequation}{\arabic{chapter}.\arabic{section}.\arabic{equation}}
  \renewcommand{\thetable}{\arabic{table}}
  \renewcommand{\thefigure}{\arabic{figure}}
  \fi
\fi
}

% Fuer tikz Autogenerierung
\newcounter{MTIKZAutofilenumber}

% Spezielle Counter fuer die Bentz-Module
\newcounter{mycounter}
\newcounter{chemapplet}
\newcounter{physapplet}

\newcounter{MSSEnd} % Ist 1 falls ein MSubsection aktiv ist, der einen MSubsectionEndMacro-Aufruf verursacht
\newcounter{MFileNumber}
\def\MLastFile{\special{html:[[!-- mfileref;;}\arabic{MFileNumber}\special{html:; //--]]}}

% Vollstaendiger Pfad ist \MMaterial / \MLastFilePath / \MLastFileName    ==   \MMaterial / \MLastFile

% Wird nur bei kompletter Baum-Erstellung ausgefuehrt!
% #1 = Lesbare Modulbezeichnung
\newcommand{\MSectionStartMacros}[1]{
\setcounter{MTestSite}{0}
\setcounter{MCaptionOn}{0}
\setcounter{MLastTypeEq}{0}
\setcounter{MSSEnd}{0}
\setcounter{MFileNumber}{0} % Preinkrekement-Counter
\setcounter{MTIKZAutofilenumber}{0}
\setcounter{mycounter}{1}
\setcounter{physapplet}{1}
\setcounter{chemapplet}{0}
\ifttm
\special{html:<!-- mdeclaresection;;}\arabic{chapter}\special{html:;;}\arabic{section}\special{html:;;}#1\special{html:;; //-->}%
\else
\setcounter{thmc}{0}
\setcounter{exmpc}{0}
\setcounter{verc}{0}
\setcounter{infoc}{0}
\fi
\setcounter{MiniMarkerCounter}{1}
\setcounter{AlignCounter}{1}
\setcounter{MXCTest}{0}
\setcounter{MCodeCounter}{0}
\setcounter{MEntryCounter}{0}
}

% Wird immer ausgefuehrt
\newcommand{\MSubsectionStartMacros}{
\ifttm\else\MPageHeaderDef\fi
\MWatermarkSettings
\setcounter{MXCounter}{0}
\setcounter{MSCounter}{0}
\setcounter{MSiteCounter}{1}
\setcounter{MExerciseCollectionCounter}{0}
% Zaehler fuer das Labelsystem zuruecksetzen (prefix-Zaehler)
\setcounter{MInfoCounter}{0}
\setcounter{MExerciseCounter}{0}
\setcounter{MExampleCounter}{0}
\setcounter{MExperimentCounter}{0}
\setcounter{MGraphicsCounter}{0}
\setcounter{MTableCounter}{0}
\setcounter{MTheoremCounter}{0}
\setcounter{MObjectCounter}{0}
\setcounter{MEquationCounter}{0}
\setcounter{MVideoCounter}{0}
\setcounter{equation}{0}
\setcounter{figure}{0}
}

\newcommand{\MSubsectionEndMacros}{
% Bei Chemiemodulen das PSE einhaengen, es soll als SContent am Ende erscheinen
\special{html:<!-- subsectionend //-->}
\ifnum\value{MSubjectArea}=3{\MIncludePSE}\fi
}


\ifttm
%\newcommand{\MEmbed}[1]{\MRegisterFile{#1}\begin{html}<embed src="\end{html}\MMaterial/\MLastFile\begin{html}" width="192" height="189"></embed>\end{html}}
\newcommand{\MEmbed}[1]{\MRegisterFile{#1}\begin{html}<embed src="\end{html}\MMaterial/\MLastFile\begin{html}"></embed>\end{html}}
\fi

%----------------- Makros fuer die Textdarstellung -----------------------------------------------

\ifttm
% MUGraphics bindet eine Grafik ein:
% Parameter 1: Dateiname der Grafik, relativ zur Position des Modul-Tex-Dokuments
% Parameter 2: Skalierungsoptionen fuer PDF (fuer includegraphics)
% Parameter 3: Titel fuer die Grafik, wird unter die Grafik mit der Grafiknummer gesetzt und kann MLabel bzw. MCopyrightLabel enthalten
% Parameter 4: Skalierungsoptionen fuer HTML (css-styles)

% ERSATZ: <img alt="My Image" src="data:image/png;base64,iVBORwA<MoreBase64SringHere>" />


\newcommand{\MUGraphics}[4]{\MRegisterFile{#1}\begin{html}
<div class="imagecenter">
<center>
<div>
<img src="\end{html}\MMaterial/\MLastFile\begin{html}" style="#4" alt="\end{html}\MMaterial/\MLastFile\begin{html}"/>
</div>
<div class="bildtext">
\end{html}
\addtocounter{MGraphicsCounter}{1}
\setcounter{MLastIndex}{\value{MGraphicsCounter}}
\setcounter{MLastType}{8}
\addtocounter{MCaptionOn}{1}
\ifnum\value{MSepNumbers}=0
\textbf{Abbildung \arabic{MGraphicsCounter}:} #3
\else
\textbf{Abbildung \arabic{section}.\arabic{subsection}.\arabic{MGraphicsCounter}:} #3
\fi
\addtocounter{MCaptionOn}{-1}
\begin{html}
</div>
</center>
</div>
<br />
\end{html}%
\special{html:<!-- mfeedbackbutton;Abbildung;}\arabic{MGraphicsCounter}\special{html:;}\arabic{section}.\arabic{subsection}.\arabic{MGraphicsCounter}\special{html:; //-->}%
}

% MVideo bindet ein Video als Einzeldatei ein:
% Parameter 1: Dateiname des Videos, relativ zur Position des Modul-Tex-Dokuments, ohne die Endung ".mp4"
% Parameter 2: Titel fuer das Video (kann MLabel oder MCopyrightLabel enthalten), wird unter das Video mit der Videonummer gesetzt
\newcommand{\MVideo}[2]{\MRegisterFile{#1.mp4}\begin{html}
<div class="imagecenter">
<center>
<div>
<video width="95\%" controls="controls"><source src="\end{html}\MMaterial/#1.mp4\begin{html}" type="video/mp4">Ihr Browser kann keine MP4-Videos abspielen!</video>
</div>
<div class="bildtext">
\end{html}
\addtocounter{MVideoCounter}{1}
\setcounter{MLastIndex}{\value{MVideoCounter}}
\setcounter{MLastType}{12}
\addtocounter{MCaptionOn}{1}
\ifnum\value{MSepNumbers}=0
\textbf{Video \arabic{MVideoCounter}:} #2
\else
\textbf{Video \arabic{section}.\arabic{subsection}.\arabic{MVideoCounter}:} #2
\fi
\addtocounter{MCaptionOn}{-1}
\begin{html}
</div>
</center>
</div>
<br />
\end{html}}

\newcommand{\MDVideo}[2]{\MRegisterFile{#1.mp4}\MRegisterFile{#1.ogv}\begin{html}
<div class="imagecenter">
<center>
<div>
<video width="70\%" controls><source src="\end{html}\MMaterial/#1.mp4\begin{html}" type="video/mp4"><source src="\end{html}\MMaterial/#1.ogv\begin{html}" type="video/ogg">Ihr Browser kann keine MP4-Videos abspielen!</video>
</div>
<br />
#2
</center>
</div>
<br />
\end{html}
}

\newcommand{\MGraphics}[3]{\MUGraphics{#1}{#2}{#3}{}}

\else

\newcommand{\MVideo}[2]{%
% Kein Video im PDF darstellbar, trotzdem so tun als ob da eines waere
\begin{center}
(Video nicht darstellbar)
\end{center}
\addtocounter{MVideoCounter}{1}
\setcounter{MLastIndex}{\value{MVideoCounter}}
\setcounter{MLastType}{12}
\addtocounter{MCaptionOn}{1}
\ifnum\value{MSepNumbers}=0
\textbf{Video \arabic{MVideoCounter}:} #2
\else
\textbf{Video \arabic{section}.\arabic{subsection}.\arabic{MVideoCounter}:} #2
\fi
\addtocounter{MCaptionOn}{-1}
}


% MGraphics bindet eine Grafik ein:
% Parameter 1: Dateiname der Grafik, relativ zur Position des Modul-Tex-Dokuments
% Parameter 2: Skalierungsoptionen fuer PDF (fuer includegraphics)
% Parameter 3: Titel fuer die Grafik, wird unter die Grafik mit der Grafiknummer gesetzt
\newcommand{\MGraphics}[3]{%
\MRegisterFile{#1}%
\ %
\begin{figure}[H]%
\centering{%
\includegraphics[#2]{\MDPrefix/#1}%
\addtocounter{MCaptionOn}{1}%
\caption{#3}%
\addtocounter{MCaptionOn}{-1}%
}%
\end{figure}%
\addtocounter{MGraphicsCounter}{1}\setcounter{MLastIndex}{\value{MGraphicsCounter}}\setcounter{MLastType}{8}\ %
%\ \\Abbildung \ifnum\value{MSepNumbers}=0\else\arabic{chapter}.\arabic{section}.\fi\arabic{MGraphicsCounter}: #3%
}

\newcommand{\MUGraphics}[4]{\MGraphics{#1}{#2}{#3}}


\fi

\newcounter{MCaptionOn} % = 1 falls eine Grafikcaption aktiv ist, = 0 sonst


% MGraphicsSolo bindet eine Grafik pur ein ohne Titel
% Parameter 1: Dateiname der Grafik, relativ zur Position des Modul-Tex-Dokuments
% Parameter 2: Skalierungsoptionen (wirken nur im PDF)
\newcommand{\MGraphicsSolo}[2]{\MUGraphicsSolo{#1}{#2}{}}

% MUGraphicsSolo bindet eine Grafik pur ein ohne Titel, aber mit HTML-Skalierung
% Parameter 1: Dateiname der Grafik, relativ zur Position des Modul-Tex-Dokuments
% Parameter 2: Skalierungsoptionen (wirken nur im PDF)
% Parameter 3: Skalierungsoptionen (wirken nur im HTML), als style-format: "width=???, height=???"
\ifttm
\newcommand{\MUGraphicsSolo}[3]{\MRegisterFile{#1}\begin{html}
<img src="\end{html}\MMaterial/\MLastFile\begin{html}" style="\end{html}#3\begin{html}" alt="\end{html}\MMaterial/\MLastFile\begin{html}"/>
\end{html}%
\special{html:<!-- mfeedbackbutton;Abbildung;}#1\special{html:;}\MMaterial/\MLastFile\special{html:; //-->}%
}
\else
\newcommand{\MUGraphicsSolo}[3]{\MRegisterFile{#1}\includegraphics[#2]{\MDPrefix/#1}}
\fi

% Externer Link mit URL
% Erster Parameter: Vollstaendige(!) URL des Links
% Zweiter Parameter: Text fuer den Link
\newcommand{\MExtLink}[2]{\ifttm\special{html:<a target="_new" href="}#1\special{html:">}#2\special{html:</a>}\else\href{#1}{#2}\fi} % ohne MINTERLINK!


% Interner Link, die verlinkte Datei muss im gleichen Verzeichnis liegen wie die Modul-Texdatei
% Erster Parameter: Dateiname
% Zweiter Parameter: Text fuer den Link
\newcommand{\MIntLink}[2]{\ifttm\MRegisterFile{#1}\special{html:<a class="MINTERLINK" target="_new" href="}\MMaterial/\MLastFile\special{html:">}#2\special{html:</a>}\else{\href{#1}{#2}}\fi}


\ifttm
\def\MMaterial{:localmaterial:}
\else
\def\MMaterial{\MDPrefix}
\fi

\ifttm
\def\MNoFile#1{:directmaterial:#1}
\else
\def\MNoFile#1{#1}
\fi

\newcommand{\MChem}[1]{$\mathrm{#1}$}

\newcommand{\MApplet}[3]{
% Bindet ein Java-Applet ein, die Parameter sind:
% (wird nur im HTML, aber nicht im PDF erstellt)
% #1 Dateiname des Applets (muss mit ".class" enden)
% #2 = Breite in Pixeln
% #3 = Hoehe in Pixeln
\ifttm
\MRegisterFile{#1}
\begin{html}
<applet code="\end{html}\MMaterial/\MLastFile\begin{html}" width="#2" height="#3" alt="[Java-Applet kann nicht gestartet werden]"></applet>
\end{html}
\fi
}

\newcommand{\MScriptPage}[2]{
% Bindet eine JavaScript-Datei ein, die eine eigene Seite bekommt
% (wird nur im HTML, aber nicht im PDF erstellt)
% #1 Dateiname des Programms (sollte mit ".js" enden)
% #2 = Kurztitel der Seite
\ifttm
\begin{MSContent}{#2}{#2}{puzzle}
\MRegisterFile{#1}
\begin{html}
<script src="\MMaterial/\MLastFile" type="text/javascript"></script>
\end{html}
\end{MSContent}
\fi
}

\newcommand{\MIncludePSE}{
% Bindet bei Chemie-Modulen das PSE ein
% (wird nur im HTML, aber nicht im PDF erstellt)
\ifttm
\special{html:<!-- includepse //-->}
\begin{MSContent}{Periodensystem der Elemente}{PSE}{table}
\MRegisterFile{../files/pse.js}
\MRegisterFile{../files/radio.png}
\begin{html}
<script src="\MMaterial/../files/pse.js" type="text/javascript"></script>
<p id="divid"><br /><br />
<script language="javascript" type="text/javascript">
    startpse("divid","\MMaterial/../files"); 
</script>
</p>
<br />
<br />
<br />
<p>Die Farben der Elementsymbole geben an: <font style="color:Red">gasf&ouml;rmig </font> <font style="color:Blue">fl&uuml;ssig </font> fest</p>
<p>Die Elemente der Gruppe 1 A, 2 A, 3 A usw. geh&ouml;ren zu den Hauptgruppenelementen.</p>
<p>Die Elemente der Gruppe 1 B, 2 B, 3 B usw. geh&ouml;ren zu den Nebengruppenelementen.</p>
<p>() kennzeichnet die Masse des stabilsten Isotops</p>
\end{html}
\end{MSContent}
\fi
}

\newcommand{\MAppletArchive}[4]{
% Bindet ein Java-Applet ein, die Parameter sind:
% (wird nur im HTML, aber nicht im PDF erstellt)
% #1 Dateiname der Klasse mit Appletaufruf (muss mit ".class" enden)
% #2 Dateiname des Archivs (muss mit ".jar" enden)
% #3 = Breite in Pixeln
% #4 = Hoehe in Pixeln
\ifttm
\MRegisterFile{#2}
\begin{html}
<applet code="#1" archive="\end{html}\MMaterial/\MLastFile\begin{html}" codebase="." width="#3" height="#4" alt="[Java-Archiv kann nicht gestartet werden]"></applet>
\end{html}
\fi
}

% Bindet in der Haupttexdatei ein MINT-Modul ein. Parameter 1 ist das Verzeichnis (relativ zur Haupttexdatei), Parameter 2 ist der Dateinahme ohne Pfad.
\newcommand{\IncludeModule}[2]{
\renewcommand{\MDPrefix}{#1}
\input{#1/#2}
\ifnum\value{MSSEnd}>0{\MSubsectionEndMacros}\addtocounter{MSSEnd}{-1}\fi
}

% Der ttm-Konverter setzt keine Makros im \input um, also muss hier getrickst werden:
% Das MDPrefix muss in den einzelnen Modulen manuell eingesetzt werden
\newcommand{\MInputFile}[1]{
\ifttm
\input{#1}
\else
\input{#1}
\fi
}


\newcommand{\MCases}[1]{\left\lbrace{\begin{array}{rl} #1 \end{array}}\right.}

\ifttm
\newenvironment{MCaseEnv}{\left\lbrace\begin{array}{rl}}{\end{array}\right.}
\else
\newenvironment{MCaseEnv}{\left\lbrace\begin{array}{rl}}{\end{array}\right.}
\fi

\def\MSkip{\ifttm\MCR\fi}

\ifttm
\def\MCR{\special{html:<br />}}
\else
\def\MCR{\ \\}
\fi


% Pragmas - Sind Schluesselwoerter, die dem Preprocessing sowie dem Konverter uebergeben werden und bestimmte
%           Aktionen ausloesen. Im Output (PDF und HTML) tauchen sie nicht auf.
\newcommand{\MPragma}[1]{%
\ifttm%
\special{html:<!-- mpragma;-;}#1\special{html:;; -->}%
\else%
% MPragmas werden vom Preprozessor direkt im LaTeX gefunden
\fi%
}

% Ersatz der Befehle textsubscript und textsuperscript, die ttm nicht kennt
\ifttm%
\newcommand{\MTextsubscript}[1]{\special{html:<sub>}#1\special{html:</sub>}}%
\newcommand{\MTextsuperscript}[1]{\special{html:<sup>}#1\special{html:</sup>}}%
\else%
\newcommand{\MTextsubscript}[1]{\textsubscript{#1}}%
\newcommand{\MTextsuperscript}[1]{\textsuperscript{#1}}%
\fi

%------------------ Einbindung von dia-Diagrammen ----------------------------------------------
% Beim preprocessing wird aus jeder dia-Datei eine tex-Datei und eine pdf-Datei erzeugt,
% diese werden hier jeweils im PDF und HTML eingebunden
% Parameter: Dateiname der mit dia erstellten Datei (OHNE die Endung .dia)
\ifttm%
\newcommand{\MDia}[1]{%
\MGraphicsSolo{#1minthtml.png}{}%
}
\else%
\newcommand{\MDia}[1]{%
\MGraphicsSolo{#1mintpdf.png}{scale=0.1667}%
}
\fi%

% subsup funktioniert im Ausdruck $D={\R}^+_0$, also \R geklammert und sup zuerst
% \ifttm
% \def\MSubsup#1#2#3{\special{html:<msubsup>} #1 #2 #3\special{html:</msubsup>}}
% \else
% \def\MSubsup#1#2#3{{#1}^{#3}_{#2}}
% \fi

%\input{local.tex}

% \ifttm
% \else
% \newwrite\mintlog
% \immediate\openout\mintlog=mintlog.txt
% \fi

% ----------------------- tikz autogenerator -------------------------------------------------------------------

\newcommand{\Mtikzexternalize}{\tikzexternalize}% wird bei Konvertierung ueber mconvert ggf. ausgehebelt!

\ifttm
\else
\tikzset%
{
  % Defines a custom style which generates pdf and converts to (low and hi-res quality) png and svg, then deletes the pdf
  % Important: DO NOT directly convert from pdf to hires-png or from svg to png with GraphViz convert as it has some problems and memory leaks
  png export/.style=%
  {
    external/system call/.add={}{; 
      pdf2svg "\image.pdf" "\image.svg" ; 
      convert -density 112.5 -transparent white "\image.pdf" "\image.png"; 
      inkscape --export-png="\image.4x.png" --export-dpi=450 --export-background-opacity=0 --without-gui "\image.svg"; 
      rm "\image.pdf"; rm "\image.log"; rm "\image.dpth"; rm "\image.idx"
    },
    external/force remake,
  }
}
\tikzset{png export}
\tikzsetexternalprefix{}
% PNGs bei externer Erzeugung in "richtiger" Groesse einbinden
\pgfkeys{/pgf/images/include external/.code={\includegraphics[scale=0.64]{#1}}}
\fi

% Spezielle Umgebung fuer Autogenerierung, Bildernamen sind nur innerhalb eines Moduls (einer MSection) eindeutig)

\newcommand{\MTIKZautofilename}{tikzautofile}

\ifttm
% HTML-Version: Vom Autogenerator erzeugte png-Datei einbinden, tikz selbst nicht ausfuehren (sprich: #1 schlucken)
\newcommand{\MTikzAuto}[1]{%
\addtocounter{MTIKZAutofilenumber}{1}
\renewcommand{\MTIKZautofilename}{mtikzauto_\arabic{MTIKZAutofilenumber}}
\MUGraphicsSolo{\MSectionID\MTIKZautofilename.4x.png}{scale=1}{\special{html:[[!-- svgstyle;}\MSectionID\MTIKZautofilename\special{html: //--]]}} % Styleinfos werden aus original-png, nicht 4x-png geholt!
%\MRegisterFile{\MSectionID\MTIKZautofilename.png} % not used right now
%\MRegisterFile{\MSectionID\MTIKZautofilename.svg}
}
\else%
% PDF-Version: Falls Autogenerator aktiv wird Datei automatisch benannt und exportiert
\newcommand{\MTikzAuto}[1]{%
\addtocounter{MTIKZAutofilenumber}{1}%
\renewcommand{\MTIKZautofilename}{mtikzauto_\arabic{MTIKZAutofilenumber}}
\tikzsetnextfilename{\MTIKZautofilename}%
#1%
}
\fi

% In einer reinen LaTeX-Uebersetzung kapselt der Preambelinclude-Befehl nur input,
% in einer konvertergesteuerten PDF/HTML-Uebersetzung wird er dagegen entfernt und
% die Preambeln an mintmod angehaengt, die Ersetzung wird von mconvert.pl vorgenommen.

\newcommand{\MPreambleInclude}[1]{\input{#1}}

% Globale Watermarksettings (werden auch nochmal zu Beginn jedes subsection gesetzt,
% muessen hier aber auch global ausgefuehrt wegen Einfuehrungsseiten und Inhaltsverzeichnis

\MWatermarkSettings
% ---------------------------------- Parametrisierte Aufgaben ----------------------------------------

\ifttm
\newenvironment{MPExercise}{%
\begin{MExercise}%
}{%
\special{html:<button name="Name_MPEX}\arabic{MExerciseCounter}\special{html:" id="MPEX}\arabic{MExerciseCounter}%
\special{html:" type="button" onclick="reroll('}\arabic{MExerciseCounter}\special{html:');">Neue Aufgabe erzeugen</button>}%
\end{MExercise}%
}
\else
\newenvironment{MPExercise}{%
\begin{MExercise}%
}{%
\end{MExercise}%
}
\fi

% Parameter: Name, Min, Max, PDF-Standard. Name in Deklaration OHNE backslash, im Code MIT Backslash
\ifttm
\newcommand{\MGlobalInteger}[4]{\special{html:%
<!-- onloadstart //-->%
MVAR.push(createGlobalInteger("}#1\special{html:",}#2\special{html:,}#3\special{html:,}#4\special{html:)); %
<!-- onloadstop //-->%
<!-- viewmodelstart //-->%
ob}#1\special{html:: ko.observable(rerollMVar("}#1\special{html:")),%
<!-- viewmodelstop //-->%
}%
}%
\else%
\newcommand{\MGlobalInteger}[4]{\newcounter{mvc_#1}\setcounter{mvc_#1}{#4}}
\fi

% Parameter: Name, Min, Max, PDF-Standard. Name in Deklaration OHNE backslash, im Code MIT Backslash, Wert ist Wurzel von value
\ifttm
\newcommand{\MGlobalSqrt}[4]{\special{html:%
<!-- onloadstart //-->%
MVAR.push(createGlobalSqrt("}#1\special{html:",}#2\special{html:,}#3\special{html:,}#4\special{html:)); %
<!-- onloadstop //-->%
<!-- viewmodelstart //-->%
ob}#1\special{html:: ko.observable(rerollMVar("}#1\special{html:")),%
<!-- viewmodelstop //-->%
}%
}%
\else%
\newcommand{\MGlobalSqrt}[4]{\newcounter{mvc_#1}\setcounter{mvc_#1}{#4}}% Funktioniert nicht als Wurzel !!!
\fi

% Parameter: Name, Min, Max, PDF-Standard zaehler, PDF-Standard nenner. Name in Deklaration OHNE backslash, im Code MIT Backslash
\ifttm
\newcommand{\MGlobalFraction}[5]{\special{html:%
<!-- onloadstart //-->%
MVAR.push(createGlobalFraction("}#1\special{html:",}#2\special{html:,}#3\special{html:,}#4\special{html:,}#5\special{html:)); %
<!-- onloadstop //-->%
<!-- viewmodelstart //-->%
ob}#1\special{html:: ko.observable(rerollMVar("}#1\special{html:")),%
<!-- viewmodelstop //-->%
}%
}%
\else%
\newcommand{\MGlobalFraction}[5]{\newcounter{mvc_#1}\setcounter{mvc_#1}{#4}} % Funktioniert nicht als Bruch !!!
\fi

% MVar darf im HTML nur in MEvalMathDisplay-Umgebungen genutzt werden oder in Strings die an den Parser uebergeben werden
\ifttm%
\newcommand{\MVar}[1]{\special{html:[var_}#1\special{html:]}}%
\else%
\newcommand{\MVar}[1]{\arabic{mvc_#1}}%
\fi

\ifttm%
\newcommand{\MRerollButton}[2]{\special{html:<button type="button" onclick="rerollMVar('}#1\special{html:');">}#2\special{html:</button>}}%
\else%
\newcommand{\MRerollButton}[2]{\relax}% Keine sinnvolle Entsprechung im PDF
\fi

% MEvalMathDisplay fuer HTML wird in mconvert.pl im preprocessing realisiert
% PDF: eine equation*-Umgebung (ueber amsmath)
% HTML: Eine Mathjax-Tex-Umgebung, deren Auswertung mit knockout-obervablen gekoppelt ist
% PDF-Version hier nur fuer pdflatex-only-Uebersetzung gegeben

\ifttm\else\newenvironment{MEvalMathDisplay}{\begin{equation*}}{\end{equation*}}\fi

% ---------------------------------- Spezialbefehle fuer AD ------------------------------------------

%Abk�rzung f�r \longrightarrow:
\newcommand{\lto}{\ensuremath{\longrightarrow}}

%Makro f�r Funktionen:
\newcommand{\exfunction}[5]
{\begin{array}{rrcl}
 #1 \colon  & #2 &\lto & #3 \\[.05cm]  
  & #4 &\longmapsto  & #5 
\end{array}}

\newcommand{\function}[5]{%
#1:\;\left\lbrace{\begin{array}{rcl}
 #2 &\lto & #3 \\
 #4 &\longmapsto  & #5 \end{array}}\right.}


%Die Identit�t:
\DeclareMathOperator{\Id}{Id}

%Die Signumfunktion:
\DeclareMathOperator{\sgn}{sgn}

%Zwei Betonungskommandos (k�nnen angepasst werden):
\newcommand{\highlight}[1]{#1}
\newcommand{\modstextbf}[1]{#1}
\newcommand{\modsemph}[1]{#1}


% ---------------------------------- Spezialbefehle fuer JL ------------------------------------------


\def\jccolorfkt{green!50!black} %Farbe des Funktionsgraphen
\def\jccolorfktarea{green!25!white} %Farbe der Fl"ache unter dem Graphen
\def\jccolorfktareahell{green!12!white} %helle Einf"arbung der Fl"ache unter dem Graphen
\def\jccolorfktwert{green!50!black} %Farbe einzelner Punkte des Graphen

\newcommand{\MPfadBilder}{Bilder}

\ifttm%
\newcommand{\jMD}{\,\MD}%
\else%
\newcommand{\jMD}{\;\MD}%
\fi%

\def\jHTMLHinweisBedienung{\MInputHint{%
Mit Hilfe der Symbole am oberen Rand des Fensters
k"onnen Sie durch die einzelnen Abschnitte navigieren.}}

\def\jHTMLHinweisEingabeText{\MInputHint{%
Geben Sie jeweils ein Wort oder Zeichen als Antwort ein.}}

\def\jHTMLHinweisEingabeTerm{\MInputHint{%
Klammern Sie Ihre Terme, um eine eindeutige Eingabe zu erhalten. 
Beispiel: Der Term $\frac{3x+1}{x-2}$ soll in der Form
\texttt{(3*x+1)/((x+2)^2}$ eingegeben werden (wobei auch Leerzeichen 
eingegeben werden k"onnen, damit eine Formel besser lesbar ist).}}

\def\jHTMLHinweisEingabeIntervalle{\MInputHint{%
Intervalle werden links mit einer "offnenden Klammer und rechts mit einer 
schlie"senden Klammer angegeben. Eine runde Klammer wird verwendet, wenn der 
Rand nicht dazu geh"ort, eine eckige, wenn er dazu geh"ort. 
Als Trennzeichen wird ein Komma oder ein Semikolon akzeptiert.
Beispiele: $(a, b)$ offenes Intervall,
$[a; b)$ links abgeschlossenes, rechts offenes Intervall von $a$ bis $b$. 
Die Eingabe $]a;b[$ f"ur ein offenes Intervall wird nicht akzeptiert.
F"ur $\infty$ kann \texttt{infty} oder \texttt{unendlich} geschrieben werden.}}

\def\jHTMLHinweisEingabeFunktionen{\MInputHint{%
Schreiben Sie Malpunkte (geschrieben als \texttt{*}) aus und setzen Sie Klammern um Argumente f�r Funktionen.
Beispiele: Polynom: \texttt{3*x + 0.1}, Sinusfunktion: \texttt{sin(x)}, 
Verkettung von cos und Wurzel: \texttt{cos(sqrt(3*x))}.}}

\def\jHTMLHinweisEingabeFunktionenSinCos{\MInputHint{%
Die Sinusfunktion $\sin x$ wird in der Form \texttt{sin(x)} angegeben, %
$\cos\left(\sqrt{3 x}\right)$ durch \texttt{cos(sqrt(3*x))}.}}

\def\jHTMLHinweisEingabeFunktionenExp{\MInputHint{%
Die Exponentialfunktion $\MEU^{3x^4 + 5}$ wird als
\texttt{exp(3 * x^4 + 5)} angegeben, %
$\ln\left(\sqrt{x} + 3.2\right)$ durch \texttt{ln(sqrt(x) + 3.2)}.}}

% ---------------------------------- Spezialbefehle fuer Fachbereich Physik --------------------------

\newcommand{\E}{{e}}
\newcommand{\ME}[1]{\cdot 10^{#1}}
\newcommand{\MU}[1]{\;\mathrm{#1}}
\newcommand{\MPG}[3]{%
  \ifnum#2=0%
    #1\ \mathrm{#3}%
  \else%
    #1\cdot 10^{#2}\ \mathrm{#3}%
  \fi}%
%

\newcommand{\MMul}{\MExponentensymbXYZl} % Nur eine Abkuerzung


% ---------------------------------- Stichwortfunktionialitaet ---------------------------------------

% mpreindexentry wird durch Auswahlroutine in conv.pl durch mindexentry substitutiert
\ifttm%
\def\MIndex#1{\index{#1}\special{html:<!-- mpreindexentry;;}#1\special{html:;;}\arabic{MSubjectArea}\special{html:;;}%
\arabic{chapter}\special{html:;;}\arabic{section}\special{html:;;}\arabic{subsection}\special{html:;;}\arabic{MEntryCounter}\special{html:; //-->}%
\setcounter{MLastIndex}{\value{MEntryCounter}}%
\addtocounter{MEntryCounter}{1}%
}%
% Copyrightliste wird als tex-Datei im preprocessing von conv.pl erzeugt und unter converter/tex/entrycollection.tex abgelegt
% Der input-Befehl funktioniert nur, wenn die aufrufende tex-Datei auf der obersten Ebene liegt (d.h. selbst kein input/include ist, insbesondere keine Moduldatei)
\def\MEntryList{} % \input funktioniert nicht, weil ttm (und damit das \input) ausgefuehrt wird, bevor Datei da ist
\else%
\def\MIndex#1{\index{#1}}
\def\MEntryList{\MAbort{Stichwortliste nur im HTML realisierbar}}%
\fi%

\def\MEntry#1#2{\textbf{#1}\MIndex{#2}} % Idee: MLastType auf neuen Entry-Typ und dann ein MLabel vergeben mit autogen-Nummer

% ---------------------------------- Befehle fuer Tests ----------------------------------------------

% MEquationItem stellt eine Eingabezeile der Form Vorgabe = Antwortfeld her, der zweite Parameter kann z.B. MSimplifyQuestion-Befehl sein
\ifttm
\newcommand{\MEquationItem}[2]{{#1}$\,=\,${#2}}%
\else%
\newcommand{\MEquationItem}[2]{{#1}$\;\;=\,${#2}}%
\fi

\ifttm
\newcommand{\MInputHint}[1]{%
\ifnum%
\if\value{MTestSite}>0%
\else%
{\color{blue}#1}%
\fi%
\fi%
}
\else
\newcommand{\MInputHint}[1]{\relax}
\fi

\ifttm
\newcommand{\MInTestHeader}{%
Dies ist ein einreichbarer Test:
\begin{itemize}
\item{Im Gegensatz zu den offenen Aufgaben werden beim Eingeben keine Hinweise zur Formulierung der mathematischen Ausdr�cke gegeben.}
\item{Der Test kann jederzeit neu gestartet oder verlassen werden.}
\item{Der Test kann durch die Buttons am Ende der Seite beendet und abgeschickt, oder zur�ckgesetzt werden.}
\item{Der Test kann mehrfach probiert werden. F�r die Statistik z�hlt die zuletzt abgeschickte Version.}
\end{itemize}
}
\else
\newcommand{\MInTestHeader}{%
\relax
}
\fi

\ifttm
\newcommand{\MInTestFooter}{%
\special{html:<button name="Name_TESTFINISH" id="TESTFINISH" type="button" onclick="finish_button('}\MTestName\special{html:');">Test auswerten</button>}%
\begin{html}
&nbsp;&nbsp;&nbsp;&nbsp;&nbsp;&nbsp;&nbsp;&nbsp;
<button name="Name_TESTRESET" id="TESTRESET" type="button" onclick="reset_button();">Test zur�cksetzen</button>
<br />
<br />
<div class="xreply">
<p name="Name_TESTEVAL" id="TESTEVAL">
Hier erscheint die Testauswertung!
<br />
</p>
</div>
\end{html}
}
\else
\newcommand{\MInTestFooter}{%
\relax
}
\fi


% ---------------------------------- Notationsmakros -------------------------------------------------------------

% Notationsmakros die nicht von der Kursvariante abhaengig sind

\newcommand{\MZahltrennzeichen}[1]{\renewcommand{\MZXYZhltrennzeichen}{#1}}

\ifttm
\newcommand{\MZahl}[3][\MZXYZhltrennzeichen]{\edef\MZXYZtemp{\noexpand\special{html:<mn>#2#1#3</mn>}}\MZXYZtemp}
\else
\newcommand{\MZahl}[3][\MZXYZhltrennzeichen]{{}#2{#1}#3}
\fi

\newcommand{\MEinheitenabstand}[1]{\renewcommand{\MEinheitenabstXYZnd}{#1}}
\ifttm
\newcommand{\MEinheit}[2][\MEinheitenabstXYZnd]{{}#1\edef\MEINHtemp{\noexpand\special{html:<mi mathvariant="normal">#2</mi>}}\MEINHtemp} 
\else
\newcommand{\MEinheit}[2][\MEinheitenabstXYZnd]{{}#1 \mathrm{#2}} 
\fi

\newcommand{\MExponentensymbol}[1]{\renewcommand{\MExponentensymbXYZl}{#1}}
\newcommand{\MExponent}[2][\MExponentensymbXYZl]{{}#1{} 10^{#2}} 

%Punkte in 2 und 3 Dimensionen
\newcommand{\MPointTwo}[3][]{#1(#2\MCoordPointSep #3{}#1)}
\newcommand{\MPointThree}[4][]{#1(#2\MCoordPointSep #3\MCoordPointSep #4{}#1)}
\newcommand{\MPointTwoAS}[2]{\left(#1\MCoordPointSep #2\right)}
\newcommand{\MPointThreeAS}[3]{\left(#1\MCoordPointSep #2\MCoordPointSep #3\right)}

% Masseinheit, Standardabstand: \,
\newcommand{\MEinheitenabstXYZnd}{\MThinspace} 

% Horizontaler Leerraum zwischen herausgestellter Formel und Interpunktion
\ifttm
\newcommand{\MDFPSpace}{\,}
\newcommand{\MDFPaSpace}{\,\,}
\newcommand{\MBlank}{\ }
\else
\newcommand{\MDFPSpace}{\;}
\newcommand{\MDFPaSpace}{\;\;}
\newcommand{\MBlank}{\ }
\fi

% Satzende in herausgestellter Formel mit horizontalem Leerraum
\newcommand{\MDFPeriod}{\MDFPSpace .}

% Separation von Aufzaehlung und Bedingung in Menge
\newcommand{\MCondSetSep}{\,:\,} %oder '\mid'

% Konverter kennt mathopen nicht
\ifttm
\def\mathopen#1{}
\fi

% -----------------------------------START Rouletteaufgaben ------------------------------------------------------------

\ifttm
% #1 = Dateiname, #2 = eindeutige ID fuer das Roulette im Kurs
\newcommand{\MDirectRouletteExercises}[2]{
\begin{MExercise}
\texttt{Im HTML erscheinen hier Aufgaben aus einer Aufgabenliste...}
\end{MExercise}
}
\else
\newcommand{\MDirectRouletteExercises}[2]{\relax} % wird durch mconvert.pl gefunden und ersetzt
\fi


% ---------------------------------- START Makros, die von der Kursvariante abhaengen ----------------------------------

\ifvariantunotation
  % unotation = An Universitaeten uebliche Notation
  \def\MVariant{unotation}

  % Trennzeichen fuer Dezimalzahlen
  \newcommand{\MZXYZhltrennzeichen}{.}

  % Exponent zur Basis 10 in der Exponentialschreibweise, 
  % Standardmalzeichen: \times
  \newcommand{\MExponentensymbXYZl}{\times} 

  % Begrenzungszeichen fuer offene Intervalle
  \newcommand{\MoIl}[1][]{\mbox{}#1(\mathopen{}} % bzw. ']'
  \newcommand{\MoIr}[1][]{#1)\mbox{}} % bzw. '['

  % Zahlen-Separation im IntervaLL
  \newcommand{\MIntvlSep}{,} %oder ';'

  % Separation von Elementen in Mengen
  \newcommand{\MElSetSep}{,} %oder ';'

  % Separation von Koordinaten in Punkten
  \newcommand{\MCoordPointSep}{,} %oder ';' oder '|', '\MThinspace|\MThinspace'

\else
  % An dieser Stelle wird angenommen, dass std-Variante aktiv ist
  % std = beschlossene Notation im TU9-Onlinekurs 
  \def\MVariant{std}

  % Trennzeichen fuer Dezimalzahlen
  \newcommand{\MZXYZhltrennzeichen}{,}

  % Exponent zur Basis 10 in der Exponentialschreibweise, 
  % Standardmalzeichen: \times
  \newcommand{\MExponentensymbXYZl}{\times} 

  % Begrenzungszeichen fuer offene Intervalle
  \newcommand{\MoIl}[1][]{\mbox{}#1]\mathopen{}} % bzw. '('
  \newcommand{\MoIr}[1][]{#1[\mbox{}} % bzw. ')'

  % Zahlen-Separation im IntervaLL
  \newcommand{\MIntvlSep}{;} %oder ','
  
  % Separation von Elementen in Mengen
  \newcommand{\MElSetSep}{;} %oder ','

  % Separation von Koordinaten in Punkten
  \newcommand{\MCoordPointSep}{;} %oder '|', '\MThinspace|\MThinspace'

\fi



% ---------------------------------- ENDE Makros, die von der Kursvariante abhaengen ----------------------------------


% diese Kommandos setzen Mathemodus vorraus
\newcommand{\MGeoAbstand}[2]{[\overline{{#1}{#2}}]}
\newcommand{\MGeoGerade}[2]{{#1}{#2}}
\newcommand{\MGeoStrecke}[2]{\overline{{#1}{#2}}}
\newcommand{\MGeoDreieck}[3]{{#1}{#2}{#3}}

%
\ifttm
\newcommand{\MOhm}{\special{html:<mn>&#x3A9;</mn>}}
\else
\newcommand{\MOhm}{\Omega} %\varOmega
\fi


\def\PERCTAG{\MAbort{PERCTAG ist zur internen verwendung in mconvert.pl reserviert, dieses Makro darf sonst nicht benutzt werden.}}

% Im Gegensatz zu einfachen html-Umgebungen werden MDirectHTML-Umgebungen von mconvert.pl am ganzen ttm-Prozess vorbeigeschleust und aus dem PDF komplett ausgeschnitten
\ifttm%
\newenvironment{MDirectHTML}{\begin{html}}{\end{html}}%
\else%
\newenvironment{MDirectHTML}{\begin{html}}{\end{html}}%
\fi

% Im Gegensatz zu einfachen Mathe-Umgebungen werden MDirectMath-Umgebungen von mconvert.pl am ganzen ttm-Prozess vorbeigeschleust, ueber MathJax realisiert, und im PDF als $$ ... $$ gesetzt
\ifttm%
\newenvironment{MDirectMath}{\begin{html}}{\end{html}}%
\else%
\newenvironment{MDirectMath}{\begin{equation*}}{\end{equation*}}% Vorsicht, auch \[ und \] werden in amsmath durch equation* redefiniert
\fi

% ---------------------------------- Location Management ---------------------------------------------

% #1 = buttonname (muss in files/images liegen und Format 48x48 haben), #2 = Vollstaendiger Einrichtungsname, #3 = Kuerzel der Einrichtung,  #4 = Name der include-texdatei
\ifttm
\newcommand{\MLocationSite}[3]{\special{html:<!-- mlocation;;}#1\special{html:;;}#2\special{html:;;}#3\special{html:;; //-->}}
\else
\newcommand{\MLocationSite}[3]{\relax}
\fi

% ---------------------------------- Copyright Management --------------------------------------------

\newcommand{\MCCLicense}{%
{\color{green}\textbf{CC BY-SA 3.0}}
}

\newcommand{\MCopyrightLabel}[1]{ (\MSRef{L_COPYRIGHTCOLLECTION}{Lizenz})\MLabel{#1}}

% Copyrightliste wird als tex-Datei im preprocessing erzeugt und unter converter/tex/copyrightcollection.tex abgelegt
% Der input-Befehl funktioniert nur, wenn die aufrufende tex-Datei auf der obersten Ebene liegt (d.h. selbst kein input/include ist, insbesondere keine Moduldatei)
\newcommand{\MCopyrightCollection}{\input{copyrightcollection.tex}}

% MCopyrightNotice fuegt eine Copyrightnotiz ein, der parser ersetzt diese durch CopyrightNoticePOST im preparsing, diese Definition wird nur fuer reine pdflatex-Uebersetzungen gebraucht
% Parameter: #1: Kurze Lizenzbeschreibung (typischerweise \MCCLicense)
%            #2: Link zum Original (http://...) oder NONE falls das Bild selbst ein Original ist, oder TIKZ falls das Bild aus einer tikz-Umgebung stammt
%            #3: Link zum Autor (http://...) oder MINT falls Original im MINT-Kolleg erstellt oder NONE falls Autor unbekannt
%            #4: Bemerkung (z.B. dass Datei mit Maple exportiert wurde)
%            #5: Labelstring fuer existierendes Label auf das copyrighted Objekt, mit MCopyrightLabel erzeugt
%            Keines der Felder darf leer sein!
\newcommand{\MCopyrightNotice}[5]{\MCopyrightNoticePOST{#1}{#2}{#3}{#4}{#5}}

\ifttm%
\newcommand{\MCopyrightNoticePOST}[5]{\relax}%
\else%
\newcommand{\MCopyrightNoticePOST}[5]{\relax}%
\fi%

% ---------------------------------- Meldungen fuer den Benutzer des Konverters ----------------------
\MPragma{mintmodversion;P0.1.0}
\MPragma{usercomment;This is file mintmod.tex version P0.1.0}


% ----------------------------------- Spezialelemente fuer Konfigurationsseite, werden nicht von mintscripts.js verwaltet --

% #1 = DOM-id der Box
\ifttm\newcommand{\MConfigbox}[1]{\special{html:<input cfieldtype="2" type="checkbox" name="Name_}#1\special{html:" id="}#1\special{html:" onchange="confHandlerChange('}#1\special{html:');"/>}}\fi % darf im PDF nicht aufgerufen werden!


\MPragma{MathSkip}

\Mtikzexternalize

\MSetSubject{\MINTMathematics}

\begin{document}

\MSection{Geometrie}
\MLabel{VBKM05}
\MSetSectionID{VBKM05} % hier identisch mit dem alten tikz-Dateien-Prefix 

\begin{MSectionStart}
\MDeclareSiteUXID{VBKM05_START}

\MModstartBox

In den ersten Abschnitten wird in die elementare Geometrie eingef"uhrt. 
Hierzu wird auch Bezug auf die bisherigen Module genommen. Im Zentrum stehen 
die Eigenschaften von Dreiecken, bevor Fl"achen von Vielecken und Volumina 
von einfachen K"orpern berechnet werden.
Weitergehende Fragestellungen k"onnen mit den Winkelfunktionen bearbeitet
werden. Diese bieten einen Ausblick auf die kommenden Module zur Analysis
und zur Analytischen Geometrie.
\end{MSectionStart}


%jgl: section 1:
\MSubsection{Grundbegriffe der ebenen Geometrie}
\MLabel{M05_Grundbegriffe}

\begin{MIntro}
\MDeclareSiteUXID{VBKM05_Grundbegriffe_Intro}

Ein klarer Sternenhimmel vermittelt einen anschaulichen Eindruck von den 
elementaren Objekten, den Punkten, in der Geometrie. Seit Menschengedenken
wurden die Lichtpunkte am Himmel in Gedanken durch Linien miteinander 
verbunden, die als Umrisse unterschiedlichster Figuren interpretiert wurden.
Vom praktischen Nutzen der Himmelsgeometrie zeugt praktisch jedes Geb"aude
mit seinen Ecken, Kanten, Fensterfl"achen, \ldots.

Die Erfindung von best"andigen Zeichenstiften, Tafeln aus Wachs, Papyrus oder 
Papier erm"oglicht es, umgekehrt das Gesehene oder eigene Gedanken {\glqq}auf
Papier{\grqq} zu festzuhalten und anderen zu zeigen. Verbunden mit dem Willen,
zum Beispiel das Gezeichnete als Bauwerk zu realisieren, war die Idee eines
Plans geboren. Der Plan ist eine Zeichnung einer idealisierten Vorstellung, 
beispielsweise wie ein Sportstadion von oben betrachtet aussehen soll.

F"ur den Bau eines Sportstadions werden markante Punkte im Gel"ande abgesteckt.
Den aktuellen Stand der Arbeiten zeigt die folgende Ansicht, in die die Umrisse 
aus einem Plan projiziert sind.

\begin{center}
%Stadion:
\MTikzAuto{%
\begin{tikzpicture}[line width=1pt]
%Spielfeld:
\coordinate (AS) at (-2.25,0);
\coordinate (BS) at (2.25,0);
\coordinate (FeldA0) at ($(AS) + (0,-1.5)$);
\coordinate (FeldA1) at ($(AS) + (0,1.5)$);
\coordinate (FeldB0) at ($(BS) + (0,-1.5)$);
\coordinate (FeldB1) at ($(BS) + (0,1.5)$);
%Stadionbereich:
\coordinate (GA0) at ($(AS) + (0,-2.5)$);
\coordinate (GA1) at ($(AS) + (0,2.5)$);
\coordinate (GB0) at ($(BS) + (0,-2.5)$);
\coordinate (GB1) at ($(BS) + (0,2.5)$);
%
\coordinate (HA0) at ($(AS) + (0,-3)$);
\coordinate (HA1) at ($(AS) + (0,3)$);
\coordinate (HB0) at ($(BS) + (0,-3)$);
\coordinate (HB1) at ($(BS) + (0,3)$);
%
\coordinate (GA01) at ($(AS) + (0,-2.25)$);
\coordinate (GA02) at ($(AS) + (0,-2)$);
\coordinate (GA03) at ($(AS) + (0,-1.75)$);
\coordinate (GB01) at ($(BS) + (0,-2.25)$);
\coordinate (GB02) at ($(BS) + (0,-2)$);
\coordinate (GB03) at ($(BS) + (0,-1.75)$);
\coordinate (GA11) at ($(AS) + (0,2.25)$);
\coordinate (GA12) at ($(AS) + (0,2)$);
\coordinate (GA13) at ($(AS) + (0,1.75)$);
\coordinate (GB11) at ($(BS) + (0,2.25)$);
\coordinate (GB12) at ($(BS) + (0,2)$);
\coordinate (GB13) at ($(BS) + (0,1.75)$);
%Aussenbereich:
\coordinate (XA) at ($(GA1) + (-2.5, 0) + (-2.5, 0)$);
\coordinate (XC) at ($(GA1) + (-2.5, 0)$);
\coordinate (XB) at ($(AS) + (-2.5,0)$);
\coordinate (XAC1) at ($(XA) + (0.5, 0)$);
\coordinate (XAC2) at ($(XA) + (1, 0)$);
\coordinate (XAC3) at ($(XA) + (1.5, 0)$);
\coordinate (XAC4) at ($(XA) + (2, 0)$);
\coordinate (XBC1) at ($(XB) + (0, 0.5)$);
\coordinate (XBC2) at ($(XB) + (0, 1)$);
\coordinate (XBC3) at ($(XB) + (0, 1.5)$);
\coordinate (XBC4) at ($(XB) + (0, 2)$);
%
\begin{scope}[style=dashed,color=black!50!white]
%Sportfeld:
\draw (FeldA0) -- (FeldB0) -- (FeldB1) -- (FeldA1) -- cycle;
%Stadionbereich:
\draw (GA03) -- (GB03);
\draw (GA02) -- (GB02);
\draw (GA01) -- (GB01);
\draw (GA0) -- (GB0);
\draw (GA13) -- (GB13);
\draw (GA12) -- (GB12);
\draw (GA11) -- (GB11);
\draw (GA1) -- (GB1);
%Aussenbereich:
\draw (XA) -- (XB) -- (XC) -- cycle;
\draw (XAC1) -- (XBC1);
\draw (XAC2) -- (XBC2);
\draw (XAC3) -- (XBC3);
\draw (XAC4) -- (XBC4);
%Links:
\draw (GA13) arc [start angle=90, end angle=270, radius=1.75cm];
\draw (GA12) arc [start angle=90, end angle=270, radius=2cm];
\draw (GA11) arc [start angle=90, end angle=270, radius=2.25cm];
\draw (GA1) arc [start angle=90, end angle=270, radius=2.5cm];
%\draw (-3,3) arc [start angle=90, end angle=270, radius=3cm];
%Rechts
\draw (GB03) arc [start angle=-90, end angle=90, radius=1.75cm];
\draw (GB02) arc [start angle=-90, end angle=90, radius=2cm];
\draw (GB01) arc [start angle=-90, end angle=90, radius=2.25cm];
\draw (GB0) arc [start angle=-90, end angle=90, radius=2.5cm];
\draw (HB0) arc [start angle=-90, end angle=90, radius=3cm];
\end{scope}
%Messpunkte:
%Stadionbereich:
\filldraw (0,0) circle(1.5pt);
\filldraw (AS) circle(1.5pt);
\filldraw (BS) circle(1.5pt);
\filldraw (GA0) circle(1.5pt);
\filldraw (GA1) circle(1.5pt);
\filldraw (GB0) circle(1.5pt);
\filldraw (GB1) circle(1.5pt);
\filldraw (HB0) circle(1.5pt);
\filldraw (HB1) circle(1.5pt);
%Sportfeld:
\filldraw (FeldA0) circle(1.5pt);
\filldraw (FeldA1) circle(1.5pt);
\filldraw (FeldB0) circle(1.5pt);
\filldraw (FeldB1) circle(1.5pt);
%Aussenbereich:
\filldraw (XA) circle(1.5pt);
\filldraw (XB) circle(1.5pt);
\filldraw (XC) circle(1.5pt);
\end{tikzpicture}
}
\par
(Mess-)Punkte und Linien aus dem Plan f"ur einen Stadionbau.
\end{center}

Die Zeichnung kann man als idealisiertes Abbild der Wirklichkeit ansehen.
In diesem Sinne wird an einige geometrische Grundbegriffe erinnert. Sie sind 
der Ausgangspunkt, um anschlie"send kompliziertere Figuren und K"orper zu 
konstruieren.
\end{MIntro}

\begin{MXContent}{Punkte und Geraden}{Punkte und Geraden}{STD}
\MDeclareSiteUXID{VBKM05_PunkteGeraden_Content}
0
Ein Standpunkt oder eine Position in einer Ebene wird in der Geometrie zum
einfachsten Objekt, einem Punkt, idealisiert. 
Ein Punkt wird mit einem gro"sen lateinischen Buchstaben bezeichnet, au"ser seiner Position besitzt der Punkte keine weitere Eigenschaften.

Betrachtet man mehrere Punkte, kann man auf verschiedene Weise Beziehungen 
zwischen den Punkten betrachten -- und man kann neue Objekte, wie zum Beispiel
Strecken und Geraden beschreiben (siehe die folgende Abbildung).
Mathematisch gesprochen, sind es Mengen von Punkten.

%Punkte, Strecken und Geraden:
\begin{center}
\MTikzAuto{%
\begin{tikzpicture}[line width=1pt,scale=1];
%\coordinate (A) at (-2,-1);
%\coordinate (B) at (2,1);
%\coordinate (C) at ($(A) + (0.5,0)$);
%\coordinate (D) at ($(B) + (0.5,0)$);
%\coordinate (GA) at ($(A) + (-1,-0.5)$);
%\coordinate (GB) at ($(B) + (1,0.5)$);
%\coordinate (HC) at ($(A) + (-1,-0.5)$);
%\coordinate (HD) at ($(B) + (1,0.5)$);
%Punkte:
\begin{scope}[xshift=-5cm]
\filldraw (-1,-0.5) circle(1pt);
\filldraw (1,0.5) circle(1pt);
\node at (0,-1.5) {Punkte};
\end{scope}
%Strecke:
\begin{scope}[xshift=-2cm]
\draw[color=blue!70!white] (-1,-0.5) -- (1,0.5);
\filldraw (-1,-0.5) circle(1pt);
\filldraw (1,0.5) circle(1pt);
\node at (0,-1.5) {Strecke};
\end{scope}
%Gerade:
\begin{scope}[xshift=2cm]
\draw[color=blue!50!white] (-2,-1) -- (2,1);
\draw[color=blue!70!white] (-1,-0.5) -- (1,0.5);
\filldraw (-1,-0.5) circle(1pt);
\filldraw (1,0.5) circle(1pt);
\node at (0,-1.5) {Gerade};
\end{scope}
%Parallele Geraden:
\begin{scope}[xshift=7cm]
\draw[color=blue!50!white] (-2,-1) -- (2,1);
\draw[color=blue!70!white] (-1,-0.5) -- (1,0.5);
\filldraw (-1,-0.5) circle(1pt);
\filldraw (1,0.5) circle(1pt);
\draw[color=green!50!white] (-1.5,-1) -- (2.5,1);
%\draw (-0.5,-0.5) circle(1pt);
%\draw (1.5,0.5) circle(1pt);
\node at (0,-1.5) {zwei Geraden};
\end{scope}
\end{tikzpicture}
}
\end{center}

Als Erstes wird eine Strecke und der Abstand zwischen Punkten betrachtet.
Dazu bedarf es noch eines Vergleichsma"sstabs, um den Abstand messen zu
k"onnen. In der Mathematik wird dazu eine Vergleichsstecke ausgew"ahlt, 
deren L"ange als Einheitsl"ange bezeichnet wird. F"ur Anwendungen wird 
man entsprechend der Aufgabe passende Einheiten wie zum Beispiel Meter 
oder Zentimeter als L"angeneinheit festlegen.

\begin{MXInfo}{Strecken und Abst"ande}
Gegeben sind zwei verschiedene Punkte $A$ und $B$. Die \MEntry{Strecke}{Strecke} 
$\MGeoStrecke{A}{B}$ zwischen $A$ und $B$ ist der k"urzeste Weg zwischen den 
Punkten $A$ und $B$ in dieser Reihenfolge. 

%Strecke:
\begin{center}
\MTikzAuto{%
\begin{tikzpicture}[line width=1pt,scale=1];
\begin{scope}[xshift=-3cm]
\draw[color=blue] (-1,-0.5) -- (1,0.5);
\filldraw (-1,-0.5) circle(1pt);
\node[below] at (-1,-0.5) {$A$};
\filldraw (1,0.5) circle(1pt);
\node[below] at (1,0.5) {$B$};
\end{scope}
\end{tikzpicture}
}
\end{center}

Die L"ange der Strecke $\MGeoStrecke{A}{B}$ wird mit $\MGeoAbstand{A}{B}$ 
notiert.  Die \MEntry{Streckenl"ange}{Streckenlänge} ist gleich dem Abstand 
zwischen den Punkten $A$ und $B$.
\end{MXInfo}

Das Bild von einem Lichtstrahl, ausgesandt von einem fernen Stern oder 
von der Sonne, ist eine passende Vorstellung f"ur einen 
\MEntry{Strahl}{Strahl}, der in einem Punkt $A$ beginnt und in Richtung eines
zweiten Punktes $B$ gedanklich immer weiter und darüber hinaus geht. Ein Strahl wird auch
\MEntry{Halbgerade}{Halbgerade} genannt.

%Strahlen und Geraden:
\begin{center}
\MTikzAuto{%
\begin{tikzpicture}[line width=1pt,scale=1];
%\coordinate (A) at (-1,-0.5);
%\coordinate (B) at (1,0.5);
%\coordinate (GA) at ($(A) + (-1,-0.5)$);
%\coordinate (GB) at ($(B) + (1,0.5)$);
%Strahl:
\begin{scope}[xshift=-3cm]
\draw (-1,-0.5) -- (2,1);
\filldraw (-1,-0.5) circle(1pt);
\node[below] at (-1,-0.5) {$A$};
\filldraw (1,0.5) circle(1pt);
\node[below] at (1,0.5) {$B$};
\node at (0,-1.5) {Strahl};
\end{scope}
%Gerade:
\begin{scope}[xshift=3cm]
\draw (-2,-1) -- (2,1);
\filldraw (-1,-0.5) circle(1pt);
\node[below] at (-1,-0.5) {$A$};
\filldraw (1,0.5) circle(1pt);
\node[below] at (1,0.5) {$B$};
\node at (0,-1.5) {Gerade};
\end{scope}
\end{tikzpicture}
}
\end{center}

Wenn man den Weg auf einer Strecke $\MGeoStrecke{A}{B}$ "uber beide Punkte hinaus verlängert, spricht man von einer Geraden.

\begin{MXInfo}{Gerade}
Seien $A$ und $B$ zwei Punkte (das hei"st, es ist $A$ ein Punkt, der vom 
Punkt $B$ verschieden ist). Dann bestimmen $A$ und $B$ genau eine 
\MEntry{Gerade}{Gerade} $\MGeoGerade{A}{B}$, die auch nur mittels einem kleinem lateinischen Buchstaben benannt werden kann.
\end{MXInfo}
Wenn man zu zwei Punkten $A$ und $B$ einen weiteren Punkt $P$ betrachtet,
kann man nach dem Abstand $d$ von $P$ zur Geraden $\MGeoGerade{A}{B}$ fragen.
Dies ist die kleinste Entfernung zwischen $P$ und den Punkten der Geraden
$\MGeoGerade{A}{B}$.

%Abstand zu einer Geraden:
\begin{center}
\MTikzAuto{%
\begin{tikzpicture}[line width=1pt,scale=1];
\coordinate (A) at (0,0);
\coordinate (B) at (2,1);
\coordinate (P) at ($(A) + (-0.5,1)$);
%Abstand von einer Geraden:
\begin{scope}[xshift=3cm]
\draw ($ (A)!-0.4!(B) $) -- ($ (A)!1.6!(B) $);
\draw[style=dotted] (A) -- node[above right]{$d$} (P);
\filldraw (P) circle(1pt);
%Beschriftung:
\node[above left] at (P) {$P$};
\filldraw (A) circle(1pt);
\node[below right] at (B) {Gerade};
\end{scope}
\end{tikzpicture}
}
\end{center}

Wenn drei Punkte $P$, $Q$ und $S$ in der Ebene betrachtet werden, sind damit
Geraden $\MGeoGerade{S}{P}$ und $\MGeoGerade{S}{Q}$ festgelegt.

Sie haben den Punkt $S$ gemeinsam. Liegt auch der Punkt $Q$ auf der Geraden 
$\MGeoGerade{S}{P}$, dann bezeichnen $\MGeoGerade{S}{Q}$ und $\MGeoGerade{S}{P}$
dieselbe Gerade. 
Wenn $Q$ nicht zur Geraden $\MGeoGerade{S}{P}$ geh"ort, sind $\MGeoGerade{S}{Q}$
und $\MGeoGerade{S}{P}$ verschiedene Geraden. Die beiden Geraden haben nur 
den Punkt $S$, den \MEntry{Schnittpunkt}{Schnittpunkt}, gemeinsam.

F"ur irgendwelche Geraden $g$ und $h$ gibt es noch den Fall, dass sie keinen 
Punkt gemeinsam haben. Der kleinste Abstand von Punkten auf $g$ beziehungsweise
$h$ ist der Abstand zwischen den Geraden $g$ und $h$. Somit haben $g$ und $h$
keinen gemeinsamen Punkt, wenn ihr Abstand gr"o"ser als $0$ ist.
Geraden hei"sen \MEntry{parallel}{parallel}, wenn jeder Punkt auf einer der 
Geraden denselben Abstand zur anderen Geraden hat.

Auch eine einzelne Gerade kann "uber den Abstand von zwei Punkten $M$ und $M'$ 
beschrieben werden: Die Menge aller Punkte, die von zwei Punkten $M$ und $M'$ 
gleich weit entfernt sind, ist eine Gerade.

Dies ist ein typische Vorgehensweise in der Geometrie: Mit Hilfe von 
Eigenschaften, wie einem Abstand, neue Objekte zu definieren. Auf diese Weise
kann auch ein Kreis ganz einfach beschrieben werden.

\begin{MXInfo}{Kreis}
Gegeben ist ein Punkt $M$ und eine positive reelle Zahl $r$.
\par
\begin{tabular}{ll}
\begin{minipage}[c]{6cm}
Dann hei"st die Menge aller Punkte, die von $M$ den Abstand $r$ haben, ein 
\MEntry{Kreis}{Kreis} um $M$ mit \MEntry{Radius}{Radius} $r$.
\end{minipage}
&
\begin{minipage}[c]{6cm}
\begin{center}
\MTikzAuto{%
\begin{tikzpicture}[line width=1pt]
\filldraw (0,0) circle(1pt);
\node[below] (0,0) {$M$};
\draw[style=dotted] (0,0) -- node[above] {$r$} (1.5,0);
\draw[color=blue] (0,0) circle(1.5cm);
\end{tikzpicture}
}
\end{center}
\end{minipage}
\end{tabular}
\end{MXInfo}
\end{MXContent}

%content: Strahlensatz.
\begin{MXContent}{Strahlens\"atze}{Strahlens\"atze}{STD}
\MDeclareSiteUXID{VBKM05_Strahlensaetze_Content}

Eine Lochkamera liefert ein kleines Bild der Umwelt. Die Gr"o"senverh"altnisse
vom Bildh"ohe $B$ zu Gegenstandsh"ohe $G$ sind proportional zu den Abst"anden $b$ 
beziehungsweise $g$, die jeweils von der Lochblende $L$ gemessen werden:
\[
\frac{B}{G} = \frac{b}{g} \MDFPeriod %%
\]

%Lochkamera:
\begin{center}
\MTikzAuto{%
\begin{tikzpicture}[line width=1pt]
\coordinate (L) at (0,0);
\coordinate (B0) at (-3,0);
\coordinate (G0) at (9,0);
\coordinate (B1) at ($(B0) + (0,-1)$);
\coordinate (G1) at ($(G0) + (0,3)$);
%\fill[color=black!50!white] (0,0) -- (3,0) -- (3,1) -- cycle;
%\fill[color=black!50!white] (0,0) -- (-3,0) -- (-3,-1) -- cycle;
%\fill[color=black!30!white] (3,0) -- (6,0) -- (6,2) -- (3,1) -- cycle;
%\fill[color=black!15!white] (6,0) -- (9,0) -- (9,3) -- (6,2) -- cycle;
\draw[style=dotted] (B0) -- node[left] {$B$} (B1);
\draw[style=dashed] (G0) -- node[right] {$G$} (G1);
\draw (B0) -- node[above] {$b$} (L) -- node[below] {$g$} (G0);
\draw (B1) -- (G1);
\node[below] at (L) {$L$};
%\node[below] at (G0) {$G_0$};
%\node[above] at (B0) {$B_0$};
%\node[above] at (G1) {$G1$};
%\node[below] at (B1) {$B_1$};
\end{tikzpicture}
}
\end{center}


Eigenschaften von Bildern einer zentrischen Streckung kann man ebenfalls
mit den Strahlens"atzen beschreiben (siehe auch die Abbildung 
\MRef{Mathematik_ElementareGeometrie_zentrischeStreckung}).

Die gemeinsame Eigenschaft der Beispiele findet sich darin, dass Strahlen (oder 
Geraden) mit einem gemeinsamen Schnittpunkt von parallelen Geraden geschnitten
werden.


\begin{MXInfo}{Strahlens\"atze}%
\MLabel{VBKM05_Satz_Strahlensatz}%
Vom gemeinsamen Punkt $S$ gehen zwei verschiedene Strahlen $s_1$ und $s_2$ aus, 
die durch die Punkte $A$ beziehungsweise $C$ verlaufen. Der Punkt $B$ 
liegt auf dem Strahl $s_1$ und $D$ liegt auf dem Strahl $s_2$. Zuerst
werden die Strecken zwischen den Punkten auf den beiden Strahlen betrachtet,
dann auch die Strecken zwischen den Strahlen.
\par
\begin{tabular}{@{}lr@{}}
\begin{minipage}[b]{7cm}
%Auf dem Strahl $s_1$ werden die Strecken $\MGeoStrecke{S}{A}$ und 
%$\MGeoStrecke{S}{B}$ sowie $\MGeoStrecke{A}{B}$ betrachtet.
F\"ur zwei Punkte~$P$ und~$Q$ sei $\MGeoStrecke{P}{Q}$ die Strecke von~$P$ 
nach~$Q$, und $\MGeoAbstand{P}{Q}$ bezeichne die L\"ange dieser Strecke.
\vspace*{2cm}
\end{minipage}
&
\MTikzAuto{%
\begin{tikzpicture}
\coordinate (S) at (0,0);
\coordinate (A) at ($ (S) + (3,0.5) $);
\coordinate (C) at ($ (S) + (4,2.5) $);
\coordinate (B) at ($ (S)!1.7!(A) $);
\coordinate (D) at ($ (S)!1.7!(C) $);
%
\path (S) node[left]{$S$} (A) node[below right]{$A$} (B) node[below right]{$B$}
                          (D) node[above left] {$D$} (C) node[above left] {$C$};
%
\draw (S) -- ($ (S)!1.1!(B) $);
\draw (S) -- ($ (S)!1.1!(D) $);
%
\draw ($ (A)!-0.2!(C) $) -- ($ (C)!-1!(A) $) node[left]{$g$};
\draw ($ (B)!-0.2!(D) $) node[right]{$h$} -- ($ (D)!-0.1!(B) $);
\end{tikzpicture}
}
\end{tabular}
\par
Wenn die Geraden~$g$ und~$h$ parallel sind, gelten die folgenden Aussagen:
\begin{itemize}
\item
Die Abschnitte auf einem Strahl verhalten sich wie die entsprechenden 
Abschnitte auf dem anderen Strahl:
\[
   \frac{\MGeoAbstand{S}{A}}{\MGeoAbstand{S}{C}}
 = \frac{\MGeoAbstand{A}{B}}{\MGeoAbstand{C}{D}}
 = \frac{\MGeoAbstand{S}{B}}{\MGeoAbstand{S}{D}} \MDFPeriod
\]
Dies kann auch so ausgedr"uckt werden:
\[
   \frac{\MGeoAbstand{S}{A}}{\MGeoAbstand{A}{B}}
 = \frac{\MGeoAbstand{S}{C}}{\MGeoAbstand{C}{D}}
\quad\text{und}\quad
   \frac{\MGeoAbstand{S}{A}}{\MGeoAbstand{S}{B}}
 = \frac{\MGeoAbstand{S}{C}}{\MGeoAbstand{S}{D}} \MDFPeriod
\]
\item
Die Abschnitte auf den Parallelen verhalten sich wie die von $S$ ausgehenden 
entsprechenden Abschnitte auf einem Strahl:
\[
   \frac{\MGeoAbstand{S}{A}}{\MGeoAbstand{S}{B}}
 = \frac{\MGeoAbstand{A}{C}}{\MGeoAbstand{B}{D}}
 = \frac{\MGeoAbstand{S}{C}}{\MGeoAbstand{S}{D}} \MDFPeriod
\]
Dies kann auch so ausgedr"uckt werden:
\[
   \frac{\MGeoAbstand{S}{A}}{\MGeoAbstand{A}{C}}
 = \frac{\MGeoAbstand{S}{B}}{\MGeoAbstand{B}{D}}
\quad\text{und}\quad
   \frac{\MGeoAbstand{S}{C}}{\MGeoAbstand{C}{A}}
 = \frac{\MGeoAbstand{S}{D}}{\MGeoAbstand{D}{B}} \MDFPSpace,
\]
wobei $\MGeoAbstand{A}{C} = \MGeoAbstand{C}{A}$
und $\MGeoAbstand{B}{D} = \MGeoAbstand{D}{B}$ gilt.
\end{itemize}
\end{MXInfo}
Die Aussagen der Strahlens"atze gelten auch, wenn anstelle von Strahlen 
zwei Geraden betrachtet werden, die sich im Punkt $S$ schneiden. Als 
Anwendungsbeispiel ist oben eine Lochkamera genannt worden. Als Merksatz f\"ur die Strahlens|"atze kann man sich einfach $\frac{lang}{kurz} = \frac{lang}{kurz}$ einpr\"agen.

Auf diese Weise k"onnen Entfernungen zwischen Punkten berechnet werden,
ohne dass die Strecke direkt gemessen wird.

\begin{MExample}
Die gegebenen Punkte $A$, $B$, $C$ und $D$ bestimmen die Geraden 
$\MGeoGerade{A}{B}$ und $\MGeoGerade{C}{D}$, die sich im Punkt $S$
schneiden. Weiter ist bekannt, dass die Geraden $\MGeoGerade{A}{C}$ und 
$\MGeoGerade{B}{D}$ parallel sind. Zwischen den Punkten wurden die
Abst"ande $\MGeoAbstand{A}{B} = 51$, $\MGeoAbstand{S}{C} = 12 $ und 
$\MGeoAbstand{C}{D} = 18$ gemessen.

\begin{center}
\MTikzAuto{%
\begin{tikzpicture}
\coordinate (S) at (0,0);
\coordinate (A) at (20:3.4);
\coordinate (B) at (20:8.5);
\coordinate (C) at (60:1.2);
\coordinate (D) at (60:3.0);
%
\path %
 (S) node[below right]{$S$} (A) node[below right]{$A$} %
     (20:1.7) node[below right]{$x$} (B) node[below right]{$B$} %
          (D) node[above left] {$D$} (C) node[above left] {$C$};
%
\draw ($ (S)!-0.1!(B) $) -- ($ (S)!1.1!(B) $);
\draw ($ (S)!-0.1!(D) $) -- ($ (S)!1.1!(D) $);
%
\draw ($ (A)!-0.2!(C) $) -- ($ (C)!-0.1!(A) $);
\draw ($ (B)!-0.2!(D) $) -- ($ (D)!-0.1!(B) $);
\end{tikzpicture}
}
\end{center}

Daraus wird berechnet, wie weit $A$ von $S$ entfernt ist. Wenn der gesuchte
Abstand mit $x$ bezeichnet wird, gilt mit den Strahlens"atzen dann
\[
   \frac{x}{\MGeoAbstand{A}{B}}
 = \frac{\MGeoAbstand{S}{C}}{\MGeoAbstand{C}{D}} \MDFPSpace,
\]
woraus
\[
x = \frac{\MGeoAbstand{S}{C}}{\MGeoAbstand{C}{D}} \cdot \MGeoAbstand{A}{B} %
 = \frac{12}{18} \cdot 51 = \frac{2}{3} \cdot 51 = 34 %%
\]
folgt.
\end{MExample}
\end{MXContent}


\begin{MExercises}
\MDeclareSiteUXID{VBKM05_Strahlensaetze_Exercises}
\begin{MExercise}
Der Sohn des Hauses beobachtet den Baum auf des Nachbarn Grundst\"uck. Er 
stellt fest, dass der Baum von der Hecke, die die beiden Grundst\"ucke trennt,
vollst\"andig verdeckt wird, wenn er nur nahe genug an die Hecke herantritt.
Jetzt sucht er den Punkt, an dem der Baum gerade so nicht mehr zu sehen ist.

Der $\MZahl{1}{40}$~Meter gro\ss e Junge muss $\MZahl{2}{50}$~Meter von der 
$\MZahl{2}{40}$~Meter hohen, $1$~Meter breiten und oben spitz zulaufenden 
Hecke entfernt stehen, damit der Baum vollst\"andig verdeckt ist.

Wie hoch ist der Baum, wenn die Mitte des Stamms $\MZahl{14}{5}$~Meter von 
der Hecke entfernt steht?

F\"uhren Sie die Rechnung bitte zun\"achst allgemein durch und setzen Sie erst
am Ende die Zahlenwerte ein!

Ergebnis: \MLParsedQuestion{10}{7.4}{1}{GEO7} $\MEinheit{m}$.

\begin{MHint}{Hinweis}
Beachten Sie die Breite der Hecke!
\end{MHint}

\begin{MHint}{L\"osung}
Es werden die in der Zeichnung eingef"uhrten Bezeichnungen f"ur die L"angen 
der jeweiligen Strecken verwendet.

\begin{center}
\MTikzAuto{%
\begin{tikzpicture}[x=0.75cm, y=0.75cm]
\coordinate (KF) at (0,0);
\coordinate (KK) at (0,1.4);
\coordinate (HF) at (3,0);
\coordinate (HK) at (3,2.4);
\coordinate (BF) at (16.5,0);
\coordinate (BK) at (16.5,7.4);
\coordinate (BM) at (16.5,1.4);
%
\begin{scope}[color=black!50, every node/.style={color=black}]
\draw (KF) -- ++ (0,-0.3) ++(0,0.15) -- node[below]{$\MZahl{2}{5}\MEinheit{m}$} ($ (HF) + (-0.5,-0.15) $) ++(0,-0.15) -- ++(0,0.3);
\draw ($ (HF) + (0.5,0) $) -- ++ (0,-0.3) ++(0,0.15) -- node[below]{$\MZahl{14}{5}\MEinheit{m}$} ($ (BF) + (0,-0.15) $) ++(0,-0.15) -- ++(0,0.3);
 \node at (HF) [below]{$1\MEinheit{m}$};
 \draw ($ (KF)!0.5!(BF) $) -- node[right]{$h_{\mathrm{K}} = \MZahl{1}{4}\MEinheit{m}$} ($ (KK)!0.5!(BM) $) ++ (-0.15,0) -- ++(0.3,0);
 \draw (BF) -- node[right]{$h_{\mathrm{B}}$} (BK) ++ (-0.15,0) -- ++(0.3,0);
 \draw ($ (HF) + (0.75,0) $) -- node[below right]{$h_{\mathrm{H}} = \MZahl{2}{4}\MEinheit{m}$} ($ (HK) + (0.75,0) $) ++ (-0.15,0) -- ++(0.3,0);
 \draw (KK) -- ++ (0,-0.3) ++(0,0.15) -- node[below]{$d_{\mathrm{KH}}$} ($ (HF) + (KK) - (KF) + (0,-0.15) $) ++(0,-0.15) -- ++(0,0.3);
 \draw (KK) -- ++ (0,0.3) ++(0,-0.15) -- node[above]{$d_{\mathrm{KB}}$} ($ (BM) + (0,0.15) $) ++(0,0.15) -- ++(0,-0.3);
\end{scope}
\draw (KF) ++ (-0.2,0) -- ++ (0.2,0.5) -- ++ (0.2,-0.5)
      (KF) ++ (0,0.5) -- ++ (0,0.5)
      (KK) ++ (0,-0.6) ++ (-0.2,-0.4) -- ++ (0.2,0.4) -- ++ (0.2,-0.4)
      (KK) ++ (0,-0.2) circle (0.2);
\draw (HK) decorate[decoration={random steps,segment length=2pt,amplitude=1pt}] {.. controls ($ (HF) + (-0.5,1.5) $) .. ($ (HF) + (-0.5,0) $)};
\draw (HK) decorate[decoration={random steps,segment length=2pt,amplitude=1pt}] {.. controls ($ (HF) + (0.5,1.5) $) .. ($ (HF) + (0.5,0) $)};
\draw (BK) decorate[decoration={saw,mirror}] {-- ($ (BF) + (-2.9,1) $) -- ++ (2.7,0)} -- ($ (BF) + (-0.3,0) $);
\draw (BK) decorate[decoration={saw}] {-- ($ (BF) + (2.9,1) $) -- ++ (-2.7,0)} -- ($ (BF) + (0.3,0) $);
\draw (KK) -- (BK) ($ (KF)!-0.05!(BF) $) -- ($ (BF)!-0.05!(KF) $);
\draw[dashed] (KK) -- (BM);
\end{tikzpicture}
}
\end{center}

Eine Anwendung des zweiten Strahlensatzes
{\glqq}$\frac{\text{komplett}}{\text{vorne}} %
 = \frac{\text{lang}}{\text{kurz}}${\grqq}
liefert:
\[
\frac{d_{\mathrm{KB}}}{d_{\mathrm{KH}}}
= \frac{h_{\mathrm{B}} - h_{\mathrm{K}}}{h_{\mathrm{H}} - h_{\mathrm{K}}}
 \qquad \text{bzw.} \qquad
 h_{\mathrm{B}} = \left( h_{\mathrm{H}} - h_{\mathrm{K}} \right) \cdot \frac{d_{\mathrm{KB}}}{d_{\mathrm{KH}}} + h_{\mathrm{K}} \MDFPeriod
\]

Nun gelten $d_{\mathrm{KH}} = \MZahl{2}{5}\MEinheit{m} + \frac{1\MEinheit{m}}{2} = 3\MEinheit{m}$
und $d_{\mathrm{KB}} = \MZahl{2}{5}\MEinheit{m} + 1\MEinheit{m} + \MZahl{14}{5}\MEinheit{m} = 18\MEinheit{m}$.
Damit folgt
\[
  h_{\mathrm{B}}
   = \left( \MZahl{2}{4}\MEinheit{m} - \MZahl{1}{4}\MEinheit{m} \right) \cdot \frac{18\MEinheit{m}}{3\MEinheit{m}} + \MZahl{1}{4}\MEinheit{m}
   = 1\MEinheit{m} \cdot 6 + \MZahl{1}{4}\MEinheit{m}
   = \MZahl{7}{4}\MEinheit{m} \MDFPeriod
\]
\end{MHint}
\end{MExercise}

\end{MExercises}

%jgl: end of section 1.


%content: section 2: Winkel und Winkelmessung.

\MSubsection{Winkel und Winkelmessung}
\MLabel{M05_Winkel}

\begin{MIntro}
\MDeclareSiteUXID{VBKM05_WinkelIntro}

Geraden, die sich in einem Punkt $S$ schneiden, teilen die Ebene in 
charakteristischer Weise auf.
Zur Beschreibung dieser Beobachtung wird der Begriff Winkel eingef"uhrt.
Die Frage nach M"oglichkeiten, Winkel zu messen, wird auf verschiedene 
Weisen beantwortet, die letztlich alle auf einer Einteilung von Kreisen 
beruhen. 
Hier werden das Gradma"s und das Bogenma"s beschrieben.

\begin{center}
\MTikzAuto{%
\begin{tikzpicture}[line width=1pt]
\coordinate (S) at (0,0);
\coordinate (A) at (6,0);
\coordinate (B) at (6,3);
\coordinate (C) at (-2,0);
\coordinate (D) at (-2,-1);
%
\fill[color=blue!50!white] (S) -- (A) -- (B) -- (S);
\fill[color=blue!25!white] (S) -- (C) -- (D) -- (S);
\fill[color=green!50!white] (S) -- (B) -- (C) -- (S);
\fill[color=green!25!white] (S) -- (D) -- (A) -- (S);
\draw (C) -- node[below]{$g$} (A);
\draw (D) -- node[above]{$h$} (B);
\node[below] at (S) {$S$};
\end{tikzpicture}
}
\par
Jeder farbige Bereich zeigt einen der Winkel, 
die durch $g$ und $h$ festgelegt werden.
\end{center}
\end{MIntro}


\begin{MXContent}{Winkel}{Winkel}{STD}
\MDeclareSiteUXID{VBKM05_Winkel_Content}
\MLabel{VBKM05_Winkel}
Zwei Strahlen (Halbgeraden) $g$ und $h$ in der Ebene, die von demselben Punkt 
$S$ ausgehen, schlie"sen einen \MEntry{Winkel}{Winkel} 
$\Mmeasuredangle\left(g, h\right)$ ein.

\begin{center}
\MTikzAuto{%
\begin{tikzpicture}[line width=1pt]
\coordinate (S) at (0,0);
\coordinate (A) at (6,0);
\coordinate (B) at (6,3);
\coordinate (C) at (-2,0);
\coordinate (D) at (-2,-1);
%
\fill[color=blue!50!white] (S) -- (A) -- (B) -- (S);
\draw (S) -- node[below]{$g$} (A);
\draw (S) -- node[above]{$h$} (B);
\node[left] at (S) {$S$};
\node at (2,0.4) {$\Mmeasuredangle(g,h)$};
\end{tikzpicture}
}
\par
Winkel zwischen den Strahlen $g$ und $h$.
\end{center}

In der Bezeichnung des Winkels $\Mmeasuredangle \left( g, h \right)$ ist die 
Reihenfolge wichtig, in der $g$ und $h$ aufgeschrieben werden.
$\Mmeasuredangle \left( g, h \right)$ bezeichnet den oben beschriebenen 
Winkel, der dadurch festgelegt ist, dass man die Halbgerade $g$ gegen 
den Uhrzeigersinn zur Halbgeraden $h$ dreht.
Mit $\Mmeasuredangle \left( h, g \right)$ wird der Winkel von $h$ zu $g$ 
bezeichnet.

\begin{center}
\MTikzAuto{%
\begin{tikzpicture}[line width=1pt,scale=0.5]
\coordinate (S) at (0,0);
\coordinate (A) at (6,0);
\coordinate (B) at (6,3);
\coordinate (CC) at (-4,1);
\coordinate (DD) at (-4,-2);
%
\fill[color=red!50!white] (S) -- (B) -- (CC) -- (DD) -- (A) -- (S);
\draw (S) -- node[below]{$g$} (A);
\draw (S) -- node[above]{$h$} (B);
\node[left] at (S) {$S$};
\node at (-2,-0.8) {$\Mmeasuredangle(h,g)$};
\end{tikzpicture}
}
\par
Winkel zwischen den Strahlen $h$ und $g$.
\end{center}

Der Punkt $S$ hei"st \MEntry{Scheitelpunkt}{Scheitelpunkt (Winkel)} des 
Winkels, und die beiden Halbgeraden, die den Winkel bilden, hei"sen 
\MEntry{Schenkel}{Schenkel} des Winkels.
Wenn $A$ ein Punkt auf der Halbgeraden $g$ und $B$ ein Punkt auf der Halbgeraden $h$
ist, so kann man auch $\Mmeasuredangle \left( A S B \right)$ f"ur 
$\Mmeasuredangle \left( g, h \right)$ schreiben. In diesem Sinne werden 
Winkel zwischen Strecken $\MGeoStrecke{S}{A}$ und $\MGeoStrecke{S}{B}$
beschrieben.

Winkel werden oft mit kleinen griechischen Buchstaben bezeichnet, soweit sie
sich vom lateinischen Alphabet unterscheiden, vgl. dazu Tabelle \MRef{VBKM01_Griechisch}
im Kapitel \MNRef{VBKM01}.
Indem man Geraden in die Betrachtungen miteinbezieht, lassen sich weitere 
Winkel entdecken.

\begin{MXInfo}{Scheitelwinkel und Nebenwinkel}%
\MLabel{VBKM05_Scheitelwinkel_Nebenwinkel}%
Es seien $g$ und $h$ zwei Geraden, die sich im Punkt $S$ schneiden.

\begin{center}
\MTikzAuto{%
\begin{tikzpicture}[line width=1pt,scale=0.5]
%\coordinate[label=below:{$P$}] (P) at (4,0);
\coordinate (S) at (0,0);
\coordinate (A) at (6,0);
\coordinate (B) at (6,3);
\coordinate (C) at (-6,0);
\coordinate (D) at (-6,-3);
%
\fill[color=blue!50!white] (S) -- (A) -- (B) -- (S);
\fill[color=blue!25!white] (S) -- (C) -- (D) -- (S);
\fill[color=green!50!white] (S) -- (B) -- (C) -- (S);
\fill[color=green!25!white] (S) -- (D) -- (A) -- (S);
\draw (C) -- (A) node[below left] {$g$};
\draw (D) --  (B) node[above right] {$h$};
%\node[below] at (A) {$g$};
%\node[above] at (B) {$h$};
\node[below] at (S) {$S$};
\node at (3,0.6) {$\Mvarphi$};
\node at (-3,-0.6) {$\Mvarphi'$};
\node at (-1.5,0.6) {$\psi$};
\node at (1.5,-0.6) {$\psi'$};
\end{tikzpicture}
}
\end{center}

\begin{itemize}
\item Die Winkel $\Mvarphi$ und $\Mvarphi'$ hei"sen 
 \MEntry{Scheitelwinkel}{Scheitelwinkel} zueinander, da sie sich am selben Scheitelpunkt gegen\"uber liegen.
\item Die Winkel $\Mvarphi$ und $\psi$ hei"sen 
 \MEntry{Nebenwinkel}{Nebenwinkel} bez"uglich $g$, da sie an der selben Gerade liegen.
\end{itemize}
\end{MXInfo}

In der obigen Zeichnung gibt es noch weitere Scheitelwinkel und
Nebenwinkel.

\begin{MExercise}
Notieren Sie alle Scheitelwinkel und alle Nebenwinkel.

\begin{MHint}{L"osung}
Zus"atzlich zu $\Mvarphi$ und $\Mvarphi'$ sind auch $\psi$ und $\psi'$ 
Scheitelwinkel.
Nebenwinkel von $g$ sind zus"atzlich zu $\Mvarphi$ und $\psi$ auch die Winkel 
$\Mvarphi'$ und $\psi'$. Au"serdem sind $\psi$ und $\Mvarphi'$ sowie
$\psi'$ und $\Mvarphi$ Nebenwinkel.
\end{MHint}
\end{MExercise}

Einige besondere Winkel erhalten eigene Namen.
Dabei ist eine Winkelhalbierende $w$ diejenige Halbgerade, deren Punkte von 
beiden gegebenen Halbgeraden $g$ und $h$ denselben Abstand haben. Dann kann 
man sagen, dass $w$ den Winkel zwischen $g$ und $h$ halbiert.

\begin{MXInfo}{Namen besonderer Winkel}
Seien $g$ und $h$ Halbgeraden mit dem Scheitelpunkt $S$.
\begin{itemize}
\item
Der Winkel, der die gesamte Ebene "uberdeckt, hei"st 
\MEntry{Vollwinkel}{Vollwinkel (Winkel)}. In diesem Fall sind die beiden Geraden identisch.
\item
Wenn $g$ und $h$ eine Gerade bilden, hei"st der Winkel zwischen $g$ und $h$
%$\Mmeasuredangle(g,h)$ gestreckter Winkel.
gestreckter Winkel.
\item
Der Winkel zwischen zwei Halbgeraden, die einen gestreckten Winkel halbieren,
hei"st \MEntry{rechter Winkel}{rechter Winkel (Winkel)}. 
Man sagt dann auch, dass $g$ und $h$  
\MEntry{ senkrecht aufeinander stehen}{senkrecht (Gerade)} oder dass 
$g$ und $h$  \MEntry{ orthogonal zueinander}{orthogonal} sind. 
\end{itemize}
\end{MXInfo}

Als N"achstes sollen nun drei verschiedene Geraden betrachtet werden, 
von denen zwei parallel sind, wohingegen die dritte nicht parallel zu diesen 
beiden ist. Es ergeben sich dann acht Schnittwinkel.
Je vier dieser Winkel sind gleich gro"s.

\begin{MXInfo}{Winkel an parallelen Geraden}%
\MLabel{Mathematik_ElementareGeometrie_StufenwinkelWechselwinkel}%
Gegeben sind zwei parallele Geraden $g$ und $h$, die von einer Geraden $j$ 
geschnitten werden.

\begin{center}
\MTikzAuto{%
\begin{tikzpicture}
\coordinate (S) at (0,0);
\coordinate (P) at (4,1);
\coordinate (T) at ($ (S) + (1,2) $);
\coordinate (Q) at ($ (P) + (1,2) $);
%
%Winkel alpha:
\draw[color=red!50!white] ({atan(2)}:{sqrt{17}/4}) arc({atan(2)}:{180+atan(0.25)}:{sqrt(17)/4});
%\filldraw[color=red!50!white] (S) -- (0.5,1) -- (-1,-0.25) -- cycle;
%Winkel beta:
%\draw[color=blue!50!white] (1,0.25) arc({atan(0.25)}:{atan(2)}:{sqrt(17)/4});
%\filldraw[color=blue!50!white] (S) -- (1,0.25) -- (0.5,1) -- cycle;
%
%Winkel alpha':
%\filldraw[color=red!50!white] (T) -- ($ (T) + (0.5,1) $) -- ($ (T) + (-1,-0.25) $)-- cycle;
\draw[color=red!50!white] ($ (T) + ({atan(2)}:{sqrt{17}/4}) $) arc({atan(2)}:{180+atan(0.25)}:{sqrt(17)/4});
%Winkel beta':
%\filldraw[color=blue!50!white] (T) -- ($ (T) +(-1,-0.25) $) -- ($ (T) + (-0.5,-1) $) -- cycle;
%\draw[color=blue!50!white] ($ (T) + (-1,-0.25) $) arc({180+atan(0.25)}:{180+atan(2)}:{sqrt(17)/4});
%
\draw ($ (S) + (0,0) $) node[above left,color=red]{$\alpha$};
\draw ($ (T) + (0,0) $) node[above left,color=red]{$\alpha'$};
%
%\draw ($ (S) + (0.2,0.05) $) node[above right,color=blue]{$\beta$};
%\draw ($ (T) + (-0.2,-0.1) $) node[below left,color=blue]{$\beta'$};
%
\draw ($ (S)!-0.7!(P) $) node[below right] {$g$} -- ($ (S)!0.7!(P) $);
\draw ($ (T)!-0.7!(Q) $) node[below right] {$h$} -- ($ (T)!0.7!(Q) $);
%
\draw ($ (S)!-0.4!(T) $) -- ($ (S)!1.6!(T) $) node[right]{$j$};
\end{tikzpicture}
}
%
\MTikzAuto{%
\begin{tikzpicture}
\coordinate (S) at (0,0);
\coordinate (P) at (4,1);
\coordinate (T) at ($ (S) + (1,2) $);
\coordinate (Q) at ($ (P) + (1,2) $);
%
%Winkel alpha:
%\draw[color=red!50!white] ({atan(2)}:{sqrt{17}/4}) arc({atan(2)}:{180+atan(0.25)}:{sqrt(17)/4});
%\filldraw[color=red!50!white] (S) -- (0.5,1) -- (-1,-0.25) -- cycle;
%Winkel beta:
\draw[color=blue!50!white] (1,0.25) arc({atan(0.25)}:{atan(2)}:{sqrt(17)/4});
%\filldraw[color=blue!50!white] (S) -- (1,0.25) -- (0.5,1) -- cycle;
%
%Winkel alpha':
%\filldraw[color=red!50!white] (T) -- ($ (T) + (0.5,1) $) -- ($ (T) + (-1,-0.25) $)-- cycle;
%\draw[color=red!50!white] ($ (T) + ({atan(2)}:{sqrt{17}/4}) $) arc({atan(2)}:{180+atan(0.25)}:{sqrt(17)/4});
%Winkel beta':
\draw[color=blue!50!white] ($ (T) + (-1,-0.25) $) arc({180+atan(0.25)}:{180+atan(2)}:{sqrt(17)/4});
%\filldraw[color=blue!50!white] (T) -- ($ (T) +(-1,-0.25) $) -- ($ (T) + (-0.5,-1) $) -- cycle;
%
%\draw ($ (S) + (0,0) $) node[above left,color=red]{$\alpha$};
%\draw ($ (T) + (0,0) $) node[above left,color=red]{$\alpha'$};
%
\draw ($ (S) + (0.2,0.05) $) node[above right,color=blue]{$\beta$};
\draw ($ (T) + (-0.2,-0.1) $) node[below left,color=blue]{$\beta'$};
%
\draw ($ (S)!-0.7!(P) $) node[below right] {$g$} -- ($ (S)!0.7!(P) $);
\draw ($ (T)!-0.7!(Q) $) node[below right] {$h$} -- ($ (T)!0.7!(Q) $);
%
\draw ($ (S)!-0.4!(T) $) -- ($ (S)!1.6!(T) $) node[right]{$j$};
\end{tikzpicture}
}
\end{center}

\begin{itemize}
\item Dann hei"st der Winkel $\alpha'$ ein \MEntry{Stufenwinkel}{Stufenwinkel} 
 zu $\alpha$, und
\item der Winkel $\beta'$ ein \MEntry{Wechselwinkel}{Wechselwinkel} zu $\beta$.
\end{itemize}
Da $g$ und $h$ parallel sind, sind die Winkel $\alpha$ und $\alpha'$ gleich
gro"s. Ebenso sind $\beta$ und $\beta'$ gleich gro"s.
\end{MXInfo}

\begin{MExercise}
In der Zeichnung werden zwei parallele Geraden $g$ und $h$ dargestellt, 
die von einer weiteren Geraden $j$ geschnitten werden.
Erl"autern Sie, welche Winkel gleich gro"s sind und welche Winkel 
Stufenwinkel beziehungsweise Wechselwinkel zueinander sind.

\begin{center}
\MTikzAuto{%
\begin{tikzpicture}
\coordinate (G) at (0,0);
\coordinate (H) at ($ (G) + (30:5) $);
\coordinate (A) at ($ (G) + (30:1.5) $);
\coordinate (B) at ($ (H) + (A) - (G) $);
\coordinate (C) at ($ (G) + (0,2) $);
\coordinate (D) at ($ (C) + (B) - (A) $);
\coordinate (E) at (0,5);
\coordinate (F) at (5,1);
\coordinate (S) at (intersection of A--B and E--F);
\coordinate (T) at (intersection of C--D and E--F);
%
\draw[dotted] (S) circle [radius=0.9] (T) circle [radius=0.9];
\draw (A) -- node[at end, below] {$g$} (B) (C) -- node[at end, above left] {$h$} (D) (F) -- node[at end, below] {$j$} (E);
%
\begin{scope}[outer sep=4pt]
 \node at (S) [right] {$\alpha$};
 \node at (S) [above] {$\beta$};
 \node at (S) [left]  {$\gamma$};
 \node at (S) [below] {$\delta$};
 \node at (T) [right] {$\Mvarepsilon$};
 \node at (T) [above] {$\chi$};
 \node at (T) [left]  {$\Mvarphi$};
 \node at (T) [below] {$\psi$};
\end{scope}
\end{tikzpicture}
}
\end{center}

\begin{MHint}{L"osung}
\begin{itemize}
\item Die Winkel $\alpha$, $\gamma$, $\Mvarepsilon$ und $\Mvarphi$ sind 
 gleich gro"s, ebenso die Winkel $\beta$, $\delta$, $\chi$ und $\psi$.
\item Es sind $\beta$ und $\psi$ bzw. $\gamma$ und $\Mvarepsilon$ Wechselwinkel.
\item Die Winkel $\alpha$ und $\Mvarepsilon$ sind Stufenwinkel, 
 ebenso $\beta$ und $\chi$, $\delta$ und $\psi$ und $\gamma$ und $\Mvarphi$.
\end{itemize}
\end{MHint}
\end{MExercise}
\end{MXContent}


\begin{MXContent}{Winkelmessung}{Winkelmessung}{STD}
\MDeclareSiteUXID{VBKM05_Winkelmessung_Content}
\MLabel{VBKM05_Winkelmessung}


Die Erkl"arung zur Bezeichnung des Winkel $\Mmeasuredangle(g,h)$, der durch
Drehung von $g$ gegen den Uhrzeigersinn zu $h$ festgelegt wird, f"uhrt 
zu einer Idee, Winkel zu messen, das hei"st quantitativ miteinander zu 
vergleichen.

So wie auf einem runden Ziffernblatt einer analogen Uhr die Markierungen zu den
zw"olf Zahlen f"ur die Stunden im gleichen Abstand voneinander angebracht sind,
kann man einen Kreis gleichm"a"sig einteilen. Auf diese Weise erh"alt man eine
Skala f"ur Winkel. Je nach der verwendeten Skalierung ergeben sich 
verschiedene Zahlen, mit denen die Gr"o"se eines Winkels angegeben werden kann.

\paragraph{Gradma"s}

Es wird eine Kreisscheibe in $360$ gleiche Segmente eingeteilt. Eine Drehung
um ein Segment beschreibt einen Winkel von $1$ Grad. Hierf"ur wird $1\MGrad$ 
geschrieben. In der folgenden Zeichnung sind Winkel mit einem Gradma"s von 
$30\MGrad$ und Vielfachen davon dargestellt.

%Winkel im Gradmass:
\begin{center}
\MTikzAuto{%
\begin{tikzpicture}[line width=1pt]
\coordinate (M) at (0,0);
\coordinate (A) at (2,0);
 \draw (M) -- (0:2cm);
\foreach \x in {30, 60, 120, 150, 210, 240, 300, 330}
 \draw[style=dotted] (M) -- ({\x}:2);
%\foreach \x in {0, 90, 180, 270}
% \draw (M) -- ({\x}:2cm);
% \draw[color=black!8!white] (M) -- (0:2cm) node[right,color=black] {$0\MGrad$};
 \draw (M) -- (0:2cm) node[right] {$0\MGrad$};
 \draw (M) -- (90:2cm) node[above] {$90\MGrad$};
 \draw (M) -- (180:2cm) node[left] {$180\MGrad$};
 \draw (M) -- (270:2cm) node[below] {$270\MGrad$};
\node at (30:2.4cm) {$30\MGrad$};
\node at (60:2.4cm) {$60\MGrad$};
\node at (120:2.4cm) {$120\MGrad$};
\node at (150:2.4cm) {$150\MGrad$};
\node at (210:2.4cm) {$210\MGrad$};
\node at (240:2.4cm) {$240\MGrad$};
\node at (300:2.4cm) {$300\MGrad$};
\node at (330:2.4cm) {$330\MGrad$};
%Kreis:
\draw (M) circle(2cm);
\end{tikzpicture}
}
\end{center}

\paragraph{Bogenma"s}

Bereits im antiken Babylonien, "Agypten und Griechenland stellte man fest, 
dass das Verh\"altnis des Umfangs $U$ eines Kreises zu seinem Durchmesser 
$D$ stets das gleiche ist, und somit Umfang und Durchmesser zueinander 
proportional sind.
Dieses Verh\"altnis wird die Kreiszahl $\pi$ genannt. 
\begin{MXInfo}{Kreiszahl}\MLabel{Kreiszahl}%
Gegeben ist ein Kreis mit Umfang $U$ und Durchmesser $D$.
Die \textbf{Kreiszahl} ist
\[
\pi = \frac{U}{D} = \frac{U}{2r} %%
\]
mit dem Kreisradius $r = \frac{1}{2} D$.
Dabei ist $\pi$ keine rationale Zahl. Sie kann nicht als endlicher oder 
periodischer Dezimalbruch geschrieben werden. Numerische Berechnungen ergeben,
dass n"aherungsweise $\pi \approx \MZahl{3}{141592653589793}$ ist.
\end{MXInfo}

Wenn der Radius $r$ des Kreises genau $1$ ist, so hat der Kreis den Umfang $2\pi$. 
Beim \MEntry{Bogenma"s}{Bogenma"s} wird die Linie des Umfang eines Kreises vom Radius $r = 1$
eingeteilt.
Als Bogenma"s eines Winkels $\Mmeasuredangle(g,h)$ wird die L"ange des 
\MEntry{Kreisbogens}{Kreisbogen} verwendet, die durch den Winkel 
{\glqq}ausgeschnitten{\grqq} wird.

Damit wird Winkeln durch das Bogenma"s eine Zahl zwischen $0$ und $2\pi$ 
zugeordnet. In der Technik wird auch die Kennzeichnung rad (f"ur Radiant)
verwendet, um explizit auszudr"ucken, dass ein Winkel im Bogenma"s gemessen
wird.

\begin{MXInfo}{Bogenma"s}\MLabel{VBKM05_Def_Bogenmass}%
Gegeben sind zwei Halbgeraden $g$ und $h$, die vom gemeinsamen Punkt $S$
ausgehen und den Winkel $\Mmeasuredangle(g,h)$ einschlie"sen.
Zeichnet man einen Kreis mit Radius~$r = 1$ um~$S$, wird der Kreis von den 
beiden Halbgeraden in zwei Teile zerschnitten. Wichtig ist nun derjenige 
Kreisbogen $x$, auf dem man von der Halbgeraden $g$ gegen den Uhrzeigersinn 
zur Halbgeraden $h$ kommt (im Bild gr\"un eingef\"arbt). Dies kann man auch 
so ausdr"ucken, dass der Scheitelpunkt $S$ stets links liegt, 
wenn man sich auf dem Kreisbogen $x$ von $g$ in Richtung $h$ bewegt.

\begin{center}
\MTikzAuto{%
\begin{tikzpicture}[scale=0.9,line width=2pt]
\coordinate[label=below:$S$] (S) at (0,0);
\coordinate[label=below left:$g$] (A) at ($ (S) + (-10:3) $);
\coordinate[label=below left:$h$] (B) at ($ (S) + (100:3) $);
%
%Winkel:
\fill[color=green!30!white] (A) -- (S) -- (B);
%Radius:
\draw[dotted, line width=1.5pt] (S) -- node[above]{$r$} ++ (-1.4,0);
%Kreisboegen:
\draw [color=green!50!black] (S) ++ (-10:1.4) arc (-10:100:1.4);
\draw [color=red] (S) ++ (100:1.4) arc (100:350:1.4);
%Halbgeraden mit Schnittpunkt:
\draw (A) -- (S) -- (B);
\filldraw (0,0) circle(1pt);
%Bezeichnung fuer den Bogen:
\node[right,color=green!50!black] at (1.4,0) {$x$};
\end{tikzpicture}
}
\end{center}

Die L\"ange des Kreisbogens $x$ ist das \MEntry{Bogenma"s}{Bogenma"s} 
des Winkels $\Mmeasuredangle \left( g, h \right)$.
\end{MXInfo}

Mit einem Winkelma"s wie dem Bogenma"s oder dem zuvor eingef"uhrten Gradma"s
kann man Winkel einfach in verschiedene Klassen einteilen und daf"ur eigene
Namen vergeben.
Zur Wiederholung werden auch bereits eingef"uhrte Bezeichnungen nochmals 
mit aufgef"uhrt.
\begin{MXInfo}{Namen f"ur verschiedene Klassen von Winkeln}
F"ur Winkel, deren Bogenma"s in einem bestimmten Bereich liegt, werden folgende
Bezeichnungen eingef"uhrt:
\begin{itemize}
\item
Ein Winkel mit einem Ma\ss\ zwischen~$0$ und $\frac{\pi}{2}$ hei"st
\MEntry{spitzer Winkel}{Winkel (spitz)}.
       
\item
Ein Winkel mit einem Ma\ss\ von~$\frac{\pi}{2}$ hei"st \textbf{rechter Winkel}.
       
\item
Ein Winkel mit einem Ma\ss\ zwischen~$\frac{\pi}{2}$ und $\pi$ hei\ss t 
\MEntry{stumpfer Winkel}{Winkel (stumpf)}.
       
\item
Ein Winkel mit einem Ma\ss\ zwischen~$\pi$ und~$2 \pi$ hei\ss t 
\MEntry{"uberstumpfer Winkel}{Winkel ("uberstumpf)}.
\end{itemize}
Man sagt, zwei Halbgeraden \textbf{stehen senkrecht aufeinander}, 
wenn sie einen rechten Winkel bilden.

Zwei Halbgeraden bilden eine Gerade, wenn sie einen Winkel vom Ma\ss~$\pi$ bilden.
\end{MXInfo}

Wenn man das Bogenma"s des Winkels $\Mmeasuredangle \left( g, h \right)$ kennt, 
kann man auch das Bogema"s des Winkels $\Mmeasuredangle \left( h, g \right)$ 
bestimmen. Aus der obigen Definition \MRef{VBKM05_Def_Bogenmass} ergibt sich, 
dass 
\[
   \Mmeasuredangle \left( h, g \right)
 = 2 \pi - \Mmeasuredangle \left( g, h \right) %%
\]
gilt. In der Zeichnung zur Definition \MRef{VBKM05_Def_Bogenmass} ist das 
Bogenma"s des Winkels $\Mmeasuredangle(h,g)$ die L"ange des rot dargestellten 
Kreisbogens des Kreises mit dem Radius $r = 1$.

Die Formulierungen der letzten S"atze klingen m"oglicherweise umst"andlich.
Dies liegt vermutlich auch daran, dass genau zwischen Winkel und einem Ma"s, 
hier dem Bogenma"s, f"ur den Winkel unterschieden wird.

Wenn es darum geht, einen gesuchten Wert zu berechnen, wird bei Strecken 
oft dieselbe Bezeichnung f"ur die Strecke und ihre L"ange verwendet. Dies 
ist meistens verst"andlich und hilft, einen Sachverhalt einfach zu beschreiben
oder in einer Zeichnung darzustellen. Wichtig ist dabei, dass die verwendete 
Einheit bekannt ist oder explizit angegeben wird.
Eine solche Vereinbarung, auch Konvention genannt, wird im Zusammenhang mit 
Winkeln ebenfalls oft verwendet, wenn aus dem Zusammenhang verst"andlich wird,
dass es um die Berechnung eines Wertes in einem bestimmten Winkelma"s geht.

\begin{MXInfo}{Konvention}\MLabel{VBKM05_Konvention_Winkel}%
Wenn eine Berechnung unabh"angig von einem bestimmten Winkelma"s ist oder 
dieses vorab festgelegt wurde, wird in Rechnungen auch kurz vom Winkel 
gesprochen und dieselbe Bezeichnung f"ur den Winkel und seinen Wert im 
gew"ahlten Winkelma"s verwendet.
\end{MXInfo}
In diesem Sinne kann dann beispielsweise $\Mmeasuredangle(g,h) = 90\MGrad$ 
geschrieben und vom rechten Winkel $\Mmeasuredangle(g,h)$ gesprochen werden, 
der von Geraden $g$ und $h$ eingeschlossen wird.
Dies gilt entsprechend, wenn das Bogenma"s verwendet wird.

Ein Wert eines Winkels im Gradma"s kann in das Bogenma"s umgerechnet werden 
(und umgekehrt), indem man die Verh"altnisse der Ma"szahlen eines Winkels zum 
Wert des Vollwinkels im jeweiligen Winkelma"s betrachtet. Die Umrechnung 
zwischen Bogenma"s und Gradma"s wird im Folgenden beschrieben.

\begin{MXInfo}{Zusammenhang zwischen Bogenma"s und Gradma"s}
Es werden Halbgeraden $g$ und $h$ betrachtet, die den Winkel 
$\Mmeasuredangle(g,h)$ einschlie"sen. Das Bogenma"s des Winkels wird mit $x$ 
bezeichnet und das Gradma"s des Winkels mit $\alpha$. 

Dann ist der Anteil $x$ von $2\pi$ gleich dem Anteil $\alpha$ von 
$360\MGrad$ und damit: 
\[
   \frac{x}{2\pi} = \frac{\alpha}{360\MGrad} \MDFPeriod
\]
Somit ist 
\[
x = \frac{\pi}{180\MGrad} \cdot \alpha %
\quad\text{und}\quad
\alpha = \frac{180\MGrad}{\pi} \cdot x \MDFPeriod
\]
\end{MXInfo}
Deshalb sind die Angaben im Bogenma"s und im Gradma"s zueinander proportional,
sodass die Umrechnung mit dem jeweiligen Proportionalit"atsfaktor 
$\frac{\pi}{180\MGrad}$ beziehungsweise $\frac{180\MGrad}{\pi}$ 
sehr einfach ist.

\begin{MExercise}
Der Winkel $\Mmeasuredangle\left(g, h\right)$ betr\"agt im Gradma"s $60\MGrad$.
Rechnen Sie den Winkel in das Bogenma\ss\ um:
\par
$\Mmeasuredangle\left(g, h\right)=$\MLParsedQuestion{10}{pi/3}{3}{M05ExAngle}.
\par
\MInputHint{Ein~$\pi$ geben Sie als \texttt{pi} ein. Sie k"onnen 
Ihr Ergebnis auch auf drei Nachkommastellen gerundet angeben.}
\par
\begin{MHint}{L\"osung}
Aus
\[
\frac{\Mmeasuredangle\left(g, h\right)}{2\pi} %
= \frac{60\MGrad}{360\MGrad} %
\]
ergibt sich
\[
 \Mmeasuredangle\left(g, h\right) %
= \frac{60\MGrad}{360\MGrad}\cdot 2\pi %
= \frac{1}{6}\cdot 2\pi=\frac{\pi}{3}\MDFPeriod 
\]
\end{MHint}

\end{MExercise}

\begin{MExercise}
Der Winkel $\beta$ betr\"agt im Bogenma"s $\pi/4$. Wie gro"s ist der 
Winkel im Gradma"s?

\par
$\beta=$\MLParsedQuestion{10}{45}{3}{PARSEDQUEST2}$\MGrad$.
\par
\begin{MHint}{L\"osung}
Aus
\[
\frac{\pi/4}{2\pi} = \frac{\beta}{360\MGrad} %%
\]
erh"alt man
\[
\beta %
 = \frac{\pi/4}{2\pi}\cdot 360\MGrad %
 = \frac{1}{8}\cdot 360\MGrad %
 = 45\MGrad \MDFPeriod %%
\]
\end{MHint}
\end{MExercise}

\begin{MExercise}
Es werden sechs Winkel $\alpha_1$ bis $\alpha_6$ betrachtet, von denen jeweils
der Wert im angegebenen Winkelma"s bekannt ist. Berechnen Sie den Wert im 
anderen genannten Winkelma"s.

\begin{center}
\begin{tabular}{l*{6}{c}}
 & $\alpha_1$ & $\alpha_2$ & $\alpha_3$ & $\alpha_4$ & $\alpha_5$ & $\alpha_6$ \\
 Bogenma"s & $\pi$               &   \MLParsedQuestion{10}{9*pi/5}{3}{GEO1} &   	$\frac{2 \pi}{3}$                &\MLParsedQuestion{10}{3*pi/2}{3}{GEO2} & $\frac{11 \pi}{12}$ & \MLParsedQuestion{10}{pi/6}{3}{GEO3} \\
 Gradma"s  & \MLParsedQuestion{10}{180}{3}{GEO4}      & $324\MGrad$    & \MLParsedQuestion{10}{120}{3}{GEO5}        &    $270\MGrad$    & \MLParsedQuestion{10}{165}{3}{GEO6} & $30\MGrad$ %%
\end{tabular}
\end{center}

\MInputHint{Ein $\pi$ geben Sie als \texttt{pi} ein. 
Sie k"onnen Ihr Ergebnis auch auf drei Nachkommastellen gerundet angeben.}

\begin{MHint}{L"osung}
Es wird der Wert eines Winkels im Bogenma"s mit $x$ und im Gradma"s mit $\alpha$
bezeichnet. Dann ist das Verh"altnis zur jeweiligen Ma"szahl des Vollwinkels 
gleich: $\frac{x}{2 \pi} = \frac{\alpha}{360\MGrad}$.
Somit ergibt sich die Umrechnung vom Bogenma"s in das Gradma"s aus
$\alpha = \frac{180\MGrad}{\pi} \cdot x$, 
und die Umrechnung vom Gradma"s in das Bogenma"s ergibt sich gem"a"s
$x = \frac{\pi}{180\MGrad} \cdot \alpha$.
F"ur die angegebenen Werte erh"alt man:

\begin{center}
\begin{tabular}{l*{6}{c}}
 & $\alpha_1$ & $\alpha_2$ & $\alpha_3$ & $\alpha_4$ & $\alpha_5$ & $\alpha_6$ \\
%\hline
 Bogenma"s &   $\pi$     & $\frac{9 \cdot \pi}{5}$ & $\frac{2 \pi}{3}$ & $\frac{3 \cdot \pi}{2}$ & $\frac{11 \pi}{12}$ & $\frac{\pi}{6}$ \\
 Gradma"s  & $180\MGrad$ &      $324\MGrad$        &    $120\MGrad$    &    $270\MGrad$        & $165\MGrad$          &   $30\MGrad$   %%
\end{tabular}
\end{center}
\end{MHint}

\end{MExercise}

\end{MXContent}
%jgl: end of section 2.



%content: section 3: Dreiecke.

\MSubsection{Rund um Dreiecke}
\MLabel{M05_Dreiecke}

\begin{MIntro}
\MDeclareSiteUXID{VBKM05_Dreiecke_Intro}

Technische Bauwerke, wie zum Beispiel Fachwerke oder manche Br"ucken, nutzen 
Dreiecke als Konstruktionselemente.

%Bruecke:
%Beispiel einer Brueckenkonstruktion mit Dreieckselementen:
\begin{center}
\MTikzAuto{%
\begin{tikzpicture}[line width=1pt]
\coordinate (A) at (-6,0);
\coordinate (B) at (6,0);
\coordinate (C) at ($(-6,0) + (60:3)$);
\coordinate (D) at ($(6,0) + (120:3)$);
%Bruecke:
\draw (-6,0) -- ++(60:3) -- ++(-60:3) -- ++(60:3) -- ++(-60:3) %
 -- ++(60:3) -- ++(-60:3) -- ++(60:3) -- ++(-60:3);
\draw[line width=2pt] (A) -- (B);
\draw[line width=2pt] (C) -- (D);
\coordinate (LA) at (-0.2,0);
\coordinate (LAA) at (-0.3,0);
\coordinate (LAU) at (-0.3,-0.1);
\coordinate (LB) at (0.2,0);
\coordinate (LBB) at (0.3,0);
\coordinate (LBU) at (0.3,-0.1);
\coordinate (LC) at ($ (LA) + (60:0.4) $);
\coordinate (LM) at ($ (LC) + (0,0.08) $);
%Lager:
\begin{scope}[xshift=-6cm,yshift=-0.5cm]
\coordinate (LA) at (-0.2,0);
\coordinate (LAA) at (-0.3,0);
\coordinate (LB) at (0.2,0);
\coordinate (LBB) at (0.3,0);
\coordinate (LC) at ($ (LA) + (60:0.4) $);
\coordinate (LM) at ($ (LC) + (0,0.08) $);
\draw (LAA) -- (LBB);
\draw (LA) -- (LC) -- (LB);
\draw (LM) circle(0.07);
\draw (-0.2,0) -- ++(-120:0.2);
\draw (-0.1,0) -- ++(-120:0.2);
\draw (0.0,0) -- ++(-120:0.2);
\draw (0.1,0) -- ++(-120:0.2);
\draw (0.2,0) -- ++(-120:0.2);
\draw (0.3,0) -- ++(-120:0.2);
\end{scope}
%
\begin{scope}[xshift=6cm,yshift=-0.5cm]
\coordinate (LA) at (-0.2,0);
\coordinate (LAA) at (-0.3,0);
\coordinate (LAU) at (-0.3,-0.1);
\coordinate (LB) at (0.2,0);
\coordinate (LBB) at (0.3,0);
\coordinate (LBU) at (0.3,-0.1);
\coordinate (LC) at ($ (LA) + (60:0.4) $);
\coordinate (LM) at ($ (LC) + (0,0.08) $);
\draw (LAA) -- (LBB);
\draw (LA) -- (LC) -- (LB);
\draw (LM) circle(0.07);
\draw (LAU) -- (LBU);
\draw (-0.2,-0.1) -- ++(-120:0.2);
\draw (-0.1,-0.1) -- ++(-120:0.2);
\draw (0.0,-0.1) -- ++(-120:0.2);
\draw (0.1,-0.1) -- ++(-120:0.2);
\draw (0.2,-0.1) -- ++(-120:0.2);
\draw (0.3,-0.1) -- ++(-120:0.2);
\end{scope}
\end{tikzpicture}
}
\end{center}

Umgekehrt kann man sich fragen, wie irgendeine Fl"ache in Dreiecke zerlegt 
werden kann. Diese Fragestellung ist f"ur viele geometrische Berechnungen
hilfreich. Einige Beispiele zeigt der Abschnitt \MRef{M05_Vielecke}. 

Au"serdem f"uhrt die Frage, wie irgendwelche Fl"achen in einfach zu 
bestimmende {\glqq}Grundformen{\grqq} zerlegt werden k"onnen, in Anwendungen 
zu konstruktiven Antworten, die weit "uber elementare geometrische 
Betrachtungen hinaus bedeutend sind. 
Einen ersten Eindruck vermittelt die Integralrechnung im Kapitel 
\MRef{VBKM08} und ihre Anwendung in der Berechnung von Fl"acheninhalten. 
Dort wird oft von einer {\glqq}n"aherungsweisen{\grqq} Zerlegung in Rechtecken 
ausgegangen (die man sich jeweils in zwei Dreiecke zerlegt denken kann, wenn 
man bei Dreiecken bleiben m"ochte). 
F"ur die dreidimensionale computerunterst"utzte Modellierung 
von Oberfl"achen von K"orpern, beispielsweise im Automobilbau von einer 
Fahrzeugkarosserie, bilden Zerlegungen in Dreiecke die Grundlage f"ur viele 
Berechnungen und t"auschend echt aussehende virtuelle Animationen.  
\end{MIntro}


\begin{MXContent}{Dreiecke}{Dreiecke}{STD}
\MDeclareSiteUXID{VBKM05_Dreiecke_Content}

Viele Aussagen "uber geometrische Figuren und K"orper ergeben sich aus 
Eigenschaften von Dreiecken, der {\glqq}einfachsten geschlossenen Figur{\grqq}, 
die durch drei Punkte bestimmt wird, die nicht auf einer Geraden liegen. 

%Eine Strecke $\MGeoStrecke{A}{B}$ ist durch ihre beiden {\glqq}Endpunkte{\grqq} 
%$A$ und $B$ bestimmt (die hier als verschieden vorausgesetzt werden). Die 
%{\glqq}einfachste Figur{\grqq}, in der neben einer L"ange, wie sie bei Strecken
%als bestimmendes Merkmal vorliegt, auch mehrere Winkel vorkommen, ergibt sich,
%wenn ein weiterer Punkt $C$ hinzugenommen wird, der nicht auf der Geraden
%\MGeoGerade{A}{B}$ liegt.

Zun"achst werden die wichtigsten Begriffe zusammengestellt, bevor Fragen 
beantwortet werden, wann Dreiecke eindeutig bestimmt sind und wie einzelne 
Seitenl"angen oder Winkel berechnet werden k"onnen. 
Hierbei sind die Strahlens"atze ein wichtiges Hilfsmittel, die auch als 
Aussagen "uber Beziehungen zwischen Dreiecken gesehen werden k"onnen.

Funktionale Beziehungen 
zwischen Seitenl"angen und Winkel werden dann im Abschnitt
\MRef{M05_Trigonometrie} betrachtet, um weitergehende Fragestellungen 
beantworten zu k"onnen, die f"ur Anwendungen relevant sind.

\begin{MXInfo}{Dreieck}\MLabel{Mathematik_ElementareGeometrie_SummeDerInnenwinkel}%
Ein \MEntry{Dreieck}{Dreieck} entsteht, wenn man drei voneinander verschiedene Punkte $A$, $B$ und $C$, 
die nicht auf einer Geraden liegen, verbindet. Dieses Dreieck wird dann mit 
$\MGeoDreieck{A}{B}{C}$ bezeichnet.
\begin{itemize}
 \item Die drei Punkte, die verbunden werden, hei"sen
       \MEntry{Ecken}{Ecken (Dreieck)} des Dreiecks, und
       die drei Verbindungslinien hei"sen
       \MEntry{Seiten}{Seiten (Dreieck)} des Dreiecks.
       
 \item Je zwei Seiten des Dreiecks bilden je zwei Winkel.
   
       Der kleinere dieser beiden Winkel hei"st
       \MEntry{Innenwinkel}{Innenwinkel} (oft kurz Winkel genannt),
       und der gr"o"sere der beiden Winkel hei"st
       \MEntry{Au"senwinkel}{Au"senwinkel}.
 
 \item Die Summe der drei Innenwinkel eines Dreiecks ist
 $180\MGrad$ beziehungsweise $\pi$.
\end{itemize}
\end{MXInfo}

\begin{tabular}{@{}lr@{}}
\begin{minipage}{10cm}
Eine oft verwendete Art, die Gr"o"sen eines Dreiecks zu bezeichnen, ist 
folgende:
Man benennt die Ecken eines Dreiecks in {\glqq}mathematisch positiver{\grqq}
Richtung (gegen den Uhrzeigersinn) mit lateinischen Gro"sbuchstaben. 
Die einer Ecke gegen\"uberliegende Seite eines Dreiecks bekommt den 
entsprechenden Kleinbuchstaben zugeordnet, und der Innenwinkel in einer Ecke 
erh"alt den entsprechenden Kleinbuchstaben des griechischen Alphabets.
\par
Da die Au"senwinkel eines Dreiecks wesentlich weniger interessant sind als
die Innenwinkel, nennt man die \textbf{Innenwinkel} eines Dreiecks auch schlicht
\MEntry{Winkel}{Winkel (Dreieck)} des Dreiecks.
\end{minipage}
&
\begin {minipage}{6cm}
%\begin{center}
\MTikzAuto{%
\begin{tikzpicture}
\coordinate[label=below left:$A$] (A) at (0,0);
\coordinate[label=right:$B$]      (B) at (4,0.5);
\coordinate[label=above:$C$]      (C) at (2,3);
\coordinate (MAB) at ($ (A)!0.5!(B) $);
\coordinate (MBC) at ($ (B)!0.5!(C) $);
\coordinate (MCA) at ($ (C)!0.5!(A) $);
%
\draw (A) -- (B) -- (C) -- cycle;
%
\path (A) -- node[near start]{$\alpha$} (MBC) node[above right]{$a$};
\path (B) -- node[near start]{$\beta$}  (MCA) node[above left] {$b$};
\path (C) -- node[near start]{$\gamma$} (MAB) node[below]      {$c$};
%
\path let \p1 = (current bounding box.east),
          \p2 = (current bounding box.west),
          \p3 = ($ (\p1) - (\p2) $),
          \n3 = {veclen(\p3)} in;
%     (current bounding box.south) node [below, text width=\n3, text centered, outer sep = 0.5\baselineskip]
 %          {Die Bezeichnungen von Ecken, Seiten und Innenwinkeln in einem Dreieck.};
\end{tikzpicture}
}
\end{minipage}
\end{tabular}

Die Summe aller (Innen-)Winkel in einem Dreieck betr"agt $180\MGrad$ 
beziehungsweise $\pi$. Somit kann h"ochstens ein Winkel gleich oder gr"o"ser 
als $90\MGrad$ beziehungsweise $\frac{\pi}{2}$ sein. Dementsprechend werden die
Dreiecke nach ihrem gr"o"sten Winkel in drei verschiedene Klassen eingeteilt:
\begin{MXInfo}{Bezeichnungen f"ur Dreiecke}%
Dreiecke werden folgenderma"sen nach ihren Winkeln benannt:
\begin{itemize}
 \item Ein Dreieck, in dem alle Winkel kleiner als~$\frac{\pi}{2}$ sind, hei"st 
  \MEntry{spitzwinklig}{Spitzwinklig (Dreieck)}.
 
 \item Ein Dreieck, das einen rechten Winkel enth"alt, hei"st 
  \MEntry{rechtwinklig}{Rechtwinklig (Dreieck)}.
 
  In einem rechtwinkligen Dreieck hei"sen die Seiten, die auf den Schenkeln des
  rechten Winkels liegen, \MEntry{Katheten}{Kathete},
  und die Seite, die dem rechten Winkel gegen"uberliegt, hei"st
  \MEntry{Hypotenuse}{Hypotenuse}.
       
 \item Ein Dreieck, das einen Winkel mit einem Ma"s von "uber~$\frac{\pi}{2}$
  besitzt, hei"st \MEntry{stumpfwinklig}{Stumpfwinklig (Dreieck)}.
\end{itemize}
\end{MXInfo}

Es wird eine einfache Konstruktion eines Wagenhebers in der Form eines
Dreiecks betrachtet: Er besteht aus zwei St"aben, die durch ein Gelenk 
miteinander verbunden sind. Die anderen Endpunkte der beiden St"abe k"onnen
zusammengezogen werden k"onnen.
Je gro"ser der Winkel eines Stabs gegen"uber der Stra"se ist, desto 
h"oher befindet sich das Gelenk "uber dem Boden.

\begin{center}
\MTikzAuto{%
\begin{tikzpicture}[line width=1pt]
\draw (0,0) -- (5,4) -- (7,0);
\draw[style=dashed] (7,0) -- (0,0);
\draw[style=dotted] (5,0) -- node[right] {$h_c$} (5,4);
\node[left] at (0,0) {$A$};
\node[right] at (7,0) {$B$};
\node[above] at (5,4) {$C$};
\node[below] at (5,0) {$D$};
\draw (5.5,0) arc(0:90:0.5);
\draw (5.2,0.2) circle(0.5pt);
\end{tikzpicture}
}
\end{center}

So wird in einem Dreieck $\MGeoDreieck{A}{B}{C}$ die k"urzeste Strecke 
zwischen der Ecke $C$ und der gegen\"uberliegenden Seite $\MGeoStrecke{A}{B}$
, die 
\MEntry{H"ohe eines Dreiecks}{H"ohe eines Dreiecks} $h_c$ auf 
die Seite $c$ genannt. Der andere Endpunkt $D$ der Strecke $h_c$ hei"st 
\MEntry{H"ohenfu"spunkt}{H"ohenfu"spunkt}.
Entsprechend werden die H"ohen $h_a$ und $h_b$ definiert.

Man kann auch sagen, dass die H"ohen diejenigen Strecken sind, die senkrecht 
auf der Geraden einer Seite stehen und bis zu einer Ecke des Dreiecks gehen.
\end{MXContent}


%content: Satz des Pythagoras.
\begin{MXContent}{Satz des Pythagoras}{Pythagoras}{STD}
\MDeclareSiteUXID{VBKM05_Pythagoras_Content}

Eine Aussage "uber die Seitenl"angen in einem rechtwinkligen Dreieck
bietet der \MEntry{Satz des Pythagoras}{Pythagoras (Satz)}. 
Dieser wird hier in einer oft verwendeten Formulierung angegeben.

\begin{MXInfo}{Satz des Pythagoras}
\MLabel{VBKM05_Pythagoras}
\begin{tabular}{@{}lr@{}}
\begin{minipage}{9cm}
Es wird ein rechtwinkliges Dreieck betrachtet, in dem der rechte Winkel bei 
$C$ liegt.
\vspace*{1cm}
\end{minipage}
&
\begin{minipage}{7cm}
\begin{center}
\MTikzAuto{%
\begin{tikzpicture}[line width=1pt]
\coordinate[label=left:$A$] (A) at (0,0);
\coordinate[label=right:$B$] (B) at ($ (A) + (4.6,0) $);
\coordinate[label=above:$C$] (C) at ($ (B) + (120:2.3) $);
%Zeichen fuer rechten Winkel:
\draw (B) ++(120:1.8) arc(300:210:0.5);
\draw (C) ++(255:0.3) circle(0.5pt);
%Dreieck:
\draw (A) -- (B) -- (C) -- cycle;
%Beschriftung des Dreiecks:
\path (A) -- node[below] {$c$} (B) %
 -- node[above right] {$a$} (C) -- node[above left] {$b$} (A);
\end{tikzpicture}
}
\end{center}
\end{minipage}
\end{tabular}

Dann ist die Summe der Quadrate \"uber den Katheten $a$ und $b$ gleich dem
Quadrat \"uber der Hypotenuse $c$. Mit den genannten Bezeichnungen gilt
somit (siehe auch das abgebildete Dreieck): 
\[
a^2 + b^2 = c^2 \MDFPeriod
\]
Werden die Seiten des Dreiecks anders bezeichnet, muss die Gleichung 
entsprechend angepasst werden!
\end{MXInfo}


\begin{MExample}
Gegeben sei ein rechtwinkliges Dreieck mit den Kathetenl\"angen $a=6$ und 
$b=8$.

Die L\"ange der Hypotenuse kann mithilfe des Satzes von Pythagoras 
berechnet werden:
\[
c = \sqrt{c^2} = \sqrt{a^2 + b^2} = \sqrt{36 + 64} = \sqrt{100} = 10 \MDFPeriod 
\]
\end{MExample}

\begin{MExercise}
Gegeben ist ein rechtwinkliges Dreieck $\MGeoDreieck{A}{B}{C}$ mit rechtem 
Winkel in $C$, der Hypotenuse $c = \frac{25}{3}$ und der H\"ohe $h_c = 4$ sowie 
der Strecke $\MGeoStrecke{D}{B}$ mit $q = \MGeoAbstand{D}{B} = 3$. Dabei 
bezeichnet $D$ den H"ohenfu"spunkt der H"ohe $h_c$.
Berechnen Sie die L\"ange der beiden Katheten $a$ und $b$. 

\begin{MHint}{L\"osung}
Es wird der Satz des Pythagoras auf das Dreieck $\MGeoDreieck{D}{B}{C}$
angewandt, das in $D$ einen rechten Winkel hat. Dann ist
\[
 a = \sqrt{h_c^2 + q^2} = \sqrt{4^2 + 3^2} = \sqrt{25} = 5 \MDFPeriod
\]
Nun wird der Satz des Pythagoras auf das gegebene rechtwinklige Dreieck
$\MGeoDreieck{A}{B}{C}$ angewandt, woraus 
\[
 b = \sqrt{c^2-a^2} = \sqrt{\left(\frac{25}{3}\right)^2-5^2} %
 = \sqrt{\frac{400}{9}} %
 = \frac{20}{3} %%
\]
folgt.
\end{MHint}
\end{MExercise}

Der \MEntry{Satz des Thales}{Thales (Satz)} ist ein weiterer wichtiger Satz, 
der eine Aussage "uber rechtwinklige Dreiecke ausdr"uckt.
\begin{MXInfo}{Satz des Thales}
\par
\begin{tabular}{@{}lr@{}}
\MTikzAuto{%
\begin{tikzpicture}[x=1.0cm, y=1.0cm] 
\draw[color=black, thick] (-3,0) -- (3,0);
\draw[color=blue, thick] (3,0) arc (0:180:3);
\draw[color=black, thick] (-3,0) -- (50:3) -- (3,0);
%\draw[color=red, thick] (-3,0) -- (100:3) -- (3,0);
\draw[color=black] (50:3) ++(295:0.6) arc (295:205:0.6);
%\draw[color=black] (100:3) ++(320:0.6) arc (320:230:0.6);
\fill[color=black] (50:3) ++(250:0.3) circle (1.0pt);
%\fill[color=black] (100:3) ++(275:0.3) circle (1.0pt);
\draw[color=black] (0,0) node[anchor=north] {$M$};
\draw[color=black] (-1.5,0) node[anchor=south] {$r$};
\draw[color=black] (1.5,0) node[anchor=south] {$r$};
\draw (0,0) -- (50:3);
%\draw (0,0) -- (100:3);
\node[anchor=north west] at (50:1.5) {$r$};
%\node[anchor=west] at (100:1.8) {$r$};
\node[left] at (-3, 0) {$A$};
\node[right] at (3, 0) {$B$};
\node[above right] at (50:3) {$C$};
\end{tikzpicture}
}
&
\begin{minipage}[b]{7cm}
Hat das Dreieck $ABC$ bei $C$ einen rechten Winkel, so liegt $C$ auf einem 
Kreis mit Radius $r$ und der Hypotenuse $\MGeoStrecke{A}{B}$ als 
Durchmesser der L"ange $2r$.
\vspace*{1.5cm}
\end{minipage}
\end{tabular}
\end{MXInfo}

Die umgekehrte Aussage gilt ebenso. Wenn man "uber einer Strecke 
$\MGeoStrecke{A}{B}$ einen Halbkreis konstruiert und dann $A$ und $B$ mit 
einem beliebigen Punkt $C$ auf dem Halbkreis verbindet, dann ist das so 
entstandene Dreieck immer rechtwinklig.

\begin{MExample}\MLabel{ThaleskreisBeispiel}%
Es soll ein rechtwinkliges Dreieck mit der Hypotenusenl\"ange $c=6\MEinheit{cm}$ 
und der H\"ohe $h_c=\MZahl{2}{5}\MEinheit{cm}$ konstruiert werden.

\begin{tabular}{@{}lr@{}}
\begin{minipage}[b]{7cm}
 \begin{enumerate}
  \item Zuerst zeichnet man die Hypotenuse \[c=\MGeoStrecke{A}{B} \MDFPeriod \]

  \item Die Mitte der Hypotenuse wird nun zum Mittelpunkt eines 
  Kreises mit dem Radius $r = c/2$.

  \item Nun zeichnet man eine Parallele zur Hypotenuse im Abstand $h_c$. 
  Es gibt zwei Schnittpunkte $C$ und $C'$ dieser Parallelen mit dem Thaleskreis. 
 \end{enumerate}
\end{minipage}
&
\MTikzAuto{%
\begin{tikzpicture}[x=1.2cm, y=1.2cm] 
\draw[color=red, thick] (-3,0) -- (3,0);
\draw[color=blue, thick] (3,0) arc (0:180:3);
\draw[color=red, thick, dashed] (-3,2.5) -- (3,2.5);
\fill[color=black, opacity=0.5] (0,0) circle (2.0pt);
\draw[color=black, thick] (-3,0) -- (-1.658312395,2.5) -- (3,0);
\draw[color=black, thick, dashed] (-3,0) -- (1.658312395,2.5) -- (3,0);
\draw[color=black] (-1.658312395,0) -- (-1.658312395,2.5);
\draw[color=gray, dashed] (1.658312395,0) -- (1.658312395,2.5);
\draw[color=black] (-3,0) node[anchor=north east] {$A$};
\draw[color=black] (3,0) node[anchor=north west] {$B$};
\draw[color=black] (0,-2pt) node[anchor=north] {$M$};
\draw[color=black] (-1.658312395,1.10) node[anchor=east] {$h_c$};
\draw[color=black] (1.658312395,1.10) node[anchor=west] {$h_c$};
\node[anchor=south east] at (-1.658312395,2.5) {$C$};
\node[anchor=south west] at (1.658312395,2.5) {$C'$};
\draw[color=red] (-1.5,0) node[anchor=north] {\large $\mathsf{1}$};
\draw[color=blue] (30:3) node[anchor=west] {\large $\mathsf{2}$};
\draw[color=red] (3,2.5) node[anchor=south east] {\large $\mathsf{3}$};
\end{tikzpicture}
}
\end{tabular}

Diese Schnittpunkte sind jeweils die dritte Ecke eines Dreiecks, das die 
geforderten Eigenschaften hat, das hei\ss t, man erh\"alt zwei L\"osungen.
W\"urde man noch einen Thaleskreis nach unten zeichen, so erg\"aben sich 
noch mal zwei L\"osungen.
Wenn es nicht um die Lage, sondern nur um die {\glqq}Form{\grqq} der Dreiecke
geht, dann sind alle diese Dreiecke {\glqq}deckungsgleich{\grqq} (siehe auch
\MRef{VBKM05_DreieckeKongruenzsaetze}). 
\end{MExample}

\begin{MExercise}
Welche H\"ohe $h_c$ kann ein rechtwinkliges Dreieck mit der Hypotenuse $c$ 
maximal haben?

\begin{MHint}{L\"osung}
Die H\"ohe $h_c$ kann maximal so gro"s werden wie der Radius des 
Thaleskreises \"uber der Hypotenuse. Es ist also $h_c \leq \frac{c}{2}$.
\end{MHint}
\end{MExercise}

%Zusatz: Hoehensatz und Kathetensatz.
\begin{MCOSHZusatz}
In einem rechtwinkligen Dreieck gelten neben dem Satz des Pythagoras weitere 
Aussagen.
Dazu werden folgende Bezeichnungen verwendet:
\par
\begin{tabular}{@{}lr@{}}
\begin{minipage}{9cm}
Es wird ein rechtwinkliges Dreieck mit rechtem Winkel bei $C$ betrachtet. 
Die H"ohe $h_c$ schneidet die Hypotenuse des Dreiecks $\MGeoDreieck{A}{B}{C}$
im Punkt $D$, dem H"ohenfu"spunkt. Weiter werden die Bezeichnungen
$p = \MGeoAbstand{A}{D}$ und $q = \MGeoAbstand{B}{D}$ vereinbart.
\vspace*{1cm}
\end{minipage}
&
\begin{minipage}{7cm}
\MTikzAuto{%
\begin{tikzpicture}
\coordinate[label=above:$C$]       (C) at (0,0);
\coordinate[label=below right:$B$] (B) at ($ (C) + (2,-4) $);
\path let \p1=($ (B) - (C) $) in 
        coordinate[label=left:$A$] (A) at ($ (C) + ({\y1*3/4}, {-\x1*3/4}) $);
\path let \p1=($ (B) - (A) $) in
        coordinate                 (K) at ($ (C) + ({\y1/5}, {- \x1/5}) $);
\coordinate[label=below:$D$]       (D) at (intersection of C--K and A--B);
%
\draw (B) -- node[sloped, above]{$a$} (C) -- node[sloped, above]{$b$} (A) -- cycle;
\draw (C) -- node[sloped, right, rotate=-90]{$h_c$} (D);
\path (A) -- node[sloped, above]{$p$} (D) -- node[sloped, above]{$q$} (B) -- node[sloped, below]{$c$} (A);
\end{tikzpicture}
}
\end{minipage}
\end{tabular}

\begin{MXInfo}{H"ohensatz}
Das Quadrat "uber der H"ohe ist fl"acheninhaltsgleich dem Rechteck aus 
den beiden Hypotenusenabschnitten: 
\[h_c^2 = p\cdot q \MDFPeriod\]	
\end{MXInfo}

\begin{MXInfo}{Kathetensatz}
Das Quadrat "uber einer Kathete ist fl"acheninhaltsgleich dem Rechteck aus 
der Hypotenuse und dem anliegenden Hypotenusenabschnitt: 
\[a^2 = c\cdot q \MDFPSpace, \MDFPaSpace b^2=c\cdot p \MDFPeriod\]
\end{MXInfo}

\begin{MExample}
Gegeben sei ein rechtwinkliges Dreieck mit den Kathetenl\"angen $a=3$ und 
$b=4$.

Die L"ange der Hypotenuse kann mithilfe des Satzes von Pythagoras berechnet 
werden:
\[
c = \sqrt{a^2 + b^2}=\sqrt{9 + 16}=\sqrt{25}=5 \MDFPeriod %%
\]
Die einzelnen Hypotenusenabschnitte $p$ und $q$ berechnen sich gem"a"s dem 
Kathetensatz zu:
\[
q=\frac{a^2}{c}=\frac{9}{5} = \MZahl{1}{8} \quad \text{und} \quad 
p=\frac{b^2}{c}=\frac{16}{5} = \MZahl{3}{2} \MDFPeriod
\]
Die H\"ohe $h_c$ erh\"alt man mit dem H\"ohensatz:
\[
h_c=\sqrt{p\cdot q}=\sqrt{\frac{9}{5}\cdot\frac{16}{5}} %
=\sqrt{\frac{144}{25}}=\frac{12}{5} = \MZahl{2}{4} \MDFPeriod\]
\end{MExample}

\begin{MExercise}
Berechnen Sie f"ur ein rechtwinkliges Dreieck mit der Hypotenuse 
$c=\MZahl{10}{5}$ und dem Hypotenusenabschnitt $q=\MZahl{3}{78}$ die L\"ange 
der beiden Katheten.

\begin{MHint}{L\"osung}
\[\text{Kathetensatz:} \quad a=\sqrt{c\cdot q}=\sqrt{\MZahl{10}{5} \cdot \MZahl{3}{78}}=\MZahl{6}{3} \MDFPSpace;\]
\[\text{Satz des Pythagoras:} \quad b=\sqrt{c^2-a^2} = \sqrt{{\MZahl{10}{5}}^2-{\MZahl{6}{3}}^2}=\MZahl{8}{4} \MDFPeriod\]
\end{MHint}
\end{MExercise}
\end{MCOSHZusatz}
%Ende Zusatz: Hoehensatz und Kathetensatz.

\end{MXContent}
%end of content: Satz des Pythagoras.


\begin{MXContent}{Kongruente und "ahnliche Dreiecke}{Kongruenz}{STD}
\MDeclareSiteUXID{VBKM05_Kongruenzsaetze_Content}

Zu einem Dreieck geh"oren unter anderem drei Seitenl"angen und drei Winkel. 
Die Au"senwinkel sind durch die Innenwinkel bereits festgelegt, sodass durch 
diese sechs Gr"o"sen die {\glqq}Form{\grqq} eines Dreiecks bestimmt ist. Wenn 
bei zwei Dreiecken alle diese Gr"o"sen "ubereinstimmen, so sind diese Dreiecke 
deckungsgleich oder \MEntry{kongruent}{kongruent (Dreieck)}. 
Dabei spielt es keine 
Rolle, wo sich die Dreiecke befinden. Kongruente Dreiecke k"onnen also durch 
Drehung, Spiegelung und Verschiebung ineinander "ubergef"uhrt werden.

Kennt man vier von den sechs Gr"o"sen, so ist das Dreieck eindeutig bestimmt 
bis auf Spielgelung oder Drehung, das hei"st bis auf die Lage des Dreiecks 
im Raum. Alle Dreiecke, die man mit diesen Angaben erh"alt, sind dann kongruent.
In einigen F"allen gen"ugen sogar drei Angaben, um das Dreieck eindeutig 
zu bestimmen.
Sie werden in den \MEntry{Kongruenzs"atzen}{Kongruenzs\"atze (Dreieck)} 
beschrieben:

\begin{MXInfo}{Kongruenzs"atze f"ur Dreiecke}%
\MLabel{VBKM05_DreieckeKongruenzsaetze}%
Ein Dreieck ist bis auf seine Lage in der Ebene eindeutig bestimmt, wenn 
eine der folgenden Situationen vorliegt:
\begin{itemize}
 \item Von den drei Winkeln und den drei Seitenl\"angen sind
       mindestens vier Angaben gegeben.
 
 \item Alle drei Seitenl\"angen sind gegeben.
       (Diesen Satz bezeichnet man gerne mit \glqq sss\grqq\ f\"ur \glqq Seite, Seite, Seite\grqq.)
 
 \item Eine Seitenl\"ange und ihre Winkel zu den anderen Seiten sind gegeben
       (\glqq wsw\grqq\ f\"ur \glqq Winkel, Seite, Winkel\grqq).
        
 \item Zwei Seitenl\"angen und der von den Seiten eingeschlossene Winkel 
       sind gegeben (\glqq sws\grqq\ f\"ur \glqq Seite, Winkel, Seite\grqq).
       
             
 \item Ein Winkel und zwei Seitenl\"angen sind so gegeben, dass nur eine der 
       Seiten auf einem Schenkel des Winkels liegt und die andere gegebene 
       Seite die l\"angere der beiden gegebenen Seiten ist.

       (Diesen Satz bezeichnet man mit \glqq Ssw\grqq\ f\"ur \glqq Seite, Seite,
        Winkel\grqq,
        wobei das gro"s geschriebene \glqq S\grqq\ signalisieren soll, dass die
        dem Winkel gegen\"uberliegende Seite die l\"angere Seite darstellt.)
\end{itemize}
\end{MXInfo}
%Referenzfehler:
%Ein Beispiel f"ur deckungsgleiche (kongruente) Dreiecke sind die L"osungen 
%in \MRef{VBKM05_BspSatzDesThales}, wo sich verschiedene Dreiecke ergaben, 
%deren Seiten aufgrund der Konstruktion so zugeordnet werden k"onnen, 
%dass diese gleich lang sind.
 
Wenn von einem Dreieck nur zwei oder drei Angaben gegeben sind, die 
keinem der oben angegebenen F"alle entsprechen, so gibt es verschiedene
Dreiecke, f"ur die die Angaben zutreffen und die nicht deckungsgleich sind.

Im Folgenden wird zuerst in einem Beispiel erl"autert, wie mit den 
Kongruenzs"atzen ein Dreieck konstruiert werden kann.
Danach wird ein Beispiel zu Dreiecken betrachtet, bei denen nur die Winkel 
gegeben sind und somit keine der obigen Bedingungen erf"ullt ist.


\begin{MExample}
\begin{tabular}{@{}lr@{}}
\begin{minipage}{10cm}
Gegeben seien die Seiten $b$ und~$c$ und der Winkel~$\alpha$.
Das Dreieck \glqq sws\grqq\ erh"alt man, indem man zun\"achst eine Seite, 
hier zum Beispiel die Seite $c$, zeichnet und an der nach der 
Bezeichnungskonvention passenden Ecke ($A$) den Winkel $\alpha$ anf"ugt.
Dann schl"agt man um diese Ecke einen Kreis, dessen Radius der
L"ange der zweiten Seite (hier $b$) entspricht. Der Schnittpunkt dieses
Kreises mit dem zweiten Schenkel des Winkels bildet die dritte
Ecke des Dreiecks ($C$).
\end{minipage}
&
\begin{minipage}{7cm}
%\begin{center}
\MTikzAuto{%
\begin{tikzpicture}
\coordinate [label=left:$A$]        (A) at (0,0);
\coordinate [label=below right:$B$] (B) at ($ (A) + (-15:3.2) $);
\coordinate [label=above:$C$]       (C) at ($ (A) + (60:2) $);
%
\draw (A) -- node[below left]{1.} (B) -- node[above right] {4.} (C) -- cycle;
\draw[dotted] (C) -- ($ (C)!-0.5!(A) $) node[below right]{2.};
\node at (A) [label=135:3., draw, dotted, circle through=(C)]{};
\end{tikzpicture}
}
\end{minipage}
\end{tabular}
\end{MExample}


\begin{MExercise}
Konstruieren Sie ein Dreieck mit einer Seite $c=5$ und den Winkeln 
$\alpha=30\MGrad$ und $\beta=120\MGrad$, wobei die oben eingef"uhrte Notation
verwendet wird.

\begin{MHint}{L\"osung}
\begin{tabular}{@{}lr@{}}
\begin{minipage}{9cm}
Man zeichnet zuerst die gegebene Strecke $c$.
Dann tr\"agt man an den beiden Enden der Strecke die zwei der Bezeichnungskonvention entsprechenden Winkel an.
Der Schnittpunkt $C$ der beiden neuen Schenkel ist die dritte Ecke des Dreiecks.
\end{minipage}
&
\begin{minipage}{7cm}
\MTikzAuto{%
\begin{tikzpicture}[scale=0.5]
\coordinate [label=left:$A$]        (A) at (0,0);
\coordinate [label=below right:$B$] (B) at ($ (A) + (10:4) $);
\coordinate [label=above left:$C$]  (C) at ($ (A) + (40:7) $);
%
\draw (A) -- node[below]{1.} (B) -- (C) -- cycle;
\draw[dotted] (C) -- ($ (C)!-0.5!(A) $) node[below right]{2.};
\draw[dotted] (C) -- ($ (C)!-0.5!(B) $) node[left]{3.};
\end{tikzpicture}
}
\end{minipage}
\end{tabular}
\end{MHint}
\end{MExercise}

\begin{MExample}\MLabel{VBKM05_BeispielAehnlichkeit}%
Gegeben seien nun die drei Winkel $\alpha=77\MGrad$, $\beta=44\MGrad$ und 
$\gamma=59\MGrad$, deren Summe $180\MGrad$ ist.
Diese Auswahl von drei Winkeln ohne Angabe zu einer Seite findet man nicht bei 
den Kongruenzs\"atzen \MRef{VBKM05_DreieckeKongruenzsaetze}. Beispiele solcher
Dreicke sind hier dargestellt:

\begin{center}
\MTikzAuto{%
\begin{tikzpicture}[x=1.0cm, y=1.0cm] 
\pgfmathparse{5*sin(44)*cos(77)/sin(59)}\let\cx=\pgfmathresult
\pgfmathparse{5*sin(44)*sin(77)/sin(59)}\let\cy=\pgfmathresult
\foreach \sx/\sy/\dsf/\ang in {0.0cm/0.0cm/1.0/-15,6.0cm/2.0cm/0.4/40,8.0cm/0.0cm/0.5/-15} {
\begin{scope}[xshift=\sx,yshift=\sy,rotate=\ang]
\coordinate (OB) at (5,0);
\coordinate (OC) at (\cx,\cy);
\coordinate (A) at (0,0);
\coordinate (B) at ($ (A)!\dsf!(OB) $);
\coordinate (C) at ($ (A)!\dsf!(OC) $);
\coordinate (MAB) at ($ (A)!0.5!(B) $);
\coordinate (MBC) at ($ (B)!0.5!(C) $);
\coordinate (MCA) at ($ (C)!0.5!(A) $);
\draw[black,thick] (A) -- (B) -- (C) -- cycle;
\pgfmathparse{0.15/\dsf}
\path (A) -- node[pos=\pgfmathresult]{$\alpha$} (MBC);
\path (B) -- node[pos=\pgfmathresult]{$\beta$}  (MCA);
\path (C) -- node[pos=\pgfmathresult]{$\gamma$} (MAB);
\end{scope}
}
\end{tikzpicture}
}
\end{center}

Es gibt sogar unendlich viele derartige Dreiecke, die die angegebenen Winkel 
haben und die nicht kongruent zueinander sind, also nicht durch Drehung oder 
Spiegelung ineinander \"ubergef\"uhrt werden k\"onnen.
\end{MExample}

Allerdings sehen diese Dreiecke irgendwie \"ahnlich aus. Solche 
\textbf{\"ahnlichen} Dreiecke erh\"alt man auch, wenn man zum Beispiel die 
Verh\"altnisse aller Seiten zueinander kennt. 
Dies ergibt sich aus den Strahlens"atzen, wie die folgende Zeichnung 
verdeutlicht: 

%Dreiecke im Strahlensatz:
\begin{center}
\MTikzAuto{%
\begin{tikzpicture}[line width=1pt]
\coordinate (A) at (-3,0);
\coordinate (B) at (9,0);
\coordinate (C) at ($(A) + (0,-1)$);
\coordinate (D) at ($(B) + (0,3)$);
\fill[color=black!50!white] (0,0) -- (3,0) -- (3,1) -- cycle;
\fill[color=black!50!white] (0,0) -- (-3,0) -- (-3,-1) -- cycle;
\fill[color=black!30!white] (3,0) -- (6,0) -- (6,2) -- (3,1) -- cycle;
\fill[color=black!15!white] (6,0) -- (9,0) -- (9,3) -- (6,2) -- cycle;
\draw (A) -- (B);
\draw (C) -- (D);
\draw (A) -- (C);
\draw (3,0) -- ++(0,1);
\draw[style=dashed] (6,0) -- ++(0,2);
\draw[style=dotted] (B) -- (D);
\node[below] at (0,0) {$A$};
\node[below] at (3,0) {$B$};
\node[below] at (6,0) {$B'$};
\node[below] at (9,0) {$B''$};
\node[above] at (-3,0) {$-B$};
\node[above] at (3,1) {$C$};
\node[above] at (6,2) {$C'$};
\node[above] at (9,3) {$C''$};
\node[below] at (-3,-1) {$-C$};
\end{tikzpicture}
}
\end{center}

\begin{MXInfo}{"Ahnlichkeitss"atze f"ur Dreiecke}%
\MLabel{VBKM05_DreieckeAehnlichkeitssaetze}
Zwei Dreiecke hei"sen zueinander \MEntry{"ahnlich}{"Ahnlich (Dreieck)}, wenn
sie
\begin{itemize}
 \item in zwei (und damit wegen der Winkelsumme in drei) 
   Winkeln "ubereinstimmen, oder
 \item in allen \textbf{Verh"altnissen} ihrer entsprechenden Seiten 
   \"ubereinstimmen, oder
 \item in einem Winkel und im \textbf{Verh"altnis} der anliegenden 
   Seiten "ubereinstimmen, oder
 \item im \textbf{Verh"altnis} zweier Seiten und im Gegenwinkel der 
   gr"o"seren Seite "ubereinstimmen.
\end{itemize}
\end{MXInfo}

Eine Besonderheit gibt es bei dem rechten und dem linken Dreieck in Beispiel
\MRef{VBKM05_BeispielAehnlichkeit}: 
Hier geht das eine Dreieck durch zentrische Streckung 
\MLabel{Mathematik_ElementareGeometrie_zentrischeStreckung} mit dem 
Streckzentrum $S$ und einem Streckfaktor $k$ in das andere \"uber.

\begin{center}
\MTikzAuto{%
\def\sxyc{0.8cm}
\begin{tikzpicture}[x=\sxyc, y=\sxyc] 
\pgfmathparse{5*sin(44)*cos(77)/sin(59)}\let\cx=\pgfmathresult
\pgfmathparse{5*sin(44)*sin(77)/sin(59)}\let\cy=\pgfmathresult
\pgfmathparse{cos(15)}\let\rc=\pgfmathresult
\pgfmathparse{sin(15)}\let\rs=\pgfmathresult
\foreach \ax/\ay/\lsf/\rsf in {\cx/\cy/-1/14,5/0/-2/15,0/0/-0.5/13.8} {
  \pgfmathparse{\lsf+(12-\lsf)/12*(\ax*\rc+\ay*\rs)}\let\cax\pgfmathresult
  \pgfmathparse{(12-\lsf)/12*(-\ax*\rs+\ay*\rc)}\let\cay\pgfmathresult
  \pgfmathparse{\rsf+(12-\rsf)/12*(\ax*\rc+\ay*\rs)}\let\cbx\pgfmathresult
  \pgfmathparse{(12-\rsf)/12*(-\ax*\rs+\ay*\rc)}\let\cby\pgfmathresult
  \draw[gray, dashed] (\cax,\cay) -- (\cbx,\cby);
}
\node[anchor=south] at (12,0) {$S$};
\foreach \sx/\sy/\dsf/\ang in {0.0cm/0.0cm/1.0/-15,4.8*\sxyc/0.0cm/0.60/-15} {
  \begin{scope}[xshift=\sx,yshift=\sy,rotate=\ang]
  \coordinate (OB) at (5,0);
  \coordinate (OC) at (\cx,\cy);
  \coordinate (A) at (0,0);
  \coordinate (B) at ($ (A)!\dsf!(OB) $);
  \coordinate (C) at ($ (A)!\dsf!(OC) $);
  \coordinate (MAB) at ($ (A)!0.5!(B) $);
  \coordinate (MBC) at ($ (B)!0.5!(C) $);
  \coordinate (MCA) at ($ (C)!0.5!(A) $);
  \draw[black,thick] (A) -- (B) -- (C) -- cycle;
  \pgfmathparse{0.15/\dsf}
  \path (A) -- node[pos=\pgfmathresult]{$\alpha$} (MBC);
  \path (B) -- node[pos=\pgfmathresult]{$\beta$}  (MCA);
  \path (C) -- node[pos=\pgfmathresult]{$\gamma$} (MAB);
  \end{scope}
}
\end{tikzpicture}
}
\end{center}

\end{MXContent}


\begin{MExercises}
\MDeclareSiteUXID{VBKM05_Kongruenzsaetze_Exercises}

\begin{MExercise}
Untersuchen Sie die folgende Figur auf Stufenwinkel und Wechselwinkel!

\begin{center}
\MTikzAuto{%
\begin{tikzpicture}
\coordinate (A) at (0,0);
\coordinate (B) at ($ (A) + ( 00:3) $);
\coordinate (C) at ($ (B) + ( 60:3) $);
\coordinate (D) at ($ (C) + (120:3) $);
\coordinate (E) at ($ (D) + (180:3) $);
\coordinate (F) at ($ (E) + (240:3) $);
\coordinate (AB) at (intersection of A--C and B--F);
\coordinate (BC) at (intersection of B--D and C--A);
\coordinate (CD) at (intersection of C--E and D--B);
\coordinate (DE) at (intersection of D--F and E--C);
\coordinate (EF) at (intersection of E--A and F--D);
\coordinate (FA) at (intersection of F--B and A--E);
%
\draw (A) -- (C) -- (E) -- cycle;
\draw (B) -- (D) -- (F) -- cycle;
%
\draw (AB) -- (DE);
\draw (BC) -- (EF);
\draw (CD) -- (FA);
\end{tikzpicture}
}
\end{center}

\begin{MHint}{L\"osung}
\begin{tabular}{@{}lr@{}}
\begin{minipage}[b]{9cm}
Die Winkel $\alpha$ und $\alpha'$ zum Beispiel sind Stufenwinkel, ebenso 
$\beta$ und $\beta'$.
\par
Die Winkel $\alpha'$ und $\beta$ zum Beispiel sind Wechselwinkel, ebenso 
$\alpha$ und $\beta'$.
\end{minipage}
&
\MTikzAuto{%
\begin{tikzpicture}
\coordinate (A) at (0,0);
\coordinate (B) at ($ (A) + ( 00:3) $);
\coordinate (C) at ($ (B) + ( 60:3) $);
\coordinate (D) at ($ (C) + (120:3) $);
\coordinate (E) at ($ (D) + (180:3) $);
\coordinate (F) at ($ (E) + (240:3) $);
\coordinate (AB) at (intersection of A--C and B--F);
\coordinate (BC) at (intersection of B--D and C--A);
\coordinate (CD) at (intersection of C--E and D--B);
\coordinate (DE) at (intersection of D--F and E--C);
\coordinate (EF) at (intersection of E--A and F--D);
\coordinate (FA) at (intersection of F--B and A--E);
%
\draw (A) -- (C) -- (E) -- cycle;
\draw (B) -- (D) -- (F) -- cycle;
%
\draw (AB) -- (DE);
\draw (BC) -- (EF);
\draw (CD) -- (FA);
\draw[color=black, thin] (DE) ++(-30:0.65) arc (-30:30:0.65);
\draw[color=black] (DE) ++(0:0.45) node {\small $\alpha'$};
\draw[color=black, thin] (DE) ++(150:0.65) arc (150:210:0.65);
\draw[color=black] (DE) ++(0:-0.45) node {\small $\beta'$};
\draw[color=black, thin] (EF) ++(-30:0.65) arc (-30:30:0.65);
\draw[color=black] (EF) ++(0:0.45) node {\small $\alpha$};
\draw[color=black, thin] (CD) ++(150:0.65) arc (150:210:0.65);
\draw[color=black] (CD) ++(0:-0.45) node {\small $\beta$};
\end{tikzpicture}
}
\end{tabular}

\end{MHint}
\end{MExercise}


\begin{MExercise}
Zeigen Sie mithilfe von Wechselwinkeln, dass die Summe der (Innen-)Winkel in 
einem Dreieck stets $\pi$ beziehungsweise $180\MGrad$ betr\"agt.

\begin{MHint}{Tipp}
Zeichnen Sie zu einer Seite des Dreiecks eine parallele Gerade, die durch
die dritte Ecke verl"auft, und betrachten Sie die Winkel an dieser Ecke.
\end{MHint}

\begin{MHint}{L\"osung}
\begin{tabular}{@{}lr@{}}
\MTikzAuto{%
\begin{tikzpicture}[x=1.0cm, y=1.0cm, scale=0.8] 
%%\draw[help lines, gray!50, xstep=0.5, ystep=0.5] (0,0) grid (9,8);
\draw[color=black] (1,0)--(9,4) (0.5,3.5)--(7.5,7.0);
\draw[color=black, very thick] (2,0.5) -- (7.5,3.25) -- (4.0,5.25) -- cycle;
\draw[color=black, thin] (2,0.5) ++(26.5660:1.2) arc (26.5650:67.1663:1.2);
\draw[color=black] (2,0.5) ++(46.865:0.8) node {\large $\alpha$};
\draw[color=black, thin] (7.5,3.25) ++(150.255:1.2) arc (150.255:205.565:1.2);
\draw[color=black] (7.5,3.25) ++(177.910:0.8) node {\large $\beta$};
\draw[color=black, thin] (4.0,5.25) ++(247.1663:0.9) arc (247.1663:330.255:0.9);
\draw[color=black] (4.0,5.25) ++(288.7107:0.6) node {\large $\gamma$};
\draw[color=black, thin] (4.0,5.25) ++(206.5660:1.2) arc (206.5650:247.1663:1.2);
\draw[color=black] (4.0,5.25) ++(226.865:0.8) node {\large $\alpha'$};
\draw[color=black, thin] (4.0,5.25) ++(-29.745:1.2) arc (-29.745:26.5660:1.2);
\draw[color=black] (4.0,5.25) ++(-1.5895:0.8) node {\large $\beta'$};
\draw[color=black] (5.75,4.25) node[anchor=south west] {\large $a$};
\draw[color=black] (3.0,2.875) node[anchor=south east] {\large $b$};
\draw[color=black] (4.75,1.875) node[anchor=north west] {\large $c$};
\end{tikzpicture}
}
&
\begin{minipage}[b]{9cm}
Zeichnet man parallel zur Seite $c$ eine Gerade durch die obere Ecke des 
Dreiecks, so erh"alt man jeweils einen Wechselwinkel $\alpha'$ zu $\alpha$ 
und $\beta'$ zu $\beta$. 

An der Geraden gilt
\[
\alpha'+\gamma+\beta' = \pi\MDFPeriod %%
\]
Weiter ist $\alpha'=\alpha$ und $\beta'=\beta$. Damit folgt 
$\alpha+\gamma+\beta=\pi$.
\end{minipage}
\end{tabular}
\end{MHint} 
\end{MExercise}
\end{MExercises}

%jgl: end of section 3.


%content: section 4: Vielecke, Flaechen und Umfang.

\MSubsection{Vielecke, Fl\"acheninhalt und Umfang}
\MLabel{M05_Vielecke}

\begin{MIntro}
\MDeclareSiteUXID{VBKM05_Flaecheninhalt_Intro}
In der Natur kann man vielf"altige Formen entdecken. Dabei sind runde Formen
sehr offensichtlich.  Wenn es darum geht, eine Fl"ache l"uckenlos auszuf"ullen, 
entdeckt man auch Begrenzungen, die n"aherungsweise Strecken sind. Ein markantes
Beispiel sind Wabenstrukturen, die von Insekten angelegt werden.
Technische Anwendungen bauen oft auf Figuren mit geraden Begrenzungen auf.

In diesem Abschnitt werden einige Spezialf"alle von Vielecken betrachtet, mit
denen geradlinig begrenzte Fl"achen beschrieben werden k"onnen. 
Dabei sollen zun"achst charakteristische Besonderheiten benannt werden.
Anschlie"send wird der Frage nachgegangen, wie der Fl"acheninhalt eines 
Vielecks einfach berechnet werden kann. 
\end{MIntro}


\begin{MXContent}{Vierecke}{Vierecke}{STD}
\MDeclareSiteUXID{VBKM05_Vierecke_Content}

Im vorherigen Abschnitt \MRef{M05_Dreiecke} wurden Dreiecke betrachtet. 
Dazu wurde von drei Punkten ausgegangen, die nicht auf einer Geraden liegen. 
Will man all diese drei Punkte durch alle m\"oglichen Kombinationen
von 3 Strecken (zwischen jeweils zwei der drei Punkte) verbinden,
ergibt sich immer ein geschlossener Weg. 
Dabei ist jeder gegebene Punkt immer eine Schnittstelle von genau zwei Strecken.
Au"serdem gibt es keine Kreuzungspunkte.

Bei mehr als drei Punkten ist dies nicht immer der Fall. Bereits vier Punkte 
k"onnen so verbunden werden, dass sich die Verbindungsstrecken kreuzen oder 
dass sich mehrere geschlossene Wege ergeben.

Hier sollen die Strecken alle gegebenen Punkte durch einen einzigen 
geschlossenen Weg miteinander verbinden, der kreuzungsfrei verl"auft.

%Verschiedene Vierecke:
\begin{center}
\MTikzAuto{%
\begin{tikzpicture}[line width=1.5pt]
\begin{scope}[xshift=-5cm]
\draw[dotted,color=blue] (-0.4,0) -- (2,0);
%\draw (0,2) -- (-1.6,-1) -- (0,0.2) -- (1.6,-1) -- cycle;
\draw (2,0) -- (-1,-1.6) -- (-0.4,0) -- (-1,1.6) -- cycle;
\end{scope}
%%
\begin{scope}[xshift=0cm]
\draw[dotted,color=blue] (-0.8,-1.2) -- (1.6,1.2);
\draw (1.6,1.2) -- (-1.6,1.2) -- (-0.8,-1.2) -- (0.8,-1.2) -- cycle;
\end{scope}
%%
\begin{scope}[xshift=5cm]
\draw[dotted,color=blue] (0:1.2cm) -- (180:1.2cm);
\draw (0:1.2cm) -- (90:1.2cm) -- (180:1.2cm) -- (270:1.2cm) -- cycle;
\end{scope}
\end{tikzpicture}
}
\end{center}

Offenbar kann ein Viereck in zwei Dreiecke zerlegt werden. Im Allgemeinen erh"alt
man zwei Dreiecke, wenn man eine Ecke mit dem gr"o"sten Winkel mit der 
gegen"uberliegenden Ecke durch eine Strecke verbindet. Eine solche Strecke 
zwischen Ecken, die nicht miteinander verbunden sind, hei"st eine 
\MEntry{Diagonale}{Diagonale (Viereck)} des Vierecks. Aus der Kenntnis, dass 
die Summe der (Innen-)Winkel eines Dreicks $\pi$ beziehungsweise $180\MGrad$ ist,
folgt damit, dass die Summe der (Innen-)Winkel eines Vierecks doppelt so gro"s
ist, also $2 \pi$ beziehungsweise $360\MGrad$ betr"agt.

\begin{MXInfo}{Vierecke}\MLabel{VBKM05_Vierecke}%
Es werden \MEntry{Vierecke}{Viereck} betrachtet, die sich dadurch ergeben,
dass vier gegebene Punkte so durch Strecken miteinander verbunden werden, dass 
ein einziger geschlossener, kreuzungsfreier Weg durch alle vier Punkte entsteht.
Dabei sollen je drei der gegebenen Punkte, die durch zwei Strecken miteinander 
verbunden sind, nicht auf einer Geraden liegen.

Wie bei Dreiecken werden auch bei Vierecken die Innenwinkel kurz als Winkel 
bezeichnet. In den Situationen, in denen neben Innenwinkeln auch Au"senwinkel 
betrachtet werden, ist es hilfreich, jeweils ausdr"ucklich anzugeben, ob ein
Innenwinkel oder ein Au"senwinkel betrachtet wird.
\end{MXInfo}

Genauso wie Dreiecke, werden auch Vierecke in vielf"altiger Weise 
in der technischen Konstruktionen verwendet. Dadurch sind weitere allt"agliche Namen 
gebr"auchlich, die verschiedene Klassen von Vierecken bezeichnen.

Wie bei Dreiecken werden auch Vierecke nach der L"ange von Seiten oder nach der 
Gr"o"se von Winkeln eingeteilt. Dabei erkennt man typische Unterschiede zu 
Dreiecken. Anders als dort kann es bei Vierecken beispielsweise parallele Seiten 
geben. Au"serdem ist es m"oglich, dass es mehr als eine Ecke mit einem rechten 
Winkel gibt.

\begin{MXInfo}{Besondere Klassen von Vierecken}\MLabel{VBKM05_ViereckeKlassen}%
F"ur Vierecke mit den folgenden weiteren Eigenschaften werden eigene Begriffe 
eingef"uhrt: Ein Viereck hei"st
\begin{itemize}
\item \MEntry{Trapez}{Trapez}, falls wenigstens zwei Seiten parallel sind;
\item \MEntry{Parallelogramm}{Parallelogramm}, falls alle beiden zwei 
 gegen\"uberliegende Seiten parallel sind;
\item \MEntry{Raute}{Raute} oder 
 \MEntry{gleichseitiges Viereck}{gleichseitiges Viereck} oder 
 auch \MEntry{Rhombus}{Rhombus}, falls alle vier Seiten gleich lang sind;
\item \MEntry{Rechteck}{Rechteck}, falls alle vier (Innen-)Winkel rechte 
 Winkel sind;
\item \MEntry{Quadrat}{Quadrat}, falls es ein Rechteck ist, bei dem alle 
 Seiten gleich lang sind;
\item \MEntry{Einheitsquadrat}{Einheitsquadrat}, falls es ein Quadrat mit 
 Seitenl\"ange $1$ ist.
\end{itemize}
\end{MXInfo}
F"ur das Einheitsquadrat muss also auch ein L"angenma"sstab vereinbart werden.

%Rechteck:
\begin{center}
\MTikzAuto{%
\begin{tikzpicture}[line width=1pt]
%\coordinate (A) at (0,0);
%Rechteck:
\begin{scope}[xshift=-6.6cm]
\draw (0,0) -- ++(2.4,0) -- ++(0,1.8) -- ++(-2.4,0) -- cycle;
\node at (1.2,-1) {Rechteck};
\end{scope}
%
%Trapez:
\begin{scope}[xshift=-3.2cm]
\draw (0,0) -- ++(2.4,0) -- ++(-0.3,1.8) -- ++(-1.2,0) -- cycle;
\node at (1.2,-1) {Trapez};
\end{scope}
%
%Parallelogramm:
\begin{scope}[xshift=0cm]
\draw (0,0) -- ++(2.4,0) -- ++(0.4,1.8) -- ++(-2.4,0) -- cycle;
\node at (1.3,-1) {Parallelogramm};
\end{scope}
%
%Raute:
\begin{scope}[xshift=4.6cm]
\draw (0,-0.2) -- (0.8,1) -- (0,2.2) -- (-0.8,1) -- cycle;
\node at (0,-1) {Raute};
\end{scope}
%
%Quadrat:
\begin{scope}[xshift=6.6cm]
\draw (0,0) -- ++(1.8,0) -- ++(0,1.8) -- ++(-1.8,0) -- cycle;
\node at (1,-1) {Quadrat};
\end{scope}
\end{tikzpicture}
}
\end{center}


Unter den gerade eingef"uhrten speziellen Vierecken gibt es eine ganze 
Reihe von Zusammenh"angen:

\begin{MXInfo}{Beziehungen zwischen Vierecken}\MLabel{VBKM05_Vierecke_Beziehungen}%
Es gelten unter anderem folgende Beziehungen zwischen Vierecken:
\begin{itemize}
 \item Jedes Quadrat ist ein Rechteck.
\item Jedes Quadrat ist eine Raute.
\item Jede Raute ist ein Parallelogramm.
\item Jedes Rechteck ist ein Parallelogramm.
\item Jedes Parallelogramm ist ein Trapez.
\end{itemize}
\end{MXInfo}

Diese Vierecke k"onnen auf vielerlei Weisen anhand von Eigenschaften der 
Seiten, Winkel oder auch Diagonalen charakterisiert werden.

\begin{MXInfo}{Parallelogramm}\MLabel{VBKM05_Parallelogramm}%
\begin{tabular}{@{}lr@{}}
\begin{minipage}{9.6cm}
Ein Viereck ist genau dann ein Parallelogramm, wenn
\begin{itemize}
\item gegen"uberliegende Seiten parallel sind;
%jgl: Es werden hier nur einfache Vierecke betrachtet:
%\item es ein einfaches Viereck ist, dessen gegen"uberliegenden Seiten 
\item gegen"uberliegende Seiten jeweils gleich lang sind;
\item gegen"uberliegende Winkel gleich gro"s sind;
\item zwei benachbarte Winkel zusammen $\pi$ bzw. $180\MGrad$ ergeben; 
\item die Diagonalen einander halbieren.
\end{itemize}
\end{minipage}
&
\begin{minipage}{6cm}
%Parallelogramm:
\begin{center}
\MTikzAuto{%
\begin{tikzpicture}[line width=2pt]
\begin{scope}[yshift=1.8cm]
\coordinate (A) at (0,0);
\coordinate (B) at ($ (A) + (10:4.5cm) $);
\coordinate (D) at ($ (A) + (60:1.6cm) $);
\coordinate (C) at ($ (D) + (A)!1!(B) $);
%
%\draw[color=blue] (A) -- (B) -- (C) -- (D) -- cycle;
\draw[color=blue] (A) -- (B);
\draw[color=blue] (C) -- (D);
\draw[color=green] (B) -- (C);
\draw[color=green] (D) -- (A);
\foreach \Punkt in {(A), (B), (C), (D)} do
\filldraw \Punkt circle(2pt);
\end{scope}
\begin{scope}[yshift=0cm]
\coordinate (A) at (0,0);
\coordinate (B) at ($ (A) + (10:4.5cm) $);
\coordinate (D) at ($ (A) + (60:1.6cm) $);
\coordinate (C) at ($ (D) + (A)!1!(B) $);
%
\draw[color=red] ($ (A)!0.2!(B) $) arc(10:60:0.9cm);
%\draw[color=red] ($ (A) + (35:0.3cm) $) circle(1pt);
\draw[color=red] ($ (C)!0.2!(D) $) arc(190:240:0.9cm);
%\draw[color=red] ($ (C) + (215:0.3cm) $) circle(1pt);
\draw[color=red!50!yellow] ($ (B)!0.2!(C) $) arc(60:190:0.32cm);
%\draw[color=red] ($ (A) + (35:0.3cm) $) circle(1pt);
\draw[color=red!50!yellow] ($ (D)!0.2!(A) $) arc(240:370:0.32cm);
%\draw[color=red] ($ (C) + (215:0.3cm) $) circle(1pt);
\draw[color=black!50!white] (A) -- (B) -- (C) -- (D) -- cycle;
%\draw[color=blue] (A) -- (B);
%\draw[color=blue] (C) -- (D);
%\draw[color=green] (B) -- (C);
%\draw[color=green] (D) -- (A);
\foreach \Punkt in {(A), (B), (C), (D)} do
\filldraw \Punkt circle(2pt);
\end{scope}
\begin{scope}[yshift=-1.8cm]
\coordinate (A) at (0,0);
\coordinate (B) at ($ (A) + (10:4.5cm) $);
\coordinate (D) at ($ (A) + (60:1.6cm) $);
\coordinate (C) at ($ (D) + (A)!1!(B) $);
\coordinate (S) at ($ (A)!0.5!(C) $);
%
\draw[color=black!60!white] (A) -- (B) -- (C) -- (D) -- cycle;
\draw[color=blue] (A) -- ($ (A)!0.5!(C) $);
%
\draw[color=blue] (A) -- (S);
\draw[color=blue!60!white] (S) -- (C);
%
\draw[color=green!60!black] (D) -- (S);
\draw[color=green!60!white] (S) -- (B);
\foreach \Punkt in {(A), (B), (C), (D)} do
\filldraw \Punkt circle(2pt);
\end{scope}
\end{tikzpicture}
}
\end{center}
\end{minipage}
\end{tabular}
\end{MXInfo}

Rauten k"onnen als besondere Parallelogramme beschrieben werden.

\begin{MXInfo}{Raute}\MLabel{VBKM05_Raute}%
\begin{tabular}{@{}lr@{}}
\begin{minipage}{9.6cm}
Ein Viereck ist genau dann eine Raute, wenn
\begin{itemize}
\item alle Seiten gleich lang sind;
\item es ein Parallelogramm ist, in welchem die Diagonalen senkrecht 
 zueinander stehen;
%jgl: Folgende Aussage gilt in jedem(!) Parallelogramm:
%\item es ein Parallelogramm ist, in welchem die Diagonalen einander halbieren.
%jgl: neue Charakterisierung:
\item wenigstens zwei benachbarte Seiten gleich lang sind und sich die 
Diagonalen einander halbieren.
\end{itemize}
\end{minipage}
&
\begin{minipage}{6cm}
%Raute:
\begin{center}
\MTikzAuto{%
\begin{tikzpicture}[line width=2pt]
\begin{scope}[yshift=1.8cm]
\coordinate (A) at (0,0);
\coordinate (B) at ($ (A) + (10:2.5cm) $);
\coordinate (D) at ($ (A) + (40:2.5cm) $);
\coordinate (C) at ($ (D) + (A)!1!(B) $);
%
\draw[color=blue] (A) -- (B) -- (C) -- (D) -- cycle;
\foreach \Punkt in {(A), (B), (C), (D)} do
\filldraw \Punkt circle(2pt);
\end{scope}
%%
\begin{scope}[yshift=0cm]
\coordinate (A) at (0,0);
\coordinate (B) at ($ (A) + (10:2.5cm) $);
\coordinate (D) at ($ (A) + (40:2.5cm) $);
\coordinate (C) at ($ (D) + (A)!1!(B) $);
\coordinate (S) at ($ (A)!0.5!(C) $);
%
\draw[color=red] ($ (S) + (25:0.4cm) $) arc(25:110:0.4cm);
\filldraw[color=red] ($ (S) + (70:0.2cm) $) circle(0.3pt);
%
\draw[color=black] (A) -- (B);
\draw[color=black] (D) -- (C);
\draw[color=black!60!white] (B) -- (C);
\draw[color=black!60!white] (A) -- (D);
%
\draw[color=blue] (A) -- (C);
\draw[color=green] (D) -- (B);
\foreach \Punkt in {(A), (B), (C), (D)} do
\filldraw \Punkt circle(2pt);
\end{scope}
%%
\begin{scope}[yshift=-1.8cm]
\coordinate (A) at (0,0);
\coordinate (B) at ($ (A) + (10:2.5cm) $);
\coordinate (D) at ($ (A) + (40:2.5cm) $);
\coordinate (C) at ($ (D) + (A)!1!(B) $);
\coordinate (S) at ($ (A)!0.5!(C) $);
%
\draw[color=black] (D) -- (A) -- (B);
\draw[color=black!60!white] (B) -- (C);
\draw[color=black!40!white] (C) -- (D);
%
%\draw[color=blue] (A) -- ($ (A)!0.5!(C) $);
\draw[color=blue] (A) -- (S);
\draw[color=blue!60!white] (S) -- (C);
\draw[color=green!50!black] (D) -- (S);
\draw[color=green] (S) -- (B);
\foreach \Punkt in {(A), (B), (C), (D)} do
\filldraw \Punkt circle(2pt);
\end{scope}
\end{tikzpicture}
}
\end{center}
\end{minipage}
\end{tabular}
\end{MXInfo}

Bei Rechtecken denkt man oft an rechte Winkel, wie es ihr Name andeutet.
Dar"uber hinaus k"onnen Rechtecke mit Hilfe von Eigenschaften ihrer 
Diagonalen einfach beschrieben werden.

\begin{MXInfo}{Rechteck}\MLabel{VBKM05_Rechteck}%
\begin{tabular}{@{}lr@{}}
\begin{minipage}{9.6cm}
Ein Viereck ist genau dann ein Rechteck, wenn
\begin{itemize}
\item alle Winkel gleich gro"s sind;
\item es ein Parallelogramm ist, bei dem wenigstens ein Winkel ein 
 rechter Winkel ist;
\item es ein Parallelogramm ist, bei dem die Diagonalen gleich lang sind; 
\item die Diagonalen einander halbieren und gleich lang sind;
%jgl: Folgende Aussage beschreibt ein spezielles(!) Recheck: ein Quadrat, sodass
%die Aequivalenz hier falsch ist:
%\item es eine Raute mit gleich langen Diagonalen ist;
\item die Diagonalen einander halbieren und wenigstens ein Winkel 
 ein rechter Winkel ist.
\end{itemize}
\end{minipage}
&
\begin{minipage}{6cm}
%Rechteck:
\begin{center}
\MTikzAuto{%
\begin{tikzpicture}[line width=2pt]
\begin{scope}[yshift=1.6cm]
\coordinate (A) at (0,0);
\coordinate (B) at ($ (A) + (10:5cm) $);
\coordinate (D) at ($ (A) + (100:2.5cm) $);
\coordinate (C) at ($ (D) + (A)!1!(B) $);
%
\draw[color=red] ($ (A)!0.1!(B) $) arc(10:100:0.5cm);
\draw[color=red] ($ (B)!0.2!(C) $) arc(100:190:0.5cm);
\draw[color=red] ($ (C)!0.1!(D) $) arc(190:280:0.5cm);
\draw[color=red] ($ (D)!0.2!(A) $) arc(280:370:0.5cm);
\filldraw[color=red!50!yellow] ($ (A) + (55:0.3cm) $) circle(0.3pt);
%
%\draw[color=red] ($ (A) + (10:0.6cm) $) arc(10:100:0.6cm);
%\draw[color=red] ($ (A) + (55:0.3cm) $) circle(0.4pt);
%
\draw[color=blue] (A) -- (B);
\draw[color=blue] (C) -- (D);
\draw[color=blue!60!white] (A) -- (D);
\draw[color=blue!60!white] (B) -- (C);
\foreach \Punkt in {(A), (B), (C), (D)} do
\filldraw \Punkt circle(2pt);
\end{scope}
%%
\begin{scope}[yshift=-1.6cm]
\coordinate (A) at (0,0);
\coordinate (B) at ($ (A) + (10:5cm) $);
\coordinate (D) at ($ (A) + (100:2.5cm) $);
\coordinate (C) at ($ (D) + (A)!1!(B) $);
\coordinate (S) at ($ (A)!0.5!(C) $);
%
\draw[color=red!50!yellow] ($ (A)!0.1!(B) $) arc(10:100:0.5cm);
\filldraw[color=red!50!yellow] ($ (A) + (70:0.3cm) $) circle(0.3pt);
%
\draw[color=black!60!white] (A) -- (B) -- (C) -- (D) -- cycle;
%\draw[color=blue] (A) -- ($ (A)!0.5!(C) $);
%
\draw[color=blue] (A) -- (S);
\draw[color=blue!80!white] (S) -- (C);
\draw[color=blue!60!white] (D) -- (S);
\draw[color=blue!30!white] (S) -- (B);
\foreach \Punkt in {(A), (B), (C), (D)} do
\filldraw \Punkt circle(2pt);
\end{scope}
\end{tikzpicture}
}
\end{center}
\end{minipage}
\end{tabular}
\end{MXInfo}

Quadrate sind sowohl besondere Recktecke als auch spezielle Rauten.

\begin{MXInfo}{Quadrat}\MLabel{VBKM05_Quadrat}%
\begin{tabular}{@{}lr@{}}
\begin{minipage}{9.6cm}
Ein Viereck ist genau dann ein Quadrat, wenn
\begin{itemize}
\item alle Seiten gleich lang sind und 
 \begin{itemize}
 \item alle Winkel gleich gro"s sind oder 
 \item wenigstens ein Winkel ein rechter Winkel ist;
 \end{itemize}
\item die Diagonalen gleich lang sind und zudem 
alle Seiten gleich lang sind;
%jgl: Aussage falsch (auch von Rechtecken erf"ullt):
%\item es ein Parallelogramm ist, in welchem die Diagonalen einander 
% halbieren (jgl: dies gilt in jedem Parallelogramm) und 
% \begin{itemize}
%  \item alle Winkel gleich gro"s sind oder
%  \item welches wenigstens einen rechten Winkel besitzt;
%\end{itemize}
%jgl: Vorschlag fuer eine weitere Charakerisierung:
\item die Diagonalen senkrecht zueinander stehen und 
 \begin{itemize}
  \item sich halbieren und gleich lang sind oder
  \item alle Winkel gleich gro"s sind;
 \end{itemize}
\item es eine Raute mit gleich langen Diagonalen ist;
\item es sowohl eine Raute als auch ein Rechteck ist.
\end{itemize}
\end{minipage}
&
\begin{minipage}{6cm}
%Quadrat:
\begin{center}
\MTikzAuto{%
\begin{tikzpicture}[line width=2pt]
\begin{scope}[yshift=3.0cm]
\coordinate (A) at (0,0);
\coordinate (B) at ($ (A) + (10:2.5cm) $);
\coordinate (D) at ($ (A) + (100:2.5cm) $);
\coordinate (C) at ($ (D) + (A)!1!(B) $);
%
\draw[color=red] ($ (A)!0.2!(B) $) arc(10:100:0.5cm);
\draw[color=red] ($ (B)!0.2!(C) $) arc(100:190:0.5cm);
\draw[color=red] ($ (C)!0.2!(D) $) arc(190:280:0.5cm);
\draw[color=red] ($ (D)!0.2!(A) $) arc(280:370:0.5cm);
\filldraw[color=red!50!yellow] ($ (A) + (55:0.3cm) $) circle(0.3pt);
%
\draw[color=blue] (A) -- (B) -- (C) -- (D) -- cycle;
\foreach \Punkt in {(A), (B), (C), (D)} do
\filldraw \Punkt circle(2pt);
\end{scope}
%%
\begin{scope}[yshift=0cm]
\coordinate (A) at (0,0);
\coordinate (B) at ($ (A) + (10:2.5cm) $);
\coordinate (D) at ($ (A) + (100:2.5cm) $);
\coordinate (C) at ($ (D) + (A)!1!(B) $);
%\coordinate (S) at ($ (A)!0.5!(C) $);
%
\draw[color=green!50!black] (A) -- (B) -- (C) -- (D) -- cycle;
%
\draw[color=blue] (A) -- (C);
\draw[color=blue] (D) -- (B);
\foreach \Punkt in {(A), (B), (C), (D)} do
\filldraw \Punkt circle(2pt);
\end{scope}
%%
\begin{scope}[yshift=-3.0cm]
\coordinate (A) at (0,0);
\coordinate (B) at ($ (A) + (10:2.5cm) $);
\coordinate (D) at ($ (A) + (100:2.5cm) $);
\coordinate (C) at ($ (D) + (A)!1!(B) $);
\coordinate (S) at ($ (A)!0.5!(C) $);
%
\draw[color=red] ($ (S) + (55:0.5cm) $) arc(55:145:0.5cm);
\filldraw[color=red] ($ (S) + (100:0.25cm) $) circle(0.3pt);
%
%\draw[color=red!60!yellow] ($ (A) + (10:0.6cm) $) arc(10:100:0.6cm);
\draw[color=red!60!yellow] ($ (A)!0.2!(B) $) arc(10:100:0.5cm);
\draw[color=red!60!yellow] ($ (B)!0.2!(C) $) arc(100:190:0.5cm);
\draw[color=red!60!yellow] ($ (C)!0.2!(D) $) arc(190:280:0.5cm);
\draw[color=red!60!yellow] ($ (D)!0.2!(A) $) arc(280:370:0.5cm);
%
%jgl: Dass die Diagonalen gleich lang sind, wird durch die gleiche Farbe
%angedeutet:
\draw[color=blue!50!white] (A) -- (C);
\draw[color=blue!50!white] (D) -- (B);
%jgl: Dass die Diagonalen sich halbieren, wird durch den hervorgehobenen
%Schnittpunkt angedeutet:
\filldraw[color=blue] (S) circle(0.3pt);
%
%\filldraw[color=red!50!yellow] (S) circle(0.3pt);
%
\draw[color=black!50!white] (A) -- (B) -- (C) -- (D) -- cycle;
%
\foreach \Punkt in {(A), (B), (C), (D)} do
\filldraw \Punkt circle(2pt);
\end{scope}
\end{tikzpicture}
}
\end{center}
\end{minipage}
\end{tabular}
\end{MXInfo}
\end{MXContent}


\begin{MXContent}{Vielecke}{Vielecke}{STD}
\MDeclareSiteUXID{VBKM05_Vielecke_Content}

Bei Dreiecken tr"agt bereits eine Ecke oder Seite wesentlich zu den 
Eigenschaften des gesamten Dreiecks bei, zum Beispiel eine Ecke mit 
einem rechten Winkel. In Vierecken ist eine Ecke nicht mehr ganz so bestimmend.
Daf"ur gibt es mehr Variationen der Formen.
%
Wenn {\glqq}viele{\grqq} Punkte durch Strecken zu einer geschlossenen 
Figur, einem Vieleck, verbunden werden, ergeben sich viele M"oglichkeiten, 
vielf"altige Fl"achen zu gestalten und sogar runde Formen anzun"ahern.

%Beispiele von Vielecken mit einer eingezeichneten Diagonale:
\begin{center}
\MTikzAuto{%
\begin{tikzpicture}[line width=1.5pt]
\begin{scope}[xshift=-5cm]
\draw[dotted,color=blue] (200:1.6cm) -- (110:2cm);
%\draw (0,0) circle(1pt);
\draw (110:2cm) -- (150:2cm) -- (200:1.6cm) -- (250:1.6cm) -- (330:1cm) %
 -- (20:1cm) -- (50:1.6cm) -- cycle;
\end{scope}
%%
\begin{scope}[xshift=0cm]
\draw[dotted,color=blue] (-1.2,2cm) -- (-0.6,-0.6);
%\draw (100:2cm) -- (240:1cm) -- (315:2.5cm) -- (0:0.5cm) -- (70:3cm) -- cycle;
\draw (-0.6,-0.6) -- (-0.6,2) -- (-1.2,2) -- (-1.2,-1.2) -- (1.2,-1.2) %
 -- (1.2,-0.6) -- cycle;
\end{scope}
%%
%\begin{scope}[xshift=2cm]
%\draw (0:2cm) %
% -- (15:2cm) -- (30:2cm) -- (45:2cm) -- (60:2cm) -- (75:2cm) -- (90:2cm) %
% -- (105:2cm) -- (120:2cm) -- (135:2cm) -- (150:2cm) %
% -- (165:2cm) -- (180:2cm) -- (195:2cm) -- (210:2cm) -- (225:2cm) -- (240:2cm) %
% -- (255:2cm) -- (270:2cm) -- (285:2cm) -- (300:2cm) -- (315:2cm) %
% -- (330:2cm) -- (345:2cm) -- cycle;
%\end{scope}
%%
\begin{scope}[xshift=5cm]
\draw[dotted,color=blue] (0,-2) -- (0,2);
\draw (0:2cm) -- (30:2cm) -- (60:2cm) -- (90:2cm) -- (120:2cm)  -- (150:2cm) %
 -- (180:2cm) -- (210:2cm) -- (240:2cm) -- (270:2cm) -- (300:2cm) 
 -- (330:2cm) -- cycle;
%\draw (0:2cm) -- (30:2cm) -- (60:2cm) -- (90:2cm) -- (120:2cm)  -- (150:2cm) %
% -- (180:2cm) -- (210:2cm) -- (240:2cm) -- (270:2cm) -- (300:2cm) 
% -- (330:2cm) -- cycle;
\end{scope}
\end{tikzpicture}
}
\end{center}

Eine so detaillierte Einteilung wie f"ur Dreiecke oder Vierecke ist dabei
kaum mehr m"oglich. Die neuen M"oglichkeiten, wie die Ann"aherung an 
runde Formen, f"uhren auch auf andere interessante Fragen. Dazu werden nicht 
einzelne Vielecke, sondern Konstruktionsprinzipien f"ur eine Abfolge von vielen
Vielecken betrachtet. Und andererseits kann jedes Vieleck bei Bedarf 
in Dreiecke zerlegt werden, wie dies schon bei Vierecken beobachtet wurde.
Eine Eigenschaft einer besonderen Ecke wird somit vielfach unter dem Aspekt 
betrachtet, was dies insgesamt f"ur das Vieleck bedeutet.

Zur Einteilung in verschiedene Klassen bietet sich hier die Fragestellung an,
ob eine gewisse Eigenschaft von \textbf{allen} Ecken oder Seiten erf"ullt
wird oder eben nicht, und was dies f"ur die Vielecke bedeutet.
Beispielsweise werden Vielecke danach klassifiziert, ob die Winkel in allen
Ecken kleiner $\pi$ beziehungsweise $180\MGrad$ sind. In diesem Fall verlaufen
dann alle Diagonalen im Innern des Vielecks. Andernfalls gibt es mindestens 
eine Diagonale au"serhalb.

Die obigen Zeichnungen sind Beispiele von Vielecken, die verschiedene
Eigenschaften zeigen. Im linken Vieleck sind alle (Innen-) Winkel kleiner 
als $\pi$ beziehungsweise $180\MGrad$. In dieser Situation spricht man auch 
von einem konvexen Vieleck. Anders ist es im mittleren Vieleck, in dem es eine
Ecke mit einem gr"o"seren Winkel gibt. 
Im rechts dargestellten Vieleck sind alle Innenwinkel gleich, woraus sich ein 
sehr gleichm"a"siger Verlauf der Seiten ergibt.
%Au"serdem bieten Vielecke die M"oglichkeit, vielf"altige Fl"achen zu gestalten 
%und sogar runde Formen anzun"ahern.

\begin{MXInfo}{Vielecke}\MLabel{VBKM05_Vielecke}%
Es seien $n$ Punkte in der Ebene gegeben, wobei $n$ eine nat"urliche Zahl mit 
$n \geq 3$ sei.
Hier werden \MEntry{Vielecke}{Vieleck} betrachtet, die dadurch entstehen, dass
die Punkte nacheinander durch Strecken miteinander verbunden werden, sodass 
ein geschlossener Weg ohne Kreuzungspunkte entsteht und jeder der gegebenen 
Punkte zu genau zwei Strecken geh"ort. 
Dabei sollen je drei Punkte, die durch aufeinander folgende Strecken verbunden
sind, nicht auf einer gemeinsamen Geraden liegen.

Ein Vieleck wird auch als \MEntry{$n$-Eck}{$n$-Eck} oder 
\MEntry{Polygon}{Polygon} bezeichnet.

\begin{itemize}
\item Die $n$ Punkte, die verbunden werden, hei"sen 
 \MEntry{Ecken}{Ecke (Vieleck)} des Vielecks, und
 die $n$ Verbindungsstrecken hei"sen
 \MEntry{Seiten}{Seite (Vieleck)} des Vielecks.

\item Jedes Vieleck kann man in $(n-2)$ einander nicht "uberlappende 
 Dreiecke zerlegen. Die Summe aller (Innen-)Winkel eines Vielecks betr"agt 
 daher $(n-2) \cdot \pi$ beziehungsweise $(n-2) \cdot 180\MGrad$.

\item Die Verbindungsstrecken zwischen je zwei Ecken, die nicht auf derselben 
 Seite des Vielecks liegen, hei"sen die \MEntry{Diagonalen}{Diagonale (Vieleck)} 
 des Vielecks.
\end{itemize}
\end{MXInfo}

Weitere Aussagen ergeben sich f"ur Vielecke mit gleich langen Seiten und 
gleich gro"sen Innenwinkeln. F"ur $n = 3$ sind dies die gleichseitigen 
Dreiecke, und von den Vierecken sind es die Quadrate.

\begin{MXInfo}{Regelm"a"sige Vielecke}
Ein Vieleck, dessen $n$ Seiten alle die gleiche L"ange haben und dessen 
(Innen-)Winkel zudem alle gleich gro"s sind, nennt man
\MEntry{regelm"a"siges Vieleck}{regelm"a"siges Vieleck}
oder \MEntry{regelm"a"siges $n$-Eck}{regelm"a"siges n-Eck}.
\end{MXInfo}
Bienenwaben sind -- von oben betrachtet -- n"aherungsweise regelm"a"sige 
Sechsecke.

\begin{center}
\MTikzAuto{%
\begin{tikzpicture}[line width=1.5pt, declare function={Radius=3;}]
\begin{scope}[xshift=-4cm]
\foreach \x in {0,1,2,...,6}
 \coordinate (P\x) at ({\x*60}:Radius);
\coordinate (M) at (0,0);
%\draw[->] (-3,0) -- (3.5,0) node[below left] {$x$};
%\draw[->] (0,-3) -- (0,3) node[below left] {$y$};
%Sechseck:
\draw[color=yellow!50!red] (P0) -- (P1) -- (P2) -- (P3) -- (P4) -- (P5) -- cycle;
%Radius:
%\draw[style=dotted,color=black!50!white] (M) -- (P5);
%Mittelpunkt:
%\filldraw[color=black!50!white] (0,0) circle(2pt);
%Punkte des Sechsecks:
%\foreach \Punkt in {P0, P1, P2, P3, P4, P5}
% \filldraw[color=black!50!white] (\Punkt) circle(2pt);
\end{scope}
%%
\begin{scope}[xshift=4cm]
\foreach \x in {0,1,2,...,5}
{
 \coordinate (P\x) at ({\x*60}:Radius);
 \coordinate (SMP\x) at ({(\x*60)+30}:{1.1*Radius});
}
\coordinate (M) at (0,0);
%\draw[->] (-3,0) -- (3.5,0) node[below left] {$x$};
%\draw[->] (0,-3) -- (0,3) node[below left] {$y$};
%Sechseck:
\draw[color=yellow!50!red] (P0) -- (P1) -- (P2) -- (P3) -- (P4) -- (P5) -- cycle;
%Umkreis:
\draw[color=black!50!white] (M) circle(Radius);
%Seitenmittensenkrechten:
\foreach \x/\y in {0/3,1/4,2/5}
 \draw[style=dotted,color=red!50!white] (SMP\x) -- (SMP\y);
%Radien:
\foreach \x in {0,1,2,...,5}
 \draw[style=dotted,color=black!50!white] (M) -- (P\x);
%Mittelpunkt:
\filldraw[color=black!50!white] (0,0) circle(2pt);
%Punkte des Sechsecks:
\foreach \Punkt in {P0, P1, P2, P3, P4, P5}
 \filldraw[color=black!50!white] (\Punkt) circle(2pt);
\end{scope}
\end{tikzpicture}
}
\end{center}

Regelm"a"sige Vielecke haben verschiedene Symmetrieeigenschaften. Alle Geraden 
durch die Seitenmitten,  die senkrecht auf einer Seite stehen, schneiden sich
in einem Punkt $M$. Eine Spiegelung an einer solchen Geraden bildet das 
Vieleck auf sich ab (Spiegelsymmetrie).

Au"serdem sind regelm"a"sige Vielecke in der Weise rotationssymmetrisch, dass das 
Vieleck auf sich abgebildet wird, wenn man es um $M$ um den Winkel 
$\frac{2 \pi}{n}$ dreht.

Die Ecken eines regelm"a"sigen Vielecks haben von $M$ alle denselben Abstand und
liegen somit auf einem Kreis um $M$.
\end{MXContent}


\begin{MXContent}{Umfang}{Umfang}{STD}
\MDeclareSiteUXID{VBKM05_Umfang_Content}
Der Umfang eines Vielecks ist die Summe der L"angen aller Verbindungsstecken.
Wenn ein Vieleck weitere Eigenschaften hat, die einen Bezug zur L"ange der 
Seiten haben, kann es weitere Aussagen zum Umfang geben.

Es werden zun"achst Vierecke betrachtet. Wenn $a$ und $b$ 
benachbarte Seiten eines Parallelogramms sind, dann ist sein Umfang
$U = a + b + a + b = 2 \cdot (a + b)$.

Bei einer Raute und auch bei einem Quadrat sind alle vier Seiten gleich lang, 
sodass sein Umfang durch $U = 4 \cdot a$ gegeben ist, wenn $a$ die L"ange einer 
Seite ist. 

Ebenso sind bei jedem regelm"a"sigen Vieleck alle Seiten gleich lang. Wenn $n$ 
die Anzahl der Eckpunkte und $a$ die L"ange einer Seite ist, dann ist der 
Umfang $U_n$ einfach durch $U_n = n \cdot a$ zu berechnen.

Als Ausblick auf die Winkelfunktionen, die im Abschnitt 
\MRef{M05_Trigonometrie}
besprochen werden, soll der Umfang eines regelm"a"sigen Vielecks noch auf eine 
andere Weise berechnet werden.
Seine Ecken liegen alle auf einem Kreis mit einem Radius $r$. 

\begin{center}
\MTikzAuto{%
\begin{tikzpicture}[line width=1.5pt, declare function={Radius=3; %
 cSeite={pow(3,0.5)/2*Radius};}]
\foreach \x in {0,1,2,5}
{
 \coordinate (P\x) at ({\x*60}:Radius);
 \coordinate (SMP\x) at ({(\x*60)+30}:{1.1*Radius});
}
\coordinate (M) at (0,0);
\coordinate (C) at (30:cSeite);
%Ausschnitt eines Sechsecks:
\draw[color=yellow!50!red] (P5) -- (P0) -- (P1) -- (P2);
%Winkel:
\filldraw[color=blue] ({0.4*Radius},0) arc(0:30:{0.4*Radius});
\node[right] at (15:{0.4*Radius}) {$\Mvarphi$};
%Umkreis:
\draw[color=black!50!white] (-75:Radius) arc(-75:135:Radius);
%Seitenmittensenkrechten:
\draw[color=red!50!white] (M) -- (C);
\draw[style=dotted,color=red!50!white] (C) -- (SMP0);
\foreach \x in {1,5}
 \draw[style=dotted,color=red!50!white] (M) -- (SMP\x);
%Radien:
 \draw[color=black!50!white] (M) -- node[below] {$r$} (P0);
\foreach \x in {1,2,5}
 \draw[style=dotted,color=black!50!white] (M) -- (P\x);
%Mittelpunkt:
\filldraw[color=black!50!white] (0,0) circle(2pt);
%Punkte des Sechsecks:
\foreach \Punkt in {P0, P1, P2, P5}
 \filldraw[color=black!50!white] (\Punkt) circle(2pt);
\node[below left] at (M) {$M$};
\node[below right] at (P0) {$A$};
\node[above right] at (P1) {$B$};
\node[left] at (C) {$C$};
\end{tikzpicture}
}
\end{center}

Der Winkel $\Mvarphi$ zwischen den Strecken, die vom Mittelpunkt des 
Kreises zu den Ecken $A$ und $B$ einer Seite verlaufen, ist der $n$-te Teil 
des Vollwinkels: $\Mvarphi = \frac{2 \pi}{n}$.
Der Mittelpunkt des Kreises, $A$ und der Mittelpunkt $C$ der Strecke
$\MGeoStrecke{A}{B}$ bilden ein rechtwinkliges Dreieck $\MGeoDreieck{M}{A}{C}$ 
mit dem Winkel 
$\Mmeasuredangle(AMC) = \frac{1}{2} \cdot \Mvarphi = \frac{\pi}{n}$.
Wenn aus 
\[
\sin(\Mmeasuredangle(AMC)) = \frac{\frac{1}{2} a}{r} %%
\]
der Wert von $a$ berechnet und in $U = n \cdot a$ eingesetzt wird, erh"alt man
die Formel
\[
U_n = n \cdot a = 2 \cdot r \cdot n \cdot \sin\left(\frac{\pi}{n}\right) %%
\]
f"ur den Umfang eines regelm"a"sigen Vielecks. Beispielsweise ist 
$U_6 = 2 \cdot r \cdot 6 \cdot \frac{1}{2} = 6 \cdot r$. 
Je gr"o"ser $n$ ist, um so mehr n"ahert sich der Umfang des Vielecks dem Wert 
$2 \cdot r \cdot \pi \approx \MZahl{6}{283} \cdot r$ an, der den Umfang eines 
Kreises mit Radius $r$ angibt. Dies kann mit weiterf"uhrenden Methoden der 
Differentialrechnung begr"undet werden, dessen Grundideen im Kapitel 
\MRef{VBKM07} eingef"uhrt wird. Die Idee hinter der beschriebenen Ann"aherung
kann man hier so beschreiben: Zu einem schwierig zu berechnenden Wert, wie dem 
Kreisumfang, sucht man nach vielen vergleichbaren Objekten, hier den regelm"a"sigen 
Vielecken, die zwei Eigenschaften haben: Ihr Umfang ist einfach zu berechnen, 
und bei hinreichend gro"ser Eckenzahl unterscheiden sie sich hinsichtlich des 
Umfangs vom Kreis um weniger als irgend eine vorgegebene positive Zahl (hier
denkt man an {\glqq}kleine{\grqq} Zahlen).
Dieses Vorgehen eignet sich auch f"ur die Berechnung von Fl"achen, die nicht 
durch gerade Strecken berandet werden (siehe Kapitel \MRef{VBKM08}).
Dazu wird im Folgenden zun"achst die Berechnung von Fl"acheninhalten von 
Vielecken erl"autert, die in dieser Hinsicht relativ einfach ist und die 
Ausgangspunkt f"ur eine Ann"aherung sein kann, wie das obige Bild eines 
Sechsecks in einem Kreis zeigt.
\end{MXContent}


\begin{MXContent}{Fl\"acheninhalt}{Fl\"acheninhalt}{STD}
\MDeclareSiteUXID{VBKM05_Flaecheninhalt_Content}
Der Inhalt einer Fl\"ache ist die Zahl der Einheitsquadrate, die man ben\"otigt,
um diese Fl\"ache vollst\"andig zu bedecken.

\begin{tabular}{@{}l@{\hspace{1.5cm}}r@{}}
\begin{minipage}{8cm}
Zuerst sollen Rechtecke betrachtet werden.
Wenn ein Rechteck eine Seite der L\"ange $a$ und eine benachbarte Seite der 
L\"ange $b$ hat, dann gibt es $b$ Reihen mit $a$ Einheitsquadraten, also 
$b \cdot a$ Einheitsquadrate.
\end{minipage}
&
\begin{minipage}{7cm}
\MTikzAuto{%
\def\sxyc{0.8cm}
\begin{tikzpicture}[x=\sxyc,y=\sxyc] 
\begin{scope}[yshift=-2.5cm]
\draw[help lines, black, xstep=1, ystep=1] (1,1) grid (8,5);
\draw[color=green!50!black, line width=2pt] (1,1)--(8,1) (1,5)--(8,5);
\draw[color=blue, line width=2pt] (1,1)--(1,5) (8,1)--(8,5);
\draw[color=green!50!black] (4.5,1) node[anchor=north] {$a$};
\draw[color=green!50!black] (4.5,5) node[anchor=south] {$a$};
\draw[color=blue] (1,3) node[anchor=east] {$b$};
\draw[color=blue] (8,3) node[anchor=west] {$b$};
\end{scope}
\end{tikzpicture}
}
\end{minipage}
\end{tabular}

\begin{MXInfo}{Fl"ache eines Rechtecks}
Die Fl\"ache $F$ eines Rechtecks mit den L"angen $a$ und $b$ benachbarter 
Seiten ist
\[
F = b\cdot a = a\cdot b \MDFPeriod %%
\]
\end{MXInfo}

\begin{tabular}{@{}lr@{}}
\begin{minipage}{10cm}
Damit l\"asst sich nun auch leicht der Fl\"acheninhalt eines rechtwinkligen 
Dreiecks berechnen.
Es sei~$\MGeoDreieck{A}{B}{C}$ ein rechtwinkliges Dreieck, welches 
um~$180\MGrad$ gedreht werde. Legt man anschlie"send das urspr\"ungliche 
und das neue Dreieck entlang der beiden Hypotenusen aneinander, so erh\"alt 
man ein Rechteck.
\end{minipage}
&
\begin{minipage}{5cm}
%\begin{center}
\MTikzAuto{%
\begin{tikzpicture}[rotate=-20]
\coordinate (A) at (0,0);
\coordinate (B) at ($ (A) + (1,-1.5) $);
\coordinate (C) at ($ (A) + (3, 2) $);
\coordinate (D) at ($ (B) + (C) - (A)$);
% \coordinate[label=right:$B_1$]      (B) at (4,-1);
\draw (A) node [left]{$A$} -- (B) node[left]{$B$} -- (C) node[right]{$C$} -- cycle;
\draw[dotted] (B) -- (D) node[right]{$D$} -- (C);
\end{tikzpicture}
}
%\end{center}
\end{minipage}
\end{tabular}

Der Fl\"acheninhalt des Dreiecks ist nun die H\"alfte des Fl\"acheninhaltes 
des Rechtecks, also $ F = \frac{1}{2}\cdot a\cdot b$.

Und wie berechnet man die Fl"ache, wenn das Dreieck nicht rechtwinklig ist?

Aus jedem beliebigen Dreieck kann man zwei rechtwinklige Dreiecke gewinnen,
indem man von einer Ecke aus eine Linie auf die gegen\"uberliegende Seite zieht,
so dass sie diese senkrecht trifft. Diese Linie nennt man die 
\textbf{H\"ohe} $h_{i}$ eines Dreiecks auf die bestimmte Seite $i$, wobei 
der Index $i$ derjenigen Seite $a$, $b$ oder $c$ entspricht, \"uber der die 
H\"ohe bestimmt wird.

Je nachdem, ob die neue Linie innerhalb oder au"serhalb des Dreiecks liegt, 
ergibt sich der Fl\"acheninhalt des Dreiecks dann aus der Summe oder der 
Differenz der Fl\"acheninhalte der beiden sich ergebenden rechtwinkligen 
Dreiecke:

\begin{center}
\MTikzAuto{%
\begin{tikzpicture}
 \coordinate[label=left:$A$]  (A) at (0,0);
 \coordinate[label=right:$B$] (B) at ($ (A) + (3,0) $);
 \coordinate[label=below:$D$] (D) at ($ (A)!0.3!(B) $);
 \coordinate[label=above:$C$] (C) at ($ (D) + (0,2) $);
 %
 \draw (A) -- node[below]{$c$} (B) -- (C) -- cycle;
 \draw[dotted] (C) -- node[right]{$h_c$} (D);
 \path (A) -- node[above]{$c_1$} (D) -- node[above]{$c_2$} (B);
\end{tikzpicture}
}
\hspace{4em}
\MTikzAuto{%
\begin{tikzpicture}
 \coordinate[label=below:$U$] (U) at (0,0);
 \coordinate[label=right:$V$] (V) at ($ (U) + (3,0) $);
 \coordinate[label=below:$X$] (X) at ($ (U)!-0.3!(V) $);
 \coordinate[label=above:$W$] (W) at ($ (X) + (0,2) $);
 %
 \draw (U) -- node[above]{$w$} (V) -- (W) -- cycle;
 \path (X) -- node[below]{$w_2$} (V);
 \draw[dotted] (W) -- node[left] {$h_w$} (X);
 \draw[dotted] (U) -- node[above]{$w_1$} (X);
\end{tikzpicture}
}
\end{center}

Links gilt also (wenn $F_{\Delta}$ den Fl\"acheninhalt des Dreiecks~$\Delta$ 
bezeichnet)
\[
   F_{\MGeoDreieck{A}{B}{C}}
 = F_{\MGeoDreieck{D}{B}{C}} + F_{\MGeoDreieck{A}{D}{C}}
 = \Mtfrac{1}{2} \cdot h_c \cdot c_2 + \Mtfrac{1}{2} \cdot h_c \cdot c_1
 = \Mtfrac{1}{2} \cdot h_c \cdot \left( c_2 + c_1 \right)
 = \Mtfrac{1}{2} \cdot h_c \cdot c \MDFPeriod
\]
Rechts gilt 
\[
   F_{\MGeoDreieck{U}{V}{W}}
 = F_{\MGeoDreieck{X}{V}{W}} - F_{\MGeoDreieck{X}{U}{W}}
 = \Mtfrac{1}{2} \cdot h_w \cdot w_2 - \Mtfrac{1}{2} \cdot h_w \cdot w_1
 = \Mtfrac{1}{2} \cdot h_w \cdot \left( w_2 - w_1 \right)
 = \Mtfrac{1}{2} \cdot h_w \cdot w \MDFPeriod
\]
Somit kann der Fl"acheninhalt stets mittels einer Seitenl"ange und der L"ange
der hierzu senkrechten H"ohe berechnet werden.

\begin{MXInfo}{Dreiecksfl"ache}
Der Fl\"acheninhalt $F_{\MGeoDreieck{A}{B}{C}}$ eines Dreiecks berechnet sich 
aus der H\"alfte des Produkts
der L\"ange einer Seite mit der L\"ange der zugeh\"origen H\"ohe des Dreiecks:
       \[
          F_{\MGeoDreieck{A}{B}{C}}
        = \frac{1}{2} \cdot a \cdot h_a
        = \frac{1}{2} \cdot b \cdot h_b
        = \frac{1}{2} \cdot c \cdot h_c \MDFPeriod
       \]
Dabei ist die 
\MEntry{H\"ohe eines Dreiecks auf einer Seite}{H\"ohe eines Dreiecks} 
die Strecke, die von dem der Seite gegen\"uberliegenden Punkt ausgeht 
und die Gerade, auf der die Seite liegt, im rechten Winkel trifft. 
%Der Punkt, auf dem die H\"ohe diese Gerade trifft, hei"st
%       \MEntry{H"ohenfu"spunkt}{H"ohenfu"spunkt} der H\"ohe.
\end{MXInfo}


\begin{MExample}
\begin{tabular}{@{}l@{\hspace{2cm}}r@{}}
\begin{minipage}{8cm}
Bei dem hier gezeigten Dreieck ist die H\"ohe gegeben, die zur Seite mit dem 
Wert $\MZahl{8}{6}$ geh\"ort.
Bei den Angaben handelt es sich jeweils um gerundete numerische Werte.
Der Fl\"acheninhalt $F$ des Dreiecks ist also rund
\[ F = \frac{\MZahl{8}{6}\cdot\MZahl{5}{5}}{2}=\MZahl{23}{65} \MDFPeriod\]
\end{minipage}
&
\begin{minipage}{7cm}
\MTikzAuto{%
\begin{tikzpicture}[x=0.6cm, y=0.6cm] 
%%\draw[help lines, gray!50, xstep=0.5, ystep=0.5] (0,0) grid (9,8);
\draw[color=black, very thick] (0,0) -- (1.7032,-6.0654) -- (7.8,0) -- cycle;
\draw[color=black, thick] (0,0) -- (3.87986,-3.89994);
\draw[color=black] (3.9,0) node[anchor=south] {\large $7{,}8$};
\draw[color=black] (0.85160,-3.0327) node[anchor=north east] {\large $6{,}3$};
\draw[color=black] (4.7516,-3.0327) node[anchor=north west] {\large $8{,}6$};
\draw[color=black] (1.93993,-1.94997) node[anchor=south west] {\large $5{,}5$};
\end{tikzpicture}
}
\end{minipage}
\end{tabular}
\end{MExample}

\begin{MExercise}
Berechnen Sie den Fl\"acheninhalt des Dreiecks:

\begin{center}
\MTikzAuto{%
\begin{tikzpicture}[x=1.2cm, y=1.2cm] 
%Koordinatensystem
\node (xMAX) at (6.0,0){};
\node (yMAX) at (0,3.2){};
\draw[help lines, gray, dashed, xstep=1, ystep=1] (0,0) grid (5.5,2.8);
\draw[->,color=black] (-0.4,0) -- (xMAX);
\foreach \x in {1, 2, 3, 4, 5}
\draw[shift={(\x,0)},color=black] (0pt,2pt) -- (0pt,-2pt) node[below] {\normalsize $\x$};
\draw[->,color=black] (0,-0.4) -- (yMAX);
\foreach \y in {1,2}
\draw[shift={(0,\y)},color=black] (2pt,0pt) -- (-2pt,0pt) node[left] {\normalsize $\y$};
%Achsenbeschriftung
\draw (xMAX) node[anchor=north east] {$x$};
\draw (yMAX) node[anchor=north east] {$y$};
%Beschriftung und Graphen
%%\clip(-2.8,-0.5) rectangle (6,3);
\draw[color=black, very thick] (1,0) -- (4,2) -- (5,0) -- cycle;
%%\draw[color=black] (7,3) node[anchor=south west] {$\MPointTwo{7}{3}$};
\end{tikzpicture}
}
\end{center}

\begin{MHint}{L\"osung}
An diesem Dreieck l"asst sich eine H"ohe leicht ablesen, und zwar die H"ohe,
die senkrecht auf der Seite steht, die auf der $x$-Achse liegt. 
Die L"ange $h$ dieser H"ohe ist $h = 2$, und die L"ange der genannten Seite 
ist $c = 5 - 1 = 4$. Damit ergibt sich der Fl"acheninhalt $F$ des Dreiecks zu
$F = \frac{1}{2} \cdot c \cdot h = \frac{1}{2} \cdot 4 \cdot 2 = 4$.
\end{MHint}
\end{MExercise}

Mit der Formel f"ur den Fl"acheninhalt von Dreiecken lassen sich auch Fl"achen 
von anderen Vielecken -- auch Polygone genannt -- bestimmen. 
Denn jedes Vieleck kann in Dreiecke unterteilt werden, indem man so 
lange Diagonalen einzeichnet, bis die Teilfl"achen Dreiecke sind.
Die Summe der Fl"acheninhalte dieser Dreiecke ergibt den Fl"acheninhalt des 
Vielecks. 
%In der Praxis wird man auch Zerlegungen, zum Beispiel mit Rechtecken betrachten,
%wenn die ben"otigten Gr"o"sen vorliegen.
Hier soll die Betrachtung jedoch auf einige einfache Formen beschr"ankt bleiben. 
Im folgenden Beispiel kann man das Vieleck in ein Dreieck und ein Rechteck 
zerlegen. Dadurch wird die Berechnung besonders einfach.

\begin{MExample}
\begin{tabular}{@{}l@{\hspace{2cm}}r@{}}
\begin {minipage}{8cm}
Man betrachte das rechts dargestellte Vieleck, ein Trapez. 
In diesem Beispiel kann man das Vieleck in ein rechtwinkliges Dreieck mit 
den Katheten $\left(a-c\right)$ und $b$ und der Hypotenuse $d$ sowie ein 
Rechteck mit den Seiten $b$ und $c$ unterteilen. 
\par
\vspace*{0.5cm}
Der Fl\"acheninhalt des Polygons ist dann:
\end{minipage}
&
\begin{minipage}{7cm}
\MTikzAuto{%
\begin{tikzpicture}[x=0.4cm, y=0.4cm] 
\draw[thick] (0,0) -- (9,0) -- (9,9) -- (4,9) -- cycle;
\draw[thick, dashed] (4,0) -- (4,9);
\node[anchor=north] at (4.5,0) {$a$};
\node[anchor=west] at (9,4.5) {$b$};
\node[anchor=south] at (6.5,9) {$c$};
\node[anchor=south east] at (2.0,4.5) {$d$};
\end{tikzpicture}
}
\end{minipage}
\end{tabular}
\[
F = F_{\text{Dreieck}} + F_{\text{Rechteck}} %
  = \frac{1}{2}\left( a-c \right)\cdot b + b\cdot c %
  = \frac{1}{2}ab-\frac{1}{2}bc+bc %
  =\frac{1}{2}\left(a+c\right)\cdot b \MDFPeriod
\]
\end{MExample}

\begin{MExercise}
\begin{tabular}{@{}l@{\hspace{0.6cm}}r@{}}
\begin{minipage}{7cm}
Berechnen Sie den Fl\"acheninhalt des dargestellten
\textbf{Parallelogramms} f\"ur $a=4$ und $h=5$.
\par
\begin{MHint}{Tipp}
Teilen Sie das Parallelogramm sinnvoll auf, und schauen Sie sich die 
entstandenen Dreiecke genau an!
\end{MHint}
\par
\vspace*{1cm}
\end{minipage}
&
\begin{minipage}{7cm}
\MTikzAuto{%
\begin{tikzpicture}[x=0.5cm, y=0.5cm] 
\draw[thick] (0,0) -- (10,0) -- ++(45:10) -- (45:10) -- cycle;
\draw[stealth'-stealth',thick] (8,0) -- (8,7.0710678);
\node[anchor=north] at (5,0) {$a$};
\node[anchor=west] at (8,3.5355339) {$h$};
\end{tikzpicture}
}
\end{minipage}
\end{tabular}

\begin{MHint}{L\"osung}

\begin{tabular}{@{}lr@{}}
\begin{minipage}[b]{7cm}
Man kann das Parallelogramm in das linke rote Dreieck, einem folgenden 
Rechteck und das rechte Dreieck aufspalten. Schneidet man das rote Dreieck 
aus und setzt es von rechts an das Parallelogramm, erh\"alt man ein Rechteck 
mit den Seiten $a$ und $h$. Der Fl\"acheninhalt ergibt sich dann zu
\[F=a\cdot h=4\cdot 5=20 \MDFPeriod\]
\end{minipage}
&
\MTikzAuto{%
\begin{tikzpicture}[x=0.5cm, y=0.5cm] 
\draw[thick] (0,0) -- (10,0) -- ++(45:10) -- (45:10) -- cycle;
\draw[thick,dashed] (10,0) -- (10,7.0710678);
\draw (7.0710678,0) -- (7.0710678,7.0710678);
\draw[red,thick] (45:10) -- (0,0) -- (7.0710678,0);
\draw[red,thick,dashed] (7.0710678,0) -- (7.0710678,7.0710678);
\draw[red,thick,dashed] (10,0) -- ++(7.0710678,0) -- ++(0,7.0710678) -- cycle;
\node[anchor=north] at (5,0) {$a$};
\node[anchor=west] at (10,3.5355339) {$h$};
\end{tikzpicture}
}
\end{tabular}
\end{MHint}
\end{MExercise}

Zum Schluss sollen noch Kreisfl\"achen berechnet werden. In \MRef{Kreiszahl} 
wurde bereits die Kreiszahl $\pi$ vorgestellt, die das Verh"altnis zwischen
Umfang und Durchmessung eines Kreises beschreibt. 
Auch in der Formel f"ur den Fl\"acheninhalt von Kreisen kommt 
die Kreiszahl vor.

\begin{MXInfo}{Fl"acheninhalt eines Kreises}
Der Fl"acheninhalt $F$ eines Kreises mit dem Radius $r$ berechnet sich zu 
\[ F =\pi\cdot r^2 \MDFPeriod\]
\end{MXInfo}

\begin{MExample}
Ein Kreis mit dem Radius $r = 2$ habe einen Fl"acheninhalt $F$ von rund 
$\MZahl{12}{566}$.
Hieraus l"asst sich die Kreiszahl $\pi$ n"aherungsweise berechnen:
Aus $F =\pi\cdot r^2$ folgt $\pi = \frac{F}{r^2}$.
Mit den angegebenen Werten ergibt sich der N"aherungswert
\[
\pi = \frac{F}{r^2} \approx \frac{\MZahl{12}{566}}{4} %
 =\MZahl{3}{1415} \MDFPeriod %%
\]
\end{MExample}

\end{MXContent}


\begin{MExercises}
\MDeclareSiteUXID{VBKM05_Flaecheninhalt_Exercises}

\begin{MExercise}
\begin{tabular}{@{}lr@{}}
\begin{minipage}[b]{8cm}
Berechnen Sie den Fl\"acheninhalt des Polygons:\\
\vspace{3.5cm}
\end{minipage}
&
\MTikzAuto{%
\begin{tikzpicture}[x=1.0cm, y=1.0cm] 
\draw[color=black, thick] (0.0,0.0) -- (3.0,-1.8) -- (5.4,0.0) -- 
(4.0,2.6) -- (1.5,2.2)-- cycle; 
\draw[color=black,style=dotted] (0.0,0.0) -- (5.4,0.0) (1.5,0.0) -- (1.5,2.2)
(4.0,0.0) -- (4.0,2.6) (3.0,0.0) -- (3.0,-1.8);
\draw[color=blue] (1.1,0.8) node {\large $F_1$};
\draw[color=blue] (2.75,1.2) node {\large $F_2$};
\draw[color=blue] (4.4,0.9) node {\large $F_3$};
\draw[color=blue] (3.0,-0.5) node[fill=white] {\large $F_4$};
\draw[color=black] (0.75,0.0) node[anchor=south] {\large $15$};
\draw[color=black] (2.75,0.0) node[anchor=south] {\large $25$};
\draw[color=black] (4.7,0.0) node[anchor=south] {\large $14$};
\draw[color=black] (1.5,1.5) node[anchor=west] {\large $22$};
\draw[color=black] (4.0,1.8) node[anchor=east] {\large $26$};
\draw[color=black] (3.0,-1.1) node[anchor=west] {\large $18$};
\end{tikzpicture}
}
\end{tabular}
\begin{MHint}{L\"osung}
Die Fl"acheninhalte der angegebenen Teilfl"achen werden separat bestimmt.
\begin{itemize}
\item
$F_1$ ist ein Dreieck: $F_1=\frac{15\cdot 22}{2}=165$.
\item
$F_2$ ist ein Trapez, welches in zwei Dreiecke mit der H"ohe 25 zerlegt 
werden kann: $F_2 = \frac{22\cdot 25}{2}+\frac{26\cdot 25}{2} = 275+325 = 600$.
\item
$F_3$ ist ein Dreieck: $F_3=\frac{14\cdot 26}{2}=182$.
\item
Auch $F_4$ ist ein Dreieck: $F_4 = \frac{(15+25+14)\cdot 18}{2} = 486$.
\end{itemize}
Damit erh"alt man den Fl"acheninhalt des gesamten Polygons als Summe der 
Fl"acheninhalte der Teilfl"achen: 
$F_1+F_2+F_3+F_4 = 165+600+182+486 = 1433$.
\end{MHint}
\end{MExercise}

\end{MExercises}

%end of content: section 4: Vielecke, Flaechen und Umfang.




%jgl: Neuer Abschnitt:
\MSubsection{Elementargeometrische K"orper}%
\MLabel{M05_Koerper}

\begin{MIntro}
\MDeclareSiteUXID{VBKM05A_ElementareKoerper_Intro}

Die Gestalt allt"aglicher Gegenst"ande wie ein Notizblock, ein Handy oder 
auch technische Konstruktionen wie Tunnel k"onnen durch einfache Grundformen
beschrieben werden, wenn man von {\glqq}abgerundeteten Ecken{\grqq} absieht.
Woran liegt das?

Wenn man einen Besen "uber einen ebenen Boden mit Sand bewegt, wird ein 
viereckiger Ausschnitt des Bodens sichtbar.
Geometrisch idealisiert formuliert, entsteht aus einer Strecke (dem Besen) 
durch Verschieben ein Viereck. Wenn man den Besen dreht, kann man eine 
Kreisscheibe erzeugen. Auf diese Weise ergeben sich aus einfachen Objekten 
kompliziertere, die trotzdem einfach beschrieben werden k"onnen.
\end{MIntro}

\begin{MXContent}{Elementargeometrische K"orper}{Elementargeometrische K"orper}{STD}%
\MDeclareSiteUXID{VBKM05A_ElementareKoerper_Content}

Punkte sind die einfachsten geometrischen Objekte. Verschiebungen
f"uhren auf Strecken, und Operationen wie das Verschieben oder Drehen von 
Strecken in der Ebene f"uhren auf elementargeometrische Figuren. 
Beispielsweise ergeben sich Vielecke und Kreise, wie sie oben beschrieben 
wurden.

Wenn man Figuren au"serhalb ihrer Ebene verschiebt oder dreht, entstehen neue
Objekte, die als K"orper bezeichnet werden. Im Folgenden werden einige einfache 
K"orper beschrieben, deren Form in vielen allt"aglichen Gegenst"anden oder 
technischen Konstruktionen erkennbar ist.

%Beispiel:
\begin{MExample}
Es wird ein Rechteck betrachtet und senkrecht zur Zeichenebene verschoben.
Dadurch entsteht ein Quader. Seine Oberfl"ache besteht aus dem gegebenen
Rechteck und einer Kopie davon. Au"serdem entstanden aus den vier Seiten
des urspr"unglichen Rechtecks jeweils vier weitere Rechtecke.
\end{MExample}

Betrachtet man irgend ein Vieleck und verschiebt dieses senkrecht, entsteht
ein Prisma genannter K"orper. Der Name wird auch f"ur durchsichtige physikalische
K"orper dieser Form verwendet, mit denen Lichtwellen gebrochen werden. Die 
einzelnen Lichtwellen von scheinbar wei"sem Licht werden in verschiedenen Farben
sichtbar.

\begin{MXInfo}{Prisma}
Gegeben ist ein Vieleck $G$.
Ein \MEntry{Prisma}{Prisma} ist ein K"orper, der aus einer senkrechten 
Verschiebung eines Vielecks $G$ um eine Strecke der L"ange $h$ entsteht. 
Die beiden Fl"achen, die aus dem gegebenen Vieleck und einer Kopie einen 
Teil der Oberfl"ache bilden, werden dann als Grundfl"ache des Prismas 
bezeichnet. Sie sind zueinander parallel. 
Die anderen Fl"achen bilden zusammen den Mantel $M$.

%Zeichnung Prisma:
\begin{center}
\MTikzAuto{%
\tdplotsetmaincoords{60}{30}
\begin{tikzpicture}[tdplot_main_coords,>=latex]
\def\zt{3}
\def\ticl{0.8}
\def\dsth{0.5}
\foreach \zc in {0,\zt} 
% coordinate number, radius, azimuth
\foreach \nc/\rc/\ac in {0/3/0,1/3/120,2/1/240} {
\pgfmathparse{\rc*cos(\ac)}\let\ax=\pgfmathresult
\pgfmathparse{\rc*sin(\ac)}\let\ay=\pgfmathresult
\node (PC-\nc-\zc) at (\ax,\ay,\zc) {};
}
%,name path=level-ground
\path[fill=gray!60,opacity=0.5] (PC-0-0.center) 
\foreach \nc in {1,2} { -- (PC-\nc-0.center) {} } -- cycle;
\draw[thick, black] (PC-2-0.center) -- (PC-0-0.center);
\draw[thick, black, dashed] (PC-0-0.center) --  (PC-1-0.center) -- (PC-2-0.center) ;
\foreach \nc in {0,2} { \draw[thick] (PC-\nc-0.center) -- (PC-\nc-\zt.center); };
\foreach \nc in {1} { \draw[thick,dashed] (PC-\nc-0.center) -- (PC-\nc-\zt.center); };
\path[draw=black,thick,fill=gray!30,fill opacity=0.2] (PC-0-\zt.center) 
\foreach \nc in {1,2} { -- (PC-\nc-\zt.center) {} } -- cycle;
\draw (PC-2-0.center) [tdplot_screen_coords] -- +(-\ticl,0);
\draw (PC-2-\zt.center) [tdplot_screen_coords] -- +(-\ticl,0);
\path (PC-2-0.center) [tdplot_screen_coords] ++(-\dsth,0) coordinate (BG);
\path (PC-2-\zt.center) [tdplot_screen_coords] ++(-\dsth,0) coordinate (BT);
\path ($ (BG)!0.5!(BT) $) node (TXT) {$h$}; % [fill=white]
\draw[<-] (BG) -- (TXT.south);
\draw[->] (TXT.north) -- (BT);
%%\tdplottransformmainscreen{\ax}{\ay}{\az}
%%\pgfpathmoveto{\pgfpoint{\tdplotresx cm}{\tdplotresy cm}}
\end{tikzpicture}}
\end{center}

In der Zeichnung ist ein Prisma mit einem Dreieck als Grundfl"ache gezeigt. Die
anderen Seitenfl"achen, die an die Grundfl"ache angrenzen, sind Rechtecke.
%Ende (Zeichnung).

Das Volumen $V$ des Prismas ist das Produkt aus dem Fl"acheninhalt des Vielecks 
$G$ und der H"ohe $h$: Es ist $V = G \cdot h$.

Der Fl"acheninhalt $O$ der Oberfl"ache ergibt sich aus dem doppelten Inhalt der 
Grundfl"ache $G$ und dem Fl"acheninhalt des Mantels $M$. Wenn $U$ der Umfang 
des gegebenen Vielecks ist, gilt $O = 2 \cdot G + M = 2 \cdot G + U \cdot h$.
\end{MXInfo}

Im einf"uhrenden Beispiel wurde ein Quader beschrieben. Mit obiger Definition
kann er als spezielles Prisma angesehen werden, n"amlich als ein Prisma mit einem 
Rechteck als Grundfl"ache. Wenn alle Seitenfl"achen Quadrate sind, spricht man
von einem W"urfel.

Das Bauprinzip kann auf verschiedene Weisen variiert werden.
Beispielsweise kann das Vieleck durch eine Kreisscheibe ersetzt werden, die 
verschoben wird. Durch die senkrechte Verschiebung entsteht ein besonders 
symmetrischer K"orper, ein Zylinder.
Eine Tunnelbohrmaschine stellt -- vereinfacht -- eine zylinderf"ormige R"ohre her.

\begin{MXInfo}{Zylinder}
Gegeben ist eine Kreisscheibe $G$.
Ein \MEntry{Zylinder}{Zylinder} ist ein K"orper, der aus einer senkrechten 
Verschiebung einer Kreisscheibe $G$ um eine Strecke der L"ange $h$ entsteht. 
Die beiden Fl"achen, die aus der gegebenen Kreisscheibe und einer Kopie einen 
Teil der Oberfl"ache bilden, werden dann als Grundfl"ache des Zylinders
bezeichnet. Sie sind zueinander parallel. 
Der gekr"ummte Teil der Oberfl"ache zwischen den beiden Kreisscheiben bildet 
den Mantel $M$ des Zylinders.

%Zeichnung Zylinder:
\begin{center}
\MTikzAuto{%
\tdplotsetmaincoords{60}{30}
\begin{tikzpicture}[tdplot_main_coords,>=latex]
\def\az{30}
%\def\pl{60}
\def\zt{3}
\def\rd{2}
\def\ticl{0.8}
\def\dsth{0.5}
\pgfmathparse{\rd*cos(\az)}\let\ax=\pgfmathresult
\pgfmathparse{\rd*sin(\az)}\let\ay=\pgfmathresult
\path[fill=gray!60,opacity=0.5] (0,0,0) circle (\rd);
\draw[thick, black,dashed] (\ax,\ay,0) arc (\az:{\az+180}:\rd);
\draw[thick, black] (-\ax,-\ay,0) arc ({\az+180}:{\az+360}:\rd);
\draw (-\ax,-\ay,0) [tdplot_screen_coords] -- +(-\ticl,0);
\draw (-\ax,-\ay,\zt) [tdplot_screen_coords] -- +(-\ticl,0);
\path (-\ax,-\ay,0) [tdplot_screen_coords] ++(-\dsth,0) coordinate (DHB);
\path (-\ax,-\ay,\zt) [tdplot_screen_coords] ++(-\dsth,0) coordinate (DHT);
\draw[thin,black,dashed] (0,0,0) -- +(0,0,\zt);
% fadings
\shade [opacity=0.5,left color=transparent!0,right color=transparent!50]
(\ax,\ay,0) arc ({\az+360}:{\az+180}:\rd)
-- (-\ax,-\ay,\zt) arc ({\az+180}:{\az+360}:\rd) -- cycle;
%
\draw[thin,black,dashed] (\ax,\ay,0) -- +(0,0,\zt) (-\ax,-\ay,0) -- +(0,0,\zt);
\path[fill=gray!30,fill opacity=0.2] (0,0,\zt) circle (\rd);
\draw[thick, black] (0,0,\zt) circle (\rd);
\path ($ (DHB)!0.5!(DHT) $) node (TXH) {$h$}; % [fill=white]
\draw[<-] (DHB) -- (TXH.south);
\draw[->] (TXH.north) -- (DHT);
\path (-\ay,\ax,\zt) [tdplot_screen_coords] ++(0,\ticl) coordinate (TRL)
++(\rd,0) coordinate (TRR);
\path (-\ay,\ax,\zt) [tdplot_screen_coords] ++(0,\dsth) coordinate (DRL)
++(\rd,0) coordinate (DRR);
\draw (0,0,\zt) -- (TRL) (\ax,\ay,\zt) -- (TRR);
\path ($ (DRL)!0.5!(DRR) $) node (TXR) {$r$}; % [fill=white]
\draw[<-] (DRL) -- (TXR.west);
\draw[->] (TXR.east) -- (DRR);
\end{tikzpicture}
}
\end{center}
%Ende (Zeichnung).

Das Volumen $V$ des Zylinders ist das Produkt aus dem Fl"acheninhalt der 
Kreisscheibe $G$ mit dem Radius $r$ und der H"ohe $h$: 
Es ist $V = G \cdot h = \pi \cdot r^2 \cdot h$.

Der Fl"acheninhalt $O$ der Oberfl"ache ergibt sich als Summe aus dem doppelten
Inhalt der Kreisscheibe $G$ und dem Inhalt des Mantels $M$.
Mit dem Umfang $U = 2 \cdot \pi \cdot r$ der gegebenen Kreisscheibe gilt 
$O = 2 \cdot G + M = 2 \cdot \pi \cdot r^2 + 2 \cdot \pi \cdot r \cdot h %
 = 2 \cdot \pi \cdot r \cdot (r + h)$.
\end{MXInfo}

Wenn die Kreisscheibe nicht verschoben, sondern rotiert wird, wobei die 
Drehachse durch den Mittelpunkt und einen Randpunkt verl"auft, ergibt sich
eine Kugel.

\begin{MXInfo}{Kugel}
Gegeben ist eine Kreisscheibe mit Mittelpunkt $M$ und Radius $r$. Rotiert die
Kreisscheibe um $M$ und einen Randpunkt, erh"alt man eine Kugel mit 
Radius $r$.

%Zeichnung Kugel:
\begin{center}
\MTikzAuto{%
\tdplotsetmaincoords{60}{30}
\begin{tikzpicture}[tdplot_main_coords,>=latex]
\def\az{30}
\def\pl{60}
\def\rd{2.5}
\def\ticl{0.8}
\def\dsth{0.5}
\pgfmathparse{\rd*cos(\az)}\let\ax=\pgfmathresult
\pgfmathparse{\rd*sin(\az)}\let\ay=\pgfmathresult
% fadings
\begin{scope}[tdplot_screen_coords]
  \shade[ball color=gray!50] (0,0,0) circle (\rd);
\end{scope}
\draw[thin,black,dashed] (0,0,0) -- +(0,0,\rd);
\draw[thick, black,dashed] (\ax,\ay,0) arc (\az:{\az+180}:\rd);
\begin{scope}[tdplot_screen_coords]
  \fill[opacity=0.5,color=white] (0,0,0) circle (\rd);
  \draw[black,dashed] (0,0,0) circle (\rd);
\end{scope}
\draw[thick,black,dashed] (-\ax,-\ay,0) arc ({\az+180}:{\az+360}:\rd);
%
\draw (-\ax,-\ay,0) [tdplot_screen_coords] -- +(-\ticl,0);
\draw (-\ax,-\ay,\rd) [tdplot_screen_coords] ++(-\ticl,0) 
[tdplot_main_coords] -- (0,0,\rd);
\path (-\ax,-\ay,0) [tdplot_screen_coords] ++(-\dsth,0) coordinate (DHB);
\path (-\ax,-\ay,\rd) [tdplot_screen_coords] ++(-\dsth,0) coordinate (DHT);
\path ($ (DHB)!0.5!(DHT) $) node (TXH) {$r$}; % [fill=white]
\draw[<-] (DHB) -- (TXH.south);
\draw[->] (TXH.north) -- (DHT);
\end{tikzpicture}
}
\end{center}
%Ende (Zeichnung).

Das Volumen $V$ der Kugel ist $V = \frac{4}{3} \cdot \pi \cdot r^3$.

Der Fl"acheninhalt $O$ der Oberfl"ache ist durch 
$O = 4 \cdot \pi \cdot r^2$ gegeben.
\end{MXInfo}

Eine Kugel kann auch als der K"orper beschrieben werden, der aus allen Punkten 
besteht, deren Abstand von $M$ kleiner oder gleich $r$ ist (siehe auch 
Kapitel \MRef{VBKM10}).

In dieser Betrachtungsweise ist ein Prisma ein K"orper aus allen Punkten, 
die auf einer Verbindungslinie von Grundfl"ache und ihrer Kopie liegen. 

Im Folgenden werden zwei Variationen dieses Vorgehens betrachtet.
Zun"achst wird wieder von einem Vieleck als Grundfl"ache ausgegangen. 
Anstelle einer Kopie der Grundfl"ache wird jetzt nur ein einzelner Punkt 
vorgegeben.

\begin{MXInfo}{Pyramide}
Gegeben ist ein Vieleck $G$ und ein Punkt $S$ mit dem Abstand $h > 0$ von
$G$.
Eine Pyramide mit der Grundfl"ache $G$ und der Spitze $S$ ist der K"orper,
der aus allen Punkten besteht, die auf einer Strecke von $S$ zu einem 
Punkt auf der Grundfl"ache $G$ liegen.

%Zeichnung Pyramide:
\begin{center}
\MTikzAuto{%
\tdplotsetmaincoords{60}{40}
\begin{tikzpicture}[tdplot_main_coords,>=latex]
\def\zt{5}
\def\rd{2.5}
\def\ticl{0.8}
\def\dsth{0.5}
% coordinate number, radius, azimuth
\foreach \nc/\rc/\ac in {0/\rd/0,1/\rd/120,2/\rd/240} {
\pgfmathparse{\rc*cos(\ac)}\let\ax=\pgfmathresult
\pgfmathparse{\rc*sin(\ac)}\let\ay=\pgfmathresult
\node (PC-\nc) at (\ax,\ay,0) {};
}
%
\path[fill=gray!60,opacity=0.5] (PC-0.center) 
\foreach \nc in {1,2} { -- (PC-\nc.center) {} } -- cycle;
\draw[thick, black] (PC-2.center) -- (PC-0.center);
\draw[thick, black, dashed] (PC-0.center) --  (PC-1.center) -- (PC-2.center);
\foreach \nc in {0,2} { \draw[thick] (PC-\nc.center) -- (0,0,\zt); };
\foreach \nc in {1} { \draw[thick,dashed] (PC-\nc.center) -- (0,0,\zt); };
\draw[thin,black,dashed] (0,0,0) -- +(0,0,\zt);
% Legende
\draw (0,0,0) [tdplot_screen_coords] -- ++({-\rd-\ticl},0);
\draw (0,0,\zt) [tdplot_screen_coords] -- ++({-\rd-\ticl},0);
\path (0,0,0) [tdplot_screen_coords] ++({-\rd-\dsth},0) coordinate (BG);
\path (0,0,\zt) [tdplot_screen_coords] ++({-\rd-\dsth},0) coordinate (BT);
\path ($ (BG)!0.5!(BT) $) node (TXT) {$h$}; % [fill=white]
\draw[<-] (BG) -- (TXT.south);
\draw[->] (TXT.north) -- (BT);
%
\path (PC-2.center) [tdplot_screen_coords] +(0,-\dsth) coordinate (DAL)
+(0,-\ticl) coordinate (TAL);
\path (PC-0.center) [tdplot_screen_coords] +(0,-\dsth) coordinate (DAR)
+(0,-\ticl) coordinate (TAR);
\draw (PC-2.center) -- (TAL) (PC-0.center) -- (TAR);
\path ($ (DAL)!0.5!(DAR) $) node (TXA) {$a$}; % [fill=white]
\draw[<-] (DAL) -- (TXA.west);
\draw[->] (TXA.east) -- (DAR);
\end{tikzpicture}
}
\end{center}

In der Zeichnung ist eine Pyramide mit einer dreieckigen Grundfl"ache 
dargestellt.
%Ende (Zeichnung).

Das Volumen $V$ der Pyramide ist proportional zum Fl"acheninhalt der 
Grundfl"ache $G$ und der H"ohe $h$. 
Es gilt $V = \frac{1}{3} \cdot G \cdot h$.

Der Fl"acheninhalt $O$ der Oberfl"ache ergibt sich als Summe aus dem Inhalt 
der Grundfl"ache $G$ und des Mantels $M$. Dabei ist der Fl"acheninhalt des 
Mantels die Summe seiner Dreiecksfl"achen $D_k$ ($1 \leq k \leq n$).
Damit ist $O = G + M = G + D_1 + \ldots + D_n$.
\end{MXInfo}

In besonderen Situationen ergeben sich einfache Formeln, mit denen 
das Volumen und die Oberfl"ache berechnet werden k"onnen. Ein Beispiel ist
die oben abgebildete Pyramide. Dort ist die Grundfl"ache ein gleichseitiges 
Dreieck. Die folgende Aufgabe bietet die Gelegenheit, f"ur einen speziellen
Fall dieser Situation eine Formel f"ur %das Volumen und 
die Oberfl"ache mit den 
Eigenschaften gleichseitiger Dreiecke zu finden.

\begin{MExercise}
Berechnen Sie %das Volumen $V$ und 
die Oberfl"ache $O$ einer Pyramide, deren 
Seiten alle gleichseitige Dreiecke mit der Seitenl"ange $a$ sind.

Antwort: 
\MEquationItem{$O$}{\MLSimplifyQuestion{30}{a*a*sqrt(3)}{10}{a}{4}{512}{ExM05TPyr2}}

\begin{MHint}{L"osung}
Eine Pyramide, deren Seiten alle gleichseitige Dreiecke sind, hat insgesamt 
vier Seiten: eine dreieckige Grundfl"ache, an die drei weitere Seitenfl"achen 
grenzen. Da alle Seitenfl"achen gleich sind, ist die Oberfl"ache der Pyramide
durch $O = 4 \cdot F$ gegeben, wobei $F$ den Fl"acheninhalt eines gleichseitigen
Dreiecks angibt. Die H"ohe $\ell$ eines gleichseitigen Dreiecks mit Seitenl"ange 
$a$ ergibt sich mit dem Satz des Pythagoras aus 
\[
a^2 = \ell^2 + \left(\frac{a}{2}\right)^2 %%
\]
zu 
$\ell = \sqrt{a^2 - \frac{1}{4} \cdot a^2} %
 = \frac{1}{2} \cdot a \cdot \sqrt{3}$.
Damit ist 
$F = \frac{1}{2} \cdot a \cdot \ell = \frac{1}{4} \cdot a^2 \cdot \sqrt{3}$, 
sodass
$O = 4 \cdot F = a^2 \cdot \sqrt{3}$ gilt.
\end{MHint}
\end{MExercise}


Die "Uberlegungen von oben, dass Prismen und Zylinder ein gemeinsames 
Bauprinzip bei unterschiedlichen Grundfl"achen haben, kann auf die neue 
Situation einer Pyramide "ubertragen werden.
Man erh"alt einen weiteren K"orper, wenn anstatt eines Vielecks wie bei der 
Pyramide eine Kreisscheibe als Grundfl"ache verwendet wird.

\begin{MXInfo}{Kegel}
Gegeben ist eine Kreisscheibe $G$ mit Radius $r$.
Au"serdem wird ein Punkt $S$ mit dem Abstand $h > 0$ von $G$ betrachtet.
Ein Kegel mit der Grundfl"ache $G$ und der Spitze $S$ ist der K"orper,
der aus allen Punkten besteht, die auf einer Strecke von $S$ zu einem 
Punkt auf der Grundfl"ache $G$ liegen.

%Zeichnung Kegel:
\begin{center}
\MTikzAuto{
\def\az{30}
\def\pl{60}
\tdplotsetmaincoords{\pl}{\az}
\begin{tikzpicture}[tdplot_main_coords,>=latex]
\def\zt{4}
\def\rd{2}
\def\ticl{0.8}
\def\dsth{0.5}
\pgfmathparse{\rd*cos(\az)}\let\ax=\pgfmathresult
\pgfmathparse{\rd*sin(\az)}\let\ay=\pgfmathresult
\pgfmathparse{\rd/\zt/tan(\pl)}\let\stm=\pgfmathresult
\pgfmathparse{less(abs(\stm),1)}\let\cres\pgfmathresult
\pgfmathparse{greater(\stm,0)}\let\crsg\pgfmathresult
\def\ccmp{1}
%%\typeout{cres=\cres, crsg=\crsg}
\if\cres\ccmp
  \pgfmathparse{\rd*cos(\pl)*\stm}\let\smy=\pgfmathresult
  \pgfmathparse{\rd*sqrt(1-\stm*\stm)}\let\smx=\pgfmathresult
  \pgfmathparse{asin(\stm)}\let\astm=\pgfmathresult
\else
  \def\smy{1}\def\smx{0}\def\astm{90}
\fi
\pgfmathparse{\rd*cos(\az+\astm)}\let\axl=\pgfmathresult
\pgfmathparse{\rd*sin(\az+\astm)}\let\ayl=\pgfmathresult
\path[fill=gray!60,opacity=0.5] (0,0,0) circle (\rd);
%%\draw[thick,black,dashed] (\ax,\ay,0) arc (\az:{\az+180}:\rd);
%%\draw[thick,black] (-\ax,-\ay,0) arc ({\az+180}:{\az+360}:\rd);
\if\crsg\ccmp
  \draw[thick,black,dashed] (\axl,\ayl,0) arc ({\az+\astm}:{\az+180-\astm}:\rd);
  \draw[thick,black] (\axl,\ayl,0) arc ({\az+360+\astm}:{\az+180-\astm}:\rd);
\else
  \draw[thick,black] (0,0,0) circle (\rd);
\fi
\draw (-\ax,-\ay,0) [tdplot_screen_coords] -- +(-\ticl,0);
\draw (-\ax,-\ay,\zt) [tdplot_screen_coords] ++(-\ticl,0) 
[tdplot_main_coords] -- (0,0,\zt);
\path (-\ax,-\ay,0) [tdplot_screen_coords] ++(-\dsth,0) coordinate (DHB);
\path (-\ax,-\ay,\zt) [tdplot_screen_coords] ++(-\dsth,0) coordinate (DHT);
\draw[thin,black,dashed] (0,0,0) -- +(0,0,\zt);
% fadings
\if\cres\ccmp
  \shade [opacity=0.5,left color=transparent!0,right color=transparent!50]
  (\axl,\ayl,0) arc ({\az+360+\astm}:{\az+180-\astm}:\rd) -- (0,0,\zt) -- cycle;
\else
  \shade [opacity=0.5,left color=transparent!0,right color=transparent!50]
  (0,0,0) circle (\rd);
\fi
%
\if\cres\ccmp
  \draw[thin,black,dashed] (0,0,\zt) [tdplot_screen_coords] -- (\smx,\smy);
  \draw[thin,black,dashed] (0,0,\zt) [tdplot_screen_coords] -- (-\smx,\smy);
\fi
\path ($ (DHB)!0.5!(DHT) $) node (TXH) {$h$}; % [fill=white]
\draw[<-] (DHB) -- (TXH.south);
\draw[->] (TXH.north) -- (DHT);
\path (0,0,\zt) [tdplot_screen_coords] ++(0,\ticl) coordinate (TRL)
++(\rd,0) coordinate (TRR);
\path (0,0,\zt) [tdplot_screen_coords] ++(0,\dsth) coordinate (DRL)
++(\rd,0) coordinate (DRR);
\draw (0,0,\zt) -- (TRL) (\ax,\ay,0) -- (TRR);
\path ($ (DRL)!0.5!(DRR) $) node (TXR) {$r$}; % [fill=white]
\draw[<-] (DRL) -- (TXR.west);
\draw[->] (TXR.east) -- (DRR);
\end{tikzpicture}
}
\end{center}
%Ende (Zeichnung).

Das Volumen $V$ des Kegels ist proportional zum Fl"acheninhalt der 
Kreisscheibe $G$ und der H"ohe $h$. 
Es gilt $V = \frac{1}{3} \cdot G \cdot h %
 = \frac{1}{3} \cdot \pi \cdot r^2 \cdot h$.

Ein Kegel in der hier betrachteten Situation, in der sich die Spitze des 
Kegels senkrecht "uber dem Mittelpunkt der Kreisscheibe befindet, wird 
genauer als \MEntry{gerader Kreiskegel}{Kreiskegel, gerade} bezeichnet.

Der Fl"acheninhalt der Oberfl"ache eines geraden Kreiskegels ist die Summe 
des Inhalts der Kreisscheibe $G$ und des Mantels $M$. Wenn $\ell$ der Abstand 
der Kegelspitze vom Rand der Kreisscheibe ist, 
gilt mit dem Kreisumfang $U =  2 \pi r$ dann
$O = G + M = \pi \cdot r^2 + \pi \cdot r \cdot \ell %
 = \pi \cdot r \cdot (r + \ell)$.
\end{MXInfo}

In obiger Zeichnung eines Kegels wird die H"ohe $h$ genannt. Die angebenen 
Formeln enthalten jedoch den Abstand der Spitze vom Rand der Kreisscheibe.
Wie h"angen beide Gr"o"sen zusammen?

\begin{MExercise}
Stellen Sie sich vor, der gerade Kegel wird von der Spitze eben 
durch den Mittelpunkt der Kreisscheibe in zwei gleiche Teile zerlegt.
Dann ist eine Seite der beiden Teile jeweils ein Dreieck, dessen Seiten durch  
die H"ohe $h$, den Abstand $\ell$ und den Radius $r$  
der Kreisscheibe bestimmt werden.
Hierbei werden $r$ und $h$ als gegeben angesehen.

\begin{MExerciseItems}
\item
Beschreiben Sie $\ell$ in Abh"angigkeit von $h$ und $r$:
\par
\MEquationItem{$\ell$}{\MLSimplifyQuestion{30}{sqrt(r^2 + h^2)}{10}{r,h}{4}{512}{ExM05Zyla}}

\begin{MHint}{L"osung}
Die L"ange $\ell$ ist die Hypotenuse des Dreiecks, dessen Katheten die H"ohe
$h$ des geraden Kegels und der Radius $r$ der Kreisschreibe sind. Aus dem Satz
des Pythagoras ergibt sich $\ell = \sqrt{r^2 + h^2}$.
\end{MHint}

\item
Geben Sie die Oberfl"ache $O$ des Kegels in Abh"angigkeit von $h$ und $r$ an:
\par
 \MEquationItem{$O$}{\MLSimplifyQuestion{30}{pi*r*(r + sqrt(r^2 + h^2))}{10}{r,h}{4}{512}{ExM05Zylb}}

\begin{MHint}{L"osung}
Es wird das Ergebnis $\ell = \sqrt{r^2 + h^2}$ der ersten Teilaufgabe in die 
oben angegebene Formel f"ur die Oberfl"ache $O$ eingesetzt. Damit erh"alt man
\[
O = \pi \cdot r \cdot (r + \ell) %
 = \pi \cdot r \cdot \left(r + \sqrt{r^2 + h^2}\right) %
 = \pi \cdot r^2 \cdot \left(1 + \sqrt{1 + \left(\frac{h}{r}\right)^2} \right). %%
\]
Mit der zuletzt genannten Formel sieht man, wie sich die Oberfl"ache $O$ im 
Vergleich zur Kreisfl"ache $G = \pi \cdot r^2$ mit dem Verh"altnis von H"ohe $h$
zu Radius $r$ "andert.
\end{MHint}
\end{MExerciseItems}
\end{MExercise}

\end{MXContent}
%end of subsection: ElementareKoerper.


%Uebungen zum Abschnitt Elementare Koerper:
\begin{MExercises}
\MDeclareSiteUXID{VBKM05_ElementareKoerper_Exercises}

\begin{MExercise}
%Berechnen Sie das Volumen eines Prismas der H"ohe $h = 8\MEinheit{cm}$, dessen 
%Grundfl"ache ein gleichseitiges Sechseck mit Seitenl"ange $a = 3\MEinheit{cm}$ 
%ist.
%\par
%Hinweis: Das Sechseck kann in sechs gleichseitige Dreiecke zerlegt werden.
Berechnen Sie das Volumen eines Prismas der H"ohe $h = 8\MEinheit{cm}$, dessen 
Grundfl"ache ein Dreieck ist, von dem zwei Seiten $5\MEinheit{cm}$ lang sind,
und eine Seite $6\MEinheit{cm}$ lang ist.
\par
Antwort: \MLParsedQuestion{15}{96}{10}{ExM05EK1}$\MEinheit{cm}^3$
\begin{MHint}{L"osung}
Die Grundfl"ache des Prismas ist ein gleichseitiges Dreieck mit den gleich
langen Seiten $a$ und $b$ und einer weiteren Seite $c$. Deshalb kann die 
H"ohe $h_c$ auf $c$ mit dem Satz des Pythagoras berechnet werden. Mit den 
angegebenen Werten erh"alt man die H"ohe 
$h_c = \sqrt{(5\MEinheit{cm})^2 - \left(\frac{6\MEinheit{cm}}{2}\right)^2} %
 = 4\MEinheit{cm}$ und damit
den Fl"acheninhalt 
$G = \frac{1}{2} \cdot 6\MEinheit{cm} \cdot 4\MEinheit{cm} = 12\MEinheit{cm}^2$.
Das gesuchte Volumen $V$ des Prismas ist
$V = G \cdot h = 12\MEinheit{cm}^2 \cdot 8\MEinheit{cm} %
 =  96\MEinheit{cm}^3$.
\end{MHint}
\end{MExercise}

\begin{MExercise}
Die Oberfl"ache eines Zylinders mit einer H"ohe von 
$h = \MZahl{6}{0}\MEinheit{cm}$ soll mit einer Farbfolie beklebt werden. 
Die Oberfl"ache soll $O = \MZahl{200}{0}\MEinheit{cm}^2$ gro"s sein. Berechnen 
Sie den Durchmesser $d$ der Kreisfl"ache und das Volumen $V$ des Zylinders.
Verwenden Sie als N"aherung f"ur $\pi$ den Wert $\MZahl{3}{1415}$ und runden 
Sie Ihre Ergebnisse auf eine Nachkommastelle (dies entspricht einer Genauigkeit
von einem Millimeter beziehungsweise Qubikmillimeter).

Antworten:
\begin{MExerciseItems}
\item Durchmesser %
%Ergebnis auf eine Nachkommastelle:
 \MEquationItem{$d$}{\MLParsedQuestion{15}{6.8}{10}{ExM05EK21}}$\MEinheit{cm}$
%Ergebnis auf zwei Nachkommastelle:
% \MEquationItem{$d$}{\MLParsedQuestion{15}{6.78}{10}{ExM05EK21}}$\MEinheit{cm}$

\begin{MHint}{L"osung}
Die Oberfl"ache eines Zylinders ergibt sich aus der Summe der Grundfl"achen und 
der Mantelfl"ache:
\[
O = 2 \cdot \pi \cdot \left(\frac{d}{2}\right)^2 + \pi \cdot d \cdot h %
 = \frac{1}{2} \cdot \pi \cdot d^2 + \pi \cdot h \cdot d %
 = \frac{\pi}{2} \cdot \left( d^2 + 2 \cdot h \cdot d \right) \MDFPeriod %
\]
In der Aufgabe sind die Oberfl"ache $O$ und die H"ohe $h$ des Zylinders gegeben.
Damit kann der gesuchte Kreisdurchmesser $d$ des Bodens berechnet werden, indem 
eine L"osungsformel f"ur eine quadratische Gleichung verwendet wird oder eine 
quadratische Erg"anzung vorgenommen wird (siehe Kapitel \MRef{VBKM01}):
Multiplikation mit $\frac{2}{\pi}$ und Addition von $h^2$ f"uhrt auf
\[
(d + h)^2 = d^2 + 2 h d + h^2 = \frac{2 \cdot O}{\pi} + h^2 \MDFPSpace, %%
\]
woraus 
\[ 
d = -h + \sqrt{\frac{2 \cdot O}{\pi} + h^2} %%
\]
folgt.
Mit den angegebenen Zahlenwerten ergibt sich
\[
d \approx -\MZahl{6}{0}\MEinheit{cm} %
 + \sqrt{\frac{2 \cdot \MZahl{200}{0}\MEinheit{cm}^2}{\MZahl{3}{1415}} %
      + \MZahl{6}{0}^2\MEinheit{cm}^2} %
  = -\MZahl{6}{0}\MEinheit{cm} %
 + \sqrt{\frac{\MZahl{400}{0}\MEinheit{cm}^2}{\MZahl{3}{1415}} %
      + \MZahl{36}{0}\MEinheit{cm}^2} %
%Ergebnis auf eine Nachkommastelle:
% = \MZahl{6}{8}\MEinheit{cm} %%
%Ergebnis auf zwei Nachkommastellen:
 \approx \MZahl{6}{8}\MEinheit{cm} %%
\]
f"ur den gesuchten Durchmesser.
\end{MHint}
\item Volumen %
%Ergebnis auf eine Nachkommastelle:
\MEquationItem{$V$}{\MLParsedQuestion{15}{217.9}{10}{ExM05EK22}}$\MEinheit{cm}^3$
(mit dem gerundeten Ergebnis der ersten Teilaufgabe)
%Ergebnis auf zwei Nachkommastellen:
%\MEquationItem{$V$}{\MLParsedQuestion{15}{216.61}{10}{ExM05EK22}}$\MEinheit{cm}^3$

\begin{MHint}{L"osung}
Mit dem in der vorherigen Teilaufgabe berechneten Durchmesser 
$d = \MZahl{6}{8}\MEinheit{cm}$ erh"alt man das Volumen $V$ des Zylinders:
\begin{eqnarray*}
V = \pi \cdot \left(\frac{d}{2}\right)^2 \cdot h %
 \approx \MZahl{3}{1415} \cdot \left(\frac{\MZahl{6}{8}\MEinheit{cm}}{2}\right)^2 %
   \cdot \MZahl{6}{0}\MEinheit{cm} %
 & = & \MZahl{3}{1415} \cdot \MZahl{11}{56}\MEinheit{cm}^{2} %
   \cdot \MZahl{6}{0}\MEinheit{cm}  \\
 & = & \MZahl{3}{1415} \cdot \MZahl{69}{36}\MEinheit{cm}^{3} \\
%Ergebnis auf eine Nachkommastelle:
 & \approx & \MZahl{217}{9}\MEinheit{cm}^{3} \MDFPeriod %
%Ergebnis auf zwei Nachkommastellen:
% & = & \MZahl{216}{61}\MEinheit{cm}^{3} \MDFPeriod %
\end{eqnarray*}
Anmerkung: Wenn das Ergebnis des Durchmessers 
$d \approx \MZahl{6}{78}\MEinheit{cm}$ mit zwei Nachkommastellen berechnet
wird, erh"alt man $V \approx \MZahl{216}{61}\MEinheit{cm}^{3}$ als Wert f"ur 
das Volumen.
\end{MHint}
\end{MExerciseItems}
\end{MExercise}

\begin{MExercise}
Es ist ein Holzst"uck in Form eines Quaders mit dem Volumen $V_0$ gegeben. 
Der Quader ist $h = 120\MEinheit{cm}$ hoch, und die Grundfl"ache 
besteht aus 
einem Quadrat mit einer Seitenl"ange von $s = 40\MEinheit{cm}$.
Aus dem Holzst"uck wird ein zylinderf"ormiges Loch der H"ohe $h$ mit einem 
Durchmesser von $d = 20\MEinheit{cm}$ {\glqq}mittig{\grqq} ausgebohrt (das 
hei"st, der Schnittpunkt der Diagonalen der quadratischen Grundfl"ache bildet 
den Mittelpunkt der Kreisscheibe des Zylinders). 
Verwenden Sie als N"aherung f"ur $\pi$ den Wert $\MZahl{3}{1415}$ und runden 
Sie Ihre Ergebnisse auf ganze Zahlen.
Berechnen Sie
\begin{MExerciseItems}
\item das Volumen $V_Z$ des ausgebohrten Hohlraums:
\par
\MEquationItem{$V_Z$}{\MLParsedQuestion{15}{37698}{10}{ExM05EK21b}}$\MEinheit{cm}^3$

\begin{MHint}{L"osung}
Das Volumen des Zylinders ist
\[
V_Z = \pi \cdot \left(\frac{d}{2}\right)^2 \cdot h %
 \approx \MZahl{3}{1415} \cdot \left(\frac{20\MEinheit{cm}}{2}\right)^2 %
   \cdot 120\MEinheit{cm} %
 = \MZahl{3}{1415} \cdot 12000\MEinheit{cm}^{3} %
 = 37698\MEinheit{cm}^{3} %
\]
\end{MHint}
%
\item den prozentualen Anteil des Volumens $V_1$ des neuen Holzst"ucks, 
das nach dem Ausbohren von $V_0$ noch vorhanden ist:
\par
Antwort: \MLParsedQuestion{15}{80}{10}{ExM05EK21c} $\%$

\begin{MHint}{L"osung}
Das Volumen $V_0$ des Holzst"ucks ist
\[
V_0 = s^2 \cdot h %
 = \left(40\MEinheit{cm}\right)^2 \cdot 120\MEinheit{cm} %
 = 1600 \cdot 120\MEinheit{cm}^{3} %
 = 16 \cdot 12000\MEinheit{cm}^{3} %
 = 192000\MEinheit{cm}^{3} %
\]
Damit ist der Anteil $p_Z$ des ausgebohrten Zylinders durch
\[
p_Z = \frac{V_Z}{V_0} %
 = \frac{\pi \cdot 12000\MEinheit{cm}^3}{16 \cdot 12000\MEinheit{cm}^3} %
 \approx \frac{\MZahl{3}{1415}}{16} \approx \MZahl{19}{6} \approx 20 \% %%
\]
und somit $p = (100 - 20) \% = 80 \%$ der prozentuale Anteil des neuen 
Holzst"ucks im Vergleich zum urspr"unglichen Holzquader.
%
%Anmerkung: Wenn mit dem gerundeten Ergebnis aus der ersten Teilaufgaben 
%weiter gerechnet wird, erh"alt man
%$p = 1 - \frac{V_Z}{V_0} %
% = 1 - \frac{37698\MEinheit{cm}^{3}}{192000\MEinheit{cm}^{3}} %
% \approx \MZahl{0}{803} \approx{80} \%$
%als prozentualen Anteil, und damit in diesem Beispiel dasselbe gerundete 
%Ergebnis.
\end{MHint}
\end{MExerciseItems}
\end{MExercise}
\end{MExercises}
%end of exercises: Uebungen zum Abschnitt Elementare Koerper (jgl).



%content: Abschnitt 6: Winkelfunktionen

\MSubsection{Winkelfunktionen: Sinus und Co.}
\MLabel{M05_Trigonometrie}

\begin{MIntro}
\MDeclareSiteUXID{VBKM05_Trigonometrie_Intro}
%Je nach Sonnenstand ist der Schatten einer Person oder eines Gegenstandes 
%unterschiedlich gro"s. Indem die Verh"altnisse in Abh"angigkeit des Winkels
%erfasst werden, kann man die zugeh"origen L"angen berechnen. Die Betrachtung
%von funktionalen Zusammenh"ange, die sich hier aus geometrischen Beobachtungen 
%ergeben und zur Definition von Sinus, Cosinus und weiteren trigonometrischer 
%Gr"o"sen f"uhren, werden im n"achsten Modul allgemein besprochen.

An Bergstra"sen werden Schilder aufgestellt, wenn es sehr steil bergab geht.
Mit einer Prozentzahl wird beschrieben, wie stark das Gel"ande relativ 
zu einer horizontalen Bewegung abf"allt.
Systematisch wurden Fragen nach den Bedingungen von Bewegungen an einer 
{\glqq}schiefen Ebene{\grqq} in der Physik von Galileo Galilei untersucht.
Die Ergebnisse sind auch f"ur technische Konstruktionen relevant. 

Als mathematisches Hilfsmittel dienen Winkelfunktionen. Sie beschreiben einen
geometrischen Sachverhalt mittels eines rechnerischen Ausdrucks.
%Im Bild von oben ausgedr"uckt, wird in diesem Abschnitt "uber Winkelfunktionen
%beschrieben, wie der Zusammenhang zwischen der prozentualen Angabe des 
Wie dieser Zusammenhang zwischen der prozentualen Angabe des Gef"alles und 
dem zugeh"origen Winkel formuliert werden kann, wird in diesem Abschnitt
beschrieben. Eine erste Untersuchung der Eigenschaften von Winkelfunktionen
gibt einen Eindruck von den vielf"altigen Anwendungsm"oglichkeiten, die
weit "uber die Geometrie hinausreichen und in den kommenden Abschnitten 
immer wieder aufgegriffen werden.

\end{MIntro}

\begin{MXContent}{Trigonometrie am Dreieck}{Dreieck}{STD}
\MLabel{Abschnitt:TrigonometrieAmDreieck}
\MDeclareSiteUXID{VBKM05_Trigonometrie_Content}

F"ahrt man eine Stra"se mit einem Gef"alle von f"unf Prozent bergab, 
nimmt die H"ohe alle hundert Meter um f"unf Meter ab. Dabei wird der 
H"ohenunterschied im Vergleich zur Horizontalen betrachtet.

\begin{center}
\MTikzAuto{%
\begin{tikzpicture}[line width=1pt]
\coordinate (A) at (0,0);
\coordinate (B) at (10,0);
\coordinate (C) at (10,0.5);
\path (A) -- node[below] {$a = 100\MEinheit{m}$} (B) %
 -- node[right] {$b = 5\MEinheit{m}$} (C) %
 -- node[above,rotate={atan(0.05)}] {Steigung $5\%$} (A);
\draw (A) -- (B) -- (C) -- cycle;
\end{tikzpicture}
}
\end{center}

Demnach betr"agt das Gef"alle $100\%$, wenn der H"ohenunterschied
$100\MEinheit{m}$ zwischen zwei Positionen betr"agt, deren horizontaler 
Abstand $100\MEinheit{m}$ betr"agt. Geometrisch formuliert, ist 
die Verbindungsstrecke zwischen den beiden Punkte eine Diagonale 
eines Quadrats. Damit hat der Winkel zwischen der horizontalen 
Vergleichsstrecke und der Diagonalen, auf der man sich bewegt, das
Winkelma"s von $45\MGrad$. 

\begin{center}
\MTikzAuto{%
\begin{tikzpicture}[line width=1pt]
\coordinate (A) at (0,0);
\coordinate (B) at (3,0);
\coordinate (C) at (3,3);
\coordinate (D) at (0,3);
\path (A) -- node[below] {$100\MEinheit{m}$} (B) -- %
 node[below,rotate=90] {$100\MEinheit{m}$}(C);
\draw (A) -- (B) -- (C) -- cycle;
\draw[style=dotted] (C) -- (D) -- (A) -- cycle;
\draw (1,0) arc(0:45:1);
\draw[style=dotted] (0,1) arc(90:45:1);
\end{tikzpicture}
}
\end{center}

Dies kann man auch so formulieren: Einem Winkel von $45\MGrad$ entspricht
eine Steigung von $\frac{100\MEinheit{m}}{100\MEinheit{m}} = 1$, das hei"st 
ein Streckenverh"altnis von $1$ von vertikaler zu horizontaler Strecke. 
Aufgrund der Strahlens"atze ist das Streckenverh"altnis von den L"angen der 
einzelnen Strecken unabh"angig. Es h"angt nur davon ab, wie die Strahlen
zueinander verlaufen, also wie gro"s der Winkel zwischen ihnen ist.
Wenn diese Zuordnung zwischen Winkel und Streckenverh"altnis auch f"ur andere
Winkel bekannt ist, kann man damit viele konstruktive Aufgabenstellungen 
l"osen. 
Beispielsweise kann die H"ohe zu einem gegebenen Winkel bestimmt werden.

Schon die Frage, welches Verh"altnis zu einem Winkel von $30\MGrad$ geh"ort,
zeigt allerdings, dass die Bestimmung der Zuordung zwischen Winkel und 
Streckenverh"altnis im Allgemeinen jedoch nicht so einfach ist.
%Schon die Berechnung, dass zu einem Gef"alle oder einer Steigung von 
%$\MZahl{57}{7}\%$ der Winkel rund $30\MGrad$ gro"s ist, ist bereits nicht 
%mehr so einfach.
%Schon die Berechnung, dass zu einem Gef"alle oder einer Steigung von $50\%$ 
%der Winkel rund $\MZahl{26}{565}\MGrad$ gro"s ist, ist bereits nicht mehr 
%so einfach.
Deshalb wurden die aufw"andig bestimmten Werte anfangs in gro"sen Tafelwerken 
aufgeschrieben, um dann einfach nachgeschlagen werden zu k"onnen.
Inzwischen sind die Werte mittels Taschenrechner und Computer praktisch 
"uberall verf"ugbar. Die gebr"auchlichsten Zuordnungen von Winkel zu einem 
Streckenverh"altnis werden im Folgenden vorgestellt. 
Sie werden Winkelfunktionen oder trigonometrische Funktionen genannt, und 
das mathematische Gebiet, das sich mit ihren Eigenschaften befasst, hei"st
\MEntry{Trigonometrie}{Trigonometrie}.

\begin{MXInfo}{Die trigonometrischen Funktionen im rechtwinkligen Dreieck}%
\MLabel{M05_DefinitionWinkelfunktionen}%
Es werden die gebr"auchlichsten \MEntry{Winkelfunktionen}{Winkelfunktionen}
als Zuordnungen zwischen Winkel und Seitenverh"altnissen in einem rechtwinkligen 
Dreieck beschrieben. Die Winkelfunktionen hei"sen auch 
\MEntry{trigonometrische Funktionen}{trigonometrische Funktion}.
Dabei bezeichnet $x$ einen Winkel in einem rechtwinkligen Dreieck, der 
kein rechter Winkel ist.
Die \MEntry{Gegenkathete}{Gegenkathete} ist die Seite, die dem Winkel $x$ 
gegen"uberliegt, und die andere Kathete wird \MEntry{Ankathete}{Ankathete} 
genannt.

\begin{center}
\MTikzAuto{%
\begin{tikzpicture}[line width=1pt]
\coordinate (A) at (0,0);
\coordinate (B) at ({3*sqrt(3)},0);
\coordinate (C) at ({3*sqrt(3)},3);
\path (A) -- node[below] {Ankathete $b$} (B) %
 -- node[below,rotate=90] {Gegenkathete $a$} (C) %
 -- node[above,rotate=30] {Hypotenuse $c$} (A);
\draw (A) -- (B) -- (C) -- cycle;
\draw ($ (B) + (0,0.5) $) arc(90:180:0.5);
\filldraw ($ (B) + (135:0.25) $) circle(0.5pt);
\draw ($ (A) + (1.2,0) $) arc(0:30:1.2);
\path ($ (A) + (1.2,0) $) arc(0:15:1.2) node[right] {$x$};
\end{tikzpicture}
}
\end{center}

\begin{itemize}
\item
Die Zuordnung zwischen Winkel $x$ und dem Verh"altnis zwischen Gegenkathete $a$ 
und Ankathete $b$ wird als Tangens 
\[
\tan(x) := \frac{\text{Gegenkathete}}{\text{Ankathete}} = \frac{a}{b} %%
\]
bezeichnet.
%
\item
Die Zuordnung zwischen Winkel $x$ und dem Verh"altnis zwischen Ankathete $b$ 
und Hypotenuse $c$ wird als Kosinus 
\[
\cos(x) := \frac{\text{Ankathete}}{\text{Hyptenuse}} = \frac{b}{c} %%
\]
bezeichnet.
%
\item
Die Zuordnung zwischen Winkel $x$ und dem Verh"altnis zwischen Gegenkathete $a$ 
und Hypotenuse $c$ wird als Sinus 
\[
\sin(x) := \frac{\text{Gegenkathete}}{\text{Hyptenuse}} = \frac{a}{c} %%
\]
bezeichnet.
\end{itemize}
\end{MXInfo}

Demnach beschreibt der Tangens die Zuordnung zwischen dem Neigungswinkel und 
dem Verh"altnis zwischen H"ohe und Breite, also der Steigung. Dies ist auch 
im Kapitel \MRef{VBKM08} im Kontext der geometrischen Interpretation der 
Ableitung von Bedeutung.

Der Tangens des Winkels $\alpha$ ist nach der Definition 
\[
\tan\left(\alpha\right)=\frac{a}{b}=\frac{a}{b}\cdot\frac{c}{c} %
 =\frac{a}{c}\cdot\frac{c}{b} %
 =\frac{\sin\left(\alpha\right)}{\cos\left(\alpha\right)} \MDFPeriod %%
\]
Somit gen"ugt es, die Werte von Sinus und Kosinus zu kennen, um auch den 
Tangens berechnen zu k"onnen.

\begin{MExample}
Von einem Dreieck ist bekannt, dass es einen rechten Winkel 
$\gamma=\frac{\pi}{2}=90\MGrad$ hat. Die Seite $c$ ist $5\MEinheit{cm}$, 
die Seite $a$ ist $\MZahl{2}{5}\MEinheit{cm}$ lang. Es sollen jeweils der Sinus,
Kosinus und Tangens des Winkels $\alpha$ bestimmt werden:

Der Sinus l\"asst sich sofort aus den Angaben berechnen:
\[
\sin\left(\alpha\right)=\frac{a}{c}
 =\frac{\MZahl{2}{5}\MEinheit{cm}}{5\MEinheit{cm}}=\MZahl{0}{5} \MDFPeriod\]
F\"ur den Kosinus wird die L\"ange der Seite $b$ ben\"otigt, welche man 
mithilfe des Satzes von Pythagoras erh\"alt:
\[
b^2 = c^2 - a^2 \MDFPeriod %%
\]
Daraus folgt:
\[
\cos\left(\alpha\right)=\frac{b}{c}=\frac{\sqrt{c^2-a^2}}{c} %
 = \frac{\sqrt{\left(5\MEinheit{cm}\right)^2 - \left(\MZahl{2}{5}\MEinheit{cm}\right)^2}}{5\MEinheit{cm}} %
 = \MZahl{0}{866} \MDFPeriod %%
\]
Damit ergibt sich f\"ur den Tangens
\[
\tan\left(\alpha\right) %
 =\frac{\sin\left(\alpha\right)}{\cos\left(\alpha\right)} %
 =\frac{\MZahl{0}{5}}{\MZahl{0}{866}}=\MZahl{0}{5773} \MDFPeriod
\]
\end{MExample}


\begin{MExercise}\MLabel{VBKM05_TrigonometrieAufgabeTabelle}
Es sollen einige Werte der Winkelfunktionen Sinus, Kosinus und Tangens
n"aherungsweise grafisch bestimmt werden. Gehen Sie von rechtwinkligen 
Dreiecken mit einer Hypotenuse $c=5$ aus. Zeichnen Sie mithilfe des 
Thaleskreises rechtwinklige Dreiecke f\"ur die Winkel
\[ 
\alpha \in \left\{10\MGrad\MElSetSep 20\MGrad\MElSetSep %
30\MGrad\MElSetSep 40\MGrad\MElSetSep 45\MGrad\MElSetSep 50\MGrad\MElSetSep %
60\MGrad\MElSetSep 70\MGrad\MElSetSep 80\MGrad \right\} \MDFPeriod %%
\]
%jgl: Anmerkung: Befehl \MEinheit nicht f"ur Umlaute geeignet.
\ifttm
Fertigen Sie Ihre Zeichnungen im Ma"sstab 
$1$ L"angeneinheit $\hat{=} 2\MEinheit{cm}$ an, und schreiben Sie 
die Messwerte zu den Seiten $a$ und $b$ in eine Tabelle. 
\else
Fertigen Sie Ihre Zeichnungen im Ma"sstab 
$1\,\text{L"angeneinheit} \,\hat{=}\, 2\MEinheit{cm}$ an, und schreiben Sie 
die Messwerte zu den Seiten $a$ und $b$ in eine Tabelle. 
\fi

Berechnen Sie zu jedem Winkel mit den gemessenen Werten den Sinus, Kosinus 
und Tangens, und "uberlegen Sie sich anschlie"send, wo auch Werte zu 
$\alpha = 0\MGrad$ und zu $\alpha = 90\MGrad$ existieren.
Tragen Sie anschlie"send die Werte von Sinus und Kosinus in 
Abh"angigkeit des Winkels $\alpha$ in ein Diagramm ein.

\begin{MHint}{L\"osung}
Beim Messen entstehen immer Messfehler! Deshalb wird Ihre Tabelle teilweise
etwas andere Werte enthalten. Die Tabelle k\"onnte folgenderma"sen aussehen:

%\ifttm\relax\else\begin{small}\fi
\begin{center}
\begin{tabular}{r|r|r|r|r|r}\hline
	$\alpha$ 		& $a$ 				& $b$ 				& $\sin\left(\alpha\right)$ & $\cos\left(\alpha\right)$ & $\tan\left(\alpha\right)$\\ \hline\hline
	0 						& $\MZahl{0}{0}$		& $\MZahl{5}{0}$ 	& $\MZahl{0}{0}$		& $\MZahl{1}{0}$	& $\MZahl{0}{0}$\\ \hline
	$10\MGrad$		& $\MZahl{0}{8}$	& $\MZahl{4}{9}$	& $\MZahl{0}{160}$	& $\MZahl{0}{98}$	& $\MZahl{0}{1633}$\\ \hline
	$20\MGrad$		& $\MZahl{1}{7}$	& $\MZahl{4}{7}$	& $\MZahl{0}{34}$	& $\MZahl{0}{94}$	& $\MZahl{0}{3617}$\\ \hline
	$30\MGrad$		& $\MZahl{2}{5}$	& $\MZahl{4}{3}$	& $\MZahl{0}{5}$	& $\MZahl{0}{86}$	& $\MZahl{0}{5814}$\\ \hline
	$40\MGrad$		& $\MZahl{3}{2}$	& $\MZahl{3}{8}$	& $\MZahl{0}{64}$	& $\MZahl{0}{76}$	& $\MZahl{0}{8421}$\\ \hline
	$45\MGrad$		& $\MZahl{3}{5}$	& $\MZahl{3}{5}$	& $\MZahl{0}{7}$	& $\MZahl{0}{7}$	& $\MZahl{1}{0}$\\ \hline
	$50\MGrad$		& $\MZahl{3}{8}$	& $\MZahl{3}{27}$	& $\MZahl{0}{76}$	& $\MZahl{0}{64}$	& $\MZahl{1}{1875}$\\ \hline
	$60\MGrad$		& $\MZahl{4}{3}$	& $\MZahl{2}{5}$	& $\MZahl{0}{86}$	& $\MZahl{0}{5}$	& $\MZahl{1}{7200}$\\ \hline
	$70\MGrad$		& $\MZahl{4}{7}$	& $\MZahl{1}{7}$	& $\MZahl{0}{94}$	& $\MZahl{0}{34}$	& $\MZahl{2}{7647}$\\ \hline
	$80\MGrad$		& $\MZahl{4}{9}$	& $\MZahl{0}{8}$	& $\MZahl{0}{98}$	& $\MZahl{0}{160}$	& $\MZahl{6}{1250}$\\ \hline
	$90\MGrad$		& $\MZahl{5}{0}$		& $\MZahl{0}{0}$	& $\MZahl{1}{0}$	& $\MZahl{0}{0}$	& -- \\ \hline
\end{tabular}
\end{center}
%\ifttm\relax\else\end{small}\fi

Das zugeh"orige Diagramm sieht dann folgenderma"sen aus:

\begin{center}
\MTikzAuto{%
\pgfkeys{/pgf/number format/set decimal separator={{{\MZXYZhltrennzeichen}}}}
\begin{tikzpicture}[x=0.1cm, y=5.4cm,scale=0.7] 
%Koordinatensystem
\node (xMAX) at (102.0,0){};
\node (yMAX) at (0,1.15){};
%%\draw[help lines, gray, dashed, xstep=1, ystep=1] (0,0) grid (5.5,2.8);
\draw[-stealth',color=black] (-5,0) -- (xMAX);
\foreach \x in {10, 20, 30, 40, 50, 60, 70, 80, 90}
\draw[shift={(\x,0)},color=black] (0pt,0pt) -- (0pt,-6pt) node[below] {\normalsize $\x\MGrad$};
\draw[-stealth',color=black] (0,-0.12) -- (yMAX);
\foreach \y in {0.25, 0.5, 0.75, 1}
\draw[shift={(0,\y)},color=black] (0pt,0pt) -- (-6pt,0pt)  node[left] {\normalsize $\pgfmathprintnumber{\y}$};
%Achsenbeschriftung
\draw (xMAX) node[anchor=north east] {$\alpha$};
\draw[color=blue] (yMAX) ++(0.,-16pt) node[anchor=south east] {$\cos(\alpha)$};
\draw[color=red] (yMAX) ++(102.0,-16pt) node[anchor=south east] {$\sin(\alpha)$};
\fill[color=red] (90,1.0) circle (2.0pt);
\fill[color=red] (80,0.98) circle (2.0pt);
\fill[color=red] (70,0.94) circle (2.0pt);
\fill[color=red] (60,0.86) circle (2.0pt);
\fill[color=red] (50,0.76) circle (2.0pt);
%\fill[color=red] (45,0.7) circle (2.0pt);
\fill[color=red] (40,0.64) circle (2.0pt);
\fill[color=red] (30,0.5) circle (2.0pt);
\fill[color=red] (20,0.34) circle (2.0pt);
\fill[color=red] (10,0.16) circle (2.0pt);
\fill[color=red] (0,0.0) circle (2.0pt);
\fill[color=blue] (0,1.0) circle (2.0pt);
\fill[color=blue] (10,0.98) circle (2.0pt);
\fill[color=blue] (20,0.94) circle (2.0pt);
\fill[color=blue] (30,0.86) circle (2.0pt);
\fill[color=blue] (40,0.76) circle (2.0pt);
%\fill[color=blue] (45,0.7) circle (2.0pt);
\fill[color=blue] (50,0.64) circle (2.0pt);
\fill[color=blue] (60,0.5) circle (2.0pt);
\fill[color=blue] (70,0.34) circle (2.0pt);
\fill[color=blue] (80,0.16) circle (2.0pt);
\fill[color=blue] (90,0.0) circle (2.0pt);
\fill[color=magenta!50!black] (45,0.7) circle (2.0pt);
%Beschriftung und Graphen
%%\clip(-2.8,-0.5) rectangle (6,3);
%%\draw[color=black] (7,3) node[anchor=south west] {$\MPointTwo{7}{3}$};
\end{tikzpicture}
}
\end{center}
\end{MHint}
\end{MExercise}

Wenn man sich die Ergebnisse aus der letzten Aufgabe nochmals genauer ansieht,
kann man auf verschiedene Ideen kommen, sie zu interpretieren, und dann einige
Zusammenh"ange erkennen. 

\begin{itemize}
\item Mit zunehmendem Winkel $\alpha$ nimmt die Gegenkathete $a$ zu und die 
Ankathete $b$ ab.

Ebenso verhalten sich $\sin\left(\alpha\right)$ zu $a$ und 
$\cos\left(\alpha\right)$ zu $b$.

\item Mit zunehmendem Winkel $\alpha$ nimmt $a$ in dem gleichen Ma\ss\ zu 
wie $b$ mit dem von $90\MGrad$ aus fallenden Winkel $\alpha$ abnimmt. Im 
Thaleskreis sind die beiden Dreiecke mit den entgegengesetzten Werten 
f\"ur $a$ und $b$ die zwei L\"osungen f\"ur die Konstruktion eines 
rechtwinkligen Dreiecks mit gegebener Hypotenuse und gegebener H\"ohe 
(siehe auch das Beispiel \MRef{ThaleskreisBeispiel}).

\item
F"ur den Winkel $\beta = 90\MGrad - \alpha$ ist die Ankathete die Seite im
rechtwinkligen Dreieck, die aus Sicht des Winkels $\alpha$ als Gegenkathete
bezeichnet wird (und umgekehrt). Somit gilt
\[
\sin\left(\alpha\right)=\cos\left(90\MGrad-\alpha\right) %
 = \cos\left(\frac{\pi}{2}-\alpha\right) %%
\]
und 
\[
\cos\left(\alpha\right) = \sin\left(90\MGrad-\alpha\right) %
 = \sin\left(\frac{\pi}{2}-\alpha\right) \MDFPeriod %%
\]
%
\item Bei $\alpha=45\MGrad$ sind die Katheten und damit auch Sinus und Kosinus 
von $\alpha$ gleich. Diese Beobachtung f"uhrte eingangs umgekehrt zur Bestimmung
des Steigungswinkels.

\item Der Tangens, also das Verh\"altnis von $a$ zu $b$, steigt mit zunehmendem 
Winkel $\alpha$ von Null ins {\glqq}Unendliche{\grqq}. 
\end{itemize}

Im folgenden Beispiel wird die "Uberlegung aus der Einleitung fortgesetzt, 
die auf ein Dreieck mit einem Winkel von $45\MGrad$ f"uhrte, um den zugeh"origen
Sinuswert exakt zu berechnen.

\begin{MExample}
Es soll der Sinus des Winkels $\alpha=45\MGrad$ nun exakt berechnet, also 
nicht wie in Aufgabe \MRef{VBKM05_TrigonometrieAufgabeTabelle} aus 
gemessenen (und damit fehlerbehafteten) Werten bestimmt werden.

Wenn im rechtwinkligen Dreieck mit $\gamma=90\MGrad$ der Winkel $\alpha$ 
gleich $45\MGrad$ ist, so muss wegen der Summe der Winkel 
$\alpha+\beta+\gamma=\pi=180\MGrad$ der Winkel $\beta$ auch gleich 
$45\MGrad=\pi/4$ sein. Folglich sind die beiden Katheten $a$ und $b$ gleich lang. 
Ein Dreieck mit zwei gleich langen Seiten nennt man 
\MEntry{gleichschenklig}{gleichschenklig (Dreieck)}.

\begin{tabular}{@{}lr@{}}
\MTikzAuto{%
\begin{tikzpicture}[x=1.0cm, y=1.0cm] 
%%\draw[help lines, gray!50, xstep=0.5, ystep=0.5] (0,0) grid (9,8);
\draw[color=black, very thick] (0,0) -- (6,0) -- (3,3) -- cycle;
\draw[color=black, thin] (0,0) ++(0:1.2) arc (0:45:1.2);
\draw[color=black] (0,0) ++(22.5:0.8) node {\large $\alpha$};
\draw[color=black, thin] (6,0) ++(135:1.2) arc (135:180:1.2);
\draw[color=black] (6,0) ++(157.5:0.8) node {\large $\beta$};
\draw[color=black, thin] (3,3) ++(225:1.2) arc (225:315:1.2);
\fill[color=black] (3,3) ++(0,-0.6) circle (1.5pt);
\draw[color=black] (4.5,1.5) node[anchor=south west] {\large $a$};
\draw[color=black] (1.5,1.5) node[anchor=south east] {\large $b$};
\draw[color=black] (3,0) node[anchor=north] {\large $c$};
\end{tikzpicture}
}
&
\begin{minipage}[b]{10cm}
Es gilt: \[\sin\left(\alpha\right) = \sin\left(45\MGrad\right) = \frac{a}{c} \MDFPeriod\]
Au"serdem gilt 
\[a^2+b^2 = 2a^2 = c^2\quad\Rightarrow\quad c=\sqrt{2}\cdot a\MDFPSpace,\]
woraus
\[
\sin\left(45\MGrad\right) = \sin\left(\pi/4\right) %
 = \frac{a}{\sqrt{2}\cdot a} = \frac{1}{2}\cdot \sqrt{2} %%
\]
\end{minipage}
\end{tabular}
folgt.
In der Aufgabe \MRef{VBKM05_TrigonometrieAufgabeTabelle} wurde der Sinus 
von $45\MGrad$ durch einen Wert von $\MZahl{0}{7}$ angen"ahert, was dem 
tats"achlichen Wert von $\frac{1}{2}\cdot \sqrt{2}$ schon recht nahe kommt.
\end{MExample}

Im n"achsten Beispiel wird der Sinuswert zum Winkel $\alpha = 60\MGrad$ 
berechnet. Hierf"ur wird zun"achst nicht ein rechtwinkliges Dreieck sondern 
ein Dreieck mit drei gleich langen Seiten betrachtet. Mit einer geschickten
Zerlegung und Berechnung einer weiteren {\glqq}Hilfsgr"o"se{\grqq} erh"alt man
daraus das gesuchte Ergebnis.

\begin{MExample}\MLabel{M05_TrigonometrieBeispiel:gleichseitigesDreieck}%
In diesem Beispiel soll ein \textbf{gleichseitiges} Dreieck betrachtet werden,
um $\sin\left(60\MGrad\right)$ zu berechnen. 
Wie der Name sagt, sind in diesem Dreieck alle Seiten gleich lang, und auch 
die Winkel sind alle gleich gro"s, n\"amlich 
$\alpha=\beta=\gamma = \frac{180\MGrad}{3} = 60\MGrad = \frac{\pi}{3}$.
Das Dreieck ist nach dem Kongruenzsatz {\glqq}sss{\grqq} mit der Angabe einer 
Seite $a$ 
eindeutig bestimmt, und man erh\"alt dieses, indem man die Seite $a$ 
zeichnet und mit dem Zirkel einen Kreis vom Radius $a$ um jede Ecke schl\"agt.
Der Schnittpunkt der Kreise ist nun die dritte Ecke.

\begin{tabular}{@{}lr@{}}
\begin{minipage}[b]{10.5cm}
Dieses Dreieck hat keinen rechten Winkel. Zeichnet man eine H\"ohe $h$ auf 
eine der Seiten $a$ ein, so erh\"alt man zwei kongruente Dreiecke mit je 
einem rechten Winkel.

Es gilt nun: 
\[
\sin\left(\alpha\right) %
 = \sin\left(60\MGrad\right) %
 = \frac{h}{a} \MDFPeriod %%
\]
Nach dem Satz von Pythagoras ist 
\[
\left(\frac{a}{2}\right)^2+h^2 %
 = a^2 \MDFPeriod %%
\]
Daraus folgt 
\[
h^2 = \frac{3}{4}a^2 
\quad \text{und somit} \quad
h = \frac{1}{2}\sqrt{3}\cdot a \MDFPeriod
\]
\end{minipage}
&
\MTikzAuto{%
\begin{tikzpicture}[x=1.0cm, y=1.0cm] 
%%\draw[help lines, gray!50, xstep=0.5, ystep=0.5] (0,0) grid (9,8);
\draw[color=black, very thick] (0,0) -- (5,0) -- (2.5,4.33) -- cycle;
\draw[color=black, thin] (0,0) ++(0:1.2) arc (0:60:1.2);
\draw[color=black] (0,0) ++(30:0.8) node {\large $\alpha$};
\draw[color=black, thin] (5,0) ++(120:1.2) arc (120:180:1.2);
\draw[color=black] (5,0) ++(150:0.8) node {\large $\alpha$};
\draw[color=black, thin] (2.5,0) -- (2.5,3.13);
\draw[color=black, gray, thin] (2.5,3.13) -- (2.5,4.33);
\draw[color=black, thin] (2.5,4.33) ++(240:1.2) arc (240:300:1.2);
\draw[color=black] (2.5,4.33) ++(270:0.8) node {\large $\alpha$};
\draw[color=black, thin] (2.5,0) ++(90:0.8) arc (90:180:0.8);
\fill[color=black] (2.5,0) ++(135:0.4) circle (1.5pt);
\draw[color=black] (3.75,2.165) node[anchor=south west] {\large $a$};
\draw[color=black] (1.25,2.165) node[anchor=south east] {\large $a$};
\draw[color=black] (2.5,0.0) node[anchor=north] {\large $a$};
\draw[color=black] (2.5,1.65) node[anchor=west] {\large $h$};
\draw[color=black, gray, thin] (0,0) ++(50:5.0) arc (50:70:5.0);
\draw[color=black, gray, thin] (5,0) ++(110:5.0) arc (110:130:5.0);
\end{tikzpicture}
}
\end{tabular}

Damit erh"alt man den gesuchten Wert
\[
\sin\left(60\MGrad\right) %
 = \sin\left(\frac{\pi}{3}\right) %
 = \frac{h}{a} %
 =\frac{1}{2}\cdot \sqrt{3} \MDFPeriod %%
\]
Aus diesem Dreieck kann man noch den Sinus eines weiteren Winkels berechnen: 
Die H\"ohe $h$ teilt den oberen Winkel in zwei gleiche Teile, sodass man 
in den beiden kleinen kongruenten Dreiecken jeweils den Winkel 
$30\MGrad = \frac{\pi}{6}$ erh\"alt.
Es ist nun 
\[
\sin\left(30\MGrad\right) %
 = \sin\left(\frac{\pi}{6}\right) %
 = \frac{a/2}{a} %
 = \frac{1}{2} \MDFPeriod
\]
\end{MExample}

\begin{MExercise}
Berechnen Sie den exakten Wert des Kosinus f\"ur die Winkel 
$\alpha_1=30\MGrad$, $\alpha_2=45\MGrad$ und $\alpha_3=60\MGrad$. 
Verwenden Sie dazu die Ergebnisse aus dem vorherigen Beispiel und aus der
Aufgabe \MRef{VBKM05_TrigonometrieAufgabeTabelle}.

\begin{MHint}{L\"osung}
Aus der Aufgabe \MRef{VBKM05_TrigonometrieAufgabeTabelle} ist bekannt, 
dass $\cos\left(\alpha\right)= \sin\left(90\MGrad-\alpha\right)$ gilt.
Mit den Ergebnissen aus den obigen Beispielen folgt daraus
\begin{eqnarray*}
\cos\left(30\MGrad\right) %
 & = & \sin\left(90\MGrad-30\MGrad\right) %
 =\sin\left(60\MGrad\right) %
 =\frac{1}{2}\cdot\sqrt{3} \MDFPSpace, \\
%
\cos\left(45\MGrad\right)
 & = & \sin\left(90\MGrad-45\MGrad\right) %
 =\sin\left(45\MGrad\right) %
 =\frac{1}{2}\cdot\sqrt{2} \MDFPSpace, \\
%
\cos\left(60\MGrad\right) %
 & = & \sin\left(90\MGrad-60\MGrad\right) %
 =\sin\left(30\MGrad\right) %
 =\frac{1}{2} \MDFPeriod \\
\end{eqnarray*}
\end{MHint}
\end{MExercise}

In einer kleinen Tabelle werden die gefundenen Werte f"ur oft verwendete Winkel 
zusammengestellt:
Hier wird in der mit $x$ bezeichneten ersten Zeile der Winkel im Bogenma"s und 
in der mit $\alpha$ bezeichneten letzten Zeile der Winkel im Gradma"s notiert.

\ifttm
\begin{MDirectHTML}
\[
\begin{array}[t]{l|*{5}{c}}
 x            & 0                          & \tfrac{\pi}{6}             & \tfrac{\pi}{4}             & \tfrac{\pi}{3}             & \tfrac{\pi}{2}             \\[1mm] \hline
         \sin & 0 = \frac{1}{2} \cdot \sqrt{0} & \frac{1}{2} = \frac{1}{2} \cdot \sqrt{1} & \frac{1}{2} \cdot \sqrt{2} & \frac{1}{2} \cdot \sqrt{3} & \frac{1}{2} \cdot \sqrt{4} = 1 \\[1mm]
         \cos & 1 = \frac{1}{2} \cdot \sqrt{4} & \frac{1}{2} \cdot \sqrt{3} & \frac{1}{2} \cdot \sqrt{2} & \frac{1}{2} \cdot \sqrt{1} = \frac{1}{2} & \frac{1}{2} \cdot \sqrt{0} = 0 \\[1mm]
         \tan & 0                          & \frac{\sqrt{3}}{3}         & 1                          & \sqrt{3}                   & -                          \\[1mm] \hline
 \alpha       & 0^{\circ}                    & 30^{\circ}                   & 45^{\circ}                   & 60^{\circ}                   & 90^{\circ} %% 
\end{array}
\]
\end{MDirectHTML}
\else
\begin{center}
       $\begin{array}[t]{l|*{5}{c}}
 x            & 0                          & \Mtfrac{\pi}{6}             & \Mtfrac{\pi}{4}             & \Mtfrac{\pi}{3}             & \Mtfrac{\pi}{2}         \\[1mm] \hline
         \sin & 0 = \frac{1}{2} \cdot \sqrt{0} & \frac{1}{2} = \frac{1}{2} \cdot \sqrt{1} & \frac{1}{2} \cdot \sqrt{2} & \frac{1}{2} \cdot \sqrt{3} & \frac{1}{2} \cdot \sqrt{4} = 1 \\[1mm]
         \cos & 1 = \frac{1}{2} \cdot \sqrt{4} & \frac{1}{2} \cdot \sqrt{3} & \frac{1}{2} \cdot \sqrt{2} & \frac{1}{2} \cdot \sqrt{1} = \frac{1}{2} & \frac{1}{2} \cdot \sqrt{0} = 0 \\[1mm]
         \tan & 0                          & \frac{\sqrt{3}}{3}         & 1                          & \sqrt{3}                   & -                          \\[1mm] \hline
 \alpha       & 0\MGrad                  & 30\MGrad                 & 45\MGrad                 & 60\MGrad                 & 90\MGrad %% 
        \end{array}$
\end{center}
\fi

Diese Werte sollte man sich merken. Die Werte der trigonometrischen Funktionen 
f"ur andere Winkel sind in Tabellen abgelegt bzw. mit dem Taschenrechner 
berechenbar.

Damit kann man dann aus einem Winkel und einem Abstand ganz einfach eine H"ohe 
berechnen. 
Ist n"amlich $s$ der Abstand zu einem Geb"aude mit Flachdach, das unter einem 
Winkel $x$ beobachtet wird, ergibt sich aus $\tan(x) = \frac{h}{s}$ n"amlich 
$h = s \cdot \tan(x)$. Ebenso k"onnen auch $\cos$ und $\sin$ verwendet werden, 
um L"angen zu berechnen. Dieser Zusammenhang zwischen Winkeln und L"angen wird
oft verwendet.

Beispielsweise kann man so einen Fl"acheninhalt berechnen, auch wenn eine 
ben"otigte L"ange nicht unmittelbar gegeben ist. Im folgenden Beispiel ist es 
eine H"ohe $h$ in einem Dreieck, die zu berechnen ist.
Da $h$, ausgehend von einer Ecke, hier $C$ genannt, senkrecht auf der Geraden 
der gegen"uberliegenden Seite $c = \MGeoStrecke{A}{B}$ steht, bilden 
die Ecken von $h$ und $A$ bzw. $B$ ein rechtwinkliges Dreieck.
Mit einer Angabe zu einem Winkel und der entsprechenden Seite kann 
dann die H"ohe aus $\sin(\alpha) = \frac{h}{b}$ oder aus 
$\sin(\beta) = \frac{h}{a}$ berechnet werden, wobei von den konventionellen
Bezeichnungen ausgegangen wurde.

\begin{MExercise}
Berechnen Sie den Fl\"acheninhalt $F$ eines Dreiecks mit den Seiten $c = 7$ 
und $b = 3$ sowie dem Winkel $\alpha = 30\MGrad$ zwischen den Seiten $c$ und $b$.

Ergebnis: %
 \MEquationItem{$F$}{\MLParsedQuestion{20}{21/4}{4}{ExM05Sec6DFlaeche}}

\begin{MHint}{L\"osung}
Der Fl\"acheninhalt $F$ kann gem"a"s $F =\frac{1}{2}\cdot c \cdot h_c$ 
berechnet werden, wobei noch $h_c$ zu bestimmen ist:
Aus $\sin\left(\alpha\right)=\frac{h_c}{b}$
ergibt sich
\[
 h_c=b\cdot\sin\left(\alpha\right) %
  = 3\cdot\sin\left(30\MGrad\right) %
  = 3\cdot\frac{1}{2} \MDFPeriod %%
\]
Damit ist
\[ 
F =\frac{1}{2}\cdot c\cdot b\cdot\sin\left(\alpha\right) %
  =\frac{1}{2}\cdot 7 \cdot3\cdot\frac{1}{2}=\frac{21}{4} \MDFPeriod %%
\]
\end{MHint}
\end{MExercise}

\end{MXContent}


\begin{MXContent}{Trigonometrie am Einheitskreis}{Einheitskreis}{STD}
\MLabel{VBKM05_Trigonometrie_Einheitskreis}
\MDeclareSiteUXID{VBKM05_TrigonometrieEinheitskreis_Content}

Im vorherigen Abschnitt wurden die trigonometrischen Funktionen anhand eines 
rechtwinkligen Dreiecks eingef"uhrt. Die beschriebenen Eigenschaften gelten 
also f\"ur einen Winkelbereich von $0\MGrad$ bis $90\MGrad$ beziehungsweise 
$0$ bis $\frac{\pi}{2}$.

Um die gewonnenen Erkenntnisse auf gr"o"sere Winkel als $\pi/2$ ausdehnen zu 
k"onnen, erweist sich der Blick auf den sogenannten Einheitskreis als 
besonders n"utzlich.

\begin{center}
\MTikzAuto{%
\begin{tikzpicture}[line width=1.5pt,scale=1.5]
\coordinate (P) at (30:1);
\coordinate (Px) at ($(P) + (-90:{1/2}) $);
\draw[style=dotted] ($(P) + (-1.6,0) $) -- ($ ({pi/6},{1/2}) + (0.3,0) $);
\draw[style=dotted] %
 ($(Px) + (-1.6,0) $) -- ($ ({sqrt(3)/2},{-pi/6}) + (-1.6,-1.9) $);
\begin{scope}[xshift=-1.6cm]
\coordinate (M) at (0,0);
\coordinate (P) at (30:1);
\coordinate (Px) at ($(P) + (-90:{1/2}) $);
\coordinate (Py) at ($(P) + (180:{1/2}) $);
\draw (-1.3,0) -- (0,0);
\draw[->] (Px)  -- (1.3,0) node[below left] {$x$};
\draw[->] (0,-1.3) -- (0,1.3) node[below left] {$y$};
\draw[color=black!50!white] (M) circle(1);
\draw (M) -- node[above,rotate=30] {$r=1$} (P);
\draw[color=green] (M) -- (Px);
\draw[color=blue] (P) -- (Px);
\draw (1,0) arc(0:30:1);
\node[right] at (15:1) {$\Mvarphi$};
\filldraw (M) circle(1pt);
\filldraw (P) circle(1pt);
\end{scope}
%
\begin{scope}[xshift=0.3cm]
\coordinate (M) at (0,0);
\coordinate (P) at (30:1);
\coordinate (Px) at ($(P) + (-90:{1/2}) $);
\coordinate (Py) at ($(P) + (180:{1/2}) $);
\draw[->,color=black!50!white] (-0.3,0) -- (6.6,0) node[above left] {$\xi$};
\draw[->] (0,-1.3) -- (0,1.3) node[below left] {$y=\sin(\xi)$};
%\draw[->] (0,-1.3) -- (0,1.3) node[below left] {$y$};
\node[below left] at (M) {$0$};
\foreach \x/\xt in {1/{\frac{\pi}{2}}, 2/{\pi}, 3/{\frac{3\pi}{2}}, 4/{2\pi}}
 \draw[color=black!50!white] ({\x*pi/2},0) -- ++(0,-0.1) node[below] {$\xt$};
\draw[domain=0:{2*pi},samples=200,color=blue!50!white] 
 plot (\x,{sin(\x r)});
\draw[color=black] (M) -- node[below] {$\Mvarphi$} ({pi/6},0);
\draw[color=blue] ({pi/6},0) -- node[right] {$\sin(\Mvarphi)$} ++(0,{1/2});
\filldraw (M) circle(1pt);
\filldraw ({pi/6}, {1/2}) circle(1pt);
\end{scope}
%
\begin{scope}[xshift=-1.6cm,yshift=-1.9cm,rotate=-90]
\coordinate (M) at (0,0);
\coordinate (P) at (30:1);
\coordinate (Px) at ($(P) + (-90:{1/2}) $);
\coordinate (Py) at ($(P) + (180:{1/2}) $);
\draw[->,color=black!50!white] (-0.3,0) -- (6.6,0) node[above right] {$\xi$};
\draw[->] (0,-1.3) -- (0,1.3); % node[above left] {$x$}; 
  % node[above right] {$x=\cos(\xi)$};
\node[above left] at (M) {$0$};
%\path[->] (M) -- node[above,fill=white] {$x=\cos(\xi)$} (0,1.3);
\node [above right] at (0,{sqrt(3)/2}) {$x=\cos(\xi)$};
\foreach \x/\xt in {1/{\frac{\pi}{2}}, 2/{\pi}, 3/{\frac{3\pi}{2}}, 4/{2\pi}}
 \draw[color=black!50!white] ({\x*pi/2},0) -- ++(0,-0.1) node[left] {$\xt$};
\draw[domain=0:{2*pi},samples=200,color=green!50!white] 
plot (\x,{cos(\x r)});
\draw[color=black] (M) -- node[left] {$\Mvarphi$} ({pi/6},0);
\draw[color=green] %
 ({pi/6},0) -- node[above] {$\cos(\Mvarphi)$} ({pi/6},{sqrt(3)/2});
\filldraw (M) circle(1pt);
\filldraw ({pi/6}, {sqrt(3)/2}) circle(1pt);
\end{scope}
\end{tikzpicture}
}
\end{center}


%%%\begin{center}
%%%\MTikzAuto{%
%%%\begin{tikzpicture}[x=0.024cm, y=2.4cm] 
%%%\begin{scope}[xshift=-4cm,xscale=100]
%%%%Koordinatensystem
%%%\node (xMAX) at (1.3,0){};
%%%\node (yMAX) at (0,1.4){};
%%%\draw[color=black] (0,0) circle (1);
%%%\draw[-stealth',color=black] (-1.2,0) -- (1.3,0);
%%%\draw[-stealth',color=black] (0,-1.4) -- (0,1.4);
%%%\draw (xMAX) node[anchor=north] {$x$};
%%%\draw (yMAX) node[anchor=north east] {$y$};
%%%\foreach \x in {-1, 1}
%%%\draw[shift={(\x,0)},thick,color=black] (0,-0.05) -- (0,0.05) (0,0) node[anchor=north west] {\normalsize $\x$};
%%%\foreach \y in {-1, 1}
%%%\draw[shift={(0,\y)},thick,color=black] (0.05,0) -- (-0.05,0) (0,0) node[anchor=north west] {\normalsize $\y$};
%%%\def\cAng{38}
%%%\draw[-stealth',thick,color=black] (0,0) -- ({cos(\cAng)},{sin(\cAng)}) node[anchor=south west] {$P$};
%%%\draw[shift={({cos(\cAng)},0)},very thick,color=blue] (0,0.05) -- (0,-0.05) node[anchor=north east] {\normalsize $\cos(\alpha)$};
%%%\draw[shift={(0,{sin(\cAng)})},very thick,color=red] (0.05,0) -- (-0.05,0) node[anchor=east] {\normalsize $\sin(\alpha)$};
%%%\draw[thick,color=blue] (0,0) -- ({cos(\cAng)},0.0);
%%%\draw[thick,color=red] ({cos(\cAng)},0) -- ({cos(\cAng)},{sin(\cAng)});
%%%\draw[color=black, thin] (0,0) ++(0:0.5) arc (0:\cAng:0.5);
%%%\draw[color=black] (0,0) ++({0.5*\cAng}:0.35) node {$\alpha$};
%%%\end{scope}
%%%%Koordinatensystem
%%%\node (xMAX) at (400.0,0){};
%%%\node (yMAX) at (0,1.4){};
%%%%%\draw[help lines, gray, dashed, xstep=1, ystep=1] (0,0) grid (5.5,2.8);
%%%\draw[-stealth',color=black] (-20,0) -- (xMAX);
%%%\foreach \x in {30, 60, 90, 120, 150, 180, 210, 240, 270, 300, 330, 360}
%%%\draw[shift={(\x,0)},color=black] (0pt,0pt) -- (0pt,-6pt) node[below] {\scriptsize $\x\MGrad$};
%%%\draw[-stealth',color=black] (0,-1.4) -- (yMAX);
%%%\foreach \y in {1}
%%%\draw[shift={(0,\y)},color=black] (0pt,0pt) -- (-6pt,0pt)  node[left] {\normalsize $\pgfmathprintnumber{\y}$};
%%%%Achsenbeschriftung
%%%\draw (xMAX) node[anchor=north east] {$\alpha$};
%%%\draw[color=red] (yMAX) node[anchor=north west] {$y=\sin(\alpha)$};
%%%\draw[smooth,samples=73,domain=0:360, line width=1pt,color=red] plot(\x,{sin(\x)}); %{cos(\x r)}
%%%%Beschriftung und Graphen
%%%%%\clip(-2.8,-0.5) rectangle (6,3);
%%%%%\draw[color=black] (7,3) node[anchor=south west] {$\MPointTwo{7}{3}$};
%%%\end{tikzpicture}
%%%}
%%%%
%%%\MTikzAuto{%
%%%\begin{tikzpicture}[x=0.024cm, y=2.4cm] 
%%%%Koordinatensystem
%%%\node (xMAX) at (400.0,0){};
%%%\node (yMAX) at (0,1.4){};
%%%%%\draw[help lines, gray, dashed, xstep=1, ystep=1] (0,0) grid (5.5,2.8);
%%%\draw[-stealth',color=black] (-20,0) -- (xMAX);
%%%\foreach \x in {30, 60, 90, 120, 150, 180, 210, 240, 270, 300, 330, 360}
%%%\draw[shift={(\x,0)},color=black] (0pt,0pt) -- (0pt,-6pt) node[below] {\scriptsize $\x\MGrad$};
%%%\draw[-stealth',color=black] (0,-1.4) -- (yMAX);
%%%\foreach \y in {1}
%%%\draw[shift={(0,\y)},color=black] (0pt,0pt) -- (-6pt,0pt)  node[left] {\normalsize $\pgfmathprintnumber{\y}$};
%%%%Achsenbeschriftung
%%%\draw (xMAX) node[anchor=north east] {$\alpha$};
%%%\draw[color=blue] (yMAX) node[anchor=north west] {$x=\cos(\alpha)$};
%%%\draw[smooth,samples=73,domain=0:360, line width=1pt,color=blue] plot(\x,{cos(\x)}); %{cos(\x r)}
%%%%Beschriftung und Graphen
%%%%%\clip(-2.8,-0.5) rectangle (6,3);
%%%%%\draw[color=black] (7,3) node[anchor=south west] {$\MPointTwo{7}{3}$};
%%%\end{tikzpicture}
%%%}
%%%\end{center}

Der Einheitskreis ist ein Kreis mit Radius $1$.
Sein Mittelpunkt wird im Nullpunkt eines kartesischen Koordinatensystems 
positioniert.
Hier wird eine Strecke vom Mittelpunkt aus mit der L\"ange $1$ betrachtet. 
Diese Strecke wird nun von ihrer horizontalen Ausgangslage auf der positiven 
$x$-Achse gegen den Uhrzeigersinn, also im mathematisch positiven Sinn, um den 
Nullpunkt gedreht. 
Dabei \"uberstreicht ihr rotierendes Ende den Einheitskreis und bildet mit der 
positiven $x$-Achse den Winkel $\alpha$, der bei der Rotation von $0$ bis 
$2\pi$ bzw. $360\MGrad$ w\"achst. Zu jedem Winkel $\Mvarphi$ geh\"ort also ein 
Punkt $P_{\Mvarphi}$ mit den Koordinaten $x_{\Mvarphi}$ und $y_{\Mvarphi}$ auf 
dem Einheitskreis.

F\"ur $\Mvarphi$ zwischen $0$ und $\frac{\pi}{2}$ kann man die Strecke, den 
zugeh\"origen Abschnitt auf der $x$-Achse und den zugeh\"origen 
$y$-Achsenabschnitt als rechtwinkliges Dreieck ansehen. 
Die Hypotenuse ist die Strecke mit der L\"ange $1$, der $x$-Achsenabschnitt 
ist die Ankathete und der $y$-Achsenabschnitt die Gegenkathete.
Dies entspricht der Situation aus dem vorherigen Abschnitt.
\par
Der Sinus des Winkels $\Mvarphi$ ist also
\[
\sin\left(\Mvarphi\right)=\frac{y_{\Mvarphi}}{1}=y_{\Mvarphi}
\]
und der Kosinus ist
\[
\cos\left(\Mvarphi\right)=\frac{x_{\Mvarphi}}{1}=x_{\Mvarphi} \MDFPeriod
\]
Anhand der obigen Beschreibung am Einheitskreis gelten diese Definitionen jetzt 
auch f\"ur Winkel $\Mvarphi > \pi/2$. 
Dabei k\"onnen die Werte f\"ur $x_{\Mvarphi}$ und $y_{\Mvarphi}$ auch negativ 
werden und damit auch Sinus und Kosinus. Tr\"agt man die $y$-Werte in 
Abh\"angigkeit vom Winkel $\Mvarphi$ in ein Diagramm ein, so erh\"alt man die 
blaue Kurve f"ur die Sinusfunktion. F\"ur die $x$-Werte erh\"alt man die 
gr"une Kurve f"ur die Kosinusfunktion.
Indem man die Strecke in umgekehrter Richtung dreht, kann man entsprechend 
Werte f"ur negative Winkel definieren.
\par
Mit dem Satz von Pythagoras gilt au"serdem
\[
x_{\Mvarphi}^{2}+y_{\Mvarphi}^{2}=1 \MDFPeriod
\]
Setzt man hier die Beziehungen f\"ur $x_{\Mvarphi}$ und $y_{\Mvarphi}$ mit den 
Winkelfunktionen ein, so ergibt sich f\"ur beliebige Winkel $\Mvarphi$ die 
wichtige Beziehung
\[
\sin^2\left(\Mvarphi\right)+\cos^2\left(\Mvarphi\right)=1 \MDFPeriod
\]
Aus der Beschreibung von Sinus und Kosinus am Einheitskreis wird zudem 
ersichtlich, dass sich die Kosinuswerte bei Spiegelung an der $x$-Achse 
nicht "andern. Somit ist der Wert von $\cos$ zum Winkel $\Mvarphi$ gleich
dem Kosinuswert zum Winkel $-\Mvarphi$ (in der Zeichnung gr"un dargestellt).
Beim Sinus f"uhrt eine Spiegelung an der $x$-Achse zur "Anderung des 
Vorzeichens des Sinuswertes (in der Zeichnung blau beziehungsweise violett
dargestellt). 

\begin{center}
\MTikzAuto{%
\begin{tikzpicture}[line width=1.5pt]
\begin{scope}[xshift=-4cm]
\draw[->] (-3,0) -- (3.5,0) node[below left] {$x$};
\draw[->] (0,-3) -- (0,3) node[below left] {$y$};
\draw[color=black!50!white] (0,0) circle(2);
\draw (0,0) -- (30:2);
%\draw[color=green] (0,0) -- node[below] {$\cos(\Mvarphi)$} ({sqrt(3)},0);
\draw[color=green] (0,0) -- ({sqrt(3)},0);
\draw[style=dashed,color=blue] (30:2) -- ++(-90:1);
\draw[style=dotted] (0,0) -- (-30:2);
\draw[style=dotted,color=blue] (-30:2) -- ++(90:1);
%Winkel:
\draw (2,0) arc(0:30:2);
\node[right] at (15:2) {$\Mvarphi$};
\draw[style=dotted] (2,0) arc(0:-30:2);
\node[right] at (-15:2) {$-\Mvarphi$};
%Punkte:
\filldraw (0,0) circle(1pt);
\filldraw (30:2) circle(1pt);
\filldraw (-30:2) circle(1pt);
\end{scope}
%
\begin{scope}[xshift=4cm]
\draw (-3,0) -- (0,0);
\draw[->] ({sqrt(3)},0) -- (3.5,0) node[below left] {$x$};
\draw[->] (0,-3) -- (0,3) node[below left] {$y$};
\draw[color=black!50!white] (0,0) circle(2);
\draw (0,0) -- (30:2);
%\draw[color=green] (0,0) -- node[below] {$\cos(\Mvarphi)$} ({sqrt(3)},0);
\draw[style=dashed,color=green] (0,0) -- ({sqrt(3)},0);
\draw[color=blue] (30:2) -- ++(-90:1);
\draw[style=dotted] (0,0) -- (-30:2);
\draw[color=blue!70!red] (-30:2) -- ++(90:1);
%Winkel:
\draw (2,0) arc(0:30:2);
\node[right] at (15:2) {$\Mvarphi$};
\draw[style=dotted] (2,0) arc(0:-30:2);
\node[right] at (-15:2) {$-\Mvarphi$};
%Punkte:
\filldraw (0,0) circle(1pt);
\filldraw (30:2) circle(1pt);
\filldraw (-30:2) circle(1pt);
\end{scope}
\end{tikzpicture}
}
\end{center}

Formelm"a"sig ausgedr"uckt bedeutet dies
\[
\cos(-\Mvarphi) = \cos(\Mvarphi)
\qquad \text{und} \qquad
\sin(-\Mvarphi) = -\sin(\Mvarphi)
\]
f"ur jeden Winkel $\Mvarphi$. 
Diese Symmetrieeigenschaften sind f"ur viele Rechnungen hilfreich. 
Ein elementares Beispiel ist die Berechnung des Winkels zwischen $x$-Achse und 
der Verbindungsstrecke vom Nullpunkt zu einem Punkt im kartesischen 
Koordinatensystem (siehe auch 
Aufgabe \MRef{VBKM05_TrigonometrieAufgabeArcsinArccos}).

\begin{MExample}
Gesucht sind jeweils die Werte des Sinus, Kosinus und Tangens des 
Winkels $\alpha = 315\MGrad$.

F"ur $\alpha = 315\MGrad$ liegt der Punkt $P_\alpha$ im vierten Quadranten. 
Er wird auf dem Einheitskreis auch durch den negativen Winkel 
$\Mvarphi = 315\MGrad - 360\MGrad = -45\MGrad$ beschrieben.
Damit ist
$\sin(315\MGrad) = \sin(-45\MGrad) = -\sin(45\MGrad) = -\frac{1}{2} \sqrt{2}$
und
$\cos(315\MGrad) = \cos(-45\MGrad) = \cos(45\MGrad) = \frac{1}{2} \sqrt{2}$
sowie $\tan(315\MGrad) = \tan(-45\MGrad) = -1$.
%Die zugeh"orige Strecke bildet mit den zugeh"origen Achsenabschnitten 
%ein gleichschenkliges Dreieck. Somit gilt 
%$\left|x_\alpha\right|=\left|y_\alpha\right|$,
%woraus 
%$2\cdot\left|x_\alpha\right|^2 %
% = \left|x_\alpha\right|^2+\left|y_\alpha\right|^2
% = 1$ 
%und somit
%$\left|x_\alpha\right| %
% =\left|y_\alpha\right| %
% =\frac{1}{\sqrt 2} %
% =\frac{1}{2}\sqrt 2$
%folgt. Damit ist
%\[
%\cos\left(315\MGrad\right) %
% =x_\alpha
% =\frac{1}{2}\sqrt 2 \MDFPSpace,
%\MDFPaSpace
%\sin\left(315\MGrad\right) %
% =y_\alpha %
% = -\frac{1}{2}\sqrt 2 \MDFPSpace,
%\MDFPaSpace
%\tan\left(315\MGrad\right) %
% =\frac{y_\alpha}{x_\alpha}=-1 \MDFPeriod
%\]
\end{MExample}

\end{MXContent}


\begin{MExercises}
\MDeclareSiteUXID{VBKM05_Trigonometrie_Exercises}
\begin{MExercise}\MLabel{VBKM05_TrigonometrieAufgabeArcsinArccos}
Wie gro"s ist der Winkel $\Mvarphi$ im Gradma"s, den der Punkt 
$P_{\Mvarphi} = \MPointTwo{-\MZahl{0}{643}}{-\MZahl{0}{766}}$ auf dem 
Einheitskreis mit dem Nullpunkt im kartesischen Koordinatensystem einschlie"st? 
Verwenden Sie dazu den Taschenrechner, aber vertrauen Sie ihm nicht blind!

Ergebnis: \MEquationItem{$\Mvarphi$}{\MLParsedQuestion{15}{230}{4}{ExM05Sec6TrigFktWinkel}$\MGrad$}

\begin{MHint}{L\"osung}
Aus den Koordinaten des Punktes $P_{\Mvarphi}$ ergibt sich:
\[
\cos\left(\alpha\right)=-\MZahl{0}{643} 
 \quad \text{und} \quad
\sin\left(\alpha\right)=-\MZahl{0}{766}
\MDFPeriod %%
\]
Wenn Sie in den Taschenrechner
\begin{itemize}
\item \texttt{invers(cos(-\MZahl{0}{643}))} bzw. $\cos^{-1}$(-\MZahl{0}{643}) 
eingeben, erhalten Sie ungef"ahr $130\MGrad$, und
\item bei Eingabe von 
\texttt{invers(sin(-\MZahl{0}{766}))} bzw. $\sin^{-1}$(-\MZahl{0}{766})
erhalten Sie ungef\"ahr $-50\MGrad$.
\end{itemize}
Au"serdem wissen Sie, dass der Punkt im dritten Quadranten liegt. Somit muss
der Wert f"ur den Winkel im Bereich zwischen $180\MGrad$ und $270\MGrad$ liegen.

\MTikzAuto{%
\begin{tikzpicture}[x=2.6cm, y=2.6cm,line width=0.7pt] 
%Koordinatensystem
\node (xMAX) at (1.3,0){};
\node (yMAX) at (0,1.3){};
\draw[color=black] (0,0) circle (1);
\draw[-stealth',color=black] (-1.2,0) -- (1.3,0);
\draw[-stealth',color=black] (0,-1.3) -- (0,1.3);
\draw (xMAX) node[anchor=north east] {$x$};
\draw (yMAX) node[anchor=north east] {$y$};
\foreach \x in {-1, 1}
\draw[shift={(\x,0)},thick,color=black] (0,-0.05) -- (0,0.05) (0,0) node[anchor=north west] {\normalsize $\x$};
\foreach \y in {-1, 1}
\draw[shift={(0,\y)},thick,color=black] (0.05,0) -- (-0.05,0) (0,0) node[anchor=north west] {\normalsize $\y$};
\def\cRad{1.3}
\def\cAng{50}
\draw[thick,color=black] (0,0) -- ({\cRad*cos(\cAng)},{-\cRad*sin(\cAng)});
\draw[thick,color=black] (0,0) -- (1,0);
\draw[thick,color=black] (0,0) -- ({-\cRad*cos(\cAng)},{\cRad*sin(\cAng)});
\draw[color=black, thin] (0,0) ++({-\cAng}:0.5) arc ({-\cAng}:{180-\cAng}:0.5);
\draw[color=black] (0,0) ++({-0.5*\cAng}:0.35) node {$-\cAng\MGrad$};
\draw[color=black] (0,0) ++(45:0.30) node {$130\MGrad$}; %({90-0.5*\cAng}:0.35)
\draw[color=magenta, thin] (0,0) ++({-\cAng}:0.60) arc ({-\cAng}:{180+\cAng}:0.60);
\draw[thick,color=magenta] (0,0) -- ({-cos(\cAng)},{-sin(\cAng)});
\draw[thick,color=blue] ({cos(\cAng)},0) -- ({cos(\cAng)},{-sin(\cAng)});
\draw[thick,color=black,dashed] ({-cos(\cAng)},0) -- ({-cos(\cAng)},{sin(\cAng)});
\draw[color=blue,dashed] ({-cos(\cAng)},0) -- ({-cos(\cAng)},{-sin(\cAng)-0.1});
\draw[color=black,dashed] ({-cos(\cAng)-0.1},{-sin(\cAng)}) -- (0,{-sin(\cAng)});
\draw[thick,color=black,blue] (-0.05,{-sin(\cAng)}) -- (0.05,{-sin(\cAng)});
\end{tikzpicture}
}
\begin{minipage}[b]{10cm}
Anhand der Zeichnung erkennt man, dass der negative Kosinuswert zum 
Winkel $-130\MGrad$ und zu
$\Mvarphi = -130\MGrad= -130\MGrad+360\MGrad = 230\MGrad$ geh"ort.
\par
Ebenso kann der negative Sinuswert zu $-50\MGrad$ und zu
$\Mvarphi = -(-50\MGrad) + 180\MGrad = 230\MGrad$ geh\"oren.
\par
Da dieser Wert im oben genannten Bereich liegt, ist $\Mvarphi=230\MGrad$ der 
gesuchte Wert des Winkels, der in der Zeichnung rosa gekennzeichnet ist.
\vspace*{2cm}
\end{minipage}
\end{MHint}

\end{MExercise}


\begin{MExercise}
\begin{enumerate}
\item
F"ur ein im Punkt $C$ rechtwinkliges Dreieck seien 
eine Kathete $b = \MZahl{2}{53}\MEinheit{cm}$ und die Hypotenuse 
$c = \MZahl{3}{88}\MEinheit{cm}$ gegeben. 
Berechnen Sie die L"ange der anderen Kathete $a$ und die Sinuswerte 
$\sin \left( \alpha \right)$ sowie $\sin \left( \beta \right)$. 
Runden Sie Ihre Ergebnisse bitte auf vier Nachkommastellen.

Ergebnisse:
\begin{itemize}
\item \MEquationItem{$a$}{\MLParsedQuestion{15}{2.9417}{4}{ExM05Sec6TrigFktWerteLaenge}$\MEinheit{cm}$}
%
\item \MEquationItem{$\sin \left( \alpha \right)$}{\MLParsedQuestion{15}{0.7582}{4}{ExM05Sec6TrigFktWerteAlpha}}
%
\item \MEquationItem{$\sin \left( \beta \right)$}{\MLParsedQuestion{15}{0.6520}{4}{ExM05Sec6TrigFktWerteBeta}}
\end{itemize}
       
\begin{MHint}{L\"osung}
Es ist
\[
a = \sqrt{c^2 - b^2}
= \sqrt{\left( \MZahl{3}{88}\MEinheit{cm} \right)^2 %
  - \left( \MZahl{2}{53}\MEinheit{cm} \right)^2}
= \sqrt{\MZahl{15}{0544}\MEinheit{cm}^2 - \MZahl{6}{4009}\MEinheit{cm}^2}
= \sqrt{\MZahl{8}{6535}}\MEinheit{cm} \MDFPSpace,
\]
und
\[
\sin \left( \alpha \right)
= \frac{a}{c}
= \frac{\sqrt{\MZahl{8}{6535}}\MEinheit{cm}}{\MZahl{3}{88}\MEinheit{cm}}
= \frac{\sqrt{86535}}{388}
\qquad \text{und} \qquad
          \sin \left( \beta \right)
= \frac{b}{c}
= \frac{\MZahl{2}{53}\MEinheit{cm}}{\MZahl{3}{88}\MEinheit{cm}}
= \frac{253}{388} \MDFPeriod
\]
Numerisch ergibt sich
$a \approx \MZahl{2}{9417}\MEinheit{cm}$, 
$\sin \left( \alpha \right) \approx \MZahl{0}{7582}$
und $\sin \left( \beta \right) \approx \MZahl{0}{6520}$.
\end{MHint}
%\end{MExercise}

%\begin{MExercise}
\item
Es wird ein Dreieck mit den Seiten
$a = 4\MEinheit{m}$ und $c = 60\MEinheit{cm}$
und dem Winkel $\beta = \Mmeasuredangle(a,c) = \frac{11 \pi}{36}$ betrachtet.
Berechnen Sie den Fl"acheninhalt $F$ des Dreiecks und runden Sie Ihr Ergebnis
auf drei Nachkommastellen. 

Ergebnis: %
 \MEquationItem{$F$}{\MLParsedQuestion{25}{4*sin(11*pi/36)*6/20}{3}{GEO8}$\MEinheit{m}^2$}
 
\MInputHint{Geben Sie Ihr Ergebnis bitte auf drei Nachkommastellen gerundet 
oder als Ausdruck an. Dabei wird der Sinus eines Winkels~$x$ als 
\texttt{sin(x)} und die Zahl~$\pi$ als \texttt{pi} geschrieben.}
 
\begin{MHint}{L\"osung}
Der Fl"acheninhalt ist durch
\[
F = \frac{1}{2} \cdot \left( a \cdot \sin \left( \beta \right) \right) \cdot c %
= \frac{1}{2} \cdot 4\MEinheit{m} \cdot \sin \left( \Mtfrac{11 \pi}{36} \right) \cdot \MZahl{0}{6}\MEinheit{m} %
= \sin \left( \Mtfrac{11 \pi}{36} \right) \cdot \MZahl{1}{2}\MEinheit{m}^2
\approx \MZahl{0}{983}\MEinheit{m}^2 %%
\]
gegeben.
\end{MHint}
\end{enumerate}
\end{MExercise}

\end{MExercises}

%end of content: section 6: Winkelfunktionen.



%begin of test:
\MSubsection{Abschlusstest}
\MLabel{M05_Abschlusstest}

\begin{MTest}{Abschlusstest Modul \arabic{section}}
\MLabel{M05_Abschlusstest_Test}
\MDeclareSiteUXID{VBKM05_Abschlusstest}

\begin{MExercise} %Testaufgabe
Kennzeichnen Sie die dargestellten Figuren m"oglichst genau, indem Sie
jeweils den Namen der Klasse (gegebenenfalls mit einem vorangestellten 
Adjektiv) angeben, die m"oglichst viele Eigenschaften der Figur beschreibt.

\begin{center}
\MTikzAuto{%
\begin{tikzpicture}[line width=1pt,scale=0.6]
\begin{scope}[xshift=-8cm] %Viereck:
\draw (-1,-1) -- (0,0) -- (1,-1) -- (0,1) -- (-1,-1);
\node at (0,-2) {$F_1$};
\end{scope}
%
\begin{scope}[xshift=-4cm] %Quadrat:
\draw (-1.5,0)  -- (0,-1.5) -- (1.5,0) -- (0,1.5) -- (-1.5,0);
\node at (0,-2) {$F_2$};
\end{scope}
%
\begin{scope}[xshift=0cm] %Parallogramm:
\draw (-0.5,-1)  -- (0.5,-2) -- (0.5,1) -- (-0.5,2) -- (-0.5,-1);
\node at (-0.5,-2) {$F_3$};
\end{scope}
%
\begin{scope}[xshift=4cm] %gleichschenkliges Dreieck:
\draw (-1.3,1)  -- (0,-2) -- (1.3,1) -- (-1.3,1);
\node at (-1,-2) {$F_4$};
\end{scope}
%
\begin{scope}[xshift=8cm] %Raute
\draw (-1.5,0)  -- (0,-1) -- (1.5,0) -- (0,1) -- (-1.5,0);
\node at (0,-2) {$F_5$};
\end{scope}
%
\end{tikzpicture}
}
\end{center}

\begin{MQuestionGroup}
\begin{tabular}[t]{cc}
Figur: & Klassenbeschreibung: \\
 $F_1$ & \MLQuestion{32}{Viereck}{ExM05TestAg11} \\
 $F_2$ & \MLQuestion{32}{Quadrat}{ExM05TestAg12} \\
 $F_3$ & \MLQuestion{32}{Parallelogramm}{ExM05TestAg13} \\
 $F_4$ & \MLQuestion{32}{gleichschenkliges Dreieck}{ExM05TestAg14} \\
 $F_5$ & \MLQuestion{32}{Raute}{ExM05TestAg15} \\
%
\end{tabular}
\end{MQuestionGroup}
%jgl: Tests aktuell ohne Loesungen (daher auskommentiert):
%jgl: Loesung erstellt:
%\begin{MHint}{L"osung}
%\begin{tabular}[t]{cc}
%Figur & Klassenbeschreibung: \\
% $F_1$ & ist ein Viereck (vier Seiten). \\
% $F_2$ & ist ein Quadrat (vier gleich lange Seiten, Diagonalen gleich lang). \\
% $F_3$ & ist ein Parallelogramm (gegen"uberliegende Seiten parallel. \\
% $F_4$ & ist ein gleichschenkliges Dreieck (drei Seiten, davon zwei 
%   anliegende Seiten gleich lang). \\
% $F_5$ & ist eine Raute (vier gleich lange Seiten). \\
%\end{tabular}
%\end{MHint}
\end{MExercise}


\begin{MExercise} %Testaufgabe
Welche der Aussagen und Ergebnisse sind richtig?
\par
\ifttm
\begin{MQuestionGroup}
\begin{tabular}{|l|l|}
 richtig? & \\
 \MLCheckbox{0}{ExM05TestAg20g} & % \MLCheckbox{1}{ExM05TestAg21} & %
 Jedes Rechteck ist eine Raute. \\
%
\MLCheckbox{1}{ExM05TestAg22} & % \MLCheckbox{0}{ExM05TestAg23} & %
 Jedes Quadrat ist ein Parallelogramm. \\
%
\MLCheckbox{1}{ExM05TestAg24} & % \MLCheckbox{0}{ExM05TestAg25} & %
 Es gibt genau ein Quadrat mit einer Diagonalen von $5\MEinheit{cm}$. \\
%
\MLCheckbox{1}{ExM05TestAg26} & % \MLCheckbox{0}{ExM05TestAg27} & %
 Ein Dreieck mit den Winkeln $36\MGrad$ und $54\MGrad$ ist rechtwinklig. \\
%
\MLCheckbox{0}{ExM05TestAg28} & % \MLCheckbox{1}{ExM05TestAg29} & %
 In einem Viereck ist die Summe aller (Innen-)Winkel im Bogenma"s gleich $4 \pi$.
\end{tabular}
%%%\begin{tabular}{|l|l|l|}
%%% richtig & falsch & \\
%%% \MLCheckbox{0}{ExM05TestAg20g} & \MLCheckbox{1}{ExM05TestAg21} & %
%%% Jedes Rechteck ist eine Raute. \\
%%%%
%%%\MLCheckbox{1}{ExM05TestAg22} & \MLCheckbox{0}{ExM05TestAg23} & %
%%% Jedes Quadrat ist ein Parallelogramm. \\
%%%%
%%%\MLCheckbox{1}{ExM05TestAg24} & \MLCheckbox{0}{ExM05TestAg25} & %
%%% Es gibt genau ein Quadrat mit einer Diagonalen von $5\MEinheit{cm}$. \\
%%%%
%%%\MLCheckbox{1}{ExM05TestAg26} & \MLCheckbox{0}{ExM05TestAg27} & %
%%% Ein Dreieck mit den Winkeln $36\MGrad$ und $54\MGrad$ ist rechtwinklig. \\
%%%%
%%%\MLCheckbox{0}{ExM05TestAg28} & \MLCheckbox{1}{ExM05TestAg29} & %
%%% In einem Viereck ist die Summe aller (Innen-)Winkel im Bogenma"s gleich $4 \pi$.
%%%\end{tabular}
\end{MQuestionGroup}
\else
\begin{MQuestionGroup}
\begin{tabular}[t]{ccp{120mm}}
 richtig & falsch & \\
\MLCheckbox{0}{ExM05TestAg20} & \MLCheckbox{1}{ExM05TestAg21} & %
 Jedes Rechteck ist eine Raute. \\
%
\MLCheckbox{1}{ExM05TestAg22} & \MLCheckbox{0}{ExM05TestAg23} & %
 Jedes Quadrat ist ein Parallelogramm. \\
%
\MLCheckbox{1}{ExM05TestAg24} & \MLCheckbox{0}{ExM05TestAg25} & %
 Es gibt genau ein Quadrat mit einer Diagonalen von $5\MEinheit{cm}$. \\
%
\MLCheckbox{1}{ExM05TestAg26} & \MLCheckbox{0}{ExM05TestAg27} & %
 Ein Dreieck mit den Winkeln $36\MGrad$ und $54\MGrad$ ist rechtwinklig. \\
%
\MLCheckbox{0}{ExM05TestAg28} & \MLCheckbox{1}{ExM05TestAg29} & %
 In einem Viereck ist die Summe aller (Innen-)Winkel im Bogenma"s gleich $4 \pi$.
\end{tabular}
\end{MQuestionGroup}
\fi
%jgl: Tests aktuell ohne Loesungen (daher auskommentiert):
%jgl: Loesung erstellt:
%\begin{MHint}{L"osung}
%\begin{itemize}
%\item Nicht jedes Rechteck ist eine Raute, da es Rechtecke gibt, bei denen
% nicht alle Seiten gleich lang sind.
%\item Jedes Quadrat ist ein Parallelogramm, da bei einem Quadrat gegen"uber 
% liegende Seiten parallel sind.
%\item Ein Quadrat ist durch die L"ange der Diagonalen eindeutig festgelegt,
% alle Winkel bekannt sind und die Diagonale das Quadrat in zwei Dreiecke 
% zerlegt. Von den beiden Dreiecken sind dann alle Winkel und eine Seite 
% bekannt sind, sodass sie eindeutig festgelegt sind.
% Somit gibt es genau ein (kongruentes) Quadrat mit einer Diagonalen von 
% $5\MEinheit{cm}$.
%\item Ein Dreieck mit den Winkeln $36\MGrad$ und $54\MGrad$ ist rechtwinklig,
% da der dritte (Innen-)Winkel $180\MGrad - (36\MGrad + 54\MGrad) = 90\MGrad$
% misst. 
%\item In einem Viereck ist die Summe aller Winkel im Gradma"s gleich 
% $2 \cdot 180\MGrad = 360\MGrad$ beziehungsweise $2 \cdot \pi = 2\pi \neq 4\pi$,
% da jedes Viereck in zwei Dreiecke zerlegt werden kann.
%\end{itemize}
\end{MExercise}


\begin{MExercise} %Testaufgabe: Gemeinsame Abschlussaufgabe der 
% TU9-Brueckenkurse von OMB+ und VE&MINT:
Im Dreieck ABC mit den Seitenl"angen 
$a = 5\MEinheit{cm}$,
$b = 6\MEinheit{cm}$ und
$c = 9\MEinheit{cm}$
sind auf der Seite $c$ ein Punkt $P$ und auf der Seite $b$ ein Punkt $Q$ so
gew"ahlt, dass $PQ$ parallel zur Seite $a$ ist und 
$\MGeoAbstand{P}{Q} = \MZahl{0}{50}\MEinheit{cm}$ gilt. Bestimmen Sie die 
Streckenl"angen $\MGeoAbstand{P}{B}$ und $\MGeoAbstand{Q}{C}$ in Zentimeter:
\par
\begin{MExerciseItems}
\item \MEquationItem{$\MGeoAbstand{P}{B}$}{\MLParsedQuestion{15}{8.1}{10}{ExM05TestAg31}$\MEinheit{cm}$}
\item \MEquationItem{$\MGeoAbstand{Q}{C}$}{\MLParsedQuestion{15}{5.4}{10}{ExM05TestAg32}$\MEinheit{cm}$}
\end{MExerciseItems}
%jgl: Tests aktuell ohne Loesungen (daher auskommentiert):
%jgl: Loesung erstellt:
%\begin{MHint}{L"osung}
%Mit den Strahlens"atzen werden zun"achst die Streckenl"angen
%$\MGeoAbstand{A}{P}$ und $\MGeoAbstand{A}{Q}$ berechnet:
%Aus
%\[
%\frac{\MGeoAbstand{A}{P}}{\MGeoAbstand{P}{Q}} = \frac{c}{a} %%
%\quad\text{und}\quad
%\frac{\MGeoAbstand{A}{Q}}{\MGeoAbstand{Q}{P}} = \frac{b}{a} %%
%\]
%folgt
%\[
% \MGeoAbstand{A}{P} = \frac{\MGeoAbstand{P}{Q} \cdot c}{a} %
% = \frac{\MZahl{0}{5} \cdot 9\MEinheit{cm}{5\MEinheit{cm} %%
% = \MZahl{0}{9}\MEinheit{cm} \MDFPeriod %%
%\]
%und
%\[
% \MGeoAbstand{A}{Q} = \frac{\MGeoAbstand{Q}{P} \cdot b}{a} %
% = \frac{\MZahl{0}{5} \cdot 6\MEinheit{cm}{5\MEinheit{cm} %
% = \MZahl{0}{6}\MEinheit{cm} \MDFPeriod %%
%\]
%Damit ergeben sich die gesuchten L"angen zu
%$\MGeoAbstand{P}{B} = 9\MEinheit{cm} - \MZahl{0}{9}\MEinheit{cm} %
% = \MZahl{8}{1}\MEinheit{cm}
%und
%$\MGeoAbstand{Q}{C} = 6\MEinheit{cm} - \MZahl{0}{6}\MEinheit{cm} %
% = \MZahl{5}{4}\MEinheit{cm}
%ergeben.
\end{MExercise}


\begin{MExercise}
\MLabel{VBKM05_A_Quadratkreise}
Ein Quadrat mit Seitenl"ange $a$ sei gegeben. Geben Sie Formeln an f"ur 
Fl"acheninhalt und Umfang des gr"o"stm"oglichen Kreises innerhalb des 
Quadrats, sowie f"ur den kleinstm"oglichen Kreis, der das Quadrat enth"alt:
\begin{MExerciseItems}
\item Umfang des Kreises im Quadrat in Abh"angigkeit von der 
Seitenl"ange $a$: \MLSimplifyQuestion{14}{pi*a}{14}{a}{4}{1}{ExM05TestAg41}
%
\item Fl"acheninhalt des Kreises im Quadrat in Abh"angigkeit von der 
Seitenl"ange $a$: \MLSimplifyQuestion{14}{(1/4)*pi*a*a}{14}{a}{4}{1}{ExM05TestAg42}
%
\item Umfang des Kreises um das Quadrat in Abh"angigkeit von der 
Seitenl"ange $a$: \MLSimplifyQuestion{14}{a*pi*sqrt(2)}{14}{a}{4}{1}{ExM05TestAg43}
%
\item Fl"acheninhalt des Kreises um das Quadrat in Abh"angigkeit von der 
Seitenl"ange $a$: \MLSimplifyQuestion{14}{pi*a*a/2}{14}{a}{4}{1}{ExM05TestAg44}
\end{MExerciseItems}
\par
In den Antwortfeldern d"urfen keine Klammern oder Wurzelausdr"ucke auftauchen.
Schreiben Sie beispielsweise $2^{0.5}$ statt $\sqrt{2}$, um die Wurzel zu 
vermeiden.
%jgl: Tests aktuell ohne Loesungen (daher auskommentiert):
%\par
%\begin{MHint}{L\"osung}
%Der mittig im Quadrat liegende Kreis besitzt den Radius $r=\frac12a$, 
%die halbe Seitenl\"ange des Quadrats. Folglich besitzt er den Umfang 
%$2\pi r =2\pi \cdot \frac12a = \pi a$ und den Fl\"acheninhalt
%$\pi r^2=\pi \cdot \frac14a^2 =\frac14\pi a^2$.\\
%\par
%Befindet sich das Quadrat dagegen mittig innerhalb des Kreises, so ist 
%sein Radius die H\"alfte der L\"ange der Diagonalen vom Quadrat. Diese 
%besitzt die L\"ange $d=\sqrt2\cdot a$. Dies folgt aus dem Satz von Pythagoras,
%da die halbe Quadratdiagonale ein rechtwinkliges Dreieck bildet mit 
%Seitenl\"ange $d$ der Hypotenuse und den beiden Katheten mit L\"angen 
%$\frac12a$: Also ist der Umfang 
%$2\pi r=2\pi\cdot \sqrt2 \cdot\frac12 a=\sqrt2\cdot \pi \cdot a$, und
%der Fl\"acheninhalt ist 
%$\pi r^2=\pi \cdot (\sqrt2\cdot \frac12\cdot a)^2=\pi \cdot\frac12 a^2$.
%\end{MHint}
\end{MExercise}

\end{MTest}
%end of test.

\end{document}
