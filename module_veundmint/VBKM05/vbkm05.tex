% MINTMOD Version P0.1.0, needs to be consistent with preprocesser object in tex2x and MPragma-Version at the end of this file

% Parameter aus Konvertierungsprozess (PDF und HTML-Erzeugung wenn vom Konverter aus gestartet) werden hier eingefuegt, Preambleincludes werden am Schluss angehaengt

\newif\ifttm                % gesetzt falls Uebersetzung in HTML stattfindet, sonst uebersetzung in PDF

% Wahl der Notationsvariante ist im PDF immer std, in der HTML-Uebersetzung wird vom Konverter die Auswahl modifiziert
\newif\ifvariantstd
\newif\ifvariantunotation
\variantstdtrue % Diese Zeile wird vom Konverter erkannt und ggf. modifiziert, daher nicht veraendern!


\def\MOutputDVI{1}
\def\MOutputPDF{2}
\def\MOutputHTML{3}
\newcounter{MOutput}

\ifttm
\usepackage{german}
\usepackage{array}
\usepackage{amsmath}
\usepackage{amssymb}
\usepackage{amsthm}
\else
\documentclass[ngerman,oneside]{scrbook}
\usepackage{etex}
\usepackage[latin1]{inputenc}
\usepackage{textcomp}
\usepackage[ngerman]{babel}
\usepackage[pdftex]{color}
\usepackage{xcolor}
\usepackage{graphicx}
\usepackage[all]{xy}
\usepackage{fancyhdr}
\usepackage{verbatim}
\usepackage{array}
\usepackage{float}
\usepackage{makeidx}
\usepackage{amsmath}
\usepackage{amstext}
\usepackage{amssymb}
\usepackage{amsthm}
\usepackage[ngerman]{varioref}
\usepackage{framed}
\usepackage{supertabular}
\usepackage{longtable}
\usepackage{maxpage}
\usepackage{tikz}
\usepackage{tikzscale}
\usepackage{tikz-3dplot}
\usepackage{bibgerm}
\usepackage{chemarrow}
\usepackage{polynom}
%\usepackage{draftwatermark}
\usepackage{pdflscape}
\usetikzlibrary{calc}
\usetikzlibrary{through}
\usetikzlibrary{shapes.geometric}
\usetikzlibrary{arrows}
\usetikzlibrary{intersections}
\usetikzlibrary{decorations.pathmorphing}
\usetikzlibrary{external}
\usetikzlibrary{patterns}
\usetikzlibrary{fadings}
\usepackage[colorlinks=true,linkcolor=blue]{hyperref} 
\usepackage[all]{hypcap}
%\usepackage[colorlinks=true,linkcolor=blue,bookmarksopen=true]{hyperref} 
\usepackage{ifpdf}

\usepackage{movie15}

\setcounter{tocdepth}{2} % In Inhaltsverzeichnis bis subsection
\setcounter{secnumdepth}{3} % Nummeriert bis subsubsection

\setlength{\LTpost}{0pt} % Fuer longtable
\setlength{\parindent}{0pt}
\setlength{\parskip}{8pt}
%\setlength{\parskip}{9pt plus 2pt minus 1pt}
\setlength{\abovecaptionskip}{-0.25ex}
\setlength{\belowcaptionskip}{-0.25ex}
\fi

\ifttm
\newcommand{\MDebugMessage}[1]{\special{html:<!-- debugprint;;}#1\special{html:; //-->}}
\else
%\newcommand{\MDebugMessage}[1]{\immediate\write\mintlog{#1}}
\newcommand{\MDebugMessage}[1]{}
\fi

\def\MPageHeaderDef{%
\pagestyle{fancy}%
\fancyhead[r]{(C) VE\&MINT-Projekt}
\fancyfoot[c]{\thepage\\--- CCL BY-SA 3.0 ---}
}


\ifttm%
\def\MRelax{}%
\else%
\def\MRelax{\relax}%
\fi%

%--------------------------- Uebernahme von speziellen XML-Versionen einiger LaTeX-Kommandos aus xmlbefehle.tex vom alten Kasseler Konverter ---------------

\newcommand{\MSep}{\left\|{\phantom{\frac1g}}\right.}

\newcommand{\ML}{L}

\newcommand{\MGGT}{\mathrm{ggT}}


\ifttm
% Verhindert dass die subsection-nummer doppelt in der toccaption auftaucht (sollte ggf. in toccaption gefixt werden so dass diese Ueberschreibung nicht notwendig ist)
\renewcommand{\thesubsection}{}
% Kommandos die ttm nicht kennt
\newcommand{\binomial}[2]{{#1 \choose #2}} %  Binomialkoeffizienten
\newcommand{\eur}{\begin{html}&euro;\end{html}}
\newcommand{\square}{\begin{html}&square;\end{html}}
\newcommand{\glqq}{"'}  \newcommand{\grqq}{"'}
\newcommand{\nRightarrow}{\special{html: &nrArr; }}
\newcommand{\nmid}{\special{html: &nmid; }}
\newcommand{\nparallel}{\begin{html}&nparallel;\end{html}}
\newcommand{\mapstoo}{\begin{html}<mo>&map;</mo>\end{html}}

% Schnitt und Vereinigungssymbole von Mengen haben zu kleine Abstaende; korrigiert:
\newcommand{\ccup}{\,\!\cup\,\!}
\newcommand{\ccap}{\,\!\cap\,\!}


% Umsetzung von mathbb im HTML
\renewcommand{\mathbb}[1]{\begin{html}<mo>&#1opf;</mo>\end{html}}
\fi

%---------------------- Strukturierung ----------------------------------------------------------------------------------------------------------------------

%---------------------- Kapselung des sectioning findet auf drei Ebenen statt:
% 1. Die LateX-Befehl
% 2. Die D-Versionen der Befehle, die nur die Grade der Abschnitte umhaengen falls notwendig
% 3. Die M-Versionen der Befehle, die zusaetzliche Formatierungen vornehmen, Skripten starten und das HTML codieren
% Im Modultext duerfen nur die M-Befehle verwendet werden!

\ifttm

  \def\Dsubsubsubsection#1{\subsubsubsection{#1}}
  \def\Dsubsubsection#1{\subsubsection{#1}\addtocounter{subsubsection}{1}} % ttm-Fehler korrigieren
  \def\Dsubsection#1{\subsection{#1}}
  \def\Dsection#1{\section{#1}} % Im HTML wird nur der Sektionstitel gegeben
  \def\Dchapter#1{\chapter{#1}}
  \def\Dsubsubsubsectionx#1{\subsubsubsection*{#1}}
  \def\Dsubsubsectionx#1{\subsubsection*{#1}}
  \def\Dsubsectionx#1{\subsection*{#1}}
  \def\Dsectionx#1{\section*{#1}}
  \def\Dchapterx#1{\chapter*{#1}}

\else

  \def\Dsubsubsubsection#1{\subsubsection{#1}}
  \def\Dsubsubsection#1{\subsection{#1}}
  \def\Dsubsection#1{\section{#1}}
  \def\Dsection#1{\chapter{#1}}
  \def\Dchapter#1{\title{#1}}
  \def\Dsubsubsubsectionx#1{\subsubsection*{#1}}
  \def\Dsubsubsectionx#1{\subsection*{#1}}
  \def\Dsubsectionx#1{\section*{#1}}
  \def\Dsectionx#1{\chapter*{#1}}

\fi

\newcommand{\MStdPoints}{4}
\newcommand{\MSetPoints}[1]{\renewcommand{\MStdPoints}{#1}}

% Befehl zum Abbruch der Erstellung (nur PDF)
\newcommand{\MAbort}[1]{\err{#1}}

% Prefix vor Dateieinbindungen, wird in der Baumdatei mit \renewcommand modifiziert
% und auf das Verzeichnisprefix gesetzt, in dem das gerade bearbeitete tex-Dokument liegt.
% Im HTML wird es auf das Verzeichnis der HTML-Datei gesetzt.
% Das Prefix muss mit / enden !
\newcommand{\MDPrefix}{.}

% MRegisterFile notiert eine Datei zur Einbindung in den HTML-Baum. Grafiken mit MGraphics werden automatisch eingebunden.
% Mit MLastFile erhaelt man eine Markierung fuer die zuletzt registrierte Datei.
% Diese Markierung wird im postprocessing durch den physikalischen Dateinamen ersetzt, aber nur den Namen (d.h. \MMaterial gehoert noch davor, vgl Definition von MGraphics)
% Parameter: Pfad/Name der Datei bzw. des Ordners, relativ zur Position des Modul-Tex-Dokuments.
\ifttm
\newcommand{\MRegisterFile}[1]{\addtocounter{MFileNumber}{1}\special{html:<!-- registerfile;;}#1\special{html:;;}\MDPrefix\special{html:;;}\arabic{MFileNumber}\special{html:; //-->}}
\else
\newcommand{\MRegisterFile}[1]{\addtocounter{MFileNumber}{1}}
\fi

% Testen welcher Uebersetzer hier am Werk ist

\ifttm
\setcounter{MOutput}{3}
\else
\ifx\pdfoutput\undefined
  \pdffalse
  \setcounter{MOutput}{\MOutputDVI}
  \message{Verarbeitung mit latex, Ausgabe in dvi.}
\else
  \setcounter{MOutput}{\MOutputPDF}
  \message{Verarbeitung mit pdflatex, Ausgabe in pdf.}
  \ifnum \pdfoutput=0
    \pdffalse
  \setcounter{MOutput}{\MOutputDVI}
  \message{Verarbeitung mit pdflatex, Ausgabe in dvi.}
  \else
    \ifnum\pdfoutput=1
    \pdftrue
  \setcounter{MOutput}{\MOutputPDF}
  \message{Verarbeitung mit pdflatex, Ausgabe in pdf.}
    \fi
  \fi
\fi
\fi

\ifnum\value{MOutput}=\MOutputPDF
\DeclareGraphicsExtensions{.pdf,.png,.jpg}
\fi

\ifnum\value{MOutput}=\MOutputDVI
\DeclareGraphicsExtensions{.eps,.png,.jpg}
\fi

\ifnum\value{MOutput}=\MOutputHTML
% Wird vom Konverter leider nicht erkannt und daher in split.pm hardcodiert!
\DeclareGraphicsExtensions{.png,.jpg,.gif}
\fi

% Umdefinition der hyperref-Nummerierung im PDF-Modus
\ifttm
\else
\renewcommand{\theHfigure}{\arabic{chapter}.\arabic{section}.\arabic{figure}}
\fi

% Makro, um in der HTML-Ausgabe die zuerst zu oeffnende Datei zu kennzeichnen
\ifttm
\newcommand{\MGlobalStart}{\special{html:<!-- mglobalstarttag -->}}
\else
\newcommand{\MGlobalStart}{}
\fi

% Makro, um bei scormlogin ein pullen des Benutzers bei Aufruf der Seite zu erzwingen (typischerweise auf der Einstiegsseite)
\ifttm
\newcommand{\MPullSite}{\special{html:<!-- pullsite //-->}}
\else
\newcommand{\MPullSite}{}
\fi

% Makro, um in der HTML-Ausgabe die Kapiteluebersicht zu kennzeichnen
\ifttm
\newcommand{\MGlobalChapterTag}{\special{html:<!-- mglobalchaptertag -->}}
\else
\newcommand{\MGlobalChapterTag}{}
\fi

% Makro, um in der HTML-Ausgabe die Konfiguration zu kennzeichnen
\ifttm
\newcommand{\MGlobalConfTag}{\special{html:<!-- mglobalconfigtag -->}}
\else
\newcommand{\MGlobalConfTag}{}
\fi

% Makro, um in der HTML-Ausgabe die Standortbeschreibung zu kennzeichnen
\ifttm
\newcommand{\MGlobalLocationTag}{\special{html:<!-- mgloballocationtag -->}}
\else
\newcommand{\MGlobalLocationTag}{}
\fi

% Makro, um in der HTML-Ausgabe die persoenlichen Daten zu kennzeichnen
\ifttm
\newcommand{\MGlobalDataTag}{\special{html:<!-- mglobaldatatag -->}}
\else
\newcommand{\MGlobalDataTag}{}
\fi

% Makro, um in der HTML-Ausgabe die Suchseite zu kennzeichnen
\ifttm
\newcommand{\MGlobalSearchTag}{\special{html:<!-- mglobalsearchtag -->}}
\else
\newcommand{\MGlobalSearchTag}{}
\fi

% Makro, um in der HTML-Ausgabe die Favoritenseite zu kennzeichnen
\ifttm
\newcommand{\MGlobalFavoTag}{\special{html:<!-- mglobalfavoritestag -->}}
\else
\newcommand{\MGlobalFavoTag}{}
\fi

% Makro, um in der HTML-Ausgabe die Eingangstestseite zu kennzeichnen
\ifttm
\newcommand{\MGlobalSTestTag}{\special{html:<!-- mglobalstesttag -->}}
\else
\newcommand{\MGlobalSTestTag}{}
\fi

% Makro, um in der PDF-Ausgabe ein Wasserzeichen zu definieren
\ifttm
\newcommand{\MWatermarkSettings}{\relax}
\else
\newcommand{\MWatermarkSettings}{%
% \SetWatermarkText{(c) MINT-Kolleg Baden-W�rttemberg 2014}
% \SetWatermarkLightness{0.85}
% \SetWatermarkScale{1.5}
}
\fi

\ifttm
\newcommand{\MBinom}[2]{\left({\begin{array}{c} #1 \\ #2 \end{array}}\right)}
\else
\newcommand{\MBinom}[2]{\binom{#1}{#2}}
\fi

\ifttm
\newcommand{\DeclareMathOperator}[2]{\def#1{\mathrm{#2}}}
\newcommand{\operatorname}[1]{\mathrm{#1}}
\fi

%----------------- Makros fuer die gemischte HTML/PDF-Konvertierung ------------------------------

\newcommand{\MTestName}{\relax} % wird durch Test-Umgebung gesetzt

% Fuer experimentelle Kursinhalte, die im Release-Umsetzungsvorgang eine Fehlermeldung
% produzieren sollen aber sonst normal umgesetzt werden
\newenvironment{MExperimental}{%
}{%
}

% Wird von ttm nicht richtig umgesetzt!!
\newenvironment{MExerciseItems}{%
\renewcommand\theenumi{\alph{enumi}}%
\begin{enumerate}%
}{%
\end{enumerate}%
}


\definecolor{infoshadecolor}{rgb}{0.75,0.75,0.75}
\definecolor{exmpshadecolor}{rgb}{0.875,0.875,0.875}
\definecolor{expeshadecolor}{rgb}{0.95,0.95,0.95}
\definecolor{framecolor}{rgb}{0.2,0.2,0.2}

% Bei PDF-Uebersetzung wird hinter den Start jeder Satz/Info-aehnlichen Umgebung eine leere mbox gesetzt, damit
% fuehrende Listen oder enums nicht den Zeilenumbruch kaputtmachen
%\ifttm
\def\MTB{}
%\else
%\def\MTB{\mbox{}}
%\fi


\ifttm
\newcommand{\MRelates}{\special{html:<mi>&wedgeq;</mi>}}
\else
\def\MRelates{\stackrel{\scriptscriptstyle\wedge}{=}}
\fi

\def\MInch{\text{''}}
\def\Mdd{\textit{''}}

\ifttm
\def\MNL{ \newline }
\newenvironment{MArray}[1]{\begin{array}{#1}}{\end{array}}
\else
\def\MNL{ \\ }
\newenvironment{MArray}[1]{\begin{array}{#1}}{\end{array}}
\fi

\newcommand{\MBox}[1]{$\mathrm{#1}$}
\newcommand{\MMBox}[1]{\mathrm{#1}}


\ifttm%
\newcommand{\Mtfrac}[2]{{\textstyle \frac{#1}{#2}}}
\newcommand{\Mdfrac}[2]{{\displaystyle \frac{#1}{#2}}}
\newcommand{\Mmeasuredangle}{\special{html:<mi>&angmsd;</mi>}}
\else%
\newcommand{\Mtfrac}[2]{\tfrac{#1}{#2}}
\newcommand{\Mdfrac}[2]{\dfrac{#1}{#2}}
\newcommand{\Mmeasuredangle}{\measuredangle}
\relax
\fi

% Matrizen und Vektoren

% Inhalt wird in der Form a & b \\ c & d erwartet
% Vorsicht: MVector = Komponentenspalte, MVec = Variablensymbol
\ifttm%
\newcommand{\MVector}[1]{\left({\begin{array}{c}#1\end{array}}\right)}
\else%
\newcommand{\MVector}[1]{\begin{pmatrix}#1\end{pmatrix}}
\fi



\newcommand{\MVec}[1]{\vec{#1}}
\newcommand{\MDVec}[1]{\overrightarrow{#1}}

%----------------- Umgebungen fuer Definitionen und Saetze ----------------------------------------

% Fuegt einen Tabellen-Zeilenumbruch ein im PDF, aber nicht im HTML
\newcommand{\TSkip}{\ifttm \else&\ \\\fi}

\newenvironment{infoshaded}{%
\def\FrameCommand{\fboxsep=\FrameSep \fcolorbox{framecolor}{infoshadecolor}}%
\MakeFramed {\advance\hsize-\width \FrameRestore}}%
{\endMakeFramed}

\newenvironment{expeshaded}{%
\def\FrameCommand{\fboxsep=\FrameSep \fcolorbox{framecolor}{expeshadecolor}}%
\MakeFramed {\advance\hsize-\width \FrameRestore}}%
{\endMakeFramed}

\newenvironment{exmpshaded}{%
\def\FrameCommand{\fboxsep=\FrameSep \fcolorbox{framecolor}{exmpshadecolor}}%
\MakeFramed {\advance\hsize-\width \FrameRestore}}%
{\endMakeFramed}

\def\STDCOLOR{black}

\ifttm%
\else%
\newtheoremstyle{MSatzStyle}
  {1cm}                   %Space above
  {1cm}                   %Space below
  {\normalfont\itshape}   %Body font
  {}                      %Indent amount (empty = no indent,
                          %\parindent = para indent)
  {\normalfont\bfseries}  %Thm head font
  {}                      %Punctuation after thm head
  {\newline}              %Space after thm head: " " = normal interword
                          %space; \newline = linebreak
  {\thmname{#1}\thmnumber{ #2}\thmnote{ (#3)}}
                          %Thm head spec (can be left empty, meaning
                          %`normal')
                          %
\newtheoremstyle{MDefStyle}
  {1cm}                   %Space above
  {1cm}                   %Space below
  {\normalfont}           %Body font
  {}                      %Indent amount (empty = no indent,
                          %\parindent = para indent)
  {\normalfont\bfseries}  %Thm head font
  {}                      %Punctuation after thm head
  {\newline}              %Space after thm head: " " = normal interword
                          %space; \newline = linebreak
  {\thmname{#1}\thmnumber{ #2}\thmnote{ (#3)}}
                          %Thm head spec (can be left empty, meaning
                          %`normal')
\fi%

\newcommand{\MInfoText}{Info}

\newcounter{MHintCounter}
\newcounter{MCodeEditCounter}

\newcounter{MLastIndex}  % Enthaelt die dritte Stelle (Indexnummer) des letzten angelegten Objekts
\newcounter{MLastType}   % Enthaelt den Typ des letzten angelegten Objekts (mithilfe der unten definierten Konstanten). Die Entscheidung, wie der Typ dargstellt wird, wird in split.pm beim Postprocessing getroffen.
\newcounter{MLastTypeEq} % =1 falls das Label in einer Matheumgebung (equation, eqnarray usw.) steht, =2 falls das Label in einer table-Umgebung steht

% Da ttm keine Zahlmakros verarbeiten kann, werden diese Nummern in den Zuweisungen hardcodiert!
\def\MTypeSection{1}          %# Zaehler ist section
\def\MTypeSubsection{2}       %# Zaehler ist subsection
\def\MTypeSubsubsection{3}    %# Zaehler ist subsubsection
\def\MTypeInfo{4}             %# Eine Infobox, Separatzaehler fuer die Chemie (auch wenn es dort nicht nummeriert wird) ist MInfoCounter
\def\MTypeExercise{5}         %# Eine Aufgabe, Separatzaehler fuer die Chemie ist MExerciseCounter
\def\MTypeExample{6}          %# Eine Beispielbox, Separatzaehler fuer die Chemie ist MExampleCounter
\def\MTypeExperiment{7}       %# Eine Versuchsbox, Separatzaehler fuer die Chemie ist MExperimentCounter
\def\MTypeGraphics{8}         %# Eine Graphik, Separatzaehler fuer alle FB ist MGraphicsCounter
\def\MTypeTable{9}            %# Eine Tabellennummer, hat keinen Zaehler da durch table gezaehlt wird
\def\MTypeEquation{10}        %# Eine Gleichungsnummer, hat keinen Zaehler da durch equation/eqnarray gezaehlt wird
\def\MTypeTheorem{11}         % Ein theorem oder xtheorem, Separatzaehler fuer die Chemie ist MTheoremCounter
\def\MTypeVideo{12}           %# Ein Video,Separatzaehler fuer alle FB ist MVideoCounter
\def\MTypeEntry{13}           %# Ein Eintrag fuer die Stichwortliste, wird nicht gezaehlt sondern erhaelt im preparsing ein unique-label 

% Zaehler fuer das Labelsystem sind prefixcounter, jeder Zaehler wird VOR dem gezaehlten Objekt inkrementiert und zaehlt daher das aktuelle Objekt
\newcounter{MInfoCounter}
\newcounter{MExerciseCounter}
\newcounter{MExampleCounter}
\newcounter{MExperimentCounter}
\newcounter{MGraphicsCounter}
\newcounter{MTableCounter}
\newcounter{MEquationCounter}  % Nur im HTML, sonst durch "equation"-counter von latex realisiert
\newcounter{MTheoremCounter}
\newcounter{MObjectCounter}   % Gemeinsamer Zaehler fuer Objekte (ausser Grafiken/Tabellen) in Mathe/Info/Physik
\newcounter{MVideoCounter}
\newcounter{MEntryCounter}

\newcounter{MTestSite} % 1 = Subsubsection ist eine Pruefungsseite, 0 = ist eine normale Seite (inkl. Hilfeseite)

\def\MCell{$\phantom{a}$}

\newenvironment{MExportExercise}{\begin{MExercise}}{\end{MExercise}} % wird von mconvert abgefangen

\def\MGenerateExNumber{%
\ifnum\value{MSepNumbers}=0%
\arabic{section}.\arabic{subsection}.\arabic{MObjectCounter}\setcounter{MLastIndex}{\value{MObjectCounter}}%
\else%
\arabic{section}.\arabic{subsection}.\arabic{MExerciseCounter}\setcounter{MLastIndex}{\value{MExerciseCounter}}%
\fi%
}%

\def\MGenerateExmpNumber{%
\ifnum\value{MSepNumbers}=0%
\arabic{section}.\arabic{subsection}.\arabic{MObjectCounter}\setcounter{MLastIndex}{\value{MObjectCounter}}%
\else%
\arabic{section}.\arabic{subsection}.\arabic{MExerciseCounter}\setcounter{MLastIndex}{\value{MExampleCounter}}%
\fi%
}%

\def\MGenerateInfoNumber{%
\ifnum\value{MSepNumbers}=0%
\arabic{section}.\arabic{subsection}.\arabic{MObjectCounter}\setcounter{MLastIndex}{\value{MObjectCounter}}%
\else%
\arabic{section}.\arabic{subsection}.\arabic{MExerciseCounter}\setcounter{MLastIndex}{\value{MInfoCounter}}%
\fi%
}%

\def\MGenerateSiteNumber{%
\arabic{section}.\arabic{subsection}.\arabic{subsubsection}%
}%

% Funktionalitaet fuer Auswahlaufgaben

\newcounter{MExerciseCollectionCounter} % = 0 falls nicht in collection-Umgebung, ansonsten Schachtelungstiefe
\newcounter{MExerciseCollectionTextCounter} % wird von MExercise-Umgebung inkrementiert und von MExerciseCollection-Umgebung auf Null gesetzt

\ifttm
% MExerciseCollection gruppiert Aufgaben, die dynamisch aus der Datenbank gezogen werden und nicht direkt in der HTML-Seite stehen
% Parameter: #1 = ID der Collection, muss eindeutig fuer alle IN DER DB VORHANDENEN collections sein unabhaengig vom Kurs
%            #2 = Optionsargument (im Moment: 1 = Iterative Auswahl, 2 = Zufallsbasierte Auswahl)
\newenvironment{MExerciseCollection}[2]{%
\addtocounter{MExerciseCollectionCounter}{1}
\setcounter{MExerciseCollectionTextCounter}{0}
\special{html:<!-- mexercisecollectionstart;;}#1\special{html:;;}#2\special{html:;; //-->}%
}{%
\special{html:<!-- mexercisecollectionstop //-->}%
\addtocounter{MExerciseCollectionCounter}{-1}
}
\else
\newenvironment{MExerciseCollection}[2]{%
\addtocounter{MExerciseCollectionCounter}{1}
\setcounter{MExerciseCollectionTextCounter}{0}
}{%
\addtocounter{MExerciseCollectionCounter}{-1}
}
\fi

% Bei Uebersetzung nach PDF werden die theorem-Umgebungen verwendet, bei Uebersetzung in HTML ein manuelles Makro
\ifttm%

  \newenvironment{MHint}[1]{  \special{html:<button name="Name_MHint}\arabic{MHintCounter}\special{html:" class="hintbutton_closed" id="MHint}\arabic{MHintCounter}\special{html:_button" %
  type="button" onclick="toggle_hint('MHint}\arabic{MHintCounter}\special{html:');">}#1\special{html:</button>}
  \special{html:<div class="hint" style="display:none" id="MHint}\arabic{MHintCounter}\special{html:"> }}{\begin{html}</div>\end{html}\addtocounter{MHintCounter}{1}}

  \newenvironment{MCOSHZusatz}{  \special{html:<button name="Name_MHint}\arabic{MHintCounter}\special{html:" class="chintbutton_closed" id="MHint}\arabic{MHintCounter}\special{html:_button" %
  type="button" onclick="toggle_hint('MHint}\arabic{MHintCounter}\special{html:');">}Weiterf�hrende Inhalte\special{html:</button>}
  \special{html:<div class="hintc" style="display:none" id="MHint}\arabic{MHintCounter}\special{html:">
  <div class="coshwarn">Diese Inhalte gehen �ber das Kursniveau hinaus und werden in den Aufgaben und Tests nicht abgefragt.</div><br />}
  \addtocounter{MHintCounter}{1}}{\begin{html}</div>\end{html}}

  
  \newenvironment{MDefinition}{\begin{definition}\setcounter{MLastIndex}{\value{definition}}\ \\}{\end{definition}}

  
  \newenvironment{MExercise}{
  \renewcommand{\MStdPoints}{4}
  \addtocounter{MExerciseCounter}{1}
  \addtocounter{MObjectCounter}{1}
  \setcounter{MLastType}{5}

  \ifnum\value{MExerciseCollectionCounter}=0\else\addtocounter{MExerciseCollectionTextCounter}{1}\special{html:<!-- mexercisetextstart;;}\arabic{MExerciseCollectionTextCounter}\special{html:;; //-->}\fi
  \special{html:<div class="aufgabe" id="ADIV_}\MGenerateExNumber\special{html:">}%
  \textbf{Aufgabe \MGenerateExNumber
  } \ \\}{
  \special{html:</div><!-- mfeedbackbutton;Aufgabe;}\arabic{MTestSite}\special{html:;}\MGenerateExNumber\special{html:; //-->}
  \ifnum\value{MExerciseCollectionCounter}=0\else\special{html:<!-- mexercisetextstop //-->}\fi
  }

  % Stellt eine Kombination aus Aufgabe, Loesungstext und Eingabefeld bereit,
  % bei der Aufgabentext und Musterloesung sowie die zugehoerigen Feldelemente
  % extern bezogen und div-aktualisiert werden, das Eingabefeld aber immer das gleiche ist.
  \newenvironment{MFetchExercise}{
  \addtocounter{MExerciseCounter}{1}
  \addtocounter{MObjectCounter}{1}
  \setcounter{MLastType}{5}

  \special{html:<div class="aufgabe" id="ADIV_}\MGenerateExNumber\special{html:">}%
  \textbf{Aufgabe \MGenerateExNumber
  } \ \\%
  \special{html:</div><div class="exfetch_text" id="ADIVTEXT_}\MGenerateExNumber\special{html:">}%
  \special{html:</div><div class="exfetch_sol" id="ADIVSOL_}\MGenerateExNumber\special{html:">}%
  \special{html:</div><div class="exfetch_input" id="ADIVINPUT_}\MGenerateExNumber\special{html:">}%
  }{
  \special{html:</div>}
  }

  \newenvironment{MExample}{
  \addtocounter{MExampleCounter}{1}
  \addtocounter{MObjectCounter}{1}
  \setcounter{MLastType}{6}
  \begin{html}
  <div class="exmp">
  <div class="exmprahmen">
  \end{html}\textbf{Beispiel
  \ifnum\value{MSepNumbers}=0
  \arabic{section}.\arabic{subsection}.\arabic{MObjectCounter}\setcounter{MLastIndex}{\value{MObjectCounter}}
  \else
  \arabic{section}.\arabic{subsection}.\arabic{MExampleCounter}\setcounter{MLastIndex}{\value{MExampleCounter}}
  \fi
  } \ \\}{\begin{html}</div>
  </div>
  \end{html}
  \special{html:<!-- mfeedbackbutton;Beispiel;}\arabic{MTestSite}\special{html:;}\MGenerateExmpNumber\special{html:; //-->}
  }

  \newenvironment{MExperiment}{
  \addtocounter{MExperimentCounter}{1}
  \addtocounter{MObjectCounter}{1}
  \setcounter{MLastType}{7}
  \begin{html}
  <div class="expe">
  <div class="experahmen">
  \end{html}\textbf{Versuch
  \ifnum\value{MSepNumbers}=0
  \arabic{section}.\arabic{subsection}.\arabic{MObjectCounter}\setcounter{MLastIndex}{\value{MObjectCounter}}
  \else
%  \arabic{MExperimentCounter}\setcounter{MLastIndex}{\value{MExperimentCounter}}
  \arabic{section}.\arabic{subsection}.\arabic{MExperimentCounter}\setcounter{MLastIndex}{\value{MExperimentCounter}}
  \fi
  } \ \\}{\begin{html}</div>
  </div>
  \end{html}}

  \newenvironment{MChemInfo}{
  \setcounter{MLastType}{4}
  \begin{html}
  <div class="info">
  <div class="inforahmen">
  \end{html}}{\begin{html}</div>
  </div>
  \end{html}}

  \newenvironment{MXInfo}[1]{
  \addtocounter{MInfoCounter}{1}
  \addtocounter{MObjectCounter}{1}
  \setcounter{MLastType}{4}
  \begin{html}
  <div class="info">
  <div class="inforahmen">
  \end{html}\textbf{#1
  \ifnum\value{MInfoNumbers}=0
  \else
    \ifnum\value{MSepNumbers}=0
    \arabic{section}.\arabic{subsection}.\arabic{MObjectCounter}\setcounter{MLastIndex}{\value{MObjectCounter}}
    \else
    \arabic{MInfoCounter}\setcounter{MLastIndex}{\value{MInfoCounter}}
    \fi
  \fi
  } \ \\}{\begin{html}</div>
  </div>
  \end{html}
  \special{html:<!-- mfeedbackbutton;Info;}\arabic{MTestSite}\special{html:;}\MGenerateInfoNumber\special{html:; //-->}
  }

  \newenvironment{MInfo}{\ifnum\value{MInfoNumbers}=0\begin{MChemInfo}\else\begin{MXInfo}{Info}\ \\ \fi}{\ifnum\value{MInfoNumbers}=0\end{MChemInfo}\else\end{MXInfo}\fi}

\else%

  \theoremstyle{MSatzStyle}
  \newtheorem{thm}{Satz}[section]
  \newtheorem{thmc}{Satz}
  \theoremstyle{MDefStyle}
  \newtheorem{defn}[thm]{Definition}
  \newtheorem{exmp}[thm]{Beispiel}
  \newtheorem{info}[thm]{\MInfoText}
  \theoremstyle{MDefStyle}
  \newtheorem{defnc}{Definition}
  \theoremstyle{MDefStyle}
  \newtheorem{exmpc}{Beispiel}[section]
  \theoremstyle{MDefStyle}
  \newtheorem{infoc}{\MInfoText}
  \theoremstyle{MDefStyle}
  \newtheorem{exrc}{Aufgabe}[section]
  \theoremstyle{MDefStyle}
  \newtheorem{verc}{Versuch}[section]
  
  \newenvironment{MFetchExercise}{}{} % kann im PDF nicht dargestellt werden
  
  \newenvironment{MExercise}{\begin{exrc}\renewcommand{\MStdPoints}{1}\MTB}{\end{exrc}}
  \newenvironment{MHint}[1]{\ \\ \underline{#1:}\\}{}
  \newenvironment{MCOSHZusatz}{\ \\ \underline{Weiterf�hrende Inhalte:}\\}{}
  \newenvironment{MDefinition}{\ifnum\value{MInfoNumbers}=0\begin{defnc}\else\begin{defn}\fi\MTB}{\ifnum\value{MInfoNumbers}=0\end{defnc}\else\end{defn}\fi}
%  \newenvironment{MExample}{\begin{exmp}}{\ \linebreak[1] \ \ \ \ $\phantom{a}$ \ \hfill $\blacklozenge$\end{exmp}}
  \newenvironment{MExample}{
    \ifnum\value{MInfoNumbers}=0\begin{exmpc}\else\begin{exmp}\fi
    \MTB
    \begin{exmpshaded}
    \ \newline
}{
    \end{exmpshaded}
    \ifnum\value{MInfoNumbers}=0\end{exmpc}\else\end{exmp}\fi
}
  \newenvironment{MChemInfo}{\begin{infoshaded}}{\end{infoshaded}}

  \newenvironment{MInfo}{\ifnum\value{MInfoNumbers}=0\begin{MChemInfo}\else\renewcommand{\MInfoText}{Info}\begin{info}\begin{infoshaded}
  \MTB
   \ \newline
    \fi
  }{\ifnum\value{MInfoNumbers}=0\end{MChemInfo}\else\end{infoshaded}\end{info}\fi}

  \newenvironment{MXInfo}[1]{
    \renewcommand{\MInfoText}{#1}
    \ifnum\value{MInfoNumbers}=0\begin{infoc}\else\begin{info}\fi%
    \MTB
    \begin{infoshaded}
    \ \newline
  }{\end{infoshaded}\ifnum\value{MInfoNumbers}=0\end{infoc}\else\end{info}\fi}

  \newenvironment{MExperiment}{
    \renewcommand{\MInfoText}{Versuch}
    \ifnum\value{MInfoNumbers}=0\begin{verc}\else\begin{info}\fi
    \MTB
    \begin{expeshaded}
    \ \newline
  }{
    \end{expeshaded}
    \ifnum\value{MInfoNumbers}=0\end{verc}\else\end{info}\fi
  }
\fi%

% MHint sollte nicht direkt fuer Loesungen benutzt werden wegen solutionselect
\newenvironment{MSolution}{\begin{MHint}{L"osung}}{\end{MHint}}

\newcounter{MCodeCounter}

\ifttm
\newenvironment{MCode}{\special{html:<!-- mcodestart -->}\ttfamily\color{blue}}{\special{html:<!-- mcodestop -->}}
\else
\newenvironment{MCode}{\begin{flushleft}\ttfamily\addtocounter{MCodeCounter}{1}}{\addtocounter{MCodeCounter}{-1}\end{flushleft}}
% Ohne color-Statement da inkompatible mit framed/shaded-Boxen aus dem framed-package
\fi

%----------------- Sonderdefinitionen fuer Symbole, die der Konverter nicht kann ----------------------------------------------

\ifttm%
\newcommand{\MUnderset}[2]{\underbrace{#2}_{#1}}%
\else%
\newcommand{\MUnderset}[2]{\underset{#1}{#2}}%
\fi%

\ifttm
\newcommand{\MThinspace}{\special{html:<mi>&#x2009;</mi>}}
\else
\newcommand{\MThinspace}{\,}
\fi

\ifttm
\newcommand{\glq}{\begin{html}&sbquo;\end{html}}
\newcommand{\grq}{\begin{html}&lsquo;\end{html}}
\newcommand{\glqq}{\begin{html}&bdquo;\end{html}}
\newcommand{\grqq}{\begin{html}&ldquo;\end{html}}
\fi

\ifttm
\newcommand{\MNdash}{\begin{html}&ndash;\end{html}}
\else
\newcommand{\MNdash}{--}
\fi

%\ifttm\def\MIU{\special{html:<mi>&#8520;</mi>}}\else\def\MIU{\mathrm{i}}\fi
\def\MIU{\mathrm{i}}
\def\MEU{e} % TU9-Onlinekurs: italic-e
%\def\MEU{\mathrm{e}} % Alte Onlinemodule: roman-e
\def\MD{d} % Kursives d in Integralen im TU9-Onlinekurs
%\def\MD{\mathrm{d}} % roman-d in den alten Onlinemodulen
\def\MDB{\|}

%zusaetzlicher Leerraum vor "\MD"
\ifttm%
\def\MDSpace{\special{html:<mi>&#x2009;</mi>}}
\else%
\def\MDSpace{\,}
\fi%
\newcommand{\MDwSp}{\MDSpace\MD}%

\ifttm
\def\Mdq{\dq}
\else
\def\Mdq{\dq}
\fi

\def\MSpan#1{\left<{#1}\right>}
\def\MSetminus{\setminus}
\def\MIM{I}

\ifttm
\newcommand{\ld}{\text{ld}}
\newcommand{\lg}{\text{lg}}
\else
\DeclareMathOperator{\ld}{ld}
%\newcommand{\lg}{\text{lg}} % in latex schon definiert
\fi


\def\Mmapsto{\ifttm\special{html:<mi>&mapsto;</mi>}\else\mapsto\fi} 
\def\Mvarphi{\ifttm\phi\else\varphi\fi}
\def\Mphi{\ifttm\varphi\else\phi\fi}
\ifttm%
\newcommand{\MEumu}{\special{html:<mi>&#x3BC;</mi>}}%
\else%
\newcommand{\MEumu}{\textrm{\textmu}}%
\fi
\def\Mvarepsilon{\ifttm\epsilon\else\varepsilon\fi}
\def\Mepsilon{\ifttm\varepsilon\else\epsilon\fi}
\def\Mvarkappa{\ifttm\kappa\else\varkappa\fi}
\def\Mkappa{\ifttm\varkappa\else\kappa\fi}
\def\Mcomplement{\ifttm\special{html:<mi>&comp;</mi>}\else\complement\fi} 
\def\MWW{\mathrm{WW}}
\def\Mmod{\ifttm\special{html:<mi>&nbsp;mod&nbsp;</mi>}\else\mod\fi} 

\ifttm%
\def\mod{\text{\;mod\;}}%
\def\MNEquiv{\special{html:<mi>&NotCongruent;</mi>}}% 
\def\MNSubseteq{\special{html:<mi>&NotSubsetEqual;</mi>}}%
\def\MEmptyset{\special{html:<mi>&empty;</mi>}}%
\def\MVDots{\special{html:<mi>&#x22EE;</mi>}}%
\def\MHDots{\special{html:<mi>&#x2026;</mi>}}%
\def\Mddag{\special{html:<mi>&#x1202;</mi>}}%
\def\sphericalangle{\special{html:<mi>&measuredangle;</mi>}}%
\def\nparallel{\special{html:<mi>&nparallel;</mi>}}%
\def\MProofEnd{\special{html:<mi>&#x25FB;</mi>}}%
\newenvironment{MProof}[1]{\underline{#1}:\MCR\MCR}{\hfill $\MProofEnd$}%
\else%
\def\MNEquiv{\not\equiv}%
\def\MNSubseteq{\not\subseteq}%
\def\MEmptyset{\emptyset}%
\def\MVDots{\vdots}%
\def\MHDots{\hdots}%
\def\Mddag{\ddag}%
\newenvironment{MProof}[1]{\begin{proof}[#1]}{\end{proof}}%
\fi%



% Spaces zum Auffuellen von Tabellenbreiten, die nur im HTML wirken
\ifttm%
\def\MTSP{\:}%
\else%
\def\MTSP{}%
\fi%

\DeclareMathOperator{\arsinh}{arsinh}
\DeclareMathOperator{\arcosh}{arcosh}
\DeclareMathOperator{\artanh}{artanh}
\DeclareMathOperator{\arcoth}{arcoth}


\newcommand{\MMathSet}[1]{\mathbb{#1}}
\def\N{\MMathSet{N}}
\def\Z{\MMathSet{Z}}
\def\Q{\MMathSet{Q}}
\def\R{\MMathSet{R}}
\def\C{\MMathSet{C}}

\newcounter{MForLoopCounter}
\newcommand{\MForLoop}[2]{\setcounter{MForLoopCounter}{#1}\ifnum\value{MForLoopCounter}=0{}\else{{#2}\addtocounter{MForLoopCounter}{-1}\MForLoop{\value{MForLoopCounter}}{#2}}\fi}

\newcounter{MSiteCounter}
\newcounter{MFieldCounter} % Kombination section.subsection.site.field ist eindeutig in allen Modulen, field alleine nicht

\newcounter{MiniMarkerCounter}

\ifttm
\newenvironment{MMiniPageP}[1]{\begin{minipage}{#1\linewidth}\special{html:<!-- minimarker;;}\arabic{MiniMarkerCounter}\special{html:;;#1; //-->}}{\end{minipage}\addtocounter{MiniMarkerCounter}{1}}
\else
\newenvironment{MMiniPageP}[1]{\begin{minipage}{#1\linewidth}}{\end{minipage}\addtocounter{MiniMarkerCounter}{1}}
\fi

\newcounter{AlignCounter}

\newcommand{\MStartJustify}{\ifttm\special{html:<!-- startalign;;}\arabic{AlignCounter}\special{html:;;justify; //-->}\fi}
\newcommand{\MStopJustify}{\ifttm\special{html:<!-- stopalign;;}\arabic{AlignCounter}\special{html:; //-->}\fi\addtocounter{AlignCounter}{1}}

\newenvironment{MJTabular}[1]{
\MStartJustify
\begin{tabular}{#1}
}{
\end{tabular}
\MStopJustify
}

\newcommand{\MImageLeft}[2]{
\begin{center}
\begin{tabular}{lc}
\MStartJustify
\begin{MMiniPageP}{0.65}
#1
\end{MMiniPageP}
\MStopJustify
&
\begin{MMiniPageP}{0.3}
#2  
\end{MMiniPageP}
\end{tabular}
\end{center}
}

\newcommand{\MImageHalf}[2]{
\begin{center}
\begin{tabular}{lc}
\MStartJustify
\begin{MMiniPageP}{0.45}
#1
\end{MMiniPageP}
\MStopJustify
&
\begin{MMiniPageP}{0.45}
#2  
\end{MMiniPageP}
\end{tabular}
\end{center}
}

\newcommand{\MBigImageLeft}[2]{
\begin{center}
\begin{tabular}{lc}
\MStartJustify
\begin{MMiniPageP}{0.25}
#1
\end{MMiniPageP}
\MStopJustify
&
\begin{MMiniPageP}{0.7}
#2  
\end{MMiniPageP}
\end{tabular}
\end{center}
}

\ifttm
\def\No{\mathbb{N}_0}
\else
\def\No{\ensuremath{\N_0}}
\fi
\def\MT{\textrm{\tiny T}}
\newcommand{\MTranspose}[1]{{#1}^{\MT}}
\ifttm
\newcommand{\MRe}{\mathsf{Re}}
\newcommand{\MIm}{\mathsf{Im}}
\else
\DeclareMathOperator{\MRe}{Re}
\DeclareMathOperator{\MIm}{Im}
\fi

\newcommand{\Mid}{\mathrm{id}}
\newcommand{\MFeinheit}{\mathrm{feinh}}

\ifttm
\newcommand{\Msubstack}[1]{\begin{array}{c}{#1}\end{array}}
\else
\newcommand{\Msubstack}[1]{\substack{#1}}
\fi

% Typen von Fragefeldern:
% 1 = Alphanumerisch, case-sensitive-Vergleich
% 2 = Ja/Nein-Checkbox, Loesung ist 0 oder 1   (OPTION = Image-id fuer Rueckmeldung)
% 3 = Reelle Zahlen Geparset
% 4 = Funktionen Geparset (mit Stuetzstellen zur ueberpruefung)

% Dieser Befehl erstellt ein interaktives Aufgabenfeld. Parameter:
% - #1 Laenge in Zeichen
% - #2 Loesungstext (alphanumerisch, case sensitive)
% - #3 AufgabenID (alphanumerisch, case sensitive)
% - #4 Typ (Kennnummer)
% - #5 String fuer Optionen (ggf. mit Semikolon getrennte Einzelstrings)
% - #6 Anzahl Punkte
% - #7 uxid (kann z.B. Loesungsstring sein)
% ACHTUNG: Die langen Zeilen bitte so lassen, Zeilenumbrueche im tex werden in div's umgesetzt
\newcommand{\MQuestionID}[7]{
\ifttm
\special{html:<!-- mdeclareuxid;;}UX#7\special{html:;;}\arabic{section}\special{html:;;}#3\special{html:;; //-->}%
\special{html:<!-- mdeclarepoints;;}\arabic{section}\special{html:;;}#3\special{html:;;}#6\special{html:;;}\arabic{MTestSite}\special{html:;;}\arabic{chapter}%
\special{html:;; //--><!-- onloadstart //-->CreateQuestionObj("}#7\special{html:",}\arabic{MFieldCounter}\special{html:,"}#2%
\special{html:","}#3\special{html:",}#4\special{html:,"}#5\special{html:",}#6\special{html:,}\arabic{MTestSite}\special{html:,}\arabic{section}%
\special{html:);<!-- onloadstop //-->}%
\special{html:<input mfieldtype="}#4\special{html:" name="Name_}#3\special{html:" id="}#3\special{html:" type="text" size="}#1\special{html:" maxlength="}#1%
\special{html:" }\ifnum\value{MGroupActive}=0\special{html:onfocus="handlerFocus(}\arabic{MFieldCounter}%
\special{html:);" onblur="handlerBlur(}\arabic{MFieldCounter}\special{html:);" onkeyup="handlerChange(}\arabic{MFieldCounter}\special{html:,0);" onpaste="handlerChange(}\arabic{MFieldCounter}\special{html:,0);" oninput="handlerChange(}\arabic{MFieldCounter}\special{html:,0);" onpropertychange="handlerChange(}\arabic{MFieldCounter}\special{html:,0);"/>}%
\special{html:<img src="images/questionmark.gif" width="20" height="20" border="0" align="absmiddle" id="}QM#3\special{html:"/>}
\else%
\special{html:onblur="handlerBlur(}\arabic{MFieldCounter}%
\special{html:);" onfocus="handlerFocus(}\arabic{MFieldCounter}\special{html:);" onkeyup="handlerChange(}\arabic{MFieldCounter}\special{html:,1);" onpaste="handlerChange(}\arabic{MFieldCounter}\special{html:,1);" oninput="handlerChange(}\arabic{MFieldCounter}\special{html:,1);" onpropertychange="handlerChange(}\arabic{MFieldCounter}\special{html:,1);"/>}%
\special{html:<img src="images/questionmark.gif" width="20" height="20" border="0" align="absmiddle" id="}QM#3\special{html:"/>}\fi%
\else%
\ifnum\value{QBoxFlag}=1\fbox{$\phantom{\MForLoop{#1}{b}}$}\else$\phantom{\MForLoop{#1}{b}}$\fi%
\fi%
}

% ACHTUNG: Die langen Zeilen bitte so lassen, Zeilenumbrueche im tex werden in div's umgesetzt
% QuestionCheckbox macht ausserhalb einer QuestionGroup keinen Sinn!
% #1 = solution (1 oder 0), ggf. mit ::smc abgetrennt auszuschliessende single-choice-boxen (UXIDs durch , getrennt), #2 = id, #3 = points, #4 = uxid
\newcommand{\MQuestionCheckbox}[4]{
\ifttm
\special{html:<!-- mdeclareuxid;;}UX#4\special{html:;;}\arabic{section}\special{html:;;}#2\special{html:;; //-->}%
\ifnum\value{MGroupActive}=0\MDebugMessage{ERROR: Checkbox Nr. \arabic{MFieldCounter}\ ist nicht in einer Kontrollgruppe, es wird niemals eine Loesung angezeigt!}\fi
\special{html: %
<!-- mdeclarepoints;;}\arabic{section}\special{html:;;}#2\special{html:;;}#3\special{html:;;}\arabic{MTestSite}\special{html:;;}\arabic{chapter}%
\special{html:;; //--><!-- onloadstart //-->CreateQuestionObj("}#4\special{html:",}\arabic{MFieldCounter}\special{html:,"}#1\special{html:","}#2\special{html:",2,"IMG}#2%
\special{html:",}#3\special{html:,}\arabic{MTestSite}\special{html:,}\arabic{section}\special{html:);<!-- onloadstop //-->}%
\special{html:<input mfieldtype="2" type="checkbox" name="Name_}#2\special{html:" id="}#2\special{html:" onchange="handlerChange(}\arabic{MFieldCounter}\special{html:,1);"/><img src="images/questionmark.gif" name="}Name_IMG#2%
\special{html:" width="20" height="20" border="0" align="absmiddle" id="}IMG#2\special{html:"/> }%
\else%
\ifnum\value{QBoxFlag}=1\fbox{$\phantom{X}$}\else$\phantom{X}$\fi%
\fi%
}

\def\MGenerateID{QFELD_\arabic{section}.\arabic{subsection}.\arabic{MSiteCounter}.QF\arabic{MFieldCounter}}

% #1 = 0/1 ggf. mit ::smc abgetrennt auszuschliessende single-choice-boxen (UXIDs durch , getrennt ohne UX), #2 = uxid ohne UX
\newcommand{\MCheckbox}[2]{
\MQuestionCheckbox{#1}{\MGenerateID}{\MStdPoints}{#2}
\addtocounter{MFieldCounter}{1}
}

% Erster Parameter: Zeichenlaenge der Eingabebox, zweiter Parameter: Loesungstext
\newcommand{\MQuestion}[2]{
\MQuestionID{#1}{#2}{\MGenerateID}{1}{0}{\MStdPoints}{#2}
\addtocounter{MFieldCounter}{1}
}

% Erster Parameter: Zeichenlaenge der Eingabebox, zweiter Parameter: Loesungstext
\newcommand{\MLQuestion}[3]{
\MQuestionID{#1}{#2}{\MGenerateID}{1}{0}{\MStdPoints}{#3}
\addtocounter{MFieldCounter}{1}
}

% Parameter: Laenge des Feldes, Loesung (wird auch geparsed), Stellen Genauigkeit hinter dem Komma, weitere Stellen werden mathematisch gerundet vor Vergleich
\newcommand{\MParsedQuestion}[3]{
\MQuestionID{#1}{#2}{\MGenerateID}{3}{#3}{\MStdPoints}{#2}
\addtocounter{MFieldCounter}{1}
}

% Parameter: Laenge des Feldes, Loesung (wird auch geparsed), Stellen Genauigkeit hinter dem Komma, weitere Stellen werden mathematisch gerundet vor Vergleich
\newcommand{\MLParsedQuestion}[4]{
\MQuestionID{#1}{#2}{\MGenerateID}{3}{#3}{\MStdPoints}{#4}
\addtocounter{MFieldCounter}{1}
}

% Parameter: Laenge des Feldes, Loesungsfunktion, Anzahl Stuetzstellen, Funktionsvariablen durch Kommata getrennt (nicht case-sensitive), Anzahl Nachkommastellen im Vergleich
\newcommand{\MFunctionQuestion}[5]{
\MQuestionID{#1}{#2}{\MGenerateID}{4}{#3;#4;#5;0}{\MStdPoints}{#2}
\addtocounter{MFieldCounter}{1}
}

% Parameter: Laenge des Feldes, Loesungsfunktion, Anzahl Stuetzstellen, Funktionsvariablen durch Kommata getrennt (nicht case-sensitive), Anzahl Nachkommastellen im Vergleich, UXID
\newcommand{\MLFunctionQuestion}[6]{
\MQuestionID{#1}{#2}{\MGenerateID}{4}{#3;#4;#5;0}{\MStdPoints}{#6}
\addtocounter{MFieldCounter}{1}
}

% Parameter: Laenge des Feldes, Loesungsintervall, Genauigkeit der Zahlenwertpruefung
\newcommand{\MIntervalQuestion}[3]{
\MQuestionID{#1}{#2}{\MGenerateID}{6}{#3}{\MStdPoints}{#2}
\addtocounter{MFieldCounter}{1}
}

% Parameter: Laenge des Feldes, Loesungsintervall, Genauigkeit der Zahlenwertpruefung, UXID
\newcommand{\MLIntervalQuestion}[4]{
\MQuestionID{#1}{#2}{\MGenerateID}{6}{#3}{\MStdPoints}{#4}
\addtocounter{MFieldCounter}{1}
}

% Parameter: Laenge des Feldes, Loesungsfunktion, Anzahl Stuetzstellen, Funktionsvariable (nicht case-sensitive), Anzahl Nachkommastellen im Vergleich, Vereinfachungsbedingung
% Vereinfachungsbedingung ist eine der Folgenden:
% 0 = Keine Vereinfachungsbedingung
% 1 = Keine Klammern (runde oder eckige) mehr im vereinfachten Ausdruck
% 2 = Faktordarstellung (Term hat Produkte als letzte Operation, Summen als vorgeschaltete Operation)
% 3 = Summendarstellung (Term hat Summen als letzte Operation, Produkte als vorgeschaltete Operation)
% Flag 512: Besondere Stuetzstellen (nur >1 und nur schwach rational), sonst symmetrisch um Nullpunkt und ganze Zahlen inkl. Null werden getroffen
\newcommand{\MSimplifyQuestion}[6]{
\MQuestionID{#1}{#2}{\MGenerateID}{4}{#3;#4;#5;#6}{\MStdPoints}{#2}
\addtocounter{MFieldCounter}{1}
}

\newcommand{\MLSimplifyQuestion}[7]{
\MQuestionID{#1}{#2}{\MGenerateID}{4}{#3;#4;#5;#6}{\MStdPoints}{#7}
\addtocounter{MFieldCounter}{1}
}

% Parameter: Laenge des Feldes, Loesung (optionaler Ausdruck), Anzahl Stuetzstellen, Funktionsvariable (nicht case-sensitive), Anzahl Nachkommastellen im Vergleich, Spezialtyp (string-id)
\newcommand{\MLSpecialQuestion}[7]{
\MQuestionID{#1}{#2}{\MGenerateID}{7}{#3;#4;#5;#6}{\MStdPoints}{#7}
\addtocounter{MFieldCounter}{1}
}

\newcounter{MGroupStart}
\newcounter{MGroupEnd}
\newcounter{MGroupActive}

\newenvironment{MQuestionGroup}{
\setcounter{MGroupStart}{\value{MFieldCounter}}
\setcounter{MGroupActive}{1}
}{
\setcounter{MGroupActive}{0}
\setcounter{MGroupEnd}{\value{MFieldCounter}}
\addtocounter{MGroupEnd}{-1}
}

\newcommand{\MGroupButton}[1]{
\ifttm
\special{html:<button name="Name_Group}\arabic{MGroupStart}\special{html:to}\arabic{MGroupEnd}\special{html:" id="Group}\arabic{MGroupStart}\special{html:to}\arabic{MGroupEnd}\special{html:" %
type="button" onclick="group_button(}\arabic{MGroupStart}\special{html:,}\arabic{MGroupEnd}\special{html:);">}#1\special{html:</button>}
\else
\phantom{#1}
\fi
}

%----------------- Makros fuer die modularisierte Darstellung ------------------------------------

\def\MyText#1{#1}

% is used internally by the conversion package, should not be used by original tex documents
\def\MOrgLabel#1{\relax}

\ifttm

% Ein MLabel wird im html codiert durch das tag <!-- mmlabel;;Labelbezeichner;;SubjectArea;;chapter;;section;;subsection;;Index;;Objekttyp; //-->
\def\MLabel#1{%
\ifnum\value{MLastType}=8%
\ifnum\value{MCaptionOn}=0%
\MDebugMessage{ERROR: Grafik \arabic{MGraphicsCounter} hat separates label: #1 (Grafiklabels sollten nur in der Caption stehen)}%
\fi
\fi
\ifnum\value{MLastType}=12%
\ifnum\value{MCaptionOn}=0%
\MDebugMessage{ERROR: Video \arabic{MVideoCounter} hat separates label: #1 (Videolabels sollten nur in der Caption stehen}%
\fi
\fi
\ifnum\value{MLastType}=10\setcounter{MLastIndex}{\value{equation}}\fi
\label{#1}\begin{html}<!-- mmlabel;;#1;;\end{html}\arabic{MSubjectArea}\special{html:;;}\arabic{chapter}\special{html:;;}\arabic{section}\special{html:;;}\arabic{subsection}\special{html:;;}\arabic{MLastIndex}\special{html:;;}\arabic{MLastType}\special{html:; //-->}}%

\else

% Sonderbehandlung im PDF fuer Abbildungen in separater aux-Datei, da MGraphics die figure-Umgebung nicht verwendet
\def\MLabel#1{%
\ifnum\value{MLastType}=8%
\ifnum\value{MCaptionOn}=0%
\MDebugMessage{ERROR: Grafik \arabic{MGraphicsCounter} hat separates label: #1 (Grafiklabels sollten nur in der Caption stehen}%
\fi
\fi
\ifnum\value{MLastType}=12%
\ifnum\value{MCaptionOn}=0%
\MDebugMessage{ERROR: Video \arabic{MVideoCounter} hat separates label: #1 (Videolabels sollten nur in der Caption stehen}%
\fi
\fi
\label{#1}%
}%

\fi

% Gibt Begriff des referenzierten Objekts mit aus, aber nur im HTML, daher nur in Ausnahmefaellen (z.B. Copyrightliste) sinnvoll
\def\MCRef#1{\ifttm\special{html:<!-- mmref;;}#1\special{html:;;1; //-->}\else\vref{#1}\fi}


\def\MRef#1{\ifttm\special{html:<!-- mmref;;}#1\special{html:;;0; //-->}\else\vref{#1}\fi}
\def\MERef#1{\ifttm\special{html:<!-- mmref;;}#1\special{html:;;0; //-->}\else\eqref{#1}\fi}
\def\MNRef#1{\ifttm\special{html:<!-- mmref;;}#1\special{html:;;0; //-->}\else\ref{#1}\fi}
\def\MSRef#1#2{\ifttm\special{html:<!-- msref;;}#1\special{html:;;}#2\special{html:; //-->}\else \if#2\empty \ref{#1} \else \hyperref[#1]{#2}\fi\fi} 

\def\MRefRange#1#2{\ifttm\MRef{#1} bis 
\MRef{#2}\else\vrefrange[\unskip]{#1}{#2}\fi}

\def\MRefTwo#1#2{\ifttm\MRef{#1} und \MRef{#2}\else%
\let\vRefTLRsav=\reftextlabelrange\let\vRefTPRsav=\reftextpagerange%
\def\reftextlabelrange##1##2{\ref{##1} und~\ref{##2}}%
\def\reftextpagerange##1##2{auf den Seiten~\pageref{#1} und~\pageref{#2}}%
\vrefrange[\unskip]{#1}{#2}%
\let\reftextlabelrange=\vRefTLRsav\let\reftextpagerange=\vRefTPRsav\fi}

% MSectionChapter definiert falls notwendig das Kapitel vor der section. Das ist notwendig, wenn nur ein Einzelmodul uebersetzt wird.
% MChaptersGiven ist ein Counter, der von mconvert.pl vordefiniert wird.
\ifttm
\newcommand{\MSectionChapter}{\ifnum\value{MChaptersGiven}=0{\Dchapter{Modul}}\else{}\fi}
\else
\newcommand{\MSectionChapter}{\ifnum\value{chapter}=0{\Dchapter{Modul}}\else{}\fi}
\fi


\def\MChapter#1{\ifnum\value{MSSEnd}>0{\MSubsectionEndMacros}\addtocounter{MSSEnd}{-1}\fi\Dchapter{#1}}
\def\MSubject#1{\MChapter{#1}} % Schluesselwort HELPSECTION ist reserviert fuer Hilfesektion

\newcommand{\MSectionID}{UNKNOWNID}

\ifttm
\newcommand{\MSetSectionID}[1]{\renewcommand{\MSectionID}{#1}}
\else
\newcommand{\MSetSectionID}[1]{\renewcommand{\MSectionID}{#1}\tikzsetexternalprefix{#1}}
\fi


\newcommand{\MSection}[1]{\MSetSectionID{MODULID}\ifnum\value{MSSEnd}>0{\MSubsectionEndMacros}\addtocounter{MSSEnd}{-1}\fi\MSectionChapter\Dsection{#1}\MSectionStartMacros{#1}\setcounter{MLastIndex}{-1}\setcounter{MLastType}{1}} % Sections werden ueber das section-Feld im mmlabel-Tag identifiziert, nicht ueber das Indexfeld

\def\MSubsection#1{\ifnum\value{MSSEnd}>0{\MSubsectionEndMacros}\addtocounter{MSSEnd}{-1}\fi\ifttm\else\clearpage\fi\Dsubsection{#1}\MSubsectionStartMacros\setcounter{MLastIndex}{-1}\setcounter{MLastType}{2}\addtocounter{MSSEnd}{1}}% Subsections werden ueber das subsection-Feld im mmlabel-Tag identifiziert, nicht ueber das Indexfeld
\def\MSubsectionx#1{\Dsubsectionx{#1}} % Nur zur Verwendung in MSectionStart gedacht
\def\MSubsubsection#1{\Dsubsubsection{#1}\setcounter{MLastIndex}{\value{subsubsection}}\setcounter{MLastType}{3}\ifttm\special{html:<!-- sectioninfo;;}\arabic{section}\special{html:;;}\arabic{subsection}\special{html:;;}\arabic{subsubsection}\special{html:;;1;;}\arabic{MTestSite}\special{html:; //-->}\fi}
\def\MSubsubsectionx#1{\Dsubsubsectionx{#1}\ifttm\special{html:<!-- sectioninfo;;}\arabic{section}\special{html:;;}\arabic{subsection}\special{html:;;}\arabic{subsubsection}\special{html:;;0;;}\arabic{MTestSite}\special{html:; //-->}\else\addcontentsline{toc}{subsection}{#1}\fi}

\ifttm
\def\MSubsubsubsectionx#1{\ \newline\textbf{#1}\special{html:<br />}}
\else
\def\MSubsubsubsectionx#1{\ \newline
\textbf{#1}\ \\
}
\fi


% Dieses Skript wird zu Beginn jedes Modulabschnitts (=Webseite) ausgefuehrt und initialisiert den Aufgabenfeldzaehler
\newcommand{\MPageScripts}{
\setcounter{MFieldCounter}{1}
\addtocounter{MSiteCounter}{1}
\setcounter{MHintCounter}{1}
\setcounter{MCodeEditCounter}{1}
\setcounter{MGroupActive}{0}
\DoQBoxes
% Feldvariablen werden im HTML-Header in conv.pl eingestellt
}

% Dieses Skript wird zum Ende jedes Modulabschnitts (=Webseite) ausgefuehrt
\ifttm
\newcommand{\MEndScripts}{\special{html:<br /><!-- mfeedbackbutton;Seite;}\arabic{MTestSite}\special{html:;}\MGenerateSiteNumber\special{html:; //-->}
}
\else
\newcommand{\MEndScripts}{\relax}
\fi


\newcounter{QBoxFlag}
\newcommand{\DoQBoxes}{\setcounter{QBoxFlag}{1}}
\newcommand{\NoQBoxes}{\setcounter{QBoxFlag}{0}}

\newcounter{MXCTest}
\newcounter{MXCounter}
\newcounter{MSCounter}



\ifttm

% Struktur des sectioninfo-Tags: <!-- sectioninfo;;section;;subsection;;subsubsection;;nr_ausgeben;;testpage; //-->

%Fuegt eine zusaetzliche html-Seite an hinter ALLEN bisherigen und zukuenftigen content-Seiten ausserhalb der vor-zurueck-Schleife (d.h. nur durch Button oder MIntLink erreichbar!)
% #1 = Titel des Modulabschnitts, #2 = Kurztitel fuer die Buttons, #3 = Buttonkennung (STD = default nehmen, NONE = Ohne Button in der Navigation)
\newenvironment{MSContent}[3]{\special{html:<div class="xcontent}\arabic{MSCounter}\special{html:"><!-- scontent;-;}\arabic{MSCounter};-;#1;-;#2;-;#3\special{html: //-->}\MPageScripts\MSubsubsectionx{#1}}{\MEndScripts\special{html:<!-- endscontent;;}\arabic{MSCounter}\special{html: //--></div>}\addtocounter{MSCounter}{1}}

% Fuegt eine zusaetzliche html-Seite ein hinter den bereits vorhandenen content-Seiten (oder als erste Seite) innerhalb der vor-zurueck-Schleife der Navigation
% #1 = Titel des Modulabschnitts, #2 = Kurztitel fuer die Buttons, #3 = Buttonkennung (STD = Defaultbutton, NONE = Ohne Button in der Navigation)
\newenvironment{MXContent}[3]{\special{html:<div class="xcontent}\arabic{MXCounter}\special{html:"><!-- xcontent;-;}\arabic{MXCounter};-;#1;-;#2;-;#3\special{html: //-->}\MPageScripts\MSubsubsection{#1}}{\MEndScripts\special{html:<!-- endxcontent;;}\arabic{MXCounter}\special{html: //--></div>}\addtocounter{MXCounter}{1}}

% Fuegt eine zusaetzliche html-Seite ein die keine subsubsection-Nummer bekommt, nur zur internen Verwendung in mintmod.tex gedacht!
% #1 = Titel des Modulabschnitts, #2 = Kurztitel fuer die Buttons, #3 = Buttonkennung (STD = Defaultbutton, NONE = Ohne Button in der Navigation)
% \newenvironment{MUContent}[3]{\special{html:<div class="xcontent}\arabic{MXCounter}\special{html:"><!-- xcontent;-;}\arabic{MXCounter};-;#1;-;#2;-;#3\special{html: //-->}\MPageScripts\MSubsubsectionx{#1}}{\MEndScripts\special{html:<!-- endxcontent;;}\arabic{MXCounter}\special{html: //--></div>}\addtocounter{MXCounter}{1}}

\newcommand{\MDeclareSiteUXID}[1]{\special{html:<!-- mdeclaresiteuxid;;}#1\special{html:;;}\arabic{chapter}\special{html:;;}\arabic{section}\special{html:;; //-->}}

\else

%\newcommand{\MSubsubsection}[1]{\refstepcounter{subsubsection} \addcontentsline{toc}{subsubsection}{\thesubsubsection. #1}}


% Fuegt eine zusaetzliche html-Seite an hinter den bereits vorhandenen content-Seiten
% #1 = Titel des Modulabschnitts, #2 = Kurztitel fuer die Buttons, #3 = Iconkennung (im PDF wirkungslos)
%\newenvironment{MUContent}[3]{\ifnum\value{MXCTest}>0{\MDebugMessage{ERROR: Geschachtelter SContent}}\fi\MPageScripts\MSubsubsectionx{#1}\addtocounter{MXCTest}{1}}{\addtocounter{MXCounter}{1}\addtocounter{MXCTest}{-1}}
\newenvironment{MXContent}[3]{\ifnum\value{MXCTest}>0{\MDebugMessage{ERROR: Geschachtelter SContent}}\fi\MPageScripts\MSubsubsection{#1}\addtocounter{MXCTest}{1}}{\addtocounter{MXCounter}{1}\addtocounter{MXCTest}{-1}}
\newenvironment{MSContent}[3]{\ifnum\value{MXCTest}>0{\MDebugMessage{ERROR: Geschachtelter XContent}}\fi\MPageScripts\MSubsubsectionx{#1}\addtocounter{MXCTest}{1}}{\addtocounter{MSCounter}{1}\addtocounter{MXCTest}{-1}}

\newcommand{\MDeclareSiteUXID}[1]{\relax}

\fi 

% GHEADER und GFOOTER werden von split.pm gefunden, aber nur, wenn nicht HELPSITE oder TESTSITE
\ifttm
\newenvironment{MSectionStart}{\special{html:<div class="xcontent0">}\MSubsubsectionx{Modul\"ubersicht}}{\setcounter{MSSEnd}{0}\special{html:</div>}}
% Darf nicht als XContent nummeriert werden, darf nicht als XContent gelabelt werden, wird aber in eine xcontent-div gesetzt fuer Python-parsing
\else
\newenvironment{MSectionStart}{\MSubsectionx{Modul\"ubersicht}}{\setcounter{MSSEnd}{0}}
\fi

\newenvironment{MIntro}{\begin{MXContent}{Einf\"uhrung}{Einf\"uhrung}{genetisch}}{\end{MXContent}}
\newenvironment{MContent}{\begin{MXContent}{Inhalt}{Inhalt}{beweis}}{\end{MXContent}}
\newenvironment{MExercises}{\ifttm\else\clearpage\fi\begin{MXContent}{Aufgaben}{Aufgaben}{aufgb}\special{html:<!-- declareexcsymb //-->}}{\end{MXContent}}

% #1 = Lesbare Testbezeichnung
\newenvironment{MTest}[1]{%
\renewcommand{\MTestName}{#1}
\ifttm\else\clearpage\fi%
\addtocounter{MTestSite}{1}%
\begin{MXContent}{#1}{#1}{STD} % {aufgb}%
\special{html:<!-- declaretestsymb //-->}
\begin{MQuestionGroup}%
\MInTestHeader
}%
{%
\end{MQuestionGroup}%
\ \\ \ \\%
\MInTestFooter
\end{MXContent}\addtocounter{MTestSite}{-1}%
}

\newenvironment{MExtra}{\ifttm\else\clearpage\fi\begin{MXContent}{Zus\"atzliche Inhalte}{Zusatz}{weiterfhrg}}{\end{MXContent}}

\makeindex

\ifttm
\def\MPrintIndex{
\ifnum\value{MSSEnd}>0{\MSubsectionEndMacros}\addtocounter{MSSEnd}{-1}\fi
\renewcommand{\indexname}{Stichwortverzeichnis}
\special{html:<p><!-- printindex //--></p>}
}
\else
\def\MPrintIndex{
\ifnum\value{MSSEnd}>0{\MSubsectionEndMacros}\addtocounter{MSSEnd}{-1}\fi
\renewcommand{\indexname}{Stichwortverzeichnis}
\addcontentsline{toc}{section}{Stichwortverzeichnis}
\printindex
}
\fi


% Konstanten fuer die Modulfaecher

\def\MINTMathematics{1}
\def\MINTInformatics{2}
\def\MINTChemistry{3}
\def\MINTPhysics{4}
\def\MINTEngineering{5}

\newcounter{MSubjectArea}
\newcounter{MInfoNumbers} % Gibt an, ob die Infoboxen nummeriert werden sollen
\newcounter{MSepNumbers} % Gibt an, ob Beispiele und Experimente separat nummeriert werden sollen
\newcommand{\MSetSubject}[1]{
 % ttm kapiert setcounter mit Parametern nicht, also per if abragen und einsetzen
\ifnum#1=1\setcounter{MSubjectArea}{1}\setcounter{MInfoNumbers}{1}\setcounter{MSepNumbers}{0}\fi
\ifnum#1=2\setcounter{MSubjectArea}{2}\setcounter{MInfoNumbers}{1}\setcounter{MSepNumbers}{0}\fi
\ifnum#1=3\setcounter{MSubjectArea}{3}\setcounter{MInfoNumbers}{0}\setcounter{MSepNumbers}{1}\fi
\ifnum#1=4\setcounter{MSubjectArea}{4}\setcounter{MInfoNumbers}{0}\setcounter{MSepNumbers}{0}\fi
\ifnum#1=5\setcounter{MSubjectArea}{5}\setcounter{MInfoNumbers}{1}\setcounter{MSepNumbers}{0}\fi
% Separate Nummerntechnik fuer unsere Chemiker: alles dreistellig
\ifnum#1=3
  \ifttm
  \renewcommand{\theequation}{\arabic{section}.\arabic{subsection}.\arabic{equation}}
  \renewcommand{\thetable}{\arabic{section}.\arabic{subsection}.\arabic{table}} 
  \renewcommand{\thefigure}{\arabic{section}.\arabic{subsection}.\arabic{figure}} 
  \else
  \renewcommand{\theequation}{\arabic{chapter}.\arabic{section}.\arabic{equation}}
  \renewcommand{\thetable}{\arabic{chapter}.\arabic{section}.\arabic{table}}
  \renewcommand{\thefigure}{\arabic{chapter}.\arabic{section}.\arabic{figure}}
  \fi
\else
  \ifttm
  \renewcommand{\theequation}{\arabic{section}.\arabic{subsection}.\arabic{equation}}
  \renewcommand{\thetable}{\arabic{table}}
  \renewcommand{\thefigure}{\arabic{figure}}
  \else
  \renewcommand{\theequation}{\arabic{chapter}.\arabic{section}.\arabic{equation}}
  \renewcommand{\thetable}{\arabic{table}}
  \renewcommand{\thefigure}{\arabic{figure}}
  \fi
\fi
}

% Fuer tikz Autogenerierung
\newcounter{MTIKZAutofilenumber}

% Spezielle Counter fuer die Bentz-Module
\newcounter{mycounter}
\newcounter{chemapplet}
\newcounter{physapplet}

\newcounter{MSSEnd} % Ist 1 falls ein MSubsection aktiv ist, der einen MSubsectionEndMacro-Aufruf verursacht
\newcounter{MFileNumber}
\def\MLastFile{\special{html:[[!-- mfileref;;}\arabic{MFileNumber}\special{html:; //--]]}}

% Vollstaendiger Pfad ist \MMaterial / \MLastFilePath / \MLastFileName    ==   \MMaterial / \MLastFile

% Wird nur bei kompletter Baum-Erstellung ausgefuehrt!
% #1 = Lesbare Modulbezeichnung
\newcommand{\MSectionStartMacros}[1]{
\setcounter{MTestSite}{0}
\setcounter{MCaptionOn}{0}
\setcounter{MLastTypeEq}{0}
\setcounter{MSSEnd}{0}
\setcounter{MFileNumber}{0} % Preinkrekement-Counter
\setcounter{MTIKZAutofilenumber}{0}
\setcounter{mycounter}{1}
\setcounter{physapplet}{1}
\setcounter{chemapplet}{0}
\ifttm
\special{html:<!-- mdeclaresection;;}\arabic{chapter}\special{html:;;}\arabic{section}\special{html:;;}#1\special{html:;; //-->}%
\else
\setcounter{thmc}{0}
\setcounter{exmpc}{0}
\setcounter{verc}{0}
\setcounter{infoc}{0}
\fi
\setcounter{MiniMarkerCounter}{1}
\setcounter{AlignCounter}{1}
\setcounter{MXCTest}{0}
\setcounter{MCodeCounter}{0}
\setcounter{MEntryCounter}{0}
}

% Wird immer ausgefuehrt
\newcommand{\MSubsectionStartMacros}{
\ifttm\else\MPageHeaderDef\fi
\MWatermarkSettings
\setcounter{MXCounter}{0}
\setcounter{MSCounter}{0}
\setcounter{MSiteCounter}{1}
\setcounter{MExerciseCollectionCounter}{0}
% Zaehler fuer das Labelsystem zuruecksetzen (prefix-Zaehler)
\setcounter{MInfoCounter}{0}
\setcounter{MExerciseCounter}{0}
\setcounter{MExampleCounter}{0}
\setcounter{MExperimentCounter}{0}
\setcounter{MGraphicsCounter}{0}
\setcounter{MTableCounter}{0}
\setcounter{MTheoremCounter}{0}
\setcounter{MObjectCounter}{0}
\setcounter{MEquationCounter}{0}
\setcounter{MVideoCounter}{0}
\setcounter{equation}{0}
\setcounter{figure}{0}
}

\newcommand{\MSubsectionEndMacros}{
% Bei Chemiemodulen das PSE einhaengen, es soll als SContent am Ende erscheinen
\special{html:<!-- subsectionend //-->}
\ifnum\value{MSubjectArea}=3{\MIncludePSE}\fi
}


\ifttm
%\newcommand{\MEmbed}[1]{\MRegisterFile{#1}\begin{html}<embed src="\end{html}\MMaterial/\MLastFile\begin{html}" width="192" height="189"></embed>\end{html}}
\newcommand{\MEmbed}[1]{\MRegisterFile{#1}\begin{html}<embed src="\end{html}\MMaterial/\MLastFile\begin{html}"></embed>\end{html}}
\fi

%----------------- Makros fuer die Textdarstellung -----------------------------------------------

\ifttm
% MUGraphics bindet eine Grafik ein:
% Parameter 1: Dateiname der Grafik, relativ zur Position des Modul-Tex-Dokuments
% Parameter 2: Skalierungsoptionen fuer PDF (fuer includegraphics)
% Parameter 3: Titel fuer die Grafik, wird unter die Grafik mit der Grafiknummer gesetzt und kann MLabel bzw. MCopyrightLabel enthalten
% Parameter 4: Skalierungsoptionen fuer HTML (css-styles)

% ERSATZ: <img alt="My Image" src="data:image/png;base64,iVBORwA<MoreBase64SringHere>" />


\newcommand{\MUGraphics}[4]{\MRegisterFile{#1}\begin{html}
<div class="imagecenter">
<center>
<div>
<img src="\end{html}\MMaterial/\MLastFile\begin{html}" style="#4" alt="\end{html}\MMaterial/\MLastFile\begin{html}"/>
</div>
<div class="bildtext">
\end{html}
\addtocounter{MGraphicsCounter}{1}
\setcounter{MLastIndex}{\value{MGraphicsCounter}}
\setcounter{MLastType}{8}
\addtocounter{MCaptionOn}{1}
\ifnum\value{MSepNumbers}=0
\textbf{Abbildung \arabic{MGraphicsCounter}:} #3
\else
\textbf{Abbildung \arabic{section}.\arabic{subsection}.\arabic{MGraphicsCounter}:} #3
\fi
\addtocounter{MCaptionOn}{-1}
\begin{html}
</div>
</center>
</div>
<br />
\end{html}%
\special{html:<!-- mfeedbackbutton;Abbildung;}\arabic{MGraphicsCounter}\special{html:;}\arabic{section}.\arabic{subsection}.\arabic{MGraphicsCounter}\special{html:; //-->}%
}

% MVideo bindet ein Video als Einzeldatei ein:
% Parameter 1: Dateiname des Videos, relativ zur Position des Modul-Tex-Dokuments, ohne die Endung ".mp4"
% Parameter 2: Titel fuer das Video (kann MLabel oder MCopyrightLabel enthalten), wird unter das Video mit der Videonummer gesetzt
\newcommand{\MVideo}[2]{\MRegisterFile{#1.mp4}\begin{html}
<div class="imagecenter">
<center>
<div>
<video width="95\%" controls="controls"><source src="\end{html}\MMaterial/#1.mp4\begin{html}" type="video/mp4">Ihr Browser kann keine MP4-Videos abspielen!</video>
</div>
<div class="bildtext">
\end{html}
\addtocounter{MVideoCounter}{1}
\setcounter{MLastIndex}{\value{MVideoCounter}}
\setcounter{MLastType}{12}
\addtocounter{MCaptionOn}{1}
\ifnum\value{MSepNumbers}=0
\textbf{Video \arabic{MVideoCounter}:} #2
\else
\textbf{Video \arabic{section}.\arabic{subsection}.\arabic{MVideoCounter}:} #2
\fi
\addtocounter{MCaptionOn}{-1}
\begin{html}
</div>
</center>
</div>
<br />
\end{html}}

\newcommand{\MDVideo}[2]{\MRegisterFile{#1.mp4}\MRegisterFile{#1.ogv}\begin{html}
<div class="imagecenter">
<center>
<div>
<video width="70\%" controls><source src="\end{html}\MMaterial/#1.mp4\begin{html}" type="video/mp4"><source src="\end{html}\MMaterial/#1.ogv\begin{html}" type="video/ogg">Ihr Browser kann keine MP4-Videos abspielen!</video>
</div>
<br />
#2
</center>
</div>
<br />
\end{html}
}

\newcommand{\MGraphics}[3]{\MUGraphics{#1}{#2}{#3}{}}

\else

\newcommand{\MVideo}[2]{%
% Kein Video im PDF darstellbar, trotzdem so tun als ob da eines waere
\begin{center}
(Video nicht darstellbar)
\end{center}
\addtocounter{MVideoCounter}{1}
\setcounter{MLastIndex}{\value{MVideoCounter}}
\setcounter{MLastType}{12}
\addtocounter{MCaptionOn}{1}
\ifnum\value{MSepNumbers}=0
\textbf{Video \arabic{MVideoCounter}:} #2
\else
\textbf{Video \arabic{section}.\arabic{subsection}.\arabic{MVideoCounter}:} #2
\fi
\addtocounter{MCaptionOn}{-1}
}


% MGraphics bindet eine Grafik ein:
% Parameter 1: Dateiname der Grafik, relativ zur Position des Modul-Tex-Dokuments
% Parameter 2: Skalierungsoptionen fuer PDF (fuer includegraphics)
% Parameter 3: Titel fuer die Grafik, wird unter die Grafik mit der Grafiknummer gesetzt
\newcommand{\MGraphics}[3]{%
\MRegisterFile{#1}%
\ %
\begin{figure}[H]%
\centering{%
\includegraphics[#2]{\MDPrefix/#1}%
\addtocounter{MCaptionOn}{1}%
\caption{#3}%
\addtocounter{MCaptionOn}{-1}%
}%
\end{figure}%
\addtocounter{MGraphicsCounter}{1}\setcounter{MLastIndex}{\value{MGraphicsCounter}}\setcounter{MLastType}{8}\ %
%\ \\Abbildung \ifnum\value{MSepNumbers}=0\else\arabic{chapter}.\arabic{section}.\fi\arabic{MGraphicsCounter}: #3%
}

\newcommand{\MUGraphics}[4]{\MGraphics{#1}{#2}{#3}}


\fi

\newcounter{MCaptionOn} % = 1 falls eine Grafikcaption aktiv ist, = 0 sonst


% MGraphicsSolo bindet eine Grafik pur ein ohne Titel
% Parameter 1: Dateiname der Grafik, relativ zur Position des Modul-Tex-Dokuments
% Parameter 2: Skalierungsoptionen (wirken nur im PDF)
\newcommand{\MGraphicsSolo}[2]{\MUGraphicsSolo{#1}{#2}{}}

% MUGraphicsSolo bindet eine Grafik pur ein ohne Titel, aber mit HTML-Skalierung
% Parameter 1: Dateiname der Grafik, relativ zur Position des Modul-Tex-Dokuments
% Parameter 2: Skalierungsoptionen (wirken nur im PDF)
% Parameter 3: Skalierungsoptionen (wirken nur im HTML), als style-format: "width=???, height=???"
\ifttm
\newcommand{\MUGraphicsSolo}[3]{\MRegisterFile{#1}\begin{html}
<img src="\end{html}\MMaterial/\MLastFile\begin{html}" style="\end{html}#3\begin{html}" alt="\end{html}\MMaterial/\MLastFile\begin{html}"/>
\end{html}%
\special{html:<!-- mfeedbackbutton;Abbildung;}#1\special{html:;}\MMaterial/\MLastFile\special{html:; //-->}%
}
\else
\newcommand{\MUGraphicsSolo}[3]{\MRegisterFile{#1}\includegraphics[#2]{\MDPrefix/#1}}
\fi

% Externer Link mit URL
% Erster Parameter: Vollstaendige(!) URL des Links
% Zweiter Parameter: Text fuer den Link
\newcommand{\MExtLink}[2]{\ifttm\special{html:<a target="_new" href="}#1\special{html:">}#2\special{html:</a>}\else\href{#1}{#2}\fi} % ohne MINTERLINK!


% Interner Link, die verlinkte Datei muss im gleichen Verzeichnis liegen wie die Modul-Texdatei
% Erster Parameter: Dateiname
% Zweiter Parameter: Text fuer den Link
\newcommand{\MIntLink}[2]{\ifttm\MRegisterFile{#1}\special{html:<a class="MINTERLINK" target="_new" href="}\MMaterial/\MLastFile\special{html:">}#2\special{html:</a>}\else{\href{#1}{#2}}\fi}


\ifttm
\def\MMaterial{:localmaterial:}
\else
\def\MMaterial{\MDPrefix}
\fi

\ifttm
\def\MNoFile#1{:directmaterial:#1}
\else
\def\MNoFile#1{#1}
\fi

\newcommand{\MChem}[1]{$\mathrm{#1}$}

\newcommand{\MApplet}[3]{
% Bindet ein Java-Applet ein, die Parameter sind:
% (wird nur im HTML, aber nicht im PDF erstellt)
% #1 Dateiname des Applets (muss mit ".class" enden)
% #2 = Breite in Pixeln
% #3 = Hoehe in Pixeln
\ifttm
\MRegisterFile{#1}
\begin{html}
<applet code="\end{html}\MMaterial/\MLastFile\begin{html}" width="#2" height="#3" alt="[Java-Applet kann nicht gestartet werden]"></applet>
\end{html}
\fi
}

\newcommand{\MScriptPage}[2]{
% Bindet eine JavaScript-Datei ein, die eine eigene Seite bekommt
% (wird nur im HTML, aber nicht im PDF erstellt)
% #1 Dateiname des Programms (sollte mit ".js" enden)
% #2 = Kurztitel der Seite
\ifttm
\begin{MSContent}{#2}{#2}{puzzle}
\MRegisterFile{#1}
\begin{html}
<script src="\MMaterial/\MLastFile" type="text/javascript"></script>
\end{html}
\end{MSContent}
\fi
}

\newcommand{\MIncludePSE}{
% Bindet bei Chemie-Modulen das PSE ein
% (wird nur im HTML, aber nicht im PDF erstellt)
\ifttm
\special{html:<!-- includepse //-->}
\begin{MSContent}{Periodensystem der Elemente}{PSE}{table}
\MRegisterFile{../files/pse.js}
\MRegisterFile{../files/radio.png}
\begin{html}
<script src="\MMaterial/../files/pse.js" type="text/javascript"></script>
<p id="divid"><br /><br />
<script language="javascript" type="text/javascript">
    startpse("divid","\MMaterial/../files"); 
</script>
</p>
<br />
<br />
<br />
<p>Die Farben der Elementsymbole geben an: <font style="color:Red">gasf&ouml;rmig </font> <font style="color:Blue">fl&uuml;ssig </font> fest</p>
<p>Die Elemente der Gruppe 1 A, 2 A, 3 A usw. geh&ouml;ren zu den Hauptgruppenelementen.</p>
<p>Die Elemente der Gruppe 1 B, 2 B, 3 B usw. geh&ouml;ren zu den Nebengruppenelementen.</p>
<p>() kennzeichnet die Masse des stabilsten Isotops</p>
\end{html}
\end{MSContent}
\fi
}

\newcommand{\MAppletArchive}[4]{
% Bindet ein Java-Applet ein, die Parameter sind:
% (wird nur im HTML, aber nicht im PDF erstellt)
% #1 Dateiname der Klasse mit Appletaufruf (muss mit ".class" enden)
% #2 Dateiname des Archivs (muss mit ".jar" enden)
% #3 = Breite in Pixeln
% #4 = Hoehe in Pixeln
\ifttm
\MRegisterFile{#2}
\begin{html}
<applet code="#1" archive="\end{html}\MMaterial/\MLastFile\begin{html}" codebase="." width="#3" height="#4" alt="[Java-Archiv kann nicht gestartet werden]"></applet>
\end{html}
\fi
}

% Bindet in der Haupttexdatei ein MINT-Modul ein. Parameter 1 ist das Verzeichnis (relativ zur Haupttexdatei), Parameter 2 ist der Dateinahme ohne Pfad.
\newcommand{\IncludeModule}[2]{
\renewcommand{\MDPrefix}{#1}
\input{#1/#2}
\ifnum\value{MSSEnd}>0{\MSubsectionEndMacros}\addtocounter{MSSEnd}{-1}\fi
}

% Der ttm-Konverter setzt keine Makros im \input um, also muss hier getrickst werden:
% Das MDPrefix muss in den einzelnen Modulen manuell eingesetzt werden
\newcommand{\MInputFile}[1]{
\ifttm
\input{#1}
\else
\input{#1}
\fi
}


\newcommand{\MCases}[1]{\left\lbrace{\begin{array}{rl} #1 \end{array}}\right.}

\ifttm
\newenvironment{MCaseEnv}{\left\lbrace\begin{array}{rl}}{\end{array}\right.}
\else
\newenvironment{MCaseEnv}{\left\lbrace\begin{array}{rl}}{\end{array}\right.}
\fi

\def\MSkip{\ifttm\MCR\fi}

\ifttm
\def\MCR{\special{html:<br />}}
\else
\def\MCR{\ \\}
\fi


% Pragmas - Sind Schluesselwoerter, die dem Preprocessing sowie dem Konverter uebergeben werden und bestimmte
%           Aktionen ausloesen. Im Output (PDF und HTML) tauchen sie nicht auf.
\newcommand{\MPragma}[1]{%
\ifttm%
\special{html:<!-- mpragma;-;}#1\special{html:;; -->}%
\else%
% MPragmas werden vom Preprozessor direkt im LaTeX gefunden
\fi%
}

% Ersatz der Befehle textsubscript und textsuperscript, die ttm nicht kennt
\ifttm%
\newcommand{\MTextsubscript}[1]{\special{html:<sub>}#1\special{html:</sub>}}%
\newcommand{\MTextsuperscript}[1]{\special{html:<sup>}#1\special{html:</sup>}}%
\else%
\newcommand{\MTextsubscript}[1]{\textsubscript{#1}}%
\newcommand{\MTextsuperscript}[1]{\textsuperscript{#1}}%
\fi

%------------------ Einbindung von dia-Diagrammen ----------------------------------------------
% Beim preprocessing wird aus jeder dia-Datei eine tex-Datei und eine pdf-Datei erzeugt,
% diese werden hier jeweils im PDF und HTML eingebunden
% Parameter: Dateiname der mit dia erstellten Datei (OHNE die Endung .dia)
\ifttm%
\newcommand{\MDia}[1]{%
\MGraphicsSolo{#1minthtml.png}{}%
}
\else%
\newcommand{\MDia}[1]{%
\MGraphicsSolo{#1mintpdf.png}{scale=0.1667}%
}
\fi%

% subsup funktioniert im Ausdruck $D={\R}^+_0$, also \R geklammert und sup zuerst
% \ifttm
% \def\MSubsup#1#2#3{\special{html:<msubsup>} #1 #2 #3\special{html:</msubsup>}}
% \else
% \def\MSubsup#1#2#3{{#1}^{#3}_{#2}}
% \fi

%\input{local.tex}

% \ifttm
% \else
% \newwrite\mintlog
% \immediate\openout\mintlog=mintlog.txt
% \fi

% ----------------------- tikz autogenerator -------------------------------------------------------------------

\newcommand{\Mtikzexternalize}{\tikzexternalize}% wird bei Konvertierung ueber mconvert ggf. ausgehebelt!

\ifttm
\else
\tikzset%
{
  % Defines a custom style which generates pdf and converts to (low and hi-res quality) png and svg, then deletes the pdf
  % Important: DO NOT directly convert from pdf to hires-png or from svg to png with GraphViz convert as it has some problems and memory leaks
  png export/.style=%
  {
    external/system call/.add={}{; 
      pdf2svg "\image.pdf" "\image.svg" ; 
      convert -density 112.5 -transparent white "\image.pdf" "\image.png"; 
      inkscape --export-png="\image.4x.png" --export-dpi=450 --export-background-opacity=0 --without-gui "\image.svg"; 
      rm "\image.pdf"; rm "\image.log"; rm "\image.dpth"; rm "\image.idx"
    },
    external/force remake,
  }
}
\tikzset{png export}
\tikzsetexternalprefix{}
% PNGs bei externer Erzeugung in "richtiger" Groesse einbinden
\pgfkeys{/pgf/images/include external/.code={\includegraphics[scale=0.64]{#1}}}
\fi

% Spezielle Umgebung fuer Autogenerierung, Bildernamen sind nur innerhalb eines Moduls (einer MSection) eindeutig)

\newcommand{\MTIKZautofilename}{tikzautofile}

\ifttm
% HTML-Version: Vom Autogenerator erzeugte png-Datei einbinden, tikz selbst nicht ausfuehren (sprich: #1 schlucken)
\newcommand{\MTikzAuto}[1]{%
\addtocounter{MTIKZAutofilenumber}{1}
\renewcommand{\MTIKZautofilename}{mtikzauto_\arabic{MTIKZAutofilenumber}}
\MUGraphicsSolo{\MSectionID\MTIKZautofilename.4x.png}{scale=1}{\special{html:[[!-- svgstyle;}\MSectionID\MTIKZautofilename\special{html: //--]]}} % Styleinfos werden aus original-png, nicht 4x-png geholt!
%\MRegisterFile{\MSectionID\MTIKZautofilename.png} % not used right now
%\MRegisterFile{\MSectionID\MTIKZautofilename.svg}
}
\else%
% PDF-Version: Falls Autogenerator aktiv wird Datei automatisch benannt und exportiert
\newcommand{\MTikzAuto}[1]{%
\addtocounter{MTIKZAutofilenumber}{1}%
\renewcommand{\MTIKZautofilename}{mtikzauto_\arabic{MTIKZAutofilenumber}}
\tikzsetnextfilename{\MTIKZautofilename}%
#1%
}
\fi

% In einer reinen LaTeX-Uebersetzung kapselt der Preambelinclude-Befehl nur input,
% in einer konvertergesteuerten PDF/HTML-Uebersetzung wird er dagegen entfernt und
% die Preambeln an mintmod angehaengt, die Ersetzung wird von mconvert.pl vorgenommen.

\newcommand{\MPreambleInclude}[1]{\input{#1}}

% Globale Watermarksettings (werden auch nochmal zu Beginn jedes subsection gesetzt,
% muessen hier aber auch global ausgefuehrt wegen Einfuehrungsseiten und Inhaltsverzeichnis

\MWatermarkSettings
% ---------------------------------- Parametrisierte Aufgaben ----------------------------------------

\ifttm
\newenvironment{MPExercise}{%
\begin{MExercise}%
}{%
\special{html:<button name="Name_MPEX}\arabic{MExerciseCounter}\special{html:" id="MPEX}\arabic{MExerciseCounter}%
\special{html:" type="button" onclick="reroll('}\arabic{MExerciseCounter}\special{html:');">Neue Aufgabe erzeugen</button>}%
\end{MExercise}%
}
\else
\newenvironment{MPExercise}{%
\begin{MExercise}%
}{%
\end{MExercise}%
}
\fi

% Parameter: Name, Min, Max, PDF-Standard. Name in Deklaration OHNE backslash, im Code MIT Backslash
\ifttm
\newcommand{\MGlobalInteger}[4]{\special{html:%
<!-- onloadstart //-->%
MVAR.push(createGlobalInteger("}#1\special{html:",}#2\special{html:,}#3\special{html:,}#4\special{html:)); %
<!-- onloadstop //-->%
<!-- viewmodelstart //-->%
ob}#1\special{html:: ko.observable(rerollMVar("}#1\special{html:")),%
<!-- viewmodelstop //-->%
}%
}%
\else%
\newcommand{\MGlobalInteger}[4]{\newcounter{mvc_#1}\setcounter{mvc_#1}{#4}}
\fi

% Parameter: Name, Min, Max, PDF-Standard. Name in Deklaration OHNE backslash, im Code MIT Backslash, Wert ist Wurzel von value
\ifttm
\newcommand{\MGlobalSqrt}[4]{\special{html:%
<!-- onloadstart //-->%
MVAR.push(createGlobalSqrt("}#1\special{html:",}#2\special{html:,}#3\special{html:,}#4\special{html:)); %
<!-- onloadstop //-->%
<!-- viewmodelstart //-->%
ob}#1\special{html:: ko.observable(rerollMVar("}#1\special{html:")),%
<!-- viewmodelstop //-->%
}%
}%
\else%
\newcommand{\MGlobalSqrt}[4]{\newcounter{mvc_#1}\setcounter{mvc_#1}{#4}}% Funktioniert nicht als Wurzel !!!
\fi

% Parameter: Name, Min, Max, PDF-Standard zaehler, PDF-Standard nenner. Name in Deklaration OHNE backslash, im Code MIT Backslash
\ifttm
\newcommand{\MGlobalFraction}[5]{\special{html:%
<!-- onloadstart //-->%
MVAR.push(createGlobalFraction("}#1\special{html:",}#2\special{html:,}#3\special{html:,}#4\special{html:,}#5\special{html:)); %
<!-- onloadstop //-->%
<!-- viewmodelstart //-->%
ob}#1\special{html:: ko.observable(rerollMVar("}#1\special{html:")),%
<!-- viewmodelstop //-->%
}%
}%
\else%
\newcommand{\MGlobalFraction}[5]{\newcounter{mvc_#1}\setcounter{mvc_#1}{#4}} % Funktioniert nicht als Bruch !!!
\fi

% MVar darf im HTML nur in MEvalMathDisplay-Umgebungen genutzt werden oder in Strings die an den Parser uebergeben werden
\ifttm%
\newcommand{\MVar}[1]{\special{html:[var_}#1\special{html:]}}%
\else%
\newcommand{\MVar}[1]{\arabic{mvc_#1}}%
\fi

\ifttm%
\newcommand{\MRerollButton}[2]{\special{html:<button type="button" onclick="rerollMVar('}#1\special{html:');">}#2\special{html:</button>}}%
\else%
\newcommand{\MRerollButton}[2]{\relax}% Keine sinnvolle Entsprechung im PDF
\fi

% MEvalMathDisplay fuer HTML wird in mconvert.pl im preprocessing realisiert
% PDF: eine equation*-Umgebung (ueber amsmath)
% HTML: Eine Mathjax-Tex-Umgebung, deren Auswertung mit knockout-obervablen gekoppelt ist
% PDF-Version hier nur fuer pdflatex-only-Uebersetzung gegeben

\ifttm\else\newenvironment{MEvalMathDisplay}{\begin{equation*}}{\end{equation*}}\fi

% ---------------------------------- Spezialbefehle fuer AD ------------------------------------------

%Abk�rzung f�r \longrightarrow:
\newcommand{\lto}{\ensuremath{\longrightarrow}}

%Makro f�r Funktionen:
\newcommand{\exfunction}[5]
{\begin{array}{rrcl}
 #1 \colon  & #2 &\lto & #3 \\[.05cm]  
  & #4 &\longmapsto  & #5 
\end{array}}

\newcommand{\function}[5]{%
#1:\;\left\lbrace{\begin{array}{rcl}
 #2 &\lto & #3 \\
 #4 &\longmapsto  & #5 \end{array}}\right.}


%Die Identit�t:
\DeclareMathOperator{\Id}{Id}

%Die Signumfunktion:
\DeclareMathOperator{\sgn}{sgn}

%Zwei Betonungskommandos (k�nnen angepasst werden):
\newcommand{\highlight}[1]{#1}
\newcommand{\modstextbf}[1]{#1}
\newcommand{\modsemph}[1]{#1}


% ---------------------------------- Spezialbefehle fuer JL ------------------------------------------


\def\jccolorfkt{green!50!black} %Farbe des Funktionsgraphen
\def\jccolorfktarea{green!25!white} %Farbe der Fl"ache unter dem Graphen
\def\jccolorfktareahell{green!12!white} %helle Einf"arbung der Fl"ache unter dem Graphen
\def\jccolorfktwert{green!50!black} %Farbe einzelner Punkte des Graphen

\newcommand{\MPfadBilder}{Bilder}

\ifttm%
\newcommand{\jMD}{\,\MD}%
\else%
\newcommand{\jMD}{\;\MD}%
\fi%

\def\jHTMLHinweisBedienung{\MInputHint{%
Mit Hilfe der Symbole am oberen Rand des Fensters
k"onnen Sie durch die einzelnen Abschnitte navigieren.}}

\def\jHTMLHinweisEingabeText{\MInputHint{%
Geben Sie jeweils ein Wort oder Zeichen als Antwort ein.}}

\def\jHTMLHinweisEingabeTerm{\MInputHint{%
Klammern Sie Ihre Terme, um eine eindeutige Eingabe zu erhalten. 
Beispiel: Der Term $\frac{3x+1}{x-2}$ soll in der Form
\texttt{(3*x+1)/((x+2)^2}$ eingegeben werden (wobei auch Leerzeichen 
eingegeben werden k"onnen, damit eine Formel besser lesbar ist).}}

\def\jHTMLHinweisEingabeIntervalle{\MInputHint{%
Intervalle werden links mit einer "offnenden Klammer und rechts mit einer 
schlie"senden Klammer angegeben. Eine runde Klammer wird verwendet, wenn der 
Rand nicht dazu geh"ort, eine eckige, wenn er dazu geh"ort. 
Als Trennzeichen wird ein Komma oder ein Semikolon akzeptiert.
Beispiele: $(a, b)$ offenes Intervall,
$[a; b)$ links abgeschlossenes, rechts offenes Intervall von $a$ bis $b$. 
Die Eingabe $]a;b[$ f"ur ein offenes Intervall wird nicht akzeptiert.
F"ur $\infty$ kann \texttt{infty} oder \texttt{unendlich} geschrieben werden.}}

\def\jHTMLHinweisEingabeFunktionen{\MInputHint{%
Schreiben Sie Malpunkte (geschrieben als \texttt{*}) aus und setzen Sie Klammern um Argumente f�r Funktionen.
Beispiele: Polynom: \texttt{3*x + 0.1}, Sinusfunktion: \texttt{sin(x)}, 
Verkettung von cos und Wurzel: \texttt{cos(sqrt(3*x))}.}}

\def\jHTMLHinweisEingabeFunktionenSinCos{\MInputHint{%
Die Sinusfunktion $\sin x$ wird in der Form \texttt{sin(x)} angegeben, %
$\cos\left(\sqrt{3 x}\right)$ durch \texttt{cos(sqrt(3*x))}.}}

\def\jHTMLHinweisEingabeFunktionenExp{\MInputHint{%
Die Exponentialfunktion $\MEU^{3x^4 + 5}$ wird als
\texttt{exp(3 * x^4 + 5)} angegeben, %
$\ln\left(\sqrt{x} + 3.2\right)$ durch \texttt{ln(sqrt(x) + 3.2)}.}}

% ---------------------------------- Spezialbefehle fuer Fachbereich Physik --------------------------

\newcommand{\E}{{e}}
\newcommand{\ME}[1]{\cdot 10^{#1}}
\newcommand{\MU}[1]{\;\mathrm{#1}}
\newcommand{\MPG}[3]{%
  \ifnum#2=0%
    #1\ \mathrm{#3}%
  \else%
    #1\cdot 10^{#2}\ \mathrm{#3}%
  \fi}%
%

\newcommand{\MMul}{\MExponentensymbXYZl} % Nur eine Abkuerzung


% ---------------------------------- Stichwortfunktionialitaet ---------------------------------------

% mpreindexentry wird durch Auswahlroutine in conv.pl durch mindexentry substitutiert
\ifttm%
\def\MIndex#1{\index{#1}\special{html:<!-- mpreindexentry;;}#1\special{html:;;}\arabic{MSubjectArea}\special{html:;;}%
\arabic{chapter}\special{html:;;}\arabic{section}\special{html:;;}\arabic{subsection}\special{html:;;}\arabic{MEntryCounter}\special{html:; //-->}%
\setcounter{MLastIndex}{\value{MEntryCounter}}%
\addtocounter{MEntryCounter}{1}%
}%
% Copyrightliste wird als tex-Datei im preprocessing von conv.pl erzeugt und unter converter/tex/entrycollection.tex abgelegt
% Der input-Befehl funktioniert nur, wenn die aufrufende tex-Datei auf der obersten Ebene liegt (d.h. selbst kein input/include ist, insbesondere keine Moduldatei)
\def\MEntryList{} % \input funktioniert nicht, weil ttm (und damit das \input) ausgefuehrt wird, bevor Datei da ist
\else%
\def\MIndex#1{\index{#1}}
\def\MEntryList{\MAbort{Stichwortliste nur im HTML realisierbar}}%
\fi%

\def\MEntry#1#2{\textbf{#1}\MIndex{#2}} % Idee: MLastType auf neuen Entry-Typ und dann ein MLabel vergeben mit autogen-Nummer

% ---------------------------------- Befehle fuer Tests ----------------------------------------------

% MEquationItem stellt eine Eingabezeile der Form Vorgabe = Antwortfeld her, der zweite Parameter kann z.B. MSimplifyQuestion-Befehl sein
\ifttm
\newcommand{\MEquationItem}[2]{{#1}$\,=\,${#2}}%
\else%
\newcommand{\MEquationItem}[2]{{#1}$\;\;=\,${#2}}%
\fi

\ifttm
\newcommand{\MInputHint}[1]{%
\ifnum%
\if\value{MTestSite}>0%
\else%
{\color{blue}#1}%
\fi%
\fi%
}
\else
\newcommand{\MInputHint}[1]{\relax}
\fi

\ifttm
\newcommand{\MInTestHeader}{%
Dies ist ein einreichbarer Test:
\begin{itemize}
\item{Im Gegensatz zu den offenen Aufgaben werden beim Eingeben keine Hinweise zur Formulierung der mathematischen Ausdr�cke gegeben.}
\item{Der Test kann jederzeit neu gestartet oder verlassen werden.}
\item{Der Test kann durch die Buttons am Ende der Seite beendet und abgeschickt, oder zur�ckgesetzt werden.}
\item{Der Test kann mehrfach probiert werden. F�r die Statistik z�hlt die zuletzt abgeschickte Version.}
\end{itemize}
}
\else
\newcommand{\MInTestHeader}{%
\relax
}
\fi

\ifttm
\newcommand{\MInTestFooter}{%
\special{html:<button name="Name_TESTFINISH" id="TESTFINISH" type="button" onclick="finish_button('}\MTestName\special{html:');">Test auswerten</button>}%
\begin{html}
&nbsp;&nbsp;&nbsp;&nbsp;&nbsp;&nbsp;&nbsp;&nbsp;
<button name="Name_TESTRESET" id="TESTRESET" type="button" onclick="reset_button();">Test zur�cksetzen</button>
<br />
<br />
<div class="xreply">
<p name="Name_TESTEVAL" id="TESTEVAL">
Hier erscheint die Testauswertung!
<br />
</p>
</div>
\end{html}
}
\else
\newcommand{\MInTestFooter}{%
\relax
}
\fi


% ---------------------------------- Notationsmakros -------------------------------------------------------------

% Notationsmakros die nicht von der Kursvariante abhaengig sind

\newcommand{\MZahltrennzeichen}[1]{\renewcommand{\MZXYZhltrennzeichen}{#1}}

\ifttm
\newcommand{\MZahl}[3][\MZXYZhltrennzeichen]{\edef\MZXYZtemp{\noexpand\special{html:<mn>#2#1#3</mn>}}\MZXYZtemp}
\else
\newcommand{\MZahl}[3][\MZXYZhltrennzeichen]{{}#2{#1}#3}
\fi

\newcommand{\MEinheitenabstand}[1]{\renewcommand{\MEinheitenabstXYZnd}{#1}}
\ifttm
\newcommand{\MEinheit}[2][\MEinheitenabstXYZnd]{{}#1\edef\MEINHtemp{\noexpand\special{html:<mi mathvariant="normal">#2</mi>}}\MEINHtemp} 
\else
\newcommand{\MEinheit}[2][\MEinheitenabstXYZnd]{{}#1 \mathrm{#2}} 
\fi

\newcommand{\MExponentensymbol}[1]{\renewcommand{\MExponentensymbXYZl}{#1}}
\newcommand{\MExponent}[2][\MExponentensymbXYZl]{{}#1{} 10^{#2}} 

%Punkte in 2 und 3 Dimensionen
\newcommand{\MPointTwo}[3][]{#1(#2\MCoordPointSep #3{}#1)}
\newcommand{\MPointThree}[4][]{#1(#2\MCoordPointSep #3\MCoordPointSep #4{}#1)}
\newcommand{\MPointTwoAS}[2]{\left(#1\MCoordPointSep #2\right)}
\newcommand{\MPointThreeAS}[3]{\left(#1\MCoordPointSep #2\MCoordPointSep #3\right)}

% Masseinheit, Standardabstand: \,
\newcommand{\MEinheitenabstXYZnd}{\MThinspace} 

% Horizontaler Leerraum zwischen herausgestellter Formel und Interpunktion
\ifttm
\newcommand{\MDFPSpace}{\,}
\newcommand{\MDFPaSpace}{\,\,}
\newcommand{\MBlank}{\ }
\else
\newcommand{\MDFPSpace}{\;}
\newcommand{\MDFPaSpace}{\;\;}
\newcommand{\MBlank}{\ }
\fi

% Satzende in herausgestellter Formel mit horizontalem Leerraum
\newcommand{\MDFPeriod}{\MDFPSpace .}

% Separation von Aufzaehlung und Bedingung in Menge
\newcommand{\MCondSetSep}{\,:\,} %oder '\mid'

% Konverter kennt mathopen nicht
\ifttm
\def\mathopen#1{}
\fi

% -----------------------------------START Rouletteaufgaben ------------------------------------------------------------

\ifttm
% #1 = Dateiname, #2 = eindeutige ID fuer das Roulette im Kurs
\newcommand{\MDirectRouletteExercises}[2]{
\begin{MExercise}
\texttt{Im HTML erscheinen hier Aufgaben aus einer Aufgabenliste...}
\end{MExercise}
}
\else
\newcommand{\MDirectRouletteExercises}[2]{\relax} % wird durch mconvert.pl gefunden und ersetzt
\fi


% ---------------------------------- START Makros, die von der Kursvariante abhaengen ----------------------------------

\ifvariantunotation
  % unotation = An Universitaeten uebliche Notation
  \def\MVariant{unotation}

  % Trennzeichen fuer Dezimalzahlen
  \newcommand{\MZXYZhltrennzeichen}{.}

  % Exponent zur Basis 10 in der Exponentialschreibweise, 
  % Standardmalzeichen: \times
  \newcommand{\MExponentensymbXYZl}{\times} 

  % Begrenzungszeichen fuer offene Intervalle
  \newcommand{\MoIl}[1][]{\mbox{}#1(\mathopen{}} % bzw. ']'
  \newcommand{\MoIr}[1][]{#1)\mbox{}} % bzw. '['

  % Zahlen-Separation im IntervaLL
  \newcommand{\MIntvlSep}{,} %oder ';'

  % Separation von Elementen in Mengen
  \newcommand{\MElSetSep}{,} %oder ';'

  % Separation von Koordinaten in Punkten
  \newcommand{\MCoordPointSep}{,} %oder ';' oder '|', '\MThinspace|\MThinspace'

\else
  % An dieser Stelle wird angenommen, dass std-Variante aktiv ist
  % std = beschlossene Notation im TU9-Onlinekurs 
  \def\MVariant{std}

  % Trennzeichen fuer Dezimalzahlen
  \newcommand{\MZXYZhltrennzeichen}{,}

  % Exponent zur Basis 10 in der Exponentialschreibweise, 
  % Standardmalzeichen: \times
  \newcommand{\MExponentensymbXYZl}{\times} 

  % Begrenzungszeichen fuer offene Intervalle
  \newcommand{\MoIl}[1][]{\mbox{}#1]\mathopen{}} % bzw. '('
  \newcommand{\MoIr}[1][]{#1[\mbox{}} % bzw. ')'

  % Zahlen-Separation im IntervaLL
  \newcommand{\MIntvlSep}{;} %oder ','
  
  % Separation von Elementen in Mengen
  \newcommand{\MElSetSep}{;} %oder ','

  % Separation von Koordinaten in Punkten
  \newcommand{\MCoordPointSep}{;} %oder '|', '\MThinspace|\MThinspace'

\fi



% ---------------------------------- ENDE Makros, die von der Kursvariante abhaengen ----------------------------------


% diese Kommandos setzen Mathemodus vorraus
\newcommand{\MGeoAbstand}[2]{[\overline{{#1}{#2}}]}
\newcommand{\MGeoGerade}[2]{{#1}{#2}}
\newcommand{\MGeoStrecke}[2]{\overline{{#1}{#2}}}
\newcommand{\MGeoDreieck}[3]{{#1}{#2}{#3}}

%
\ifttm
\newcommand{\MOhm}{\special{html:<mn>&#x3A9;</mn>}}
\else
\newcommand{\MOhm}{\Omega} %\varOmega
\fi


\def\PERCTAG{\MAbort{PERCTAG ist zur internen verwendung in mconvert.pl reserviert, dieses Makro darf sonst nicht benutzt werden.}}

% Im Gegensatz zu einfachen html-Umgebungen werden MDirectHTML-Umgebungen von mconvert.pl am ganzen ttm-Prozess vorbeigeschleust und aus dem PDF komplett ausgeschnitten
\ifttm%
\newenvironment{MDirectHTML}{\begin{html}}{\end{html}}%
\else%
\newenvironment{MDirectHTML}{\begin{html}}{\end{html}}%
\fi

% Im Gegensatz zu einfachen Mathe-Umgebungen werden MDirectMath-Umgebungen von mconvert.pl am ganzen ttm-Prozess vorbeigeschleust, ueber MathJax realisiert, und im PDF als $$ ... $$ gesetzt
\ifttm%
\newenvironment{MDirectMath}{\begin{html}}{\end{html}}%
\else%
\newenvironment{MDirectMath}{\begin{equation*}}{\end{equation*}}% Vorsicht, auch \[ und \] werden in amsmath durch equation* redefiniert
\fi

% ---------------------------------- Location Management ---------------------------------------------

% #1 = buttonname (muss in files/images liegen und Format 48x48 haben), #2 = Vollstaendiger Einrichtungsname, #3 = Kuerzel der Einrichtung,  #4 = Name der include-texdatei
\ifttm
\newcommand{\MLocationSite}[3]{\special{html:<!-- mlocation;;}#1\special{html:;;}#2\special{html:;;}#3\special{html:;; //-->}}
\else
\newcommand{\MLocationSite}[3]{\relax}
\fi

% ---------------------------------- Copyright Management --------------------------------------------

\newcommand{\MCCLicense}{%
{\color{green}\textbf{CC BY-SA 3.0}}
}

\newcommand{\MCopyrightLabel}[1]{ (\MSRef{L_COPYRIGHTCOLLECTION}{Lizenz})\MLabel{#1}}

% Copyrightliste wird als tex-Datei im preprocessing erzeugt und unter converter/tex/copyrightcollection.tex abgelegt
% Der input-Befehl funktioniert nur, wenn die aufrufende tex-Datei auf der obersten Ebene liegt (d.h. selbst kein input/include ist, insbesondere keine Moduldatei)
\newcommand{\MCopyrightCollection}{\input{copyrightcollection.tex}}

% MCopyrightNotice fuegt eine Copyrightnotiz ein, der parser ersetzt diese durch CopyrightNoticePOST im preparsing, diese Definition wird nur fuer reine pdflatex-Uebersetzungen gebraucht
% Parameter: #1: Kurze Lizenzbeschreibung (typischerweise \MCCLicense)
%            #2: Link zum Original (http://...) oder NONE falls das Bild selbst ein Original ist, oder TIKZ falls das Bild aus einer tikz-Umgebung stammt
%            #3: Link zum Autor (http://...) oder MINT falls Original im MINT-Kolleg erstellt oder NONE falls Autor unbekannt
%            #4: Bemerkung (z.B. dass Datei mit Maple exportiert wurde)
%            #5: Labelstring fuer existierendes Label auf das copyrighted Objekt, mit MCopyrightLabel erzeugt
%            Keines der Felder darf leer sein!
\newcommand{\MCopyrightNotice}[5]{\MCopyrightNoticePOST{#1}{#2}{#3}{#4}{#5}}

\ifttm%
\newcommand{\MCopyrightNoticePOST}[5]{\relax}%
\else%
\newcommand{\MCopyrightNoticePOST}[5]{\relax}%
\fi%

% ---------------------------------- Meldungen fuer den Benutzer des Konverters ----------------------
\MPragma{mintmodversion;P0.1.0}
\MPragma{usercomment;This is file mintmod.tex version P0.1.0}


% ----------------------------------- Spezialelemente fuer Konfigurationsseite, werden nicht von mintscripts.js verwaltet --

% #1 = DOM-id der Box
\ifttm\newcommand{\MConfigbox}[1]{\special{html:<input cfieldtype="2" type="checkbox" name="Name_}#1\special{html:" id="}#1\special{html:" onchange="confHandlerChange('}#1\special{html:');"/>}}\fi % darf im PDF nicht aufgerufen werden!


\MPragma{MathSkip}

%Einkommentieren zur Erzeugung der pngs aus den tikz-Umgebungen
\Mtikzexternalize

\MSetSubject{\MINTMathematics}

\begin{document}

\MSection{Geometrie}
\MSetSectionID{VBKM05} % hier identisch mit dem alten tikz-Dateien-Prefix 
\MLabel{VBKM05}

\begin{MSectionStart}
\MDeclareSiteUXID{VBKM05_START}
Die Darstellung der Elementaren Geometrie beginnt mit der Einf\"uhrung von Winkeln im Bogenma\ss\ und im Gradma\ss\ 
und wird \"uber Stufen- und Wechselwinkel zu den Dreiecken hin fortgesetzt. Es werden kongruente und \"ahnliche Dreiecke betrachtet, ihre Fl\"acheninhalte und die von anderen Figuren berechnet, worauf man dann zu den Strahlens\"atzen kommt, die die Grundlage f\"ur die Einf\"uhrung der trigonometrischen Funktionen
am rechtwinkligen Dreieck bilden. Nachdem man sich auch die Trigonometrie am Einheitskreis angesehen hat, wird die dritte Dimension hinzugenommen und es werden Volumina von Zylindern, Kegeln und Kugeln behandelt.

Dieses Modul gliedert sich in folgende Abschnitte:

\begin{itemize}
\item{\MSRef{M05_Winkel}{Winkel, Kongruenz und \"Ahnlichkeit}}
\item{\MSRef{M05_Flaecheninhalt}{Fl\"acheninhalt und Strahlens\"atze}}
\item{\MSRef{Mathematik_ElementareGeometrie_Sec:Trigonometrie}{Trigonometrie}}
\item{\MSRef{M05_Abschlusstest}{Abschlusstest} f\"ur Modul \MNRef{VBKM05}.}
\end{itemize}


\end{MSectionStart}

\MSubsection{Winkel, Kongruenz und \"Ahnlichkeit}
\MLabel{M05_Winkel}

\begin{MIntro}
\MDeclareSiteUXID{VBKM05_WinkelIntro}

Das Bild zeigt einen Ausschnitt des Stra\ss enplans von Ludwigshafen am Rhein. An ihm lassen
sich einige Erkenntnisse \"uber die Geometrie in der Ebene ablesen.

\begin{center}
\MGraphicsSolo{Stadtplan.png}{scale=1} \MLabel{Stadtplan}
\end{center}

Man kann hier unter anderem das Bogenma\ss\ eines Winkels entdecken und Stufen- und Wechselwinkel aufsp\"uren. 

Im ersten Abschnitt dieses Moduls wird an einige geometrische Grundbegriffe erinnert, ehe im darauf folgenden 
Abschnitt das Bogenma\ss\ und das Gradma\ss\ eingef\"uhrt,
Stufen- und Wechselwinkel betrachtet und die Winkelsumme in Dreiecken bestimmt werden. Ferner werden Kriterien, wann zwei Dreiecke kongruent oder \"ahnlich sind, vorgestellt.

\end{MIntro}


\begin{MXContent}{Geometrische Grundbegriffe}{Geometrische Grundbegriffe}{STD}

\MDeclareSiteUXID{VBKM05_Gerade}

\begin{MInfo}
\MLabel{VBKM05_Gerade}
Eine \MEntry{Gerade}{Gerade} in der Ebene ist anschaulich gesprochen eine gerade, unendlich d\"unne Linie, auf der man beliebig entlang laufen kann, ohne an eine Begrenzung zu sto\ss en, d.h., sie erstreckt sich in beiden Richtungen ins Unendliche.
\begin{itemize}
 \item Eine mathematisch korrekte Erkl\"arung dieses Begriffes definiert eine Gerade in der Ebene als die L\"osungmenge einer linearen Gleichung der Form $ax+by=0$, wobei $a$ und $b$ reelle Zahlen sind, die nicht beide zugleich $0$ sein d\"urfen.
 \item   Eine Gerade ist stets durch die Angabe zweier voneinander verschiedener Punkte, die auf ihr liegen, eindeutig bestimmt.
\item Zwei Geraden in der Ebene
\begin{itemize}
 \item haben entweder genau einen gemeinsamen Punkt (der sogenannte Schnittpunkt der Geraden $g$ und $h$) oder
\item sind gleich oder
 \item haben \"uberhaupt keinen gemeinsamen Punkt.
\end{itemize}
In den beiden letzten F\"allen nennt man die beiden Geraden \MEntry{parallel}{parallele Geraden}.
\end{itemize}
\end{MInfo}

\MDeclareSiteUXID{VBKM05_Halbgerade}

\begin{MInfo}
\MLabel{VBKM05_Halbgerade}
Eine \MEntry{Halbgerade}{Halbgerade} (auch \MEntry{Strahl}{Strahl} genannt) ist \"ahnlich wie eine Gerade eine gerade, unendlich d\"unne Linie, auf der man nun allerdings lediglich in eine Richtung beliebig entlang laufen kann, ohne an eine Begrenzung zu sto\ss en, d.h., sie erstreckt sich nur in einer Richtung ins Unendliche und besitzt einen Anfangspunkt.
\begin{itemize}
 \item Halbgeraden sind genau diejenigen Punktmengen, die man erh\"alt, indem man die rechte reelle Zahlenachse (einschlie\ss lich $0$) in der Ebene dreht oder verschiebt (oder beides).
\item Mathematisch formuliert gestattet jede Halbgerade wenigstens eine der folgenden Darstellungen:
\begin{itemize}
 \item als L\"osungsmenge der Gleichung $ax+by=0$ unter der Nebenbedingung $x\geq c$;
 \item als L\"osungsmenge der Gleichung $ax+by=0$ unter der Nebenbedingung $x\leq c$;
 \item als L\"osungsmenge der Gleichung $ax+by=0$ unter der Nebenbedingung $y\geq c$;
 \item als L\"osungsmenge der Gleichung $ax+by=0$ unter der Nebenbedingung $y\leq c$;
\end{itemize}
mit reellen Zahlen $a$, $b$ und $c$, wobei $a$ und $b$ nicht beide zugleich $0$ sein d\"urfen.
 \item Eine Halbgerade ist durch die Angabe ihres Anfangspunktes und eines weiteren Punktes, der auf der Halbgeraden liegt, eindeutig bestimmt.
 \item Zu jeder Halbgeraden gibt es genau eine Gerade, auf der diese Halbgerade liegt.
 \item Zu jeder Halbgeraden $g$ gibt es die sogenannte \MEntry{entgegengesetzte Halbgerade}{entgegengesetzte Halbgerade}, die hier mit $g^-$ bezeichnet wird.
Die entgegengesetzte Halbgerade $g^-$ erh\"alt man, indem man im Anfangspunkt der Halbgeraden $g$ startend in die genau entgegengesetzte Richtung l\"auft.
Oder anders ausgedr\"uckt: Ist $g$ eine Halbgerade mit Anfangspunkt $P$ und ist $h$ die eindeutig bestimmte Gerade, auf der $g$ liegt, so erh\"alt man $g^-$, indem bis auf den Punkt $P$ alle Punkte, die zu $g$ geh\"oren, aus $h$ entfernt. Die entgegengesetzte Halbgerade hat den gleichen Anfangspunkt wie die urspr\"ungliche Halbgerade.
\end{itemize}
\end{MInfo}


\MDeclareSiteUXID{VBKM05_Strecke}

\begin{MInfo}
\MLabel{VBKM05_Strecke}
Eine \MEntry{Strecke}{Strecke} ist anschaulich gesprochen diejenige gerade, unendlich d\"unne Linie, die zwei Punkte direkt miteinander verbindet. Man nennt diese beiden Punkt die Endpunkte der Strecke.
\begin{itemize}
 \item Strecken sind genau diejenigen nichtleeren Punktmengen, die als Schnittmengen zweier Halbgeraden, von denen keine die jeweils andere enth\"alt, auftreten. Man beachte jedoch, dass bei dieser Definition auch einzelne Punkte als Strecken aufgefasst werden.
\item Eine Strecke ist stets durch die Angabe ihrer Endpunkte eindeutig bestimmt. Man sagt auch, dass eine Strecke ihre Endpunkte miteinander verbindet.
 \item Strecken haben stets eine endliche L\"ange. Diese L\"ange ist genau der Abstand zwischen ihren Endpunkten.
 \item Zu jeder Strecke, die aus mehr als einem Punkt besteht, gibt es genau eine Gerade, auf der diese Strecke liegt.
\item Zwei Strecken, die jeweils beide mehr als einen Punkt besitzen und zudem auf parallelen Geraden liegen, nennt man \MEntry{parallel}{parallel (Strecken)}.
\end{itemize}
\end{MInfo}  


\MDeclareSiteUXID{VBKM05_Kreis}

\begin{MInfo}
\MLabel{VBKM05_Strecke}
Ein \MEntry{Kreis}{Kreis} (oder eine \MEntry{Kreislinie}{Kreislinie}) mit Mittelpunkt $M$ und Radius $r>0$ ist die Menge aller Punkte der Ebene, welche zu $M$ genau den Abstand $r$ besitzen.
\begin{itemize}
\item Jeder Kreis l\"asst sich mathematisch als die L\"osungsmenge einer Gleichung der Form $$(x-a)^2+(y-b)^2=r^2$$ mit reellen Zahlen $a$ und $b$ beschreiben.
 \item Jeder Kreis ist durch die Angabe dreier voneinander verschiedener Punkte, die auf ihm liegen, eindeutig bestimmt.
\item Zwei voneinander verschiedene Kreise schneiden sich entweder in keinem, in genau einem oder in genau zwei Punkten.
 \item Der \MEntry{Durchmesser}{Durchmesser} $d$ eines Kreises mit Radius $r$ ist $2r$.
\end{itemize}
\end{MInfo}  
\end{MXContent}

\begin{MXContent}{Winkel}{Winkel}{STD}

\MDeclareSiteUXID{VBKM05_WinkelContent}
\MLabel{VBKM05_Winkel}
Zwei Halbgeraden ~$g$ und~$h$ in der Ebene, die von demselben Punkt ~$P$ ausgehen, bilden einen \MEntry{Winkel}{Winkel}.

\begin{center}
\MTikzAuto{%
\begin{tikzpicture}[scale=0.9,line width=2pt]
 \coordinate[label=left:$P$] (P) at (0,0);
 \coordinate[label=above:$g$] (A) at ($ (P) + (-20:3) $);
 \coordinate[label=below left:$h$] (B) at ($ (P) + (110:3) $);
 %
 \draw [color=green!50!black] (P) ++ (-20:1) arc (-20:110:1);
 \draw [color=red] (P) ++ (110:1) arc (110:340:1);
 \draw (A) -- (P) -- (B);
 \draw[dashed, line width=1.5pt] (P) -- node[left]{$1$} ++ (0,-1);
\end{tikzpicture}
}
\end{center}

Zeichnet man einen Kreis mit Radius~$1$ um~$P$, wird dieser von den beiden Halbgeraden in zwei Teile zerschnitten. Wichtig ist nun derjenige Kreisbogen, der entsteht, wenn man von der Geraden $g$ gegen den Uhrzeigersinn zur Geraden $h$ geht (im obigen Bild gr\"un eingef\"arbt):

\begin{MInfo}%
\MLabel{VBKM05_Bogenmass}
Der \MEntry{Kreisbogen}{Kreisbogen}, der entsteht, wenn man von der Geraden $g$ gegen den Uhrzeigersinn zur Geraden $h$ geht, bezeichnet den \MEntry{Winkel}{Winkel} von $g$ zu $h$.
  \begin{itemize}
    \item Die L\"ange des Kreisbogens mit dem Radius $1$ ist das sogenannte \MEntry{Bogenma\ss }{Bogenma\ss } $\Mmeasuredangle \left( g, h \right)$.
    \item $P$ hei\ss t \MEntry{Scheitelpunkt}{Scheitelpunkt (Winkel)} des Winkels, und die beiden Halbgeraden, die den Winkel bilden, hei\ss en \MEntry{Schenkel}{Schenkel} des Winkels.
    \item Hat man einen Punkt $A$ auf der Geraden $g$ und einen Punkt $B$ auf der Geraden $h$, so kann man auch $\Mmeasuredangle \left( A P B \right)$ statt $\Mmeasuredangle \left( g, h \right)$ schreiben.
    \item Winkel werden h\"aufig mit kleinen griechischen Buchstaben bezeichnet, z. Bsp. $\alpha$.
  \end{itemize}
\end{MInfo}

Man beachte, dass es bei der Bedeutung des Symbols $\Mmeasuredangle \left( g, h \right)$ wesentlich auf die Reihenfolge, in welcher $g$ und $h$ aufgef\"uhrt werden, ankommt:
$\Mmeasuredangle \left( h, g \right)$ bezeichnet n\"amlich die L\"ange desjenigen Kreisbogens, der entsteht, wenn man von der Geraden $h$ gegen den Uhrzeigersinn zur Geraden $g$ l\"auft (im obigen Bild rot eingef\"arbt), und diese L\"ange ist in der Regel von $\Mmeasuredangle \left( g, h \right)$ verschieden!


\begin{MHint}{griechische Buchstaben}
Falls Sie das griechische Alphabet noch nicht so gut kennen, gibt es hier eine kleine (nach mathematischen Gesichtspunkten geordnete) \"Ubersicht,
die auch die Gro\ss buchstaben enth\"alt:
\begin{center}
\begin{tabular}{|*{5}{cc|}}
\ifttm\hline\else\firsthline\fi
%Die Grossbuchstaben, die nicht auch vom lateinischen Alphabet genutzt
%werden, druckt pdfLaTeX aufrecht, MathML dagegen kursiv.
%Die Sortierung ist so geloest, dass oft zusammen verwendete Buchstaben
%in derselben Spalte landen.
 $\alpha$,       \ifttm$A$\else$\mathrm{A}$\fi & \glqq alpha\grqq         &
 $\xi$,          $\Xi$                         & \glqq xi\grqq            &
 $\iota$,        \ifttm$I$\else$\mathrm{I}$\fi & \glqq iota\grqq          &
 $o$,            \ifttm$O$\else$\mathrm{O}$\fi & \glqq omikron\grqq       &
 $\pi$,          $\Pi$                         & \glqq pi\grqq            \\
 $\beta$,        \ifttm$B$\else$\mathrm{B}$\fi & \glqq beta\grqq          &
 $\zeta$,        \ifttm$Z$\else$\mathrm{Z}$\fi & \glqq zeta\grqq          &
 $\Mvarkappa$,   \ifttm$K$\else$\mathrm{K}$\fi & \glqq kappa\grqq         &
 $\omega$,       $\Omega$                      & \glqq omega\grqq         &
 $\Mvarphi$,     $\Phi$                        & \glqq phi\grqq           \\
 $\gamma$,       $\Gamma$                      & \glqq gamma\grqq         &
 $\eta$,         \ifttm$H$\else$\mathrm{H}$\fi & \glqq eta\grqq           &
 $\lambda$,      $\Lambda$                     & \glqq lambda\grqq        &
 $\varrho$,      \ifttm$P$\else$\mathrm{P}$\fi & \glqq rho\grqq           &
 $\psi$,         $\Psi$                        & \glqq psi\grqq           \\
 $\delta$,       $\Delta$                      & \glqq delta\grqq         &
 $\vartheta$,    $\Theta$                      & \glqq theta\grqq         &
 $\mu$,          \ifttm$M$\else$\mathrm{M}$\fi & \glqq m\"u\grqq          &
 $\sigma$,       $\Sigma$                      & \glqq sigma\grqq         &
 $\chi$,         \ifttm$X$\else$\mathrm{X}$\fi & \glqq chi\grqq           \\
 $\Mvarepsilon$, \ifttm$E$\else$\mathrm{E}$\fi & \glqq epsilon\grqq       &
 \special{html:&nbsp;}                         & \special{html:&nbsp;}    &
 $\nu$,          \ifttm$N$\else$\mathrm{N}$\fi & \glqq n\"u\grqq          &
 $\tau$,         \ifttm$T$\else$\mathrm{T}$\fi & \glqq tau\grqq           &
 $\upsilon$,     $\Upsilon$                    & \glqq \"upsilon\grqq     \\
 \ifttm\hline\else\lasthline\fi
\end{tabular}

\end{center}
\end{MHint}

Indem man auch noch die entgegengesetzten Halbgeraden in die Betrachtungen miteinbezieht, lassen sich weitere Winkel entdecken.

\begin{MInfo}%
\MLabel{VBKM05_Scheitelwinkel_Nebenwinkel}%
Es seien $g$ und $h$ zwei Halbgeraden mit gleichem Anfangspunkt $P$ und mit den entgegengesetzten Halbgeraden $g^-$ bzw. $h^-$.
  \begin{itemize}
    \item Der Winkel $\Mmeasuredangle \left( g^-, h^- \right)$ hei\ss t \MEntry{Scheitelwinkel}{Scheitelwinkel}.
		\item Der Winkel $\Mmeasuredangle \left( g, h \right)$ und der Scheitelwinkel $\Mmeasuredangle \left( g^-, h^- \right)$ sind stets gleich gro\ss.
\item Die Winkel $\Mmeasuredangle \left( g^-, h \right)$ und $\Mmeasuredangle \left( g, h^- \right)$ hei\ss en \MEntry{Nebenwinkel}{Nebenwinkel}.
\item Die Nebenwinkel sind immer gleich gro\ss.
  \end{itemize}
\end{MInfo}

%\begin{MExample}
%Ein sch\"ones Beispiel f\"ur einen Winkel sieht man im Ludwigshafener Stadtplan bei der Kurf\"urstenstra\ss e und der Saarlandstra\ss e:

%\begin{center}
%\MGraphicsSolo{../Bilder/Mathematik_ElementareGeometrie/Mathematik_ElementareGeometrie_Stadtplan_Radialmass.png}{scale=1}
%\MGraphicsSolo{Mathematik_ElementareGeometrie_Stadtplan_Radialmass.png}{scale=1}
%\end{center}

%Hat der Weg von der Kreuzung der beiden Stra\ss en
%\"uber die Kurf\"urstenstra\ss e zur Sebastian-Bach-Stra\ss e eine L\"ange von~$1$, und geht man nun auf der Sebastian-Bach-Stra\ss e von der Kurf\"urstenstra\ss e aus bis zur Saarlandstra\ss e, so hat man genau einen Weg von
%\[
% \Mmeasuredangle \left( \text{Kurf\"urstenstra\ss e}, \text{\ Saarlandstra\ss e} \right)
%\]
%zur\"uckgelegt. (Dabei gehen wir davon aus, dass die Sebastian-Bach-Stra\ss e kreisf\"ormig um
%die Kreuzung Kurf\"urstenstra\ss e und Saarlandstra\ss e herumf\"uhrt.)
%\end{MExample}

Nun soll das Bogenma\ss\ bestimmt werden.

Bereits die Griechen stellten fest, dass das Verh\"altnis des
Umfangs~$U$ eines Kreises zu seinem Radius~$r$ stets das gleiche ist.
Sie definierten dieses Verh\"altnis \"uber die Kreiszahl~$\pi$: 
\begin{MInfo}\MLabel{Kreiszahl}%
Die \textbf{Kreiszahl} ist
\[
    \pi = \frac{U}{2r} \MDFPeriod
\]

Dabei ist $\pi$ keine rationale Zahl, sie kann nicht als endlicher oder periodischer Dezimalbruch geschrieben werden. N\"aherungsrechnungen haben ergeben, dass
$\pi \approx \MZahl{3}{141592653589793}$ ist.
\end{MInfo}

Ist der Radius des Kreises genau $1$, so hat der Kreis den Umfang $2\pi$. Die L\"ange eines Kreisbogens, also das Bogenma\ss\ des Winkels, ist dann eine Zahl zwischen $0$ und $2\pi$.

\begin{MExample}
Der Winkel zwischen zwei Halbgeraden, die einen Halbkreis ausschneiden, betr\"agt $2\pi/2=\pi$.\\
Der Winkel zwischen zwei Halbgeraden, die einen Viertelkreis ausschneiden, betr\"agt $2\pi/4=\pi/2$.\\
Der Winkel zwischen zwei Halbgeraden, die einen Achtelkreis ausschneiden, betr\"agt $2\pi/8=\pi/4$.
\end{MExample}


Wenn man den Winkel $\Mmeasuredangle \left( g, h \right)$ kennt, so kann man nun auch leicht den Winkel $\Mmeasuredangle \left( h, g \right)$ bestimmen, der ja nach dem Bild \MRef{Bogenmass} durch den anderen Kreisbogen bestimmt ist:
\[
   \Mmeasuredangle \left( h, g \right)
 = 2 \pi - \Mmeasuredangle \left( g, h \right) \MDFPeriod
\]


Ein anderes sehr gebr\"auchliches Winkelma\ss\ erh\"alt man, indem man den Kreis in 360~gleich
gro\ss e Teile zerlegt und dann misst, wie viele dieser Teile
\"uberstrichen werden, wenn~$g$ mathematisch positiv auf~$h$ gedreht wird.
Dieses \textbf{Gradma\ss} eines Winkels kann leicht in das
Bogenma\ss\ \"uberf\"uhrt werden:
\[
   \Mmeasuredangle \left( g, h \right)
 = 2 \pi \cdot \frac{\alpha}{360^{\circ}} \MDFPSpace,
\]
wenn $\alpha$ das Gradma\ss\ des Winkels zwischen~$g$ und~$h$ angibt.

\begin{MExercise}
Der Winkel $\Mmeasuredangle\left(g, h\right)$ betr\"agt im Gradma\ss\ $60^\circ$. Rechnen Sie den Winkel in das Bogenma\ss\ um:
\ifttm%
(Ein~$\pi$ geben Sie als \glqq\texttt{pi}\grqq\ ein. Bitte runden Sie Ihre Ergebnisse auf drei Nachkommastellen!)
\fi

\ \\
$\Mmeasuredangle\left(g, h\right)=$\MLParsedQuestion{10}{pi/3}{3}{GEO1}.
\ \\
\begin{MHint}{L\"osung}
\[\frac{60^\circ}{360^\circ}=\frac{\Mmeasuredangle\left(g, h\right)}{2\pi} \quad\Rightarrow\quad \Mmeasuredangle\left(g, h\right)=\frac{60^\circ}{360^\circ}\cdot 2\pi=\frac{1}{6}\cdot 2\pi=\frac{\pi}{3}\MDFPeriod\]
\end{MHint}

\end{MExercise}

\begin{MExercise}
Der Winkel $\beta$ betr\"agt im Bogenma\ss\ $\pi/4$. Wie gro\ss\ ist der Winkel im Gradma\ss ?

\ \\
$\beta=$\MLParsedQuestion{10}{45}{3}{PARSEDQUEST2}$^\circ$.
\ \\
\begin{MHint}{L\"osung}
\[\frac{\pi/4}{2\pi}=\frac{\beta}{360^\circ}\quad\Rightarrow\quad \beta=\frac{\pi/4}{2\pi}\cdot 360^\circ=\frac{1}{8}\cdot 360^\circ=45^\circ \MDFPeriod\]
\end{MHint}
\end{MExercise}

\begin{MInfo}%
Die folgenden Winkelformen bekommen spezielle Namen:
\begin{itemize}
\item
Ein Winkel mit einem Ma\ss\ zwischen~$0$ und $\frac{\pi}{2}$ hei\ss t \MEntry{spitzer Winkel}{Winkel (spitz)}.
       
Ein Winkel mit einem Ma\ss\ von~$\frac{\pi}{2}$ hei\ss t \textbf{rechter Winkel}.
       
Ein Winkel mit einem Ma\ss\ zwischen~$\frac{\pi}{2}$ und $\pi$ hei\ss t \MEntry{stumpfer Winkel}{Winkel (stumpf)}.
       
Ein Winkel mit einem Ma\ss\ zwischen~$\pi$ und~$2 \pi$ hei\ss t \textbf{\"uberstumpfer Winkel}.

 \item Zwei Halbgeraden bilden eine Gerade, wenn sie einen Winkel vom Ma\ss~$\pi$ bilden.
 
 \item Zwei Halbgeraden \textbf{stehen senkrecht aufeinander}, wenn sie einen rechten Winkel bilden.
\end{itemize}
\end{MInfo}
  

Als n\"achstes sollen nun drei verschiedene Geraden betrachtet werden, von denen zwei parallel sind,
wohingegen die dritte nicht parallel zu diesen beiden ist. Es ergeben sich dann acht Schnittwinkel.
Je vier dieser Winkel sind gleich gro\ss .

\begin{MInfo}\MLabel{Mathematik_ElementareGeometrie_StufenwinkelWechselwinkel}%

\begin{center}
\MTikzAuto{%
\begin{tikzpicture}
\coordinate (G) at (0,0);
\coordinate (H) at ($ (G) + (30:5) $);
\coordinate (A) at ($ (G) + (30:1.5) $);
\coordinate (B) at ($ (H) + (A) - (G) $);
\coordinate (C) at ($ (G) + (0,2) $);
\coordinate (D) at ($ (C) + (B) - (A) $);
\coordinate (E) at (0,5);
\coordinate (F) at (5,1);
\coordinate (S) at (intersection of A--B and E--F);
\coordinate (T) at (intersection of C--D and E--F);
%
\draw[dotted] (S) circle [radius=0.9] (T) circle [radius=0.9];
\draw (A) -- node[at end, below] {$g$} (B) (C) -- node[at end, above left] {$h$} (D) (F) -- node[at end, below] {$j$} (E);
%
\begin{scope}[outer sep=4pt]
 \node at (S) [right] {$\alpha$};
 \node at (S) [above] {$\beta$};
 \node at (S) [left]  {$\gamma$};
 \node at (S) [below] {$\delta$};
 \node at (T) [right] {$\Mvarepsilon$};
 \node at (T) [above] {$\chi$};
 \node at (T) [left]  {$\Mvarphi$};
 \node at (T) [below] {$\psi$};
\end{scope}
\end{tikzpicture}
}
\end{center}

\begin{itemize}
  \item Die Winkel $\alpha, \gamma, \Mvarepsilon$ und $\Mvarphi$ sind gleich gro\ss , ebenso die Winkel $\beta, \delta, \chi$ und $\psi$.
  \item Dabei nennt man $\beta$ und $\psi$ bzw. $\gamma$ und $\Mvarepsilon$ \textbf{Wechselwinkel}.
  \item Die Winkel $\alpha$ und $\Mvarepsilon$ hei\ss en \textbf{Stufenwinkel}, ebenso $\beta$ und $\chi$, $\delta$ und $\psi$ und $\gamma$ und $\Mvarphi$.
\end{itemize}

\end{MInfo}

Eine einfache Figur mit Winkeln ist das Dreieck:
\begin{MInfo}\MLabel{Mathematik_ElementareGeometrie_SummeDerInnenwinkel}%
\begin{itemize}
 \item Ein \MEntry{Dreieck}{Dreieck} entsteht,
       wenn man drei Punkte, die nicht auf einer Geraden liegen, verbindet.
       
 \item Die drei Punkte, die verbunden werden, hei\ss en
       \MEntry{Ecken}{Ecken (Dreieck)} des Dreiecks, und
       die drei Verbindungslinien hei\ss en
       \MEntry{Seiten}{Seiten (Dreieck)} des Dreiecks.
       
 \item Je zwei Seiten des Dreiecks bilden je zwei Winkel.
   
       Der kleinere dieser beiden Winkel hei\ss t
       \MEntry{Innenwinkel}{Innenwinkel},
       und der gr\"o\ss ere der beiden Winkel hei\ss t
       \MEntry{Au\ss enwinkel}{Au\ss enwinkel}.
 
 \item Die Summe der drei Innenwinkel eines Dreiecks betr\"agt stets~$\pi$ bzw. $180^\circ$.
\end{itemize}
\end{MInfo}

\begin{tabular}{lr}
\begin{minipage}{10cm}
Man benennt die Ecken eines Dreiecks in mathematisch positiver Richtung mit lateinischen Gro\ss buchstaben. Die einem Punkt gegen\"uberliegende
Seite eines Dreiecks bekommt den entsprechenden Kleinbuchstaben zugeordnet, und
der Innenwinkel in einer Ecke erh\"alt den entsprechenden Kleinbuchstaben des
griechischen Alphabets.

Da die Au\ss enwinkel eines Dreiecks wesentlich weniger interessant sind als
die Innenwinkel, nennt man die Innenwinkel eines Dreiecks auch schlicht
\MEntry{Winkel}{Winkel (Dreieck)} des Dreiecks.
\end{minipage}
&
\begin {minipage}{6cm}
%\begin{center}
\MTikzAuto{%
\begin{tikzpicture}
\coordinate[label=below left:$A$] (A) at (0,0);
\coordinate[label=right:$B$]      (B) at (4,0.5);
\coordinate[label=above:$C$]      (C) at (2,3);
\coordinate (MAB) at ($ (A)!0.5!(B) $);
\coordinate (MBC) at ($ (B)!0.5!(C) $);
\coordinate (MCA) at ($ (C)!0.5!(A) $);
%
\draw (A) -- (B) -- (C) -- cycle;
%
\path (A) -- node[near start]{$\alpha$} (MBC) node[above right]{$a$};
\path (B) -- node[near start]{$\beta$}  (MCA) node[above left] {$b$};
\path (C) -- node[near start]{$\gamma$} (MAB) node[below]      {$c$};
%
\path let \p1 = (current bounding box.east),
          \p2 = (current bounding box.west),
          \p3 = ($ (\p1) - (\p2) $),
          \n3 = {veclen(\p3)} in;
%     (current bounding box.south) node [below, text width=\n3, text centered, outer sep = 0.5\baselineskip]
 %          {Die Bezeichnungen von Ecken, Seiten und Innenwinkeln in einem Dreieck.};
\end{tikzpicture}
}
\end{minipage}
\end{tabular}
%\qquad\qquad
%\ifttm%
%\MGraphicsSolo{Mathematik_ElementareGeometrie_Winkelsumme.png}{scale=1}
%\else%
%\tikzsetnextfilename{Winkelsumme}
%\begin{tikzpicture}
%\coordinate[label=below left:$A$] (A) at (0,0);
%\coordinate[label=right:$B$] (B)      at (4,0.5);
%\coordinate[label=above:$C$] (C)      at (2,3);
%\coordinate (MAB) at ($ (A)!0.5!(B) $);
%\coordinate (MBC) at ($ (B)!0.5!(C) $);
%\coordinate (MCA) at ($ (C)!0.5!(A) $);
%\coordinate (Hilfspunkt) at ($ (B) - (A) $);
%\coordinate (PA) at ($ (C) - (Hilfspunkt)!0.5!(0,0) $);
%\coordinate (PB) at ($ (C) + (Hilfspunkt)!0.5!(0,0) $);
%
%\draw (A) -- (B) -- (C) -- cycle;
%
%\draw[dotted] (PA) -- (PB);
%\path (C) -- node[near start]{$\delta$}      ($ (A)!0.6!(PA) $);
%\path (C) -- node[near start]{$\Mvarepsilon$} ($ (B)!0.6!(PB) $);
%
%\path (A) -- node[near start]{$\alpha$} (MBC) node[above right]{$a$};
%\path (B) -- node[near start]{$\beta$}  (MCA) node[above left] {$b$};
%\path (C) -- node[near start]{$\gamma$} (MAB) node[below]      {$c$};
%
%\path let \p1 = (current bounding box.east),
%          \p2 = (current bounding box.west),
%          \p3 = ($ (\p1) - (\p2) $),
%          \n3 = {veclen(\p3)} in
%      (current bounding box.south) node [below, text width=\n3, text centered, outer sep = 0.5\baselineskip]
%           {Skizze f\"ur den Beweis des Innenwinkelsatzes.};
%\end{tikzpicture}
%\fi
%\end{center}

%Der letzte Satz von Info~\MRef{Mathematik_ElementareGeometrie_SummeDerInnenwinkel}
%ist keine Definition, sondern eine Aussage. Als solche wollen wir sie beweisen.
%Wir rechnen die Innenwinkelsumme eines Dreiecks kurz aus.

%In der rechten Skizze ist die gepunktete Linie parallel zur
%Seite~$c$ des Dreiecks. Also sind $\alpha$ und $\delta$ bzw.\ 
%$\beta$ und $\Mvarepsilon$ Wechselwinkel und damit gleich gro\ss .

%Verl\"angert man die gepunktete Linie auf beiden Seiten ins
%Unendliche, so erh\"alt man eine Gerade und deshalb gilt
%\[
% \delta + \gamma + \Mvarepsilon = \pi.
%\]
%Wegen $\alpha = \delta$ und $\beta = \Mvarepsilon$ folgt die Behauptung.
\ \\
Da die Summe aller Winkel in einem Dreieck~$\pi$ betr\"agt, kann h\"ochstens
ein Winkel gleich oder gr\"o\ss er als~$\frac{\pi}{2}$ sein. Dadurch werden die
Dreiecke nach ihrem gr\"o\ss ten Winkel in drei verschiedene Klassen eingeteilt:
\begin{MInfo}%
\begin{itemize}
 \item Ein Dreieck, in dem alle Winkel kleiner als~$\frac{\pi}{2}$ sind, hei\ss t \MEntry{spitzwinklig}{Spitzwinklig (Dreieck)}.
 
 \item Ein Dreieck, das einen rechten Winkel enth\"alt, hei\ss t \MEntry{rechtwinklig}{Rechtwinklig (Dreieck)}.
 
       In einem rechtwinkligen Dreieck hei\ss en die Seiten, die auf den Schenkeln des
       rechten Winkels liegen, \MEntry{Katheten}{Kathete},
       und die Seite, die dem rechten Winkel gegen\"uberliegt, hei\ss t
       \MEntry{Hypotenuse}{Hypotenuse}.
       
 \item Ein Dreieck, das einen Winkel mit einem Ma\ss\ von \"uber~$\frac{\pi}{2}$
       besitzt, hei\ss t \textbf{stumpfwinklig}.
\end{itemize}
\end{MInfo}

Neben Dreiecken kann man nat\"urlich auch Figuren mit mehr Ecken betrachten:

\begin{MInfo}\MLabel{VBKM05_Vielecke}%
Es sei $n$ eine nat\"urliche Zahl, die mindestens $3$ betrage.
\begin{itemize}
 \item 
Ein \MEntry{$n$-Eck}{$n$-Eck} (auch als \MEntry{Vieleck}{Vieleck} oder \MEntry{Polygon}{Polygon} bezeichnet) entsteht,
      wenn man $n$ Punkte in der Ebene nacheinander in einer zuvor festgelegten Reihenfolge durch Strecken miteinander verbindet,
			wobei der letzte Punkt wieder mit dem ersten Punkt verbunden wird und je drei aufeinander folgende Punkte nicht auf einer gemeinsamen Geraden
			liegen d\"urfen.
       
 \item Die $n$ Punkte, die verbunden werden, hei\ss en
       \MEntry{Ecken}{Ecken ($n$-Eck)} des $n$-Ecks, und
       die $n$ Verbindungsstrecken hei\ss en
       \MEntry{Seiten}{Seiten ($n$-Eck)} des $n$-Ecks.
   \item Wenn sich die Seiten eines $n$-Ecks h\"ochstens in den Ecken schneiden, so spricht man von einem \MEntry{einfachen $n$-Eck}{einfaches $n$-Eck} (bzw. von einem \MEntry{einfachen Vieleck}{einfaches Vieleck} oder \MEntry{einfachem Polygon}{einfaches Polygon}).  
 \item Je zwei Seiten eines $n$-Ecks bilden je zwei Winkel.
Der kleinere dieser beiden Winkel hei\ss t
\MEntry{Innenwinkel}{Innenwinkel ($n$-Eck)},
       und der gr\"o\ss ere der beiden Winkel hei\ss t
       \MEntry{Au\ss enwinkel}{Au\ss enwinkel ($n$-Eck)}.
		\item Jedes einfache $n$-Eck l\"asst sich in $(n-2)$ einander nicht \"uberlappende Dreiecke zerlegen. Die Summe aller Innenwinkel eines einfachen $n$-Ecks betr\"agt daher stets $(n-2)\pi$ bzw. $(n-2)180^\circ$.
 \item Ein einfaches $n$-Eck, dessen Seiten alle die gleiche L\"ange haben und dessen Innenwinkel zudem alle gleich gro\ss\ sind, nennt man
\MEntry{regelm\"a\ss iges $n$-Eck}{regelm\"a\ss iges $n$-Eck} (bzw. \MEntry{regelm\"a\ss iges Vieleck}{regelm\"a\ss iges Vieleck} oder \MEntry{regelm\"a\ss iges Polygon}{regelm\"a\ss iges Polygon}).
\item Die Verbindungsstrecken zwischen je zwei Ecken, die nicht auf derselben Seite des $n$-Ecks liegen, hei\ss en die \MEntry{Diagonalen}{Diagonale ($n$-Eck)} des $n$-Ecks.
\end{itemize}
\end{MInfo}

Unter den Vielecken nehmen die Vierecke eine besondere Stellung ein, da diese in noch feinere Klassen eingeteilt werden.

\begin{MInfo}\MLabel{VBKM05_Vierecke}%
Ein Viereck hei\ss t
\begin{itemize}
 \item \MEntry{Trapez}{Trapez}, falls wenigstens zwei Seiten parallel sind;
 \item \MEntry{Parallelogramm}{Parallelogramm}, falls je zwei gegen\"uberliegende Seiten parallel sind;
 \item \MEntry{Raute}{Raute} oder \MEntry{gleichseitiges Viereck}{gleichseitiges Viereck} oder auch \MEntry{Rhombus}{Rhombus}, falls alle vier Seiten gleich lang sind;
 \item \MEntry{Rechteck}{Rechteck}, falls alle vier Winkel rechte Winkel sind;
 \item \MEntry{Quadrat}{Quadrat}, falls es ein Rechteck ist, bei dem alle Seiten gleich lang sind;
 \item \MEntry{Einheitsquadrat}{Einheitsquadrat}, falls es ein Quadrat mit Seitenl\"ange $1$ ist.
\end{itemize}
\end{MInfo}

Unter den gerade eingef\"uhrten speziellen Vierecken gibt es eine ganze Reihe von Zusammenh\"angen:

\begin{MInfo}\MLabel{VBKM05_Vierecke_Beziehungen}%
\begin{itemize}
 \item Jedes Quadrat ist ein Rechteck.
\item Jedes Quadrat ist eine Raute.
\item Jedes Rechteck ist ein Parallelogramm.
\item Jedes Parallelogramm ist ein Trapez.
\item Jede Raute ist ein Parallelogramm.
\end{itemize}
\end{MInfo}

Ferner lassen sich diese Vierecke auf vielerlei Arten charakterisieren.

\begin{MInfo}\MLabel{VBKM05_Parallelogramm}%
Ein Viereck ist genau dann ein Parallelogramm, wenn
\begin{itemize}
 \item es ein einfaches Viereck ist, dessen gegen\"uberliegenden Seiten jeweils gleich lang sind;
\item gegen\"uberliegende Innenwinkel gleich gro\ss\ sind;
\item zwei benachbarte Innenwinkel zusammen $\pi$ bzw. $180^\circ$ ergeben; 
\item die Diagonalen einander halbieren.
\end{itemize}
\end{MInfo}


\begin{MInfo}\MLabel{VBKM05_Raute}%
Ein Viereck ist genau dann eine Raute, wenn
\begin{itemize}
 \item es ein Parallelogramm ist, in welchem die Diagonalen senkrecht zueinander stehen;
 \item es ein Parallelogramm ist, in welchem die Diagonalen einander halbieren.
\end{itemize}
\end{MInfo}


\begin{MInfo}\MLabel{VBKM05_Rechteck}%
Ein Viereck ist genau dann ein Rechteck, wenn
\begin{itemize}
 \item alle Innenwinkel gleich gro\ss\ sind;
\item es ein Parallelogramm ist, bei dem wenigstens ein Innenwinkel ein rechter Winkel ist;
\item es ein Parallelogramm ist, bei dem die Diagonalen gleich lang sind; 
\item die Diagonalen einander halbieren und gleich lang sind;
\item es eine Raute mit gleich langen Diagonalen ist;
\item die Diagonalen einander halbieren und wenigstens ein Innenwinkel ein rechter Winkel ist.
\end{itemize}
\end{MInfo}


\begin{MInfo}\MLabel{VBKM05_Quadrat}%
Ein Viereck ist genau dann eine Quadrat, wenn
\begin{itemize}
\item alle Seiten gleich lang und alle Winkel gleich gro\ss\ sind;
 \item die Diagonalen einander halbieren und gleich lang sind und zudem alle Seiten gleich lang sind;
 \item es ein Parallelogramm ist, in welchem die Diagonalen senkrecht zueinander stehen und welches wenigstens einen rechten Winkel besitzt;
 \item es ein Parallelogramm ist, in welchem die Diagonalen einander halbieren und welches wenigstens einen rechten Winkel besitzt;
 \item es ein Parallelogramm ist, in welchem die Diagonalen senkrecht zueinander stehen und alle Innenwinkel gleich gro\ss\ sind;
 \item es ein Parallelogramm ist, in welchem die Diagonalen einander halbieren und alle Innenwinkel gleich gro\ss\ sind;
\item es sowohl eine Raute als auch ein Rechteck ist.
\end{itemize}
\end{MInfo}

\end{MXContent}



\begin{MXContent}{Kongruenzs\"atze}{Kongruenz}{weiterfhrg}
\MDeclareSiteUXID{VBKM05_Kongruenzsaetze}

Zu einem Dreieck geh\"oren drei Seitenl\"angen und drei Winkel, also sechs Gr\"o\ss en. Wenn bei zwei Dreiecken alle diese Gr\"o\ss en \"ubereinstimmen, so sind diese Dreiecke \textbf{kongruent} oder deckungsgleich, dabei spielt es keine Rolle, wo sich die Dreiecke befinden. Kongruente Dreiecke k\"onnen also durch Drehung, Spiegelung und Verschiebung ineinander \"uberf\"uhrt werden.\\
Kennt man vier von den sechs Gr\"o\ss en, so ist das Dreieck eindeutig bestimmt bis auf Spielgelung oder Drehung, das hei\ss t bis auf die Lage des Dreiecks im Raum. Alle Dreiecke, die man mit diesen Angaben erh\"alt, sind dann kongruent. In einigen F\"allen gen\"ugen sogar drei Angaben, um das Dreieck eindeutig zu bestimmen.
Diese F\"alle werden mit den \MEntry{Kongruenzs\"atzen}{Kongruenzs\"atze} beschrieben:

\begin{MInfo}\MLabel{Mathematik_ElementareGeometrie_Satz:Kongruenzsaetze}%
Ein Dreieck ist eindeutig bestimmt, wenn
\begin{itemize}
 \item von den drei Winkeln und den drei Seitenl\"angen
       mindestens vier Angaben gegeben sind.
 
 \item alle drei Seitenl\"angen gegeben sind.
       (Diesen Satz bezeichnet man gerne mit \glqq sss\grqq\ f\"ur \glqq Seite, Seite, Seite\grqq.)
 
 \item eine Seitenl\"ange und ihre Winkel zu den anderen Seiten gegeben
       sind (\glqq wsw\grqq\ f\"ur \glqq Winkel, Seite, Winkel\grqq).
        
 \item zwei Seitenl\"angen und der von den Seiten eingeschlossenen
       Winkel gegeben sind (\glqq sws\grqq\ f\"ur \glqq Seite, Winkel, Seite\grqq).
       
             
 \item ein Winkel und zwei Seitenl\"angen so gegeben sind,
       dass nur eine der Seiten auf einem Schenkel des Winkels liegt und die andere gegebene Seite die l\"angere der beiden gegebenen Seiten ist.
       
       (Diesen Satz bezeichnet man mit \glqq Ssw\grqq\ f\"ur \glqq Seite, Seite, Winkel\grqq,
        wobei das gro\ss\ geschriebene \glqq S\grqq\ signalisieren soll, dass die
        dem Winkel gegen\"uberliegende Seite die l\"angere Seite darstellt.)
\end{itemize}
\end{MInfo}
 
Hat man von einem Dreieck nur zwei oder drei Angaben, die 
keinem der oben angegebenen F\"alle entsprechen, gegeben, so gibt es verschiedene
 Dreiecke, f\"ur die die Angaben zutreffen.

%%\ifttm\relax\else\newpage\fi

\begin{MExample}
\begin{tabular}{lr}
\begin{minipage}{10cm}
Gegeben seien der Winkel~$\alpha$ und die Seiten $b$ und~$c$.
Das Dreieck \glqq sws\grqq\ erh\"alt man, indem man zun\"achst eine Seite, hier zum Beispiel die Seite $c$, zeichnet und an der
nach der Bezeichnungskonvention korrekten Ecke ($A$) den Winkel $\alpha$ anf\"ugt.
Dann schl\"agt man um diese Ecke einen Kreis, dessen Radius der
L\"ange der zweiten Seite (hier $b$) entspricht. Der Schnittpunkt dieses
Kreises mit dem zweiten Schenkel des Winkels bildet die dritte
Ecke des Dreiecks ($C$).
\end{minipage}
&
\begin{minipage}{7cm}
%\begin{center}
\MTikzAuto{%
\begin{tikzpicture}
\coordinate [label=left:$A$]        (A) at (0,0);
\coordinate [label=below right:$B$] (B) at ($ (A) + (-15:3.2) $);
\coordinate [label=above:$C$]       (C) at ($ (A) + (60:2) $);
%
\draw (A) -- node[below left]{1.} (B) -- node[above right] {4.} (C) -- cycle;
\draw[dotted] (C) -- ($ (C)!-0.5!(A) $) node[below right]{2.};
\node at (A) [label=135:3., draw, dotted, circle through=(C)]{};
\end{tikzpicture}
}
\end{minipage}
\end{tabular}
\end{MExample}


\begin{MExercise}
Konstruieren Sie ein Dreieck mit der Seite $c=5$ und den Winkeln $\alpha=30^\circ$ und $\beta=120^\circ$.

\begin{MHint}{L\"osung}
\begin{tabular}{lr}
\begin{minipage}{9cm}
Man zeichnet zuerst die gegebene Strecke $c$.
Dann tr\"agt man an den beiden Enden der Strecke die zwei der Bezeichnungskonvention entsprechenden Winkel an.
Der Schnittpunkt der beiden neuen Schenkel ist die dritte Ecke des Dreiecks.
\end{minipage}
&
\begin{minipage}{7cm}
\MTikzAuto{%
\begin{tikzpicture}[scale=0.75]
\coordinate [label=left:$A$]        (A) at (0,0);
\coordinate [label=below right:$B$] (B) at ($ (A) + (10:4) $);
\coordinate [label=above left:$C$]  (C) at ($ (A) + (40:7) $);
%
\draw (A) -- node[below]{1.} (B) -- (C) -- cycle;
\draw[dotted] (C) -- ($ (C)!-0.5!(A) $) node[below right]{2.};
\draw[dotted] (C) -- ($ (C)!-0.5!(B) $) node[left]{3.};
\end{tikzpicture}
}
\end{minipage}
\end{tabular}

\end{MHint}
\end{MExercise}

\begin{MExample}\MLabel{Mathematik_ElementareGeometrie_Beispiel:Aehnlichkeit}

Gegeben seien nun die drei Winkel $\alpha=77^\circ$, $\beta=44^\circ$ und $\gamma=59^\circ$.
Diese Angaben findet man nicht bei den Kongruenzs\"atzen \MRef{Mathematik_ElementareGeometrie_Satz:Kongruenzsaetze}. Dennoch kann man entsprechende Dreiecke konstruieren:
\begin{center}
\MTikzAuto{%
\begin{tikzpicture}[x=1.0cm, y=1.0cm] 
\pgfmathparse{5*sin(44)*cos(77)/sin(59)}\let\cx=\pgfmathresult
\pgfmathparse{5*sin(44)*sin(77)/sin(59)}\let\cy=\pgfmathresult
\foreach \sx/\sy/\dsf/\ang in {0.0cm/0.0cm/1.0/-15,6.0cm/2.0cm/0.4/40,8.0cm/0.0cm/0.5/-15} {
\begin{scope}[xshift=\sx,yshift=\sy,rotate=\ang]
\coordinate (OB) at (5,0);
\coordinate (OC) at (\cx,\cy);
\coordinate (A) at (0,0);
\coordinate (B) at ($ (A)!\dsf!(OB) $);
\coordinate (C) at ($ (A)!\dsf!(OC) $);
\coordinate (MAB) at ($ (A)!0.5!(B) $);
\coordinate (MBC) at ($ (B)!0.5!(C) $);
\coordinate (MCA) at ($ (C)!0.5!(A) $);
\draw[black,thick] (A) -- (B) -- (C) -- cycle;
\pgfmathparse{0.15/\dsf}
\path (A) -- node[pos=\pgfmathresult]{$\alpha$} (MBC);
\path (B) -- node[pos=\pgfmathresult]{$\beta$}  (MCA);
\path (C) -- node[pos=\pgfmathresult]{$\gamma$} (MAB);
\end{scope}
}
\end{tikzpicture}
}
%\ifttm%
%-%\MGraphicsSolo{Mathematik_ElementareGeometrie_www_1.png}{scale=1}
%\else%
%\tikzsetnextfilename{wwGross}
%\begin{tikzpicture}
% \coordinate[label=left:$A_1$]       (A) at (0,0);
% \coordinate[label=right:$B_1$]      (B) at (4,-1);
% \coordinate[label=above left:$C_1$] (C) at (1.5,3);
% \draw (A) -- (B) -- (C) -- cycle;
%\end{tikzpicture}
%\fi
%\qquad\qquad
%\ifttm%
%\MGraphicsSolo{Mathematik_ElementareGeometrie_wwKlein.png}{scale=1}
%\else%
%\tikzsetnextfilename{wwKlein}
%\begin{tikzpicture}[scale=0.5]
% \coordinate[label=left:$A_3$]       (D) at (7,0);
% \coordinate[label=right:$B_3$]      (E) at (9,-0.5);
% \coordinate[label=above left:$C_3$] (F) at (7.75,1.5);
% \draw (D) -- (E) -- (F) -- cycle;

%\end{tikzpicture}
%\fi
\end{center}
Es stellt sich heraus, dass es sogar unendlich viele derartige Dreiecke gibt, die nicht kongruent zueinander sind, also nicht durch Drehung oder Spiegelung ineinander \"ubergef\"uhrt werden k\"onnen.
\end{MExample}

Allerdings sehen diese Dreiecke irgendwie \"ahnlich aus. Solche \textbf{\"ahnlichen} Dreiecke erh\"alt man auch, wenn man zum Beispiel die Verh\"altnisse aller Seiten zueinander kennt.

\begin{MInfo}%
\textbf{\"Ahnlichkeitss\"atze f\"ur Dreiecke}\\
\ \\
Zwei Dreiecke sind zueinander \"ahnlich, wenn
\begin{itemize}
 \item	sie in zwei (und damit wegen der Winkelinnensumme in drei) Winkeln \"ubereinstimmen.
 \item  sie in allen \textbf{Verh\"altnissen} ihrer entsprechenden Seiten \"ubereinstimmen.
 \item  sie in einem Winkel und im \textbf{Verh\"altnis} der anliegenden Seiten \"ubereinstimmen.
 \item  sie im \textbf{Verh\"altnis} zweier Seiten und im Gegenwinkel der gr\"o\ss eren Seite \"ubereinstimmen.
\end{itemize}
\end{MInfo}

Eine Besonderheit gibt es bei dem rechten und dem linken Dreieck in Beispiel  \MRef{Mathematik_ElementareGeometrie_Beispiel:Aehnlichkeit}: Hier geht das eine Dreieck durch zentrische Streckung \MLabel{Mathematik_ElementareGeometrie_zentrischeStreckung} mit dem Streckzentrum $S$ und einem Streckfaktor $k$ in das andere \"uber.
\begin{center}
\MTikzAuto{%
\begin{tikzpicture}[x=1.0cm, y=1.0cm] 
\pgfmathparse{5*sin(44)*cos(77)/sin(59)}\let\cx=\pgfmathresult
\pgfmathparse{5*sin(44)*sin(77)/sin(59)}\let\cy=\pgfmathresult
\pgfmathparse{cos(15)}\let\rc=\pgfmathresult
\pgfmathparse{sin(15)}\let\rs=\pgfmathresult
\foreach \ax/\ay/\lsf/\rsf in {\cx/\cy/-1/14,5/0/-2/15,0/0/-0.5/13.8} {
  \pgfmathparse{\lsf+(12-\lsf)/12*(\ax*\rc+\ay*\rs)}\let\cax\pgfmathresult
  \pgfmathparse{(12-\lsf)/12*(-\ax*\rs+\ay*\rc)}\let\cay\pgfmathresult
  \pgfmathparse{\rsf+(12-\rsf)/12*(\ax*\rc+\ay*\rs)}\let\cbx\pgfmathresult
  \pgfmathparse{(12-\rsf)/12*(-\ax*\rs+\ay*\rc)}\let\cby\pgfmathresult
  \draw[gray, dashed] (\cax,\cay) -- (\cbx,\cby);
}
\node[anchor=south] at (12,0) {$S$};
\foreach \sx/\sy/\dsf/\ang in {0.0cm/0.0cm/1.0/-15,4.8cm/0.0cm/0.60/-15} {
  \begin{scope}[xshift=\sx,yshift=\sy,rotate=\ang]
  \coordinate (OB) at (5,0);
  \coordinate (OC) at (\cx,\cy);
  \coordinate (A) at (0,0);
  \coordinate (B) at ($ (A)!\dsf!(OB) $);
  \coordinate (C) at ($ (A)!\dsf!(OC) $);
  \coordinate (MAB) at ($ (A)!0.5!(B) $);
  \coordinate (MBC) at ($ (B)!0.5!(C) $);
  \coordinate (MCA) at ($ (C)!0.5!(A) $);
  \draw[black,thick] (A) -- (B) -- (C) -- cycle;
  \pgfmathparse{0.15/\dsf}
  \path (A) -- node[pos=\pgfmathresult]{$\alpha$} (MBC);
  \path (B) -- node[pos=\pgfmathresult]{$\beta$}  (MCA);
  \path (C) -- node[pos=\pgfmathresult]{$\gamma$} (MAB);
  \end{scope}
}
\end{tikzpicture}
}
\end{center}

\end{MXContent}

\begin{MExercises}
\MDeclareSiteUXID{VBKM05_Kongruenzsaetze_Exercises}
\begin{MExercise}

Geben Sie die in der folgenden Tabelle fehlenden Werte an! Dabei soll in
einer Spalte stets der gleiche Winkel stehen.

\ifttm%
(Ein~$\pi$ geben Sie als \glqq\texttt{pi}\grqq\ ein. Bitte runden Sie Ihre Ergebnisse auf drei Nachkommastellen!)
\fi

\begin{center}
\begin{tabular}{l*{6}{c}}
 Bogenma\ss\ & $\pi$               &   \MLParsedQuestion{10}{9*pi/5}{3}{GEO2} &   	$\frac{2 \pi}{3}$                &\MLParsedQuestion{10}{3*pi/2}{3}{GEO3} & $\frac{11 \pi}{12}$\\
 Gradma\ss\   & \MLParsedQuestion{10}{180}{3}{GEO4}      & $324$    & \MLParsedQuestion{10}{120}{3}{GEO5}        &    $270$    & \MLParsedQuestion{10}{165}{3}{GEO6}    \\
\end{tabular}
\end{center}
\end{MExercise}

\begin{MExercise}
Untersuchen Sie die folgende Figur auf Stufenwinkel und Wechselwinkel!
\begin{center}
\MTikzAuto{%
\begin{tikzpicture}
\coordinate (A) at (0,0);
\coordinate (B) at ($ (A) + ( 00:3) $);
\coordinate (C) at ($ (B) + ( 60:3) $);
\coordinate (D) at ($ (C) + (120:3) $);
\coordinate (E) at ($ (D) + (180:3) $);
\coordinate (F) at ($ (E) + (240:3) $);
\coordinate (AB) at (intersection of A--C and B--F);
\coordinate (BC) at (intersection of B--D and C--A);
\coordinate (CD) at (intersection of C--E and D--B);
\coordinate (DE) at (intersection of D--F and E--C);
\coordinate (EF) at (intersection of E--A and F--D);
\coordinate (FA) at (intersection of F--B and A--E);
%
\draw (A) -- (C) -- (E) -- cycle;
\draw (B) -- (D) -- (F) -- cycle;
%
\draw (AB) -- (DE);
\draw (BC) -- (EF);
\draw (CD) -- (FA);
\end{tikzpicture}
}
\end{center}

\begin{MHint}{L\"osung}
\begin{tabular}{lc}
\begin{minipage}[b]{9cm}
Die Winkel $\alpha$ und $\alpha'$ zum Beispiel sind Stufenwinkel, ebenso $\beta$ und $\beta'$.\\
\ \\
Die Winkel $\alpha'$ und $\beta$ zum Beispiel sind Wechselwinkel, ebenso $\alpha$ und $\beta'$.
\end{minipage}
&
\MTikzAuto{%
\begin{tikzpicture}
\coordinate (A) at (0,0);
\coordinate (B) at ($ (A) + ( 00:3) $);
\coordinate (C) at ($ (B) + ( 60:3) $);
\coordinate (D) at ($ (C) + (120:3) $);
\coordinate (E) at ($ (D) + (180:3) $);
\coordinate (F) at ($ (E) + (240:3) $);
\coordinate (AB) at (intersection of A--C and B--F);
\coordinate (BC) at (intersection of B--D and C--A);
\coordinate (CD) at (intersection of C--E and D--B);
\coordinate (DE) at (intersection of D--F and E--C);
\coordinate (EF) at (intersection of E--A and F--D);
\coordinate (FA) at (intersection of F--B and A--E);
%
\draw (A) -- (C) -- (E) -- cycle;
\draw (B) -- (D) -- (F) -- cycle;
%
\draw (AB) -- (DE);
\draw (BC) -- (EF);
\draw (CD) -- (FA);
\draw[color=black, thin] (DE) ++(-30:0.65) arc (-30:30:0.65);
\draw[color=black] (DE) ++(0:0.45) node {\small $\alpha'$};
\draw[color=black, thin] (DE) ++(150:0.65) arc (150:210:0.65);
\draw[color=black] (DE) ++(0:-0.45) node {\small $\beta'$};
\draw[color=black, thin] (EF) ++(-30:0.65) arc (-30:30:0.65);
\draw[color=black] (EF) ++(0:0.45) node {\small $\alpha$};
\draw[color=black, thin] (CD) ++(150:0.65) arc (150:210:0.65);
\draw[color=black] (CD) ++(0:-0.45) node {\small $\beta$};
\end{tikzpicture}
}
\end{tabular}

\end{MHint}
\end{MExercise}


\begin{MExercise}
Zeigen Sie mithilfe von Wechselwinkeln, dass die Summe der Innenwinkel in einem Dreieck stets $\pi$ betr\"agt.

\begin{MHint}{L\"osung}
\begin{tabular}{lc}
\MTikzAuto{%
\begin{tikzpicture}[x=1.0cm, y=1.0cm] 
%%\draw[help lines, gray!50, xstep=0.5, ystep=0.5] (0,0) grid (9,8);
\draw[color=black] (1,0)--(9,4) (0.5,3.5)--(7.5,7.0);
\draw[color=black, very thick] (2,0.5) -- (7.5,3.25) -- (4.0,5.25) -- cycle;
\draw[color=black, thin] (2,0.5) ++(26.5660:1.2) arc (26.5650:67.1663:1.2);
\draw[color=black] (2,0.5) ++(46.865:0.8) node {\large $\alpha$};
\draw[color=black, thin] (7.5,3.25) ++(150.255:1.2) arc (150.255:205.565:1.2);
\draw[color=black] (7.5,3.25) ++(177.910:0.8) node {\large $\beta$};
\draw[color=black, thin] (4.0,5.25) ++(247.1663:0.9) arc (247.1663:330.255:0.9);
\draw[color=black] (4.0,5.25) ++(288.7107:0.6) node {\large $\gamma$};
\draw[color=black, thin] (4.0,5.25) ++(206.5660:1.2) arc (206.5650:247.1663:1.2);
\draw[color=black] (4.0,5.25) ++(226.865:0.8) node {\large $\alpha'$};
\draw[color=black, thin] (4.0,5.25) ++(-29.745:1.2) arc (-29.745:26.5660:1.2);
\draw[color=black] (4.0,5.25) ++(-1.5895:0.8) node {\large $\beta'$};
\draw[color=black] (5.75,4.25) node[anchor=south west] {\large $a$};
\draw[color=black] (3.0,2.875) node[anchor=south east] {\large $b$};
\draw[color=black] (4.75,1.875) node[anchor=north west] {\large $c$};
\end{tikzpicture}
}
&
\begin{minipage}[b]{9cm}
Zeichnet man parallel zur Seite $c$ eine Gerade durch die obere Ecke des Dreiecks, so erh\"alt man jeweils einen Wechselwinkel $\alpha'$ zu $\alpha$ und $\beta'$ zu $\beta$. \\
\ \\
An der Geraden gilt
\[\alpha'+\gamma+\beta'=\pi\MDFPeriod\]
Da $\alpha'=\alpha\quad\text{und}\quad\beta'=\beta\quad\text{folgt}\quad\alpha+\gamma+\beta=\pi$.
\ \\
\end{minipage}
\end{tabular}
\end{MHint} 
\end{MExercise}
\end{MExercises}



\MSubsection{Fl\"acheninhalt und Strahlens\"atze}
\MLabel{M05_Flaecheninhalt}

\begin{MIntro}
\MDeclareSiteUXID{VBKM05_Flaecheninhalt_Intro}
Das Bauamt von Ludwigshafen m\"ochte die Grundst\"ucksgr\"o\ss en des im zuvor betrachteten
Ausschnitt des Stadtplans (Bild \MRef{Stadtplan}) abgebildeten Viertels neu berechnen.


Die meisten der Grundst\"uck haben eine (ann\"ahernd) vieleckige Grundfl\"ache,
die sich, wie man zeigen kann, in Dreiecke aufspalten l\"asst. Das Bauamt teilt diese Grundst\"ucke
also in Dreiecke auf, berechnet die Fl\"acheninhalte der Dreiecke und erh\"alt
so die Gesamtfl\"ache der Grundst\"ucke durch die Addition der Dreiecksfl\"acheninhalte.

In diesem Abschnitt wird wiederholt, wie man den Umfang und Fl\"acheninhalt eines Dreiecks und anderer Figuren berechnet. 

Im Anschluss daran werden die sogenannten Strahlens\"atze eingef\"uhrt,
die zum Beispiel bei der Skalierung einer Zeichnung oder bei der Berechnung
von H\"ohen bzw.\ Abst\"anden zum Einsatz kommen.
\end{MIntro}

\ifttm\relax\else\newpage\fi

\begin{MXContent}{Fl\"acheninhalt}{Fl\"acheninhalt}{STD}
\MDeclareSiteUXID{VBKM05_Flaecheninhalt_Content}
Der Inhalt einer Fl\"ache ist die Zahl der Einheitsquadrate, die man ben\"otigt, um diese Fl\"ache vollst\"andig zu bedecken.\\
\ \\
Zuerst sollen Rechtecke betrachtet werden.
%\begin{MInfo}
%Ein Rechteck ist ein Viereck, bei dem alle vier Innenwinkel rechte Winkel sind.
%\end{MInfo}

\begin{tabular}{lc}
\MTikzAuto{%
\begin{tikzpicture}[x=0.8cm, y=0.8cm] 
\draw[help lines, black, xstep=1, ystep=1] (1,1) grid (8,5);
\draw[color=blue, line width=2pt] (1,1)--(8,1) (1,5)--(8,5);
\draw[color=red, line width=2pt] (1,1)--(1,5) (8,1)--(8,5);
\draw[color=blue] (4.5,1) node[anchor=north] {\large $a$};
\draw[color=blue] (4.5,5) node[anchor=south] {\large $a$};
\draw[color=red] (1,3) node[anchor=east] {\large $b$};
\draw[color=red] (8,3) node[anchor=west] {\large $b$};
\end{tikzpicture}
}
&
\begin{minipage}[b]{10cm}
 Wenn ein Rechteck eine Seite der L\"ange $a$ und eine Seite der L\"ange $b$ hat, dann gibt es $b$ Reihen mit $a$ Einheitsquadraten, also $a\cdot b$ Einheitsquadrate.\\
 \ \\
\end{minipage}
\end{tabular}

\begin{MInfo}
 Die Fl\"ache $A$ des Rechtecks ist 
 \[A=b\cdot a=a\cdot b \MDFPeriod\]
\end{MInfo}

\begin{tabular}{lr}
\begin{minipage}{10cm}
Damit l\"asst sich nun auch leicht der Fl\"acheninhalt eines rechtwinkligen Dreiecks berechnen.
Es sei~$ABC$ ein rechtwinkliges Dreieck, welches um~$180^\circ$ gedreht werde. Legt man anschlie\ss end das urspr\"ungliche und das neue Dreieck
entlang der beiden Hypotenusen aneinander, so erh\"alt man ein Rechteck.
\end{minipage}
&
\begin{minipage}{5cm}
%\begin{center}
\MTikzAuto{%
\begin{tikzpicture}[rotate=-20]
\coordinate (A) at (0,0);
\coordinate (B) at ($ (A) + (1,-1.5) $);
\coordinate (C) at ($ (A) + (3, 2) $);
\coordinate (D) at ($ (B) + (C) - (A)$);
% \coordinate[label=right:$B_1$]      (B) at (4,-1);
\draw (A) node [left]{$A$} -- (B) node[left]{$B$} -- (C) node[right]{$C$} -- cycle;
\draw[dotted] (B) -- (D) node[right]{$D$} -- (C);
\end{tikzpicture}
}
%\end{center}
\end{minipage}
\end{tabular}

Der Fl\"acheninhalt des Dreiecks ist nun die H\"alfte des Fl\"acheninhaltes des Rechtecks, also $A=\frac{1}{2}\cdot a\cdot b$

Doch was ist zu tun,
wenn das Dreieck nicht rechtwinklig ist?
%Der Stadtplan gibt uns hier einen entscheidenden Tipp:

%\begin{center}
%\MGraphicsSolo{../Bilder/Mathematik_ElementareGeometrie/Mathematik_ElementareGeometrie_Stadtplan_Flaecheninhalt.png}{scale=1}
%\MGraphicsSolo{Mathematik_ElementareGeometrie_Stadtplan_Flaecheninhalt.png}{scale=1}
%\end{center}

%Die Sch\"utzenstra\ss e bildet mit der Rottstra\ss e und der Wittelsbachstra\ss e ein Dreieck. Durch den kurzen Weg von der Ecke Sch\"utzenstra\ss e und Rottstra\ss e auf die Wittelsbachstra\ss e wird dieses Dreieck in zwei rechtwinklige Dreiecke zerteilt.

Aus jedem beliebigen Dreieck kann man zwei rechtwinklige Dreiecke gewinnen, indem man von einer Ecke aus eine Linie auf die gegen\"uberliegende Seite zieht, so dass sie diese senkrecht trifft. Diese Linie nennt man die \textbf{H\"ohe} $h_{i}$ eines Dreiecks auf die bestimmte Seite $i$, wobei der Index $i$ derjenigen Seite $a$, $b$ oder $c$ entspricht, \"uber der die H\"ohe bestimmt wird.

Je nachdem, ob die neue Linie innerhalb oder au\ss erhalb des Dreiecks liegt, ergibt sich der Fl\"acheninhalt des Dreiecks dann aus der Summe oder der Differenz der Fl\"acheninhalte der beiden sich ergebenden rechtwinkligen Dreiecke:
\begin{center}
\MTikzAuto{%
\begin{tikzpicture}
 \coordinate[label=left:$A$]  (A) at (0,0);
 \coordinate[label=right:$B$] (B) at ($ (A) + (3,0) $);
 \coordinate[label=below:$D$] (D) at ($ (A)!0.3!(B) $);
 \coordinate[label=above:$C$] (C) at ($ (D) + (0,2) $);
 %
 \draw (A) -- node[below]{$c$} (B) -- (C) -- cycle;
 \draw[dotted] (C) -- node[right]{$h_c$} (D);
 \path (A) -- node[above]{$c_1$} (D) -- node[above]{$c_2$} (B);
\end{tikzpicture}
}
\hspace{4em}
\MTikzAuto{%
\begin{tikzpicture}
 \coordinate[label=below:$U$] (U) at (0,0);
 \coordinate[label=right:$V$] (V) at ($ (U) + (3,0) $);
 \coordinate[label=below:$X$] (X) at ($ (U)!-0.3!(V) $);
 \coordinate[label=above:$W$] (W) at ($ (X) + (0,2) $);
 %
 \draw (U) -- node[above]{$w$} (V) -- (W) -- cycle;
 \path (X) -- node[below]{$w_2$} (V);
 \draw[dotted] (W) -- node[left] {$h_w$} (X);
 \draw[dotted] (U) -- node[above]{$w_1$} (X);
\end{tikzpicture}
}
\end{center}

Links gilt also (wenn $A_{\Delta}$ den Fl\"acheninhalt des Dreiecks~$\Delta$ bezeichnet)
\[
   A_{ABC}
 = A_{DBC} + A_{ADC}
 = \Mtfrac{1}{2} \cdot h_c \cdot c_2 + \Mtfrac{1}{2} \cdot h_c \cdot c_1
 = \Mtfrac{1}{2} \cdot h_c \cdot \left( c_2 + c_1 \right)
 = \Mtfrac{1}{2} \cdot h_c \cdot c \MDFPeriod
\]
Rechts gilt genauso
\[
   A_{UVW}
 = A_{XVW} - A_{XUW}
 = \Mtfrac{1}{2} \cdot h_w \cdot w_2 - \Mtfrac{1}{2} \cdot h_w \cdot w_1
 = \Mtfrac{1}{2} \cdot h_w \cdot \left( w_2 - w_1 \right)
 = \Mtfrac{1}{2} \cdot h_w \cdot w \MDFPeriod
\]

\begin{MInfo}%
\begin{itemize}
 \item Die \MEntry{H\"ohe eines Dreiecks auf einer Seite}{H\"ohe} ist die 
       Strecke, die von dem der Seite gegen\"uberliegenden Punkt ausgeht und die Gerade,
       auf der die Seite liegt, im rechten Winkel trifft. Der Punkt, auf dem die H\"ohe diese Gerade trifft, hei\ss t
       \MEntry{Lotfu\ss punkt}{Lotfu\ss punkt} der H\"ohe.
 
 \item Der Fl\"acheninhalt eines Dreiecks berechnet sich aus der H\"alfte des Produkts
       der L\"ange einer Seite mit der L\"ange der zugeh\"origen H\"ohe des Dreiecks
       \[
          A_{ABC}
        = \frac{a \cdot h_a}{2}
        = \frac{b \cdot h_b}{2}
        = \frac{c \cdot h_c}{2} \MDFPeriod
       \]
\end{itemize}
\end{MInfo}


\begin{MExample}
\begin{tabular}{lc}
%-%\MGraphicsSolo{Mathematik_ElementareGeometrie_DreieckFlaeche_Bsp.png}{scale=1}
\MTikzAuto{%
\begin{tikzpicture}[x=0.6cm, y=0.6cm] 
%%\draw[help lines, gray!50, xstep=0.5, ystep=0.5] (0,0) grid (9,8);
\draw[color=black, very thick] (0,0) -- (1.7032,-6.0654) -- (7.8,0) -- cycle;
\draw[color=black, thick] (0,0) -- (3.87986,-3.89994);
\draw[color=black] (3.9,0) node[anchor=south] {\large $7{,}8$};
\draw[color=black] (0.85160,-3.0327) node[anchor=north east] {\large $6{,}3$};
\draw[color=black] (4.7516,-3.0327) node[anchor=north west] {\large $8{,}6$};
\draw[color=black] (1.93993,-1.94997) node[anchor=south west] {\large $5{,}5$};
\end{tikzpicture}
}
&
\begin{minipage}[b]{10cm}
Bei dem hier gezeigten Dreieck ist die H\"ohe gegeben, die zur Seite mit dem Wert $\MZahl{8}{5}$ geh\"ort.
(Bei den Angaben handelt es sich jeweils um gerundete numerische Werte.)
Der Fl\"acheninhalt des Dreiecks ist also
\[A=\frac{\MZahl{8}{6}\cdot\MZahl{5}{5}}{2}=\MZahl{23}{65} \MDFPeriod\]
\ \\
\end{minipage}\\
\end{tabular}
\end{MExample}

\begin{MExercise}
Berechnen Sie den Fl\"acheninhalt des Dreiecks:\\
\ \\
%-%\MGraphicsSolo{Mathematik_ElementareGeometrie_DreieckFlaeche_Aufgabe.png}{scale=1}
\MTikzAuto{%
\begin{tikzpicture}[x=1.2cm, y=1.2cm] 
%Koordinatensystem
\node (xMAX) at (6.0,0){};
\node (yMAX) at (0,3.2){};
\draw[help lines, gray, dashed, xstep=1, ystep=1] (0,0) grid (5.5,2.8);
\draw[->,color=black] (-0.4,0) -- (xMAX);
\foreach \x in {1, 2, 3, 4, 5}
\draw[shift={(\x,0)},color=black] (0pt,2pt) -- (0pt,-2pt) node[below] {\normalsize $\x$};
\draw[->,color=black] (0,-0.4) -- (yMAX);
\foreach \y in {1,2}
\draw[shift={(0,\y)},color=black] (2pt,0pt) -- (-2pt,0pt) node[left] {\normalsize $\y$};
%Achsenbeschriftung
\draw (xMAX) node[anchor=north east] {$x$};
\draw (yMAX) node[anchor=north east] {$y$};
%Beschriftung und Graphen
%%\clip(-2.8,-0.5) rectangle (6,3);
\draw[color=black, very thick] (1,0) -- (4,2) -- (5,0) -- cycle;
%%\draw[color=black] (7,3) node[anchor=south west] {$\MPointTwo{7}{3}$};
\end{tikzpicture}
}
\begin{MHint}{L\"osung}
An diesem Dreieck l\"asst sich die zur Seite, die auf der $x$-Achse liegt, zugeh\"orige H\"ohe ablesen:
\[\text{Seite}=4, \quad \text{H\"ohe}=2\quad
\Rightarrow\quad\ A=\frac{4\cdot 2}{2}=4 \MDFPeriod\] 
\end{MHint}
\end{MExercise}

Nun lassen sich auch die Fl\"achen von anderen Vielecken, auch \textbf{Polygone} genannt, bestimmen. Hier soll die Betrachtung jedoch auf einige einfache Formen beschr\"ankt bleiben. Polygone k\"onnen in Dreiecke unterteilt werden. Die Summe der Fl\"acheninhalte dieser Dreiecke ergibt den Fl\"acheninhalt des Polygons.

\begin{MExample}
\begin{tabular}{lc}
%-%\MGraphicsSolo{Mathematik_ElementareGeometrie_Polygone.png}{scale=1}
\MTikzAuto{%
\begin{tikzpicture}[x=0.5cm, y=0.5cm] 
\draw[thick] (0,0) -- (9,0) -- (9,9) -- (4,9) -- cycle;
\draw[thick, dashed] (4,0) -- (4,9);
\node[anchor=north] at (4.5,0) {$a$};
\node[anchor=west] at (9,4.5) {$b$};
\node[anchor=south] at (6.5,9) {$c$};
\node[anchor=south east] at (2.0,4.5) {$d$};
\end{tikzpicture}
}
&
\begin {minipage}[b]{10cm}
Man betrachte das links dargestellte Polygon. In diesem Beispiel kann man das Polygon in ein rechtwinkliges Dreieck mit den Katheten $\left(a-c\right)$ und $b$ und der Hypotenuse $d$ sowie ein Rechteck mit den Seiten $b$ und $c$ unterteilen. Der Fl\"acheninhalt des Polygons ist dann:
\ \\
\end{minipage}
\end{tabular}
\[
  A   =  A_{\text{Dreieck}}+A_{\text{Rechteck}} = \frac{1}{2}\left( a-c \right)\cdot b + b\cdot c 
 = \frac{1}{2}ab-\frac{1}{2}bc+bc=\frac{1}{2}\left(a+b\right)\cdot c \MDFPeriod
\]
\end{MExample}

\begin{MExercise}
\begin{tabular}{lr}
\begin{minipage}[b]{7cm}
Berechnen Sie den Fl\"acheninhalt des\newline \textbf{Parallelogramms} f\"ur $a=4$ und $h=5$.\\
Tipp: Teilen Sie es sinnvoll auf und schauen Sie sich die entstandenen Dreiecke gut an!\\
\vspace*{1cm}
\end{minipage}
&
%-%\MGraphicsSolo{Mathematik_ElementareGeometrie_Polygone_Parallelogramm.png}{scale=1}
\MTikzAuto{%
\begin{tikzpicture}[x=0.5cm, y=0.5cm] 
\draw[thick] (0,0) -- (10,0) -- ++(45:10) -- (45:10) -- cycle;
\draw[stealth'-stealth',thick] (8,0) -- (8,7.0710678);
\node[anchor=north] at (5,0) {$a$};
\node[anchor=west] at (8,3.5355339) {$h$};
\end{tikzpicture}
}
\end{tabular}

\begin{MHint}{L\"osung}

\begin{tabular}{lr}
\begin{minipage}[b]{7cm}
Man kann das Parallelogramm in das linke rote Dreieck, einem folgenden Rechteck und das rechte Dreieck aufspalten. Schneidet man das rote Dreieck aus und setzt es von rechts an das Parallelogramm, erh\"alt man ein Rechteck mit den Seiten $a$ und $h$. Der Fl\"acheninhalt ergibt sich dann zu
\[A=a\cdot b=4\cdot 5=20 \MDFPeriod\]
\end{minipage}
&
%-%\MGraphicsSolo{Mathematik_ElementareGeometrie_Polygone_Parallelogramm_Loesung.png}{scale=1}
\MTikzAuto{%
\begin{tikzpicture}[x=0.5cm, y=0.5cm] 
\draw[thick] (0,0) -- (10,0) -- ++(45:10) -- (45:10) -- cycle;
\draw[thick,dashed] (10,0) -- (10,7.0710678);
\draw (7.0710678,0) -- (7.0710678,7.0710678);
\draw[red,thick] (45:10) -- (0,0) -- (7.0710678,0);
\draw[red,thick,dashed] (7.0710678,0) -- (7.0710678,7.0710678);
\draw[red,thick,dashed] (10,0) -- ++(7.0710678,0) -- ++(0,7.0710678) -- cycle;
\node[anchor=north] at (5,0) {$a$};
\node[anchor=west] at (10,3.5355339) {$h$};
\end{tikzpicture}
}
\end{tabular}
\end{MHint}
\end{MExercise}

Zum Schluss sollen noch Kreisfl\"achen berechnet werden. In \MRef{Kreiszahl} wurde bereits die Kreiszahl $\pi$ vorgestellt, die \"uber den Umfang des Kreises definiert ist. 
Ebenso h\"angt die Kreiszahl mit dem Fl\"acheninhalt von Kreisen zusammen.
\begin{MInfo}
Der Fl\"acheninhalt eines Kreises mit dem Radius $r$ berechnet sich zu 
\[A=\pi\cdot r^2 \MDFPeriod\]
\end{MInfo}

\begin{MExample}
Ein Kreis habe einen Fl\"acheninhalt von $\MZahl{12}{566}$ bei einem Radius von $r=2$. Hieraus l\"asst sich die Kreiszahl $\pi$ berechnen:
\[A=\pi\cdot r^2\quad\Rightarrow\quad \pi=\frac{A}{r^2}=\frac{\MZahl{12}{566}}{4}=\MZahl{3}{1415} \MDFPeriod\]
\end{MExample}

\end{MXContent}


\begin{MExercises}
\MDeclareSiteUXID{VBKM05_Flaecheninhalt_Exercises}

\begin{MExercise}
\begin{tabular}{lr}
\begin{minipage}[b]{8cm}
Berechnen Sie den Fl\"acheninhalt des Polygons:\\
\vspace{3.5cm}
\end{minipage}
&
%-%\MGraphicsSolo{Mathematik_ElementareGeometrie_Polygone_Aufgabe.png}{scale=1}
\MTikzAuto{%
\begin{tikzpicture}[x=1.0cm, y=1.0cm] 
\draw[color=black, thick] (0.0,0.0) -- (3.0,-1.8) -- (5.4,0.0) -- 
(4.0,2.6) -- (1.5,2.2)-- cycle 
(0.0,0.0) -- (5.4,0.0) (1.5,0.0) -- (1.5,2.2)
(4.0,0.0) -- (4.0,2.6) (3.0,0.0) -- (3.0,-1.8);
\draw[color=black] (1.1,0.8) node {\large $A_1$};
\draw[color=black] (2.75,1.2) node {\large $A_2$};
\draw[color=black] (4.4,0.9) node {\large $A_3$};
\draw[color=black] (2.2,-0.6) node {\large $A_4$};
\draw[color=black] (3.9,-0.5) node {\large $A_5$};
\draw[color=black] (0.75,0.0) node[anchor=south] {\large $15$};
\draw[color=black] (2.75,0.0) node[anchor=south] {\large $25$};
\draw[color=black] (4.7,0.0) node[anchor=south] {\large $14$};
\draw[color=black] (1.5,1.5) node[anchor=west] {\large $22$};
\draw[color=black] (4.0,1.8) node[anchor=east] {\large $26$};
\draw[color=black] (3.0,-0.9) node[anchor=west] {\large $18$};
\end{tikzpicture}
}
\end{tabular}
\end{MExercise}

\end{MExercises}


\begin{MXContent}{Die Strahlens\"atze}{Strahlens\"atze}{STD}
\MDeclareSiteUXID{VBKM05_Strahlensaetze}

Die Strahlens\"atze haben etwas mit der zentrischen Streckung zu tun (siehe \MRef{Mathematik_ElementareGeometrie_zentrischeStreckung}).

%\MGraphicsSolo{../Bilder/Mathematik_ElementareGeometrie/Mathematik_ElementareGeometrie_Stadtplan_Strahlensatz.png}{scale=1}

\ifttm%
\relax
\else%
\newcommand{\Strahlensatzfigur}[1][]
{%
\coordinate (S) at (0,0);
\coordinate (A) at ($ (S) + (3,0.5) $);
\coordinate (D) at ($ (S) + (4,2.5) $);
\coordinate (B) at ($ (S)!1.7!(A) $);
\coordinate (C) at ($ (S)!1.7!(D) $);
%
\path (S) node[left]{$S$} (A) node[below right]{$A$} (B) node[below right]{$B$}
                          (C) node[above left] {$C$} (D) node[above left] {$D$};
%
\draw (S) -- ($ (S)!1.1!(B) $);
\draw (S) -- ($ (S)!1.1!(C) $);
%
\draw[#1] ($ (A)!-0.2!(D) $) -- ($ (D)!-1!(A) $) node[left]{$g$};
\draw ($ (B)!-0.2!(C) $) node[right]{$h$} -- ($ (C)!-0.1!(B) $);
}
\fi

\begin{MInfo}\MLabel{Mathematik_ElementareGeometrie_Satz:ErsterStrahlensatz}%
\textbf{Strahlens\"atze}

%\begin{center}
\begin{tabular}{lc}
\mbox{
\MTikzAuto{%
\begin{tikzpicture}
\Strahlensatzfigur
\end{tikzpicture}
}
}
%\end{center}
&
\begin{minipage}[b]{7cm}
F\"ur zwei Punkte~$P$ und~$Q$ seien $\overline{PQ}$ die Strecke von~$P$ nach~$Q$
und $\left| \overline{PQ} \right|$ die L\"ange dieser Strecke.
\vspace*{2cm}
\end{minipage}
\end{tabular}


Sind in dem obigen Bild die Geraden~$g$ und~$h$ parallel, so gilt:
\begin{itemize}
\item
Die Abschnitte auf einem Strahl verhalten sich wie die entsprechenden Abschnitte auf dem anderen Strahl:
\[
   \frac{\left| \overline{SA} \right|}{\left| \overline{SD} \right|}
 = \frac{\left| \overline{SB} \right|}{\left| \overline{SC} \right|}
 = \frac{\left| \overline{AB} \right|}{\left| \overline{CD} \right|} \MDFPeriod
\]
\item
Die Abschnitte auf den Parallelen  verhalten sich wie die von $S$ ausgehenden entsprechenden Abschnitte auf einem Strahl:
\[
   \frac{\left| \overline{SA} \right|}{\left| \overline{SB} \right|}
 = \frac{\left| \overline{SD} \right|}{\left| \overline{SC} \right|}
 = \frac{\left| \overline{AD} \right|}{\left| \overline{BC} \right|} \MDFPeriod
\]

\end{itemize}
\end{MInfo}


%Im Stadtplan (Bild \MRef{Stadtplan}) findet sich noch eine Stelle, an der die Strahlens\"atze angewendet werden k\"onnen. Wo ist sie?

%\begin{MHint}{L\"osung}
%\begin{center}
%\MGraphicsSolo{../Bilder/Mathematik_ElementareGeometrie/Mathematik_ElementareGeometrie_Stadtplan_nochmalStrahlensatz.png}{scale=1}
%\MGraphicsSolo{Mathematik_ElementareGeometrie_Stadtplan_nochmalStrahlensatz.png}{scale=1}
%\end{center}
%\end{MHint}

Mit dem Strahlensatz lassen sich auch einige wichtige S\"atze herleiten, die f\"ur ein rechtwinkliges Dreieck gelten, zum Beispiel die \textbf{Satzgruppe des Pythagoras}. Diese sollen hier aber ohne Herleitung angegeben werden:

\begin{MInfo}
\MLabel{VBKM05_Pythagoras}
\begin{tabular}{lr}
\begin{minipage}{9cm}
Ist in einem rechtwinkligen Dreieck der rechte Winkel bei $C$, $D$~der Lotfu\ss punkt der H\"ohe~$h_c$ auf~$c$,
$p = \left| \overline{AD} \right|$ und $q = \left| \overline{BD} \right|$, so gilt:
\vspace*{1cm}
\end{minipage}
&
\begin{minipage}{7cm}
\MTikzAuto{%
\begin{tikzpicture}
\coordinate[label=above:$C$]       (C) at (0,0);
\coordinate[label=below right:$B$] (B) at ($ (C) + (2,-4) $);
\path let \p1=($ (B) - (C) $) in 
        coordinate[label=left:$A$] (A) at ($ (C) + ({\y1*3/4}, {-\x1*3/4}) $);
\path let \p1=($ (B) - (A) $) in
        coordinate                 (K) at ($ (C) + ({\y1/5}, {- \x1/5}) $);
\coordinate[label=below:$D$]       (D) at (intersection of C--K and A--B);
%
\draw (B) -- node[sloped, above]{$a$} (C) -- node[sloped, above]{$b$} (A) -- cycle;
\draw (C) -- node[sloped, right, rotate=-90]{$h_c$} (D);
\path (A) -- node[sloped, above]{$p$} (D) -- node[sloped, above]{$q$} (B) -- node[sloped, below]{$c$} (A);
\end{tikzpicture}
}
\end{minipage}
\end{tabular}
\begin{itemize}
	\item \textbf{Satz des Pythagoras}\\
		Die Summe der Quadrate \"uber den Katheten haben den gleichen Fl\"acheninhalt wie das Quadrat \"uber der Hypotenuse. So gilt f\"ur das hier abgebildete Dreieck: \[a^2+b^2=c^2 \MDFPeriod\]
		Werden die Seiten des Dreiecks anders bezeichnet, muss die Gleichung entsprechend angepasst werden!
	\item \textbf{Kathetensatz}\\
		Das Quadrat \"uber einer Kathete ist fl\"acheninhaltsgleich dem Rechteck aus der Hypotenuse und dem anliegenden Hypotenusenabschnitt: \[a^2=c\cdot q \MDFPSpace, \MDFPaSpace b^2=c\cdot p \MDFPeriod\]
	\item \textbf{H\"ohensatz}\\
		Das Quadrat \"uber der H\"ohe ist fl\"acheninhaltsgleich dem Rechteck aus den beiden Hypotenusenabschnitten: \[h^2=p\cdot q \MDFPeriod\]	
\end{itemize}
\end{MInfo}

\begin{MExample}
Gegeben sei ein rechtwinkliges Dreieck mit den Kathetenl\"angen $a=3$ und $b=4$.\\
\ \\
Die L\"ange der Hypotenuse kann mithilfe des Satzes von Pythagoras berechnet werden:
\[c^2=\sqrt{a^2 + b^2}=\sqrt{9 + 16}=\sqrt{25}=5 \MDFPeriod\]
Die einzelnen Hypotenusenabschnitte $p$ und $q$ berechnen sich gem\"a\ss\ dem Kathetensatz zu:
\[q=\frac{a^2}{c}=\frac{9}{5} \MDFPSpace, \MDFPaSpace p=\frac{b^2}{c}=\frac{16}{5} \MDFPeriod\]
Die H\"ohe $h_c$ erh\"alt man mit dem H\"ohensatz:
\[h_c=\sqrt{p\cdot q}=\sqrt{\frac{9}{5}\cdot\frac{16}{5}}=\sqrt{\frac{144}{25}}=\frac{12}{5} \MDFPeriod\]
\end{MExample}

\begin{MExercise}
Berechnen Sie f\"ur ein rechtwinkliges Dreieck mit der Hypotenuse $c=\MZahl{10}{5}$, der H\"ohe $h_c=\MZahl{5}{04}$ und dem Hypotenusenabschnitt $q=\MZahl{3}{78}$ die L\"ange der beiden Katheten.

\begin{MHint}{L\"osung}
\[\text{Kathetensatz:} \quad a=\sqrt{c\cdot q}=\sqrt{\MZahl{10}{5} \cdot \MZahl{3}{78}}=\MZahl{6}{3} \MDFPSpace;\]
\[\text{Satz des Pythagoras:} \quad b=\sqrt{c^2-a^2} = \sqrt{{\MZahl{10}{5}}^2-{\MZahl{6}{3}}^2}=\MZahl{8}{4} \MDFPeriod\]
\end{MHint}
\end{MExercise}

Es gibt noch einen weiteren wichtigen Satz, der f\"ur rechtwinklige Dreiecke g\"ultig ist:
\begin{MInfo}
\textbf{Satz des Thales}\\
\ \\
\begin{tabular}{lr}
%-%\MGraphicsSolo{Mathematik_ElementareGeometrie_Thaleskreis.png}{scale=1}
\MTikzAuto{%
\begin{tikzpicture}[x=1.0cm, y=1.0cm] 
\draw[color=black, thick] (-3,0) -- (3,0);
\draw[color=black, thick] (3,0) arc (0:180:3);
\draw[color=red, thick] (-3,0) -- (50:3) -- (3,0);
\draw[color=red, thick] (-3,0) -- (100:3) -- (3,0);
\draw[color=black] (50:3) ++(295:0.6) arc (295:205:0.6);
\draw[color=black] (100:3) ++(320:0.6) arc (320:230:0.6);
\fill[color=black] (50:3) ++(250:0.3) circle (1.0pt);
\fill[color=black] (100:3) ++(275:0.3) circle (1.0pt);
\draw[color=black] (0,0) node[anchor=north] {$M$};
\draw[color=black] (-1.5,0) node[anchor=south] {$r$};
\draw[color=black] (1.5,0) node[anchor=south] {$r$};
\draw (0,0) -- (50:3) (0,0) -- (100:3);
\node[anchor=north west] at (50:1.5) {$r$};
\node[anchor=west] at (100:1.8) {$r$};
\end{tikzpicture}
}
&
\begin{minipage}[b]{7cm}
Hat das Dreieck $ABC$ bei $C$ einen rechten Winkel, so liegt $C$ auf einem Kreis mit der Hypotenuse $c=\overline{AB}$ als Durchmesser.
\vspace*{1.5cm}
\end{minipage}
\end{tabular}
\end{MInfo}

Wenn man also \"uber einer Strecke $\overline{AB}$ einen Halbkreis konstruiert und dann $A$ und $B$ mit einem beliebigen Punkt $C$ auf dem Halbkreis verbindet, dann ist das so entstandene Dreieck immer rechtwinklig.

\begin{MExample}\MLabel{ThaleskreisAufgabe}
Es soll ein rechtwinkliges Dreieck mit der Hypotenusenl\"ange $c=6\MEinheit{cm}$ und der H\"ohe $h_c=\MZahl{2}{5}\MEinheit{cm}$ konstruiert werden.\\
\ \\
\begin{tabular}{lr}
\begin{minipage}[b]{7cm}
  \begin{enumerate}
    \item Zuerst zeichnet man die Hypotenuse \newline $c=\overline{AB}$.\\
    \item Die Mitte der Hypotenuse wird nun zum Mittelpunkt eines Kreises mit der L\"ange $c/2$.\\
    \item Nun zeichnet man eine Parallelle zur Hypotenuse im Abstand $h_c$. Es gibt zwei Schnittpunkte $C$ und $C'$ dieser Parallelen mit dem Thaleskreis. 
  \end{enumerate}
\end{minipage}
&
%-%\MGraphicsSolo{Mathematik_ElementareGeometrie_Thaleskreis_Konstruktion.png}{scale=1}
\MTikzAuto{%
\begin{tikzpicture}[x=1.2cm, y=1.2cm] 
\draw[color=red, thick] (-3,0) -- (3,0);
\draw[color=blue, thick] (3,0) arc (0:180:3);
\draw[color=red, thick, dashed] (-3,2.5) -- (3,2.5);
\fill[color=black, opacity=0.5] (0,0) circle (2.0pt);
\draw[color=black, thick] (-3,0) -- (-1.658312395,2.5) -- (3,0);
\draw[color=black, thick, dashed] (-3,0) -- (1.658312395,2.5) -- (3,0);
\draw[color=black] (-1.658312395,0) -- (-1.658312395,2.5);
\draw[color=gray, dashed] (1.658312395,0) -- (1.658312395,2.5);
\draw[color=black] (-3,0) node[anchor=north east] {$A$};
\draw[color=black] (3,0) node[anchor=north west] {$B$};
\draw[color=black] (0,-2pt) node[anchor=north] {$M$};
\draw[color=black] (-1.658312395,1.10) node[anchor=east] {$h_c$};
\draw[color=black] (1.658312395,1.10) node[anchor=west] {$h_c$};
\node[anchor=south east] at (-1.658312395,2.5) {$C$};
\node[anchor=south west] at (1.658312395,2.5) {$C'$};
\draw[color=red] (-1.5,0) node[anchor=north] {\large $\mathsf{1}$};
\draw[color=blue] (30:3) node[anchor=west] {\large $\mathsf{2}$};
\draw[color=red] (3,2.5) node[anchor=south east] {\large $\mathsf{3}$};
\end{tikzpicture}
}
\end{tabular}\\
Diese sind jeweils die dritte Ecke eines Dreiecks, das die geforderten Eigenschaften hat, das hei\ss t, man erh\"alt zwei L\"osungen.
W\"urde man noch einen Thaleskreis nach unten zeichen, so erg\"aben sich noch mal zwei L\"osungen.
\end{MExample}

\begin{MExercise}
Welche H\"ohe $h_c$ kann ein rechtwinkliges Dreieck mit der Hypotenuse $c$ maximal haben?

\begin{MHint}{L\"osung}
Die H\"ohe $h_c$ kann maximal so gro\ss\ werden wie der Radius des Thaleskreises \"uber der Hypotenuse, also $h_c\leq c/2$.
\end{MHint}
\end{MExercise}

\end{MXContent}

\begin{MExercises}
\MDeclareSiteUXID{VBKM05_Strahlensaetze_Exercises}
\begin{MExercise}
Der Sohn des Hauses beobachtet den Baum auf des Nachbarn Grundst\"uck.
Er stellt fest, dass der Baum von der Hecke, die die beiden Grundst\"ucke trennt,
vollst\"andig verdeckt wird, wenn er nur nahe genug an die Hecke herantritt.
Jetzt sucht er den Punkt, an dem der Baum gerade so nicht mehr zu sehen ist.

Der $\MZahl{1}{40}$~Meter gro\ss e Junge muss $\MZahl{2}{50}$~Meter von der $\MZahl{2}{40}$~Meter hohen, $1$~Meter breiten
und oben spitz zulaufenden Hecke
entfernt stehen, damit der Baum vollst\"andig verdeckt ist.

Wie hoch ist der Baum, wenn die Mitte des Stamms $\MZahl{14}{5}$~Meter von der Hecke entfernt steht?
\ifttm (Angabe in Metern) \MLParsedQuestion{10}{7.4}{1}{GEO7}\else\relax\fi

F\"uhren Sie die Rechnung bitte zun\"achst allgemein durch und setzen Sie erst
am Ende die Zahlenwerte ein!

\begin{MHint}{Hinweis}
Beachten Sie die Breite der Hecke!
\end{MHint}

\begin{MHint}{L\"osung}
\begin{center}
\MTikzAuto{%
\begin{tikzpicture}[x=0.75cm, y=0.75cm]
\coordinate (KF) at (0,0);
\coordinate (KK) at (0,1.4);
\coordinate (HF) at (3,0);
\coordinate (HK) at (3,2.4);
\coordinate (BF) at (16.5,0);
\coordinate (BK) at (16.5,7.4);
\coordinate (BM) at (16.5,1.4);
%
\begin{scope}[color=black!50, every node/.style={color=black}]
\draw (KF) -- ++ (0,-0.3) ++(0,0.15) -- node[below]{$\MZahl{2}{5}\MEinheit{m}$} ($ (HF) + (-0.5,-0.15) $) ++(0,-0.15) -- ++(0,0.3);
\draw ($ (HF) + (0.5,0) $) -- ++ (0,-0.3) ++(0,0.15) -- node[below]{$\MZahl{14}{5}\MEinheit{m}$} ($ (BF) + (0,-0.15) $) ++(0,-0.15) -- ++(0,0.3);
 \node at (HF) [below]{$1\MEinheit{m}$};
 \draw ($ (KF)!0.5!(BF) $) -- node[right]{$h_{\mathrm{K}} = \MZahl{1}{4}\MEinheit{m}$} ($ (KK)!0.5!(BM) $) ++ (-0.15,0) -- ++(0.3,0);
 \draw (BF) -- node[right]{$h_{\mathrm{B}}$} (BK) ++ (-0.15,0) -- ++(0.3,0);
 \draw ($ (HF) + (0.75,0) $) -- node[below right]{$h_{\mathrm{H}} = \MZahl{2}{4}\MEinheit{m}$} ($ (HK) + (0.75,0) $) ++ (-0.15,0) -- ++(0.3,0);
 \draw (KK) -- ++ (0,-0.3) ++(0,0.15) -- node[below]{$d_{\mathrm{KH}}$} ($ (HF) + (KK) - (KF) + (0,-0.15) $) ++(0,-0.15) -- ++(0,0.3);
 \draw (KK) -- ++ (0,0.3) ++(0,-0.15) -- node[above]{$d_{\mathrm{KB}}$} ($ (BM) + (0,0.15) $) ++(0,0.15) -- ++(0,-0.3);
\end{scope}
\draw (KF) ++ (-0.2,0) -- ++ (0.2,0.5) -- ++ (0.2,-0.5)
      (KF) ++ (0,0.5) -- ++ (0,0.5)
      (KK) ++ (0,-0.6) ++ (-0.2,-0.4) -- ++ (0.2,0.4) -- ++ (0.2,-0.4)
      (KK) ++ (0,-0.2) circle (0.2);
\draw (HK) decorate[decoration={random steps,segment length=2pt,amplitude=1pt}] {.. controls ($ (HF) + (-0.5,1.5) $) .. ($ (HF) + (-0.5,0) $)};
\draw (HK) decorate[decoration={random steps,segment length=2pt,amplitude=1pt}] {.. controls ($ (HF) + (0.5,1.5) $) .. ($ (HF) + (0.5,0) $)};
\draw (BK) decorate[decoration={saw,mirror}] {-- ($ (BF) + (-2.9,1) $) -- ++ (2.7,0)} -- ($ (BF) + (-0.3,0) $);
\draw (BK) decorate[decoration={saw}] {-- ($ (BF) + (2.9,1) $) -- ++ (-2.7,0)} -- ($ (BF) + (0.3,0) $);
\draw (KK) -- (BK) ($ (KF)!-0.05!(BF) $) -- ($ (BF)!-0.05!(KF) $);
\draw[dashed] (KK) -- (BM);
\end{tikzpicture}
}
\end{center}

Eine Anwendung des zweiten Strahlensatz $\left( \frac{\text{komplett}}{\text{vorne}} = \frac{\text{lang}}{\text{kurz}} \right)$
liefert:
\[
\frac{d_{\mathrm{KB}}}{d_{\mathrm{KH}}}
= \frac{h_{\mathrm{B}} - h_{\mathrm{K}}}{h_{\mathrm{H}} - h_{\mathrm{K}}}
 \qquad \text{bzw.} \qquad
 h_{\mathrm{B}} = \left( h_{\mathrm{H}} - h_{\mathrm{K}} \right) \cdot \frac{d_{\mathrm{KB}}}{d_{\mathrm{KH}}} + h_{\mathrm{K}} \MDFPeriod
\]

Nun gelten $d_{\mathrm{KH}} = \MZahl{2}{5}\MEinheit{m} + \frac{1\MEinheit{m}}{2} = 3\MEinheit{m}$
und $d_{\mathrm{KB}} = \MZahl{2}{5}\MEinheit{m} + 1\MEinheit{m} + \MZahl{14}{5}\MEinheit{m} = 18\MEinheit{m}$.
Damit folgt
\[
  h_{\mathrm{B}}
   = \left( \MZahl{2}{4}\MEinheit{m} - \MZahl{1}{4}\MEinheit{m} \right) \cdot \frac{18\MEinheit{m}}{3\MEinheit{m}} + \MZahl{1}{4}\MEinheit{m}
   = 1\MEinheit{m} \cdot 6 + \MZahl{1}{4}\MEinheit{m}
   = \MZahl{7}{4}\MEinheit{m} \MDFPeriod
\]
\end{MHint}
\end{MExercise}

\end{MExercises}




\MSubsection{Trigonometrie}\MLabel{Mathematik_ElementareGeometrie_Sec:Trigonometrie}


\begin{MIntro}
\MDeclareSiteUXID{VBKM05_Trigonometrie_Intro}
Der vorherige Abschnitt besch\"aftigte sich unter anderem mit den Seiten rechtwinkliger Dreiecke.
Im nun folgenden Abschnitt sollen auch die beiden anderen Winkel im rechtwinkligen Dreieck in die Betrachtung miteinbezogen und deren Beziehung zu den Seiten erhellt werden. Dies erm\"oglicht es dann auch die H\"ohe $h$ eines nicht rechtwinkligen Dreiecks zu berechnen, die man einerseits f\"ur die Fl\"achenberechnung ben\"otigt, die jedoch andererseits oftmals gar nicht explizit gegeben ist.
\end{MIntro}

\begin{MXContent}{Trigonometrie am Dreieck}{Dreieck}{STD}
\MLabel{Abschnitt:TrigonometrieAmDreieck}
\MDeclareSiteUXID{VBKM05_Trigonometrie_Content}

Mithilfe der Strahlens\"atze konnte gezeigt werden, dass die Verh\"altnisse der
Seiten in einem Dreieck lediglich von den Winkeln des Dreiecks abh\"angen. \"Andert man
in einem Strahlensatz den Winkel bei~$S$ oder den Winkel, in dem die
parallelen Geraden einen Strahl schneiden, so \"andern sich nat\"urlich
auch die Verh\"altnisse.
Legt man hingegen einen der beiden Winkel fest, so kann man die
Verh\"altnisse in Abh\"angigkeit von dem anderen Winkel als Funktion
von einer Variablen darstellen.\\
\ \\
Es sei nun der Winkel $\gamma$ auf $\gamma\,=\,\pi/2$ festgelegt, das hei\ss t es werde ein rechtwinkliges Dreieck betrachtet mit rechtem Winkel bei $\gamma$. Nun sind die Seitenverh\"altnisse nur vom Winkel $\alpha$ abh\"angig, der Winkel $\beta$ ergibt sich n\"amlich aus der Winkelinnensumme des Dreiecks. Dies f\"uhrt zu einer ganzen Schar von \"ahnlichen Dreiecken, wobei die Seiten $c$, $c'$, $\ldots$ die Hypotenusen sind. Im Hinblick auf den betrachteten Winkel $\alpha$ bezeichnet man die dem Winkel gegen\"uberliegen Seiten
$a$, $a'$, $\ldots$ als \textbf{Gegenkatheten} und die am Winkel anliegenden Seiten $b$, $b'$, $\ldots$ als \textbf{Ankatheten}.\\
Wendet man nun die Strahlens\"atze an, so gelangt man zu folgenden Erkenntnissen: 

\begin{MInfo}\MLabel{Mathematik_ElementareGeometrie_Def:Trigonometrie}%
\textbf{Die trigonometrischen Funktionen im rechtwinkligen Dreieck}

\begin{center}
%-%\MGraphicsSolo{Mathematik_ElementareGeometrie_TrigonometrieDefinition_1.png}{scale=1}
\MTikzAuto{%
\begin{tikzpicture}[x=1.0cm, y=1.0cm] 
\pgfmathparse{sin(39)/sin(91)}\let\laf\pgfmathresult
\pgfmathparse{sin(50)/sin(91)}\let\lbf\pgfmathresult
\draw[color=black, thick] (0,0) -- (23:9) (0,0) -- (62:7);
\draw[color=black, thick] (23:5.4) -- +(153:{5.4*\laf});
\draw[color=black, thin] (0,0) ++(23:1.2) arc (23:62:1.2);
\draw[color=black] (0,0) ++(42.5:0.8) node {$\alpha$};
\draw[color=black, thin] (23:5.4) ++(153:1.2) arc (153:203:1.2);
\draw[color=black] (23:5.4) ++(178:0.8) node {$\beta$};
\draw[color=black, thin] (23:5.4) ++(153:{5.4*\laf}) ++(242:0.9) arc (242:333:0.9);
\draw[color=black] (23:5.4) ++(153:{5.4*\laf}) ++(287.5:0.6) node {$\gamma$};
\path (23:5.4) ++(153:{5.4*0.5*\laf}) node[anchor=south west] {$a$};
\path (62:{5.4*0.5*\lbf}) node[anchor=south east] {$b$};
\path (23:{5.4*0.5}) node[anchor=north west] {$c$};
\draw[color=black, thick, loosely dashed] (23:7.0) -- +(153:{7.0*\laf});
\path (23:7.0) ++(153:{7.0*0.5*\laf}) node[anchor=south west] {$a'$};
\path (62:{6.4*\lbf}) node[anchor=south east] {$b'$};
\path (23:{6.4}) node[anchor=north west] {$c'$};
\draw[color=black, thick, loosely dotted] (23:8.4) -- +(153:{8.4*\laf});
\path (23:8.4) ++(153:{8.4*0.5*\laf}) node[anchor=south west] {${a'}'$};
\path (62:{7.7*\lbf}) node[anchor=south east] {${b'}'$};
\path (23:{7.7}) node[anchor=north west] {${c'}'$};
\end{tikzpicture}
}
\end{center}

\begin{itemize}
 \item Das Verh\"altnis \[\frac{a}{c}=\frac{a'}{c'}=\frac{  {a'}'}{{c'}'}=\ldots=\frac{\text{Gegenkathete}}{\text{Hypotenuse}}:=\sin\left(\alpha\right)\]
 bezeichnet man als den \textbf{Sinus} des Winkels $\alpha$.
 
 \item  Das Verh\"altnis \[\frac{b}{c}=\frac{b'}{c'}=\frac{{b'}'}{{c'}'}=\ldots=\frac{\text{Ankathete}}{\text{Hypotenuse}}:=\cos\left(\alpha\right)\]
 bezeichnet man als den \textbf{Kosinus} des Winkels $\alpha$.
 
 \item Das Verh\"altnis \[\frac{a}{b}=\frac{a'}{b'}=\frac{{a'}'}{{b'}'}=\ldots=\frac{\text{Gegenkathete}}{\text{Ankathete}}:=\tan\left(\alpha\right)\]
 bezeichnet man als den \textbf{Tangens} des Winkels $\alpha$.
\end{itemize}

\end{MInfo}

Der Tangens des Winkels $\alpha$ ist nach der Definition \[\tan\left(\alpha\right)=\frac{a}{b}=\frac{a}{b}\cdot\frac{c}{c}=\frac{a}{c}\cdot\frac{c}{b}=\frac{\sin\left(\alpha\right)}{\cos\left(\alpha\right)} \MDFPeriod\]

\begin{MExample}
Von einem Dreieck ist bekannt, dass es einen rechten Winkel $\gamma=\frac{\pi}{2}=90^\circ$ hat. Die Seite $c$ ist $5\MEinheit{cm}$, die Seite $a$ ist $\MZahl{2}{5}\MEinheit{cm}$ lang. Es sollen jeweils der Sinus, Kosinus und Tangens des Winkels $\alpha$ bestimmt werden:

Der Sinus l\"asst sich sofort aus den Angaben berechnen:
\[\sin\left(\alpha\right)=\frac{a}{c}=\frac{\MZahl{2}{5}\MEinheit{cm}}{5\MEinheit{cm}}=\MZahl{0}{5} \MDFPeriod\]
F\"ur den Kosinus wird die L\"ange der Seite $b$ ben\"otigt, welche man mithilfe des Satzes von Pythagoras erh\"alt:
\[b^2=c^2-a^2\quad\Rightarrow\quad\cos\left(\alpha\right)=\frac{b}{c}=\frac{\sqrt{c^2-a^2}}{c}=\frac{\sqrt{\left(5\MEinheit{cm}\right)^2-\left(\MZahl{2}{5}\MEinheit{cm}\right)^2}}{5\MEinheit{cm}}=\MZahl{0}{866} \MDFPeriod\]
Daraus folgt f\"ur den Tangens
\[\tan\left(\alpha\right)=\frac{\sin\left(\alpha\right)}{\cos\left(\alpha\right)}=\frac{\MZahl{0}{5}}{\MZahl{0}{866}}=\MZahl{0}{5773} \MDFPeriod\]
\end{MExample}


\begin{MExercise}\MLabel{Mathematik_ElementareGeometrie:Trigonometrie_Aufgabe:Tabelle}
Die Hypotenuse $c=5$ ist vorgegeben. Zeichnen Sie mithilfe des Thaleskreises (Ma\ss stab $ 1\hat{=}2\MEinheit{cm}$
die rechtwinkligen Dreiecke f\"ur die Winkel $\alpha\in\left\{0^\circ\MElSetSep 10^\circ\MElSetSep 20^\circ\MElSetSep 30^\circ\MElSetSep 40^\circ\MElSetSep 45^\circ\MElSetSep
50^\circ\MElSetSep 60^\circ\MElSetSep 70^\circ\MElSetSep 80^\circ\MElSetSep 90^\circ\right\}$.

Messen Sie die Seiten $a$ und $b$ und schreiben Sie sie in eine Tabelle. Berechnen Sie zu jedem Dreieck den Sinus, Kosinus und Tangens.

Schauen Sie sich die Werte genauer an und versuchen Sie, sie zu interpretieren.

Tragen Sie die Werte von Sinus, Kosinus und Tangens in Abh\"angigkeit des Winkels $\alpha$ in ein Diagramm.

\begin{MHint}{L\"osung}
Beim Messen entstehen immer Messfehler!
Die Tabelle k\"onnte folgenderma\ss en aussehen:

\begin{center}
\begin{tabular}{r|r|r|r|r|r}\hline
	$\alpha$ 		& $a$ 				& $b$ 				& $\sin\left(\alpha\right)$ & $\cos\left(\alpha\right)$ & $\tan\left(\alpha\right)$\\ \hline\hline
	0 						& $\MZahl{0}{0}$		& $\MZahl{5}{0}$ 	& $\MZahl{0}{0}$		& $\MZahl{1}{0}$	& $\MZahl{0}{0}$\\ \hline
	$10^\circ$		& $\MZahl{0}{8}$	& $\MZahl{4}{9}$	& $\MZahl{0}{160}$	& $\MZahl{0}{98}$	& $\MZahl{0}{1633}$\\ \hline
	$20^\circ$		& $\MZahl{1}{7}$	& $\MZahl{4}{7}$	& $\MZahl{0}{34}$	& $\MZahl{0}{94}$	& $\MZahl{0}{3617}$\\ \hline
	$30^\circ$		& $\MZahl{2}{5}$	& $\MZahl{4}{3}$	& $\MZahl{0}{5}$	& $\MZahl{0}{86}$	& $\MZahl{0}{5814}$\\ \hline
	$40^\circ$		& $\MZahl{3}{2}$	& $\MZahl{3}{8}$	& $\MZahl{0}{64}$	& $\MZahl{0}{76}$	& $\MZahl{0}{8421}$\\ \hline
	$45^\circ$		& $\MZahl{3}{5}$	& $\MZahl{3}{5}$	& $\MZahl{0}{7}$	& $\MZahl{0}{7}$	& $\MZahl{1}{0}$\\ \hline
	$50^\circ$		& $\MZahl{3}{8}$	& $\MZahl{3}{27}$	& $\MZahl{0}{76}$	& $\MZahl{0}{64}$	& $\MZahl{1}{1875}$\\ \hline
	$60^\circ$		& $\MZahl{4}{3}$	& $\MZahl{2}{5}$	& $\MZahl{0}{86}$	& $\MZahl{0}{5}$	& $\MZahl{1}{7200}$\\ \hline
	$70^\circ$		& $\MZahl{4}{7}$	& $\MZahl{1}{7}$	& $\MZahl{0}{94}$	& $\MZahl{0}{34}$	& $\MZahl{2}{7647}$\\ \hline
	$80^\circ$		& $\MZahl{4}{9}$	& $\MZahl{0}{8}$	& $\MZahl{0}{98}$	& $\MZahl{0}{160}$	& $\MZahl{6}{1250}$\\ \hline
	$90^\circ$		& $\MZahl{5}{0}$		& $\MZahl{0}{0}$	& $\MZahl{1}{0}$	& $\MZahl{0}{0}$	& $\rightarrow\infty$\\ \hline
\end{tabular}
\end{center}

\begin{itemize}
\item Mit zunehmendem Winkel $\alpha$ nimmt die Gegenkathete $a$ zu und die Ankathete $b$ ab.

Ebenso verhalten sich $\sin\left(\alpha\right)\sim a$ und $\cos\left(\alpha\right)\sim b$.

\item Mit zunehmendem Winkel $\alpha$ nimmt $a$ in dem gleichen Ma\ss\ zu wie $b$ mit dem von $90^\circ$ aus fallenden Winkel $\alpha$ abnimmt. Im Thaleskreis sind die beiden Dreiecke mit den entgegengesetzten Werten f\"ur $a$ und $b$ die zwei L\"osungen f\"ur die Konstruktion eines rechtwinkligen Dreiecks mit gegebener Hypotenuse und gegebener H\"ohe (Aufgabe \MRef{ThaleskreisAufgabe}).

Ebenso verhalten sich Sinus und Kosinus zueinander: es ist also

$\sin\left(\alpha\right)=\cos\left(90^\circ-\alpha\right)=\cos\left(\pi/2-\alpha\right)$
bzw. 

$\cos\left(\alpha\right)=\sin\left(90^\circ-\alpha\right)=\sin\left(\pi/2-\alpha\right)$.

\item Bei $\alpha=45^\circ$ sind die Katheten und damit auch Sinus und Kosinus von $\alpha$ gleich.

\item Der Tangens, also das Verh\"altnis von $a$ zu $b$, steigt mit zunehmendem Winkel $\alpha$ von Null ins Unendliche. 
\end{itemize}

Das Diagramm sieht folgenderma\ss en aus:

%-%\MGraphicsSolo{Mathematik_ElementareGeometrie_Trigonometrie_Diagramm.png}{scale=1}
\MTikzAuto{%
\pgfkeys{/pgf/number format/set decimal separator={{{\MZXYZhltrennzeichen}}}}
\begin{tikzpicture}[x=0.1cm, y=5.4cm] 
%Koordinatensystem
\node (xMAX) at (102.0,0){};
\node (yMAX) at (0,1.15){};
%%\draw[help lines, gray, dashed, xstep=1, ystep=1] (0,0) grid (5.5,2.8);
\draw[-stealth',color=black] (-5,0) -- (xMAX);
\foreach \x in {10, 20, 30, 40, 50, 60, 70, 80, 90}
\draw[shift={(\x,0)},color=black] (0pt,0pt) -- (0pt,-6pt) node[below] {\normalsize $\x^\circ$};
\draw[-stealth',color=black] (0,-0.12) -- (yMAX);
\foreach \y in {0.25, 0.5, 0.75, 1}
\draw[shift={(0,\y)},color=black] (0pt,0pt) -- (-6pt,0pt)  node[left] {\normalsize $\pgfmathprintnumber{\y}$};
%Achsenbeschriftung
\draw (xMAX) node[anchor=north east] {$\alpha$};
\draw[color=blue] (yMAX) ++(0.,-16pt) node[anchor=south east] {$\cos(\alpha)$};
\draw[color=red] (yMAX) ++(0.,-5pt) node[anchor=south east] {$\sin(\alpha)$};
\fill[color=red] (90,1.0) circle (2.0pt);
\fill[color=red] (80,0.98) circle (2.0pt);
\fill[color=red] (70,0.94) circle (2.0pt);
\fill[color=red] (60,0.86) circle (2.0pt);
\fill[color=red] (50,0.76) circle (2.0pt);
%\fill[color=red] (45,0.7) circle (2.0pt);
\fill[color=red] (40,0.64) circle (2.0pt);
\fill[color=red] (30,0.5) circle (2.0pt);
\fill[color=red] (20,0.34) circle (2.0pt);
\fill[color=red] (10,0.16) circle (2.0pt);
\fill[color=red] (0,0.0) circle (2.0pt);
\fill[color=blue] (0,1.0) circle (2.0pt);
\fill[color=blue] (10,0.98) circle (2.0pt);
\fill[color=blue] (20,0.94) circle (2.0pt);
\fill[color=blue] (30,0.86) circle (2.0pt);
\fill[color=blue] (40,0.76) circle (2.0pt);
%\fill[color=blue] (45,0.7) circle (2.0pt);
\fill[color=blue] (50,0.64) circle (2.0pt);
\fill[color=blue] (60,0.5) circle (2.0pt);
\fill[color=blue] (70,0.34) circle (2.0pt);
\fill[color=blue] (80,0.16) circle (2.0pt);
\fill[color=blue] (90,0.0) circle (2.0pt);
\fill[color=magenta!50!black] (45,0.7) circle (2.0pt);
%Beschriftung und Graphen
%%\clip(-2.8,-0.5) rectangle (6,3);
%%\draw[color=black] (7,3) node[anchor=south west] {$\MPointTwo{7}{3}$};
\end{tikzpicture}
}

\end{MHint}
\end{MExercise}


\begin{MExample}
Es soll der Sinus des Winkels $\alpha=45^\circ$ nun exakt berechnet, also nicht wie in Aufgabe \MRef{Mathematik_ElementareGeometrie:Trigonometrie_Aufgabe:Tabelle} aus gemessenen (und damit fehlerbehafteten) Werten bestimmt werden.

Wenn im rechtwinkligen Dreieck mit $\gamma=90^\circ$ der Winkel $\alpha$ gleich $45^\circ$ ist, so muss wegen der Innenwinkelsumme $\alpha+\beta+\gamma=\pi=180^\circ$ der Winkel $\beta$ auch gleich $45^\circ=\pi/4$ sein, und die beiden Katheten $a$ und $b$ sind gleich lang. Ein Dreieck mit zwei gleich langen Seiten nennt man \textbf{gleichschenklig}:

\begin{tabular}{lr}
%-%\MGraphicsSolo{Mathematik_ElementareGeometrie_Trigonometrie45Grad.png}{scale=1}
\MTikzAuto{%
\begin{tikzpicture}[x=1.0cm, y=1.0cm] 
%%\draw[help lines, gray!50, xstep=0.5, ystep=0.5] (0,0) grid (9,8);
\draw[color=black, very thick] (0,0) -- (6,0) -- (3,3) -- cycle;
\draw[color=black, thin] (0,0) ++(0:1.2) arc (0:45:1.2);
\draw[color=black] (0,0) ++(22.5:0.8) node {\large $\alpha$};
\draw[color=black, thin] (6,0) ++(135:1.2) arc (135:180:1.2);
\draw[color=black] (6,0) ++(157.5:0.8) node {\large $\beta$};
\draw[color=black, thin] (3,3) ++(225:1.2) arc (225:315:1.2);
\fill[color=black] (3,3) ++(0,-0.6) circle (1.5pt);
\draw[color=black] (4.5,1.5) node[anchor=south west] {\large $a$};
\draw[color=black] (1.5,1.5) node[anchor=south east] {\large $b$};
\draw[color=black] (3,0) node[anchor=north] {\large $c$};
\end{tikzpicture}
}
&
\begin{minipage}[b]{10cm}
Es gilt: \[\sin\left(\alpha\right) = \sin\left(45^\circ\right) = \frac{a}{c} \MDFPeriod\]
Au\ss erdem gilt: \[a^2+b^2 = 2a^2 = c^2\quad\Rightarrow\quad c=\sqrt{2}\cdot a\]
\[\Rightarrow\quad \sin\left(45^\circ\right) = \sin\left(\pi/4\right)=\frac{a}{\sqrt{2}\cdot a} = \frac{1}{2}\cdot \sqrt{2} \MDFPeriod\]
\end{minipage}
\end{tabular}
In der Aufgabe \MRef{Mathematik_ElementareGeometrie:Trigonometrie_Aufgabe:Tabelle} wurde der Sinus von $45^\circ$ durch einen Wert von $\MZahl{0}{7}$ angen\"hert, was dem tats\"achlichen Wert von $\frac{1}{2}\cdot \sqrt{2}$ schon recht nahe kommt.
\end{MExample}

\begin{MExample}\MLabel{Mathematik_ElementareGeometrie:Trigonometrie_Beispiel:gleichseitigesDreieck}%
In diesem Beispiel soll ein \textbf{gleichseitiges} Dreieck betrachtet werden. Wie der Name sagt, sind in diesem Dreieck alle Seiten gleich lang, und auch die Winkel sind alle gleich gro\ss , n\"amlich $\alpha=\beta=\gamma = \frac{180^\circ}{3} = 60^\circ = \frac{\pi}{3}$.
Das Dreieck ist nach dem Kongruenzsatz "`sss"' mit der Angabe einer Seite $a$ eindeutig bestimmt, und man erh\"alten dieses, indem man die Seite $a$ zeichnet und mit dem Zirkel einen Kreis vom Radius $a$ um jede Ecke schl\"agt. Der Schnittpunkt der Kreise ist nun die dritte Ecke.

\begin{tabular}{lr}
\begin{minipage}[b]{10.5cm}
Dieses Dreieck hat keinen rechten Winkel. Zeichnet man eine H\"ohe $h$ auf eine der Seiten $a$ ein, so erh\"alt man zwei kongruente Dreiecke mit je einem rechten Winkel.

Es gilt nun: \[\sin\left(\alpha\right)=\sin\left(60^\circ\right)=\frac{h}{a} \MDFPeriod\]
Nach dem Satz von Pythagoras ist \[\left(\frac{a}{2}\right)^2+h^2=a^2 \quad\Rightarrow\quad h^2=\frac{3}{4}a^2 \Rightarrow\quad h=\frac{1}{2}\sqrt{3}\cdot a\]
\end{minipage}
&
%-%\MGraphicsSolo{Mathematik_ElementareGeometrie_Trigonometrie60Grad.png}{scale=1}
\MTikzAuto{%
\begin{tikzpicture}[x=1.0cm, y=1.0cm] 
%%\draw[help lines, gray!50, xstep=0.5, ystep=0.5] (0,0) grid (9,8);
\draw[color=black, very thick] (0,0) -- (5,0) -- (2.5,4.33) -- cycle;
\draw[color=black, thin] (0,0) ++(0:1.2) arc (0:60:1.2);
\draw[color=black] (0,0) ++(30:0.8) node {\large $\alpha$};
\draw[color=black, thin] (5,0) ++(120:1.2) arc (120:180:1.2);
\draw[color=black] (5,0) ++(150:0.8) node {\large $\alpha$};
\draw[color=black, thin] (2.5,0) -- (2.5,3.13);
\draw[color=black, gray, thin] (2.5,3.13) -- (2.5,4.33);
\draw[color=black, thin] (2.5,4.33) ++(240:1.2) arc (240:300:1.2);
\draw[color=black] (2.5,4.33) ++(270:0.8) node {\large $\alpha$};
\draw[color=black, thin] (2.5,0) ++(90:0.8) arc (90:180:0.8);
\fill[color=black] (2.5,0) ++(135:0.4) circle (1.5pt);
\draw[color=black] (3.75,2.165) node[anchor=south west] {\large $a$};
\draw[color=black] (1.25,2.165) node[anchor=south east] {\large $a$};
\draw[color=black] (2.5,0.0) node[anchor=north] {\large $a$};
\draw[color=black] (2.5,1.65) node[anchor=west] {\large $h$};
\draw[color=black, gray, thin] (0,0) ++(50:5.0) arc (50:70:5.0);
\draw[color=black, gray, thin] (5,0) ++(110:5.0) arc (110:130:5.0);
\end{tikzpicture}
}
\end{tabular}
 \[\Rightarrow\quad\sin\left(60^\circ\right)=\sin\left(\frac{\pi}{3}\right)=\frac{1}{2}\cdot \sqrt{3} \MDFPeriod\]
 \ \\
Aus diesem Dreieck kann man noch den Sinus eines weiteren Winkels berechnen: 
Die H\"ohe $h$ teilt den oberen Winkel in zwei gleiche Teile, sodass man in den beiden kleinen kongruenten Dreiecken jeweils den Winkel $30^\circ = \frac{\pi}{6}$ erh\"alt.
Es ist nun \[\sin\left(30^\circ\right)=\sin\left(\frac{\pi}{6}\right)=\frac{a/2}{a}=\frac{1}{2} \MDFPeriod\]
\end{MExample}

\begin{MExercise}
Berechnen Sie, auf die vorher gezeigten Beispiele Bezug nehmend, den exakten Wert des Kosinus f\"ur die Winkel $\alpha_1=30^\circ, \alpha_2=45^\circ$ und $\alpha_3=60^\circ$. Verwenden Sie die Erkenntnisse aus Aufgabe \MRef{Mathematik_ElementareGeometrie:Trigonometrie_Aufgabe:Tabelle}.

\begin{MHint}{L\"osung}
Aus der Aufgabe \MRef{Mathematik_ElementareGeometrie:Trigonometrie_Aufgabe:Tabelle} ist bekannt, dass $\cos\left(\alpha\right)= \sin\left(90^\circ-\alpha\right)$.

Daraus folgt
\[\cos\left(30^\circ\right)=\sin\left(90^\circ-30^\circ\right)=\sin\left(60^\circ\right)=\frac{1}{2}\cdot\sqrt{3} \MDFPSpace, \]
\[\cos\left(45^\circ\right)=\sin\left(90^\circ-45^\circ\right)=\sin\left(45^\circ\right)=\frac{1}{2}\cdot\sqrt{2} \MDFPSpace, \]
\[\cos\left(60^\circ\right)=\sin\left(90^\circ-60^\circ\right)=\sin\left(30^\circ\right)=\frac{1}{2} \MDFPeriod\]
\end{MHint}
\end{MExercise}

In einer kleinen Tabelle k\"onnen die gefundenen Werte f\"ur markante Winkel zusammengetragen werden:

\ifttm
\begin{MDirectHTML}
\[
\begin{array}[t]{l|*{5}{c}}
              & 0                          & \tfrac{\pi}{6}             & \tfrac{\pi}{4}             & \tfrac{\pi}{3}             & \tfrac{\pi}{2}             \\[1mm] \hline
         \sin & \frac{1}{2} \cdot \sqrt{0} & \frac{1}{2} \cdot \sqrt{1} & \frac{1}{2} \cdot \sqrt{2} & \frac{1}{2} \cdot \sqrt{3} & \frac{1}{2} \cdot \sqrt{4} \\[1mm]
         \cos & \frac{1}{2} \cdot \sqrt{4} & \frac{1}{2} \cdot \sqrt{3} & \frac{1}{2} \cdot \sqrt{2} & \frac{1}{2} \cdot \sqrt{1} & \frac{1}{2} \cdot \sqrt{0} \\[1mm]
         \tan & 0                          & \frac{\sqrt{3}}{3}         & 1                          & \sqrt{3}                   & -                          \\[1mm]
              & 0^{\circ}                  & 30^{\circ}                 & 45^{\circ}                 & 60^{\circ}                 & 90^{\circ}                
\end{array}
\]
\end{MDirectHTML}
\else
\begin{center}
       $\begin{array}[t]{l|*{5}{c}}
              & 0                          & \Mtfrac{\pi}{6}             & \Mtfrac{\pi}{4}             & \Mtfrac{\pi}{3}             & \Mtfrac{\pi}{2}             \\[1mm] \hline
         \sin & \frac{1}{2} \cdot \sqrt{0} & \frac{1}{2} \cdot \sqrt{1} & \frac{1}{2} \cdot \sqrt{2} & \frac{1}{2} \cdot \sqrt{3} & \frac{1}{2} \cdot \sqrt{4} \\[1mm]
         \cos & \frac{1}{2} \cdot \sqrt{4} & \frac{1}{2} \cdot \sqrt{3} & \frac{1}{2} \cdot \sqrt{2} & \frac{1}{2} \cdot \sqrt{1} & \frac{1}{2} \cdot \sqrt{0} \\[1mm]
         \tan & 0                          & \frac{\sqrt{3}}{3}         & 1                          & \sqrt{3}                   & -                          \\[1mm]
              & 0^{\circ}                  & 30^{\circ}                 & 45^{\circ}                 & 60^{\circ}                 & 90^{\circ}                
        \end{array}$
\end{center}
\fi

Diese Werte sollte man sich merken. Die trigonometrischen Funktionen f\"ur andere Winkel sind in Tabellen bzw. im Taschenrechener gespeichert.

Im Beispiel \MRef{Mathematik_ElementareGeometrie:Trigonometrie_Beispiel:gleichseitigesDreieck} wurde mithilfe der trigonometrischen Funktionen die H\"ohe $h$ des Dreiecks berechnet. Diese Vorgehensweise gilt f\"ur alle beliebigen Dreiecke, da die H\"ohe $h$ das Dreieck immer in zwei rechtwinklige Dreiecke teilt und somit die trigonometrischen Funktionen angewandt werden k\"onnen.

\begin{MExercise}
Berechnen Sie den Fl\"acheninhalt eines Dreiecks mit den Seiten $c=7$ und $b=3$, sowie dem Winkel $\alpha=30^\circ$.

\begin{MHint}{L\"osung}
Der Fl\"acheninhalt ist $A=\frac{1}{2}\cdot c \cdot h_c$.
\[\sin\left(\alpha\right)=\frac{h_c}{b}\quad\Rightarrow\quad h_c=b\cdot\sin\left(\alpha\right)=3\cdot\sin\left(30^\circ\right)= 3\cdot\frac{1}{2}\]
\[\Rightarrow\quad A=\frac{1}{2}\cdot c\cdot b\cdot\sin\left(\alpha\right)=\frac{1}{2}\cdot 7 \cdot3\cdot\frac{1}{2}=\frac{21}{4} \MDFPeriod\]
\end{MHint}
\end{MExercise}

\end{MXContent}


\begin{MXContent}{Trigonometrie am Einheitskreis}{Einheitskreis}{STD}
\MLabel{VBKM05_Trigonometrie_Einheitskreis}
\MDeclareSiteUXID{VBKM05_Trigonometrie_Einheitskreis}


Im letzten Abschnitt wurden die trigonometrischen Funktionen anhand eines rechtwinkligen Dreiecks betrachtet. Die gefundenen Erkenntnisse gelten also f\"ur einen Winkelbereich von $0$ bis $90^\circ=\frac{\pi}{2}$.

Um die gewonnenen Erkenntnisse auf gr\"o\ss ere Winkel als $\pi/2$ ausdehnen zu k\"onnen, erweist sich der Blick auf den sogenannten Einheitskreis als besonders n\"utzlich.

\ifttm\else\newpage\fi

\begin{center}
%-%\MGraphicsSolo{Mathematik_ElementareGeometrie_Trigonometrie_Einheitskreis_a.png}{scale=0.8}
\MTikzAuto{%
\begin{tikzpicture}[x=0.024cm, y=2.4cm] 
\begin{scope}[xshift=-4cm,xscale=100]
%Koordinatensystem
\node (xMAX) at (1.3,0){};
\node (yMAX) at (0,1.4){};
\draw[color=black] (0,0) circle (1);
\draw[-stealth',color=black] (-1.2,0) -- (1.3,0);
\draw[-stealth',color=black] (0,-1.4) -- (0,1.4);
\draw (xMAX) node[anchor=north] {$x$};
\draw (yMAX) node[anchor=north east] {$y$};
\foreach \x in {-1, 1}
\draw[shift={(\x,0)},thick,color=black] (0,-0.05) -- (0,0.05) (0,0) node[anchor=north west] {\normalsize $\x$};
\foreach \y in {-1, 1}
\draw[shift={(0,\y)},thick,color=black] (0.05,0) -- (-0.05,0) (0,0) node[anchor=north west] {\normalsize $\y$};
\def\cAng{38}
\draw[-stealth',thick,color=black] (0,0) -- ({cos(\cAng)},{sin(\cAng)}) node[anchor=south west] {$P$};
\draw[shift={({cos(\cAng)},0)},very thick,color=blue] (0,0.05) -- (0,-0.05) node[anchor=north east] {\normalsize $\cos(\alpha)$};
\draw[shift={(0,{sin(\cAng)})},very thick,color=red] (0.05,0) -- (-0.05,0) node[anchor=east] {\normalsize $\sin(\alpha)$};
\draw[thick,color=blue] (0,0) -- ({cos(\cAng)},0.0);
\draw[thick,color=red] ({cos(\cAng)},0) -- ({cos(\cAng)},{sin(\cAng)});
\draw[color=black, thin] (0,0) ++(0:0.5) arc (0:\cAng:0.5);
\draw[color=black] (0,0) ++({0.5*\cAng}:0.35) node {$\alpha$};
\end{scope}
%Koordinatensystem
\node (xMAX) at (400.0,0){};
\node (yMAX) at (0,1.4){};
%%\draw[help lines, gray, dashed, xstep=1, ystep=1] (0,0) grid (5.5,2.8);
\draw[-stealth',color=black] (-20,0) -- (xMAX);
\foreach \x in {30, 60, 90, 120, 150, 180, 210, 240, 270, 300, 330, 360}
\draw[shift={(\x,0)},color=black] (0pt,0pt) -- (0pt,-6pt) node[below] {\scriptsize $\x^\circ$};
\draw[-stealth',color=black] (0,-1.4) -- (yMAX);
\foreach \y in {1}
\draw[shift={(0,\y)},color=black] (0pt,0pt) -- (-6pt,0pt)  node[left] {\normalsize $\pgfmathprintnumber{\y}$};
%Achsenbeschriftung
\draw (xMAX) node[anchor=north east] {$\alpha$};
\draw[color=red] (yMAX) node[anchor=north west] {$y=\sin(\alpha)$};
\draw[smooth,samples=73,domain=0:360, line width=1pt,color=red] plot(\x,{sin(\x)}); %{cos(\x r)}
%Beschriftung und Graphen
%%\clip(-2.8,-0.5) rectangle (6,3);
%%\draw[color=black] (7,3) node[anchor=south west] {$\MPointTwo{7}{3}$};
\end{tikzpicture}
}
%-%\MGraphicsSolo{Mathematik_ElementareGeometrie_Trigonometrie_Einheitskreis_b.png}{scale=0.8}
\MTikzAuto{%
\begin{tikzpicture}[x=0.024cm, y=2.4cm] 
%Koordinatensystem
\node (xMAX) at (400.0,0){};
\node (yMAX) at (0,1.4){};
%%\draw[help lines, gray, dashed, xstep=1, ystep=1] (0,0) grid (5.5,2.8);
\draw[-stealth',color=black] (-20,0) -- (xMAX);
\foreach \x in {30, 60, 90, 120, 150, 180, 210, 240, 270, 300, 330, 360}
\draw[shift={(\x,0)},color=black] (0pt,0pt) -- (0pt,-6pt) node[below] {\scriptsize $\x^\circ$};
\draw[-stealth',color=black] (0,-1.4) -- (yMAX);
\foreach \y in {1}
\draw[shift={(0,\y)},color=black] (0pt,0pt) -- (-6pt,0pt)  node[left] {\normalsize $\pgfmathprintnumber{\y}$};
%Achsenbeschriftung
\draw (xMAX) node[anchor=north east] {$\alpha$};
\draw[color=blue] (yMAX) node[anchor=north west] {$x=\cos(\alpha)$};
\draw[smooth,samples=73,domain=0:360, line width=1pt,color=blue] plot(\x,{cos(\x)}); %{cos(\x r)}
%Beschriftung und Graphen
%%\clip(-2.8,-0.5) rectangle (6,3);
%%\draw[color=black] (7,3) node[anchor=south west] {$\MPointTwo{7}{3}$};
\end{tikzpicture}
}
\end{center}

Der Einheitskreis ist ein Kreis mit Radius $1$ und Mittelpunkt im Ursprung des kartesischen Koordinatensystems.\\
\ \\
 Es werde ein Vektor vom Ursprung aus mit der L\"ange $1$ betrachtet. Dieser Vektor werde nun von seiner Ausgangslage auf der positiven $x$-Achse gegen den Uhrzeigersinn, also im mathematisch positiven Sinn um den Nullpunkt rotiert. Dabei \"uberstreicht seine Spitze den Einheitskreis, und er bildet mit der positiven $x$-Achse den Winkel $\alpha$, der bei der Rotation von $0$ bis $2\pi$ bzw. $360^\circ$ w\"achst. Zu jedem Winkel $\alpha$ geh\"ort also ein Punkt $P_\alpha$ mit den Koordinaten $x_\alpha$ und $y_\alpha$ auf dem Einheitskreis.
\ \\
F\"ur $\alpha\in\left[0\MIntvlSep \pi/2\right]$ kann man den Vektor, den zugeh\"origen $x$-Achsenabschnitt und den zugeh\"origen $y$-Achsenabschnitt als rechtwinkliges Dreieck ansehen, wie es vom letzten Kapitel her bekannt ist. Die Hypotenuse ist der Vektor mit der L\"ange $1$, der $x$-Achsenabschnitt ist die Ankathete und der $y$-Achsenabschnitt die Gegenkathete.\\
\ \\
Der Sinus des Winkels $\alpha$ ist also
\[\sin\left(\alpha\right)=\frac{y_\alpha}{1}=y_\alpha\]
und der Kosinus ist
\[\cos\left(\alpha\right)=\frac{x_\alpha}{1}=x_\alpha \MDFPeriod\]

Diese Definitionen gelten auch f\"ur die Winkel $\alpha>\pi/2$. Dabei k\"onnen die Werte f\"ur $x_\alpha$ und $y_\alpha$ auch negativ werden und damit auch der Kosinus bzw. Sinus. Tr\"agt man die $y$-Werte in Abh\"angigkeit vom Winkel $\alpha$ in ein Diagramm, so erh\"alt man die rote Kurve, f\"ur die $x$-Werte erh\"alt man die blaue Kurve.
\ \\
Mit dem Satz von Pythagoras gilt au\ss erdem
\[x_{\alpha}^{2}+y_{\alpha}^{2}=1 \MDFPeriod\]
Setzt man hier die Beziehungen f\"ur $x_{\alpha}$ und $y_{\alpha}$ mit den Winkelfunktionen ein, so ergibt sich f\"ur beliebige Winkel $\alpha$ die wichtige Beziehung
\[\sin^2\left(\alpha\right)+\cos^2\left(\alpha\right)=1 \MDFPeriod\]


\begin{MExample}
Gesucht sind jeweils die Werte des Sinus, Kosinus und Tangens des Winkels $\alpha=315^\circ$.\\

F\"ur $\alpha=315^\circ$ liegt der Punkt $P_\alpha$ im 4. Quadranten, der zugeh\"orige Vektor bildet mit den zugeh\"origen Achsenabschnitten ein gleichschenkliges Dreieck. Es gilt: \[\left|x_\alpha\right|=\left|y_\alpha\right|\quad\Rightarrow\quad \left|x_\alpha\right|^2+\left|y_\alpha\right|^2=2\cdot\left|x_\alpha\right|^2=1\]
\[\Rightarrow\quad \left|x_\alpha\right|=\left|y_\alpha\right|=\frac{1}{\sqrt 2}=\frac{1}{2}\sqrt 2 \MDFPSpace,\]
\[\cos\left(\alpha\right)=x_\alpha=\frac{1}{2}\sqrt 2 \MDFPSpace,	\MDFPaSpace	\sin\left(\alpha\right)=y_\alpha=-\frac{1}{2}\sqrt 2 \MDFPSpace,	\MDFPaSpace	\tan\left(\alpha\right)=\frac{y_\alpha}{x_\alpha}=-1 \MDFPeriod\]
\end{MExample}

\end{MXContent}


\begin{MExercises}
\MDeclareSiteUXID{VBKM05_Trigonometrie_Exercises}
\begin{MExercise}
Welcher Winkel geh\"ort zu dem Punkt $P_\alpha\left(-\MZahl{0}{643}, -\MZahl{0}{766}\right)$? 
\begin{MHint}{Hinweis}
Verwenden Sie dazu den Taschenrechner, aber vertrauen Sie ihm nicht blind!
\end{MHint}

\begin{MHint}{L\"osung}
Aus den Koordinaten des Punktes $P_\alpha$ ergibt sich:
\[\sin\left(\alpha\right)=-\MZahl{0}{766}\quad	\text{und}	\quad\cos\left(\alpha\right)=-\MZahl{0}{643} \MDFPeriod\]
Tippen Sie in den Taschenrechner:\\
invers(sin(-\MZahl{0}{766})) bzw. $\sin^{-1}$(-\MZahl{0}{766}), so erhalten Sie ungef\"ahr $-50^\circ$, und \\
invers(cos(-\MZahl{0}{643})) bzw. $\sin^{-1}$(-\MZahl{0}{643}), so erhalten Sie ungef\"ahr $130^\circ$.\\
Au\ss erdem wissen Sie, dass der Punkt im 3. Quadranten ist, also ein Winkel im Bereich zwischen $180^\circ$ und $270^\circ$ herauskommen muss.

%-%\MGraphicsSolo{Mathematik_ElementareGeometrie_Trigonometrie_Aufgabe_Einheitskreis.png}{scale=1}
\MTikzAuto{%
\begin{tikzpicture}[x=2.6cm, y=2.6cm] 
%Koordinatensystem
\node (xMAX) at (1.3,0){};
\node (yMAX) at (0,1.3){};
\draw[color=black] (0,0) circle (1);
\draw[-stealth',color=black] (-1.2,0) -- (1.3,0);
\draw[-stealth',color=black] (0,-1.3) -- (0,1.3);
\draw (xMAX) node[anchor=north east] {$x$};
\draw (yMAX) node[anchor=north east] {$y$};
\foreach \x in {-1, 1}
\draw[shift={(\x,0)},thick,color=black] (0,-0.05) -- (0,0.05) (0,0) node[anchor=north west] {\normalsize $\x$};
\foreach \y in {-1, 1}
\draw[shift={(0,\y)},thick,color=black] (0.05,0) -- (-0.05,0) (0,0) node[anchor=north west] {\normalsize $\y$};
\def\cRad{1.3}
\def\cAng{50}
\draw[thick,color=black] (0,0) -- ({\cRad*cos(\cAng)},{-\cRad*sin(\cAng)});
\draw[thick,color=black] (0,0) -- (1,0);
\draw[thick,color=black] (0,0) -- ({-\cRad*cos(\cAng)},{\cRad*sin(\cAng)});
\draw[color=black, thin] (0,0) ++({-\cAng}:0.5) arc ({-\cAng}:{180-\cAng}:0.5);
\draw[color=black] (0,0) ++({-0.5*\cAng}:0.35) node {$-\cAng^\circ$};
\draw[color=black] (0,0) ++(45:0.30) node {$130^\circ$}; %({90-0.5*\cAng}:0.35)
\draw[color=magenta, thin] (0,0) ++({-\cAng}:0.60) arc ({-\cAng}:{180+\cAng}:0.60);
\draw[thick,color=magenta] (0,0) -- ({-cos(\cAng)},{-sin(\cAng)});
\draw[thick,color=red] ({cos(\cAng)},0) -- ({cos(\cAng)},{-sin(\cAng)});
\draw[thick,color=black,dashed] ({-cos(\cAng)},0) -- ({-cos(\cAng)},{sin(\cAng)});
\draw[color=black,dashed] ({-cos(\cAng)},0) -- ({-cos(\cAng)},{-sin(\cAng)-0.1});
\draw[color=black,dashed] ({-cos(\cAng)-0.1},{-sin(\cAng)}) -- (0,{-sin(\cAng)});
\draw[thick,color=black,red] (-0.05,{-sin(\cAng)}) -- (0.05,{-sin(\cAng)});
\end{tikzpicture}
}
\begin{minipage}[b]{10cm}
Anhand des Bildes kann man erkennen, dass der negative Sinuswert zwar zum Winkel $-50^\circ$, aber auch zu $\alpha=\left(180^\circ+50^\circ\right)=230^\circ$ geh\"ort.\\
\ \\
Ebenso kann der negative Kosinuswert zu $130^\circ$, aber auch zu $\alpha=-130^\circ=\left(360^\circ-130^\circ\right)=230^\circ$ geh\"oren.\\
\ \\
Der richtige Winkel ist also $\alpha=230^\circ$ (rosa).
\vspace*{2cm}
\end{minipage}
\end{MHint}

\end{MExercise}


\begin{MExercise}
\begin{enumerate}
\item F\"ur ein bei~$C$ rechtwinkliges Dreieck seien $b = \MZahl{2}{53}\MEinheit{cm}$ und
	$c = \MZahl{3}{88}\MEinheit{cm}$ gegeben. Geben Sie $\sin \left( \alpha \right)$,
       $\sin \left( \beta \right)$ und $a$ an!
       
       \begin{MHint}{L\"osung}
       \[
          a
        = \sqrt{c^2 - b^2}
	= \sqrt{\left( \MZahl{3}{88}\MEinheit{cm} \right)^2 - \left( \MZahl{2}{53}\MEinheit{cm} \right)^2}
	= \sqrt{\MZahl{15}{0544}\MEinheit{cm}^2 - \MZahl{6}{4009}\MEinheit{cm}^2}
	= \sqrt{\MZahl{8}{6535}}\MEinheit{cm} \MDFPSpace,
       \]
       \[
          \sin \left( \alpha \right)
        = \frac{a}{c}
	= \frac{\sqrt{\MZahl{8}{6535}}\MEinheit{cm}}{\MZahl{3}{88}\MEinheit{cm}}
        = \frac{\sqrt{86535}}{388}
        \qquad \text{und} \qquad
          \sin \left( \beta \right)
        = \frac{b}{c}
	= \frac{\MZahl{2}{53}\MEinheit{cm}}{\MZahl{3}{88}\MEinheit{cm}}
        = \frac{253}{388} \MDFPeriod
       \]
       Numerisch ergibt sich
       $a \approx \MZahl{2}{9417}\MEinheit{cm}$, $\sin \left( \alpha \right) \approx \MZahl{0}{7587}$
       und $\sin \left( \beta \right) \approx \MZahl{0}{65201}$.
       \end{MHint}
 \item Berechnen Sie den Fl\"acheninhalt eines Dreiecks mit $\beta = \frac{11 \pi}{36}$,
       $a = 4\MEinheit{m}$ und $c = 60\MEinheit{cm}$!
       
       \ifttm(Geben Sie Ihr Ergebnis auf drei Nachkommastellen genau oder als Ausdruck an!
              Dabei wird der Sinus eines Winkels~$\Mvarphi$ als \glqq\texttt{sin(}$\Mvarphi$\texttt{)}\grqq\ 
              und die Zahl~$\pi$ als \glqq\texttt{pi}\grqq\ geschrieben.)
	      Anzahl der $\MEinheit{m}^2$: \MLParsedQuestion{25}{4*sin(11*pi/36)*6/20}{3}{GEO8}
       \else\relax\fi
       \begin{MHint}{L\"osung}
       \[
                \frac{\left( a \cdot \sin \left( \beta \right) \right) \cdot c}{2}
	=    \sin \left( \Mtfrac{11 \pi}{36} \right) \cdot \MZahl{1}{2}\MEinheit{m}^2
	\approx \MZahl{0}{98298}\MEinheit{m}^2 \MDFPeriod
       \]
       \end{MHint}
\end{enumerate}
\end{MExercise}

\end{MExercises}


\MSubsection{Abschlusstest}

\begin{MTest}{Abschlusstest Modul \arabic{section}}
\MLabel{M05_Abschlusstest}
\MDeclareSiteUXID{VBKM05_Abschlusstest}

\begin{MExercise}
In der folgenden Tabelle sind f\"ur ein rechtwinkliges Dreieck einige Gr\"o\ss en gegeben, berechnen Sie die \"ubrigen Gr\"o\ss en. Verwenden Sie f\"ur die Winkel stets das Bogenma\ss :\\
\MInputHint{Geben Sie Kreiszahl einfach als Text ein, zum Beispiel tippen Sie $\frac23\pi$ ein als \texttt{2/3*pi}.}
\ \\
\begin{center}
\MTikzAuto{%
\begin{tikzpicture}[x=1.0cm, y=1.0cm] 
\def\anga{31}\def\angb{53}\def\lenc{6.5}
\pgfmathparse{180-\anga-\angb}\let\angc=\pgfmathresult
\pgfmathparse{\lenc*sin(\angb)*cos(\anga)/sin(\angc)}\let\cx=\pgfmathresult
\pgfmathparse{\lenc*sin(\angb)*sin(\anga)/sin(\angc)}\let\cy=\pgfmathresult
%-%\foreach \sx/\sy/\dsf/\ang in {0.0cm/0.0cm/1.0/-12} {
\def\sx{0.0cm}\def\sy{0.0cm}\def\dsf{1.0}\def\ang{-12}
\begin{scope}[xshift=\sx,yshift=\sy,rotate=\ang]
\coordinate (OB) at (\lenc,0);
\coordinate (OC) at (\cx,\cy);
\coordinate (A) at (0,0);
\coordinate (B) at ($ (A)!\dsf!(OB) $);
\coordinate (C) at ($ (A)!\dsf!(OC) $);
\coordinate (MAB) at ($ (A)!0.5!(B) $);
\coordinate (MBC) at ($ (B)!0.5!(C) $);
\coordinate (MCA) at ($ (C)!0.5!(A) $);
\fill[gray] (A) -- (B) -- (C) -- cycle;
\draw[color=black] (MBC) ++({90-\angb}:0.3) node {$a$};
\draw[color=black] (MCA) ++({90+\anga}:0.3) node {$b$};
\draw[color=black] (MAB) ++(270:0.3) node {$c$};
\pgfmathparse{0.75/sin(\anga)}\let\sncia=\pgfmathresult
\pgfmathparse{0.75/sin(\angb)}\let\sncib=\pgfmathresult
\pgfmathparse{0.75/sin(\angc)}\let\sncic=\pgfmathresult
\fill[color=gray!50!white] (A) -- ++(0:\sncia) arc (0:\anga:\sncia) -- cycle;
\fill[color=gray!50!white] (B) -- ++({180-\angb}:\sncib) arc ({180-\angb}:180:\sncib) -- cycle;
\fill[color=gray!50!white] (C) -- ++({180+\anga}:\sncic) arc ({180+\anga}:{360-\angb}:\sncic) -- cycle;
\draw[color=black] (A) ++(0:\sncia) arc (0:\anga:\sncia);
\draw[color=black] (B) ++({180-\angb}:\sncib) arc ({180-\angb}:180:\sncib);
\draw[color=black] (C) ++({180+\anga}:\sncic) arc ({180+\anga}:{360-\angb}:\sncic);
\pgfmathparse{0.50/sin(\anga)}\let\sncia=\pgfmathresult
\pgfmathparse{0.50/sin(\angb)}\let\sncib=\pgfmathresult
\pgfmathparse{0.50/sin(\angc)}\let\sncic=\pgfmathresult
\draw[color=black] (A) ++({0.5*\anga}:\sncia) node {$\alpha$};
\draw[color=black] (B) ++({180-0.5*\angb}:\sncib) node {$\beta$};
\draw[color=black] (C) ++({270+0.5*(\anga-\angb)}:\sncic) node {$\gamma$};
%
\draw[black,very thick] (A) -- (B) -- (C) -- cycle;
\end{scope}
%%}
\end{tikzpicture}
}
\par
Winkel erhalten die griechischen Buchstaben zu den gegen\"uberliegenden Seiten
\end{center}
%-%\MUGraphics{dreieck_einfach.png}{width=0.4\linewidth}{Winkel erhalten die griechischen Buchstaben zu den gegen\"uberliegenden Seiten\MCopyrightLabel{VBKM05_Abbildung_dreieck_einfach}}{width:400px}
%-%\MCopyrightNotice{\MCCLicense}{http://commons.wikimedia.org/wiki/File:Dreieck.svg#mediaviewer/Datei:Dreieck.svg}{http://de.wikipedia.org/wiki/Benutzer:Thire}{Modifiziert mit GIMP}{VBKM05_Abbildung_dreieck_einfach}
%\MCopyrightHTML{CC BY-SA 5.0}{http://commons.wikimedia.org/wiki/File:Dreieck.svg#mediaviewer/Datei:Dreieck.svg}{http://de.wikipedia.org/wiki/Benutzer:Thire}

\ifttm
\def\VBKMFUENFWIDTHDREIECKE{9}
\else
\def\VBKMFUENFWIDTHDREIECKE{5}
\fi

\begin{center}
\begin{tabular}{|c|c|c|c|c|c|c|c|c|}
\hline
$a$ & $b$ & $c$ & $\alpha$ & $\beta$ & $\gamma$& $\sin(\alpha)$ & $\sin(\beta)$ & $\sin(\gamma)$ \\ \hline
$1$ & $1$ & $1$ & $\frac13\pi$ & $\frac13\pi$ & $\frac13\pi$ & $\frac12\sqrt3$& $\frac12\sqrt3$& $\frac12\sqrt3$ \\
$1$ & $1$ & \MLParsedQuestion{\VBKMFUENFWIDTHDREIECKE}{sqrt(2)}{5}{WR1} & \MLParsedQuestion{\VBKMFUENFWIDTHDREIECKE}{1/4*pi}{5}{WR21} & \MLParsedQuestion{\VBKMFUENFWIDTHDREIECKE}{1/4*pi}{5}{WR8} & $\frac12\pi$ & \MLParsedQuestion{\VBKMFUENFWIDTHDREIECKE}{1/sqrt(2)}{5}{WR13} & \MLParsedQuestion{\VBKMFUENFWIDTHDREIECKE}{1/sqrt(2)}{5}{PARSEDQUEST3} & \MLParsedQuestion{\VBKMFUENFWIDTHDREIECKE}{1}{5}{WR17}   \\ 
$3$ & $4$ & \MLParsedQuestion{\VBKMFUENFWIDTHDREIECKE}{5}{5}{WR2} & $*$ & $*$ & $\frac12\pi$ & \MLParsedQuestion{\VBKMFUENFWIDTHDREIECKE}{4/5}{5}{WR7} & \MLParsedQuestion{\VBKMFUENFWIDTHDREIECKE}{3/5}{5}{PARSEDQUEST4} & \MLParsedQuestion{\VBKMFUENFWIDTHDREIECKE}{1}{5}{WR12} \\
\MLParsedQuestion{\VBKMFUENFWIDTHDREIECKE}{sqrt(15)}{5}{WR3} & $\sqrt7$ & $\sqrt8$ & \MLParsedQuestion{\VBKMFUENFWIDTHDREIECKE}{pi/2}{5}{WR6} & $*$ & $*$ & $1$ & \MLParsedQuestion{\VBKMFUENFWIDTHDREIECKE}{sqrt(8/15)}{5}{WR11} & \MLParsedQuestion{\VBKMFUENFWIDTHDREIECKE}{sqrt(7/15)}{5}{WR14} \\
\MLParsedQuestion{\VBKMFUENFWIDTHDREIECKE}{9/sqrt(3)}{5}{WR4} & \MLParsedQuestion{\VBKMFUENFWIDTHDREIECKE}{9/sqrt(3)}{5}{WR5} & $9$ & \MLParsedQuestion{\VBKMFUENFWIDTHDREIECKE}{pi/6}{5}{WR9} & \MLParsedQuestion{\VBKMFUENFWIDTHDREIECKE}{pi/6}{5}{WR10}& \MLParsedQuestion{\VBKMFUENFWIDTHDREIECKE}{2/3*pi}{5}{WR15}& $\frac12$ & $\frac12$ & $*$\\
\hline
\end{tabular}
\end{center}
Die mit $*$ gekennzeichneten Felder m\"ussen nicht ausgef\"ullt werden.

\begin{MHint}{L\"osung}
Die ausgef\"ullte Tabelle lautet wie folgt:
\begin{center}
\begin{tabular}{|c|c|c|c|c|c|c|c|c|}
\hline
$a$ & $b$ & $c$ & $\alpha$ & $\beta$ & $\gamma$& $\sin(\alpha)$ & $\sin(\beta)$ & $\sin(\gamma)$ \\ \hline
$1$ & $1$ & $1$ & $\frac13\pi$ & $\frac13\pi$ & $\frac13\pi$ & $\frac12\sqrt3$& $\frac12\sqrt3$& $\frac12\sqrt3$ \\
$1$ & $1$ & $\sqrt2$ & $\frac14\pi$ & $\frac14\pi$ & $\frac12\pi$ & $\frac1{\sqrt2}$ & $\frac1{\sqrt2}$ & $1$   \\ 
$3$ & $4$ & $5$ & $*$ & $*$ & $\frac12\pi$ & $\frac45$ & $\frac35$ & $1$ \\
$\sqrt{15}$ & $\sqrt7$ & $\sqrt8$ & $\frac12\pi$ & $*$ & $*$ & $1$ & $\sqrt{8/15}$ & $\sqrt{7/15}$ \\
$\frac{9}{\sqrt3}$ & $\frac{9}{\sqrt3}$ & $9$ & $\frac16\pi$ &$\frac16\pi$ & $\frac23\pi$ & $\frac12$ & $\frac12$ & $*$\\
\hline
\end{tabular}
\end{center}
In der dritten Zeile benutzt man den rechten Winkel $\gamma=\frac12\pi$, der \"uber den Satz von Pythagoras auf $c=\sqrt{a^2+b^2}=\sqrt2$ f\"uhrt.
Das Dreieck ist gleichschenklig, also auch $\alpha=\beta$. Da die Winkelsumme
im Dreieck $\pi$ betr\"agt, muss $\alpha=\beta=\frac14\pi$ sein. In den beiden letzten L\"osungsfeldern ist dann $\sin(\frac14\pi)=\frac{\text{Gegenkathete}}{\text{Hypotenuse}}=\frac1{\sqrt2}$.
\ \\ \ \\
In der vierten Zeile ist das Dreieck rechtwinklig bei $\gamma=\pi$ und der Satz des Pythagoras f\"uhrt auf $c=\sqrt{a^2+b^2}=5$. Aus den Seiten kann man die Sinuswerte der Winkel bestimmen: $\sin(\alpha)=\frac45$, $\sin(\beta)=\frac35$ und $\sin(\gamma)=1$.
\ \\ \ \\
In der f\"unften Zeile ist das Dreieck wieder rechtwinklig, diesmal aber bei $\alpha=\pi$, daher spielen die Seiten nun andere Rollen: $a$ ist die L\"ange der Hypotenuse.
Der Satz des Pythagoras liefert in dieser Situation $a=\sqrt{b^2+c^2}=\sqrt{15}$. Da alle Seiten vorliegen, lassen sich die Winkel wieder bestimmen (jetzt mit $a$ als Hypotenuse): $\sin(\alpha)=1$, $\sin(\beta)=\sqrt{\frac{8}{15}}$ und $\sin(\gamma)=\sqrt{\frac{}{15}}$.
\ \\ \ \\
In der letzten Zeile rechnet man aus $\sin(\alpha)=\sin(\beta)=\frac12$ die Winkel $\alpha=\beta=\frac16\pi$ aus. Wegen der Winkelsumme $\alpha+\beta+\gamma=\pi$ folgt $\gamma=\pi-\frac13\pi=\frac23\pi$.
Dieses Dreieck ist gleichschenklig, aber nicht rechtwinklig, daher d\"urfen die trigonometrischen Funktionen nicht direkt an den Seiten angesetzt werden. Als Hilfsmittel kann man die H\"ohe auf der Seite $c$
benutzen, die das Dreieck in zwei rechtwinklige Dreiecke zerlegt:

\begin{center}
\MTikzAuto{%
\begin{tikzpicture}[x=1.0cm, y=1.0cm] 
\def\anga{68}\def\angb{68}\def\lenc{3.5}\def\lenrf{0.60}\def\lenlf{0.40}
\pgfmathparse{180-\anga-\angb}\let\angc=\pgfmathresult
\pgfmathparse{\lenc*sin(\angb)*cos(\anga)/sin(\angc)}\let\cx=\pgfmathresult
\pgfmathparse{\lenc*sin(\angb)*sin(\anga)/sin(\angc)}\let\cy=\pgfmathresult
%-%\foreach \sx/\sy/\dsf/\ang in {0.0cm/0.0cm/1.0/-12} {
\def\sx{0.0cm}\def\sy{0.0cm}\def\dsf{1.0}\def\ang{0}
\begin{scope}[xshift=\sx,yshift=\sy,rotate=\ang]
\coordinate (OB) at (\lenc,0);
\coordinate (OC) at (\cx,\cy);
\coordinate (A) at (0,0);
\coordinate (B) at ($ (A)!\dsf!(OB) $);
\coordinate (C) at ($ (A)!\dsf!(OC) $);
\coordinate (MAB) at ($ (A)!0.5!(B) $);
\coordinate (MBC) at ($ (B)!0.5!(C) $);
\coordinate (MCA) at ($ (C)!0.5!(A) $);
\fill[gray!6.25!white] (A) -- (B) -- (C) -- cycle;
\draw[color=black] (MBC) ++({90-\angb}:0.3) node {$a$};
\draw[color=black] (MCA) ++({90+\anga}:0.3) node {$b$};
\draw[color=black] (MAB) ++(270:0.3) node {$c$};
\pgfmathparse{\lenrf/sin(\anga)}\let\sncia=\pgfmathresult
\pgfmathparse{\lenrf/sin(\angb)}\let\sncib=\pgfmathresult
\pgfmathparse{\lenrf/sin(\angc)}\let\sncic=\pgfmathresult
\fill[color=gray!25!white] (A) -- ++(0:\sncia) arc (0:\anga:\sncia) -- cycle;
\fill[color=gray!25!white] (B) -- ++({180-\angb}:\sncib) arc ({180-\angb}:180:\sncib) -- cycle;
\fill[color=white] (C) -- ++({180+\anga}:\sncic) arc ({180+\anga}:{360-\angb}:\sncic) -- cycle;
\draw[color=black] (A) ++(0:\sncia) arc (0:\anga:\sncia);
\draw[color=black] (B) ++({180-\angb}:\sncib) arc ({180-\angb}:180:\sncib);
\draw[color=black] (C) ++({180+\anga}:\sncic) arc ({180+\anga}:{360-\angb}:\sncic);
\pgfmathparse{\lenlf/sin(\anga)}\let\sncia=\pgfmathresult
\pgfmathparse{\lenlf/sin(\angb)}\let\sncib=\pgfmathresult
\pgfmathparse{\lenlf/sin(\angc)}\let\sncic=\pgfmathresult
\draw[color=black] (A) ++({0.5*\anga}:\sncia) node {$\alpha$};
\draw[color=black] (B) ++({180-0.5*\angb}:\sncib) node {$\beta$};
%%\draw[color=black] (C) ++({270+0.5*(\anga-\angb)}:\sncic) node {$\gamma$};
%
\draw[black,very thick] (A) -- (B) -- (C) -- cycle;
\draw[black] (MAB) -- (C);
\end{scope}
%%}
\end{tikzpicture}
}
\par
Einteilung eines gleichschenkligen Dreiecks in zwei rechtwinklige Dreiecke.
\end{center}
%-%\MUGraphics{dreieck_haelfte.png}{width=0.2\linewidth}{Einteilung eines gleichschenkligen Dreiecks in zwei rechtwinklige Dreiecke.}{width:150px}
%\MCopyrightHTML{Wikipedia ver\"andert}{<a href="http://commons.wikimedia.org/wiki/File:Isosceles-triangle-tikz.svg#mediaviewer/Datei:Isosceles-triangle-tikz.svg">Isosceles-triangle-tikz</a> von <a href="//commons.wikimedia.org/wiki/User:MartinThoma" title="User:MartinThoma">MartinThoma</a> - <span class="int-own-work">Eigenes Werk</span>. Lizenziert unter <a title="Creative Commons Attribution 3.0" href="http://creativecommons.org/licenses/by/3.0">CC BY 3.0</a> \"uber <a href="//commons.wikimedia.org/wiki/">Wikimedia Commons</a>}

Im linken Teildreieck gibt es einen rechten Winkel, dort ist nun $b$ die Hypotenuse und Ankathete von $\alpha$ besitzt die bekannte L\"ange $\frac{9}{2}$. Wegen $\cos(\frac16\pi)=\frac12\sqrt3$ kann man
nach $b$ aufl\"osen:
$$
\frac12\sqrt3 \;=\; \cos(\frac16\pi) \;=\; \frac{\text{Ankathete im Teildreieck}}{\text{Hypotenuse im Teildreieck}} \;=\; \frac{\frac92}{b} \;\; \Rightarrow\;\;
b \;=\; \frac{9}{\sqrt3} \MDFPeriod
$$
und weil das urspr\"ungliche Dreieck gleichschenklig ist gilt auch $a=\frac{9}{\sqrt3}$.
\end{MHint}

\end{MExercise}



\begin{MExercise}
\MLabel{VBKM05_A_Quadratkreise}
Ein Quadrat mit Seitenl\"ange $a$ sei gegeben. Geben Sie Formeln an f\"ur Fl\"acheninhalt und Umfang des gr\"o\ss tm\"oglichen Kreises innerhalb des Quadrats, sowie f\"ur
den kleinstm\"oglichen Kreis, der das Quadrat enth\"alt:
\begin{MExerciseItems}
\item{Umfang des Kreises im Quadrat in Abh\"angigkeit von der Seitenl\"ange $a$: \MLSimplifyQuestion{14}{pi*a}{14}{a}{4}{1}{NGEOM1}}
\item{Fl\"acheninhalt des Kreises im Quadrat in Abh\"angigkeit von der Seitenl\"ange $a$: \MLSimplifyQuestion{14}{(1/4)*pi*a*a}{14}{a}{4}{1}{NGEOM2}}
\item{Umfang des Kreises um das Quadrat in Abh\"angigkeit von der Seitenl\"ange $a$: \MLSimplifyQuestion{14}{sqrt(2)*pi*a}{14}{a}{4}{1}{NGEOM3}}
\item{Fl\"acheninhalt des Kreises um das Quadrat in Abh\"angigkeit von der Seitenl\"ange $a$: \MLSimplifyQuestion{14}{pi*a*a/2}{14}{a}{4}{1}{NGEOM4}}
\end{MExerciseItems}
\ \\ \ \\
In den Antwortfeldern d\"urfen keine Klammern oder Wurzelausdr\"ucke auftauchen, schreiben Sie beispielsweise $2^{0.5}$ statt $\sqrt{2}$ um die Wurzel zu vermeiden.
\ \\ \ \\
\begin{MHint}{L\"osung}
Der mittig im Quadrat liegende Kreis besitzt den Radius $r=\frac12a$, die halbe Seitenl\"ange des Quadrats. Folglich besitzt er den Umfang $2\pi r =2\pi \cdot \frac12a = \pi a$ und den Fl\"acheninhalt
$\pi r^2=\pi \cdot \frac14a^2 =\frac14\pi a^2$.\\
\ \\
Befindet sich das Quadrat dagegen mittig innerhalb des Kreises, so ist sein Radius die H\"alfte der L\"ange der Diagonalen vom Quadrat. Diese besitzt
die L\"ange $d=\sqrt2\cdot a$. Dies folgt aus dem Satz von Pythagoras, da die halbe Quadratdiagonale ein rechtwinkliges Dreieck bildet
mit Seitenl\"ange $d$ der Hypotenuse und den beiden Katheten mit L\"angen $\frac12a$: Also ist der Umfang $2\pi r=2\pi\cdot \sqrt2 \cdot\frac12 a=\sqrt2\cdot \pi \cdot a$,
der Fl\"acheninhalt ist $\pi r^2=\pi \cdot (\sqrt2\cdot \frac12\cdot a)^2=\pi \cdot\frac12 a^2$.
\end{MHint}
\end{MExercise}

\end{MTest}

\end{document}
