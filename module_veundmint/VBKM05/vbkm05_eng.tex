% MINTMOD Version P0.1.0, needs to be consistent with preprocesser object in tex2x and MPragma-Version at the end of this file

% Parameter aus Konvertierungsprozess (PDF und HTML-Erzeugung wenn vom Konverter aus gestartet) werden hier eingefuegt, Preambleincludes werden am Schluss angehaengt

\newif\ifttm                % gesetzt falls Uebersetzung in HTML stattfindet, sonst uebersetzung in PDF

% Wahl der Notationsvariante ist im PDF immer std, in der HTML-Uebersetzung wird vom Konverter die Auswahl modifiziert
\newif\ifvariantstd
\newif\ifvariantunotation
\variantstdtrue % Diese Zeile wird vom Konverter erkannt und ggf. modifiziert, daher nicht veraendern!


\def\MOutputDVI{1}
\def\MOutputPDF{2}
\def\MOutputHTML{3}
\newcounter{MOutput}

\ifttm
\usepackage{german}
\usepackage{array}
\usepackage{amsmath}
\usepackage{amssymb}
\usepackage{amsthm}
\else
\documentclass[ngerman,oneside]{scrbook}
\usepackage{etex}
\usepackage[latin1]{inputenc}
\usepackage{textcomp}
\usepackage[ngerman]{babel}
\usepackage[pdftex]{color}
\usepackage{xcolor}
\usepackage{graphicx}
\usepackage[all]{xy}
\usepackage{fancyhdr}
\usepackage{verbatim}
\usepackage{array}
\usepackage{float}
\usepackage{makeidx}
\usepackage{amsmath}
\usepackage{amstext}
\usepackage{amssymb}
\usepackage{amsthm}
\usepackage[ngerman]{varioref}
\usepackage{framed}
\usepackage{supertabular}
\usepackage{longtable}
\usepackage{maxpage}
\usepackage{tikz}
\usepackage{tikzscale}
\usepackage{tikz-3dplot}
\usepackage{bibgerm}
\usepackage{chemarrow}
\usepackage{polynom}
%\usepackage{draftwatermark}
\usepackage{pdflscape}
\usetikzlibrary{calc}
\usetikzlibrary{through}
\usetikzlibrary{shapes.geometric}
\usetikzlibrary{arrows}
\usetikzlibrary{intersections}
\usetikzlibrary{decorations.pathmorphing}
\usetikzlibrary{external}
\usetikzlibrary{patterns}
\usetikzlibrary{fadings}
\usepackage[colorlinks=true,linkcolor=blue]{hyperref} 
\usepackage[all]{hypcap}
%\usepackage[colorlinks=true,linkcolor=blue,bookmarksopen=true]{hyperref} 
\usepackage{ifpdf}

\usepackage{movie15}

\setcounter{tocdepth}{2} % In Inhaltsverzeichnis bis subsection
\setcounter{secnumdepth}{3} % Nummeriert bis subsubsection

\setlength{\LTpost}{0pt} % Fuer longtable
\setlength{\parindent}{0pt}
\setlength{\parskip}{8pt}
%\setlength{\parskip}{9pt plus 2pt minus 1pt}
\setlength{\abovecaptionskip}{-0.25ex}
\setlength{\belowcaptionskip}{-0.25ex}
\fi

\ifttm
\newcommand{\MDebugMessage}[1]{\special{html:<!-- debugprint;;}#1\special{html:; //-->}}
\else
%\newcommand{\MDebugMessage}[1]{\immediate\write\mintlog{#1}}
\newcommand{\MDebugMessage}[1]{}
\fi

\def\MPageHeaderDef{%
\pagestyle{fancy}%
\fancyhead[r]{(C) VE\&MINT-Projekt}
\fancyfoot[c]{\thepage\\--- CCL BY-SA 3.0 ---}
}


\ifttm%
\def\MRelax{}%
\else%
\def\MRelax{\relax}%
\fi%

%--------------------------- Uebernahme von speziellen XML-Versionen einiger LaTeX-Kommandos aus xmlbefehle.tex vom alten Kasseler Konverter ---------------

\newcommand{\MSep}{\left\|{\phantom{\frac1g}}\right.}

\newcommand{\ML}{L}

\newcommand{\MGGT}{\mathrm{ggT}}


\ifttm
% Verhindert dass die subsection-nummer doppelt in der toccaption auftaucht (sollte ggf. in toccaption gefixt werden so dass diese Ueberschreibung nicht notwendig ist)
\renewcommand{\thesubsection}{}
% Kommandos die ttm nicht kennt
\newcommand{\binomial}[2]{{#1 \choose #2}} %  Binomialkoeffizienten
\newcommand{\eur}{\begin{html}&euro;\end{html}}
\newcommand{\square}{\begin{html}&square;\end{html}}
\newcommand{\glqq}{"'}  \newcommand{\grqq}{"'}
\newcommand{\nRightarrow}{\special{html: &nrArr; }}
\newcommand{\nmid}{\special{html: &nmid; }}
\newcommand{\nparallel}{\begin{html}&nparallel;\end{html}}
\newcommand{\mapstoo}{\begin{html}<mo>&map;</mo>\end{html}}

% Schnitt und Vereinigungssymbole von Mengen haben zu kleine Abstaende; korrigiert:
\newcommand{\ccup}{\,\!\cup\,\!}
\newcommand{\ccap}{\,\!\cap\,\!}


% Umsetzung von mathbb im HTML
\renewcommand{\mathbb}[1]{\begin{html}<mo>&#1opf;</mo>\end{html}}
\fi

%---------------------- Strukturierung ----------------------------------------------------------------------------------------------------------------------

%---------------------- Kapselung des sectioning findet auf drei Ebenen statt:
% 1. Die LateX-Befehl
% 2. Die D-Versionen der Befehle, die nur die Grade der Abschnitte umhaengen falls notwendig
% 3. Die M-Versionen der Befehle, die zusaetzliche Formatierungen vornehmen, Skripten starten und das HTML codieren
% Im Modultext duerfen nur die M-Befehle verwendet werden!

\ifttm

  \def\Dsubsubsubsection#1{\subsubsubsection{#1}}
  \def\Dsubsubsection#1{\subsubsection{#1}\addtocounter{subsubsection}{1}} % ttm-Fehler korrigieren
  \def\Dsubsection#1{\subsection{#1}}
  \def\Dsection#1{\section{#1}} % Im HTML wird nur der Sektionstitel gegeben
  \def\Dchapter#1{\chapter{#1}}
  \def\Dsubsubsubsectionx#1{\subsubsubsection*{#1}}
  \def\Dsubsubsectionx#1{\subsubsection*{#1}}
  \def\Dsubsectionx#1{\subsection*{#1}}
  \def\Dsectionx#1{\section*{#1}}
  \def\Dchapterx#1{\chapter*{#1}}

\else

  \def\Dsubsubsubsection#1{\subsubsection{#1}}
  \def\Dsubsubsection#1{\subsection{#1}}
  \def\Dsubsection#1{\section{#1}}
  \def\Dsection#1{\chapter{#1}}
  \def\Dchapter#1{\title{#1}}
  \def\Dsubsubsubsectionx#1{\subsubsection*{#1}}
  \def\Dsubsubsectionx#1{\subsection*{#1}}
  \def\Dsubsectionx#1{\section*{#1}}
  \def\Dsectionx#1{\chapter*{#1}}

\fi

\newcommand{\MStdPoints}{4}
\newcommand{\MSetPoints}[1]{\renewcommand{\MStdPoints}{#1}}

% Befehl zum Abbruch der Erstellung (nur PDF)
\newcommand{\MAbort}[1]{\err{#1}}

% Prefix vor Dateieinbindungen, wird in der Baumdatei mit \renewcommand modifiziert
% und auf das Verzeichnisprefix gesetzt, in dem das gerade bearbeitete tex-Dokument liegt.
% Im HTML wird es auf das Verzeichnis der HTML-Datei gesetzt.
% Das Prefix muss mit / enden !
\newcommand{\MDPrefix}{.}

% MRegisterFile notiert eine Datei zur Einbindung in den HTML-Baum. Grafiken mit MGraphics werden automatisch eingebunden.
% Mit MLastFile erhaelt man eine Markierung fuer die zuletzt registrierte Datei.
% Diese Markierung wird im postprocessing durch den physikalischen Dateinamen ersetzt, aber nur den Namen (d.h. \MMaterial gehoert noch davor, vgl Definition von MGraphics)
% Parameter: Pfad/Name der Datei bzw. des Ordners, relativ zur Position des Modul-Tex-Dokuments.
\ifttm
\newcommand{\MRegisterFile}[1]{\addtocounter{MFileNumber}{1}\special{html:<!-- registerfile;;}#1\special{html:;;}\MDPrefix\special{html:;;}\arabic{MFileNumber}\special{html:; //-->}}
\else
\newcommand{\MRegisterFile}[1]{\addtocounter{MFileNumber}{1}}
\fi

% Testen welcher Uebersetzer hier am Werk ist

\ifttm
\setcounter{MOutput}{3}
\else
\ifx\pdfoutput\undefined
  \pdffalse
  \setcounter{MOutput}{\MOutputDVI}
  \message{Verarbeitung mit latex, Ausgabe in dvi.}
\else
  \setcounter{MOutput}{\MOutputPDF}
  \message{Verarbeitung mit pdflatex, Ausgabe in pdf.}
  \ifnum \pdfoutput=0
    \pdffalse
  \setcounter{MOutput}{\MOutputDVI}
  \message{Verarbeitung mit pdflatex, Ausgabe in dvi.}
  \else
    \ifnum\pdfoutput=1
    \pdftrue
  \setcounter{MOutput}{\MOutputPDF}
  \message{Verarbeitung mit pdflatex, Ausgabe in pdf.}
    \fi
  \fi
\fi
\fi

\ifnum\value{MOutput}=\MOutputPDF
\DeclareGraphicsExtensions{.pdf,.png,.jpg}
\fi

\ifnum\value{MOutput}=\MOutputDVI
\DeclareGraphicsExtensions{.eps,.png,.jpg}
\fi

\ifnum\value{MOutput}=\MOutputHTML
% Wird vom Konverter leider nicht erkannt und daher in split.pm hardcodiert!
\DeclareGraphicsExtensions{.png,.jpg,.gif}
\fi

% Umdefinition der hyperref-Nummerierung im PDF-Modus
\ifttm
\else
\renewcommand{\theHfigure}{\arabic{chapter}.\arabic{section}.\arabic{figure}}
\fi

% Makro, um in der HTML-Ausgabe die zuerst zu oeffnende Datei zu kennzeichnen
\ifttm
\newcommand{\MGlobalStart}{\special{html:<!-- mglobalstarttag -->}}
\else
\newcommand{\MGlobalStart}{}
\fi

% Makro, um bei scormlogin ein pullen des Benutzers bei Aufruf der Seite zu erzwingen (typischerweise auf der Einstiegsseite)
\ifttm
\newcommand{\MPullSite}{\special{html:<!-- pullsite //-->}}
\else
\newcommand{\MPullSite}{}
\fi

% Makro, um in der HTML-Ausgabe die Kapiteluebersicht zu kennzeichnen
\ifttm
\newcommand{\MGlobalChapterTag}{\special{html:<!-- mglobalchaptertag -->}}
\else
\newcommand{\MGlobalChapterTag}{}
\fi

% Makro, um in der HTML-Ausgabe die Konfiguration zu kennzeichnen
\ifttm
\newcommand{\MGlobalConfTag}{\special{html:<!-- mglobalconfigtag -->}}
\else
\newcommand{\MGlobalConfTag}{}
\fi

% Makro, um in der HTML-Ausgabe die Standortbeschreibung zu kennzeichnen
\ifttm
\newcommand{\MGlobalLocationTag}{\special{html:<!-- mgloballocationtag -->}}
\else
\newcommand{\MGlobalLocationTag}{}
\fi

% Makro, um in der HTML-Ausgabe die persoenlichen Daten zu kennzeichnen
\ifttm
\newcommand{\MGlobalDataTag}{\special{html:<!-- mglobaldatatag -->}}
\else
\newcommand{\MGlobalDataTag}{}
\fi

% Makro, um in der HTML-Ausgabe die Suchseite zu kennzeichnen
\ifttm
\newcommand{\MGlobalSearchTag}{\special{html:<!-- mglobalsearchtag -->}}
\else
\newcommand{\MGlobalSearchTag}{}
\fi

% Makro, um in der HTML-Ausgabe die Favoritenseite zu kennzeichnen
\ifttm
\newcommand{\MGlobalFavoTag}{\special{html:<!-- mglobalfavoritestag -->}}
\else
\newcommand{\MGlobalFavoTag}{}
\fi

% Makro, um in der HTML-Ausgabe die Eingangstestseite zu kennzeichnen
\ifttm
\newcommand{\MGlobalSTestTag}{\special{html:<!-- mglobalstesttag -->}}
\else
\newcommand{\MGlobalSTestTag}{}
\fi

% Makro, um in der PDF-Ausgabe ein Wasserzeichen zu definieren
\ifttm
\newcommand{\MWatermarkSettings}{\relax}
\else
\newcommand{\MWatermarkSettings}{%
% \SetWatermarkText{(c) MINT-Kolleg Baden-W�rttemberg 2014}
% \SetWatermarkLightness{0.85}
% \SetWatermarkScale{1.5}
}
\fi

\ifttm
\newcommand{\MBinom}[2]{\left({\begin{array}{c} #1 \\ #2 \end{array}}\right)}
\else
\newcommand{\MBinom}[2]{\binom{#1}{#2}}
\fi

\ifttm
\newcommand{\DeclareMathOperator}[2]{\def#1{\mathrm{#2}}}
\newcommand{\operatorname}[1]{\mathrm{#1}}
\fi

%----------------- Makros fuer die gemischte HTML/PDF-Konvertierung ------------------------------

\newcommand{\MTestName}{\relax} % wird durch Test-Umgebung gesetzt

% Fuer experimentelle Kursinhalte, die im Release-Umsetzungsvorgang eine Fehlermeldung
% produzieren sollen aber sonst normal umgesetzt werden
\newenvironment{MExperimental}{%
}{%
}

% Wird von ttm nicht richtig umgesetzt!!
\newenvironment{MExerciseItems}{%
\renewcommand\theenumi{\alph{enumi}}%
\begin{enumerate}%
}{%
\end{enumerate}%
}


\definecolor{infoshadecolor}{rgb}{0.75,0.75,0.75}
\definecolor{exmpshadecolor}{rgb}{0.875,0.875,0.875}
\definecolor{expeshadecolor}{rgb}{0.95,0.95,0.95}
\definecolor{framecolor}{rgb}{0.2,0.2,0.2}

% Bei PDF-Uebersetzung wird hinter den Start jeder Satz/Info-aehnlichen Umgebung eine leere mbox gesetzt, damit
% fuehrende Listen oder enums nicht den Zeilenumbruch kaputtmachen
%\ifttm
\def\MTB{}
%\else
%\def\MTB{\mbox{}}
%\fi


\ifttm
\newcommand{\MRelates}{\special{html:<mi>&wedgeq;</mi>}}
\else
\def\MRelates{\stackrel{\scriptscriptstyle\wedge}{=}}
\fi

\def\MInch{\text{''}}
\def\Mdd{\textit{''}}

\ifttm
\def\MNL{ \newline }
\newenvironment{MArray}[1]{\begin{array}{#1}}{\end{array}}
\else
\def\MNL{ \\ }
\newenvironment{MArray}[1]{\begin{array}{#1}}{\end{array}}
\fi

\newcommand{\MBox}[1]{$\mathrm{#1}$}
\newcommand{\MMBox}[1]{\mathrm{#1}}


\ifttm%
\newcommand{\Mtfrac}[2]{{\textstyle \frac{#1}{#2}}}
\newcommand{\Mdfrac}[2]{{\displaystyle \frac{#1}{#2}}}
\newcommand{\Mmeasuredangle}{\special{html:<mi>&angmsd;</mi>}}
\else%
\newcommand{\Mtfrac}[2]{\tfrac{#1}{#2}}
\newcommand{\Mdfrac}[2]{\dfrac{#1}{#2}}
\newcommand{\Mmeasuredangle}{\measuredangle}
\relax
\fi

% Matrizen und Vektoren

% Inhalt wird in der Form a & b \\ c & d erwartet
% Vorsicht: MVector = Komponentenspalte, MVec = Variablensymbol
\ifttm%
\newcommand{\MVector}[1]{\left({\begin{array}{c}#1\end{array}}\right)}
\else%
\newcommand{\MVector}[1]{\begin{pmatrix}#1\end{pmatrix}}
\fi



\newcommand{\MVec}[1]{\vec{#1}}
\newcommand{\MDVec}[1]{\overrightarrow{#1}}

%----------------- Umgebungen fuer Definitionen und Saetze ----------------------------------------

% Fuegt einen Tabellen-Zeilenumbruch ein im PDF, aber nicht im HTML
\newcommand{\TSkip}{\ifttm \else&\ \\\fi}

\newenvironment{infoshaded}{%
\def\FrameCommand{\fboxsep=\FrameSep \fcolorbox{framecolor}{infoshadecolor}}%
\MakeFramed {\advance\hsize-\width \FrameRestore}}%
{\endMakeFramed}

\newenvironment{expeshaded}{%
\def\FrameCommand{\fboxsep=\FrameSep \fcolorbox{framecolor}{expeshadecolor}}%
\MakeFramed {\advance\hsize-\width \FrameRestore}}%
{\endMakeFramed}

\newenvironment{exmpshaded}{%
\def\FrameCommand{\fboxsep=\FrameSep \fcolorbox{framecolor}{exmpshadecolor}}%
\MakeFramed {\advance\hsize-\width \FrameRestore}}%
{\endMakeFramed}

\def\STDCOLOR{black}

\ifttm%
\else%
\newtheoremstyle{MSatzStyle}
  {1cm}                   %Space above
  {1cm}                   %Space below
  {\normalfont\itshape}   %Body font
  {}                      %Indent amount (empty = no indent,
                          %\parindent = para indent)
  {\normalfont\bfseries}  %Thm head font
  {}                      %Punctuation after thm head
  {\newline}              %Space after thm head: " " = normal interword
                          %space; \newline = linebreak
  {\thmname{#1}\thmnumber{ #2}\thmnote{ (#3)}}
                          %Thm head spec (can be left empty, meaning
                          %`normal')
                          %
\newtheoremstyle{MDefStyle}
  {1cm}                   %Space above
  {1cm}                   %Space below
  {\normalfont}           %Body font
  {}                      %Indent amount (empty = no indent,
                          %\parindent = para indent)
  {\normalfont\bfseries}  %Thm head font
  {}                      %Punctuation after thm head
  {\newline}              %Space after thm head: " " = normal interword
                          %space; \newline = linebreak
  {\thmname{#1}\thmnumber{ #2}\thmnote{ (#3)}}
                          %Thm head spec (can be left empty, meaning
                          %`normal')
\fi%

\newcommand{\MInfoText}{Info}

\newcounter{MHintCounter}
\newcounter{MCodeEditCounter}

\newcounter{MLastIndex}  % Enthaelt die dritte Stelle (Indexnummer) des letzten angelegten Objekts
\newcounter{MLastType}   % Enthaelt den Typ des letzten angelegten Objekts (mithilfe der unten definierten Konstanten). Die Entscheidung, wie der Typ dargstellt wird, wird in split.pm beim Postprocessing getroffen.
\newcounter{MLastTypeEq} % =1 falls das Label in einer Matheumgebung (equation, eqnarray usw.) steht, =2 falls das Label in einer table-Umgebung steht

% Da ttm keine Zahlmakros verarbeiten kann, werden diese Nummern in den Zuweisungen hardcodiert!
\def\MTypeSection{1}          %# Zaehler ist section
\def\MTypeSubsection{2}       %# Zaehler ist subsection
\def\MTypeSubsubsection{3}    %# Zaehler ist subsubsection
\def\MTypeInfo{4}             %# Eine Infobox, Separatzaehler fuer die Chemie (auch wenn es dort nicht nummeriert wird) ist MInfoCounter
\def\MTypeExercise{5}         %# Eine Aufgabe, Separatzaehler fuer die Chemie ist MExerciseCounter
\def\MTypeExample{6}          %# Eine Beispielbox, Separatzaehler fuer die Chemie ist MExampleCounter
\def\MTypeExperiment{7}       %# Eine Versuchsbox, Separatzaehler fuer die Chemie ist MExperimentCounter
\def\MTypeGraphics{8}         %# Eine Graphik, Separatzaehler fuer alle FB ist MGraphicsCounter
\def\MTypeTable{9}            %# Eine Tabellennummer, hat keinen Zaehler da durch table gezaehlt wird
\def\MTypeEquation{10}        %# Eine Gleichungsnummer, hat keinen Zaehler da durch equation/eqnarray gezaehlt wird
\def\MTypeTheorem{11}         % Ein theorem oder xtheorem, Separatzaehler fuer die Chemie ist MTheoremCounter
\def\MTypeVideo{12}           %# Ein Video,Separatzaehler fuer alle FB ist MVideoCounter
\def\MTypeEntry{13}           %# Ein Eintrag fuer die Stichwortliste, wird nicht gezaehlt sondern erhaelt im preparsing ein unique-label 

% Zaehler fuer das Labelsystem sind prefixcounter, jeder Zaehler wird VOR dem gezaehlten Objekt inkrementiert und zaehlt daher das aktuelle Objekt
\newcounter{MInfoCounter}
\newcounter{MExerciseCounter}
\newcounter{MExampleCounter}
\newcounter{MExperimentCounter}
\newcounter{MGraphicsCounter}
\newcounter{MTableCounter}
\newcounter{MEquationCounter}  % Nur im HTML, sonst durch "equation"-counter von latex realisiert
\newcounter{MTheoremCounter}
\newcounter{MObjectCounter}   % Gemeinsamer Zaehler fuer Objekte (ausser Grafiken/Tabellen) in Mathe/Info/Physik
\newcounter{MVideoCounter}
\newcounter{MEntryCounter}

\newcounter{MTestSite} % 1 = Subsubsection ist eine Pruefungsseite, 0 = ist eine normale Seite (inkl. Hilfeseite)

\def\MCell{$\phantom{a}$}

\newenvironment{MExportExercise}{\begin{MExercise}}{\end{MExercise}} % wird von mconvert abgefangen

\def\MGenerateExNumber{%
\ifnum\value{MSepNumbers}=0%
\arabic{section}.\arabic{subsection}.\arabic{MObjectCounter}\setcounter{MLastIndex}{\value{MObjectCounter}}%
\else%
\arabic{section}.\arabic{subsection}.\arabic{MExerciseCounter}\setcounter{MLastIndex}{\value{MExerciseCounter}}%
\fi%
}%

\def\MGenerateExmpNumber{%
\ifnum\value{MSepNumbers}=0%
\arabic{section}.\arabic{subsection}.\arabic{MObjectCounter}\setcounter{MLastIndex}{\value{MObjectCounter}}%
\else%
\arabic{section}.\arabic{subsection}.\arabic{MExerciseCounter}\setcounter{MLastIndex}{\value{MExampleCounter}}%
\fi%
}%

\def\MGenerateInfoNumber{%
\ifnum\value{MSepNumbers}=0%
\arabic{section}.\arabic{subsection}.\arabic{MObjectCounter}\setcounter{MLastIndex}{\value{MObjectCounter}}%
\else%
\arabic{section}.\arabic{subsection}.\arabic{MExerciseCounter}\setcounter{MLastIndex}{\value{MInfoCounter}}%
\fi%
}%

\def\MGenerateSiteNumber{%
\arabic{section}.\arabic{subsection}.\arabic{subsubsection}%
}%

% Funktionalitaet fuer Auswahlaufgaben

\newcounter{MExerciseCollectionCounter} % = 0 falls nicht in collection-Umgebung, ansonsten Schachtelungstiefe
\newcounter{MExerciseCollectionTextCounter} % wird von MExercise-Umgebung inkrementiert und von MExerciseCollection-Umgebung auf Null gesetzt

\ifttm
% MExerciseCollection gruppiert Aufgaben, die dynamisch aus der Datenbank gezogen werden und nicht direkt in der HTML-Seite stehen
% Parameter: #1 = ID der Collection, muss eindeutig fuer alle IN DER DB VORHANDENEN collections sein unabhaengig vom Kurs
%            #2 = Optionsargument (im Moment: 1 = Iterative Auswahl, 2 = Zufallsbasierte Auswahl)
\newenvironment{MExerciseCollection}[2]{%
\addtocounter{MExerciseCollectionCounter}{1}
\setcounter{MExerciseCollectionTextCounter}{0}
\special{html:<!-- mexercisecollectionstart;;}#1\special{html:;;}#2\special{html:;; //-->}%
}{%
\special{html:<!-- mexercisecollectionstop //-->}%
\addtocounter{MExerciseCollectionCounter}{-1}
}
\else
\newenvironment{MExerciseCollection}[2]{%
\addtocounter{MExerciseCollectionCounter}{1}
\setcounter{MExerciseCollectionTextCounter}{0}
}{%
\addtocounter{MExerciseCollectionCounter}{-1}
}
\fi

% Bei Uebersetzung nach PDF werden die theorem-Umgebungen verwendet, bei Uebersetzung in HTML ein manuelles Makro
\ifttm%

  \newenvironment{MHint}[1]{  \special{html:<button name="Name_MHint}\arabic{MHintCounter}\special{html:" class="hintbutton_closed" id="MHint}\arabic{MHintCounter}\special{html:_button" %
  type="button" onclick="toggle_hint('MHint}\arabic{MHintCounter}\special{html:');">}#1\special{html:</button>}
  \special{html:<div class="hint" style="display:none" id="MHint}\arabic{MHintCounter}\special{html:"> }}{\begin{html}</div>\end{html}\addtocounter{MHintCounter}{1}}

  \newenvironment{MCOSHZusatz}{  \special{html:<button name="Name_MHint}\arabic{MHintCounter}\special{html:" class="chintbutton_closed" id="MHint}\arabic{MHintCounter}\special{html:_button" %
  type="button" onclick="toggle_hint('MHint}\arabic{MHintCounter}\special{html:');">}Weiterf�hrende Inhalte\special{html:</button>}
  \special{html:<div class="hintc" style="display:none" id="MHint}\arabic{MHintCounter}\special{html:">
  <div class="coshwarn">Diese Inhalte gehen �ber das Kursniveau hinaus und werden in den Aufgaben und Tests nicht abgefragt.</div><br />}
  \addtocounter{MHintCounter}{1}}{\begin{html}</div>\end{html}}

  
  \newenvironment{MDefinition}{\begin{definition}\setcounter{MLastIndex}{\value{definition}}\ \\}{\end{definition}}

  
  \newenvironment{MExercise}{
  \renewcommand{\MStdPoints}{4}
  \addtocounter{MExerciseCounter}{1}
  \addtocounter{MObjectCounter}{1}
  \setcounter{MLastType}{5}

  \ifnum\value{MExerciseCollectionCounter}=0\else\addtocounter{MExerciseCollectionTextCounter}{1}\special{html:<!-- mexercisetextstart;;}\arabic{MExerciseCollectionTextCounter}\special{html:;; //-->}\fi
  \special{html:<div class="aufgabe" id="ADIV_}\MGenerateExNumber\special{html:">}%
  \textbf{Aufgabe \MGenerateExNumber
  } \ \\}{
  \special{html:</div><!-- mfeedbackbutton;Aufgabe;}\arabic{MTestSite}\special{html:;}\MGenerateExNumber\special{html:; //-->}
  \ifnum\value{MExerciseCollectionCounter}=0\else\special{html:<!-- mexercisetextstop //-->}\fi
  }

  % Stellt eine Kombination aus Aufgabe, Loesungstext und Eingabefeld bereit,
  % bei der Aufgabentext und Musterloesung sowie die zugehoerigen Feldelemente
  % extern bezogen und div-aktualisiert werden, das Eingabefeld aber immer das gleiche ist.
  \newenvironment{MFetchExercise}{
  \addtocounter{MExerciseCounter}{1}
  \addtocounter{MObjectCounter}{1}
  \setcounter{MLastType}{5}

  \special{html:<div class="aufgabe" id="ADIV_}\MGenerateExNumber\special{html:">}%
  \textbf{Aufgabe \MGenerateExNumber
  } \ \\%
  \special{html:</div><div class="exfetch_text" id="ADIVTEXT_}\MGenerateExNumber\special{html:">}%
  \special{html:</div><div class="exfetch_sol" id="ADIVSOL_}\MGenerateExNumber\special{html:">}%
  \special{html:</div><div class="exfetch_input" id="ADIVINPUT_}\MGenerateExNumber\special{html:">}%
  }{
  \special{html:</div>}
  }

  \newenvironment{MExample}{
  \addtocounter{MExampleCounter}{1}
  \addtocounter{MObjectCounter}{1}
  \setcounter{MLastType}{6}
  \begin{html}
  <div class="exmp">
  <div class="exmprahmen">
  \end{html}\textbf{Beispiel
  \ifnum\value{MSepNumbers}=0
  \arabic{section}.\arabic{subsection}.\arabic{MObjectCounter}\setcounter{MLastIndex}{\value{MObjectCounter}}
  \else
  \arabic{section}.\arabic{subsection}.\arabic{MExampleCounter}\setcounter{MLastIndex}{\value{MExampleCounter}}
  \fi
  } \ \\}{\begin{html}</div>
  </div>
  \end{html}
  \special{html:<!-- mfeedbackbutton;Beispiel;}\arabic{MTestSite}\special{html:;}\MGenerateExmpNumber\special{html:; //-->}
  }

  \newenvironment{MExperiment}{
  \addtocounter{MExperimentCounter}{1}
  \addtocounter{MObjectCounter}{1}
  \setcounter{MLastType}{7}
  \begin{html}
  <div class="expe">
  <div class="experahmen">
  \end{html}\textbf{Versuch
  \ifnum\value{MSepNumbers}=0
  \arabic{section}.\arabic{subsection}.\arabic{MObjectCounter}\setcounter{MLastIndex}{\value{MObjectCounter}}
  \else
%  \arabic{MExperimentCounter}\setcounter{MLastIndex}{\value{MExperimentCounter}}
  \arabic{section}.\arabic{subsection}.\arabic{MExperimentCounter}\setcounter{MLastIndex}{\value{MExperimentCounter}}
  \fi
  } \ \\}{\begin{html}</div>
  </div>
  \end{html}}

  \newenvironment{MChemInfo}{
  \setcounter{MLastType}{4}
  \begin{html}
  <div class="info">
  <div class="inforahmen">
  \end{html}}{\begin{html}</div>
  </div>
  \end{html}}

  \newenvironment{MXInfo}[1]{
  \addtocounter{MInfoCounter}{1}
  \addtocounter{MObjectCounter}{1}
  \setcounter{MLastType}{4}
  \begin{html}
  <div class="info">
  <div class="inforahmen">
  \end{html}\textbf{#1
  \ifnum\value{MInfoNumbers}=0
  \else
    \ifnum\value{MSepNumbers}=0
    \arabic{section}.\arabic{subsection}.\arabic{MObjectCounter}\setcounter{MLastIndex}{\value{MObjectCounter}}
    \else
    \arabic{MInfoCounter}\setcounter{MLastIndex}{\value{MInfoCounter}}
    \fi
  \fi
  } \ \\}{\begin{html}</div>
  </div>
  \end{html}
  \special{html:<!-- mfeedbackbutton;Info;}\arabic{MTestSite}\special{html:;}\MGenerateInfoNumber\special{html:; //-->}
  }

  \newenvironment{MInfo}{\ifnum\value{MInfoNumbers}=0\begin{MChemInfo}\else\begin{MXInfo}{Info}\ \\ \fi}{\ifnum\value{MInfoNumbers}=0\end{MChemInfo}\else\end{MXInfo}\fi}

\else%

  \theoremstyle{MSatzStyle}
  \newtheorem{thm}{Satz}[section]
  \newtheorem{thmc}{Satz}
  \theoremstyle{MDefStyle}
  \newtheorem{defn}[thm]{Definition}
  \newtheorem{exmp}[thm]{Beispiel}
  \newtheorem{info}[thm]{\MInfoText}
  \theoremstyle{MDefStyle}
  \newtheorem{defnc}{Definition}
  \theoremstyle{MDefStyle}
  \newtheorem{exmpc}{Beispiel}[section]
  \theoremstyle{MDefStyle}
  \newtheorem{infoc}{\MInfoText}
  \theoremstyle{MDefStyle}
  \newtheorem{exrc}{Aufgabe}[section]
  \theoremstyle{MDefStyle}
  \newtheorem{verc}{Versuch}[section]
  
  \newenvironment{MFetchExercise}{}{} % kann im PDF nicht dargestellt werden
  
  \newenvironment{MExercise}{\begin{exrc}\renewcommand{\MStdPoints}{1}\MTB}{\end{exrc}}
  \newenvironment{MHint}[1]{\ \\ \underline{#1:}\\}{}
  \newenvironment{MCOSHZusatz}{\ \\ \underline{Weiterf�hrende Inhalte:}\\}{}
  \newenvironment{MDefinition}{\ifnum\value{MInfoNumbers}=0\begin{defnc}\else\begin{defn}\fi\MTB}{\ifnum\value{MInfoNumbers}=0\end{defnc}\else\end{defn}\fi}
%  \newenvironment{MExample}{\begin{exmp}}{\ \linebreak[1] \ \ \ \ $\phantom{a}$ \ \hfill $\blacklozenge$\end{exmp}}
  \newenvironment{MExample}{
    \ifnum\value{MInfoNumbers}=0\begin{exmpc}\else\begin{exmp}\fi
    \MTB
    \begin{exmpshaded}
    \ \newline
}{
    \end{exmpshaded}
    \ifnum\value{MInfoNumbers}=0\end{exmpc}\else\end{exmp}\fi
}
  \newenvironment{MChemInfo}{\begin{infoshaded}}{\end{infoshaded}}

  \newenvironment{MInfo}{\ifnum\value{MInfoNumbers}=0\begin{MChemInfo}\else\renewcommand{\MInfoText}{Info}\begin{info}\begin{infoshaded}
  \MTB
   \ \newline
    \fi
  }{\ifnum\value{MInfoNumbers}=0\end{MChemInfo}\else\end{infoshaded}\end{info}\fi}

  \newenvironment{MXInfo}[1]{
    \renewcommand{\MInfoText}{#1}
    \ifnum\value{MInfoNumbers}=0\begin{infoc}\else\begin{info}\fi%
    \MTB
    \begin{infoshaded}
    \ \newline
  }{\end{infoshaded}\ifnum\value{MInfoNumbers}=0\end{infoc}\else\end{info}\fi}

  \newenvironment{MExperiment}{
    \renewcommand{\MInfoText}{Versuch}
    \ifnum\value{MInfoNumbers}=0\begin{verc}\else\begin{info}\fi
    \MTB
    \begin{expeshaded}
    \ \newline
  }{
    \end{expeshaded}
    \ifnum\value{MInfoNumbers}=0\end{verc}\else\end{info}\fi
  }
\fi%

% MHint sollte nicht direkt fuer Loesungen benutzt werden wegen solutionselect
\newenvironment{MSolution}{\begin{MHint}{L"osung}}{\end{MHint}}

\newcounter{MCodeCounter}

\ifttm
\newenvironment{MCode}{\special{html:<!-- mcodestart -->}\ttfamily\color{blue}}{\special{html:<!-- mcodestop -->}}
\else
\newenvironment{MCode}{\begin{flushleft}\ttfamily\addtocounter{MCodeCounter}{1}}{\addtocounter{MCodeCounter}{-1}\end{flushleft}}
% Ohne color-Statement da inkompatible mit framed/shaded-Boxen aus dem framed-package
\fi

%----------------- Sonderdefinitionen fuer Symbole, die der Konverter nicht kann ----------------------------------------------

\ifttm%
\newcommand{\MUnderset}[2]{\underbrace{#2}_{#1}}%
\else%
\newcommand{\MUnderset}[2]{\underset{#1}{#2}}%
\fi%

\ifttm
\newcommand{\MThinspace}{\special{html:<mi>&#x2009;</mi>}}
\else
\newcommand{\MThinspace}{\,}
\fi

\ifttm
\newcommand{\glq}{\begin{html}&sbquo;\end{html}}
\newcommand{\grq}{\begin{html}&lsquo;\end{html}}
\newcommand{\glqq}{\begin{html}&bdquo;\end{html}}
\newcommand{\grqq}{\begin{html}&ldquo;\end{html}}
\fi

\ifttm
\newcommand{\MNdash}{\begin{html}&ndash;\end{html}}
\else
\newcommand{\MNdash}{--}
\fi

%\ifttm\def\MIU{\special{html:<mi>&#8520;</mi>}}\else\def\MIU{\mathrm{i}}\fi
\def\MIU{\mathrm{i}}
\def\MEU{e} % TU9-Onlinekurs: italic-e
%\def\MEU{\mathrm{e}} % Alte Onlinemodule: roman-e
\def\MD{d} % Kursives d in Integralen im TU9-Onlinekurs
%\def\MD{\mathrm{d}} % roman-d in den alten Onlinemodulen
\def\MDB{\|}

%zusaetzlicher Leerraum vor "\MD"
\ifttm%
\def\MDSpace{\special{html:<mi>&#x2009;</mi>}}
\else%
\def\MDSpace{\,}
\fi%
\newcommand{\MDwSp}{\MDSpace\MD}%

\ifttm
\def\Mdq{\dq}
\else
\def\Mdq{\dq}
\fi

\def\MSpan#1{\left<{#1}\right>}
\def\MSetminus{\setminus}
\def\MIM{I}

\ifttm
\newcommand{\ld}{\text{ld}}
\newcommand{\lg}{\text{lg}}
\else
\DeclareMathOperator{\ld}{ld}
%\newcommand{\lg}{\text{lg}} % in latex schon definiert
\fi


\def\Mmapsto{\ifttm\special{html:<mi>&mapsto;</mi>}\else\mapsto\fi} 
\def\Mvarphi{\ifttm\phi\else\varphi\fi}
\def\Mphi{\ifttm\varphi\else\phi\fi}
\ifttm%
\newcommand{\MEumu}{\special{html:<mi>&#x3BC;</mi>}}%
\else%
\newcommand{\MEumu}{\textrm{\textmu}}%
\fi
\def\Mvarepsilon{\ifttm\epsilon\else\varepsilon\fi}
\def\Mepsilon{\ifttm\varepsilon\else\epsilon\fi}
\def\Mvarkappa{\ifttm\kappa\else\varkappa\fi}
\def\Mkappa{\ifttm\varkappa\else\kappa\fi}
\def\Mcomplement{\ifttm\special{html:<mi>&comp;</mi>}\else\complement\fi} 
\def\MWW{\mathrm{WW}}
\def\Mmod{\ifttm\special{html:<mi>&nbsp;mod&nbsp;</mi>}\else\mod\fi} 

\ifttm%
\def\mod{\text{\;mod\;}}%
\def\MNEquiv{\special{html:<mi>&NotCongruent;</mi>}}% 
\def\MNSubseteq{\special{html:<mi>&NotSubsetEqual;</mi>}}%
\def\MEmptyset{\special{html:<mi>&empty;</mi>}}%
\def\MVDots{\special{html:<mi>&#x22EE;</mi>}}%
\def\MHDots{\special{html:<mi>&#x2026;</mi>}}%
\def\Mddag{\special{html:<mi>&#x1202;</mi>}}%
\def\sphericalangle{\special{html:<mi>&measuredangle;</mi>}}%
\def\nparallel{\special{html:<mi>&nparallel;</mi>}}%
\def\MProofEnd{\special{html:<mi>&#x25FB;</mi>}}%
\newenvironment{MProof}[1]{\underline{#1}:\MCR\MCR}{\hfill $\MProofEnd$}%
\else%
\def\MNEquiv{\not\equiv}%
\def\MNSubseteq{\not\subseteq}%
\def\MEmptyset{\emptyset}%
\def\MVDots{\vdots}%
\def\MHDots{\hdots}%
\def\Mddag{\ddag}%
\newenvironment{MProof}[1]{\begin{proof}[#1]}{\end{proof}}%
\fi%



% Spaces zum Auffuellen von Tabellenbreiten, die nur im HTML wirken
\ifttm%
\def\MTSP{\:}%
\else%
\def\MTSP{}%
\fi%

\DeclareMathOperator{\arsinh}{arsinh}
\DeclareMathOperator{\arcosh}{arcosh}
\DeclareMathOperator{\artanh}{artanh}
\DeclareMathOperator{\arcoth}{arcoth}


\newcommand{\MMathSet}[1]{\mathbb{#1}}
\def\N{\MMathSet{N}}
\def\Z{\MMathSet{Z}}
\def\Q{\MMathSet{Q}}
\def\R{\MMathSet{R}}
\def\C{\MMathSet{C}}

\newcounter{MForLoopCounter}
\newcommand{\MForLoop}[2]{\setcounter{MForLoopCounter}{#1}\ifnum\value{MForLoopCounter}=0{}\else{{#2}\addtocounter{MForLoopCounter}{-1}\MForLoop{\value{MForLoopCounter}}{#2}}\fi}

\newcounter{MSiteCounter}
\newcounter{MFieldCounter} % Kombination section.subsection.site.field ist eindeutig in allen Modulen, field alleine nicht

\newcounter{MiniMarkerCounter}

\ifttm
\newenvironment{MMiniPageP}[1]{\begin{minipage}{#1\linewidth}\special{html:<!-- minimarker;;}\arabic{MiniMarkerCounter}\special{html:;;#1; //-->}}{\end{minipage}\addtocounter{MiniMarkerCounter}{1}}
\else
\newenvironment{MMiniPageP}[1]{\begin{minipage}{#1\linewidth}}{\end{minipage}\addtocounter{MiniMarkerCounter}{1}}
\fi

\newcounter{AlignCounter}

\newcommand{\MStartJustify}{\ifttm\special{html:<!-- startalign;;}\arabic{AlignCounter}\special{html:;;justify; //-->}\fi}
\newcommand{\MStopJustify}{\ifttm\special{html:<!-- stopalign;;}\arabic{AlignCounter}\special{html:; //-->}\fi\addtocounter{AlignCounter}{1}}

\newenvironment{MJTabular}[1]{
\MStartJustify
\begin{tabular}{#1}
}{
\end{tabular}
\MStopJustify
}

\newcommand{\MImageLeft}[2]{
\begin{center}
\begin{tabular}{lc}
\MStartJustify
\begin{MMiniPageP}{0.65}
#1
\end{MMiniPageP}
\MStopJustify
&
\begin{MMiniPageP}{0.3}
#2  
\end{MMiniPageP}
\end{tabular}
\end{center}
}

\newcommand{\MImageHalf}[2]{
\begin{center}
\begin{tabular}{lc}
\MStartJustify
\begin{MMiniPageP}{0.45}
#1
\end{MMiniPageP}
\MStopJustify
&
\begin{MMiniPageP}{0.45}
#2  
\end{MMiniPageP}
\end{tabular}
\end{center}
}

\newcommand{\MBigImageLeft}[2]{
\begin{center}
\begin{tabular}{lc}
\MStartJustify
\begin{MMiniPageP}{0.25}
#1
\end{MMiniPageP}
\MStopJustify
&
\begin{MMiniPageP}{0.7}
#2  
\end{MMiniPageP}
\end{tabular}
\end{center}
}

\ifttm
\def\No{\mathbb{N}_0}
\else
\def\No{\ensuremath{\N_0}}
\fi
\def\MT{\textrm{\tiny T}}
\newcommand{\MTranspose}[1]{{#1}^{\MT}}
\ifttm
\newcommand{\MRe}{\mathsf{Re}}
\newcommand{\MIm}{\mathsf{Im}}
\else
\DeclareMathOperator{\MRe}{Re}
\DeclareMathOperator{\MIm}{Im}
\fi

\newcommand{\Mid}{\mathrm{id}}
\newcommand{\MFeinheit}{\mathrm{feinh}}

\ifttm
\newcommand{\Msubstack}[1]{\begin{array}{c}{#1}\end{array}}
\else
\newcommand{\Msubstack}[1]{\substack{#1}}
\fi

% Typen von Fragefeldern:
% 1 = Alphanumerisch, case-sensitive-Vergleich
% 2 = Ja/Nein-Checkbox, Loesung ist 0 oder 1   (OPTION = Image-id fuer Rueckmeldung)
% 3 = Reelle Zahlen Geparset
% 4 = Funktionen Geparset (mit Stuetzstellen zur ueberpruefung)

% Dieser Befehl erstellt ein interaktives Aufgabenfeld. Parameter:
% - #1 Laenge in Zeichen
% - #2 Loesungstext (alphanumerisch, case sensitive)
% - #3 AufgabenID (alphanumerisch, case sensitive)
% - #4 Typ (Kennnummer)
% - #5 String fuer Optionen (ggf. mit Semikolon getrennte Einzelstrings)
% - #6 Anzahl Punkte
% - #7 uxid (kann z.B. Loesungsstring sein)
% ACHTUNG: Die langen Zeilen bitte so lassen, Zeilenumbrueche im tex werden in div's umgesetzt
\newcommand{\MQuestionID}[7]{
\ifttm
\special{html:<!-- mdeclareuxid;;}UX#7\special{html:;;}\arabic{section}\special{html:;;}#3\special{html:;; //-->}%
\special{html:<!-- mdeclarepoints;;}\arabic{section}\special{html:;;}#3\special{html:;;}#6\special{html:;;}\arabic{MTestSite}\special{html:;;}\arabic{chapter}%
\special{html:;; //--><!-- onloadstart //-->CreateQuestionObj("}#7\special{html:",}\arabic{MFieldCounter}\special{html:,"}#2%
\special{html:","}#3\special{html:",}#4\special{html:,"}#5\special{html:",}#6\special{html:,}\arabic{MTestSite}\special{html:,}\arabic{section}%
\special{html:);<!-- onloadstop //-->}%
\special{html:<input mfieldtype="}#4\special{html:" name="Name_}#3\special{html:" id="}#3\special{html:" type="text" size="}#1\special{html:" maxlength="}#1%
\special{html:" }\ifnum\value{MGroupActive}=0\special{html:onfocus="handlerFocus(}\arabic{MFieldCounter}%
\special{html:);" onblur="handlerBlur(}\arabic{MFieldCounter}\special{html:);" onkeyup="handlerChange(}\arabic{MFieldCounter}\special{html:,0);" onpaste="handlerChange(}\arabic{MFieldCounter}\special{html:,0);" oninput="handlerChange(}\arabic{MFieldCounter}\special{html:,0);" onpropertychange="handlerChange(}\arabic{MFieldCounter}\special{html:,0);"/>}%
\special{html:<img src="images/questionmark.gif" width="20" height="20" border="0" align="absmiddle" id="}QM#3\special{html:"/>}
\else%
\special{html:onblur="handlerBlur(}\arabic{MFieldCounter}%
\special{html:);" onfocus="handlerFocus(}\arabic{MFieldCounter}\special{html:);" onkeyup="handlerChange(}\arabic{MFieldCounter}\special{html:,1);" onpaste="handlerChange(}\arabic{MFieldCounter}\special{html:,1);" oninput="handlerChange(}\arabic{MFieldCounter}\special{html:,1);" onpropertychange="handlerChange(}\arabic{MFieldCounter}\special{html:,1);"/>}%
\special{html:<img src="images/questionmark.gif" width="20" height="20" border="0" align="absmiddle" id="}QM#3\special{html:"/>}\fi%
\else%
\ifnum\value{QBoxFlag}=1\fbox{$\phantom{\MForLoop{#1}{b}}$}\else$\phantom{\MForLoop{#1}{b}}$\fi%
\fi%
}

% ACHTUNG: Die langen Zeilen bitte so lassen, Zeilenumbrueche im tex werden in div's umgesetzt
% QuestionCheckbox macht ausserhalb einer QuestionGroup keinen Sinn!
% #1 = solution (1 oder 0), ggf. mit ::smc abgetrennt auszuschliessende single-choice-boxen (UXIDs durch , getrennt), #2 = id, #3 = points, #4 = uxid
\newcommand{\MQuestionCheckbox}[4]{
\ifttm
\special{html:<!-- mdeclareuxid;;}UX#4\special{html:;;}\arabic{section}\special{html:;;}#2\special{html:;; //-->}%
\ifnum\value{MGroupActive}=0\MDebugMessage{ERROR: Checkbox Nr. \arabic{MFieldCounter}\ ist nicht in einer Kontrollgruppe, es wird niemals eine Loesung angezeigt!}\fi
\special{html: %
<!-- mdeclarepoints;;}\arabic{section}\special{html:;;}#2\special{html:;;}#3\special{html:;;}\arabic{MTestSite}\special{html:;;}\arabic{chapter}%
\special{html:;; //--><!-- onloadstart //-->CreateQuestionObj("}#4\special{html:",}\arabic{MFieldCounter}\special{html:,"}#1\special{html:","}#2\special{html:",2,"IMG}#2%
\special{html:",}#3\special{html:,}\arabic{MTestSite}\special{html:,}\arabic{section}\special{html:);<!-- onloadstop //-->}%
\special{html:<input mfieldtype="2" type="checkbox" name="Name_}#2\special{html:" id="}#2\special{html:" onchange="handlerChange(}\arabic{MFieldCounter}\special{html:,1);"/><img src="images/questionmark.gif" name="}Name_IMG#2%
\special{html:" width="20" height="20" border="0" align="absmiddle" id="}IMG#2\special{html:"/> }%
\else%
\ifnum\value{QBoxFlag}=1\fbox{$\phantom{X}$}\else$\phantom{X}$\fi%
\fi%
}

\def\MGenerateID{QFELD_\arabic{section}.\arabic{subsection}.\arabic{MSiteCounter}.QF\arabic{MFieldCounter}}

% #1 = 0/1 ggf. mit ::smc abgetrennt auszuschliessende single-choice-boxen (UXIDs durch , getrennt ohne UX), #2 = uxid ohne UX
\newcommand{\MCheckbox}[2]{
\MQuestionCheckbox{#1}{\MGenerateID}{\MStdPoints}{#2}
\addtocounter{MFieldCounter}{1}
}

% Erster Parameter: Zeichenlaenge der Eingabebox, zweiter Parameter: Loesungstext
\newcommand{\MQuestion}[2]{
\MQuestionID{#1}{#2}{\MGenerateID}{1}{0}{\MStdPoints}{#2}
\addtocounter{MFieldCounter}{1}
}

% Erster Parameter: Zeichenlaenge der Eingabebox, zweiter Parameter: Loesungstext
\newcommand{\MLQuestion}[3]{
\MQuestionID{#1}{#2}{\MGenerateID}{1}{0}{\MStdPoints}{#3}
\addtocounter{MFieldCounter}{1}
}

% Parameter: Laenge des Feldes, Loesung (wird auch geparsed), Stellen Genauigkeit hinter dem Komma, weitere Stellen werden mathematisch gerundet vor Vergleich
\newcommand{\MParsedQuestion}[3]{
\MQuestionID{#1}{#2}{\MGenerateID}{3}{#3}{\MStdPoints}{#2}
\addtocounter{MFieldCounter}{1}
}

% Parameter: Laenge des Feldes, Loesung (wird auch geparsed), Stellen Genauigkeit hinter dem Komma, weitere Stellen werden mathematisch gerundet vor Vergleich
\newcommand{\MLParsedQuestion}[4]{
\MQuestionID{#1}{#2}{\MGenerateID}{3}{#3}{\MStdPoints}{#4}
\addtocounter{MFieldCounter}{1}
}

% Parameter: Laenge des Feldes, Loesungsfunktion, Anzahl Stuetzstellen, Funktionsvariablen durch Kommata getrennt (nicht case-sensitive), Anzahl Nachkommastellen im Vergleich
\newcommand{\MFunctionQuestion}[5]{
\MQuestionID{#1}{#2}{\MGenerateID}{4}{#3;#4;#5;0}{\MStdPoints}{#2}
\addtocounter{MFieldCounter}{1}
}

% Parameter: Laenge des Feldes, Loesungsfunktion, Anzahl Stuetzstellen, Funktionsvariablen durch Kommata getrennt (nicht case-sensitive), Anzahl Nachkommastellen im Vergleich, UXID
\newcommand{\MLFunctionQuestion}[6]{
\MQuestionID{#1}{#2}{\MGenerateID}{4}{#3;#4;#5;0}{\MStdPoints}{#6}
\addtocounter{MFieldCounter}{1}
}

% Parameter: Laenge des Feldes, Loesungsintervall, Genauigkeit der Zahlenwertpruefung
\newcommand{\MIntervalQuestion}[3]{
\MQuestionID{#1}{#2}{\MGenerateID}{6}{#3}{\MStdPoints}{#2}
\addtocounter{MFieldCounter}{1}
}

% Parameter: Laenge des Feldes, Loesungsintervall, Genauigkeit der Zahlenwertpruefung, UXID
\newcommand{\MLIntervalQuestion}[4]{
\MQuestionID{#1}{#2}{\MGenerateID}{6}{#3}{\MStdPoints}{#4}
\addtocounter{MFieldCounter}{1}
}

% Parameter: Laenge des Feldes, Loesungsfunktion, Anzahl Stuetzstellen, Funktionsvariable (nicht case-sensitive), Anzahl Nachkommastellen im Vergleich, Vereinfachungsbedingung
% Vereinfachungsbedingung ist eine der Folgenden:
% 0 = Keine Vereinfachungsbedingung
% 1 = Keine Klammern (runde oder eckige) mehr im vereinfachten Ausdruck
% 2 = Faktordarstellung (Term hat Produkte als letzte Operation, Summen als vorgeschaltete Operation)
% 3 = Summendarstellung (Term hat Summen als letzte Operation, Produkte als vorgeschaltete Operation)
% Flag 512: Besondere Stuetzstellen (nur >1 und nur schwach rational), sonst symmetrisch um Nullpunkt und ganze Zahlen inkl. Null werden getroffen
\newcommand{\MSimplifyQuestion}[6]{
\MQuestionID{#1}{#2}{\MGenerateID}{4}{#3;#4;#5;#6}{\MStdPoints}{#2}
\addtocounter{MFieldCounter}{1}
}

\newcommand{\MLSimplifyQuestion}[7]{
\MQuestionID{#1}{#2}{\MGenerateID}{4}{#3;#4;#5;#6}{\MStdPoints}{#7}
\addtocounter{MFieldCounter}{1}
}

% Parameter: Laenge des Feldes, Loesung (optionaler Ausdruck), Anzahl Stuetzstellen, Funktionsvariable (nicht case-sensitive), Anzahl Nachkommastellen im Vergleich, Spezialtyp (string-id)
\newcommand{\MLSpecialQuestion}[7]{
\MQuestionID{#1}{#2}{\MGenerateID}{7}{#3;#4;#5;#6}{\MStdPoints}{#7}
\addtocounter{MFieldCounter}{1}
}

\newcounter{MGroupStart}
\newcounter{MGroupEnd}
\newcounter{MGroupActive}

\newenvironment{MQuestionGroup}{
\setcounter{MGroupStart}{\value{MFieldCounter}}
\setcounter{MGroupActive}{1}
}{
\setcounter{MGroupActive}{0}
\setcounter{MGroupEnd}{\value{MFieldCounter}}
\addtocounter{MGroupEnd}{-1}
}

\newcommand{\MGroupButton}[1]{
\ifttm
\special{html:<button name="Name_Group}\arabic{MGroupStart}\special{html:to}\arabic{MGroupEnd}\special{html:" id="Group}\arabic{MGroupStart}\special{html:to}\arabic{MGroupEnd}\special{html:" %
type="button" onclick="group_button(}\arabic{MGroupStart}\special{html:,}\arabic{MGroupEnd}\special{html:);">}#1\special{html:</button>}
\else
\phantom{#1}
\fi
}

%----------------- Makros fuer die modularisierte Darstellung ------------------------------------

\def\MyText#1{#1}

% is used internally by the conversion package, should not be used by original tex documents
\def\MOrgLabel#1{\relax}

\ifttm

% Ein MLabel wird im html codiert durch das tag <!-- mmlabel;;Labelbezeichner;;SubjectArea;;chapter;;section;;subsection;;Index;;Objekttyp; //-->
\def\MLabel#1{%
\ifnum\value{MLastType}=8%
\ifnum\value{MCaptionOn}=0%
\MDebugMessage{ERROR: Grafik \arabic{MGraphicsCounter} hat separates label: #1 (Grafiklabels sollten nur in der Caption stehen)}%
\fi
\fi
\ifnum\value{MLastType}=12%
\ifnum\value{MCaptionOn}=0%
\MDebugMessage{ERROR: Video \arabic{MVideoCounter} hat separates label: #1 (Videolabels sollten nur in der Caption stehen}%
\fi
\fi
\ifnum\value{MLastType}=10\setcounter{MLastIndex}{\value{equation}}\fi
\label{#1}\begin{html}<!-- mmlabel;;#1;;\end{html}\arabic{MSubjectArea}\special{html:;;}\arabic{chapter}\special{html:;;}\arabic{section}\special{html:;;}\arabic{subsection}\special{html:;;}\arabic{MLastIndex}\special{html:;;}\arabic{MLastType}\special{html:; //-->}}%

\else

% Sonderbehandlung im PDF fuer Abbildungen in separater aux-Datei, da MGraphics die figure-Umgebung nicht verwendet
\def\MLabel#1{%
\ifnum\value{MLastType}=8%
\ifnum\value{MCaptionOn}=0%
\MDebugMessage{ERROR: Grafik \arabic{MGraphicsCounter} hat separates label: #1 (Grafiklabels sollten nur in der Caption stehen}%
\fi
\fi
\ifnum\value{MLastType}=12%
\ifnum\value{MCaptionOn}=0%
\MDebugMessage{ERROR: Video \arabic{MVideoCounter} hat separates label: #1 (Videolabels sollten nur in der Caption stehen}%
\fi
\fi
\label{#1}%
}%

\fi

% Gibt Begriff des referenzierten Objekts mit aus, aber nur im HTML, daher nur in Ausnahmefaellen (z.B. Copyrightliste) sinnvoll
\def\MCRef#1{\ifttm\special{html:<!-- mmref;;}#1\special{html:;;1; //-->}\else\vref{#1}\fi}


\def\MRef#1{\ifttm\special{html:<!-- mmref;;}#1\special{html:;;0; //-->}\else\vref{#1}\fi}
\def\MERef#1{\ifttm\special{html:<!-- mmref;;}#1\special{html:;;0; //-->}\else\eqref{#1}\fi}
\def\MNRef#1{\ifttm\special{html:<!-- mmref;;}#1\special{html:;;0; //-->}\else\ref{#1}\fi}
\def\MSRef#1#2{\ifttm\special{html:<!-- msref;;}#1\special{html:;;}#2\special{html:; //-->}\else \if#2\empty \ref{#1} \else \hyperref[#1]{#2}\fi\fi} 

\def\MRefRange#1#2{\ifttm\MRef{#1} bis 
\MRef{#2}\else\vrefrange[\unskip]{#1}{#2}\fi}

\def\MRefTwo#1#2{\ifttm\MRef{#1} und \MRef{#2}\else%
\let\vRefTLRsav=\reftextlabelrange\let\vRefTPRsav=\reftextpagerange%
\def\reftextlabelrange##1##2{\ref{##1} und~\ref{##2}}%
\def\reftextpagerange##1##2{auf den Seiten~\pageref{#1} und~\pageref{#2}}%
\vrefrange[\unskip]{#1}{#2}%
\let\reftextlabelrange=\vRefTLRsav\let\reftextpagerange=\vRefTPRsav\fi}

% MSectionChapter definiert falls notwendig das Kapitel vor der section. Das ist notwendig, wenn nur ein Einzelmodul uebersetzt wird.
% MChaptersGiven ist ein Counter, der von mconvert.pl vordefiniert wird.
\ifttm
\newcommand{\MSectionChapter}{\ifnum\value{MChaptersGiven}=0{\Dchapter{Modul}}\else{}\fi}
\else
\newcommand{\MSectionChapter}{\ifnum\value{chapter}=0{\Dchapter{Modul}}\else{}\fi}
\fi


\def\MChapter#1{\ifnum\value{MSSEnd}>0{\MSubsectionEndMacros}\addtocounter{MSSEnd}{-1}\fi\Dchapter{#1}}
\def\MSubject#1{\MChapter{#1}} % Schluesselwort HELPSECTION ist reserviert fuer Hilfesektion

\newcommand{\MSectionID}{UNKNOWNID}

\ifttm
\newcommand{\MSetSectionID}[1]{\renewcommand{\MSectionID}{#1}}
\else
\newcommand{\MSetSectionID}[1]{\renewcommand{\MSectionID}{#1}\tikzsetexternalprefix{#1}}
\fi


\newcommand{\MSection}[1]{\MSetSectionID{MODULID}\ifnum\value{MSSEnd}>0{\MSubsectionEndMacros}\addtocounter{MSSEnd}{-1}\fi\MSectionChapter\Dsection{#1}\MSectionStartMacros{#1}\setcounter{MLastIndex}{-1}\setcounter{MLastType}{1}} % Sections werden ueber das section-Feld im mmlabel-Tag identifiziert, nicht ueber das Indexfeld

\def\MSubsection#1{\ifnum\value{MSSEnd}>0{\MSubsectionEndMacros}\addtocounter{MSSEnd}{-1}\fi\ifttm\else\clearpage\fi\Dsubsection{#1}\MSubsectionStartMacros\setcounter{MLastIndex}{-1}\setcounter{MLastType}{2}\addtocounter{MSSEnd}{1}}% Subsections werden ueber das subsection-Feld im mmlabel-Tag identifiziert, nicht ueber das Indexfeld
\def\MSubsectionx#1{\Dsubsectionx{#1}} % Nur zur Verwendung in MSectionStart gedacht
\def\MSubsubsection#1{\Dsubsubsection{#1}\setcounter{MLastIndex}{\value{subsubsection}}\setcounter{MLastType}{3}\ifttm\special{html:<!-- sectioninfo;;}\arabic{section}\special{html:;;}\arabic{subsection}\special{html:;;}\arabic{subsubsection}\special{html:;;1;;}\arabic{MTestSite}\special{html:; //-->}\fi}
\def\MSubsubsectionx#1{\Dsubsubsectionx{#1}\ifttm\special{html:<!-- sectioninfo;;}\arabic{section}\special{html:;;}\arabic{subsection}\special{html:;;}\arabic{subsubsection}\special{html:;;0;;}\arabic{MTestSite}\special{html:; //-->}\else\addcontentsline{toc}{subsection}{#1}\fi}

\ifttm
\def\MSubsubsubsectionx#1{\ \newline\textbf{#1}\special{html:<br />}}
\else
\def\MSubsubsubsectionx#1{\ \newline
\textbf{#1}\ \\
}
\fi


% Dieses Skript wird zu Beginn jedes Modulabschnitts (=Webseite) ausgefuehrt und initialisiert den Aufgabenfeldzaehler
\newcommand{\MPageScripts}{
\setcounter{MFieldCounter}{1}
\addtocounter{MSiteCounter}{1}
\setcounter{MHintCounter}{1}
\setcounter{MCodeEditCounter}{1}
\setcounter{MGroupActive}{0}
\DoQBoxes
% Feldvariablen werden im HTML-Header in conv.pl eingestellt
}

% Dieses Skript wird zum Ende jedes Modulabschnitts (=Webseite) ausgefuehrt
\ifttm
\newcommand{\MEndScripts}{\special{html:<br /><!-- mfeedbackbutton;Seite;}\arabic{MTestSite}\special{html:;}\MGenerateSiteNumber\special{html:; //-->}
}
\else
\newcommand{\MEndScripts}{\relax}
\fi


\newcounter{QBoxFlag}
\newcommand{\DoQBoxes}{\setcounter{QBoxFlag}{1}}
\newcommand{\NoQBoxes}{\setcounter{QBoxFlag}{0}}

\newcounter{MXCTest}
\newcounter{MXCounter}
\newcounter{MSCounter}



\ifttm

% Struktur des sectioninfo-Tags: <!-- sectioninfo;;section;;subsection;;subsubsection;;nr_ausgeben;;testpage; //-->

%Fuegt eine zusaetzliche html-Seite an hinter ALLEN bisherigen und zukuenftigen content-Seiten ausserhalb der vor-zurueck-Schleife (d.h. nur durch Button oder MIntLink erreichbar!)
% #1 = Titel des Modulabschnitts, #2 = Kurztitel fuer die Buttons, #3 = Buttonkennung (STD = default nehmen, NONE = Ohne Button in der Navigation)
\newenvironment{MSContent}[3]{\special{html:<div class="xcontent}\arabic{MSCounter}\special{html:"><!-- scontent;-;}\arabic{MSCounter};-;#1;-;#2;-;#3\special{html: //-->}\MPageScripts\MSubsubsectionx{#1}}{\MEndScripts\special{html:<!-- endscontent;;}\arabic{MSCounter}\special{html: //--></div>}\addtocounter{MSCounter}{1}}

% Fuegt eine zusaetzliche html-Seite ein hinter den bereits vorhandenen content-Seiten (oder als erste Seite) innerhalb der vor-zurueck-Schleife der Navigation
% #1 = Titel des Modulabschnitts, #2 = Kurztitel fuer die Buttons, #3 = Buttonkennung (STD = Defaultbutton, NONE = Ohne Button in der Navigation)
\newenvironment{MXContent}[3]{\special{html:<div class="xcontent}\arabic{MXCounter}\special{html:"><!-- xcontent;-;}\arabic{MXCounter};-;#1;-;#2;-;#3\special{html: //-->}\MPageScripts\MSubsubsection{#1}}{\MEndScripts\special{html:<!-- endxcontent;;}\arabic{MXCounter}\special{html: //--></div>}\addtocounter{MXCounter}{1}}

% Fuegt eine zusaetzliche html-Seite ein die keine subsubsection-Nummer bekommt, nur zur internen Verwendung in mintmod.tex gedacht!
% #1 = Titel des Modulabschnitts, #2 = Kurztitel fuer die Buttons, #3 = Buttonkennung (STD = Defaultbutton, NONE = Ohne Button in der Navigation)
% \newenvironment{MUContent}[3]{\special{html:<div class="xcontent}\arabic{MXCounter}\special{html:"><!-- xcontent;-;}\arabic{MXCounter};-;#1;-;#2;-;#3\special{html: //-->}\MPageScripts\MSubsubsectionx{#1}}{\MEndScripts\special{html:<!-- endxcontent;;}\arabic{MXCounter}\special{html: //--></div>}\addtocounter{MXCounter}{1}}

\newcommand{\MDeclareSiteUXID}[1]{\special{html:<!-- mdeclaresiteuxid;;}#1\special{html:;;}\arabic{chapter}\special{html:;;}\arabic{section}\special{html:;; //-->}}

\else

%\newcommand{\MSubsubsection}[1]{\refstepcounter{subsubsection} \addcontentsline{toc}{subsubsection}{\thesubsubsection. #1}}


% Fuegt eine zusaetzliche html-Seite an hinter den bereits vorhandenen content-Seiten
% #1 = Titel des Modulabschnitts, #2 = Kurztitel fuer die Buttons, #3 = Iconkennung (im PDF wirkungslos)
%\newenvironment{MUContent}[3]{\ifnum\value{MXCTest}>0{\MDebugMessage{ERROR: Geschachtelter SContent}}\fi\MPageScripts\MSubsubsectionx{#1}\addtocounter{MXCTest}{1}}{\addtocounter{MXCounter}{1}\addtocounter{MXCTest}{-1}}
\newenvironment{MXContent}[3]{\ifnum\value{MXCTest}>0{\MDebugMessage{ERROR: Geschachtelter SContent}}\fi\MPageScripts\MSubsubsection{#1}\addtocounter{MXCTest}{1}}{\addtocounter{MXCounter}{1}\addtocounter{MXCTest}{-1}}
\newenvironment{MSContent}[3]{\ifnum\value{MXCTest}>0{\MDebugMessage{ERROR: Geschachtelter XContent}}\fi\MPageScripts\MSubsubsectionx{#1}\addtocounter{MXCTest}{1}}{\addtocounter{MSCounter}{1}\addtocounter{MXCTest}{-1}}

\newcommand{\MDeclareSiteUXID}[1]{\relax}

\fi 

% GHEADER und GFOOTER werden von split.pm gefunden, aber nur, wenn nicht HELPSITE oder TESTSITE
\ifttm
\newenvironment{MSectionStart}{\special{html:<div class="xcontent0">}\MSubsubsectionx{Modul\"ubersicht}}{\setcounter{MSSEnd}{0}\special{html:</div>}}
% Darf nicht als XContent nummeriert werden, darf nicht als XContent gelabelt werden, wird aber in eine xcontent-div gesetzt fuer Python-parsing
\else
\newenvironment{MSectionStart}{\MSubsectionx{Modul\"ubersicht}}{\setcounter{MSSEnd}{0}}
\fi

\newenvironment{MIntro}{\begin{MXContent}{Einf\"uhrung}{Einf\"uhrung}{genetisch}}{\end{MXContent}}
\newenvironment{MContent}{\begin{MXContent}{Inhalt}{Inhalt}{beweis}}{\end{MXContent}}
\newenvironment{MExercises}{\ifttm\else\clearpage\fi\begin{MXContent}{Aufgaben}{Aufgaben}{aufgb}\special{html:<!-- declareexcsymb //-->}}{\end{MXContent}}

% #1 = Lesbare Testbezeichnung
\newenvironment{MTest}[1]{%
\renewcommand{\MTestName}{#1}
\ifttm\else\clearpage\fi%
\addtocounter{MTestSite}{1}%
\begin{MXContent}{#1}{#1}{STD} % {aufgb}%
\special{html:<!-- declaretestsymb //-->}
\begin{MQuestionGroup}%
\MInTestHeader
}%
{%
\end{MQuestionGroup}%
\ \\ \ \\%
\MInTestFooter
\end{MXContent}\addtocounter{MTestSite}{-1}%
}

\newenvironment{MExtra}{\ifttm\else\clearpage\fi\begin{MXContent}{Zus\"atzliche Inhalte}{Zusatz}{weiterfhrg}}{\end{MXContent}}

\makeindex

\ifttm
\def\MPrintIndex{
\ifnum\value{MSSEnd}>0{\MSubsectionEndMacros}\addtocounter{MSSEnd}{-1}\fi
\renewcommand{\indexname}{Stichwortverzeichnis}
\special{html:<p><!-- printindex //--></p>}
}
\else
\def\MPrintIndex{
\ifnum\value{MSSEnd}>0{\MSubsectionEndMacros}\addtocounter{MSSEnd}{-1}\fi
\renewcommand{\indexname}{Stichwortverzeichnis}
\addcontentsline{toc}{section}{Stichwortverzeichnis}
\printindex
}
\fi


% Konstanten fuer die Modulfaecher

\def\MINTMathematics{1}
\def\MINTInformatics{2}
\def\MINTChemistry{3}
\def\MINTPhysics{4}
\def\MINTEngineering{5}

\newcounter{MSubjectArea}
\newcounter{MInfoNumbers} % Gibt an, ob die Infoboxen nummeriert werden sollen
\newcounter{MSepNumbers} % Gibt an, ob Beispiele und Experimente separat nummeriert werden sollen
\newcommand{\MSetSubject}[1]{
 % ttm kapiert setcounter mit Parametern nicht, also per if abragen und einsetzen
\ifnum#1=1\setcounter{MSubjectArea}{1}\setcounter{MInfoNumbers}{1}\setcounter{MSepNumbers}{0}\fi
\ifnum#1=2\setcounter{MSubjectArea}{2}\setcounter{MInfoNumbers}{1}\setcounter{MSepNumbers}{0}\fi
\ifnum#1=3\setcounter{MSubjectArea}{3}\setcounter{MInfoNumbers}{0}\setcounter{MSepNumbers}{1}\fi
\ifnum#1=4\setcounter{MSubjectArea}{4}\setcounter{MInfoNumbers}{0}\setcounter{MSepNumbers}{0}\fi
\ifnum#1=5\setcounter{MSubjectArea}{5}\setcounter{MInfoNumbers}{1}\setcounter{MSepNumbers}{0}\fi
% Separate Nummerntechnik fuer unsere Chemiker: alles dreistellig
\ifnum#1=3
  \ifttm
  \renewcommand{\theequation}{\arabic{section}.\arabic{subsection}.\arabic{equation}}
  \renewcommand{\thetable}{\arabic{section}.\arabic{subsection}.\arabic{table}} 
  \renewcommand{\thefigure}{\arabic{section}.\arabic{subsection}.\arabic{figure}} 
  \else
  \renewcommand{\theequation}{\arabic{chapter}.\arabic{section}.\arabic{equation}}
  \renewcommand{\thetable}{\arabic{chapter}.\arabic{section}.\arabic{table}}
  \renewcommand{\thefigure}{\arabic{chapter}.\arabic{section}.\arabic{figure}}
  \fi
\else
  \ifttm
  \renewcommand{\theequation}{\arabic{section}.\arabic{subsection}.\arabic{equation}}
  \renewcommand{\thetable}{\arabic{table}}
  \renewcommand{\thefigure}{\arabic{figure}}
  \else
  \renewcommand{\theequation}{\arabic{chapter}.\arabic{section}.\arabic{equation}}
  \renewcommand{\thetable}{\arabic{table}}
  \renewcommand{\thefigure}{\arabic{figure}}
  \fi
\fi
}

% Fuer tikz Autogenerierung
\newcounter{MTIKZAutofilenumber}

% Spezielle Counter fuer die Bentz-Module
\newcounter{mycounter}
\newcounter{chemapplet}
\newcounter{physapplet}

\newcounter{MSSEnd} % Ist 1 falls ein MSubsection aktiv ist, der einen MSubsectionEndMacro-Aufruf verursacht
\newcounter{MFileNumber}
\def\MLastFile{\special{html:[[!-- mfileref;;}\arabic{MFileNumber}\special{html:; //--]]}}

% Vollstaendiger Pfad ist \MMaterial / \MLastFilePath / \MLastFileName    ==   \MMaterial / \MLastFile

% Wird nur bei kompletter Baum-Erstellung ausgefuehrt!
% #1 = Lesbare Modulbezeichnung
\newcommand{\MSectionStartMacros}[1]{
\setcounter{MTestSite}{0}
\setcounter{MCaptionOn}{0}
\setcounter{MLastTypeEq}{0}
\setcounter{MSSEnd}{0}
\setcounter{MFileNumber}{0} % Preinkrekement-Counter
\setcounter{MTIKZAutofilenumber}{0}
\setcounter{mycounter}{1}
\setcounter{physapplet}{1}
\setcounter{chemapplet}{0}
\ifttm
\special{html:<!-- mdeclaresection;;}\arabic{chapter}\special{html:;;}\arabic{section}\special{html:;;}#1\special{html:;; //-->}%
\else
\setcounter{thmc}{0}
\setcounter{exmpc}{0}
\setcounter{verc}{0}
\setcounter{infoc}{0}
\fi
\setcounter{MiniMarkerCounter}{1}
\setcounter{AlignCounter}{1}
\setcounter{MXCTest}{0}
\setcounter{MCodeCounter}{0}
\setcounter{MEntryCounter}{0}
}

% Wird immer ausgefuehrt
\newcommand{\MSubsectionStartMacros}{
\ifttm\else\MPageHeaderDef\fi
\MWatermarkSettings
\setcounter{MXCounter}{0}
\setcounter{MSCounter}{0}
\setcounter{MSiteCounter}{1}
\setcounter{MExerciseCollectionCounter}{0}
% Zaehler fuer das Labelsystem zuruecksetzen (prefix-Zaehler)
\setcounter{MInfoCounter}{0}
\setcounter{MExerciseCounter}{0}
\setcounter{MExampleCounter}{0}
\setcounter{MExperimentCounter}{0}
\setcounter{MGraphicsCounter}{0}
\setcounter{MTableCounter}{0}
\setcounter{MTheoremCounter}{0}
\setcounter{MObjectCounter}{0}
\setcounter{MEquationCounter}{0}
\setcounter{MVideoCounter}{0}
\setcounter{equation}{0}
\setcounter{figure}{0}
}

\newcommand{\MSubsectionEndMacros}{
% Bei Chemiemodulen das PSE einhaengen, es soll als SContent am Ende erscheinen
\special{html:<!-- subsectionend //-->}
\ifnum\value{MSubjectArea}=3{\MIncludePSE}\fi
}


\ifttm
%\newcommand{\MEmbed}[1]{\MRegisterFile{#1}\begin{html}<embed src="\end{html}\MMaterial/\MLastFile\begin{html}" width="192" height="189"></embed>\end{html}}
\newcommand{\MEmbed}[1]{\MRegisterFile{#1}\begin{html}<embed src="\end{html}\MMaterial/\MLastFile\begin{html}"></embed>\end{html}}
\fi

%----------------- Makros fuer die Textdarstellung -----------------------------------------------

\ifttm
% MUGraphics bindet eine Grafik ein:
% Parameter 1: Dateiname der Grafik, relativ zur Position des Modul-Tex-Dokuments
% Parameter 2: Skalierungsoptionen fuer PDF (fuer includegraphics)
% Parameter 3: Titel fuer die Grafik, wird unter die Grafik mit der Grafiknummer gesetzt und kann MLabel bzw. MCopyrightLabel enthalten
% Parameter 4: Skalierungsoptionen fuer HTML (css-styles)

% ERSATZ: <img alt="My Image" src="data:image/png;base64,iVBORwA<MoreBase64SringHere>" />


\newcommand{\MUGraphics}[4]{\MRegisterFile{#1}\begin{html}
<div class="imagecenter">
<center>
<div>
<img src="\end{html}\MMaterial/\MLastFile\begin{html}" style="#4" alt="\end{html}\MMaterial/\MLastFile\begin{html}"/>
</div>
<div class="bildtext">
\end{html}
\addtocounter{MGraphicsCounter}{1}
\setcounter{MLastIndex}{\value{MGraphicsCounter}}
\setcounter{MLastType}{8}
\addtocounter{MCaptionOn}{1}
\ifnum\value{MSepNumbers}=0
\textbf{Abbildung \arabic{MGraphicsCounter}:} #3
\else
\textbf{Abbildung \arabic{section}.\arabic{subsection}.\arabic{MGraphicsCounter}:} #3
\fi
\addtocounter{MCaptionOn}{-1}
\begin{html}
</div>
</center>
</div>
<br />
\end{html}%
\special{html:<!-- mfeedbackbutton;Abbildung;}\arabic{MGraphicsCounter}\special{html:;}\arabic{section}.\arabic{subsection}.\arabic{MGraphicsCounter}\special{html:; //-->}%
}

% MVideo bindet ein Video als Einzeldatei ein:
% Parameter 1: Dateiname des Videos, relativ zur Position des Modul-Tex-Dokuments, ohne die Endung ".mp4"
% Parameter 2: Titel fuer das Video (kann MLabel oder MCopyrightLabel enthalten), wird unter das Video mit der Videonummer gesetzt
\newcommand{\MVideo}[2]{\MRegisterFile{#1.mp4}\begin{html}
<div class="imagecenter">
<center>
<div>
<video width="95\%" controls="controls"><source src="\end{html}\MMaterial/#1.mp4\begin{html}" type="video/mp4">Ihr Browser kann keine MP4-Videos abspielen!</video>
</div>
<div class="bildtext">
\end{html}
\addtocounter{MVideoCounter}{1}
\setcounter{MLastIndex}{\value{MVideoCounter}}
\setcounter{MLastType}{12}
\addtocounter{MCaptionOn}{1}
\ifnum\value{MSepNumbers}=0
\textbf{Video \arabic{MVideoCounter}:} #2
\else
\textbf{Video \arabic{section}.\arabic{subsection}.\arabic{MVideoCounter}:} #2
\fi
\addtocounter{MCaptionOn}{-1}
\begin{html}
</div>
</center>
</div>
<br />
\end{html}}

\newcommand{\MDVideo}[2]{\MRegisterFile{#1.mp4}\MRegisterFile{#1.ogv}\begin{html}
<div class="imagecenter">
<center>
<div>
<video width="70\%" controls><source src="\end{html}\MMaterial/#1.mp4\begin{html}" type="video/mp4"><source src="\end{html}\MMaterial/#1.ogv\begin{html}" type="video/ogg">Ihr Browser kann keine MP4-Videos abspielen!</video>
</div>
<br />
#2
</center>
</div>
<br />
\end{html}
}

\newcommand{\MGraphics}[3]{\MUGraphics{#1}{#2}{#3}{}}

\else

\newcommand{\MVideo}[2]{%
% Kein Video im PDF darstellbar, trotzdem so tun als ob da eines waere
\begin{center}
(Video nicht darstellbar)
\end{center}
\addtocounter{MVideoCounter}{1}
\setcounter{MLastIndex}{\value{MVideoCounter}}
\setcounter{MLastType}{12}
\addtocounter{MCaptionOn}{1}
\ifnum\value{MSepNumbers}=0
\textbf{Video \arabic{MVideoCounter}:} #2
\else
\textbf{Video \arabic{section}.\arabic{subsection}.\arabic{MVideoCounter}:} #2
\fi
\addtocounter{MCaptionOn}{-1}
}


% MGraphics bindet eine Grafik ein:
% Parameter 1: Dateiname der Grafik, relativ zur Position des Modul-Tex-Dokuments
% Parameter 2: Skalierungsoptionen fuer PDF (fuer includegraphics)
% Parameter 3: Titel fuer die Grafik, wird unter die Grafik mit der Grafiknummer gesetzt
\newcommand{\MGraphics}[3]{%
\MRegisterFile{#1}%
\ %
\begin{figure}[H]%
\centering{%
\includegraphics[#2]{\MDPrefix/#1}%
\addtocounter{MCaptionOn}{1}%
\caption{#3}%
\addtocounter{MCaptionOn}{-1}%
}%
\end{figure}%
\addtocounter{MGraphicsCounter}{1}\setcounter{MLastIndex}{\value{MGraphicsCounter}}\setcounter{MLastType}{8}\ %
%\ \\Abbildung \ifnum\value{MSepNumbers}=0\else\arabic{chapter}.\arabic{section}.\fi\arabic{MGraphicsCounter}: #3%
}

\newcommand{\MUGraphics}[4]{\MGraphics{#1}{#2}{#3}}


\fi

\newcounter{MCaptionOn} % = 1 falls eine Grafikcaption aktiv ist, = 0 sonst


% MGraphicsSolo bindet eine Grafik pur ein ohne Titel
% Parameter 1: Dateiname der Grafik, relativ zur Position des Modul-Tex-Dokuments
% Parameter 2: Skalierungsoptionen (wirken nur im PDF)
\newcommand{\MGraphicsSolo}[2]{\MUGraphicsSolo{#1}{#2}{}}

% MUGraphicsSolo bindet eine Grafik pur ein ohne Titel, aber mit HTML-Skalierung
% Parameter 1: Dateiname der Grafik, relativ zur Position des Modul-Tex-Dokuments
% Parameter 2: Skalierungsoptionen (wirken nur im PDF)
% Parameter 3: Skalierungsoptionen (wirken nur im HTML), als style-format: "width=???, height=???"
\ifttm
\newcommand{\MUGraphicsSolo}[3]{\MRegisterFile{#1}\begin{html}
<img src="\end{html}\MMaterial/\MLastFile\begin{html}" style="\end{html}#3\begin{html}" alt="\end{html}\MMaterial/\MLastFile\begin{html}"/>
\end{html}%
\special{html:<!-- mfeedbackbutton;Abbildung;}#1\special{html:;}\MMaterial/\MLastFile\special{html:; //-->}%
}
\else
\newcommand{\MUGraphicsSolo}[3]{\MRegisterFile{#1}\includegraphics[#2]{\MDPrefix/#1}}
\fi

% Externer Link mit URL
% Erster Parameter: Vollstaendige(!) URL des Links
% Zweiter Parameter: Text fuer den Link
\newcommand{\MExtLink}[2]{\ifttm\special{html:<a target="_new" href="}#1\special{html:">}#2\special{html:</a>}\else\href{#1}{#2}\fi} % ohne MINTERLINK!


% Interner Link, die verlinkte Datei muss im gleichen Verzeichnis liegen wie die Modul-Texdatei
% Erster Parameter: Dateiname
% Zweiter Parameter: Text fuer den Link
\newcommand{\MIntLink}[2]{\ifttm\MRegisterFile{#1}\special{html:<a class="MINTERLINK" target="_new" href="}\MMaterial/\MLastFile\special{html:">}#2\special{html:</a>}\else{\href{#1}{#2}}\fi}


\ifttm
\def\MMaterial{:localmaterial:}
\else
\def\MMaterial{\MDPrefix}
\fi

\ifttm
\def\MNoFile#1{:directmaterial:#1}
\else
\def\MNoFile#1{#1}
\fi

\newcommand{\MChem}[1]{$\mathrm{#1}$}

\newcommand{\MApplet}[3]{
% Bindet ein Java-Applet ein, die Parameter sind:
% (wird nur im HTML, aber nicht im PDF erstellt)
% #1 Dateiname des Applets (muss mit ".class" enden)
% #2 = Breite in Pixeln
% #3 = Hoehe in Pixeln
\ifttm
\MRegisterFile{#1}
\begin{html}
<applet code="\end{html}\MMaterial/\MLastFile\begin{html}" width="#2" height="#3" alt="[Java-Applet kann nicht gestartet werden]"></applet>
\end{html}
\fi
}

\newcommand{\MScriptPage}[2]{
% Bindet eine JavaScript-Datei ein, die eine eigene Seite bekommt
% (wird nur im HTML, aber nicht im PDF erstellt)
% #1 Dateiname des Programms (sollte mit ".js" enden)
% #2 = Kurztitel der Seite
\ifttm
\begin{MSContent}{#2}{#2}{puzzle}
\MRegisterFile{#1}
\begin{html}
<script src="\MMaterial/\MLastFile" type="text/javascript"></script>
\end{html}
\end{MSContent}
\fi
}

\newcommand{\MIncludePSE}{
% Bindet bei Chemie-Modulen das PSE ein
% (wird nur im HTML, aber nicht im PDF erstellt)
\ifttm
\special{html:<!-- includepse //-->}
\begin{MSContent}{Periodensystem der Elemente}{PSE}{table}
\MRegisterFile{../files/pse.js}
\MRegisterFile{../files/radio.png}
\begin{html}
<script src="\MMaterial/../files/pse.js" type="text/javascript"></script>
<p id="divid"><br /><br />
<script language="javascript" type="text/javascript">
    startpse("divid","\MMaterial/../files"); 
</script>
</p>
<br />
<br />
<br />
<p>Die Farben der Elementsymbole geben an: <font style="color:Red">gasf&ouml;rmig </font> <font style="color:Blue">fl&uuml;ssig </font> fest</p>
<p>Die Elemente der Gruppe 1 A, 2 A, 3 A usw. geh&ouml;ren zu den Hauptgruppenelementen.</p>
<p>Die Elemente der Gruppe 1 B, 2 B, 3 B usw. geh&ouml;ren zu den Nebengruppenelementen.</p>
<p>() kennzeichnet die Masse des stabilsten Isotops</p>
\end{html}
\end{MSContent}
\fi
}

\newcommand{\MAppletArchive}[4]{
% Bindet ein Java-Applet ein, die Parameter sind:
% (wird nur im HTML, aber nicht im PDF erstellt)
% #1 Dateiname der Klasse mit Appletaufruf (muss mit ".class" enden)
% #2 Dateiname des Archivs (muss mit ".jar" enden)
% #3 = Breite in Pixeln
% #4 = Hoehe in Pixeln
\ifttm
\MRegisterFile{#2}
\begin{html}
<applet code="#1" archive="\end{html}\MMaterial/\MLastFile\begin{html}" codebase="." width="#3" height="#4" alt="[Java-Archiv kann nicht gestartet werden]"></applet>
\end{html}
\fi
}

% Bindet in der Haupttexdatei ein MINT-Modul ein. Parameter 1 ist das Verzeichnis (relativ zur Haupttexdatei), Parameter 2 ist der Dateinahme ohne Pfad.
\newcommand{\IncludeModule}[2]{
\renewcommand{\MDPrefix}{#1}
\input{#1/#2}
\ifnum\value{MSSEnd}>0{\MSubsectionEndMacros}\addtocounter{MSSEnd}{-1}\fi
}

% Der ttm-Konverter setzt keine Makros im \input um, also muss hier getrickst werden:
% Das MDPrefix muss in den einzelnen Modulen manuell eingesetzt werden
\newcommand{\MInputFile}[1]{
\ifttm
\input{#1}
\else
\input{#1}
\fi
}


\newcommand{\MCases}[1]{\left\lbrace{\begin{array}{rl} #1 \end{array}}\right.}

\ifttm
\newenvironment{MCaseEnv}{\left\lbrace\begin{array}{rl}}{\end{array}\right.}
\else
\newenvironment{MCaseEnv}{\left\lbrace\begin{array}{rl}}{\end{array}\right.}
\fi

\def\MSkip{\ifttm\MCR\fi}

\ifttm
\def\MCR{\special{html:<br />}}
\else
\def\MCR{\ \\}
\fi


% Pragmas - Sind Schluesselwoerter, die dem Preprocessing sowie dem Konverter uebergeben werden und bestimmte
%           Aktionen ausloesen. Im Output (PDF und HTML) tauchen sie nicht auf.
\newcommand{\MPragma}[1]{%
\ifttm%
\special{html:<!-- mpragma;-;}#1\special{html:;; -->}%
\else%
% MPragmas werden vom Preprozessor direkt im LaTeX gefunden
\fi%
}

% Ersatz der Befehle textsubscript und textsuperscript, die ttm nicht kennt
\ifttm%
\newcommand{\MTextsubscript}[1]{\special{html:<sub>}#1\special{html:</sub>}}%
\newcommand{\MTextsuperscript}[1]{\special{html:<sup>}#1\special{html:</sup>}}%
\else%
\newcommand{\MTextsubscript}[1]{\textsubscript{#1}}%
\newcommand{\MTextsuperscript}[1]{\textsuperscript{#1}}%
\fi

%------------------ Einbindung von dia-Diagrammen ----------------------------------------------
% Beim preprocessing wird aus jeder dia-Datei eine tex-Datei und eine pdf-Datei erzeugt,
% diese werden hier jeweils im PDF und HTML eingebunden
% Parameter: Dateiname der mit dia erstellten Datei (OHNE die Endung .dia)
\ifttm%
\newcommand{\MDia}[1]{%
\MGraphicsSolo{#1minthtml.png}{}%
}
\else%
\newcommand{\MDia}[1]{%
\MGraphicsSolo{#1mintpdf.png}{scale=0.1667}%
}
\fi%

% subsup funktioniert im Ausdruck $D={\R}^+_0$, also \R geklammert und sup zuerst
% \ifttm
% \def\MSubsup#1#2#3{\special{html:<msubsup>} #1 #2 #3\special{html:</msubsup>}}
% \else
% \def\MSubsup#1#2#3{{#1}^{#3}_{#2}}
% \fi

%\input{local.tex}

% \ifttm
% \else
% \newwrite\mintlog
% \immediate\openout\mintlog=mintlog.txt
% \fi

% ----------------------- tikz autogenerator -------------------------------------------------------------------

\newcommand{\Mtikzexternalize}{\tikzexternalize}% wird bei Konvertierung ueber mconvert ggf. ausgehebelt!

\ifttm
\else
\tikzset%
{
  % Defines a custom style which generates pdf and converts to (low and hi-res quality) png and svg, then deletes the pdf
  % Important: DO NOT directly convert from pdf to hires-png or from svg to png with GraphViz convert as it has some problems and memory leaks
  png export/.style=%
  {
    external/system call/.add={}{; 
      pdf2svg "\image.pdf" "\image.svg" ; 
      convert -density 112.5 -transparent white "\image.pdf" "\image.png"; 
      inkscape --export-png="\image.4x.png" --export-dpi=450 --export-background-opacity=0 --without-gui "\image.svg"; 
      rm "\image.pdf"; rm "\image.log"; rm "\image.dpth"; rm "\image.idx"
    },
    external/force remake,
  }
}
\tikzset{png export}
\tikzsetexternalprefix{}
% PNGs bei externer Erzeugung in "richtiger" Groesse einbinden
\pgfkeys{/pgf/images/include external/.code={\includegraphics[scale=0.64]{#1}}}
\fi

% Spezielle Umgebung fuer Autogenerierung, Bildernamen sind nur innerhalb eines Moduls (einer MSection) eindeutig)

\newcommand{\MTIKZautofilename}{tikzautofile}

\ifttm
% HTML-Version: Vom Autogenerator erzeugte png-Datei einbinden, tikz selbst nicht ausfuehren (sprich: #1 schlucken)
\newcommand{\MTikzAuto}[1]{%
\addtocounter{MTIKZAutofilenumber}{1}
\renewcommand{\MTIKZautofilename}{mtikzauto_\arabic{MTIKZAutofilenumber}}
\MUGraphicsSolo{\MSectionID\MTIKZautofilename.4x.png}{scale=1}{\special{html:[[!-- svgstyle;}\MSectionID\MTIKZautofilename\special{html: //--]]}} % Styleinfos werden aus original-png, nicht 4x-png geholt!
%\MRegisterFile{\MSectionID\MTIKZautofilename.png} % not used right now
%\MRegisterFile{\MSectionID\MTIKZautofilename.svg}
}
\else%
% PDF-Version: Falls Autogenerator aktiv wird Datei automatisch benannt und exportiert
\newcommand{\MTikzAuto}[1]{%
\addtocounter{MTIKZAutofilenumber}{1}%
\renewcommand{\MTIKZautofilename}{mtikzauto_\arabic{MTIKZAutofilenumber}}
\tikzsetnextfilename{\MTIKZautofilename}%
#1%
}
\fi

% In einer reinen LaTeX-Uebersetzung kapselt der Preambelinclude-Befehl nur input,
% in einer konvertergesteuerten PDF/HTML-Uebersetzung wird er dagegen entfernt und
% die Preambeln an mintmod angehaengt, die Ersetzung wird von mconvert.pl vorgenommen.

\newcommand{\MPreambleInclude}[1]{\input{#1}}

% Globale Watermarksettings (werden auch nochmal zu Beginn jedes subsection gesetzt,
% muessen hier aber auch global ausgefuehrt wegen Einfuehrungsseiten und Inhaltsverzeichnis

\MWatermarkSettings
% ---------------------------------- Parametrisierte Aufgaben ----------------------------------------

\ifttm
\newenvironment{MPExercise}{%
\begin{MExercise}%
}{%
\special{html:<button name="Name_MPEX}\arabic{MExerciseCounter}\special{html:" id="MPEX}\arabic{MExerciseCounter}%
\special{html:" type="button" onclick="reroll('}\arabic{MExerciseCounter}\special{html:');">Neue Aufgabe erzeugen</button>}%
\end{MExercise}%
}
\else
\newenvironment{MPExercise}{%
\begin{MExercise}%
}{%
\end{MExercise}%
}
\fi

% Parameter: Name, Min, Max, PDF-Standard. Name in Deklaration OHNE backslash, im Code MIT Backslash
\ifttm
\newcommand{\MGlobalInteger}[4]{\special{html:%
<!-- onloadstart //-->%
MVAR.push(createGlobalInteger("}#1\special{html:",}#2\special{html:,}#3\special{html:,}#4\special{html:)); %
<!-- onloadstop //-->%
<!-- viewmodelstart //-->%
ob}#1\special{html:: ko.observable(rerollMVar("}#1\special{html:")),%
<!-- viewmodelstop //-->%
}%
}%
\else%
\newcommand{\MGlobalInteger}[4]{\newcounter{mvc_#1}\setcounter{mvc_#1}{#4}}
\fi

% Parameter: Name, Min, Max, PDF-Standard. Name in Deklaration OHNE backslash, im Code MIT Backslash, Wert ist Wurzel von value
\ifttm
\newcommand{\MGlobalSqrt}[4]{\special{html:%
<!-- onloadstart //-->%
MVAR.push(createGlobalSqrt("}#1\special{html:",}#2\special{html:,}#3\special{html:,}#4\special{html:)); %
<!-- onloadstop //-->%
<!-- viewmodelstart //-->%
ob}#1\special{html:: ko.observable(rerollMVar("}#1\special{html:")),%
<!-- viewmodelstop //-->%
}%
}%
\else%
\newcommand{\MGlobalSqrt}[4]{\newcounter{mvc_#1}\setcounter{mvc_#1}{#4}}% Funktioniert nicht als Wurzel !!!
\fi

% Parameter: Name, Min, Max, PDF-Standard zaehler, PDF-Standard nenner. Name in Deklaration OHNE backslash, im Code MIT Backslash
\ifttm
\newcommand{\MGlobalFraction}[5]{\special{html:%
<!-- onloadstart //-->%
MVAR.push(createGlobalFraction("}#1\special{html:",}#2\special{html:,}#3\special{html:,}#4\special{html:,}#5\special{html:)); %
<!-- onloadstop //-->%
<!-- viewmodelstart //-->%
ob}#1\special{html:: ko.observable(rerollMVar("}#1\special{html:")),%
<!-- viewmodelstop //-->%
}%
}%
\else%
\newcommand{\MGlobalFraction}[5]{\newcounter{mvc_#1}\setcounter{mvc_#1}{#4}} % Funktioniert nicht als Bruch !!!
\fi

% MVar darf im HTML nur in MEvalMathDisplay-Umgebungen genutzt werden oder in Strings die an den Parser uebergeben werden
\ifttm%
\newcommand{\MVar}[1]{\special{html:[var_}#1\special{html:]}}%
\else%
\newcommand{\MVar}[1]{\arabic{mvc_#1}}%
\fi

\ifttm%
\newcommand{\MRerollButton}[2]{\special{html:<button type="button" onclick="rerollMVar('}#1\special{html:');">}#2\special{html:</button>}}%
\else%
\newcommand{\MRerollButton}[2]{\relax}% Keine sinnvolle Entsprechung im PDF
\fi

% MEvalMathDisplay fuer HTML wird in mconvert.pl im preprocessing realisiert
% PDF: eine equation*-Umgebung (ueber amsmath)
% HTML: Eine Mathjax-Tex-Umgebung, deren Auswertung mit knockout-obervablen gekoppelt ist
% PDF-Version hier nur fuer pdflatex-only-Uebersetzung gegeben

\ifttm\else\newenvironment{MEvalMathDisplay}{\begin{equation*}}{\end{equation*}}\fi

% ---------------------------------- Spezialbefehle fuer AD ------------------------------------------

%Abk�rzung f�r \longrightarrow:
\newcommand{\lto}{\ensuremath{\longrightarrow}}

%Makro f�r Funktionen:
\newcommand{\exfunction}[5]
{\begin{array}{rrcl}
 #1 \colon  & #2 &\lto & #3 \\[.05cm]  
  & #4 &\longmapsto  & #5 
\end{array}}

\newcommand{\function}[5]{%
#1:\;\left\lbrace{\begin{array}{rcl}
 #2 &\lto & #3 \\
 #4 &\longmapsto  & #5 \end{array}}\right.}


%Die Identit�t:
\DeclareMathOperator{\Id}{Id}

%Die Signumfunktion:
\DeclareMathOperator{\sgn}{sgn}

%Zwei Betonungskommandos (k�nnen angepasst werden):
\newcommand{\highlight}[1]{#1}
\newcommand{\modstextbf}[1]{#1}
\newcommand{\modsemph}[1]{#1}


% ---------------------------------- Spezialbefehle fuer JL ------------------------------------------


\def\jccolorfkt{green!50!black} %Farbe des Funktionsgraphen
\def\jccolorfktarea{green!25!white} %Farbe der Fl"ache unter dem Graphen
\def\jccolorfktareahell{green!12!white} %helle Einf"arbung der Fl"ache unter dem Graphen
\def\jccolorfktwert{green!50!black} %Farbe einzelner Punkte des Graphen

\newcommand{\MPfadBilder}{Bilder}

\ifttm%
\newcommand{\jMD}{\,\MD}%
\else%
\newcommand{\jMD}{\;\MD}%
\fi%

\def\jHTMLHinweisBedienung{\MInputHint{%
Mit Hilfe der Symbole am oberen Rand des Fensters
k"onnen Sie durch die einzelnen Abschnitte navigieren.}}

\def\jHTMLHinweisEingabeText{\MInputHint{%
Geben Sie jeweils ein Wort oder Zeichen als Antwort ein.}}

\def\jHTMLHinweisEingabeTerm{\MInputHint{%
Klammern Sie Ihre Terme, um eine eindeutige Eingabe zu erhalten. 
Beispiel: Der Term $\frac{3x+1}{x-2}$ soll in der Form
\texttt{(3*x+1)/((x+2)^2}$ eingegeben werden (wobei auch Leerzeichen 
eingegeben werden k"onnen, damit eine Formel besser lesbar ist).}}

\def\jHTMLHinweisEingabeIntervalle{\MInputHint{%
Intervalle werden links mit einer "offnenden Klammer und rechts mit einer 
schlie"senden Klammer angegeben. Eine runde Klammer wird verwendet, wenn der 
Rand nicht dazu geh"ort, eine eckige, wenn er dazu geh"ort. 
Als Trennzeichen wird ein Komma oder ein Semikolon akzeptiert.
Beispiele: $(a, b)$ offenes Intervall,
$[a; b)$ links abgeschlossenes, rechts offenes Intervall von $a$ bis $b$. 
Die Eingabe $]a;b[$ f"ur ein offenes Intervall wird nicht akzeptiert.
F"ur $\infty$ kann \texttt{infty} oder \texttt{unendlich} geschrieben werden.}}

\def\jHTMLHinweisEingabeFunktionen{\MInputHint{%
Schreiben Sie Malpunkte (geschrieben als \texttt{*}) aus und setzen Sie Klammern um Argumente f�r Funktionen.
Beispiele: Polynom: \texttt{3*x + 0.1}, Sinusfunktion: \texttt{sin(x)}, 
Verkettung von cos und Wurzel: \texttt{cos(sqrt(3*x))}.}}

\def\jHTMLHinweisEingabeFunktionenSinCos{\MInputHint{%
Die Sinusfunktion $\sin x$ wird in der Form \texttt{sin(x)} angegeben, %
$\cos\left(\sqrt{3 x}\right)$ durch \texttt{cos(sqrt(3*x))}.}}

\def\jHTMLHinweisEingabeFunktionenExp{\MInputHint{%
Die Exponentialfunktion $\MEU^{3x^4 + 5}$ wird als
\texttt{exp(3 * x^4 + 5)} angegeben, %
$\ln\left(\sqrt{x} + 3.2\right)$ durch \texttt{ln(sqrt(x) + 3.2)}.}}

% ---------------------------------- Spezialbefehle fuer Fachbereich Physik --------------------------

\newcommand{\E}{{e}}
\newcommand{\ME}[1]{\cdot 10^{#1}}
\newcommand{\MU}[1]{\;\mathrm{#1}}
\newcommand{\MPG}[3]{%
  \ifnum#2=0%
    #1\ \mathrm{#3}%
  \else%
    #1\cdot 10^{#2}\ \mathrm{#3}%
  \fi}%
%

\newcommand{\MMul}{\MExponentensymbXYZl} % Nur eine Abkuerzung


% ---------------------------------- Stichwortfunktionialitaet ---------------------------------------

% mpreindexentry wird durch Auswahlroutine in conv.pl durch mindexentry substitutiert
\ifttm%
\def\MIndex#1{\index{#1}\special{html:<!-- mpreindexentry;;}#1\special{html:;;}\arabic{MSubjectArea}\special{html:;;}%
\arabic{chapter}\special{html:;;}\arabic{section}\special{html:;;}\arabic{subsection}\special{html:;;}\arabic{MEntryCounter}\special{html:; //-->}%
\setcounter{MLastIndex}{\value{MEntryCounter}}%
\addtocounter{MEntryCounter}{1}%
}%
% Copyrightliste wird als tex-Datei im preprocessing von conv.pl erzeugt und unter converter/tex/entrycollection.tex abgelegt
% Der input-Befehl funktioniert nur, wenn die aufrufende tex-Datei auf der obersten Ebene liegt (d.h. selbst kein input/include ist, insbesondere keine Moduldatei)
\def\MEntryList{} % \input funktioniert nicht, weil ttm (und damit das \input) ausgefuehrt wird, bevor Datei da ist
\else%
\def\MIndex#1{\index{#1}}
\def\MEntryList{\MAbort{Stichwortliste nur im HTML realisierbar}}%
\fi%

\def\MEntry#1#2{\textbf{#1}\MIndex{#2}} % Idee: MLastType auf neuen Entry-Typ und dann ein MLabel vergeben mit autogen-Nummer

% ---------------------------------- Befehle fuer Tests ----------------------------------------------

% MEquationItem stellt eine Eingabezeile der Form Vorgabe = Antwortfeld her, der zweite Parameter kann z.B. MSimplifyQuestion-Befehl sein
\ifttm
\newcommand{\MEquationItem}[2]{{#1}$\,=\,${#2}}%
\else%
\newcommand{\MEquationItem}[2]{{#1}$\;\;=\,${#2}}%
\fi

\ifttm
\newcommand{\MInputHint}[1]{%
\ifnum%
\if\value{MTestSite}>0%
\else%
{\color{blue}#1}%
\fi%
\fi%
}
\else
\newcommand{\MInputHint}[1]{\relax}
\fi

\ifttm
\newcommand{\MInTestHeader}{%
Dies ist ein einreichbarer Test:
\begin{itemize}
\item{Im Gegensatz zu den offenen Aufgaben werden beim Eingeben keine Hinweise zur Formulierung der mathematischen Ausdr�cke gegeben.}
\item{Der Test kann jederzeit neu gestartet oder verlassen werden.}
\item{Der Test kann durch die Buttons am Ende der Seite beendet und abgeschickt, oder zur�ckgesetzt werden.}
\item{Der Test kann mehrfach probiert werden. F�r die Statistik z�hlt die zuletzt abgeschickte Version.}
\end{itemize}
}
\else
\newcommand{\MInTestHeader}{%
\relax
}
\fi

\ifttm
\newcommand{\MInTestFooter}{%
\special{html:<button name="Name_TESTFINISH" id="TESTFINISH" type="button" onclick="finish_button('}\MTestName\special{html:');">Test auswerten</button>}%
\begin{html}
&nbsp;&nbsp;&nbsp;&nbsp;&nbsp;&nbsp;&nbsp;&nbsp;
<button name="Name_TESTRESET" id="TESTRESET" type="button" onclick="reset_button();">Test zur�cksetzen</button>
<br />
<br />
<div class="xreply">
<p name="Name_TESTEVAL" id="TESTEVAL">
Hier erscheint die Testauswertung!
<br />
</p>
</div>
\end{html}
}
\else
\newcommand{\MInTestFooter}{%
\relax
}
\fi


% ---------------------------------- Notationsmakros -------------------------------------------------------------

% Notationsmakros die nicht von der Kursvariante abhaengig sind

\newcommand{\MZahltrennzeichen}[1]{\renewcommand{\MZXYZhltrennzeichen}{#1}}

\ifttm
\newcommand{\MZahl}[3][\MZXYZhltrennzeichen]{\edef\MZXYZtemp{\noexpand\special{html:<mn>#2#1#3</mn>}}\MZXYZtemp}
\else
\newcommand{\MZahl}[3][\MZXYZhltrennzeichen]{{}#2{#1}#3}
\fi

\newcommand{\MEinheitenabstand}[1]{\renewcommand{\MEinheitenabstXYZnd}{#1}}
\ifttm
\newcommand{\MEinheit}[2][\MEinheitenabstXYZnd]{{}#1\edef\MEINHtemp{\noexpand\special{html:<mi mathvariant="normal">#2</mi>}}\MEINHtemp} 
\else
\newcommand{\MEinheit}[2][\MEinheitenabstXYZnd]{{}#1 \mathrm{#2}} 
\fi

\newcommand{\MExponentensymbol}[1]{\renewcommand{\MExponentensymbXYZl}{#1}}
\newcommand{\MExponent}[2][\MExponentensymbXYZl]{{}#1{} 10^{#2}} 

%Punkte in 2 und 3 Dimensionen
\newcommand{\MPointTwo}[3][]{#1(#2\MCoordPointSep #3{}#1)}
\newcommand{\MPointThree}[4][]{#1(#2\MCoordPointSep #3\MCoordPointSep #4{}#1)}
\newcommand{\MPointTwoAS}[2]{\left(#1\MCoordPointSep #2\right)}
\newcommand{\MPointThreeAS}[3]{\left(#1\MCoordPointSep #2\MCoordPointSep #3\right)}

% Masseinheit, Standardabstand: \,
\newcommand{\MEinheitenabstXYZnd}{\MThinspace} 

% Horizontaler Leerraum zwischen herausgestellter Formel und Interpunktion
\ifttm
\newcommand{\MDFPSpace}{\,}
\newcommand{\MDFPaSpace}{\,\,}
\newcommand{\MBlank}{\ }
\else
\newcommand{\MDFPSpace}{\;}
\newcommand{\MDFPaSpace}{\;\;}
\newcommand{\MBlank}{\ }
\fi

% Satzende in herausgestellter Formel mit horizontalem Leerraum
\newcommand{\MDFPeriod}{\MDFPSpace .}

% Separation von Aufzaehlung und Bedingung in Menge
\newcommand{\MCondSetSep}{\,:\,} %oder '\mid'

% Konverter kennt mathopen nicht
\ifttm
\def\mathopen#1{}
\fi

% -----------------------------------START Rouletteaufgaben ------------------------------------------------------------

\ifttm
% #1 = Dateiname, #2 = eindeutige ID fuer das Roulette im Kurs
\newcommand{\MDirectRouletteExercises}[2]{
\begin{MExercise}
\texttt{Im HTML erscheinen hier Aufgaben aus einer Aufgabenliste...}
\end{MExercise}
}
\else
\newcommand{\MDirectRouletteExercises}[2]{\relax} % wird durch mconvert.pl gefunden und ersetzt
\fi


% ---------------------------------- START Makros, die von der Kursvariante abhaengen ----------------------------------

\ifvariantunotation
  % unotation = An Universitaeten uebliche Notation
  \def\MVariant{unotation}

  % Trennzeichen fuer Dezimalzahlen
  \newcommand{\MZXYZhltrennzeichen}{.}

  % Exponent zur Basis 10 in der Exponentialschreibweise, 
  % Standardmalzeichen: \times
  \newcommand{\MExponentensymbXYZl}{\times} 

  % Begrenzungszeichen fuer offene Intervalle
  \newcommand{\MoIl}[1][]{\mbox{}#1(\mathopen{}} % bzw. ']'
  \newcommand{\MoIr}[1][]{#1)\mbox{}} % bzw. '['

  % Zahlen-Separation im IntervaLL
  \newcommand{\MIntvlSep}{,} %oder ';'

  % Separation von Elementen in Mengen
  \newcommand{\MElSetSep}{,} %oder ';'

  % Separation von Koordinaten in Punkten
  \newcommand{\MCoordPointSep}{,} %oder ';' oder '|', '\MThinspace|\MThinspace'

\else
  % An dieser Stelle wird angenommen, dass std-Variante aktiv ist
  % std = beschlossene Notation im TU9-Onlinekurs 
  \def\MVariant{std}

  % Trennzeichen fuer Dezimalzahlen
  \newcommand{\MZXYZhltrennzeichen}{,}

  % Exponent zur Basis 10 in der Exponentialschreibweise, 
  % Standardmalzeichen: \times
  \newcommand{\MExponentensymbXYZl}{\times} 

  % Begrenzungszeichen fuer offene Intervalle
  \newcommand{\MoIl}[1][]{\mbox{}#1]\mathopen{}} % bzw. '('
  \newcommand{\MoIr}[1][]{#1[\mbox{}} % bzw. ')'

  % Zahlen-Separation im IntervaLL
  \newcommand{\MIntvlSep}{;} %oder ','
  
  % Separation von Elementen in Mengen
  \newcommand{\MElSetSep}{;} %oder ','

  % Separation von Koordinaten in Punkten
  \newcommand{\MCoordPointSep}{;} %oder '|', '\MThinspace|\MThinspace'

\fi



% ---------------------------------- ENDE Makros, die von der Kursvariante abhaengen ----------------------------------


% diese Kommandos setzen Mathemodus vorraus
\newcommand{\MGeoAbstand}[2]{[\overline{{#1}{#2}}]}
\newcommand{\MGeoGerade}[2]{{#1}{#2}}
\newcommand{\MGeoStrecke}[2]{\overline{{#1}{#2}}}
\newcommand{\MGeoDreieck}[3]{{#1}{#2}{#3}}

%
\ifttm
\newcommand{\MOhm}{\special{html:<mn>&#x3A9;</mn>}}
\else
\newcommand{\MOhm}{\Omega} %\varOmega
\fi


\def\PERCTAG{\MAbort{PERCTAG ist zur internen verwendung in mconvert.pl reserviert, dieses Makro darf sonst nicht benutzt werden.}}

% Im Gegensatz zu einfachen html-Umgebungen werden MDirectHTML-Umgebungen von mconvert.pl am ganzen ttm-Prozess vorbeigeschleust und aus dem PDF komplett ausgeschnitten
\ifttm%
\newenvironment{MDirectHTML}{\begin{html}}{\end{html}}%
\else%
\newenvironment{MDirectHTML}{\begin{html}}{\end{html}}%
\fi

% Im Gegensatz zu einfachen Mathe-Umgebungen werden MDirectMath-Umgebungen von mconvert.pl am ganzen ttm-Prozess vorbeigeschleust, ueber MathJax realisiert, und im PDF als $$ ... $$ gesetzt
\ifttm%
\newenvironment{MDirectMath}{\begin{html}}{\end{html}}%
\else%
\newenvironment{MDirectMath}{\begin{equation*}}{\end{equation*}}% Vorsicht, auch \[ und \] werden in amsmath durch equation* redefiniert
\fi

% ---------------------------------- Location Management ---------------------------------------------

% #1 = buttonname (muss in files/images liegen und Format 48x48 haben), #2 = Vollstaendiger Einrichtungsname, #3 = Kuerzel der Einrichtung,  #4 = Name der include-texdatei
\ifttm
\newcommand{\MLocationSite}[3]{\special{html:<!-- mlocation;;}#1\special{html:;;}#2\special{html:;;}#3\special{html:;; //-->}}
\else
\newcommand{\MLocationSite}[3]{\relax}
\fi

% ---------------------------------- Copyright Management --------------------------------------------

\newcommand{\MCCLicense}{%
{\color{green}\textbf{CC BY-SA 3.0}}
}

\newcommand{\MCopyrightLabel}[1]{ (\MSRef{L_COPYRIGHTCOLLECTION}{Lizenz})\MLabel{#1}}

% Copyrightliste wird als tex-Datei im preprocessing erzeugt und unter converter/tex/copyrightcollection.tex abgelegt
% Der input-Befehl funktioniert nur, wenn die aufrufende tex-Datei auf der obersten Ebene liegt (d.h. selbst kein input/include ist, insbesondere keine Moduldatei)
\newcommand{\MCopyrightCollection}{\input{copyrightcollection.tex}}

% MCopyrightNotice fuegt eine Copyrightnotiz ein, der parser ersetzt diese durch CopyrightNoticePOST im preparsing, diese Definition wird nur fuer reine pdflatex-Uebersetzungen gebraucht
% Parameter: #1: Kurze Lizenzbeschreibung (typischerweise \MCCLicense)
%            #2: Link zum Original (http://...) oder NONE falls das Bild selbst ein Original ist, oder TIKZ falls das Bild aus einer tikz-Umgebung stammt
%            #3: Link zum Autor (http://...) oder MINT falls Original im MINT-Kolleg erstellt oder NONE falls Autor unbekannt
%            #4: Bemerkung (z.B. dass Datei mit Maple exportiert wurde)
%            #5: Labelstring fuer existierendes Label auf das copyrighted Objekt, mit MCopyrightLabel erzeugt
%            Keines der Felder darf leer sein!
\newcommand{\MCopyrightNotice}[5]{\MCopyrightNoticePOST{#1}{#2}{#3}{#4}{#5}}

\ifttm%
\newcommand{\MCopyrightNoticePOST}[5]{\relax}%
\else%
\newcommand{\MCopyrightNoticePOST}[5]{\relax}%
\fi%

% ---------------------------------- Meldungen fuer den Benutzer des Konverters ----------------------
\MPragma{mintmodversion;P0.1.0}
\MPragma{usercomment;This is file mintmod.tex version P0.1.0}


% ----------------------------------- Spezialelemente fuer Konfigurationsseite, werden nicht von mintscripts.js verwaltet --

% #1 = DOM-id der Box
\ifttm\newcommand{\MConfigbox}[1]{\special{html:<input cfieldtype="2" type="checkbox" name="Name_}#1\special{html:" id="}#1\special{html:" onchange="confHandlerChange('}#1\special{html:');"/>}}\fi % darf im PDF nicht aufgerufen werden!


\MPragma{MathSkip}
\newcommand{\MGrad}{^{\circ}}
\usepackage{ngerman}

\Mtikzexternalize

\MSetSubject{\MINTMathematics}

\begin{document}

\MSection{Geometry}
\MLabel{VBKM05}
\MSetSectionID{VBKM05} % hier identisch mit dem alten tikz-Dateien-Prefix 

\begin{MSectionStart}
\MDeclareSiteUXID{VBKM05_START}

\MModstartBox

The first sections of this chapter will introduce you to elementary geometry, while referring 
to the previous chapters. As a main topic, we first deal with the properties of triangles before 
calculating areas of polygons and volumes of simple geometric solids. Advanced problems are solved 
by means of trigonometric functions. These will give us a first taste of the later modules 
on calculus and analytic geometry.

 
\end{MSectionStart}


%jgl: section 1:
\MSubsection{Elements of Plane Geometry}
\MLabel{M05_Grundbegriffe}

\begin{MIntro}
\MDeclareSiteUXID{VBKM05_Grundbegriffe_Intro}


Looking at the stars on a clear, moonless night conveys a vivid impression of the 
elementary objects of geometry, namely the points. Since time immemorial, people mentally 
connected the points of light in the night sky by lines they subsequently interpreted 
as the contours of highly diverse characters. Every building, with its vertex corners, 
edges, and faces, provides evidence of the practical use of this ``heavenly'' geometry, \ldots.

On the other hand, the invention of pencils, wax tablets, papyrus or paper enabled
people to capture their thoughts and observations ``on paper'' and to show them to
others. For example, the desire to realise a drawing as a physical building resulted in the 
concept of a plan. A plan is a drawing of an idealised image showing, for example, how 
a stadium shall look from above.

For the construction of a stadium, significant points are staked out in the terrain. The 
current status of the project is shown in the following drawing containing the contours and 
significant points from a plan. 


\begin{center}
%Stadion:
\MTikzAuto{%
\begin{tikzpicture}[line width=1pt]
%Spielfeld:
\coordinate (AS) at (-2.25,0);
\coordinate (BS) at (2.25,0);
\coordinate (FeldA0) at ($(AS) + (0,-1.5)$);
\coordinate (FeldA1) at ($(AS) + (0,1.5)$);
\coordinate (FeldB0) at ($(BS) + (0,-1.5)$);
\coordinate (FeldB1) at ($(BS) + (0,1.5)$);
%Stadionbereich:
\coordinate (GA0) at ($(AS) + (0,-2.5)$);
\coordinate (GA1) at ($(AS) + (0,2.5)$);
\coordinate (GB0) at ($(BS) + (0,-2.5)$);
\coordinate (GB1) at ($(BS) + (0,2.5)$);
%
\coordinate (HA0) at ($(AS) + (0,-3)$);
\coordinate (HA1) at ($(AS) + (0,3)$);
\coordinate (HB0) at ($(BS) + (0,-3)$);
\coordinate (HB1) at ($(BS) + (0,3)$);
%
\coordinate (GA01) at ($(AS) + (0,-2.25)$);
\coordinate (GA02) at ($(AS) + (0,-2)$);
\coordinate (GA03) at ($(AS) + (0,-1.75)$);
\coordinate (GB01) at ($(BS) + (0,-2.25)$);
\coordinate (GB02) at ($(BS) + (0,-2)$);
\coordinate (GB03) at ($(BS) + (0,-1.75)$);
\coordinate (GA11) at ($(AS) + (0,2.25)$);
\coordinate (GA12) at ($(AS) + (0,2)$);
\coordinate (GA13) at ($(AS) + (0,1.75)$);
\coordinate (GB11) at ($(BS) + (0,2.25)$);
\coordinate (GB12) at ($(BS) + (0,2)$);
\coordinate (GB13) at ($(BS) + (0,1.75)$);
%Aussenbereich:
\coordinate (XA) at ($(GA1) + (-2.5, 0) + (-2.5, 0)$);
\coordinate (XC) at ($(GA1) + (-2.5, 0)$);
\coordinate (XB) at ($(AS) + (-2.5,0)$);
\coordinate (XAC1) at ($(XA) + (0.5, 0)$);
\coordinate (XAC2) at ($(XA) + (1, 0)$);
\coordinate (XAC3) at ($(XA) + (1.5, 0)$);
\coordinate (XAC4) at ($(XA) + (2, 0)$);
\coordinate (XBC1) at ($(XB) + (0, 0.5)$);
\coordinate (XBC2) at ($(XB) + (0, 1)$);
\coordinate (XBC3) at ($(XB) + (0, 1.5)$);
\coordinate (XBC4) at ($(XB) + (0, 2)$);
%
\begin{scope}[style=dashed,color=black!50!white]
%Sportfeld:
\draw (FeldA0) -- (FeldB0) -- (FeldB1) -- (FeldA1) -- cycle;
%Stadionbereich:
\draw (GA03) -- (GB03);
\draw (GA02) -- (GB02);
\draw (GA01) -- (GB01);
\draw (GA0) -- (GB0);
\draw (GA13) -- (GB13);
\draw (GA12) -- (GB12);
\draw (GA11) -- (GB11);
\draw (GA1) -- (GB1);
%Aussenbereich:
\draw (XA) -- (XB) -- (XC) -- cycle;
\draw (XAC1) -- (XBC1);
\draw (XAC2) -- (XBC2);
\draw (XAC3) -- (XBC3);
\draw (XAC4) -- (XBC4);
%Links:
\draw (GA13) arc [start angle=90, end angle=270, radius=1.75cm];
\draw (GA12) arc [start angle=90, end angle=270, radius=2cm];
\draw (GA11) arc [start angle=90, end angle=270, radius=2.25cm];
\draw (GA1) arc [start angle=90, end angle=270, radius=2.5cm];
%\draw (-3,3) arc [start angle=90, end angle=270, radius=3cm];
%Rechts
\draw (GB03) arc [start angle=-90, end angle=90, radius=1.75cm];
\draw (GB02) arc [start angle=-90, end angle=90, radius=2cm];
\draw (GB01) arc [start angle=-90, end angle=90, radius=2.25cm];
\draw (GB0) arc [start angle=-90, end angle=90, radius=2.5cm];
\draw (HB0) arc [start angle=-90, end angle=90, radius=3cm];
\end{scope}
%Messpunkte:
%Stadionbereich:
\filldraw (0,0) circle(1.5pt);
\filldraw (AS) circle(1.5pt);
\filldraw (BS) circle(1.5pt);
\filldraw (GA0) circle(1.5pt);
\filldraw (GA1) circle(1.5pt);
\filldraw (GB0) circle(1.5pt);
\filldraw (GB1) circle(1.5pt);
\filldraw (HB0) circle(1.5pt);
\filldraw (HB1) circle(1.5pt);
%Sportfeld:
\filldraw (FeldA0) circle(1.5pt);
\filldraw (FeldA1) circle(1.5pt);
\filldraw (FeldB0) circle(1.5pt);
\filldraw (FeldB1) circle(1.5pt);
%Aussenbereich:
\filldraw (XA) circle(1.5pt);
\filldraw (XB) circle(1.5pt);
\filldraw (XC) circle(1.5pt);
\end{tikzpicture}
}
\par
(Measurement) points and lines from a construction plan of a stadium
\end{center}

The drawing can be considered as an idealised image of reality. Along these lines, 
we will first recapitulate some basic concepts of geometry. Then, applying these concepts, 
we will construct more complicated figures and geometric solids.


\end{MIntro}

\begin{MXContent}{Points and Lines}{Points and Lines}{STD}
\MDeclareSiteUXID{VBKM05_PunkteGeraden_Content}

In geometry, a place or a position in a plane is idealised to the 
most basic object, namely a point. A single point itself cannot be characterised any further. 

For several points, relations between these points can be considered in different ways --- 
and points can be used to define new objects such as line segments and lines (see figure below).
Mathematically, these objects are sets of points.

%Punkte, Strecken und Geraden:
\begin{center}
\MTikzAuto{%
\begin{tikzpicture}[line width=1pt,scale=1];
%\coordinate (A) at (-2,-1);
%\coordinate (B) at (2,1);
%\coordinate (C) at ($(A) + (0.5,0)$);
%\coordinate (D) at ($(B) + (0.5,0)$);
%\coordinate (GA) at ($(A) + (-1,-0.5)$);
%\coordinate (GB) at ($(B) + (1,0.5)$);
%\coordinate (HC) at ($(A) + (-1,-0.5)$);
%\coordinate (HD) at ($(B) + (1,0.5)$);
%Punkte:
\begin{scope}[xshift=-5cm]
\filldraw (-1,-0.5) circle(1pt);
\filldraw (1,0.5) circle(1pt);
\node at (0,-1.5) {points};
\end{scope}
%Strecke:
\begin{scope}[xshift=-2cm]
\draw[color=blue!70!white] (-1,-0.5) -- (1,0.5);
\filldraw (-1,-0.5) circle(1pt);
\filldraw (1,0.5) circle(1pt);
\node at (0,-1.5) {line segment};
\end{scope}
%Gerade:
\begin{scope}[xshift=2cm]
\draw[color=blue!50!white] (-2,-1) -- (2,1);
\draw[color=blue!70!white] (-1,-0.5) -- (1,0.5);
\filldraw (-1,-0.5) circle(1pt);
\filldraw (1,0.5) circle(1pt);
\node at (0,-1.5) {line};
\end{scope}
%Parallele Geraden:
\begin{scope}[xshift=7cm]
\draw[color=blue!50!white] (-2,-1) -- (2,1);
\draw[color=blue!70!white] (-1,-0.5) -- (1,0.5);
\filldraw (-1,-0.5) circle(1pt);
\filldraw (1,0.5) circle(1pt);
\draw[color=green!50!white] (-1.5,-1) -- (2.5,1);
%\draw (-0.5,-0.5) circle(1pt);
%\draw (1.5,0.5) circle(1pt);
\node at (0,-1.5) {two lines};
\end{scope}
\end{tikzpicture}
}
\end{center}

First, we consider a line segment and the distance between points. To do this, 
we need a comparison tool for measuring distance. In mathematics,
this tool is a comparative length called the unit length. For applications, 
appropriate length units such as metres or centimetres are chosen, depending 
on the task in hand.

\begin{MXInfo}{Line Segments and Distances}
Given two points $A$ and $B$, the \MEntry{line segment}{line segment} $\MGeoStrecke{A}{B}$
between $A$ and $B$ is the shortest path between the two points $A$ and $B$.

%Strecke:
\begin{center}
\MTikzAuto{%
\begin{tikzpicture}[line width=1pt,scale=1];
\begin{scope}[xshift=-3cm]
\draw[color=blue] (-1,-0.5) -- (1,0.5);
\filldraw (-1,-0.5) circle(1pt);
\node[below] at (-1,-0.5) {$A$};
\filldraw (1,0.5) circle(1pt);
\node[below] at (1,0.5) {$B$};
\end{scope}
\end{tikzpicture}
}
\end{center}

The length of the line segment $\MGeoStrecke{A}{B}$ is denoted by $\MGeoAbstand{A}{B}$. 
The \MEntry{line length}{line length} equals the distance between the two points $A$ and $B$.
\end{MXInfo}

A ray of light emitted by a distant star or by the sun is an appropriate notion of a 
\MEntry{ray}{ray} starting at the initial point $A$ and proceeding through a second point $B$ indefinitely. A
ray is also called \MEntry{half-line}{half-line}.

%Strahlen und Geraden:
\begin{center}
\MTikzAuto{%
\begin{tikzpicture}[line width=1pt,scale=1];
%\coordinate (A) at (-1,-0.5);
%\coordinate (B) at (1,0.5);
%\coordinate (GA) at ($(A) + (-1,-0.5)$);
%\coordinate (GB) at ($(B) + (1,0.5)$);
%Strahl:
\begin{scope}[xshift=-3cm]
\draw (-1,-0.5) -- (2,1);
\filldraw (-1,-0.5) circle(1pt);
\node[below] at (-1,-0.5) {$A$};
\filldraw (1,0.5) circle(1pt);
\node[below] at (1,0.5) {$B$};
\node at (0,-1.5) {ray};
\end{scope}
%Gerade:
\begin{scope}[xshift=3cm]
\draw (-2,-1) -- (2,1);
\filldraw (-1,-0.5) circle(1pt);
\node[below] at (-1,-0.5) {$A$};
\filldraw (1,0.5) circle(1pt);
\node[below] at (1,0.5) {$B$};
\node at (0,-1.5) {line};
\end{scope}
\end{tikzpicture}
}
\end{center}

Continuing the path of a line segment $\MGeoStrecke{A}{B}$ on both ends indefinitely results in 
a line.

\begin{MXInfo}{Line}
Let $A$ and $B$ be two points (i.e. point $A$ is different from point $B$). Then, $A$ and $B$ define
exactly one \MEntry{line}{line} $\MGeoGerade{A}{B}$.
\end{MXInfo}

Considering, beside $A$ and $B$, an additional point $P$, we can ask for the distance $d$ of the point 
$P$ from the line $\MGeoGerade{A}{B}$, which is defined as the shortest path between $P$ and one of 
the points of the line $\MGeoGerade{A}{B}$.

%Abstand zu einer Geraden:
\begin{center}
\MTikzAuto{%
\begin{tikzpicture}[line width=1pt,scale=1];
\coordinate (A) at (0,0);
\coordinate (B) at (2,1);
\coordinate (P) at ($(A) + (-0.5,1)$);
%Abstand von einer Geraden:
\begin{scope}[xshift=3cm]
\draw ($ (A)!-0.4!(B) $) -- ($ (A)!1.6!(B) $);
\draw[style=dotted] (A) -- node[above right]{$d$} (P);
\filldraw (P) circle(1pt);
%Beschriftung:
\node[above left] at (P) {$P$};
\filldraw (A) circle(1pt);
\node[below right] at (B) {line};
\end{scope}
\end{tikzpicture}
}
\end{center}

Given three points $P$, $Q$, and $S$ in the plane, the lines $\MGeoGerade{S}{P}$ and 
$\MGeoGerade{S}{Q}$ can be defined.

The two lines have the point $S$ in common. If the point $Q$ is also on the line $\MGeoGerade{S}{P}$,
then $\MGeoGerade{S}{Q}$ and $\MGeoGerade{S}{P}$ denote one and the same line. If the 
point $Q$ does not belong to the line $\MGeoGerade{S}{P}$, the line $\MGeoGerade{S}{Q}$ is
different from the line $\MGeoGerade{S}{P}$. Then, the two lines have only the point $S$ in common.
The point $S$ is called \MEntry{intersection point}{intersection point}. 

If any two lines $g$ and $h$ do not have any points in common, the smallest 
distance between points on $g$ and $h$, respectively, is called the distance between 
the lines $g$ and $h$. Hence, $g$ and $h$ do not have any point in common if they have 
a distance larger than $0$. Two lines are called \MEntry{parallel}{parallel} if every point 
on one of the two lines has the same distance from the other line.

A single line can be described by the distance of two points $M$ and $M'$ as well: The set 
of all points with the same distance from two points $M$ and $M'$ is a line.

In geometry, it is a typical approach to define new objects by means of certain properties such 
as the distance. In this way, a circle can also be described very easily.


\begin{MXInfo}{Circle}
Let a point $M$ and a positive real number $r$ be given.

\par
\begin{tabular}{ll}
\begin{minipage}[c]{6cm}
Then, the set of all points at distance $r$ from point $M$ is 
a \MEntry{circle}{circle} around $M$ with \MEntry{radius}{radius} $r$.

\end{minipage}
&
\begin{minipage}[c]{6cm}
\begin{center}
\MTikzAuto{%
\begin{tikzpicture}[line width=1pt]
\filldraw (0,0) circle(1pt);
\node[below] (0,0) {$M$};
\draw[style=dotted] (0,0) -- node[above] {$r$} (1.5,0);
\draw[color=blue] (0,0) circle(1.5cm);
\end{tikzpicture}
}
\end{center}
\end{minipage}
\end{tabular}
\end{MXInfo}
\end{MXContent}

%content: Strahlensatz.
\begin{MXContent}{Intercept Theorems}{Intercept Theorems}{STD}
\MDeclareSiteUXID{VBKM05_Strahlensaetze_Content}

A pinhole camera provides a small image of the outside space. The ratio 
of the size of the image $B$ to the size of the object $G$ equals 
the ratio of the distance $b$ from the pinhole $L$ to the distance $g$
from $L$:
\[
\frac{B}{G} = \frac{b}{g} \MDFPeriod %%
\]

%Lochkamera:
\begin{center}
\MTikzAuto{%
\begin{tikzpicture}[line width=1pt]
\coordinate (L) at (0,0);
\coordinate (B0) at (-3,0);
\coordinate (G0) at (9,0);
\coordinate (B1) at ($(B0) + (0,-1)$);
\coordinate (G1) at ($(G0) + (0,3)$);
%\fill[color=black!50!white] (0,0) -- (3,0) -- (3,1) -- cycle;
%\fill[color=black!50!white] (0,0) -- (-3,0) -- (-3,-1) -- cycle;
%\fill[color=black!30!white] (3,0) -- (6,0) -- (6,2) -- (3,1) -- cycle;
%\fill[color=black!15!white] (6,0) -- (9,0) -- (9,3) -- (6,2) -- cycle;
\draw[style=dotted] (B0) -- node[left] {$B$} (B1);
\draw[style=dashed] (G0) -- node[right] {$G$} (G1);
\draw (B0) -- node[above] {$b$} (L) -- node[below] {$g$} (G0);
\draw (B1) -- (G1);
\node[below] at (L) {$L$};
%\node[below] at (G0) {$G_0$};
%\node[above] at (B0) {$B_0$};
%\node[above] at (G1) {$G1$};
%\node[below] at (B1) {$B_1$};
\end{tikzpicture}
}
\end{center}

Properties of images arising from uniform scaling can also 
be described by means of the intercept theorems (see also 
Figure~\MRef{Mathematik_ElementareGeometrie_zentrischeStreckung}).

What all examples applying the intercept theorems have in common is that rays (or lines)
with an intersection point are intersected by parallel lines.

\begin{MXInfo}{Intercept Theorems}%
\MLabel{VBKM05_Satz_Strahlensatz}%

Let $S$ be the common emanating point of the two rays $s_1$ and $s_2$ proceeding through the points $A$ 
and $C$, respectively. The point $B$ is on the ray $s_1$ and the point $D$ is on the ray $s_2$.
First, we consider the line segments between the points on the two rays and then the line segments between 
the rays. 

\par
\begin{tabular}{@{}lr@{}}
\begin{minipage}[b]{7cm}
%Auf dem Strahl $s_1$ werden die Strecken $\MGeoStrecke{S}{A}$ und 
%$\MGeoStrecke{S}{B}$ sowie $\MGeoStrecke{A}{B}$ betrachtet.
For two points $P$ and $Q$, $\MGeoStrecke{P}{Q}$ is the line segment 
from $P$ to $Q$ and $\MGeoAbstand{P}{Q}$ denotes the length of this line segment. 

\vspace*{2cm}
\end{minipage}
&
\MTikzAuto{%
\begin{tikzpicture}
\coordinate (S) at (0,0);
\coordinate (A) at ($ (S) + (3,0.5) $);
\coordinate (C) at ($ (S) + (4,2.5) $);
\coordinate (B) at ($ (S)!1.7!(A) $);
\coordinate (D) at ($ (S)!1.7!(C) $);
%
\path (S) node[left]{$S$} (A) node[below right]{$A$} (B) node[below right]{$B$}
                          (D) node[above left] {$D$} (C) node[above left] {$C$};
%
\draw (S) -- ($ (S)!1.1!(B) $);
\draw (S) -- ($ (S)!1.1!(D) $);
%
\draw ($ (A)!-0.2!(C) $) -- ($ (C)!-1!(A) $) node[left]{$g$};
\draw ($ (B)!-0.2!(D) $) node[right]{$h$} -- ($ (D)!-0.1!(B) $);
\end{tikzpicture}
}
\end{tabular}
\par
If the lines $g$ and $h$ are parallel, the following statements hold:

\begin{itemize}
\item
The ratio of the line segments on one of the two rays equals the 
corresponding ratio of the line segments on the other:
\[
   \frac{\MGeoAbstand{S}{A}}{\MGeoAbstand{S}{C}}
 = \frac{\MGeoAbstand{A}{B}}{\MGeoAbstand{C}{D}}
 = \frac{\MGeoAbstand{S}{B}}{\MGeoAbstand{S}{D}} \MDFPeriod
\]
This can also be expressed in the form:
\[
   \frac{\MGeoAbstand{S}{A}}{\MGeoAbstand{A}{B}}
 = \frac{\MGeoAbstand{S}{C}}{\MGeoAbstand{C}{D}}
\quad\text{and}\quad
   \frac{\MGeoAbstand{S}{A}}{\MGeoAbstand{S}{B}}
 = \frac{\MGeoAbstand{S}{C}}{\MGeoAbstand{S}{D}} \MDFPeriod
\]
\item
The ratio of the line segments on the parallel lines equals the ratio of 
the corresponding line segments emanating from $S$ on a single ray
\[
   \frac{\MGeoAbstand{S}{A}}{\MGeoAbstand{S}{B}}
 = \frac{\MGeoAbstand{A}{C}}{\MGeoAbstand{B}{D}}
 = \frac{\MGeoAbstand{S}{C}}{\MGeoAbstand{S}{D}} \MDFPeriod
\]
This can also be expressed in the form:
\[
   \frac{\MGeoAbstand{S}{A}}{\MGeoAbstand{A}{C}}
 = \frac{\MGeoAbstand{S}{B}}{\MGeoAbstand{B}{D}}
\quad\text{and}\quad
   \frac{\MGeoAbstand{S}{C}}{\MGeoAbstand{C}{A}}
 = \frac{\MGeoAbstand{S}{D}}{\MGeoAbstand{D}{B}} \MDFPSpace,
\]
where $\MGeoAbstand{A}{C} = \MGeoAbstand{C}{A}$
and $\MGeoAbstand{B}{D} = \MGeoAbstand{D}{B}$.
\end{itemize}
\end{MXInfo}

The statements of the intercept theorems also hold if two lines intersecting in a point $S$ are 
considered instead of the two rays. 
An application example of this case is the pinhole camera mentioned above.

In this way, distances between points can be calculated without 
measuring the length of the line segments directly.

\begin{MExample}
Let four points $A$, $B$, $C$, and $D$ be given. These points define the
two lines $\MGeoGerade{A}{B}$ and $\MGeoGerade{C}{D}$ intersecting at the point $S$. 
Furthermore, it is known that the lines $\MGeoGerade{A}{C}$ and $\MGeoGerade{B}{D}$ 
are parallel. Between the points the following distances were measured: 
$\MGeoAbstand{A}{B} = 51$, $\MGeoAbstand{S}{C} = 12 $, and 
$\MGeoAbstand{C}{D} = 18$.

\begin{center}
\MTikzAuto{%
\begin{tikzpicture}
\coordinate (S) at (0,0);
\coordinate (A) at (20:3.4);
\coordinate (B) at (20:8.5);
\coordinate (C) at (60:1.2);
\coordinate (D) at (60:3.0);
%
\path %
 (S) node[below right]{$S$} (A) node[below right]{$A$} %
     (20:1.7) node[below right]{$x$} (B) node[below right]{$B$} %
          (D) node[above left] {$D$} (C) node[above left] {$C$};
%
\draw ($ (S)!-0.1!(B) $) -- ($ (S)!1.1!(B) $);
\draw ($ (S)!-0.1!(D) $) -- ($ (S)!1.1!(D) $);
%
\draw ($ (A)!-0.2!(C) $) -- ($ (C)!-0.1!(A) $);
\draw ($ (B)!-0.2!(D) $) -- ($ (D)!-0.1!(B) $);
\end{tikzpicture}
}
\end{center}

>From this, the distance between $A$ and $S$ can be calculated. Let $x$ denote the 
required distance. Then, according to the intercept theorems, we have
\[
   \frac{x}{\MGeoAbstand{A}{B}}
 = \frac{\MGeoAbstand{S}{C}}{\MGeoAbstand{C}{D}} \MDFPSpace,
\]
from which
\[
x = \frac{\MGeoAbstand{S}{C}}{\MGeoAbstand{C}{D}} \cdot \MGeoAbstand{A}{B} %
 = \frac{12}{18} \cdot 51 = \frac{2}{3} \cdot 51 = 34 %%
\]
follows.
\end{MExample}
\end{MXContent}


\begin{MExercises}
\MDeclareSiteUXID{VBKM05_Strahlensaetze_Exercises}
\begin{MExercise}
The son of the house is looking at the tree on the neighbouring property. He observes 
that the tree is completely covered by the hedge separating the two properties only if
he stands close enough to the hedge. Now he is looking for the point at which 
he \textit{just} cannot see the tree any more. 

The boy is $\MZahl{1}{40}$~metres tall. If the boy stands
$\MZahl{2}{50}$~metres away from the hedge, which is $\MZahl{2}{40}$~metres 
high, $1$~metre wide and clipped into a pointed shape at the top, the tree 
disappears from his sight.

What is the height of the tree if the middle of the trunk is $\MZahl{14}{5}$~metres away from 
the hedge?

Please carry out the calculation using variables and insert the values only at the end!

Result: \MLParsedQuestion{10}{7.4}{1}{GEO7} $\MEinheit{m}$.

\begin{MHint}{Hint}
Take the width of the hedge into account!
\end{MHint}

\begin{MHint}{Solution}
The line segments are denoted as shown in the figure below.

\begin{center}
\MTikzAuto{%
\begin{tikzpicture}[x=0.75cm, y=0.75cm]
\coordinate (KF) at (0,0);
\coordinate (KK) at (0,1.4);
\coordinate (HF) at (3,0);
\coordinate (HK) at (3,2.4);
\coordinate (BF) at (16.5,0);
\coordinate (BK) at (16.5,7.4);
\coordinate (BM) at (16.5,1.4);
%
\begin{scope}[color=black!50, every node/.style={color=black}]
\draw (KF) -- ++ (0,-0.3) ++(0,0.15) -- node[below]{$\MZahl{2}{5}\MEinheit{m}$} ($ (HF) + (-0.5,-0.15) $) ++(0,-0.15) -- ++(0,0.3);
\draw ($ (HF) + (0.5,0) $) -- ++ (0,-0.3) ++(0,0.15) -- node[below]{$\MZahl{14}{5}\MEinheit{m}$} ($ (BF) + (0,-0.15) $) ++(0,-0.15) -- ++(0,0.3);
 \node at (HF) [below]{$1\MEinheit{m}$};
 \draw ($ (KF)!0.5!(BF) $) -- node[right]{$h_{\mathrm{K}} = \MZahl{1}{4}\MEinheit{m}$} ($ (KK)!0.5!(BM) $) ++ (-0.15,0) -- ++(0.3,0);
 \draw (BF) -- node[right]{$h_{\mathrm{B}}$} (BK) ++ (-0.15,0) -- ++(0.3,0);
 \draw ($ (HF) + (0.75,0) $) -- node[below right]{$h_{\mathrm{H}} = \MZahl{2}{4}\MEinheit{m}$} ($ (HK) + (0.75,0) $) ++ (-0.15,0) -- ++(0.3,0);
 \draw (KK) -- ++ (0,-0.3) ++(0,0.15) -- node[below]{$d_{\mathrm{KH}}$} ($ (HF) + (KK) - (KF) + (0,-0.15) $) ++(0,-0.15) -- ++(0,0.3);
 \draw (KK) -- ++ (0,0.3) ++(0,-0.15) -- node[above]{$d_{\mathrm{KB}}$} ($ (BM) + (0,0.15) $) ++(0,0.15) -- ++(0,-0.3);
\end{scope}
\draw (KF) ++ (-0.2,0) -- ++ (0.2,0.5) -- ++ (0.2,-0.5)
      (KF) ++ (0,0.5) -- ++ (0,0.5)
      (KK) ++ (0,-0.6) ++ (-0.2,-0.4) -- ++ (0.2,0.4) -- ++ (0.2,-0.4)
      (KK) ++ (0,-0.2) circle (0.2);
\draw (HK) decorate[decoration={random steps,segment length=2pt,amplitude=1pt}] {.. controls ($ (HF) + (-0.5,1.5) $) .. ($ (HF) + (-0.5,0) $)};
\draw (HK) decorate[decoration={random steps,segment length=2pt,amplitude=1pt}] {.. controls ($ (HF) + (0.5,1.5) $) .. ($ (HF) + (0.5,0) $)};
\draw (BK) decorate[decoration={saw,mirror}] {-- ($ (BF) + (-2.9,1) $) -- ++ (2.7,0)} -- ($ (BF) + (-0.3,0) $);
\draw (BK) decorate[decoration={saw}] {-- ($ (BF) + (2.9,1) $) -- ++ (-2.7,0)} -- ($ (BF) + (0.3,0) $);
\draw (KK) -- (BK) ($ (KF)!-0.05!(BF) $) -- ($ (BF)!-0.05!(KF) $);
\draw[dashed] (KK) -- (BM);
\end{tikzpicture}
}
\end{center}

Applying the second intercept theorem results in
{\glqq}$\frac{\text{full}}{\text{at front}} %
 = \frac{\text{long}}{\text{short}}${\grqq}
:
\[
\frac{d_{\mathrm{KB}}}{d_{\mathrm{KH}}}
= \frac{h_{\mathrm{B}} - h_{\mathrm{K}}}{h_{\mathrm{H}} - h_{\mathrm{K}}}
 \qquad \text{or} \qquad
 h_{\mathrm{B}} = \left( h_{\mathrm{H}} - h_{\mathrm{K}} \right) \cdot \frac{d_{\mathrm{KB}}}{d_{\mathrm{KH}}} + h_{\mathrm{K}} \MDFPeriod
\]

The values are $d_{\mathrm{KH}} = \MZahl{2}{5}\MEinheit{m} + \frac{1\MEinheit{m}}{2} = 3\MEinheit{m}$
and $d_{\mathrm{KB}} = \MZahl{2}{5}\MEinheit{m} + 1\MEinheit{m} + \MZahl{14}{5}\MEinheit{m} = 18\MEinheit{m}$.
Hence, it follows that
\[
  h_{\mathrm{B}}
   = \left( \MZahl{2}{4}\MEinheit{m} - \MZahl{1}{4}\MEinheit{m} \right) \cdot \frac{18\MEinheit{m}}{3\MEinheit{m}} + \MZahl{1}{4}\MEinheit{m}
   = 1\MEinheit{m} \cdot 6 + \MZahl{1}{4}\MEinheit{m}
   = \MZahl{7}{4}\MEinheit{m} \MDFPeriod
\]
\end{MHint}
\end{MExercise}

\end{MExercises}

%jgl: end of section 1.


%content: section 2: Winkel und Winkelmessung.

\MSubsection{Angles and Angle Measurement}
\MLabel{M05_Winkel}

\begin{MIntro}
\MDeclareSiteUXID{VBKM05_WinkelIntro}

Lines intersecting at a point $S$ divide the plane in a characteristic way. 
To describe this observation the concept of an angle is introduced. 
The question of how to measure angles can be answered in different ways, 
which in the end are all based on the subdivision of circles.

In this module, degree measure and radian measure are described. 


\begin{center}
\MTikzAuto{%
\begin{tikzpicture}[line width=1pt]
\coordinate (S) at (0,0);
\coordinate (A) at (6,0);
\coordinate (B) at (6,3);
\coordinate (C) at (-2,0);
\coordinate (D) at (-2,-1);
%
\fill[color=blue!50!white] (S) -- (A) -- (B) -- (S);
\fill[color=blue!25!white] (S) -- (C) -- (D) -- (S);
\fill[color=green!50!white] (S) -- (B) -- (C) -- (S);
\fill[color=green!25!white] (S) -- (D) -- (A) -- (S);
\draw (C) -- node[below]{$g$} (A);
\draw (D) -- node[above]{$h$} (B);
\node[below] at (S) {$S$};
\end{tikzpicture}
}
\par
Every coloured region represents one of the angles defined by the lines
$g$ and $h$.
\end{center}
\end{MIntro}


\begin{MXContent}{Angles}{Angles}{STD}
\MDeclareSiteUXID{VBKM05_Winkel_Content}
\MLabel{VBKM05_Winkel}
Two rays (half-lines) $g$ and $h$ in the plane emanating from the common
point $S$ enclose an \MEntry{angle}{angle} $\Mmeasuredangle\left(g, h\right)$.

\begin{center}
\MTikzAuto{%
\begin{tikzpicture}[line width=1pt]
\coordinate (S) at (0,0);
\coordinate (A) at (6,0);
\coordinate (B) at (6,3);
\coordinate (C) at (-2,0);
\coordinate (D) at (-2,-1);
%
\fill[color=blue!50!white] (S) -- (A) -- (B) -- (S);
\draw (S) -- node[below]{$g$} (A);
\draw (S) -- node[above]{$h$} (B);
\node[left] at (S) {$S$};
\node at (2,0.4) {$\Mmeasuredangle(g,h)$};
\end{tikzpicture}
}
\par
Angle enclosed by the rays $g$ and $h$.
\end{center}

For the notation of the angle $\Mmeasuredangle \left( g, h \right)$, 
the order of $g$ and $h$ is relevant. $\Mmeasuredangle \left( g, h \right)$
denotes the angle shown in the figure above. It is defined by
turning the half-line $g$ counter-clockwise to the half-line $h$.

In contrast, $\Mmeasuredangle \left( h, g \right)$ denotes the angle from 
$h$ to $g$ as illustrated by the figure below.


\begin{center}
\MTikzAuto{%
\begin{tikzpicture}[line width=1pt,scale=0.5]
\coordinate (S) at (0,0);
\coordinate (A) at (6,0);
\coordinate (B) at (6,3);
\coordinate (CC) at (-4,1);
\coordinate (DD) at (-4,-2);
%
\fill[color=red!50!white] (S) -- (B) -- (CC) -- (DD) -- (A) -- (S);
\draw (S) -- node[below]{$g$} (A);
\draw (S) -- node[above]{$h$} (B);
\node[left] at (S) {$S$};
\node at (-2,-0.8) {$\Mmeasuredangle(h,g)$};
\end{tikzpicture}
}
\par
Angle enclosed by the rays $h$ and $g$
\end{center}

The point $S$ is called \MEntry{vertex}{vertex (angle)} of the angle, and 
the two half-lines enclosing the angle are called \MEntry{arms}{arms} of the angle. 
If $A$ is a point on the line $g$ and $B$ is a point on the line $h$, then 
the angle $\Mmeasuredangle \left( g, h \right)$ can also be denoted by 
$\Mmeasuredangle \left( A S B \right)$. In this way, angles between line segments 
$\MGeoStrecke{S}{A}$ and $\MGeoStrecke{S}{B}$ are described.

Angles are often denoted by lower-case Greek letters to distinguish them from variables, 
which are generally denoted by lower-case Latin letters (see Table~\MRef{VBKM01_Griechisch} 
in module~\MNRef{VBKM01}). Further angles can be found by considering angles formed by intersecting lines. 

\begin{MXInfo}{Vertical Angles and Supplementary Angles}%
\MLabel{VBKM05_Scheitelwinkel_Nebenwinkel}%


Let $g$ and $h$ be two lines intersecting in a point $S$.

\begin{center}
\MTikzAuto{%
\begin{tikzpicture}[line width=1pt,scale=0.5]
%\coordinate[label=below:{$P$}] (P) at (4,0);
\coordinate (S) at (0,0);
\coordinate (A) at (6,0);
\coordinate (B) at (6,3);
\coordinate (C) at (-6,0);
\coordinate (D) at (-6,-3);
%
\fill[color=blue!50!white] (S) -- (A) -- (B) -- (S);
\fill[color=blue!25!white] (S) -- (C) -- (D) -- (S);
\fill[color=green!50!white] (S) -- (B) -- (C) -- (S);
\fill[color=green!25!white] (S) -- (D) -- (A) -- (S);
\draw (C) -- (A) node[below left] {$g$};
\draw (D) --  (B) node[above right] {$h$};
%\node[below] at (A) {$g$};
%\node[above] at (B) {$h$};
\node[below] at (S) {$S$};
\node at (3,0.6) {$\Mvarphi$};
\node at (-3,-0.6) {$\Mvarphi'$};
\node at (-1.5,0.6) {$\psi$};
\node at (1.5,-0.6) {$\psi'$};
\end{tikzpicture}
}
\end{center}

\begin{itemize}
\item The angles $\Mvarphi$ and $\Mvarphi'$ are called
 \MEntry{vertical angles}{vertical angle}.
\item The angles $\Mvarphi$ and $\psi$ are called 
 \MEntry{supplementary angles}{supplementary} with respect to $g$.
\end{itemize}
\end{MXInfo}

The figure above contains further vertical and supplementary angles.


\begin{MExercise}
Find all vertical and supplementary angles occurring in the figure above.


\begin{MHint}{Solution}
In addition to $\Mvarphi$ and $\Mvarphi'$, $\psi$ and $\psi'$ are also vertical angles. 
Beside the angles $\Mvarphi$ and $\psi$, the angles $\Mvarphi'$ and $\psi'$ are also
supplementary angles of $g$. Moreover, $\psi$ and $\Mvarphi'$ as well as $\psi'$ and 
$\Mvarphi$ are supplementary angles.
\end{MHint}
\end{MExercise}

Some special angles have their own dedicated name. For example, the angle bisector $w$
is the half-line whose points have the same distance from the two given half-lines $g$ 
and $h$. Then, it can be said that $w$ bisects the angle between $g$ and $h$.

\begin{MXInfo}{Names of Special Angles}
Let $g$ and $h$ be half-lines with the intersection point $S$.

\begin{itemize}
\item
The angle covering the entire plane is called \MEntry{complete angle}{complete angle (angle)}.

\item
If the rays $g$ and $h$ form a line, the angle between $g$ and $h$ is called
%$\Mmeasuredangle(g,h)$ gestreckter Winkel.
straight angle.
\item
The angle between two half-lines bisecting a straight angle is called 
 \MEntry{right angle}{right angle (angle)}. 
One also says that $g$ and $h$ \MEntry{are perpendicular (or orthogonal) to each other}{perpendicular (line)}.
\end{itemize}
\end{MXInfo}

Next, three lines are considered. Two of the three lines are parallel, while the 
third line is not parallel to the others. It is called a transversal. These lines form
eight cutting angles. Four of the eight angles are equal.

\begin{MXInfo}{Angles at Parallel Lines}%
\MLabel{Mathematik_ElementareGeometrie_StufenwinkelWechselwinkel}%
Let two parallel lines $g$ and $h$  be given cut by another transversal line $j$.


\begin{center}
\MTikzAuto{%
\begin{tikzpicture}
\coordinate (S) at (0,0);
\coordinate (P) at (4,1);
\coordinate (T) at ($ (S) + (1,2) $);
\coordinate (Q) at ($ (P) + (1,2) $);
%
%Winkel alpha:
\draw[color=red!50!white] ({atan(2)}:{sqrt{17}/4}) arc({atan(2)}:{180+atan(0.25)}:{sqrt(17)/4});
%\filldraw[color=red!50!white] (S) -- (0.5,1) -- (-1,-0.25) -- cycle;
%Winkel beta:
%\draw[color=blue!50!white] (1,0.25) arc({atan(0.25)}:{atan(2)}:{sqrt(17)/4});
%\filldraw[color=blue!50!white] (S) -- (1,0.25) -- (0.5,1) -- cycle;
%
%Winkel alpha':
%\filldraw[color=red!50!white] (T) -- ($ (T) + (0.5,1) $) -- ($ (T) + (-1,-0.25) $)-- cycle;
\draw[color=red!50!white] ($ (T) + ({atan(2)}:{sqrt{17}/4}) $) arc({atan(2)}:{180+atan(0.25)}:{sqrt(17)/4});
%Winkel beta':
%\filldraw[color=blue!50!white] (T) -- ($ (T) +(-1,-0.25) $) -- ($ (T) + (-0.5,-1) $) -- cycle;
%\draw[color=blue!50!white] ($ (T) + (-1,-0.25) $) arc({180+atan(0.25)}:{180+atan(2)}:{sqrt(17)/4});
%
\draw ($ (S) + (0,0) $) node[above left,color=red]{$\alpha$};
\draw ($ (T) + (0,0) $) node[above left,color=red]{$\alpha'$};
%
%\draw ($ (S) + (0.2,0.05) $) node[above right,color=blue]{$\beta$};
%\draw ($ (T) + (-0.2,-0.1) $) node[below left,color=blue]{$\beta'$};
%
\draw ($ (S)!-0.7!(P) $) node[below right] {$g$} -- ($ (S)!0.7!(P) $);
\draw ($ (T)!-0.7!(Q) $) node[below right] {$h$} -- ($ (T)!0.7!(Q) $);
%
\draw ($ (S)!-0.4!(T) $) -- ($ (S)!1.6!(T) $) node[right]{$j$};
\end{tikzpicture}
}
%
\MTikzAuto{%
\begin{tikzpicture}
\coordinate (S) at (0,0);
\coordinate (P) at (4,1);
\coordinate (T) at ($ (S) + (1,2) $);
\coordinate (Q) at ($ (P) + (1,2) $);
%
%Winkel alpha:
%\draw[color=red!50!white] ({atan(2)}:{sqrt{17}/4}) arc({atan(2)}:{180+atan(0.25)}:{sqrt(17)/4});
%\filldraw[color=red!50!white] (S) -- (0.5,1) -- (-1,-0.25) -- cycle;
%Winkel beta:
\draw[color=blue!50!white] (1,0.25) arc({atan(0.25)}:{atan(2)}:{sqrt(17)/4});
%\filldraw[color=blue!50!white] (S) -- (1,0.25) -- (0.5,1) -- cycle;
%
%Winkel alpha':
%\filldraw[color=red!50!white] (T) -- ($ (T) + (0.5,1) $) -- ($ (T) + (-1,-0.25) $)-- cycle;
%\draw[color=red!50!white] ($ (T) + ({atan(2)}:{sqrt{17}/4}) $) arc({atan(2)}:{180+atan(0.25)}:{sqrt(17)/4});
%Winkel beta':
\draw[color=blue!50!white] ($ (T) + (-1,-0.25) $) arc({180+atan(0.25)}:{180+atan(2)}:{sqrt(17)/4});
%\filldraw[color=blue!50!white] (T) -- ($ (T) +(-1,-0.25) $) -- ($ (T) + (-0.5,-1) $) -- cycle;
%
%\draw ($ (S) + (0,0) $) node[above left,color=red]{$\alpha$};
%\draw ($ (T) + (0,0) $) node[above left,color=red]{$\alpha'$};
%
\draw ($ (S) + (0.2,0.05) $) node[above right,color=blue]{$\beta$};
\draw ($ (T) + (-0.2,-0.1) $) node[below left,color=blue]{$\beta'$};
%
\draw ($ (S)!-0.7!(P) $) node[below right] {$g$} -- ($ (S)!0.7!(P) $);
\draw ($ (T)!-0.7!(Q) $) node[below right] {$h$} -- ($ (T)!0.7!(Q) $);
%
\draw ($ (S)!-0.4!(T) $) -- ($ (S)!1.6!(T) $) node[right]{$j$};
\end{tikzpicture}
}
\end{center}

\begin{itemize}
\item Then the angle $\alpha'$ is called a \MEntry{corresponding angle}{corresponding angle} 
of $\alpha$ and 

\item the angle $\beta'$ is called an \MEntry{alternate angle}{alternate angle} of $\beta$.
\end{itemize}
Since the lines $g$ and $h$ are parallel, the angles $\alpha$ and $\alpha'$ are equal. Likewise, 
the angles $\beta$ and $\beta'$ are equal.
\end{MXInfo}

\begin{MExercise}
The figure shows two parallel lines $g$ and $h$ cut by another line $j$. 
Explain which angles are equal and which angles are corresponding angles or alternate 
angles to each other, respectively.

\begin{center}
\MTikzAuto{%
\begin{tikzpicture}
\coordinate (G) at (0,0);
\coordinate (H) at ($ (G) + (30:5) $);
\coordinate (A) at ($ (G) + (30:1.5) $);
\coordinate (B) at ($ (H) + (A) - (G) $);
\coordinate (C) at ($ (G) + (0,2) $);
\coordinate (D) at ($ (C) + (B) - (A) $);
\coordinate (E) at (0,5);
\coordinate (F) at (5,1);
\coordinate (S) at (intersection of A--B and E--F);
\coordinate (T) at (intersection of C--D and E--F);
%
\draw[dotted] (S) circle [radius=0.9] (T) circle [radius=0.9];
\draw (A) -- node[at end, below] {$g$} (B) (C) -- node[at end, above left] {$h$} (D) (F) -- node[at end, below] {$j$} (E);
%
\begin{scope}[outer sep=4pt]
 \node at (S) [right] {$\alpha$};
 \node at (S) [above] {$\beta$};
 \node at (S) [left]  {$\gamma$};
 \node at (S) [below] {$\delta$};
 \node at (T) [right] {$\Mvarepsilon$};
 \node at (T) [above] {$\chi$};
 \node at (T) [left]  {$\Mvarphi$};
 \node at (T) [below] {$\psi$};
\end{scope}
\end{tikzpicture}
}
\end{center}

\begin{MHint}{Solution}
\begin{itemize}
\item The angles $\alpha$, $\gamma$, $\Mvarepsilon$, and $\Mvarphi$ are 
 equal as well as the angles $\beta$, $\delta$, $\chi$, and $\psi$.
\item The angles $\beta$ and $\psi$ as well as the angles $\gamma$ and $\Mvarepsilon$ are alternate angles.
\item The angles $\alpha$ and $\Mvarepsilon$ are corresponding angles, likewise the angles  
 $\beta$ and $\chi$, $\delta$ and $\psi$ as well as $\gamma$ and $\Mvarphi$.
\end{itemize}
\end{MHint}
\end{MExercise}
\end{MXContent}


\begin{MXContent}{Angle Measurement}{Angle Measurement}{STD}
\MDeclareSiteUXID{VBKM05_Winkelmessung_Content}
\MLabel{VBKM05_Winkelmessung}

We already explained the notation $\Mmeasuredangle(g,h)$ for the angle defined by turning 
$g$ counter-clockwise to $h$. This explanation provides an idea of how to measure angles, i.e.
how to compare angles quantitatively.

Think of the face of an analogue watch with its twelve evenly spaced hour marks. Likewise, 
the circumference of a circle can be evenly subdivided. In this way, a certain scale for 
angles is obtained. Depending on the applied scaling, the magnitude of an angle can be specified in 
different units. 

\paragraph{Degree Measure.}

A disk is subdivided into $360$ equal segments. A rotation by one segment defines an
angle of $1$ degree. This is written as $1\MGrad$. The figure below shows angles 
of multiples of $30\MGrad$.

%Winkel im Gradmass:
\begin{center}
\MTikzAuto{%
\begin{tikzpicture}[line width=1pt]
\coordinate (M) at (0,0);
\coordinate (A) at (2,0);
 \draw (M) -- (0:2cm);
\foreach \x in {30, 60, 120, 150, 210, 240, 300, 330}
 \draw[style=dotted] (M) -- ({\x}:2);
%\foreach \x in {0, 90, 180, 270}
% \draw (M) -- ({\x}:2cm);
% \draw[color=black!8!white] (M) -- (0:2cm) node[right,color=black] {$0\MGrad$};
 \draw (M) -- (0:2cm) node[right] {$0\MGrad$};
 \draw (M) -- (90:2cm) node[above] {$90\MGrad$};
 \draw (M) -- (180:2cm) node[left] {$180\MGrad$};
 \draw (M) -- (270:2cm) node[below] {$270\MGrad$};
\node at (30:2.4cm) {$30\MGrad$};
\node at (60:2.4cm) {$60\MGrad$};
\node at (120:2.4cm) {$120\MGrad$};
\node at (150:2.4cm) {$150\MGrad$};
\node at (210:2.4cm) {$210\MGrad$};
\node at (240:2.4cm) {$240\MGrad$};
\node at (300:2.4cm) {$300\MGrad$};
\node at (330:2.4cm) {$330\MGrad$};
%Kreis:
\draw (M) circle(2cm);
\end{tikzpicture}
}
\end{center}

\paragraph{Radian Measure.}

In ancient Babylonia, Egypt, and Greece people had already observed that
the ratio of the circumference $U$ of a circle to its diameter $D$ is always 
the same, and hence circumference and diameter of a circle are proportional to 
each other. This ratio is called $\pi$.


\begin{MXInfo}{The Number $\pi$}\MLabel{Kreiszahl}%
Let a circle with circumference $U$ and diameter $D$ be given. 
Then, the ratio of the circumference $U$ of a circle to its diameter $D$
is
\[
\pi = \frac{U}{D} = \frac{U}{2r}\, ,%%
\]
where $r = \frac{1}{2} D$ is the radius of the circle.

The number $\pi$ is not a rational number. It cannot be expressed as a 
finite or periodic decimal fraction. From numerical calculations we know that
the value of $\pi$ is approximately $\pi \approx \MZahl{3}{141592653589793}$.
\end{MXInfo}

If the circle has a radius of exactly $1$, the circumference is $2\pi$. Now, for the 
\MEntry{radian measure}{radian measure} the circumference of a circle with radius $1$ is subdivided. 
For the radian measure of an angle $\Mmeasuredangle(g,h)$ the length of an \MEntry{arc}{arc} 
``cut'' by this angle is used. As a result, the radian measure assigns to each angle a number 
between $0$ and $2\pi$. In scientific applications, the symbol rad is used to
express explicitly that the angle is measured in radian measure.

\begin{MXInfo}{Radian Measure}\MLabel{VBKM05_Def_Bogenmass}%
Let $g$ and $h$ be two half-lines emanating from the common vertex $S$ and enclosing 
the angle $\Mmeasuredangle(g,h)$. If a circle with radius~$r = 1$ is drawn around $S$, the
two half-lines cut the circle into two pieces. Now, the angle is described by the one arc $x$ 
that transforms $g$ into $h$ by a counter-clockwise rotation (indicated by a green line in the 
figure below). In other words, vertex $S$ is always on the left if one moves on the 
arc $x$ from $g$ towards $h$.

\begin{center}
\MTikzAuto{%
\begin{tikzpicture}[scale=0.9,line width=2pt]
\coordinate[label=below:$S$] (S) at (0,0);
\coordinate[label=below left:$g$] (A) at ($ (S) + (-10:3) $);
\coordinate[label=below left:$h$] (B) at ($ (S) + (100:3) $);
%
%Winkel:
\fill[color=green!30!white] (A) -- (S) -- (B);
%Radius:
\draw[dotted, line width=1.5pt] (S) -- node[above]{$r$} ++ (-1.4,0);
%Kreisboegen:
\draw [color=green!50!black] (S) ++ (-10:1.4) arc (-10:100:1.4);
\draw [color=red] (S) ++ (100:1.4) arc (100:350:1.4);
%Halbgeraden mit Schnittpunkt:
\draw (A) -- (S) -- (B);
\filldraw (0,0) circle(1pt);
%Bezeichnung fuer den Bogen:
\node[right,color=green!50!black] at (1.4,0) {$x$};
\end{tikzpicture}
}
\end{center}

The length of the arc $x$ is the \MEntry{radian measure}{radian measure} of the
angle $\Mmeasuredangle \left( g, h \right)$.
\end{MXInfo}

By means of an angle measure (such as the radian and degree measures
introduced previously), angles can simply be classified into different types and named 
accordingly. 

For repetition and completeness, all names, including ones previously discussed, are listed 
below.

\begin{MXInfo}{Names of Different Types of Angles}
For angles whose radian measure is in a certain range, the following names 
are introduced:

\begin{itemize}
\item
An angle with a radian measure greater than~$0$ and less than $\frac{\pi}{2}$ is called an
\MEntry{acute angle}{angle (acute)}.
       
\item
An angle with a radian measure of exactly~$\frac{\pi}{2}$ is called a \textbf{right angle}.
  
\item
An angle with a radian measure greater than~$\frac{\pi}{2}$ and less than $\pi$ is called 
an \MEntry{obtuse angle}{angle (obtuse)}.
       
\item
An angle with a radian measure greater than~$\pi$ and less than~$2 \pi$ is called
a \MEntry{reflex angle}{angle (reflex)}.
\end{itemize}

Two half-lines are said to be \textbf{perpendicular to each other} if they form a right angle.

Two half-lines form a line if they enclose an angle of radian measure~$\pi$.
\end{MXInfo}

>From the radian measure of the angle $\Mmeasuredangle \left( g, h \right)$, the 
radian measure of the angle $\Mmeasuredangle \left( h, g \right)$  can also be determined. 
>From definition~\MRef{VBKM05_Def_Bogenmass} it is known that
\[
   \Mmeasuredangle \left( h, g \right)
 = 2 \pi - \Mmeasuredangle \left( g, h \right)\,. %%
\]
In the figure of definition~\MRef{VBKM05_Def_Bogenmass} the radian measure 
of the angle $\Mmeasuredangle(h,g)$ is the length of the red arc of the circle 
with radius $r = 1$.


The wording in the last sentences might seem awkward. The reason for that lies probably in the fact
that we do distinguish precisely between an angle and its measure, e.g. the radian measure in this case.

When it comes to calculating a required value for line segments, the same notation is often
used for a segment and its length. Mostly this is clear, and it helps to describe or to illustrate 
a problem efficiently. Importantly, the unit of the angle has to be known or 
explicitly specified. Often, such an agreement -- a so called convention -- is also used if it is 
known from the context that a certain angle has to be calculated using a certain angle measure. 

\begin{MXInfo}{Convention}\MLabel{VBKM05_Konvention_Winkel}%
If a calculation does not depend on a certain measure or the unit of the angles
is specified in advance, the term angle is used for short denoting both the angle itself and its value in 
the specified measure.


\end{MXInfo}
Hence, for example, we can write $\Mmeasuredangle(g,h) = 90\MGrad$ and speak about the 
right angle $\Mmeasuredangle(g,h)$ enclosed by the lines $g$ and $h$ at the same time.
The same idea applies for the radian measure.

The value of an angle can be converted from radian measure to degree measure (and vice versa) by
considering the ratios of its value to the value of the complete angle in the 
respective angle measure. 

The conversion from radian measure to degree measure is described below.

\begin{MXInfo}{Relation between Radian Measure and Degree Measure}
Let $g$ and $h$ be two half-lines enclosing the angle $\Mmeasuredangle(g,h)$.
The radian measure of the angle is denoted by $x$ and the degree measure of 
the angle is denoted by $\alpha$.

Then, the ratio of $x$ to $2\pi$ equals the ratio of $\alpha$ to $360\MGrad$, and thus:
\[
   \frac{x}{2\pi} = \frac{\alpha}{360\MGrad} \MDFPeriod
\]
Hence,
\[
x = \frac{\pi}{180\MGrad} \cdot \alpha %
\quad\text{and}\quad
\alpha = \frac{180\MGrad}{\pi} \cdot x \MDFPeriod
\]
\end{MXInfo}

Therefore, the values in radian measure are proportional to the ones in degree measure.
Thus, the conversion using the respective proportionality factors $\frac{\pi}{180\MGrad}$ and 
$\frac{180\MGrad}{\pi}$ is very simple.


\begin{MExercise}
The angle $\Mmeasuredangle\left(g, h\right)$ equals $60\MGrad$ in degree measure. Calculate the angle in radian measure: 
\par
$\Mmeasuredangle\left(g, h\right)=$\MLParsedQuestion{10}{pi/3}{3}{M05ExAngle}.
\par
\MInputHint{Enter~$\pi$ as \texttt{pi}. Alternatively, you can enter your result rounded to three decimal digits.}
\par
\begin{MHint}{Solution}
From
\[
\frac{\Mmeasuredangle\left(g, h\right)}{2\pi} %
= \frac{60\MGrad}{360\MGrad} %
\]
we have
\[
 \Mmeasuredangle\left(g, h\right) %
= \frac{60\MGrad}{360\MGrad}\cdot 2\pi %
= \frac{1}{6}\cdot 2\pi=\frac{\pi}{3}\MDFPeriod 
\]
\end{MHint}

\end{MExercise}

\begin{MExercise}
The angle $\beta$ equals $\pi/4$ in radian measure. 
Find its value in degree measure.
\par
$\beta=$\MLParsedQuestion{10}{45}{3}{PARSEDQUEST2}$\MGrad$.
\par
\begin{MHint}{Solution}
From
\[
\frac{\pi/4}{2\pi} = \frac{\beta}{360\MGrad} %%
\]
we obtain
\[
\beta %
 = \frac{\pi/4}{2\pi}\cdot 360\MGrad %
 = \frac{1}{8}\cdot 360\MGrad %
 = 45\MGrad \MDFPeriod %%
\]
\end{MHint}
\end{MExercise}

\begin{MExercise}
The values of the six angles $\alpha_1,\ldots, \alpha_6$ are specified either in degree measure or in radian measure. 
Convert their values to the other measure. 

\begin{center}
\begin{tabular}{l*{6}{c}}
 & $\alpha_1$ & $\alpha_2$ & $\alpha_3$ & $\alpha_4$ & $\alpha_5$ & $\alpha_6$ \\
 Radian measure & $\pi$               &   \MLParsedQuestion{10}{9*pi/5}{3}{GEO1} &   	$\frac{2 \pi}{3}$                &\MLParsedQuestion{10}{3*pi/2}{3}{GEO2} & $\frac{11 \pi}{12}$ & \MLParsedQuestion{10}{pi/6}{3}{GEO3} \\
 Degree measure   & \MLParsedQuestion{10}{180}{3}{GEO4}      & $324$    & \MLParsedQuestion{10}{120}{3}{GEO5}        &    $270$    & \MLParsedQuestion{10}{165}{3}{GEO6} & $30$ \\
\end{tabular}
\end{center}

\MInputHint{Enter~$\pi$ as \texttt{pi}. Alternatively, you can enter your result rounded to three decimal digits.}
\end{MExercise}

\end{MXContent}
%jgl: end of section 2.


%content: section 3: Dreiecke.

\MSubsection{All about Triangles}
\MLabel{M05_Dreiecke}

\begin{MIntro}
\MDeclareSiteUXID{VBKM05_Dreiecke_Intro}

Technical structures such as trusses and some bridges use triangles as their 
constructing elements (see figure below).


%Bruecke:
%Beispiel einer Brueckenkonstruktion mit Dreieckselementen:
\begin{center}
\MTikzAuto{%
\begin{tikzpicture}[line width=1pt]
\coordinate (A) at (-6,0);
\coordinate (B) at (6,0);
\coordinate (C) at ($(-6,0) + (60:3)$);
\coordinate (D) at ($(6,0) + (120:3)$);
%Bruecke:
\draw (-6,0) -- ++(60:3) -- ++(-60:3) -- ++(60:3) -- ++(-60:3) %
 -- ++(60:3) -- ++(-60:3) -- ++(60:3) -- ++(-60:3);
\draw[line width=2pt] (A) -- (B);
\draw[line width=2pt] (C) -- (D);
\coordinate (LA) at (-0.2,0);
\coordinate (LAA) at (-0.3,0);
\coordinate (LAU) at (-0.3,-0.1);
\coordinate (LB) at (0.2,0);
\coordinate (LBB) at (0.3,0);
\coordinate (LBU) at (0.3,-0.1);
\coordinate (LC) at ($ (LA) + (60:0.4) $);
\coordinate (LM) at ($ (LC) + (0,0.08) $);
%Lager:
\begin{scope}[xshift=-6cm,yshift=-0.5cm]
\coordinate (LA) at (-0.2,0);
\coordinate (LAA) at (-0.3,0);
\coordinate (LB) at (0.2,0);
\coordinate (LBB) at (0.3,0);
\coordinate (LC) at ($ (LA) + (60:0.4) $);
\coordinate (LM) at ($ (LC) + (0,0.08) $);
\draw (LAA) -- (LBB);
\draw (LA) -- (LC) -- (LB);
\draw (LM) circle(0.07);
\draw (-0.2,0) -- ++(-120:0.2);
\draw (-0.1,0) -- ++(-120:0.2);
\draw (0.0,0) -- ++(-120:0.2);
\draw (0.1,0) -- ++(-120:0.2);
\draw (0.2,0) -- ++(-120:0.2);
\draw (0.3,0) -- ++(-120:0.2);
\end{scope}
%
\begin{scope}[xshift=6cm,yshift=-0.5cm]
\coordinate (LA) at (-0.2,0);
\coordinate (LAA) at (-0.3,0);
\coordinate (LAU) at (-0.3,-0.1);
\coordinate (LB) at (0.2,0);
\coordinate (LBB) at (0.3,0);
\coordinate (LBU) at (0.3,-0.1);
\coordinate (LC) at ($ (LA) + (60:0.4) $);
\coordinate (LM) at ($ (LC) + (0,0.08) $);
\draw (LAA) -- (LBB);
\draw (LA) -- (LC) -- (LB);
\draw (LM) circle(0.07);
\draw (LAU) -- (LBU);
\draw (-0.2,-0.1) -- ++(-120:0.2);
\draw (-0.1,-0.1) -- ++(-120:0.2);
\draw (0.0,-0.1) -- ++(-120:0.2);
\draw (0.1,-0.1) -- ++(-120:0.2);
\draw (0.2,-0.1) -- ++(-120:0.2);
\draw (0.3,-0.1) -- ++(-120:0.2);
\end{scope}
\end{tikzpicture}
}
\end{center}


Conversely, the question arises how an arbitrary surface can be subdivided into triangles. 
For many geometrical calculations this question is useful. 
Some examples are given in Section~\MRef{M05_Vielecke}.

Furthermore, the question of how to partition arbitrary surfaces into simple 
``basic elements'' results in constructive answers in applications that are relevant 
far beyond simple geometric considerations. A first impression of such relevance gives 
us the integral calculus described in chapter~\MRef{VBKM08} together with its application 
to the calculation of surface areas. There, the first approximation to the integral is 
a partition of the area into rectangles (each consisting of two triangles, to stay on  topic). For the three-dimensional computer aided modelling of surfaces,
for example in the manufacturing of car bodies, partitions into triangles 
(triangulations) are the basis of many calculations and deceptively realistic 
looking virtual animations. 
\end{MIntro}


\begin{MXContent}{Triangles}{Triangles}{STD}
\MDeclareSiteUXID{VBKM05_Dreiecke_Content}

Many statements on geometric figures and solids arise from the properties of 
triangles. A triangle is the ``simplest closed figure'' which can be determined by 
three non-collinear points (i.e. the points do not lie on a single straight line).

%Eine Strecke $\MGeoStrecke{A}{B}$ ist durch ihre beiden {\glqq}Endpunkte{\grqq} 
%$A$ und $B$ bestimmt (die hier als verschieden vorausgesetzt werden). Die 
%{\glqq}einfachste Figur{\grqq}, in der neben einer L"ange, wie sie bei Strecken
%als bestimmendes Merkmal vorliegt, auch mehrere Winkel vorkommen, ergibt sich,
%wenn ein weiterer Punkt $C$ hinzugenommen wird, der nicht auf der Geraden
%\MGeoGerade{A}{B}$ liegt.


First, we will present the important terms. Then we will answer the question of
under which conditions a triangle is uniquely defined and how individual 
angles and sides can be calculated. Here, the intercept theorems are an important tool 
since they can also be considered as statements on relations between different triangles.

In Section~\MRef{M05_Trigonometrie} we will then investigate functional relations 
between side lengths and angles enabling us to answer advanced questions 
relevant to applications.

\begin{MXInfo}{Triangle}\MLabel{Mathematik_ElementareGeometrie_SummeDerInnenwinkel}%

A \MEntry{triangle}{triangle} is constructed by joining three non-collinear points $A$, $B$, and $C$. 
The resulting triangle is denoted by $\MGeoDreieck{A}{B}{C}$.

\begin{itemize}
 \item The three points are called \MEntry{vertices}{vertex (triangle)} of the triangle, and the 
three lines are called \MEntry{sides}{side (triangle)} of the triangle.
    
 \item Each two sides of the triangle form two angles. The smaller angle is called 
\MEntry{interior angle}{interior angle} (or simply angle for short) and the greater 
angle is called \MEntry{exterior angle}{exterior angle}.
 
 \item The sum of the three interior angles is always $180\MGrad$ or $\pi$.
\end{itemize}
\end{MXInfo}

\begin{tabular}{@{}lr@{}}
\begin{minipage}{10cm}

The vertices and sides of a triangle are often denoted as follows:
vertices are denoted by upper-case Latin letters
in mathematical positive direction (counter-clockwise). 
The side opposite a vertex is denoted by its lower-case Latin letter, 
and the interior angle of the vertex is denoted by the corresponding lower-case Greek letter. 
\par
Since exterior angles are far less important than interior angles, 
the \textbf{interior angles} are simply called \MEntry{angles}{angle (triangle)} 
of the triangle.
\end{minipage}
&
\begin {minipage}{6cm}
%\begin{center}
\MTikzAuto{%
\begin{tikzpicture}
\coordinate[label=below left:$A$] (A) at (0,0);
\coordinate[label=right:$B$]      (B) at (4,0.5);
\coordinate[label=above:$C$]      (C) at (2,3);
\coordinate (MAB) at ($ (A)!0.5!(B) $);
\coordinate (MBC) at ($ (B)!0.5!(C) $);
\coordinate (MCA) at ($ (C)!0.5!(A) $);
%
\draw (A) -- (B) -- (C) -- cycle;
%
\path (A) -- node[near start]{$\alpha$} (MBC) node[above right]{$a$};
\path (B) -- node[near start]{$\beta$}  (MCA) node[above left] {$b$};
\path (C) -- node[near start]{$\gamma$} (MAB) node[below]      {$c$};
%
\path let \p1 = (current bounding box.east),
          \p2 = (current bounding box.west),
          \p3 = ($ (\p1) - (\p2) $),
          \n3 = {veclen(\p3)} in;
%     (current bounding box.south) node [below, text width=\n3, text centered, outer sep = 0.5\baselineskip]
 %          {Die Bezeichnungen von Ecken, Seiten und Innenwinkeln in einem Dreieck.};
\end{tikzpicture}
}
\end{minipage}
\end{tabular}

The sum of all (interior) angles is always $180\MGrad$ or $\pi$. Hence, at most one angle can be 
equal to or greater than $90\MGrad$ or $\frac{\pi}{2}$. Consequently, triangles are classified 
according to their greatest interior angle into three types:


\begin{MXInfo}{Names of Triangles}%
Triangles are named according to their angles as follows:
\begin{itemize}
 \item A triangle that only has angles less than~$\frac{\pi}{2}$ is called 
\MEntry{acute}{acute (triangle)}.
 
 \item A triangle that has a right angle is called \MEntry{right-angled}{right-angled (triangle)} 
  triangle or simply right triangle.

  In a right triangle the two sides enclosing the right angle are called 
  \MEntry{catheti}{cathetus} or \textbf{legs}, and the side opposite to the right angle
  is called \MEntry{hypotenuse}{hypotenuse}.    

 \item A triangle that has an angle greater than~$\frac{\pi}{2}$ is called 
  \MEntry{obtuse}{obtuse (triangle)}.
\end{itemize}
\end{MXInfo}

As an example, let us consider the simple structure of a car jack with the shape of a triangle (see figure below):
It consists of two rods connected by a joint. The two other endpoints of the rods can be pulled 
together. The greater the angle of a rod with respect to the street is, the higher the joint is 
above the ground.

\begin{center}
\MTikzAuto{%
\begin{tikzpicture}[line width=1pt]
\draw (0,0) -- (5,4) -- (7,0);
\draw[style=dashed] (7,0) -- (0,0);
\draw[style=dotted] (5,0) -- node[right] {$h_c$} (5,4);
\node[left] at (0,0) {$A$};
\node[right] at (7,0) {$B$};
\node[above] at (5,4) {$C$};
\node[below] at (5,0) {$D$};
\draw (5.5,0) arc(0:90:0.5);
\draw (5.2,0.2) circle(0.5pt);
\end{tikzpicture}
}
\end{center}

Thus, in a triangle $\MGeoDreieck{A}{B}{C}$ the shortest line segment between vertex $C$ and 
the line defined by the side $c$ opposite to $C$ is called  
\MEntry{altitude (or height) of the triangle}{altitude of a triangle} $h_c$ on the (base) side $c$.
The second endpoint $D$ of the line segment $h_c$ is called the \MEntry{perpendicular foot}{perpendicular foot}. The altitudes $h_a$ and 
$h_b$ are defined accordingly.

One can also say that altitudes are those line segments that are perpendicular to the 
line of a side and have the vertex opposite to the relevant side as an endpoint. 
\end{MXContent}


%content: Satz des Pythagoras.
\begin{MXContent}{Pythagoras' Theorem}{Pythagoras}{STD}
\MDeclareSiteUXID{VBKM05_Pythagoras_Content}

One statement relating the lengths of the sides in a right triangle is provided by 
\MEntry{Pythagoras' theorem}{Pythagoras (theorem)}. A commonly-used formulation of the theorem is given here.

\begin{MXInfo}{Pythagoras' Theorem}
\MLabel{VBKM05_Pythagoras}
\begin{tabular}{@{}lr@{}}
\begin{minipage}{9cm}

Consider a right triangle with the right angle at vertex~$C$.

\vspace*{1cm}
\end{minipage}
&
\begin{minipage}{7cm}
\begin{center}
\MTikzAuto{%
\begin{tikzpicture}[line width=1pt]
\coordinate[label=left:$A$] (A) at (0,0);
\coordinate[label=right:$B$] (B) at ($ (A) + (4.6,0) $);
\coordinate[label=above:$C$] (C) at ($ (B) + (120:2.3) $);
%Zeichen fuer rechten Winkel:
\draw (B) ++(120:1.8) arc(300:210:0.5);
\draw (C) ++(255:0.3) circle(0.5pt);
%Dreieck:
\draw (A) -- (B) -- (C) -- cycle;
%Beschriftung des Dreiecks:
\path (A) -- node[below] {$c$} (B) %
 -- node[above right] {$a$} (C) -- node[above left] {$b$} (A);
\end{tikzpicture}
}
\end{center}
\end{minipage}
\end{tabular}

Then, the sum of the areas of the squares on the legs a and b equals the area 
of the square on the hypotenuse c. This statement can be written as an equation 
(see also the triangle in the figure):
\[
a^2 + b^2 = c^2 \MDFPeriod
\]
If the sides of the triangle are denoted in another way, the equation has to be 
adapted accordingly!
\end{MXInfo}


\begin{MExample}
Let a right triangle with legs of length $a=6$ and $b=8$ be given.

The length of the hypotenuse can be calculated by means of Pythagoras' theorem:

\[
c = \sqrt{c^2} = \sqrt{a^2 + b^2} = \sqrt{36 + 64} = \sqrt{100} = 10 \MDFPeriod 
\]
\end{MExample}

\begin{MExercise}
Consider a right triangle $\MGeoDreieck{A}{B}{C}$ with the right angle at vertex
$C$, hypotenuse $c = \frac{25}{3}$, and altitude (height) $h_c = 4$. The line segment 
$\MGeoStrecke{D}{B}$ has the length $q = \MGeoAbstand{D}{B} = 3$. Here, $D$ is
the perpendicular foot of the altitude $h_c$. Calculate the length of the two legs 
$a$ and $b$.

\begin{MHint}{Solution}
We apply Pythagoras' theorem to the triangle $\MGeoDreieck{D}{B}{C}$ that has a right 
angle at the vertex $D$. Then, we have
\[
 a = \sqrt{h_c^2 + q^2} = \sqrt{4^2 + 3^2} = \sqrt{25} = 5 \MDFPeriod
\]
Now, we apply Pythagoras' theorem to the given right triangle $\MGeoDreieck{A}{B}{C}$:
\[
 b = \sqrt{c^2-a^2} = \sqrt{\left(\frac{25}{3}\right)^2-5^2} %
 = \sqrt{\frac{400}{9}} %
 = \frac{20}{3}. %%
\]

\end{MHint}
\end{MExercise}
\MEntry{Thales' theorem}{Thales (theorem)} is another important theorem that
makes a statement on right triangles.


\begin{MXInfo}{Thales' Theorem}
\par
\begin{tabular}{@{}lr@{}}
\MTikzAuto{%
\begin{tikzpicture}[x=1.0cm, y=1.0cm] 
\draw[color=black, thick] (-3,0) -- (3,0);
\draw[color=blue, thick] (3,0) arc (0:180:3);
\draw[color=black, thick] (-3,0) -- (50:3) -- (3,0);
%\draw[color=red, thick] (-3,0) -- (100:3) -- (3,0);
\draw[color=black] (50:3) ++(295:0.6) arc (295:205:0.6);
%\draw[color=black] (100:3) ++(320:0.6) arc (320:230:0.6);
\fill[color=black] (50:3) ++(250:0.3) circle (1.0pt);
%\fill[color=black] (100:3) ++(275:0.3) circle (1.0pt);
\draw[color=black] (0,0) node[anchor=north] {$M$};
\draw[color=black] (-1.5,0) node[anchor=south] {$r$};
\draw[color=black] (1.5,0) node[anchor=south] {$r$};
\draw (0,0) -- (50:3);
%\draw (0,0) -- (100:3);
\node[anchor=north west] at (50:1.5) {$r$};
%\node[anchor=west] at (100:1.8) {$r$};
\node[left] at (-3, 0) {$A$};
\node[right] at (3, 0) {$B$};
\node[above right] at (50:3) {$C$};
\end{tikzpicture}
}
&
\begin{minipage}[b]{7cm}
If the triangle $ABC$ has a right angle at the vertex $C$, then vertex $C$ 
lies on a circle with radius $r$ whose diameter $2r$ is the 
hypotenuse $\MGeoStrecke{A}{B}$.
\vspace*{1.5cm}
\end{minipage}
\end{tabular}
\end{MXInfo}

The converse statement is also true. Construct a half-circle above a line segment $\MGeoStrecke{A}{B}$.
If the points $A$ and $B$ are joint to an arbitrary point $C$ on the half-circle, then the 
resulting triangle $ABC$ is always right-angled.

\begin{MExample}\MLabel{ThaleskreisBeispiel}%
Construct a right triangle with a given hypotenuse $c=6\MEinheit{cm}$ and altitude 
$h_c=\MZahl{2}{5}\MEinheit{cm}$.


\begin{tabular}{@{}lr@{}}
\begin{minipage}[b]{7cm}
 \begin{enumerate}
  \item First, draw the hypotenuse \[c=\MGeoStrecke{A}{B} \MDFPeriod \]

  \item Let the middle of the hypotenuse be the centre of a circle with radius 
  $r = c/2$.

  \item Then draw a parallel to the hypotenuse at distance $h_c$. This 
  parallel intersects Thales' circle in two points $C$ and $C'$.
 \end{enumerate}
\end{minipage}
&
\MTikzAuto{%
\begin{tikzpicture}[x=1.2cm, y=1.2cm] 
\draw[color=red, thick] (-3,0) -- (3,0);
\draw[color=blue, thick] (3,0) arc (0:180:3);
\draw[color=red, thick, dashed] (-3,2.5) -- (3,2.5);
\fill[color=black, opacity=0.5] (0,0) circle (2.0pt);
\draw[color=black, thick] (-3,0) -- (-1.658312395,2.5) -- (3,0);
\draw[color=black, thick, dashed] (-3,0) -- (1.658312395,2.5) -- (3,0);
\draw[color=black] (-1.658312395,0) -- (-1.658312395,2.5);
\draw[color=gray, dashed] (1.658312395,0) -- (1.658312395,2.5);
\draw[color=black] (-3,0) node[anchor=north east] {$A$};
\draw[color=black] (3,0) node[anchor=north west] {$B$};
\draw[color=black] (0,-2pt) node[anchor=north] {$M$};
\draw[color=black] (-1.658312395,1.10) node[anchor=east] {$h_c$};
\draw[color=black] (1.658312395,1.10) node[anchor=west] {$h_c$};
\node[anchor=south east] at (-1.658312395,2.5) {$C$};
\node[anchor=south west] at (1.658312395,2.5) {$C'$};
\draw[color=red] (-1.5,0) node[anchor=north] {\large $\mathsf{1}$};
\draw[color=blue] (30:3) node[anchor=west] {\large $\mathsf{2}$};
\draw[color=red] (3,2.5) node[anchor=south east] {\large $\mathsf{3}$};
\end{tikzpicture}
}
\end{tabular}
Together with the points $A$ and $B$ these intersections points each form 
a triangle possessing the required properties, i.e. two solutions exist.
Two further solutions are obtained if the construction is repeated 
drawing a second parallel below the hypotenuse. The constructed triangles are 
different in position but concerning shape and size 
these triangles are ``congruent'' (see also Section~\MRef{VBKM05_DreieckeKongruenzsaetze}).
\end{MExample}

\begin{MExercise}
Find the maximum altitude (height) $h_c$ of a right triangle with hypotenuse $c$.

\begin{MHint}{Solution}
The maximum altitude $h_c$ is the radius of the Thales circle on the 
hypotenuse. Hence, $h_c \leq \frac{c}{2}$.
\end{MHint}
\end{MExercise}

%Zusatz: Hoehensatz und Kathetensatz.
\begin{MCOSHZusatz}

In a right triangle, some statements beyond Pythagoras' theorem hold.

To study them, we will use the notation illustrated below:
\par
\begin{tabular}{@{}lr@{}}
\begin{minipage}{9cm}
Consider a right triangle with the right angle at the vertex $C$. The altitude
$h_c$ intersects the hypotenuse of the triangle $\MGeoDreieck{A}{B}{C}$ in 
the point $D$, called perpendicular foot. Furthermore, let $p = \MGeoAbstand{A}{D}$ 
and $q = \MGeoAbstand{B}{D}$.
\vspace*{1cm}
\end{minipage}
&
\begin{minipage}{7cm}
\MTikzAuto{%
\begin{tikzpicture}
\coordinate[label=above:$C$]       (C) at (0,0);
\coordinate[label=below right:$B$] (B) at ($ (C) + (2,-4) $);
\path let \p1=($ (B) - (C) $) in 
        coordinate[label=left:$A$] (A) at ($ (C) + ({\y1*3/4}, {-\x1*3/4}) $);
\path let \p1=($ (B) - (A) $) in
        coordinate                 (K) at ($ (C) + ({\y1/5}, {- \x1/5}) $);
\coordinate[label=below:$D$]       (D) at (intersection of C--K and A--B);
%
\draw (B) -- node[sloped, above]{$a$} (C) -- node[sloped, above]{$b$} (A) -- cycle;
\draw (C) -- node[sloped, right, rotate=-90]{$h_c$} (D);
\path (A) -- node[sloped, above]{$p$} (D) -- node[sloped, above]{$q$} (B) -- node[sloped, below]{$c$} (A);
\end{tikzpicture}
}
\end{minipage}
\end{tabular}

\begin{MXInfo}{Right Triangle Altitude Theorem}
The area of the square on the altitude equals the area of the rectangle created
by the two hypotenuse segments: 
\[h^2 = p\cdot q \MDFPeriod\]	
\end{MXInfo}

\begin{MXInfo}{Cathetus Theorem}
The area of the square on a leg (cathetus) equals the area of the rectangle created 
by the hypotenuse and the hypotenuse segment adjacent to the leg:
\[a^2 = c\cdot q \MDFPSpace, \MDFPaSpace b^2=c\cdot p \MDFPeriod\]
\end{MXInfo}

\begin{MExample}
Let a right triangle with the legs $a=3$ and $b=4$ be given.

The length of the hypotenuse can be calculated by means of Pythagoras' theorem:
\[
c = \sqrt{a^2 + b^2}=\sqrt{9 + 16}=\sqrt{25}=5 \MDFPeriod %%
\]
According to the cathetus theorem the hypotenuse segments  $p$ and $q$ are:
\[
q=\frac{a^2}{c}=\frac{9}{5} = \MZahl{1}{8} \quad \text{and} \quad 
p=\frac{b^2}{c}=\frac{16}{5} = \MZahl{3}{2} \MDFPeriod
\]
According to the altitude theorem the altitude $h_c$ is:
\[
h_c=\sqrt{p\cdot q}=\sqrt{\frac{9}{5}\cdot\frac{16}{5}} %
=\sqrt{\frac{144}{25}}=\frac{12}{5} = \MZahl{2}{4} \MDFPeriod\]
\end{MExample}

\begin{MExercise}
Find the length of the two legs of a given right triangle with hypotenuse
$c=\MZahl{10}{5}$, altitude $h_c=\MZahl{5}{04}$,
and hypotenuse segment $q=\MZahl{3}{78}$.

\begin{MHint}{Solution}
\[\text{Cathetus theorem:} \quad a=\sqrt{c\cdot q}=\sqrt{\MZahl{10}{5} \cdot \MZahl{3}{78}}=\MZahl{6}{3} \MDFPSpace;\]
\[\text{Pythagoras' theorem:} \quad b=\sqrt{c^2-a^2} = \sqrt{{\MZahl{10}{5}}^2-{\MZahl{6}{3}}^2}=\MZahl{8}{4} \MDFPeriod\]
\end{MHint}
\end{MExercise}
\end{MCOSHZusatz}
%Ende Zusatz: Hoehensatz und Kathetensatz.

\end{MXContent}
%end of content: Satz des Pythagoras.

\begin{MXContent}{Congruence and Similar Triangles}{Kongruenz}{STD}
\MDeclareSiteUXID{VBKM05_Kongruenzsaetze_Content}

Each triangle includes three sides and three angles. The 
exterior angles are already defined by the interior angles such that 
the ``shape'' of a triangle is determined by six characteristics. If two 
triangles coincide in all these characteristics, they are said to be 
\MEntry{congruent}{congruent (triangle)}. For that, the position of the 
triangles is not relevant, i.e. congruent triangles can be transformed into 
each other by rotation, reflection, and translation.

If four of the six characteristics are known, the triangle is uniquely determined
up to rotation or reflection, i.e. its position in the plane. Then, all triangles with these 
characteristics are congruent. In some cases, only three characteristics are sufficient
to determine the triangle uniquely. These cases are described by the following 
\MEntry{theorems for congruent triangles}{theorem (congruent triangles)}.


\begin{MXInfo}{Theorems for Congruent Triangles}%
\MLabel{VBKM05_DreieckeKongruenzsaetze}%

Up to its position in the plane, a triangle is uniquely defined if one of the 
following situations is at hand:

\begin{itemize}
 \item At least four of the six characteristics (three angles and three sides) are known.
 
 \item The lengths of all three sides are known. \\ (This theorem is usually called 
  ``sss'' for ``side, side, side''.)

 
 \item Two angles and the length of the included side are known. \\ (This theorem is 
  usually called ``asa'' for ``angle, side, angle''.)
        
 \item The lengths of two sides and the included angle are known. \\ (This theorem is 
  usually called ``sas'' for ``side, angle, side''.)
       
             
 \item The lengths of two sides and a non-included angle are known such that 
  only one side is a leg of the given angle and the second side is greater than the 
  given leg. 

  (This theorem is called ``Ssa'', where the upper-case ``S''  indicates that 
  the side opposite to the given angle is the greater one.)
\end{itemize}
\end{MXInfo}
%Referenzfehler:
%Ein Beispiel f"ur deckungsgleiche (kongruente) Dreiecke sind die L"osungen 
%in \MRef{VBKM05_BspSatzDesThales}, wo sich verschiedene Dreiecke ergaben, 
%deren Seiten aufgrund der Konstruktion so zugeordnet werden k"onnen, 
%dass diese gleich lang sind.

If only two characteristics of a triangle are known, or three characteristics are known
that do not correspond to one of the cases described above, than a number of different triangles 
with these characteristics exist which are not congruent.

The next example will illustrate how a triangle can be constructed applying the theorems 
for congruent triangles. Then another example will be considered, where only 
three angles are known and hence none of the theorems described above apply.

\begin{MExample}
\begin{tabular}{@{}lr@{}}
\begin{minipage}{10cm}
Let the sides $b$, $c$, and the angle~$\alpha$ be given. According to the
``sas'' theorem the triangle is constructed as follows: 
1. Draw a line, 
in this example side $c$. 2. Attach the angle 
$\alpha$ to the corresponding vertex ($A$). 
3. Draw a circle around vertex $A$ with a radius
corresponding to the length of the second side (in this case, side $b$). 
4. The intersection point of this circle with the second leg of the angle~$\alpha$
is the third vertex ($C$) of the triangle. (The first leg of $\alpha$
is the side $c$.) 
\end{minipage}
&
\begin{minipage}{7cm}
%\begin{center}
\MTikzAuto{%
\begin{tikzpicture}
\coordinate [label=left:$A$]        (A) at (0,0);
\coordinate [label=below right:$B$] (B) at ($ (A) + (-15:3.2) $);
\coordinate [label=above:$C$]       (C) at ($ (A) + (60:2) $);
%
\draw (A) -- node[below left]{1.} (B) -- node[above right] {4.} (C) -- cycle;
\draw[dotted] (C) -- ($ (C)!-0.5!(A) $) node[below right]{2.};
\node at (A) [label=135:3., draw, dotted, circle through=(C)]{};
\end{tikzpicture}
}
\end{minipage}
\end{tabular}
\end{MExample}


\begin{MExercise}
Construct a triangle with side $c=5$ and the two angles 
$\alpha=30\MGrad$ and $\beta=120\MGrad$ using the notation introduced above.


\begin{MHint}{Solution}
\begin{tabular}{@{}lr@{}}
\begin{minipage}{9cm}
1. Draw the given line segment $c$. 2. Attach the corresponding angles $\alpha$ and $\beta$ to either side of the segment 
. 3. The intersection point of 
the two new legs is the third vertex $C$ of the triangle.
\end{minipage}
&
\begin{minipage}{7cm}
\MTikzAuto{%
\begin{tikzpicture}[scale=0.5]
\coordinate [label=left:$A$]        (A) at (0,0);
\coordinate [label=below right:$B$] (B) at ($ (A) + (10:4) $);
\coordinate [label=above left:$C$]  (C) at ($ (A) + (40:7) $);
%
\draw (A) -- node[below]{1.} (B) -- (C) -- cycle;
\draw[dotted] (C) -- ($ (C)!-0.5!(A) $) node[below right]{2.};
\draw[dotted] (C) -- ($ (C)!-0.5!(B) $) node[left]{3.};
\end{tikzpicture}
}
\end{minipage}
\end{tabular}
\end{MHint}
\end{MExercise}

\begin{MExample}\MLabel{VBKM05_BeispielAehnlichkeit}%
Let three angles $\alpha=77\MGrad$, $\beta=44\MGrad$, and 
$\gamma=59\MGrad$ be given summing up to $180\MGrad$. 
This case does not correspond to one of the cases in the theorems for congruent 
triangles~\MRef{VBKM05_DreieckeKongruenzsaetze}. A few 
examples for triangles with the given angles are shown below. 

\begin{center}
\MTikzAuto{%
\begin{tikzpicture}[x=1.0cm, y=1.0cm] 
\pgfmathparse{5*sin(44)*cos(77)/sin(59)}\let\cx=\pgfmathresult
\pgfmathparse{5*sin(44)*sin(77)/sin(59)}\let\cy=\pgfmathresult
\foreach \sx/\sy/\dsf/\ang in {0.0cm/0.0cm/1.0/-15,6.0cm/2.0cm/0.4/40,8.0cm/0.0cm/0.5/-15} {
\begin{scope}[xshift=\sx,yshift=\sy,rotate=\ang]
\coordinate (OB) at (5,0);
\coordinate (OC) at (\cx,\cy);
\coordinate (A) at (0,0);
\coordinate (B) at ($ (A)!\dsf!(OB) $);
\coordinate (C) at ($ (A)!\dsf!(OC) $);
\coordinate (MAB) at ($ (A)!0.5!(B) $);
\coordinate (MBC) at ($ (B)!0.5!(C) $);
\coordinate (MCA) at ($ (C)!0.5!(A) $);
\draw[black,thick] (A) -- (B) -- (C) -- cycle;
\pgfmathparse{0.15/\dsf}
\path (A) -- node[pos=\pgfmathresult]{$\alpha$} (MBC);
\path (B) -- node[pos=\pgfmathresult]{$\beta$}  (MCA);
\path (C) -- node[pos=\pgfmathresult]{$\gamma$} (MAB);
\end{scope}
}
\end{tikzpicture}
}
\end{center}

Actually, an infinite number of triangles with the given angles do exist. They 
are not congruent to each other, i.e. they cannot be transformed into each other 
by rotation or reflection.
\end{MExample}


However, the triangles look similar in a way. Such \textbf{similar} triangles are 
also obtained if, for example, all the side ratios are known. This fact results
from the intercept theorems as illustrated by the figure below. 

%Dreiecke im Strahlensatz:
\begin{center}
\MTikzAuto{%
\begin{tikzpicture}[line width=1pt]
\coordinate (A) at (-3,0);
\coordinate (B) at (9,0);
\coordinate (C) at ($(A) + (0,-1)$);
\coordinate (D) at ($(B) + (0,3)$);
\fill[color=black!50!white] (0,0) -- (3,0) -- (3,1) -- cycle;
\fill[color=black!50!white] (0,0) -- (-3,0) -- (-3,-1) -- cycle;
\fill[color=black!30!white] (3,0) -- (6,0) -- (6,2) -- (3,1) -- cycle;
\fill[color=black!15!white] (6,0) -- (9,0) -- (9,3) -- (6,2) -- cycle;
\draw (A) -- (B);
\draw (C) -- (D);
\draw (A) -- (C);
\draw (3,0) -- ++(0,1);
\draw[style=dashed] (6,0) -- ++(0,2);
\draw[style=dotted] (B) -- (D);
\node[below] at (0,0) {$A$};
\node[below] at (3,0) {$B$};
\node[below] at (6,0) {$B'$};
\node[below] at (9,0) {$B''$};
\node[above] at (-3,0) {$-B$};
\node[above] at (3,1) {$C$};
\node[above] at (6,2) {$C'$};
\node[above] at (9,3) {$C''$};
\node[below] at (-3,-1) {$-C$};
\end{tikzpicture}
}
\end{center}

\begin{MXInfo}{Similarity Theorems for Triangles}%
\MLabel{VBKM05_DreieckeAehnlichkeitssaetze}

Two triangles are called \MEntry{similar}{similarity (triangles)} to each other
if they
\begin{itemize}
 \item have two (and because of the triangle postulate also three) congruent angles, or
 \item have three sides whose lengths have the same \textbf{ratio}, or
 \item have one congruent angle and two adjacent sides whose 
  lengths have the same \textbf{ratio}, or
 \item have two sides whose 
  lengths have the same \textbf{ratio} and the angles
  opposite to the greater side are congruent.
\end{itemize}
\end{MXInfo}

The right and the left triangle in Example~\MRef{VBKM05_BeispielAehnlichkeit}
have a special relationship. The left triangle is transformed into the other by 
uniform scaling\MLabel{Mathematik_ElementareGeometrie_zentrischeStreckung}
with the centre of enlargement $S$ and the scaling factor $k$.

\begin{center}
\MTikzAuto{%
\def\sxyc{0.8cm}
\begin{tikzpicture}[x=\sxyc, y=\sxyc] 
\pgfmathparse{5*sin(44)*cos(77)/sin(59)}\let\cx=\pgfmathresult
\pgfmathparse{5*sin(44)*sin(77)/sin(59)}\let\cy=\pgfmathresult
\pgfmathparse{cos(15)}\let\rc=\pgfmathresult
\pgfmathparse{sin(15)}\let\rs=\pgfmathresult
\foreach \ax/\ay/\lsf/\rsf in {\cx/\cy/-1/14,5/0/-2/15,0/0/-0.5/13.8} {
  \pgfmathparse{\lsf+(12-\lsf)/12*(\ax*\rc+\ay*\rs)}\let\cax\pgfmathresult
  \pgfmathparse{(12-\lsf)/12*(-\ax*\rs+\ay*\rc)}\let\cay\pgfmathresult
  \pgfmathparse{\rsf+(12-\rsf)/12*(\ax*\rc+\ay*\rs)}\let\cbx\pgfmathresult
  \pgfmathparse{(12-\rsf)/12*(-\ax*\rs+\ay*\rc)}\let\cby\pgfmathresult
  \draw[gray, dashed] (\cax,\cay) -- (\cbx,\cby);
}
\node[anchor=south] at (12,0) {$S$};
\foreach \sx/\sy/\dsf/\ang in {0.0cm/0.0cm/1.0/-15,4.8*\sxyc/0.0cm/0.60/-15} {
  \begin{scope}[xshift=\sx,yshift=\sy,rotate=\ang]
  \coordinate (OB) at (5,0);
  \coordinate (OC) at (\cx,\cy);
  \coordinate (A) at (0,0);
  \coordinate (B) at ($ (A)!\dsf!(OB) $);
  \coordinate (C) at ($ (A)!\dsf!(OC) $);
  \coordinate (MAB) at ($ (A)!0.5!(B) $);
  \coordinate (MBC) at ($ (B)!0.5!(C) $);
  \coordinate (MCA) at ($ (C)!0.5!(A) $);
  \draw[black,thick] (A) -- (B) -- (C) -- cycle;
  \pgfmathparse{0.15/\dsf}
  \path (A) -- node[pos=\pgfmathresult]{$\alpha$} (MBC);
  \path (B) -- node[pos=\pgfmathresult]{$\beta$}  (MCA);
  \path (C) -- node[pos=\pgfmathresult]{$\gamma$} (MAB);
  \end{scope}
}
\end{tikzpicture}
}
\end{center}

\end{MXContent}


\begin{MExercises}
\MDeclareSiteUXID{VBKM05_Kongruenzsaetze_Exercises}

\begin{MExercise}
Find corresponding angles and alternate angles in the figure below.

\begin{center}
\MTikzAuto{%
\begin{tikzpicture}
\coordinate (A) at (0,0);
\coordinate (B) at ($ (A) + ( 00:3) $);
\coordinate (C) at ($ (B) + ( 60:3) $);
\coordinate (D) at ($ (C) + (120:3) $);
\coordinate (E) at ($ (D) + (180:3) $);
\coordinate (F) at ($ (E) + (240:3) $);
\coordinate (AB) at (intersection of A--C and B--F);
\coordinate (BC) at (intersection of B--D and C--A);
\coordinate (CD) at (intersection of C--E and D--B);
\coordinate (DE) at (intersection of D--F and E--C);
\coordinate (EF) at (intersection of E--A and F--D);
\coordinate (FA) at (intersection of F--B and A--E);
%
\draw (A) -- (C) -- (E) -- cycle;
\draw (B) -- (D) -- (F) -- cycle;
%
\draw (AB) -- (DE);
\draw (BC) -- (EF);
\draw (CD) -- (FA);
\end{tikzpicture}
}
\end{center}

\begin{MHint}{Solution}
\begin{tabular}{@{}lr@{}}
\begin{minipage}[b]{9cm}
For example, the angles $\alpha$ and $\alpha'$ are corresponding angles. Likewise, 
angles $\beta$ and $\beta'$.
\par
For example, the angles $\alpha'$ and $\beta$ are alternate angles. Likewise,
angles $\alpha$ and $\beta'$.
\end{minipage}
&
\MTikzAuto{%
\begin{tikzpicture}
\coordinate (A) at (0,0);
\coordinate (B) at ($ (A) + ( 00:3) $);
\coordinate (C) at ($ (B) + ( 60:3) $);
\coordinate (D) at ($ (C) + (120:3) $);
\coordinate (E) at ($ (D) + (180:3) $);
\coordinate (F) at ($ (E) + (240:3) $);
\coordinate (AB) at (intersection of A--C and B--F);
\coordinate (BC) at (intersection of B--D and C--A);
\coordinate (CD) at (intersection of C--E and D--B);
\coordinate (DE) at (intersection of D--F and E--C);
\coordinate (EF) at (intersection of E--A and F--D);
\coordinate (FA) at (intersection of F--B and A--E);
%
\draw (A) -- (C) -- (E) -- cycle;
\draw (B) -- (D) -- (F) -- cycle;
%
\draw (AB) -- (DE);
\draw (BC) -- (EF);
\draw (CD) -- (FA);
\draw[color=black, thin] (DE) ++(-30:0.65) arc (-30:30:0.65);
\draw[color=black] (DE) ++(0:0.45) node {\small $\alpha'$};
\draw[color=black, thin] (DE) ++(150:0.65) arc (150:210:0.65);
\draw[color=black] (DE) ++(0:-0.45) node {\small $\beta'$};
\draw[color=black, thin] (EF) ++(-30:0.65) arc (-30:30:0.65);
\draw[color=black] (EF) ++(0:0.45) node {\small $\alpha$};
\draw[color=black, thin] (CD) ++(150:0.65) arc (150:210:0.65);
\draw[color=black] (CD) ++(0:-0.45) node {\small $\beta$};
\end{tikzpicture}
}
\end{tabular}

\end{MHint}
\end{MExercise}


\begin{MExercise}
Prove that the sum of interior angles in a triangle is always $\pi$
or $180\MGrad$ using the concept of alternate angles.

\begin{MHint}{Hint}
Draw a parallel to one of the sides of the triangle passing through 
the third vertex and consider the angles at this vertex.
\end{MHint}

\begin{MHint}{Solution}
\begin{tabular}{@{}lr@{}}
\MTikzAuto{%
\begin{tikzpicture}[x=1.0cm, y=1.0cm, scale=0.8] 
%%\draw[help lines, gray!50, xstep=0.5, ystep=0.5] (0,0) grid (9,8);
\draw[color=black] (1,0)--(9,4) (0.5,3.5)--(7.5,7.0);
\draw[color=black, very thick] (2,0.5) -- (7.5,3.25) -- (4.0,5.25) -- cycle;
\draw[color=black, thin] (2,0.5) ++(26.5660:1.2) arc (26.5650:67.1663:1.2);
\draw[color=black] (2,0.5) ++(46.865:0.8) node {\large $\alpha$};
\draw[color=black, thin] (7.5,3.25) ++(150.255:1.2) arc (150.255:205.565:1.2);
\draw[color=black] (7.5,3.25) ++(177.910:0.8) node {\large $\beta$};
\draw[color=black, thin] (4.0,5.25) ++(247.1663:0.9) arc (247.1663:330.255:0.9);
\draw[color=black] (4.0,5.25) ++(288.7107:0.6) node {\large $\gamma$};
\draw[color=black, thin] (4.0,5.25) ++(206.5660:1.2) arc (206.5650:247.1663:1.2);
\draw[color=black] (4.0,5.25) ++(226.865:0.8) node {\large $\alpha'$};
\draw[color=black, thin] (4.0,5.25) ++(-29.745:1.2) arc (-29.745:26.5660:1.2);
\draw[color=black] (4.0,5.25) ++(-1.5895:0.8) node {\large $\beta'$};
\draw[color=black] (5.75,4.25) node[anchor=south west] {\large $a$};
\draw[color=black] (3.0,2.875) node[anchor=south east] {\large $b$};
\draw[color=black] (4.75,1.875) node[anchor=north west] {\large $c$};
\end{tikzpicture}
}
&
\begin{minipage}[b]{9cm}
Drawing a parallel to the side $c$ passing through 
the vertex $C$ one obtains an alternate angle $\alpha'$ to $\alpha$
and an alternate angle $\beta'$ to $\beta$. The angles $\alpha'$, $\gamma$, and 
$\beta'$ form a straight angle. Therefore,
\[
\alpha'+\gamma+\beta' = \pi\MDFPeriod %%
\]
Furthermore, it is known that $\alpha'=\alpha$ and $\beta'=\beta$. Hence, 
$\alpha+\gamma+\beta=\pi$.
\end{minipage}
\end{tabular}
\end{MHint} 
\end{MExercise}
\end{MExercises}

%jgl: end of section 3.

%content: section 4: Vielecke, Flaechen und Umfang.

\MSubsection{Polygons, Area and Circumference}
\MLabel{M05_Vielecke}

\begin{MIntro}
\MDeclareSiteUXID{VBKM05_Flaecheninhalt_Intro}

In nature, various figures in different shapes can be found. There, 
rounded shapes are particularly evident. When it comes to partitioning a surface 
completely, some boundaries can be found that can be approximated as line 
segments. The honeycomb structures created by 
insects are a famous example. Technical applications are often based on figures bounded by straight 
line segments.
 

In this section we consider some special cases of polygons which can be 
used to describe surfaces bounded by straight line segments. To do this, we
will first specify some characteristic features. Then, we will address the 
question how the area of a polygon can be calculated easily.
\end{MIntro}


\begin{MXContent}{Quadrilaterals}{Quadrilaterals}{STD}
\MDeclareSiteUXID{VBKM05_Vierecke_Content}

In the previous section~\MRef{M05_Dreiecke}, triangles were considered.
They were defined by three non-collinear points. Connecting each two of the three points 
by a line segment always results in a single closed path in which
every point connects exactly two line segments. The line segments in the path
have only their endpoints in common. Furthermore, the line segments do not intersect. 

For more than three points this is not always true. Even only four points 
can be connected in such a way that line segments intersect or more than one
closed path exists.

In the figure below all given points are to be connected by a single closed path 
without any intersections.

%Verschiedene Vierecke:
\begin{center}
\MTikzAuto{%
\begin{tikzpicture}[line width=1.5pt]
\begin{scope}[xshift=-5cm]
\draw[dotted,color=blue] (-0.4,0) -- (2,0);
%\draw (0,2) -- (-1.6,-1) -- (0,0.2) -- (1.6,-1) -- cycle;
\draw (2,0) -- (-1,-1.6) -- (-0.4,0) -- (-1,1.6) -- cycle;
\end{scope}
%%
\begin{scope}[xshift=0cm]
\draw[dotted,color=blue] (-0.8,-1.2) -- (1.6,1.2);
\draw (1.6,1.2) -- (-1.6,1.2) -- (-0.8,-1.2) -- (0.8,-1.2) -- cycle;
\end{scope}
%%
\begin{scope}[xshift=5cm]
\draw[dotted,color=blue] (0:1.2cm) -- (180:1.2cm);
\draw (0:1.2cm) -- (90:1.2cm) -- (180:1.2cm) -- (270:1.2cm) -- cycle;
\end{scope}
\end{tikzpicture}
}
\end{center}

Obviously, a quadrilateral can be divided into two triangles. Generally, one 
obtains two triangles if the vertex with the greatest angle is connected to 
the opposite vertex by a line segment. Such an additional line segment between 
two vertices of the quadrilateral which are not connected to each other is called a 
\MEntry{diagonal}{diagonal (quadrilateral)} of the quadrilateral. From the fact that 
the sum of (interior) angles in a triangle equals $\pi$ or $180\MGrad$ 
then results that the sum of (interior) angles in a quadrilateral is twice 
this sum, i.e. $2 \pi$ or $360\MGrad$. 


\begin{MXInfo}{Quadrilaterals}\MLabel{VBKM05_Vierecke}%

Consider \MEntry{quadrilaterals}{quadrilaterals} constructed by connecting the 
four given points by line segments forming a single, closed and 
non-intersecting path trough these four points. There, each three of 
the four points connected by two line segments must be non-collinear. 

As for triangles, the interior angles of quadrilaterals are simply denoted 
as angles if not otherwise specified in context.
\end{MXInfo}

Like triangles, quadrilaterals are used in technical structures in many
ways. Therefore, additional terms are commonly used to specify different types of quadrilaterals. 

Also as for triangles, quadrilaterals are classified by the lengths of their 
sides or by the magnitudes of angles. There are differences between the classifications of triangles and quadrilaterals. For example, quadrilaterals can have parallel sides, or have more than one vertex with a right angle. 

\begin{MXInfo}{Special Types of Quadrilaterals}\MLabel{VBKM05_ViereckeKlassen}%

Quadrilaterals with the following properties have their own terms:
A quadrilateral is called

\begin{itemize}
\item \MEntry{trapezoid}{trapezoid} if at least one pair of opposite sides is parallel;
\item \MEntry{parallelogram}{parallelogram} if two pairs of opposite sides are parallel;
\item \MEntry{rhombus}{rhombus} or 
 \MEntry{equilateral quadrilateral}{equilateral quadrilateral} or
 \MEntry{diamond}{diamond} if all four sides are of equal length;
\item \MEntry{rectangle}{rectangle} if all four (interior) angles are right angles;
\item \MEntry{square}{square} if it is a rectangle with four sides  of equal length;
\item \MEntry{unit square}{unit square} if it is a square with sides of length $1$.
\end{itemize}
\end{MXInfo}

Thus, for the unit square also a measure has to be fixed.

%Rechteck:
\begin{center}
\MTikzAuto{%
\begin{tikzpicture}[line width=1pt]
%\coordinate (A) at (0,0);
%Rechteck:
\begin{scope}[xshift=-6.6cm]
\draw (0,0) -- ++(2.4,0) -- ++(0,1.8) -- ++(-2.4,0) -- cycle;
\node at (1.2,-1) {rectangle};
\end{scope}
%
%Trapez:
\begin{scope}[xshift=-3.2cm]
\draw (0,0) -- ++(2.4,0) -- ++(-0.3,1.8) -- ++(-1.2,0) -- cycle;
\node at (1.2,-1) {trapezoid};
\end{scope}
%
%Parallelogramm:
\begin{scope}[xshift=0cm]
\draw (0,0) -- ++(2.4,0) -- ++(0.4,1.8) -- ++(-2.4,0) -- cycle;
\node at (1.3,-1) {parallelogram};
\end{scope}
%
%Raute:
\begin{scope}[xshift=4.6cm]
\draw (0,-0.2) -- (0.8,1) -- (0,2.2) -- (-0.8,1) -- cycle;
\node at (0,-1) {rhombus};
\end{scope}
%
%Quadrat:
\begin{scope}[xshift=6.6cm]
\draw (0,0) -- ++(1.8,0) -- ++(0,1.8) -- ++(-1.8,0) -- cycle;
\node at (1,-1) {square};
\end{scope}
\end{tikzpicture}
}
\end{center}

There are several relations between the quadrilaterals introduced above:
\begin{MXInfo}{Relations between rectangles}\MLabel{VBKM05_Vierecke_Beziehungen}%
Between different quadrilaterals the following relations exist:
\begin{itemize}
\item Every square is a rectangle.
\item Every square is a rhombus.
\item Every rhombus is a parallelogram.
\item Every rectangle is a parallelogram.
\item Every parallelogram is a trapezoid.
\end{itemize}
\end{MXInfo}

These quadrilaterals can be characterised by means of the properties of 
their sides, angles, or diagonals in many ways.

\begin{MXInfo}{Parallelogram}\MLabel{VBKM05_Parallelogramm}%
\begin{tabular}{@{}lr@{}}
\begin{minipage}{9.6cm}
A quadrilateral is a parallelogram if and only if
\begin{itemize}
\item opposite sides are parallel;
%jgl: Es werden hier nur einfache Vierecke betrachtet:
%\item es ein einfaches Viereck ist, dessen gegen"uberliegenden Seiten 
\item opposite sides are of equal length;
\item opposite (interior) angles are equal;
\item two adjacent (interior) angles sum up to $\pi$ or $180\MGrad$, respectively; 
\item diagonals bisect each other.
\end{itemize}
\end{minipage}
&
\begin{minipage}{6cm}
%Parallelogramm:
\begin{center}
\MTikzAuto{%
\begin{tikzpicture}[line width=2pt]
\begin{scope}[yshift=1.8cm]
\coordinate (A) at (0,0);
\coordinate (B) at ($ (A) + (10:4.5cm) $);
\coordinate (D) at ($ (A) + (60:1.6cm) $);
\coordinate (C) at ($ (D) + (A)!1!(B) $);
%
%\draw[color=blue] (A) -- (B) -- (C) -- (D) -- cycle;
\draw[color=blue] (A) -- (B);
\draw[color=blue] (C) -- (D);
\draw[color=green] (B) -- (C);
\draw[color=green] (D) -- (A);
\foreach \Punkt in {(A), (B), (C), (D)} do
\filldraw \Punkt circle(2pt);
\end{scope}
\begin{scope}[yshift=0cm]
\coordinate (A) at (0,0);
\coordinate (B) at ($ (A) + (10:4.5cm) $);
\coordinate (D) at ($ (A) + (60:1.6cm) $);
\coordinate (C) at ($ (D) + (A)!1!(B) $);
%
\draw[color=red] ($ (A)!0.2!(B) $) arc(10:60:0.9cm);
%\draw[color=red] ($ (A) + (35:0.3cm) $) circle(1pt);
\draw[color=red] ($ (C)!0.2!(D) $) arc(190:240:0.9cm);
%\draw[color=red] ($ (C) + (215:0.3cm) $) circle(1pt);
\draw[color=red!50!yellow] ($ (B)!0.2!(C) $) arc(60:190:0.32cm);
%\draw[color=red] ($ (A) + (35:0.3cm) $) circle(1pt);
\draw[color=red!50!yellow] ($ (D)!0.2!(A) $) arc(240:370:0.32cm);
%\draw[color=red] ($ (C) + (215:0.3cm) $) circle(1pt);
\draw[color=black!50!white] (A) -- (B) -- (C) -- (D) -- cycle;
%\draw[color=blue] (A) -- (B);
%\draw[color=blue] (C) -- (D);
%\draw[color=green] (B) -- (C);
%\draw[color=green] (D) -- (A);
\foreach \Punkt in {(A), (B), (C), (D)} do
\filldraw \Punkt circle(2pt);
\end{scope}
\begin{scope}[yshift=-1.8cm]
\coordinate (A) at (0,0);
\coordinate (B) at ($ (A) + (10:4.5cm) $);
\coordinate (D) at ($ (A) + (60:1.6cm) $);
\coordinate (C) at ($ (D) + (A)!1!(B) $);
\coordinate (S) at ($ (A)!0.5!(C) $);
%
\draw[color=black!60!white] (A) -- (B) -- (C) -- (D) -- cycle;
\draw[color=blue] (A) -- ($ (A)!0.5!(C) $);
%
\draw[color=blue] (A) -- (S);
\draw[color=blue!60!white] (S) -- (C);
%
\draw[color=green!60!black] (D) -- (S);
\draw[color=green!60!white] (S) -- (B);
\foreach \Punkt in {(A), (B), (C), (D)} do
\filldraw \Punkt circle(2pt);
\end{scope}
\end{tikzpicture}
}
\end{center}
\end{minipage}
\end{tabular}
\end{MXInfo}

Rhombuses can be described as a special type of parallelograms.

\begin{MXInfo}{Rhombus}\MLabel{VBKM05_Raute}%
\begin{tabular}{@{}lr@{}}
\begin{minipage}{9.6cm}

A quadrilateral is a rhombus if and only if
\begin{itemize}
\item all sides are of equal length;
\item it is a parallelogram  in which the diagonals are perpendicular;
%jgl: Folgende Aussage gilt in jedem(!) Parallelogramm:
%\item es ein Parallelogramm ist, in welchem die Diagonalen einander halbieren.
%jgl: neue Charakterisierung:
\item at least two adjacent sides are of equal length and the diagonals bisect each other.
\end{itemize}
\end{minipage}
&
\begin{minipage}{6cm}
%Raute:
\begin{center}
\MTikzAuto{%
\begin{tikzpicture}[line width=2pt]
\begin{scope}[yshift=1.8cm]
\coordinate (A) at (0,0);
\coordinate (B) at ($ (A) + (10:2.5cm) $);
\coordinate (D) at ($ (A) + (40:2.5cm) $);
\coordinate (C) at ($ (D) + (A)!1!(B) $);
%
\draw[color=blue] (A) -- (B) -- (C) -- (D) -- cycle;
\foreach \Punkt in {(A), (B), (C), (D)} do
\filldraw \Punkt circle(2pt);
\end{scope}
%%
\begin{scope}[yshift=0cm]
\coordinate (A) at (0,0);
\coordinate (B) at ($ (A) + (10:2.5cm) $);
\coordinate (D) at ($ (A) + (40:2.5cm) $);
\coordinate (C) at ($ (D) + (A)!1!(B) $);
\coordinate (S) at ($ (A)!0.5!(C) $);
%
\draw[color=red] ($ (S) + (25:0.4cm) $) arc(25:110:0.4cm);
\filldraw[color=red] ($ (S) + (70:0.2cm) $) circle(0.3pt);
%
\draw[color=black] (A) -- (B);
\draw[color=black] (D) -- (C);
\draw[color=black!60!white] (B) -- (C);
\draw[color=black!60!white] (A) -- (D);
%
\draw[color=blue] (A) -- (C);
\draw[color=green] (D) -- (B);
\foreach \Punkt in {(A), (B), (C), (D)} do
\filldraw \Punkt circle(2pt);
\end{scope}
%%
\begin{scope}[yshift=-1.8cm]
\coordinate (A) at (0,0);
\coordinate (B) at ($ (A) + (10:2.5cm) $);
\coordinate (D) at ($ (A) + (40:2.5cm) $);
\coordinate (C) at ($ (D) + (A)!1!(B) $);
\coordinate (S) at ($ (A)!0.5!(C) $);
%
\draw[color=black] (D) -- (A) -- (B);
\draw[color=black!60!white] (B) -- (C);
\draw[color=black!40!white] (C) -- (D);
%
%\draw[color=blue] (A) -- ($ (A)!0.5!(C) $);
\draw[color=blue] (A) -- (S);
\draw[color=blue!60!white] (S) -- (C);
\draw[color=green!50!black] (D) -- (S);
\draw[color=green] (S) -- (B);
\foreach \Punkt in {(A), (B), (C), (D)} do
\filldraw \Punkt circle(2pt);
\end{scope}
\end{tikzpicture}
}
\end{center}
\end{minipage}
\end{tabular}
\end{MXInfo}

In the case of rectangles one often thinks of right angles since the term
rectangle comes from the Latin word rectangulus, 
which is a combination of rectus (right) and angulus (angle).
Apart from that, rectangles can simply be described by means of 
the properties of their diagonals.

\begin{MXInfo}{Rectangle}\MLabel{VBKM05_Rechteck}%
\begin{tabular}{@{}lr@{}}
\begin{minipage}{9.6cm}
A quadrilateral is a rectangle if and only if

\begin{itemize}
\item all (interior) angles are equal;
\item it is a parallelogram containing at least one right angle;
\item it is a parallelogram whose diagonals are of equal length; 
\item the diagonals are of equal length and bisect each other;
%jgl: Folgende Aussage beschreibt ein spezielles(!) Recheck: ein Quadrat, sodass
%die Aequivalenz hier falsch ist:
%\item es eine Raute mit gleich langen Diagonalen ist;
\item the diagonals bisect each other and at least one (interior) angle 
  is a right angle.
\end{itemize}
\end{minipage}
&
\begin{minipage}{6cm}
%Rechteck:
\begin{center}
\MTikzAuto{%
\begin{tikzpicture}[line width=2pt]
\begin{scope}[yshift=1.6cm]
\coordinate (A) at (0,0);
\coordinate (B) at ($ (A) + (10:5cm) $);
\coordinate (D) at ($ (A) + (100:2.5cm) $);
\coordinate (C) at ($ (D) + (A)!1!(B) $);
%
\draw[color=red] ($ (A)!0.1!(B) $) arc(10:100:0.5cm);
\draw[color=red] ($ (B)!0.2!(C) $) arc(100:190:0.5cm);
\draw[color=red] ($ (C)!0.1!(D) $) arc(190:280:0.5cm);
\draw[color=red] ($ (D)!0.2!(A) $) arc(280:370:0.5cm);
\filldraw[color=red!50!yellow] ($ (A) + (55:0.3cm) $) circle(0.3pt);
%
%\draw[color=red] ($ (A) + (10:0.6cm) $) arc(10:100:0.6cm);
%\draw[color=red] ($ (A) + (55:0.3cm) $) circle(0.4pt);
%
\draw[color=blue] (A) -- (B);
\draw[color=blue] (C) -- (D);
\draw[color=blue!60!white] (A) -- (D);
\draw[color=blue!60!white] (B) -- (C);
\foreach \Punkt in {(A), (B), (C), (D)} do
\filldraw \Punkt circle(2pt);
\end{scope}
%%
\begin{scope}[yshift=-1.6cm]
\coordinate (A) at (0,0);
\coordinate (B) at ($ (A) + (10:5cm) $);
\coordinate (D) at ($ (A) + (100:2.5cm) $);
\coordinate (C) at ($ (D) + (A)!1!(B) $);
\coordinate (S) at ($ (A)!0.5!(C) $);
%
\draw[color=red!50!yellow] ($ (A)!0.1!(B) $) arc(10:100:0.5cm);
\filldraw[color=red!50!yellow] ($ (A) + (70:0.3cm) $) circle(0.3pt);
%
\draw[color=black!60!white] (A) -- (B) -- (C) -- (D) -- cycle;
%\draw[color=blue] (A) -- ($ (A)!0.5!(C) $);
%
\draw[color=blue] (A) -- (S);
\draw[color=blue!80!white] (S) -- (C);
\draw[color=blue!60!white] (D) -- (S);
\draw[color=blue!30!white] (S) -- (B);
\foreach \Punkt in {(A), (B), (C), (D)} do
\filldraw \Punkt circle(2pt);
\end{scope}
\end{tikzpicture}
}
\end{center}
\end{minipage}
\end{tabular}
\end{MXInfo}

Squares are both special types of rectangles and special types of rhombuses.

\begin{MXInfo}{Square}\MLabel{VBKM05_Quadrat}%
\begin{tabular}{@{}lr@{}}
\begin{minipage}{9.6cm}

A quadrilateral is a square if and only if

\begin{itemize}
\item all sides are of equal length and 
 \begin{itemize}
 \item all (interior) angles are equal or
 \item at least one (interior) angle is a right angle;
 \end{itemize}
\item the diagonals are of equal length and in addition all sides are 
of equal length;
%jgl: Aussage falsch (auch von Rechtecken erf"ullt):
%\item es ein Parallelogramm ist, in welchem die Diagonalen einander 
% halbieren (jgl: dies gilt in jedem Parallelogramm) und 
% \begin{itemize}
%  \item alle Innenwinkel gleich gro"s sind oder
%  \item welches wenigstens einen rechten Winkel besitzt;
%\end{itemize}
%jgl: Vorschlag fuer eine weitere Charakerisierung:
\item the diagonals are perpendicular and 
 \begin{itemize}
  \item bisect each other and are of equal length or 
  \item all (interior) angles are equal;
 \end{itemize}
\item it is a rhombus whose diagonals are of equal length;
\item it is both a rhombus and a rectangle.
\end{itemize}
\end{minipage}
&
\begin{minipage}{6cm}
%Quadrat:
\begin{center}
\MTikzAuto{%
\begin{tikzpicture}[line width=2pt]
\begin{scope}[yshift=3.0cm]
\coordinate (A) at (0,0);
\coordinate (B) at ($ (A) + (10:2.5cm) $);
\coordinate (D) at ($ (A) + (100:2.5cm) $);
\coordinate (C) at ($ (D) + (A)!1!(B) $);
%
\draw[color=red] ($ (A)!0.2!(B) $) arc(10:100:0.5cm);
\draw[color=red] ($ (B)!0.2!(C) $) arc(100:190:0.5cm);
\draw[color=red] ($ (C)!0.2!(D) $) arc(190:280:0.5cm);
\draw[color=red] ($ (D)!0.2!(A) $) arc(280:370:0.5cm);
\filldraw[color=red!50!yellow] ($ (A) + (55:0.3cm) $) circle(0.3pt);
%
\draw[color=blue] (A) -- (B) -- (C) -- (D) -- cycle;
\foreach \Punkt in {(A), (B), (C), (D)} do
\filldraw \Punkt circle(2pt);
\end{scope}
%%
\begin{scope}[yshift=0cm]
\coordinate (A) at (0,0);
\coordinate (B) at ($ (A) + (10:2.5cm) $);
\coordinate (D) at ($ (A) + (100:2.5cm) $);
\coordinate (C) at ($ (D) + (A)!1!(B) $);
%\coordinate (S) at ($ (A)!0.5!(C) $);
%
\draw[color=green!50!black] (A) -- (B) -- (C) -- (D) -- cycle;
%
\draw[color=blue] (A) -- (C);
\draw[color=blue] (D) -- (B);
\foreach \Punkt in {(A), (B), (C), (D)} do
\filldraw \Punkt circle(2pt);
\end{scope}
%%
\begin{scope}[yshift=-3.0cm]
\coordinate (A) at (0,0);
\coordinate (B) at ($ (A) + (10:2.5cm) $);
\coordinate (D) at ($ (A) + (100:2.5cm) $);
\coordinate (C) at ($ (D) + (A)!1!(B) $);
\coordinate (S) at ($ (A)!0.5!(C) $);
%
\draw[color=red] ($ (S) + (55:0.5cm) $) arc(55:145:0.5cm);
\filldraw[color=red] ($ (S) + (100:0.25cm) $) circle(0.3pt);
%
%\draw[color=red!60!yellow] ($ (A) + (10:0.6cm) $) arc(10:100:0.6cm);
\draw[color=red!60!yellow] ($ (A)!0.2!(B) $) arc(10:100:0.5cm);
\draw[color=red!60!yellow] ($ (B)!0.2!(C) $) arc(100:190:0.5cm);
\draw[color=red!60!yellow] ($ (C)!0.2!(D) $) arc(190:280:0.5cm);
\draw[color=red!60!yellow] ($ (D)!0.2!(A) $) arc(280:370:0.5cm);
%
%jgl: Dass die Diagonalen gleich lang sind, wird durch die gleiche Farbe
%angedeutet:
\draw[color=blue!50!white] (A) -- (C);
\draw[color=blue!50!white] (D) -- (B);
%jgl: Dass die Diagonalen sich halbieren, wird durch den hervorgehobenen
%Schnittpunkt angedeutet:
\filldraw[color=blue] (S) circle(0.3pt);
%
%\filldraw[color=red!50!yellow] (S) circle(0.3pt);
%
\draw[color=black!50!white] (A) -- (B) -- (C) -- (D) -- cycle;
%
\foreach \Punkt in {(A), (B), (C), (D)} do
\filldraw \Punkt circle(2pt);
\end{scope}
\end{tikzpicture}
}
\end{center}
\end{minipage}
\end{tabular}
\end{MXInfo}
\end{MXContent}

\begin{MXContent}{Polygons}{Polygons}{STD}
\MDeclareSiteUXID{VBKM05_Vielecke_Content}

For triangles, even one vertex or side contributes to the essential properties 
of the whole triangle, for example, one vertex with a right angle. For quadrilaterals, 
a single vertex no longer has such strong specifying properties. Instead, there is a greater variety of shapes.
If ``many'' points are connected to a closed figure by line segments, there are many 
possibilities to create various figures and even to approximate round shapes. 


%Beispiele von Vielecken mit einer eingezeichneten Diagonale:
\begin{center}
\MTikzAuto{%
\begin{tikzpicture}[line width=1.5pt]
\begin{scope}[xshift=-5cm]
\draw[dotted,color=blue] (200:1.6cm) -- (110:2cm);
%\draw (0,0) circle(1pt);
\draw (110:2cm) -- (150:2cm) -- (200:1.6cm) -- (250:1.6cm) -- (330:1cm) %
 -- (20:1cm) -- (50:1.6cm) -- cycle;
\end{scope}
%%
\begin{scope}[xshift=0cm]
\draw[dotted,color=blue] (-1.2,2cm) -- (-0.6,-0.6);
%\draw (100:2cm) -- (240:1cm) -- (315:2.5cm) -- (0:0.5cm) -- (70:3cm) -- cycle;
\draw (-0.6,-0.6) -- (-0.6,2) -- (-1.2,2) -- (-1.2,-1.2) -- (1.2,-1.2) %
 -- (1.2,-0.6) -- cycle;
\end{scope}
%%
%\begin{scope}[xshift=2cm]
%\draw (0:2cm) %
% -- (15:2cm) -- (30:2cm) -- (45:2cm) -- (60:2cm) -- (75:2cm) -- (90:2cm) %
% -- (105:2cm) -- (120:2cm) -- (135:2cm) -- (150:2cm) %
% -- (165:2cm) -- (180:2cm) -- (195:2cm) -- (210:2cm) -- (225:2cm) -- (240:2cm) %
% -- (255:2cm) -- (270:2cm) -- (285:2cm) -- (300:2cm) -- (315:2cm) %
% -- (330:2cm) -- (345:2cm) -- cycle;
%\end{scope}
%%
\begin{scope}[xshift=5cm]
\draw[dotted,color=blue] (0,-2) -- (0,2);
\draw (0:2cm) -- (30:2cm) -- (60:2cm) -- (90:2cm) -- (120:2cm)  -- (150:2cm) %
 -- (180:2cm) -- (210:2cm) -- (240:2cm) -- (270:2cm) -- (300:2cm) 
 -- (330:2cm) -- cycle;
%\draw (0:2cm) -- (30:2cm) -- (60:2cm) -- (90:2cm) -- (120:2cm)  -- (150:2cm) %
% -- (180:2cm) -- (210:2cm) -- (240:2cm) -- (270:2cm) -- (300:2cm) 
% -- (330:2cm) -- cycle;
\end{scope}
\end{tikzpicture}
}
\end{center}

Here, such a detailed classification as for triangles or quadrilaterals is barely 
possible. The new possibilities such as the approximation of round shapes do also lead 
to new interesting questions. One does not consider single polygons but construction 
principles for a series of many polygons. On the other hand, every polygon can be 
divided into triangles if required, as we have seen already for quadrilaterals.
Thus, a property of a single vertex is often considered in terms of what this means
for the polygon in the whole.

For classification, the question is convenient whether a certain condition is satisfied by 
\textbf{all} vertices or not, and what this means for the polygons. For example, polygons 
are classified according to the magnitudes of their angles, e.g. whether all angles of the 
vertices are less than $\pi$ or $180\MGrad$. If so, all diagonals pass through the inside of the polygon. 
Otherwise, at least one diagonal exists in the outside.

The figure above shows examples of polygons exhibiting different properties. 
In the polygon to the left all (interior) angles are less than $\pi$ or $180\MGrad$. In this 
case the polygon is said to be convex. In contrast, the polygon in the middle contains a vertex 
with an angle greater than $\pi$ or $180\MGrad$. In the polygon to the right all angles are 
equal, leading to a very evenly shaped polygon.
%Au"serdem bieten Vielecke die M"oglichkeit, vielf"altige Fl"achen zu gestalten 
%und sogar runde Formen anzun"ahern.

\begin{MXInfo}{Polygons}\MLabel{VBKM05_Vielecke}%

Let $n$~points in the plane be given, where $n$ is a natural number with $n \geq 3$.
Here, we consider \MEntry{polygons}{polygon} constructed by connecting points by 
line segments such that a closed, non-self-intersecting (simple) path is formed, and 
every point is adjacent to exactly two segments, where every three points connected 
by successive segments are to be non-collinear. 

A polygon is also called \MEntry{$n$-gon}{$n$-gon}. 


\begin{itemize}
\item The $n$ points that are connected are called \MEntry{vertices}{vertex (polygon)}
of the polygon, and the $n$ connecting line segments are called \MEntry{sides}{side (polygon)}
of the polygon.
\item Every polygon can be divided into $(n-2)$ non-overlapping triangles. Hence, the sum of the 
interior angles of a polygon is $(n-2) \cdot \pi$ or $(n-2) \cdot 180\MGrad$.
\item Line segments connecting two vertices not adjacent to the same side of the 
polygon are called \MEntry{diagonals}{diagonal (polygon)} of the polygon.
\end{itemize}
\end{MXInfo}

Further statements hold for polygons with sides of equal length and equal 
interior angles. For $n=3$, these are equilateral triangles, end for 
quadrilaterals these are squares.

\begin{MXInfo}{Regular Polygons}
A polygon that is  equilateral (all sides have the same length) and
equiangular (all angles are equal in measure) is called 
\MEntry{regular polygon}{regular polygon} or 
\MEntry{regelar $n$-gon}{regular $n$-gon}.
\end{MXInfo}

Honeycombs are -- when seen from above -- approximately regular hexagons.
\begin{center}
\MTikzAuto{%
\begin{tikzpicture}[line width=1.5pt, declare function={Radius=3;}]
\begin{scope}[xshift=-4cm]
\foreach \x in {0,1,2,...,6}
 \coordinate (P\x) at ({\x*60}:Radius);
\coordinate (M) at (0,0);
%\draw[->] (-3,0) -- (3.5,0) node[below left] {$x$};
%\draw[->] (0,-3) -- (0,3) node[below left] {$y$};
%Sechseck:
\draw[color=yellow!50!red] (P0) -- (P1) -- (P2) -- (P3) -- (P4) -- (P5) -- cycle;
%Radius:
%\draw[style=dotted,color=black!50!white] (M) -- (P5);
%Mittelpunkt:
%\filldraw[color=black!50!white] (0,0) circle(2pt);
%Punkte des Sechsecks:
%\foreach \Punkt in {P0, P1, P2, P3, P4, P5}
% \filldraw[color=black!50!white] (\Punkt) circle(2pt);
\end{scope}
%%
\begin{scope}[xshift=4cm]
\foreach \x in {0,1,2,...,5}
{
 \coordinate (P\x) at ({\x*60}:Radius);
 \coordinate (SMP\x) at ({(\x*60)+30}:{1.1*Radius});
}
\coordinate (M) at (0,0);
%\draw[->] (-3,0) -- (3.5,0) node[below left] {$x$};
%\draw[->] (0,-3) -- (0,3) node[below left] {$y$};
%Sechseck:
\draw[color=yellow!50!red] (P0) -- (P1) -- (P2) -- (P3) -- (P4) -- (P5) -- cycle;
%Umkreis:
\draw[color=black!50!white] (M) circle(Radius);
%Seitenmittensenkrechten:
\foreach \x/\y in {0/3,1/4,2/5}
 \draw[style=dotted,color=red!50!white] (SMP\x) -- (SMP\y);
%Radien:
\foreach \x in {0,1,2,...,5}
 \draw[style=dotted,color=black!50!white] (M) -- (P\x);
%Mittelpunkt:
\filldraw[color=black!50!white] (0,0) circle(2pt);
%Punkte des Sechsecks:
\foreach \Punkt in {P0, P1, P2, P3, P4, P5}
 \filldraw[color=black!50!white] (\Punkt) circle(2pt);
\end{scope}
\end{tikzpicture}
}
\end{center}

Regular polygons have various symmetry properties. All lines perpendicular to the sides,
passing through the midpoint of the respective side, intersect in a point $M$. Reflecting 
a polygon across such a line maps it onto itself.

Furthermore, regular polygons have rotational symmetry, i.e. a $n$-gon maps onto itself
if it is rotated around $M$ by an angle of $\frac{2 \pi}{n}$.

The vertices of a regular polygon have all the same distance from $M$ and thus lie 
all on a circle around~$M$.
\end{MXContent}


\begin{MXContent}{Circumference}{Circumference}{STD}
\MDeclareSiteUXID{VBKM05_Umfang_Content}

The circumference of a polygon is the sum of the lengths of all its
line segments. If a polygon has further properties concerning the side lengths, 
more statements concerning the circumference can hold.

First, quadrilaterals are considered. If $a$ and $b$ are adjacent sides 
of a parallelogram, then its circumference is 
$U = a + b + a + b = 2 \cdot a \cdot b$.

For a rhombus (and therefore for a square), all four sides have the same length $a$ such
that its circumference is $U = 4 \cdot a$.

Likewise, for every regular polygon, all sides have the same length. If $n$ is the number 
of vertices and $a$ is the length of a side, then the circumference $U_n$ can simply be
calculated by $U_n = n \cdot a$.

As an outlook to trigonometric functions described in Section~\MRef{M05_Trigonometrie} 
the circumference of a regular polygon shall now be calculated in another way. 


\begin{center}
\MTikzAuto{%
\begin{tikzpicture}[line width=1.5pt, declare function={Radius=3; %
 cSeite={pow(3,0.5)/2*Radius};}]
\foreach \x in {0,1,2,5}
{
 \coordinate (P\x) at ({\x*60}:Radius);
 \coordinate (SMP\x) at ({(\x*60)+30}:{1.1*Radius});
}
\coordinate (M) at (0,0);
\coordinate (C) at (30:cSeite);
%Ausschnitt eines Sechsecks:
\draw[color=yellow!50!red] (P5) -- (P0) -- (P1) -- (P2);
%Winkel:
\filldraw[color=blue] ({0.4*Radius},0) arc(0:30:{0.4*Radius});
\node[right] at (15:{0.4*Radius}) {$\Mvarphi$};
%Umkreis:
\draw[color=black!50!white] (-75:Radius) arc(-75:135:Radius);
%Seitenmittensenkrechten:
\draw[color=red!50!white] (M) -- (C);
\draw[style=dotted,color=red!50!white] (C) -- (SMP0);
\foreach \x in {1,5}
 \draw[style=dotted,color=red!50!white] (M) -- (SMP\x);
%Radien:
 \draw[color=black!50!white] (M) -- node[below] {$r$} (P0);
\foreach \x in {1,2,5}
 \draw[style=dotted,color=black!50!white] (M) -- (P\x);
%Mittelpunkt:
\filldraw[color=black!50!white] (0,0) circle(2pt);
%Punkte des Sechsecks:
\foreach \Punkt in {P0, P1, P2, P5}
 \filldraw[color=black!50!white] (\Punkt) circle(2pt);
\node[below left] at (M) {$M$};
\node[below right] at (P0) {$A$};
\node[above right] at (P1) {$B$};
\node[left] at (C) {$C$};
\end{tikzpicture}
}
\end{center}

The vertices of 
a regular polygon all lie on a common circle with radius $r$. 
The angle $\Mvarphi$ between the line segments connecting the centre of the circle 
to the vertices $A$ and $B$ of a side is the $n$-th part of the complete angle:
$\Mvarphi = \frac{2 \pi}{n}$. The centre of the circle and the midpoint $C$ of the 
line segment $\MGeoStrecke{A}{B}$ form a right triangle $\MGeoDreieck{M}{A}{C}$
with the angle $\Mmeasuredangle(AMC) = \frac{1}{2} \cdot \Mvarphi = \frac{\pi}{n}$.
If the value of $a$ is calculated by 
\[
\sin(\Mmeasuredangle(AMC)) = \frac{\frac{1}{2} a}{r} %%
\]
and is inserted in $U = n \cdot a$, then we obtain the formula 
\[
U_n = n \cdot a = 2 \cdot r \cdot n \cdot \sin\left(\frac{\pi}{n}\right) %%
\]
for the circumference of a regular polygon. For example, 
$U_6 = 2 \cdot r \cdot 6 \cdot \frac{1}{2} = 6 \cdot r$. 
The larger $n$ is, the closer the circumference is to the value 
$2 \cdot r \cdot \pi \approx \MZahl{6}{283} \cdot r$ describing the 
circumference of a circle with radius $r$. This can be 
shown by means of more advanced methods of calculus, the basic ideas of which are introduced in Chapter~\MRef{VBKM07}. The approach described here is
based on the following idea: it is difficult to calculate the value 
of the circumference of a circle. Therefore, one looks for 
similar objects, in this case the regular polygons, with two properties:
their circumference can be calculated easily, and if the number of vertices 
is sufficiently large, then the circumference of the polygon differs from the 
circumference of a circle less than any given positive number (here, one 
thinks of ``small'' numbers). This approach can also be used to calculate 
the area of surfaces that are not bounded by line segments 
(see Chapter~\MRef{VBKM08}). For this purpose, it will be illustrated here how to calculate the area of polygons, which is in this respect relatively 
easy. Further, this can be used as the starting point of an approximation, as the 
figure above showing a circle inscribed into a hexagon suggests. 
\end{MXContent}


\begin{MXContent}{Area}{Area}{STD}
\MDeclareSiteUXID{VBKM05_Flaecheninhalt_Content}

The area of a surface equals the number of unit squares required to cover this surface 
completely.

\begin{tabular}{@{}l@{\hspace{1.5cm}}r@{}}
\begin{minipage}{8cm}

Let us first consider rectangles. If the sides of the rectangle are of
lengths $a$ and $b$, then the rectangle contains $b$ rows with $a$ unit squares, i.e. 
$b \cdot a$ unit squares.
\end{minipage}
&
\begin{minipage}{7cm}
\MTikzAuto{%
\def\sxyc{0.8cm}
\begin{tikzpicture}[x=\sxyc,y=\sxyc] 
\begin{scope}[yshift=-2.5cm]
\draw[help lines, black, xstep=1, ystep=1] (1,1) grid (8,5);
\draw[color=green!50!black, line width=2pt] (1,1)--(8,1) (1,5)--(8,5);
\draw[color=blue, line width=2pt] (1,1)--(1,5) (8,1)--(8,5);
\draw[color=green!50!black] (4.5,1) node[anchor=north] {$a$};
\draw[color=green!50!black] (4.5,5) node[anchor=south] {$a$};
\draw[color=blue] (1,3) node[anchor=east] {$b$};
\draw[color=blue] (8,3) node[anchor=west] {$b$};
\end{scope}
\end{tikzpicture}
}
\end{minipage}
\end{tabular}

\begin{MXInfo}{Area of a Rectangle}
The area $F$ of a rectangle with sides of lengths $a$ and $b$ is
\[
F = b\cdot a = a\cdot b \MDFPeriod %%
\]
\end{MXInfo}

\begin{tabular}{@{}lr@{}}
\begin{minipage}{10cm}

With this, the area of a right triangle can be calculated easily. Let 
$\MGeoDreieck{A}{B}{C}$ be a right triangle rotated by an angle of $180\MGrad$. 
If the original and the rotated triangle are merged along the hypotenuse,
one obtains a rectangle.
\end{minipage}
&
\begin{minipage}{5cm}
%\begin{center}
\MTikzAuto{%
\begin{tikzpicture}[rotate=-20]
\coordinate (A) at (0,0);
\coordinate (B) at ($ (A) + (1,-1.5) $);
\coordinate (C) at ($ (A) + (3, 2) $);
\coordinate (D) at ($ (B) + (C) - (A)$);
% \coordinate[label=right:$B_1$]      (B) at (4,-1);
\draw (A) node [left]{$A$} -- (B) node[left]{$B$} -- (C) node[right]{$C$} -- cycle;
\draw[dotted] (B) -- (D) node[right]{$D$} -- (C);
\end{tikzpicture}
}
%\end{center}
\end{minipage}
\end{tabular}

The area of the right triangle is then half the area of the rectangle, i.e. 
$ F = \frac{1}{2}\cdot a\cdot b$.

And how is the area calculated if the triangle is not right-angled?

Every triangle can be divided into two right triangles by drawing a line 
from one vertex to the opposite side such that this line is perpendicular to the side. 
This line is called the \textbf{altitude} $h_{i}$ of a triangle on a specific side $i$,
where $i$ is the index of the side $a$, $b$, or $c$. 

Depending on whether the new line is interior or exterior to the triangle,
the area of the triangle equals the sum or the difference of the areas of the two resulting 
right triangles:

\begin{center}
\MTikzAuto{%
\begin{tikzpicture}
 \coordinate[label=left:$A$]  (A) at (0,0);
 \coordinate[label=right:$B$] (B) at ($ (A) + (3,0) $);
 \coordinate[label=below:$D$] (D) at ($ (A)!0.3!(B) $);
 \coordinate[label=above:$C$] (C) at ($ (D) + (0,2) $);
 %
 \draw (A) -- node[below]{$c$} (B) -- (C) -- cycle;
 \draw[dotted] (C) -- node[right]{$h_c$} (D);
 \path (A) -- node[above]{$c_1$} (D) -- node[above]{$c_2$} (B);
\end{tikzpicture}
}
\hspace{4em}
\MTikzAuto{%
\begin{tikzpicture}
 \coordinate[label=below:$U$] (U) at (0,0);
 \coordinate[label=right:$V$] (V) at ($ (U) + (3,0) $);
 \coordinate[label=below:$X$] (X) at ($ (U)!-0.3!(V) $);
 \coordinate[label=above:$W$] (W) at ($ (X) + (0,2) $);
 %
 \draw (U) -- node[above]{$w$} (V) -- (W) -- cycle;
 \path (X) -- node[below]{$w_2$} (V);
 \draw[dotted] (W) -- node[left] {$h_w$} (X);
 \draw[dotted] (U) -- node[above]{$w_1$} (X);
\end{tikzpicture}
}
\end{center}

Thus, on the left, we have (if $F_{\Delta}$ is the area of the triangle~$\Delta$)
\[
   F_{\MGeoDreieck{A}{B}{C}}
 = F_{\MGeoDreieck{D}{B}{C}} + F_{\MGeoDreieck{A}{D}{C}}
 = \Mtfrac{1}{2} \cdot h_c \cdot c_2 + \Mtfrac{1}{2} \cdot h_c \cdot c_1
 = \Mtfrac{1}{2} \cdot h_c \cdot \left( c_2 + c_1 \right)
 = \Mtfrac{1}{2} \cdot h_c \cdot c \MDFPeriod
\]
On the right, we have
\[
   F_{\MGeoDreieck{U}{V}{W}}
 = F_{\MGeoDreieck{X}{V}{W}} - F_{\MGeoDreieck{X}{U}{W}}
 = \Mtfrac{1}{2} \cdot h_w \cdot w_2 - \Mtfrac{1}{2} \cdot h_w \cdot w_1
 = \Mtfrac{1}{2} \cdot h_w \cdot \left( w_2 - w_1 \right)
 = \Mtfrac{1}{2} \cdot h_w \cdot w \MDFPeriod
\]
Thus the area can always be calculated from the length of one side and the 
length of the altitude perpendicular to the corresponding side.

\begin{MXInfo}{Area of a Triangle}

The area $F_{\MGeoDreieck{A}{B}{C}}$ of a triangle equals half the product 
of the length of a side and the length of the corresponding altitude of the triangle:
       \[
          F_{\MGeoDreieck{A}{B}{C}}
        = \frac{1}{2} \cdot a \cdot h_a
        = \frac{1}{2} \cdot b \cdot h_b
        = \frac{1}{2} \cdot c \cdot h_c \MDFPeriod
       \]
Here, the \MEntry{altitude of a triangle on a side}{altitude of a triangle} 
denotes the line segment from the vertex opposite the side 
to the line containing the side itself, perpendicular to this side. 
%Der Punkt, auf dem die H\"ohe diese Gerade trifft, hei"st
%       \MEntry{H"ohenfu"spunkt}{H"ohenfu"spunkt} der H\"ohe.
\end{MXInfo}


\begin{MExample}
\begin{tabular}{@{}l@{\hspace{2cm}}r@{}}
\begin{minipage}{8cm}
For the triangle to the right, the altitude corresponding to the side of length  
$\MZahl{8}{6}$ is given. The given values are rounded numerical values.
Hence, the area $F$ of the triangle is approximately 
\[ F = \frac{\MZahl{8}{6}\cdot\MZahl{5}{5}}{2}=\MZahl{23}{65} \MDFPeriod\]
\end{minipage}
&
\begin{minipage}{7cm}
\MTikzAuto{%
\begin{tikzpicture}[x=0.6cm, y=0.6cm] 
%%\draw[help lines, gray!50, xstep=0.5, ystep=0.5] (0,0) grid (9,8);
\draw[color=black, very thick] (0,0) -- (1.7032,-6.0654) -- (7.8,0) -- cycle;
\draw[color=black, thick] (0,0) -- (3.87986,-3.89994);
\draw[color=black] (3.9,0) node[anchor=south] {\large $7{,}8$};
\draw[color=black] (0.85160,-3.0327) node[anchor=north east] {\large $6{,}3$};
\draw[color=black] (4.7516,-3.0327) node[anchor=north west] {\large $8{,}6$};
\draw[color=black] (1.93993,-1.94997) node[anchor=south west] {\large $5{,}5$};
\end{tikzpicture}
}
\end{minipage}
\end{tabular}
\end{MExample}

\begin{MExercise}
Calculate the area of the triangle below.


\begin{center}
\MTikzAuto{%
\begin{tikzpicture}[x=1.2cm, y=1.2cm] 
%Koordinatensystem
\node (xMAX) at (6.0,0){};
\node (yMAX) at (0,3.2){};
\draw[help lines, gray, dashed, xstep=1, ystep=1] (0,0) grid (5.5,2.8);
\draw[->,color=black] (-0.4,0) -- (xMAX);
\foreach \x in {1, 2, 3, 4, 5}
\draw[shift={(\x,0)},color=black] (0pt,2pt) -- (0pt,-2pt) node[below] {\normalsize $\x$};
\draw[->,color=black] (0,-0.4) -- (yMAX);
\foreach \y in {1,2}
\draw[shift={(0,\y)},color=black] (2pt,0pt) -- (-2pt,0pt) node[left] {\normalsize $\y$};
%Achsenbeschriftung
\draw (xMAX) node[anchor=north east] {$x$};
\draw (yMAX) node[anchor=north east] {$y$};
%Beschriftung und Graphen
%%\clip(-2.8,-0.5) rectangle (6,3);
\draw[color=black, very thick] (1,0) -- (4,2) -- (5,0) -- cycle;
%%\draw[color=black] (7,3) node[anchor=south west] {$\MPointTwo{7}{3}$};
\end{tikzpicture}
}
\end{center}

\begin{MHint}{Solution}
For this triangle, one altitude can be read off easily, namely the altitude perpendicular  
the side on the $x$-axes. The length $h$ of this altitude is $h = 2$,
and the length of the corresponding side is $c = 5 - 1 = 4$. Hence, the area $F$ of the 
triangle is 
$F = \frac{1}{2} \cdot c \cdot h = \frac{1}{2} \cdot 4 \cdot 2 = 4$.
\end{MHint}
\end{MExercise}

Using the formula for the area of triangles, areas of polygons can also be 
calculated. This is due to the fact that every polygon can be divided into 
triangles by adding diagonals to the polygon until all subareas are triangles. However, 
the considerations will remain restricted here to a few simple shapes. In the
following example, the polygon can be divided into a triangle and a rectangle. 
As a result, the calculation will be particularly easy. 

\begin{MExample}
\begin{tabular}{@{}l@{\hspace{2cm}}r@{}}
\begin {minipage}{8cm}
Consider the polygon to the right, namely a trapezoid. In this example,
the polygon can be divided into a right triangle with the legs $\left(a-c\right)$
and $b$ and the hypotenuse $d$ as well as a rectangle with sides of length $b$ 
and $c$.
\par
\vspace*{0.5cm}
Then, the area of the polygon is:
\end{minipage}
&
\begin{minipage}{7cm}
\MTikzAuto{%
\begin{tikzpicture}[x=0.4cm, y=0.4cm] 
\draw[thick] (0,0) -- (9,0) -- (9,9) -- (4,9) -- cycle;
\draw[thick, dashed] (4,0) -- (4,9);
\node[anchor=north] at (4.5,0) {$a$};
\node[anchor=west] at (9,4.5) {$b$};
\node[anchor=south] at (6.5,9) {$c$};
\node[anchor=south east] at (2.0,4.5) {$d$};
\end{tikzpicture}
}
\end{minipage}
\end{tabular}
\[
F = F_{\text{triangle}} + F_{\text{rectangle}} %
  = \frac{1}{2}\left( a-c \right)\cdot b + b\cdot c %
  = \frac{1}{2}ab-\frac{1}{2}bc+bc %
  =\frac{1}{2}\left(a+c\right)\cdot b \MDFPeriod
\]
\end{MExample}

\begin{MExercise}

\begin{tabular}{@{}l@{\hspace{0.6cm}}r@{}}\\
\begin{minipage}{7cm}
Calculate the area of the \textbf{parallelogram} 
to the right for $a=4$ and $h=5$.
\par
\begin{MHint}{Hint}
Divide the parallelogram appropriately and look at the resulting 
triangles carefully!
\end{MHint}
\par
\vspace*{1.5cm}
\end{minipage}
&
\begin{minipage}{7cm}
\MTikzAuto{%
\begin{tikzpicture}[x=0.5cm, y=0.5cm] 
\draw[thick] (0,0) -- (10,0) -- ++(45:10) -- (45:10) -- cycle;
\draw[stealth'-stealth',thick] (8,0) -- (8,7.0710678);
\node[anchor=north] at (5,0) {$a$};
\node[anchor=west] at (8,3.5355339) {$h$};
\end{tikzpicture}
}
\end{minipage}
\end{tabular}

\begin{MHint}{Solution}

\begin{tabular}{@{}lr@{}}
\begin{minipage}[b]{7cm}
The parallelogram can be divided into the left red rectangle, a rectangle, and 
the right triangle. Shifting the left red triangle to the right one obtains 
a rectangle with sides of lengths $a$ and $h$. Then, the area of the parallelogram is
\[F=a\cdot h=4\cdot 5=20 \MDFPeriod\]
\end{minipage}
&
\MTikzAuto{%
\begin{tikzpicture}[x=0.5cm, y=0.5cm] 
\draw[thick] (0,0) -- (10,0) -- ++(45:10) -- (45:10) -- cycle;
\draw[thick,dashed] (10,0) -- (10,7.0710678);
\draw (7.0710678,0) -- (7.0710678,7.0710678);
\draw[red,thick] (45:10) -- (0,0) -- (7.0710678,0);
\draw[red,thick,dashed] (7.0710678,0) -- (7.0710678,7.0710678);
\draw[red,thick,dashed] (10,0) -- ++(7.0710678,0) -- ++(0,7.0710678) -- cycle;
\node[anchor=north] at (5,0) {$a$};
\node[anchor=west] at (10,3.5355339) {$h$};
\end{tikzpicture}
}
\end{tabular}
\end{MHint}
\end{MExercise}

Finally, we will calculate the area of a circle. Info 
Box~\MRef{Kreiszahl} introduced the number $\pi$ describing the 
ratio of the circumference of a circle to its radius. The formula
for the area of the circle also involves $\pi$.

\begin{MXInfo}{Area of a Circle}
The area of a circle with radius $r$ is
\[ F =\pi\cdot r^2 \MDFPeriod\]
\end{MXInfo}

\begin{MExample}
Let the area of a circle with radius $r=2$ be $\MZahl{12}{566}$. 
This fact can be used to calculate an approximate value 
of the number $\pi$:
We have $F =\pi\cdot r^2$, hence $\pi = \frac{F}{r^2}$.
Inserting the given values results in the approximate value 
\[
\pi = \frac{F}{r^2} \approx \frac{\MZahl{12}{566}}{4} %
 =\MZahl{3}{1415} \MDFPeriod %%
\]
\end{MExample}

\end{MXContent}


\begin{MExercises}
\MDeclareSiteUXID{VBKM05_Flaecheninhalt_Exercises}

\begin{MExercise}
\begin{tabular}{@{}lr@{}}
\begin{minipage}[b]{8cm}
Calculate the area of the polygon to the right.\\
\vspace{3.5cm}
\end{minipage}
&
\MTikzAuto{%
\begin{tikzpicture}[x=1.0cm, y=1.0cm] 
\draw[color=black, thick] (0.0,0.0) -- (3.0,-1.8) -- (5.4,0.0) -- 
(4.0,2.6) -- (1.5,2.2)-- cycle; 
\draw[color=black,style=dotted] (0.0,0.0) -- (5.4,0.0) (1.5,0.0) -- (1.5,2.2)
(4.0,0.0) -- (4.0,2.6) (3.0,0.0) -- (3.0,-1.8);
\draw[color=blue] (1.1,0.8) node {\large $F_1$};
\draw[color=blue] (2.75,1.2) node {\large $F_2$};
\draw[color=blue] (4.4,0.9) node {\large $F_3$};
\draw[color=blue] (3.0,-0.5) node[fill=white] {\large $F_4$};
\draw[color=black] (0.75,0.0) node[anchor=south] {\large $15$};
\draw[color=black] (2.75,0.0) node[anchor=south] {\large $25$};
\draw[color=black] (4.7,0.0) node[anchor=south] {\large $14$};
\draw[color=black] (1.5,1.5) node[anchor=west] {\large $22$};
\draw[color=black] (4.0,1.8) node[anchor=east] {\large $26$};
\draw[color=black] (3.0,-1.1) node[anchor=west] {\large $18$};
\end{tikzpicture}
}
\end{tabular}
\begin{MHint}{Solution}
The values of the indicated subareas are calculated separately.
\begin{itemize}
\item
$F_1$ is a triangle: $F_1=\frac{15\cdot 22}{2}=165$.
\item
$F_2$ is a trapezoid that can be divided into two triangles with the altitude
 $25$: $F_2 = \frac{22\cdot 25}{2}+\frac{26\cdot 25}{2} = 275+325 = 600$.
\item
$F_3$ is a triangle: $F_3=\frac{14\cdot 26}{2}=182$.
\item
The surface $F_4$ is also a triangle: $F_4 = \frac{(15+25+14)\cdot 18}{2} = 486$.
\end{itemize}
Finally, we obtain the area of the entire polygon by summing up all these subareas: 
$F_1+F_2+F_3+F_4 = 165+600+182+486 = 1433$.
\end{MHint}
\end{MExercise}

\end{MExercises}

%end of content: section 4: Vielecke, Flaechen und Umfang.




%jgl: Neuer Abschnitt:
\MSubsection{Simple Geometric Solids}%
\MLabel{M05_Koerper}

\begin{MIntro}
\MDeclareSiteUXID{VBKM05A_ElementareKoerper_Intro}

The shapes of common objects such as a notepad or a mobile phone as well as of
technical structures such as tunnels can be described by simple basic solids,
apart form the ``rounded vertices''. Why is that?

If a broom is moved straight across a plane floor covered with dust, 
a rectangular section of the clean floor becomes visible. Geometrically idealised, 
a rectangle is formed by shifting a line segment (the broom). If the 
broom is rotated, a circle can be created. In this way, from simple 
objects more complex objects are constructed that can nevertheless be described easily.

\end{MIntro}

\begin{MXContent}{Simple Geometric Solids}{Simple Geometric Solids}{STD}%
\MDeclareSiteUXID{VBKM05A_ElementareKoerper_Content}

Points are the simplest basic geometric objects. Translations of points result in
line segments, and transformations such as translation or rotation of 
line segments result in simple geometric figures. For example, polygons and circles 
are obtained in a way described above. 

If figures are shifted or rotated out of their plane, then new objects are 
created that are denoted as solids. In this section some simple solids 
will be described, whose shapes can be identified easily in many everyday objects 
and technical constructions.

%Beispiel:
\begin{MExample}
Let us consider a rectangle and shift it perpendicular to the drawing plane.
In this way, a rectangular cuboid (or informally a rectangular box) is constructed. 
Its surface consists of the given rectangle and a copy of that (two faces). Four further rectangles (faces) are formed by the four sides of the given rectangle.
\end{MExample}

Taking any polygon and shifting it perpendicular to the drawing plane, results in a solid 
that is called a prism. The term also denotes a transparent optical element 
of this shape used to refract light waves. Because the angle of refraction
depends on the wavelength (i.e. the colour of the light), the different wave lengths of seemingly white 
light are refracted differently. In this way, the different colours of white light become visible.

\begin{MXInfo}{Prism}

Let a polygon $G$ be given. A \MEntry{prism}{prism} is a solid resulting from 
a perpendicular translation of a polygon $G$ by a line segment of length $h$.
The two faces, i.e. the given polygon and the shifted copy of this polygon,
are then called base faces. They are parallel to each other. All other 
faces together form the lateral surface $M$.

%Zeichnung Prisma:
\begin{center}
\MTikzAuto{%
\tdplotsetmaincoords{60}{30}
\begin{tikzpicture}[tdplot_main_coords,>=latex]
\def\zt{3}
\def\ticl{0.8}
\def\dsth{0.5}
\foreach \zc in {0,\zt} 
% coordinate number, radius, azimuth
\foreach \nc/\rc/\ac in {0/3/0,1/3/120,2/1/240} {
\pgfmathparse{\rc*cos(\ac)}\let\ax=\pgfmathresult
\pgfmathparse{\rc*sin(\ac)}\let\ay=\pgfmathresult
\node (PC-\nc-\zc) at (\ax,\ay,\zc) {};
}
%,name path=level-ground
\path[fill=gray!60,opacity=0.5] (PC-0-0.center) 
\foreach \nc in {1,2} { -- (PC-\nc-0.center) {} } -- cycle;
\draw[thick, black] (PC-2-0.center) -- (PC-0-0.center);
\draw[thick, black, dashed] (PC-0-0.center) --  (PC-1-0.center) -- (PC-2-0.center) ;
\foreach \nc in {0,2} { \draw[thick] (PC-\nc-0.center) -- (PC-\nc-\zt.center); };
\foreach \nc in {1} { \draw[thick,dashed] (PC-\nc-0.center) -- (PC-\nc-\zt.center); };
\path[draw=black,thick,fill=gray!30,fill opacity=0.2] (PC-0-\zt.center) 
\foreach \nc in {1,2} { -- (PC-\nc-\zt.center) {} } -- cycle;
\draw (PC-2-0.center) [tdplot_screen_coords] -- +(-\ticl,0);
\draw (PC-2-\zt.center) [tdplot_screen_coords] -- +(-\ticl,0);
\path (PC-2-0.center) [tdplot_screen_coords] ++(-\dsth,0) coordinate (BG);
\path (PC-2-\zt.center) [tdplot_screen_coords] ++(-\dsth,0) coordinate (BT);
\path ($ (BG)!0.5!(BT) $) node (TXT) {$h$}; % [fill=white]
\draw[<-] (BG) -- (TXT.south);
\draw[->] (TXT.north) -- (BT);
%%\tdplottransformmainscreen{\ax}{\ay}{\az}
%%\pgfpathmoveto{\pgfpoint{\tdplotresx cm}{\tdplotresy cm}}
\end{tikzpicture}}
\end{center}

The figure above shows a prism with a triangle as its base. The other faces adjacent
to the base face are rectangles.
%Ende (Zeichnung).

The volume $V$ of the prism is the product of the area of the polygon $G$ and the height 
$h$: We have $V = G \cdot h$.

The area $O$ of the surface is the sum of twice the area of the base face $G$ and the 
area of the lateral surface $M$. If $U$ is the circumference of the given polygon, we
have $O = 2 \cdot G + M = 2 \cdot G + U \cdot h$.
\end{MXInfo}

In the introductory example a rectangular cuboid was described. Using the definition 
above, it can be considered as a special case of a prism, namely a prism with a rectangle 
as its base face. If all faces are squares, the prism is called a cube.

The construction principle can be varied in different ways. For example, the 
polygon can be replaced by a disk that is shifted. By a perpendicular translation 
of the disk a solid is created that is especially symmetric, namely a cylinder. 
A tunnel drilling machine creates -- if considered in a simplified manner -- 
a cylindrical tube.

\begin{MXInfo}{Cylinder}
Let a disk $G$ be given. A \MEntry{cylinder}{cylinder} is a solid created 
by a perpendicular translation of a disk $G$ by a line segment $h$. The two 
faces, i.e. the given disk and its copy, are then called the base faces 
of the cylinder. They are parallel to each other. The curved part of the surface 
between the two disks forms the lateral face $M$ of the cylinder.


%Zeichnung Zylinder:
\begin{center}
\MTikzAuto{%
\tdplotsetmaincoords{60}{30}
\begin{tikzpicture}[tdplot_main_coords,>=latex]
\def\az{30}
%\def\pl{60}
\def\zt{3}
\def\rd{2}
\def\ticl{0.8}
\def\dsth{0.5}
\pgfmathparse{\rd*cos(\az)}\let\ax=\pgfmathresult
\pgfmathparse{\rd*sin(\az)}\let\ay=\pgfmathresult
\path[fill=gray!60,opacity=0.5] (0,0,0) circle (\rd);
\draw[thick, black,dashed] (\ax,\ay,0) arc (\az:{\az+180}:\rd);
\draw[thick, black] (-\ax,-\ay,0) arc ({\az+180}:{\az+360}:\rd);
\draw (-\ax,-\ay,0) [tdplot_screen_coords] -- +(-\ticl,0);
\draw (-\ax,-\ay,\zt) [tdplot_screen_coords] -- +(-\ticl,0);
\path (-\ax,-\ay,0) [tdplot_screen_coords] ++(-\dsth,0) coordinate (DHB);
\path (-\ax,-\ay,\zt) [tdplot_screen_coords] ++(-\dsth,0) coordinate (DHT);
\draw[thin,black,dashed] (0,0,0) -- +(0,0,\zt);
% fadings
\shade [opacity=0.5,left color=transparent!0,right color=transparent!50]
(\ax,\ay,0) arc ({\az+360}:{\az+180}:\rd)
-- (-\ax,-\ay,\zt) arc ({\az+180}:{\az+360}:\rd) -- cycle;
%
\draw[thin,black,dashed] (\ax,\ay,0) -- +(0,0,\zt) (-\ax,-\ay,0) -- +(0,0,\zt);
\path[fill=gray!30,fill opacity=0.2] (0,0,\zt) circle (\rd);
\draw[thick, black] (0,0,\zt) circle (\rd);
\path ($ (DHB)!0.5!(DHT) $) node (TXH) {$h$}; % [fill=white]
\draw[<-] (DHB) -- (TXH.south);
\draw[->] (TXH.north) -- (DHT);
\path (-\ay,\ax,\zt) [tdplot_screen_coords] ++(0,\ticl) coordinate (TRL)
++(\rd,0) coordinate (TRR);
\path (-\ay,\ax,\zt) [tdplot_screen_coords] ++(0,\dsth) coordinate (DRL)
++(\rd,0) coordinate (DRR);
\draw (0,0,\zt) -- (TRL) (\ax,\ay,\zt) -- (TRR);
\path ($ (DRL)!0.5!(DRR) $) node (TXR) {$r$}; % [fill=white]
\draw[<-] (DRL) -- (TXR.west);
\draw[->] (TXR.east) -- (DRR);
\end{tikzpicture}
}
\end{center}
%Ende (Zeichnung).

The volume $V$ of the cylinder is the product of the area of the disk 
$G$ with radius $r$ and the height $h$ of the cylinder: $V = G \cdot h = \pi \cdot r^2 \cdot h$.

The area $O$ of the surface is the sum of twice the area of the disk $G$ and 
the area of the lateral surface $M$. With the circumference $U = 2 \cdot \pi \cdot r$
of the disk we have $O = 2 \cdot G + M = 2 \cdot \pi \cdot r^2 + 2 \cdot \pi \cdot r \cdot h %
 = 2 \cdot \pi \cdot r \cdot (r + h)$.
\end{MXInfo}


If the disk is not translated but rotated, where the axis of rotation passes 
trough the centre of the disk and one of its boundary points, then the resulting 
solid is a sphere.


\begin{MXInfo}{Sphere}
Let a disk with centre $M$ and radius $r$ be given. If the disk $M$ is 
rotated around an axis trough $M$ and a boundary point of the disk, the resulting 
solid is a sphere with radius $r$.

%Zeichnung Kugel:
\begin{center}
\MTikzAuto{%
\tdplotsetmaincoords{60}{30}
\begin{tikzpicture}[tdplot_main_coords,>=latex]
\def\az{30}
\def\pl{60}
\def\rd{2.5}
\def\ticl{0.8}
\def\dsth{0.5}
\pgfmathparse{\rd*cos(\az)}\let\ax=\pgfmathresult
\pgfmathparse{\rd*sin(\az)}\let\ay=\pgfmathresult
% fadings
\begin{scope}[tdplot_screen_coords]
  \shade[ball color=gray!50] (0,0,0) circle (\rd);
\end{scope}
\draw[thin,black,dashed] (0,0,0) -- +(0,0,\rd);
\draw[thick, black,dashed] (\ax,\ay,0) arc (\az:{\az+180}:\rd);
\begin{scope}[tdplot_screen_coords]
  \fill[opacity=0.5,color=white] (0,0,0) circle (\rd);
  \draw[black,dashed] (0,0,0) circle (\rd);
\end{scope}
\draw[thick,black,dashed] (-\ax,-\ay,0) arc ({\az+180}:{\az+360}:\rd);
%
\draw (-\ax,-\ay,0) [tdplot_screen_coords] -- +(-\ticl,0);
\draw (-\ax,-\ay,\rd) [tdplot_screen_coords] ++(-\ticl,0) 
[tdplot_main_coords] -- (0,0,\rd);
\path (-\ax,-\ay,0) [tdplot_screen_coords] ++(-\dsth,0) coordinate (DHB);
\path (-\ax,-\ay,\rd) [tdplot_screen_coords] ++(-\dsth,0) coordinate (DHT);
\path ($ (DHB)!0.5!(DHT) $) node (TXH) {$r$}; % [fill=white]
\draw[<-] (DHB) -- (TXH.south);
\draw[->] (TXH.north) -- (DHT);
\end{tikzpicture}
}
\end{center}
%Ende (Zeichnung).

The volume $V$ of the sphere is $V = \frac{4}{3} \cdot \pi \cdot r^3$.

The area $O$ of the surface is given by
$O = 4 \cdot \pi \cdot r^2$.
\end{MXInfo}

A sphere can also be described as a solid consisting of all points that
have a distance less than or equal to $r$ from $M$ (see also Chapter~\MRef{VBKM10}).

In this approach, a prism is a solid consisting of all points that 
lie on a connecting line between the base face and its copy. 

In the following, two variations of this approach will be considered. 
We start again with a polygon as base face. Moreover, instead of a copy of the 
base face, only a point is given.

\begin{MXInfo}{Pyramid}
Let a polygon $G$ and a point $S$ with distance $h > 0$ from $G$ be given. 
A pyramid with the base $G$ and the apex $S$ is a solid consisting of all 
points lying on a line segment between $S$ and a point of the base face $G$.

%Zeichnung Pyramide:
\begin{center}
\MTikzAuto{%
\tdplotsetmaincoords{60}{40}
\begin{tikzpicture}[tdplot_main_coords,>=latex]
\def\zt{5}
\def\rd{2.5}
\def\ticl{0.8}
\def\dsth{0.5}
% coordinate number, radius, azimuth
\foreach \nc/\rc/\ac in {0/\rd/0,1/\rd/120,2/\rd/240} {
\pgfmathparse{\rc*cos(\ac)}\let\ax=\pgfmathresult
\pgfmathparse{\rc*sin(\ac)}\let\ay=\pgfmathresult
\node (PC-\nc) at (\ax,\ay,0) {};
}
%
\path[fill=gray!60,opacity=0.5] (PC-0.center) 
\foreach \nc in {1,2} { -- (PC-\nc.center) {} } -- cycle;
\draw[thick, black] (PC-2.center) -- (PC-0.center);
\draw[thick, black, dashed] (PC-0.center) --  (PC-1.center) -- (PC-2.center);
\foreach \nc in {0,2} { \draw[thick] (PC-\nc.center) -- (0,0,\zt); };
\foreach \nc in {1} { \draw[thick,dashed] (PC-\nc.center) -- (0,0,\zt); };
\draw[thin,black,dashed] (0,0,0) -- +(0,0,\zt);
% Legende
\draw (0,0,0) [tdplot_screen_coords] -- ++({-\rd-\ticl},0);
\draw (0,0,\zt) [tdplot_screen_coords] -- ++({-\rd-\ticl},0);
\path (0,0,0) [tdplot_screen_coords] ++({-\rd-\dsth},0) coordinate (BG);
\path (0,0,\zt) [tdplot_screen_coords] ++({-\rd-\dsth},0) coordinate (BT);
\path ($ (BG)!0.5!(BT) $) node (TXT) {$h$}; % [fill=white]
\draw[<-] (BG) -- (TXT.south);
\draw[->] (TXT.north) -- (BT);
%
\path (PC-2.center) [tdplot_screen_coords] +(0,-\dsth) coordinate (DAL)
+(0,-\ticl) coordinate (TAL);
\path (PC-0.center) [tdplot_screen_coords] +(0,-\dsth) coordinate (DAR)
+(0,-\ticl) coordinate (TAR);
\draw (PC-2.center) -- (TAL) (PC-0.center) -- (TAR);
\path ($ (DAL)!0.5!(DAR) $) node (TXA) {$a$}; % [fill=white]
\draw[<-] (DAL) -- (TXA.west);
\draw[->] (TXA.east) -- (DAR);
\end{tikzpicture}
}
\end{center}

The figure above shows a pyramid with a triangular base face.
%Ende (Zeichnung).

The volume $V$ of the pyramid is proportional to the area of the base face 
$G$ and the height $h$: $V = \frac{1}{3} \cdot G \cdot h$.

The area $O$ of the surface is the sum of the area of the base face $G$ and 
the area of the lateral surface $M$, where the area of the lateral surfaces 
is the sum of the areas of its triangular faces $D_k$ ($1 \leq k \leq n$).
Thus, we have $O = G + M = G + D_1 + \ldots + D_n$.
\end{MXInfo}

In special situations one obtains simple formulas that can be used to 
calculate the volume and the surface area of the solid. One example is 
the pyramid shown above. There, the base face is an equilateral triangle. 
The following exercise illustrates how to derive a formula for the 
surface area of a special case of a pyramid from the properties of 
equilateral triangles.  

\begin{MExercise}
Calculate the surface area $O$ of a pyramid whose faces are all equilateral 
triangles with sides of length $a$.

Answer: 
%\begin{MExerciseItems}
%\item \MEquationItem{$V$}{\MLSimplifyQuestion{30}{ ? }{10}{a}{4}{32}{ExM05TPyr1}}
%\item 
\MEquationItem{$O$}{\MLSimplifyQuestion{30}{a*a*sqrt(3)}{10}{a}{4}{32}{ExM05TPyr2}}
%\end{MExerciseItems}

\begin{MHint}{Solution}
A pyramid whose faces are all equilateral triangles has in total four faces: 
a triangular base face and three further adjacent faces. Since all faces are equal,
the surface area of this pyramid is given by $O = 4 \cdot F$, where $F$ is the 
area if a single equilateral triangle. The height (altitude) $\ell$ of an equilateral
triangle with sides of length $a$ is, according to Pythagoras' theorem,
\[
a^2 = \ell^2 + \left(\frac{a}{2}\right)^2 \, , %%
\]
equal to
$\ell = \sqrt{a^2 - \frac{1}{4} \cdot a^2} %
 = \frac{1}{2} \cdot a \cdot \sqrt{3}$.
Hence,
$F = \frac{1}{2} \cdot a \cdot \ell = \frac{1}{4} \cdot a^2 \cdot \sqrt{3}$, 
so
$O = 4 \cdot F = a^2 \cdot \sqrt{3}$.%
\end{MHint}
\end{MExercise}

The considerations above, that a prism and a cylinder share the same 
constructing principle for different base faces, can be applied to the new 
situation of a pyramid as well. One obtains another solid if instead of a polygon 
(as for the case of a pyramid) a disk is now used as base face. 

\begin{MXInfo}{Cone}
Let a disk $G$ with radius $r$ and a point $S$ with distance $h > 0$ from $G$ be given.
A cone with the base face $G$ and the apex $S$ is the solid consisting of
all points lying on a line segment between $S$ and a point of the base face $G$.

%Zeichnung Kegel:
\begin{center}
\MTikzAuto{
\def\az{30}
\def\pl{60}
\tdplotsetmaincoords{\pl}{\az}
\begin{tikzpicture}[tdplot_main_coords,>=latex]
\def\zt{4}
\def\rd{2}
\def\ticl{0.8}
\def\dsth{0.5}
\pgfmathparse{\rd*cos(\az)}\let\ax=\pgfmathresult
\pgfmathparse{\rd*sin(\az)}\let\ay=\pgfmathresult
\pgfmathparse{\rd/\zt/tan(\pl)}\let\stm=\pgfmathresult
\pgfmathparse{less(abs(\stm),1)}\let\cres\pgfmathresult
\pgfmathparse{greater(\stm,0)}\let\crsg\pgfmathresult
\def\ccmp{1}
%%\typeout{cres=\cres, crsg=\crsg}
\if\cres\ccmp
  \pgfmathparse{\rd*cos(\pl)*\stm}\let\smy=\pgfmathresult
  \pgfmathparse{\rd*sqrt(1-\stm*\stm)}\let\smx=\pgfmathresult
  \pgfmathparse{asin(\stm)}\let\astm=\pgfmathresult
\else
  \def\smy{1}\def\smx{0}\def\astm{90}
\fi
\pgfmathparse{\rd*cos(\az+\astm)}\let\axl=\pgfmathresult
\pgfmathparse{\rd*sin(\az+\astm)}\let\ayl=\pgfmathresult
\path[fill=gray!60,opacity=0.5] (0,0,0) circle (\rd);
%%\draw[thick,black,dashed] (\ax,\ay,0) arc (\az:{\az+180}:\rd);
%%\draw[thick,black] (-\ax,-\ay,0) arc ({\az+180}:{\az+360}:\rd);
\if\crsg\ccmp
  \draw[thick,black,dashed] (\axl,\ayl,0) arc ({\az+\astm}:{\az+180-\astm}:\rd);
  \draw[thick,black] (\axl,\ayl,0) arc ({\az+360+\astm}:{\az+180-\astm}:\rd);
\else
  \draw[thick,black] (0,0,0) circle (\rd);
\fi
\draw (-\ax,-\ay,0) [tdplot_screen_coords] -- +(-\ticl,0);
\draw (-\ax,-\ay,\zt) [tdplot_screen_coords] ++(-\ticl,0) 
[tdplot_main_coords] -- (0,0,\zt);
\path (-\ax,-\ay,0) [tdplot_screen_coords] ++(-\dsth,0) coordinate (DHB);
\path (-\ax,-\ay,\zt) [tdplot_screen_coords] ++(-\dsth,0) coordinate (DHT);
\draw[thin,black,dashed] (0,0,0) -- +(0,0,\zt);
% fadings
\if\cres\ccmp
  \shade [opacity=0.5,left color=transparent!0,right color=transparent!50]
  (\axl,\ayl,0) arc ({\az+360+\astm}:{\az+180-\astm}:\rd) -- (0,0,\zt) -- cycle;
\else
  \shade [opacity=0.5,left color=transparent!0,right color=transparent!50]
  (0,0,0) circle (\rd);
\fi
%
\if\cres\ccmp
  \draw[thin,black,dashed] (0,0,\zt) [tdplot_screen_coords] -- (\smx,\smy);
  \draw[thin,black,dashed] (0,0,\zt) [tdplot_screen_coords] -- (-\smx,\smy);
\fi
\path ($ (DHB)!0.5!(DHT) $) node (TXH) {$h$}; % [fill=white]
\draw[<-] (DHB) -- (TXH.south);
\draw[->] (TXH.north) -- (DHT);
\path (0,0,\zt) [tdplot_screen_coords] ++(0,\ticl) coordinate (TRL)
++(\rd,0) coordinate (TRR);
\path (0,0,\zt) [tdplot_screen_coords] ++(0,\dsth) coordinate (DRL)
++(\rd,0) coordinate (DRR);
\draw (0,0,\zt) -- (TRL) (\ax,\ay,0) -- (TRR);
\path ($ (DRL)!0.5!(DRR) $) node (TXR) {$r$}; % [fill=white]
\draw[<-] (DRL) -- (TXR.west);
\draw[->] (TXR.east) -- (DRR);
\end{tikzpicture}
}
\end{center}
%Ende (Zeichnung).

The volume $V$ of the cone is proportional to the area of the disk $G$ and 
its height $h$. We have $V = \frac{1}{3} \cdot G \cdot h %
 = \frac{1}{3} \cdot \pi \cdot r^2 \cdot h$.

A cone whose apex is perpendicularly above the centre of the disk is called 
\MEntry{right circular cone}{circular cone, right}.

The area of the surface of a right circular cone is the sum of the 
area of the disk $G$ and the area of the lateral surface $M$. If $\ell$
is the distance of the apex from the boundary of the disk, then with the 
circumference of a circle $U =  2 \pi r$ we have
$O = G + M = \pi \cdot r^2 + \pi \cdot r \cdot \ell %
 = \pi \cdot r \cdot (r + \ell)$.
\end{MXInfo}

In the figure above the height $h$ of the cone is given. However, 
the formulas given above contain the distance between the apex and 
the boundary of the disk. How these quantities are related?

\begin{MExercise}
Let's imagine that the right circular cone is divided into two equal pieces by a plane 
trough the apex perpendicular to the base face. Then, one face of the two pieces is a triangle
whose sides are defined by the height $h$, the distance $\ell$, and the radius $r$ or the diameter of 
the circle, respectively. Here, $r$ and $h$ are given.

\begin{MExerciseItems}
\item
Describe $\ell$ as a function of $h$ and $r$:
\par
\MEquationItem{$\ell$}{\MLSimplifyQuestion{30}{sqrt(r*r + h*h)}{10}{r,h}{4}{32}{ExM05Zyla}}

\begin{MHint}{Solution}
The length of $\ell$ is the hypotenuse of the triangle whose legs are the height
$h$ of the circular cone and the radius $r$ of the disk. Form Pythagoras' theorem
we have $\ell = \sqrt{r^2 + h^2}$.
\end{MHint}

\item
Describe the surface area $O$ of the cone as a function of $h$ and $r$:
\par
 \MEquationItem{$O$}{\MLSimplifyQuestion{30}{pi*r*(r + sqrt(r*r + h*h))}{10}{r,h}{4}{32}{ExM05Zylb}}

\begin{MHint}{Solution}
The result for $\ell = \sqrt{r^2 + h^2}$ obtained in the first part of the exercise is inserted 
in the formula for the surface area $O$ given above. Thus, we have
\[
O = \pi \cdot r \cdot (r + \ell) %
 = \pi \cdot r \cdot \left(r + \sqrt{r^2 + h^2}\right) %
 = \pi \cdot r^2 \cdot \left(1 + \sqrt{1 + \left(\frac{h}{r}\right)} \right). %%
\]
The last formula shows how the relation between surface area of the cone  and the area $G = \pi \cdot r^2$ of the disk depends on the ratio of the height 
$h$ to the radius $r$.
\end{MHint}
\end{MExerciseItems}
\end{MExercise}

\end{MXContent}
%end of subsection: ElementareKoerper.


%Uebungen zum Abschnitt Elementare Koerper:
\begin{MExercises}
\MDeclareSiteUXID{VBKM05_ElementareKoerper_Exercises}

\begin{MExercise}
%Berechnen Sie das Volumen eines Prismas der H"ohe $h = 8\MEinheit{cm}$, dessen 
%Grundfl"ache ein gleichseitiges Sechseck mit Seitenl"ange $a = 3\MEinheit{cm}$ 
%ist.
%\par
%Hinweis: Das Sechseck kann in sechs gleichseitige Dreiecke zerlegt werden.
Calculate the volume of a prism of height $h = 8\MEinheit{cm}$ with a triangle
as its base. Two sides of this triangle are of length $5\MEinheit{cm}$, 
and one side is of length $6\MEinheit{cm}$.
\par
Answer: \MLParsedQuestion{15}{96}{10}{ExM05EK1}$\MEinheit{cm}^3$
\end{MExercise}

\begin{MExercise}
The surface area of a cylinder of height $h = 6\MEinheit{cm}$ is to be covered 
with a coloured sheet. The surface area shall be $O = 200\MEinheit{cm}^2$. 
Calculate the diameter $d$ of the disk and the volume of the cylinder. Use the approximate value $\MZahl{3}{1415}$ for 
$\pi$ and round off your result 
to the nearest millimetre.


Answers:
\begin{MExerciseItems}
\item %
 \MEquationItem{$d$}{\MLParsedQuestion{15}{177}{10}{ExM05EK21}}$\MEinheit{cm}$
\item %
 \MEquationItem{$V$}{\MLParsedQuestion{15}{177}{10}{ExM05EK22}}$\MEinheit{cm}^3$
\end{MExerciseItems}
\end{MExercise}

\begin{MExercise}
Consider a piece of wood with the shape of a rectangular cuboid with the volume
$V$. The height of the cuboid is $h = 120\MEinheit{cm}$, and the 
base face is a square with sides of length $s = 40\MEinheit{cm}$.
>From the piece of wood, a cylindrical hole of height $g$ with a diameter 
$d = 20\MEinheit{cm}$ is drilled ``centrically'' (i.e. the intersection 
point of the diagonals of the quadratic base face is the centre of the 
base disk of the cylinder). Use the approximate value 
$\MZahl{3}{1415}$ for $\pi$ and round off your result to integers. Calculate

\begin{MExerciseItems}
\item the volume $V_Z$ of the drilled hole:
\par
\MEquationItem{$V_Z$}{\MLParsedQuestion{15}{37698}{10}{ExM05EK21b}}$\MEinheit{cm}^3$
\begin{MHint}{Solution}
The volume of the cylinder is
\[
V_Z = \pi \cdot \left(\frac{d}{2}\right)^2 \cdot h %
 = \MZahl{3}{1415} \cdot \left(\frac{20\MEinheit{cm}}{2}\right)^2 %
   \cdot 120\MEinheit{cm} %
 = \MZahl{3}{1415} \cdot 12000\MEinheit{cm}^{3} %
 = 37698\MEinheit{cm}^{3} %
\]
\end{MHint}
%
\item the percentage of the volume $V_1$ of the new piece of wood remaining
after drilling of the volume $V_0$:
\par
Answer: \MLParsedQuestion{15}{81}{10}{ExM05EK21c} $\%$
\begin{MHint}{Solution}
The volume $V$ of the piece of wood is
\[
V = s^2 \cdot h %
 = \left(40\MEinheit{cm}\right)^2 \cdot 120\MEinheit{cm} %
 = 1600 \cdot 120\MEinheit{cm}^{3} %
 = 16 \cdot 12000\MEinheit{cm}^{3} %
\]
The percentage $p_Z$ of the drilled cylinder is 
\[
p_Z = \frac{V_Z}{V} %
 = \frac{\pi \cdot 12000\MEinheit{cm}^3}{16 \cdot 12000\MEinheit{cm}^3} %
 \approx \frac{\MZahl{3}{1415}}{16} \approx 19 \% %%
\]
and thus, $p = (100 - 19) \% = 81 \%$ is the percentage original wooden rectangular cuboid which forms the new piece of wood.
\end{MHint}
\end{MExerciseItems}
\end{MExercise}
\end{MExercises}
%end of exercises: Uebungen zum Abschnitt Elementare Koerper (jgl).



%content: Abschnitt 6: Winkelfunktionen

\MSubsection{Trigonometric Functions: Sine, et cetera}
\MLabel{M05_Trigonometrie}

\begin{MIntro}
\MDeclareSiteUXID{VBKM05_Trigonometrie_Intro}
%Je nach Sonnenstand ist der Schatten einer Person oder eines Gegenstandes 
%unterschiedlich gro"s. Indem die Verh"altnisse in Abh"angigkeit des Winkels
%erfasst werden, kann man die zugeh"origen L"angen berechnen. Die Betrachtung
%von funktionalen Zusammenh"ange, die sich hier aus geometrischen Beobachtungen 
%ergeben und zur Definition von Sinus, Cosinus und weiteren trigonometrischer 
%Gr"o"sen f"uhren, werden im n"achsten Modul allgemein besprochen.

On mountain roads, warning signs are put up if the road goes steeply downhill.
The percentage describes how steep the terrain slopes compared to a horizontal 
movement. Questions for the conditions of movements on an inclined plane in physics 
have been investigated by Galileo Galilei. The results are also relevant for 
technical constructions.

Trigonometric functions serve as a mathematical tool: they describe a geometric 
situation by means of a mathematical expression.
%Im Bild von oben ausgedr"uckt, wird in diesem Abschnitt "uber Winkelfunktionen
%beschrieben, wie der Zusammenhang zwischen der prozentualen Angabe des 
This section describes how the relation between the percentage of the slope and 
the corresponding angle can be expressed. A first investigation of the properties 
of the trigonometric functions gives an idea of the various possible applications
far beyond geometry, which will be revisited repeatedly in the later sections. 
\end{MIntro}

\begin{MXContent}{Trigonometry in Triangles}{Triangle}{STD}
\MLabel{Abschnitt:TrigonometrieAmDreieck}
\MDeclareSiteUXID{VBKM05_Trigonometrie_Content}

If one drives downhill on a road with a slope of five percent, then
the height falls five metres for every 100 metres travelled horizontally. Here, the difference in height
is considered in comparison to the horizontal line.

\begin{center}
\MTikzAuto{%
\begin{tikzpicture}[line width=1pt]
\coordinate (A) at (0,0);
\coordinate (B) at (10,0);
\coordinate (C) at (10,0.5);
\path (A) -- node[below] {$a = 100\MEinheit{m}$} (B) %
 -- node[right] {$b = 5\MEinheit{m}$} (C) %
 -- node[above,rotate={atan(0.05)}] {slope of $5\%$} (A);
\draw (A) -- (B) -- (C) -- cycle;
\end{tikzpicture}
}
\end{center}

Accordingly, the slope is $100\%$ if the difference in height 
between two positions with a horizontal distance of $100\MEinheit{m}$
is $100\MEinheit{m}$. Geometrically, the connecting line segment between 
the two points is a diagonal of a square. Hence, the angle between the horizontal 
line and the diagonal, i.e. the road on which ones moves, has a degree measure 
of $45\MGrad$.



\begin{center}
\MTikzAuto{%
\begin{tikzpicture}[line width=1pt]
\coordinate (A) at (0,0);
\coordinate (B) at (3,0);
\coordinate (C) at (3,3);
\coordinate (D) at (0,3);
\path (A) -- node[below] {$100\MEinheit{m}$} (B) -- %
 node[below,rotate=90] {$100\MEinheit{m}$}(C);
\draw (A) -- (B) -- (C) -- cycle;
\draw[style=dotted] (C) -- (D) -- (A) -- cycle;
\draw (1,0) arc(0:45:1);
\draw[style=dotted] (0,1) arc(90:45:1);
\end{tikzpicture}
}
\end{center}

In other words: An angle of $45\MGrad$ corresponds to a slope of 
$\frac{100\MEinheit{m}}{100\MEinheit{m}} = 1$, i.e. the ratio 
of the horizontal line segment to the vertical line segment is $1$. 
According to the intercept theorem, this ratio does not depend 
on the lengths of the individual segments. It only depends on 
the position of the two rays with respect to each other, i.e. 
the measure of the angle they enclose. If this assignment of a 
ratio of the line segments to an angle is also known for other angles, many constructive 
problems can be solved. For example, for a given angle the height can be
determined. 

Even the question of which ratio corresponds to an angle of $30\MGrad$ shows, however,
that in general it is not that simple to determine the assignment of a ratio
of line segments to an angle.
%Schon die Berechnung, dass zu einem Gef"alle oder einer Steigung von 
%$\MZahl{57}{7}\%$ der Winkel rund $30\MGrad$ gro"s ist, ist bereits nicht 
%mehr so einfach.
%Schon die Berechnung, dass zu einem Gef"alle oder einer Steigung von $50\%$ 
%der Winkel rund $\MZahl{26}{565}\MGrad$ gro"s ist, ist bereits nicht mehr 
%so einfach.
Therefore, the time-consumingly determined values that we considered initially
were listed in mathematical tables such that they could be looked up later again easily.
Now, these values are available practically everywhere, provided by calculators and 
computers. The most common assignments of an angle to a ratio of line segments are presented 
below. They are called circular functions or trigonometric functions, 
the branch of mathematics dealing with their properties is called 
\MEntry{trigonometry}{trigonometry}.


\begin{MXInfo}{Trigonometric Functions in the Right Triangle}%
\MLabel{M05_DefinitionWinkelfunktionen}%

Here, the most common \MEntry{circular functions}{circular functions}
are described as assignments of ratios of the sides in a right triangle to an angle. 
The circular functions are also called \MEntry{trigonometric functions}{trigonometric function}.
Here, $x$ denotes an angle in a right triangle that is not a right angle.
The \MEntry{opposite (side)}{opposite (side)} is the side opposite the angle $x$, 
and the other leg is called the \MEntry{adjacent (side)}{adjacent (side)}. 

\begin{center}
\MTikzAuto{%
\begin{tikzpicture}[line width=1pt]
\coordinate (A) at (0,0);
\coordinate (B) at ({3*sqrt(3)},0);
\coordinate (C) at ({3*sqrt(3)},3);
\path (A) -- node[below] {adjacent side $b$} (B) %
 -- node[below,rotate=90] {opposite side $a$} (C) %
 -- node[above,rotate=30] {hypotenuse $c$} (A);
\draw (A) -- (B) -- (C) -- cycle;
\draw ($ (B) + (0,0.5) $) arc(90:180:0.5);
\filldraw ($ (B) + (135:0.25) $) circle(0.5pt);
\draw ($ (A) + (1.2,0) $) arc(0:30:1.2);
\path ($ (A) + (1.2,0) $) arc(0:15:1.2) node[right] {$x$};
\end{tikzpicture}
}
\end{center}

\begin{itemize}
\item
The ratio of the opposite side $a$ to the adjacent side $b$ to an angle 
is called tangent function:
\[
\tan(x) := \frac{\text{opposite side}}{\text{adjacent side}} = \frac{a}{b} %%
\]
%
\item
The ratio of the adjacent side $b$ to the hypotenuse $c$ to an angle 
is called cosine function:
\[
\cos(x) := \frac{\text{adjacent side}}{\text{hypotenuse}} = \frac{b}{c} %%
\]
%
\item
The ratio of the opposite side $a$ to the hypotenuse $c$ 
to an angle is called sine function:
\[
\sin(x) := \frac{\text{opposite side}}{\text{hypotenuse}} = \frac{a}{c} %%
\]
\end{itemize}
\end{MXInfo}

The tangent function describes the assignment of the ratio 
of height to width to the angle of inclination, i.e. the slope. In 
Chapter~\MRef{VBKM08} this is also relevant in the context to the 
geometrical interpretation of the derivative.

According to the definition, the tangent function of the angle 
$\alpha$ is
\[
\tan\left(\alpha\right)=\frac{a}{b}=\frac{a}{b}\cdot\frac{c}{c} %
 =\frac{a}{c}\cdot\frac{c}{b} %
 =\frac{\sin\left(\alpha\right)}{\cos\left(\alpha\right)} \MDFPeriod %%
\]
Thus, it suffices to know the values of sine and cosine   
to be able to calculate the tangent function.

\begin{MExample}
Let a triangle with a right angle $\gamma=\frac{\pi}{2}=90\MGrad$ be given.
The side $c$ is of length $5\MEinheit{cm}$, and the side $a$ is of length 
$\MZahl{2}{5}\MEinheit{cm}$. Calculate the sine, cosine and tangent function 
of the angle $\alpha$.

The sine can be calculated immediately from the given values:
\[
\sin\left(\alpha\right)=\frac{a}{c}
 =\frac{\MZahl{2}{5}\MEinheit{cm}}{5\MEinheit{cm}}=\MZahl{0}{5} \MDFPeriod\]
To calculate the cosine the length of the side $b$ is required - it can be obtained
by means of Pythagoras' theorem:
\[
b^2 = c^2 - a^2 %%
\]
Hence,
\[
\cos\left(\alpha\right)=\frac{b}{c}=\frac{\sqrt{c^2-a^2}}{c} %
 =\frac{\sqrt{\left(5\MEinheit{cm}\right)^2 - \left(\MZahl{2}{5}\MEinheit{cm}\right)^2}}{5\MEinheit{cm}}=\MZahl{0}{866} \MDFPeriod %%
\]
Thus, the tangent of the angle $\alpha$ is 
\[
\tan\left(\alpha\right) %
 =\frac{\sin\left(\alpha\right)}{\cos\left(\alpha\right)} %
 =\frac{\MZahl{0}{5}}{\MZahl{0}{866}}=\MZahl{0}{5773} \MDFPeriod
\]
\end{MExample}


\begin{MExercise}\MLabel{VBKM05_TrigonometrieAufgabeTabelle}
Determine some approximate values of the trigonometric functions sine, cosine 
and tangent graphically. Let a right triangle with the hypotenuse $c=5$ be given. 
Use Thales' circle to draw right triangles for the angles
\[ 
\alpha \in \left\{10\MGrad\MElSetSep 20\MGrad\MElSetSep %
30\MGrad\MElSetSep 40\MGrad\MElSetSep 45\MGrad\MElSetSep 50\MGrad\MElSetSep %
60\MGrad\MElSetSep 70\MGrad\MElSetSep 80\MGrad \right\} \MDFPeriod %%
\]
%jgl: Anmerkung: Befehl \MEinheit nicht f"ur Umlaute geeignet.
\ifttm
Use a drawing scale of $1$ unit length $\hat{=} 2\MEinheit{cm}$, and fill in the 
measured values for the sides $a$ and $b$ in a table. 
\else
Use a drawing scale of $1\,\text{unit length} \,\hat{=}\, 2\MEinheit{cm}$, and fill in the 
measured values for the sides $a$ and $b$ in a table. 
\fi
>From the measured values, calculate the sine, cosine, and tangent of each angle
and decide for which functions also values for $\alpha = 0\MGrad$ 
and $\alpha = 90\MGrad$ exist. After that, plot the calculated values of sine 
and cosine against the angle $\alpha$.

\begin{MHint}{Solution}
In the process of measurement, errors will always occur! Therefore, the 
values in your table will be slightly different from the ones given in 
the table below. The table could look as follows:

%\ifttm\relax\else\begin{small}\fi
\begin{center}
\begin{tabular}{r|r|r|r|r|r}\hline
	$\alpha$ 		& $a$ 				& $b$ 				& $\sin\left(\alpha\right)$ & $\cos\left(\alpha\right)$ & $\tan\left(\alpha\right)$\\ \hline\hline
	0 						& $\MZahl{0}{0}$		& $\MZahl{5}{0}$ 	& $\MZahl{0}{0}$		& $\MZahl{1}{0}$	& $\MZahl{0}{0}$\\ \hline
	$10\MGrad$		& $\MZahl{0}{8}$	& $\MZahl{4}{9}$	& $\MZahl{0}{160}$	& $\MZahl{0}{98}$	& $\MZahl{0}{1633}$\\ \hline
	$20\MGrad$		& $\MZahl{1}{7}$	& $\MZahl{4}{7}$	& $\MZahl{0}{34}$	& $\MZahl{0}{94}$	& $\MZahl{0}{3617}$\\ \hline
	$30\MGrad$		& $\MZahl{2}{5}$	& $\MZahl{4}{3}$	& $\MZahl{0}{5}$	& $\MZahl{0}{86}$	& $\MZahl{0}{5814}$\\ \hline
	$40\MGrad$		& $\MZahl{3}{2}$	& $\MZahl{3}{8}$	& $\MZahl{0}{64}$	& $\MZahl{0}{76}$	& $\MZahl{0}{8421}$\\ \hline
	$45\MGrad$		& $\MZahl{3}{5}$	& $\MZahl{3}{5}$	& $\MZahl{0}{7}$	& $\MZahl{0}{7}$	& $\MZahl{1}{0}$\\ \hline
	$50\MGrad$		& $\MZahl{3}{8}$	& $\MZahl{3}{27}$	& $\MZahl{0}{76}$	& $\MZahl{0}{64}$	& $\MZahl{1}{1875}$\\ \hline
	$60\MGrad$		& $\MZahl{4}{3}$	& $\MZahl{2}{5}$	& $\MZahl{0}{86}$	& $\MZahl{0}{5}$	& $\MZahl{1}{7200}$\\ \hline
	$70\MGrad$		& $\MZahl{4}{7}$	& $\MZahl{1}{7}$	& $\MZahl{0}{94}$	& $\MZahl{0}{34}$	& $\MZahl{2}{7647}$\\ \hline
	$80\MGrad$		& $\MZahl{4}{9}$	& $\MZahl{0}{8}$	& $\MZahl{0}{98}$	& $\MZahl{0}{160}$	& $\MZahl{6}{1250}$\\ \hline
	$90\MGrad$		& $\MZahl{5}{0}$		& $\MZahl{0}{0}$	& $\MZahl{1}{0}$	& $\MZahl{0}{0}$	& -- \\ \hline
\end{tabular}
\end{center}
%\ifttm\relax\else\end{small}\fi

Then, the corresponding diagram looks as follows:


\begin{center}
\MTikzAuto{%
\pgfkeys{/pgf/number format/set decimal separator={{{\MZXYZhltrennzeichen}}}}
\begin{tikzpicture}[x=0.1cm, y=5.4cm,scale=0.7] 
%Koordinatensystem
\node (xMAX) at (102.0,0){};
\node (yMAX) at (0,1.15){};
%%\draw[help lines, gray, dashed, xstep=1, ystep=1] (0,0) grid (5.5,2.8);
\draw[-stealth',color=black] (-5,0) -- (xMAX);
\foreach \x in {10, 20, 30, 40, 50, 60, 70, 80, 90}
\draw[shift={(\x,0)},color=black] (0pt,0pt) -- (0pt,-6pt) node[below] {\normalsize $\x\MGrad$};
\draw[-stealth',color=black] (0,-0.12) -- (yMAX);
\foreach \y in {0.25, 0.5, 0.75, 1}
\draw[shift={(0,\y)},color=black] (0pt,0pt) -- (-6pt,0pt)  node[left] {\normalsize $\pgfmathprintnumber{\y}$};
%Achsenbeschriftung
\draw (xMAX) node[anchor=north east] {$\alpha$};
\draw[color=blue] (yMAX) ++(0.,-16pt) node[anchor=south east] {$\cos(\alpha)$};
\draw[color=red] (yMAX) ++(102.0,-16pt) node[anchor=south east] {$\sin(\alpha)$};
\fill[color=red] (90,1.0) circle (2.0pt);
\fill[color=red] (80,0.98) circle (2.0pt);
\fill[color=red] (70,0.94) circle (2.0pt);
\fill[color=red] (60,0.86) circle (2.0pt);
\fill[color=red] (50,0.76) circle (2.0pt);
%\fill[color=red] (45,0.7) circle (2.0pt);
\fill[color=red] (40,0.64) circle (2.0pt);
\fill[color=red] (30,0.5) circle (2.0pt);
\fill[color=red] (20,0.34) circle (2.0pt);
\fill[color=red] (10,0.16) circle (2.0pt);
\fill[color=red] (0,0.0) circle (2.0pt);
\fill[color=blue] (0,1.0) circle (2.0pt);
\fill[color=blue] (10,0.98) circle (2.0pt);
\fill[color=blue] (20,0.94) circle (2.0pt);
\fill[color=blue] (30,0.86) circle (2.0pt);
\fill[color=blue] (40,0.76) circle (2.0pt);
%\fill[color=blue] (45,0.7) circle (2.0pt);
\fill[color=blue] (50,0.64) circle (2.0pt);
\fill[color=blue] (60,0.5) circle (2.0pt);
\fill[color=blue] (70,0.34) circle (2.0pt);
\fill[color=blue] (80,0.16) circle (2.0pt);
\fill[color=blue] (90,0.0) circle (2.0pt);
\fill[color=magenta!50!black] (45,0.7) circle (2.0pt);
%Beschriftung und Graphen
%%\clip(-2.8,-0.5) rectangle (6,3);
%%\draw[color=black] (7,3) node[anchor=south west] {$\MPointTwo{7}{3}$};
\end{tikzpicture}
}
\end{center}
\end{MHint}
\end{MExercise}

If we once again look closer at the results obtained in the last exercise, 
we can find different ways to interpret them, and then identify 
some relations.


\begin{itemize}
\item With increasing angle $\alpha$ the opposite side $a$ increases and 
the adjacent side $b$ decreases.

Likewise, $\sin\left(\alpha\right)\sim a$ and 
$\cos\left(\alpha\right)\sim b$.

\item With increasing angle $\alpha$ the opposite side $a$ increases 
to the same extent as the adjacent side $b$ decreases with the angle $\alpha$
decreasing from $90\MGrad$. In the Thales circle, the two triangles 
with the opposite values of $a$ and $b$ are two solutions for 
the construction of a right triangle with a given hypotenuse and a given 
altitude (see also Example~\MRef{ThaleskreisBeispiel}).

\item In the right triangle the adjacent side of the angle $\beta = 90\MGrad - \alpha$ is 
the same side as the opposite side of the angle $\alpha$
(and vice versa). Thus, 
\[
\sin\left(\alpha\right)=\cos\left(90\MGrad-\alpha\right) %
 = \cos\left(\frac{\pi}{2}-\alpha\right) %%
\]
and 
\[
\cos\left(\alpha\right) = \sin\left(90\MGrad-\alpha\right) %
 = \sin\left(\frac{\pi}{2}-\alpha\right) \MDFPeriod %%
\]
%
\item For $\alpha=45\MGrad$ the opposite side and adjacent side are equal, and thus sine and 
cosine are equal as well. This observation was used at the beginning of this 
section for the determination of the slope.

\item The tangent function, i.e. the ratio of $a$ to $b$, increases with increasing 
angle $\alpha$ from zero to ``infinity''. 
\end{itemize}

In the following example we will continue our considerations from the beginning of this 
section and use a triangle with an angle of $45\MGrad$ to calculate the value 
of the corresponding sine value exactly.

\begin{MExample}
Calculate the sine of the angle $\alpha=45\MGrad$ now exactly, i.e. unlike as in 
Exercise~\MRef{VBKM05_TrigonometrieAufgabeTabelle}, where the sine was calculated from 
measured (and hence error-prone) values.

If in a right triangle with $\gamma=90\MGrad$ the angle $\alpha$ is equal to $45\MGrad$, then, 
because of the formula for the sum of interior angles in a right triangle, $\alpha+\beta+\gamma=\pi=180\MGrad$,  
the angle $\beta$ also needs to be equal to $45\MGrad=\pi/4$, and the two legs $a$ and $b$ are 
of equal length. A triangle with two sides of equal length is called \textbf{isosceles}.

\begin{tabular}{@{}lr@{}}
\MTikzAuto{%
\begin{tikzpicture}[x=1.0cm, y=1.0cm] 
%%\draw[help lines, gray!50, xstep=0.5, ystep=0.5] (0,0) grid (9,8);
\draw[color=black, very thick] (0,0) -- (6,0) -- (3,3) -- cycle;
\draw[color=black, thin] (0,0) ++(0:1.2) arc (0:45:1.2);
\draw[color=black] (0,0) ++(22.5:0.8) node {\large $\alpha$};
\draw[color=black, thin] (6,0) ++(135:1.2) arc (135:180:1.2);
\draw[color=black] (6,0) ++(157.5:0.8) node {\large $\beta$};
\draw[color=black, thin] (3,3) ++(225:1.2) arc (225:315:1.2);
\fill[color=black] (3,3) ++(0,-0.6) circle (1.5pt);
\draw[color=black] (4.5,1.5) node[anchor=south west] {\large $a$};
\draw[color=black] (1.5,1.5) node[anchor=south east] {\large $b$};
\draw[color=black] (3,0) node[anchor=north] {\large $c$};
\end{tikzpicture}
}
&
\begin{minipage}[b]{10cm}
We have: \[\sin\left(\alpha\right) = \sin\left(45\MGrad\right) = \frac{a}{c} \MDFPeriod\]
Moreover: \[a^2+b^2 = 2a^2 = c^2\quad\Rightarrow\quad c=\sqrt{2}\cdot a\]
\[\Rightarrow\quad \sin\left(45\MGrad\right) = \sin\left(\pi/4\right)=\frac{a}{\sqrt{2}\cdot a} = \frac{1}{2}\cdot \sqrt{2} \MDFPeriod\]
\end{minipage}
\end{tabular}
In Exercise~\MRef{VBKM05_TrigonometrieAufgabeTabelle} the value of the sine of 
$45\MGrad$ was approximated by a value of $\MZahl{0}{7}$ which is quite close to 
the actual value of $\frac{1}{2}\cdot \sqrt{2}$.
\end{MExample}

In the next example we will calculate the sine of the angle $\alpha = 60\MGrad$. 
For this purpose, we first do not consider a right triangle but an equilateral 
triangle. By a clever decomposition of the triangle and by using another 
``auxiliary quantity'' we will obtain the required result. 


\begin{MExample}\MLabel{M05_TrigonometrieBeispiel:gleichseitigesDreieck}%
Consider a \textbf{equilateral} triangle to calculate $\sin\left(60\MGrad\right)$. 
As the name implies, the sides of this triangle are all of equal length, and the angles are 
also all of the same magnitude, namely $\alpha=\beta=\gamma = \frac{180\MGrad}{3} = 60\MGrad = \frac{\pi}{3}$.
According to the theorem for congruent triangles ``sss'', the triangle is defined uniquely 
by the specification of a side $a$. This triangle is constructed by drawing the side $a$ 
and then drawing a circle with radius $r$ around both endpoints of the side. Now, the intersection 
point of the two circles is the third vertex.

\begin{tabular}{@{}lr@{}}
\begin{minipage}[b]{10.5cm}
This triangle is not right-angled. If an altitude $h$ is drawn on one 
of the sides $a$, the triangle can be divided into two congruent right triangles.

We have: 
\[
\sin\left(\alpha\right) %
 = \sin\left(60\MGrad\right) %
 = \frac{h}{a} \MDFPeriod %%
\]
According to Pythagoras' theorem we have 
\[
\left(\frac{a}{2}\right)^2+h^2 %
 = a^2 \MDFPeriod %%
\]
Therefore,
\[
h^2 = \frac{3}{4}a^2 
\quad \text{and hence} \quad
h = \frac{1}{2}\sqrt{3}\cdot a \MDFPeriod
\]
\end{minipage}
&
\MTikzAuto{%
\begin{tikzpicture}[x=1.0cm, y=1.0cm] 
%%\draw[help lines, gray!50, xstep=0.5, ystep=0.5] (0,0) grid (9,8);
\draw[color=black, very thick] (0,0) -- (5,0) -- (2.5,4.33) -- cycle;
\draw[color=black, thin] (0,0) ++(0:1.2) arc (0:60:1.2);
\draw[color=black] (0,0) ++(30:0.8) node {\large $\alpha$};
\draw[color=black, thin] (5,0) ++(120:1.2) arc (120:180:1.2);
\draw[color=black] (5,0) ++(150:0.8) node {\large $\alpha$};
\draw[color=black, thin] (2.5,0) -- (2.5,3.13);
\draw[color=black, gray, thin] (2.5,3.13) -- (2.5,4.33);
\draw[color=black, thin] (2.5,4.33) ++(240:1.2) arc (240:300:1.2);
\draw[color=black] (2.5,4.33) ++(270:0.8) node {\large $\alpha$};
\draw[color=black, thin] (2.5,0) ++(90:0.8) arc (90:180:0.8);
\fill[color=black] (2.5,0) ++(135:0.4) circle (1.5pt);
\draw[color=black] (3.75,2.165) node[anchor=south west] {\large $a$};
\draw[color=black] (1.25,2.165) node[anchor=south east] {\large $a$};
\draw[color=black] (2.5,0.0) node[anchor=north] {\large $a$};
\draw[color=black] (2.5,1.65) node[anchor=west] {\large $h$};
\draw[color=black, gray, thin] (0,0) ++(50:5.0) arc (50:70:5.0);
\draw[color=black, gray, thin] (5,0) ++(110:5.0) arc (110:130:5.0);
\end{tikzpicture}
}
\end{tabular}
As a result we obtain the required value
\[
\sin\left(60\MGrad\right) %
 = \sin\left(\frac{\pi}{3}\right) %
 = \frac{h}{a} %
 =\frac{1}{2}\cdot \sqrt{3} \MDFPeriod %%
\]
>From this triangle the sine of another angle can also be calculated: the altitude
$h$ bisects the above angle such that in the two congruent smaller triangles the 
above angle is $30\MGrad = \frac{\pi}{6}$. Now we have
\[
\sin\left(30\MGrad\right) %
 = \sin\left(\frac{\pi}{6}\right) %
 = \frac{a/2}{a} %
 = \frac{1}{2} \MDFPeriod
\]
\end{MExample}

\begin{MExercise}
Calculate the exact value of the cosine of the angles $\alpha_1=30\MGrad$, 
$\alpha_2=45\MGrad$, and $\alpha_3=60\MGrad$. To do this, use the results obtained 
in the example above and in Exercise~\MRef{VBKM05_TrigonometrieAufgabeTabelle}.

\begin{MHint}{Solution}
>From Exercise~\MRef{VBKM05_TrigonometrieAufgabeTabelle} it is known  
that $\cos\left(\alpha\right)= \sin\left(90\MGrad-\alpha\right)$.
With the results obtained in the example above it follows
\begin{eqnarray*}
\cos\left(30\MGrad\right) %
 & = & \sin\left(90\MGrad-30\MGrad\right) %
 =\sin\left(60\MGrad\right) %
 =\frac{1}{2}\cdot\sqrt{3} \MDFPSpace, \\
%
\cos\left(45\MGrad\right)
 & = & \sin\left(90\MGrad-45\MGrad\right) %
 =\sin\left(45\MGrad\right) %
 =\frac{1}{2}\cdot\sqrt{2} \MDFPSpace, \\
%
\cos\left(60\MGrad\right) %
 & = & \sin\left(90\MGrad-60\MGrad\right) %
 =\sin\left(30\MGrad\right) %
 =\frac{1}{2} \MDFPeriod \\
\end{eqnarray*}
\end{MHint}
\end{MExercise}

The following small table lists the values for frequently used angles:
In the first row denoted by $x$ the angle is given in degree measure, and 
in the last row denoted by $\alpha$ the angle is given in radian measure.

\ifttm
\begin{MDirectHTML}
\[
\begin{array}[t]{l|*{5}{c}}
 x            & 0                          & \tfrac{\pi}{6}             & \tfrac{\pi}{4}             & \tfrac{\pi}{3}             & \tfrac{\pi}{2}             \\[1mm] \hline
         \sin & 0 = \frac{1}{2} \cdot \sqrt{0} & \frac{1}{2} = \frac{1}{2} \cdot \sqrt{1} & \frac{1}{2} \cdot \sqrt{2} & \frac{1}{2} \cdot \sqrt{3} & \frac{1}{2} \cdot \sqrt{4} = 1 \\[1mm]
         \cos & 1 = \frac{1}{2} \cdot \sqrt{4} & \frac{1}{2} \cdot \sqrt{3} & \frac{1}{2} \cdot \sqrt{2} & \frac{1}{2} \cdot \sqrt{1} = \frac{1}{2} & \frac{1}{2} \cdot \sqrt{0} = 0 \\[1mm]
         \tan & 0                          & \frac{\sqrt{3}}{3}         & 1                          & \sqrt{3}                   & -                          \\[1mm] \hline
 \alpha       & 0^{\circ}                    & 30^{\circ}                   & 45^{\circ}                   & 60^{\circ}                   & 90^{\circ} %% 
\end{array}
\]
\end{MDirectHTML}
\else
\begin{center}
       $\begin{array}[t]{l|*{5}{c}}
 x            & 0                          & \Mtfrac{\pi}{6}             & \Mtfrac{\pi}{4}             & \Mtfrac{\pi}{3}             & \Mtfrac{\pi}{2}         \\[1mm] \hline
         \sin & 0 = \frac{1}{2} \cdot \sqrt{0} & \frac{1}{2} = \frac{1}{2} \cdot \sqrt{1} & \frac{1}{2} \cdot \sqrt{2} & \frac{1}{2} \cdot \sqrt{3} & \frac{1}{2} \cdot \sqrt{4} = 1 \\[1mm]
         \cos & 1 = \frac{1}{2} \cdot \sqrt{4} & \frac{1}{2} \cdot \sqrt{3} & \frac{1}{2} \cdot \sqrt{2} & \frac{1}{2} \cdot \sqrt{1} = \frac{1}{2} & \frac{1}{2} \cdot \sqrt{0} = 0 \\[1mm]
         \tan & 0                          & \frac{\sqrt{3}}{3}         & 1                          & \sqrt{3}                   & -                          \\[1mm] \hline
 \alpha       & 0\MGrad                  & 30\MGrad                 & 45\MGrad                 & 60\MGrad                 & 90\MGrad %% 
        \end{array}$
\end{center}
\fi

You should learn these values by heart. The values of the trigonometric functions for other
angles are listed in tables or saved in your calculator.

Hence, a height can be calculated very easily from an angle and a distance. 
Namely, if $s$ is the distance of a building with a flat roof, which is observed 
at an angle of $x$, then from $\tan(x) = \frac{h}{s}$ we have  $h = s \cdot \tan(x)$. 
Likewise, sine and cosine can be used to calculate lengths. This relation between 
angles and lengths is often used.

For example, an area can be calculated in this way even if the required 
length is not given directly. In the following example, the altitude
of a triangle is to be calculated. Since $h$ emanating from a vertex $C$ 
is perpendicular to the line of the opposite side
$c = \MGeoStrecke{A}{B}$, the vertices of $h$ and  $A$ or $B$, respectively, 
form a right triangle. If an angle and the adjacent side are given, then 
the altitude can be calculated from $\sin(\alpha) = \frac{h}{b}$ or from
$\sin(\beta) = \frac{h}{a}$, using standard notation.


\begin{MExercise}
Calculate the area $F$ of a triangle with the sides $c = 7$, $b = 3$,  and 
the angle $\alpha = 30\MGrad$ between the two sides $c$ and $b$.

Result: %
 \MEquationItem{$F$}{\MLParsedQuestion{20}{21/4}{4}{ExM05Sec6DFlaeche}}

\begin{MHint}{Solution}
The area $F$ can be calculated from $F =\frac{1}{2}\cdot c \cdot h_c$, where we
still need to determine $h_c$. From $\sin\left(\alpha\right)=\frac{h_c}{b}$
we have
\[
 h_c=b\cdot\sin\left(\alpha\right) %
  = 3\cdot\sin\left(30\MGrad\right) %
  = 3\cdot\frac{1}{2} \MDFPeriod %%
\]
Hence,
\[ 
F =\frac{1}{2}\cdot c\cdot b\cdot\sin\left(\alpha\right) %
  =\frac{1}{2}\cdot 7 \cdot3\cdot\frac{1}{2}=\frac{21}{4} \MDFPeriod %%
\]
\end{MHint}
\end{MExercise}

\end{MXContent}


\begin{MXContent}{Trigonometry in the Unit Circle}{Unit Circle}{STD}
\MLabel{VBKM05_Trigonometrie_Einheitskreis}
\MDeclareSiteUXID{VBKM05_TrigonometrieEinheitskreis_Content}

In the previous section the trigonometric functions were introduced by means 
of a right triangle. Hence, the properties described above are valid 
for an angle ranging from $0\MGrad$ to $90\MGrad$ or 
from $0$ to $\frac{\pi}{2}$, respectively.

To extend the acquired insights to angles greater than $\pi/2$, it is 
particularly useful to investigate the so called unit circle.

\begin{center}
\MTikzAuto{%
\begin{tikzpicture}[line width=1.5pt,scale=1.5]
\coordinate (P) at (30:1);
\coordinate (Px) at ($(P) + (-90:{1/2}) $);
\draw[style=dotted] ($(P) + (-1.6,0) $) -- ($ ({pi/6},{1/2}) + (0.3,0) $);
\draw[style=dotted] %
 ($(Px) + (-1.6,0) $) -- ($ ({sqrt(3)/2},{-pi/6}) + (-1.6,-1.9) $);
\begin{scope}[xshift=-1.6cm]
\coordinate (M) at (0,0);
\coordinate (P) at (30:1);
\coordinate (Px) at ($(P) + (-90:{1/2}) $);
\coordinate (Py) at ($(P) + (180:{1/2}) $);
\draw (-1.3,0) -- (0,0);
\draw[->] (Px)  -- (1.3,0) node[below left] {$x$};
\draw[->] (0,-1.3) -- (0,1.3) node[below left] {$y$};
\draw[color=black!50!white] (M) circle(1);
\draw (M) -- node[above,rotate=30] {$r=1$} (P);
\draw[color=green] (M) -- (Px);
\draw[color=blue] (P) -- (Px);
\draw (1,0) arc(0:30:1);
\node[right] at (15:1) {$\Mvarphi$};
\filldraw (M) circle(1pt);
\filldraw (P) circle(1pt);
\end{scope}
%
\begin{scope}[xshift=0.3cm]
\coordinate (M) at (0,0);
\coordinate (P) at (30:1);
\coordinate (Px) at ($(P) + (-90:{1/2}) $);
\coordinate (Py) at ($(P) + (180:{1/2}) $);
\draw[->,color=black!50!white] (-0.3,0) -- (6.6,0) node[above left] {$\xi$};
\draw[->] (0,-1.3) -- (0,1.3) node[below left] {$y=\sin(\xi)$};
%\draw[->] (0,-1.3) -- (0,1.3) node[below left] {$y$};
\node[below left] at (M) {$0$};
\foreach \x/\xt in {1/{\frac{\pi}{2}}, 2/{\pi}, 3/{\frac{3\pi}{2}}, 4/{2\pi}}
 \draw[color=black!50!white] ({\x*pi/2},0) -- ++(0,-0.1) node[below] {$\xt$};
\draw[domain=0:{2*pi},samples=200,color=blue!50!white] 
 plot (\x,{sin(\x r)});
\draw[color=black] (M) -- node[below] {$\Mvarphi$} ({pi/6},0);
\draw[color=blue] ({pi/6},0) -- node[right] {$\sin(\Mvarphi)$} ++(0,{1/2});
\filldraw (M) circle(1pt);
\filldraw ({pi/6}, {1/2}) circle(1pt);
\end{scope}
%
\begin{scope}[xshift=-1.6cm,yshift=-1.9cm,rotate=-90]
\coordinate (M) at (0,0);
\coordinate (P) at (30:1);
\coordinate (Px) at ($(P) + (-90:{1/2}) $);
\coordinate (Py) at ($(P) + (180:{1/2}) $);
\draw[->,color=black!50!white] (-0.3,0) -- (6.6,0) node[above right] {$\xi$};
\draw[->] (0,-1.3) -- (0,1.3); % node[above left] {$x$}; 
  % node[above right] {$x=\cos(\xi)$};
\node[above left] at (M) {$0$};
%\path[->] (M) -- node[above,fill=white] {$x=\cos(\xi)$} (0,1.3);
\node [above right] at (0,{sqrt(3)/2}) {$x=\cos(\xi)$};
\foreach \x/\xt in {1/{\frac{\pi}{2}}, 2/{\pi}, 3/{\frac{3\pi}{2}}, 4/{2\pi}}
 \draw[color=black!50!white] ({\x*pi/2},0) -- ++(0,-0.1) node[left] {$\xt$};
\draw[domain=0:{2*pi},samples=200,color=green!50!white] 
plot (\x,{cos(\x r)});
\draw[color=black] (M) -- node[left] {$\Mvarphi$} ({pi/6},0);
\draw[color=green] %
 ({pi/6},0) -- node[above] {$\cos(\Mvarphi)$} ({pi/6},{sqrt(3)/2});
\filldraw (M) circle(1pt);
\filldraw ({pi/6}, {sqrt(3)/2}) circle(1pt);
\end{scope}
\end{tikzpicture}
}
\end{center}


%%%\begin{center}
%%%\MTikzAuto{%
%%%\begin{tikzpicture}[x=0.024cm, y=2.4cm] 
%%%\begin{scope}[xshift=-4cm,xscale=100]
%%%%Koordinatensystem
%%%\node (xMAX) at (1.3,0){};
%%%\node (yMAX) at (0,1.4){};
%%%\draw[color=black] (0,0) circle (1);
%%%\draw[-stealth',color=black] (-1.2,0) -- (1.3,0);
%%%\draw[-stealth',color=black] (0,-1.4) -- (0,1.4);
%%%\draw (xMAX) node[anchor=north] {$x$};
%%%\draw (yMAX) node[anchor=north east] {$y$};
%%%\foreach \x in {-1, 1}
%%%\draw[shift={(\x,0)},thick,color=black] (0,-0.05) -- (0,0.05) (0,0) node[anchor=north west] {\normalsize $\x$};
%%%\foreach \y in {-1, 1}
%%%\draw[shift={(0,\y)},thick,color=black] (0.05,0) -- (-0.05,0) (0,0) node[anchor=north west] {\normalsize $\y$};
%%%\def\cAng{38}
%%%\draw[-stealth',thick,color=black] (0,0) -- ({cos(\cAng)},{sin(\cAng)}) node[anchor=south west] {$P$};
%%%\draw[shift={({cos(\cAng)},0)},very thick,color=blue] (0,0.05) -- (0,-0.05) node[anchor=north east] {\normalsize $\cos(\alpha)$};
%%%\draw[shift={(0,{sin(\cAng)})},very thick,color=red] (0.05,0) -- (-0.05,0) node[anchor=east] {\normalsize $\sin(\alpha)$};
%%%\draw[thick,color=blue] (0,0) -- ({cos(\cAng)},0.0);
%%%\draw[thick,color=red] ({cos(\cAng)},0) -- ({cos(\cAng)},{sin(\cAng)});
%%%\draw[color=black, thin] (0,0) ++(0:0.5) arc (0:\cAng:0.5);
%%%\draw[color=black] (0,0) ++({0.5*\cAng}:0.35) node {$\alpha$};
%%%\end{scope}
%%%%Koordinatensystem
%%%\node (xMAX) at (400.0,0){};
%%%\node (yMAX) at (0,1.4){};
%%%%%\draw[help lines, gray, dashed, xstep=1, ystep=1] (0,0) grid (5.5,2.8);
%%%\draw[-stealth',color=black] (-20,0) -- (xMAX);
%%%\foreach \x in {30, 60, 90, 120, 150, 180, 210, 240, 270, 300, 330, 360}
%%%\draw[shift={(\x,0)},color=black] (0pt,0pt) -- (0pt,-6pt) node[below] {\scriptsize $\x\MGrad$};
%%%\draw[-stealth',color=black] (0,-1.4) -- (yMAX);
%%%\foreach \y in {1}
%%%\draw[shift={(0,\y)},color=black] (0pt,0pt) -- (-6pt,0pt)  node[left] {\normalsize $\pgfmathprintnumber{\y}$};
%%%%Achsenbeschriftung
%%%\draw (xMAX) node[anchor=north east] {$\alpha$};
%%%\draw[color=red] (yMAX) node[anchor=north west] {$y=\sin(\alpha)$};
%%%\draw[smooth,samples=73,domain=0:360, line width=1pt,color=red] plot(\x,{sin(\x)}); %{cos(\x r)}
%%%%Beschriftung und Graphen
%%%%%\clip(-2.8,-0.5) rectangle (6,3);
%%%%%\draw[color=black] (7,3) node[anchor=south west] {$\MPointTwo{7}{3}$};
%%%\end{tikzpicture}
%%%}
%%%%
%%%\MTikzAuto{%
%%%\begin{tikzpicture}[x=0.024cm, y=2.4cm] 
%%%%Koordinatensystem
%%%\node (xMAX) at (400.0,0){};
%%%\node (yMAX) at (0,1.4){};
%%%%%\draw[help lines, gray, dashed, xstep=1, ystep=1] (0,0) grid (5.5,2.8);
%%%\draw[-stealth',color=black] (-20,0) -- (xMAX);
%%%\foreach \x in {30, 60, 90, 120, 150, 180, 210, 240, 270, 300, 330, 360}
%%%\draw[shift={(\x,0)},color=black] (0pt,0pt) -- (0pt,-6pt) node[below] {\scriptsize $\x\MGrad$};
%%%\draw[-stealth',color=black] (0,-1.4) -- (yMAX);
%%%\foreach \y in {1}
%%%\draw[shift={(0,\y)},color=black] (0pt,0pt) -- (-6pt,0pt)  node[left] {\normalsize $\pgfmathprintnumber{\y}$};
%%%%Achsenbeschriftung
%%%\draw (xMAX) node[anchor=north east] {$\alpha$};
%%%\draw[color=blue] (yMAX) node[anchor=north west] {$x=\cos(\alpha)$};
%%%\draw[smooth,samples=73,domain=0:360, line width=1pt,color=blue] plot(\x,{cos(\x)}); %{cos(\x r)}
%%%%Beschriftung und Graphen
%%%%%\clip(-2.8,-0.5) rectangle (6,3);
%%%%%\draw[color=black] (7,3) node[anchor=south west] {$\MPointTwo{7}{3}$};
%%%\end{tikzpicture}
%%%}
%%%\end{center}

The unit circle is a circle with a radius of $1$. Its centre is positioned  
at the origin in the Cartesian coordinate system. Consider a line segment 
of length $1$ emanating from the centre. From its horizontal initial 
position on the positive $x$-axis, this segment is now rotated counter-clockwise, i.e.
in the mathematical positive direction, around its centre. In this process, its rotating 
end point is sweeping the unit circle enclosing the angle $\Mvarphi$ with the positive 
$x$-axis. During rotation, the angle $\Mvarphi$ increases from $0$ to $2\pi$ or 
$360\MGrad$, respectively. Thus, to any angle $\Mvarphi$ there corresponds a point 
with the coordinates $x_{\Mvarphi}$ and $y_{\Mvarphi}$ on the unit circle. 

For $\Mvarphi$ from $0$ to $\frac{\pi}{2}$, the line segment, the corresponding
segment on the $x$-axis, and the the corresponding segment on the $y$-axis
can be regarded as a right triangle. The hypotenuse is the line segment of length $1$, the 
$x$-intercept is the adjacent side, and the $y$-intercept is the opposite side. 
This matches the situation described in the previous section.
\par
Hence, the sine of the angle $\Mvarphi$ is
\[
\sin\left(\Mvarphi\right)=\frac{y_{\Mvarphi}}{1}=y_{\Mvarphi}
\]
and the cosine is
\[
\cos\left(\Mvarphi\right)=\frac{x_{\Mvarphi}}{1}=x_{\Mvarphi} \MDFPeriod
\]
Based on the description above, these definitions 
now are also valid for angles $\Mvarphi > \pi/2$. Here, the values of 
$x_{\Mvarphi}$ and $y_{\Mvarphi}$ can be negative as well, hence also sine
and cosine can be negative. If the $y$-values are plotted against the angle 
$\Mvarphi$, one obtains for the sine function the blue curve. Plotting
$y$-values against the angle $\Mvarphi$ one obtains for the cosine function 
the green curve. If the line segment is rotated in the opposite direction,
values for negative angles can be defined accordingly. 
\par

Furthermore, using Pythagoras' theorem, we have
\[
x_{\Mvarphi}^{2}+y_{\Mvarphi}^{2}=1 \MDFPeriod
\]
Replacing $x_{\Mvarphi}$ and $y_{\Mvarphi}$ by the corresponding relations 
to the trigonometric functions results for any $\Mvarphi$ in the important relation 
\[
\sin^2\left(\Mvarphi\right)+\cos^2\left(\Mvarphi\right)=1 \MDFPeriod
\]
Additionally, from the description of the sine and cosine function, it can 
be seen that the values of the cosine function do not change if 
the line segment is reflected across the $x$-axis. Hence, the cosine value of 
the angle $\Mvarphi$ is equal to the cosine value of the angle $-\Mvarphi$
(indicated in the figure below by the green line). For the sine function, 
a reflection across the $x$-axis results in a change of sign of the sine 
value (indicated in the figure below by the blue line and the violet line, 
respectively)
 

\begin{center}
\MTikzAuto{%
\begin{tikzpicture}[line width=1.5pt]
\begin{scope}[xshift=-4cm]
\draw[->] (-3,0) -- (3.5,0) node[below left] {$x$};
\draw[->] (0,-3) -- (0,3) node[below left] {$y$};
\draw[color=black!50!white] (0,0) circle(2);
\draw (0,0) -- (30:2);
%\draw[color=green] (0,0) -- node[below] {$\cos(\Mvarphi)$} ({sqrt(3)},0);
\draw[color=green] (0,0) -- ({sqrt(3)},0);
\draw[style=dashed,color=blue] (30:2) -- ++(-90:1);
\draw[style=dotted] (0,0) -- (-30:2);
\draw[style=dotted,color=blue] (-30:2) -- ++(90:1);
%Winkel:
\draw (2,0) arc(0:30:2);
\node[right] at (15:2) {$\Mvarphi$};
\draw[style=dotted] (2,0) arc(0:-30:2);
\node[right] at (-15:2) {$-\Mvarphi$};
%Punkte:
\filldraw (0,0) circle(1pt);
\filldraw (30:2) circle(1pt);
\filldraw (-30:2) circle(1pt);
\end{scope}
%
\begin{scope}[xshift=4cm]
\draw (-3,0) -- (0,0);
\draw[->] ({sqrt(3)},0) -- (3.5,0) node[below left] {$x$};
\draw[->] (0,-3) -- (0,3) node[below left] {$y$};
\draw[color=black!50!white] (0,0) circle(2);
\draw (0,0) -- (30:2);
%\draw[color=green] (0,0) -- node[below] {$\cos(\Mvarphi)$} ({sqrt(3)},0);
\draw[style=dashed,color=green] (0,0) -- ({sqrt(3)},0);
\draw[color=blue] (30:2) -- ++(-90:1);
\draw[style=dotted] (0,0) -- (-30:2);
\draw[color=blue!70!red] (-30:2) -- ++(90:1);
%Winkel:
\draw (2,0) arc(0:30:2);
\node[right] at (15:2) {$\Mvarphi$};
\draw[style=dotted] (2,0) arc(0:-30:2);
\node[right] at (-15:2) {$-\Mvarphi$};
%Punkte:
\filldraw (0,0) circle(1pt);
\filldraw (30:2) circle(1pt);
\filldraw (-30:2) circle(1pt);
\end{scope}
\end{tikzpicture}
}
\end{center}

Expressed in formulas, this is
\[
\cos(-\Mvarphi) = \cos(\Mvarphi)
\qquad \text{and} \qquad
\sin(-\Mvarphi) = -\sin(\Mvarphi)
\]
for every angle $\Mvarphi$. 
These symmetry properties are useful for many calculations. 
An elementary example is the calculation of the angle between the 
$x$-axis and the connecting line from the origin to a point in the 
Cartesian coordinate system (see also Exercise~\MRef{VBKM05_TrigonometrieAufgabeArcsinArccos}).

\begin{MExample}
Find the values of the sine, cosine, and tangent function of the angle
 $\alpha=315\MGrad$.

For $\alpha=315\MGrad$, the point $P_\alpha$ lies in the fourth quadrant.
On the unit circle it es also described by the negative angle 
$\Mvarphi = 315\MGrad - 360\MGrad = -45\MGrad$.
Therefore, we have
$\sin(315\MGrad) = \sin(-45\MGrad) = -\sin(45\MGrad)=\frac{1}{2} \sqrt{2}$
and
$\cos(315\MGrad) = \cos(-45\MGrad) = \cos(45\MGrad)=\frac{1}{2} \sqrt{2}$
as well as $\tan(315\MGrad) = \tan(-45\MGrad) = -1$.
%Die zugeh"orige Strecke bildet mit den zugeh"origen Achsenabschnitten 
%ein gleichschenkliges Dreieck. Somit gilt 
%$\left|x_\alpha\right|=\left|y_\alpha\right|$,
%woraus 
%$2\cdot\left|x_\alpha\right|^2 %
% = \left|x_\alpha\right|^2+\left|y_\alpha\right|^2
% = 1$ 
%und somit
%$\left|x_\alpha\right| %
% =\left|y_\alpha\right| %
% =\frac{1}{\sqrt 2} %
% =\frac{1}{2}\sqrt 2$
%folgt. Damit ist
%\[
%\cos\left(315\MGrad\right) %
% =x_\alpha
% =\frac{1}{2}\sqrt 2 \MDFPSpace,
%\MDFPaSpace
%\sin\left(315\MGrad\right) %
% =y_\alpha %
% = -\frac{1}{2}\sqrt 2 \MDFPSpace,
%\MDFPaSpace
%\tan\left(315\MGrad\right) %
% =\frac{y_\alpha}{x_\alpha}=-1 \MDFPeriod
%\]
\end{MExample}

\end{MXContent}


\begin{MExercises}
\MDeclareSiteUXID{VBKM05_Trigonometrie_Exercises}
\begin{MExercise}\MLabel{VBKM05_TrigonometrieAufgabeArcsinArccos}
What is the degree measure of the angle $\Mvarphi$ between the 
$x$-axis and the connecting line from the origin in the 
Cartesian coordinate system to the point 
$P_{\Mvarphi} = \MPointTwo{-\MZahl{0}{643}}{-\MZahl{0}{766}}$ on the unit circle?
Use a calculator, but do not trust it blindly!

Result: \MEquationItem{$\Mvarphi$}{\MLParsedQuestion{15}{230}{4}{ExM05Sec6TrigFktWinkel}$\MGrad$}

\begin{MHint}{Solution}
>From the coordinates of the point $P_{\Mvarphi}$ we have
\[
\cos\left(\alpha\right)=-\MZahl{0}{643} 
 \quad \text{and} \quad
\sin\left(\alpha\right)=-\MZahl{0}{766}
\MDFPeriod %%
\]
If you enter
\begin{itemize}
\item \texttt{invers(cos(-\MZahl{0}{643}))} or $\cos^{-1}$(-\MZahl{0}{643}) in the calculator,
you obtain approximately $130\MGrad$
\item \texttt{invers(sin(-\MZahl{0}{766}))} or $\sin^{-1}$(-\MZahl{0}{766}) in the calculator,
you obtain approximately $-50\MGrad$.
\end{itemize}
Moreover, you know that the point lies in the third quadrant. Thus, the angle must be in the 
range from $180\MGrad$ to $270\MGrad$.

\MTikzAuto{%
\begin{tikzpicture}[x=2.6cm, y=2.6cm,line width=0.7pt] 
%Koordinatensystem
\node (xMAX) at (1.3,0){};
\node (yMAX) at (0,1.3){};
\draw[color=black] (0,0) circle (1);
\draw[-stealth',color=black] (-1.2,0) -- (1.3,0);
\draw[-stealth',color=black] (0,-1.3) -- (0,1.3);
\draw (xMAX) node[anchor=north east] {$x$};
\draw (yMAX) node[anchor=north east] {$y$};
\foreach \x in {-1, 1}
\draw[shift={(\x,0)},thick,color=black] (0,-0.05) -- (0,0.05) (0,0) node[anchor=north west] {\normalsize $\x$};
\foreach \y in {-1, 1}
\draw[shift={(0,\y)},thick,color=black] (0.05,0) -- (-0.05,0) (0,0) node[anchor=north west] {\normalsize $\y$};
\def\cRad{1.3}
\def\cAng{50}
\draw[thick,color=black] (0,0) -- ({\cRad*cos(\cAng)},{-\cRad*sin(\cAng)});
\draw[thick,color=black] (0,0) -- (1,0);
\draw[thick,color=black] (0,0) -- ({-\cRad*cos(\cAng)},{\cRad*sin(\cAng)});
\draw[color=black, thin] (0,0) ++({-\cAng}:0.5) arc ({-\cAng}:{180-\cAng}:0.5);
\draw[color=black] (0,0) ++({-0.5*\cAng}:0.35) node {$-\cAng\MGrad$};
\draw[color=black] (0,0) ++(45:0.30) node {$130\MGrad$}; %({90-0.5*\cAng}:0.35)
\draw[color=magenta, thin] (0,0) ++({-\cAng}:0.60) arc ({-\cAng}:{180+\cAng}:0.60);
\draw[thick,color=magenta] (0,0) -- ({-cos(\cAng)},{-sin(\cAng)});
\draw[thick,color=blue] ({cos(\cAng)},0) -- ({cos(\cAng)},{-sin(\cAng)});
\draw[thick,color=black,dashed] ({-cos(\cAng)},0) -- ({-cos(\cAng)},{sin(\cAng)});
\draw[color=blue,dashed] ({-cos(\cAng)},0) -- ({-cos(\cAng)},{-sin(\cAng)-0.1});
\draw[color=black,dashed] ({-cos(\cAng)-0.1},{-sin(\cAng)}) -- (0,{-sin(\cAng)});
\draw[thick,color=black,blue] (-0.05,{-sin(\cAng)}) -- (0.05,{-sin(\cAng)});
\end{tikzpicture}
}
\begin{minipage}[b]{10cm}
The figure to the left shows that the negative cosine value corresponds to the angle $-130\MGrad$ 
and to the angle $\Mvarphi = -130\MGrad= -130\MGrad+360\MGrad = 230\MGrad$.
\par
Likewise, the negative sine value can correspond to the angle $-50\MGrad$ and to
the angle $\Mvarphi = -(-50\MGrad) + 180\MGrad = 230\MGrad$.
\par
Since this last value lies in the range stated above, the required value of the angle is 
$\Mvarphi=230\MGrad$ indicated in the figure by a pink line.
\vspace*{2cm}
\end{minipage}
\end{MHint}

\end{MExercise}


\begin{MExercise}
\begin{enumerate}
\item Let a right triangle with the right angle at the vertex $C$ and the sides 
$b = \MZahl{2}{53}\MEinheit{cm}$ and $c = \MZahl{3}{88}\MEinheit{cm}$ be given.
Calculate the values of $\sin \left( \alpha \right)$, $\sin \left( \beta \right)$,
and $a$.

Results:
\begin{itemize}
\item \MEquationItem{$\sin \left( \alpha \right)$}{\MLParsedQuestion{15}{0.7587}{4}{ExM05Sec6TrigFktWerteAlpha}}
\item \MEquationItem{$\sin \left( \beta \right)$}{\MLParsedQuestion{15}{0.65201}{4}{ExM05Sec6TrigFktWerteBeta}}
\item \MEquationItem{$a$}{\MLParsedQuestion{15}{2.9417}{4}{ExM05Sec6TrigFktWerteLaenge}$\MEinheit{cm}$}
\end{itemize}
       
\begin{MHint}{Solution}
We have
\[
a = \sqrt{c^2 - b^2}
= \sqrt{\left( \MZahl{3}{88}\MEinheit{cm} \right)^2 - \left( \MZahl{2}{53}\MEinheit{cm} \right)^2}
= \sqrt{\MZahl{15}{0544}\MEinheit{cm}^2 - \MZahl{6}{4009}\MEinheit{cm}^2}
= \sqrt{\MZahl{8}{6535}}\MEinheit{cm} \MDFPSpace,
\]
and
\[
\sin \left( \alpha \right)
= \frac{a}{c}
= \frac{\sqrt{\MZahl{8}{6535}}\MEinheit{cm}}{\MZahl{3}{88}\MEinheit{cm}}
= \frac{\sqrt{86535}}{388}
\qquad \text{and} \qquad
          \sin \left( \beta \right)
= \frac{b}{c}
= \frac{\MZahl{2}{53}\MEinheit{cm}}{\MZahl{3}{88}\MEinheit{cm}}
= \frac{253}{388} \MDFPeriod
\]
Numerically, we obtain
$a \approx \MZahl{2}{9417}\MEinheit{cm}$, $\sin \left( \alpha \right) \approx \MZahl{0}{7587}$,
and $\sin \left( \beta \right) \approx \MZahl{0}{65201}$.
\end{MHint}
\item Calculate the area $F$ of a triangle with the sides 
$a = 4\MEinheit{m}$, $c = 60\MEinheit{cm}$,
and the angle $\beta = \Mmeasuredangle(a,c) = \frac{11 \pi}{36}$.

Result: \MEquationItem{$F$}{\MLParsedQuestion{25}{4*sin(11*pi/36)*6/20}{3}{GEO8}$\MEinheit{m}^2$}
 
\MInputHint{Please, enter your result rounded to three decimal digits or as an expression. Enter
the sine of an angle as \texttt{sin(x)} and the number $\pi$ as \texttt{pi}.}
 
\begin{MHint}{Solution}
\[
\frac{\left( a \cdot \sin \left( \beta \right) \right) \cdot c}{2}
= \sin \left( \Mtfrac{11 \pi}{36} \right) \cdot \MZahl{1}{2}\MEinheit{m}^2
\approx \MZahl{0}{98298}\MEinheit{m}^2 \MDFPeriod
\]
\end{MHint}
\end{enumerate}
\end{MExercise}

\end{MExercises}

%end of content: section 6: Winkelfunktionen.



%begin of test:
\MSubsection{Final Test}
\MLabel{M05_Abschlusstest}

\begin{MTest}{Final Test Module \arabic{section}}
\MLabel{M05_Abschlusstest_Test}
\MDeclareSiteUXID{VBKM05_Abschlusstest}

\begin{MExercise} %Testaufgabe
Identify the figures below as precisely as possible by specifying the name of the type 
(preceded by an adjective if necessary) and describing as many properties 
of the figure as possible.

\begin{center}
\MTikzAuto{%
\begin{tikzpicture}[line width=1pt,scale=0.6]
\begin{scope}[xshift=-8cm] %Viereck:
\draw (-1,-1) -- (0,0) -- (1,-1) -- (0,1) -- (-1,-1);
\node at (0,-2) {$F_1$};
\end{scope}
%
\begin{scope}[xshift=-4cm] %Quadrat:
\draw (-1.5,0)  -- (0,-1.5) -- (1.5,0) -- (0,1.5) -- (-1.5,0);
\node at (0,-2) {$F_2$};
\end{scope}
%
\begin{scope}[xshift=0cm] %Parallogramm:
\draw (-0.5,-1)  -- (0.5,-2) -- (0.5,1) -- (-0.5,2) -- (-0.5,-1);
\node at (-0.5,-2) {$F_3$};
\end{scope}
%
\begin{scope}[xshift=4cm] %gleichschenkliges Dreieck:
\draw (-1.3,1)  -- (0,-2) -- (1.3,1) -- (-1.3,1);
\node at (-1,-2) {$F_4$};
\end{scope}
%
\begin{scope}[xshift=8cm] %Raute
\draw (-1.5,0)  -- (0,-1) -- (1.5,0) -- (0,1) -- (-1.5,0);
\node at (0,-2) {$F_5$};
\end{scope}
%
\end{tikzpicture}
}
\end{center}

\begin{MQuestionGroup}
\begin{tabular}[t]{cc}
Figure: & Description of the Type: \\
 $F_1$ & \MLQuestion{32}{rectangle}{ExM05TestAg11} \\
 $F_2$ & \MLQuestion{32}{square}{ExM05TestAg12} \\
 $F_3$ & \MLQuestion{32}{parallelogram}{ExM05TestAg13} \\
 $F_4$ & \MLQuestion{32}{isosceles triangle}{ExM05TestAg14} \\
 $F_5$ & \MLQuestion{32}{rhombus}{ExM05TestAg15} \\
%
\end{tabular}
\end{MQuestionGroup}
%jgl: Tests aktuell ohne Loesungen (daher auskommentiert):
%jgl: Loesung erstellt:
%\begin{MHint}{L"osung}
%\begin{tabular}[t]{cc}
%Figur & Klassenbeschreibung: \\
% $F_1$ & ist ein Viereck (vier Seiten). \\
% $F_2$ & ist ein Quadrat (vier gleich lange Seiten, Diagonalen gleich lang). \\
% $F_3$ & ist ein Parallelogramm (gegen"uberliegende Seiten parallel. \\
% $F_4$ & ist ein gleichschenkliges Dreieck (drei Seiten, davon zwei 
%   anliegende Seiten gleich lang). \\
% $F_5$ & ist eine Raute (vier gleich lange Seiten). \\
%\end{tabular}
%\end{MHint}
\end{MExercise}


\begin{MExercise} %Testaufgabe
Are the following results and statements right or wrong?
\par
\ifttm
\begin{MQuestionGroup}
\begin{tabular}{|l|l|}
 right? & \\
 \MLCheckbox{0}{ExM05TestAg20g} & % \MLCheckbox{1}{ExM05TestAg21} & %
 Every rectangle is a rhombus. \\
%
\MLCheckbox{1}{ExM05TestAg22} & % \MLCheckbox{0}{ExM05TestAg23} & %
 Every square is a parallelogram. \\
%
\MLCheckbox{1}{ExM05TestAg24} & % \MLCheckbox{0}{ExM05TestAg25} & %
 There exists exactly one square with a diagonal of $5\MEinheit{cm}$. \\
%
\MLCheckbox{1}{ExM05TestAg26} & % \MLCheckbox{0}{ExM05TestAg27} & %
 A triangle with the angles $36\MGrad$ and $54\MGrad$ is right-angled. \\
%
\MLCheckbox{0}{ExM05TestAg28} & % \MLCheckbox{1}{ExM05TestAg29} & %
 In a rectangle the sum of all (interior) angles in radian measure is equal to $4 \pi$.
\end{tabular}
%%%\begin{tabular}{|l|l|l|}
%%% richtig & falsch & \\
%%% \MLCheckbox{0}{ExM05TestAg20g} & \MLCheckbox{1}{ExM05TestAg21} & %
%%% Jedes Rechteck ist eine Raute. \\
%%%%
%%%\MLCheckbox{1}{ExM05TestAg22} & \MLCheckbox{0}{ExM05TestAg23} & %
%%% Jedes Quadrat ist ein Parallelogramm. \\
%%%%
%%%\MLCheckbox{1}{ExM05TestAg24} & \MLCheckbox{0}{ExM05TestAg25} & %
%%% Es gibt genau ein Quadrat mit einer Diagonalen von $5\MEinheit{cm}$. \\
%%%%
%%%\MLCheckbox{1}{ExM05TestAg26} & \MLCheckbox{0}{ExM05TestAg27} & %
%%% Ein Dreieck mit den Winkeln $36\MGrad$ und $54\MGrad$ ist rechtwinklig. \\
%%%%
%%%\MLCheckbox{0}{ExM05TestAg28} & \MLCheckbox{1}{ExM05TestAg29} & %
%%% In einem Viereck ist die Summe aller (Innen-)Winkel im Bogenma"s gleich $4 \pi$.
%%%\end{tabular}
\end{MQuestionGroup}
\else
\begin{MQuestionGroup}
\begin{tabular}[t]{ccp{120mm}}
 right & wrong & \\
\MLCheckbox{0}{ExM05TestAg20} & \MLCheckbox{1}{ExM05TestAg21} & %
 Every rectangle is a rhombus. \\
%
\MLCheckbox{1}{ExM05TestAg22} & \MLCheckbox{0}{ExM05TestAg23} & %
 Every square is a parallelogram. \\
%
\MLCheckbox{1}{ExM05TestAg24} & \MLCheckbox{0}{ExM05TestAg25} & %
 It exists exactly one square with a diagonal of $5\MEinheit{cm}$.  \\
%
\MLCheckbox{1}{ExM05TestAg26} & \MLCheckbox{0}{ExM05TestAg27} & %
 A triangle with the angles $36\MGrad$ and $54\MGrad$ is right-angled.  \\
%
\MLCheckbox{0}{ExM05TestAg28} & \MLCheckbox{1}{ExM05TestAg29} & %
 In a rectangle the sum of all (interior) angles in radian measure is equal to $4 \pi$.
\end{tabular}
\end{MQuestionGroup}
\fi
%jgl: Tests aktuell ohne Loesungen (daher auskommentiert):
%jgl: Loesung erstellt:
%\begin{MHint}{L"osung}
%\begin{itemize}
%\item Nicht jedes Rechteck ist eine Raute, da es Rechtecke gibt, bei denen
% nicht alle Seiten gleich lang sind.
%\item Jedes Quadrat ist ein Parallelogramm, da bei einem Quadrat gegen"uber 
% liegende Seiten parallel sind.
%\item Ein Quadrat ist durch die L"ange der Diagonalen eindeutig festgelegt,
% alle Winkel bekannt sind und die Diagonale das Quadrat in zwei Dreiecke 
% zerlegt. Von den beiden Dreiecken sind dann alle Winkel und eine Seite 
% bekannt sind, sodass sie eindeutig festgelegt sind.
% Somit gibt es genau ein (kongruentes) Quadrat mit einer Diagonalen von 
% $5\MEinheit{cm}$.
%\item Ein Dreieck mit den Winkeln $36\MGrad$ und $54\MGrad$ ist rechtwinklig,
% da der dritte (Innen-)Winkel $180\MGrad - (36\MGrad + 54\MGrad) = 90\MGrad$
% misst. 
%\item In einem Viereck ist die Summe aller Winkel im Gradma"s gleich 
% $2 \cdot 180\MGrad = 360\MGrad$ beziehungsweise $2 \cdot \pi = 2\pi \neq 4\pi$,
% da jedes Viereck in zwei Dreiecke zerlegt werden kann.
%\end{itemize}
\end{MExercise}


\begin{MExercise} %Testaufgabe: Gemeinsame Abschlussaufgabe der 
% TU9-Brueckenkurse von OMB+ und VE&MINT:
Consider a triangle ABC with side lengths
$a = 5\MEinheit{cm}$,
$b = 6\MEinheit{cm}$, and
$c = 9\MEinheit{cm}$. On the side $c$ a point $P$ and on the side $b$ a point $Q$ 
are chosen such that $PQ$ is parallel to the side $a$ and 
$\MGeoAbstand{P}{Q} = \MZahl{0}{50}\MEinheit{cm}$. Calculate the lengths 
of the line segments $\MGeoAbstand{P}{B}$ and $\MGeoAbstand{Q}{C}$ specified in centimetre.
\par
\begin{MExerciseItems}
\item \MEquationItem{$\MGeoAbstand{P}{B}$}{\MLParsedQuestion{15}{8.1}{10}{ExM05TestAg31}$\MEinheit{cm}$}
\item \MEquationItem{$\MGeoAbstand{Q}{C}$}{\MLParsedQuestion{15}{5.4}{10}{ExM05TestAg32}$\MEinheit{cm}$}
\end{MExerciseItems}
%jgl: Tests aktuell ohne Loesungen (daher auskommentiert):
%jgl: Loesung erstellt:
%\begin{MHint}{L"osung}
%Mit den Strahlens"atzen werden zun"achst die Streckenl"angen
%$\MGeoAbstand{A}{P}$ und $\MGeoAbstand{A}{Q}$ berechnet:
%Aus
%\[
%\frac{\MGeoAbstand{A}{P}}{\MGeoAbstand{P}{Q}} = \frac{c}{a} %%
%\quad\text{und}\quad
%\frac{\MGeoAbstand{A}{Q}}{\MGeoAbstand{Q}{P}} = \frac{b}{a} %%
%\]
%folgt
%\[
% \MGeoAbstand{A}{P} = \frac{\MGeoAbstand{P}{Q} \cdot c}{a} %
% = \frac{\MZahl{0}{5} \cdot 9\MEinheit{cm}{5\MEinheit{cm} %%
% = \MZahl{0}{9}\MEinheit{cm} \MDFPeriod %%
%\]
%und
%\[
% \MGeoAbstand{A}{Q} = \frac{\MGeoAbstand{Q}{P} \cdot b}{a} %
% = \frac{\MZahl{0}{5} \cdot 6\MEinheit{cm}{5\MEinheit{cm} %
% = \MZahl{0}{6}\MEinheit{cm} \MDFPeriod %%
%\]
%Damit ergeben sich die gesuchten L"angen zu
%$\MGeoAbstand{P}{B} = 9\MEinheit{cm} - \MZahl{0}{9}\MEinheit{cm} %
% = \MZahl{8}{1}\MEinheit{cm}
%und
%$\MGeoAbstand{Q}{C} = 6\MEinheit{cm} - \MZahl{0}{6}\MEinheit{cm} %
% = \MZahl{5}{4}\MEinheit{cm}
%ergeben.
\end{MExercise}


\begin{MExercise}
\MLabel{VBKM05_A_Quadratkreise}
Consider a square with sides of length $a$. Find the formulas for the area and the circumference
for the largest circle inscribed to the square as well as for the smallest circle containing the 
square completely:
\begin{MExerciseItems}
\item{Circumference of the circle within the square as a function of the side length $a$: \MLSimplifyQuestion{14}{pi*a}{14}{a}{4}{1}{ExM05TestAg41}}
%
\item{Area of the circle within the square as a function of the side length $a$: \MLSimplifyQuestion{14}{(1/4)*pi*a*a}{14}{a}{4}{1}{ExM05TestAg42}}
%
\item{Circumference of the circle around the square as a function of the side length $a$: \MLSimplifyQuestion{14}{sqrt(2)*pi*a}{14}{a}{4}{1}{ExM05TestAg43}}
%
\item{Area of the circle around the square as a function of the side length $a$: \MLSimplifyQuestion{14}{pi*a*a/2}{14}{a}{4}{1}{ExM05TestAg44}}
\end{MExerciseItems}
\par

Do not enter any brackets or radical terms. Enter, for example, $2^{0.5}$ instead of $\sqrt{2}$
to avoid the radical.
%jgl: Tests aktuell ohne Loesungen (daher auskommentiert):
%\par
%\begin{MHint}{L\"osung}
%Der mittig im Quadrat liegende Kreis besitzt den Radius $r=\frac12a$, 
%die halbe Seitenl\"ange des Quadrats. Folglich besitzt er den Umfang 
%$2\pi r =2\pi \cdot \frac12a = \pi a$ und den Fl\"acheninhalt
%$\pi r^2=\pi \cdot \frac14a^2 =\frac14\pi a^2$.\\
%\par
%Befindet sich das Quadrat dagegen mittig innerhalb des Kreises, so ist 
%sein Radius die H\"alfte der L\"ange der Diagonalen vom Quadrat. Diese 
%besitzt die L\"ange $d=\sqrt2\cdot a$. Dies folgt aus dem Satz von Pythagoras,
%da die halbe Quadratdiagonale ein rechtwinkliges Dreieck bildet mit 
%Seitenl\"ange $d$ der Hypotenuse und den beiden Katheten mit L\"angen 
%$\frac12a$: Also ist der Umfang 
%$2\pi r=2\pi\cdot \sqrt2 \cdot\frac12 a=\sqrt2\cdot \pi \cdot a$, und
%der Fl\"acheninhalt ist 
%$\pi r^2=\pi \cdot (\sqrt2\cdot \frac12\cdot a)^2=\pi \cdot\frac12 a^2$.
%\end{MHint}
\end{MExercise}

\end{MTest}
%end of test.

\printindex
\end{document}
