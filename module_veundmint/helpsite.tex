\ifttm
\MSetSubject{\MINTPhysics}
\MSubject{HELPSECTION}
\MSection{Willkommen}

\begin{MSectionStart}
\MDeclareSiteUXID{VBKM_FIRSTPAGE}
\MGlobalStart
\MPullSite

\special{html:<div class="jumbotron">
<h1 class="start" >Onlinebrückenkurs Mathematik</h1>
<p>
Willkommen!
</p>
<p>
Kursdaten werden im Browser automatisch gespeichert, eine explizite Anmeldung zum Kurs ist nicht erforderlich.
<noscript>
<b>JavaScript ist auf diesem Browser nicht aktiviert, ohne Javascript kann der Kurs nicht ausgeführt werden!</b>
</noscript>
<br />
<center><button class="stdbutton btn btn-primary btn-lg" type="button" onclick="opensite('sectionx1.1.0.html');">Gehen Sie hier direkt zum Kursanfang</button></center>
<br />
</p>
<p>
oder informieren Sie sich weiter unten über die verschiedenen Weisen, mit dem Kurs zu arbeiten.
</p>
</div>
}


\special{html:<b>Direkt in ein spezielles Lernmodul einsteigen</b>
<p>Einzelne Lernmodule und deren Abschnitte können jederzeit manuell angesteuert und bearbeitet werden:</p>}

\begin{itemize}
\item{Modul 01: \MSRef{VBKM01}{Elementares Rechnen}}
\item{Modul 02: \MSRef{VBKM02}{Gleichungen in einer Unbekannten}}
\item{Modul 03: \MSRef{VBKM03}{Ungleichungen in einer Unbekannten}}
\item{Modul 04: \MSRef{VBKM04}{Lineare Gleichungssysteme}}
\item{Modul 05: \MSRef{VBKM05}{Elementare Geometrie}}
\item{Modul 06: \MSRef{VBKM06}{Elementare Funktionen}}
\item{Modul 07: \MSRef{VBKM07}{Differentialrechnung}}
\item{Modul 08: \MSRef{VBKM08}{Integralrechnung}}
\item{Modul 09: \MSRef{VBKM09}{Orientierung im zweidimensionalen Koordinatensystem}}
\item{Modul 10: \MSRef{VBKM10}{Grundlagen der anschaulichen Vektorgeometrie}}
\item{Modul 11: \MSRef{VBKM_STOCH}{Sprechweisen der Statistik}}
\end{itemize}

Auch die Tests und die Übungsaufgaben können in beliebiger Reihenfolge und auch mehrfach bearbeitet werden.
Dazu sind einfach die Einträge im Inhaltsverzeichnis im linken Randbereich anzuklicken.



\special{html:<dl>
<dt>Mit einem ausführlichen Test starten</dt>
<dd>In <a href="test.html">diesem Modus</a> wird ein Onlinetest über die Inhalte der Module im Umfang von ca. einer Stunde bearbeitet.
Aufgrund des Testergebnisses wird eine Empfehlung zur Arbeit mit diesem Onlinekurs gegeben. </dd>}

\special{html:<dt>Module in Reihenfolge starten</dt>
<dd>In <a href="sectionx1.1.0.html">diesem Modus</a> werden die Module nacheinander abgearbeitet. Jedes Modul enthält einen
kurzen Eingangstest sowie einen Abschlusstest, so dass der Lernerfolg überprüft werden kann.</dd>}

\special{html:<dt>Nach einem Stichwort suchen</dt>
<dd>Die <a href="search.html">Stichwortsuche</a> liefert Links zu Modulinhalten vom Kurs passend zum Stichwort.
Damit können punktuelle Wissenslücken gezielt geschlossen werden.</dd>}

%\special{html:<dt>Einstellungen überprüfen</dt>
%<dd>Auf der <a href="signup.html">Einstellungsseite</a> kann festgelegt werden, wie und welche Kursdaten gespeichert werden.
%Die Standardeinstellung ist, dass die Kursdaten (insbesondere eingegebene Lösungen und Punktezahlen) im Browser
%abgespeichert werden und Lösungsdaten nur anonymisiert zu statistischen Zwecken an unsere Server geschickt werden.</dd>
%<dl>}




\special{html:<center><a href="http://creativecommons.org/licenses/by-sa/3.0/de/deed.en" target="_new"><img src="../images/cclbysa.png" alt="Bild CCL-BY-SA"></a></center>}\\
Die Bestandteile dieses Kurses stehen unter der Creative Commons License CC BY-SA 3.0 und können
kopiert oder nach Bearbeitung weiterverwendet werden solange der Ursprung (dieser Kurs) zitiert wird,
weitere Informationen zur Lizenz und Weiterverwendung finden sich \MSRef{L_COPYRIGHTCOLLECTION}{hier}.
\ \\ \ \\
Dieser Kurs wurde im Rahmen des \MExtLink{http://www.ve-und-mint.de}{VE\&MINT-Projekts} entwickelt.
\ \\ \ \\

Kursversion: \MSignatureMain (\MSignatureVersion) \\
Erstellung: \MSignatureDate \\
Lokale Version: \MSignatureLocalization \\ 
Kursvariante: \MSignatureVariant \\

\begin{html}
<br />
<img src="../images/logo_tuberlin.png" style="height:68px">&nbsp;
<img src="../images/logo_unidarmstadt.png" style="height:68px">&nbsp;
<img src="../images/logo_unihannover.jpg" style="height:68px">&nbsp;
<img src="../images/logo_kit.png" style="height:68px">&nbsp;
<img src="../images/logo_unikassel.png" style="height:68px">&nbsp;
<img src="../images/logo_unipaderborn.png" style="height:68px">&nbsp;
<img src="../images/logo_unistuttgart.jpg" style="height:68px">
\end{html}
\ \\
\begin{html}
<p>
<h3  class="start">Welche Browser kann man verwenden?</h3>
Die folgenden Browser k&ouml;nnen verwendet werden: Firefox, Internet Explorer, Chrome, Safari, Opera.<br />
Andere Browser haben zum Teil Probleme, unsere Modulseiten richtig anzuzeigen.
<br />
Wir empfehlen, nur die neusten Versionen dieser Browser mit den aktuellen Updates einzusetzen,
insbesondere kann der Kurs nicht mit veralteten Browsern (Internet Explorer 8 oder früher) bearbeitet werden.
<br />
<br />
<script type="text/javascript">
document.write("Ihr Browsertyp: " + navigator.appName + ", Browserkennung: " + navigator.userAgent);
</script>
<br />
Unsere Seiten verwenden aktive Inhalte, insbesondere <a class="start" href="http://www.mathjax.org" target="_new">MathJax</a>. Dazu muss JavaScript im Browser aktiviert sein.
<br />
<a href="http://www.mathjax.org"><img title="Powered by MathJax" src="../images/mj_logo.png" border="0" alt="Powered by MathJax" /></a><br />
<br />
<script type="text/javascript">
document.write("JavaScript ist auf diesem Browser aktiviert.");
</script>
</p>
<br/>
<p >
Bei technischen Problemen und Fragen wenden Sie sich bitte an die <a class="start" href="mailto:brueckenkurs@innocampus.tu-berlin.de">Kursbetreuung</a>.<br/>
<br/>
<br/>
</p>
\end{html}

\end{MSectionStart}


\MSubsection{Informationen und Impressum}

\begin{MXContent}{Informationen zum Kurs}{Kursinformationen}{STD}
\MDeclareSiteUXID{VBKM_COURSEINFORMATION}

Der Onlinebrückenkurs dient Studieninteressierten zur Vorbereitung sowie zur Überprüfung des fachlichen Kenntnisstands in Mathematik
und verweist auf geeignete Angebote der Hochschulen vor Ort. Der Kurs beinhaltet neben einem umfangreichen Lernmaterial diagnostische Tests zur Selbsteinschätzung
sowie zahlreiche Lernvideos und interaktive Aufgaben. Der Kurs wurde im Rahmen des VE\&MINT-Projekts erstellt.
\ \\ \ \\
VE\&MINT ist ein Kooperationsprojekt des MINT-Kollegs Baden-Württemberg
mit dem VEMINT-Konsortium, der Leibniz Universität Hannover und der Technischen Universität Berlin mit dem Ziel,
einen bundesweit frei zugänglichen Mathematik-Onlinebrückenkurs auf Basis einer freien Lizenz anzubieten und allgemein
den Austausch von Lehr- und Lernmaterial sowie Software unter den Standorten zu fördern.
\ \\ \ \\
Weitere Informationen zum Onlinekurs und zum Projekt gibt es unter \MExtLink{http://www.ve-und-mint.de}{www.ve-und-mint.de}.

Die englisch- bzw. zweisprachige Version des Kurses wurde an der TU Berlin zur Bereitstellung als Brückenkurs
für Flüchtlinge entwickelt.
\end{MXContent}

\begin{MXContent}{Autorenliste}{Autorenliste}{STD}
\MDeclareSiteUXID{VBKM_AUTHORS}
An der Erstellung dieses Kurses sind folgenden Einrichtungen beteiligt gewesen:

% MAIL bei stuttgartern noch inkorrekt
\MExtLink{http://www.mint-kolleg.de}{MINT-Kolleg Baden-Württemberg} mit Standorten \MExtLink{http://www.kit.edu}{KIT} und \MExtLink{http://www.uni-stuttgart.de}{Universität Stuttgart}:\\
\ \\
Leitung: Dr. Claudia Goll (KA/S), Dr. Tobias Bentz (KA), Dr. Norbert Röhrl (S)\\ \ \\
\ \\ \ \\
Technischer und inhaltlicher Ansprechpartner: \MExtLink{mailto:daniel.haase@kit.edu}{Dr. Daniel Haase}
\ \\ \ \\
Dozenten/Autoren:\\
\begin{tabular}{lll}
Akkar, Zineb&Dr.& \\
App, Andreas&Dr.& \\
Beer, Julia&Dipl. Geoökol.& \\
Dege, Christopher&M.A.& \\
Deißler, Juliane&Dipl.-Math.& \\
Dirmeier, Alexander&Dr.& \\
Feiler, Simon&Dr.& \\
Gulino, Harriet& \\
Haase, Daniel&Dr.& \\
Hägele, Constanze&Dr.& \\
Hankele, Vera&Dr.&\\
Hardy, Edme H.&PD Dr.& \\
Häußling, Rainer&PD Dr.& \\
Heidbüchel, Jörg&Dr.&\\
Helfrich-Schkarbanenko, Andreas&Dr.& \\
Herold, Heike&Dipl.-Ing.&\\
Hoffmann, Heiko&Dr.&\\
Karl, Inge&Dipl.-Ing.& \\
Kempf, Sonja&Dipl.-Inf.& \\
Kleb, Joachim&Dr.& \\
Koß, Rainer&Dipl.-Math.& \\
Liedtke, Jürgen&Dr.&\\
Lilli, Markus&Dr.& \\
Merkt, Domnic&Dr.& \\
Nese, Chandrasekhar&Dr.& \\
Pintschovius, Ursel&Dipl.-Biol.& \\
Pohl, Tanja&Dipl.-Ing.& \\
Rapedius, Kevin&Dr.& \\
Rutka, Vita&Dr.& \\
Schulz, Monika&Dr.& \\
Schüpp-Niewa, Barbara&Dr.& \\
Sternal, Oliver&Dr.& \\
Stroh, Tilo&Dr.& \\
Vettin, Laura&Dipl.-Math.& \\
Walliser, Nils-Ole&Dr.& \\
Weyreter, Gunther&Dr.-Ing.& \\
Ziebarth, Eva&Dr.&
\end{tabular}
\ \\ \ \\
\MExtLink{http://www.vemint.de}{VEMINT-Konsortium} mit Standorten Kassel, Paderborn und Darmstadt:\\
\begin{itemize}
\item{Prof. Dr. Wolfram Koepf (KS)}
\item{Prof. Dr. Rolf Biehler (PB)}
\item{Prof. Dr. Regina Bruder (DA)}
\end{itemize}
% Dozenten/Autoren:\\
% \begin{tabular}{lll}
% ? & ? &?
% \end{tabular}
\ \\ \ \\
\MExtLink{http://www.innocampus.tu-berlin.de}{innoCampus} an der \MExtLink{http://www.tu-berlin.de}{TU Berlin}:\\
\begin{tabular}{lll}
Born, Stefan & Dr.& \MExtLink{mailto:born@math.tu-berlin.de}{eMail}\\
Zorn, Erhard & Dipl. Phys. & \MExtLink{mailto:erhard@math.tu-berlin.de}{eMail}\\
\end{tabular}
An der Entwicklung des zweisprachigen Kurses waren beteiligt:
\begin{tabular}{ll}
Ortiz Troncoso, Alvaro & Programmierer \\
Plessing, Niklas & Programmierer \\
Hummel, Clara & Tutorin \\
Jesche, Tobias & Tutor \\
Hauß, Jella & Koordination \\
\end{tabular}
Die englische Übersetzung wurde im Auftrage der TU Berlin von Frau Micaela Krieger-Hauwede 
durchgeführt.
\ \\ \ \\
Die \MExtLink{http://www.uni-hannover.de}{Leibniz Universität Hannover} über
\begin{itemize}
\item{das Institut für Didaktik der Mathematik und Physik (Prof. Dr. Reinhard Hochmuth),}
\item{das Institut für Algebraische Geometrie (Dr. Anne Frühbis-Krüger).}
\end{itemize}
\end{MXContent}

\begin{MXContent}{Impressum}{Impressum}{info}
\MDeclareSiteUXID{VBKM_IMPRESSUM}
\MLabel{L_COPYRIGHTCOLLECTION}

Sämtliche Inhalte und Materialien dieses Kurses werden unter
der offenen Lizenz \MExtLink{http://creativecommons.org/licenses/by-sa/3.0/de/deed.en}{CC BY-SA 3.0}
veröffentlicht und können (bei Nennung des Urhebers) weiterverwendet oder angepasst werden. Autoren- und Urheberrechte des MINT-Kollegs Baden-Württemberg und
der VEMINT-Standorte bleiben davon unberührt.
\ \\ \ \\
Bei der Weiterverwendung von Kursmaterialien sind anzugeben: Der Name des Urhebers (das VE\&MINT-Projekt), die verwendete Lizenz samt einem URI/URL, sowie ein Hinweis falls
die Inhalte verändert wurden. Hier ein Beispiel:

\begin{center}
\begin{html}
<img src="../../images/tbeispiel.png" alt="Beispielbild CC BY-SA"><br />
\end{html}
Unter Verwendung von Materialien des VE\&MINT-Projekts erstellt von H. Mustermann, Lizenz \MExtLink{http://creativecommons.org/licenses/by-sa/3.0/de/deed.en}{CC BY-SA 3.0 de}.
\end{center}
\ \\

In den Onlinemodulen werden folgende unter einer freien Lizenz stehenden Materialien eingesetzt:
\begin{itemize}
\item{Icons, die ausgehend von Material aus der \MExtLink{http://openiconlibrary.sourceforge.net}{Open Icon Library} erstellt bzw. angepasst wurden.}
\item{Grafiken, die mit \MExtLink{http://www.gimp.org}{GIMP} erstellt wurden.}
\item{Funktionsplots, die mit \MExtLink{http://pgf.sourceforge.net}{PGF/TikZ} generiert oder aus einem CAS exportiert wurden.}
\item{Videos, die am \MExtLink{http://www.zml.kit.edu}{Zentrum für mediales Lernen} am \MExtLink{http://www.kit.edu}{KIT} erstellt wurden.}
\end{itemize}


Dieser Onlinekurs wurde im Rahmen des Projekts \MExtLink{http://www.ve-und-mint.de}{VE\&MINT} von den am Projekt beteiligten Partneruniversitäten entwickelt,
Sie sehen die bundesweite und durch das \MExtLink{http://www.kit.edu}{Karlsruher Institut für Technologie} angebotene Version des Kurses. An den Partneruniversitäten
werden teilweise auf die dortigen Studiengänge angepasste Versionen des Kurses angeboten.
\ \\ \ \\
Verantwortlich für den Inhalt dieses Kurses: \MExtLink{http://www.mint-kolleg.de}{MINT-Kolleg Baden-Württemberg}, Dr. Claudia Goll. 
\ \\ \ \\
Beachten Sie die Hinweise zum \MSRef{L_HAS}{Haftungsauschluss}.
\ \\ \ \\
Einzelnachweis der Medieninhalte in diesem Kurs:\ \\
\MCopyrightCollection



\end{MXContent}

\begin{MXContent}{Haftungsauschluss}{Haftungsauschluss}{STD}
\MLabel{L_HAS}
\MDeclareSiteUXID{VBKM_LEGAL}

\MSubsubsubsectionx{Funktionalität und Gewährleistung}
Die im Rahmen dieses Kurses eingesetzten Materialien und Softwarebausteine unterliegen einer freien Lizenz, aber erheben keinen Anspruch auf inhaltliche oder technische Fehlerfreiheit. Die Inhalte wurden
sorgsam überprüft, dennoch kann insbesondere während der Anlaufphase des Kursbetriebs keinerlei Gewähr auf Richtigkeit oder technische Fehlerfreiheit der Inhalte und Software sowie auf
Verfügbarkeit der Betriebsserver übernommen werden. Dies bestrifft sowohl die clientseitigen Softwarekomponenten (insbesondere HTML5 und JS-Code) sowie die über unsere Server bezogenen Inhalte.
Für auftretende clientseitige Probleme wie z.B. fehlerhaftes Browserverhalten oder optisch verfälschende Wiedergabe unserer Inhalte können wir aufgrund der Vielzahl der Browser und der häufigen
Sicherheitsupdates für die gängigen Betriebssysteme ebenfalls keine Gewährleistung geben.

\MSubsubsubsectionx{Inhalt und Informationsübermittlung}
Der Diensteanbieter ist für eigene Inhalte auf diesen Seiten nach den allgemeinen Gesetzen verantwortlich,
jedoch nicht verpflichtet, übermittelte oder gespeicherte fremde Informationen zu überwachen oder nach Umständen zu forschen,
die auf eine rechtswidrige Tätigkeit hinweisen. Verpflichtungen zur Entfernung oder Sperrung der Nutzung von Informationen nach den allgemeinen Gesetzen bleiben hiervon unberührt.
Eine diesbezügliche Haftung ist jedoch erst ab dem Zeitpunkt der Kenntnis einer konkreten Rechtsverletzung möglich. Bei Bekanntwerden von entsprechenden Rechtsverletzungen werden wir diese Inhalte umgehend entfernen.

\MSubsubsubsectionx{Haftung für Links}
Unser Angebot enthält Links zu externen Webseiten Dritter, auf deren Inhalte wir keinen Einfluss haben. Deshalb können wir für diese fremden Inhalte auch keine Gewähr übernehmen.
Für die Inhalte der verlinkten Seiten ist stets der jeweilige Anbieter oder Betreiber der Seiten verantwortlich. Die verlinkten Seiten wurden zum Zeitpunkt der Verlinkung
auf mögliche Rechtsverstöße überprüft. Rechtswidrige Inhalte waren zum Zeitpunkt der Verlinkung nicht erkennbar. Eine permanente inhaltliche Kontrolle der verlinkten Seiten ist jedoch ohne
konkrete Anhaltspunkte einer Rechtsverletzung nicht zumutbar. Bei Bekanntwerden von Rechtsverletzungen werden wir derartige Links umgehend entfernen.

\MSubsubsubsectionx{Urheberrecht}
Die durch das VE\&MINT-Projekt erstellten Inhalte sowie die eingesetzte Softare auf diesen Seiten unterliegen dem deutschen Urheberrecht.
Die Vervielfältigung, Bearbeitung, Verbreitung ist gemäß der \MExtLink{http://de.wikipedia.org/wiki/Creative_Commons}{Creative Commons License} CC BY-SA in der Version 3.0 erlaubt.
Von Dritten eingebrachte oder übernommene Inhalte stehen ebenfalls unter dieser Lizenz. Urheber- und Autorenrechte bleiben davon unberührt.
Sollten Sie auf eine Urheberrechtsverletzung aufmerksam werden, bitten wir um einen entsprechenden Hinweis. Bei Bekanntwerden von Rechtsverletzungen werden wir derartige Inhalte umgehend entfernen. 
\end{MXContent}



\MSubsection{Formeldarstellung}

\begin{MXContent}{Formeldarstellung in den Modulen}{Formeldarstellung}{STD}
\MDeclareSiteUXID{VBKM_DISPLAYINFO}

\begin{html}
<p>
Die Formeln in den Modulen werden mit Hilfe der <a href="http://www.mathjax.org" target="_new">MathJax</a>-Bibliothek dargestellt. Diese Bibliothek bietet zwei Arten der Darstellung an:
<ul>
<li> <b> MathML:</b> Dieser Modus stellt die Formelsymbole passend zum Text durch MathML-Tags dar.</li><br />
<li> <b>HTML-CSS:</b> In diesem Modus werden die Formelsymbole als Grafiken eingebunden. Diese Darstellung ist optisch angenehmer, jedoch
kostet der Aufbau der Formeln im Browser etwas Zeit. Wir empfehlen diese Darstellung der Formeln, insbesondere
wenn Ihr Browser keine ausreichende MathML-Darstellung bietet. Dies ist vor allem bei alten Browserversionen der Fall.</li>
</ul>
Die Art der Formeldarstellung k&ouml;nnen Sie einstellen, indem Sie mit der rechten Maustaste auf eine Formel klicken
und im MathJax-Menu die Punkte Settings und Math Renderer anw&auml;hlen.

<br />
<br />
Unsere Onlinemodule setzen eine aktive Internetverbindung voraus. Fehlende Formelzeichen auf den Modulseiten deuten darauf hin, dass Ihre Verbindung unterbrochen wurde oder
dass Ihr Browser das Laden der Symbole aus Sicherheitsgr&uuml;nden unterbindet. Gehen Sie in diesem Fall wie im n&auml;chsten  Abschnitt beschrieben vor.
</p>

<h3 class="start">Welche Browser kann man verwenden?</h3>
<p>
Die folgenden Browser k&ouml;nnen verwendet werden: Firefox, Internet Explorer, Opera.<br />
Wir empfehlen die Verwendung von <a href="http://www.mozilla.org/de/firefox/new/" target="_new">Firefox</a> (ab Version 3.6).<br /><br />
Der Internet-Explorer akzeptiert teilweise die Nachladung spezieller Formelelemente nicht,<br />
es wird dann bei entsprechenen Seiten eine Aufforderung an den Benutzer ausgegeben,<br />
um das Nachladen nicht abgesicherter Inhalte zu erlauben.<br /><br />
Diese Meldungen k&ouml;nnen Sie in den Sicherheitseinstellungen des Browsers ausschalten. <br />

<br />
<br />

<p>
Bei technischen Problemen und Fragen wenden Sie sich bitte an die <a class="start" href="mailto:brueckenkurs@innocampus.tu-berlin.de"> Betreuung des Kurses.<br/>
<br />
<b r/>
</p>
\end{html}

\end{MXContent}


\fi
