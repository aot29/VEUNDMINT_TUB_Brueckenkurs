\ifttm
\MSetSubject{\MINTPhysics}
\MSubject{Standortbeschreibung}
\MSection{Karlsruher Institut f�r Technologie}

\begin{MSectionStart}
\MLabel{L_LOCATION}
\MDeclareSiteUXID{LOCATION_KIT_START}
Auf den folgenden Seiten finden sich
\begin{itemize}
\item{die Beschreibung des Karlsruher Institut f�r Technologie,}
\item{die Angebote des MINT-Kollegs Baden-W�rttemberg f�r Studieninteressierte vor Ort.}
\end{itemize}
\end{MSectionStart}

\MSubsection{Das Karlsruher Institut f�r Technologie}

\begin{MXContent}{Karlsruher Institut f�r Technologie}{KIT}{STD}
\MDeclareSiteUXID{LOCATION_KIT_DESCRIPTION}
\MGlobalLocationTag
Das \MExtLink{http://www.kit.edu}{Karlsruher Institut f�r Technologie} 
ist am 1. Oktober 2009 durch den Zusammenschluss des Forschungszentrums Karlsruhe und der Universit�t Karlsruhe entstanden.
Das KIT vereint die Aufgaben einer Universit�t des Landes Baden-W�rttemberg und einer Forschungseinrichtung der Helmholtz-Gemeinschaft in Forschung, Lehre und Innovation.
Der Zusammenschluss zum KIT stellt die konsequente Fortf�hrung einer �ber Jahre andauernden engen Zusammenarbeit zweier
traditionsreicher Forschungs- und Lehrinstitutionen dar. Die Universit�t Karlsruhe, 1825 als Polytechnische Hochschule entstanden, entwickelte sich
zu einer modernen St�tte der natur-, ingenieur-, wirtschafts-, sozial- und geisteswissenschaftlichen Forschung und Lehre mit elf Fakult�ten.

\ \\ \ \\
Das KIT bietet folgende MINT-Bachelorstudieng�nge an:
\begin{itemize}
\item{Angewandte Geowissenschaften}
\item{Architektur}
\item{Bauingenieurwesen}
\item{Bioingenieurwesen}
\item{Biologie}
\item{Chemie}
\item{Chemieingenieurwesen und Verfahrenstechnik}
\item{Chemische Biologie}
\item{Elektrotechnik und Informationstechnik}
\item{Geod�sie und Geoinformatik}
\item{Geo�kologie}
\item{Geophysik}
\item{Informatik}
\item{Informationswirtschaft}
\item{Ingenieurp�dagogik Bautechnik}
\item{Ingenieurp�dagogik Elektrotechnik}
\item{Ingenieurp�dagogik Metalltechnik}
\item{Lebensmittelchemie}
\item{Maschinenbau}
\item{Materialwissenschaft und Werkstofftechnik}
\item{Mathematik}
\item{Mechanical Engineering}
\item{Mechatronik und Informationstechnik}
\item{Meteorologie}
\item{Physik}
\item{Technische Volkswirtschaftslehre}
\item{Wirtschaftsingenieurwesen}
\item{Wissenschaft - Medien - Kommunikation}
\end{itemize}

F�r diese Studieng�nge stehen f�r Studieninteressierte die Vorbereitungsangebote des \MExtLink{http://www.mint-kolleg.de}{MINT-Kollegs Baden-W�rttemberg} zur Verf�gung.
\end{MXContent}

\begin{MXContent}{MINT-Kolleg}{MINT-Kolleg}{STD}
\MDeclareSiteUXID{LOCATION_KIT_MINT}

Als zentrale wissenschaftliche Einrichtung am KIT unterst�tzt das MINT-Kolleg seit dem Wintersemester 2011/2012 Studieninteressierte und Studierende
in den ersten Fachsemestern mit einem zus�tzlichen Lehrangebot in den MINT-F�chern (Mathematik, Informatik, Naturwissenschaft und Technik).
Ziel ist es, den �bergang von der Schule an die Hochschule zu erleichtern, Studienanf�nger auf die besonderen Anforderungen eines technischen bzw.
natur- und ingenieurwissenschaftlichen Studiums vorzubereiten und somit ihren pers�nlichen Studienerfolg zu optimieren.

Mit dem individuell gestaltbaren Kursangebot des MINT-Kollegs k�nnen Sie in der Studieneingangsphase Ihre Kenntnisse in den MINT-F�chern auffrischen und festigen.
Unser Ziel ist es, Sie in der Studieneingangsphase zu unterst�tzen - f�r ein erfolgreiches Studium.

Genauere Informationen finden sich \MExtLink{http://www.mint-kolleg.kit.edu/studienvorbereitung.php}{hier}.

Vor Beginn des Wintersemesters wird ein Pr�senzvorkurs Mathematik angeboten, der unabh�ngig von der Teilnahme am Onlinekurs belegt werden kann.

\end{MXContent}


\fi
