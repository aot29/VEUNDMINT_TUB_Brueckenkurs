\ifttm
\MSetSubject{\MINTPhysics}
\MSubject{Sonstiges}
\MSection{Benutzereinstellungen}

\begin{MSectionStart}
Auf den folgenden Seiten finden sich
\begin{itemize}
\item{die \MSRef{L_CHAPTER}{Kapitel�bersicht} des Kurses,}
\item{die pers�nlichen \MSRef{L_CDATA}{Kursdaten},}
\item{die \MSRef{L_CONFIG}{Einstellungen} f�r den Onlinekurs.}
\end{itemize}
\end{MSectionStart}

\MSubsection{Kapitel�bersicht}

\begin{MXContent}{Kapitel�bersicht}{Kapitel}{STD}
\MLabel{L_CHAPTER}
\MGlobalChapterTag
Einzelne Lernmodule und deren Abschnitte k�nnen jederzeit manuell angesteuert und bearbeitet werden:
\begin{itemize}
\item{Modul 01: \MSRef{VBKM01}{Elementares Rechnen}}
\item{Modul 02: \MSRef{VBKM02}{Gleichungen in einer Unbekannten}}
\item{Modul 03: \MSRef{VBKM03}{Ungleichungen in einer Unbekannten}}
\item{Modul 04: \MSRef{VBKM04}{Lineare Gleichungssysteme}}
\item{Modul 05: \MSRef{VBKM05}{Elementare Geometrie}}
\item{Modul 06: \MSRef{VBKM06}{Elementare Funktionen}}
\item{Modul 07: \MSRef{VBKM07}{Differentialrechnung}}
\item{Modul 08: \MSRef{VBKM08}{Integralrechnung}}
\item{Modul 09: \MSRef{VBKM09}{Orientierung im zweidimensionalen Koordinatensystem}}
\item{Modul 10: \MSRef{VBKM_STOCH}{Sprechweisen der Statistik}}
\end{itemize}

Auch die Tests und die �bungsaufgaben k�nnen in beliebiger Reihenfolge und auch mehrfach bearbeitet werden.
Dazu sind einfach die Eintr�ge im Inhaltsverzeichnis im linken Randbereich anzuklicken.
\end{MXContent}

\MSubsection{Meine Kursdaten}

\begin{MXContent}{Meine Kursdaten}{Kursdaten}{STD}
\MLabel{L_CDATA}
\MGlobalDataTag

Die pers�nlichen Daten k�nnen auf der \MSRef{L_CONFIG}{Einstellungsseite} eingerichtet werden.

Der Bearbeitungsstand der Kursmodule wird aufgrund der bearbeiteten �bungsaufgaben sowie der Ergebnisse
der Abschlusstests gemessen.

\color{red}
Sie m�ssen nicht alle Module vollst�ndig durcharbeiten, Sie sollten aber die Abschlusstests erfolgreich bearbeitet haben.
\color{black}

\begin{html}
<div>
<p id="CDATAS">
Kann nicht auf Browserdaten zugreifen!
</p>
</div>
\end{html}

% \begin{html}
%  <div class="progress">
%   <div class="progress-bar progress-bar-striped active" role="progressbar" aria-valuenow="40" aria-valuemin="0" aria-valuemax="100" style="width:40\%">
%     40\%
%   </div>
% </div>
% \end{html}

\end{MXContent}


\MSubsection{Einstellungen}

\begin{MXContent}{Einstellungen}{Einstellungen}{STD}
\MLabel{L_CONFIG}
\MGlobalConfTag

\begin{html}
<div class="xreply">
<p id="LOGINFIELD">
Kann nicht auf Browserdaten zugreifen!
</p>
</div>

\end{html}

% <button type="button" id="LOGINBUTTON" style="height:60px;width:250px;background:#D0E0E0" onclick="userlogin_click();">Benutzer wechseln</button><br />


\begin{MInfo}
\MLabel{L_LS}
\textbf{Information zur Datenspeicherung}
\ \\ \ \\
Bei Aktivierung der Datenspeicherung werden die Kursdaten (insbesondere eingegebene L�sungen) in Ihrem Browser sowie auf einem Server des KIT gespeichert.
Die Testergebnisse gehen in keiner Weise in Ihr Studium ein, werden aber statistisch ausgewertet um Informationen �ber den Jahrgang und den Erfolg des Vorkurses zu erhalten.
Zudem wird Ihr Nutzungsverhalten anonym kommuniziert um den Kurs zu verbessern, beispielsweise welche mathematischen Symbole Sie bei der L�sung der Aufgaben einsetzen.

Ihre Testergebnisse werden den Tutoren Ihrer Gruppe kommuniziert, damit diese das Niveau des Tutoriums einstellen k�nnen.

Falls Sie die Speicherung Ihrer Daten nicht w�nschen k�nnen Sie in den folgenden Einstellungsoptionen das Verhalten des Kurses anpassen.
\end{MInfo}

Daten werden wie folgt kommuniziert oder gespeichert:

\begin{itemize}
\item{\MConfigbox{CF_LOCAL} Kursdaten werden im Browser und auf einem KIT-Server gespeichert.}
\item{\MConfigbox{CF_USAGE} Nutzungsverhalten im Kurs wird anonym kommuniziert.}
\item{\MConfigbox{CF_TESTS} Ergebnisse der Tests werden kommuniziert.}
\end{itemize}
\begin{html}
<div class="xreply">
<p id="CHECKIS">
Kann nicht auf Browserdaten zugreifen!
</p>
</div>
\end{html}


% \ \\ \ \\
% Ihre lokal gespeicherten Daten:
% 
% \begin{html}
% <textarea name="Name_OBJARRAYS" id="OBJARRAYS" rows="40" cols="80" style="background-color:#F0F0F0"></textarea>
% <script>
% var e = document.getElementById("OBJARRAYS");
% e.value = "?";
% e.readOnly = true;
% </script>
% \end{html}
% 
% \ \\ \ \\
% Das intersiteobj:
% 
% \begin{html}
% <textarea name="Name_OBJOUT" id="OBJOUT" rows="40" cols="80" style="background-color:#F0F0F0"></textarea>
% <script>
% var mys = JSON.stringify(intersiteobj);
% console.log("Decoded intersiteobj");
% var e = document.getElementById("OBJOUT");
% e.value = mys;
% e.readOnly = true;
% </script>
% \end{html}


\end{MXContent}

% \MSubsection{Lesezeichen}
% 
% \begin{MSContent}{Lesezeichen}{Lesezeichen}{STD}
% \MLabel{L_FAVORITES}
% \MGlobalFavoTag
% 
% ?
% 
% \end{MSContent}

\MSubsection{Suchen}

\begin{MSContent}{Suchen}{Suchen}{STD}
\MLabel{L_SEARCHSITE}
\MGlobalSearchTag

Mit Strg-F sucht der Browser nach Schlagw�rtern in der Liste. Folgende Stichw�rter werden im Onlinekurs erkl�rt:

\special{html:<!-- msearchtable //-->}

\end{MSContent}


% \MSubsection{Hinweise Beta-Version}
% 
% \begin{MSContent}{Hinweise Beta-Version}{Hinweise}{STD}
% \MLabel{L_BETA}
% \MGlobalBetaTag
% 
% \textbf{Dieser Kurs befindet sich in der Beta-Version.}
% \ \\ \ \\
% Einige Features sind noch nicht verf�gbar und technische sowie inhaltiche Fehler k�nnen trotz sorgf�ltiger Pr�fung des Materials nicht ausgeschlossen werden.
% \ \\ \ \\
% �ber den Feedback-Button kann eine Mail an den Kursadministrator geschickt werden, bitte melden Sie inhaltliche Fehler oder technische Probleme. Beachten Sie dabei,
% dass der Kurs nur mit den neueren Versionen der auf der Startseite gelisteten Browser bearbeitet werden kann und f�r eine Nutzung an einem PC oder Notebook/Tablet mit
% einer gen�gend hohen Aufl�sung (mindestens 1024x1020) konzipiert ist. F�r mobile Endger�te mit kleinerer Aufl�sung ist der Kurs nicht geeignet.
% \ \\ \ \\
% Der endg�ltige Release sowie der Einsatz im Massenbetrieb an den Standorten des VE\&MINT-Projekts ist zum Juli 2015 geplant.
% \end{MSContent}
% 
% \fi
