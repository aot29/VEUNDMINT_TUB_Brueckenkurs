% MINTMOD Version P0.1.0, needs to be consistent with preprocesser object in tex2x and MPragma-Version at the end of this file

% Parameter aus Konvertierungsprozess (PDF und HTML-Erzeugung wenn vom Konverter aus gestartet) werden hier eingefuegt, Preambleincludes werden am Schluss angehaengt

\newif\ifttm                % gesetzt falls Uebersetzung in HTML stattfindet, sonst uebersetzung in PDF

% Wahl der Notationsvariante ist im PDF immer std, in der HTML-Uebersetzung wird vom Konverter die Auswahl modifiziert
\newif\ifvariantstd
\newif\ifvariantunotation
\variantstdtrue % Diese Zeile wird vom Konverter erkannt und ggf. modifiziert, daher nicht veraendern!


\def\MOutputDVI{1}
\def\MOutputPDF{2}
\def\MOutputHTML{3}
\newcounter{MOutput}

\ifttm
\usepackage{german}
\usepackage{array}
\usepackage{amsmath}
\usepackage{amssymb}
\usepackage{amsthm}
\else
\documentclass[ngerman,oneside]{scrbook}
\usepackage{etex}
\usepackage[latin1]{inputenc}
\usepackage{textcomp}
\usepackage[ngerman]{babel}
\usepackage[pdftex]{color}
\usepackage{xcolor}
\usepackage{graphicx}
\usepackage[all]{xy}
\usepackage{fancyhdr}
\usepackage{verbatim}
\usepackage{array}
\usepackage{float}
\usepackage{makeidx}
\usepackage{amsmath}
\usepackage{amstext}
\usepackage{amssymb}
\usepackage{amsthm}
\usepackage[ngerman]{varioref}
\usepackage{framed}
\usepackage{supertabular}
\usepackage{longtable}
\usepackage{maxpage}
\usepackage{tikz}
\usepackage{tikzscale}
\usepackage{tikz-3dplot}
\usepackage{bibgerm}
\usepackage{chemarrow}
\usepackage{polynom}
%\usepackage{draftwatermark}
\usepackage{pdflscape}
\usetikzlibrary{calc}
\usetikzlibrary{through}
\usetikzlibrary{shapes.geometric}
\usetikzlibrary{arrows}
\usetikzlibrary{intersections}
\usetikzlibrary{decorations.pathmorphing}
\usetikzlibrary{external}
\usetikzlibrary{patterns}
\usetikzlibrary{fadings}
\usepackage[colorlinks=true,linkcolor=blue]{hyperref} 
\usepackage[all]{hypcap}
%\usepackage[colorlinks=true,linkcolor=blue,bookmarksopen=true]{hyperref} 
\usepackage{ifpdf}

\usepackage{movie15}

\setcounter{tocdepth}{2} % In Inhaltsverzeichnis bis subsection
\setcounter{secnumdepth}{3} % Nummeriert bis subsubsection

\setlength{\LTpost}{0pt} % Fuer longtable
\setlength{\parindent}{0pt}
\setlength{\parskip}{8pt}
%\setlength{\parskip}{9pt plus 2pt minus 1pt}
\setlength{\abovecaptionskip}{-0.25ex}
\setlength{\belowcaptionskip}{-0.25ex}
\fi

\ifttm
\newcommand{\MDebugMessage}[1]{\special{html:<!-- debugprint;;}#1\special{html:; //-->}}
\else
%\newcommand{\MDebugMessage}[1]{\immediate\write\mintlog{#1}}
\newcommand{\MDebugMessage}[1]{}
\fi

\def\MPageHeaderDef{%
\pagestyle{fancy}%
\fancyhead[r]{(C) VE\&MINT-Projekt}
\fancyfoot[c]{\thepage\\--- CCL BY-SA 3.0 ---}
}


\ifttm%
\def\MRelax{}%
\else%
\def\MRelax{\relax}%
\fi%

%--------------------------- Uebernahme von speziellen XML-Versionen einiger LaTeX-Kommandos aus xmlbefehle.tex vom alten Kasseler Konverter ---------------

\newcommand{\MSep}{\left\|{\phantom{\frac1g}}\right.}

\newcommand{\ML}{L}

\newcommand{\MGGT}{\mathrm{ggT}}


\ifttm
% Verhindert dass die subsection-nummer doppelt in der toccaption auftaucht (sollte ggf. in toccaption gefixt werden so dass diese Ueberschreibung nicht notwendig ist)
\renewcommand{\thesubsection}{}
% Kommandos die ttm nicht kennt
\newcommand{\binomial}[2]{{#1 \choose #2}} %  Binomialkoeffizienten
\newcommand{\eur}{\begin{html}&euro;\end{html}}
\newcommand{\square}{\begin{html}&square;\end{html}}
\newcommand{\glqq}{"'}  \newcommand{\grqq}{"'}
\newcommand{\nRightarrow}{\special{html: &nrArr; }}
\newcommand{\nmid}{\special{html: &nmid; }}
\newcommand{\nparallel}{\begin{html}&nparallel;\end{html}}
\newcommand{\mapstoo}{\begin{html}<mo>&map;</mo>\end{html}}

% Schnitt und Vereinigungssymbole von Mengen haben zu kleine Abstaende; korrigiert:
\newcommand{\ccup}{\,\!\cup\,\!}
\newcommand{\ccap}{\,\!\cap\,\!}


% Umsetzung von mathbb im HTML
\renewcommand{\mathbb}[1]{\begin{html}<mo>&#1opf;</mo>\end{html}}
\fi

%---------------------- Strukturierung ----------------------------------------------------------------------------------------------------------------------

%---------------------- Kapselung des sectioning findet auf drei Ebenen statt:
% 1. Die LateX-Befehl
% 2. Die D-Versionen der Befehle, die nur die Grade der Abschnitte umhaengen falls notwendig
% 3. Die M-Versionen der Befehle, die zusaetzliche Formatierungen vornehmen, Skripten starten und das HTML codieren
% Im Modultext duerfen nur die M-Befehle verwendet werden!

\ifttm

  \def\Dsubsubsubsection#1{\subsubsubsection{#1}}
  \def\Dsubsubsection#1{\subsubsection{#1}\addtocounter{subsubsection}{1}} % ttm-Fehler korrigieren
  \def\Dsubsection#1{\subsection{#1}}
  \def\Dsection#1{\section{#1}} % Im HTML wird nur der Sektionstitel gegeben
  \def\Dchapter#1{\chapter{#1}}
  \def\Dsubsubsubsectionx#1{\subsubsubsection*{#1}}
  \def\Dsubsubsectionx#1{\subsubsection*{#1}}
  \def\Dsubsectionx#1{\subsection*{#1}}
  \def\Dsectionx#1{\section*{#1}}
  \def\Dchapterx#1{\chapter*{#1}}

\else

  \def\Dsubsubsubsection#1{\subsubsection{#1}}
  \def\Dsubsubsection#1{\subsection{#1}}
  \def\Dsubsection#1{\section{#1}}
  \def\Dsection#1{\chapter{#1}}
  \def\Dchapter#1{\title{#1}}
  \def\Dsubsubsubsectionx#1{\subsubsection*{#1}}
  \def\Dsubsubsectionx#1{\subsection*{#1}}
  \def\Dsubsectionx#1{\section*{#1}}
  \def\Dsectionx#1{\chapter*{#1}}

\fi

\newcommand{\MStdPoints}{4}
\newcommand{\MSetPoints}[1]{\renewcommand{\MStdPoints}{#1}}

% Befehl zum Abbruch der Erstellung (nur PDF)
\newcommand{\MAbort}[1]{\err{#1}}

% Prefix vor Dateieinbindungen, wird in der Baumdatei mit \renewcommand modifiziert
% und auf das Verzeichnisprefix gesetzt, in dem das gerade bearbeitete tex-Dokument liegt.
% Im HTML wird es auf das Verzeichnis der HTML-Datei gesetzt.
% Das Prefix muss mit / enden !
\newcommand{\MDPrefix}{.}

% MRegisterFile notiert eine Datei zur Einbindung in den HTML-Baum. Grafiken mit MGraphics werden automatisch eingebunden.
% Mit MLastFile erhaelt man eine Markierung fuer die zuletzt registrierte Datei.
% Diese Markierung wird im postprocessing durch den physikalischen Dateinamen ersetzt, aber nur den Namen (d.h. \MMaterial gehoert noch davor, vgl Definition von MGraphics)
% Parameter: Pfad/Name der Datei bzw. des Ordners, relativ zur Position des Modul-Tex-Dokuments.
\ifttm
\newcommand{\MRegisterFile}[1]{\addtocounter{MFileNumber}{1}\special{html:<!-- registerfile;;}#1\special{html:;;}\MDPrefix\special{html:;;}\arabic{MFileNumber}\special{html:; //-->}}
\else
\newcommand{\MRegisterFile}[1]{\addtocounter{MFileNumber}{1}}
\fi

% Testen welcher Uebersetzer hier am Werk ist

\ifttm
\setcounter{MOutput}{3}
\else
\ifx\pdfoutput\undefined
  \pdffalse
  \setcounter{MOutput}{\MOutputDVI}
  \message{Verarbeitung mit latex, Ausgabe in dvi.}
\else
  \setcounter{MOutput}{\MOutputPDF}
  \message{Verarbeitung mit pdflatex, Ausgabe in pdf.}
  \ifnum \pdfoutput=0
    \pdffalse
  \setcounter{MOutput}{\MOutputDVI}
  \message{Verarbeitung mit pdflatex, Ausgabe in dvi.}
  \else
    \ifnum\pdfoutput=1
    \pdftrue
  \setcounter{MOutput}{\MOutputPDF}
  \message{Verarbeitung mit pdflatex, Ausgabe in pdf.}
    \fi
  \fi
\fi
\fi

\ifnum\value{MOutput}=\MOutputPDF
\DeclareGraphicsExtensions{.pdf,.png,.jpg}
\fi

\ifnum\value{MOutput}=\MOutputDVI
\DeclareGraphicsExtensions{.eps,.png,.jpg}
\fi

\ifnum\value{MOutput}=\MOutputHTML
% Wird vom Konverter leider nicht erkannt und daher in split.pm hardcodiert!
\DeclareGraphicsExtensions{.png,.jpg,.gif}
\fi

% Umdefinition der hyperref-Nummerierung im PDF-Modus
\ifttm
\else
\renewcommand{\theHfigure}{\arabic{chapter}.\arabic{section}.\arabic{figure}}
\fi

% Makro, um in der HTML-Ausgabe die zuerst zu oeffnende Datei zu kennzeichnen
\ifttm
\newcommand{\MGlobalStart}{\special{html:<!-- mglobalstarttag -->}}
\else
\newcommand{\MGlobalStart}{}
\fi

% Makro, um bei scormlogin ein pullen des Benutzers bei Aufruf der Seite zu erzwingen (typischerweise auf der Einstiegsseite)
\ifttm
\newcommand{\MPullSite}{\special{html:<!-- pullsite //-->}}
\else
\newcommand{\MPullSite}{}
\fi

% Makro, um in der HTML-Ausgabe die Kapiteluebersicht zu kennzeichnen
\ifttm
\newcommand{\MGlobalChapterTag}{\special{html:<!-- mglobalchaptertag -->}}
\else
\newcommand{\MGlobalChapterTag}{}
\fi

% Makro, um in der HTML-Ausgabe die Konfiguration zu kennzeichnen
\ifttm
\newcommand{\MGlobalConfTag}{\special{html:<!-- mglobalconfigtag -->}}
\else
\newcommand{\MGlobalConfTag}{}
\fi

% Makro, um in der HTML-Ausgabe die Standortbeschreibung zu kennzeichnen
\ifttm
\newcommand{\MGlobalLocationTag}{\special{html:<!-- mgloballocationtag -->}}
\else
\newcommand{\MGlobalLocationTag}{}
\fi

% Makro, um in der HTML-Ausgabe die persoenlichen Daten zu kennzeichnen
\ifttm
\newcommand{\MGlobalDataTag}{\special{html:<!-- mglobaldatatag -->}}
\else
\newcommand{\MGlobalDataTag}{}
\fi

% Makro, um in der HTML-Ausgabe die Suchseite zu kennzeichnen
\ifttm
\newcommand{\MGlobalSearchTag}{\special{html:<!-- mglobalsearchtag -->}}
\else
\newcommand{\MGlobalSearchTag}{}
\fi

% Makro, um in der HTML-Ausgabe die Favoritenseite zu kennzeichnen
\ifttm
\newcommand{\MGlobalFavoTag}{\special{html:<!-- mglobalfavoritestag -->}}
\else
\newcommand{\MGlobalFavoTag}{}
\fi

% Makro, um in der HTML-Ausgabe die Eingangstestseite zu kennzeichnen
\ifttm
\newcommand{\MGlobalSTestTag}{\special{html:<!-- mglobalstesttag -->}}
\else
\newcommand{\MGlobalSTestTag}{}
\fi

% Makro, um in der PDF-Ausgabe ein Wasserzeichen zu definieren
\ifttm
\newcommand{\MWatermarkSettings}{\relax}
\else
\newcommand{\MWatermarkSettings}{%
% \SetWatermarkText{(c) MINT-Kolleg Baden-W�rttemberg 2014}
% \SetWatermarkLightness{0.85}
% \SetWatermarkScale{1.5}
}
\fi

\ifttm
\newcommand{\MBinom}[2]{\left({\begin{array}{c} #1 \\ #2 \end{array}}\right)}
\else
\newcommand{\MBinom}[2]{\binom{#1}{#2}}
\fi

\ifttm
\newcommand{\DeclareMathOperator}[2]{\def#1{\mathrm{#2}}}
\newcommand{\operatorname}[1]{\mathrm{#1}}
\fi

%----------------- Makros fuer die gemischte HTML/PDF-Konvertierung ------------------------------

\newcommand{\MTestName}{\relax} % wird durch Test-Umgebung gesetzt

% Fuer experimentelle Kursinhalte, die im Release-Umsetzungsvorgang eine Fehlermeldung
% produzieren sollen aber sonst normal umgesetzt werden
\newenvironment{MExperimental}{%
}{%
}

% Wird von ttm nicht richtig umgesetzt!!
\newenvironment{MExerciseItems}{%
\renewcommand\theenumi{\alph{enumi}}%
\begin{enumerate}%
}{%
\end{enumerate}%
}


\definecolor{infoshadecolor}{rgb}{0.75,0.75,0.75}
\definecolor{exmpshadecolor}{rgb}{0.875,0.875,0.875}
\definecolor{expeshadecolor}{rgb}{0.95,0.95,0.95}
\definecolor{framecolor}{rgb}{0.2,0.2,0.2}

% Bei PDF-Uebersetzung wird hinter den Start jeder Satz/Info-aehnlichen Umgebung eine leere mbox gesetzt, damit
% fuehrende Listen oder enums nicht den Zeilenumbruch kaputtmachen
%\ifttm
\def\MTB{}
%\else
%\def\MTB{\mbox{}}
%\fi


\ifttm
\newcommand{\MRelates}{\special{html:<mi>&wedgeq;</mi>}}
\else
\def\MRelates{\stackrel{\scriptscriptstyle\wedge}{=}}
\fi

\def\MInch{\text{''}}
\def\Mdd{\textit{''}}

\ifttm
\def\MNL{ \newline }
\newenvironment{MArray}[1]{\begin{array}{#1}}{\end{array}}
\else
\def\MNL{ \\ }
\newenvironment{MArray}[1]{\begin{array}{#1}}{\end{array}}
\fi

\newcommand{\MBox}[1]{$\mathrm{#1}$}
\newcommand{\MMBox}[1]{\mathrm{#1}}


\ifttm%
\newcommand{\Mtfrac}[2]{{\textstyle \frac{#1}{#2}}}
\newcommand{\Mdfrac}[2]{{\displaystyle \frac{#1}{#2}}}
\newcommand{\Mmeasuredangle}{\special{html:<mi>&angmsd;</mi>}}
\else%
\newcommand{\Mtfrac}[2]{\tfrac{#1}{#2}}
\newcommand{\Mdfrac}[2]{\dfrac{#1}{#2}}
\newcommand{\Mmeasuredangle}{\measuredangle}
\relax
\fi

% Matrizen und Vektoren

% Inhalt wird in der Form a & b \\ c & d erwartet
% Vorsicht: MVector = Komponentenspalte, MVec = Variablensymbol
\ifttm%
\newcommand{\MVector}[1]{\left({\begin{array}{c}#1\end{array}}\right)}
\else%
\newcommand{\MVector}[1]{\begin{pmatrix}#1\end{pmatrix}}
\fi



\newcommand{\MVec}[1]{\vec{#1}}
\newcommand{\MDVec}[1]{\overrightarrow{#1}}

%----------------- Umgebungen fuer Definitionen und Saetze ----------------------------------------

% Fuegt einen Tabellen-Zeilenumbruch ein im PDF, aber nicht im HTML
\newcommand{\TSkip}{\ifttm \else&\ \\\fi}

\newenvironment{infoshaded}{%
\def\FrameCommand{\fboxsep=\FrameSep \fcolorbox{framecolor}{infoshadecolor}}%
\MakeFramed {\advance\hsize-\width \FrameRestore}}%
{\endMakeFramed}

\newenvironment{expeshaded}{%
\def\FrameCommand{\fboxsep=\FrameSep \fcolorbox{framecolor}{expeshadecolor}}%
\MakeFramed {\advance\hsize-\width \FrameRestore}}%
{\endMakeFramed}

\newenvironment{exmpshaded}{%
\def\FrameCommand{\fboxsep=\FrameSep \fcolorbox{framecolor}{exmpshadecolor}}%
\MakeFramed {\advance\hsize-\width \FrameRestore}}%
{\endMakeFramed}

\def\STDCOLOR{black}

\ifttm%
\else%
\newtheoremstyle{MSatzStyle}
  {1cm}                   %Space above
  {1cm}                   %Space below
  {\normalfont\itshape}   %Body font
  {}                      %Indent amount (empty = no indent,
                          %\parindent = para indent)
  {\normalfont\bfseries}  %Thm head font
  {}                      %Punctuation after thm head
  {\newline}              %Space after thm head: " " = normal interword
                          %space; \newline = linebreak
  {\thmname{#1}\thmnumber{ #2}\thmnote{ (#3)}}
                          %Thm head spec (can be left empty, meaning
                          %`normal')
                          %
\newtheoremstyle{MDefStyle}
  {1cm}                   %Space above
  {1cm}                   %Space below
  {\normalfont}           %Body font
  {}                      %Indent amount (empty = no indent,
                          %\parindent = para indent)
  {\normalfont\bfseries}  %Thm head font
  {}                      %Punctuation after thm head
  {\newline}              %Space after thm head: " " = normal interword
                          %space; \newline = linebreak
  {\thmname{#1}\thmnumber{ #2}\thmnote{ (#3)}}
                          %Thm head spec (can be left empty, meaning
                          %`normal')
\fi%

\newcommand{\MInfoText}{Info}

\newcounter{MHintCounter}
\newcounter{MCodeEditCounter}

\newcounter{MLastIndex}  % Enthaelt die dritte Stelle (Indexnummer) des letzten angelegten Objekts
\newcounter{MLastType}   % Enthaelt den Typ des letzten angelegten Objekts (mithilfe der unten definierten Konstanten). Die Entscheidung, wie der Typ dargstellt wird, wird in split.pm beim Postprocessing getroffen.
\newcounter{MLastTypeEq} % =1 falls das Label in einer Matheumgebung (equation, eqnarray usw.) steht, =2 falls das Label in einer table-Umgebung steht

% Da ttm keine Zahlmakros verarbeiten kann, werden diese Nummern in den Zuweisungen hardcodiert!
\def\MTypeSection{1}          %# Zaehler ist section
\def\MTypeSubsection{2}       %# Zaehler ist subsection
\def\MTypeSubsubsection{3}    %# Zaehler ist subsubsection
\def\MTypeInfo{4}             %# Eine Infobox, Separatzaehler fuer die Chemie (auch wenn es dort nicht nummeriert wird) ist MInfoCounter
\def\MTypeExercise{5}         %# Eine Aufgabe, Separatzaehler fuer die Chemie ist MExerciseCounter
\def\MTypeExample{6}          %# Eine Beispielbox, Separatzaehler fuer die Chemie ist MExampleCounter
\def\MTypeExperiment{7}       %# Eine Versuchsbox, Separatzaehler fuer die Chemie ist MExperimentCounter
\def\MTypeGraphics{8}         %# Eine Graphik, Separatzaehler fuer alle FB ist MGraphicsCounter
\def\MTypeTable{9}            %# Eine Tabellennummer, hat keinen Zaehler da durch table gezaehlt wird
\def\MTypeEquation{10}        %# Eine Gleichungsnummer, hat keinen Zaehler da durch equation/eqnarray gezaehlt wird
\def\MTypeTheorem{11}         % Ein theorem oder xtheorem, Separatzaehler fuer die Chemie ist MTheoremCounter
\def\MTypeVideo{12}           %# Ein Video,Separatzaehler fuer alle FB ist MVideoCounter
\def\MTypeEntry{13}           %# Ein Eintrag fuer die Stichwortliste, wird nicht gezaehlt sondern erhaelt im preparsing ein unique-label 

% Zaehler fuer das Labelsystem sind prefixcounter, jeder Zaehler wird VOR dem gezaehlten Objekt inkrementiert und zaehlt daher das aktuelle Objekt
\newcounter{MInfoCounter}
\newcounter{MExerciseCounter}
\newcounter{MExampleCounter}
\newcounter{MExperimentCounter}
\newcounter{MGraphicsCounter}
\newcounter{MTableCounter}
\newcounter{MEquationCounter}  % Nur im HTML, sonst durch "equation"-counter von latex realisiert
\newcounter{MTheoremCounter}
\newcounter{MObjectCounter}   % Gemeinsamer Zaehler fuer Objekte (ausser Grafiken/Tabellen) in Mathe/Info/Physik
\newcounter{MVideoCounter}
\newcounter{MEntryCounter}

\newcounter{MTestSite} % 1 = Subsubsection ist eine Pruefungsseite, 0 = ist eine normale Seite (inkl. Hilfeseite)

\def\MCell{$\phantom{a}$}

\newenvironment{MExportExercise}{\begin{MExercise}}{\end{MExercise}} % wird von mconvert abgefangen

\def\MGenerateExNumber{%
\ifnum\value{MSepNumbers}=0%
\arabic{section}.\arabic{subsection}.\arabic{MObjectCounter}\setcounter{MLastIndex}{\value{MObjectCounter}}%
\else%
\arabic{section}.\arabic{subsection}.\arabic{MExerciseCounter}\setcounter{MLastIndex}{\value{MExerciseCounter}}%
\fi%
}%

\def\MGenerateExmpNumber{%
\ifnum\value{MSepNumbers}=0%
\arabic{section}.\arabic{subsection}.\arabic{MObjectCounter}\setcounter{MLastIndex}{\value{MObjectCounter}}%
\else%
\arabic{section}.\arabic{subsection}.\arabic{MExerciseCounter}\setcounter{MLastIndex}{\value{MExampleCounter}}%
\fi%
}%

\def\MGenerateInfoNumber{%
\ifnum\value{MSepNumbers}=0%
\arabic{section}.\arabic{subsection}.\arabic{MObjectCounter}\setcounter{MLastIndex}{\value{MObjectCounter}}%
\else%
\arabic{section}.\arabic{subsection}.\arabic{MExerciseCounter}\setcounter{MLastIndex}{\value{MInfoCounter}}%
\fi%
}%

\def\MGenerateSiteNumber{%
\arabic{section}.\arabic{subsection}.\arabic{subsubsection}%
}%

% Funktionalitaet fuer Auswahlaufgaben

\newcounter{MExerciseCollectionCounter} % = 0 falls nicht in collection-Umgebung, ansonsten Schachtelungstiefe
\newcounter{MExerciseCollectionTextCounter} % wird von MExercise-Umgebung inkrementiert und von MExerciseCollection-Umgebung auf Null gesetzt

\ifttm
% MExerciseCollection gruppiert Aufgaben, die dynamisch aus der Datenbank gezogen werden und nicht direkt in der HTML-Seite stehen
% Parameter: #1 = ID der Collection, muss eindeutig fuer alle IN DER DB VORHANDENEN collections sein unabhaengig vom Kurs
%            #2 = Optionsargument (im Moment: 1 = Iterative Auswahl, 2 = Zufallsbasierte Auswahl)
\newenvironment{MExerciseCollection}[2]{%
\addtocounter{MExerciseCollectionCounter}{1}
\setcounter{MExerciseCollectionTextCounter}{0}
\special{html:<!-- mexercisecollectionstart;;}#1\special{html:;;}#2\special{html:;; //-->}%
}{%
\special{html:<!-- mexercisecollectionstop //-->}%
\addtocounter{MExerciseCollectionCounter}{-1}
}
\else
\newenvironment{MExerciseCollection}[2]{%
\addtocounter{MExerciseCollectionCounter}{1}
\setcounter{MExerciseCollectionTextCounter}{0}
}{%
\addtocounter{MExerciseCollectionCounter}{-1}
}
\fi

% Bei Uebersetzung nach PDF werden die theorem-Umgebungen verwendet, bei Uebersetzung in HTML ein manuelles Makro
\ifttm%

  \newenvironment{MHint}[1]{  \special{html:<button name="Name_MHint}\arabic{MHintCounter}\special{html:" class="hintbutton_closed" id="MHint}\arabic{MHintCounter}\special{html:_button" %
  type="button" onclick="toggle_hint('MHint}\arabic{MHintCounter}\special{html:');">}#1\special{html:</button>}
  \special{html:<div class="hint" style="display:none" id="MHint}\arabic{MHintCounter}\special{html:"> }}{\begin{html}</div>\end{html}\addtocounter{MHintCounter}{1}}

  \newenvironment{MCOSHZusatz}{  \special{html:<button name="Name_MHint}\arabic{MHintCounter}\special{html:" class="chintbutton_closed" id="MHint}\arabic{MHintCounter}\special{html:_button" %
  type="button" onclick="toggle_hint('MHint}\arabic{MHintCounter}\special{html:');">}Weiterf�hrende Inhalte\special{html:</button>}
  \special{html:<div class="hintc" style="display:none" id="MHint}\arabic{MHintCounter}\special{html:">
  <div class="coshwarn">Diese Inhalte gehen �ber das Kursniveau hinaus und werden in den Aufgaben und Tests nicht abgefragt.</div><br />}
  \addtocounter{MHintCounter}{1}}{\begin{html}</div>\end{html}}

  
  \newenvironment{MDefinition}{\begin{definition}\setcounter{MLastIndex}{\value{definition}}\ \\}{\end{definition}}

  
  \newenvironment{MExercise}{
  \renewcommand{\MStdPoints}{4}
  \addtocounter{MExerciseCounter}{1}
  \addtocounter{MObjectCounter}{1}
  \setcounter{MLastType}{5}

  \ifnum\value{MExerciseCollectionCounter}=0\else\addtocounter{MExerciseCollectionTextCounter}{1}\special{html:<!-- mexercisetextstart;;}\arabic{MExerciseCollectionTextCounter}\special{html:;; //-->}\fi
  \special{html:<div class="aufgabe" id="ADIV_}\MGenerateExNumber\special{html:">}%
  \textbf{Aufgabe \MGenerateExNumber
  } \ \\}{
  \special{html:</div><!-- mfeedbackbutton;Aufgabe;}\arabic{MTestSite}\special{html:;}\MGenerateExNumber\special{html:; //-->}
  \ifnum\value{MExerciseCollectionCounter}=0\else\special{html:<!-- mexercisetextstop //-->}\fi
  }

  % Stellt eine Kombination aus Aufgabe, Loesungstext und Eingabefeld bereit,
  % bei der Aufgabentext und Musterloesung sowie die zugehoerigen Feldelemente
  % extern bezogen und div-aktualisiert werden, das Eingabefeld aber immer das gleiche ist.
  \newenvironment{MFetchExercise}{
  \addtocounter{MExerciseCounter}{1}
  \addtocounter{MObjectCounter}{1}
  \setcounter{MLastType}{5}

  \special{html:<div class="aufgabe" id="ADIV_}\MGenerateExNumber\special{html:">}%
  \textbf{Aufgabe \MGenerateExNumber
  } \ \\%
  \special{html:</div><div class="exfetch_text" id="ADIVTEXT_}\MGenerateExNumber\special{html:">}%
  \special{html:</div><div class="exfetch_sol" id="ADIVSOL_}\MGenerateExNumber\special{html:">}%
  \special{html:</div><div class="exfetch_input" id="ADIVINPUT_}\MGenerateExNumber\special{html:">}%
  }{
  \special{html:</div>}
  }

  \newenvironment{MExample}{
  \addtocounter{MExampleCounter}{1}
  \addtocounter{MObjectCounter}{1}
  \setcounter{MLastType}{6}
  \begin{html}
  <div class="exmp">
  <div class="exmprahmen">
  \end{html}\textbf{Beispiel
  \ifnum\value{MSepNumbers}=0
  \arabic{section}.\arabic{subsection}.\arabic{MObjectCounter}\setcounter{MLastIndex}{\value{MObjectCounter}}
  \else
  \arabic{section}.\arabic{subsection}.\arabic{MExampleCounter}\setcounter{MLastIndex}{\value{MExampleCounter}}
  \fi
  } \ \\}{\begin{html}</div>
  </div>
  \end{html}
  \special{html:<!-- mfeedbackbutton;Beispiel;}\arabic{MTestSite}\special{html:;}\MGenerateExmpNumber\special{html:; //-->}
  }

  \newenvironment{MExperiment}{
  \addtocounter{MExperimentCounter}{1}
  \addtocounter{MObjectCounter}{1}
  \setcounter{MLastType}{7}
  \begin{html}
  <div class="expe">
  <div class="experahmen">
  \end{html}\textbf{Versuch
  \ifnum\value{MSepNumbers}=0
  \arabic{section}.\arabic{subsection}.\arabic{MObjectCounter}\setcounter{MLastIndex}{\value{MObjectCounter}}
  \else
%  \arabic{MExperimentCounter}\setcounter{MLastIndex}{\value{MExperimentCounter}}
  \arabic{section}.\arabic{subsection}.\arabic{MExperimentCounter}\setcounter{MLastIndex}{\value{MExperimentCounter}}
  \fi
  } \ \\}{\begin{html}</div>
  </div>
  \end{html}}

  \newenvironment{MChemInfo}{
  \setcounter{MLastType}{4}
  \begin{html}
  <div class="info">
  <div class="inforahmen">
  \end{html}}{\begin{html}</div>
  </div>
  \end{html}}

  \newenvironment{MXInfo}[1]{
  \addtocounter{MInfoCounter}{1}
  \addtocounter{MObjectCounter}{1}
  \setcounter{MLastType}{4}
  \begin{html}
  <div class="info">
  <div class="inforahmen">
  \end{html}\textbf{#1
  \ifnum\value{MInfoNumbers}=0
  \else
    \ifnum\value{MSepNumbers}=0
    \arabic{section}.\arabic{subsection}.\arabic{MObjectCounter}\setcounter{MLastIndex}{\value{MObjectCounter}}
    \else
    \arabic{MInfoCounter}\setcounter{MLastIndex}{\value{MInfoCounter}}
    \fi
  \fi
  } \ \\}{\begin{html}</div>
  </div>
  \end{html}
  \special{html:<!-- mfeedbackbutton;Info;}\arabic{MTestSite}\special{html:;}\MGenerateInfoNumber\special{html:; //-->}
  }

  \newenvironment{MInfo}{\ifnum\value{MInfoNumbers}=0\begin{MChemInfo}\else\begin{MXInfo}{Info}\ \\ \fi}{\ifnum\value{MInfoNumbers}=0\end{MChemInfo}\else\end{MXInfo}\fi}

\else%

  \theoremstyle{MSatzStyle}
  \newtheorem{thm}{Satz}[section]
  \newtheorem{thmc}{Satz}
  \theoremstyle{MDefStyle}
  \newtheorem{defn}[thm]{Definition}
  \newtheorem{exmp}[thm]{Beispiel}
  \newtheorem{info}[thm]{\MInfoText}
  \theoremstyle{MDefStyle}
  \newtheorem{defnc}{Definition}
  \theoremstyle{MDefStyle}
  \newtheorem{exmpc}{Beispiel}[section]
  \theoremstyle{MDefStyle}
  \newtheorem{infoc}{\MInfoText}
  \theoremstyle{MDefStyle}
  \newtheorem{exrc}{Aufgabe}[section]
  \theoremstyle{MDefStyle}
  \newtheorem{verc}{Versuch}[section]
  
  \newenvironment{MFetchExercise}{}{} % kann im PDF nicht dargestellt werden
  
  \newenvironment{MExercise}{\begin{exrc}\renewcommand{\MStdPoints}{1}\MTB}{\end{exrc}}
  \newenvironment{MHint}[1]{\ \\ \underline{#1:}\\}{}
  \newenvironment{MCOSHZusatz}{\ \\ \underline{Weiterf�hrende Inhalte:}\\}{}
  \newenvironment{MDefinition}{\ifnum\value{MInfoNumbers}=0\begin{defnc}\else\begin{defn}\fi\MTB}{\ifnum\value{MInfoNumbers}=0\end{defnc}\else\end{defn}\fi}
%  \newenvironment{MExample}{\begin{exmp}}{\ \linebreak[1] \ \ \ \ $\phantom{a}$ \ \hfill $\blacklozenge$\end{exmp}}
  \newenvironment{MExample}{
    \ifnum\value{MInfoNumbers}=0\begin{exmpc}\else\begin{exmp}\fi
    \MTB
    \begin{exmpshaded}
    \ \newline
}{
    \end{exmpshaded}
    \ifnum\value{MInfoNumbers}=0\end{exmpc}\else\end{exmp}\fi
}
  \newenvironment{MChemInfo}{\begin{infoshaded}}{\end{infoshaded}}

  \newenvironment{MInfo}{\ifnum\value{MInfoNumbers}=0\begin{MChemInfo}\else\renewcommand{\MInfoText}{Info}\begin{info}\begin{infoshaded}
  \MTB
   \ \newline
    \fi
  }{\ifnum\value{MInfoNumbers}=0\end{MChemInfo}\else\end{infoshaded}\end{info}\fi}

  \newenvironment{MXInfo}[1]{
    \renewcommand{\MInfoText}{#1}
    \ifnum\value{MInfoNumbers}=0\begin{infoc}\else\begin{info}\fi%
    \MTB
    \begin{infoshaded}
    \ \newline
  }{\end{infoshaded}\ifnum\value{MInfoNumbers}=0\end{infoc}\else\end{info}\fi}

  \newenvironment{MExperiment}{
    \renewcommand{\MInfoText}{Versuch}
    \ifnum\value{MInfoNumbers}=0\begin{verc}\else\begin{info}\fi
    \MTB
    \begin{expeshaded}
    \ \newline
  }{
    \end{expeshaded}
    \ifnum\value{MInfoNumbers}=0\end{verc}\else\end{info}\fi
  }
\fi%

% MHint sollte nicht direkt fuer Loesungen benutzt werden wegen solutionselect
\newenvironment{MSolution}{\begin{MHint}{L"osung}}{\end{MHint}}

\newcounter{MCodeCounter}

\ifttm
\newenvironment{MCode}{\special{html:<!-- mcodestart -->}\ttfamily\color{blue}}{\special{html:<!-- mcodestop -->}}
\else
\newenvironment{MCode}{\begin{flushleft}\ttfamily\addtocounter{MCodeCounter}{1}}{\addtocounter{MCodeCounter}{-1}\end{flushleft}}
% Ohne color-Statement da inkompatible mit framed/shaded-Boxen aus dem framed-package
\fi

%----------------- Sonderdefinitionen fuer Symbole, die der Konverter nicht kann ----------------------------------------------

\ifttm%
\newcommand{\MUnderset}[2]{\underbrace{#2}_{#1}}%
\else%
\newcommand{\MUnderset}[2]{\underset{#1}{#2}}%
\fi%

\ifttm
\newcommand{\MThinspace}{\special{html:<mi>&#x2009;</mi>}}
\else
\newcommand{\MThinspace}{\,}
\fi

\ifttm
\newcommand{\glq}{\begin{html}&sbquo;\end{html}}
\newcommand{\grq}{\begin{html}&lsquo;\end{html}}
\newcommand{\glqq}{\begin{html}&bdquo;\end{html}}
\newcommand{\grqq}{\begin{html}&ldquo;\end{html}}
\fi

\ifttm
\newcommand{\MNdash}{\begin{html}&ndash;\end{html}}
\else
\newcommand{\MNdash}{--}
\fi

%\ifttm\def\MIU{\special{html:<mi>&#8520;</mi>}}\else\def\MIU{\mathrm{i}}\fi
\def\MIU{\mathrm{i}}
\def\MEU{e} % TU9-Onlinekurs: italic-e
%\def\MEU{\mathrm{e}} % Alte Onlinemodule: roman-e
\def\MD{d} % Kursives d in Integralen im TU9-Onlinekurs
%\def\MD{\mathrm{d}} % roman-d in den alten Onlinemodulen
\def\MDB{\|}

%zusaetzlicher Leerraum vor "\MD"
\ifttm%
\def\MDSpace{\special{html:<mi>&#x2009;</mi>}}
\else%
\def\MDSpace{\,}
\fi%
\newcommand{\MDwSp}{\MDSpace\MD}%

\ifttm
\def\Mdq{\dq}
\else
\def\Mdq{\dq}
\fi

\def\MSpan#1{\left<{#1}\right>}
\def\MSetminus{\setminus}
\def\MIM{I}

\ifttm
\newcommand{\ld}{\text{ld}}
\newcommand{\lg}{\text{lg}}
\else
\DeclareMathOperator{\ld}{ld}
%\newcommand{\lg}{\text{lg}} % in latex schon definiert
\fi


\def\Mmapsto{\ifttm\special{html:<mi>&mapsto;</mi>}\else\mapsto\fi} 
\def\Mvarphi{\ifttm\phi\else\varphi\fi}
\def\Mphi{\ifttm\varphi\else\phi\fi}
\ifttm%
\newcommand{\MEumu}{\special{html:<mi>&#x3BC;</mi>}}%
\else%
\newcommand{\MEumu}{\textrm{\textmu}}%
\fi
\def\Mvarepsilon{\ifttm\epsilon\else\varepsilon\fi}
\def\Mepsilon{\ifttm\varepsilon\else\epsilon\fi}
\def\Mvarkappa{\ifttm\kappa\else\varkappa\fi}
\def\Mkappa{\ifttm\varkappa\else\kappa\fi}
\def\Mcomplement{\ifttm\special{html:<mi>&comp;</mi>}\else\complement\fi} 
\def\MWW{\mathrm{WW}}
\def\Mmod{\ifttm\special{html:<mi>&nbsp;mod&nbsp;</mi>}\else\mod\fi} 

\ifttm%
\def\mod{\text{\;mod\;}}%
\def\MNEquiv{\special{html:<mi>&NotCongruent;</mi>}}% 
\def\MNSubseteq{\special{html:<mi>&NotSubsetEqual;</mi>}}%
\def\MEmptyset{\special{html:<mi>&empty;</mi>}}%
\def\MVDots{\special{html:<mi>&#x22EE;</mi>}}%
\def\MHDots{\special{html:<mi>&#x2026;</mi>}}%
\def\Mddag{\special{html:<mi>&#x1202;</mi>}}%
\def\sphericalangle{\special{html:<mi>&measuredangle;</mi>}}%
\def\nparallel{\special{html:<mi>&nparallel;</mi>}}%
\def\MProofEnd{\special{html:<mi>&#x25FB;</mi>}}%
\newenvironment{MProof}[1]{\underline{#1}:\MCR\MCR}{\hfill $\MProofEnd$}%
\else%
\def\MNEquiv{\not\equiv}%
\def\MNSubseteq{\not\subseteq}%
\def\MEmptyset{\emptyset}%
\def\MVDots{\vdots}%
\def\MHDots{\hdots}%
\def\Mddag{\ddag}%
\newenvironment{MProof}[1]{\begin{proof}[#1]}{\end{proof}}%
\fi%



% Spaces zum Auffuellen von Tabellenbreiten, die nur im HTML wirken
\ifttm%
\def\MTSP{\:}%
\else%
\def\MTSP{}%
\fi%

\DeclareMathOperator{\arsinh}{arsinh}
\DeclareMathOperator{\arcosh}{arcosh}
\DeclareMathOperator{\artanh}{artanh}
\DeclareMathOperator{\arcoth}{arcoth}


\newcommand{\MMathSet}[1]{\mathbb{#1}}
\def\N{\MMathSet{N}}
\def\Z{\MMathSet{Z}}
\def\Q{\MMathSet{Q}}
\def\R{\MMathSet{R}}
\def\C{\MMathSet{C}}

\newcounter{MForLoopCounter}
\newcommand{\MForLoop}[2]{\setcounter{MForLoopCounter}{#1}\ifnum\value{MForLoopCounter}=0{}\else{{#2}\addtocounter{MForLoopCounter}{-1}\MForLoop{\value{MForLoopCounter}}{#2}}\fi}

\newcounter{MSiteCounter}
\newcounter{MFieldCounter} % Kombination section.subsection.site.field ist eindeutig in allen Modulen, field alleine nicht

\newcounter{MiniMarkerCounter}

\ifttm
\newenvironment{MMiniPageP}[1]{\begin{minipage}{#1\linewidth}\special{html:<!-- minimarker;;}\arabic{MiniMarkerCounter}\special{html:;;#1; //-->}}{\end{minipage}\addtocounter{MiniMarkerCounter}{1}}
\else
\newenvironment{MMiniPageP}[1]{\begin{minipage}{#1\linewidth}}{\end{minipage}\addtocounter{MiniMarkerCounter}{1}}
\fi

\newcounter{AlignCounter}

\newcommand{\MStartJustify}{\ifttm\special{html:<!-- startalign;;}\arabic{AlignCounter}\special{html:;;justify; //-->}\fi}
\newcommand{\MStopJustify}{\ifttm\special{html:<!-- stopalign;;}\arabic{AlignCounter}\special{html:; //-->}\fi\addtocounter{AlignCounter}{1}}

\newenvironment{MJTabular}[1]{
\MStartJustify
\begin{tabular}{#1}
}{
\end{tabular}
\MStopJustify
}

\newcommand{\MImageLeft}[2]{
\begin{center}
\begin{tabular}{lc}
\MStartJustify
\begin{MMiniPageP}{0.65}
#1
\end{MMiniPageP}
\MStopJustify
&
\begin{MMiniPageP}{0.3}
#2  
\end{MMiniPageP}
\end{tabular}
\end{center}
}

\newcommand{\MImageHalf}[2]{
\begin{center}
\begin{tabular}{lc}
\MStartJustify
\begin{MMiniPageP}{0.45}
#1
\end{MMiniPageP}
\MStopJustify
&
\begin{MMiniPageP}{0.45}
#2  
\end{MMiniPageP}
\end{tabular}
\end{center}
}

\newcommand{\MBigImageLeft}[2]{
\begin{center}
\begin{tabular}{lc}
\MStartJustify
\begin{MMiniPageP}{0.25}
#1
\end{MMiniPageP}
\MStopJustify
&
\begin{MMiniPageP}{0.7}
#2  
\end{MMiniPageP}
\end{tabular}
\end{center}
}

\ifttm
\def\No{\mathbb{N}_0}
\else
\def\No{\ensuremath{\N_0}}
\fi
\def\MT{\textrm{\tiny T}}
\newcommand{\MTranspose}[1]{{#1}^{\MT}}
\ifttm
\newcommand{\MRe}{\mathsf{Re}}
\newcommand{\MIm}{\mathsf{Im}}
\else
\DeclareMathOperator{\MRe}{Re}
\DeclareMathOperator{\MIm}{Im}
\fi

\newcommand{\Mid}{\mathrm{id}}
\newcommand{\MFeinheit}{\mathrm{feinh}}

\ifttm
\newcommand{\Msubstack}[1]{\begin{array}{c}{#1}\end{array}}
\else
\newcommand{\Msubstack}[1]{\substack{#1}}
\fi

% Typen von Fragefeldern:
% 1 = Alphanumerisch, case-sensitive-Vergleich
% 2 = Ja/Nein-Checkbox, Loesung ist 0 oder 1   (OPTION = Image-id fuer Rueckmeldung)
% 3 = Reelle Zahlen Geparset
% 4 = Funktionen Geparset (mit Stuetzstellen zur ueberpruefung)

% Dieser Befehl erstellt ein interaktives Aufgabenfeld. Parameter:
% - #1 Laenge in Zeichen
% - #2 Loesungstext (alphanumerisch, case sensitive)
% - #3 AufgabenID (alphanumerisch, case sensitive)
% - #4 Typ (Kennnummer)
% - #5 String fuer Optionen (ggf. mit Semikolon getrennte Einzelstrings)
% - #6 Anzahl Punkte
% - #7 uxid (kann z.B. Loesungsstring sein)
% ACHTUNG: Die langen Zeilen bitte so lassen, Zeilenumbrueche im tex werden in div's umgesetzt
\newcommand{\MQuestionID}[7]{
\ifttm
\special{html:<!-- mdeclareuxid;;}UX#7\special{html:;;}\arabic{section}\special{html:;;}#3\special{html:;; //-->}%
\special{html:<!-- mdeclarepoints;;}\arabic{section}\special{html:;;}#3\special{html:;;}#6\special{html:;;}\arabic{MTestSite}\special{html:;;}\arabic{chapter}%
\special{html:;; //--><!-- onloadstart //-->CreateQuestionObj("}#7\special{html:",}\arabic{MFieldCounter}\special{html:,"}#2%
\special{html:","}#3\special{html:",}#4\special{html:,"}#5\special{html:",}#6\special{html:,}\arabic{MTestSite}\special{html:,}\arabic{section}%
\special{html:);<!-- onloadstop //-->}%
\special{html:<input mfieldtype="}#4\special{html:" name="Name_}#3\special{html:" id="}#3\special{html:" type="text" size="}#1\special{html:" maxlength="}#1%
\special{html:" }\ifnum\value{MGroupActive}=0\special{html:onfocus="handlerFocus(}\arabic{MFieldCounter}%
\special{html:);" onblur="handlerBlur(}\arabic{MFieldCounter}\special{html:);" onkeyup="handlerChange(}\arabic{MFieldCounter}\special{html:,0);" onpaste="handlerChange(}\arabic{MFieldCounter}\special{html:,0);" oninput="handlerChange(}\arabic{MFieldCounter}\special{html:,0);" onpropertychange="handlerChange(}\arabic{MFieldCounter}\special{html:,0);"/>}%
\special{html:<img src="images/questionmark.gif" width="20" height="20" border="0" align="absmiddle" id="}QM#3\special{html:"/>}
\else%
\special{html:onblur="handlerBlur(}\arabic{MFieldCounter}%
\special{html:);" onfocus="handlerFocus(}\arabic{MFieldCounter}\special{html:);" onkeyup="handlerChange(}\arabic{MFieldCounter}\special{html:,1);" onpaste="handlerChange(}\arabic{MFieldCounter}\special{html:,1);" oninput="handlerChange(}\arabic{MFieldCounter}\special{html:,1);" onpropertychange="handlerChange(}\arabic{MFieldCounter}\special{html:,1);"/>}%
\special{html:<img src="images/questionmark.gif" width="20" height="20" border="0" align="absmiddle" id="}QM#3\special{html:"/>}\fi%
\else%
\ifnum\value{QBoxFlag}=1\fbox{$\phantom{\MForLoop{#1}{b}}$}\else$\phantom{\MForLoop{#1}{b}}$\fi%
\fi%
}

% ACHTUNG: Die langen Zeilen bitte so lassen, Zeilenumbrueche im tex werden in div's umgesetzt
% QuestionCheckbox macht ausserhalb einer QuestionGroup keinen Sinn!
% #1 = solution (1 oder 0), ggf. mit ::smc abgetrennt auszuschliessende single-choice-boxen (UXIDs durch , getrennt), #2 = id, #3 = points, #4 = uxid
\newcommand{\MQuestionCheckbox}[4]{
\ifttm
\special{html:<!-- mdeclareuxid;;}UX#4\special{html:;;}\arabic{section}\special{html:;;}#2\special{html:;; //-->}%
\ifnum\value{MGroupActive}=0\MDebugMessage{ERROR: Checkbox Nr. \arabic{MFieldCounter}\ ist nicht in einer Kontrollgruppe, es wird niemals eine Loesung angezeigt!}\fi
\special{html: %
<!-- mdeclarepoints;;}\arabic{section}\special{html:;;}#2\special{html:;;}#3\special{html:;;}\arabic{MTestSite}\special{html:;;}\arabic{chapter}%
\special{html:;; //--><!-- onloadstart //-->CreateQuestionObj("}#4\special{html:",}\arabic{MFieldCounter}\special{html:,"}#1\special{html:","}#2\special{html:",2,"IMG}#2%
\special{html:",}#3\special{html:,}\arabic{MTestSite}\special{html:,}\arabic{section}\special{html:);<!-- onloadstop //-->}%
\special{html:<input mfieldtype="2" type="checkbox" name="Name_}#2\special{html:" id="}#2\special{html:" onchange="handlerChange(}\arabic{MFieldCounter}\special{html:,1);"/><img src="images/questionmark.gif" name="}Name_IMG#2%
\special{html:" width="20" height="20" border="0" align="absmiddle" id="}IMG#2\special{html:"/> }%
\else%
\ifnum\value{QBoxFlag}=1\fbox{$\phantom{X}$}\else$\phantom{X}$\fi%
\fi%
}

\def\MGenerateID{QFELD_\arabic{section}.\arabic{subsection}.\arabic{MSiteCounter}.QF\arabic{MFieldCounter}}

% #1 = 0/1 ggf. mit ::smc abgetrennt auszuschliessende single-choice-boxen (UXIDs durch , getrennt ohne UX), #2 = uxid ohne UX
\newcommand{\MCheckbox}[2]{
\MQuestionCheckbox{#1}{\MGenerateID}{\MStdPoints}{#2}
\addtocounter{MFieldCounter}{1}
}

% Erster Parameter: Zeichenlaenge der Eingabebox, zweiter Parameter: Loesungstext
\newcommand{\MQuestion}[2]{
\MQuestionID{#1}{#2}{\MGenerateID}{1}{0}{\MStdPoints}{#2}
\addtocounter{MFieldCounter}{1}
}

% Erster Parameter: Zeichenlaenge der Eingabebox, zweiter Parameter: Loesungstext
\newcommand{\MLQuestion}[3]{
\MQuestionID{#1}{#2}{\MGenerateID}{1}{0}{\MStdPoints}{#3}
\addtocounter{MFieldCounter}{1}
}

% Parameter: Laenge des Feldes, Loesung (wird auch geparsed), Stellen Genauigkeit hinter dem Komma, weitere Stellen werden mathematisch gerundet vor Vergleich
\newcommand{\MParsedQuestion}[3]{
\MQuestionID{#1}{#2}{\MGenerateID}{3}{#3}{\MStdPoints}{#2}
\addtocounter{MFieldCounter}{1}
}

% Parameter: Laenge des Feldes, Loesung (wird auch geparsed), Stellen Genauigkeit hinter dem Komma, weitere Stellen werden mathematisch gerundet vor Vergleich
\newcommand{\MLParsedQuestion}[4]{
\MQuestionID{#1}{#2}{\MGenerateID}{3}{#3}{\MStdPoints}{#4}
\addtocounter{MFieldCounter}{1}
}

% Parameter: Laenge des Feldes, Loesungsfunktion, Anzahl Stuetzstellen, Funktionsvariablen durch Kommata getrennt (nicht case-sensitive), Anzahl Nachkommastellen im Vergleich
\newcommand{\MFunctionQuestion}[5]{
\MQuestionID{#1}{#2}{\MGenerateID}{4}{#3;#4;#5;0}{\MStdPoints}{#2}
\addtocounter{MFieldCounter}{1}
}

% Parameter: Laenge des Feldes, Loesungsfunktion, Anzahl Stuetzstellen, Funktionsvariablen durch Kommata getrennt (nicht case-sensitive), Anzahl Nachkommastellen im Vergleich, UXID
\newcommand{\MLFunctionQuestion}[6]{
\MQuestionID{#1}{#2}{\MGenerateID}{4}{#3;#4;#5;0}{\MStdPoints}{#6}
\addtocounter{MFieldCounter}{1}
}

% Parameter: Laenge des Feldes, Loesungsintervall, Genauigkeit der Zahlenwertpruefung
\newcommand{\MIntervalQuestion}[3]{
\MQuestionID{#1}{#2}{\MGenerateID}{6}{#3}{\MStdPoints}{#2}
\addtocounter{MFieldCounter}{1}
}

% Parameter: Laenge des Feldes, Loesungsintervall, Genauigkeit der Zahlenwertpruefung, UXID
\newcommand{\MLIntervalQuestion}[4]{
\MQuestionID{#1}{#2}{\MGenerateID}{6}{#3}{\MStdPoints}{#4}
\addtocounter{MFieldCounter}{1}
}

% Parameter: Laenge des Feldes, Loesungsfunktion, Anzahl Stuetzstellen, Funktionsvariable (nicht case-sensitive), Anzahl Nachkommastellen im Vergleich, Vereinfachungsbedingung
% Vereinfachungsbedingung ist eine der Folgenden:
% 0 = Keine Vereinfachungsbedingung
% 1 = Keine Klammern (runde oder eckige) mehr im vereinfachten Ausdruck
% 2 = Faktordarstellung (Term hat Produkte als letzte Operation, Summen als vorgeschaltete Operation)
% 3 = Summendarstellung (Term hat Summen als letzte Operation, Produkte als vorgeschaltete Operation)
% Flag 512: Besondere Stuetzstellen (nur >1 und nur schwach rational), sonst symmetrisch um Nullpunkt und ganze Zahlen inkl. Null werden getroffen
\newcommand{\MSimplifyQuestion}[6]{
\MQuestionID{#1}{#2}{\MGenerateID}{4}{#3;#4;#5;#6}{\MStdPoints}{#2}
\addtocounter{MFieldCounter}{1}
}

\newcommand{\MLSimplifyQuestion}[7]{
\MQuestionID{#1}{#2}{\MGenerateID}{4}{#3;#4;#5;#6}{\MStdPoints}{#7}
\addtocounter{MFieldCounter}{1}
}

% Parameter: Laenge des Feldes, Loesung (optionaler Ausdruck), Anzahl Stuetzstellen, Funktionsvariable (nicht case-sensitive), Anzahl Nachkommastellen im Vergleich, Spezialtyp (string-id)
\newcommand{\MLSpecialQuestion}[7]{
\MQuestionID{#1}{#2}{\MGenerateID}{7}{#3;#4;#5;#6}{\MStdPoints}{#7}
\addtocounter{MFieldCounter}{1}
}

\newcounter{MGroupStart}
\newcounter{MGroupEnd}
\newcounter{MGroupActive}

\newenvironment{MQuestionGroup}{
\setcounter{MGroupStart}{\value{MFieldCounter}}
\setcounter{MGroupActive}{1}
}{
\setcounter{MGroupActive}{0}
\setcounter{MGroupEnd}{\value{MFieldCounter}}
\addtocounter{MGroupEnd}{-1}
}

\newcommand{\MGroupButton}[1]{
\ifttm
\special{html:<button name="Name_Group}\arabic{MGroupStart}\special{html:to}\arabic{MGroupEnd}\special{html:" id="Group}\arabic{MGroupStart}\special{html:to}\arabic{MGroupEnd}\special{html:" %
type="button" onclick="group_button(}\arabic{MGroupStart}\special{html:,}\arabic{MGroupEnd}\special{html:);">}#1\special{html:</button>}
\else
\phantom{#1}
\fi
}

%----------------- Makros fuer die modularisierte Darstellung ------------------------------------

\def\MyText#1{#1}

% is used internally by the conversion package, should not be used by original tex documents
\def\MOrgLabel#1{\relax}

\ifttm

% Ein MLabel wird im html codiert durch das tag <!-- mmlabel;;Labelbezeichner;;SubjectArea;;chapter;;section;;subsection;;Index;;Objekttyp; //-->
\def\MLabel#1{%
\ifnum\value{MLastType}=8%
\ifnum\value{MCaptionOn}=0%
\MDebugMessage{ERROR: Grafik \arabic{MGraphicsCounter} hat separates label: #1 (Grafiklabels sollten nur in der Caption stehen)}%
\fi
\fi
\ifnum\value{MLastType}=12%
\ifnum\value{MCaptionOn}=0%
\MDebugMessage{ERROR: Video \arabic{MVideoCounter} hat separates label: #1 (Videolabels sollten nur in der Caption stehen}%
\fi
\fi
\ifnum\value{MLastType}=10\setcounter{MLastIndex}{\value{equation}}\fi
\label{#1}\begin{html}<!-- mmlabel;;#1;;\end{html}\arabic{MSubjectArea}\special{html:;;}\arabic{chapter}\special{html:;;}\arabic{section}\special{html:;;}\arabic{subsection}\special{html:;;}\arabic{MLastIndex}\special{html:;;}\arabic{MLastType}\special{html:; //-->}}%

\else

% Sonderbehandlung im PDF fuer Abbildungen in separater aux-Datei, da MGraphics die figure-Umgebung nicht verwendet
\def\MLabel#1{%
\ifnum\value{MLastType}=8%
\ifnum\value{MCaptionOn}=0%
\MDebugMessage{ERROR: Grafik \arabic{MGraphicsCounter} hat separates label: #1 (Grafiklabels sollten nur in der Caption stehen}%
\fi
\fi
\ifnum\value{MLastType}=12%
\ifnum\value{MCaptionOn}=0%
\MDebugMessage{ERROR: Video \arabic{MVideoCounter} hat separates label: #1 (Videolabels sollten nur in der Caption stehen}%
\fi
\fi
\label{#1}%
}%

\fi

% Gibt Begriff des referenzierten Objekts mit aus, aber nur im HTML, daher nur in Ausnahmefaellen (z.B. Copyrightliste) sinnvoll
\def\MCRef#1{\ifttm\special{html:<!-- mmref;;}#1\special{html:;;1; //-->}\else\vref{#1}\fi}


\def\MRef#1{\ifttm\special{html:<!-- mmref;;}#1\special{html:;;0; //-->}\else\vref{#1}\fi}
\def\MERef#1{\ifttm\special{html:<!-- mmref;;}#1\special{html:;;0; //-->}\else\eqref{#1}\fi}
\def\MNRef#1{\ifttm\special{html:<!-- mmref;;}#1\special{html:;;0; //-->}\else\ref{#1}\fi}
\def\MSRef#1#2{\ifttm\special{html:<!-- msref;;}#1\special{html:;;}#2\special{html:; //-->}\else \if#2\empty \ref{#1} \else \hyperref[#1]{#2}\fi\fi} 

\def\MRefRange#1#2{\ifttm\MRef{#1} bis 
\MRef{#2}\else\vrefrange[\unskip]{#1}{#2}\fi}

\def\MRefTwo#1#2{\ifttm\MRef{#1} und \MRef{#2}\else%
\let\vRefTLRsav=\reftextlabelrange\let\vRefTPRsav=\reftextpagerange%
\def\reftextlabelrange##1##2{\ref{##1} und~\ref{##2}}%
\def\reftextpagerange##1##2{auf den Seiten~\pageref{#1} und~\pageref{#2}}%
\vrefrange[\unskip]{#1}{#2}%
\let\reftextlabelrange=\vRefTLRsav\let\reftextpagerange=\vRefTPRsav\fi}

% MSectionChapter definiert falls notwendig das Kapitel vor der section. Das ist notwendig, wenn nur ein Einzelmodul uebersetzt wird.
% MChaptersGiven ist ein Counter, der von mconvert.pl vordefiniert wird.
\ifttm
\newcommand{\MSectionChapter}{\ifnum\value{MChaptersGiven}=0{\Dchapter{Modul}}\else{}\fi}
\else
\newcommand{\MSectionChapter}{\ifnum\value{chapter}=0{\Dchapter{Modul}}\else{}\fi}
\fi


\def\MChapter#1{\ifnum\value{MSSEnd}>0{\MSubsectionEndMacros}\addtocounter{MSSEnd}{-1}\fi\Dchapter{#1}}
\def\MSubject#1{\MChapter{#1}} % Schluesselwort HELPSECTION ist reserviert fuer Hilfesektion

\newcommand{\MSectionID}{UNKNOWNID}

\ifttm
\newcommand{\MSetSectionID}[1]{\renewcommand{\MSectionID}{#1}}
\else
\newcommand{\MSetSectionID}[1]{\renewcommand{\MSectionID}{#1}\tikzsetexternalprefix{#1}}
\fi


\newcommand{\MSection}[1]{\MSetSectionID{MODULID}\ifnum\value{MSSEnd}>0{\MSubsectionEndMacros}\addtocounter{MSSEnd}{-1}\fi\MSectionChapter\Dsection{#1}\MSectionStartMacros{#1}\setcounter{MLastIndex}{-1}\setcounter{MLastType}{1}} % Sections werden ueber das section-Feld im mmlabel-Tag identifiziert, nicht ueber das Indexfeld

\def\MSubsection#1{\ifnum\value{MSSEnd}>0{\MSubsectionEndMacros}\addtocounter{MSSEnd}{-1}\fi\ifttm\else\clearpage\fi\Dsubsection{#1}\MSubsectionStartMacros\setcounter{MLastIndex}{-1}\setcounter{MLastType}{2}\addtocounter{MSSEnd}{1}}% Subsections werden ueber das subsection-Feld im mmlabel-Tag identifiziert, nicht ueber das Indexfeld
\def\MSubsectionx#1{\Dsubsectionx{#1}} % Nur zur Verwendung in MSectionStart gedacht
\def\MSubsubsection#1{\Dsubsubsection{#1}\setcounter{MLastIndex}{\value{subsubsection}}\setcounter{MLastType}{3}\ifttm\special{html:<!-- sectioninfo;;}\arabic{section}\special{html:;;}\arabic{subsection}\special{html:;;}\arabic{subsubsection}\special{html:;;1;;}\arabic{MTestSite}\special{html:; //-->}\fi}
\def\MSubsubsectionx#1{\Dsubsubsectionx{#1}\ifttm\special{html:<!-- sectioninfo;;}\arabic{section}\special{html:;;}\arabic{subsection}\special{html:;;}\arabic{subsubsection}\special{html:;;0;;}\arabic{MTestSite}\special{html:; //-->}\else\addcontentsline{toc}{subsection}{#1}\fi}

\ifttm
\def\MSubsubsubsectionx#1{\ \newline\textbf{#1}\special{html:<br />}}
\else
\def\MSubsubsubsectionx#1{\ \newline
\textbf{#1}\ \\
}
\fi


% Dieses Skript wird zu Beginn jedes Modulabschnitts (=Webseite) ausgefuehrt und initialisiert den Aufgabenfeldzaehler
\newcommand{\MPageScripts}{
\setcounter{MFieldCounter}{1}
\addtocounter{MSiteCounter}{1}
\setcounter{MHintCounter}{1}
\setcounter{MCodeEditCounter}{1}
\setcounter{MGroupActive}{0}
\DoQBoxes
% Feldvariablen werden im HTML-Header in conv.pl eingestellt
}

% Dieses Skript wird zum Ende jedes Modulabschnitts (=Webseite) ausgefuehrt
\ifttm
\newcommand{\MEndScripts}{\special{html:<br /><!-- mfeedbackbutton;Seite;}\arabic{MTestSite}\special{html:;}\MGenerateSiteNumber\special{html:; //-->}
}
\else
\newcommand{\MEndScripts}{\relax}
\fi


\newcounter{QBoxFlag}
\newcommand{\DoQBoxes}{\setcounter{QBoxFlag}{1}}
\newcommand{\NoQBoxes}{\setcounter{QBoxFlag}{0}}

\newcounter{MXCTest}
\newcounter{MXCounter}
\newcounter{MSCounter}



\ifttm

% Struktur des sectioninfo-Tags: <!-- sectioninfo;;section;;subsection;;subsubsection;;nr_ausgeben;;testpage; //-->

%Fuegt eine zusaetzliche html-Seite an hinter ALLEN bisherigen und zukuenftigen content-Seiten ausserhalb der vor-zurueck-Schleife (d.h. nur durch Button oder MIntLink erreichbar!)
% #1 = Titel des Modulabschnitts, #2 = Kurztitel fuer die Buttons, #3 = Buttonkennung (STD = default nehmen, NONE = Ohne Button in der Navigation)
\newenvironment{MSContent}[3]{\special{html:<div class="xcontent}\arabic{MSCounter}\special{html:"><!-- scontent;-;}\arabic{MSCounter};-;#1;-;#2;-;#3\special{html: //-->}\MPageScripts\MSubsubsectionx{#1}}{\MEndScripts\special{html:<!-- endscontent;;}\arabic{MSCounter}\special{html: //--></div>}\addtocounter{MSCounter}{1}}

% Fuegt eine zusaetzliche html-Seite ein hinter den bereits vorhandenen content-Seiten (oder als erste Seite) innerhalb der vor-zurueck-Schleife der Navigation
% #1 = Titel des Modulabschnitts, #2 = Kurztitel fuer die Buttons, #3 = Buttonkennung (STD = Defaultbutton, NONE = Ohne Button in der Navigation)
\newenvironment{MXContent}[3]{\special{html:<div class="xcontent}\arabic{MXCounter}\special{html:"><!-- xcontent;-;}\arabic{MXCounter};-;#1;-;#2;-;#3\special{html: //-->}\MPageScripts\MSubsubsection{#1}}{\MEndScripts\special{html:<!-- endxcontent;;}\arabic{MXCounter}\special{html: //--></div>}\addtocounter{MXCounter}{1}}

% Fuegt eine zusaetzliche html-Seite ein die keine subsubsection-Nummer bekommt, nur zur internen Verwendung in mintmod.tex gedacht!
% #1 = Titel des Modulabschnitts, #2 = Kurztitel fuer die Buttons, #3 = Buttonkennung (STD = Defaultbutton, NONE = Ohne Button in der Navigation)
% \newenvironment{MUContent}[3]{\special{html:<div class="xcontent}\arabic{MXCounter}\special{html:"><!-- xcontent;-;}\arabic{MXCounter};-;#1;-;#2;-;#3\special{html: //-->}\MPageScripts\MSubsubsectionx{#1}}{\MEndScripts\special{html:<!-- endxcontent;;}\arabic{MXCounter}\special{html: //--></div>}\addtocounter{MXCounter}{1}}

\newcommand{\MDeclareSiteUXID}[1]{\special{html:<!-- mdeclaresiteuxid;;}#1\special{html:;;}\arabic{chapter}\special{html:;;}\arabic{section}\special{html:;; //-->}}

\else

%\newcommand{\MSubsubsection}[1]{\refstepcounter{subsubsection} \addcontentsline{toc}{subsubsection}{\thesubsubsection. #1}}


% Fuegt eine zusaetzliche html-Seite an hinter den bereits vorhandenen content-Seiten
% #1 = Titel des Modulabschnitts, #2 = Kurztitel fuer die Buttons, #3 = Iconkennung (im PDF wirkungslos)
%\newenvironment{MUContent}[3]{\ifnum\value{MXCTest}>0{\MDebugMessage{ERROR: Geschachtelter SContent}}\fi\MPageScripts\MSubsubsectionx{#1}\addtocounter{MXCTest}{1}}{\addtocounter{MXCounter}{1}\addtocounter{MXCTest}{-1}}
\newenvironment{MXContent}[3]{\ifnum\value{MXCTest}>0{\MDebugMessage{ERROR: Geschachtelter SContent}}\fi\MPageScripts\MSubsubsection{#1}\addtocounter{MXCTest}{1}}{\addtocounter{MXCounter}{1}\addtocounter{MXCTest}{-1}}
\newenvironment{MSContent}[3]{\ifnum\value{MXCTest}>0{\MDebugMessage{ERROR: Geschachtelter XContent}}\fi\MPageScripts\MSubsubsectionx{#1}\addtocounter{MXCTest}{1}}{\addtocounter{MSCounter}{1}\addtocounter{MXCTest}{-1}}

\newcommand{\MDeclareSiteUXID}[1]{\relax}

\fi 

% GHEADER und GFOOTER werden von split.pm gefunden, aber nur, wenn nicht HELPSITE oder TESTSITE
\ifttm
\newenvironment{MSectionStart}{\special{html:<div class="xcontent0">}\MSubsubsectionx{Modul\"ubersicht}}{\setcounter{MSSEnd}{0}\special{html:</div>}}
% Darf nicht als XContent nummeriert werden, darf nicht als XContent gelabelt werden, wird aber in eine xcontent-div gesetzt fuer Python-parsing
\else
\newenvironment{MSectionStart}{\MSubsectionx{Modul\"ubersicht}}{\setcounter{MSSEnd}{0}}
\fi

\newenvironment{MIntro}{\begin{MXContent}{Einf\"uhrung}{Einf\"uhrung}{genetisch}}{\end{MXContent}}
\newenvironment{MContent}{\begin{MXContent}{Inhalt}{Inhalt}{beweis}}{\end{MXContent}}
\newenvironment{MExercises}{\ifttm\else\clearpage\fi\begin{MXContent}{Aufgaben}{Aufgaben}{aufgb}\special{html:<!-- declareexcsymb //-->}}{\end{MXContent}}

% #1 = Lesbare Testbezeichnung
\newenvironment{MTest}[1]{%
\renewcommand{\MTestName}{#1}
\ifttm\else\clearpage\fi%
\addtocounter{MTestSite}{1}%
\begin{MXContent}{#1}{#1}{STD} % {aufgb}%
\special{html:<!-- declaretestsymb //-->}
\begin{MQuestionGroup}%
\MInTestHeader
}%
{%
\end{MQuestionGroup}%
\ \\ \ \\%
\MInTestFooter
\end{MXContent}\addtocounter{MTestSite}{-1}%
}

\newenvironment{MExtra}{\ifttm\else\clearpage\fi\begin{MXContent}{Zus\"atzliche Inhalte}{Zusatz}{weiterfhrg}}{\end{MXContent}}

\makeindex

\ifttm
\def\MPrintIndex{
\ifnum\value{MSSEnd}>0{\MSubsectionEndMacros}\addtocounter{MSSEnd}{-1}\fi
\renewcommand{\indexname}{Stichwortverzeichnis}
\special{html:<p><!-- printindex //--></p>}
}
\else
\def\MPrintIndex{
\ifnum\value{MSSEnd}>0{\MSubsectionEndMacros}\addtocounter{MSSEnd}{-1}\fi
\renewcommand{\indexname}{Stichwortverzeichnis}
\addcontentsline{toc}{section}{Stichwortverzeichnis}
\printindex
}
\fi


% Konstanten fuer die Modulfaecher

\def\MINTMathematics{1}
\def\MINTInformatics{2}
\def\MINTChemistry{3}
\def\MINTPhysics{4}
\def\MINTEngineering{5}

\newcounter{MSubjectArea}
\newcounter{MInfoNumbers} % Gibt an, ob die Infoboxen nummeriert werden sollen
\newcounter{MSepNumbers} % Gibt an, ob Beispiele und Experimente separat nummeriert werden sollen
\newcommand{\MSetSubject}[1]{
 % ttm kapiert setcounter mit Parametern nicht, also per if abragen und einsetzen
\ifnum#1=1\setcounter{MSubjectArea}{1}\setcounter{MInfoNumbers}{1}\setcounter{MSepNumbers}{0}\fi
\ifnum#1=2\setcounter{MSubjectArea}{2}\setcounter{MInfoNumbers}{1}\setcounter{MSepNumbers}{0}\fi
\ifnum#1=3\setcounter{MSubjectArea}{3}\setcounter{MInfoNumbers}{0}\setcounter{MSepNumbers}{1}\fi
\ifnum#1=4\setcounter{MSubjectArea}{4}\setcounter{MInfoNumbers}{0}\setcounter{MSepNumbers}{0}\fi
\ifnum#1=5\setcounter{MSubjectArea}{5}\setcounter{MInfoNumbers}{1}\setcounter{MSepNumbers}{0}\fi
% Separate Nummerntechnik fuer unsere Chemiker: alles dreistellig
\ifnum#1=3
  \ifttm
  \renewcommand{\theequation}{\arabic{section}.\arabic{subsection}.\arabic{equation}}
  \renewcommand{\thetable}{\arabic{section}.\arabic{subsection}.\arabic{table}} 
  \renewcommand{\thefigure}{\arabic{section}.\arabic{subsection}.\arabic{figure}} 
  \else
  \renewcommand{\theequation}{\arabic{chapter}.\arabic{section}.\arabic{equation}}
  \renewcommand{\thetable}{\arabic{chapter}.\arabic{section}.\arabic{table}}
  \renewcommand{\thefigure}{\arabic{chapter}.\arabic{section}.\arabic{figure}}
  \fi
\else
  \ifttm
  \renewcommand{\theequation}{\arabic{section}.\arabic{subsection}.\arabic{equation}}
  \renewcommand{\thetable}{\arabic{table}}
  \renewcommand{\thefigure}{\arabic{figure}}
  \else
  \renewcommand{\theequation}{\arabic{chapter}.\arabic{section}.\arabic{equation}}
  \renewcommand{\thetable}{\arabic{table}}
  \renewcommand{\thefigure}{\arabic{figure}}
  \fi
\fi
}

% Fuer tikz Autogenerierung
\newcounter{MTIKZAutofilenumber}

% Spezielle Counter fuer die Bentz-Module
\newcounter{mycounter}
\newcounter{chemapplet}
\newcounter{physapplet}

\newcounter{MSSEnd} % Ist 1 falls ein MSubsection aktiv ist, der einen MSubsectionEndMacro-Aufruf verursacht
\newcounter{MFileNumber}
\def\MLastFile{\special{html:[[!-- mfileref;;}\arabic{MFileNumber}\special{html:; //--]]}}

% Vollstaendiger Pfad ist \MMaterial / \MLastFilePath / \MLastFileName    ==   \MMaterial / \MLastFile

% Wird nur bei kompletter Baum-Erstellung ausgefuehrt!
% #1 = Lesbare Modulbezeichnung
\newcommand{\MSectionStartMacros}[1]{
\setcounter{MTestSite}{0}
\setcounter{MCaptionOn}{0}
\setcounter{MLastTypeEq}{0}
\setcounter{MSSEnd}{0}
\setcounter{MFileNumber}{0} % Preinkrekement-Counter
\setcounter{MTIKZAutofilenumber}{0}
\setcounter{mycounter}{1}
\setcounter{physapplet}{1}
\setcounter{chemapplet}{0}
\ifttm
\special{html:<!-- mdeclaresection;;}\arabic{chapter}\special{html:;;}\arabic{section}\special{html:;;}#1\special{html:;; //-->}%
\else
\setcounter{thmc}{0}
\setcounter{exmpc}{0}
\setcounter{verc}{0}
\setcounter{infoc}{0}
\fi
\setcounter{MiniMarkerCounter}{1}
\setcounter{AlignCounter}{1}
\setcounter{MXCTest}{0}
\setcounter{MCodeCounter}{0}
\setcounter{MEntryCounter}{0}
}

% Wird immer ausgefuehrt
\newcommand{\MSubsectionStartMacros}{
\ifttm\else\MPageHeaderDef\fi
\MWatermarkSettings
\setcounter{MXCounter}{0}
\setcounter{MSCounter}{0}
\setcounter{MSiteCounter}{1}
\setcounter{MExerciseCollectionCounter}{0}
% Zaehler fuer das Labelsystem zuruecksetzen (prefix-Zaehler)
\setcounter{MInfoCounter}{0}
\setcounter{MExerciseCounter}{0}
\setcounter{MExampleCounter}{0}
\setcounter{MExperimentCounter}{0}
\setcounter{MGraphicsCounter}{0}
\setcounter{MTableCounter}{0}
\setcounter{MTheoremCounter}{0}
\setcounter{MObjectCounter}{0}
\setcounter{MEquationCounter}{0}
\setcounter{MVideoCounter}{0}
\setcounter{equation}{0}
\setcounter{figure}{0}
}

\newcommand{\MSubsectionEndMacros}{
% Bei Chemiemodulen das PSE einhaengen, es soll als SContent am Ende erscheinen
\special{html:<!-- subsectionend //-->}
\ifnum\value{MSubjectArea}=3{\MIncludePSE}\fi
}


\ifttm
%\newcommand{\MEmbed}[1]{\MRegisterFile{#1}\begin{html}<embed src="\end{html}\MMaterial/\MLastFile\begin{html}" width="192" height="189"></embed>\end{html}}
\newcommand{\MEmbed}[1]{\MRegisterFile{#1}\begin{html}<embed src="\end{html}\MMaterial/\MLastFile\begin{html}"></embed>\end{html}}
\fi

%----------------- Makros fuer die Textdarstellung -----------------------------------------------

\ifttm
% MUGraphics bindet eine Grafik ein:
% Parameter 1: Dateiname der Grafik, relativ zur Position des Modul-Tex-Dokuments
% Parameter 2: Skalierungsoptionen fuer PDF (fuer includegraphics)
% Parameter 3: Titel fuer die Grafik, wird unter die Grafik mit der Grafiknummer gesetzt und kann MLabel bzw. MCopyrightLabel enthalten
% Parameter 4: Skalierungsoptionen fuer HTML (css-styles)

% ERSATZ: <img alt="My Image" src="data:image/png;base64,iVBORwA<MoreBase64SringHere>" />


\newcommand{\MUGraphics}[4]{\MRegisterFile{#1}\begin{html}
<div class="imagecenter">
<center>
<div>
<img src="\end{html}\MMaterial/\MLastFile\begin{html}" style="#4" alt="\end{html}\MMaterial/\MLastFile\begin{html}"/>
</div>
<div class="bildtext">
\end{html}
\addtocounter{MGraphicsCounter}{1}
\setcounter{MLastIndex}{\value{MGraphicsCounter}}
\setcounter{MLastType}{8}
\addtocounter{MCaptionOn}{1}
\ifnum\value{MSepNumbers}=0
\textbf{Abbildung \arabic{MGraphicsCounter}:} #3
\else
\textbf{Abbildung \arabic{section}.\arabic{subsection}.\arabic{MGraphicsCounter}:} #3
\fi
\addtocounter{MCaptionOn}{-1}
\begin{html}
</div>
</center>
</div>
<br />
\end{html}%
\special{html:<!-- mfeedbackbutton;Abbildung;}\arabic{MGraphicsCounter}\special{html:;}\arabic{section}.\arabic{subsection}.\arabic{MGraphicsCounter}\special{html:; //-->}%
}

% MVideo bindet ein Video als Einzeldatei ein:
% Parameter 1: Dateiname des Videos, relativ zur Position des Modul-Tex-Dokuments, ohne die Endung ".mp4"
% Parameter 2: Titel fuer das Video (kann MLabel oder MCopyrightLabel enthalten), wird unter das Video mit der Videonummer gesetzt
\newcommand{\MVideo}[2]{\MRegisterFile{#1.mp4}\begin{html}
<div class="imagecenter">
<center>
<div>
<video width="95\%" controls="controls"><source src="\end{html}\MMaterial/#1.mp4\begin{html}" type="video/mp4">Ihr Browser kann keine MP4-Videos abspielen!</video>
</div>
<div class="bildtext">
\end{html}
\addtocounter{MVideoCounter}{1}
\setcounter{MLastIndex}{\value{MVideoCounter}}
\setcounter{MLastType}{12}
\addtocounter{MCaptionOn}{1}
\ifnum\value{MSepNumbers}=0
\textbf{Video \arabic{MVideoCounter}:} #2
\else
\textbf{Video \arabic{section}.\arabic{subsection}.\arabic{MVideoCounter}:} #2
\fi
\addtocounter{MCaptionOn}{-1}
\begin{html}
</div>
</center>
</div>
<br />
\end{html}}

\newcommand{\MDVideo}[2]{\MRegisterFile{#1.mp4}\MRegisterFile{#1.ogv}\begin{html}
<div class="imagecenter">
<center>
<div>
<video width="70\%" controls><source src="\end{html}\MMaterial/#1.mp4\begin{html}" type="video/mp4"><source src="\end{html}\MMaterial/#1.ogv\begin{html}" type="video/ogg">Ihr Browser kann keine MP4-Videos abspielen!</video>
</div>
<br />
#2
</center>
</div>
<br />
\end{html}
}

\newcommand{\MGraphics}[3]{\MUGraphics{#1}{#2}{#3}{}}

\else

\newcommand{\MVideo}[2]{%
% Kein Video im PDF darstellbar, trotzdem so tun als ob da eines waere
\begin{center}
(Video nicht darstellbar)
\end{center}
\addtocounter{MVideoCounter}{1}
\setcounter{MLastIndex}{\value{MVideoCounter}}
\setcounter{MLastType}{12}
\addtocounter{MCaptionOn}{1}
\ifnum\value{MSepNumbers}=0
\textbf{Video \arabic{MVideoCounter}:} #2
\else
\textbf{Video \arabic{section}.\arabic{subsection}.\arabic{MVideoCounter}:} #2
\fi
\addtocounter{MCaptionOn}{-1}
}


% MGraphics bindet eine Grafik ein:
% Parameter 1: Dateiname der Grafik, relativ zur Position des Modul-Tex-Dokuments
% Parameter 2: Skalierungsoptionen fuer PDF (fuer includegraphics)
% Parameter 3: Titel fuer die Grafik, wird unter die Grafik mit der Grafiknummer gesetzt
\newcommand{\MGraphics}[3]{%
\MRegisterFile{#1}%
\ %
\begin{figure}[H]%
\centering{%
\includegraphics[#2]{\MDPrefix/#1}%
\addtocounter{MCaptionOn}{1}%
\caption{#3}%
\addtocounter{MCaptionOn}{-1}%
}%
\end{figure}%
\addtocounter{MGraphicsCounter}{1}\setcounter{MLastIndex}{\value{MGraphicsCounter}}\setcounter{MLastType}{8}\ %
%\ \\Abbildung \ifnum\value{MSepNumbers}=0\else\arabic{chapter}.\arabic{section}.\fi\arabic{MGraphicsCounter}: #3%
}

\newcommand{\MUGraphics}[4]{\MGraphics{#1}{#2}{#3}}


\fi

\newcounter{MCaptionOn} % = 1 falls eine Grafikcaption aktiv ist, = 0 sonst


% MGraphicsSolo bindet eine Grafik pur ein ohne Titel
% Parameter 1: Dateiname der Grafik, relativ zur Position des Modul-Tex-Dokuments
% Parameter 2: Skalierungsoptionen (wirken nur im PDF)
\newcommand{\MGraphicsSolo}[2]{\MUGraphicsSolo{#1}{#2}{}}

% MUGraphicsSolo bindet eine Grafik pur ein ohne Titel, aber mit HTML-Skalierung
% Parameter 1: Dateiname der Grafik, relativ zur Position des Modul-Tex-Dokuments
% Parameter 2: Skalierungsoptionen (wirken nur im PDF)
% Parameter 3: Skalierungsoptionen (wirken nur im HTML), als style-format: "width=???, height=???"
\ifttm
\newcommand{\MUGraphicsSolo}[3]{\MRegisterFile{#1}\begin{html}
<img src="\end{html}\MMaterial/\MLastFile\begin{html}" style="\end{html}#3\begin{html}" alt="\end{html}\MMaterial/\MLastFile\begin{html}"/>
\end{html}%
\special{html:<!-- mfeedbackbutton;Abbildung;}#1\special{html:;}\MMaterial/\MLastFile\special{html:; //-->}%
}
\else
\newcommand{\MUGraphicsSolo}[3]{\MRegisterFile{#1}\includegraphics[#2]{\MDPrefix/#1}}
\fi

% Externer Link mit URL
% Erster Parameter: Vollstaendige(!) URL des Links
% Zweiter Parameter: Text fuer den Link
\newcommand{\MExtLink}[2]{\ifttm\special{html:<a target="_new" href="}#1\special{html:">}#2\special{html:</a>}\else\href{#1}{#2}\fi} % ohne MINTERLINK!


% Interner Link, die verlinkte Datei muss im gleichen Verzeichnis liegen wie die Modul-Texdatei
% Erster Parameter: Dateiname
% Zweiter Parameter: Text fuer den Link
\newcommand{\MIntLink}[2]{\ifttm\MRegisterFile{#1}\special{html:<a class="MINTERLINK" target="_new" href="}\MMaterial/\MLastFile\special{html:">}#2\special{html:</a>}\else{\href{#1}{#2}}\fi}


\ifttm
\def\MMaterial{:localmaterial:}
\else
\def\MMaterial{\MDPrefix}
\fi

\ifttm
\def\MNoFile#1{:directmaterial:#1}
\else
\def\MNoFile#1{#1}
\fi

\newcommand{\MChem}[1]{$\mathrm{#1}$}

\newcommand{\MApplet}[3]{
% Bindet ein Java-Applet ein, die Parameter sind:
% (wird nur im HTML, aber nicht im PDF erstellt)
% #1 Dateiname des Applets (muss mit ".class" enden)
% #2 = Breite in Pixeln
% #3 = Hoehe in Pixeln
\ifttm
\MRegisterFile{#1}
\begin{html}
<applet code="\end{html}\MMaterial/\MLastFile\begin{html}" width="#2" height="#3" alt="[Java-Applet kann nicht gestartet werden]"></applet>
\end{html}
\fi
}

\newcommand{\MScriptPage}[2]{
% Bindet eine JavaScript-Datei ein, die eine eigene Seite bekommt
% (wird nur im HTML, aber nicht im PDF erstellt)
% #1 Dateiname des Programms (sollte mit ".js" enden)
% #2 = Kurztitel der Seite
\ifttm
\begin{MSContent}{#2}{#2}{puzzle}
\MRegisterFile{#1}
\begin{html}
<script src="\MMaterial/\MLastFile" type="text/javascript"></script>
\end{html}
\end{MSContent}
\fi
}

\newcommand{\MIncludePSE}{
% Bindet bei Chemie-Modulen das PSE ein
% (wird nur im HTML, aber nicht im PDF erstellt)
\ifttm
\special{html:<!-- includepse //-->}
\begin{MSContent}{Periodensystem der Elemente}{PSE}{table}
\MRegisterFile{../files/pse.js}
\MRegisterFile{../files/radio.png}
\begin{html}
<script src="\MMaterial/../files/pse.js" type="text/javascript"></script>
<p id="divid"><br /><br />
<script language="javascript" type="text/javascript">
    startpse("divid","\MMaterial/../files"); 
</script>
</p>
<br />
<br />
<br />
<p>Die Farben der Elementsymbole geben an: <font style="color:Red">gasf&ouml;rmig </font> <font style="color:Blue">fl&uuml;ssig </font> fest</p>
<p>Die Elemente der Gruppe 1 A, 2 A, 3 A usw. geh&ouml;ren zu den Hauptgruppenelementen.</p>
<p>Die Elemente der Gruppe 1 B, 2 B, 3 B usw. geh&ouml;ren zu den Nebengruppenelementen.</p>
<p>() kennzeichnet die Masse des stabilsten Isotops</p>
\end{html}
\end{MSContent}
\fi
}

\newcommand{\MAppletArchive}[4]{
% Bindet ein Java-Applet ein, die Parameter sind:
% (wird nur im HTML, aber nicht im PDF erstellt)
% #1 Dateiname der Klasse mit Appletaufruf (muss mit ".class" enden)
% #2 Dateiname des Archivs (muss mit ".jar" enden)
% #3 = Breite in Pixeln
% #4 = Hoehe in Pixeln
\ifttm
\MRegisterFile{#2}
\begin{html}
<applet code="#1" archive="\end{html}\MMaterial/\MLastFile\begin{html}" codebase="." width="#3" height="#4" alt="[Java-Archiv kann nicht gestartet werden]"></applet>
\end{html}
\fi
}

% Bindet in der Haupttexdatei ein MINT-Modul ein. Parameter 1 ist das Verzeichnis (relativ zur Haupttexdatei), Parameter 2 ist der Dateinahme ohne Pfad.
\newcommand{\IncludeModule}[2]{
\renewcommand{\MDPrefix}{#1}
\input{#1/#2}
\ifnum\value{MSSEnd}>0{\MSubsectionEndMacros}\addtocounter{MSSEnd}{-1}\fi
}

% Der ttm-Konverter setzt keine Makros im \input um, also muss hier getrickst werden:
% Das MDPrefix muss in den einzelnen Modulen manuell eingesetzt werden
\newcommand{\MInputFile}[1]{
\ifttm
\input{#1}
\else
\input{#1}
\fi
}


\newcommand{\MCases}[1]{\left\lbrace{\begin{array}{rl} #1 \end{array}}\right.}

\ifttm
\newenvironment{MCaseEnv}{\left\lbrace\begin{array}{rl}}{\end{array}\right.}
\else
\newenvironment{MCaseEnv}{\left\lbrace\begin{array}{rl}}{\end{array}\right.}
\fi

\def\MSkip{\ifttm\MCR\fi}

\ifttm
\def\MCR{\special{html:<br />}}
\else
\def\MCR{\ \\}
\fi


% Pragmas - Sind Schluesselwoerter, die dem Preprocessing sowie dem Konverter uebergeben werden und bestimmte
%           Aktionen ausloesen. Im Output (PDF und HTML) tauchen sie nicht auf.
\newcommand{\MPragma}[1]{%
\ifttm%
\special{html:<!-- mpragma;-;}#1\special{html:;; -->}%
\else%
% MPragmas werden vom Preprozessor direkt im LaTeX gefunden
\fi%
}

% Ersatz der Befehle textsubscript und textsuperscript, die ttm nicht kennt
\ifttm%
\newcommand{\MTextsubscript}[1]{\special{html:<sub>}#1\special{html:</sub>}}%
\newcommand{\MTextsuperscript}[1]{\special{html:<sup>}#1\special{html:</sup>}}%
\else%
\newcommand{\MTextsubscript}[1]{\textsubscript{#1}}%
\newcommand{\MTextsuperscript}[1]{\textsuperscript{#1}}%
\fi

%------------------ Einbindung von dia-Diagrammen ----------------------------------------------
% Beim preprocessing wird aus jeder dia-Datei eine tex-Datei und eine pdf-Datei erzeugt,
% diese werden hier jeweils im PDF und HTML eingebunden
% Parameter: Dateiname der mit dia erstellten Datei (OHNE die Endung .dia)
\ifttm%
\newcommand{\MDia}[1]{%
\MGraphicsSolo{#1minthtml.png}{}%
}
\else%
\newcommand{\MDia}[1]{%
\MGraphicsSolo{#1mintpdf.png}{scale=0.1667}%
}
\fi%

% subsup funktioniert im Ausdruck $D={\R}^+_0$, also \R geklammert und sup zuerst
% \ifttm
% \def\MSubsup#1#2#3{\special{html:<msubsup>} #1 #2 #3\special{html:</msubsup>}}
% \else
% \def\MSubsup#1#2#3{{#1}^{#3}_{#2}}
% \fi

%\input{local.tex}

% \ifttm
% \else
% \newwrite\mintlog
% \immediate\openout\mintlog=mintlog.txt
% \fi

% ----------------------- tikz autogenerator -------------------------------------------------------------------

\newcommand{\Mtikzexternalize}{\tikzexternalize}% wird bei Konvertierung ueber mconvert ggf. ausgehebelt!

\ifttm
\else
\tikzset%
{
  % Defines a custom style which generates pdf and converts to (low and hi-res quality) png and svg, then deletes the pdf
  % Important: DO NOT directly convert from pdf to hires-png or from svg to png with GraphViz convert as it has some problems and memory leaks
  png export/.style=%
  {
    external/system call/.add={}{; 
      pdf2svg "\image.pdf" "\image.svg" ; 
      convert -density 112.5 -transparent white "\image.pdf" "\image.png"; 
      inkscape --export-png="\image.4x.png" --export-dpi=450 --export-background-opacity=0 --without-gui "\image.svg"; 
      rm "\image.pdf"; rm "\image.log"; rm "\image.dpth"; rm "\image.idx"
    },
    external/force remake,
  }
}
\tikzset{png export}
\tikzsetexternalprefix{}
% PNGs bei externer Erzeugung in "richtiger" Groesse einbinden
\pgfkeys{/pgf/images/include external/.code={\includegraphics[scale=0.64]{#1}}}
\fi

% Spezielle Umgebung fuer Autogenerierung, Bildernamen sind nur innerhalb eines Moduls (einer MSection) eindeutig)

\newcommand{\MTIKZautofilename}{tikzautofile}

\ifttm
% HTML-Version: Vom Autogenerator erzeugte png-Datei einbinden, tikz selbst nicht ausfuehren (sprich: #1 schlucken)
\newcommand{\MTikzAuto}[1]{%
\addtocounter{MTIKZAutofilenumber}{1}
\renewcommand{\MTIKZautofilename}{mtikzauto_\arabic{MTIKZAutofilenumber}}
\MUGraphicsSolo{\MSectionID\MTIKZautofilename.4x.png}{scale=1}{\special{html:[[!-- svgstyle;}\MSectionID\MTIKZautofilename\special{html: //--]]}} % Styleinfos werden aus original-png, nicht 4x-png geholt!
%\MRegisterFile{\MSectionID\MTIKZautofilename.png} % not used right now
%\MRegisterFile{\MSectionID\MTIKZautofilename.svg}
}
\else%
% PDF-Version: Falls Autogenerator aktiv wird Datei automatisch benannt und exportiert
\newcommand{\MTikzAuto}[1]{%
\addtocounter{MTIKZAutofilenumber}{1}%
\renewcommand{\MTIKZautofilename}{mtikzauto_\arabic{MTIKZAutofilenumber}}
\tikzsetnextfilename{\MTIKZautofilename}%
#1%
}
\fi

% In einer reinen LaTeX-Uebersetzung kapselt der Preambelinclude-Befehl nur input,
% in einer konvertergesteuerten PDF/HTML-Uebersetzung wird er dagegen entfernt und
% die Preambeln an mintmod angehaengt, die Ersetzung wird von mconvert.pl vorgenommen.

\newcommand{\MPreambleInclude}[1]{\input{#1}}

% Globale Watermarksettings (werden auch nochmal zu Beginn jedes subsection gesetzt,
% muessen hier aber auch global ausgefuehrt wegen Einfuehrungsseiten und Inhaltsverzeichnis

\MWatermarkSettings
% ---------------------------------- Parametrisierte Aufgaben ----------------------------------------

\ifttm
\newenvironment{MPExercise}{%
\begin{MExercise}%
}{%
\special{html:<button name="Name_MPEX}\arabic{MExerciseCounter}\special{html:" id="MPEX}\arabic{MExerciseCounter}%
\special{html:" type="button" onclick="reroll('}\arabic{MExerciseCounter}\special{html:');">Neue Aufgabe erzeugen</button>}%
\end{MExercise}%
}
\else
\newenvironment{MPExercise}{%
\begin{MExercise}%
}{%
\end{MExercise}%
}
\fi

% Parameter: Name, Min, Max, PDF-Standard. Name in Deklaration OHNE backslash, im Code MIT Backslash
\ifttm
\newcommand{\MGlobalInteger}[4]{\special{html:%
<!-- onloadstart //-->%
MVAR.push(createGlobalInteger("}#1\special{html:",}#2\special{html:,}#3\special{html:,}#4\special{html:)); %
<!-- onloadstop //-->%
<!-- viewmodelstart //-->%
ob}#1\special{html:: ko.observable(rerollMVar("}#1\special{html:")),%
<!-- viewmodelstop //-->%
}%
}%
\else%
\newcommand{\MGlobalInteger}[4]{\newcounter{mvc_#1}\setcounter{mvc_#1}{#4}}
\fi

% Parameter: Name, Min, Max, PDF-Standard. Name in Deklaration OHNE backslash, im Code MIT Backslash, Wert ist Wurzel von value
\ifttm
\newcommand{\MGlobalSqrt}[4]{\special{html:%
<!-- onloadstart //-->%
MVAR.push(createGlobalSqrt("}#1\special{html:",}#2\special{html:,}#3\special{html:,}#4\special{html:)); %
<!-- onloadstop //-->%
<!-- viewmodelstart //-->%
ob}#1\special{html:: ko.observable(rerollMVar("}#1\special{html:")),%
<!-- viewmodelstop //-->%
}%
}%
\else%
\newcommand{\MGlobalSqrt}[4]{\newcounter{mvc_#1}\setcounter{mvc_#1}{#4}}% Funktioniert nicht als Wurzel !!!
\fi

% Parameter: Name, Min, Max, PDF-Standard zaehler, PDF-Standard nenner. Name in Deklaration OHNE backslash, im Code MIT Backslash
\ifttm
\newcommand{\MGlobalFraction}[5]{\special{html:%
<!-- onloadstart //-->%
MVAR.push(createGlobalFraction("}#1\special{html:",}#2\special{html:,}#3\special{html:,}#4\special{html:,}#5\special{html:)); %
<!-- onloadstop //-->%
<!-- viewmodelstart //-->%
ob}#1\special{html:: ko.observable(rerollMVar("}#1\special{html:")),%
<!-- viewmodelstop //-->%
}%
}%
\else%
\newcommand{\MGlobalFraction}[5]{\newcounter{mvc_#1}\setcounter{mvc_#1}{#4}} % Funktioniert nicht als Bruch !!!
\fi

% MVar darf im HTML nur in MEvalMathDisplay-Umgebungen genutzt werden oder in Strings die an den Parser uebergeben werden
\ifttm%
\newcommand{\MVar}[1]{\special{html:[var_}#1\special{html:]}}%
\else%
\newcommand{\MVar}[1]{\arabic{mvc_#1}}%
\fi

\ifttm%
\newcommand{\MRerollButton}[2]{\special{html:<button type="button" onclick="rerollMVar('}#1\special{html:');">}#2\special{html:</button>}}%
\else%
\newcommand{\MRerollButton}[2]{\relax}% Keine sinnvolle Entsprechung im PDF
\fi

% MEvalMathDisplay fuer HTML wird in mconvert.pl im preprocessing realisiert
% PDF: eine equation*-Umgebung (ueber amsmath)
% HTML: Eine Mathjax-Tex-Umgebung, deren Auswertung mit knockout-obervablen gekoppelt ist
% PDF-Version hier nur fuer pdflatex-only-Uebersetzung gegeben

\ifttm\else\newenvironment{MEvalMathDisplay}{\begin{equation*}}{\end{equation*}}\fi

% ---------------------------------- Spezialbefehle fuer AD ------------------------------------------

%Abk�rzung f�r \longrightarrow:
\newcommand{\lto}{\ensuremath{\longrightarrow}}

%Makro f�r Funktionen:
\newcommand{\exfunction}[5]
{\begin{array}{rrcl}
 #1 \colon  & #2 &\lto & #3 \\[.05cm]  
  & #4 &\longmapsto  & #5 
\end{array}}

\newcommand{\function}[5]{%
#1:\;\left\lbrace{\begin{array}{rcl}
 #2 &\lto & #3 \\
 #4 &\longmapsto  & #5 \end{array}}\right.}


%Die Identit�t:
\DeclareMathOperator{\Id}{Id}

%Die Signumfunktion:
\DeclareMathOperator{\sgn}{sgn}

%Zwei Betonungskommandos (k�nnen angepasst werden):
\newcommand{\highlight}[1]{#1}
\newcommand{\modstextbf}[1]{#1}
\newcommand{\modsemph}[1]{#1}


% ---------------------------------- Spezialbefehle fuer JL ------------------------------------------


\def\jccolorfkt{green!50!black} %Farbe des Funktionsgraphen
\def\jccolorfktarea{green!25!white} %Farbe der Fl"ache unter dem Graphen
\def\jccolorfktareahell{green!12!white} %helle Einf"arbung der Fl"ache unter dem Graphen
\def\jccolorfktwert{green!50!black} %Farbe einzelner Punkte des Graphen

\newcommand{\MPfadBilder}{Bilder}

\ifttm%
\newcommand{\jMD}{\,\MD}%
\else%
\newcommand{\jMD}{\;\MD}%
\fi%

\def\jHTMLHinweisBedienung{\MInputHint{%
Mit Hilfe der Symbole am oberen Rand des Fensters
k"onnen Sie durch die einzelnen Abschnitte navigieren.}}

\def\jHTMLHinweisEingabeText{\MInputHint{%
Geben Sie jeweils ein Wort oder Zeichen als Antwort ein.}}

\def\jHTMLHinweisEingabeTerm{\MInputHint{%
Klammern Sie Ihre Terme, um eine eindeutige Eingabe zu erhalten. 
Beispiel: Der Term $\frac{3x+1}{x-2}$ soll in der Form
\texttt{(3*x+1)/((x+2)^2}$ eingegeben werden (wobei auch Leerzeichen 
eingegeben werden k"onnen, damit eine Formel besser lesbar ist).}}

\def\jHTMLHinweisEingabeIntervalle{\MInputHint{%
Intervalle werden links mit einer "offnenden Klammer und rechts mit einer 
schlie"senden Klammer angegeben. Eine runde Klammer wird verwendet, wenn der 
Rand nicht dazu geh"ort, eine eckige, wenn er dazu geh"ort. 
Als Trennzeichen wird ein Komma oder ein Semikolon akzeptiert.
Beispiele: $(a, b)$ offenes Intervall,
$[a; b)$ links abgeschlossenes, rechts offenes Intervall von $a$ bis $b$. 
Die Eingabe $]a;b[$ f"ur ein offenes Intervall wird nicht akzeptiert.
F"ur $\infty$ kann \texttt{infty} oder \texttt{unendlich} geschrieben werden.}}

\def\jHTMLHinweisEingabeFunktionen{\MInputHint{%
Schreiben Sie Malpunkte (geschrieben als \texttt{*}) aus und setzen Sie Klammern um Argumente f�r Funktionen.
Beispiele: Polynom: \texttt{3*x + 0.1}, Sinusfunktion: \texttt{sin(x)}, 
Verkettung von cos und Wurzel: \texttt{cos(sqrt(3*x))}.}}

\def\jHTMLHinweisEingabeFunktionenSinCos{\MInputHint{%
Die Sinusfunktion $\sin x$ wird in der Form \texttt{sin(x)} angegeben, %
$\cos\left(\sqrt{3 x}\right)$ durch \texttt{cos(sqrt(3*x))}.}}

\def\jHTMLHinweisEingabeFunktionenExp{\MInputHint{%
Die Exponentialfunktion $\MEU^{3x^4 + 5}$ wird als
\texttt{exp(3 * x^4 + 5)} angegeben, %
$\ln\left(\sqrt{x} + 3.2\right)$ durch \texttt{ln(sqrt(x) + 3.2)}.}}

% ---------------------------------- Spezialbefehle fuer Fachbereich Physik --------------------------

\newcommand{\E}{{e}}
\newcommand{\ME}[1]{\cdot 10^{#1}}
\newcommand{\MU}[1]{\;\mathrm{#1}}
\newcommand{\MPG}[3]{%
  \ifnum#2=0%
    #1\ \mathrm{#3}%
  \else%
    #1\cdot 10^{#2}\ \mathrm{#3}%
  \fi}%
%

\newcommand{\MMul}{\MExponentensymbXYZl} % Nur eine Abkuerzung


% ---------------------------------- Stichwortfunktionialitaet ---------------------------------------

% mpreindexentry wird durch Auswahlroutine in conv.pl durch mindexentry substitutiert
\ifttm%
\def\MIndex#1{\index{#1}\special{html:<!-- mpreindexentry;;}#1\special{html:;;}\arabic{MSubjectArea}\special{html:;;}%
\arabic{chapter}\special{html:;;}\arabic{section}\special{html:;;}\arabic{subsection}\special{html:;;}\arabic{MEntryCounter}\special{html:; //-->}%
\setcounter{MLastIndex}{\value{MEntryCounter}}%
\addtocounter{MEntryCounter}{1}%
}%
% Copyrightliste wird als tex-Datei im preprocessing von conv.pl erzeugt und unter converter/tex/entrycollection.tex abgelegt
% Der input-Befehl funktioniert nur, wenn die aufrufende tex-Datei auf der obersten Ebene liegt (d.h. selbst kein input/include ist, insbesondere keine Moduldatei)
\def\MEntryList{} % \input funktioniert nicht, weil ttm (und damit das \input) ausgefuehrt wird, bevor Datei da ist
\else%
\def\MIndex#1{\index{#1}}
\def\MEntryList{\MAbort{Stichwortliste nur im HTML realisierbar}}%
\fi%

\def\MEntry#1#2{\textbf{#1}\MIndex{#2}} % Idee: MLastType auf neuen Entry-Typ und dann ein MLabel vergeben mit autogen-Nummer

% ---------------------------------- Befehle fuer Tests ----------------------------------------------

% MEquationItem stellt eine Eingabezeile der Form Vorgabe = Antwortfeld her, der zweite Parameter kann z.B. MSimplifyQuestion-Befehl sein
\ifttm
\newcommand{\MEquationItem}[2]{{#1}$\,=\,${#2}}%
\else%
\newcommand{\MEquationItem}[2]{{#1}$\;\;=\,${#2}}%
\fi

\ifttm
\newcommand{\MInputHint}[1]{%
\ifnum%
\if\value{MTestSite}>0%
\else%
{\color{blue}#1}%
\fi%
\fi%
}
\else
\newcommand{\MInputHint}[1]{\relax}
\fi

\ifttm
\newcommand{\MInTestHeader}{%
Dies ist ein einreichbarer Test:
\begin{itemize}
\item{Im Gegensatz zu den offenen Aufgaben werden beim Eingeben keine Hinweise zur Formulierung der mathematischen Ausdr�cke gegeben.}
\item{Der Test kann jederzeit neu gestartet oder verlassen werden.}
\item{Der Test kann durch die Buttons am Ende der Seite beendet und abgeschickt, oder zur�ckgesetzt werden.}
\item{Der Test kann mehrfach probiert werden. F�r die Statistik z�hlt die zuletzt abgeschickte Version.}
\end{itemize}
}
\else
\newcommand{\MInTestHeader}{%
\relax
}
\fi

\ifttm
\newcommand{\MInTestFooter}{%
\special{html:<button name="Name_TESTFINISH" id="TESTFINISH" type="button" onclick="finish_button('}\MTestName\special{html:');">Test auswerten</button>}%
\begin{html}
&nbsp;&nbsp;&nbsp;&nbsp;&nbsp;&nbsp;&nbsp;&nbsp;
<button name="Name_TESTRESET" id="TESTRESET" type="button" onclick="reset_button();">Test zur�cksetzen</button>
<br />
<br />
<div class="xreply">
<p name="Name_TESTEVAL" id="TESTEVAL">
Hier erscheint die Testauswertung!
<br />
</p>
</div>
\end{html}
}
\else
\newcommand{\MInTestFooter}{%
\relax
}
\fi


% ---------------------------------- Notationsmakros -------------------------------------------------------------

% Notationsmakros die nicht von der Kursvariante abhaengig sind

\newcommand{\MZahltrennzeichen}[1]{\renewcommand{\MZXYZhltrennzeichen}{#1}}

\ifttm
\newcommand{\MZahl}[3][\MZXYZhltrennzeichen]{\edef\MZXYZtemp{\noexpand\special{html:<mn>#2#1#3</mn>}}\MZXYZtemp}
\else
\newcommand{\MZahl}[3][\MZXYZhltrennzeichen]{{}#2{#1}#3}
\fi

\newcommand{\MEinheitenabstand}[1]{\renewcommand{\MEinheitenabstXYZnd}{#1}}
\ifttm
\newcommand{\MEinheit}[2][\MEinheitenabstXYZnd]{{}#1\edef\MEINHtemp{\noexpand\special{html:<mi mathvariant="normal">#2</mi>}}\MEINHtemp} 
\else
\newcommand{\MEinheit}[2][\MEinheitenabstXYZnd]{{}#1 \mathrm{#2}} 
\fi

\newcommand{\MExponentensymbol}[1]{\renewcommand{\MExponentensymbXYZl}{#1}}
\newcommand{\MExponent}[2][\MExponentensymbXYZl]{{}#1{} 10^{#2}} 

%Punkte in 2 und 3 Dimensionen
\newcommand{\MPointTwo}[3][]{#1(#2\MCoordPointSep #3{}#1)}
\newcommand{\MPointThree}[4][]{#1(#2\MCoordPointSep #3\MCoordPointSep #4{}#1)}
\newcommand{\MPointTwoAS}[2]{\left(#1\MCoordPointSep #2\right)}
\newcommand{\MPointThreeAS}[3]{\left(#1\MCoordPointSep #2\MCoordPointSep #3\right)}

% Masseinheit, Standardabstand: \,
\newcommand{\MEinheitenabstXYZnd}{\MThinspace} 

% Horizontaler Leerraum zwischen herausgestellter Formel und Interpunktion
\ifttm
\newcommand{\MDFPSpace}{\,}
\newcommand{\MDFPaSpace}{\,\,}
\newcommand{\MBlank}{\ }
\else
\newcommand{\MDFPSpace}{\;}
\newcommand{\MDFPaSpace}{\;\;}
\newcommand{\MBlank}{\ }
\fi

% Satzende in herausgestellter Formel mit horizontalem Leerraum
\newcommand{\MDFPeriod}{\MDFPSpace .}

% Separation von Aufzaehlung und Bedingung in Menge
\newcommand{\MCondSetSep}{\,:\,} %oder '\mid'

% Konverter kennt mathopen nicht
\ifttm
\def\mathopen#1{}
\fi

% -----------------------------------START Rouletteaufgaben ------------------------------------------------------------

\ifttm
% #1 = Dateiname, #2 = eindeutige ID fuer das Roulette im Kurs
\newcommand{\MDirectRouletteExercises}[2]{
\begin{MExercise}
\texttt{Im HTML erscheinen hier Aufgaben aus einer Aufgabenliste...}
\end{MExercise}
}
\else
\newcommand{\MDirectRouletteExercises}[2]{\relax} % wird durch mconvert.pl gefunden und ersetzt
\fi


% ---------------------------------- START Makros, die von der Kursvariante abhaengen ----------------------------------

\ifvariantunotation
  % unotation = An Universitaeten uebliche Notation
  \def\MVariant{unotation}

  % Trennzeichen fuer Dezimalzahlen
  \newcommand{\MZXYZhltrennzeichen}{.}

  % Exponent zur Basis 10 in der Exponentialschreibweise, 
  % Standardmalzeichen: \times
  \newcommand{\MExponentensymbXYZl}{\times} 

  % Begrenzungszeichen fuer offene Intervalle
  \newcommand{\MoIl}[1][]{\mbox{}#1(\mathopen{}} % bzw. ']'
  \newcommand{\MoIr}[1][]{#1)\mbox{}} % bzw. '['

  % Zahlen-Separation im IntervaLL
  \newcommand{\MIntvlSep}{,} %oder ';'

  % Separation von Elementen in Mengen
  \newcommand{\MElSetSep}{,} %oder ';'

  % Separation von Koordinaten in Punkten
  \newcommand{\MCoordPointSep}{,} %oder ';' oder '|', '\MThinspace|\MThinspace'

\else
  % An dieser Stelle wird angenommen, dass std-Variante aktiv ist
  % std = beschlossene Notation im TU9-Onlinekurs 
  \def\MVariant{std}

  % Trennzeichen fuer Dezimalzahlen
  \newcommand{\MZXYZhltrennzeichen}{,}

  % Exponent zur Basis 10 in der Exponentialschreibweise, 
  % Standardmalzeichen: \times
  \newcommand{\MExponentensymbXYZl}{\times} 

  % Begrenzungszeichen fuer offene Intervalle
  \newcommand{\MoIl}[1][]{\mbox{}#1]\mathopen{}} % bzw. '('
  \newcommand{\MoIr}[1][]{#1[\mbox{}} % bzw. ')'

  % Zahlen-Separation im IntervaLL
  \newcommand{\MIntvlSep}{;} %oder ','
  
  % Separation von Elementen in Mengen
  \newcommand{\MElSetSep}{;} %oder ','

  % Separation von Koordinaten in Punkten
  \newcommand{\MCoordPointSep}{;} %oder '|', '\MThinspace|\MThinspace'

\fi



% ---------------------------------- ENDE Makros, die von der Kursvariante abhaengen ----------------------------------


% diese Kommandos setzen Mathemodus vorraus
\newcommand{\MGeoAbstand}[2]{[\overline{{#1}{#2}}]}
\newcommand{\MGeoGerade}[2]{{#1}{#2}}
\newcommand{\MGeoStrecke}[2]{\overline{{#1}{#2}}}
\newcommand{\MGeoDreieck}[3]{{#1}{#2}{#3}}

%
\ifttm
\newcommand{\MOhm}{\special{html:<mn>&#x3A9;</mn>}}
\else
\newcommand{\MOhm}{\Omega} %\varOmega
\fi


\def\PERCTAG{\MAbort{PERCTAG ist zur internen verwendung in mconvert.pl reserviert, dieses Makro darf sonst nicht benutzt werden.}}

% Im Gegensatz zu einfachen html-Umgebungen werden MDirectHTML-Umgebungen von mconvert.pl am ganzen ttm-Prozess vorbeigeschleust und aus dem PDF komplett ausgeschnitten
\ifttm%
\newenvironment{MDirectHTML}{\begin{html}}{\end{html}}%
\else%
\newenvironment{MDirectHTML}{\begin{html}}{\end{html}}%
\fi

% Im Gegensatz zu einfachen Mathe-Umgebungen werden MDirectMath-Umgebungen von mconvert.pl am ganzen ttm-Prozess vorbeigeschleust, ueber MathJax realisiert, und im PDF als $$ ... $$ gesetzt
\ifttm%
\newenvironment{MDirectMath}{\begin{html}}{\end{html}}%
\else%
\newenvironment{MDirectMath}{\begin{equation*}}{\end{equation*}}% Vorsicht, auch \[ und \] werden in amsmath durch equation* redefiniert
\fi

% ---------------------------------- Location Management ---------------------------------------------

% #1 = buttonname (muss in files/images liegen und Format 48x48 haben), #2 = Vollstaendiger Einrichtungsname, #3 = Kuerzel der Einrichtung,  #4 = Name der include-texdatei
\ifttm
\newcommand{\MLocationSite}[3]{\special{html:<!-- mlocation;;}#1\special{html:;;}#2\special{html:;;}#3\special{html:;; //-->}}
\else
\newcommand{\MLocationSite}[3]{\relax}
\fi

% ---------------------------------- Copyright Management --------------------------------------------

\newcommand{\MCCLicense}{%
{\color{green}\textbf{CC BY-SA 3.0}}
}

\newcommand{\MCopyrightLabel}[1]{ (\MSRef{L_COPYRIGHTCOLLECTION}{Lizenz})\MLabel{#1}}

% Copyrightliste wird als tex-Datei im preprocessing erzeugt und unter converter/tex/copyrightcollection.tex abgelegt
% Der input-Befehl funktioniert nur, wenn die aufrufende tex-Datei auf der obersten Ebene liegt (d.h. selbst kein input/include ist, insbesondere keine Moduldatei)
\newcommand{\MCopyrightCollection}{\input{copyrightcollection.tex}}

% MCopyrightNotice fuegt eine Copyrightnotiz ein, der parser ersetzt diese durch CopyrightNoticePOST im preparsing, diese Definition wird nur fuer reine pdflatex-Uebersetzungen gebraucht
% Parameter: #1: Kurze Lizenzbeschreibung (typischerweise \MCCLicense)
%            #2: Link zum Original (http://...) oder NONE falls das Bild selbst ein Original ist, oder TIKZ falls das Bild aus einer tikz-Umgebung stammt
%            #3: Link zum Autor (http://...) oder MINT falls Original im MINT-Kolleg erstellt oder NONE falls Autor unbekannt
%            #4: Bemerkung (z.B. dass Datei mit Maple exportiert wurde)
%            #5: Labelstring fuer existierendes Label auf das copyrighted Objekt, mit MCopyrightLabel erzeugt
%            Keines der Felder darf leer sein!
\newcommand{\MCopyrightNotice}[5]{\MCopyrightNoticePOST{#1}{#2}{#3}{#4}{#5}}

\ifttm%
\newcommand{\MCopyrightNoticePOST}[5]{\relax}%
\else%
\newcommand{\MCopyrightNoticePOST}[5]{\relax}%
\fi%

% ---------------------------------- Meldungen fuer den Benutzer des Konverters ----------------------
\MPragma{mintmodversion;P0.1.0}
\MPragma{usercomment;This is file mintmod.tex version P0.1.0}


% ----------------------------------- Spezialelemente fuer Konfigurationsseite, werden nicht von mintscripts.js verwaltet --

% #1 = DOM-id der Box
\ifttm\newcommand{\MConfigbox}[1]{\special{html:<input cfieldtype="2" type="checkbox" name="Name_}#1\special{html:" id="}#1\special{html:" onchange="confHandlerChange('}#1\special{html:');"/>}}\fi % darf im PDF nicht aufgerufen werden!


\MPragma{MathSkip}
\Mtikzexternalize

\begin{document}

\MSection{Grundlagen der anschaulichen Vektorgeometrie}
\MLabel{VBKM10}
\MSetSectionID{VBKM10} % wird fuer tikz-Dateien verwendet

\begin{MSectionStart}
\MDeclareSiteUXID{VBKM10_START}

\MModstartBox
\end{MSectionStart}

\MSubsection{Vom Pfeil zum Vektor}
\MLabel{VBKM10_PfeilVektor}

\begin{MIntro}
\MLabel{VBKM10_PfeilVektor_Intro}
\MDeclareSiteUXID{VBKM10_PfeilVektor_Intro}
Eine grundlegende Idee hinter dem mathematischen Konzept des Vektors ist eine physikalische. Es gibt in der naturwissenschaftlichen Praxis einerseits Größen, die nur durch eine Zahl beschrieben werden, die also einen bestimmten Betrag haben, wie etwa Spannung, Arbeit oder Leistung. Diese Größen werden mathematisch einfach durch Elemente aus der Menge der reellen Zahlen beschrieben. Andererseits existieren auch Größen, die neben einem Betrag auch eine bestimmte \textit{Richtung} besitzen, wie etwa Kraft oder Geschwindigkeit. So kann man sich zum Beispiel eine Kraft, die mit einem bestimmten Betrag an einem Punkt angreift, als Pfeil der entsprechenden Länge, der am Angriffspunkt sitzt, veranschaulichen. Die Richtung des Pfeils entspricht in diesem Fall der Wirkungsrichtung der Kraft:

\begin{center}
\MTikzAuto{%
\begin{tikzpicture} 
%Pfeil
\draw[color=black,->] (0,0) -- (3,1);
\draw[color=black] (1.5,0.8) node[anchor=east] {\footnotesize Länge $\hat{=}$ Betrag der Kraft};
\draw[color=black] (1.5,0.2) node[anchor=west] {\footnotesize Richtung $\hat{=}$ Wirkungsrichtung der Kraft};
\draw[color=black] (3,1) node[anchor=west] {\footnotesize Pfeil $\hat{=}$ Kraft};
%Punkt
\draw[fill=red] (0,0) circle (2pt);
\draw[color=red] (0,0) node[anchor=north east] {\footnotesize Angriffspunkt};
\end{tikzpicture}       
}%
\end{center}

Solche Größen werden mathematisch durch Vektoren beschrieben. In der naturwissenschaftlichen Anwendung müsste man für den Betrag von Vektoren bestimmte Maßeinheiten benutzen (zum Beispiel die Einheit Newton  für Kräfte), diese Einheiten spielen aber für rein mathematische Betrachtungen keine Rolle und werden deshalb im Folgenden stets weggelassen. Das Konzept des Vektors ist der wesentliche Inhalt dieses Abschnitts und wird im zweiten Abschnitt \MNRef{VBKM10_Vektoren} behandelt. Da Vektoren nicht nur in zwei Dimensionen (also in der Ebene), sondern auch in drei Dimensionen (also im Raum) betrachtet werden sollen, muss zunächst das Konzept der Koordinatensysteme und Punkte aus Kapitel \MNRef{VBKM09} auf drei Dimensionen erweitert werden. Dies geschieht im ersten Abschnitt \MNRef{VBKM10_Raumkoordinaten}. Schließlich stellt man fest, dass man mit Vektoren auch bestimmte Rechenoperationen ausführen kann. Das Rechnen mit Vektoren wird dann im dritten Abschnitt \MNRef{VBKM10_Vektorrechnung} untersucht.    
\end{MIntro}

\begin{MXContent}{Koordinatensysteme im Raum}{Raumkoordinaten}{STD}
\MLabel{VBKM10_Raumkoordinaten}
\MDeclareSiteUXID{VBKM10_Raumkoordinaten}
Im vorhergehenden Kapitel \MNRef{VBKM09} wurden im Abschnitt \MNRef{M09_1kartesisch} kartesische Koordinatensysteme und Punkte in der Ebene mit Koordinaten bezüglich dieser Systeme eingeführt. Ein sicherer Umgang mit diesen Konzepten wird hier nun vorausgesetzt. Zur eindeutigen Beschreibung eines \MEntry{Punktes im Raum}{Punkt (3-D)} braucht man drei -- statt nur zwei -- \MEntry{Koordinaten}{Koordinaten (3-D)}. Damit benötigt ein \MEntry{Koordinatensystem}{Koordinatensystem (3-D)} im Raum ebenfalls drei \MEntry{Achsen}{Achsen (3-D)}, die als $x$-, $y$- und $z$-Achse (oder manchmal auch als $x_1$-, $x_2$- und $x_3$-Achse) bezeichnet werden. Für Punkte werden wieder üblicherweise Großbuchstaben $P,Q,R,\MHDots$ und für ihre Koordinaten Kleinbuchstaben $a,b,c,x,y,z,\MHDots$ als Variablen verwendet. In Erweiterung der Schreibweise aus Kapitel \MNRef{VBKM09} werden die Koordinaten von Punkten zum Beispiel wie folgt notiert:
\[
 P=\MPointThree{1}{3}{0}
\]
oder
\[
 Q=\MPointThree{-2}{2}{3}\MDFPeriod
\]
Hier ist $Q$ beispielsweise ein Punkt mit der $x$-Koordinate $-2$, der $y$-Koordinate $2$ und der $z$-Koordinate $3$. Für den Punkt mit den Koordinaten $\MPointThree{0}{0}{0}$ ist die Bezeichnung \textbf{Ursprung} und das Symbol $O$ (vom englischen \textit{origin}) reserviert. Alle diese Punkte sind im folgenden Bild dargestellt:  

\begin{center}
\MTikzAuto{%
\begin{tikzpicture}[>=stealth]
% Dashed lines for the points P, Q
\draw[dashed, color=gray] (1,0) -- (1,3);
\draw[dashed, color=gray] (1,3) -- (0,3);
\draw[dashed, color=gray] (xyz cs:x=0,z=3) -- (xyz cs:x=-2,z=3);
\draw[dashed, color=gray] (xyz cs:x=-2,z=3) -- (xyz cs:x=-2,z=0);
\draw[dashed, color=gray] (xyz cs:x=-2,y=0,z=3) -- (xyz cs:x=-2,y=2,z=3);
\draw[dashed, color=gray] (xyz cs:x=0,y=2,z=3) -- (xyz cs:x=-2,y=2,z=3);
\draw[dashed, color=gray] (xyz cs:x=-2,y=2,z=3) -- (xyz cs:x=-2,y=2,z=0);
\draw[dashed, color=gray] (xyz cs:x=0,y=2,z=3) -- (xyz cs:x=0,y=2,z=0);
\draw[dashed, color=gray] (xyz cs:x=0,y=2,z=0) -- (xyz cs:x=-2,y=2,z=0);
\draw[dashed, color=gray] (xyz cs:y=0,z=3) -- (xyz cs:y=2,z=3); 
% 
% % Dots and labels for P, Q
\draw[fill=red] (1,3) circle (1.5pt);
\draw[color=red] (1,3) node[right] {\footnotesize $P=\MPointThree{1}{3}{0}$};
\draw[fill=violet] (xyz cs:x=-2,y=2,z=3) circle (1.5pt);
\draw[color=violet] (xyz cs:x=-2,y=2,z=3) node[left] {\footnotesize $Q=\MPointThree{-2}{2}{3}$};
% The origin
\node[align=center] at (2,-2) (ori) {\footnotesize $O=\MPointThree{0}{0}{0}$\\(Ursprung)};
\draw[->,help lines,shorten >=3pt] (ori) .. controls (0.5,-1.5) and (0.8,-1) .. (0,0,0);
% The axes
\draw[->] (xyz cs:x=-3.5) -- (xyz cs:x=3.5) node[above] {\footnotesize $x$};
\draw[->] (xyz cs:y=-3.5) -- (xyz cs:y=3.5) node[right] {\footnotesize $y$};
\draw[->] (xyz cs:z=-3.5) -- (xyz cs:z=3.5) node[left] {\footnotesize $z$};
% The ticks
\foreach \coo in {-3,-2,-1,1,2,3}
{
  \draw (\coo,-3pt) -- (\coo,3pt) node[below=4pt] {\footnotesize \coo};
  \draw (-3pt,\coo) -- (3pt,\coo) node[left=4pt] {\footnotesize \coo};
  \draw (xyz cs:y=-0.1pt,z=\coo) -- (xyz cs:y=0.1pt,z=\coo) node[below=3pt] {\scriptsize \coo};
}
\end{tikzpicture}
}
\end{center}
Die gestrichelten Hilfslinien in diesem Bild geben einen Hinweis darauf, wie in einer solchen dreidimensionalen Darstellung die Koordinaten von Punkten korrekt eingezeichnet und abgelesen werden können. Man beachte, dass die Hilfslinien alle parallel zu den Koordinatenachsen verlaufen.

Es werden nur Koordinatensysteme im Raum betrachtet, deren Achsen alle senkrecht aufeinander stehen. Diese heißen auch \MEntry{kartesische Koordinatensysteme}{kartesisches Koordinatensystem (3-D)}. Es wird hier außerdem die übliche mathematische Konvention benutzt, dass die Koordinatensysteme im Raum \textbf{rechtshändig} oder sogenannte \textbf{Rechtssysteme} sein sollen. Dies bedeutet, dass die positiven Achsenrichtungen von $x$-, $y$- und $z$-Achse durch die Drei-Finger-Regel der \textit{rechten} Hand bestimmt werden:

\begin{center}
\MUGraphicsSolo{rechtehand.png}{scale=0.4}{width:368px}
\end{center}
%Bemerkung: Bild von hier https://de.wikipedia.org/wiki/Rechtssystem_%28Mathematik%29, ist public domain, deshalb keine Lizenz nötig.

Trotzdem gibt es hier noch unterschiedliche mögliche Darstellungsformen: zum Beispiel ein Verlauf der $x$-Achse nach rechts, der $y$-Achse nach oben und der $z$-Achse nach vorne aus der Zeichenebene heraus -- wie im Koordinatensystem mit den Punkten $P$ und $Q$ oben -- oder ein Verlauf der $x$-Achse nach rechts, der $y$-Achse nach hinten in die Zeichenebene hinein und der $z$-Achse nach oben -- wie im Drei-Finger-Bild oben. Die Rechtshändigkeit ist aber in beiden Fällen gegeben.  

\begin{MExercise}
Geben Sie die Koordinaten der im folgenden Bild eingezeichneten Punkte an. Überlegen Sie sich außerdem, wie man alle eingezeichneten Punkte zu einem mathematischen Objekt zusammenfassen kann.
\begin{center}
\MTikzAuto{%
\begin{tikzpicture}[>=stealth]
% Dashed lines for the points 
%A,B,C,D:
\draw[dashed, color=gray] (2,0) -- (2,1);
\draw[dashed, color=gray] (2,0) -- (2,-2);
\draw[color=green] (2,-2,0) -- (2,-2,1);
\draw[dashed, color=gray] (2,-2,0) -- (0,0,0);
\draw[dashed, color=gray] (2,-2,1) -- (0,0,1);
%P:
\draw[dashed, color=gray] (-2,0,0) -- (-2,1,0);
\draw[dashed, color=gray] (-2,1,0) -- (-2,1,-1);
%\draw[dashed, color=gray] (-2,1,0) -- (0,1,0);
\draw[dashed, color=gray] (0,1,-1) -- (-2,1,-1);
\draw[dashed, color=gray] (0,1,-1) -- (0,0,-1);
% 

% The axes
\draw[->] (xyz cs:x=-3.5) -- (xyz cs:x=3.5) node[above] {\footnotesize $x$};
\draw[->] (xyz cs:y=-3.5) -- (xyz cs:y=3.5) node[right] {\footnotesize $y$};
\draw[->] (xyz cs:z=-3.5) -- (xyz cs:z=3.5) node[left] {\footnotesize $z$};
% The ticks
\foreach \coo in {-3,-2,-1,1,2,3}
{
  \draw (\coo,-3pt) -- (\coo,3pt) node[below=4pt] {\footnotesize \coo};
  \draw (-3pt,\coo) -- (3pt,\coo) node[left=4pt] {\footnotesize \coo};
  \draw (xyz cs:y=-0.1pt,z=\coo) -- (xyz cs:y=0.1pt,z=\coo) node[below=3pt] {\scriptsize \coo};
}
% % Dots and labels for the points
\draw[fill=red] (2,0,0) circle (1.5pt);
\draw[color=red] (2,0,0) node[anchor=south west] {\footnotesize $A$};
\draw[fill=blue] (2,1,0) circle (1.5pt);
\draw[color=blue] (2,1,0) node[right] {\footnotesize $B$};
\draw[fill=violet] (2,-2,0) circle (1.5pt);
\draw[color=violet] (2,-2,0) node[right] {\footnotesize $C$};
\draw[fill=green] (2,-2,1) circle (1.5pt);
\draw[color=green] (2,-2,1) node[below] {\footnotesize $D$};
\draw[fill=black] (-2,1,-1) circle (1.5pt);
\draw[color=black] (-2,1,-1) node[left] {\footnotesize $P$};
\draw[color=green] (0,0,0) -- (0,0,1);
\end{tikzpicture}
}
\end{center}

\begin{MExerciseItems}
\item{\MEquationItem{$A$}{\MLFunctionQuestion{15}{(2,0,0)}{5}{x}{5}{Point1}}.}
\item{\MEquationItem{$B$}{\MLFunctionQuestion{15}{(2,1,0)}{5}{x}{5}{Point2}}.}
\item{\MEquationItem{$C$}{\MLFunctionQuestion{15}{(2,-2,0)}{5}{x}{5}{Point3}}.}
\item{\MEquationItem{$D$}{\MLFunctionQuestion{15}{(2,-2,1)}{5}{x}{5}{Point4}}.}
\item{\MEquationItem{$P$}{\MLFunctionQuestion{15}{(-2,1,-1)}{5}{x}{5}{Point5}}.}
\end{MExerciseItems}
\MInputHint{Geben Sie Punkte in der Form $(x;y;z)$ ein, beispielsweise \texttt{(8;-4;15)} für den Punkt mit der $x$-Koordinate $8$, der $y$-Koordinate $-4$ und der $z$-Koordinate $15$.}

\begin{MHint}{\iSolution}
Die Koordinatentripel der gefragten Punkte lauten:
\[
 A=\MPointThree{2}{0}{0} \MDFPSpace,
\]
\[
 B=\MPointThree{2}{1}{0} \MDFPSpace,
\]
\[
 C=\MPointThree{2}{-2}{0} \MDFPSpace,
\]
\[
 D=\MPointThree{2}{-2}{1} \MDFPSpace,
\]
\[
 P=\MPointThree{-2}{1}{-1}\MDFPeriod
\]
Die Menge aller im obigen Bild eingezeichneten Punkte ist
\[
 \{A\MElSetSep B\MElSetSep C\MElSetSep D\MElSetSep P\} = \{\MPointThree{2}{0}{0}\MElSetSep\MPointThree{2}{1}{0} \MElSetSep\MPointThree{2}{-2}{0} \MElSetSep\MPointThree{2}{-2}{1} \MElSetSep\MPointThree{-2}{1}{-1}\} \MDFPeriod
\]

\end{MHint}
\end{MExercise}

Analog zum zweidimensionalen Fall in Abschnitt \MNRef{VBKM09_Punkte} kann man also auch hier im dreidimensionalen Fall eine beliebige Anzahl von Punkten im Raum wieder zu einer \MEntry{Punktmenge}{Punktmenge (3-D)} zusammenfassen. Insbesondere gibt es auch hier wieder die folgende Bezeichnung:

\begin{MInfo}
Die Menge aller Punkte im Raum, als Koordinatentripel bezüglich eines gegebenen kartesischen Koordinatensystems, wird mit
\[
 \R^3 := \{ \MPointThree{x}{y}{z}\MCondSetSep x\in\R\wedge y\in\R\wedge z\in\R \}
\]
bezeichnet. Das Symbol $\R^3$ wird dabei als \glqq$\R$ drei\grqq oder \glqq$\R$ hoch drei\grqq\ gesprochen. Dies spiegelt wieder, dass jeder Punkt eindeutig durch ein Koordinatentripel, bestehend aus drei reellen Zahlen, beschrieben werden kann.    
\end{MInfo}

\end{MXContent}


\begin{MXContent}{Vektoren in der Ebene und im Raum}{Vektoren}{STD}
\MLabel{VBKM10_Vektoren}
\MDeclareSiteUXID{VBKM10_Vektoren}
Punkte in der Ebene und im Raum, die als Koordinatenpaare oder -tripel bezüglich eines vorgegebenen Koordinatensystems gegeben sind, können durch Strecken verbunden werden. Gibt man diesen Strecken zusätzlich eine Orientierung (d.h. legt man einen der Punkte als Fußpunkt und einen der Punkte als Spitze fest), so erhält man Pfeile, die von einem Punkt zum anderen zeigen:

\begin{center}
\begin{tabular}{cc}
\MTikzAuto{
\begin{tikzpicture}[>=stealth]
%Koordinatensystem
\draw[->,color=black] (-1.5,0) -- (4.5,0);
\foreach \x in {-1,1,2,3,4}
\draw[shift={(\x,0)},color=black] (0pt,2pt) -- (0pt,-2pt) node[below] {\footnotesize $\x$};
\draw[->,color=black] (0,-1.5) -- (0,4.5);
\foreach \y in {-1,1,2,3,4}
\draw[shift={(0,\y)},color=black] (2pt,0pt) -- (-2pt,0pt) node[left] {\footnotesize $\y$};
\draw[color=black] (-10pt,-8pt) node[right] {\footnotesize $0$};
%Achsenbeschriftung
\draw (4.5,0) node[anchor=north west] {$x$};
\draw (-0.5,4.8) node[anchor=north west] {$y$};
%Punkte
\draw [fill = black] (0,0) circle (1.5pt);
\draw [color=black] (0,0) node[anchor=south east] {\footnotesize $O=\MPointTwo{0}{0}$};
\draw [fill = red] (2,1) circle (1.5pt);
\draw [color=red] (2,1) node[anchor=north west] {\footnotesize $P=\MPointTwo{2}{1}$};
\draw [fill = blue] (4,1) circle (1.5pt);
\draw [color=blue] (4,1) node[anchor=south west] {\footnotesize $Q=\MPointTwo{4}{1}$};
\draw [fill = violet] (1,3) circle (1.5pt);
\draw [color=violet] (1,3) node[anchor=south] {\footnotesize $R=\MPointTwo{1}{3}$};
%Pfeile
\draw[->, line width = 1.5pt] (0,0) -- (2,1);
\draw[->, line width = 1.5pt] (4,1) -- (1,3);
\end{tikzpicture}
}

&

\MTikzAuto{%
\begin{tikzpicture}[>=stealth]

% The axes
\draw[->] (xyz cs:x=-2.5) -- (xyz cs:x=3.5) node[above] {\footnotesize $x$};
\draw[->] (xyz cs:y=-2.5) -- (xyz cs:y=3.5) node[right] {\footnotesize $y$};
\draw[->] (xyz cs:z=-2.5) -- (xyz cs:z=3.5) node[left] {\footnotesize $z$};
% The ticks
\foreach \coo in {-2,-1,1,2,3}
{
  \draw (\coo,-3pt) -- (\coo,3pt) node[below=4pt] {\footnotesize \coo};
  \draw (-3pt,\coo) -- (3pt,\coo) node[left=4pt] {\footnotesize \coo};
  \draw (xyz cs:y=-0.1pt,z=\coo) -- (xyz cs:y=0.1pt,z=\coo) node[below=3pt] {\scriptsize \coo};
}
% % Dots and labels for the points
\draw[fill=black] (0,0,0) circle (1.5pt);
\draw[color=black] (0,0,0) node[anchor=south east] {\footnotesize $O=\MPointThree{0}{0}{0}$};
\draw[fill=red] (2,-1,2) circle (1.5pt);
\draw[color=red] (2,-1,2) node[anchor=north] {\footnotesize $P=\MPointThree{2}{-1}{2}$};
\draw[fill=blue] (3,1,0) circle (1.5pt);
\draw[color=blue] (3,1,0) node[anchor=west] {\footnotesize $Q=\MPointThree{3}{1}{0}$};
\draw[fill=violet] (1,3,-1) circle (1.5pt);
\draw[color=violet] (1,3,-1) node[anchor=west] {\footnotesize $R=\MPointThree{1}{3}{-1}$};
%Arrows
\draw[->, line width = 1.5pt] (0,0,0) -- (2,-1,2);
\draw[->, line width = 1.5pt] (3,1,0) -- (1,3,-1);
\end{tikzpicture}
}
\end{tabular}
\end{center}

Im Sinne dieser Abbildungen kann man den Informationsgehalt eines Pfeils folgendermaßen charakterisieren: Ein Pfeil gibt an, wie man von einem Punkt (am Fußpunkt des Pfeils) zu einem anderen Punkt (an der Spitze des Pfeils) gelangt. So gibt der Pfeil, welcher im zweidimensionalen Fall $Q$ mit $R$ verbindet, wieder, dass man von $Q$ aus um $3$ Längeneinheiten nach links und um $2$ Längeneinheiten nach oben gehen muss, um $R$ zu erreichen. Oder in einer mathematischeren Sprechweise: Man gehe von $Q$ aus um $-3$ in $x$-Richtung und um $2$ in $y$-Richtung. Für den Pfeil, der im zweidimensionalen Fall den Ursprung mit dem Punkt $P$ verbindet, ist dies sogar noch einfacher, denn um von $O$ nach $P$ zu gelangen, geht man einfach um $2$ in $x$-Richtung und um $1$ in $y$-Richtung. Diese Werte entsprechen natürlich genau den Koordinaten des Punktes $P$.

\begin{MExercise}
Bestimmen Sie für die Pfeile in der obigen dreidimensionalen Abbildung die (vorzeichenbehafteten) Bewegungen in die drei Koordinatenrichtungen, die nötig sind, um vom Fußpunkt zur Spitze des jeweiligen Pfeils zu gelangen. Gehen Sie analog zu den oben stehenden Erläuterungen im zweidimensionalen Fall vor.

\begin{itemize}
 \item[{}] Für den Pfeil von $O$ nach $P$: 
 
  \begin{MExerciseItems}
  \item{in $x$-Richtung: \MLFunctionQuestion{15}{2}{5}{x}{5}{Pfeil1x},}
  \item{in $y$-Richtung: \MLFunctionQuestion{15}{-1}{5}{x}{5}{Pfeil1y},}
  \item{in $z$-Richtung: \MLFunctionQuestion{15}{2}{5}{x}{5}{Pfeil1z}.}
  \end{MExerciseItems} 
  \item[{}] Für den Pfeil von $Q$ nach $R$:
  
  \begin{MExerciseItems}
  \item{in $x$-Richtung: \MLFunctionQuestion{15}{-2}{5}{x}{5}{Pfeil2x},}
  \item{in $y$-Richtung: \MLFunctionQuestion{15}{2}{5}{x}{5}{Pfeil2y},}
  \item{in $z$-Richtung: \MLFunctionQuestion{15}{-1}{5}{x}{5}{Pfeil2z}.}
  \end{MExerciseItems} 
\end{itemize}

\begin{MHint}{\iSolution}

\begin{itemize}
 \item[{}] Für den Pfeil von $O$ nach $P$: 
 
  \begin{MExerciseItems}
  \item{in $x$-Richtung: $2$,}
  \item{in $y$-Richtung: $-1$,}
  \item{in $z$-Richtung: $2$.}
  \end{MExerciseItems} 
  \item[{}] Für den Pfeil von $Q$ nach $R$:
  
  \begin{MExerciseItems}
  \item{in $x$-Richtung: $-2$,}
  \item{in $y$-Richtung: $2$,}
  \item{in $z$-Richtung: $-1$.}
  \end{MExerciseItems} 
\end{itemize}

\end{MHint}
\end{MExercise}

Nun ist ersichtlich, dass auch im Fall der Pfeile, die jeweils $Q$ mit $R$ verbinden, die Bewegungen in die jeweiligen Koordinatenrichtungen durch die Koordinaten der Punkte an Fußpunkt und Spitze der Pfeile bestimmt werden. Es gilt offenbar im zweidimensionalen Fall:
\[
 \left.\begin{array}{r}R=\MPointTwo{1}{3} \\ Q=\MPointTwo{4}{1}\end{array}\right\}\MDFPSpace\Rightarrow\MDFPSpace\left\{\begin{array}{l}\textrm{in}\MBlank x\textrm{-Richtung:}\MDFPSpace-3=1-4 \\ \textrm{in}\MBlank y\textrm{-Richtung:}\MDFPSpace2=3-1\end{array}\right.
\]
und im dreidimensionalen Fall:
\[
 \left.\begin{array}{r}R=\MPointThree{1}{3}{-1} \\ Q=\MPointThree{3}{1}{0}\end{array}\right\}\MDFPSpace\Rightarrow\MDFPSpace\left\{\begin{array}{l}\textrm{in}\MBlank x\textrm{-Richtung:}\MDFPSpace-2=1-3 \\ \textrm{in}\MBlank y\textrm{-Richtung:}\MDFPSpace2=3-1 \\ \textrm{in}\MBlank z\textrm{-Richtung:}\MDFPSpace-1=-1-0\MDFPeriod\end{array}\right.
\]
Die Bewegungen in die verschiedenen Koordinatenrichtungen ergeben sich also als die Differenzen der Koordinaten des Punkts an der Spitze des Pfeils und des Punkts am Fußpunkt des Pfeils. Dies bedeutet aber, dass sich alle Pfeile, welche Punktepaare miteinander verbinden, für die sich jeweils gleiche Koordinatendifferenzen ergeben, nur durch eine Parallelverschiebung -- unter Beibehaltung ihrer Orientierung -- unterscheiden. Die Punktepaare $P$ und $Q$, $A$ und $B$ sowie $O$ und $R$ im folgenden zweidimensionalen Bild werden jeweils durch Pfeile verbunden, die durch Parallelverschiebung zur Deckung gebracht werden können:
\begin{center}
\MTikzAuto{
\begin{tikzpicture}[>=stealth]
%Koordinatensystem
\draw[->,color=black] (-3.5,0) -- (4.5,0);
\foreach \x in {-3,-2,-1,1,2,3,4}
\draw[shift={(\x,0)},color=black] (0pt,2pt) -- (0pt,-2pt) node[below] {\footnotesize $\x$};
\draw[->,color=black] (0,-2.5) -- (0,3.5);
\foreach \y in {-2,-1,1,2,3}
\draw[shift={(0,\y)},color=black] (2pt,0pt) -- (-2pt,0pt) node[left] {\footnotesize $\y$};
\draw[color=black] (-10pt,-8pt) node[right] {\footnotesize $0$};
%Achsenbeschriftung
\draw (4.5,0) node[anchor=north west] {$x$};
\draw (-0.5,3.8) node[anchor=north west] {$y$};
%Punkte
\draw [fill = black] (0,0) circle (1.5pt);
\draw [color=black] (0,0) node[anchor=south west] {\footnotesize $O=\MPointTwo{0}{0}$};
\draw [fill = red] (-3,1) circle (1.5pt);
\draw [color=red] (-3,1) node[anchor=south west] {\footnotesize $R=\MPointTwo{-3}{1}$};
\draw [fill = blue] (4,1) circle (1.5pt);
\draw [color=blue] (4,1) node[anchor=south west] {\footnotesize $P=\MPointTwo{4}{1}$};
\draw [fill = violet] (1,2) circle (1.5pt);
\draw [color=violet] (1,2) node[anchor=south] {\footnotesize $Q=\MPointTwo{1}{2}$};
\draw [fill = green] (2,-2) circle (1.5pt);
\draw [color=green] (2,-2) node[anchor=north west] {\footnotesize $A=\MPointTwo{2}{-2}$};
\draw [fill = brown] (-1,-1) circle (1.5pt);
\draw [color=brown] (-1,-1) node[anchor=east] {\footnotesize $B=\MPointTwo{-1}{-1}$};
%Pfeile
\draw[->, line width = 1.5pt] (4,1) -- (1,2);
\draw[->, line width = 1.5pt] (0,0) -- (-3,1);
\draw[->, line width = 1.5pt] (2,-2) -- (-1,-1);
\end{tikzpicture}
} 
\end{center}

Man kann hier natürlich noch beliebig viele weitere Punktepaare finden, die durch einen ebensolchen Pfeil verbunden werden, und das Ganze funktioniert selbstverständlich analog auch im dreidimensionalen Fall. 

Jeder Pfeil in der obigen Abbildung gibt also den gleichen Informationsgehalt, nämlich $-3$ in $x$-Richtung und $1$ in $y$-Richtung, wieder. Was liegt also mathematisch näher, als jeden dieser Pfeile nur als Vertreter (einen sogenannten \textbf{Repräsentanten}) eines viel grundlegenderen Objekts aufzufassen? Dieses grundlegendere mathematische Objekt heißt \textbf{Vektor} und hat in diesem Fall die \textbf{Komponenten} $-3$ ($x$-Komponente) und $1$ ($y$-Komponente), welche als sogenanntes $2$-\textbf{Tupel} übereinander geschrieben werden:
\[
 \textrm{Vektor mit den Repräsentanten aus obiger Abbildung}\MBlank = \MVector{-3\\1}\MDFPeriod
\]
Die folgende Infobox fasst diese Erkenntnis und noch einige weitere Schreib- und Sprechweisen zu Vektoren zusammen.

\begin{MInfo}
Ein zwei- bzw. dreidimensionaler \MEntry{Vektor}{Vektor} ist ein $2$- bzw. $3$-\MEntry{Tupel}{Tupel} mit $2$ bzw. $3$ \MEntry{Komponenten}{Komponenten (von Vektoren)}, die als $x$-, $y$- (und $z$-)Komponenten bezeichnet werden. Als Variablen für Vektoren werden für gewöhnlich Kleinbuchstaben mit Pfeil benutzt. Als Variable für ihre Komponenten wird oft der gleiche Kleinbuchstabe mit der zugehörigen Koordinatenrichtung als Index zur Unterscheidung verwendet: 
\[
 \MVec{a}=\MVector{a_x\\a_y},\MDFPaSpace \MVec{b}=\MVector{b_x\\b_y\\b_z}\MDFPeriod
\]
Ein Pfeil in der Ebene oder im Raum heißt \MEntry{Repräsentant}{Repräsentant (eines Vektors)} des Vektors, wenn der Pfeil zwei Punkte in der Ebene oder im Raum verbindet, die so liegen, dass die Differenzen der Koordinaten des Punkts an der Spitze des Pfeils und des Punkts am Fußpunkt des Pfeils genau die Komponenten des Vektors ergeben.  
\end{MInfo}

Oft ist man in der Situation, dass man einen Punkt $P$ in der Ebene oder im Raum bzw. zwei Punkte $Q$ und $R$ in der Ebene oder im Raum gegeben hat, und man möchte nun den Vektor angeben, von welchem der Pfeil vom Ursprung $O$ zu dem gegebenen Punkt $P$ bzw. von welchem der Pfeil, der $Q$ mit $R$ verbindet, jeweils ein Repräsentant ist. Die dafür benutzte Schreib- und Sprechweise sowie die nötige Rechenoperation gibt die folgende Infobox wieder:

\begin{MInfo}\MLabel{VBKM10_Info_OrtsundVerbindungsvektor}
\begin{itemize}
 \item \textbf{Zweidimensionaler Fall:}\\
 Sind $P=\MPointTwo{p_x}{p_y}$ sowie $Q=\MPointTwo{q_x}{q_y}$ und $R=\MPointTwo{r_x}{r_y}$ Punkte in der Ebene, so heißt der Vektor
 \[
  \MDVec{Q R} := \MVector{r_x-q_x \\ r_y-q_y}
 \]
 \MEntry{Verbindungsvektor vom Punkt}{Verbindungsvektor} $Q$ \textbf{zum Punkt} $R$ und 
 \[
  \MDVec{P} := \MDVec{O P} = \MVector{p_x\\p_y}
 \]
 \MEntry{Ortsvektor des Punkts}{Ortsvektor} $P$. Dies sind genau diejenigen Vektoren, welche als Repräsentanten unter anderem die entsprechenden Verbindungspfeile der Punkte besitzen:
 \begin{center}
\MTikzAuto{
\begin{tikzpicture}[>=stealth]
%Koordinatensystem
\draw[->,color=black] (-1.5,0) -- (4.5,0);
% \foreach \x in {-3,-2,-1,1,2,3,4}
% \draw[shift={(\x,0)},color=black] (0pt,2pt) -- (0pt,-2pt) node[below] {\footnotesize $\x$};
\draw[->,color=black] (0,-1.5) -- (0,4.5);
\foreach \y in {-2,-1,1,2,3}
% \draw[shift={(0,\y)},color=black] (2pt,0pt) -- (-2pt,0pt) node[left] {\footnotesize $\y$};
% \draw[color=black] (-10pt,-8pt) node[right] {\footnotesize $0$};
%Achsenbeschriftung
\draw (4.5,0) node[anchor=north west] {$x$};
\draw (-0.5,4.8) node[anchor=north west] {$y$};
%Koordinaten
\draw[dashed, color=red] (0.5,1) -- (0.5,0);
\draw[dashed, color=red] (0.5,1) -- (0,1);
\draw[color=red] (0.5,-2pt) -- (0.5,2pt);
\draw[color=red] (0.5,0) node[anchor=north] {\scriptsize $p_x$};
\draw[color=red] (-2pt,1) -- (2pt,1);
\draw[color=red] (0,1) node[anchor=east] {\scriptsize $p_y$};

\draw[dashed, color=blue] (4,1.5) -- (4,0);
\draw[dashed, color=blue] (4,1.5) -- (0,1.5);
\draw[color=blue] (4,-2pt) -- (4,2pt);
\draw[color=blue] (4,0) node[anchor=north] {\scriptsize $q_x$};
\draw[color=blue] (-2pt,1.5) -- (2pt,1.5);
\draw[color=blue] (0,1.5) node[anchor=east] {\scriptsize $q_y$};

\draw[dashed, color=violet] (1,3.5) -- (0,3.5);
\draw[dashed, color=violet] (1,3.5) -- (1,0);
\draw[color=violet] (1,-2pt) -- (1,2pt);
\draw[color=violet] (1,0) node[anchor=north] {\scriptsize $r_x$};
\draw[color=violet] (-2pt,3.5) -- (2pt,3.5);
\draw[color=violet] (0,3.5) node[anchor=east] {\scriptsize $r_y$};
%Punkte
\draw [fill = black] (0,0) circle (1.5pt);
\draw [color=black] (0,0) node[anchor=south east] {\footnotesize $O=\MPointTwo{0}{0}$};
\draw [fill = red] (0.5,1) circle (1.5pt);
\draw [color=red] (0.5,1) node[anchor=south] {\footnotesize $P$};
\draw [fill = blue] (4,1.5) circle (1.5pt);
\draw [color=blue] (4,1.5) node[anchor=south west] {\footnotesize $Q$};
\draw [fill = violet] (1,3.5) circle (1.5pt);
\draw [color=violet] (1,3.5) node[anchor=south] {\footnotesize $R$};
%Pfeile
\draw[->, line width = 1.5pt] (0,0) -- (0.5,1);
\draw[->, line width = 1.5pt] (4,1.5) -- (1,3.5);
\end{tikzpicture}
} 
\end{center}

 \item \textbf{Dreidimensionaler Fall:}\\
 Sind $P=\MPointThree{p_x}{p_y}{p_z}$ sowie $Q=\MPointThree{q_x}{q_y}{q_z}$ und $R=\MPointThree{r_x}{r_y}{r_z}$ Punkte im Raum, so heißt der Vektor
 \[
  \MDVec{Q R} := \MVector{r_x-q_x \\ r_y-q_y \\ r_z-q_z}
 \]
 \MEntry{Verbindungsvektor vom Punkt}{Verbindungsvektor} $Q$ \textbf{zum Punkt} $R$ und 
 \[
  \MDVec{P} := \MDVec{O P} = \MVector{p_x\\p_y\\p_z}
 \]
 \MEntry{Ortsvektor des Punkts}{Ortsvektor} $P$. Dies sind genau diejenigen Vektoren, welche als Repräsentanten unter anderem die entsprechenden Verbindungspfeile der Punkte besitzen:
\begin{center}
\MTikzAuto{%
\begin{tikzpicture}[>=stealth]

% The axes
\draw[->] (xyz cs:x=-2.5) -- (xyz cs:x=3.5) node[above] {\footnotesize $x$};
\draw[->] (xyz cs:y=-2.5) -- (xyz cs:y=3.5) node[right] {\footnotesize $y$};
\draw[->] (xyz cs:z=-2.5) -- (xyz cs:z=3.5) node[left] {\footnotesize $z$};
% The ticks
% \foreach \coo in {-2,-1,1,2,3}
% {
%   \draw (\coo,-3pt) -- (\coo,3pt) node[below=4pt] {\footnotesize \coo};
%   \draw (-3pt,\coo) -- (3pt,\coo) node[left=4pt] {\footnotesize \coo};
%   \draw (xyz cs:y=-0.1pt,z=\coo) -- (xyz cs:y=0.1pt,z=\coo) node[below=3pt] {\scriptsize \coo};
% }
%Coordinates
\draw[dashed, color=red] (0.5,2,-1) -- (0.5,2,0);
\draw[dashed, color=red] (0.5,2,-1) -- (0.5,0,-1);
\draw[dashed, color=red] (0.5,2,-1) -- (0,2,-1);
\draw[dashed, color=red] (0.5,2,0) -- (0,2,0);
\draw[dashed, color=red] (0.5,0,-1) -- (0.5,0,0);
\draw[dashed, color=red] (0,2,-1) -- (0,2,0);
\draw[dashed, color=red] (0.5,0,-1) -- (0,0,-1);
\draw[color=red] (0,-0.06,-1) -- (0,0.06,-1);
\draw[color=red] (0,0,-1) node[anchor=north] {\scriptsize $p_z$};
\draw[color=red] (0.5,-2pt) -- (0.5,2pt);
\draw[color=red] (0.5,0,0) node[anchor=north] {\scriptsize $p_x$};
\draw[color=red] (-2pt,2) -- (2pt,2);
\draw[color=red] (0,2,0) node[anchor=east] {\scriptsize $p_y$};

\draw[dashed, color=violet] (1,-1,2) -- (0,-1,2);
\draw[dashed, color=violet] (1,-1,2)-- (1,0,2);
\draw[dashed, color=violet] (1,-1,2)-- (1,-1,0);
\draw[dashed, color=violet] (0,-1,2) -- (0,-1,0);
\draw[dashed, color=violet] (1,0,2) -- (1,0,0);
\draw[dashed, color=violet] (1,0,2) -- (0,0,2);
\draw[dashed, color=violet] (1,-1,0) -- (0,-1,0);
\draw[color=violet] (1,-2pt) -- (1,2pt);
\draw[color=violet] (1,0,0) node[anchor=south] {\scriptsize $r_x$};
\draw[color=violet] (-2pt,-1) -- (2pt,-1);
\draw[color=violet] (0,-1,0) node[anchor=east] {\scriptsize $r_y$};
\draw[color=violet] (0,-0.06,2) -- (0,0.06,2);
\draw[color=violet] (0,0,2) node[anchor=east] {\scriptsize $r_z$};

\draw[dashed, color=blue] (3,1,1) -- (0,1,1);
\draw[dashed, color=blue] (3,1,1) -- (3,0,1);
\draw[dashed, color=blue] (3,1,1) -- (3,1,0);
\draw[dashed, color=blue] (0,1,1) -- (0,1,0);
\draw[dashed, color=blue] (3,1,0) -- (0,1,0);
\draw[dashed, color=blue] (3,0,1) -- (0,0,1);
\draw[dashed, color=blue] (3,0,1) -- (3,0,0);
\draw[color=blue] (3,-2pt) -- (3,2pt);
\draw[color=blue] (3,0,0) node[anchor=south] {\scriptsize $q_x$};
\draw[color=blue] (-2pt,1) -- (2pt,1);
\draw[color=blue] (0,1,0) node[anchor=east] {\scriptsize $q_y$};
\draw[color=blue] (0,-0.06,1) -- (0,0.06,1);
\draw[color=blue] (0,0,1) node[anchor=east] {\scriptsize $q_z$};

% % Points:
\draw[fill=black] (0,0,0) circle (1.5pt);
\draw[color=black] (0,0,0) node[anchor=south east] {\footnotesize $O=\MPointThree{0}{0}{0}$};
\draw[fill=red] (0.5,2,-1) circle (1.5pt);
\draw[color=red] (0.5,2,-1) node[anchor=south] {\footnotesize $P$};
\draw[fill=violet] (1,-1,2) circle (1.5pt);
\draw[color=violet] (1,-1,2) node[anchor=north] {\footnotesize $R$};
\draw[fill=blue] (3,1,1) circle (1.5pt);
\draw[color=blue] (3,1,1) node[anchor=west] {\footnotesize $Q$};

%Arrows
\draw[->, line width = 1.5pt] (0,0,0) -- (0.5,2,-1);
\draw[->, line width = 1.5pt] (3,1,1) -- (1,-1,2);
\end{tikzpicture}
}
\end{center}

\end{itemize}
 
\end{MInfo}

\begin{MExample}
\begin{itemize}
 \item \textbf{Zweidimensionaler Fall:}\\
 Der Punkt $P=\MPointTwo{-1}{-2}$ besitzt den Ortsvektor 
 \[
  \MDVec{P} = \MVector{-1\\-2} \MDFPeriod
 \]
 Der Vektor 
 \[
  \MVec{v}=\MVector{2\\0}
 \]
 ist zum Beispiel Verbindungsvektor vom Punkt $A=\MPointTwo{1}{1}$ zum Punkt $B=\MPointTwo{3}{1}$, also gilt
 \[
  \MVec{v}=\MDVec{A B} \MDFPeriod 
 \]
 Allerdings ist $\MVec{v}\neq\MDVec{B A}$, denn 
 \[
  \MDVec{B A} = \MVector{1-3\\1-1} = \MVector{-2\\0}\neq\MVector{2\\0} \MDFPeriod
 \]
 \item \textbf{Dreidimensionaler Fall:}\\
 Zu den beiden Punkten $Q=\MPointThree{1}{1}{1}$ und $R=\MPointThree{-2}{0}{2}$ ist der Verbindungsvektor von $Q$ nach $R$:
 \[
  \MDVec{Q R}=\MVector{-2-1\\0-1\\2-1}= \MVector{-3\\-1\\1}\MDFPeriod
 \]
 Aber der Verbindungsvektor von $R$ nach $Q$ ist
 \[
  \MDVec{R Q}=\MVector{1-(-2)\\1-0\\1-2}=\MVector{3\\1\\-1} \MDFPeriod
 \]
 Der Vektor 
 \[
  \MVector{3\\1\\-1}
 \]
 ist natürlich gleichzeitig auch Ortsvektor des Punktes $\MPointThree{3}{1}{-1}$.
\end{itemize}
 
\end{MExample}

Das obige Beispiel zeigt bereits folgende interessante Tatsache: Dreht man die Orientierung eines Vektors (und damit die Orientierung aller Pfeile, die ihn repräsentieren) um, so erhält man einen Vektor, in dem die Komponenten das entgegengesetzte Vorzeichen besitzen. Man spricht hier auch vom sogenannten \textbf{Gegenvektor}. Dies ist ein erster Hinweis darauf, dass man mit Vektoren komponentenweise rechnen kann. Dies wird ausführlich im nachfolgenden Abschnitt \MNRef{VBKM10_Vektorrechnung} behandelt.  

Außerdem gibt es offenbar im zwei- und im dreidimensionalen Fall jeweils einen Vektor, dessen Komponenten alle gleich $0$ sind:
\[
 \MVector{0\\0}\MDFPaSpace\textrm{bzw.}\MDFPaSpace\MVector{0\\0\\0}\MDFPeriod
\]
Dieser Vektor wird jeweils als zwei- bzw. dreidimensionaler \textbf{Nullvektor} bezeichnet. Man kann sich vorstellen, dass seine Repräsentanten \glqq Pfeile der Länge $0$\grqq\ sind, also solche die einen Punkt mit sich selbst verbinden. Oder anders gesagt: Der Nullvektor ist der Ortsvektor des Ursprungs.

\begin{MExercise}
Gegeben sind die Punkte
\[
 A=\MPointTwo[\Big]{-1}{\frac{3}{2}}\MDFPaSpace\textrm{und}\MDFPaSpace B=\MPointTwo{\pi}{-2}
\]
in der Ebene, die Punkte
\[
 P=\MPointThree{\MZahl{0}{5}}{1}{-1}\MDFPaSpace\textrm{und}\MDFPaSpace Q=\MPointThree[\Big]{\frac{1}{2}}{-1}{1}
\]
im Raum sowie die zwei- bzw. dreidimensionalen Vektoren
\[
 \MVec{a}=\MVector{\pi\\-1}\MDFPaSpace\textrm{und}\MDFPaSpace \MVec{v}=\MVector{0\\3\\-3} \MDFPeriod
\]
\begin{itemize}
 \item Bestimmen Sie die folgenden Vektoren:\\
 \begin{MExerciseItems}
\item{$\MDVec{A B}=$\MLFunctionQuestion{15}{(pi+1,-(7/2))}{5}{x}{5}{VECI1}} 
\item{$\MDVec{B A}=$\MLFunctionQuestion{15}{(-1-pi,(7/2))}{5}{x}{5}{VECI2}} 
\item{$\MDVec{P Q}=$\MLFunctionQuestion{15}{(0,-2,2)}{5}{x}{5}{VECI3}} 
\item{$\MDVec{Q P}=$\MLFunctionQuestion{15}{(0,2,-2)}{5}{x}{5}{VECI4}} 
\end{MExerciseItems}
\MInputHint{Vektoren können in der Form \texttt{(a;b)} bzw. \texttt{(a;b;c)} eingegeben werden, zum Beispiel \texttt{(1;0)} für den Vektor $\MVector{1\\0}$. Die Zahl $\pi$ kann als \texttt{pi} eingegeben werden.} 

\item Bestimmen Sie Punkte $C$ in der Ebene und $R$ im Raum, so dass folgende Aussagen wahr sind:\\
\begin{MExerciseItems}
\item{$\MVec{a}=\MDVec{C B}\MDFPSpace\Leftrightarrow\MDFPSpace C=$\MLFunctionQuestion{15}{(0,-1)}{5}{x}{5}{POINTI1}} 
\item{$\MVec{v}=\MDVec{Q R}\MDFPSpace\Leftrightarrow\MDFPSpace R=$\MLFunctionQuestion{15}{(0.5,2,-2)}{5}{x}{5}{POINTI2}} 
\end{MExerciseItems}
\MInputHint{Auch Punkte können in der Form \texttt{(a;b)} bzw. \texttt{(a;b;c)} eingegeben werden.} 

\item Zeichnen Sie mindestens drei Repräsentanten des Vektors $a$. 
\end{itemize}


\begin{MHint}{\iSolution}
\begin{itemize}
 \item Die gesuchten Vektoren sind:\\
 \begin{MExerciseItems}
\item{$\MDVec{A B}=\MVector{\pi+1\\-\frac{7}{2}}$} 
\item{$\MDVec{B A}=\MVector{-1-\pi\\ \frac{7}{2}}$}
\item{$\MDVec{P Q}=\MVector{0\\-2\\2}$} 
\item{$\MDVec{Q P}=\MVector{0\\2\\-2}$} 
\end{MExerciseItems}
 \item Die gesuchten Punkte sind:\\
  \begin{MExerciseItems}
\item{$C=\MPointTwo{0}{-1}$} 
\item{$R=\MPointThree[\Big]{\frac{1}{2}}{2}{-2}$}
\end{MExerciseItems}
 \item Das folgende Bild zeigt drei mögliche Repräsentanten von $\MVec{a}=\MVector{\pi\\-1}$:
 
\begin{center}
\MTikzAuto{
\begin{tikzpicture}[>=stealth]
%Koordinatensystem
\draw[->,color=black] (-2.5,0) -- (6.5,0);
\foreach \x in {-2,-1,1,2,3,4,5,6}
\draw[shift={(\x,0)},color=black] (0pt,2pt) -- (0pt,-2pt) node[below] {\footnotesize $\x$};
\draw[->,color=black] (0,-1.5) -- (0,3.5);
\foreach \y in {-1,1,2,3}
\draw[shift={(0,\y)},color=black] (2pt,0pt) -- (-2pt,0pt) node[left] {\footnotesize $\y$};
\draw[color=black] (-10pt,-8pt) node[right] {\footnotesize $0$};
%Achsenbeschriftung
\draw (6.5,0) node[anchor=north west] {$x$};
\draw (-0.5,3.8) node[anchor=north west] {$y$};
%Punkte

%Pfeile
\draw[->, line width = 1.5pt] (0,0) -- (3.14,-1);
\draw[->, line width = 1.5pt] (-2,2) -- (1.14,1);
\draw[->, line width = 1.5pt] (2,3) -- (5.14,2);
\end{tikzpicture}
} 
\end{center}

\end{itemize}

\end{MHint}

\end{MExercise}

Die vorhergehende Einführung des Vektorkonzepts zeigt die nahe Verwandtschaft von Vektoren und Punkten. Tatsächlich gibt es
sogar eine Eins-zu-eins-Korrespondenz zwischen Punkten und Ortsvektoren: Zu jedem Punkt gibt es genau einen Vektor, der Ortsvektor
dieses Punktes ist, und umgekehrt gibt es zu jedem Vektor genau einen Punkt, für den dieser Vektor der zugehörige Ortsvektor ist.
Dies gilt im zwei- sowie im dreidimensionalen Fall. Das rechtfertigt nun die -- in den folgenden Abschnitten oft benutzte -- Konvention,
Punkte durch ihre Ortsvektoren zu beschreiben. So beschreibt man zum Beispiel einen Punkt $P=\MPointTwo{2}{1}$ statt durch seine
Koordinaten $\MPointTwo{2}{1}$ oft durch seinen zugehörigen Ortsvektor $\MDVec{P}=\MVector{2\\1}$. Dies bringt bei der Untersuchung
geometrischer Objekte wie Geraden oder Ebenen (insbesondere im dreidimensionalen Fall) dann auch gewisse Vorteile in der Beschreibung
mit sich (vgl. z.B.  Abschnitt \MNRef{VBKM10_GeradenEbene}).   

Weiterhin rechtfertigt die Eins-zu-eins-Korrespondenz zwischen Punkten und Ortsvektoren auch, die Abkürzungen $\R^2$ und $\R^3$ nicht nur für die Menge aller Punkte in der Ebene bzw. im Raum zu verwenden, sondern auch für die Menge aller zwei- bzw. dreidimensionaler Vektoren. Davon wird in den nächsten Abschnitten ebenfalls Gebrauch gemacht.

\end{MXContent}

\begin{MXContent}{Rechnen mit Vektoren}{Vektorrechnung}{STD}
\MLabel{VBKM10_Vektorrechnung}
\MDeclareSiteUXID{VBKM10_Vektorrechnung}
In diesem Abschnitt wird behandelt, wie man mit den -- im vorigen Abschnitt \MNRef{VBKM10_Vektoren} eingeführten -- Vektoren rechnen kann. Das Rechnen mit Vektoren hat mehrere Aspekte: Zunächst kann man mit Vektoren, die man als 2- oder 3-Tupel angibt, die Rechenoperationen Addition, Subtraktion und -- mit einer gewissen Einschränkung -- auch Multiplikation durchführen, indem man diese Operationen komponentenweise ausführt. Dann haben diese Rechenoperationen aber auch eine geometrische Bedeutung für die Pfeile, welche die Vektoren repräsentieren. Man kann die Rechenoperationen für Vektoren auch als geometrische Operationen mit ihren Repräsentanten interpretieren. Schließlich führt diese geometrische Betrachtung des Rechnens mit Vektoren auch auf ein tieferes Verständnis von Orts- und Verbindungsvektoren von Punkten.

Das Rechnen mit zwei- und dreidimensionalen Vektoren funktioniert im Wesentlichen auf die gleiche Art und Weise. Bei den Rechenoperationen in Komponentenschreibweise werden im Folgenden immer beide Fälle (zwei- und dreidimensional) aufgeführt. Bei den zugehörigen Bildern, welche die geometrische Bedeutung der Operationen für die Repräsentanten der Vektoren sowie für Orts- und Verbindungsvektoren veranschaulichen, werden in diesem Abschnitt nur Pfeile und Punkte, keine Koordinatensysteme dargestellt. So sind diese Bilder sowohl für den zwei- als auch für den dreidimensionalen Fall gültig. 

Da beim Rechnen mit Vektoren im Folgenden Vektoren durch Rechnung und Äquivalenzumformungen auseinander hervorgehen sollen, muss einmal festgehalten werden, unter welchen Umständen zwei Vektoren gleich sein sollen. 

\begin{MInfo}
Zwei Vektoren $\MVec{a},\MVec{b}\in\R^2\MBlank \textrm{oder}\MBlank \R^3$ sind genau dann \MEntry{gleich}{Gleichheit (von Vektoren)} (symbolisch: $\MVec{a}=\MVec{b}$), wenn eine (und damit alle) der folgenden äquivalenten Bedingungen zutrifft (zutreffen):
\begin{itemize}
 \item $\MVec{a}$ und $\MVec{b}$ besitzen die gleichen Komponenten, also 
 \[
  \MVec{a}=\MVec{b}\MDFPSpace\Leftrightarrow\MDFPSpace\MVector{a_x\\a_y}=\MVector{b_x\\b_y}\MDFPSpace\Leftrightarrow\MDFPSpace a_x=b_x\MBlank \textrm{und}\MBlank  a_y=b_y
 \]
 im zweidimensionalen Fall bzw.
 \[
  \MVec{a}=\MVec{b}\MDFPSpace\Leftrightarrow\MDFPSpace\MVector{a_x\\a_y\\a_z}=\MVector{b_x\\b_y\\b_z}\MDFPSpace\Leftrightarrow\MDFPSpace a_x=b_x\MBlank \textrm{und}\MBlank  a_y=b_y\MBlank \textrm{und}\MBlank  a_z=b_z
 \]
 im dreidimensionalen Fall. Dies ist auch unter der Bezeichnung \textbf{Koordinaten-} oder \textbf{Komponentenvergleich} bekannt.
 \item $\MVec{a}$ und $\MVec{b}$ besitzen einen gleichen Repräsentanten.
 \item $\MVec{a}$ und $\MVec{b}$ sind beide Ortsvektor des gleichen Punktes.
 \item $\MVec{a}$ und $\MVec{b}$ sind beide Verbindungsvektor der gleichen zwei Punkte.
\end{itemize}
\end{MInfo}

Diese Infobox macht auch klar, dass zwei Vektoren unterschiedlicher Dimension (also etwa $\MVec{a}\in\R^2$ und $\MVec{b}\in\R^3$) niemals gleich sein können. Diese Vektoren sind wegen der unterschiedlichen Anzahl an Komponenten sogar nicht einmal miteinander vergleichbar. Rechnungen mit Vektoren finden also immer nur mit einer festen Anzahl an Komponenten statt (hier entweder zwei oder drei), und die Ergebnisse der Rechnungen sind dann immer auch Vektoren mit ebendieser festen Anzahl an Komponenten.

\begin{MInfo}
Die \MEntry{Addition zweier Vektoren}{Addition (von Vektoren)} bedeutet die Addition aller ihrer Komponenten. Also
\[
 \MVec{a}+\MVec{b} = \MVector{a_x\\a_y}+\MVector{b_x\\b_y} = \MVector{a_x+b_x\\a_y+b_y}
\]
im zweidimensionalen Fall und
\[
 \MVec{a}+\MVec{b} = \MVector{a_x\\a_y\\a_z}+\MVector{b_x\\b_y\\b_z} = \MVector{a_x+b_x\\a_y+b_y\\a_z+b_z}
\]
im dreidimensionalen Fall. Geometrisch kann die Vektoraddition als \glqq Aneinanderhängen\grqq\ von Pfeilen oder als Ergänzung von zwei Pfeilen zu einem Parallelogramm interpretiert werden, je nachdem welche Repräsentanten der Vektoren man verwendet:
\begin{center}
\begin{tabular}{lr}
\MTikzAuto{%
\begin{tikzpicture}[>=stealth] 
%Pfeile
\draw[color=red,->] (0,0) -- (3,1);
\draw[color=red] (1.5,0.5) node[anchor=north] {\footnotesize $\MVec{a}$};
\draw[color=blue,->] (3,1) -- (4,3);
\draw[color=blue] (3.5,2) node[anchor=west] {\footnotesize $\MVec{b}$};
%Summe
\draw[color=violet,->] (0,0) -- (4,3);
\draw[color=violet] (1.9,1.5) node[anchor=east] {\footnotesize $\MVec{a}+\MVec{b}$};
\end{tikzpicture}       
}%
&
\MTikzAuto{%
\begin{tikzpicture}[>=stealth] 
%Pfeile
\draw[color=red,->] (0,0) -- (3,1);
\draw[color=red] (1.5,0.5) node[anchor=north] {\footnotesize $\MVec{a}$};
\draw[color=blue,->] (0,0) -- (1,2);
\draw[color=blue] (0.5,1) node[anchor=east] {\footnotesize $\MVec{b}$};
%Parallelogramm
\draw[color=red, dashed] (1,2) -- (4,3);
\draw[color=blue, dashed] (3,1) -- (4,3); 
%Summe
\draw[color=violet,->] (0,0) -- (4,3);
\draw[color=violet] (2.1,1.5) node[anchor=north] {\footnotesize $\MVec{a}+\MVec{b}$};
\end{tikzpicture}       
}%
\end{tabular}
\end{center}
\end{MInfo}

Auch bei der Addition von Vektoren gilt, analog zum Rechnen mit reellen Zahlen, das Kommutativ- und das Assoziativgesetz (vgl. Abschnitt \MNRef{VBKM01_TermeUmformen}) 
\[
 \MVec{a}+\MVec{b} = \MVec{b}+\MVec{a}
\]
sowie
\[
 \MVec{a}+\MVec{b}+\MVec{c} = (\MVec{a}+\MVec{b})+\MVec{c} = \MVec{a}+(\MVec{b}+\MVec{c}) \MDFPeriod
\]
Und der \MEntry{Nullvektor}{Nullvektor} $\MDVec{O}=\MVector{0\\0}$ bzw. $\MDVec{O}=\MVector{0\\0\\0}$ erfüllt die analoge Funktion für Vektoren, die die $0$ für reelle Zahlen erfüllt:
\[
 \MVec{a}+\MDVec{O} = \MDVec{O} + \MVec{a} = \MVec{a}\MDFPeriod
\]


\begin{MExample}
Gegeben sind die Vektoren $\MVec{v}=\MVector{2\\-1}$ und $\MVec{w}=\MVector{-1\\2}$. Dann gilt
\[
 \MVec{v}+\MVec{w} = \MVector{2\\-1} + \MVector{-1\\2} = \MVector{2-1\\-1+2} = \MVector{1\\1} \MDFPeriod
\]
Bild hierzu:
\begin{center}
\MTikzAuto{
\begin{tikzpicture}[>=stealth]
%Koordinatensystem
\draw[->,color=black] (-2.5,0) -- (3.5,0);
\foreach \x in {-2,-1,1,2,3}
\draw[shift={(\x,0)},color=black] (0pt,2pt) -- (0pt,-2pt) node[below] {\footnotesize $\x$};
\draw[->,color=black] (0,-2.5) -- (0,3.5);
\foreach \y in {-2,-1,1,2,3}
\draw[shift={(0,\y)},color=black] (2pt,0pt) -- (-2pt,0pt) node[left] {\footnotesize $\y$};
\draw[color=black] (-10pt,-8pt) node[right] {\footnotesize $0$};
%Achsenbeschriftung
\draw (3.5,0) node[anchor=north west] {$x$};
\draw (-0.5,3.8) node[anchor=north west] {$y$};
%Pfeile
\draw[->, color=red] (0,0) -- (2,-1);
\draw[color=red] (1,-0.5) node[anchor=north] {\footnotesize $\MVec{v}$};
\draw[dashed, color=red] (-1,2) -- (1,1);
\draw[->, color=blue] (0,0) -- (-1,2);
\draw[color=blue] (-0.5,1) node[anchor=east] {\footnotesize $\MVec{w}$};
\draw[dashed, color=blue] (2,-1) -- (1,1);
\draw[->, color=violet] (0,0) -- (1,1);
\draw[color=violet] (1,1) node[anchor=south west] {\footnotesize $\MVec{v}+\MVec{w}$};
\end{tikzpicture}
} 
\end{center}
Weiterhin ist $\MVec{w}$ der Verbindungsvektor vom Punkt $P=\MPointTwo{2}{0}$ zum Punkt $Q=\MPointTwo{1}{2}$, also $\MVec{w}=\MDVec{P Q}$ und $\MVec{v}$ ist der Verbindungsvektor vom Punkt $Q=\MPointTwo{1}{2}$ zum Punkt $R=\MPointTwo{3}{1}$, also $\MVec{v}=\MDVec{Q R}$. Dann stellt man fest, dass
\[
 \MVec{w}+\MVec{v} = \MDVec{P Q} + \MDVec{Q R} = \MDVec{P R}
\]
gilt:
\begin{center}
\MTikzAuto{
\begin{tikzpicture}[>=stealth]
%Koordinatensystem
\draw[->,color=black] (-1.5,0) -- (4.5,0);
\foreach \x in {-1,1,2,3,4}
\draw[shift={(\x,0)},color=black] (0pt,2pt) -- (0pt,-2pt) node[below] {\footnotesize $\x$};
\draw[->,color=black] (0,-1.5) -- (0,3.5);
\foreach \y in {-1,1,2,3}
\draw[shift={(0,\y)},color=black] (2pt,0pt) -- (-2pt,0pt) node[left] {\footnotesize $\y$};
\draw[color=black] (-10pt,-8pt) node[right] {\footnotesize $0$};
%Achsenbeschriftung
\draw (4.5,0) node[anchor=north west] {$x$};
\draw (-0.5,3.8) node[anchor=north west] {$y$};
%Punkte
\draw[fill=black] (2,0) circle (1.5pt);
\draw[color=black] (2,0) node[anchor=north west] {\footnotesize $P$};
\draw[fill=black] (1,2) circle (1.5pt);
\draw[color=black] (1,2) node[anchor=south] {\footnotesize $Q$};
\draw[fill=black] (3,1) circle (1.5pt);
\draw[color=black] (3,1) node[anchor=west] {\footnotesize $R$};
%Pfeile
\draw[->, color=blue] (2,0) -- (1,2);
\draw[color=blue] (1.5,1) node[anchor=east] {\footnotesize $\MVec{w}=\MDVec{P Q}$};
\draw[->, color=red] (1,2) -- (3,1);
\draw[color=red] (2,1.5) node[anchor=south west] {\footnotesize $\MVec{v}=\MDVec{Q R}$};
\draw[->, color=violet] (2,0) -- (3,1);
\draw[color=violet] (2.5,0.5) node[anchor=west] {\footnotesize $\MVec{w}+\MVec{v}= \MDVec{P R}$};
\end{tikzpicture}
} 
\end{center}
\end{MExample}
Das obige Beispiel zeigt, dass man mit Hilfe der Vektorrechnung auch Ausdrücke für Verbindungsvektoren von Punkten vereinfachen kann. Dies wird weiter unten bei der Subtraktion von Vektoren nochmals vertieft. 

\begin{MExercise}

\begin{MExerciseItems}
\item{Gegeben sind die Vektoren $\MVec{u}=\MVector{1\\0\\-8}$ und $\MVec{v}=\MVector{-3\\-4\\3}$. Berechnen Sie $\MVec{v}+\MVec{u}$.\\$\MVec{v}+\MVec{u}=$\MLFunctionQuestion{15}{(-2,-4,-5)}{5}{x}{5}{VEC21}} 
\item{Gegeben sind die Punkte $P$, $Q$ und $R$. Welche der folgenden Ausdrücke sind gleich dem Ausdruck $(\MDVec{P}+\MDVec{P Q})+\MDVec{Q R}$ ?\\
\begin{MQuestionGroup}
\begin{tabular}{lll}
\MLCheckbox{1}{VEC22} & \MBlank & $\MDVec{P}+(\MDVec{P Q}+\MDVec{Q R})$\\
\MLCheckbox{0}{VEC23} & \MBlank & $\MDVec{P R}$\\
\MLCheckbox{0}{VEC24} & \MBlank & $\MDVec{Q R}$\\
\MLCheckbox{1}{VEC25} & \MBlank & $\MDVec{O R}$\\
\MLCheckbox{1}{VEC26} & \MBlank & $\MDVec{R}$\\
\MLCheckbox{1}{VEC27} & \MBlank & $\MDVec{Q}+\MDVec{Q R}$\\
\end{tabular}
\end{MQuestionGroup}
\MGroupButton{Antworten kontrollieren}
}
\end{MExerciseItems}

\begin{MHint}{\iSolution}
\begin{MExerciseItems}
\item{\[\MVec{v}+\MVec{u}=\MVector{-3\\-4\\3}+\MVector{1\\0\\-8} = \MVector{-3+1\\-4+0\\3-8} =\MVector{-2\\-4\\-5} \MDFPeriod\]} 
\item{\[ (\MDVec{P}+\MDVec{P Q})+\MDVec{Q R} = \MDVec{P}+(\MDVec{P Q}+\MDVec{Q R})\MBlank \textrm{(Assoziativgesetz)}\] \[= (\MDVec{O P}+\MDVec{P Q})+\MDVec{Q R} = \MDVec{O Q}+\MDVec{Q R} = \MDVec{Q}+\MDVec{Q R} = \MDVec{O R} = \MDVec{R} \MDFPeriod\]}
\end{MExerciseItems} 
\end{MHint}
\end{MExercise}

Untersucht man mögliche Rechenoperationen für Vektoren nun weiter, so stellt man fest, dass die komponentenweise Multiplikation oder Division von Vektoren keine sinnvolle Operation ist. Dies genauer zu verstehen, würde aber den mathematischen Rahmen dieses Brückenkurses sprengen. So muss man an dieser Stelle einfach akzeptieren, dass man Vektoren nicht so einfach miteinander multiplizieren und schon gar nicht durcheinander dividieren kann. Was allerdings möglich ist, ist die Multiplikation von Vektoren mit reellen Zahlen und -- darauf aufbauend -- auch die Subtraktion von Vektoren. Wichtig ist noch Folgendes: Wird im Nachfolgenden von der Länge eines Vektors gesprochen, so ist damit die geometrische Länge der Pfeile gemeint, die ihn repräsentieren. Das Konzept der Länge oder des Betrags eines Vektors wird weiter unten noch genauer untersucht.

\begin{MInfo}
Die \MEntry{Multiplikation eines Vektors mit einer reellen Zahl}{Multiplikation (Vektor mit Skalar)} bedeutet die Multiplikation in jeder Komponente. Ist also $\MVec{a}$ ein Vektor und $s\in\R$, so gilt
\[
 s\cdot\MVec{a} = s\MVec{a}=s\MVector{a_x\\a_y} = \MVector{s a_x\\s a_y}
\]
im zweidimensionalen Fall und
\[
 s\cdot\MVec{a} = s\MVec{a}=s\MVector{a_x\\a_y\\a_z} = \MVector{s a_x\\s a_y\\s a_z}
\]
im dreidimensionalen Fall. Die Division eines Vektors durch eine Zahl $0\neq s\in\R$ ist dann einfach gegeben durch die Multiplikation mit dem Kehrwert $\frac{1}{s}$:
\[
 \frac{\MVec{a}}{s}=\frac{1}{s}\MVec{a}\MDFPeriod 
\]
Bei der Multiplikation eines Vektors mit einer reellen Zahl $s\in\R$ erhält man also einen gleich orientierten Vektor $s$-facher Länge, wenn $s>0$ ist. Wenn $s<0$ ist, hat der resultierende Vektor ebenfalls $s$-fache Länge, ist aber entgegengesetzt orientiert. Im Spezialfall $s=0$, gilt offenbar \[0\MVec{a}=\MDVec{O}\] für jeden Vektor $\MVec{a}$. Zwei weitere wichtige Fälle sind die Multiplikation mit $s=1$: \[1\cdot\MVec{a}=\MVec{a}\] -- dies lässt den Vektor offenbar gleich -- und die Multiplikation mit $s=-1$: \[-1\cdot\MVec{a}=-\MVec{a}\] -- dies ergibt den sogenannten \MEntry{Gegenvektor}{Gegenvektor}, ein Vektor gleicher Länge aber entgegengesetzter Orientierung. Bild hierzu:
\begin{center}
\MTikzAuto{%
\begin{tikzpicture}[>=stealth] 
%Pfeile
\draw[color=red, ->] (0,0) -- (2,1);
\draw[color=black, ->] (0.2,0) -- (3.2,1.5);
\draw[color=blue, ->] (0.4,0) -- (4.4,2);
\draw[color=violet, ->] (0,0) -- (-2,-1);
\draw[color=green, ->] (0.2,0) -- (-3.8,-2);
%Linie
\draw[line width = 1pt] (0,0) -- (0.4,0);
%Beschriftung
\draw[color=red] (2,1) node[anchor=south east] {\footnotesize $\MVec{a}=1\MVec{a}$};
\draw[color=black] (3.1,1.5) node[anchor= south east] {\footnotesize $\frac{3}{2}\MVec{a}$};
\draw[color=blue] (4.4,2) node[anchor=south east] {\footnotesize $2\MVec{a}$};
\draw[color=violet] (-2,-1) node[anchor=south east] {\footnotesize $-\MVec{a}=-1\cdot\MVec{a}$};
\draw[color=green] (-3.8,-2) node[anchor=south east] {\footnotesize $-2\MVec{a}$};
\end{tikzpicture}       
}% 
\end{center}
Da reelle Zahlen bei Multiplikation die Länge von Vektoren ändern, sie also \textit{skalieren}, nennt man reelle Zahlen in Bezug auf Vektoren oft auch \MEntry{Skalare}{Skalar} und spricht bei der Multiplikation einer reellen Zahl mit einem Vektor von \textbf{Skalarmultiplikation}. 
\end{MInfo}

\begin{MExample}
Gegeben ist der Vektor $\MVec{v}=\MVector{3\\\frac{3}{2}}$. Dann gilt zum Beispiel
\[
 2\MVec{v}=2\MVector{3\\\frac{3}{2}}=\MVector{2\cdot3\\2\cdot\frac{3}{2}}=\MVector{6\\3}
\]
und
\[
 -\frac{\MVec{v}}{3}=-\frac{1}{3}\MVec{v}=-\frac{1}{3}\MVector{3\\\frac{3}{2}}=\MVector{-1\\-\frac{1}{2}} \MDFPeriod
\]
Außerdem gilt zum Beispiel $\MVec{v}=\MDVec{P Q}$ für $P=\MPointTwo[\Big]{3}{\frac{1}{2}}$ und $Q=\MPointTwo{6}{2}$, da
\[
 \MVec{v}=\MVector{3\\\frac{3}{2}}=\MVector{6-3\\2-\frac{1}{2}} \MDFPeriod
\]
Dann ist aber
\[
 -\MVec{v} = -\MDVec{P Q} = \MDVec{Q P} \MDFPSpace,
\]
da 
\[
 -\MVec{v} = \MVector{-3\\-\frac{3}{2}}=\MVector{3-6\\\frac{1}{2}-2}
\]
gilt. Bild dazu:
\begin{center}
\MTikzAuto{
\begin{tikzpicture}[>=stealth]
%Koordinatensystem
\draw[->,color=black] (-1.5,0) -- (7.5,0);
\foreach \x in {-1,1,2,3,4,5,6,7}
\draw[shift={(\x,0)},color=black] (0pt,2pt) -- (0pt,-2pt) node[below] {\footnotesize $\x$};
\draw[->,color=black] (0,-1.5) -- (0,4.5);
\foreach \y in {-1,1,2,3,4}
\draw[shift={(0,\y)},color=black] (2pt,0pt) -- (-2pt,0pt) node[left] {\footnotesize $\y$};
\draw[color=black] (-10pt,-8pt) node[right] {\footnotesize $0$};
%Achsenbeschriftung
\draw (7.5,0) node[anchor=north west] {$x$};
\draw (-0.5,4.8) node[anchor=north west] {$y$};
%Punkte
\draw[fill=black] (3,0.5) circle (1.5pt);
\draw[color=black] (3,0.5) node[anchor=north west] {\footnotesize $P$};
\draw[fill=black] (6,2) circle (1.5pt);
\draw[color=black] (6,2) node[anchor=south] {\footnotesize $Q$};
%Pfeile
\draw[->, color=red, line width = 1.2pt] (0,0) -- (3,1.5);
\draw[color = red] (3,1.5) node[anchor=south east] {\footnotesize $\MVec{v}$};
\draw[->, color=violet] (0,0) -- (6,3);
\draw[color = violet] (6,3) node[anchor=south east] {\footnotesize $2\MVec{v}$};
\draw[->, color=blue] (0,0) -- (-1,-0.5);
\draw[color = blue] (-1,-0.5) node[anchor=north] {\footnotesize $-\frac{1}{3}\MVec{v}$};
%Verbindungspfeile
\draw[->, color=red, line width = 2pt] (3.04,0.506) -- (5.96,1.97);
\draw[color = red] (4.8,1.22) node[anchor=south east] {\footnotesize $\MVec{v}=\MDVec{P Q}$};
\draw[->, color=black] (5.96,1.97) -- (3.04,0.506);
\draw[color = black] (4.2,1.27) node[anchor=north west] {\footnotesize $-\MVec{v}=\MDVec{Q P}$};
\end{tikzpicture}
} 
\end{center}
\end{MExample}

Für das Rechnen mit Vielfachen von Vektoren gilt nun der folgende Satz von Rechengesetzen.

\begin{MInfo}
Sind $r$ und $s$ reelle Zahlen sowie $\MVec{a}$ und $\MVec{b}$ Vektoren, so gelten die folgenden Rechengesetze:
\begin{enumerate}
 \item $r\MVec{a}=\MVec{a}r$
 \item $r s\MVec{a}=(r s)\MVec{a}=r(s\MVec{a})$
 \item $(r+s)\MVec{a}=r\MVec{a}+s\MVec{a}$
 \item $r(\MVec{a}+\MVec{b})=r\MVec{a}+r\MVec{b}$
 \item $r(-\MVec{a})=(-r)\MVec{a}=-(r\MVec{a})$
 \item $r\MVec{a}=0\MDFPSpace\Leftrightarrow\MDFPSpace r=0$ oder $\MVec{a}=\MDVec{O}$
\end{enumerate}
Die Nummer 1 heißt auch Kommutativgesetz der Skalarmultiplikation, die Nummer 2 Assoziativgesetz der Skalarmultiplikation und die Nummern 3 und 4 Distributivgesetze der Skalarmultiplikation. 
\end{MInfo}


Mit Hilfe des Konzeptes des Gegenvektors, kann nun auch festgehalten werden, was die Subtraktion von Vektoren bedeutet.

\begin{MInfo}
Sind $\MVec{a}$ und $\MVec{b}$ Vektoren, so ist ihre \MEntry{Differenz}{Subtraktion (von Vektoren)} $\MVec{a}-\MVec{b}$ gegeben als die Summe von $\MVec{a}$ und dem Gegenvektor von $\MVec{b}$. Also gilt
\[
 \MVec{a}-\MVec{b}=\MVec{a}+(-\MVec{b})=\MVector{a_x\\a_y}+\MVector{-b_x\\-b_y}=\MVector{a_x-b_x\\a_y-b_y}
\]
im zweidimensionalen Fall und
\[
 \MVec{a}-\MVec{b}=\MVec{a}+(-\MVec{b})=\MVector{a_x\\a_y\\a_z}+\MVector{-b_x\\-b_y\\-b_z}=\MVector{a_x-b_x\\a_y-b_y\\a_z-b_z}
\]
im dreidimensionalen Fall. Damit kann die Differenz von Vektoren auch geometrisch mit Hilfe ihrer Repräsentanten interpretiert werden:
\begin{center}
\MTikzAuto{%
\begin{tikzpicture}[>=stealth] 
%Hilfslinien
\draw[color=black, dotted] (-1,-2) -- (1,-2);
\draw[color=black, dotted] (2,0) -- (1,-2);
%Pfeile
\draw[color=blue, ->] (0,0) -- (2,0);
\draw[color=red, ->] (0,0) -- (1,2);
\draw[color=red, ->, dashed] (0,0) -- (-1,-2);
\draw[color=violet, ->] (0,0) -- (1,-2);
%Ursprung
\draw[fill=black] (0,0) circle (1.5pt);
%Beschriftung
\draw[color=blue] (2,0) node[anchor=south] {\footnotesize $\MVec{a}$};
\draw[color=red] (1,2) node[anchor= south] {\footnotesize $\MVec{b}$};
\draw[color=red] (-1,-2) node[anchor=north] {\footnotesize $-\MVec{b}$};
\draw[color=violet] (0.9,-1.9) node[anchor=north west] {\footnotesize $\MVec{a}+(-\MVec{b})=\MVec{a}-\MVec{b}$};
\end{tikzpicture}       
}% 
\end{center}
\end{MInfo}

Betrachtet man in dieser Infobox nur die Differenz von Vektoren anhand ihrer Komponenten, so kann man sich fragen, wozu hierfür überhaupt das Konzept des Gegenvektors notwendig ist. Tatsächlich könnte man die Differenz von Vektoren in Komponenten auch ohne die Idee des Gegenvektors in Analogie zur Summe hinschreiben. Überlegt man sich dann aber die geometrische Interpretation der Differenz mit Hilfe von Repräsentanten (vgl. die Abbildung in der Infobox), so sieht man, dass dies nur mit Hilfe des Gegenvektors möglich ist.   

\begin{MExample}
Es werden in diesem Beispiel einige typische Aufgabenstellungen betrachtet, welche die bisher behandelten Rechengesetze für Vektoren beinhalten.
\begin{enumerate}
 \item Es sind die folgenden Vektorausdrücke zu vereinfachen:
 \begin{itemize}
  \item[(i)] $\MVector{1\\-2\\\MZahl{0}{5}}-\frac{1}{2}\MVector{2\\-3\\1}$,
  \item[(ii)] $2(\MVec{v}-\MVec{w})+3r\MVec{w}-r\cdot(-2\MVec{v})$ für $r\in\R$.
  \end{itemize}
  Anwenden der Rechengesetze ergibt:
  \begin{itemize}
   \item[(i)]
   \[
    \MVector{1\\-2\\\MZahl{0}{5}}-\frac{1}{2}\MVector{2\\-3\\1}=\MVector{1\\-2\\\MZahl{0}{5}}-\MVector{1\\-\frac{3}{2}\\\frac{1}{2}}=\MVector{1-1\\-2-(-\frac{3}{2})\\\MZahl{0}{5}-\frac{1}{2}}
    =\MVector{0\\-\frac{1}{2}\\0} \MDFPeriod
   \]
   \item[(ii)]
   \[
    2(\MVec{v}-\MVec{w})+3r\MVec{w}-r\cdot(-2\MVec{v})=2\MVec{v}-2\MVec{w}+3r\MVec{w}+2r\MVec{v}=2(r+1)\MVec{v}+(3r-2)\MVec{w} \MDFPeriod
   \]
   \end{itemize}
 \item Für $\MVec{a}=\MVector{1\\2}$ und $\MVec{b}=\MVector{-8\\3}$ ist der unbekannte Vektor $\MVec{x}$ in der Gleichung
 \[
  \MVec{a}-2\MVec{b}-(3\MVec{a}+\MVec{x})=\MVector{0\\-1}
 \]
 zu bestimmen.
 
 Auflösen nach $\MVec{x}$ sowie Einsetzen von $\MVec{a}$ und $\MVec{b}$ ergibt:
 \[
  -(3\MVec{a}+\MVec{x})=\MVector{0\\-1}-\MVec{a}+2\MVec{b}\MDFPaSpace\Leftrightarrow\MDFPaSpace-\MVec{x}=\MVector{0\\-1}-\MVec{a}+2\MVec{b}+3\MVec{a}=\MVector{0\\-1}+2(\MVec{a}+\MVec{b})
 \]
 \[
  \Leftrightarrow\MDFPaSpace\MVec{x}=-\MVector{0\\-1}-2\left(\MVector{1\\2}+\MVector{-8\\3}\right)=\MVector{0\\1}-2\MVector{-7\\5}\MDFPaSpace\Leftrightarrow\MDFPaSpace\MVec{x}=\MVector{14\\-9} \MDFPeriod
 \]
 \item Unter Benutzung der Differenz von Vektoren soll der Verbindungsvektor $\MDVec{P Q}$ zweier Punkte $P$ und $Q$ mit Hilfe der Ortsvektoren $\MDVec{P}$ und $\MDVec{Q}$ ausgedrückt werden. Es gilt:
 \[
  \MDVec{Q}-\MDVec{P} = \MDVec{O Q} - \MDVec{O P} = -\MDVec{Q O} - \MDVec{O P} = -(\MDVec{Q O} + \MDVec{O P}) = -\MDVec{Q P} = \MDVec{P Q}\MDFPeriod
 \]
 Der Verbindungsvektor $\MDVec{P Q}$ von einem Punkt $P$ zu einem Punkt $Q$ ergibt sich also immer als die Differenz des Ortsvektor $\MDVec{Q}$ (zum Endpunkt des Verbindungsvektors) und des Ortsvektors $\MDVec{P}$ (zum Anfangspunkt des Verbindungsvektors). Dies zeigt auch nochmals das folgende Bild und ein Vergleich mit der Rechenregel für Verbindungsvektoren aus \MNRef{VBKM10_Info_OrtsundVerbindungsvektor}:
\begin{center}
\MTikzAuto{%
\begin{tikzpicture}[>=stealth] 
%Punkte
\draw[fill=black] (0,0) circle (1.5pt);
\draw[color=red,fill=red] (2,0) circle (1.5pt);
\draw[color=blue,fill=blue] (1,2) circle (1.5pt);
%Pfeile
\draw[color=blue, ->] (0,0) -- (1,2);
\draw[color=red, ->] (0,0) -- (2,0);
\draw[color=violet, ->] (2,0) -- (1,2);
%Beschriftung
\draw[color=blue] (1,2) node[anchor=south] {\footnotesize $Q$};
\draw[color=red] (2,0) node[anchor= north] {\footnotesize $P$};
\draw[color=black] (0,0) node[anchor=north] {\footnotesize $O$};

\draw[color=blue] (0.5,1) node[anchor=east] {\footnotesize $\MDVec{Q}$};
\draw[color=red] (1,0) node[anchor= north] {\footnotesize $\MDVec{P}$};
\draw[color=violet] (1.4,0.8) node[anchor=south west] {\footnotesize $\MDVec{P Q}=\MDVec{Q}-\MDVec{P}$};
\end{tikzpicture}       
}% 
\end{center}
\item Die Punkte $A=\MPointTwo{2}{1}$, $B=\MPointTwo{4}{2}$ und $C=\MPointTwo{3}{3}$ bilden die Ecken eines Dreiecks. Der (geometrische) Schwerpunkt $S$ dieses Dreiecks kann mit Hilfe der zugehörigen Ortsvektoren berechnet werden:
\[
 \MDVec{S}=\frac{1}{3}(\MDVec{A}+\MDVec{B}+\MDVec{C}) = \frac{1}{3}\left(\MVector{2\\1}+\MVector{4\\2}+\MVector{3\\3}\right)=\frac{1}{3}\MVector{9\\6}=\MVector{3\\2}\MDFPeriod
\]
Somit ist $S=\MPointTwo{3}{2}$. Bild hierzu:
\begin{center}
\MTikzAuto{
\begin{tikzpicture}[>=stealth]
%Koordinatensystem
\draw[->,color=black] (-0.5,0) -- (5.5,0);
\foreach \x in {1,2,3,4,5}
\draw[shift={(\x,0)},color=black] (0pt,2pt) -- (0pt,-2pt) node[below] {\footnotesize $\x$};
\draw[->,color=black] (0,-0.5) -- (0,4.5);
\foreach \y in {1,2,3,4}
\draw[shift={(0,\y)},color=black] (2pt,0pt) -- (-2pt,0pt) node[left] {\footnotesize $\y$};
\draw[color=black] (-10pt,-8pt) node[right] {\footnotesize $0$};
%Achsenbeschriftung
\draw (5.5,0) node[anchor=north west] {$x$};
\draw (-0.5,4.8) node[anchor=north west] {$y$};
%Punkte
\draw[fill=black] (2,1) circle (1.5pt);
\draw[color=black] (2,1) node[anchor=north] {\footnotesize $A$};
\draw[fill=black] (4,2) circle (1.5pt);
\draw[color=black] (4,2) node[anchor=west] {\footnotesize $B$};
\draw[fill=black] (3,3) circle (1.5pt);
\draw[color=black] (3,3) node[anchor=south] {\footnotesize $C$};
\draw[color=red,fill=red] (3,2) circle (1.5pt);
\draw[color=red] (3,2) node[anchor=south]  {\footnotesize $S$};
%Dreiecks
\draw[color=black] (2,1) -- (4,2);
\draw[color=black] (4,2) -- (3,3);
\draw[color=black] (3,3) -- (2,1);
\end{tikzpicture}
} 
\end{center}
\end{enumerate}
 
\end{MExample}

\begin{MExercise}
\begin{MExerciseItems}
\item{Es sind $P$, $Q$, $R$ und $S$ Punkte im Raum. Vereinfachen Sie den Ausdruck
\[
 \MDVec{P Q}-(\MDVec{P Q}-\MDVec{Q R})+\MDVec{R S}
\]
so weit wie möglich.} 
\item{Zeigen Sie, dass die Punkte $A=\MPointTwo{1}{2}$, $B=\MPointTwo{4}{3}$ und $C=\MPointTwo{3}{1}$ zusammen mit dem Ursprung die Ecken eines Parallelogramms bilden.}
\end{MExerciseItems}



\begin{MHint}{\iSolution}
\begin{MExerciseItems}
\item{
\[
 \MDVec{P Q}-(\MDVec{P Q}-\MDVec{Q R})+\MDVec{R S}=\MDVec{P Q}-\MDVec{P Q}+\MDVec{Q R}+\MDVec{R S}=\MVec{O}+\MDVec{Q R}+\MDVec{R S}=\MDVec{Q S} \MDFPeriod
\]}
\item{Nach der geometrischen Interpretation der Vektoraddition, ergibt sich ein Parallelogramm, wenn einer der Ortsvektoren $\MDVec{A}$, $\MDVec{B}$ oder $\MDVec{C}$ die Summe der jeweils anderen beiden Ortsvektoren ist. Da 
\[
 \MDVec{A} + \MDVec{C} = \MVector{1\\2}+\MVector{3\\1} = \MVector{4\\3} = \MDVec{B}
\]
gilt, bilden die drei Punkte zusammen mit dem Ursprung ein Parallelogramm:
\begin{center}
\MTikzAuto{
\begin{tikzpicture}[>=stealth]
%Koordinatensystem
\draw[->,color=black] (-0.5,0) -- (5.5,0);
\foreach \x in {1,2,3,4,5}
\draw[shift={(\x,0)},color=black] (0pt,2pt) -- (0pt,-2pt) node[below] {\footnotesize $\x$};
\draw[->,color=black] (0,-0.5) -- (0,4.5);
\foreach \y in {1,2,3,4}
\draw[shift={(0,\y)},color=black] (2pt,0pt) -- (-2pt,0pt) node[left] {\footnotesize $\y$};
\draw[color=black] (-10pt,-8pt) node[right] {\footnotesize $0$};
%Achsenbeschriftung
\draw (5.5,0) node[anchor=north west] {$x$};
\draw (-0.5,4.8) node[anchor=north west] {$y$};
%Hilfslinien
\draw[color=red, dashed] (3,1) -- (4,3);
\draw[color=blue, dashed] (1,2) -- (4,3);
%Punkte
\draw[fill=black] (0,0) circle (1.5pt);
\draw[color=black] (0,0) node[anchor=south east] {\footnotesize $O$};
\draw[fill=red,color=red] (1,2) circle (1.5pt);
\draw[color=red] (1,2) node[anchor=south] {\footnotesize $A$};
\draw[fill=blue,color=blue] (3,1) circle (1.5pt);
\draw[color=blue] (3,1) node[anchor=north] {\footnotesize $C$};
\draw[color=violet,fill=violet] (4,3) circle (1.5pt);
\draw[color=violet] (4,3) node[anchor=south]  {\footnotesize $B$};
%Pfeile
\draw[color=red,->] (0,0) -- (1,2);
\draw[color=blue,->] (0,0) -- (3,1);
\draw[color=violet,->] (0,0) -- (4,3);
\end{tikzpicture}
} 
\end{center}
}
\end{MExerciseItems}
\end{MHint}

\end{MExercise}

\begin{MExercise}
Vereinfachen Sie jeweils so weit wie möglich:
\begin{MExerciseItems}
\item{$2\MVector{-1\\4\\2}-3\MVector{1\\6\\-2}=$\MLFunctionQuestion{15}{(-5,-10,10)}{5}{x}{5}{SVEC1}.} 
\item{$-2\MVector{-t\\3}-\left(\MVector{-1\\0}+\frac{t}{2}\MVector{4\\-42}\right)=$\MLFunctionQuestion{15}{(1,21*t-6)}{5}{t}{5}{SVEC2}.}
\end{MExerciseItems}
\MInputHint{Vektoren können in der Form \texttt{(a;b)} bzw. \texttt{(a;b;c)} eingegeben werden.}
\end{MExercise}

\begin{MHint}{\iSolution}
\begin{MExerciseItems}
\item{
\[
 2\MVector{-1\\4\\2}-3\MVector{1\\6\\-2}=\MVector{-2\\8\\4}-\MVector{3\\18\\-6}=\MVector{-5\\-10\\10} \MDFPeriod
\]
} 
\item{
\[
 -2\MVector{-t\\3}-\left(\MVector{-1\\0}+\frac{t}{2}\MVector{4\\-42}\right)=\MVector{2t\\-6}+\MVector{1\\0}-\MVector{2t\\-21t}=\MVector{1\\21t-6} \MDFPeriod
\]

}
\end{MExerciseItems}
 
\end{MHint}

\begin{MExercise}
Bestimmen Sie den Vektor $\MVec{y}$ in der Gleichung
\[
 3\left(\MVector{1\\-1\\1}-\MVec{y}\right)=-8\MVector{\MZahl{0}{25}\\\MZahl{0}{25}\\-\MZahl{0}{25}}+\MVec{y} \MDFPeriod
\]
$\MVec{y}=$\MLFunctionQuestion{15}{(5/4,-1/4,1/4)}{5}{x}{5}{GVEC1}
\end{MExercise}

\begin{MHint}{\iSolution}
\[
 3\left(\MVector{1\\-1\\1}-\MVec{y}\right)=-8\MVector{\MZahl{0}{25}\\\MZahl{0}{25}\\-\MZahl{0}{25}}+\MVec{y}\MDFPaSpace\Leftrightarrow\MDFPaSpace
 \MVector{3\\-3\\3}+2\MVector{1\\1\\-1}=4\MVec{y}
\]
\[
 \MDFPaSpace\Leftrightarrow\MDFPaSpace4\MVec{y}=\MVector{5\\-1\\1}\MDFPaSpace\Leftrightarrow\MDFPaSpace
 \MVec{y}=\MVector{\frac{5}{4}\\-\frac{1}{4}\\\frac{1}{4}} \MDFPeriod
\]
 
\end{MHint}

Da Vektoren durch beliebig viele Pfeile repräsentiert werden, die alle durch Parallelverschiebung auseinander hervorgehen, haben alle diese Repräsentanten die gleiche geometrische Länge (nämlich immer den Abstand der beiden Punkte, die sie verbinden). Deshalb ist es auch sinnvoll von der Länge eines Vektors zu sprechen. Die Länge eines Vektors trägt in der Mathematik die Bezeichnung \textbf{Betrag} oder \textbf{Norm}.

\begin{MInfo}
Der \MEntry{Betrag}{Betrag (eines Vektors)} oder die \MEntry{Norm}{Norm} eines Vektors $\MVec{a}$ wird als $|\MVec{a}|$ geschrieben und ist der Abstand desjenigen Punktes $P$ vom Ursprung $O$, zu dem der Vektor $\MVec{a}$ der zugehörige Ortsvektor ist (also $\MDVec{P}=\MVec{a}$). Also gilt
\[
 |\MVec{a}|=[\overline{O P}]
\]
und folglich im zweidimensionalen Fall
\[
 |\MVec{a}|=\left|\MVector{a_x\\a_y}\right|=\sqrt{a_x^2+a_y^2}
\]
sowie im dreidimensionalen Fall
\[
 |\MVec{a}|=\left|\MVector{a_x\\a_y\\a_z}\right|=\sqrt{a_x^2+a_y^2+a_z^2} \MDFPeriod
\]
Ein Vektor mit Betrag $1$ wird auch \MEntry{Einheitsvektor}{Einheitsvektor} genannt.
\end{MInfo}
Durch einen Vergleich mit der Infobox \MNRef{VBKM09_Abstand} zum Abstand zweier Punkte in einem zweidimensionalen Koordinatensystem in Kapitel \MNRef{VBKM09}, ist die Formel für den zweidimensionalen Fall unmittelbar einsichtig:
\begin{center}
\MTikzAuto{
\begin{tikzpicture}[>=stealth]
%Koordinatensystem
\draw[->,color=black] (-0.5,0) -- (4.5,0);
\draw[->,color=black] (0,-0.5) -- (0,3.5);
\draw[color=black] (-12pt,-6pt) node[right] {\footnotesize $O$};
%Achsenbeschriftung
\draw (4.5,0) node[anchor=north west] {$x$};
\draw (-0.5,3.8) node[anchor=north west] {$y$};
%Hilfslinien
\draw[color=red, dashed] (3,2) -- (3,-0.1);
\draw[color=red, dashed] (3,2) -- (-0.1,2);
\draw[color=red] (3,0) -- (3,-0.1);
\draw[color=black] (3,0.2) -- (2.8,0.2);
\draw[color=black] (2.8,0.2) -- (2.8,0);
\draw[color=black,fill=black] (2.9,0.1) circle (0.5pt);
\draw[color=red] (3,-0.1) node[anchor=north] {\footnotesize $a_x$};
\draw[color=red] (-0.1,2) node[anchor=east] {\footnotesize $a_y$};
%Punkte
\draw[fill=black] (0,0) circle (1.5pt);
\draw[fill=red,color=red] (3,2) circle (1.5pt);
\draw[color=red] (3,2) node[anchor=south] {\footnotesize $P$};
%Pfeil
\draw[color=black,->] (0,0) -- (3,2);
\draw[color=black] (2.1,1.3) node[anchor=south east] {\footnotesize $\MVec{a}=\MDVec{P}$};
\draw[color=black] (1.5,1) node[anchor=north] {\footnotesize $|\MVec{a}|$};
\end{tikzpicture}
} 
\end{center}
Es handelt sich wieder um eine einfache Anwendung des \MSRef{VBKM05_Pythagoras}{Satzes des Pythagoras}. Im dreidimensionalen Fall ist die Lage nicht wesentlich komplizierter. Auch hier hilft der Satz des Pythagoras weiter:
\begin{center}
\MTikzAuto{%
\begin{tikzpicture}[>=stealth]

% The axes
\draw[->] (xyz cs:x=-2.5) -- (xyz cs:x=3.5) node[above] {\footnotesize $x$};
\draw[->] (xyz cs:y=-1.5) -- (xyz cs:y=2.5) node[right] {\footnotesize $y$};
\draw[->] (xyz cs:z=-3) -- (xyz cs:z=2.5) node[left] {\footnotesize $z$};
% The ticks
% \foreach \coo in {-2,-1,1,2,3}
% {
%   \draw (\coo,-3pt) -- (\coo,3pt) node[below=4pt] {\footnotesize \coo};
%   \draw (-3pt,\coo) -- (3pt,\coo) node[left=4pt] {\footnotesize \coo};
%   \draw (xyz cs:y=-0.1pt,z=\coo) -- (xyz cs:y=0.1pt,z=\coo) node[below=3pt] {\scriptsize \coo};
% }
%Coordinates
\draw[dashed, color=red] (3,1,1) -- (0,1,1);
\draw[color=red] (3,1,1) -- (3,0,1);
\draw[dashed, color=red] (3,1,1) -- (3,1,0);
\draw[dashed, color=red] (0,1,1) -- (0,1,0);
\draw[dashed, color=red] (3,1,0) -- (0,1,0);
\draw[color=red] (3,0,1) -- (0,0,1);
\draw[color=red] (3,0,1) -- (3,0,0);
\draw[dashed, color=red] (0,1,1) -- (0,0,1);
\draw[dashed, color=red] (3,1,0) -- (3,0,0);

%Labels
\draw[color=red] (1.5,0,1) node[below=0.2pt] {\footnotesize $a_x$};
\draw[color=red] (3,0,0.6) node[right=0.2pt] {\footnotesize $a_z$};
\draw[color=red] (3.1,0.6,1) node[left=0.2pt] {\footnotesize $a_y$};

% % Points:
\draw[fill=black] (0,0,0) circle (1.5pt);
\draw[color=black] (0.1,0,0) node[anchor=south east] {\footnotesize $O$};
\draw[fill=red] (3,1,1) circle (1.5pt);
\draw[color=black] (3,1,1) node[anchor=west] {\footnotesize $\MDVec{P}=\MVec{a}$};

%Arrows and lines
\draw[->, line width = 1.2pt] (0,0,0) -- (3,1,1);
\draw[color=blue, line width=1.2pt] (0,0,0) -- (3,0,1);
\draw[color=black] (0,0,0.8) -- (0.2,0,0.8);
\draw[color=black] (0.2,0,0.8) -- (0.2,0,1);
\draw[fill=black] (0.1,0,0.9) circle (0.3pt);
\draw[color=blue] (3,0.2,1) -- (2.81,0.2,0.94);
\draw[color=blue] (2.81,0.2,0.94) -- (2.81,0,0.94);
\draw[fill=blue] (2.9,0.1,0.97) circle (0.5pt);
\end{tikzpicture}
}
\end{center}
In dieser Abbildung sind zwei rechtwinklige Dreiecke zu sehen. Das rechtwinklige Dreieck in der $xz$-Ebene ergibt eine Länge von $\sqrt{a_x^2+a_z^2}$ für die blaue Linie. Das zweite rechtwinklige Dreieck gibt dann für die Streckenlänge von $O$ nach $P$, also für $|\MVec{a}|$:
\[
 \sqrt{a_y^2+\left(\sqrt{a_x^2+a_z^2}\right)^2}=\sqrt{a_x^2+a_y^2+a_z^2}\MDFPeriod
\]

Für Normen von Vektoren gilt der folgende Satz von Rechenregeln.

\begin{MInfo}
Sind $\MVec{a}$ und $\MVec{b}$ Vektoren (beide aus dem $\R^2$ oder beide aus dem $\R^3$) und ist $r\in\R$, so gilt
\begin{enumerate}
 \item $|\MVec{a}|\geq 0$ und $|\MVec{a}|=0\MDFPSpace\Leftrightarrow\MDFPSpace\MVec{a}=\MDVec{O}$,
 \item $|r\MVec{a}|=|r|\cdot|\MVec{a}|$ und
 \item $|\MVec{a}+\MVec{b}|\leq |\MVec{a}|+|\MVec{b}|$.
\end{enumerate}
Nummer 1 stellt fest, dass Normen immer nicht-negativ sind und dass nur der Nullvektor die Norm $0$ hat. Nummer 2 ist besonders hilfreich für das Berechnen von Normen von Vielfachen von Vektoren. Nummer 3 heißt \MEntry{Dreiecksungleichung}{Dreiecksungleichung}.
\end{MInfo}

\begin{MExample}
\begin{itemize}
 \item Der Betrag des Vektors $\frac{1}{4}\MVector{3\\-1\\\sqrt{6}}$ berechnet sich zu
 \[
  \left|\frac{1}{4}\MVector{3\\-1\\\sqrt{6}}\right|=\left|\frac{1}{4}\right|\left|\MVector{3\\-1\\\sqrt{6}}\right|=\frac{1}{4}\sqrt{3^2+(-1)^2+6}=\frac{\sqrt{16}}{4}=1 \MDFPeriod
 \]
 Es handelt sich also um einen Einheitsvektor.
 \item Es ist eine Zahl $q\in\R$ zu finden, so dass $\left|\MVector{q^2-2\\4}-2\MVector{q-1\\q}\right|=0$ gilt.
 \[
  \left|\MVector{q^2-2\\4}-2\MVector{q-1\\q}\right|=\left|\MVector{q^2-2q\\4-2q}\right|=0\MDFPaSpace\Leftrightarrow\MDFPaSpace\MVector{q^2-2q\\4-2q}=\MVector{0\\0}
 \]
 \[
  \Leftrightarrow\MDFPaSpace q^2-2q=0\MDFPSpace\textrm{und}\MDFPSpace4-2q=0\MDFPaSpace\Leftrightarrow\MDFPaSpace q(q-2)=0 \MDFPSpace\textrm{und}\MDFPSpace 2(2-q)=0
  \MDFPaSpace\Leftrightarrow\MDFPaSpace q=2 \MDFPeriod
 \]

\end{itemize}
 
\end{MExample}

\begin{MExercise}
Berechnen Sie\\
$\left|-\frac{1}{3}\MVector{2\\-14\\5}\right|=$\MLFunctionQuestion{15}{5}{5}{x}{5}{NORM1}.
\begin{MHint}{\iSolution}
\[
 \left|-\frac{1}{3}\MVector{2\\-14\\5}\right|=\left|-\frac{1}{3}\right|\cdot\left|\MVector{2\\-14\\5}\right|=\frac{1}{3}\sqrt{2^2+(-14)^2+5^2}=\frac{1}{3}\sqrt{225}=\frac{15}{3}=5 \MDFPeriod
\]
 
\end{MHint}

\end{MExercise}


\begin{MExercise}
Finden Sie die Zahl $\chi>3$, für die $\left|\MVector{3\\\chi}-\MVector{\chi\\3}\right|=2\sqrt{2}$ gilt:\\
$\chi=$\MLFunctionQuestion{15}{5}{5}{x}{5}{ZAHL1}.
\end{MExercise}

\begin{MHint}{\iSolution}
\[
\left|\MVector{3\\\chi}-\MVector{\chi\\3}\right| =\left|\MVector{3-\chi\\\chi-3}\right|=\sqrt{(3-\chi)^2+(\chi-3)^2}=2\sqrt{2}\MDFPaSpace\Leftrightarrow\MDFPaSpace\sqrt{2(3-\chi)^2}=2\sqrt{2}
\]
\[
 \Leftrightarrow\MDFPaSpace|3-\chi|=2\MDFPaSpace\Leftrightarrow\MDFPaSpace\chi=1\MDFPSpace\textrm{oder}\MDFPSpace\chi=5 \MDFPeriod
\]
Da $\chi>3$ vorgegeben ist, folgt $\chi=5$.
\end{MHint}

\begin{MExercise}
Zeigen Sie, dass die Punkte $A=\MPointThree{4}{2}{7}$, $B=\MPointThree{3}{1}{9}$ und $C=\MPointThree{2}{3}{8}$ die Ecken eines gleichseitigen Dreiecks bilden. 
\end{MExercise}

\begin{MHint}{\iSolution}
Das Dreieck ist gleichseitig, falls
\[
 |\MDVec{A B}| = |\MDVec{A C}| = |\MDVec{B C}|
\]
gilt.

\[
 |\MDVec{A B}| = |\MDVec{B}-\MDVec{A}|=\left|\MVector{3\\1\\9}-\MVector{4\\2\\7}\right|=\left|\MVector{-1\\-1\\2}\right|=\sqrt{(-1)^2+(-1)^2+2^2}=\sqrt{6} \MDFPSpace,
\]
\[
 |\MDVec{A C}| = |\MDVec{C}-\MDVec{A}|=\left|\MVector{2\\3\\8}-\MVector{4\\2\\7}\right|=\left|\MVector{-2\\1\\1}\right|=\sqrt{(-2)^2+1^2+1^2}=\sqrt{6} \MDFPSpace,
\]
\[
 |\MDVec{B C}| = |\MDVec{C}-\MDVec{B}|=\left|\MVector{2\\3\\8}-\MVector{3\\1\\9}\right|=\left|\MVector{-1\\2\\-1}\right|=\sqrt{(-1)^2+2^2+(-1)^2}=\sqrt{6} \MDFPeriod
\]
Somit ist das Dreieck gleichseitig.
\end{MHint}


\end{MXContent}

\MSubsection{Geraden und Ebenen}
\MLabel{VBKM10_GeradenEbenen}

\begin{MIntro}
\MDeclareSiteUXID{VBKM10_GeradenEbenen_Intro}
In diesem Abschnitt werden nun Vektoren genutzt, um (zunächst) Geraden in der Ebene zu beschreiben. Man stellt dann fest, dass sich diese Beschreibung von Geraden unmittelbar auf den dreidimensionalen Fall übertragen lässt, man also damit auch Geraden im Raum beschreiben kann. Im Raum gibt es dann, neben den Geraden, weitere mathematisch interessante und mit Hilfe von Vektoren einfach zu beschreibende Objekte, nämlich Ebenen. Schließlich kann man sich überlegen, wie Punkte, Geraden und Ebenen im Raum relativ zueinander liegen können.

Hierfür sind die -- in der folgenden Infobox zusammengefassten -- Konzepte wichtig.

\begin{MInfo}\MLabel{info:kollinearkomplanar}
\begin{itemize}
 \item Zwei Vektoren $\MVec{a}$ und $\MVec{b}$ aus dem $\R^2$ oder $\R^3$ ($\MVec{a},\MVec{b}\neq\MDVec{O}$) heißen \MEntry{kollinear}{kollinear}, falls es eine Zahl $s\in\R$ gibt, so dass 
 \[
  \MVec{a}=s\MVec{b}
 \]
 gilt.
 \item Drei Vektoren $\MVec{a}$, $\MVec{b}$ und $\MVec{c}$ aus dem $\R^3$ ($\MVec{a},\MVec{b},\MVec{c}\neq\MDVec{O}$) heißen \MEntry{komplanar}{komplanar}, falls  es zwei Zahlen $s,t\in\R$ gibt, so dass
 \[
  \MVec{a}=s\MVec{b}+t\MVec{c}
 \]
 gilt.
\end{itemize}
\end{MInfo}

\begin{MHint}{Weiterführende Bemerkung}
Der Nullvektor wird im Rahmen dieses Kurses aus der Definition der Kollinearität und Komplanarität ausgeschlossen, da sich dies in der hier benutzten Verwendung zur Beschreibung von Geraden und Ebenen als ausreichend erweisen wird. Zum Preis von etwas komplizierteren Bedingungen für Kollinearität und Komplanarität kann man den Nullvektor hinzunehmen und wird dann ganz natürlich auf die (wichtigen) Begriffe der \textbf{linearen Unabhängigkeit} und der \textbf{linearen Hülle} geführt, die aber den Rahmen dieses Kurses sprengen.   
\end{MHint}


Die folgenden Überlegungen und Abbildungen machen klar, warum die Konzepte der Kollinearität und Komplanarität insbesondere für die Betrachtung von Geraden und Ebenen wichtig sind: 

Kollineare Vektoren sind Vielfache voneinander, das heißt die Repräsentanten kollinearer Vektoren mit gleichem Basispunkt liegen auf einer Geraden. Zum Beispiel sind die Vektoren
\[
 \MVec{x}=\MVector{2\\-1}
\]
und 
\[
 \MVec{y}=\MVector{-1\\\frac{1}{2}}
\]
kollinear, da 
\[
 \MVec{y}=\MVector{-1\\\frac{1}{2}}=-\frac{1}{2}\MVector{2\\-1}=-\frac{1}{2}\MVec{x}
\]
(oder auch $\MVec{x}=-2\MVec{y}$) gilt. Weitere Vektoren die zu $\MVec{x}$ und auch zu $\MVec{y}$ kollinear sind, sind beispielsweise $\MVector{4\\-2}$ oder $\MVector{-2\\1}$. Allerdings ist zum Beispiel der Vektor $\MVector{1\\1}$ nicht kollinear zu $\MVec{x}$ (und damit auch nicht zu $\MVec{y}$), da es \textit{keine} Zahl $s\in\R$ gibt, die die Gleichung
\[
 \MVec{x}=\MVector{2\\-1}=s\MVector{1\\1}
\]
erfüllt. Repräsentanten kollinearer Vektoren mit gleichem Basispunkt, etwa die Pfeile der zugehörigen Ortsvektoren (siehe unten), liegen alle auf einer Geraden:
\begin{center}
\MTikzAuto{
\begin{tikzpicture}
%Koordinatensystem
\draw[->,color=black] (-2.5,0) -- (4.5,0);
\foreach \x in {-2,-1,1,2,3,4}
\draw[shift={(\x,0)},color=black] (0pt,2pt) -- (0pt,-2pt) node[below] {\footnotesize $\x$};
\draw[->,color=black] (0,-2.5) -- (0,1.5);
\foreach \y in {-2,-1,1}
\draw[shift={(0,\y)},color=black] (2pt,0pt) -- (-2pt,0pt) node[left] {\footnotesize $\y$};
\draw[color=black] (-10pt,-8pt) node[right] {\footnotesize $0$};
%Achsenbeschriftung
\draw (4.5,0) node[anchor=north west] {$x$};
\draw (-0.5,1.8) node[anchor=north west] {$y$};
%Pfeile
\draw[color=red, ->, line width=2pt] (0,0) -- (4,-2);
\draw[color=blue, ->, line width=2pt] (0,0) -- (2,-1);
\draw[color=violet, ->, line width=2pt] (0,0) -- (-2,1);
\draw[color=green, ->, line width=2pt] (0,0) -- (-1,0.5);
%Beschriftung
\draw[color=red] (4,-1.9) node[anchor=south] {\scriptsize $\MVector{4\\-2}$};
\draw[color=blue] (2,-1.1) node[anchor=north] {\scriptsize $\MVector{2\\-1}$};
\draw[color=violet] (-2,1.1) node[anchor=south] {\scriptsize $\MVector{-2\\1}$};
\draw[color=green] (-1,0.6) node[anchor=south] {\scriptsize $\MVector{-1\\\frac{1}{2}}$};
\draw[color=black] (4.5,-2.25) node[anchor=north] {\scriptsize Gerade};
%Gerade
\draw[color=black] (-2.5,1.25) -- (4.5,-2.25);
\end{tikzpicture}
} 
\end{center}
Komplanare Vektoren im Raum sind derart, dass ihre Repräsentanten in der gleichen Ebene liegen, falls diese den gleichen Basispunkt haben. So sind beispielsweise die Vektoren 
\[
 \MVec{e}_1=\MVector{1\\0\\0}\MDFPSpace,\MDFPaSpace\MVec{e}_2=\MVector{0\\1\\0}\MDFPSpace\textrm{und}\MDFPSpace\MVector{2\\3\\0}
\]
komplanar, da 
\[
 \MVector{2\\3\\0}=2\MVec{e}_1 + 3\MVec{e}_2=2\MVector{1\\0\\0}+3\MVector{0\\1\\0}
\]
gilt. Die Pfeile ihrer zugehörigen Ortsvektoren als Repräsentanten liegen alle in der $x y$-Ebene eines Koordinatensystems im Raum. Dahingegen sind beispielsweise die Vektoren 
\[
 \MVec{e}_1=\MVector{1\\0\\0},\MDFPSpace\MVec{e}_2=\MVector{0\\1\\0}\MDFPSpace\textrm{und}\MDFPSpace\MVector{2\\3\\2}
\]
\textit{nicht} komplanar, da $\MVector{2\\3\\2}$ eine von Null verschiedene $z$-Komponente aufweist, seine Repräsentanten also immer aus der $x y$-Ebene herauszeigen. Man sieht leicht ein, dass es keine Zahlen $s,t\in\R$ geben kann, so dass die Gleichung
\[
 \MVector{2\\3\\2}=s\MVector{1\\0\\0}+t\MVector{0\\1\\0}
\]
erfüllt ist. Bild dazu:
\begin{center}
\MTikzAuto{%
\begin{tikzpicture}[>=stealth]
% The axes
\draw[->] (xyz cs:x=-1.5) -- (xyz cs:x=3.5) node[above] {\footnotesize $x$};
\draw[->] (xyz cs:y=-1.5) -- (xyz cs:y=3.5) node[right] {\footnotesize $y$};
\draw[->] (xyz cs:z=-1.5) -- (xyz cs:z=3.5) node[left] {\footnotesize $z$};
% The ticks
\foreach \coo in {-1,1,2,3}
{
  \draw (\coo,-3pt) -- (\coo,3pt) node[below=4pt] {\footnotesize \coo};
  \draw (-3pt,\coo) -- (3pt,\coo) node[left=4pt] {\footnotesize \coo};
  \draw (xyz cs:y=-0.1pt,z=\coo) -- (xyz cs:y=0.1pt,z=\coo) node[below=3pt] {\scriptsize \coo};
}
%Plane
\def \q1{(-1.5,-1.5,0) -- (3.5,-1.5,0) -- (3.5,3.5,0) -- (-1.5,3.5,0)}
\fill[color=red,fill=red,fill opacity=0.15] \q1;
\draw[color=red] (2,2,0) node[right] {\scriptsize $x y$-Ebene};
%Hilfslinien
\draw[color=gray,dashed,line width=1pt] (0,0,2) -- (2,0,2);
\draw[color=gray,dashed,line width=1pt] (2,0,0) -- (2,0,2);
\draw[color=gray,dashed,line width=1pt] (2,0,2) -- (2,3,2);
%Arrows
\draw[color=red,->,line width=1pt] (0,0,0) -- (1,0,0);
\draw[color=red,->,line width=1pt] (0,0,0) -- (0,1,0);
\draw[color=green,->,line width=1pt] (0,0,0) -- (2,3,0);
\draw[color=black,->,line width=1pt] (0,0,0) -- (2,3,2);
%Labels
\draw[color=red] (0.5,0,0) node[below] {\scriptsize $e_1$};
\draw[color=red] (0,0.5,0) node[left] {\scriptsize $e_1$};
\draw[color=black] (2,3,2) node[above] {\scriptsize $\MVector{2\\3\\2}$};
\draw[color=green] (2,3,0) node[above] {\scriptsize $\MVector{2\\3\\0}$};
\end{tikzpicture}
}
\end{center}

\end{MIntro}

\begin{MXContent}{Geraden in der Ebene und im Raum}{Geraden Ebene Raum}{STD}
\MLabel{VBKM10_GeradenEbene}
\MDeclareSiteUXID{VBKM10_GeradenEbene}
In Kapitel \MNRef{VBKM09} wurden Geraden in der Ebene mittels Koordinatengleichungen für die Punkte auf den Geraden bezüglich eines festen Koordinatensystems beschrieben. Zum Beispiel ist in dieser Beschreibung eine Gerade $g$ mit Steigung $\frac{1}{2}$ und Achsenabschnitt $1$ gegeben als Punktmenge
\[
 g=\{\MPointTwo{x}{y}\in\R^2\MCondSetSep y=\frac{1}{2}x+1\} \MDFPSpace,
\]
wofür man oft kurz auch nur die Koordinatengleichung (hier in \MSRef{VBKM09_Info_Normalform}{Normalform}) angibt:
\[
 g\colon y=\frac{1}{2}x+1\MDFPeriod
\]
Das folgende Bild zeigt die Gerade:
\begin{center}
\MTikzAuto{
\begin{tikzpicture}
%Koordinatensystem
\draw[->,color=black] (-3.5,0) -- (4.5,0);
\foreach \x in {-3,-2,-1,1,2,3,4}
\draw[shift={(\x,0)},color=black] (0pt,2pt) -- (0pt,-2pt) node[below] {\footnotesize $\x$};
\draw[->,color=black] (0,-1.5) -- (0,3.5);
\foreach \y in {-1,1,2,3}
\draw[shift={(0,\y)},color=black] (2pt,0pt) -- (-2pt,0pt) node[left] {\footnotesize $\y$};
\draw[color=black] (-10pt,-8pt) node[right] {\footnotesize $0$};
%Achsenbeschriftung
\draw (4.5,0) node[anchor=north west] {$x$};
\draw (-0.5,3.8) node[anchor=north west] {$y$};
%Gerade
\draw[color=red] (-3.5,-0.75) -- (4.5,3.25);
\draw[color=red] (4.5,3.25) node[anchor=north] {\footnotesize $g$};
\end{tikzpicture}
} 
\end{center}

Im Folgenden sollen die Punkte auf der Geraden nun durch ihre zugehörigen Ortsvektoren beschrieben werden. Die nachstehende Überlegung führt auf diese Beschreibung: Die Punkte $\MPointTwo{x}{y}$ auf der obigen Geraden $g$ erfüllen die Gleichung
\[
 y=\frac{1}{2}x+1
\]
für ihre Koordinaten. Man kann diese Gleichung für die $y$-Koordinaten in die Punkte einsetzen und erhält, dass die Gerade $g$ durch Punkte der Form $\MPointTwo[\Big]{x}{\frac{1}{2}x+1}$ mit $x\in\R$ gebildet wird. Die zu diesen Punkten gehörenden Ortsvektoren sollen mit $\MVec{r}$ bezeichnet sein. Dann gilt
\[
 \MVec{r}=\MVector{x\\\frac{1}{2}x+1}=x\MVector{1\\\frac{1}{2}}+\MVector{0\\1}
\]
mit $x\in\R$. Das heißt die Gerade $g$ kann unter Benutzung von Ortsvektoren auch beschrieben werden mittels
\[
 g\colon \MVec{r}=x\MVector{1\\\frac{1}{2}}+\MVector{0\\1} \MDFPSpace,\MDFPaSpace x\in\R \MDFPeriod
\]
Mit anderen Worten, die Punkte auf $g$ werden gebildet durch die Summe des Vektors $\MVec{a}=\MVector{0\\1}$ und allen möglichen Vielfachen des Vektors $\MVec{u}=\MVector{1\\\frac{1}{2}}$, also allen zu  $\MVector{1\\\frac{1}{2}}$ kollinearen Vektoren. Das folgende Bild stellt diese Sichtweise auf Geraden dar:

\begin{center}
\MTikzAuto{
\begin{tikzpicture}
%Koordinatensystem
\draw[->,color=black] (-3.5,0) -- (4.5,0);
\foreach \x in {-3,-2,-1,1,2,3,4}
\draw[shift={(\x,0)},color=black] (0pt,2pt) -- (0pt,-2pt) node[below] {\footnotesize $\x$};
\draw[->,color=black] (0,-1.5) -- (0,3.5);
\foreach \y in {-1,1,2,3}
\draw[shift={(0,\y)},color=black] (2pt,0pt) -- (-2pt,0pt) node[left] {\footnotesize $\y$};
\draw[color=black] (-10pt,-8pt) node[right] {\footnotesize $0$};
%Achsenbeschriftung
\draw (4.5,0) node[anchor=north west] {$x$};
\draw (-0.5,3.8) node[anchor=north west] {$y$};
%Gerade
\draw[color=red] (-3.5,-0.75) -- (4.5,3.25);
\draw[color=red] (4.5,3.25) node[anchor=south] {\footnotesize $g$};
%Hilfslinien
\draw[color=gray] (0,1) -- (-0.45,1.89);
\draw[color=gray] (2,2) -- (1.55,2.89);
\draw[color=gray] (3.5,2.75) -- (3.05,3.64);
\draw[color=gray] (1,1.5) -- (0.55,2.39);
\draw[color=gray] (-2,0) -- (-2.44,0.89);
%Richtungsvektoren
\draw[fill=red] (1,1.5) circle (1.5pt);
\draw[fill=red] (-2,0) circle (1.5pt);
\draw[fill=red] (2,2) circle (1.5pt);
\draw[fill=red] (3.5,2.75) circle (1.5pt);
\draw[color=violet,->] (0,1) -- (1,1.5);
\draw[color=violet,->] (0,1) -- (-2,0);
\draw[color=violet,->] (-0.09,1.18) -- (1.91,2.18);
\draw[color=violet,->] (-0.18,1.36) -- (3.32,3.11);
%Aufpunkt
\draw[fill=blue] (0,1) circle (1.5pt);
\draw[color=blue,->,line width=0.8pt] (0,0) -- (0,1);
\draw[color=blue] (0,0.5) node[anchor=east] {\scriptsize $\MVec{a}$};
\draw[color=blue] (0,0.95) node[anchor=west] {\scriptsize $\MPointTwo{0}{1}$};
%Beschriftung
\draw[color=violet] (-2.44,0.89) node[anchor=south] {\scriptsize $-2\MVec{u}$};
\draw[color=violet] (-0.45,1.89) node[anchor=south east] {\scriptsize $0\cdot\MVec{u}$};
\draw[color=violet] (0.55,2.39) node[anchor=south] {\scriptsize $\MVec{u}$};
\draw[color=violet] (1.55,2.89) node[anchor=south] {\scriptsize $2\MVec{u}$};
\draw[color=violet] (3.05,3.64) node[anchor=south] {\scriptsize $\MZahl{3}{5}\cdot\MVec{u}$};
\draw[color=red] (0.9,1.6) node[anchor=north west] {\scriptsize $\MPointTwo{1}{\frac{3}{2}}$}; 
\draw[color=red] (1.9,2.1) node[anchor=north west] {\scriptsize $\MPointTwo{2}{2}$};
\draw[color=red] (3.4,2.85) node[anchor=north west] {\scriptsize $\MPointTwo{\frac{7}{2}}{\frac{11}{4}}$};
\draw[color=red] (-2.05,-0.05) node[anchor=south east] {\scriptsize $\MPointTwo{-2}{0}$};
\end{tikzpicture}
} 
\end{center}

Die nächste Infobox stellt die wichtigsten Begriffe, Methoden und Konzepte zu dieser als \textbf{Punkt-Richtungsform} oder \textbf{Parameterform} bezeichneten Darstellungsweise von Geraden zusammen.

\begin{MInfo}
\begin{itemize}
 \item Eine Gerade $g$ in der Ebene ist in \MEntry{Punkt-Richtungsform}{Punkt-Richtungsform} oder \MEntry{Parameterform}{Parameterform} gegeben als Menge von Ortsvektoren
\[
 g=\left\{\MVec{r}=\lambda\MVec{u}+\MVec{a}\in\R^2\MCondSetSep\lambda\in\R\right\} \MDFPSpace,
\]
oft kurz geschrieben als
\[
 g\colon \MVec{r}=\lambda\MVec{u}+\MVec{a}\MDFPSpace,\MDFPaSpace\lambda\in\R\MDFPeriod
\]
Hierbei wird $\lambda$ als \textbf{Parameter}, $\MVec{a}$ als \MEntry{Aufpunktvektor}{Aufpunktvektor} und $\MVec{u}\neq\MDVec{O}$ als \MEntry{Richtungsvektor}{Richtungsvektor} der Geraden bezeichnet. Die Ortsvektoren $\MVec{r}$ zeigen dann zu den einzelnen Punkten auf der Geraden. Der Aufpunktvektor $\MVec{a}$ ist der Ortsvektor eines festen Punktes auf der Geraden, der als \MEntry{Aufpunkt}{Aufpunkt} bezeichnet wird. Die Vielfachen $\lambda\MVec{u}$ von $\MVec{u}$ sind alle Vektoren, die kollinear zu $\MVec{u}$ sind:
\begin{center}
\MTikzAuto{
\begin{tikzpicture}
%Koordinatensystem
\draw[->,color=black] (-1.5,0) -- (4.5,0);
%\foreach \x in {-3,-2,-1,1,2,3,4}
%\draw[shift={(\x,0)},color=black] (0pt,2pt) -- (0pt,-2pt) node[below] {\footnotesize $\x$};
\draw[->,color=black] (0,-1.5) -- (0,3.5);
%\foreach \y in {-1,1,2,3}
%\draw[shift={(0,\y)},color=black] (2pt,0pt) -- (-2pt,0pt) node[left] {\footnotesize $\y$};
%\draw[color=black] (-10pt,-8pt) node[right] {\footnotesize $0$};
%Achsenbeschriftung
\draw (4.5,0) node[anchor=north west] {$x$};
\draw (-0.5,3.8) node[anchor=north west] {$y$};
%Gerade
\draw[color=red] (-1.5,0.75) -- (4.5,3.75);
\draw[color=red] (4.5,3.75) node[anchor=north] {\footnotesize $g$};
%Vektoren
\draw[color=blue,->,line width= 1pt] (0,0) -- (1,2); 
\draw[color=violet,->,line width= 1pt] (1,2) -- (3,3); 
\draw[color=red,->,line width= 1pt] (0,0) -- (3,3); 
%Beschriftung
\draw[color=blue] (0.5,1) node[anchor=east] {\footnotesize $\MVec{a}$};
\draw[color=violet] (2,2.5) node[anchor=south] {\footnotesize $\lambda\MVec{u}$};
\draw[color=red] (1.5,1.5) node[anchor=west] {\footnotesize $\MVec{r}$};
\end{tikzpicture}
} 
\end{center}
\item Für eine Gerade $g$, die durch Angabe einer Geradengleichung in Normalform
\[
 g\colon y=m x+b
\]
vorliegt, kann eine Punkt-Richtungsform angegeben werden, indem die Ortsvektoren $\MVector{x\\m x+b}=x\MVector{1\\m}+\MVector{0\\b}$ gebildet werden. Die Punkt-Richtungsform lautet dann 
\[
 g\colon \MVec{r}=x\MVector{1\\m}+\MVector{0\\b}\MDFPSpace,\MDFPaSpace x\in\R
\]
mit dem Richtungsvektor $\MVec{u}=\MVector{1\\m}$ und dem Aufpunktvektor $\MVec{a}=\MVector{0\\b}$.
\item Für eine Gerade $g$, die in Parameterform
\[
 g\colon \MVec{r}=\lambda\MVec{u}+\MVec{a}\MDFPSpace,\MDFPaSpace\lambda\in\R
\]
vorliegt, kann eine zugehörige Geradengleichung folgendermaßen ermittelt werden: Der Richtungsvektor $\MVec{u}=\MVector{u_x\\u_y}$ liefert durch Anlegen eines Steigungsdreiecks sofort die Steigung der Geraden. Es gilt
\[
 m=\frac{u_y}{u_x} \MDFPeriod
\]
\begin{center}
\MTikzAuto{
\begin{tikzpicture}
%Koordinatensystem
\draw[->,color=black] (-1.5,0) -- (4.5,0);
%\foreach \x in {-3,-2,-1,1,2,3,4}
%\draw[shift={(\x,0)},color=black] (0pt,2pt) -- (0pt,-2pt) node[below] {\footnotesize $\x$};
\draw[->,color=black] (0,-1.5) -- (0,3.5);
%\foreach \y in {-1,1,2,3}
%\draw[shift={(0,\y)},color=black] (2pt,0pt) -- (-2pt,0pt) node[left] {\footnotesize $\y$};
%\draw[color=black] (-10pt,-8pt) node[right] {\footnotesize $0$};
%Achsenbeschriftung
\draw (4.5,0) node[anchor=north west] {$x$};
\draw (-0.5,3.8) node[anchor=north west] {$y$};
%Gerade
\draw[color=red] (-1.5,0.75) -- (4.5,3.75);
\draw[color=red] (4.5,3.75) node[anchor=north] {\footnotesize $g$};
%Vektoren
\draw[color=blue,->,line width= 1pt] (0,0) -- (1,2); 
\draw[color=violet,->,line width= 1pt] (1,2) -- (3,3); 
%\draw[color=red,->,line width= 1pt] (0,0) -- (3,3); 
%Beschriftung
\draw[color=blue] (0.5,1) node[anchor=east] {\footnotesize $\MVec{a}$};
\draw[color=violet] (2,2.5) node[anchor=south] {\footnotesize $\MVec{u}$};
%\draw[color=red] (1.5,1.5) node[anchor=west] {\footnotesize $\MVec{r}$};
%Steigungsdreieck
\draw[color=violet] (1,2) -- (3,2);
\draw[color=violet] (3,2) -- (3,3);
\draw[color=violet] (2,2) node[anchor=north] {\footnotesize $u_x$};
\draw[color=violet] (3,2.5) node[anchor=west] {\footnotesize $u_y$};
\end{tikzpicture}
} 
\end{center}
(Hierfür muss $u_x\neq0$ sein. Der Spezialfall $u_x=0$ wird im Beispiel unten behandelt.) Nach den Methoden aus Abschnitt \MNRef{VBKM09_Koordinatengleichungen} benötigt man zur Angabe der Geradengleichung in Normalform nun nur noch einen Punkt auf der Geraden, aus dem man den Achsenabschnitt $b$ bestimmt. Hierfür benutzt man am einfachsten den Aufpunktvektor $\MVec{a}$.
\end{itemize}


\end{MInfo}

Man stellt hier sofort fest, dass die Parameterform einer Geraden nicht eindeutig ist. Als Aufpunkt kann jeder Punkt auf der Geraden dienen, und auch als Richtungsvektor hat man beliebig viele 
kollineare Vektoren zur Auswahl. So wird zum Beispiel die Gerade $g$ mit der Koordinatengleichung
\[
 g\colon\frac{1}{2}x+1 
\]
aus dem einführenden Beispiel nicht nur durch 
\[
 g\colon \MVec{r}=x\MVector{1\\\frac{1}{2}}+\MVector{0\\1}\MDFPSpace,\MDFPaSpace x\in\R
\]
in Parameterform dargestellt, sondern auch durch 
\[
 g\colon \MVec{r}=\lambda\MVector{2\\1}+\MVector{2\\2}\MDFPSpace,\MDFPaSpace \lambda\in\R
\]
oder
\[
 g\colon \MVec{r}=\nu\MVector{-2\\-1}+\MVector{-2\\0}\MDFPSpace,\MDFPaSpace \nu\in\R \MDFPeriod
\]
Oft benutzt man Darstellungen mit einem möglichst einfachen Richtungsvektor. Es sollte nur darauf geachtet werden, dass bei Darstellungen der gleichen Geraden mittels unterschiedlicher Richtungs- oder Aufpunktvektoren jeweils andere Variablen für den Parameter verwendet werden, da gleiche Parameterwerte in unterschiedlichen Darstellungen zu unterschiedlichen Punkten auf der Geraden führen. So ergibt beispielsweise der Parameterwert $\lambda=1$ in der entsprechenden obigen Parameterform für $g$ den Punkt
\[
 1\cdot\MVector{2\\1}+\MVector{2\\2}=\MVector{4\\3}
\]
auf $g$, der Parameterwert $\nu=1$ der entsprechenden obigen Parameterform für $g$ den Punkt
\[
 1\cdot\MVector{-2\\-1}+\MVector{-2\\0}=\MVector{-4\\-1}
\]
auf $g$.

Das folgende Beispiel zeigt einige Anwendungen der Punkt-Richtungsform.

\begin{MExample}
\begin{itemize}
 \item Für die Gerade $g$ in der Ebene, welche durch die Geradengleichung 
 \[
  g\colon 2y-3x=6
 \]
 gegeben ist, sollen zwei verschiedene Parameterformen ermittelt werden.
 
 Zunächst wird die Geradengleichung auf Normalform gebracht:
 \[
  2y-3x=6\MDFPaSpace\Leftrightarrow\MDFPaSpace y=\frac{3}{2}x+3 \MDFPeriod
 \]
 Punkte auf $g$ haben also die Form $\MPointTwo[\Big]{x}{\frac{3}{2}x+3}$ mit $x\in\R$, welche durch Ortsvektoren $\MVector{x\\\frac{3}{2}x+3}$ mit $x\in\R$ beschrieben werden. Folglich ist eine mögliche Parameterform durch
 \[
  g\colon \MVec{r}=x\MVector{1\\\frac{3}{2}}+\MVector{0\\3}\MDFPSpace,\MDFPaSpace x\in\R
 \]
 gegeben. Für eine andere Parameterform können ein beliebiger anderer Richtungsvektor, der kollinear zu $\MVector{1\\\frac{3}{2}}$ ist, und ein beliebiger anderer Aufpunkt auf $g$ gewählt werden. Zum Beispiel ist $\MVector{2\\3}$ kollinear zu $\MVector{1\\\frac{3}{2}}$, da $\MVector{2\\3}=2\MVector{1\\\frac{3}{2}}$ gilt. Und $\MVector{2\\6}$ ist ein anderer passender Aufpunktvektor, da der Punkt $\MPointTwo{2}{6}$ offenbar die Geradengleichung erfüllt. Folglich ist 
 \[
  g\colon \MVec{r}=\sigma\MVector{2\\3}+\MVector{2\\6}\MDFPSpace,\MDFPaSpace \sigma\in\R
 \]
 eine weitere mögliche Parameterform der Geraden $g$.
 \item Auch für eine Gerade, deren Koordinatengleichung nicht auf Normalform gebracht werden kann, wie etwa
 \[
  h\colon x=2 \MDFPSpace,
 \]
 kann eine Punkt-Richtungsform angegeben werden.
 
 Punkte auf der Geraden $h$ haben alle die Form $\MPointTwo{2}{y}$ für $y\in\R$ mit zugehörigen Ortsvektoren $\MVector{2\\y}$ für $y\in\R$. Da $\MVector{2\\y}=y\MVector{0\\1}+\MVector{2\\0}$ gilt, ist eine mögliche Punkt-Richtungsform für $h$ durch
 \[
  h\colon \MVec{r}=y\MVector{0\\1}+\MVector{2\\0}\MDFPSpace,\MDFPaSpace y\in\R
 \]
 gegeben.
 \item Für die Gerade $\alpha$ in Parameterform mit
 \[
  \alpha\colon \MVec{r}=\mu\MVector{-3\\2} + \MVector{1\\1}\MDFPSpace,\MDFPaSpace \mu\in\R
 \]
 soll die zugehörige Geradengleichung in Normalform ermittelt werden.
 
 Der Richtungsvektor $\MVector{-3\\2}$ liefert die Steigung $m=\frac{2}{-3}=-\frac{2}{3}$. Somit hat die Geradengleichung in Normalform die Form
 \[
  \alpha\colon y=-\frac{2}{3}x+b \MDFPeriod
 \]
 Der Aufpunktvektor von $\alpha$ lautet $\MVector{1\\1}$. Somit kann zur Bestimmung des Achsenabschnitts $b$ der Aufpunkt $\MPointTwo{1}{1}$ eingesetzt werden:
 \[
  1=-\frac{2}{3}\cdot1+b\MDFPSpace\Leftrightarrow\MDFPSpace b=\frac{5}{3}\MDFPeriod
 \]
 Somit ergibt sich
 \[
  \alpha\colon y=-\frac{2}{3}x+\frac{5}{3} \MDFPeriod
 \]
 \item Auch für eine Gerade in Parameterform wie etwa
 \[
  \beta\colon \MVec{r}=\lambda\MVector{0\\-2} + \MVector{-1\\1}\MDFPSpace,\MDFPaSpace \lambda\in\R \MDFPSpace ,
 \]
 für welche die $x$-Komponente des Richtungsvektors $0$ ist, kann eine zugehörige Geradengleichung ermittelt werden.
 
 Der Richtungsvektor mit $x$-Komponente gleich $0$ impliziert, dass die Gerade parallel zur $y$-Achse verläuft. Folglich hat die Geradengleichung die Form
 \[
  \beta\colon x=c \MDFPeriod
 \]
 Die Konstante $c$ kann wieder durch Einsetzen des Aufpunkts $\MPointTwo{-1}{1}$ bestimmt werden. Man erhält $-1=c$, und folglich gilt
 \[
  \beta\colon x=-1 \MDFPeriod
 \]
 \item Zu den beiden Punkten $P=\MPointTwo{-1}{-1}$ und $Q=\MPointTwo{3}{2}$ ist die Gerade $P Q$ in Parameterform zu bestimmen. 
 
 Als Richtungsvektor dient hier der Verbindungsvektor
 \[
  \MVec{u}=\MDVec{P Q}=\MDVec{Q}-\MDVec{P}=\MVector{3\\2}-\MVector{-1\\-1}=\MVector{4\\3}
 \]
 und als Aufpunktvektor der Ortsvektor von einem der gegebenen Punkte, zum Beispiel
 \[
  \MVec{a}=\MDVec{P}=\MVector{-1\\-1}\MDFPeriod
 \]
 Somit gilt
 \[
  P Q\colon \MVec{r}=\lambda\MVector{4\\3}+\MVector{-1\\-1}\MDFPSpace,\MDFPaSpace\lambda\in\R \MDFPeriod
 \]
 Bild hierzu:
 \begin{center}
\MTikzAuto{
\begin{tikzpicture}
%Koordinatensystem
\draw[->,color=black] (-2.5,0) -- (4.5,0);
\foreach \x in {-2,-1,1,2,3,4}
\draw[shift={(\x,0)},color=black] (0pt,2pt) -- (0pt,-2pt) node[below] {\footnotesize $\x$};
\draw[->,color=black] (0,-2.5) -- (0,3.5);
\foreach \y in {-2,-1,1,2,3}
\draw[shift={(0,\y)},color=black] (2pt,0pt) -- (-2pt,0pt) node[left] {\footnotesize $\y$};
\draw[color=black] (10pt,5pt) node[left] {\footnotesize $0$};
%Achsenbeschriftung
\draw (4.5,0) node[anchor=north west] {$x$};
\draw (-0.5,3.8) node[anchor=north west] {$y$};
%Gerade
\draw[color=red] (-2,-1.75) -- (4.5,3.125);
\draw[color=red] (4.5,3.125) node[anchor=south] {\footnotesize $P Q$};

%Punkte
\draw[fill=red] (-1,-1) circle (1.5pt); 
\draw[color=red] (-1,-1) node[anchor=north] {\footnotesize $P$};
\draw[fill=red] (3,2) circle (1.5pt); 
\draw[color=red] (3,2) node[anchor=south] {\footnotesize $Q$};
%Vektoren
\draw[color=violet,->,line width=0.8pt] (-1,-1) -- (3,2);
\draw[color=violet] (1,0.5) node[anchor=south] {\footnotesize $\MVec{u}$};
\draw[color=blue,->,line width=0.8pt] (0,0) -- (-1,-1);
\draw[color=blue] (-0.5,-0.5) node[anchor=south] {\footnotesize $\MVec{a}$};

\end{tikzpicture}
} 
\end{center}
\end{itemize}

\end{MExample}

\begin{MExercise}
\begin{MExerciseItems}
\item{
Gegeben ist die Gerade $g$ mittels
\[
 g\colon \MVec{r}=t\MVector{-1\\5}+\MVector{2\\5}\MDFPSpace,\MDFPaSpace t\in\R
\]
in Parameterform. Bestimmen Sie die Koordinatengleichung von $g$ in Normalform:\\
$g\colon y=$\MLFunctionQuestion{15}{-5*x+15}{5}{x}{5}{PARA1}.
}
\item{
Die Gerade $h$ mit der Koordinatengleichung
\[
 h\colon\frac{1}{2}y+x+2=0
\]
hat die Parameterform
\[
 h\colon \MVec{r}=s\MVector{a\\2}+\MVector{b\\-5}\MDFPSpace,\MDFPaSpace s\in\R \MDFPeriod
\]
Bestimmen Sie die fehlenden Werte $a$ und $b$.\\
$a=$\MLFunctionQuestion{15}{-1}{5}{x}{5}{PARA2}\\
$b=$\MLFunctionQuestion{15}{0.5}{5}{x}{5}{PARA3}
}
\item{
Gegeben sind die beiden Punkte $A=\MPointTwo{-2}{-1}$ und $B=\MPointTwo{3}{-\frac{3}{2}}$. Welche der folgenden Parameterformen sind korrekte Darstellungen der Geraden $A B$?

\begin{MQuestionGroup}
\begin{tabular}{lll}
\MLCheckbox{1}{PARA4} & (i) & $A B:\MVec{r}=s\MVector{5\\-\frac{1}{2}}+\MVector{0\\-\frac{6}{5}},\MDFPSpace s\in\R$\\
\MLCheckbox{0}{PARA5} & (ii) & $A B:\MVec{r}=t\MVector{5\\\frac{1}{2}}+\MVector{-2\\-1},\MDFPSpace t\in\R$\\
\MLCheckbox{1}{PARA6} & (iii) & $A B:\MVec{r}=u\MVector{-5\\\frac{1}{2}}+\MVector{-12\\0},\MDFPSpace u\in\R$\\
\MLCheckbox{1}{PARA7} & (iv) & $A B:\MVec{r}=v\MVector{10\\-1}+\MVector{4\\-\frac{8}{5}},\MDFPSpace v\in\R$\\
\MLCheckbox{0}{PARA8} & (v) & $A B:\MVec{r}=w\MVector{-1\\10}+\MVector{-22\\1},\MDFPSpace w\in\R$\\
\MLCheckbox{0}{PARA9} & (vi) & $A B:\MVec{r}=z\MVector{\frac{5}{2}\\-\frac{1}{4}}+\MVector{\frac{6}{5}\\-1},\MDFPSpace z\in\R$\\
\end{tabular} 
\end{MQuestionGroup}

\MGroupButton{Antworten kontrollieren}
}
\item{
Für welchen Wert von $\psi$ liegt der Punkt $P$ mit dem Ortsvektor
\[
 \MDVec{P}=\MVector{-2\\\psi}
\]
auf der Geraden 
\[
 i\colon \MVec{r}=\tau\MVector{1\\-3}+\MVector{-1\\2}\MDFPSpace,\MDFPaSpace\tau\in\R \MDFPSpace ,
\]
und für welchen Parameterwert $\tau$ gilt dann $\MVec{r}=\MDVec{P}$?\\
$\psi=$\MLFunctionQuestion{15}{5}{5}{x}{5}{PARA10}\\
$\tau=$\MLFunctionQuestion{15}{-1}{5}{x}{5}{PARA11}
}
\end{MExerciseItems}

\begin{MHint}{\iSolution}
 \begin{MExerciseItems}
\item{
Die Steigung der Geraden wird aus dem Richtungsvektor $\MVector{-1\\5}$ ermittelt: $m=\frac{5}{-1}=-5$. Damit gilt
\[
 g\colon y=-5x+b \MDFPeriod
\]
Der Achsenabschnitt ergibt sich durch Einsetzen des Aufpunkts $\MPointTwo{2}{5}$:
\[
 5=-5\cdot 2+b\MDFPSpace\Leftrightarrow\MDFPSpace b=15\MDFPeriod
\]
Somit:
\[
 g\colon y=-5x+15\MDFPeriod
\]
}
\item{
Überführen der Geradengleichung in Normalform ergibt
\[
 \frac{1}{2}y+x+2=0\MDFPSpace\Leftrightarrow\MDFPSpace y=-2x-4 \MDFPeriod
\]
Der zum Aufpunktvektor $\MVector{b\\-5}$ gehörende Aufpunkt $\MPointTwo{b}{-5}$ muss auf der Geraden liegen, also die Geradengleichung erfüllen. Damit ergibt sich
\[
 -5 = -2b-4\MDFPSpace\Leftrightarrow\MDFPSpace b=\frac{1}{2} \MDFPeriod
\]
Die Ortsvektoren der Punkte auf $h$ haben die Form $\MVector{x\\-2x-4}=x\MVector{1\\-2}+\MVector{0\\-4}$. Somit ist $\MVector{1\\-2}$ ein Richtungsvektor von $h$. Weitere Richtungsvektoren von $h$ sind zu diesem kollinear. Da
\[
 -1\cdot\MVector{1\\-2}=\MVector{-1\\2}
\]
gilt, folgt $a=-1$.
}
\item{
Aus den gegebenen Punkten $A$ und $B$ folgt der Richtungsvektor
\[
 \MDVec{A B}=\MDVec{B}-\MDVec{A}=\MVector{3\\-\frac{3}{2}}-\MVector{-2\\-1}=\MVector{5\\-\frac{1}{2}} \MDFPeriod
\]
Die Richtungsvektoren in den Fällen (ii) und (v) sind nicht zu diesem Vektor kollinear. Damit sind (ii) und (v) sicher keine korrekten Darstellungen der Geraden $A B$. Aus dem Richtungsvektor $\MDVec{A B}=\MVector{5\\-\frac{1}{2}}$ folgt die Steigung $m=\frac{-\frac{1}{2}}{5}=-\frac{1}{10}$ der Geraden $A B$. Damit lautet die Geradengleichung
\[
 A B\colon y=-\frac{1}{10}x+b \MDFPSpace ,
\]
und Einsetzen von $A$ liefert den  Achsenabschnitt $b$:
\[
 -1=-\frac{1}{10}\cdot(-2)+b\MDFPSpace\Leftrightarrow\MDFPSpace b=-\frac{6}{5}\MDFPeriod
\]
Es gilt also
\[
 A B\colon y=-\frac{1}{10}x-\frac{6}{5} \MDFPeriod
\]
Diese Geradengleichung wird von den Aufpunkten in den Fällen (i) bis (v) erfüllt, im Fall (vi) aber nicht. Damit sind insgesamt die Fälle (i), (iii) und (iv) korrekte Darstellungen der Geraden, die Fälle (ii), (v) und (vi) aber nicht. 
}
\item{
Die Bedingung 
\[
 \MVector{-2\\\psi} = \tau\MVector{1\\-3}+\MVector{-1\\2}
\]
führt auf die beiden Gleichungen
\[
 -2=\tau-1\MDFPSpace\textrm{ und }\MDFPSpace\psi=-3\tau+2 \MDFPeriod
\]
Damit ergibt sich zunächst $\tau=-1$ und somit $\psi=-3\cdot(-1)+2=5$.
}
\end{MExerciseItems}
\end{MHint}

\end{MExercise}

Geraden im Raum lassen sich im Gegensatz zu Geraden in der Ebene \textit{nicht} mit Hilfe einer Koordinatengleichung darstellen. Hier hilft aber die Punkt-Richtungsform, die sich problemlos vom zwei- auf den dreidimensionalen Fall überträgt:

\begin{MInfo}
Eine Gerade $g$ im Raum ist in \MEntry{Punkt-Richtungsform}{Punkt-Richtungsform} oder \MEntry{Parameterform}{Parameterform} gegeben als Menge von Ortsvektoren
\[
 g=\left\{\MVec{r}=\lambda\MVec{u}+\MVec{a}\in\R^3\MCondSetSep\lambda\in\R\right\} \MDFPSpace,
\]
oft kurz geschrieben als
\[
 g\colon \MVec{r}=\lambda\MVec{u}+\MVec{a}\MDFPSpace,\MDFPaSpace\lambda\in\R\MDFPeriod
\]
Hier gelten die gleichen Bezeichnungen wie im zweidimensionalen Fall: $\lambda$ heißt \textbf{Parameter}, $\MVec{a}$ heißt \MEntry{Aufpunktvektor}{Aufpunktvektor} und $\MVec{u}\neq\MDVec{O}$ heißt \MEntry{Richtungsvektor}{Richtungsvektor} der Geraden:
\begin{center}
\MTikzAuto{%
\begin{tikzpicture}[>=stealth]

% The axes
\draw[->] (xyz cs:x=-2.5) -- (xyz cs:x=3.5) node[above] {\footnotesize $x$};
\draw[->] (xyz cs:y=-2.5) -- (xyz cs:y=3.5) node[right] {\footnotesize $y$};
\draw[->] (xyz cs:z=-2.5) -- (xyz cs:z=3.5) node[left] {\footnotesize $z$};
% The ticks
\foreach \coo in {-2,-1,1,2,3}
{
  \draw (\coo,-3pt) -- (\coo,3pt) node[below=4pt] {\footnotesize \coo};
  \draw (-3pt,\coo) -- (3pt,\coo) node[left=4pt] {\footnotesize \coo};
  \draw (xyz cs:y=-0.1pt,z=\coo) -- (xyz cs:y=0.1pt,z=\coo) node[below=3pt] {\scriptsize \coo};
}
\draw[color=red] (-2,-1,2) -- (3,3,-1);
\draw[color=blue,->,line width = 0.8pt] (0,0,0) -- (0.5,1,0.5);
\draw[color=violet,->,line width=0.8pt] (0.5,1,0.5) -- (1.75,2,-0.25);
\draw[color=red,->,line width=0.8pt] (0,0,0) -- (1.75,2,-0.25);
\draw[color=red] (3,3,-1) node[below] {\scriptsize $g$};
\draw[color=blue] (0.25,0.4,0.25) node[left] {\scriptsize $\MVec{a}$};
\draw[color=violet] (1,1.5,0.375) node[above] {\scriptsize $\lambda\MVec{u}$};
\draw[color=red] (0.875,1,-0.125) node[right] {\scriptsize $\MVec{r}$};
\end{tikzpicture}
}
\end{center}
\end{MInfo}

Auch hier im dreidimensionalen Fall ist -- analog zur Situation bei Geraden in der Ebene -- die Parameterform einer Geraden nie eindeutig. Das folgende Beispiel zeigt die Anwendung von Geraden in Parameterform im Raum.

\begin{MExample}
Zu den beiden gegebenen Punkten $P=\MPointThree{-1}{-2}{\frac{1}{2}}$ und $Q=\MPointThree{2}{0}{8}$ sind zwei unterschiedliche Darstellungen der Geraden $P Q$ in Parameterform anzugeben.

Der Verbindungsvektor $\MDVec{P Q}$ dient als Richtungsvektor:
\[
 \MDVec{P Q}=\MDVec{Q}-\MDVec{P}=\MVector{2\\0\\8}-\MVector{-1\\-2\\\frac{1}{2}}=\MVector{3\\2\\\frac{15}{2}}\MDFPeriod
\]
Der Punkt $Q$ kann als Aufpunkt benutzt werden. Damit ergibt sich die Parameterform
\[
 P Q\colon \MVec{r}=t\MVector{3\\2\\\frac{15}{2}}+\MVector{2\\0\\8}\MDFPSpace,\MDFPaSpace t\in\R\MDFPeriod
\]
Weitere zulässige Richtungsvektoren müssen kollinear zu $\MDVec{P Q}$ sein. Zum Beispiel:
\[
 \MVector{-6\\-4\\-15}=-2\MVector{3\\2\\\frac{15}{2}}\MDFPeriod
\]
Dann kann zum Beispiel auch der Punkt $P$ als Aufpunkt benutzt werden, und man erhält
\[
 P Q\colon \MVec{r}=s\MVector{-6\\-4\\-15}+\MVector{-1\\-2\\\frac{1}{2}}\MDFPSpace,\MDFPaSpace s\in\R
\]
als weitere korrekte Punkt-Richtungsform für die Gerade $P Q$.
\end{MExample}

\begin{MExercise}
\begin{MExerciseItems}
\item{
Die Gerade $h=A B$, welche durch die gegebenen Punkte $A=\MPointThree{-1}{-1}{0}$ und $B=\MPointThree{-3}{0}{1}$ verläuft,
hat die Parameterform
\[
 h\colon \MVec{r}=\lambda\MVector{4\\a\\b}+\MVector{c\\d\\-4}\MDFPSpace,\MDFPaSpace\lambda\in\R \MDFPeriod
\]
Bestimmen Sie die fehlenden Werte $a$, $b$, $c$ und $d$.\\
$a=$\MLFunctionQuestion{15}{-2}{5}{x}{5}{PARA31}\\
$b=$\MLFunctionQuestion{15}{-2}{5}{x}{5}{PARA32}\\
$c=$\MLFunctionQuestion{15}{7}{5}{x}{5}{PARA33}\\
$d=$\MLFunctionQuestion{15}{-5}{5}{x}{5}{PARA34}\\
}
\item{
Für welchen Wert von $\chi$ liegt der Punkt $P$ mit dem Ortsvektor
\[
 \MDVec{P}=\MVector{-2\\\chi\\-8}
\]
auf der Geraden 
\[
 g\colon \MVec{r}=\nu\MVector{1\\-3\\8}+\MVector{-1\\2\\0}\MDFPSpace,\MDFPaSpace\nu\in\R \MDFPSpace ,
\]
und für welchen Parameterwert $\nu$ gilt dann $\MVec{r}=\MDVec{P}$?\\
$\chi=$\MLFunctionQuestion{15}{5}{5}{x}{5}{PARA35}\\
$\nu=$\MLFunctionQuestion{15}{-1}{5}{x}{5}{PARA36}
}
\end{MExerciseItems}

\begin{MHint}{\iSolution}
 \begin{MExerciseItems}
\item{
Aus den gegebenen Punkten $A$ und $B$ folgt der Richtungsvektor
\[
 \MDVec{A B}=\MDVec{B}-\MDVec{A}=\MVector{-3\\0\\1}-\MVector{-1\\-1\\0}=\MVector{-2\\1\\1} \MDFPeriod
\]
Somit ist $\MVector{4\\a\\b}$ kollinear zu $\MVector{-2\\1\\1}$ und damit ebenfalls ein zulässiger Richtungsvektor für $h$, falls $a=b=-2$ gilt, denn dann hat man
\[
 \MVector{4\\-2\\-2} = -2\cdot\MVector{-2\\1\\1} \MDFPeriod
\]
Der Aufpunktvektor $\MVector{c\\d\\-4}$ muss zu einem Punkt auf $h$ gehören. Eine mögliche Parameterform der Geraden $h$ ergibt sich unter Benutzung von $\MDVec{A B}$ als Richtungsvektor und $\MDVec{A}$ als Aufpunktvektor:
\[
 h\colon\MVec{r}=t\MVector{-2\\1\\1}+\MVector{-1\\-1\\0}\MDFPSpace,\MDFPaSpace t\in\R \MDFPeriod
\]
Dies führt auf die Gleichung
\[
 \MVector{c\\d\\-4}=t\MVector{-2\\1\\1}+\MVector{-1\\-1\\0}=\MVector{-2t-1\\t-1\\t} \MDFPSpace ,
\]
aus deren dritter Komponente sofort $t=-4$ abzulesen ist. Damit muss
\[
 \MVector{c\\d\\-4}=-4\cdot\MVector{-2\\1\\1}+\MVector{-1\\-1\\0}=\MVector{7\\-5\\-4}
\]
und folglich $c=7$ sowie $d=-5$ gelten.
}
\item{
Die Bedingung 
\[
 \MVector{-2\\\chi\\-8} = \nu\MVector{1\\-3\\8}+\MVector{-1\\2\\0} = \MVector{\nu-1\\-3\nu+2\\8\nu}
\]
führt in der ersten und dritten Komponente auf $\nu=-1$, womit in der zweiten Komponente $\chi=-3\cdot(-1)+2=5$ gilt.
}
\end{MExerciseItems}
\end{MHint}

\end{MExercise}

\end{MXContent}


\begin{MXContent}{Ebenen im Raum}{Ebenen Raum}{STD}
\MLabel{VBKM10_EbenenRaum}
\MDeclareSiteUXID{VBKM10_EbenenRaum}

Startet man mit einem Vektor $\MVec{u}$ im Raum und betrachtet alle Vielfachen $\lambda\MVec{u}$, $\lambda\in\R$ dieses Vektors, so erhält man alle Vektoren, die kollinear zu $\MVec{u}$ sind (vgl. Infobox \MNRef{info:kollinearkomplanar}). Zusammen mit einem Aufpunktvektor -- und interpretiert als Ortsvektoren -- bilden alle diese Vektoren dann die Parameterform einer Geraden, wie sie im vorigen Abschnitt \MNRef{VBKM10_GeradenEbene} untersucht wurde. Aufbauend darauf ist es nun natürlich zu fragen, was man erhält, wenn man mit zwei festen (aber \textit{nicht} kollinearen) Vektoren $\MVec{u}$ und $\MVec{v}$ startet und dann alle möglichen Vektoren betrachtet, die zu diesen komplanar sind, also alle Vektoren, die man durch $\lambda\MVec{u}+\mu\MVec{v}$; $\lambda,\mu\in\R$ erhält (vgl. wieder Infobox \MNRef{info:kollinearkomplanar}). Zusammen mit einem Aufpunktvektor ergibt dies eine Verallgemeinerung des Konzepts der Parameterform einer Gerade, nämlich die Parameterform einer Ebene im Raum, welche in der unten 
stehenden Infobox 
beschrieben wird.  

Für Ebenen werden für gewöhnlich Großbuchstaben ($E$, $F$, $G$, $\MHDots$) als Variablen verwendet. Natürlich ist das Konzept einer Ebene nur im $\R^3$ sinnvoll.

\begin{MInfo}
Eine Ebene $E$ im Raum ist in \MEntry{Punkt-Richtungsform}{Punkt-Richtungsform (einer Ebene)} oder \MEntry{Parameterform}{Parameterform (einer Ebene)} gegeben als Menge von Ortsvektoren
\[
 E=\{\MVec{r}=\MVec{a}+\lambda\MVec{u}+\mu\MVec{v}\MCondSetSep \lambda,\mu\in\R\} \MDFPSpace ,
\]
oft kurz geschrieben als
\[
 E\colon\MVec{r}=\MVec{a}+\lambda\MVec{u}+\mu\MVec{v}\MDFPSpace;\MDFPaSpace \lambda,\mu\in\R \MDFPeriod
\]
Hierbei werden $\lambda$ und $\mu$ als \textbf{Parameter}, $\MVec{a}$ als \MEntry{Aufpunktvektor}{Aufpunktvektor (einer Ebene)} und $\MVec{u},\MVec{v}\neq\MDVec{O}$ als \MEntry{Richtungsvektoren}{Richtungsvektoren (einer Ebene)} der Ebene bezeichnet. Die Richtungsvektoren $\MVec{u}$ und $\MVec{v}$ sind dabei \textit{nicht} kollinear. Die Ortsvektoren $\MVec{r}$ zeigen dann zu den einzelnen Punkten in der Ebene. Der Aufpunktvektor $\MVec{a}$ ist der Ortsvektor eines festen Punktes auf der Ebene, der als \MEntry{Aufpunkt}{Aufpunkt (einer Ebene)} bezeichnet wird:

% \begin{center}
% \MTikzAuto{%
% \begin{tikzpicture}[>=stealth]
% 
% % The axes
% \draw[->] (xyz cs:x=-3.5) -- (xyz cs:x=3.5) node[above] {\footnotesize $x$};
% \draw[->] (xyz cs:y=-3.5) -- (xyz cs:y=3.5) node[right] {\footnotesize $y$};
% \draw[->] (xyz cs:z=-3.5) -- (xyz cs:z=3.5) node[left] {\footnotesize $z$};
% % The ticks
% \foreach \coo in {-3,-2,-1,1,2,3}
% {
%   \draw (\coo,-3pt) -- (\coo,3pt) node[below=4pt] {\footnotesize \coo};
%   \draw (-3pt,\coo) -- (3pt,\coo) node[left=4pt] {\footnotesize \coo};
%   \draw (xyz cs:y=-0.1pt,z=\coo) -- (xyz cs:y=0.1pt,z=\coo) node[below=3pt] {\scriptsize \coo};
% }
% \draw[color=blue,->,line width = 0.8pt] (0,0,0) -- (2,1,0);
% \draw[color=violet,->,line width=0.8pt] (2,1,0) -- (2,3,1);
% \draw[color=green,->,line width=0.8pt] (2,1,0) -- (3,2,-1);
% % \draw[color=red] (3,3,-1) node[below] {\scriptsize $g$};
% % \draw[color=blue] (0.25,0.4,0.25) node[left] {\scriptsize $\MVec{a}$};
% % \draw[color=violet] (1,1.5,0.375) node[above] {\scriptsize $\lambda\MVec{u}$};
% % \draw[color=red] (0.875,1,-0.125) node[right] {\scriptsize $\MVec{r}$};
% \end{tikzpicture}
% }
% \end{center}

(Diese Abbildung erscheint in Kürze.)

\end{MInfo}

Während zwei gegebene Punkte im Raum eine Gerade eindeutig festlegen (siehe Abschnitt \MNRef{VBKM10_GeradenEbene}), so legen drei gegebene Punkte im Raum eine Ebene eindeutig fest. Aus drei gegebenen Punkten kann relativ einfach die Parameterform der zugehörigen Ebene bestimmt werden. Die Punkt-Richtungsform einer Ebene ist -- wie auch diejenige einer Geraden -- für eine gegebene Ebene nicht eindeutig. Es gibt immer viele gleichwertige Punkt-Richtungsformen, um eine Ebene darzustellen. Das folgende Beispiel zeigt einige typische Anwendungen.

\begin{MExample}
\begin{itemize}
 \item Der Aufpunktvektor $\MVec{a}=\MVector{0\\1\\0}$ und die Richtungsvektoren $\MVec{u}=\MVector{1\\0\\0}$, $\MVec{v}=\MVector{0\\0\\1}$ ergeben eine Ebene
 \[
  E\colon \MVec{r}=\MVec{a}+\lambda\MVec{u}+\mu\MVec{v}=\MVector{0\\1\\0}+\lambda\MVector{1\\0\\0}+\mu\MVector{0\\0\\1}\MDFPSpace;\MDFPaSpace\lambda,\mu\in\R
 \]
 in Parameterform, die in der Höhe $1$ parallel zur $x z$-Ebene im Koordinatensystem liegt:
 
 (Diese Abbildung erscheint in Kürze.)
 
 Die oben angegebene Parameterform für $E$ ist nicht die einzig mögliche. Jeder andere Punkt in $E$ ist ebenfalls als Aufpunkt möglich. Zum Beispiel liegt der Punkt, welcher durch den Ortsvektor $\MVec{a}^\prime=\MVector{1\\1\\1}$ gegeben ist, in $E$, denn es gilt für $\lambda=\mu=1$:
 \[
  \MVector{1\\1\\1}=\MVector{0\\1\\0}+1\cdot\MVector{1\\0\\0}+1\cdot\MVector{0\\0\\1}\MDFPeriod
 \]
 Dieser kann als Aufpunktvektor verwendet werden. Als andere Richtungsvektoren können alle Vektoren verwendet werden, die zu $\MVec{u}$ und $\MVec{v}$ komplanar, zueinander aber nicht kollinear sind, zum Beispiel $\MVec{u}^\prime=\MVector{1\\0\\1}=1\cdot\MVector{1\\0\\0}+1\cdot\MVector{0\\0\\1}$ und $\MVec{v}^\prime=\MVector{1\\0\\-1}=1\cdot\MVector{1\\0\\0}-1\cdot\MVector{0\\0\\1}$. Dann ist eine weitere Darstellung von $E$ in Parameterform durch
 \[
  E\colon \MVec{r}=\MVec{a}^\prime+s\MVec{u}^\prime+t\MVec{v}^\prime=\MVector{1\\1\\1}+s\MVector{1\\0\\1}+t\MVector{1\\0\\-1}\MDFPSpace;\MDFPaSpace s,t\in\R
 \]
 möglich.
 \item Gegeben sind die drei Punkte $A=\MPointThree{1}{0}{-2}$, $B=\MPointThree{4}{1}{2}$ und $C=\MPointThree{0}{2}{1}$. Es ist eine Parameterform der Ebene $F$ anzugeben, die durch diese drei Punkte festgelegt wird.
 
 Einer der drei Punkte, zum Beispiel $A$, wird als Aufpunkt benutzt. Dann ist $\MDVec{A}=\MVector{1\\0\\-2}$ der Aufpunktvektor. Als Richtungsvektoren dienen dann die Verbindungsvektoren vom Aufpunkt zu den anderen beiden Punkten:
 \[
  \MDVec{A B} = \MDVec{B}-\MDVec{A}=\MVector{4\\1\\2}-\MVector{1\\0\\-2}=\MVector{3\\1\\4}\MDFPSpace,
 \]
 \[
  \MDVec{A C} = \MDVec{C}-\MDVec{A}=\MVector{0\\2\\1}-\MVector{1\\0\\-2}=\MVector{-1\\2\\3}\MDFPeriod
 \]
 Folglich ist
 \[
  F\colon\MVec{r}=\MVector{1\\0\\-2}+\rho\MVector{3\\1\\4}+\sigma\MVector{-1\\2\\3}\MDFPSpace;\MDFPaSpace\rho,\sigma\in\R
 \]
 eine korrekte Darstellung von $F$ in Parameterform.
  
 (Diese Abbildung erscheint in Kürze.)
 
 \item Von zwei Punkten $P=\MPointThree{1}{2}{3}$ und $Q=\MPointThree{2}{6}{6}$ ist zu überprüfen, ob sie in der Ebene $G$, die in Parameterform durch
 \[
  G\colon\MVec{r}=\MVector{0\\3\\2}+\mu\MVector{1\\2\\3}+\nu\MVector{0\\1\\2}\MDFPSpace;\MDFPaSpace\mu,\nu\in\R
 \]
 gegeben ist, liegen.
 
 Damit $P$ bzw. $Q$ in $G$ liegen, müssen sich ihre Ortsvektoren jeweils für bestimmte Parameterwerte $\mu$ und $\nu$ als Ortsvektoren ergeben, es müsste also $\MDVec{P}=\MVec{r}$ bzw. $\MDVec{Q}=\MVec{r}$ für jeweils geeignete $\mu$ und $\nu$ gelten. Es ergibt sich für $P$:
 \[
  \MDVec{P}=\MVector{1\\2\\3}=\MVector{0\\3\\2}+\mu\MVector{1\\2\\3}+\nu\MVector{0\\1\\2}=\MVector{\mu\\3+2\mu+\nu\\2+3\mu+2\nu}\MDFPeriod
 \]
 Die erste Komponente dieser Vektorgleichung liefert offenbar $\mu=1$. Dies in die zweite und dritte Komponente eingesetzt liefert zwei Gleichungen für $\nu$, die sich gegenseitig widersprechen:
 \[
  2=3+2\cdot 1+\nu\MDFPSpace\Leftrightarrow\MDFPSpace\nu=-3
 \]
 und
 \[
  3=2+3\cdot 1+2\nu\MDFPSpace\Leftrightarrow\MDFPSpace\nu=-1\MDFPeriod 
 \]
 Somit kann es keine Parameterwerte $\mu$ und $\nu$ geben, die in der Parameterform der Ebene $G$ den Ortsvektor $\MDVec{P}$ liefern. Folglich liegt $P$ nicht in $G$. Für $Q$ hingegen berechnet man:
 \[
  \MDVec{Q}=\MVector{2\\6\\6}=\MVector{0\\3\\2}+\mu\MVector{1\\2\\3}+\nu\MVector{0\\1\\2}=\MVector{\mu\\3+2\mu+\nu\\2+3\mu+2\nu}\MDFPeriod
 \]
 Die erste Komponente liefert nun $\mu=2$, was eingesetzt in die zweite und dritte Komponente auf
 \[
  6=3+2\cdot 2+\nu\MDFPSpace\Leftrightarrow\MDFPSpace\nu=-1
 \]
 und
 \[
  6=2+3\cdot 2+2\nu\MDFPSpace\Leftrightarrow\MDFPSpace\nu=-1
 \]
 führt. Hier ergibt sich also kein Widerspruch, sondern es stellt sich heraus, dass genau die Parameterwerte $\mu=2$ und $\nu=-1$ den Ortsvektor $\MDVec{Q}$ liefern. Somit liegt $Q$ in $G$.
 
 (Diese Abbildung erscheint in Kürze.)
 
\end{itemize}
\end{MExample}

Neben der Möglichkeit mittels dreier fester Punkte kann eine Ebene im Raum auch durch eine Gerade und einen Punkt, der nicht auf der Gerade liegt, festgelegt werden. Das folgende Beispiel zeigt, wie dies auf den Fall von drei gegebenen Punkten zurückgeführt werden kann.  

\begin{MExample}
Gegeben ist der Punkt $P=\MPointThree{2}{1}{-3}$ und die Gerade $g$ in Parameterform durch
\[
 g\colon \MVec{r}=\MVector{0\\-1\\0}+t\MVector{2\\0\\-1}\MDFPSpace,\MDFPaSpace t\in\R \MDFPeriod
\]
Der Punkt $P$ befindet sich nicht auf $g$, da es keinen Parameter $t\in\R$ gibt, so dass
\[
 \MDVec{P}=\MVector{2\\1\\-3}=\MVector{0\\-1\\0}+t\MVector{2\\0\\-1}=\MVector{2t\\-1\\-t}
\]
gilt, denn schon die zweite Komponente dieser Vektorgleichung enthält den Widerspruch $1=-1$. So legen der Punkt $P$ und die Gerade $g$ eine Ebene $E$ eindeutig fest, die sowohl $P$ als auch $g$ enthält. Eine Parameterform dieser Ebene erhält man, indem man sich zum Punkt $P$, der als Aufpunkt benutzt werden kann, noch zwei weitere Punkte auf $g$ wählt und dann genauso wie im obigen Beispiel bei gegebenen drei Punkten vorgeht. Folglich ist hier der Aufpunktvektor
\[
 \MDVec{P}=\MVector{2\\1\\-3} \MDFPSpace,
\]
und zwei weitere Punkte $Q_1$ und $Q_2$ auf $g$ ergeben sich für zwei verschiedene Werte des Parameters $t$, zum Beispiel $t=0$ und $t=1$. Die Wahl $t=0$ ergibt den Aufpunkt der Geraden. Als Ortsvektor:
\[
 \MDVec{Q}_1=\MVector{0\\-1\\0}+0\cdot\MVector{2\\0\\-1}=\MVector{0\\-1\\0} \MDFPeriod
\]
Die Wahl $t=1$ führt auf
\[
 \MDVec{Q}_2=\MVector{0\\-1\\0}+1\cdot\MVector{2\\0\\-1}=\MVector{2\\-1\\-1} \MDFPeriod
\]
Damit ergeben sich die Richtungsvektoren
\[
 \MDVec{P Q}_1=\MDVec{Q}_1 - \MDVec{P} = \MVector{0\\-1\\0} - \MVector{2\\1\\-3} = \MVector{-2\\-2\\3}
\]
und
\[
 \MDVec{P Q}_2=\MDVec{Q}_2 - \MDVec{P} = \MVector{2\\-1\\-1} - \MVector{2\\1\\-3} = \MVector{0\\-2\\2} \MDFPeriod
\]
Somit lautet eine Punkt-Richtungsform der Ebene $E$:
\[
 E\colon\MVec{r}=\MVector{2\\1\\-3}+v\MVector{-2\\-2\\3}+w\MVector{0\\-2\\2}\MDFPSpace;\MDFPaSpace v,w\in\R \MDFPeriod
\]

(Diese Abbildung erscheint in Kürze.)
\end{MExample}

Weitere Lagebeziehungen von Ebenen und Geraden -- sowie daraus abgeleitet weitere Daten, mit Hilfe derer eine Ebene eindeutig festgelegt werden kann -- werden im folgenden Abschnitt \MNRef{VBKM10_Lagebeziehung} untersucht.

\begin{MExercise}
Die Ebene $E$, welche durch die drei Punkte $A=\MPointThree{0}{0}{8}$, $B=\MPointThree{3}{-1}{10}$ und $C=\MPointThree{-1}{-2}{11}$ eindeutig festgelegt wird, hat die Parameterform
\[
 E\colon\MVec{r}=\MVector{2\\-3\\x}+s\MVector{y\\1\\-1}+t\MVector{5\\z\\-4}\MDFPSpace;\MDFPaSpace s,t\in\R \MDFPeriod
\]
Bestimmen Sie die fehlenden Komponenten $x$, $y$ und $z$.\\
$x=$\MLFunctionQuestion{15}{13}{5}{x}{5}{PLANE4}\\
$y=$\MLFunctionQuestion{15}{4}{5}{x}{5}{PLANE5}\\
$z=$\MLFunctionQuestion{15}{3}{5}{x}{5}{PLANE6}\\

\begin{MHint}{\iSolution}
Der Aufpunktvektor $\MDVec{A}=\MVector{0\\0\\8}$ und die Richtungsvektoren
\[
 \MDVec{A B}=\MDVec{B}-\MDVec{A}=\MVector{3\\-1\\10}-\MVector{0\\0\\8}=\MVector{3\\-1\\2} \MDFPSpace,
\]
\[
 \MDVec{A C}=\MDVec{C}-\MDVec{A}=\MVector{-1\\-2\\11}-\MVector{0\\0\\8}=\MVector{-1\\-2\\3}
\]
ergeben zunächst die folgende Parameterform:
\[
 E\colon\MVec{r}=\MVector{0\\0\\8}+\mu\MVector{3\\-1\\2}+\nu\MVector{-1\\-2\\3}\MDFPSpace;\MDFPaSpace \mu,\nu\in\R\MDFPeriod
\]
Damit auch der Aufpunktvektor $\MVector{2\\-3\\x}$ zu einem Punkt in $E$ gehört, muss gelten:
\[
 \MVector{2\\-3\\x}=\MVector{0\\0\\8}+\mu\MVector{3\\-1\\2}+\nu\MVector{-1\\-2\\3}=\MVector{3\mu-\nu\\-\mu-2\nu\\8+2\mu+3\nu}\MDFPeriod
\]
Dies ist ein lineares Gleichungssystem mit den drei Unbekannten $\mu$, $\nu$ und $x$, das mit den Methoden aus Abschnitt \MNRef{M04_3_Unbekannte} gelöst werden kann. Hier führt die Betrachtung der ersten und zweiten Komponente auf die beiden Gleichungen
\[
 2=3\mu-\nu\MDFPaSpace\textrm{und}\MDFPaSpace -3=-\mu-2\nu \MDFPSpace,
\]
woraus man $\mu=\nu=1$ berechnet. Dies eingesetzt in die dritte Komponente liefert
\[
 x=8+2+3=13 \MDFPeriod
\]
Damit auch $\MVector{y\\1\\-1}$ und $\MVector{5\\z\\-4}$ Richtungsvektoren der Ebene $E$ sind, müssen diese jeweils zu $\MDVec{A B}$ und $\MDVec{A C}$ komplanar sein. Für den ersten Vektor erhält man
\[
 \MVector{y\\1\\-1}=a\MVector{3\\-1\\2}+b\MVector{-1\\-2\\3}=\MVector{3a-b\\-a-2b\\2a+3b} \MDFPSpace,
\]
also wiederum ein lineares Gleichungssystem mit drei Unbekannten. In diesem Fall liefern die zweite und die dritte Komponente die beiden Gleichungen
\[
 1=-a-2b\MDFPaSpace\textrm{und}\MDFPaSpace -1=2a+3b \MDFPSpace,
\]
woraus man $a=1$ und $b=-1$ berechnet. Dies führt auf
\[
 y= 3+ (-1)\cdot(-1)=4 \MDFPeriod
\]
Völlig analog für gilt den zweiten Vektor:
\[
 \MVector{5\\z\\-4}=a\MVector{3\\-1\\2}+b\MVector{-1\\-2\\3}=\MVector{3a-b\\-a-2b\\2a+3b}\MDFPeriod
\]
Man extrahiert
\[
 5=3a-b\MDFPaSpace\textrm{und}\MDFPaSpace -4=2a+3b
\]
und berechnet $a=1$ sowie $b=-2$, was auf 
\[
 z=-1-2\cdot(-2)=3
\]
führt.
\end{MHint}

\end{MExercise}


\begin{MExercise}
Gegeben sind die Punkte $P=\MPointThree{h}{2}{-2}$, $Q=\MPointThree{1}{i}{6}$ und $R=\MPointThree{-3}{2}{j}$ sowie die Ebene $E$ in Parameterform:
\[
 E\colon\MVec{r}=\MVector{3\\0\\2}+s\MVector{2\\1\\7}+t\MVector{3\\2\\5}\MDFPSpace;\MDFPaSpace s,t\in\R\MDFPeriod
\]
Bestimmen Sie die fehlenden Komponenten $h$, $i$ und $j$, so dass die Punkte $P$, $Q$ und $R$ in der Ebene $E$ liegen.\\
$h=$\MLFunctionQuestion{15}{5}{5}{x}{5}{PLANE1}\\
$i=$\MLFunctionQuestion{15}{-2}{5}{x}{5}{PLANE2}\\
$j=$\MLFunctionQuestion{15}{-74}{5}{x}{5}{PLANE3}\\

\begin{MHint}{\iSolution}
Für die Ebene $E$ gilt
\[
 \MVec{r}=\MVector{3\\0\\2}+s\MVector{2\\1\\7}+t\MVector{3\\2\\5}=\MVector{3+2s+3t\\s+2t\\2+7s+5t} \MDFPeriod
\]
Die Bedingungen 
\[
 \MDVec{P}=\MVec{r}\MDFPSpace\Leftrightarrow\MDFPSpace\MVector{h\\2\\-2}=\MVector{3+2s+3t\\s+2t\\2+7s+5t} \MDFPSpace,
\]
\[
 \MDVec{Q}=\MVec{r}\MDFPSpace\Leftrightarrow\MDFPSpace\MVector{1\\i\\6}=\MVector{3+2s+3t\\s+2t\\2+7s+5t}
\]
und
\[
 \MDVec{R}=\MVec{r}\MDFPSpace\Leftrightarrow\MDFPSpace\MVector{-3\\2\\j}=\MVector{3+2s+3t\\s+2t\\2+7s+5t}
\]
führen jeweils auf ein lineares Gleichungssystem in den Parametern $s$, $t$ und $h$ bzw. $i$ bzw. $j$, das jeweils mit den Methoden aus Abschnitt \MNRef{M04_3_Unbekannte} gelöst werden kann.

Für $P$ ergibt sich: Die zweite und dritte Komponente bilden zwei lineare Gleichungen mit den zwei Unbekannten $s$ und $t$ der Form
\[
 2=s+2t\MDFPaSpace\textrm{und}\MDFPaSpace -4=7s+5t \MDFPSpace,
\]
woraus man die Lösung $s=-2$ und $t=2$ berechnet. Einsetzen in die erste Komponente führt auf
\[
 h=3+2\cdot(-2)+3\cdot 2=5 \MDFPeriod
\]
Für $Q$ ergibt sich: Die erste und dritte Komponente bilden zwei lineare Gleichungen mit den zwei Unbekannten $s$ und $t$ der Form
\[
 -2=2s+3t\MDFPaSpace\textrm{und}\MDFPaSpace 4=7s+5t \MDFPSpace,
\]
woraus man die Lösung $s=2$ und $t=-2$ berechnet. Einsetzen in die zweite Komponente führt auf
\[
 i=2+2\cdot(-2)=-2 \MDFPeriod
\]
Für $R$ ergibt sich: Die erste und zweite Komponente bilden zwei lineare Gleichungen mit den zwei Unbekannten $s$ und $t$ der Form
\[
 -6=2s+3t\MDFPaSpace\textrm{und}\MDFPaSpace 2=s+2t \MDFPSpace,
\]
woraus man die Lösung $s=-18$ und $t=10$ berechnet. Einsetzen in die dritte Komponente führt auf
\[
 j=2+7\cdot(-18)+5\cdot10=-74 \MDFPeriod
\]
\end{MHint}

\end{MExercise}

\end{MXContent}

\begin{MXContent}{Lagebeziehung von Geraden und Ebenen im Raum}{Lagebeziehung}{STD}
\MLabel{VBKM10_Lagebeziehung}
\MDeclareSiteUXID{VBKM10_Lagebeziehung}
Während es für die Lagebeziehung zweier Geraden in der Ebene nur drei Möglichkeiten gibt (die Geraden sind parallel, identisch oder sie schneiden sich, vgl. Abschnitt \MNRef{VBKM09_Lagebeziehungen}), existieren für zwei Geraden im Raum vier Möglichkeiten. Diese werden in der folgenden Infobox zusammengefasst. 

\begin{MInfo}\MLabel{info:schnitti}
Gegeben sind zwei Geraden im Raum in Punkt-Richtungsform, $g$ mit Aufpunktvektor $\MVec{a}$ und Richtungsvektor $\MVec{u}$ sowie $h$ mit Aufpunktvektor $\MVec{b}$ und Richtungsvektor $\MVec{v}$:
\[
 g\colon\MVec{r}=\MVec{a}+s\MVec{u}\MDFPSpace;\MDFPaSpace s\in\R \MDFPSpace,
\]
\[
 h\colon\MVec{r}=\MVec{b}+t\MVec{v}\MDFPSpace;\MDFPaSpace t\in\R\MDFPeriod
\]
Für die relative Lage von $g$ und $h$ gibt es genau vier Möglichkeiten:
\begin{enumerate}
 \item Die Geraden sind \textbf{identisch}. In diesem Fall haben $g$ und $h$ alle Punkte gemeinsam, fallen also zusammen. Dies ist genau dann der Fall, wenn die beiden Richtungsvektoren $\MVec{u}$ und $\MVec{v}$ kollinear sind und es einen gemeinsamen Punkt gibt.
 \item Die Geraden sind \MEntry{parallel}{Parallelität (zweier Geraden im Raum)}. Dies ist genau dann der Fall, wenn die beiden Richtungsvektoren $\MVec{u}$ und $\MVec{v}$ kollinear sind und es \textit{keinen} gemeinsamen Punkt gibt.
 \item Die Geraden schneiden sich. In diesem Fall haben $g$ und $h$ genau einen gemeinsamen Punkt, der \MEntry{Schnittpunkt}{Schnittpunkt (zweier Geraden im Raum)} genannt wird. Dies ist genau dann der Fall, wenn die beiden Richtungsvektoren $\MVec{u}$ und $\MVec{v}$ \textit{nicht} kollinear sind und es einen gemeinsamen Punkt gibt.
 \item Geraden, die weder identisch noch parallel sind und sich auch nicht schneiden, heißen \MEntry{windschief}{windschief}. Dies ist genau dann der Fall, wenn die beiden Richtungsvektoren $\MVec{u}$ und $\MVec{v}$ \textit{nicht} kollinear sind und es \textit{keinen} gemeinsamen Punkt gibt.
\end{enumerate}

(Diese Abbildung erscheint in Kürze.)

\end{MInfo}

Nach den Kriterien, die in obiger Infobox für die vier Möglichkeiten der relativen Lage zweier Geraden aufgeführt sind, geht man in der Praxis so vor, dass man zunächst die beiden Richtungsvektoren auf Kollinearität prüft und dann nach möglichen gemeinsamen Punkten der beiden Geraden sucht. Dies legt schließlich einen der vier Fälle eindeutig fest. Das folgende Beispiel zeigt Anwendungen dieses Vorgehens für alle vier Fälle.

\begin{MExample}
Gegeben sind die vier Geraden $g$, $h$, $i$ und $j$ in Parameterform:
\[
 g\colon\MVec{r}=\MVector{-1\\0\\3}+s\MVector{-2\\2\\-4}\MDFPSpace;\MDFPaSpace s\in\R \MDFPSpace,
\]
\[
 h\colon\MVec{r}=\MVector{1\\-2\\7}+t\MVector{1\\-1\\2} \MDFPSpace;\MDFPaSpace t\in\R \MDFPSpace,
\]
\[
 i\colon\MVec{r}=\MVector{4\\0\\8}+u\MVector{-3\\3\\-6} \MDFPSpace;\MDFPaSpace u\in\R
\]
und
\[
 j\colon\MVec{r}=\MVector{1\\3\\2}+v\MVector{1\\3\\-3} \MDFPSpace;\MDFPaSpace v\in\R \MDFPeriod
\]
\begin{itemize}
\item Die Geraden $g$ und $h$ sind identisch. Die beiden Richtungsvektoren $\MVector{-2\\2\\-4}$ von $g$ und $\MVector{1\\-1\\2}$ von $h$ sind kollinear. Es gilt
\[
  \MVector{-2\\2\\-4}= -2\cdot\MVector{1\\-1\\2} \MDFPeriod
\]
Außerdem ist der Punkt, welcher durch den Ortvektor $\MVector{1\\-2\\7}$ beschrieben wird, in $h$ (als Aufpunkt) und $g$ enthalten, denn für $g$ gilt:
\[
 \MVector{1\\-2\\7}=\MVector{-1\\0\\3}+s\MVector{-2\\2\\-4}=\MVector{-1-2s\\2s\\3-4s}\MDFPSpace\Leftrightarrow\MDFPSpace s=-1\MDFPeriod
\]
Also ergibt sich $\MVector{1\\-2\\7}$ in $g$ für den Parameterwert $s=-1$.

\item Die Geraden $h$ und $i$ (und damit natürlich auch $g$ und $i$) sind parallel. Die beiden Richtungsvektoren $\MVector{1\\-1\\2}$ von $h$ und $\MVector{-3\\3\\-6}$ von $i$ sind kollinear. Es gilt
\[
  \MVector{-3\\3\\-6}= -3\cdot\MVector{1\\-1\\2} \MDFPeriod
\]
Allerdings haben $h$ und $i$ keine Punkte gemeinsam. Dies sieht man in diesem Fall folgendermaßen: Der Aufpunkt einer der beiden Geraden ist kein Punkt auf der anderen Geraden. Dann haben die beiden Geraden gar keine gemeinsamen Punkte. Hier kann man zum Beispiel testen, ob sich der Aufpunktvektor $\MVector{4\\0\\8}$ der Gerade $i$ als Ortsvektor der Gerade $h$ ergeben kann:
\[
 \MVector{4\\0\\8}=\MVector{1\\-2\\7}+t\MVector{1\\-1\\2}=\MVector{1+t\\-2-t\\7+2t}\MDFPeriod
\]
In dieser Vektorgleichung ergäbe sich in der ersten Komponente $t=3$ und in der zweiten Komponente $t=-2$, was bereits ein Widerspruch ist. Folglich haben die beiden Geraden keine gemeinsamen Punkte.

\item Die Geraden $i$ und $j$ schneiden sich. Zunächst sind für diese beiden Geraden die Richtungsvektoren $\MVector{-3\\3\\-6}$ und $\MVector{1\\3\\-3}$ nicht kollinear. Es gibt keine Zahl $a\in\R$, so dass
\[
 \MVector{-3\\3\\-6}=a\MVector{1\\3\\-3}
\]
gilt, denn für die erste Komponente müsste $a=-3$ und für die zweite Komponente $a=1$ gelten, was bereits ein Widerspruch ist. Allerdings haben die beiden Geraden einen gemeinsamen Punkt, den man durch Gleichsetzen der Ortsvektoren für $i$ und $j$ findet:
\[
 \MVector{4\\0\\8}+u\MVector{-3\\3\\-6}=\MVector{4-3u\\3u\\8-6u}=\MVector{1+v\\3+3v\\2-3v}=\MVector{1\\3\\2}+v\MVector{1\\3\\-3}\MDFPeriod
\]
Hier führen die ersten beiden Komponenten zu den Gleichungen
\[
 3-3u=v\MDFPaSpace\textrm{und}\MDFPaSpace u=1+v
\]
für $u$ und $v$, woraus man $v=0$ und $u=1$ berechnet. Dies eingesetzt in die dritte Komponente ergibt
\[
 8-6\cdot 1=2-3\cdot 0\MDFPSpace\Leftrightarrow\MDFPSpace 2=2 \MDFPeriod
\]
Die Vektorgleichung für die Ortsvektoren ist also für die Parameterwerte $u=1$ und $v=0$ erfüllt. Folglich ergibt sich der Ortsvektor des Schnittpunkts für den Parameterwert $u=1$ in der Gerade $i$ oder auch für den Parameterwert $v=0$ in der Gerade $j$. Man erhält als  Schnittpunkt den Punkt $\MPointThree{1}{3}{2}$.

\item Die Geraden $g$ und $j$ (und damit natürlich auch $h$ und $j$) sind windschief. Auch in diesem Fall erhält man analog zum Fall mit Schnittpunkt oben recht schnell, dass die beiden Richtungsvektoren $\MVector{-2\\2\\-4}$ von $g$ und $\MVector{1\\3\\-3}$ von $j$ nicht kollinear sind. Nun haben die beiden Geraden aber keinen gemeinsamen Punkt, was man wieder durch Gleichsetzen der Ortsvektoren findet:
\[
 \MVector{-1\\0\\3}+s\MVector{-2\\2\\-4}=\MVector{-1-2s\\2s\\3-4s}=\MVector{1+v\\3+3v\\2-3v}=\MVector{1\\3\\2}+v\MVector{1\\3\\-3}\MDFPeriod
\]
Diese Vektorgleichung ist widersprüchlich, das heißt man findet keine Parameterwerte $s$ und $v$, die sie erfüllen, und folglich haben $g$ und $j$ keine gemeinsamen Punkte. Die erste und die zweite Komponente führen auf die beiden Gleichungen
\[
 -2s=2+v\MDFPaSpace\textrm{und}\MDFPaSpace 2s=3+3v \MDFPSpace,
\]
woraus man $v=-\frac{5}{4}$ und $s=-\frac{3}{8}$ berechnet. Setzt man dies aber in die dritte Komponente ein, so ergibt sich der Widerspruch
\[
 3-4(-\frac{3}{8})=2-3(-\frac{5}{4})\MDFPSpace\Leftrightarrow\MDFPSpace \frac{9}{2}=\frac{23}{4} \MDFPeriod
\]

\end{itemize}

\end{MExample}

\begin{MExercise}\MLabel{ex:twolines}
Die beiden Geraden 
\[
 g\colon\MVec{r}=\MVector{1\\2\\4}+x\MVector{-5\\10\\-15}\MDFPSpace;\MDFPaSpace x\in\R
\]
und
\[
 h\colon\MVec{r}=\MVector{4\\0\\7}+y\MVector{3\\-2\\3} \MDFPSpace;\MDFPaSpace y\in\R
\]
schneiden sich, da 

\begin{MQuestionGroup}
\begin{tabular}{lll}
\MLCheckbox{0}{INTER1} & \MBlank & die beiden Richtungsvektoren kollinear sind,\\
\MLCheckbox{0}{INTER2} & \MBlank & die beiden Richtungsvektoren nicht kollinear sind,\\
\MLCheckbox{0}{INTER3} & \MBlank & die beiden Richtungsvektoren kollinear sind und die Geraden einen gemeinsamen Punkt besitzen,\\
\MLCheckbox{1}{INTER4} & \MBlank & die beiden Richtungsvektoren nicht kollinear sind und die Geraden einen gemeinsamen Punkt besitzen,\\
\MLCheckbox{0}{INTER5} & \MBlank & die beiden Richtungsvektoren nicht kollinear sind und die Geraden keinen gemeinsamen Punkt besitzen.\\
\end{tabular}
\end{MQuestionGroup}
\MGroupButton{Antworten kontrollieren}


Berechnen Sie den Schnittpunkt $S$ der beiden Geraden $g$ und $h$.\\
$S=$\MLFunctionQuestion{15}{(1,2,4)}{5}{x}{5}{INTER0}

Der Ortsvektor des Schnittpunkts $\MDVec{S}$ ergibt sich in den Geraden $g$ und $h$ jeweils für die Parameterwerte\\
$x=$\MLFunctionQuestion{10}{0}{5}{x}{5}{INTER0a} und\\
$y=$\MLFunctionQuestion{10}{-1}{5}{x}{5}{INTER0b}.

\begin{MHint}{\iSolution}
Die beiden Richtungsvektoren $\MVector{-5\\10\\-15}$ und $\MVector{3\\-2\\3}$ sind nicht kollinear, da es keine Zahl $a\in\R$ gibt, so dass
\[
 \MVector{-5\\10\\-15}=a\MVector{3\\-2\\3}
\]
erfüllt ist. Denn betrachtet man die zweite Komponente dieser Vektorgleichung, so müsste $a=-5$ gelten, aus der ersten Komponente müsste allerdings $a=-\frac{5}{3}$ folgen; ein Widerspruch. Weiterhin besitzen die Geraden einen gemeinsamen Punkt, den Schnittpunkt $S$, der unten berechnet wird. Diese beiden Tatsachen, und nur diese, sind nach Infobox \MNRef{info:schnitti} hinreichend dafür, dass die beiden Geraden sich schneiden. 

Der Schnittpunkt ergibt sich durch Gleichsetzen der beiden Ortsvektoren:
\[
 \MVector{1\\2\\4}+x\MVector{-5\\10\\-15}=\MVector{1-5x\\2+10x\\4-15x}=\MVector{4+3y\\-2y\\7+3y}=\MVector{4\\0\\7}+y\MVector{3\\-2\\3}\MDFPeriod
\]
Die ersten beiden Komponenten dieser Vektorgleichung liefern die Gleichungen
\[
 -5x=3+3y\MDFPaSpace\textrm{und}\MDFPaSpace 2+10x=-2y \MDFPSpace,
\]
woraus man $x=0$ und $y=-1$ berechnet. Setzt man diese Werte in die dritte Komponente ein, so ergibt sich
\[
 4-15\cdot 0=7+3\cdot(-1)\MDFPSpace\Leftrightarrow\MDFPSpace 4=4 \MDFPeriod
\]
Somit ist die Vektorgleichung für diese Parameterwerte erfüllt, und es existiert ein gemeinsamer Punkt der beiden Geraden $g$ und $h$. Dessen Ortsvektor ergibt sich zum Beispiel durch Einsetzen des Parameterwerts $x=0$ in $g$:
\[
 \MDVec{S}=\MVector{1\\2\\4}+0\cdot\MVector{-5\\10\\-15}=\MVector{1\\2\\4}\MDFPeriod
\]
\end{MHint}
\end{MExercise}


\begin{MExercise}\MLabel{ex:twolines2}
Die beiden Geraden
\[
 \gamma\colon\MVec{r}=\MVector{-4\\6\\0}+s\MVector{3\\-2\\-2}\MDFPSpace;\MDFPaSpace s\in\R 
\]
und
\[
 \kappa\colon\MVec{r}=\MVector{a\\b\\4}+t\MVector{-\frac{3}{2}\\c\\1}\MDFPSpace;\MDFPaSpace t\in\R 
\]
sind parallel. Bestimmen Sie den fehlenden Eintrag $c$ und geben Sie an, welche Werte $a$ und $b$ für die Parallelität \textbf{nicht} gleichzeitig annehmen dürfen. 

$a\neq$\MLFunctionQuestion{10}{-10}{5}{x}{5}{PAR1}\\
$b\neq$\MLFunctionQuestion{10}{10}{5}{x}{5}{PAR2}\\
$c=$\MLFunctionQuestion{10}{1}{5}{x}{5}{PAR3}
\end{MExercise}

\begin{MHint}{\iSolution}
Für Parallelität müssen die beiden Richtungsvektoren kollinear sein. Man findet aus der Bedingung
\[
 \MVector{-\frac{3}{2}\\c\\1}=s\MVector{3\\-2\\-2} \MDFPSpace,
\]
den Wert $s=-\frac{1}{2}$ und damit $c=1$. Damit die Geraden wirklich parallel und nicht identisch sind, darf der Aufpunkt von $\kappa$ nicht auf $\gamma$ liegen, die Werte für $a$ und $b$ müssen also so sein, dass die Vektorgleichung
\[
 \MVector{a\\b\\4}=\MVector{-4\\6\\0}+s\MVector{3\\-2\\-2}=\MVector{-4+3s\\6-2s\\-2s}
\]
für keinen Parameterwert $s$ erfüllbar ist. Die dritte Komponente liefert sofort $s=-2$ für die Erfüllbarkeit der Gleichung. Dies eingesetzt in die erste und zweite Komponente ergibt $a=-10$ und $b=10$. Folglich muss für echte Parallelität $a\neq-10$ oder $b\neq10$ gelten.
\end{MHint}

Hat man zwei Geraden im Raum gegeben, die echt parallel sind oder sich schneiden, so stellt man fest, dass auch diese jeweils eine Ebene eindeutig festlegen:

(Diese Abbildung erscheint in Kürze.)

Es handelt sich dabei jeweils um diejenige Ebene, die beide Geraden beinhaltet.

Das folgende Beispiel zeigt, wie man aus zwei echt parallelen oder sich schneidenden Geraden die Parameterform der Ebene bekommt, die durch diese eindeutig festgelegt wird.

\begin{MExample}
\begin{itemize}
\item Die beiden Geraden 
\[
 g\colon\MVec{r}=\MVector{1\\2\\4}+s\MVector{-5\\10\\-15}\MDFPSpace;\MDFPaSpace s\in\R
\]
und
\[
 h\colon\MVec{r}=\MVector{4\\0\\7}+t\MVector{3\\-2\\3}\MDFPSpace;\MDFPaSpace t\in\R
\]
schneiden sich im Punkt $S=\MPointThree{1}{2}{4}$ (vgl. Aufgabe \MNRef{ex:twolines}). Dadurch wird eine Ebene $E$ eindeutig festgelegt, die sowohl $g$ als auch $h$ beinhaltet. Für eine Parameterform von $E$ benutzt man die Ortsvektoren von drei geeigneten Punkten, die man aus den Geraden $g$ und $h$ berechnet. Der Schnittpunkt eignet sich als Aufpunkt von $E$ mit dem Aufpunktvektor
\[
 \MDVec{S}=\MVector{1\\2\\4}.
\]
Dann ergeben sich noch Ortsvektoren von zwei Punkten $P$ und $Q$ dadurch, dass man zum Beispiel die Parameterwerte $s=1$ und $t=1$ in die Parameterformen der Geraden einsetzt:
\[
 \MDVec{P}=\MVector{1\\2\\4}+1\cdot\MVector{-5\\10\\-15}=\MVector{-4\\12\\-11} \MDFPSpace,
\]
\[
 \MDVec{Q}=\MVector{4\\0\\7}+1\cdot\MVector{3\\-2\\3}=\MVector{7\\-2\\10} \MDFPeriod
\]
Dies ergibt die Richtungsvektoren
\[
 \MDVec{S P}=\MDVec{P}-\MDVec{S}= \MVector{-4\\12\\-11}-\MVector{1\\2\\4}=\MVector{-5\\10\\-15}
\]
und
\[
 \MDVec{S Q}=\MDVec{Q}-\MDVec{S}= \MVector{7\\-2\\10}-\MVector{1\\2\\4}=\MVector{6\\-4\\11} \MDFPeriod
\]
Somit ist eine mögliche Parameterform von $E$ durch
\[
 E\colon \MVec{r}=\MVector{1\\2\\4}+\mu\MVector{-5\\10\\-15}+\nu\MVector{6\\-4\\11}\MDFPSpace;\MDFPaSpace \mu,\nu\in\R
\]
gegeben.

\item Die beiden Geraden
\[
 g\colon\MVec{r}=\MVector{-4\\6\\0}+s\MVector{3\\-2\\-2}\MDFPSpace;\MDFPaSpace s\in\R 
\]
und
\[
 h\colon\MVec{r}=\MVector{1\\1\\4}+t\MVector{-\frac{3}{2}\\1\\1}\MDFPSpace;\MDFPaSpace t\in\R 
\]
sind parallel (vgl. Aufgabe \MNRef{ex:twolines2}). Dadurch wird eine Ebene $F$ eindeutig festgelegt, die sowohl $g$ als auch $h$ beinhaltet. Für eine Parameterform von $F$ benutzt man die Ortsvektoren von drei geeigneten Punkten, die man aus den Geraden $g$ und $h$ berechnet. Es eignen sich die Ortsvektoren von Punkten auf $g$ bzw. $h$, die sich für drei Parameterwerte ergeben, zum Beispiel $s=0$, $s=1$ und $t=0$:
\[
 \MVec{A}=\MVector{-4\\6\\0}+0\cdot\MVector{3\\-2\\-2}=\MVector{-4\\6\\0} \MDFPSpace,
\]
dient als Aufpunktvektor. Dann erhält man sofort unter Benutzung von 
\[
 \MVec{B}=\MVector{-4\\6\\0}+1\cdot\MVector{3\\-2\\-2}=\MVector{-1\\4\\-2} \MDFPSpace,
\]
dass für den ersten Richtungsvektor $\MDVec{A B}$ der Ebene $F$ genau der Richtungsvektor $\MVector{3\\-2\\-2}$ von $g$ benutzt werden kann. Der zweite Richtungsvektor der Ebene ergibt sich aus dem Ortsvektor eines Punktes $B$ auf $h$ für den Parameterwert $t=0$, also den Aufpunkt von $h$:
\[
 \MDVec{B}=\MVector{1\\1\\4} \MDFPeriod
\]
Und folglich ist
\[
 \MDVec{A B}=\MDVec{B}-\MDVec{A}=\MVector{1\\1\\4}-\MVector{-4\\6\\0}=\MVector{5\\-5\\4}
\]
der zweite Richtungsvektor von $F$. Eine Parameterform von $F$ ist also gegeben durch 
\[
 F\colon \MVec{r}=\MVector{-4\\6\\0}+\lambda\MVector{3\\-2\\-2}+\mu\MVector{5\\-5\\4}\MDFPSpace;\MDFPaSpace \lambda,\mu\in\R \MDFPeriod
\]

\end{itemize}
 
\end{MExample}

Für die Lagebeziehung einer Geraden und einer Ebene im Raum gibt es wieder nur drei Möglichkeiten. Diese sind in der folgenden Infobox zusammengefasst.

\begin{MInfo}\MLabel{info:plane_line}
Sind eine Gerade $g$ mit Aufpunktvektor $\MVec{a}$ und Richtungsvektor $\MVec{u}$ sowie eine Ebene $E$ mit Aufpunktvektor $\MVec{b}$ und Richtungsvektoren $\MVec{v}$ und $\MVec{w}$ im Raum in Parameterform gegeben durch
\[
 g\colon\MVec{r}=\MVec{a}+\lambda\MVec{u}\MDFPSpace;\MDFPaSpace\lambda\in\R
\]
und
\[
 E\colon\MVec{r}=\MVec{b}+\mu\MVec{v}+\nu\MVec{w}\MDFPSpace;\MDFPaSpace\mu,\nu\in\R\MDFPSpace,
\]
so gibt es für die relative Lage von $g$ und $E$ genau drei Möglichkeiten:
\begin{enumerate}
 \item Die Gerade $g$ liegt in der Ebene $E$. Dies ist genau dann der Fall, wenn die drei Richtungsvektoren $\MVec{u}$, $\MVec{v}$ und $\MVec{w}$ komplanar sind und der Aufpunkt der Geraden in der Ebene liegt.
 \item Die Gerade $g$ liegt parallel zur Ebene $E$. Dies ist genau dann der Fall, wenn die drei Richtungsvektoren $\MVec{u}$, $\MVec{v}$ und $\MVec{w}$ komplanar sind und der Aufpunkt der Geraden \textit{nicht} in der Ebene liegt.
 \item Die Gerade $g$ und die Ebene $E$ schneiden sich. Dies ist genau dann der Fall, wenn die drei Richtungsvektoren $\MVec{u}$, $\MVec{v}$ und $\MVec{w}$ nicht komplanar sind.
\end{enumerate}

(Diese Abbildung erscheint in Kürze.)

\end{MInfo}

Hat man eine Gerade und eine Ebene gegeben und möchte man deren Lagebeziehung bestimmen, so überprüft man zunächst die drei Richtungsvektoren auf Komplanarität. Ist diese gegeben, so untersucht man den Aufpunkt der Geraden darauf hin, ob er in der Ebene enthalten ist. Dies legt letztlich einen der drei möglichen Fälle eindeutig fest. Falls sich Gerade und Ebene schneiden, kann man dann noch den Schnittpunkt berechnen. Das folgende Beispiel zeigt einige dafür benutzte Vorgehensweisen.

\begin{MExample}
Gegeben ist eine Ebene $E$ in Parameterform durch
\[
 E\colon\MVec{r}=\MVector{2\\2\\2}+s\MVector{3\\-1\\0}+t\MVector{0\\0\\2}\MDFPSpace;\MDFPaSpace s,t\in\R \MDFPeriod
\]
\begin{itemize}
 \item Eine Gerade, die den Vektor $\MVector{3\\-1\\-4}$ als Richtungsvektor aufweist, liegt entweder in der Ebene $E$ oder ist zu dieser parallel, da $\MVector{3\\-1\\-4}$ zu den beiden Richtungsvektoren von $E$ komplanar ist. Man findet aus der Bedingung
 \[
  \MVector{3\\-1\\-4}=s\MVector{3\\-1\\0}+t\MVector{0\\0\\2}
 \]
 die Zahlen $s=1$ und $t=-2$. Folglich liegt die Gerade
 \[
  g\colon\MVec{r}=\MVector{-1\\3\\0}+x\MVector{3\\-1\\-4}\MDFPSpace;\MDFPaSpace x\in\R
 \]
 in der Ebene $E$, denn der Aufpunkt $\MPointThree{-1}{3}{0}$ liegt in $E$, da man berechnet:
 \[
  \MVector{-1\\3\\0}=\MVector{2\\2\\2}+s\MVector{3\\-1\\0}+t\MVector{0\\0\\2}=\MVector{2+3s\\2-s\\2+2t}\MDFPSpace\Leftrightarrow\MDFPSpace s=t=-1\MDFPeriod
 \]
 Der Ortsvektor $\MVector{-1\\3\\0}$ der Geraden ergibt sich also für die Parameterwerte $s=t=-1$ in der Ebene. Demhingegen ist die Gerade
 \[
  h\colon\MVec{r}=y\MVector{3\\-1\\-4}\MDFPSpace;\MDFPaSpace y\in\R
 \]
 parallel zu Ebene $E$, denn $h$ weist als Aufpunkt den Ursprung $\MPointThree{0}{0}{0}$ auf. Der Ursprung liegt aber nicht in $E$, denn für die Vektorgleichung
 \[
  \MVector{0\\0\\0}=\MVector{2\\2\\2}+s\MVector{3\\-1\\0}+t\MVector{0\\0\\2}=\MVector{2+3s\\2-s\\2+2t}
 \]
 gibt es keine Parameterwerte $s$ und $t$, die diese erfüllen. Die erste Komponente würde $s=-\frac{2}{3}$ implizieren und die zweite Komponente $s=2$; ein Widerspruch.
 
 \item Jede Gerade mit einem Richtungsvektor, der nicht komplanar zu den beiden Richtungsvektoren $\MVector{3\\-1\\0}$ und $\MVector{0\\0\\2}$ von $E$ ist, schneidet die Ebene $E$ in genau einem Punkt. Ein Beispiel einer solchen Geraden ist
 \[
  k\colon\MVec{r}=\MVector{-3\\1\\0}+\mu\MVector{1\\1\\1}\MDFPSpace;\MDFPaSpace\mu\in\R \MDFPeriod
 \]
 Der Richtungsvektor $\MVector{1\\1\\1}$ ist nicht komplanar zu $\MVector{3\\-1\\0}$ und $\MVector{0\\0\\2}$, denn die Bedingung
 \[
  \MVector{1\\1\\1}= a\MVector{3\\-1\\0}+b\MVector{0\\0\\2}=\MVector{3a\\-a\\2b}
 \]
 ist durch keine Zahlen $a,b\in\R$ erfüllbar. Die erste Komponente würde $a=\frac{1}{3}$ und die zweite $a=-1$ implizieren; ein Widerspruch. Nun kann durch Gleichsetzen der Ortsvektoren der Geraden $k$ und der Ebene $E$ der Schnittpunkt der beiden berechnet werden:
 \[
  \MVector{-3\\1\\0}+\mu\MVector{1\\1\\1}=\MVector{-3+\mu\\1+\mu\\\mu}=\MVector{2+3s\\2-s\\2+2t}=\MVector{2\\2\\2}+s\MVector{3\\-1\\0}+t\MVector{0\\0\\2}\MDFPeriod
 \]
 Interessiert man sich nur für den Schnittpunkt, so genügt es den Parameterwert der Geraden zu bestimmen, für den diese Vektorgleichung erfüllt ist. Der Ortsvektor des Schnittpunkts ergibt sich dann durch Einsetzen des bestimmten Parameterwerts in die Gerade. Die ersten beiden Komponenten dieser Vektorgleichung liefern zwei Gleichungen für die Unbekannten $\mu$ und $s$:
 \[
  \mu=5+3s\MDFPaSpace\textrm{und}\MDFPaSpace \mu=1-s \MDFPSpace,
 \]
 woraus man $\mu=2$ erhält. Damit hat der Schnittpunkt den Ortsvektor
 \[
  \MVector{-3\\1\\0}+2\MVector{1\\1\\1}=\MVector{-1\\3\\2} \MDFPeriod
 \]
\end{itemize}
\end{MExample}

\begin{MExercise}
Gegeben ist die Ebene
\[
 E\colon\MVec{r}=\MVector{8\\-2\\0}+s\MVector{1\\3\\2}+t\MVector{-1\\1\\-1}\MDFPSpace;\MDFPaSpace s,t\in\R
\]
und die Gerade 
\[
 g\colon \MVec{r}=\MVector{0\\2\\1}+u\MVector{0\\4\\c}\MDFPSpace;\MDFPaSpace u\in\R \MDFPSpace,
\]
deren Aufpunkt nicht in $E$ liegt.

Bestimmen Sie den fehlenden Eintrag $c$, so dass $g$ parallel zu $E$ ist.\\
$c=$\MLFunctionQuestion{10}{1}{5}{x}{5}{BLUBB1}

Berechnen Sie für alle anderen Werte von $c$ den Schnittpunkt $S=\MPointThree{x}{y}{z}$ in Abhängigkeit von $c$. Geben Sie die drei Komponenten von $S$ getrennt an.\\
$x=$\MLFunctionQuestion{20}{0}{5}{c}{5}{BLUBB2x}\\
$y=$\MLFunctionQuestion{20}{2+40/(1-c)}{5}{c}{5}{BLUBB2y}\\
$z=$\MLFunctionQuestion{20}{1+10*c/(1-c)}{5}{c}{5}{BLUBB2z}

\begin{MHint}{\iSolution}
Damit $g$ und $E$ parallel sind, muss der Richtungsvektor $\MVector{0\\4\\c}$ von $g$ zu den beiden Richtungsvektoren $\MVector{1\\3\\2}$ und $\MVector{-1\\1\\-1}$ von $E$ komplanar sein. Die Bedingung
\[
 \MVector{0\\4\\c}=s\MVector{1\\3\\2}+t\MVector{-1\\1\\-1}
\]
ist nur für $c=1$ erfüllbar, denn die erste und zweite Komponente liefern das lineare Gleichungssystem 
\[
 s-t=0\MDFPSpace,\MDFPaSpace 3s+t=4
\]
für die Unbekannten $s$ und $t$, das die Lösung $s=t=1$ besitzt. Dies erzwingt in der dritten Komponente
\[
 c=2\cdot1-1=1 \MDFPeriod
\]
Für den Schnittpunkt werden die Ortsvektoren von Gerade und Ebene gleichgesetzt:
\[
 \MVector{0\\2\\1}+u\MVector{0\\4\\c}=\MVector{0\\2+4u\\1+c u}=\MVector{8+s-t\\-2+3s+t\\2s-t}=\MVector{8\\-2\\0}+s\MVector{1\\3\\2}+t\MVector{-1\\1\\-1}\MDFPeriod
\]
Diese Vektorgleichung entspricht einem linearen Gleichungssystem in den Variablen $s$, $t$ und $u$ mit dem Parameter $c$, das mit den Methoden aus Abschnitt \MNRef{M04_freier_Parameter} gelöst werden kann. Man erhält für die Unbekannte $u$:
\[
 u=\frac{10}{1-c}\MDFPeriod
\]
Damit ergibt sich der Ortsvektor zum Schnittpunkt $S$ durch Einsetzen in die Gerade $g$:
\[
 \MDVec{S}=\MVector{0\\2\\1}+\frac{10}{1-c}\MVector{0\\4\\c}=\MVector{0\\2+\frac{40}{1-c}\\1+\frac{10c}{1-c}}\MDFPeriod
\]

\end{MHint}


\end{MExercise}

\begin{MExercise}
Gegeben ist die Gerade
\[
 h\colon\MVec{r}=\MVector{3\\2\\1}+\rho\MVector{-8\\9\\1}\MDFPSpace;\MDFPaSpace\rho\in\R\MDFPeriod
\]
Bestimmen Sie folgende Werte des Parameters $\rho$:
\begin{MExerciseItems}
\item{Wert des Parameters $\rho$ für den $h$ die $x y$-Ebene schneidet:\ $\rho=$\MLFunctionQuestion{10}{-1}{5}{x}{5}{BLAA1}}
\item{Wert des Parameters $\rho$ für den $h$ die $y z$-Ebene schneidet:\ $\rho=$\MLFunctionQuestion{10}{3/8}{5}{x}{5}{BLAA2}}
\item{Wert des Parameters $\rho$ für den $h$ die $x z$-Ebene schneidet:\ $\rho=$\MLFunctionQuestion{10}{-2/9}{5}{x}{5}{BLAA3}}
\end{MExerciseItems}

\begin{MHint}{\iSolution}
\begin{MExerciseItems}
\item{In der $x y$-Ebene ist $z=0$, also muss die dritte Komponente des Ortsvektors von $h$ gleich Null sein:
\[
 1+\rho=0\MDFPSpace\Leftrightarrow\MDFPSpace\rho=-1\MDFPeriod
\]
} 
\item{In der $y z$-Ebene ist $x=0$, also muss die erste Komponente des Ortsvektors von $h$ gleich Null sein:
\[
 3-8\rho=0\MDFPSpace\Leftrightarrow\MDFPSpace\rho=\frac{3}{8}\MDFPeriod
\]
} 
\item{In der $x z$-Ebene ist $y=0$, also muss die zweite Komponente des Ortsvektors von $h$ gleich Null sein:
\[
 2+9\rho=0\MDFPSpace\Leftrightarrow\MDFPSpace\rho=-\frac{2}{9}\MDFPeriod
\]
} 
\end{MExerciseItems}
 
\end{MHint}

\end{MExercise}

Betrachtet man zwei Ebenen im Raum, so gibt es für ihre Lagebeziehung drei Fälle, die sich analog zu den drei möglichen relativen Lagen zweier Geraden in der Ebene aus Abschnitt \MNRef{VBKM09_Lagebeziehungen} verhalten. Die folgende Infobox stellt diese Fälle zusammen.

\begin{MInfo}
Gegeben sind zwei Ebenen $E_1$ mit dem Aufpunktvektor $\MVec{a}_1$ und den beiden Richtungsvektoren $\MVec{u}_1$ und $\MVec{v}_1$ sowie $E_2$ mit dem Aufpunktvektor $\MVec{a}_2$ und den beiden Richtungsvektoren $\MVec{u}_2$ und $\MVec{v}_2$ durch
\[
 E_1\colon \MVec{r}=\MVec{a}_1+\mu\MVec{u}_1+\nu\MVec{v}_1\MDFPSpace;\MDFPaSpace\mu,\nu\in\R
\]
und
\[
 E_2\colon \MVec{r}=\MVec{a}_2+\rho\MVec{u}_2+\sigma\MVec{v}_2\MDFPSpace;\MDFPaSpace\rho,\sigma\in\R\MDFPeriod
\]
Für die relative Lage von $E_1$ und $E_2$ gibt es genau drei Möglichkeiten:
\begin{enumerate}
 \item Die Ebenen $E_1$ und $E_2$ sind identisch, falls sie alle Punkte gemeinsam haben. Dies ist genau dann der Fall, wenn die drei Richtungsvektoren $\MVec{u}_1$, $\MVec{v}_1$, $\MVec{u}_2$ sowie die drei Richtungsvektoren $\MVec{u}_1$, $\MVec{v}_1$, $\MVec{v}_2$ komplanar sind und der Aufpunkt von $E_1$ in $E_2$ enthalten ist.  
 \item Die Ebenen $E_1$ und $E_2$ sind parallel, falls sie keine Punkte gemeinsam haben. Dies ist genau dann der Fall, wenn die drei Richtungsvektoren $\MVec{u}_1$, $\MVec{v}_1$, $\MVec{u}_2$ sowie die drei Richtungsvektoren $\MVec{u}_1$, $\MVec{v}_1$, $\MVec{v}_2$ komplanar sind und der Aufpunkt von $E_1$ \textit{nicht} in $E_2$ enthalten ist. 
 \item Die Ebenen $E_1$ und $E_2$ schneiden sich, falls ihre gemeinsamen Punkte eine Gerade bilden. Dies ist genau dann der Fall, wenn die drei Richtungsvektoren $\MVec{u}_1$, $\MVec{v}_1$, $\MVec{u}_2$ \textit{oder} die drei Richtungsvektoren $\MVec{u}_1$, $\MVec{v}_1$, $\MVec{v}_2$ \textit{nicht} komplanar sind. 
\end{enumerate}

(Diese Abbildung erscheint in Kürze.)

\end{MInfo}

Natürlich sind in den Bedingungen für die drei Fälle, die in obiger Infobox angegeben sind, die Rollen der beiden Ebenen auch vertauschbar; man kann also beispielsweise auch überprüfen, ob der Aufpunkt von $E_2$ in $E_1$ enthalten ist; dies macht keinen Unterschied. Falls die Ebenen sich schneiden, kann die Schnittgerade berechnet werden. Schnittmengen von Ebenen wurden bereits in Abschnitt \MNRef{M04_3_Unbekannte} im Rahmen der geometrischen Interpretation der Lösbarkeit linearer Gleichungssysteme mit drei Unbekannten behandelt. Die Vertrautheit mit diesem Abschnitt wird im Folgenden vorausgesetzt und dessen kurze Wiederholung wird wärmstens empfohlen. Das folgende Beispiel zeigt die Vorgehensweise bei der Untersuchung der relativen Lage zweier Ebenen.

\begin{MExample}
Gegeben sind die drei Ebenen
\[
 E\colon\MVec{r}=\MVector{0\\2\\-2}+a\MVector{1\\-2\\1}+b\MVector{4\\0\\-2}\MDFPSpace;\MDFPaSpace a,b\in\R\MDFPSpace,
\]
\[
 F\colon\MVec{r}=\MVector{5\\1\\0}+c\MVector{5\\-2\\-1}+d\MVector{-3\\-2\\3}\MDFPSpace;\MDFPaSpace c,d\in\R
\]
und
\[
 G\colon\MVec{r}=\MVector{5\\0\\1}+x\MVector{1\\-2\\0}+y\MVector{0\\0\\3}\MDFPSpace;\MDFPaSpace x,y\in\R \MDFPeriod
\]
\begin{itemize}
 \item Die Ebenen $E$ und $F$ sind parallel. Die Richtungsvektoren $\MVector{1\\-2\\1}$ und $\MVector{4\\0\\-2}$ von $E$ und der erste Richtungsvektor $\MVector{5\\-2\\-1}$ von $F$ sind komplanar, denn die Bedingung
 \[
  a\MVector{1\\-2\\1}+b\MVector{4\\0\\-2}=\MVector{5\\-2\\-1}
 \]
 ist durch die Zahlen $a=b=1$ erfüllt. Genauso sind die Richtungsvektoren $\MVector{1\\-2\\1}$ und $\MVector{4\\0\\-2}$ von $E$ und der zweite Richtungsvektor $\MVector{-3\\-2\\3}$ von $F$ sind komplanar, denn die Bedingung
 \[
  a\MVector{1\\-2\\1}+b\MVector{4\\0\\-2}=\MVector{-3\\-2\\3}
 \]
 ist durch die Zahlen $a=1$ und $b=-1$ erfüllt. Weiterhin ist der Aufpunkt von $F$ nicht in $E$ enthalten, denn die Bedingung
 \[
  \MVector{5\\1\\0}=\MVector{0\\2\\-2}+a\MVector{1\\-2\\1}+b\MVector{4\\0\\-2}=\MVector{a+4b\\2-2a\\-2+a-2b}
 \]
 ist für keine Parameterwerte $a$ und $b$ erfüllbar. Aus der zweiten Komponente würde $a=\frac{1}{2}$ folgen, was eingesetzt in die erste Komponente auf $b=\frac{9}{8}$ führt. Dies ergibt aber in der dritten Komponente den Widerspruch $0=-2+\frac{1}{2}-\frac{9}{4}$. Würde man allerdings in $F$ einen anderen Aufpunkt wählen, der in $E$ enthalten ist, zum Beispiel den gleichen Aufpunkt wie in $E$, so würde man eine zu $E$ identische Ebene erhalten, also eine andere Parameterdarstellung der gleichen Ebene. Zum Beispiel ist
 \[
  F^{\prime}\colon\MVec{r}=\MVector{0\\2\\-2}+\alpha\MVector{5\\-2\\-1}+\beta\MVector{-3\\-2\\3}\MDFPSpace;\MDFPaSpace \alpha,\beta\in\R
 \]
 eine solche Ebene.
 
 \item Die Ebenen $E$ und $G$ schneiden sich. Beide Richtungsvektoren $\MVector{1\\-2\\0}$ und $\MVector{0\\0\\3}$ von $G$ sind nicht komplanar zu den beiden Richtungsvektoren $\MVector{1\\-2\\1}$ und $\MVector{4\\0\\-2}$ von $E$. Zum Beispiel gilt für den zweiten Richtungsvektor von $G$, dass die Bedingung
 \[
  \MVector{0\\0\\3}=a\MVector{1\\-2\\1}+b\MVector{4\\0\\-2}
 \]
 durch keine Zahlen $a$ und $b$ erfüllbar ist. Die ersten beiden Komponenten würden $a=b=0$ erzwingen, was der dritten Komponente widerspricht. Die Schnittgerade der beiden Ebenen berechnet man durch Gleichsetzen der Ortsvektoren. Man erhält hier:
 \[
  \MVector{0\\2\\-2}+a\MVector{1\\-2\\1}+b\MVector{4\\0\\-2}=\MVector{a+4b\\2-2a\\-2+a-2b}=\MVector{5+x\\-2x\\1+3y}=\MVector{5\\0\\1}+x\MVector{1\\-2\\0}+y\MVector{0\\0\\3}\MDFPeriod
 \]
 Diese Vektorgleichung entspricht einem System von drei linearen Gleichungen mit den vier Unbekannten $x$, $y$, $a$ und $b$. Dies wird nun mit den Methoden aus Abschnitt \MNRef{M04_freier_Parameter} gelöst, indem man eine der Unbekannten als Parameter auffasst und die anderen Unbekannten in Abhängigkeit von diesem berechnet. Dieser übrige Parameter wird am Ende der Parameter in der Punkt-Richtungsform der zu bestimmenden Schnittgeraden werden. Welche der Unbekannten man als Parameter auffasst, ist egal. Hier wird nun $x$ als Parameter benutzt. Dann führen die ersten beiden Komponenten der Vektorgleichung auf die beiden Gleichungen
 \[
  a+4b=5+x\MDFPaSpace\textrm{und}\MDFPaSpace 2-2a=-2x \MDFPSpace,
 \]
 aus denen man $a=1+x$ und $b=1$ berechnet. Dies in die dritte Komponente eingesetzt ergibt
 \[
  -2 + (1+x) - 2\cdot 1=1+3y\MDFPSpace\Leftrightarrow\MDFPSpace y=\frac{1}{3} x-\frac{4}{3} \MDFPeriod
 \]
 Nun kann $y=\frac{1}{3}x-\frac{4}{3}$ oder $a=1+x$ und $b=1$ in die Ebene $G$ oder $E$ eingesetzt werden; dies führt -- bei gleicher Parameterwahl -- auf dieselbe Parameterdarstellung der Geraden $h$, nämlich der Schnittgeraden der beiden Ebenen. Für das Einsetzen in $G$ ergibt sich:
 \[
  h\colon\MVec{r}=\MVector{5\\0\\1}+x\MVector{1\\-2\\0}+(\Mtfrac{1}{3}x-\Mtfrac{4}{3})\MVector{0\\0\\3}=\MVector{5\\0\\-3}+x\MVector{1\\-2\\1}\MDFPSpace;\MDFPaSpace x\in\R\MDFPeriod
 \]

\end{itemize}

\end{MExample}

\begin{MExercise}
Gegeben sind die beiden Ebenen 
\[
 E\colon\MVec{r}=\MVector{-1\\3\\0}+a\MVector{2\\-5\\8}+b\MVector{0\\-1\\4}\MDFPSpace;\MDFPaSpace a,b\in\R
\]
und
\[
 F\colon\MVec{r}=\MVector{5\\0\\0}+c\MVector{-2\\3\\x}+d\MVector{2\\y\\12}\MDFPSpace;\MDFPaSpace c,d\in\R \MDFPSpace,
\]
wobei der Aufpunkt von $F$ nicht in $E$ liegt.

Bestimmen Sie die fehlenden Komponenten $x$ und $y$ von $F$, so dass $F$ und $E$ parallel sind.\\
$x=$\MLFunctionQuestion{10}{0}{5}{x}{5}{GNATZ1}\\
$y=$\MLFunctionQuestion{10}{-6}{5}{x}{5}{GNATZ2}

\begin{MHint}{\iSolution}
Für Parallelität ist erforderlich, dass beiden Richtungsvektoren von $F$ jeweils zu den beiden Richtungsvektoren von $E$ komplanar sind. Für den ersten Richtungsvektor von $F$ liefert dies die Bedingung
\[
 \MVector{-2\\3\\x}=a\MVector{2\\-5\\8}+b\MVector{0\\-1\\4} \MDFPeriod
\]
Hier berechnet man aus der ersten und zweiten Komponente die Werte $a=-1$ und $b=2$, was in der dritten Komponente den Wert
\[
 x=-8+2\cdot 4=0
\]
erzwingt. Für den zweiten Richtungsvektor von $F$ liefert dies die Bedingung
\[
 \MVector{2\\y\\12}=a\MVector{2\\-5\\8}+b\MVector{0\\-1\\4} \MDFPeriod
\]
Hier berechnet man aus der ersten und dritten Komponente die Werte $a=b=1$, was in der zweiten Komponente den Wert
\[
 y=-5-1=-6
\]
erzwingt. 
\end{MHint}


\end{MExercise}

\begin{MExercise}
Gegeben sind die beiden Ebenen 
\[
 E\colon\MVec{r}=a\MVector{2\\-5\\8}+b\MVector{0\\-1\\4}\MDFPSpace;\MDFPaSpace a,b\in\R
\]
und
\[
 F\colon\MVec{r}=c\MVector{0\\3\\4}+d\MVector{2\\-1\\0}\MDFPSpace;\MDFPaSpace c,d\in\R\MDFPSpace,
\]
welche sich schneiden und die Schnittgerade
\[
 g\colon\MVec{r}=\xi\MVector{4\\x\\y}\MDFPSpace;\MDFPaSpace \xi\in\R
\]
besitzen.

Bestimmen Sie die fehlenden Komponenten $x$ und $y$ des Richtungsvektors der Schnittgerade.\\
$x=$\MLFunctionQuestion{10}{-5}{5}{x}{5}{GNATZ3}\\
$y=$\MLFunctionQuestion{10}{-4}{5}{x}{5}{GNATZ4}

\begin{MHint}{\iSolution}
Gleichsetzen der Ortsvektoren der beiden Ebenen führt auf
\[
 a\MVector{2\\-5\\8}+b\MVector{0\\-1\\4}=\MVector{2a\\-5a-b\\8a+4b}=\MVector{2d\\3c-d\\4c}=c\MVector{0\\3\\4}+d\MVector{2\\-1\\0} \MDFPeriod
\]
Lösen des zugehörigen Gleichungssystems mit $d$ als Parameter führt auf $a=d$, $b=-\frac{5}{2}d$ und $c=-\frac{1}{2}d$. Die Bedingung $c=-\frac{1}{2}d$ in die Ebene $F$ eingesetzt führt auf
\[
 -\Mtfrac{1}{2}d\MVector{0\\3\\4}+d\MVector{2\\-1\\0}=d\MVector{2\\-\Mtfrac{5}{2}\\-2} \MDFPeriod
\]
Ein geeigneter Richtungsvektor für die Schnittgerade ist also $\MVector{2\\-\Mtfrac{5}{2}\\-2}$. Der Richtungsvektor in der angegebenen Parameterform von $g$ hat als erste Komponente allerdings eine $4$ und ist zu diesem Vektor also kollinear; dies führt auf den Vektor $\MVector{4\\-5\\-4}$ als Richtungsvektor, also $x=-5$ und $y=-4$. 
\end{MHint}


\end{MExercise}

\end{MXContent}


\MSubsection{Abschlusstest}
\MLabel{VBKM10_Abschlusstest}

\begin{MTest}{Abschlusstest Kapitel \arabic{section}}
\MDeclareSiteUXID{VBKM10_Abschlusstest}

\begin{MExercise}
Geben Sie die im Diagramm dargestellten Pfeilklassen als Vektoren an: 

\begin{center}
\MTikzAuto{
\begin{tikzpicture}[>=stealth]
%Koordinatensystem
\draw[->,color=black] (-3.5,0) -- (3.5,0);
\foreach \x in {-3,-2,-1,1,2,3}
\draw[shift={(\x,0)},color=black] (0pt,2pt) -- (0pt,-2pt) node[below] {\footnotesize $\x$};
\draw[->,color=black] (0,-1.5) -- (0,2.5);
\foreach \y in {-1,1,2}
\draw[shift={(0,\y)},color=black] (2pt,0pt) -- (-2pt,0pt) node[left] {\footnotesize $\y$};
\draw[color=black] (-10pt,-8pt) node[right] {\footnotesize $0$};
%Achsenbeschriftung
\draw (3.5,0) node[anchor=north west] {$x$};
\draw (-0.5,2.8) node[anchor=north west] {$y$};
%Hilflinien
\draw[color=gray, dotted] (-3,-1.5) -- (-3,2.5);
\draw[color=gray, dotted] (-2,-1.5) -- (-2,2.5);
\draw[color=gray, dotted] (-1,-1.5) -- (-1,2.5);
\draw[color=gray, dotted] (1,-1.5) -- (1,2.5);
\draw[color=gray, dotted] (2,-1.5) -- (2,2.5);
\draw[color=gray, dotted] (3,-1.5) -- (3,2.5);
\draw[color=gray, dotted] (-3.5,-1) -- (3.5,-1);
\draw[color=gray, dotted] (-3.5,1) -- (3.5,1);
\draw[color=gray, dotted] (-3.5,2) -- (3.5,2);
%Pfeile
\draw[color=red, ->, line width = 1.5pt] (-3,-1) -- (-2,1);
\draw[color=blue, ->, line width = 1.5pt] (-1,0) -- (1,2);
\draw[color=violet, ->, line width = 1.5pt] (0,0) -- (-1,2);
\draw[color=green, ->, line width = 1.5pt] (2,0) -- (3,1);
\draw[color=black, ->, line width = 1.5pt] (2,-1) -- (3,0);
\end{tikzpicture}
} 
\end{center}

%\MUGraphics{vektoren1.png}{width=0.5\linewidth}{Pfeile in der Ebene}{width:400px}

% Beachten Sie die Beschriftung der Achsen (vertikale Achse gehört
% zur ersten Komponente der Vektoren), 
% beispielsweise gehört zum roten Pfeil der Vektor $\MVector{1\\2}$.

\begin{MExerciseItems}
\item{Roter Vektor: \MLFunctionQuestion{15}{(1,2)}{5}{x}{5}{OR20}.} 
\item{Violetter Vektor: \MLFunctionQuestion{15}{(-1,2)}{5}{x}{5}{OR20b}.} 
\item{Blauer Vektor: \MLFunctionQuestion{15}{(2,2)}{5}{x}{5}{OR21}.} 
\item{Grüner Vektor: \MLFunctionQuestion{15}{(1,1)}{5}{x}{5}{OR22}.} 
\item{Schwarzer Vektor: \MLFunctionQuestion{15}{(1,1)}{5}{x}{5}{OR23}.}
\end{MExerciseItems}
\MInputHint{Vektoren können in der Form \texttt{(a;b)} eingegeben werden, zum Beispiel \texttt{(8;-9)} für den Vektor $\MVector{8\\-9}$.} 

% \begin{MHint}{\iSolution}
% Der rote Pfeil bewegt um eine Einheit nach rechts und um zwei Einheiten nach oben, er wird also durch $\MVector{1\\2}$ beschrieben.
% Alternativ kann man auch Startpunkte der Pfeile von den Endpunkten abziehen, um den Vektor zu bestimmen. Dabei ist dann die Beschriftung
% der Achsen zu beachten:
% $$
% \MVector{-2\\1}-\MVector{-3\\-1} \;=\; \MVector{-2-(-3)\\1-(-1)} \;=\; \MVector{1\\2}\: .
% $$
% Ebenso erhält man $\MVector{-1\\2}$ für den gelben Vektor, $\MVector{2\\2}$ für den blauen und $\MVector{1\\1}$ für den grünen sowie den schwarzen Vektor.
% Dass die Pfeile im Diagramm verschiedene Start- und Endpunkte haben spielt keine Rolle, als Vektoren (Pfeilklassen) sind beide gleich.
% \end{MHint}

\end{MExercise}

\begin{MExercise}
% Modifizierte  COSH-Frage
Ein Sportflugzeug würde bei Windstille mit einer Geschwindigkeit von
150 Kilometer pro Stunde genau nach Süden fliegen. Es wird jedoch von einem Wind,
der mit der Geschwindigkeit 30 Kilometer pro Stunde aus Richtung Westen weht,
abgetrieben. Stellen Sie die Geschwindigkeit des Flugzeugs als Summe von zwei Vektoren in der Ebene dar, wobei
die zweite Komponente zur Nord-Süd-Achse (positive Werte für Norden) und die erste Komponente
zur Ost-West-Achse gehört (positive Werte für Osten). Lassen Sie die Einheit (Kilometer pro Stunde)
in der Rechnung weg:
\begin{MExerciseItems}
\item{Bei Windstille ist die Geschwindigkeit \MLFunctionQuestion{15}{(0,-150)}{5}{x}{5}{OR0}.}
\item{Der Wind verursacht eine zusätzliche Geschwindigkeit von \MLFunctionQuestion{15}{(30,0)}{5}{x}{5}{OR3}.}
\item{Das abgetriebene Flugzeug hat insgesamt den Geschwindigkeitsvektor \MLFunctionQuestion{15}{(30,-150)}{5}{x}{5}{OR1}.}
\item{Die Länge dieses Vektors (der Betrag der Geschwindigkeit) ist \MLParsedQuestion{15}{sqrt(30*30+150*150)}{5}{OR2}.\\
\MInputHint{Wurzelausdrücke müssen Sie nicht auswerten.}}
\end{MExerciseItems}
\MInputHint{Vektoren können in der Form \texttt{(a;b)} eingegeben werden, zum Beispiel \texttt{(8;-9)} für den Vektor $\MVector{8\\-9}$.} 
% \begin{MHint}{\iSolution}
% Der Wind verursacht eine zusätzliche Geschwindigkeit von $\MVector{30\\0}$, damit ergibt sich die Geschwindigkeitsvektor
% $$
% \MVector{0\\-150}+\MVector{30\\0} \;=\; \MVector{30\\-150}
% $$
% durch komponentenweises Addieren der Vektoren.
% Die Länge dieses Vektors ist
% $$
% \sqrt{30^2+(-150)^2} \;=\; \sqrt{23400}\: .
% $$
% Als Zahlenwert ist das gerundet $152.9706$.
% \end{MHint}
\end{MExercise}

\begin{MExercise}
Gegeben sind die Punkte $P=\MPointTwo{3}{4}$, $Q=\MPointTwo{1}{0}$ und $R=\MPointTwo{-2}{1}$ in der Ebene. Berechnen Sie die folgenden Vektoren:
\begin{MExerciseItems}
\item{\MEquationItem{$\MDVec{P Q}$}{\MLFunctionQuestion{15}{(-2,-4)}{5}{x}{5}{OR4}}.}
\item{\MEquationItem{$\MDVec{Q R}$}{\MLFunctionQuestion{15}{(-3,1)}{5}{x}{5}{OR4a}}.}
\item{\MEquationItem{$\MDVec{R R}$}{\MLFunctionQuestion{15}{(0,0)}{5}{x}{5}{OR4b}}.}
\item{\MEquationItem{$\MDVec{Q P}$}{\MLFunctionQuestion{15}{(2,4)}{5}{x}{5}{OR4c}}.}
\item{\MEquationItem{$\MDVec{R P}$}{\MLFunctionQuestion{15}{(5,3)}{5}{x}{5}{OR4d}}.}
\end{MExerciseItems}
% \begin{MHint}{\iSolution}
% Die Richtungsvektoren ergeben sich aus der Differenz zwischen End- und Startkoordinaten der Punkte:
% \begin{eqnarray*}
% \MDVec{P Q} &=& \MVector{1\\0}-\MVector{3\\4} \;=\; \MVector{-2\\-4}\: ,\ \\
% \end{eqnarray*}
% \end{MHint}
\end{MExercise}


\begin{MExercise}
Gegeben sind die Punkte $P=\MPointThree{1}{2}{3}$, $Q=\MPointThree{3}{0}{0}$ und $R=\MPointThree{-1}{2}{2}$ im Raum. Berechnen Sie die folgenden Vektoren:
\begin{MExerciseItems}
\item{\MEquationItem{$\MDVec{P Q}$}{\MLFunctionQuestion{15}{(2,-2,-3)}{5}{x}{5}{FKT1}}.}
\item{\MEquationItem{$\MDVec{R Q}$}{\MLFunctionQuestion{15}{(4,-2,-2)}{5}{x}{5}{FKT2}}.}
\end{MExerciseItems}

Bestimmen Sie den Ortsvektor $\MDVec{M}$ des Mittelpunkts $M$ der Strecke $\overline{P R}$:\ \MEquationItem{$\MDVec{M}$}{\MLFunctionQuestion{15}{(0,2,2.5)}{5}{x}{5}{FKT3}}.
% LOESUNG STIMMT NOCH NICHT XXX !! Scheint jetzt OK. TS
\end{MExercise}

\begin{MExercise}
Finden Sie den Schnittpunkt $S$ der beiden in Punkt-Richtungsform gegebenen Geraden
$$
\MVec{r} \;=\; \MVector{1\\1\\2}+\alpha\cdot \MVector{1\\1\\1}\MDFPSpace;\MDFPaSpace\alpha\in\R \MBlank\MBlank \text{und} \MBlank\MBlank
\MVec{r} \;=\; \MVector{0\\1\\-2}+\beta\cdot \MVector{-2\\-4\\4}\MDFPSpace;\MDFPaSpace\beta\in\R \MDFPeriod
$$
\begin{MExerciseItems}
\item{Der Ortsvektor des Schnittpunkts ist $\MDVec{S}=$\MLFunctionQuestion{15}{(-1,-1,0)}{5}{x}{5}{FKT4}.}
\item{Man erhält ihn als Punkt der ersten Geraden für den Parameter \MEquationItem{$\alpha$}{\MLParsedQuestion{7}{-2}{3}{PARSEDQUEST10}}.}
\item{Man erhält ihn als Punkt der zweiten Geraden für den Parameter \MEquationItem{$\beta$}{\MLParsedQuestion{7}{1/2}{3}{PARSEDQUEST11}}.}
\end{MExerciseItems}
\end{MExercise}

\end{MTest}

\newpage
\MPrintIndex

\end{document}
