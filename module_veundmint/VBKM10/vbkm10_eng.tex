% MINTMOD Version P0.1.0, needs to be consistent with preprocesser object in tex2x and MPragma-Version at the end of this file

% Parameter aus Konvertierungsprozess (PDF und HTML-Erzeugung wenn vom Konverter aus gestartet) werden hier eingefuegt, Preambleincludes werden am Schluss angehaengt

\newif\ifttm                % gesetzt falls Uebersetzung in HTML stattfindet, sonst uebersetzung in PDF

% Wahl der Notationsvariante ist im PDF immer std, in der HTML-Uebersetzung wird vom Konverter die Auswahl modifiziert
\newif\ifvariantstd
\newif\ifvariantunotation
\variantstdtrue % Diese Zeile wird vom Konverter erkannt und ggf. modifiziert, daher nicht veraendern!


\def\MOutputDVI{1}
\def\MOutputPDF{2}
\def\MOutputHTML{3}
\newcounter{MOutput}

\ifttm
\usepackage{german}
\usepackage{array}
\usepackage{amsmath}
\usepackage{amssymb}
\usepackage{amsthm}
\else
\documentclass[ngerman,oneside]{scrbook}
\usepackage{etex}
\usepackage[latin1]{inputenc}
\usepackage{textcomp}
\usepackage[ngerman]{babel}
\usepackage[pdftex]{color}
\usepackage{xcolor}
\usepackage{graphicx}
\usepackage[all]{xy}
\usepackage{fancyhdr}
\usepackage{verbatim}
\usepackage{array}
\usepackage{float}
\usepackage{makeidx}
\usepackage{amsmath}
\usepackage{amstext}
\usepackage{amssymb}
\usepackage{amsthm}
\usepackage[ngerman]{varioref}
\usepackage{framed}
\usepackage{supertabular}
\usepackage{longtable}
\usepackage{maxpage}
\usepackage{tikz}
\usepackage{tikzscale}
\usepackage{tikz-3dplot}
\usepackage{bibgerm}
\usepackage{chemarrow}
\usepackage{polynom}
%\usepackage{draftwatermark}
\usepackage{pdflscape}
\usetikzlibrary{calc}
\usetikzlibrary{through}
\usetikzlibrary{shapes.geometric}
\usetikzlibrary{arrows}
\usetikzlibrary{intersections}
\usetikzlibrary{decorations.pathmorphing}
\usetikzlibrary{external}
\usetikzlibrary{patterns}
\usetikzlibrary{fadings}
\usepackage[colorlinks=true,linkcolor=blue]{hyperref} 
\usepackage[all]{hypcap}
%\usepackage[colorlinks=true,linkcolor=blue,bookmarksopen=true]{hyperref} 
\usepackage{ifpdf}

\usepackage{movie15}

\setcounter{tocdepth}{2} % In Inhaltsverzeichnis bis subsection
\setcounter{secnumdepth}{3} % Nummeriert bis subsubsection

\setlength{\LTpost}{0pt} % Fuer longtable
\setlength{\parindent}{0pt}
\setlength{\parskip}{8pt}
%\setlength{\parskip}{9pt plus 2pt minus 1pt}
\setlength{\abovecaptionskip}{-0.25ex}
\setlength{\belowcaptionskip}{-0.25ex}
\fi

\ifttm
\newcommand{\MDebugMessage}[1]{\special{html:<!-- debugprint;;}#1\special{html:; //-->}}
\else
%\newcommand{\MDebugMessage}[1]{\immediate\write\mintlog{#1}}
\newcommand{\MDebugMessage}[1]{}
\fi

\def\MPageHeaderDef{%
\pagestyle{fancy}%
\fancyhead[r]{(C) VE\&MINT-Projekt}
\fancyfoot[c]{\thepage\\--- CCL BY-SA 3.0 ---}
}


\ifttm%
\def\MRelax{}%
\else%
\def\MRelax{\relax}%
\fi%

%--------------------------- Uebernahme von speziellen XML-Versionen einiger LaTeX-Kommandos aus xmlbefehle.tex vom alten Kasseler Konverter ---------------

\newcommand{\MSep}{\left\|{\phantom{\frac1g}}\right.}

\newcommand{\ML}{L}

\newcommand{\MGGT}{\mathrm{ggT}}


\ifttm
% Verhindert dass die subsection-nummer doppelt in der toccaption auftaucht (sollte ggf. in toccaption gefixt werden so dass diese Ueberschreibung nicht notwendig ist)
\renewcommand{\thesubsection}{}
% Kommandos die ttm nicht kennt
\newcommand{\binomial}[2]{{#1 \choose #2}} %  Binomialkoeffizienten
\newcommand{\eur}{\begin{html}&euro;\end{html}}
\newcommand{\square}{\begin{html}&square;\end{html}}
\newcommand{\glqq}{"'}  \newcommand{\grqq}{"'}
\newcommand{\nRightarrow}{\special{html: &nrArr; }}
\newcommand{\nmid}{\special{html: &nmid; }}
\newcommand{\nparallel}{\begin{html}&nparallel;\end{html}}
\newcommand{\mapstoo}{\begin{html}<mo>&map;</mo>\end{html}}

% Schnitt und Vereinigungssymbole von Mengen haben zu kleine Abstaende; korrigiert:
\newcommand{\ccup}{\,\!\cup\,\!}
\newcommand{\ccap}{\,\!\cap\,\!}


% Umsetzung von mathbb im HTML
\renewcommand{\mathbb}[1]{\begin{html}<mo>&#1opf;</mo>\end{html}}
\fi

%---------------------- Strukturierung ----------------------------------------------------------------------------------------------------------------------

%---------------------- Kapselung des sectioning findet auf drei Ebenen statt:
% 1. Die LateX-Befehl
% 2. Die D-Versionen der Befehle, die nur die Grade der Abschnitte umhaengen falls notwendig
% 3. Die M-Versionen der Befehle, die zusaetzliche Formatierungen vornehmen, Skripten starten und das HTML codieren
% Im Modultext duerfen nur die M-Befehle verwendet werden!

\ifttm

  \def\Dsubsubsubsection#1{\subsubsubsection{#1}}
  \def\Dsubsubsection#1{\subsubsection{#1}\addtocounter{subsubsection}{1}} % ttm-Fehler korrigieren
  \def\Dsubsection#1{\subsection{#1}}
  \def\Dsection#1{\section{#1}} % Im HTML wird nur der Sektionstitel gegeben
  \def\Dchapter#1{\chapter{#1}}
  \def\Dsubsubsubsectionx#1{\subsubsubsection*{#1}}
  \def\Dsubsubsectionx#1{\subsubsection*{#1}}
  \def\Dsubsectionx#1{\subsection*{#1}}
  \def\Dsectionx#1{\section*{#1}}
  \def\Dchapterx#1{\chapter*{#1}}

\else

  \def\Dsubsubsubsection#1{\subsubsection{#1}}
  \def\Dsubsubsection#1{\subsection{#1}}
  \def\Dsubsection#1{\section{#1}}
  \def\Dsection#1{\chapter{#1}}
  \def\Dchapter#1{\title{#1}}
  \def\Dsubsubsubsectionx#1{\subsubsection*{#1}}
  \def\Dsubsubsectionx#1{\subsection*{#1}}
  \def\Dsubsectionx#1{\section*{#1}}
  \def\Dsectionx#1{\chapter*{#1}}

\fi

\newcommand{\MStdPoints}{4}
\newcommand{\MSetPoints}[1]{\renewcommand{\MStdPoints}{#1}}

% Befehl zum Abbruch der Erstellung (nur PDF)
\newcommand{\MAbort}[1]{\err{#1}}

% Prefix vor Dateieinbindungen, wird in der Baumdatei mit \renewcommand modifiziert
% und auf das Verzeichnisprefix gesetzt, in dem das gerade bearbeitete tex-Dokument liegt.
% Im HTML wird es auf das Verzeichnis der HTML-Datei gesetzt.
% Das Prefix muss mit / enden !
\newcommand{\MDPrefix}{.}

% MRegisterFile notiert eine Datei zur Einbindung in den HTML-Baum. Grafiken mit MGraphics werden automatisch eingebunden.
% Mit MLastFile erhaelt man eine Markierung fuer die zuletzt registrierte Datei.
% Diese Markierung wird im postprocessing durch den physikalischen Dateinamen ersetzt, aber nur den Namen (d.h. \MMaterial gehoert noch davor, vgl Definition von MGraphics)
% Parameter: Pfad/Name der Datei bzw. des Ordners, relativ zur Position des Modul-Tex-Dokuments.
\ifttm
\newcommand{\MRegisterFile}[1]{\addtocounter{MFileNumber}{1}\special{html:<!-- registerfile;;}#1\special{html:;;}\MDPrefix\special{html:;;}\arabic{MFileNumber}\special{html:; //-->}}
\else
\newcommand{\MRegisterFile}[1]{\addtocounter{MFileNumber}{1}}
\fi

% Testen welcher Uebersetzer hier am Werk ist

\ifttm
\setcounter{MOutput}{3}
\else
\ifx\pdfoutput\undefined
  \pdffalse
  \setcounter{MOutput}{\MOutputDVI}
  \message{Verarbeitung mit latex, Ausgabe in dvi.}
\else
  \setcounter{MOutput}{\MOutputPDF}
  \message{Verarbeitung mit pdflatex, Ausgabe in pdf.}
  \ifnum \pdfoutput=0
    \pdffalse
  \setcounter{MOutput}{\MOutputDVI}
  \message{Verarbeitung mit pdflatex, Ausgabe in dvi.}
  \else
    \ifnum\pdfoutput=1
    \pdftrue
  \setcounter{MOutput}{\MOutputPDF}
  \message{Verarbeitung mit pdflatex, Ausgabe in pdf.}
    \fi
  \fi
\fi
\fi

\ifnum\value{MOutput}=\MOutputPDF
\DeclareGraphicsExtensions{.pdf,.png,.jpg}
\fi

\ifnum\value{MOutput}=\MOutputDVI
\DeclareGraphicsExtensions{.eps,.png,.jpg}
\fi

\ifnum\value{MOutput}=\MOutputHTML
% Wird vom Konverter leider nicht erkannt und daher in split.pm hardcodiert!
\DeclareGraphicsExtensions{.png,.jpg,.gif}
\fi

% Umdefinition der hyperref-Nummerierung im PDF-Modus
\ifttm
\else
\renewcommand{\theHfigure}{\arabic{chapter}.\arabic{section}.\arabic{figure}}
\fi

% Makro, um in der HTML-Ausgabe die zuerst zu oeffnende Datei zu kennzeichnen
\ifttm
\newcommand{\MGlobalStart}{\special{html:<!-- mglobalstarttag -->}}
\else
\newcommand{\MGlobalStart}{}
\fi

% Makro, um bei scormlogin ein pullen des Benutzers bei Aufruf der Seite zu erzwingen (typischerweise auf der Einstiegsseite)
\ifttm
\newcommand{\MPullSite}{\special{html:<!-- pullsite //-->}}
\else
\newcommand{\MPullSite}{}
\fi

% Makro, um in der HTML-Ausgabe die Kapiteluebersicht zu kennzeichnen
\ifttm
\newcommand{\MGlobalChapterTag}{\special{html:<!-- mglobalchaptertag -->}}
\else
\newcommand{\MGlobalChapterTag}{}
\fi

% Makro, um in der HTML-Ausgabe die Konfiguration zu kennzeichnen
\ifttm
\newcommand{\MGlobalConfTag}{\special{html:<!-- mglobalconfigtag -->}}
\else
\newcommand{\MGlobalConfTag}{}
\fi

% Makro, um in der HTML-Ausgabe die Standortbeschreibung zu kennzeichnen
\ifttm
\newcommand{\MGlobalLocationTag}{\special{html:<!-- mgloballocationtag -->}}
\else
\newcommand{\MGlobalLocationTag}{}
\fi

% Makro, um in der HTML-Ausgabe die persoenlichen Daten zu kennzeichnen
\ifttm
\newcommand{\MGlobalDataTag}{\special{html:<!-- mglobaldatatag -->}}
\else
\newcommand{\MGlobalDataTag}{}
\fi

% Makro, um in der HTML-Ausgabe die Suchseite zu kennzeichnen
\ifttm
\newcommand{\MGlobalSearchTag}{\special{html:<!-- mglobalsearchtag -->}}
\else
\newcommand{\MGlobalSearchTag}{}
\fi

% Makro, um in der HTML-Ausgabe die Favoritenseite zu kennzeichnen
\ifttm
\newcommand{\MGlobalFavoTag}{\special{html:<!-- mglobalfavoritestag -->}}
\else
\newcommand{\MGlobalFavoTag}{}
\fi

% Makro, um in der HTML-Ausgabe die Eingangstestseite zu kennzeichnen
\ifttm
\newcommand{\MGlobalSTestTag}{\special{html:<!-- mglobalstesttag -->}}
\else
\newcommand{\MGlobalSTestTag}{}
\fi

% Makro, um in der PDF-Ausgabe ein Wasserzeichen zu definieren
\ifttm
\newcommand{\MWatermarkSettings}{\relax}
\else
\newcommand{\MWatermarkSettings}{%
% \SetWatermarkText{(c) MINT-Kolleg Baden-W�rttemberg 2014}
% \SetWatermarkLightness{0.85}
% \SetWatermarkScale{1.5}
}
\fi

\ifttm
\newcommand{\MBinom}[2]{\left({\begin{array}{c} #1 \\ #2 \end{array}}\right)}
\else
\newcommand{\MBinom}[2]{\binom{#1}{#2}}
\fi

\ifttm
\newcommand{\DeclareMathOperator}[2]{\def#1{\mathrm{#2}}}
\newcommand{\operatorname}[1]{\mathrm{#1}}
\fi

%----------------- Makros fuer die gemischte HTML/PDF-Konvertierung ------------------------------

\newcommand{\MTestName}{\relax} % wird durch Test-Umgebung gesetzt

% Fuer experimentelle Kursinhalte, die im Release-Umsetzungsvorgang eine Fehlermeldung
% produzieren sollen aber sonst normal umgesetzt werden
\newenvironment{MExperimental}{%
}{%
}

% Wird von ttm nicht richtig umgesetzt!!
\newenvironment{MExerciseItems}{%
\renewcommand\theenumi{\alph{enumi}}%
\begin{enumerate}%
}{%
\end{enumerate}%
}


\definecolor{infoshadecolor}{rgb}{0.75,0.75,0.75}
\definecolor{exmpshadecolor}{rgb}{0.875,0.875,0.875}
\definecolor{expeshadecolor}{rgb}{0.95,0.95,0.95}
\definecolor{framecolor}{rgb}{0.2,0.2,0.2}

% Bei PDF-Uebersetzung wird hinter den Start jeder Satz/Info-aehnlichen Umgebung eine leere mbox gesetzt, damit
% fuehrende Listen oder enums nicht den Zeilenumbruch kaputtmachen
%\ifttm
\def\MTB{}
%\else
%\def\MTB{\mbox{}}
%\fi


\ifttm
\newcommand{\MRelates}{\special{html:<mi>&wedgeq;</mi>}}
\else
\def\MRelates{\stackrel{\scriptscriptstyle\wedge}{=}}
\fi

\def\MInch{\text{''}}
\def\Mdd{\textit{''}}

\ifttm
\def\MNL{ \newline }
\newenvironment{MArray}[1]{\begin{array}{#1}}{\end{array}}
\else
\def\MNL{ \\ }
\newenvironment{MArray}[1]{\begin{array}{#1}}{\end{array}}
\fi

\newcommand{\MBox}[1]{$\mathrm{#1}$}
\newcommand{\MMBox}[1]{\mathrm{#1}}


\ifttm%
\newcommand{\Mtfrac}[2]{{\textstyle \frac{#1}{#2}}}
\newcommand{\Mdfrac}[2]{{\displaystyle \frac{#1}{#2}}}
\newcommand{\Mmeasuredangle}{\special{html:<mi>&angmsd;</mi>}}
\else%
\newcommand{\Mtfrac}[2]{\tfrac{#1}{#2}}
\newcommand{\Mdfrac}[2]{\dfrac{#1}{#2}}
\newcommand{\Mmeasuredangle}{\measuredangle}
\relax
\fi

% Matrizen und Vektoren

% Inhalt wird in der Form a & b \\ c & d erwartet
% Vorsicht: MVector = Komponentenspalte, MVec = Variablensymbol
\ifttm%
\newcommand{\MVector}[1]{\left({\begin{array}{c}#1\end{array}}\right)}
\else%
\newcommand{\MVector}[1]{\begin{pmatrix}#1\end{pmatrix}}
\fi



\newcommand{\MVec}[1]{\vec{#1}}
\newcommand{\MDVec}[1]{\overrightarrow{#1}}

%----------------- Umgebungen fuer Definitionen und Saetze ----------------------------------------

% Fuegt einen Tabellen-Zeilenumbruch ein im PDF, aber nicht im HTML
\newcommand{\TSkip}{\ifttm \else&\ \\\fi}

\newenvironment{infoshaded}{%
\def\FrameCommand{\fboxsep=\FrameSep \fcolorbox{framecolor}{infoshadecolor}}%
\MakeFramed {\advance\hsize-\width \FrameRestore}}%
{\endMakeFramed}

\newenvironment{expeshaded}{%
\def\FrameCommand{\fboxsep=\FrameSep \fcolorbox{framecolor}{expeshadecolor}}%
\MakeFramed {\advance\hsize-\width \FrameRestore}}%
{\endMakeFramed}

\newenvironment{exmpshaded}{%
\def\FrameCommand{\fboxsep=\FrameSep \fcolorbox{framecolor}{exmpshadecolor}}%
\MakeFramed {\advance\hsize-\width \FrameRestore}}%
{\endMakeFramed}

\def\STDCOLOR{black}

\ifttm%
\else%
\newtheoremstyle{MSatzStyle}
  {1cm}                   %Space above
  {1cm}                   %Space below
  {\normalfont\itshape}   %Body font
  {}                      %Indent amount (empty = no indent,
                          %\parindent = para indent)
  {\normalfont\bfseries}  %Thm head font
  {}                      %Punctuation after thm head
  {\newline}              %Space after thm head: " " = normal interword
                          %space; \newline = linebreak
  {\thmname{#1}\thmnumber{ #2}\thmnote{ (#3)}}
                          %Thm head spec (can be left empty, meaning
                          %`normal')
                          %
\newtheoremstyle{MDefStyle}
  {1cm}                   %Space above
  {1cm}                   %Space below
  {\normalfont}           %Body font
  {}                      %Indent amount (empty = no indent,
                          %\parindent = para indent)
  {\normalfont\bfseries}  %Thm head font
  {}                      %Punctuation after thm head
  {\newline}              %Space after thm head: " " = normal interword
                          %space; \newline = linebreak
  {\thmname{#1}\thmnumber{ #2}\thmnote{ (#3)}}
                          %Thm head spec (can be left empty, meaning
                          %`normal')
\fi%

\newcommand{\MInfoText}{Info}

\newcounter{MHintCounter}
\newcounter{MCodeEditCounter}

\newcounter{MLastIndex}  % Enthaelt die dritte Stelle (Indexnummer) des letzten angelegten Objekts
\newcounter{MLastType}   % Enthaelt den Typ des letzten angelegten Objekts (mithilfe der unten definierten Konstanten). Die Entscheidung, wie der Typ dargstellt wird, wird in split.pm beim Postprocessing getroffen.
\newcounter{MLastTypeEq} % =1 falls das Label in einer Matheumgebung (equation, eqnarray usw.) steht, =2 falls das Label in einer table-Umgebung steht

% Da ttm keine Zahlmakros verarbeiten kann, werden diese Nummern in den Zuweisungen hardcodiert!
\def\MTypeSection{1}          %# Zaehler ist section
\def\MTypeSubsection{2}       %# Zaehler ist subsection
\def\MTypeSubsubsection{3}    %# Zaehler ist subsubsection
\def\MTypeInfo{4}             %# Eine Infobox, Separatzaehler fuer die Chemie (auch wenn es dort nicht nummeriert wird) ist MInfoCounter
\def\MTypeExercise{5}         %# Eine Aufgabe, Separatzaehler fuer die Chemie ist MExerciseCounter
\def\MTypeExample{6}          %# Eine Beispielbox, Separatzaehler fuer die Chemie ist MExampleCounter
\def\MTypeExperiment{7}       %# Eine Versuchsbox, Separatzaehler fuer die Chemie ist MExperimentCounter
\def\MTypeGraphics{8}         %# Eine Graphik, Separatzaehler fuer alle FB ist MGraphicsCounter
\def\MTypeTable{9}            %# Eine Tabellennummer, hat keinen Zaehler da durch table gezaehlt wird
\def\MTypeEquation{10}        %# Eine Gleichungsnummer, hat keinen Zaehler da durch equation/eqnarray gezaehlt wird
\def\MTypeTheorem{11}         % Ein theorem oder xtheorem, Separatzaehler fuer die Chemie ist MTheoremCounter
\def\MTypeVideo{12}           %# Ein Video,Separatzaehler fuer alle FB ist MVideoCounter
\def\MTypeEntry{13}           %# Ein Eintrag fuer die Stichwortliste, wird nicht gezaehlt sondern erhaelt im preparsing ein unique-label 

% Zaehler fuer das Labelsystem sind prefixcounter, jeder Zaehler wird VOR dem gezaehlten Objekt inkrementiert und zaehlt daher das aktuelle Objekt
\newcounter{MInfoCounter}
\newcounter{MExerciseCounter}
\newcounter{MExampleCounter}
\newcounter{MExperimentCounter}
\newcounter{MGraphicsCounter}
\newcounter{MTableCounter}
\newcounter{MEquationCounter}  % Nur im HTML, sonst durch "equation"-counter von latex realisiert
\newcounter{MTheoremCounter}
\newcounter{MObjectCounter}   % Gemeinsamer Zaehler fuer Objekte (ausser Grafiken/Tabellen) in Mathe/Info/Physik
\newcounter{MVideoCounter}
\newcounter{MEntryCounter}

\newcounter{MTestSite} % 1 = Subsubsection ist eine Pruefungsseite, 0 = ist eine normale Seite (inkl. Hilfeseite)

\def\MCell{$\phantom{a}$}

\newenvironment{MExportExercise}{\begin{MExercise}}{\end{MExercise}} % wird von mconvert abgefangen

\def\MGenerateExNumber{%
\ifnum\value{MSepNumbers}=0%
\arabic{section}.\arabic{subsection}.\arabic{MObjectCounter}\setcounter{MLastIndex}{\value{MObjectCounter}}%
\else%
\arabic{section}.\arabic{subsection}.\arabic{MExerciseCounter}\setcounter{MLastIndex}{\value{MExerciseCounter}}%
\fi%
}%

\def\MGenerateExmpNumber{%
\ifnum\value{MSepNumbers}=0%
\arabic{section}.\arabic{subsection}.\arabic{MObjectCounter}\setcounter{MLastIndex}{\value{MObjectCounter}}%
\else%
\arabic{section}.\arabic{subsection}.\arabic{MExerciseCounter}\setcounter{MLastIndex}{\value{MExampleCounter}}%
\fi%
}%

\def\MGenerateInfoNumber{%
\ifnum\value{MSepNumbers}=0%
\arabic{section}.\arabic{subsection}.\arabic{MObjectCounter}\setcounter{MLastIndex}{\value{MObjectCounter}}%
\else%
\arabic{section}.\arabic{subsection}.\arabic{MExerciseCounter}\setcounter{MLastIndex}{\value{MInfoCounter}}%
\fi%
}%

\def\MGenerateSiteNumber{%
\arabic{section}.\arabic{subsection}.\arabic{subsubsection}%
}%

% Funktionalitaet fuer Auswahlaufgaben

\newcounter{MExerciseCollectionCounter} % = 0 falls nicht in collection-Umgebung, ansonsten Schachtelungstiefe
\newcounter{MExerciseCollectionTextCounter} % wird von MExercise-Umgebung inkrementiert und von MExerciseCollection-Umgebung auf Null gesetzt

\ifttm
% MExerciseCollection gruppiert Aufgaben, die dynamisch aus der Datenbank gezogen werden und nicht direkt in der HTML-Seite stehen
% Parameter: #1 = ID der Collection, muss eindeutig fuer alle IN DER DB VORHANDENEN collections sein unabhaengig vom Kurs
%            #2 = Optionsargument (im Moment: 1 = Iterative Auswahl, 2 = Zufallsbasierte Auswahl)
\newenvironment{MExerciseCollection}[2]{%
\addtocounter{MExerciseCollectionCounter}{1}
\setcounter{MExerciseCollectionTextCounter}{0}
\special{html:<!-- mexercisecollectionstart;;}#1\special{html:;;}#2\special{html:;; //-->}%
}{%
\special{html:<!-- mexercisecollectionstop //-->}%
\addtocounter{MExerciseCollectionCounter}{-1}
}
\else
\newenvironment{MExerciseCollection}[2]{%
\addtocounter{MExerciseCollectionCounter}{1}
\setcounter{MExerciseCollectionTextCounter}{0}
}{%
\addtocounter{MExerciseCollectionCounter}{-1}
}
\fi

% Bei Uebersetzung nach PDF werden die theorem-Umgebungen verwendet, bei Uebersetzung in HTML ein manuelles Makro
\ifttm%

  \newenvironment{MHint}[1]{  \special{html:<button name="Name_MHint}\arabic{MHintCounter}\special{html:" class="hintbutton_closed" id="MHint}\arabic{MHintCounter}\special{html:_button" %
  type="button" onclick="toggle_hint('MHint}\arabic{MHintCounter}\special{html:');">}#1\special{html:</button>}
  \special{html:<div class="hint" style="display:none" id="MHint}\arabic{MHintCounter}\special{html:"> }}{\begin{html}</div>\end{html}\addtocounter{MHintCounter}{1}}

  \newenvironment{MCOSHZusatz}{  \special{html:<button name="Name_MHint}\arabic{MHintCounter}\special{html:" class="chintbutton_closed" id="MHint}\arabic{MHintCounter}\special{html:_button" %
  type="button" onclick="toggle_hint('MHint}\arabic{MHintCounter}\special{html:');">}Weiterf�hrende Inhalte\special{html:</button>}
  \special{html:<div class="hintc" style="display:none" id="MHint}\arabic{MHintCounter}\special{html:">
  <div class="coshwarn">Diese Inhalte gehen �ber das Kursniveau hinaus und werden in den Aufgaben und Tests nicht abgefragt.</div><br />}
  \addtocounter{MHintCounter}{1}}{\begin{html}</div>\end{html}}

  
  \newenvironment{MDefinition}{\begin{definition}\setcounter{MLastIndex}{\value{definition}}\ \\}{\end{definition}}

  
  \newenvironment{MExercise}{
  \renewcommand{\MStdPoints}{4}
  \addtocounter{MExerciseCounter}{1}
  \addtocounter{MObjectCounter}{1}
  \setcounter{MLastType}{5}

  \ifnum\value{MExerciseCollectionCounter}=0\else\addtocounter{MExerciseCollectionTextCounter}{1}\special{html:<!-- mexercisetextstart;;}\arabic{MExerciseCollectionTextCounter}\special{html:;; //-->}\fi
  \special{html:<div class="aufgabe" id="ADIV_}\MGenerateExNumber\special{html:">}%
  \textbf{Aufgabe \MGenerateExNumber
  } \ \\}{
  \special{html:</div><!-- mfeedbackbutton;Aufgabe;}\arabic{MTestSite}\special{html:;}\MGenerateExNumber\special{html:; //-->}
  \ifnum\value{MExerciseCollectionCounter}=0\else\special{html:<!-- mexercisetextstop //-->}\fi
  }

  % Stellt eine Kombination aus Aufgabe, Loesungstext und Eingabefeld bereit,
  % bei der Aufgabentext und Musterloesung sowie die zugehoerigen Feldelemente
  % extern bezogen und div-aktualisiert werden, das Eingabefeld aber immer das gleiche ist.
  \newenvironment{MFetchExercise}{
  \addtocounter{MExerciseCounter}{1}
  \addtocounter{MObjectCounter}{1}
  \setcounter{MLastType}{5}

  \special{html:<div class="aufgabe" id="ADIV_}\MGenerateExNumber\special{html:">}%
  \textbf{Aufgabe \MGenerateExNumber
  } \ \\%
  \special{html:</div><div class="exfetch_text" id="ADIVTEXT_}\MGenerateExNumber\special{html:">}%
  \special{html:</div><div class="exfetch_sol" id="ADIVSOL_}\MGenerateExNumber\special{html:">}%
  \special{html:</div><div class="exfetch_input" id="ADIVINPUT_}\MGenerateExNumber\special{html:">}%
  }{
  \special{html:</div>}
  }

  \newenvironment{MExample}{
  \addtocounter{MExampleCounter}{1}
  \addtocounter{MObjectCounter}{1}
  \setcounter{MLastType}{6}
  \begin{html}
  <div class="exmp">
  <div class="exmprahmen">
  \end{html}\textbf{Beispiel
  \ifnum\value{MSepNumbers}=0
  \arabic{section}.\arabic{subsection}.\arabic{MObjectCounter}\setcounter{MLastIndex}{\value{MObjectCounter}}
  \else
  \arabic{section}.\arabic{subsection}.\arabic{MExampleCounter}\setcounter{MLastIndex}{\value{MExampleCounter}}
  \fi
  } \ \\}{\begin{html}</div>
  </div>
  \end{html}
  \special{html:<!-- mfeedbackbutton;Beispiel;}\arabic{MTestSite}\special{html:;}\MGenerateExmpNumber\special{html:; //-->}
  }

  \newenvironment{MExperiment}{
  \addtocounter{MExperimentCounter}{1}
  \addtocounter{MObjectCounter}{1}
  \setcounter{MLastType}{7}
  \begin{html}
  <div class="expe">
  <div class="experahmen">
  \end{html}\textbf{Versuch
  \ifnum\value{MSepNumbers}=0
  \arabic{section}.\arabic{subsection}.\arabic{MObjectCounter}\setcounter{MLastIndex}{\value{MObjectCounter}}
  \else
%  \arabic{MExperimentCounter}\setcounter{MLastIndex}{\value{MExperimentCounter}}
  \arabic{section}.\arabic{subsection}.\arabic{MExperimentCounter}\setcounter{MLastIndex}{\value{MExperimentCounter}}
  \fi
  } \ \\}{\begin{html}</div>
  </div>
  \end{html}}

  \newenvironment{MChemInfo}{
  \setcounter{MLastType}{4}
  \begin{html}
  <div class="info">
  <div class="inforahmen">
  \end{html}}{\begin{html}</div>
  </div>
  \end{html}}

  \newenvironment{MXInfo}[1]{
  \addtocounter{MInfoCounter}{1}
  \addtocounter{MObjectCounter}{1}
  \setcounter{MLastType}{4}
  \begin{html}
  <div class="info">
  <div class="inforahmen">
  \end{html}\textbf{#1
  \ifnum\value{MInfoNumbers}=0
  \else
    \ifnum\value{MSepNumbers}=0
    \arabic{section}.\arabic{subsection}.\arabic{MObjectCounter}\setcounter{MLastIndex}{\value{MObjectCounter}}
    \else
    \arabic{MInfoCounter}\setcounter{MLastIndex}{\value{MInfoCounter}}
    \fi
  \fi
  } \ \\}{\begin{html}</div>
  </div>
  \end{html}
  \special{html:<!-- mfeedbackbutton;Info;}\arabic{MTestSite}\special{html:;}\MGenerateInfoNumber\special{html:; //-->}
  }

  \newenvironment{MInfo}{\ifnum\value{MInfoNumbers}=0\begin{MChemInfo}\else\begin{MXInfo}{Info}\ \\ \fi}{\ifnum\value{MInfoNumbers}=0\end{MChemInfo}\else\end{MXInfo}\fi}

\else%

  \theoremstyle{MSatzStyle}
  \newtheorem{thm}{Satz}[section]
  \newtheorem{thmc}{Satz}
  \theoremstyle{MDefStyle}
  \newtheorem{defn}[thm]{Definition}
  \newtheorem{exmp}[thm]{Beispiel}
  \newtheorem{info}[thm]{\MInfoText}
  \theoremstyle{MDefStyle}
  \newtheorem{defnc}{Definition}
  \theoremstyle{MDefStyle}
  \newtheorem{exmpc}{Beispiel}[section]
  \theoremstyle{MDefStyle}
  \newtheorem{infoc}{\MInfoText}
  \theoremstyle{MDefStyle}
  \newtheorem{exrc}{Aufgabe}[section]
  \theoremstyle{MDefStyle}
  \newtheorem{verc}{Versuch}[section]
  
  \newenvironment{MFetchExercise}{}{} % kann im PDF nicht dargestellt werden
  
  \newenvironment{MExercise}{\begin{exrc}\renewcommand{\MStdPoints}{1}\MTB}{\end{exrc}}
  \newenvironment{MHint}[1]{\ \\ \underline{#1:}\\}{}
  \newenvironment{MCOSHZusatz}{\ \\ \underline{Weiterf�hrende Inhalte:}\\}{}
  \newenvironment{MDefinition}{\ifnum\value{MInfoNumbers}=0\begin{defnc}\else\begin{defn}\fi\MTB}{\ifnum\value{MInfoNumbers}=0\end{defnc}\else\end{defn}\fi}
%  \newenvironment{MExample}{\begin{exmp}}{\ \linebreak[1] \ \ \ \ $\phantom{a}$ \ \hfill $\blacklozenge$\end{exmp}}
  \newenvironment{MExample}{
    \ifnum\value{MInfoNumbers}=0\begin{exmpc}\else\begin{exmp}\fi
    \MTB
    \begin{exmpshaded}
    \ \newline
}{
    \end{exmpshaded}
    \ifnum\value{MInfoNumbers}=0\end{exmpc}\else\end{exmp}\fi
}
  \newenvironment{MChemInfo}{\begin{infoshaded}}{\end{infoshaded}}

  \newenvironment{MInfo}{\ifnum\value{MInfoNumbers}=0\begin{MChemInfo}\else\renewcommand{\MInfoText}{Info}\begin{info}\begin{infoshaded}
  \MTB
   \ \newline
    \fi
  }{\ifnum\value{MInfoNumbers}=0\end{MChemInfo}\else\end{infoshaded}\end{info}\fi}

  \newenvironment{MXInfo}[1]{
    \renewcommand{\MInfoText}{#1}
    \ifnum\value{MInfoNumbers}=0\begin{infoc}\else\begin{info}\fi%
    \MTB
    \begin{infoshaded}
    \ \newline
  }{\end{infoshaded}\ifnum\value{MInfoNumbers}=0\end{infoc}\else\end{info}\fi}

  \newenvironment{MExperiment}{
    \renewcommand{\MInfoText}{Versuch}
    \ifnum\value{MInfoNumbers}=0\begin{verc}\else\begin{info}\fi
    \MTB
    \begin{expeshaded}
    \ \newline
  }{
    \end{expeshaded}
    \ifnum\value{MInfoNumbers}=0\end{verc}\else\end{info}\fi
  }
\fi%

% MHint sollte nicht direkt fuer Loesungen benutzt werden wegen solutionselect
\newenvironment{MSolution}{\begin{MHint}{L"osung}}{\end{MHint}}

\newcounter{MCodeCounter}

\ifttm
\newenvironment{MCode}{\special{html:<!-- mcodestart -->}\ttfamily\color{blue}}{\special{html:<!-- mcodestop -->}}
\else
\newenvironment{MCode}{\begin{flushleft}\ttfamily\addtocounter{MCodeCounter}{1}}{\addtocounter{MCodeCounter}{-1}\end{flushleft}}
% Ohne color-Statement da inkompatible mit framed/shaded-Boxen aus dem framed-package
\fi

%----------------- Sonderdefinitionen fuer Symbole, die der Konverter nicht kann ----------------------------------------------

\ifttm%
\newcommand{\MUnderset}[2]{\underbrace{#2}_{#1}}%
\else%
\newcommand{\MUnderset}[2]{\underset{#1}{#2}}%
\fi%

\ifttm
\newcommand{\MThinspace}{\special{html:<mi>&#x2009;</mi>}}
\else
\newcommand{\MThinspace}{\,}
\fi

\ifttm
\newcommand{\glq}{\begin{html}&sbquo;\end{html}}
\newcommand{\grq}{\begin{html}&lsquo;\end{html}}
\newcommand{\glqq}{\begin{html}&bdquo;\end{html}}
\newcommand{\grqq}{\begin{html}&ldquo;\end{html}}
\fi

\ifttm
\newcommand{\MNdash}{\begin{html}&ndash;\end{html}}
\else
\newcommand{\MNdash}{--}
\fi

%\ifttm\def\MIU{\special{html:<mi>&#8520;</mi>}}\else\def\MIU{\mathrm{i}}\fi
\def\MIU{\mathrm{i}}
\def\MEU{e} % TU9-Onlinekurs: italic-e
%\def\MEU{\mathrm{e}} % Alte Onlinemodule: roman-e
\def\MD{d} % Kursives d in Integralen im TU9-Onlinekurs
%\def\MD{\mathrm{d}} % roman-d in den alten Onlinemodulen
\def\MDB{\|}

%zusaetzlicher Leerraum vor "\MD"
\ifttm%
\def\MDSpace{\special{html:<mi>&#x2009;</mi>}}
\else%
\def\MDSpace{\,}
\fi%
\newcommand{\MDwSp}{\MDSpace\MD}%

\ifttm
\def\Mdq{\dq}
\else
\def\Mdq{\dq}
\fi

\def\MSpan#1{\left<{#1}\right>}
\def\MSetminus{\setminus}
\def\MIM{I}

\ifttm
\newcommand{\ld}{\text{ld}}
\newcommand{\lg}{\text{lg}}
\else
\DeclareMathOperator{\ld}{ld}
%\newcommand{\lg}{\text{lg}} % in latex schon definiert
\fi


\def\Mmapsto{\ifttm\special{html:<mi>&mapsto;</mi>}\else\mapsto\fi} 
\def\Mvarphi{\ifttm\phi\else\varphi\fi}
\def\Mphi{\ifttm\varphi\else\phi\fi}
\ifttm%
\newcommand{\MEumu}{\special{html:<mi>&#x3BC;</mi>}}%
\else%
\newcommand{\MEumu}{\textrm{\textmu}}%
\fi
\def\Mvarepsilon{\ifttm\epsilon\else\varepsilon\fi}
\def\Mepsilon{\ifttm\varepsilon\else\epsilon\fi}
\def\Mvarkappa{\ifttm\kappa\else\varkappa\fi}
\def\Mkappa{\ifttm\varkappa\else\kappa\fi}
\def\Mcomplement{\ifttm\special{html:<mi>&comp;</mi>}\else\complement\fi} 
\def\MWW{\mathrm{WW}}
\def\Mmod{\ifttm\special{html:<mi>&nbsp;mod&nbsp;</mi>}\else\mod\fi} 

\ifttm%
\def\mod{\text{\;mod\;}}%
\def\MNEquiv{\special{html:<mi>&NotCongruent;</mi>}}% 
\def\MNSubseteq{\special{html:<mi>&NotSubsetEqual;</mi>}}%
\def\MEmptyset{\special{html:<mi>&empty;</mi>}}%
\def\MVDots{\special{html:<mi>&#x22EE;</mi>}}%
\def\MHDots{\special{html:<mi>&#x2026;</mi>}}%
\def\Mddag{\special{html:<mi>&#x1202;</mi>}}%
\def\sphericalangle{\special{html:<mi>&measuredangle;</mi>}}%
\def\nparallel{\special{html:<mi>&nparallel;</mi>}}%
\def\MProofEnd{\special{html:<mi>&#x25FB;</mi>}}%
\newenvironment{MProof}[1]{\underline{#1}:\MCR\MCR}{\hfill $\MProofEnd$}%
\else%
\def\MNEquiv{\not\equiv}%
\def\MNSubseteq{\not\subseteq}%
\def\MEmptyset{\emptyset}%
\def\MVDots{\vdots}%
\def\MHDots{\hdots}%
\def\Mddag{\ddag}%
\newenvironment{MProof}[1]{\begin{proof}[#1]}{\end{proof}}%
\fi%



% Spaces zum Auffuellen von Tabellenbreiten, die nur im HTML wirken
\ifttm%
\def\MTSP{\:}%
\else%
\def\MTSP{}%
\fi%

\DeclareMathOperator{\arsinh}{arsinh}
\DeclareMathOperator{\arcosh}{arcosh}
\DeclareMathOperator{\artanh}{artanh}
\DeclareMathOperator{\arcoth}{arcoth}


\newcommand{\MMathSet}[1]{\mathbb{#1}}
\def\N{\MMathSet{N}}
\def\Z{\MMathSet{Z}}
\def\Q{\MMathSet{Q}}
\def\R{\MMathSet{R}}
\def\C{\MMathSet{C}}

\newcounter{MForLoopCounter}
\newcommand{\MForLoop}[2]{\setcounter{MForLoopCounter}{#1}\ifnum\value{MForLoopCounter}=0{}\else{{#2}\addtocounter{MForLoopCounter}{-1}\MForLoop{\value{MForLoopCounter}}{#2}}\fi}

\newcounter{MSiteCounter}
\newcounter{MFieldCounter} % Kombination section.subsection.site.field ist eindeutig in allen Modulen, field alleine nicht

\newcounter{MiniMarkerCounter}

\ifttm
\newenvironment{MMiniPageP}[1]{\begin{minipage}{#1\linewidth}\special{html:<!-- minimarker;;}\arabic{MiniMarkerCounter}\special{html:;;#1; //-->}}{\end{minipage}\addtocounter{MiniMarkerCounter}{1}}
\else
\newenvironment{MMiniPageP}[1]{\begin{minipage}{#1\linewidth}}{\end{minipage}\addtocounter{MiniMarkerCounter}{1}}
\fi

\newcounter{AlignCounter}

\newcommand{\MStartJustify}{\ifttm\special{html:<!-- startalign;;}\arabic{AlignCounter}\special{html:;;justify; //-->}\fi}
\newcommand{\MStopJustify}{\ifttm\special{html:<!-- stopalign;;}\arabic{AlignCounter}\special{html:; //-->}\fi\addtocounter{AlignCounter}{1}}

\newenvironment{MJTabular}[1]{
\MStartJustify
\begin{tabular}{#1}
}{
\end{tabular}
\MStopJustify
}

\newcommand{\MImageLeft}[2]{
\begin{center}
\begin{tabular}{lc}
\MStartJustify
\begin{MMiniPageP}{0.65}
#1
\end{MMiniPageP}
\MStopJustify
&
\begin{MMiniPageP}{0.3}
#2  
\end{MMiniPageP}
\end{tabular}
\end{center}
}

\newcommand{\MImageHalf}[2]{
\begin{center}
\begin{tabular}{lc}
\MStartJustify
\begin{MMiniPageP}{0.45}
#1
\end{MMiniPageP}
\MStopJustify
&
\begin{MMiniPageP}{0.45}
#2  
\end{MMiniPageP}
\end{tabular}
\end{center}
}

\newcommand{\MBigImageLeft}[2]{
\begin{center}
\begin{tabular}{lc}
\MStartJustify
\begin{MMiniPageP}{0.25}
#1
\end{MMiniPageP}
\MStopJustify
&
\begin{MMiniPageP}{0.7}
#2  
\end{MMiniPageP}
\end{tabular}
\end{center}
}

\ifttm
\def\No{\mathbb{N}_0}
\else
\def\No{\ensuremath{\N_0}}
\fi
\def\MT{\textrm{\tiny T}}
\newcommand{\MTranspose}[1]{{#1}^{\MT}}
\ifttm
\newcommand{\MRe}{\mathsf{Re}}
\newcommand{\MIm}{\mathsf{Im}}
\else
\DeclareMathOperator{\MRe}{Re}
\DeclareMathOperator{\MIm}{Im}
\fi

\newcommand{\Mid}{\mathrm{id}}
\newcommand{\MFeinheit}{\mathrm{feinh}}

\ifttm
\newcommand{\Msubstack}[1]{\begin{array}{c}{#1}\end{array}}
\else
\newcommand{\Msubstack}[1]{\substack{#1}}
\fi

% Typen von Fragefeldern:
% 1 = Alphanumerisch, case-sensitive-Vergleich
% 2 = Ja/Nein-Checkbox, Loesung ist 0 oder 1   (OPTION = Image-id fuer Rueckmeldung)
% 3 = Reelle Zahlen Geparset
% 4 = Funktionen Geparset (mit Stuetzstellen zur ueberpruefung)

% Dieser Befehl erstellt ein interaktives Aufgabenfeld. Parameter:
% - #1 Laenge in Zeichen
% - #2 Loesungstext (alphanumerisch, case sensitive)
% - #3 AufgabenID (alphanumerisch, case sensitive)
% - #4 Typ (Kennnummer)
% - #5 String fuer Optionen (ggf. mit Semikolon getrennte Einzelstrings)
% - #6 Anzahl Punkte
% - #7 uxid (kann z.B. Loesungsstring sein)
% ACHTUNG: Die langen Zeilen bitte so lassen, Zeilenumbrueche im tex werden in div's umgesetzt
\newcommand{\MQuestionID}[7]{
\ifttm
\special{html:<!-- mdeclareuxid;;}UX#7\special{html:;;}\arabic{section}\special{html:;;}#3\special{html:;; //-->}%
\special{html:<!-- mdeclarepoints;;}\arabic{section}\special{html:;;}#3\special{html:;;}#6\special{html:;;}\arabic{MTestSite}\special{html:;;}\arabic{chapter}%
\special{html:;; //--><!-- onloadstart //-->CreateQuestionObj("}#7\special{html:",}\arabic{MFieldCounter}\special{html:,"}#2%
\special{html:","}#3\special{html:",}#4\special{html:,"}#5\special{html:",}#6\special{html:,}\arabic{MTestSite}\special{html:,}\arabic{section}%
\special{html:);<!-- onloadstop //-->}%
\special{html:<input mfieldtype="}#4\special{html:" name="Name_}#3\special{html:" id="}#3\special{html:" type="text" size="}#1\special{html:" maxlength="}#1%
\special{html:" }\ifnum\value{MGroupActive}=0\special{html:onfocus="handlerFocus(}\arabic{MFieldCounter}%
\special{html:);" onblur="handlerBlur(}\arabic{MFieldCounter}\special{html:);" onkeyup="handlerChange(}\arabic{MFieldCounter}\special{html:,0);" onpaste="handlerChange(}\arabic{MFieldCounter}\special{html:,0);" oninput="handlerChange(}\arabic{MFieldCounter}\special{html:,0);" onpropertychange="handlerChange(}\arabic{MFieldCounter}\special{html:,0);"/>}%
\special{html:<img src="images/questionmark.gif" width="20" height="20" border="0" align="absmiddle" id="}QM#3\special{html:"/>}
\else%
\special{html:onblur="handlerBlur(}\arabic{MFieldCounter}%
\special{html:);" onfocus="handlerFocus(}\arabic{MFieldCounter}\special{html:);" onkeyup="handlerChange(}\arabic{MFieldCounter}\special{html:,1);" onpaste="handlerChange(}\arabic{MFieldCounter}\special{html:,1);" oninput="handlerChange(}\arabic{MFieldCounter}\special{html:,1);" onpropertychange="handlerChange(}\arabic{MFieldCounter}\special{html:,1);"/>}%
\special{html:<img src="images/questionmark.gif" width="20" height="20" border="0" align="absmiddle" id="}QM#3\special{html:"/>}\fi%
\else%
\ifnum\value{QBoxFlag}=1\fbox{$\phantom{\MForLoop{#1}{b}}$}\else$\phantom{\MForLoop{#1}{b}}$\fi%
\fi%
}

% ACHTUNG: Die langen Zeilen bitte so lassen, Zeilenumbrueche im tex werden in div's umgesetzt
% QuestionCheckbox macht ausserhalb einer QuestionGroup keinen Sinn!
% #1 = solution (1 oder 0), ggf. mit ::smc abgetrennt auszuschliessende single-choice-boxen (UXIDs durch , getrennt), #2 = id, #3 = points, #4 = uxid
\newcommand{\MQuestionCheckbox}[4]{
\ifttm
\special{html:<!-- mdeclareuxid;;}UX#4\special{html:;;}\arabic{section}\special{html:;;}#2\special{html:;; //-->}%
\ifnum\value{MGroupActive}=0\MDebugMessage{ERROR: Checkbox Nr. \arabic{MFieldCounter}\ ist nicht in einer Kontrollgruppe, es wird niemals eine Loesung angezeigt!}\fi
\special{html: %
<!-- mdeclarepoints;;}\arabic{section}\special{html:;;}#2\special{html:;;}#3\special{html:;;}\arabic{MTestSite}\special{html:;;}\arabic{chapter}%
\special{html:;; //--><!-- onloadstart //-->CreateQuestionObj("}#4\special{html:",}\arabic{MFieldCounter}\special{html:,"}#1\special{html:","}#2\special{html:",2,"IMG}#2%
\special{html:",}#3\special{html:,}\arabic{MTestSite}\special{html:,}\arabic{section}\special{html:);<!-- onloadstop //-->}%
\special{html:<input mfieldtype="2" type="checkbox" name="Name_}#2\special{html:" id="}#2\special{html:" onchange="handlerChange(}\arabic{MFieldCounter}\special{html:,1);"/><img src="images/questionmark.gif" name="}Name_IMG#2%
\special{html:" width="20" height="20" border="0" align="absmiddle" id="}IMG#2\special{html:"/> }%
\else%
\ifnum\value{QBoxFlag}=1\fbox{$\phantom{X}$}\else$\phantom{X}$\fi%
\fi%
}

\def\MGenerateID{QFELD_\arabic{section}.\arabic{subsection}.\arabic{MSiteCounter}.QF\arabic{MFieldCounter}}

% #1 = 0/1 ggf. mit ::smc abgetrennt auszuschliessende single-choice-boxen (UXIDs durch , getrennt ohne UX), #2 = uxid ohne UX
\newcommand{\MCheckbox}[2]{
\MQuestionCheckbox{#1}{\MGenerateID}{\MStdPoints}{#2}
\addtocounter{MFieldCounter}{1}
}

% Erster Parameter: Zeichenlaenge der Eingabebox, zweiter Parameter: Loesungstext
\newcommand{\MQuestion}[2]{
\MQuestionID{#1}{#2}{\MGenerateID}{1}{0}{\MStdPoints}{#2}
\addtocounter{MFieldCounter}{1}
}

% Erster Parameter: Zeichenlaenge der Eingabebox, zweiter Parameter: Loesungstext
\newcommand{\MLQuestion}[3]{
\MQuestionID{#1}{#2}{\MGenerateID}{1}{0}{\MStdPoints}{#3}
\addtocounter{MFieldCounter}{1}
}

% Parameter: Laenge des Feldes, Loesung (wird auch geparsed), Stellen Genauigkeit hinter dem Komma, weitere Stellen werden mathematisch gerundet vor Vergleich
\newcommand{\MParsedQuestion}[3]{
\MQuestionID{#1}{#2}{\MGenerateID}{3}{#3}{\MStdPoints}{#2}
\addtocounter{MFieldCounter}{1}
}

% Parameter: Laenge des Feldes, Loesung (wird auch geparsed), Stellen Genauigkeit hinter dem Komma, weitere Stellen werden mathematisch gerundet vor Vergleich
\newcommand{\MLParsedQuestion}[4]{
\MQuestionID{#1}{#2}{\MGenerateID}{3}{#3}{\MStdPoints}{#4}
\addtocounter{MFieldCounter}{1}
}

% Parameter: Laenge des Feldes, Loesungsfunktion, Anzahl Stuetzstellen, Funktionsvariablen durch Kommata getrennt (nicht case-sensitive), Anzahl Nachkommastellen im Vergleich
\newcommand{\MFunctionQuestion}[5]{
\MQuestionID{#1}{#2}{\MGenerateID}{4}{#3;#4;#5;0}{\MStdPoints}{#2}
\addtocounter{MFieldCounter}{1}
}

% Parameter: Laenge des Feldes, Loesungsfunktion, Anzahl Stuetzstellen, Funktionsvariablen durch Kommata getrennt (nicht case-sensitive), Anzahl Nachkommastellen im Vergleich, UXID
\newcommand{\MLFunctionQuestion}[6]{
\MQuestionID{#1}{#2}{\MGenerateID}{4}{#3;#4;#5;0}{\MStdPoints}{#6}
\addtocounter{MFieldCounter}{1}
}

% Parameter: Laenge des Feldes, Loesungsintervall, Genauigkeit der Zahlenwertpruefung
\newcommand{\MIntervalQuestion}[3]{
\MQuestionID{#1}{#2}{\MGenerateID}{6}{#3}{\MStdPoints}{#2}
\addtocounter{MFieldCounter}{1}
}

% Parameter: Laenge des Feldes, Loesungsintervall, Genauigkeit der Zahlenwertpruefung, UXID
\newcommand{\MLIntervalQuestion}[4]{
\MQuestionID{#1}{#2}{\MGenerateID}{6}{#3}{\MStdPoints}{#4}
\addtocounter{MFieldCounter}{1}
}

% Parameter: Laenge des Feldes, Loesungsfunktion, Anzahl Stuetzstellen, Funktionsvariable (nicht case-sensitive), Anzahl Nachkommastellen im Vergleich, Vereinfachungsbedingung
% Vereinfachungsbedingung ist eine der Folgenden:
% 0 = Keine Vereinfachungsbedingung
% 1 = Keine Klammern (runde oder eckige) mehr im vereinfachten Ausdruck
% 2 = Faktordarstellung (Term hat Produkte als letzte Operation, Summen als vorgeschaltete Operation)
% 3 = Summendarstellung (Term hat Summen als letzte Operation, Produkte als vorgeschaltete Operation)
% Flag 512: Besondere Stuetzstellen (nur >1 und nur schwach rational), sonst symmetrisch um Nullpunkt und ganze Zahlen inkl. Null werden getroffen
\newcommand{\MSimplifyQuestion}[6]{
\MQuestionID{#1}{#2}{\MGenerateID}{4}{#3;#4;#5;#6}{\MStdPoints}{#2}
\addtocounter{MFieldCounter}{1}
}

\newcommand{\MLSimplifyQuestion}[7]{
\MQuestionID{#1}{#2}{\MGenerateID}{4}{#3;#4;#5;#6}{\MStdPoints}{#7}
\addtocounter{MFieldCounter}{1}
}

% Parameter: Laenge des Feldes, Loesung (optionaler Ausdruck), Anzahl Stuetzstellen, Funktionsvariable (nicht case-sensitive), Anzahl Nachkommastellen im Vergleich, Spezialtyp (string-id)
\newcommand{\MLSpecialQuestion}[7]{
\MQuestionID{#1}{#2}{\MGenerateID}{7}{#3;#4;#5;#6}{\MStdPoints}{#7}
\addtocounter{MFieldCounter}{1}
}

\newcounter{MGroupStart}
\newcounter{MGroupEnd}
\newcounter{MGroupActive}

\newenvironment{MQuestionGroup}{
\setcounter{MGroupStart}{\value{MFieldCounter}}
\setcounter{MGroupActive}{1}
}{
\setcounter{MGroupActive}{0}
\setcounter{MGroupEnd}{\value{MFieldCounter}}
\addtocounter{MGroupEnd}{-1}
}

\newcommand{\MGroupButton}[1]{
\ifttm
\special{html:<button name="Name_Group}\arabic{MGroupStart}\special{html:to}\arabic{MGroupEnd}\special{html:" id="Group}\arabic{MGroupStart}\special{html:to}\arabic{MGroupEnd}\special{html:" %
type="button" onclick="group_button(}\arabic{MGroupStart}\special{html:,}\arabic{MGroupEnd}\special{html:);">}#1\special{html:</button>}
\else
\phantom{#1}
\fi
}

%----------------- Makros fuer die modularisierte Darstellung ------------------------------------

\def\MyText#1{#1}

% is used internally by the conversion package, should not be used by original tex documents
\def\MOrgLabel#1{\relax}

\ifttm

% Ein MLabel wird im html codiert durch das tag <!-- mmlabel;;Labelbezeichner;;SubjectArea;;chapter;;section;;subsection;;Index;;Objekttyp; //-->
\def\MLabel#1{%
\ifnum\value{MLastType}=8%
\ifnum\value{MCaptionOn}=0%
\MDebugMessage{ERROR: Grafik \arabic{MGraphicsCounter} hat separates label: #1 (Grafiklabels sollten nur in der Caption stehen)}%
\fi
\fi
\ifnum\value{MLastType}=12%
\ifnum\value{MCaptionOn}=0%
\MDebugMessage{ERROR: Video \arabic{MVideoCounter} hat separates label: #1 (Videolabels sollten nur in der Caption stehen}%
\fi
\fi
\ifnum\value{MLastType}=10\setcounter{MLastIndex}{\value{equation}}\fi
\label{#1}\begin{html}<!-- mmlabel;;#1;;\end{html}\arabic{MSubjectArea}\special{html:;;}\arabic{chapter}\special{html:;;}\arabic{section}\special{html:;;}\arabic{subsection}\special{html:;;}\arabic{MLastIndex}\special{html:;;}\arabic{MLastType}\special{html:; //-->}}%

\else

% Sonderbehandlung im PDF fuer Abbildungen in separater aux-Datei, da MGraphics die figure-Umgebung nicht verwendet
\def\MLabel#1{%
\ifnum\value{MLastType}=8%
\ifnum\value{MCaptionOn}=0%
\MDebugMessage{ERROR: Grafik \arabic{MGraphicsCounter} hat separates label: #1 (Grafiklabels sollten nur in der Caption stehen}%
\fi
\fi
\ifnum\value{MLastType}=12%
\ifnum\value{MCaptionOn}=0%
\MDebugMessage{ERROR: Video \arabic{MVideoCounter} hat separates label: #1 (Videolabels sollten nur in der Caption stehen}%
\fi
\fi
\label{#1}%
}%

\fi

% Gibt Begriff des referenzierten Objekts mit aus, aber nur im HTML, daher nur in Ausnahmefaellen (z.B. Copyrightliste) sinnvoll
\def\MCRef#1{\ifttm\special{html:<!-- mmref;;}#1\special{html:;;1; //-->}\else\vref{#1}\fi}


\def\MRef#1{\ifttm\special{html:<!-- mmref;;}#1\special{html:;;0; //-->}\else\vref{#1}\fi}
\def\MERef#1{\ifttm\special{html:<!-- mmref;;}#1\special{html:;;0; //-->}\else\eqref{#1}\fi}
\def\MNRef#1{\ifttm\special{html:<!-- mmref;;}#1\special{html:;;0; //-->}\else\ref{#1}\fi}
\def\MSRef#1#2{\ifttm\special{html:<!-- msref;;}#1\special{html:;;}#2\special{html:; //-->}\else \if#2\empty \ref{#1} \else \hyperref[#1]{#2}\fi\fi} 

\def\MRefRange#1#2{\ifttm\MRef{#1} bis 
\MRef{#2}\else\vrefrange[\unskip]{#1}{#2}\fi}

\def\MRefTwo#1#2{\ifttm\MRef{#1} und \MRef{#2}\else%
\let\vRefTLRsav=\reftextlabelrange\let\vRefTPRsav=\reftextpagerange%
\def\reftextlabelrange##1##2{\ref{##1} und~\ref{##2}}%
\def\reftextpagerange##1##2{auf den Seiten~\pageref{#1} und~\pageref{#2}}%
\vrefrange[\unskip]{#1}{#2}%
\let\reftextlabelrange=\vRefTLRsav\let\reftextpagerange=\vRefTPRsav\fi}

% MSectionChapter definiert falls notwendig das Kapitel vor der section. Das ist notwendig, wenn nur ein Einzelmodul uebersetzt wird.
% MChaptersGiven ist ein Counter, der von mconvert.pl vordefiniert wird.
\ifttm
\newcommand{\MSectionChapter}{\ifnum\value{MChaptersGiven}=0{\Dchapter{Modul}}\else{}\fi}
\else
\newcommand{\MSectionChapter}{\ifnum\value{chapter}=0{\Dchapter{Modul}}\else{}\fi}
\fi


\def\MChapter#1{\ifnum\value{MSSEnd}>0{\MSubsectionEndMacros}\addtocounter{MSSEnd}{-1}\fi\Dchapter{#1}}
\def\MSubject#1{\MChapter{#1}} % Schluesselwort HELPSECTION ist reserviert fuer Hilfesektion

\newcommand{\MSectionID}{UNKNOWNID}

\ifttm
\newcommand{\MSetSectionID}[1]{\renewcommand{\MSectionID}{#1}}
\else
\newcommand{\MSetSectionID}[1]{\renewcommand{\MSectionID}{#1}\tikzsetexternalprefix{#1}}
\fi


\newcommand{\MSection}[1]{\MSetSectionID{MODULID}\ifnum\value{MSSEnd}>0{\MSubsectionEndMacros}\addtocounter{MSSEnd}{-1}\fi\MSectionChapter\Dsection{#1}\MSectionStartMacros{#1}\setcounter{MLastIndex}{-1}\setcounter{MLastType}{1}} % Sections werden ueber das section-Feld im mmlabel-Tag identifiziert, nicht ueber das Indexfeld

\def\MSubsection#1{\ifnum\value{MSSEnd}>0{\MSubsectionEndMacros}\addtocounter{MSSEnd}{-1}\fi\ifttm\else\clearpage\fi\Dsubsection{#1}\MSubsectionStartMacros\setcounter{MLastIndex}{-1}\setcounter{MLastType}{2}\addtocounter{MSSEnd}{1}}% Subsections werden ueber das subsection-Feld im mmlabel-Tag identifiziert, nicht ueber das Indexfeld
\def\MSubsectionx#1{\Dsubsectionx{#1}} % Nur zur Verwendung in MSectionStart gedacht
\def\MSubsubsection#1{\Dsubsubsection{#1}\setcounter{MLastIndex}{\value{subsubsection}}\setcounter{MLastType}{3}\ifttm\special{html:<!-- sectioninfo;;}\arabic{section}\special{html:;;}\arabic{subsection}\special{html:;;}\arabic{subsubsection}\special{html:;;1;;}\arabic{MTestSite}\special{html:; //-->}\fi}
\def\MSubsubsectionx#1{\Dsubsubsectionx{#1}\ifttm\special{html:<!-- sectioninfo;;}\arabic{section}\special{html:;;}\arabic{subsection}\special{html:;;}\arabic{subsubsection}\special{html:;;0;;}\arabic{MTestSite}\special{html:; //-->}\else\addcontentsline{toc}{subsection}{#1}\fi}

\ifttm
\def\MSubsubsubsectionx#1{\ \newline\textbf{#1}\special{html:<br />}}
\else
\def\MSubsubsubsectionx#1{\ \newline
\textbf{#1}\ \\
}
\fi


% Dieses Skript wird zu Beginn jedes Modulabschnitts (=Webseite) ausgefuehrt und initialisiert den Aufgabenfeldzaehler
\newcommand{\MPageScripts}{
\setcounter{MFieldCounter}{1}
\addtocounter{MSiteCounter}{1}
\setcounter{MHintCounter}{1}
\setcounter{MCodeEditCounter}{1}
\setcounter{MGroupActive}{0}
\DoQBoxes
% Feldvariablen werden im HTML-Header in conv.pl eingestellt
}

% Dieses Skript wird zum Ende jedes Modulabschnitts (=Webseite) ausgefuehrt
\ifttm
\newcommand{\MEndScripts}{\special{html:<br /><!-- mfeedbackbutton;Seite;}\arabic{MTestSite}\special{html:;}\MGenerateSiteNumber\special{html:; //-->}
}
\else
\newcommand{\MEndScripts}{\relax}
\fi


\newcounter{QBoxFlag}
\newcommand{\DoQBoxes}{\setcounter{QBoxFlag}{1}}
\newcommand{\NoQBoxes}{\setcounter{QBoxFlag}{0}}

\newcounter{MXCTest}
\newcounter{MXCounter}
\newcounter{MSCounter}



\ifttm

% Struktur des sectioninfo-Tags: <!-- sectioninfo;;section;;subsection;;subsubsection;;nr_ausgeben;;testpage; //-->

%Fuegt eine zusaetzliche html-Seite an hinter ALLEN bisherigen und zukuenftigen content-Seiten ausserhalb der vor-zurueck-Schleife (d.h. nur durch Button oder MIntLink erreichbar!)
% #1 = Titel des Modulabschnitts, #2 = Kurztitel fuer die Buttons, #3 = Buttonkennung (STD = default nehmen, NONE = Ohne Button in der Navigation)
\newenvironment{MSContent}[3]{\special{html:<div class="xcontent}\arabic{MSCounter}\special{html:"><!-- scontent;-;}\arabic{MSCounter};-;#1;-;#2;-;#3\special{html: //-->}\MPageScripts\MSubsubsectionx{#1}}{\MEndScripts\special{html:<!-- endscontent;;}\arabic{MSCounter}\special{html: //--></div>}\addtocounter{MSCounter}{1}}

% Fuegt eine zusaetzliche html-Seite ein hinter den bereits vorhandenen content-Seiten (oder als erste Seite) innerhalb der vor-zurueck-Schleife der Navigation
% #1 = Titel des Modulabschnitts, #2 = Kurztitel fuer die Buttons, #3 = Buttonkennung (STD = Defaultbutton, NONE = Ohne Button in der Navigation)
\newenvironment{MXContent}[3]{\special{html:<div class="xcontent}\arabic{MXCounter}\special{html:"><!-- xcontent;-;}\arabic{MXCounter};-;#1;-;#2;-;#3\special{html: //-->}\MPageScripts\MSubsubsection{#1}}{\MEndScripts\special{html:<!-- endxcontent;;}\arabic{MXCounter}\special{html: //--></div>}\addtocounter{MXCounter}{1}}

% Fuegt eine zusaetzliche html-Seite ein die keine subsubsection-Nummer bekommt, nur zur internen Verwendung in mintmod.tex gedacht!
% #1 = Titel des Modulabschnitts, #2 = Kurztitel fuer die Buttons, #3 = Buttonkennung (STD = Defaultbutton, NONE = Ohne Button in der Navigation)
% \newenvironment{MUContent}[3]{\special{html:<div class="xcontent}\arabic{MXCounter}\special{html:"><!-- xcontent;-;}\arabic{MXCounter};-;#1;-;#2;-;#3\special{html: //-->}\MPageScripts\MSubsubsectionx{#1}}{\MEndScripts\special{html:<!-- endxcontent;;}\arabic{MXCounter}\special{html: //--></div>}\addtocounter{MXCounter}{1}}

\newcommand{\MDeclareSiteUXID}[1]{\special{html:<!-- mdeclaresiteuxid;;}#1\special{html:;;}\arabic{chapter}\special{html:;;}\arabic{section}\special{html:;; //-->}}

\else

%\newcommand{\MSubsubsection}[1]{\refstepcounter{subsubsection} \addcontentsline{toc}{subsubsection}{\thesubsubsection. #1}}


% Fuegt eine zusaetzliche html-Seite an hinter den bereits vorhandenen content-Seiten
% #1 = Titel des Modulabschnitts, #2 = Kurztitel fuer die Buttons, #3 = Iconkennung (im PDF wirkungslos)
%\newenvironment{MUContent}[3]{\ifnum\value{MXCTest}>0{\MDebugMessage{ERROR: Geschachtelter SContent}}\fi\MPageScripts\MSubsubsectionx{#1}\addtocounter{MXCTest}{1}}{\addtocounter{MXCounter}{1}\addtocounter{MXCTest}{-1}}
\newenvironment{MXContent}[3]{\ifnum\value{MXCTest}>0{\MDebugMessage{ERROR: Geschachtelter SContent}}\fi\MPageScripts\MSubsubsection{#1}\addtocounter{MXCTest}{1}}{\addtocounter{MXCounter}{1}\addtocounter{MXCTest}{-1}}
\newenvironment{MSContent}[3]{\ifnum\value{MXCTest}>0{\MDebugMessage{ERROR: Geschachtelter XContent}}\fi\MPageScripts\MSubsubsectionx{#1}\addtocounter{MXCTest}{1}}{\addtocounter{MSCounter}{1}\addtocounter{MXCTest}{-1}}

\newcommand{\MDeclareSiteUXID}[1]{\relax}

\fi 

% GHEADER und GFOOTER werden von split.pm gefunden, aber nur, wenn nicht HELPSITE oder TESTSITE
\ifttm
\newenvironment{MSectionStart}{\special{html:<div class="xcontent0">}\MSubsubsectionx{Modul\"ubersicht}}{\setcounter{MSSEnd}{0}\special{html:</div>}}
% Darf nicht als XContent nummeriert werden, darf nicht als XContent gelabelt werden, wird aber in eine xcontent-div gesetzt fuer Python-parsing
\else
\newenvironment{MSectionStart}{\MSubsectionx{Modul\"ubersicht}}{\setcounter{MSSEnd}{0}}
\fi

\newenvironment{MIntro}{\begin{MXContent}{Einf\"uhrung}{Einf\"uhrung}{genetisch}}{\end{MXContent}}
\newenvironment{MContent}{\begin{MXContent}{Inhalt}{Inhalt}{beweis}}{\end{MXContent}}
\newenvironment{MExercises}{\ifttm\else\clearpage\fi\begin{MXContent}{Aufgaben}{Aufgaben}{aufgb}\special{html:<!-- declareexcsymb //-->}}{\end{MXContent}}

% #1 = Lesbare Testbezeichnung
\newenvironment{MTest}[1]{%
\renewcommand{\MTestName}{#1}
\ifttm\else\clearpage\fi%
\addtocounter{MTestSite}{1}%
\begin{MXContent}{#1}{#1}{STD} % {aufgb}%
\special{html:<!-- declaretestsymb //-->}
\begin{MQuestionGroup}%
\MInTestHeader
}%
{%
\end{MQuestionGroup}%
\ \\ \ \\%
\MInTestFooter
\end{MXContent}\addtocounter{MTestSite}{-1}%
}

\newenvironment{MExtra}{\ifttm\else\clearpage\fi\begin{MXContent}{Zus\"atzliche Inhalte}{Zusatz}{weiterfhrg}}{\end{MXContent}}

\makeindex

\ifttm
\def\MPrintIndex{
\ifnum\value{MSSEnd}>0{\MSubsectionEndMacros}\addtocounter{MSSEnd}{-1}\fi
\renewcommand{\indexname}{Stichwortverzeichnis}
\special{html:<p><!-- printindex //--></p>}
}
\else
\def\MPrintIndex{
\ifnum\value{MSSEnd}>0{\MSubsectionEndMacros}\addtocounter{MSSEnd}{-1}\fi
\renewcommand{\indexname}{Stichwortverzeichnis}
\addcontentsline{toc}{section}{Stichwortverzeichnis}
\printindex
}
\fi


% Konstanten fuer die Modulfaecher

\def\MINTMathematics{1}
\def\MINTInformatics{2}
\def\MINTChemistry{3}
\def\MINTPhysics{4}
\def\MINTEngineering{5}

\newcounter{MSubjectArea}
\newcounter{MInfoNumbers} % Gibt an, ob die Infoboxen nummeriert werden sollen
\newcounter{MSepNumbers} % Gibt an, ob Beispiele und Experimente separat nummeriert werden sollen
\newcommand{\MSetSubject}[1]{
 % ttm kapiert setcounter mit Parametern nicht, also per if abragen und einsetzen
\ifnum#1=1\setcounter{MSubjectArea}{1}\setcounter{MInfoNumbers}{1}\setcounter{MSepNumbers}{0}\fi
\ifnum#1=2\setcounter{MSubjectArea}{2}\setcounter{MInfoNumbers}{1}\setcounter{MSepNumbers}{0}\fi
\ifnum#1=3\setcounter{MSubjectArea}{3}\setcounter{MInfoNumbers}{0}\setcounter{MSepNumbers}{1}\fi
\ifnum#1=4\setcounter{MSubjectArea}{4}\setcounter{MInfoNumbers}{0}\setcounter{MSepNumbers}{0}\fi
\ifnum#1=5\setcounter{MSubjectArea}{5}\setcounter{MInfoNumbers}{1}\setcounter{MSepNumbers}{0}\fi
% Separate Nummerntechnik fuer unsere Chemiker: alles dreistellig
\ifnum#1=3
  \ifttm
  \renewcommand{\theequation}{\arabic{section}.\arabic{subsection}.\arabic{equation}}
  \renewcommand{\thetable}{\arabic{section}.\arabic{subsection}.\arabic{table}} 
  \renewcommand{\thefigure}{\arabic{section}.\arabic{subsection}.\arabic{figure}} 
  \else
  \renewcommand{\theequation}{\arabic{chapter}.\arabic{section}.\arabic{equation}}
  \renewcommand{\thetable}{\arabic{chapter}.\arabic{section}.\arabic{table}}
  \renewcommand{\thefigure}{\arabic{chapter}.\arabic{section}.\arabic{figure}}
  \fi
\else
  \ifttm
  \renewcommand{\theequation}{\arabic{section}.\arabic{subsection}.\arabic{equation}}
  \renewcommand{\thetable}{\arabic{table}}
  \renewcommand{\thefigure}{\arabic{figure}}
  \else
  \renewcommand{\theequation}{\arabic{chapter}.\arabic{section}.\arabic{equation}}
  \renewcommand{\thetable}{\arabic{table}}
  \renewcommand{\thefigure}{\arabic{figure}}
  \fi
\fi
}

% Fuer tikz Autogenerierung
\newcounter{MTIKZAutofilenumber}

% Spezielle Counter fuer die Bentz-Module
\newcounter{mycounter}
\newcounter{chemapplet}
\newcounter{physapplet}

\newcounter{MSSEnd} % Ist 1 falls ein MSubsection aktiv ist, der einen MSubsectionEndMacro-Aufruf verursacht
\newcounter{MFileNumber}
\def\MLastFile{\special{html:[[!-- mfileref;;}\arabic{MFileNumber}\special{html:; //--]]}}

% Vollstaendiger Pfad ist \MMaterial / \MLastFilePath / \MLastFileName    ==   \MMaterial / \MLastFile

% Wird nur bei kompletter Baum-Erstellung ausgefuehrt!
% #1 = Lesbare Modulbezeichnung
\newcommand{\MSectionStartMacros}[1]{
\setcounter{MTestSite}{0}
\setcounter{MCaptionOn}{0}
\setcounter{MLastTypeEq}{0}
\setcounter{MSSEnd}{0}
\setcounter{MFileNumber}{0} % Preinkrekement-Counter
\setcounter{MTIKZAutofilenumber}{0}
\setcounter{mycounter}{1}
\setcounter{physapplet}{1}
\setcounter{chemapplet}{0}
\ifttm
\special{html:<!-- mdeclaresection;;}\arabic{chapter}\special{html:;;}\arabic{section}\special{html:;;}#1\special{html:;; //-->}%
\else
\setcounter{thmc}{0}
\setcounter{exmpc}{0}
\setcounter{verc}{0}
\setcounter{infoc}{0}
\fi
\setcounter{MiniMarkerCounter}{1}
\setcounter{AlignCounter}{1}
\setcounter{MXCTest}{0}
\setcounter{MCodeCounter}{0}
\setcounter{MEntryCounter}{0}
}

% Wird immer ausgefuehrt
\newcommand{\MSubsectionStartMacros}{
\ifttm\else\MPageHeaderDef\fi
\MWatermarkSettings
\setcounter{MXCounter}{0}
\setcounter{MSCounter}{0}
\setcounter{MSiteCounter}{1}
\setcounter{MExerciseCollectionCounter}{0}
% Zaehler fuer das Labelsystem zuruecksetzen (prefix-Zaehler)
\setcounter{MInfoCounter}{0}
\setcounter{MExerciseCounter}{0}
\setcounter{MExampleCounter}{0}
\setcounter{MExperimentCounter}{0}
\setcounter{MGraphicsCounter}{0}
\setcounter{MTableCounter}{0}
\setcounter{MTheoremCounter}{0}
\setcounter{MObjectCounter}{0}
\setcounter{MEquationCounter}{0}
\setcounter{MVideoCounter}{0}
\setcounter{equation}{0}
\setcounter{figure}{0}
}

\newcommand{\MSubsectionEndMacros}{
% Bei Chemiemodulen das PSE einhaengen, es soll als SContent am Ende erscheinen
\special{html:<!-- subsectionend //-->}
\ifnum\value{MSubjectArea}=3{\MIncludePSE}\fi
}


\ifttm
%\newcommand{\MEmbed}[1]{\MRegisterFile{#1}\begin{html}<embed src="\end{html}\MMaterial/\MLastFile\begin{html}" width="192" height="189"></embed>\end{html}}
\newcommand{\MEmbed}[1]{\MRegisterFile{#1}\begin{html}<embed src="\end{html}\MMaterial/\MLastFile\begin{html}"></embed>\end{html}}
\fi

%----------------- Makros fuer die Textdarstellung -----------------------------------------------

\ifttm
% MUGraphics bindet eine Grafik ein:
% Parameter 1: Dateiname der Grafik, relativ zur Position des Modul-Tex-Dokuments
% Parameter 2: Skalierungsoptionen fuer PDF (fuer includegraphics)
% Parameter 3: Titel fuer die Grafik, wird unter die Grafik mit der Grafiknummer gesetzt und kann MLabel bzw. MCopyrightLabel enthalten
% Parameter 4: Skalierungsoptionen fuer HTML (css-styles)

% ERSATZ: <img alt="My Image" src="data:image/png;base64,iVBORwA<MoreBase64SringHere>" />


\newcommand{\MUGraphics}[4]{\MRegisterFile{#1}\begin{html}
<div class="imagecenter">
<center>
<div>
<img src="\end{html}\MMaterial/\MLastFile\begin{html}" style="#4" alt="\end{html}\MMaterial/\MLastFile\begin{html}"/>
</div>
<div class="bildtext">
\end{html}
\addtocounter{MGraphicsCounter}{1}
\setcounter{MLastIndex}{\value{MGraphicsCounter}}
\setcounter{MLastType}{8}
\addtocounter{MCaptionOn}{1}
\ifnum\value{MSepNumbers}=0
\textbf{Abbildung \arabic{MGraphicsCounter}:} #3
\else
\textbf{Abbildung \arabic{section}.\arabic{subsection}.\arabic{MGraphicsCounter}:} #3
\fi
\addtocounter{MCaptionOn}{-1}
\begin{html}
</div>
</center>
</div>
<br />
\end{html}%
\special{html:<!-- mfeedbackbutton;Abbildung;}\arabic{MGraphicsCounter}\special{html:;}\arabic{section}.\arabic{subsection}.\arabic{MGraphicsCounter}\special{html:; //-->}%
}

% MVideo bindet ein Video als Einzeldatei ein:
% Parameter 1: Dateiname des Videos, relativ zur Position des Modul-Tex-Dokuments, ohne die Endung ".mp4"
% Parameter 2: Titel fuer das Video (kann MLabel oder MCopyrightLabel enthalten), wird unter das Video mit der Videonummer gesetzt
\newcommand{\MVideo}[2]{\MRegisterFile{#1.mp4}\begin{html}
<div class="imagecenter">
<center>
<div>
<video width="95\%" controls="controls"><source src="\end{html}\MMaterial/#1.mp4\begin{html}" type="video/mp4">Ihr Browser kann keine MP4-Videos abspielen!</video>
</div>
<div class="bildtext">
\end{html}
\addtocounter{MVideoCounter}{1}
\setcounter{MLastIndex}{\value{MVideoCounter}}
\setcounter{MLastType}{12}
\addtocounter{MCaptionOn}{1}
\ifnum\value{MSepNumbers}=0
\textbf{Video \arabic{MVideoCounter}:} #2
\else
\textbf{Video \arabic{section}.\arabic{subsection}.\arabic{MVideoCounter}:} #2
\fi
\addtocounter{MCaptionOn}{-1}
\begin{html}
</div>
</center>
</div>
<br />
\end{html}}

\newcommand{\MDVideo}[2]{\MRegisterFile{#1.mp4}\MRegisterFile{#1.ogv}\begin{html}
<div class="imagecenter">
<center>
<div>
<video width="70\%" controls><source src="\end{html}\MMaterial/#1.mp4\begin{html}" type="video/mp4"><source src="\end{html}\MMaterial/#1.ogv\begin{html}" type="video/ogg">Ihr Browser kann keine MP4-Videos abspielen!</video>
</div>
<br />
#2
</center>
</div>
<br />
\end{html}
}

\newcommand{\MGraphics}[3]{\MUGraphics{#1}{#2}{#3}{}}

\else

\newcommand{\MVideo}[2]{%
% Kein Video im PDF darstellbar, trotzdem so tun als ob da eines waere
\begin{center}
(Video nicht darstellbar)
\end{center}
\addtocounter{MVideoCounter}{1}
\setcounter{MLastIndex}{\value{MVideoCounter}}
\setcounter{MLastType}{12}
\addtocounter{MCaptionOn}{1}
\ifnum\value{MSepNumbers}=0
\textbf{Video \arabic{MVideoCounter}:} #2
\else
\textbf{Video \arabic{section}.\arabic{subsection}.\arabic{MVideoCounter}:} #2
\fi
\addtocounter{MCaptionOn}{-1}
}


% MGraphics bindet eine Grafik ein:
% Parameter 1: Dateiname der Grafik, relativ zur Position des Modul-Tex-Dokuments
% Parameter 2: Skalierungsoptionen fuer PDF (fuer includegraphics)
% Parameter 3: Titel fuer die Grafik, wird unter die Grafik mit der Grafiknummer gesetzt
\newcommand{\MGraphics}[3]{%
\MRegisterFile{#1}%
\ %
\begin{figure}[H]%
\centering{%
\includegraphics[#2]{\MDPrefix/#1}%
\addtocounter{MCaptionOn}{1}%
\caption{#3}%
\addtocounter{MCaptionOn}{-1}%
}%
\end{figure}%
\addtocounter{MGraphicsCounter}{1}\setcounter{MLastIndex}{\value{MGraphicsCounter}}\setcounter{MLastType}{8}\ %
%\ \\Abbildung \ifnum\value{MSepNumbers}=0\else\arabic{chapter}.\arabic{section}.\fi\arabic{MGraphicsCounter}: #3%
}

\newcommand{\MUGraphics}[4]{\MGraphics{#1}{#2}{#3}}


\fi

\newcounter{MCaptionOn} % = 1 falls eine Grafikcaption aktiv ist, = 0 sonst


% MGraphicsSolo bindet eine Grafik pur ein ohne Titel
% Parameter 1: Dateiname der Grafik, relativ zur Position des Modul-Tex-Dokuments
% Parameter 2: Skalierungsoptionen (wirken nur im PDF)
\newcommand{\MGraphicsSolo}[2]{\MUGraphicsSolo{#1}{#2}{}}

% MUGraphicsSolo bindet eine Grafik pur ein ohne Titel, aber mit HTML-Skalierung
% Parameter 1: Dateiname der Grafik, relativ zur Position des Modul-Tex-Dokuments
% Parameter 2: Skalierungsoptionen (wirken nur im PDF)
% Parameter 3: Skalierungsoptionen (wirken nur im HTML), als style-format: "width=???, height=???"
\ifttm
\newcommand{\MUGraphicsSolo}[3]{\MRegisterFile{#1}\begin{html}
<img src="\end{html}\MMaterial/\MLastFile\begin{html}" style="\end{html}#3\begin{html}" alt="\end{html}\MMaterial/\MLastFile\begin{html}"/>
\end{html}%
\special{html:<!-- mfeedbackbutton;Abbildung;}#1\special{html:;}\MMaterial/\MLastFile\special{html:; //-->}%
}
\else
\newcommand{\MUGraphicsSolo}[3]{\MRegisterFile{#1}\includegraphics[#2]{\MDPrefix/#1}}
\fi

% Externer Link mit URL
% Erster Parameter: Vollstaendige(!) URL des Links
% Zweiter Parameter: Text fuer den Link
\newcommand{\MExtLink}[2]{\ifttm\special{html:<a target="_new" href="}#1\special{html:">}#2\special{html:</a>}\else\href{#1}{#2}\fi} % ohne MINTERLINK!


% Interner Link, die verlinkte Datei muss im gleichen Verzeichnis liegen wie die Modul-Texdatei
% Erster Parameter: Dateiname
% Zweiter Parameter: Text fuer den Link
\newcommand{\MIntLink}[2]{\ifttm\MRegisterFile{#1}\special{html:<a class="MINTERLINK" target="_new" href="}\MMaterial/\MLastFile\special{html:">}#2\special{html:</a>}\else{\href{#1}{#2}}\fi}


\ifttm
\def\MMaterial{:localmaterial:}
\else
\def\MMaterial{\MDPrefix}
\fi

\ifttm
\def\MNoFile#1{:directmaterial:#1}
\else
\def\MNoFile#1{#1}
\fi

\newcommand{\MChem}[1]{$\mathrm{#1}$}

\newcommand{\MApplet}[3]{
% Bindet ein Java-Applet ein, die Parameter sind:
% (wird nur im HTML, aber nicht im PDF erstellt)
% #1 Dateiname des Applets (muss mit ".class" enden)
% #2 = Breite in Pixeln
% #3 = Hoehe in Pixeln
\ifttm
\MRegisterFile{#1}
\begin{html}
<applet code="\end{html}\MMaterial/\MLastFile\begin{html}" width="#2" height="#3" alt="[Java-Applet kann nicht gestartet werden]"></applet>
\end{html}
\fi
}

\newcommand{\MScriptPage}[2]{
% Bindet eine JavaScript-Datei ein, die eine eigene Seite bekommt
% (wird nur im HTML, aber nicht im PDF erstellt)
% #1 Dateiname des Programms (sollte mit ".js" enden)
% #2 = Kurztitel der Seite
\ifttm
\begin{MSContent}{#2}{#2}{puzzle}
\MRegisterFile{#1}
\begin{html}
<script src="\MMaterial/\MLastFile" type="text/javascript"></script>
\end{html}
\end{MSContent}
\fi
}

\newcommand{\MIncludePSE}{
% Bindet bei Chemie-Modulen das PSE ein
% (wird nur im HTML, aber nicht im PDF erstellt)
\ifttm
\special{html:<!-- includepse //-->}
\begin{MSContent}{Periodensystem der Elemente}{PSE}{table}
\MRegisterFile{../files/pse.js}
\MRegisterFile{../files/radio.png}
\begin{html}
<script src="\MMaterial/../files/pse.js" type="text/javascript"></script>
<p id="divid"><br /><br />
<script language="javascript" type="text/javascript">
    startpse("divid","\MMaterial/../files"); 
</script>
</p>
<br />
<br />
<br />
<p>Die Farben der Elementsymbole geben an: <font style="color:Red">gasf&ouml;rmig </font> <font style="color:Blue">fl&uuml;ssig </font> fest</p>
<p>Die Elemente der Gruppe 1 A, 2 A, 3 A usw. geh&ouml;ren zu den Hauptgruppenelementen.</p>
<p>Die Elemente der Gruppe 1 B, 2 B, 3 B usw. geh&ouml;ren zu den Nebengruppenelementen.</p>
<p>() kennzeichnet die Masse des stabilsten Isotops</p>
\end{html}
\end{MSContent}
\fi
}

\newcommand{\MAppletArchive}[4]{
% Bindet ein Java-Applet ein, die Parameter sind:
% (wird nur im HTML, aber nicht im PDF erstellt)
% #1 Dateiname der Klasse mit Appletaufruf (muss mit ".class" enden)
% #2 Dateiname des Archivs (muss mit ".jar" enden)
% #3 = Breite in Pixeln
% #4 = Hoehe in Pixeln
\ifttm
\MRegisterFile{#2}
\begin{html}
<applet code="#1" archive="\end{html}\MMaterial/\MLastFile\begin{html}" codebase="." width="#3" height="#4" alt="[Java-Archiv kann nicht gestartet werden]"></applet>
\end{html}
\fi
}

% Bindet in der Haupttexdatei ein MINT-Modul ein. Parameter 1 ist das Verzeichnis (relativ zur Haupttexdatei), Parameter 2 ist der Dateinahme ohne Pfad.
\newcommand{\IncludeModule}[2]{
\renewcommand{\MDPrefix}{#1}
\input{#1/#2}
\ifnum\value{MSSEnd}>0{\MSubsectionEndMacros}\addtocounter{MSSEnd}{-1}\fi
}

% Der ttm-Konverter setzt keine Makros im \input um, also muss hier getrickst werden:
% Das MDPrefix muss in den einzelnen Modulen manuell eingesetzt werden
\newcommand{\MInputFile}[1]{
\ifttm
\input{#1}
\else
\input{#1}
\fi
}


\newcommand{\MCases}[1]{\left\lbrace{\begin{array}{rl} #1 \end{array}}\right.}

\ifttm
\newenvironment{MCaseEnv}{\left\lbrace\begin{array}{rl}}{\end{array}\right.}
\else
\newenvironment{MCaseEnv}{\left\lbrace\begin{array}{rl}}{\end{array}\right.}
\fi

\def\MSkip{\ifttm\MCR\fi}

\ifttm
\def\MCR{\special{html:<br />}}
\else
\def\MCR{\ \\}
\fi


% Pragmas - Sind Schluesselwoerter, die dem Preprocessing sowie dem Konverter uebergeben werden und bestimmte
%           Aktionen ausloesen. Im Output (PDF und HTML) tauchen sie nicht auf.
\newcommand{\MPragma}[1]{%
\ifttm%
\special{html:<!-- mpragma;-;}#1\special{html:;; -->}%
\else%
% MPragmas werden vom Preprozessor direkt im LaTeX gefunden
\fi%
}

% Ersatz der Befehle textsubscript und textsuperscript, die ttm nicht kennt
\ifttm%
\newcommand{\MTextsubscript}[1]{\special{html:<sub>}#1\special{html:</sub>}}%
\newcommand{\MTextsuperscript}[1]{\special{html:<sup>}#1\special{html:</sup>}}%
\else%
\newcommand{\MTextsubscript}[1]{\textsubscript{#1}}%
\newcommand{\MTextsuperscript}[1]{\textsuperscript{#1}}%
\fi

%------------------ Einbindung von dia-Diagrammen ----------------------------------------------
% Beim preprocessing wird aus jeder dia-Datei eine tex-Datei und eine pdf-Datei erzeugt,
% diese werden hier jeweils im PDF und HTML eingebunden
% Parameter: Dateiname der mit dia erstellten Datei (OHNE die Endung .dia)
\ifttm%
\newcommand{\MDia}[1]{%
\MGraphicsSolo{#1minthtml.png}{}%
}
\else%
\newcommand{\MDia}[1]{%
\MGraphicsSolo{#1mintpdf.png}{scale=0.1667}%
}
\fi%

% subsup funktioniert im Ausdruck $D={\R}^+_0$, also \R geklammert und sup zuerst
% \ifttm
% \def\MSubsup#1#2#3{\special{html:<msubsup>} #1 #2 #3\special{html:</msubsup>}}
% \else
% \def\MSubsup#1#2#3{{#1}^{#3}_{#2}}
% \fi

%\input{local.tex}

% \ifttm
% \else
% \newwrite\mintlog
% \immediate\openout\mintlog=mintlog.txt
% \fi

% ----------------------- tikz autogenerator -------------------------------------------------------------------

\newcommand{\Mtikzexternalize}{\tikzexternalize}% wird bei Konvertierung ueber mconvert ggf. ausgehebelt!

\ifttm
\else
\tikzset%
{
  % Defines a custom style which generates pdf and converts to (low and hi-res quality) png and svg, then deletes the pdf
  % Important: DO NOT directly convert from pdf to hires-png or from svg to png with GraphViz convert as it has some problems and memory leaks
  png export/.style=%
  {
    external/system call/.add={}{; 
      pdf2svg "\image.pdf" "\image.svg" ; 
      convert -density 112.5 -transparent white "\image.pdf" "\image.png"; 
      inkscape --export-png="\image.4x.png" --export-dpi=450 --export-background-opacity=0 --without-gui "\image.svg"; 
      rm "\image.pdf"; rm "\image.log"; rm "\image.dpth"; rm "\image.idx"
    },
    external/force remake,
  }
}
\tikzset{png export}
\tikzsetexternalprefix{}
% PNGs bei externer Erzeugung in "richtiger" Groesse einbinden
\pgfkeys{/pgf/images/include external/.code={\includegraphics[scale=0.64]{#1}}}
\fi

% Spezielle Umgebung fuer Autogenerierung, Bildernamen sind nur innerhalb eines Moduls (einer MSection) eindeutig)

\newcommand{\MTIKZautofilename}{tikzautofile}

\ifttm
% HTML-Version: Vom Autogenerator erzeugte png-Datei einbinden, tikz selbst nicht ausfuehren (sprich: #1 schlucken)
\newcommand{\MTikzAuto}[1]{%
\addtocounter{MTIKZAutofilenumber}{1}
\renewcommand{\MTIKZautofilename}{mtikzauto_\arabic{MTIKZAutofilenumber}}
\MUGraphicsSolo{\MSectionID\MTIKZautofilename.4x.png}{scale=1}{\special{html:[[!-- svgstyle;}\MSectionID\MTIKZautofilename\special{html: //--]]}} % Styleinfos werden aus original-png, nicht 4x-png geholt!
%\MRegisterFile{\MSectionID\MTIKZautofilename.png} % not used right now
%\MRegisterFile{\MSectionID\MTIKZautofilename.svg}
}
\else%
% PDF-Version: Falls Autogenerator aktiv wird Datei automatisch benannt und exportiert
\newcommand{\MTikzAuto}[1]{%
\addtocounter{MTIKZAutofilenumber}{1}%
\renewcommand{\MTIKZautofilename}{mtikzauto_\arabic{MTIKZAutofilenumber}}
\tikzsetnextfilename{\MTIKZautofilename}%
#1%
}
\fi

% In einer reinen LaTeX-Uebersetzung kapselt der Preambelinclude-Befehl nur input,
% in einer konvertergesteuerten PDF/HTML-Uebersetzung wird er dagegen entfernt und
% die Preambeln an mintmod angehaengt, die Ersetzung wird von mconvert.pl vorgenommen.

\newcommand{\MPreambleInclude}[1]{\input{#1}}

% Globale Watermarksettings (werden auch nochmal zu Beginn jedes subsection gesetzt,
% muessen hier aber auch global ausgefuehrt wegen Einfuehrungsseiten und Inhaltsverzeichnis

\MWatermarkSettings
% ---------------------------------- Parametrisierte Aufgaben ----------------------------------------

\ifttm
\newenvironment{MPExercise}{%
\begin{MExercise}%
}{%
\special{html:<button name="Name_MPEX}\arabic{MExerciseCounter}\special{html:" id="MPEX}\arabic{MExerciseCounter}%
\special{html:" type="button" onclick="reroll('}\arabic{MExerciseCounter}\special{html:');">Neue Aufgabe erzeugen</button>}%
\end{MExercise}%
}
\else
\newenvironment{MPExercise}{%
\begin{MExercise}%
}{%
\end{MExercise}%
}
\fi

% Parameter: Name, Min, Max, PDF-Standard. Name in Deklaration OHNE backslash, im Code MIT Backslash
\ifttm
\newcommand{\MGlobalInteger}[4]{\special{html:%
<!-- onloadstart //-->%
MVAR.push(createGlobalInteger("}#1\special{html:",}#2\special{html:,}#3\special{html:,}#4\special{html:)); %
<!-- onloadstop //-->%
<!-- viewmodelstart //-->%
ob}#1\special{html:: ko.observable(rerollMVar("}#1\special{html:")),%
<!-- viewmodelstop //-->%
}%
}%
\else%
\newcommand{\MGlobalInteger}[4]{\newcounter{mvc_#1}\setcounter{mvc_#1}{#4}}
\fi

% Parameter: Name, Min, Max, PDF-Standard. Name in Deklaration OHNE backslash, im Code MIT Backslash, Wert ist Wurzel von value
\ifttm
\newcommand{\MGlobalSqrt}[4]{\special{html:%
<!-- onloadstart //-->%
MVAR.push(createGlobalSqrt("}#1\special{html:",}#2\special{html:,}#3\special{html:,}#4\special{html:)); %
<!-- onloadstop //-->%
<!-- viewmodelstart //-->%
ob}#1\special{html:: ko.observable(rerollMVar("}#1\special{html:")),%
<!-- viewmodelstop //-->%
}%
}%
\else%
\newcommand{\MGlobalSqrt}[4]{\newcounter{mvc_#1}\setcounter{mvc_#1}{#4}}% Funktioniert nicht als Wurzel !!!
\fi

% Parameter: Name, Min, Max, PDF-Standard zaehler, PDF-Standard nenner. Name in Deklaration OHNE backslash, im Code MIT Backslash
\ifttm
\newcommand{\MGlobalFraction}[5]{\special{html:%
<!-- onloadstart //-->%
MVAR.push(createGlobalFraction("}#1\special{html:",}#2\special{html:,}#3\special{html:,}#4\special{html:,}#5\special{html:)); %
<!-- onloadstop //-->%
<!-- viewmodelstart //-->%
ob}#1\special{html:: ko.observable(rerollMVar("}#1\special{html:")),%
<!-- viewmodelstop //-->%
}%
}%
\else%
\newcommand{\MGlobalFraction}[5]{\newcounter{mvc_#1}\setcounter{mvc_#1}{#4}} % Funktioniert nicht als Bruch !!!
\fi

% MVar darf im HTML nur in MEvalMathDisplay-Umgebungen genutzt werden oder in Strings die an den Parser uebergeben werden
\ifttm%
\newcommand{\MVar}[1]{\special{html:[var_}#1\special{html:]}}%
\else%
\newcommand{\MVar}[1]{\arabic{mvc_#1}}%
\fi

\ifttm%
\newcommand{\MRerollButton}[2]{\special{html:<button type="button" onclick="rerollMVar('}#1\special{html:');">}#2\special{html:</button>}}%
\else%
\newcommand{\MRerollButton}[2]{\relax}% Keine sinnvolle Entsprechung im PDF
\fi

% MEvalMathDisplay fuer HTML wird in mconvert.pl im preprocessing realisiert
% PDF: eine equation*-Umgebung (ueber amsmath)
% HTML: Eine Mathjax-Tex-Umgebung, deren Auswertung mit knockout-obervablen gekoppelt ist
% PDF-Version hier nur fuer pdflatex-only-Uebersetzung gegeben

\ifttm\else\newenvironment{MEvalMathDisplay}{\begin{equation*}}{\end{equation*}}\fi

% ---------------------------------- Spezialbefehle fuer AD ------------------------------------------

%Abk�rzung f�r \longrightarrow:
\newcommand{\lto}{\ensuremath{\longrightarrow}}

%Makro f�r Funktionen:
\newcommand{\exfunction}[5]
{\begin{array}{rrcl}
 #1 \colon  & #2 &\lto & #3 \\[.05cm]  
  & #4 &\longmapsto  & #5 
\end{array}}

\newcommand{\function}[5]{%
#1:\;\left\lbrace{\begin{array}{rcl}
 #2 &\lto & #3 \\
 #4 &\longmapsto  & #5 \end{array}}\right.}


%Die Identit�t:
\DeclareMathOperator{\Id}{Id}

%Die Signumfunktion:
\DeclareMathOperator{\sgn}{sgn}

%Zwei Betonungskommandos (k�nnen angepasst werden):
\newcommand{\highlight}[1]{#1}
\newcommand{\modstextbf}[1]{#1}
\newcommand{\modsemph}[1]{#1}


% ---------------------------------- Spezialbefehle fuer JL ------------------------------------------


\def\jccolorfkt{green!50!black} %Farbe des Funktionsgraphen
\def\jccolorfktarea{green!25!white} %Farbe der Fl"ache unter dem Graphen
\def\jccolorfktareahell{green!12!white} %helle Einf"arbung der Fl"ache unter dem Graphen
\def\jccolorfktwert{green!50!black} %Farbe einzelner Punkte des Graphen

\newcommand{\MPfadBilder}{Bilder}

\ifttm%
\newcommand{\jMD}{\,\MD}%
\else%
\newcommand{\jMD}{\;\MD}%
\fi%

\def\jHTMLHinweisBedienung{\MInputHint{%
Mit Hilfe der Symbole am oberen Rand des Fensters
k"onnen Sie durch die einzelnen Abschnitte navigieren.}}

\def\jHTMLHinweisEingabeText{\MInputHint{%
Geben Sie jeweils ein Wort oder Zeichen als Antwort ein.}}

\def\jHTMLHinweisEingabeTerm{\MInputHint{%
Klammern Sie Ihre Terme, um eine eindeutige Eingabe zu erhalten. 
Beispiel: Der Term $\frac{3x+1}{x-2}$ soll in der Form
\texttt{(3*x+1)/((x+2)^2}$ eingegeben werden (wobei auch Leerzeichen 
eingegeben werden k"onnen, damit eine Formel besser lesbar ist).}}

\def\jHTMLHinweisEingabeIntervalle{\MInputHint{%
Intervalle werden links mit einer "offnenden Klammer und rechts mit einer 
schlie"senden Klammer angegeben. Eine runde Klammer wird verwendet, wenn der 
Rand nicht dazu geh"ort, eine eckige, wenn er dazu geh"ort. 
Als Trennzeichen wird ein Komma oder ein Semikolon akzeptiert.
Beispiele: $(a, b)$ offenes Intervall,
$[a; b)$ links abgeschlossenes, rechts offenes Intervall von $a$ bis $b$. 
Die Eingabe $]a;b[$ f"ur ein offenes Intervall wird nicht akzeptiert.
F"ur $\infty$ kann \texttt{infty} oder \texttt{unendlich} geschrieben werden.}}

\def\jHTMLHinweisEingabeFunktionen{\MInputHint{%
Schreiben Sie Malpunkte (geschrieben als \texttt{*}) aus und setzen Sie Klammern um Argumente f�r Funktionen.
Beispiele: Polynom: \texttt{3*x + 0.1}, Sinusfunktion: \texttt{sin(x)}, 
Verkettung von cos und Wurzel: \texttt{cos(sqrt(3*x))}.}}

\def\jHTMLHinweisEingabeFunktionenSinCos{\MInputHint{%
Die Sinusfunktion $\sin x$ wird in der Form \texttt{sin(x)} angegeben, %
$\cos\left(\sqrt{3 x}\right)$ durch \texttt{cos(sqrt(3*x))}.}}

\def\jHTMLHinweisEingabeFunktionenExp{\MInputHint{%
Die Exponentialfunktion $\MEU^{3x^4 + 5}$ wird als
\texttt{exp(3 * x^4 + 5)} angegeben, %
$\ln\left(\sqrt{x} + 3.2\right)$ durch \texttt{ln(sqrt(x) + 3.2)}.}}

% ---------------------------------- Spezialbefehle fuer Fachbereich Physik --------------------------

\newcommand{\E}{{e}}
\newcommand{\ME}[1]{\cdot 10^{#1}}
\newcommand{\MU}[1]{\;\mathrm{#1}}
\newcommand{\MPG}[3]{%
  \ifnum#2=0%
    #1\ \mathrm{#3}%
  \else%
    #1\cdot 10^{#2}\ \mathrm{#3}%
  \fi}%
%

\newcommand{\MMul}{\MExponentensymbXYZl} % Nur eine Abkuerzung


% ---------------------------------- Stichwortfunktionialitaet ---------------------------------------

% mpreindexentry wird durch Auswahlroutine in conv.pl durch mindexentry substitutiert
\ifttm%
\def\MIndex#1{\index{#1}\special{html:<!-- mpreindexentry;;}#1\special{html:;;}\arabic{MSubjectArea}\special{html:;;}%
\arabic{chapter}\special{html:;;}\arabic{section}\special{html:;;}\arabic{subsection}\special{html:;;}\arabic{MEntryCounter}\special{html:; //-->}%
\setcounter{MLastIndex}{\value{MEntryCounter}}%
\addtocounter{MEntryCounter}{1}%
}%
% Copyrightliste wird als tex-Datei im preprocessing von conv.pl erzeugt und unter converter/tex/entrycollection.tex abgelegt
% Der input-Befehl funktioniert nur, wenn die aufrufende tex-Datei auf der obersten Ebene liegt (d.h. selbst kein input/include ist, insbesondere keine Moduldatei)
\def\MEntryList{} % \input funktioniert nicht, weil ttm (und damit das \input) ausgefuehrt wird, bevor Datei da ist
\else%
\def\MIndex#1{\index{#1}}
\def\MEntryList{\MAbort{Stichwortliste nur im HTML realisierbar}}%
\fi%

\def\MEntry#1#2{\textbf{#1}\MIndex{#2}} % Idee: MLastType auf neuen Entry-Typ und dann ein MLabel vergeben mit autogen-Nummer

% ---------------------------------- Befehle fuer Tests ----------------------------------------------

% MEquationItem stellt eine Eingabezeile der Form Vorgabe = Antwortfeld her, der zweite Parameter kann z.B. MSimplifyQuestion-Befehl sein
\ifttm
\newcommand{\MEquationItem}[2]{{#1}$\,=\,${#2}}%
\else%
\newcommand{\MEquationItem}[2]{{#1}$\;\;=\,${#2}}%
\fi

\ifttm
\newcommand{\MInputHint}[1]{%
\ifnum%
\if\value{MTestSite}>0%
\else%
{\color{blue}#1}%
\fi%
\fi%
}
\else
\newcommand{\MInputHint}[1]{\relax}
\fi

\ifttm
\newcommand{\MInTestHeader}{%
Dies ist ein einreichbarer Test:
\begin{itemize}
\item{Im Gegensatz zu den offenen Aufgaben werden beim Eingeben keine Hinweise zur Formulierung der mathematischen Ausdr�cke gegeben.}
\item{Der Test kann jederzeit neu gestartet oder verlassen werden.}
\item{Der Test kann durch die Buttons am Ende der Seite beendet und abgeschickt, oder zur�ckgesetzt werden.}
\item{Der Test kann mehrfach probiert werden. F�r die Statistik z�hlt die zuletzt abgeschickte Version.}
\end{itemize}
}
\else
\newcommand{\MInTestHeader}{%
\relax
}
\fi

\ifttm
\newcommand{\MInTestFooter}{%
\special{html:<button name="Name_TESTFINISH" id="TESTFINISH" type="button" onclick="finish_button('}\MTestName\special{html:');">Test auswerten</button>}%
\begin{html}
&nbsp;&nbsp;&nbsp;&nbsp;&nbsp;&nbsp;&nbsp;&nbsp;
<button name="Name_TESTRESET" id="TESTRESET" type="button" onclick="reset_button();">Test zur�cksetzen</button>
<br />
<br />
<div class="xreply">
<p name="Name_TESTEVAL" id="TESTEVAL">
Hier erscheint die Testauswertung!
<br />
</p>
</div>
\end{html}
}
\else
\newcommand{\MInTestFooter}{%
\relax
}
\fi


% ---------------------------------- Notationsmakros -------------------------------------------------------------

% Notationsmakros die nicht von der Kursvariante abhaengig sind

\newcommand{\MZahltrennzeichen}[1]{\renewcommand{\MZXYZhltrennzeichen}{#1}}

\ifttm
\newcommand{\MZahl}[3][\MZXYZhltrennzeichen]{\edef\MZXYZtemp{\noexpand\special{html:<mn>#2#1#3</mn>}}\MZXYZtemp}
\else
\newcommand{\MZahl}[3][\MZXYZhltrennzeichen]{{}#2{#1}#3}
\fi

\newcommand{\MEinheitenabstand}[1]{\renewcommand{\MEinheitenabstXYZnd}{#1}}
\ifttm
\newcommand{\MEinheit}[2][\MEinheitenabstXYZnd]{{}#1\edef\MEINHtemp{\noexpand\special{html:<mi mathvariant="normal">#2</mi>}}\MEINHtemp} 
\else
\newcommand{\MEinheit}[2][\MEinheitenabstXYZnd]{{}#1 \mathrm{#2}} 
\fi

\newcommand{\MExponentensymbol}[1]{\renewcommand{\MExponentensymbXYZl}{#1}}
\newcommand{\MExponent}[2][\MExponentensymbXYZl]{{}#1{} 10^{#2}} 

%Punkte in 2 und 3 Dimensionen
\newcommand{\MPointTwo}[3][]{#1(#2\MCoordPointSep #3{}#1)}
\newcommand{\MPointThree}[4][]{#1(#2\MCoordPointSep #3\MCoordPointSep #4{}#1)}
\newcommand{\MPointTwoAS}[2]{\left(#1\MCoordPointSep #2\right)}
\newcommand{\MPointThreeAS}[3]{\left(#1\MCoordPointSep #2\MCoordPointSep #3\right)}

% Masseinheit, Standardabstand: \,
\newcommand{\MEinheitenabstXYZnd}{\MThinspace} 

% Horizontaler Leerraum zwischen herausgestellter Formel und Interpunktion
\ifttm
\newcommand{\MDFPSpace}{\,}
\newcommand{\MDFPaSpace}{\,\,}
\newcommand{\MBlank}{\ }
\else
\newcommand{\MDFPSpace}{\;}
\newcommand{\MDFPaSpace}{\;\;}
\newcommand{\MBlank}{\ }
\fi

% Satzende in herausgestellter Formel mit horizontalem Leerraum
\newcommand{\MDFPeriod}{\MDFPSpace .}

% Separation von Aufzaehlung und Bedingung in Menge
\newcommand{\MCondSetSep}{\,:\,} %oder '\mid'

% Konverter kennt mathopen nicht
\ifttm
\def\mathopen#1{}
\fi

% -----------------------------------START Rouletteaufgaben ------------------------------------------------------------

\ifttm
% #1 = Dateiname, #2 = eindeutige ID fuer das Roulette im Kurs
\newcommand{\MDirectRouletteExercises}[2]{
\begin{MExercise}
\texttt{Im HTML erscheinen hier Aufgaben aus einer Aufgabenliste...}
\end{MExercise}
}
\else
\newcommand{\MDirectRouletteExercises}[2]{\relax} % wird durch mconvert.pl gefunden und ersetzt
\fi


% ---------------------------------- START Makros, die von der Kursvariante abhaengen ----------------------------------

\ifvariantunotation
  % unotation = An Universitaeten uebliche Notation
  \def\MVariant{unotation}

  % Trennzeichen fuer Dezimalzahlen
  \newcommand{\MZXYZhltrennzeichen}{.}

  % Exponent zur Basis 10 in der Exponentialschreibweise, 
  % Standardmalzeichen: \times
  \newcommand{\MExponentensymbXYZl}{\times} 

  % Begrenzungszeichen fuer offene Intervalle
  \newcommand{\MoIl}[1][]{\mbox{}#1(\mathopen{}} % bzw. ']'
  \newcommand{\MoIr}[1][]{#1)\mbox{}} % bzw. '['

  % Zahlen-Separation im IntervaLL
  \newcommand{\MIntvlSep}{,} %oder ';'

  % Separation von Elementen in Mengen
  \newcommand{\MElSetSep}{,} %oder ';'

  % Separation von Koordinaten in Punkten
  \newcommand{\MCoordPointSep}{,} %oder ';' oder '|', '\MThinspace|\MThinspace'

\else
  % An dieser Stelle wird angenommen, dass std-Variante aktiv ist
  % std = beschlossene Notation im TU9-Onlinekurs 
  \def\MVariant{std}

  % Trennzeichen fuer Dezimalzahlen
  \newcommand{\MZXYZhltrennzeichen}{,}

  % Exponent zur Basis 10 in der Exponentialschreibweise, 
  % Standardmalzeichen: \times
  \newcommand{\MExponentensymbXYZl}{\times} 

  % Begrenzungszeichen fuer offene Intervalle
  \newcommand{\MoIl}[1][]{\mbox{}#1]\mathopen{}} % bzw. '('
  \newcommand{\MoIr}[1][]{#1[\mbox{}} % bzw. ')'

  % Zahlen-Separation im IntervaLL
  \newcommand{\MIntvlSep}{;} %oder ','
  
  % Separation von Elementen in Mengen
  \newcommand{\MElSetSep}{;} %oder ','

  % Separation von Koordinaten in Punkten
  \newcommand{\MCoordPointSep}{;} %oder '|', '\MThinspace|\MThinspace'

\fi



% ---------------------------------- ENDE Makros, die von der Kursvariante abhaengen ----------------------------------


% diese Kommandos setzen Mathemodus vorraus
\newcommand{\MGeoAbstand}[2]{[\overline{{#1}{#2}}]}
\newcommand{\MGeoGerade}[2]{{#1}{#2}}
\newcommand{\MGeoStrecke}[2]{\overline{{#1}{#2}}}
\newcommand{\MGeoDreieck}[3]{{#1}{#2}{#3}}

%
\ifttm
\newcommand{\MOhm}{\special{html:<mn>&#x3A9;</mn>}}
\else
\newcommand{\MOhm}{\Omega} %\varOmega
\fi


\def\PERCTAG{\MAbort{PERCTAG ist zur internen verwendung in mconvert.pl reserviert, dieses Makro darf sonst nicht benutzt werden.}}

% Im Gegensatz zu einfachen html-Umgebungen werden MDirectHTML-Umgebungen von mconvert.pl am ganzen ttm-Prozess vorbeigeschleust und aus dem PDF komplett ausgeschnitten
\ifttm%
\newenvironment{MDirectHTML}{\begin{html}}{\end{html}}%
\else%
\newenvironment{MDirectHTML}{\begin{html}}{\end{html}}%
\fi

% Im Gegensatz zu einfachen Mathe-Umgebungen werden MDirectMath-Umgebungen von mconvert.pl am ganzen ttm-Prozess vorbeigeschleust, ueber MathJax realisiert, und im PDF als $$ ... $$ gesetzt
\ifttm%
\newenvironment{MDirectMath}{\begin{html}}{\end{html}}%
\else%
\newenvironment{MDirectMath}{\begin{equation*}}{\end{equation*}}% Vorsicht, auch \[ und \] werden in amsmath durch equation* redefiniert
\fi

% ---------------------------------- Location Management ---------------------------------------------

% #1 = buttonname (muss in files/images liegen und Format 48x48 haben), #2 = Vollstaendiger Einrichtungsname, #3 = Kuerzel der Einrichtung,  #4 = Name der include-texdatei
\ifttm
\newcommand{\MLocationSite}[3]{\special{html:<!-- mlocation;;}#1\special{html:;;}#2\special{html:;;}#3\special{html:;; //-->}}
\else
\newcommand{\MLocationSite}[3]{\relax}
\fi

% ---------------------------------- Copyright Management --------------------------------------------

\newcommand{\MCCLicense}{%
{\color{green}\textbf{CC BY-SA 3.0}}
}

\newcommand{\MCopyrightLabel}[1]{ (\MSRef{L_COPYRIGHTCOLLECTION}{Lizenz})\MLabel{#1}}

% Copyrightliste wird als tex-Datei im preprocessing erzeugt und unter converter/tex/copyrightcollection.tex abgelegt
% Der input-Befehl funktioniert nur, wenn die aufrufende tex-Datei auf der obersten Ebene liegt (d.h. selbst kein input/include ist, insbesondere keine Moduldatei)
\newcommand{\MCopyrightCollection}{\input{copyrightcollection.tex}}

% MCopyrightNotice fuegt eine Copyrightnotiz ein, der parser ersetzt diese durch CopyrightNoticePOST im preparsing, diese Definition wird nur fuer reine pdflatex-Uebersetzungen gebraucht
% Parameter: #1: Kurze Lizenzbeschreibung (typischerweise \MCCLicense)
%            #2: Link zum Original (http://...) oder NONE falls das Bild selbst ein Original ist, oder TIKZ falls das Bild aus einer tikz-Umgebung stammt
%            #3: Link zum Autor (http://...) oder MINT falls Original im MINT-Kolleg erstellt oder NONE falls Autor unbekannt
%            #4: Bemerkung (z.B. dass Datei mit Maple exportiert wurde)
%            #5: Labelstring fuer existierendes Label auf das copyrighted Objekt, mit MCopyrightLabel erzeugt
%            Keines der Felder darf leer sein!
\newcommand{\MCopyrightNotice}[5]{\MCopyrightNoticePOST{#1}{#2}{#3}{#4}{#5}}

\ifttm%
\newcommand{\MCopyrightNoticePOST}[5]{\relax}%
\else%
\newcommand{\MCopyrightNoticePOST}[5]{\relax}%
\fi%

% ---------------------------------- Meldungen fuer den Benutzer des Konverters ----------------------
\MPragma{mintmodversion;P0.1.0}
\MPragma{usercomment;This is file mintmod.tex version P0.1.0}


% ----------------------------------- Spezialelemente fuer Konfigurationsseite, werden nicht von mintscripts.js verwaltet --

% #1 = DOM-id der Box
\ifttm\newcommand{\MConfigbox}[1]{\special{html:<input cfieldtype="2" type="checkbox" name="Name_}#1\special{html:" id="}#1\special{html:" onchange="confHandlerChange('}#1\special{html:');"/>}}\fi % darf im PDF nicht aufgerufen werden!


\MPragma{MathSkip}
\Mtikzexternalize

\begin{document}

\MSection{Basic Concepts of Descriptive Vector Geometry}
\MLabel{VBKM10}
\MSetSectionID{VBKM10} % wird fuer tikz-Dateien verwendet

\begin{MSectionStart}
\MDeclareSiteUXID{VBKM10_START}

\MModstartBox
\end{MSectionStart}

\MSubsection{From Arrows to Vectors}
\MLabel{VBKM10_PfeilVektor}

\begin{MIntro}
\MLabel{VBKM10_PfeilVektor_Intro}
\MDeclareSiteUXID{VBKM10_PfeilVektor_Intro}

The basic idea underlying the mathematical concept of a vector is rooted in physics. In science there are quantities
that are described by one single number, their magnitude. Such quantities include voltage, work, 
or power. In mathematical terms, these quantities are simply described by elements of the set of real numbers. 
There are other quantities which have not only a certain magnitude but also a certain \textit{direction}, 
such as force or velocity. For example, a force that acts on a body at a certain point can be visualised as 
an arrow of a corresponding length starting at that point. The direction of the arrow then corresponds to the direction 
in which the force acts. This is illustrated in the figure below.

\begin{center}
\MTikzAuto{%
\begin{tikzpicture} 
%Pfeil
\draw[color=black,->] (0,0) -- (3,1);
\draw[color=black] (1.5,0.8) node[anchor=east] {\footnotesize Length $\hat{=}$ Magnitude of the force};
\draw[color=black] (1.5,0.2) node[anchor=west] {\footnotesize Direction $\hat{=}$ Direction in which the force acts};
\draw[color=black] (3,1) node[anchor=west] {\footnotesize Arrow $\hat{=}$ Force};
%Punkt
\draw[fill=red] (0,0) circle (2pt);
\draw[color=red] (0,0) node[anchor=north east] {\footnotesize Initial point};
\end{tikzpicture}       
}%
\end{center}

In mathematics, such quantities are described by vectors. In the sciences, the magnitudes of vectors have certain 
units of measure (e.g. forces are measured in Newton). From a purely mathematical point of view, however, these units of 
measure are not relevant, so they are omitted here. The concept of a vector is the main topic of 
this section and will be discussed in detail in Subsection~\MNRef{VBKM10_Vektoren}. Since vectors are considered not only 
in two-dimensional space (i.e. in the plane) but also in three-dimensional space (i.e. in the space), the concept 
of coordinate systems and points introduced in Module~\MNRef{VBKM09} will be extended to three dimensions.
This is done in Subsection~\MNRef{VBKM10_Raumkoordinaten}. Finally, we see that certain operations can be applied
to vectors. In Subsection~\MNRef{VBKM10_Vektorrechnung} we will study how these vector operations are carried out. 
\end{MIntro}

\begin{MXContent}{Coordinate Systems in Three-Dimensional Space}{Space Coordinates}{STD}
\MLabel{VBKM10_Raumkoordinaten}
\MDeclareSiteUXID{VBKM10_Raumkoordinaten}

In Section~\MNRef{M09_1kartesisch} of the previous Module~\MNRef{VBKM09} we introduced Cartesian coordinate systems
and points in the plane described by coordinates with respect to these coordinate systems. A solid understanding
of these concepts is now presumed in this Module. To describe a \MEntry{point in three-dimensional space}{point (three-dimensional)},
three \MEntry{coordinates}{coordinates (in three dimensions)} are required. Thus, a 
\MEntry{coordinate system}{coordinate system (three-dimensional)} in three-dimensional space needs 
three \MEntry{axes}{axes (in three-dimensional space)}, the $x$-axis, $y$-axis and $z$-axis 
(sometimes called the $x_1$-axis, $x_2$-axis, and $x_3$-axis). Usually, points will be denoted by 
upper-case Latin letters $P,Q,R,\MHDots$, and their coordinates will be denoted by lower-case Latin letters 
$a,b,c,x,y,z,\MHDots$ as variables. Extending the notation from Module~\MNRef{VBKM09}, the coordinates of a 
point are written, for example, as follows:
\[
 P=\MPointThree{1}{3}{0}
\]
or
\[
 Q=\MPointThree{-2}{2}{3}\MDFPeriod
\]
Here, the $x$-coordinate of the point $Q$ is $-2$, its $y$-coordinate is $2$ and its $z$-coordinate is~$3$.
The point with the coordinates $\MPointThree{0}{0}{0}$ is called the \textbf{origin}, and it is denoted by the symbol
$O$. All these points are drawn in the figure below.

\begin{center}
\MTikzAuto{%
\begin{tikzpicture}[>=stealth]
% Dashed lines for the points P, Q
\draw[dashed, color=gray] (1,0) -- (1,3);
\draw[dashed, color=gray] (1,3) -- (0,3);
\draw[dashed, color=gray] (xyz cs:x=0,z=3) -- (xyz cs:x=-2,z=3);
\draw[dashed, color=gray] (xyz cs:x=-2,z=3) -- (xyz cs:x=-2,z=0);
\draw[dashed, color=gray] (xyz cs:x=-2,y=0,z=3) -- (xyz cs:x=-2,y=2,z=3);
\draw[dashed, color=gray] (xyz cs:x=0,y=2,z=3) -- (xyz cs:x=-2,y=2,z=3);
\draw[dashed, color=gray] (xyz cs:x=-2,y=2,z=3) -- (xyz cs:x=-2,y=2,z=0);
\draw[dashed, color=gray] (xyz cs:x=0,y=2,z=3) -- (xyz cs:x=0,y=2,z=0);
\draw[dashed, color=gray] (xyz cs:x=0,y=2,z=0) -- (xyz cs:x=-2,y=2,z=0);
\draw[dashed, color=gray] (xyz cs:y=0,z=3) -- (xyz cs:y=2,z=3); 
% 
% % Dots and labels for P, Q
\draw[fill=red] (1,3) circle (1.5pt);
\draw[color=red] (1,3) node[right] {\footnotesize $P=\MPointThree{1}{3}{0}$};
\draw[fill=violet] (xyz cs:x=-2,y=2,z=3) circle (1.5pt);
\draw[color=violet] (xyz cs:x=-2,y=2,z=3) node[left] {\footnotesize $Q=\MPointThree{-2}{2}{3}$};
% The origin
\node[align=center] at (2,-2) (ori) {\footnotesize $O=\MPointThree{0}{0}{0}$\\(Origin)};
\draw[->,help lines,shorten >=3pt] (ori) .. controls (0.5,-1.5) and (0.8,-1) .. (0,0,0);
% The axes
\draw[->] (xyz cs:x=-3.5) -- (xyz cs:x=3.5) node[above] {\footnotesize $x$};
\draw[->] (xyz cs:y=-3.5) -- (xyz cs:y=3.5) node[right] {\footnotesize $y$};
\draw[->] (xyz cs:z=-3.5) -- (xyz cs:z=3.5) node[left] {\footnotesize $z$};
% The ticks
\foreach \coo in {-3,-2,-1,1,2,3}
{
  \draw (\coo,-3pt) -- (\coo,3pt) node[below=4pt] {\footnotesize \coo};
  \draw (-3pt,\coo) -- (3pt,\coo) node[left=4pt] {\footnotesize \coo};
  \draw (xyz cs:y=-0.1pt,z=\coo) -- (xyz cs:y=0.1pt,z=\coo) node[below=3pt] {\scriptsize \coo};
}
\end{tikzpicture}
}
\end{center}
The dashed lines in this figure indicate how the coordinates of points in such a three-dimensional 
representation can be drawn and read off. Note that these lines are all parallel to the coordinate axes.

We will only consider coordinate systems in three-dimensional space with perpendicular coordinate axes - these are \MEntry{Cartesian coordinate systems}{Cartesian coordinate system (three-dimensional)}.
Furthermore, we will use the common mathematical convention that coordinate systems in three-dimensional space
are \textbf{right-handed}. Sometimes these are also called \textbf{positively oriented}. This means that the
positive directions of the $x$, $y$, and $z$-axis can be determined by means of the \textit{right-hand} rule 
as illustrated in the figure below.

\begin{center}
\MUGraphicsSolo{rechtehand.png}{scale=0.4}{width:368px}
\end{center}
%Bemerkung: Bild von hier https://de.wikipedia.org/wiki/Rechtssystem_%28Mathematik%29, ist public domain, deshalb keine Lizenz nötig.

However, there are various possible representations. In the figure above showing the points $P$ and $Q$,
the $x$-axis points to the right, the $y$-axis points up, and the $z$-axis points perpendicularly outwards from the drawing plane. 
In the figure which illustrates the right-hand rule, the $x$-axis points to the right, the $y$-axis points 
backwards into the drawing plane, and the $z$-axis points up. However, both coordinate systems are right-handed.

\begin{MExercise}
Specify the coordinates of the points indicated in the figure below. Consider how all indicated points can be collected 
into one mathematical object.
\begin{center}
\MTikzAuto{%
\begin{tikzpicture}[>=stealth]
% Dashed lines for the points 
%A,B,C,D:
\draw[dashed, color=gray] (2,0) -- (2,1);
\draw[dashed, color=gray] (2,0) -- (2,-2);
\draw[color=green] (2,-2,0) -- (2,-2,1);
\draw[dashed, color=gray] (2,-2,0) -- (0,0,0);
\draw[dashed, color=gray] (2,-2,1) -- (0,0,1);
%P:
\draw[dashed, color=gray] (-2,0,0) -- (-2,1,0);
\draw[dashed, color=gray] (-2,1,0) -- (-2,1,-1);
%\draw[dashed, color=gray] (-2,1,0) -- (0,1,0);
\draw[dashed, color=gray] (0,1,-1) -- (-2,1,-1);
\draw[dashed, color=gray] (0,1,-1) -- (0,0,-1);
% 

% The axes
\draw[->] (xyz cs:x=-3.5) -- (xyz cs:x=3.5) node[above] {\footnotesize $x$};
\draw[->] (xyz cs:y=-3.5) -- (xyz cs:y=3.5) node[right] {\footnotesize $y$};
\draw[->] (xyz cs:z=-3.5) -- (xyz cs:z=3.5) node[left] {\footnotesize $z$};
% The ticks
\foreach \coo in {-3,-2,-1,1,2,3}
{
  \draw (\coo,-3pt) -- (\coo,3pt) node[below=4pt] {\footnotesize \coo};
  \draw (-3pt,\coo) -- (3pt,\coo) node[left=4pt] {\footnotesize \coo};
  \draw (xyz cs:y=-0.1pt,z=\coo) -- (xyz cs:y=0.1pt,z=\coo) node[below=3pt] {\scriptsize \coo};
}
% % Dots and labels for the points
\draw[fill=red] (2,0,0) circle (1.5pt);
\draw[color=red] (2,0,0) node[anchor=south west] {\footnotesize $A$};
\draw[fill=blue] (2,1,0) circle (1.5pt);
\draw[color=blue] (2,1,0) node[right] {\footnotesize $B$};
\draw[fill=violet] (2,-2,0) circle (1.5pt);
\draw[color=violet] (2,-2,0) node[right] {\footnotesize $C$};
\draw[fill=green] (2,-2,1) circle (1.5pt);
\draw[color=green] (2,-2,1) node[below] {\footnotesize $D$};
\draw[fill=black] (-2,1,-1) circle (1.5pt);
\draw[color=black] (-2,1,-1) node[left] {\footnotesize $P$};
\draw[color=green] (0,0,0) -- (0,0,1);
\end{tikzpicture}
}
\end{center}

\begin{MExerciseItems}
\item{\MEquationItem{$A$}{\MLFunctionQuestion{15}{(2,0,0)}{5}{x}{5}{Point1}}.}
\item{\MEquationItem{$B$}{\MLFunctionQuestion{15}{(2,1,0)}{5}{x}{5}{Point2}}.}
\item{\MEquationItem{$C$}{\MLFunctionQuestion{15}{(2,-2,0)}{5}{x}{5}{Point3}}.}
\item{\MEquationItem{$D$}{\MLFunctionQuestion{15}{(2,-2,1)}{5}{x}{5}{Point4}}.}
\item{\MEquationItem{$P$}{\MLFunctionQuestion{15}{(-2,1,-1)}{5}{x}{5}{Point5}}.}
\end{MExerciseItems}
\MInputHint{Enter points in the form $(x;y;z)$. Enter, for example, \texttt{(8;-4;15)} for the point with $x$-coordinate $8$, $y$-coordinate $-4$, and $z$-coordinate $15$.}

\begin{MHint}{Solution}
The coordinate triples of the indicated points are:
\[
 A=\MPointThree{2}{0}{0} \MDFPSpace,
\]
\[
 B=\MPointThree{2}{1}{0} \MDFPSpace,
\]
\[
 C=\MPointThree{2}{-2}{0} \MDFPSpace,
\]
\[
 D=\MPointThree{2}{-2}{1} \MDFPSpace,
\]
\[
 P=\MPointThree{-2}{1}{-1}\MDFPeriod
\]
The set of all points indicated in the figure above is
\[
 \{A\MElSetSep B\MElSetSep C\MElSetSep D\MElSetSep P\} = \{\MPointThree{2}{0}{0}\MElSetSep\MPointThree{2}{1}{0} \MElSetSep\MPointThree{2}{-2}{0} \MElSetSep\MPointThree{2}{-2}{1} \MElSetSep\MPointThree{-2}{1}{-1}\} \MDFPeriod
\]

\end{MHint}
\end{MExercise}

As in the two-dimensional case discussed in Section~\MNRef{VBKM09_Punkte}, an arbitrary number of points in three-dimensional space can be collected into a \MEntry{set of points}{set of points (in three dimensions)}.
The following notation is used:

\begin{MInfo}
The set of all points (in space) specified as coordinate triples with respect to a given Cartesian coordinate system is 
denoted by
\[
 \R^3 := \{ \MPointThree{x}{y}{z}\MCondSetSep x\in\R\wedge y\in\R\wedge z\in\R \} \MDFPeriod
\]
The symbol $\R^3$ reads as ``$\R$ three'' or ``$\R$ to the power of three''. This indicates that a point can be uniquely 
described by a coordinate triple (also known as an ordered triple) consisting of three real numbers.   
\end{MInfo}

\end{MXContent}


\begin{MXContent}{Vectors in the Plane and in the Space}{Vectors}{STD}
\MLabel{VBKM10_Vektoren}
\MDeclareSiteUXID{VBKM10_Vektoren}
Points on the plane or in space that are defined as ordered pairs or triples with respect 
to a given coordinate system can be connected by line segments. Assigning a direction to these line segments 
(one of the end points of the segment is specified as the initial point and 
the other one is specified as the end point) results in arrows that point from one point to the other 
(see left and right figure below for the two-dimensional and the three-dimensional cases).

\begin{center}
\begin{tabular}{cc}
\MTikzAuto{
\begin{tikzpicture}[>=stealth]
%Koordinatensystem
\draw[->,color=black] (-1.5,0) -- (4.5,0);
\foreach \x in {-1,1,2,3,4}
\draw[shift={(\x,0)},color=black] (0pt,2pt) -- (0pt,-2pt) node[below] {\footnotesize $\x$};
\draw[->,color=black] (0,-1.5) -- (0,4.5);
\foreach \y in {-1,1,2,3,4}
\draw[shift={(0,\y)},color=black] (2pt,0pt) -- (-2pt,0pt) node[left] {\footnotesize $\y$};
\draw[color=black] (-10pt,-8pt) node[right] {\footnotesize $0$};
%Achsenbeschriftung
\draw (4.5,0) node[anchor=north west] {$x$};
\draw (-0.5,4.8) node[anchor=north west] {$y$};
%Punkte
\draw [fill = black] (0,0) circle (1.5pt);
\draw [color=black] (0,0) node[anchor=south east] {\footnotesize $O=\MPointTwo{0}{0}$};
\draw [fill = red] (2,1) circle (1.5pt);
\draw [color=red] (2,1) node[anchor=north west] {\footnotesize $P=\MPointTwo{2}{1}$};
\draw [fill = blue] (4,1) circle (1.5pt);
\draw [color=blue] (4,1) node[anchor=south west] {\footnotesize $Q=\MPointTwo{4}{1}$};
\draw [fill = violet] (1,3) circle (1.5pt);
\draw [color=violet] (1,3) node[anchor=south] {\footnotesize $R=\MPointTwo{1}{3}$};
%Pfeile
\draw[->, line width = 1.5pt] (0,0) -- (2,1);
\draw[->, line width = 1.5pt] (4,1) -- (1,3);
\end{tikzpicture}
}

&

\MTikzAuto{%
\begin{tikzpicture}[>=stealth]

% The axes
\draw[->] (xyz cs:x=-2.5) -- (xyz cs:x=3.5) node[above] {\footnotesize $x$};
\draw[->] (xyz cs:y=-2.5) -- (xyz cs:y=3.5) node[right] {\footnotesize $y$};
\draw[->] (xyz cs:z=-2.5) -- (xyz cs:z=3.5) node[left] {\footnotesize $z$};
% The ticks
\foreach \coo in {-2,-1,1,2,3}
{
  \draw (\coo,-3pt) -- (\coo,3pt) node[below=4pt] {\footnotesize \coo};
  \draw (-3pt,\coo) -- (3pt,\coo) node[left=4pt] {\footnotesize \coo};
  \draw (xyz cs:y=-0.1pt,z=\coo) -- (xyz cs:y=0.1pt,z=\coo) node[below=3pt] {\scriptsize \coo};
}
% % Dots and labels for the points
\draw[fill=black] (0,0,0) circle (1.5pt);
\draw[color=black] (0,0,0) node[anchor=south east] {\footnotesize $O=\MPointThree{0}{0}{0}$};
\draw[fill=red] (2,-1,2) circle (1.5pt);
\draw[color=red] (2,-1,2) node[anchor=north] {\footnotesize $P=\MPointThree{2}{-1}{2}$};
\draw[fill=blue] (3,1,0) circle (1.5pt);
\draw[color=blue] (3,1,0) node[anchor=west] {\footnotesize $Q=\MPointThree{3}{1}{0}$};
\draw[fill=violet] (1,3,-1) circle (1.5pt);
\draw[color=violet] (1,3,-1) node[anchor=west] {\footnotesize $R=\MPointThree{1}{3}{-1}$};
%Arrows
\draw[->, line width = 1.5pt] (0,0,0) -- (2,-1,2);
\draw[->, line width = 1.5pt] (3,1,0) -- (1,3,-1);
\end{tikzpicture}
}
\end{tabular}
\end{center}
According to these figures, an arrow provides the following information: it specifies how to get
from the initial point (at the foot of the arrow) to the end point (at the tip of the arrow).
For example, the arrow that connects the point $Q$ to the point $R$ in the left figure above specifies 
that starting from the initial point $Q$ one has to move $3$ units to the left and $2$ units  
upwards to get to the terminal point $R$. In more mathematical terms: starting from $Q$, shift by $-3$ in 
the $x$-direction and by $2$ in the $y$-direction. It is even simpler for the arrow that connects the origin to the 
point $P$ in the left figure above: to get from $O$ to $P$ move $2$ units in the $x$-direction and 
$1$ unit in the $y$-direction. Of course, these values are exactly the coordinates of the point $P$. 

\begin{MExercise}
For the arrows in the right-hand side figure above (three-dimensional case), specify the (signed) movements 
in the three coordinate directions that are required to get from the initial point to the terminal 
point of the corresponding arrow. Proceed in the same way as explained above for two dimensions.

\begin{itemize}
 \item[{}] For the arrow from $O$ to $P$, we have: 
 
  \begin{MExerciseItems}
  \item{in the $x$-direction: \MLFunctionQuestion{15}{2}{5}{x}{5}{Pfeil1x},}
  \item{in the $y$-direction: \MLFunctionQuestion{15}{-1}{5}{x}{5}{Pfeil1y},}
  \item{in the $z$-direction: \MLFunctionQuestion{15}{2}{5}{x}{5}{Pfeil1z}.}
  \end{MExerciseItems} 
  \item[{}] For the arrow from $Q$ to $R$, we have:
  
  \begin{MExerciseItems}
  \item{in the $x$-direction: \MLFunctionQuestion{15}{-2}{5}{x}{5}{Pfeil2x},}
  \item{in the $y$-direction: \MLFunctionQuestion{15}{2}{5}{x}{5}{Pfeil2y},}
  \item{in the $z$-direction: \MLFunctionQuestion{15}{-1}{5}{x}{5}{Pfeil2z}.}
  \end{MExerciseItems} 
\end{itemize}

\begin{MHint}{Solution}

\begin{itemize}
 \item[{}] For the arrow from $O$ to $P$, we have: 
 
  \begin{MExerciseItems}
  \item{in the $x$-direction: $2$,}
  \item{in the $y$-direction: $-1$,}
  \item{in the $z$-direction: $2$.}
  \end{MExerciseItems} 
  \item[{}] For the arrow from $Q$ to $R$, we have:
  
  \begin{MExerciseItems}
  \item{in the $x$-direction: $-2$,}
  \item{in the $y$-direction: $2$,}
  \item{in the $z$-direction: $-1$.}
  \end{MExerciseItems} 
\end{itemize}

\end{MHint}
\end{MExercise}

For the arrows connecting $Q$ to $R$, movements in the corresponding coordinate directions 
are determined by the coordinates of the points at the foot and at the tip of the arrows. 
Thus, in the two-dimensional case, we have:
\[
 \left.\begin{array}{r}R=\MPointTwo{1}{3} \\ Q=\MPointTwo{4}{1}\end{array}\right\}\MDFPSpace\Rightarrow\MDFPSpace\left\{\begin{array}{l}\textrm{in}\MBlank x\textrm{-direction:}\MDFPSpace-3=1-4 \MDFPSpace\\ \textrm{in}\MBlank y\textrm{-direction:}\MDFPSpace2=3-1 \MDFPSpace ,\end{array}\right.
\]
and in the three-dimensional case:
\[
 \left.\begin{array}{r}R=\MPointThree{1}{3}{-1} \\ Q=\MPointThree{3}{1}{0}\end{array}\right\}\MDFPSpace\Rightarrow\MDFPSpace\left\{\begin{array}{l}\textrm{in}\MBlank x\textrm{-direction:}\MDFPSpace-2=1-3 \\ \textrm{in}\MBlank y\textrm{-direction:}\MDFPSpace2=3-1 \\ \textrm{in}\MBlank z\textrm{-direction:}\MDFPSpace-1=-1-0\MDFPeriod\end{array}\right.
\]
The movements in the different coordinate directions are the differences of the coordinates 
of the terminal point and the initial point of the arrow. This means that all arrows connecting 
pairs of points with equal coordinate differences only differ from each other by a parallel translation, i.e. 
they retain their direction. The pairs of points $P$ and $Q$, $A$ and $B$, $O$ and $R$ indicated in 
the figure below are each connected by arrows that can be made to coincide by parallel translations.
\begin{center}
\MTikzAuto{
\begin{tikzpicture}[>=stealth]
%Koordinatensystem
\draw[->,color=black] (-3.5,0) -- (4.5,0);
\foreach \x in {-3,-2,-1,1,2,3,4}
\draw[shift={(\x,0)},color=black] (0pt,2pt) -- (0pt,-2pt) node[below] {\footnotesize $\x$};
\draw[->,color=black] (0,-2.5) -- (0,3.5);
\foreach \y in {-2,-1,1,2,3}
\draw[shift={(0,\y)},color=black] (2pt,0pt) -- (-2pt,0pt) node[left] {\footnotesize $\y$};
\draw[color=black] (-10pt,-8pt) node[right] {\footnotesize $0$};
%Achsenbeschriftung
\draw (4.5,0) node[anchor=north west] {$x$};
\draw (-0.5,3.8) node[anchor=north west] {$y$};
%Punkte
\draw [fill = black] (0,0) circle (1.5pt);
\draw [color=black] (0,0) node[anchor=south west] {\footnotesize $O=\MPointTwo{0}{0}$};
\draw [fill = red] (-3,1) circle (1.5pt);
\draw [color=red] (-3,1) node[anchor=south west] {\footnotesize $R=\MPointTwo{-3}{1}$};
\draw [fill = blue] (4,1) circle (1.5pt);
\draw [color=blue] (4,1) node[anchor=south west] {\footnotesize $P=\MPointTwo{4}{1}$};
\draw [fill = violet] (1,2) circle (1.5pt);
\draw [color=violet] (1,2) node[anchor=south] {\footnotesize $Q=\MPointTwo{1}{2}$};
\draw [fill = green] (2,-2) circle (1.5pt);
\draw [color=green] (2,-2) node[anchor=north west] {\footnotesize $A=\MPointTwo{2}{-2}$};
\draw [fill = brown] (-1,-1) circle (1.5pt);
\draw [color=brown] (-1,-1) node[anchor=east] {\footnotesize $B=\MPointTwo{-1}{-1}$};
%Pfeile
\draw[->, line width = 1.5pt] (4,1) -- (1,2);
\draw[->, line width = 1.5pt] (0,0) -- (-3,1);
\draw[->, line width = 1.5pt] (2,-2) -- (-1,-1);
\end{tikzpicture}
} 
\end{center}
Here, an infinite number of pairs of points can be found that are connected by such an arrow. 
This idea works analogously in the three-dimensional case.

Each arrow in the figure above provides the same information, namely a shift by $-3$ in 
the $x$-direction and $1$ in the $y$-direction. So what could be more natural than regarding each 
of these arrows only as a representation (a so-called \textbf{representative}) of a more basic object?
This basic mathematical object is called \textbf{vector}, and in this case it has the 
two \textbf{components}: $-3$ ($x$-component) and $1$ ($y$-component) that are written as a so-called 
$2$-\textbf{tuple} one above the other:
\[
 \textrm{vector represented in the figure above}\MBlank = \MVector{-3\\1}\MDFPeriod
\]
The Info Box below outlines these conclusions and a few more notations and dictions concerning vectors.

\begin{MInfo}
A two- or three-dimensional \MEntry{vector}{vector} is a $2$- or $3$-\MEntry{tuple}{tuple} with 
$2$ or $3$ \MEntry{components}{components (of vectors)} called the $x$-, $y$- (and $z$-)components. 
In general, vectors are denoted by lowercase italic letters accented by a right arrow or 
by upright boldface lowercase letters. The components of a vector are often denoted by the same 
lowercase italic letter as the vector, with the corresponding coordinate direction as its index:
\[
 \MVec{a}=\MVector{a_x\\a_y},\MDFPaSpace \MVec{b}=\MVector{b_x\\b_y\\b_z}\MDFPeriod
\]
An arrow in the plane or in space is called a \MEntry{representative}{representative (of a vector)} of the 
vector if the arrow connects two points in the plane or in space such that the differences between the 
coordinates at the initial and end points of the 
arrow give the components of the vector. 
\end{MInfo}

Often, a point $P$ or two points $Q$ and $R$ in the plane or in space are given and one wants to specify the vector 
that has the arrow from the origin $O$ to the given point $P$ or the arrow connecting $Q$ to $R$ as its representatives. 
The Info Box below outlines the notation and diction as well as the required vector operation:


\begin{MInfo}\MLabel{VBKM10_Info_OrtsundVerbindungsvektor}
\begin{itemize}
 \item \textbf{Two-dimensional case:}\\
 Let $P=\MPointTwo{p_x}{p_y}$, $Q=\MPointTwo{q_x}{q_y}$, and $R=\MPointTwo{r_x}{r_y}$ be points in the plane. Then the 
 vector
 \[
  \MDVec{Q R} := \MVector{r_x-q_x \\ r_y-q_y}
 \]
 is called the \MEntry{connecting vector from the initial point}{connecting vector} $Q$ \textbf{to the terminal point} $R$, and 
 \[
  \MDVec{P} := \MDVec{O P} = \MVector{p_x\\p_y}
 \]
 is called the \MEntry{position vector of the point}{position vector} $P$. These are exactly those vectors whose representatives 
 include the connecting arrows of the points (see figure below).
 \begin{center}
\MTikzAuto{
\begin{tikzpicture}[>=stealth]
%Koordinatensystem
\draw[->,color=black] (-1.5,0) -- (4.5,0);
% \foreach \x in {-3,-2,-1,1,2,3,4}
% \draw[shift={(\x,0)},color=black] (0pt,2pt) -- (0pt,-2pt) node[below] {\footnotesize $\x$};
\draw[->,color=black] (0,-1.5) -- (0,4.5);
\foreach \y in {-2,-1,1,2,3}
% \draw[shift={(0,\y)},color=black] (2pt,0pt) -- (-2pt,0pt) node[left] {\footnotesize $\y$};
% \draw[color=black] (-10pt,-8pt) node[right] {\footnotesize $0$};
%Achsenbeschriftung
\draw (4.5,0) node[anchor=north west] {$x$};
\draw (-0.5,4.8) node[anchor=north west] {$y$};
%Koordinaten
\draw[dashed, color=red] (0.5,1) -- (0.5,0);
\draw[dashed, color=red] (0.5,1) -- (0,1);
\draw[color=red] (0.5,-2pt) -- (0.5,2pt);
\draw[color=red] (0.5,0) node[anchor=north] {\scriptsize $p_x$};
\draw[color=red] (-2pt,1) -- (2pt,1);
\draw[color=red] (0,1) node[anchor=east] {\scriptsize $p_y$};

\draw[dashed, color=blue] (4,1.5) -- (4,0);
\draw[dashed, color=blue] (4,1.5) -- (0,1.5);
\draw[color=blue] (4,-2pt) -- (4,2pt);
\draw[color=blue] (4,0) node[anchor=north] {\scriptsize $q_x$};
\draw[color=blue] (-2pt,1.5) -- (2pt,1.5);
\draw[color=blue] (0,1.5) node[anchor=east] {\scriptsize $q_y$};

\draw[dashed, color=violet] (1,3.5) -- (0,3.5);
\draw[dashed, color=violet] (1,3.5) -- (1,0);
\draw[color=violet] (1,-2pt) -- (1,2pt);
\draw[color=violet] (1,0) node[anchor=north] {\scriptsize $r_x$};
\draw[color=violet] (-2pt,3.5) -- (2pt,3.5);
\draw[color=violet] (0,3.5) node[anchor=east] {\scriptsize $r_y$};
%Punkte
\draw [fill = black] (0,0) circle (1.5pt);
\draw [color=black] (0,0) node[anchor=south east] {\footnotesize $O=\MPointTwo{0}{0}$};
\draw [fill = red] (0.5,1) circle (1.5pt);
\draw [color=red] (0.5,1) node[anchor=south] {\footnotesize $P$};
\draw [fill = blue] (4,1.5) circle (1.5pt);
\draw [color=blue] (4,1.5) node[anchor=south west] {\footnotesize $Q$};
\draw [fill = violet] (1,3.5) circle (1.5pt);
\draw [color=violet] (1,3.5) node[anchor=south] {\footnotesize $R$};
%Pfeile
\draw[->, line width = 1.5pt] (0,0) -- (0.5,1);
\draw[->, line width = 1.5pt] (4,1.5) -- (1,3.5);
\end{tikzpicture}
} 
\end{center}

 \item \textbf{Three-dimensional case:}\\
 Let $P=\MPointThree{p_x}{p_y}{p_z}$, $Q=\MPointThree{q_x}{q_y}{q_z}$, and $R=\MPointThree{r_x}{r_y}{r_z}$ be points in space. 
 Then the vector
 \[
  \MDVec{Q R} := \MVector{r_x-q_x \\ r_y-q_y \\ r_z-q_z}
 \]
 is called the \MEntry{connecting vector from the initial point}{connecting vector} $Q$ \textbf{to the point} $R$, and 
 \[
  \MDVec{P} := \MDVec{O P} = \MVector{p_x\\p_y\\p_z}
 \]
 is called the \MEntry{position vector of the point}{position vector} $P$. These are exactly those vectors whose representatives 
 include the connecting arrows of the points (see figure below).
\begin{center}
\MTikzAuto{%
\begin{tikzpicture}[>=stealth]

% The axes
\draw[->] (xyz cs:x=-2.5) -- (xyz cs:x=3.5) node[above] {\footnotesize $x$};
\draw[->] (xyz cs:y=-2.5) -- (xyz cs:y=3.5) node[right] {\footnotesize $y$};
\draw[->] (xyz cs:z=-2.5) -- (xyz cs:z=3.5) node[left] {\footnotesize $z$};
% The ticks
% \foreach \coo in {-2,-1,1,2,3}
% {
%   \draw (\coo,-3pt) -- (\coo,3pt) node[below=4pt] {\footnotesize \coo};
%   \draw (-3pt,\coo) -- (3pt,\coo) node[left=4pt] {\footnotesize \coo};
%   \draw (xyz cs:y=-0.1pt,z=\coo) -- (xyz cs:y=0.1pt,z=\coo) node[below=3pt] {\scriptsize \coo};
% }
%Coordinates
\draw[dashed, color=red] (0.5,2,-1) -- (0.5,2,0);
\draw[dashed, color=red] (0.5,2,-1) -- (0.5,0,-1);
\draw[dashed, color=red] (0.5,2,-1) -- (0,2,-1);
\draw[dashed, color=red] (0.5,2,0) -- (0,2,0);
\draw[dashed, color=red] (0.5,0,-1) -- (0.5,0,0);
\draw[dashed, color=red] (0,2,-1) -- (0,2,0);
\draw[dashed, color=red] (0.5,0,-1) -- (0,0,-1);
\draw[color=red] (0,-0.06,-1) -- (0,0.06,-1);
\draw[color=red] (0,0,-1) node[anchor=north] {\scriptsize $p_z$};
\draw[color=red] (0.5,-2pt) -- (0.5,2pt);
\draw[color=red] (0.5,0,0) node[anchor=north] {\scriptsize $p_x$};
\draw[color=red] (-2pt,2) -- (2pt,2);
\draw[color=red] (0,2,0) node[anchor=east] {\scriptsize $p_y$};

\draw[dashed, color=violet] (1,-1,2) -- (0,-1,2);
\draw[dashed, color=violet] (1,-1,2)-- (1,0,2);
\draw[dashed, color=violet] (1,-1,2)-- (1,-1,0);
\draw[dashed, color=violet] (0,-1,2) -- (0,-1,0);
\draw[dashed, color=violet] (1,0,2) -- (1,0,0);
\draw[dashed, color=violet] (1,0,2) -- (0,0,2);
\draw[dashed, color=violet] (1,-1,0) -- (0,-1,0);
\draw[color=violet] (1,-2pt) -- (1,2pt);
\draw[color=violet] (1,0,0) node[anchor=south] {\scriptsize $r_x$};
\draw[color=violet] (-2pt,-1) -- (2pt,-1);
\draw[color=violet] (0,-1,0) node[anchor=east] {\scriptsize $r_y$};
\draw[color=violet] (0,-0.06,2) -- (0,0.06,2);
\draw[color=violet] (0,0,2) node[anchor=east] {\scriptsize $r_z$};

\draw[dashed, color=blue] (3,1,1) -- (0,1,1);
\draw[dashed, color=blue] (3,1,1) -- (3,0,1);
\draw[dashed, color=blue] (3,1,1) -- (3,1,0);
\draw[dashed, color=blue] (0,1,1) -- (0,1,0);
\draw[dashed, color=blue] (3,1,0) -- (0,1,0);
\draw[dashed, color=blue] (3,0,1) -- (0,0,1);
\draw[dashed, color=blue] (3,0,1) -- (3,0,0);
\draw[color=blue] (3,-2pt) -- (3,2pt);
\draw[color=blue] (3,0,0) node[anchor=south] {\scriptsize $q_x$};
\draw[color=blue] (-2pt,1) -- (2pt,1);
\draw[color=blue] (0,1,0) node[anchor=east] {\scriptsize $q_y$};
\draw[color=blue] (0,-0.06,1) -- (0,0.06,1);
\draw[color=blue] (0,0,1) node[anchor=east] {\scriptsize $q_z$};

% % Points:
\draw[fill=black] (0,0,0) circle (1.5pt);
\draw[color=black] (0,0,0) node[anchor=south east] {\footnotesize $O=\MPointThree{0}{0}{0}$};
\draw[fill=red] (0.5,2,-1) circle (1.5pt);
\draw[color=red] (0.5,2,-1) node[anchor=south] {\footnotesize $P$};
\draw[fill=violet] (1,-1,2) circle (1.5pt);
\draw[color=violet] (1,-1,2) node[anchor=north] {\footnotesize $R$};
\draw[fill=blue] (3,1,1) circle (1.5pt);
\draw[color=blue] (3,1,1) node[anchor=west] {\footnotesize $Q$};

%Arrows
\draw[->, line width = 1.5pt] (0,0,0) -- (0.5,2,-1);
\draw[->, line width = 1.5pt] (3,1,1) -- (1,-1,2);
\end{tikzpicture}
}
\end{center}

\end{itemize}
 
\end{MInfo}

\begin{MExample}
\begin{itemize}
 \item \textbf{Two-dimensional case:}\\
 The point $P=\MPointTwo{-1}{-2}$ has the position vector 
 \[
  \MDVec{P} = \MVector{-1\\-2} \MDFPeriod
 \]
 The vector
 \[
  \MVec{v}=\MVector{2\\0}
 \]
 is the connecting vector from the point $A=\MPointTwo{1}{1}$ to the point $B=\MPointTwo{3}{1}$. Thus, we have
 \[
  \MVec{v}=\MDVec{A B} \MDFPeriod 
 \]
 However, $\MVec{v}\neq\MDVec{B A}$ since
 \[
  \MDVec{B A} = \MVector{1-3\\1-1} = \MVector{-2\\0}\neq\MVector{2\\0} \MDFPeriod
 \]
 \item \textbf{Three-dimensional case:}\\
 Consider the two points $Q=\MPointThree{1}{1}{1}$ and $R=\MPointThree{-2}{0}{2}$. The connecting vector from $Q$ to $R$ is:
 \[
  \MDVec{Q R}=\MVector{-2-1\\0-1\\2-1}= \MVector{-3\\-1\\1}\MDFPeriod
 \]
 However, the connecting vector from $R$ to $Q$ is
 \[
  \MDVec{R Q}=\MVector{1-(-2)\\1-0\\1-2}=\MVector{3\\1\\-1} \MDFPeriod
 \]
 Obviously, the vector
 \[
  \MVector{3\\1\\-1}
 \]
 is also the position vector of the point $\MPointThree{3}{1}{-1}$.
\end{itemize}
 
\end{MExample}

The example above reveals an interesting fact: reversing the orientation of a vector (and thus the 
orientation of all its representative arrows) results in a vector in which all components have the opposite sign. 
This vector is also called the \textbf{opposite vector}. This suggests that vector calculations can be carried out 
component-wise. This will be discussed in detail in Subsection~\MNRef{VBKM10_Vektorrechnung}.  

Obviously, the also exists a vector with all its components equal to $0$ in the two- and three-dimensional cases:
\[
 \MVector{0\\0}\MDFPaSpace\textrm{or}\MDFPaSpace\MVector{0\\0\\0}\MDFPeriod
\]
This vector is called the (two-dimensional or three-dimensional) \textbf{zero vector}. One can imagine the zero vector 
as having ``arrows of zero length'' as its representatives, i.e. arrows that connect a point with itself. In other words, the zero vector is the position vector of the origin.

\begin{MExercise}
Let the points
\[
 A=\MPointTwo[\Big]{-1}{\frac{3}{2}}\MDFPaSpace\textrm{and}\MDFPaSpace B=\MPointTwo{\pi}{-2}
\]
be given in the plane, and the points
\[
 P=\MPointThree{\MZahl{0}{5}}{1}{-1}\MDFPaSpace\textrm{and}\MDFPaSpace Q=\MPointThree[\Big]{\frac{1}{2}}{-1}{1}
\]
in space as well, as the (two- and three-dimensional) vectors
\[
 \MVec{a}=\MVector{\pi\\-1}\MDFPaSpace\textrm{and}\MDFPaSpace \MVec{v}=\MVector{0\\3\\-3} \MDFPeriod
\]
\begin{itemize}
 \item Find the following vectors:\\
 \begin{MExerciseItems}
\item{$\MDVec{A B}=$\MLFunctionQuestion{15}{(pi+1,-(7/2))}{5}{x}{5}{VECI1}} 
\item{$\MDVec{B A}=$\MLFunctionQuestion{15}{(-1-pi,(7/2))}{5}{x}{5}{VECI2}} 
\item{$\MDVec{P Q}=$\MLFunctionQuestion{15}{(0,-2,2)}{5}{x}{5}{VECI3}} 
\item{$\MDVec{Q P}=$\MLFunctionQuestion{15}{(0,2,-2)}{5}{x}{5}{VECI4}} 
\end{MExerciseItems}
\MInputHint{Vektoren können in der Form \texttt{(a;b)} bzw. \texttt{(a;b;c)} eingegeben werden, zum Beispiel \texttt{(1;0)} für den Vektor $\MVector{1\\0}$. Die Zahl $\pi$ kann als \texttt{pi} eingegeben werden.} 

\item Find the points $C$ in the plane and $R$ in space such that the following statements are true:\\
\begin{MExerciseItems}
\item{$\MVec{a}=\MDVec{C B}\MDFPSpace\Leftrightarrow\MDFPSpace C=$\MLFunctionQuestion{15}{(0,-1)}{5}{x}{5}{POINTI1}} 
\item{$\MVec{v}=\MDVec{Q R}\MDFPSpace\Leftrightarrow\MDFPSpace R=$\MLFunctionQuestion{15}{(0.5,2,-2)}{5}{x}{5}{POINTI2}} 
\end{MExerciseItems}
\MInputHint{Points can be entered in the form \texttt{(a;b)} or \texttt{(a;b;c)} as well.} 

\item Draw at least three representatives of the vector $a$.
\end{itemize}


\begin{MHint}{Solution}
\begin{itemize}
 \item The required vectors are:\\
 \begin{MExerciseItems}
\item{$\MDVec{A B}=\MVector{\pi+1\\-\frac{7}{2}}$} 
\item{$\MDVec{B A}=\MVector{-1-\pi\\ \frac{7}{2}}$}
\item{$\MDVec{P Q}=\MVector{0\\-2\\2}$} 
\item{$\MDVec{Q P}=\MVector{0\\2\\-2}$} 
\end{MExerciseItems}
 \item The required points are:\\
  \begin{MExerciseItems}
\item{$C=\MPointTwo{0}{-1}$} 
\item{$R=\MPointThree[\Big]{\frac{1}{2}}{2}{-2}$}
\end{MExerciseItems}
 \item The figure below shows three possible representatives of $\MVec{a}=\MVector{\pi\\-1}$:
 
\begin{center}
\MTikzAuto{
\begin{tikzpicture}[>=stealth]
%Koordinatensystem
\draw[->,color=black] (-2.5,0) -- (6.5,0);
\foreach \x in {-2,-1,1,2,3,4,5,6}
\draw[shift={(\x,0)},color=black] (0pt,2pt) -- (0pt,-2pt) node[below] {\footnotesize $\x$};
\draw[->,color=black] (0,-1.5) -- (0,3.5);
\foreach \y in {-1,1,2,3}
\draw[shift={(0,\y)},color=black] (2pt,0pt) -- (-2pt,0pt) node[left] {\footnotesize $\y$};
\draw[color=black] (-10pt,-8pt) node[right] {\footnotesize $0$};
%Achsenbeschriftung
\draw (6.5,0) node[anchor=north west] {$x$};
\draw (-0.5,3.8) node[anchor=north west] {$y$};
%Punkte

%Pfeile
\draw[->, line width = 1.5pt] (0,0) -- (3.14,-1);
\draw[->, line width = 1.5pt] (-2,2) -- (1.14,1);
\draw[->, line width = 1.5pt] (2,3) -- (5.14,2);
\end{tikzpicture}
} 
\end{center}

\end{itemize}

\end{MHint}

\end{MExercise}

The previous introduction of the concept of a vector reveals the close relationship between vectors and points. Indeed, there 
is a one-to-one correspondence between points and position vectors: for every point there exists exactly one vector that is the position vector
of this point. Conversely, for every vector there exists exactly one point that has this vector as its position vector. 
This is true in both the two-dimensional and the three-dimensional cases. This justifies the convention we will follow below: 
describing points by their position vectors. A point $P=\MPointTwo{2}{1}$, for example, is often described by its corresponding 
position vector $\MDVec{P}=\MVector{2\\1}$ instead of its coordinates $\MPointTwo{2}{1}$. If geometric objects such as lines 
and planes are investigated (see e.g. Subsection~\MNRef{VBKM10_GeradenEbene}), this convention also involves certain 
advantages in the description of these objects (in particular in the three-dimensional case).

Moreover, this one-to-one correspondence between points and position vectors also justifies to use the abbreviations 
$\R^2$ and $\R^3$ not only for the set of all points in the plane or in space but also for the set of all 
two-dimensional or three-dimensional vectors. These abbreviations will also be used throughout the following sections.

\end{MXContent}

\begin{MXContent}{Simple Vector Operations}{Vector Operations}{STD}
\MLabel{VBKM10_Vektorrechnung}
\MDeclareSiteUXID{VBKM10_Vektorrechnung}

In this section we discuss what kinds of vector operations we can carry out on the vectors  
introduced in the previous Section~\MNRef{VBKM10_Vektoren}. We can view vector operations in two different ways. First you can carry out the vector operations of addition, subtraction and -- with a certain
restriction -- multiplication on vectors specified as $2$- or $3$-tuples. On the other hand, these operations can 
be interpreted graphically with respect to the arrows representing the vectors. That is, the vector operations on vectors
can be interpreted as geometric operations on their representatives. This geometric interpretation of 
vector operations leads to a deeper understanding of position vectors and connecting vectors of points.

Vector operations in two and three dimensions are carried out in essentially the same way. For all component-wise 
vector operations presented below both cases (two-dimensional and three-dimensional) are considered. The 
corresponding figures illustrate the geometric interpretation of the operations on the representatives 
of the vectors and visualise the position and connecting vectors. They will mostly show only arrows and points without 
a coordinate system. In this way, the figures apply to both the two-dimensional and the three-dimensional case.

Since vectors will be transformed below by equivalent transformations into each other, we need to specify under which conditions two vectors are considered to be equal.

\begin{MInfo}
Two vectors $\MVec{a},\MVec{b}\in\R^2\MBlank \textrm{or}\MBlank \R^3$ are \MEntry{equal}{equality (of vectors)} 
(written as $\MVec{a}=\MVec{b}$) if and only if they satisfy one (and hence all) of the following equivalent conditions:

\begin{itemize}
 \item $\MVec{a}$ and $\MVec{b}$ have the same components:
 \[
  \MVec{a}=\MVec{b}\MDFPSpace\Leftrightarrow\MDFPSpace\MVector{a_x\\a_y}=\MVector{b_x\\b_y}\MDFPSpace\Leftrightarrow\MDFPSpace a_x=b_x\MBlank \textrm{and}\MBlank  a_y=b_y
 \]
 in the two-dimensional case or
 \[
  \MVec{a}=\MVec{b}\MDFPSpace\Leftrightarrow\MDFPSpace\MVector{a_x\\a_y\\a_z}=\MVector{b_x\\b_y\\b_z}\MDFPSpace\Leftrightarrow\MDFPSpace a_x=b_x\MBlank \textrm{and}\MBlank  a_y=b_y\MBlank \textrm{and}\MBlank  a_z=b_z
 \]
 in the three-dimensional case. This is also known as \textbf{comparison of coordinates} or \textbf{components}.
 \item $\MVec{a}$ and $\MVec{b}$ have the same representatives.
 \item $\MVec{a}$ and $\MVec{b}$ are both the position vector of the same point.
 \item $\MVec{a}$ and $\MVec{b}$ are both the connecting vector of the same two points.
\end{itemize}
\end{MInfo}

From the statements in the Info Box above we see that two vectors of different dimensions (i.e. $\MVec{a}\in\R^2$ 
and $\MVec{b}\in\R^3$) can never be equal. Since these vectors have a different number of components, they are not even 
comparable. Thus, vector operations are only carried out on vectors with an equal number (two or three) of components 
, and these operations will always result in a vector with this fixed number of components.

\begin{MInfo}
The \MEntry{addition of two vectors}{addition (of vectors)} involves the addition of all their components, i.e.
\[
 \MVec{a}+\MVec{b} = \MVector{a_x\\a_y}+\MVector{b_x\\b_y} = \MVector{a_x+b_x\\a_y+b_y}
\]
in the two-dimensional case and 
\[
 \MVec{a}+\MVec{b} = \MVector{a_x\\a_y\\a_z}+\MVector{b_x\\b_y\\b_z} = \MVector{a_x+b_x\\a_y+b_y\\a_z+b_z}
\]
in the three-dimensional case. Geometrically, vector addition can be interpreted as either a``linking'' of two arrows 
or a completion of two arrows to a parallelogram, depending on which representatives of the vectors are used:
\begin{center}
\begin{tabular}{lr}
\MTikzAuto{%
\begin{tikzpicture}[>=stealth] 
%Pfeile
\draw[color=red,->] (0,0) -- (3,1);
\draw[color=red] (1.5,0.5) node[anchor=north] {\footnotesize $\MVec{a}$};
\draw[color=blue,->] (3,1) -- (4,3);
\draw[color=blue] (3.5,2) node[anchor=west] {\footnotesize $\MVec{b}$};
%Summe
\draw[color=violet,->] (0,0) -- (4,3);
\draw[color=violet] (1.9,1.5) node[anchor=east] {\footnotesize $\MVec{a}+\MVec{b}$};
\end{tikzpicture}       
}%
&
\MTikzAuto{%
\begin{tikzpicture}[>=stealth] 
%Pfeile
\draw[color=red,->] (0,0) -- (3,1);
\draw[color=red] (1.5,0.5) node[anchor=north] {\footnotesize $\MVec{a}$};
\draw[color=blue,->] (0,0) -- (1,2);
\draw[color=blue] (0.5,1) node[anchor=east] {\footnotesize $\MVec{b}$};
%Parallelogramm
\draw[color=red, dashed] (1,2) -- (4,3);
\draw[color=blue, dashed] (3,1) -- (4,3); 
%Summe
\draw[color=violet,->] (0,0) -- (4,3);
\draw[color=violet] (2.1,1.5) node[anchor=north] {\footnotesize $\MVec{a}+\MVec{b}$};
\end{tikzpicture}       
}%
\end{tabular}
\end{center}
\end{MInfo}

The laws of associativity and commutativity apply to the addition of two vectors as to the addition of real 
numbers (see Section~\MNRef{VBKM01_TermeUmformen}):
\[
 \MVec{a}+\MVec{b} = \MVec{b}+\MVec{a}
\]
and
\[
 \MVec{a}+\MVec{b}+\MVec{c} = (\MVec{a}+\MVec{b})+\MVec{c} = \MVec{a}+(\MVec{b}+\MVec{c}) \MDFPeriod
\]
The \MEntry{zero vector}{zero vector} $\MDVec{O}=\MVector{0\\0}$ or $\MDVec{O}=\MVector{0\\0\\0}$ 
plays the same role for vectors as the number $0$ for the real numbers:
\[
 \MVec{a}+\MDVec{O} = \MDVec{O} + \MVec{a} = \MVec{a}\MDFPeriod
\]


\begin{MExample}
Consider two vectors $\MVec{v}=\MVector{2\\-1}$ and $\MVec{w}=\MVector{-1\\2}$. Then, we have
\[
 \MVec{v}+\MVec{w} = \MVector{2\\-1} + \MVector{-1\\2} = \MVector{2-1\\-1+2} = \MVector{1\\1} \MDFPeriod
\]
This is illustrated by the figure below.
\begin{center}
\MTikzAuto{
\begin{tikzpicture}[>=stealth]
%Koordinatensystem
\draw[->,color=black] (-2.5,0) -- (3.5,0);
\foreach \x in {-2,-1,1,2,3}
\draw[shift={(\x,0)},color=black] (0pt,2pt) -- (0pt,-2pt) node[below] {\footnotesize $\x$};
\draw[->,color=black] (0,-2.5) -- (0,3.5);
\foreach \y in {-2,-1,1,2,3}
\draw[shift={(0,\y)},color=black] (2pt,0pt) -- (-2pt,0pt) node[left] {\footnotesize $\y$};
\draw[color=black] (-10pt,-8pt) node[right] {\footnotesize $0$};
%Achsenbeschriftung
\draw (3.5,0) node[anchor=north west] {$x$};
\draw (-0.5,3.8) node[anchor=north west] {$y$};
%Pfeile
\draw[->, color=red] (0,0) -- (2,-1);
\draw[color=red] (1,-0.5) node[anchor=north] {\footnotesize $\MVec{v}$};
\draw[dashed, color=red] (-1,2) -- (1,1);
\draw[->, color=blue] (0,0) -- (-1,2);
\draw[color=blue] (-0.5,1) node[anchor=east] {\footnotesize $\MVec{w}$};
\draw[dashed, color=blue] (2,-1) -- (1,1);
\draw[->, color=violet] (0,0) -- (1,1);
\draw[color=violet] (1,1) node[anchor=south west] {\footnotesize $\MVec{v}+\MVec{w}$};
\end{tikzpicture}
} 
\end{center}
Furthermore, $\MVec{w}$ is the connecting vector from the point $P=\MPointTwo{2}{0}$ to the point 
$Q=\MPointTwo{1}{2}$, i.e. $\MVec{w}=\MDVec{P Q}$, and $\MVec{v}$ is the connecting vector 
from the point $Q=\MPointTwo{1}{2}$ to the point $R=\MPointTwo{3}{1}$, i.e. $\MVec{v}=\MDVec{Q R}$.
\[
 \MVec{w}+\MVec{v} = \MDVec{P Q} + \MDVec{Q R} = \MDVec{P R}\MDFPeriod
\]
This is illustrated by the figure below.
\begin{center}
\MTikzAuto{
\begin{tikzpicture}[>=stealth]
%Koordinatensystem
\draw[->,color=black] (-1.5,0) -- (4.5,0);
\foreach \x in {-1,1,2,3,4}
\draw[shift={(\x,0)},color=black] (0pt,2pt) -- (0pt,-2pt) node[below] {\footnotesize $\x$};
\draw[->,color=black] (0,-1.5) -- (0,3.5);
\foreach \y in {-1,1,2,3}
\draw[shift={(0,\y)},color=black] (2pt,0pt) -- (-2pt,0pt) node[left] {\footnotesize $\y$};
\draw[color=black] (-10pt,-8pt) node[right] {\footnotesize $0$};
%Achsenbeschriftung
\draw (4.5,0) node[anchor=north west] {$x$};
\draw (-0.5,3.8) node[anchor=north west] {$y$};
%Punkte
\draw[fill=black] (2,0) circle (1.5pt);
\draw[color=black] (2,0) node[anchor=north west] {\footnotesize $P$};
\draw[fill=black] (1,2) circle (1.5pt);
\draw[color=black] (1,2) node[anchor=south] {\footnotesize $Q$};
\draw[fill=black] (3,1) circle (1.5pt);
\draw[color=black] (3,1) node[anchor=west] {\footnotesize $R$};
%Pfeile
\draw[->, color=blue] (2,0) -- (1,2);
\draw[color=blue] (1.5,1) node[anchor=east] {\footnotesize $\MVec{w}=\MDVec{P Q}$};
\draw[->, color=red] (1,2) -- (3,1);
\draw[color=red] (2,1.5) node[anchor=south west] {\footnotesize $\MVec{v}=\MDVec{Q R}$};
\draw[->, color=violet] (2,0) -- (3,1);
\draw[color=violet] (2.5,0.5) node[anchor=west] {\footnotesize $\MVec{w}+\MVec{v}= \MDVec{P R}$};
\end{tikzpicture}
} 
\end{center}
\end{MExample}
The example above shows that expressions for connecting vectors of points can also be simplified by means of 
vector operations. This will be discussed for the subtraction of vectors in more detail below.


\begin{MExercise}

\begin{MExerciseItems}
\item{Let the vectors $\MVec{u}=\MVector{1\\0\\-8}$ and $\MVec{v}=\MVector{-3\\-4\\3}$ be given. Calculate $\MVec{v}+\MVec{u}$.\\$\MVec{v}+\MVec{u}=$\MLFunctionQuestion{15}{(-2,-4,-5)}{5}{x}{5}{VEC21}} 
\item{Let the points $P$, $Q$, and $R$ be given. Which of the following expressions are equivalent to the expression
 $(\MDVec{P}+\MDVec{P Q})+\MDVec{Q R}$?\\
\begin{MQuestionGroup}
\begin{tabular}{lll}
\MLCheckbox{1}{VEC22} & \MBlank & $\MDVec{P}+(\MDVec{P Q}+\MDVec{Q R})$\\
\MLCheckbox{0}{VEC23} & \MBlank & $\MDVec{P R}$\\
\MLCheckbox{0}{VEC24} & \MBlank & $\MDVec{Q R}$\\
\MLCheckbox{1}{VEC25} & \MBlank & $\MDVec{O R}$\\
\MLCheckbox{1}{VEC26} & \MBlank & $\MDVec{R}$\\
\MLCheckbox{1}{VEC27} & \MBlank & $\MDVec{Q}+\MDVec{Q R}$\\
\end{tabular}
\end{MQuestionGroup}
\MGroupButton{Check input}
}
\end{MExerciseItems}

\begin{MHint}{Solution}
\begin{MExerciseItems}
\item{\[\MVec{v}+\MVec{u}=\MVector{-3\\-4\\3}+\MVector{1\\0\\-8} = \MVector{-3+1\\-4+0\\3-8} =\MVector{-2\\-4\\-5} \MDFPeriod\]} 
\item{\[ (\MDVec{P}+\MDVec{P Q})+\MDVec{Q R} = \MDVec{P}+(\MDVec{P Q}+\MDVec{Q R})\MBlank \textrm{(associativity)}\] \[= (\MDVec{O P}+\MDVec{P Q})+\MDVec{Q R} = \MDVec{O Q}+\MDVec{Q R} = \MDVec{Q}+\MDVec{Q R} = \MDVec{O R} = \MDVec{R} \MDFPeriod\]}
\end{MExerciseItems} 
\end{MHint}
\end{MExercise}


When we study possible operations on vectors further, we will see that the component-wise multiplication or division of vectors is not a 
meaningful operation. To understand this, however, would go far beyond the scope of this course. At this point you simply have  
to accept that vectors cannot be multiplied that simply, let alone divided. What we certainly can do 
is multiply vectors by real numbers and -- based on that -- subtract vectors. Please note: if we speak of the 
length of a vector in the following section, we mean the geometric length of the arrows representing this vector. The concept 
of the length (a.k.a. absolute value or norm) of a vector will be discussed in more detail later on.

\begin{MInfo}
The \MEntry{multiplication of a vector by a real number}{multiplication (vector and scalar)} involves
the multiplication of each component by this real number. If $\MVec{a}$ is a vector and $s\in\R$, then we have
\[
 s\cdot\MVec{a} = s\MVec{a}=s\MVector{a_x\\a_y} = \MVector{s a_x\\s a_y}
\]
in the two-dimensional case and
\[
 s\cdot\MVec{a} = s\MVec{a}=s\MVector{a_x\\a_y\\a_z} = \MVector{s a_x\\s a_y\\s a_z}
\]
in the three-dimensional case. The division of a vector by a number $0\neq s\in\R$ is then simply defined as its multiplication
by the reciprocal $\frac{1}{s}$:
\[
 \frac{\MVec{a}}{s}=\frac{1}{s}\MVec{a}\MDFPeriod 
\]
Thus, multiplying a vector by a real number $s\in\R$, $s>0$ results in an equally oriented vector that is stretched by a factor 
of $s$ . For $s\in\R$, $s<0$, the resulting vector is also stretched by a factor of $s$ but it is flipped around by an angle of $\pi$.
In the special case of $s=0$, we obviously have \[0\MVec{a}=\MDVec{O}\] for every vector $\MVec{a}$. Two other relevant cases are the multiplication by a factor of $s=1$
\[1\cdot\MVec{a}=\MVec{a}\]
 -- which obviously leaves the vector unchanged -- and the multiplication by $s=-1$
\[-1\cdot\MVec{a}=-\MVec{a}\]
 -- which results in the so-called \MEntry{opposite vector}{opposite vector}. This is a vector of 
equal length and opposite orientation. This is illustrated in the figure below.
\begin{center}
\MTikzAuto{%
\begin{tikzpicture}[>=stealth] 
%Pfeile
\draw[color=red, ->] (0,0) -- (2,1);
\draw[color=black, ->] (0.2,0) -- (3.2,1.5);
\draw[color=blue, ->] (0.4,0) -- (4.4,2);
\draw[color=violet, ->] (0,0) -- (-2,-1);
\draw[color=green, ->] (0.2,0) -- (-3.8,-2);
%Linie
\draw[line width = 1pt] (0,0) -- (0.4,0);
%Beschriftung
\draw[color=red] (2,1) node[anchor=south east] {\footnotesize $\MVec{a}=1\MVec{a}$};
\draw[color=black] (3.1,1.5) node[anchor= south east] {\footnotesize $\frac{3}{2}\MVec{a}$};
\draw[color=blue] (4.4,2) node[anchor=south east] {\footnotesize $2\MVec{a}$};
\draw[color=violet] (-2,-1) node[anchor=south east] {\footnotesize $-\MVec{a}=-1\cdot\MVec{a}$};
\draw[color=green] (-3.8,-2) node[anchor=south east] {\footnotesize $-2\MVec{a}$};
\end{tikzpicture}       
}% 
\end{center}
Since the multiplication by real numbers changes the length (\textit{scales}
the vectors), real numbers with respect to vectors are often called \MEntry{scalars}{scalar}, and the 
multiplication of a vector by a real number is called \textbf{scalar multiplication}. 
\end{MInfo}

\begin{MExample}
Let the vector $\MVec{v}=\MVector{3\\\frac{3}{2}}$ be given. Then, for example, we have
\[
 2\MVec{v}=2\MVector{3\\\frac{3}{2}}=\MVector{2\cdot3\\2\cdot\frac{3}{2}}=\MVector{6\\3}
\]
and
\[
 -\frac{\MVec{v}}{3}=-\frac{1}{3}\MVec{v}=-\frac{1}{3}\MVector{3\\\frac{3}{2}}=\MVector{-1\\-\frac{1}{2}} \MDFPeriod
\]
Moreover, $\MVec{v}=\MDVec{P Q}$ for $P=\MPointTwo[\Big]{3}{\frac{1}{2}}$ and $Q=\MPointTwo{6}{2}$ since
\[
 \MVec{v}=\MVector{3\\\frac{3}{2}}=\MVector{6-3\\2-\frac{1}{2}} \MDFPeriod
\]
Then we have
\[
 -\MVec{v} = -\MDVec{P Q} = \MDVec{Q P} 
\]
since
\[
 -\MVec{v} = \MVector{-3\\-\frac{3}{2}}=\MVector{3-6\\\frac{1}{2}-2} \MDFPeriod
\]
This is illustrated in the figure below.
\begin{center}
\MTikzAuto{
\begin{tikzpicture}[>=stealth]
%Koordinatensystem
\draw[->,color=black] (-1.5,0) -- (7.5,0);
\foreach \x in {-1,1,2,3,4,5,6,7}
\draw[shift={(\x,0)},color=black] (0pt,2pt) -- (0pt,-2pt) node[below] {\footnotesize $\x$};
\draw[->,color=black] (0,-1.5) -- (0,4.5);
\foreach \y in {-1,1,2,3,4}
\draw[shift={(0,\y)},color=black] (2pt,0pt) -- (-2pt,0pt) node[left] {\footnotesize $\y$};
\draw[color=black] (-10pt,-8pt) node[right] {\footnotesize $0$};
%Achsenbeschriftung
\draw (7.5,0) node[anchor=north west] {$x$};
\draw (-0.5,4.8) node[anchor=north west] {$y$};
%Punkte
\draw[fill=black] (3,0.5) circle (1.5pt);
\draw[color=black] (3,0.5) node[anchor=north west] {\footnotesize $P$};
\draw[fill=black] (6,2) circle (1.5pt);
\draw[color=black] (6,2) node[anchor=south] {\footnotesize $Q$};
%Pfeile
\draw[->, color=red, line width = 1.2pt] (0,0) -- (3,1.5);
\draw[color = red] (3,1.5) node[anchor=south east] {\footnotesize $\MVec{v}$};
\draw[->, color=violet] (0,0) -- (6,3);
\draw[color = violet] (6,3) node[anchor=south east] {\footnotesize $2\MVec{v}$};
\draw[->, color=blue] (0,0) -- (-1,-0.5);
\draw[color = blue] (-1,-0.5) node[anchor=north] {\footnotesize $-\frac{1}{3}\MVec{v}$};
%Verbindungspfeile
\draw[->, color=red, line width = 2pt] (3.04,0.506) -- (5.96,1.97);
\draw[color = red] (4.8,1.22) node[anchor=south east] {\footnotesize $\MVec{v}=\MDVec{P Q}$};
\draw[->, color=black] (5.96,1.97) -- (3.04,0.506);
\draw[color = black] (4.2,1.27) node[anchor=north west] {\footnotesize $-\MVec{v}=\MDVec{Q P}$};
\end{tikzpicture}
} 
\end{center}
\end{MExample}

The following calculation rules apply to scalar multiplication:


\begin{MInfo}
Let two real numbers $r$ and $s$ and two vectors $\MVec{a}$ and $\MVec{b}$ be given. Then 
the following calculation rules apply:
\begin{enumerate}
 \item $r\MVec{a}=\MVec{a}r$
 \item $r s\MVec{a}=(r s)\MVec{a}=r(s\MVec{a})$
 \item $(r+s)\MVec{a}=r\MVec{a}+s\MVec{a}$
 \item $r(\MVec{a}+\MVec{b})=r\MVec{a}+r\MVec{b}$
 \item $r(-\MVec{a})=(-r)\MVec{a}=-(r\MVec{a})$
 \item $r\MVec{a}=0\MDFPSpace\Leftrightarrow\MDFPSpace r=0$ or $\MVec{a}=\MDVec{O}$
\end{enumerate}
The first law is called the commutativity law of scalar multiplication, the second is the 
associativity law of scalar multiplication, and the third and fourth are the distributive laws 
of scalar multiplication.
\end{MInfo}

Using the concept of an opposite vector we are now able to specify what the subtraction of vectors means.

\begin{MInfo}
Let two vectors $\MVec{a}$ and $\MVec{b}$ be given. Then, their \MEntry{difference}{subtraction (of vectors)} 
$\MVec{a}-\MVec{b}$ is the sum of $\MVec{a}$ and the opposite vector of $\MVec{b}$. Thus, we have
\[
 \MVec{a}-\MVec{b}=\MVec{a}+(-\MVec{b})=\MVector{a_x\\a_y}+\MVector{-b_x\\-b_y}=\MVector{a_x-b_x\\a_y-b_y}
\]
in the two-dimensional case and
\[
 \MVec{a}-\MVec{b}=\MVec{a}+(-\MVec{b})=\MVector{a_x\\a_y\\a_z}+\MVector{-b_x\\-b_y\\-b_z}=\MVector{a_x-b_x\\a_y-b_y\\a_z-b_z}
\]
in the three-dimensional case. The difference of vectors can also be interpreted geometrically by means 
of the representatives as illustrated in the figure below.
\begin{center}
\MTikzAuto{%
\begin{tikzpicture}[>=stealth] 
%Hilfslinien
\draw[color=black, dotted] (-1,-2) -- (1,-2);
\draw[color=black, dotted] (2,0) -- (1,-2);
%Pfeile
\draw[color=blue, ->] (0,0) -- (2,0);
\draw[color=red, ->] (0,0) -- (1,2);
\draw[color=red, ->, dashed] (0,0) -- (-1,-2);
\draw[color=violet, ->] (0,0) -- (1,-2);
%Ursprung
\draw[fill=black] (0,0) circle (1.5pt);
%Beschriftung
\draw[color=blue] (2,0) node[anchor=south] {\footnotesize $\MVec{a}$};
\draw[color=red] (1,2) node[anchor= south] {\footnotesize $\MVec{b}$};
\draw[color=red] (-1,-2) node[anchor=north] {\footnotesize $-\MVec{b}$};
\draw[color=violet] (0.9,-1.9) node[anchor=north west] {\footnotesize $\MVec{a}+(-\MVec{b})=\MVec{a}-\MVec{b}$};
\end{tikzpicture}       
}% 
\end{center}
\end{MInfo}

If we only consider the difference of vectors by means of their components,
the question arises what the concept of an opposite vector here is required for. 
Indeed, the difference of vectors in componentwise notation could also be written analogously to the sum 
without using the concept of an opposite vector. However, if we think of the geometrical interpretation 
of the difference by means of representatives (see figure in the Info Box above), we see that a geometrical 
interpretation is only possible using the concept of an opposite vector. 

\begin{MExample}
This example shows typical problems involving the calculation rules for vectors 
presented so far.
\begin{enumerate}
 \item Simplify the following vector expressions:
 \begin{itemize}
  \item[(i)] $\MVector{1\\-2\\\MZahl{0}{5}}-\frac{1}{2}\MVector{2\\-3\\1}$,
  \item[(ii)] $2(\MVec{v}-\MVec{w})+3r\MVec{w}-r\cdot(-2\MVec{v})$ für $r\in\R$.
  \end{itemize}
  Applying the calculation rules results in
  \begin{itemize}
   \item[(i)]
   \[
    \MVector{1\\-2\\\MZahl{0}{5}}-\frac{1}{2}\MVector{2\\-3\\1}=\MVector{1\\-2\\\MZahl{0}{5}}-\MVector{1\\-\frac{3}{2}\\\frac{1}{2}}=\MVector{1-1\\-2-(-\frac{3}{2})\\\MZahl{0}{5}-\frac{1}{2}}
    =\MVector{0\\-\frac{1}{2}\\0} \MDFPeriod
   \]
   \item[(ii)]
   \[
    2(\MVec{v}-\MVec{w})+3r\MVec{w}-r\cdot(-2\MVec{v})=2\MVec{v}-2\MVec{w}+3r\MVec{w}+2r\MVec{v}=2(r+1)\MVec{v}+(3r-2)\MVec{w} \MDFPeriod
   \]
   \end{itemize}
 \item Let two vectors $\MVec{a}=\MVector{1\\2}$ and $\MVec{b}=\MVector{-8\\3}$ be given. Find the unknown vector $\MVec{x}$ 
  in the equation
 \[
  \MVec{a}-2\MVec{b}-(3\MVec{a}+\MVec{x})=\MVector{0\\-1}\MDFPeriod
 \]
 
 Solving for $\MVec{x}$ and substituting $\MVec{a}$ and $\MVec{b}$ results in
 \[
  -(3\MVec{a}+\MVec{x})=\MVector{0\\-1}-\MVec{a}+2\MVec{b}\MDFPaSpace\Leftrightarrow\MDFPaSpace-\MVec{x}=\MVector{0\\-1}-\MVec{a}+2\MVec{b}+3\MVec{a}=\MVector{0\\-1}+2(\MVec{a}+\MVec{b})
 \]
 \[
  \Leftrightarrow\MDFPaSpace\MVec{x}=-\MVector{0\\-1}-2\left(\MVector{1\\2}+\MVector{-8\\3}\right)=\MVector{0\\1}-2\MVector{-7\\5}\MDFPaSpace\Leftrightarrow\MDFPaSpace\MVec{x}=\MVector{14\\-9} \MDFPeriod
 \]
 \item Using the difference of vectors, specify the connecting vector $\MDVec{P Q}$ of the two points $P$ and $Q$ 
  by means of the position vectors $\MDVec{P}$ and $\MDVec{Q}$. We have:
 \[
  \MDVec{Q}-\MDVec{P} = \MDVec{O Q} - \MDVec{O P} = -\MDVec{Q O} - \MDVec{O P} = -(\MDVec{Q O} + \MDVec{O P}) = -\MDVec{Q P} = \MDVec{P Q}\MDFPeriod
 \]
 The connecting vector $\MDVec{P Q}$ from a point $P$ to a point $Q$ is always the difference 
  of the position vector $\MDVec{Q}$ (to the terminal point of the connecting vector) and the 
  position vector $\MDVec{P}$ (to the initial point of the connecting vector). This is illustrated 
  in the figure below and can also be seen from the calculation rule for connecting vectors outlined in Info 
  Box~\MNRef{VBKM10_Info_OrtsundVerbindungsvektor}.
\begin{center}
\MTikzAuto{%
\begin{tikzpicture}[>=stealth] 
%Punkte
\draw[fill=black] (0,0) circle (1.5pt);
\draw[color=red,fill=red] (2,0) circle (1.5pt);
\draw[color=blue,fill=blue] (1,2) circle (1.5pt);
%Pfeile
\draw[color=blue, ->] (0,0) -- (1,2);
\draw[color=red, ->] (0,0) -- (2,0);
\draw[color=violet, ->] (2,0) -- (1,2);
%Beschriftung
\draw[color=blue] (1,2) node[anchor=south] {\footnotesize $Q$};
\draw[color=red] (2,0) node[anchor= north] {\footnotesize $P$};
\draw[color=black] (0,0) node[anchor=north] {\footnotesize $O$};

\draw[color=blue] (0.5,1) node[anchor=east] {\footnotesize $\MDVec{Q}$};
\draw[color=red] (1,0) node[anchor= north] {\footnotesize $\MDVec{P}$};
\draw[color=violet] (1.4,0.8) node[anchor=south west] {\footnotesize $\MDVec{P Q}=\MDVec{Q}-\MDVec{P}$};
\end{tikzpicture}       
}% 
\end{center}
\item The points $A=\MPointTwo{2}{1}$, $B=\MPointTwo{4}{2}$, and $C=\MPointTwo{3}{3}$ are the vertices of a triangle. 
  The (geometric) centroid $S$ of this triangle can be calculated by means of the corresponding position vectors:
\[
 \MDVec{S}=\frac{1}{3}(\MDVec{A}+\MDVec{B}+\MDVec{C}) = \frac{1}{3}\left(\MVector{2\\1}+\MVector{4\\2}+\MVector{3\\3}\right)=\frac{1}{3}\MVector{9\\6}=\MVector{3\\2}\MDFPeriod
\]
Thus, we have $S=\MPointTwo{3}{2}$. This is illustrated in the figure below.
\begin{center}
\MTikzAuto{
\begin{tikzpicture}[>=stealth]
%Koordinatensystem
\draw[->,color=black] (-0.5,0) -- (5.5,0);
\foreach \x in {1,2,3,4,5}
\draw[shift={(\x,0)},color=black] (0pt,2pt) -- (0pt,-2pt) node[below] {\footnotesize $\x$};
\draw[->,color=black] (0,-0.5) -- (0,4.5);
\foreach \y in {1,2,3,4}
\draw[shift={(0,\y)},color=black] (2pt,0pt) -- (-2pt,0pt) node[left] {\footnotesize $\y$};
\draw[color=black] (-10pt,-8pt) node[right] {\footnotesize $0$};
%Achsenbeschriftung
\draw (5.5,0) node[anchor=north west] {$x$};
\draw (-0.5,4.8) node[anchor=north west] {$y$};
%Punkte
\draw[fill=black] (2,1) circle (1.5pt);
\draw[color=black] (2,1) node[anchor=north] {\footnotesize $A$};
\draw[fill=black] (4,2) circle (1.5pt);
\draw[color=black] (4,2) node[anchor=west] {\footnotesize $B$};
\draw[fill=black] (3,3) circle (1.5pt);
\draw[color=black] (3,3) node[anchor=south] {\footnotesize $C$};
\draw[color=red,fill=red] (3,2) circle (1.5pt);
\draw[color=red] (3,2) node[anchor=south]  {\footnotesize $S$};
%Dreiecks
\draw[color=black] (2,1) -- (4,2);
\draw[color=black] (4,2) -- (3,3);
\draw[color=black] (3,3) -- (2,1);
\end{tikzpicture}
} 
\end{center}
\end{enumerate}
 
\end{MExample}

\begin{MExercise}
\begin{MExerciseItems}
\item{Let $P$, $Q$, $R$, and $S$ be points in space. Simplify the expression
\[
 \MDVec{P Q}-(\MDVec{P Q}-\MDVec{Q R})+\MDVec{R S}
\]
as far as possible.} 
\item{Show that the points $A=\MPointTwo{1}{2}$, $B=\MPointTwo{4}{3}$, and $C=\MPointTwo{3}{1}$ 
  together with the origin form the vertices of a parallelogram.}
\end{MExerciseItems}

\begin{MHint}{Solution}
\begin{MExerciseItems}
\item{
\[
 \MDVec{P Q}-(\MDVec{P Q}-\MDVec{Q R})+\MDVec{R S}=\MDVec{P Q}-\MDVec{P Q}+\MDVec{Q R}+\MDVec{R S}=\MVec{O}+\MDVec{Q R}+\MDVec{R S}=\MDVec{Q S} \MDFPeriod
\]}
\item{According to the geometrical interpretation of vector addition, the four points form a parallelogram 
  if one of the position vectors $\MDVec{A}$, $\MDVec{B}$, or $\MDVec{C}$ is the sum of the two other 
  position vectors. Since we have 
\[
 \MDVec{A} + \MDVec{C} = \MVector{1\\2}+\MVector{3\\1} = \MVector{4\\3} = \MDVec{B} \MDFPaSpace ,
\]
the three points together with the origin form the vertices of a parallelogram. This is illustrated in the figure below.
\begin{center}
\MTikzAuto{
\begin{tikzpicture}[>=stealth]
%Koordinatensystem
\draw[->,color=black] (-0.5,0) -- (5.5,0);
\foreach \x in {1,2,3,4,5}
\draw[shift={(\x,0)},color=black] (0pt,2pt) -- (0pt,-2pt) node[below] {\footnotesize $\x$};
\draw[->,color=black] (0,-0.5) -- (0,4.5);
\foreach \y in {1,2,3,4}
\draw[shift={(0,\y)},color=black] (2pt,0pt) -- (-2pt,0pt) node[left] {\footnotesize $\y$};
\draw[color=black] (-10pt,-8pt) node[right] {\footnotesize $0$};
%Achsenbeschriftung
\draw (5.5,0) node[anchor=north west] {$x$};
\draw (-0.5,4.8) node[anchor=north west] {$y$};
%Hilfslinien
\draw[color=red, dashed] (3,1) -- (4,3);
\draw[color=blue, dashed] (1,2) -- (4,3);
%Punkte
\draw[fill=black] (0,0) circle (1.5pt);
\draw[color=black] (0,0) node[anchor=south east] {\footnotesize $O$};
\draw[fill=red,color=red] (1,2) circle (1.5pt);
\draw[color=red] (1,2) node[anchor=south] {\footnotesize $A$};
\draw[fill=blue,color=blue] (3,1) circle (1.5pt);
\draw[color=blue] (3,1) node[anchor=north] {\footnotesize $C$};
\draw[color=violet,fill=violet] (4,3) circle (1.5pt);
\draw[color=violet] (4,3) node[anchor=south]  {\footnotesize $B$};
%Pfeile
\draw[color=red,->] (0,0) -- (1,2);
\draw[color=blue,->] (0,0) -- (3,1);
\draw[color=violet,->] (0,0) -- (4,3);
\end{tikzpicture}
} 
\end{center}
}
\end{MExerciseItems}
\end{MHint}

\end{MExercise}

\begin{MExercise}
Simplify the following expressions as far as possible:
\begin{MExerciseItems}
\item{$2\MVector{-1\\4\\2}-3\MVector{1\\6\\-2}=$\MLFunctionQuestion{15}{(-5,-10,10)}{5}{x}{5}{SVEC1}.} 
\item{$-2\MVector{-t\\3}-\left(\MVector{-1\\0}+\frac{t}{2}\MVector{4\\-42}\right)=$\MLFunctionQuestion{15}{(1,21*t-6)}{5}{t}{5}{SVEC2}.}
\end{MExerciseItems}
\MInputHint{Vectors can be entered as \texttt{(a;b)} or \texttt{(a;b;c)}.}
\end{MExercise}

\begin{MHint}{Solution}
\begin{MExerciseItems}
\item{
\[
 2\MVector{-1\\4\\2}-3\MVector{1\\6\\-2}=\MVector{-2\\8\\4}-\MVector{3\\18\\-6}=\MVector{-5\\-10\\10} \MDFPeriod
\]
} 
\item{
\[
 -2\MVector{-t\\3}-\left(\MVector{-1\\0}+\frac{t}{2}\MVector{4\\-42}\right)=\MVector{2t\\-6}+\MVector{1\\0}-\MVector{2t\\-21t}=\MVector{1\\21t-6} \MDFPeriod
\]

}
\end{MExerciseItems}
 
\end{MHint}

\begin{MExercise}
Find the vector $\MVec{y}$ in the equation
\[
 3\left(\MVector{1\\-1\\1}-\MVec{y}\right)=-8\MVector{\MZahl{0}{25}\\\MZahl{0}{25}\\-\MZahl{0}{25}}+\MVec{y} \MDFPeriod
\]
$\MVec{y}=$\MLFunctionQuestion{15}{(5/4,-1/4,1/4)}{5}{x}{5}{GVEC1}
\end{MExercise}

\begin{MHint}{Solution}
\[
 3\left(\MVector{1\\-1\\1}-\MVec{y}\right)=-8\MVector{\MZahl{0}{25}\\\MZahl{0}{25}\\-\MZahl{0}{25}}+\MVec{y}\MDFPaSpace\Leftrightarrow\MDFPaSpace
 \MVector{3\\-3\\3}+2\MVector{1\\1\\-1}=4\MVec{y}
\]
\[
 \MDFPaSpace\Leftrightarrow\MDFPaSpace4\MVec{y}=\MVector{5\\-1\\1}\MDFPaSpace\Leftrightarrow\MDFPaSpace
 \MVec{y}=\MVector{\frac{5}{4}\\-\frac{1}{4}\\\frac{1}{4}} \MDFPeriod
\]
 
\end{MHint}

Since any vector has an arbitrary number of arrows as its representatives (which arise from each other 
by parallel translation), all these representatives have the same geometrical length (which is always the distance 
between the two points they connect). Therefore, it is reasonable to speak of the length of a vector. In mathematics, 
the length of a vector is called the \textbf{absolute value} or \textbf{norm}.

\begin{MInfo}
The \MEntry{absolute value}{absolute value (of a vector)} or \MEntry{norm}{norm} of a vector $\MVec{a}$
is denoted by $|\MVec{a}|$ and equals the distance between the origin $O$ and the point $P$ that has the vector 
$\MVec{a}$ as its corresponding position vector (i.e. $\MDVec{P}=\MVec{a}$). Thus, we have generally
\[
 |\MVec{a}|=[\overline{O P}]
\]
and hence in the two-dimensional case
\[
 |\MVec{a}|=\left|\MVector{a_x\\a_y}\right|=\sqrt{a_x^2+a_y^2}
\]
or in the three-dimensional case
\[
 |\MVec{a}|=\left|\MVector{a_x\\a_y\\a_z}\right|=\sqrt{a_x^2+a_y^2+a_z^2} \MDFPeriod
\]
A vector with an absolute value of $1$ is also called the \MEntry{unit vector}{unit vector}.
\end{MInfo}
Applying the formula given in the Info Box~\MNRef{VBKM09_Abstand} in Module~\MNRef{VBKM09} 
for the distance of two points in a two-dimensional coordinate system immediately results in 
the formula for the two-dimensional case:
\begin{center}
\MTikzAuto{
\begin{tikzpicture}[>=stealth]
%Koordinatensystem
\draw[->,color=black] (-0.5,0) -- (4.5,0);
\draw[->,color=black] (0,-0.5) -- (0,3.5);
\draw[color=black] (-12pt,-6pt) node[right] {\footnotesize $O$};
%Achsenbeschriftung
\draw (4.5,0) node[anchor=north west] {$x$};
\draw (-0.5,3.8) node[anchor=north west] {$y$};
%Hilfslinien
\draw[color=red, dashed] (3,2) -- (3,-0.1);
\draw[color=red, dashed] (3,2) -- (-0.1,2);
\draw[color=red] (3,0) -- (3,-0.1);
\draw[color=black] (3,0.2) -- (2.8,0.2);
\draw[color=black] (2.8,0.2) -- (2.8,0);
\draw[color=black,fill=black] (2.9,0.1) circle (0.5pt);
\draw[color=red] (3,-0.1) node[anchor=north] {\footnotesize $a_x$};
\draw[color=red] (-0.1,2) node[anchor=east] {\footnotesize $a_y$};
%Punkte
\draw[fill=black] (0,0) circle (1.5pt);
\draw[fill=red,color=red] (3,2) circle (1.5pt);
\draw[color=red] (3,2) node[anchor=south] {\footnotesize $P$};
%Pfeil
\draw[color=black,->] (0,0) -- (3,2);
\draw[color=black] (2.1,1.3) node[anchor=south east] {\footnotesize $\MVec{a}=\MDVec{P}$};
\draw[color=black] (1.5,1) node[anchor=north] {\footnotesize $|\MVec{a}|$};
\end{tikzpicture}
} 
\end{center}
This is another simple application of \MSRef{VBKM05_Pythagoras}{Pythagoras' theorem}. The three-dimensional 
case is not much more complex: Pythagoras' theorem can still be applied:
\begin{center}
\MTikzAuto{%
\begin{tikzpicture}[>=stealth]

% The axes
\draw[->] (xyz cs:x=-2.5) -- (xyz cs:x=3.5) node[above] {\footnotesize $x$};
\draw[->] (xyz cs:y=-1.5) -- (xyz cs:y=2.5) node[right] {\footnotesize $y$};
\draw[->] (xyz cs:z=-3) -- (xyz cs:z=2.5) node[left] {\footnotesize $z$};
% The ticks
% \foreach \coo in {-2,-1,1,2,3}
% {
%   \draw (\coo,-3pt) -- (\coo,3pt) node[below=4pt] {\footnotesize \coo};
%   \draw (-3pt,\coo) -- (3pt,\coo) node[left=4pt] {\footnotesize \coo};
%   \draw (xyz cs:y=-0.1pt,z=\coo) -- (xyz cs:y=0.1pt,z=\coo) node[below=3pt] {\scriptsize \coo};
% }
%Coordinates
\draw[dashed, color=red] (3,1,1) -- (0,1,1);
\draw[color=red] (3,1,1) -- (3,0,1);
\draw[dashed, color=red] (3,1,1) -- (3,1,0);
\draw[dashed, color=red] (0,1,1) -- (0,1,0);
\draw[dashed, color=red] (3,1,0) -- (0,1,0);
\draw[color=red] (3,0,1) -- (0,0,1);
\draw[color=red] (3,0,1) -- (3,0,0);
\draw[dashed, color=red] (0,1,1) -- (0,0,1);
\draw[dashed, color=red] (3,1,0) -- (3,0,0);

%Labels
\draw[color=red] (1.5,0,1) node[below=0.2pt] {\footnotesize $a_x$};
\draw[color=red] (3,0,0.6) node[right=0.2pt] {\footnotesize $a_z$};
\draw[color=red] (3.1,0.6,1) node[left=0.2pt] {\footnotesize $a_y$};

% % Points:
\draw[fill=black] (0,0,0) circle (1.5pt);
\draw[color=black] (0.1,0,0) node[anchor=south east] {\footnotesize $O$};
\draw[fill=red] (3,1,1) circle (1.5pt);
\draw[color=black] (3,1,1) node[anchor=west] {\footnotesize $\MDVec{P}=\MVec{a}$};

%Arrows and lines
\draw[->, line width = 1.2pt] (0,0,0) -- (3,1,1);
\draw[color=blue, line width=1.2pt] (0,0,0) -- (3,0,1);
\draw[color=black] (0,0,0.8) -- (0.2,0,0.8);
\draw[color=black] (0.2,0,0.8) -- (0.2,0,1);
\draw[fill=black] (0.1,0,0.9) circle (0.3pt);
\draw[color=blue] (3,0.2,1) -- (2.81,0.2,0.94);
\draw[color=blue] (2.81,0.2,0.94) -- (2.81,0,0.94);
\draw[fill=blue] (2.9,0.1,0.97) circle (0.5pt);
\end{tikzpicture}
}
\end{center}
In the figure above two right triangles can be identified. From the right triangle lying in the 
$xz$-plane we obtain $\sqrt{a_x^2+a_z^2}$ for the length of the blue line. From the second right triangle 
we then obtain for the length of the segment from $O$ to $P$, i.e. for $|\MVec{a}|$:
\[
 \sqrt{a_y^2+\left(\sqrt{a_x^2+a_z^2}\right)^2}=\sqrt{a_x^2+a_y^2+a_z^2}\MDFPeriod
\]

For norms of vectors the following calculation rules apply:

\begin{MInfo}
Let two vectors $\MVec{a}$ and $\MVec{b}$ (both in $\R^2$ or in $\R^3$) and a real number $r\in\R$ be given. Then, we have:
\begin{enumerate}
 \item $|\MVec{a}|\geq 0$ and $|\MVec{a}|=0\MDFPSpace\Leftrightarrow\MDFPSpace\MVec{a}=\MDVec{O}$,
 \item $|r\MVec{a}|=|r|\cdot|\MVec{a}|$, and
 \item $|\MVec{a}+\MVec{b}|\leq |\MVec{a}|+|\MVec{b}|$.
\end{enumerate}
The first rule states that norms are always non-negative and that only the norm of the zero vector is $0$. The second rule 
is especially useful in calculating the norms of constant multiples of vectors. The third rule is called the 
\MEntry{triangle inequality}{triangle inequality}.
\end{MInfo}

\begin{MExample}
\begin{itemize}
 \item The absolute value of the vector $\frac{1}{4}\MVector{3\\-1\\\sqrt{6}}$ is 
 \[
  \left|\frac{1}{4}\MVector{3\\-1\\\sqrt{6}}\right|=\left|\frac{1}{4}\right|\left|\MVector{3\\-1\\\sqrt{6}}\right|=\frac{1}{4}\sqrt{3^2+(-1)^2+6}=\frac{\sqrt{16}}{4}=1 \MDFPeriod
 \]
 Hence, this is a unit vector.
 \item Find a number $q\in\R$ such that $\left|\MVector{q^2-2\\4}-2\MVector{q-1\\q}\right|=0$.
 \[
  \left|\MVector{q^2-2\\4}-2\MVector{q-1\\q}\right|=\left|\MVector{q^2-2q\\4-2q}\right|=0\MDFPaSpace\Leftrightarrow\MDFPaSpace\MVector{q^2-2q\\4-2q}=\MVector{0\\0}
 \]
 \[
  \Leftrightarrow\MDFPaSpace q^2-2q=0\MDFPSpace\textrm{and}\MDFPSpace4-2q=0\MDFPaSpace\Leftrightarrow\MDFPaSpace q(q-2)=0 \MDFPSpace\textrm{and}\MDFPSpace 2(2-q)=0
  \MDFPaSpace\Leftrightarrow\MDFPaSpace q=2 \MDFPeriod
 \]

\end{itemize}
 
\end{MExample}

\begin{MExercise}
Calculate\\
$\left|-\frac{1}{3}\MVector{2\\-14\\5}\right|=$\MLFunctionQuestion{15}{5}{5}{x}{5}{NORM1}.
\begin{MHint}{Solution}
\[
 \left|-\frac{1}{3}\MVector{2\\-14\\5}\right|=\left|-\frac{1}{3}\right|\cdot\left|\MVector{2\\-14\\5}\right|=\frac{1}{3}\sqrt{2^2+(-14)^2+5^2}=\frac{1}{3}\sqrt{225}=\frac{15}{3}=5 \MDFPeriod
\]
 
\end{MHint}

\end{MExercise}


\begin{MExercise}
Find the number $\chi>3$ such that $\left|\MVector{3\\\chi}-\MVector{\chi\\3}\right|=2\sqrt{2}$.\\
$\chi=$\MLFunctionQuestion{15}{5}{5}{x}{5}{ZAHL1}.
\end{MExercise}

\begin{MHint}{Solution}
\[
\left|\MVector{3\\\chi}-\MVector{\chi\\3}\right| =\left|\MVector{3-\chi\\\chi-3}\right|=\sqrt{(3-\chi)^2+(\chi-3)^2}=2\sqrt{2}\MDFPaSpace\Leftrightarrow\MDFPaSpace\sqrt{2(3-\chi)^2}=2\sqrt{2}
\]
\[
 \Leftrightarrow\MDFPaSpace|3-\chi|=2\MDFPaSpace\Leftrightarrow\MDFPaSpace\chi=1\MDFPSpace\textrm{or}\MDFPSpace\chi=5 \MDFPeriod
\]
From $\chi>3$, it follows that $\chi=5$.
\end{MHint}

\begin{MExercise}
Show that the points $A=\MPointThree{4}{2}{7}$, $B=\MPointThree{3}{1}{9}$, and $C=\MPointThree{2}{3}{8}$ 
are the vertices of an equilateral triangle.
\end{MExercise}

\begin{MHint}{Solution}
The triangle is equilateral if
\[
 |\MDVec{A B}| = |\MDVec{A C}| = |\MDVec{B C}| \MDFPeriod
\]

\[
 |\MDVec{A B}| = |\MDVec{B}-\MDVec{A}|=\left|\MVector{3\\1\\9}-\MVector{4\\2\\7}\right|=\left|\MVector{-1\\-1\\2}\right|=\sqrt{(-1)^2+(-1)^2+2^2}=\sqrt{6} \MDFPSpace,
\]
\[
 |\MDVec{A C}| = |\MDVec{C}-\MDVec{A}|=\left|\MVector{2\\3\\8}-\MVector{4\\2\\7}\right|=\left|\MVector{-2\\1\\1}\right|=\sqrt{(-2)^2+1^2+1^2}=\sqrt{6} \MDFPSpace,
\]
\[
 |\MDVec{B C}| = |\MDVec{C}-\MDVec{B}|=\left|\MVector{2\\3\\8}-\MVector{3\\1\\9}\right|=\left|\MVector{-1\\2\\-1}\right|=\sqrt{(-1)^2+2^2+(-1)^2}=\sqrt{6} \MDFPeriod
\]
Thus, the triangle is equilateral.
\end{MHint}
\end{MXContent}

\MSubsection{Lines and Planes}
\MLabel{VBKM10_GeradenEbenen}

\begin{MIntro}
\MDeclareSiteUXID{VBKM10_GeradenEbenen_Intro}

In this section vectors are used (first of all) to describe lines in the plane. Then we will see that 
this description of lines can be extended to the three-dimensional case. Besides lines, there are other mathematical objects in space which can easily be 
described by means of vectors, namely planes. Finally, we will discuss the possible relative positions of points, 
lines, and planes with respect to each other.
 
For this purpose, the concepts outlined in the Info Box below are important.

\begin{MInfo}\MLabel{info:kollinearkomplanar}
\begin{itemize}
 \item Two vectors $\MVec{a}$ and $\MVec{b}$ in $\R^2$ or $\R^3$ ($\MVec{a},\MVec{b}\neq\MDVec{O}$) are called 
  \MEntry{collinear}{collinear} if there exists a number $s\in\R$ such that
 \[
  \MVec{a}=s\MVec{b}\MDFPeriod
 \]
 \item Three vectors $\MVec{a}$, $\MVec{b}$, and $\MVec{c}$ in $\R^3$ ($\MVec{a},\MVec{b},\MVec{c}\neq\MDVec{O}$) 
  are called \MEntry{coplanar}{coplanar} if there exist two numbers $s,t\in\R$ such that
 \[
  \MVec{a}=s\MVec{b}+t\MVec{c}\MDFPeriod
 \]
\end{itemize}
\end{MInfo}

\begin{MHint}{Further contents}
In this course, the zero vector is excluded from the definition of collinearity and coplanarity
since it will be not needed for our simple description of lines and planes. If the conditions 
for collinearity and coplanarity are made slightly more complex, the definition also extends 
to the zero vector. The required considerations will naturally involve the (important) terms of 
\textbf{linear independence} and \textbf{linear span} that, however, go far beyond the scope of this 
course. 
\end{MHint}

The following considerations and figures illustrate why the concepts of collinearity and coplanarity
are relevant for the investigation of lines and planes.

Collinear vectors are multiples of each other. The representatives of collinear vectors with 
the same initial point lie on the same line. For example, the vectors 
\[
 \MVec{x}=\MVector{2\\-1}
\]
and 
\[
 \MVec{y}=\MVector{-1\\\frac{1}{2}}
\]
are collinear since 
\[
 \MVec{y}=\MVector{-1\\\frac{1}{2}}=-\frac{1}{2}\MVector{2\\-1}=-\frac{1}{2}\MVec{x}
\]
(or also $\MVec{x}=-2\MVec{y}$). Further vectors that are collinear to both $\MVec{x}$ and 
$\MVec{y}$ are, for example, $\MVector{4\\-2}$ or $\MVector{-2\\1}$. In contrast, the vector 
$\MVector{1\\1}$ is not collinear to $\MVec{x}$ (and hence not collinear to $\MVec{y}$)
since there \textit{cannot} be any number $s\in\R$ that satisfies the equation 
\[
 \MVec{x}=\MVector{2\\-1}=s\MVector{1\\1}\MDFPeriod
\]
Representatives of collinear vectors with the same initial point, such as the arrows 
of the corresponding position vectors (see figure below), all lie on the same line.
\begin{center}
\MTikzAuto{
\begin{tikzpicture}
%Koordinatensystem
\draw[->,color=black] (-2.5,0) -- (4.5,0);
\foreach \x in {-2,-1,1,2,3,4}
\draw[shift={(\x,0)},color=black] (0pt,2pt) -- (0pt,-2pt) node[below] {\footnotesize $\x$};
\draw[->,color=black] (0,-2.5) -- (0,1.5);
\foreach \y in {-2,-1,1}
\draw[shift={(0,\y)},color=black] (2pt,0pt) -- (-2pt,0pt) node[left] {\footnotesize $\y$};
\draw[color=black] (-10pt,-8pt) node[right] {\footnotesize $0$};
%Achsenbeschriftung
\draw (4.5,0) node[anchor=north west] {$x$};
\draw (-0.5,1.8) node[anchor=north west] {$y$};
%Pfeile
\draw[color=red, ->, line width=2pt] (0,0) -- (4,-2);
\draw[color=blue, ->, line width=2pt] (0,0) -- (2,-1);
\draw[color=violet, ->, line width=2pt] (0,0) -- (-2,1);
\draw[color=green, ->, line width=2pt] (0,0) -- (-1,0.5);
%Beschriftung
\draw[color=red] (4,-1.9) node[anchor=south] {\scriptsize $\MVector{4\\-2}$};
\draw[color=blue] (2,-1.1) node[anchor=north] {\scriptsize $\MVector{2\\-1}$};
\draw[color=violet] (-2,1.1) node[anchor=south] {\scriptsize $\MVector{-2\\1}$};
\draw[color=green] (-1,0.6) node[anchor=south] {\scriptsize $\MVector{-1\\\frac{1}{2}}$};
\draw[color=black] (4.5,-2.25) node[anchor=north] {\scriptsize Line};
%Gerade
\draw[color=black] (-2.5,1.25) -- (4.5,-2.25);
\end{tikzpicture}
} 
\end{center}
For coplanar vectors in space, their representatives lie in the same 
plane if they have the same initial point. For example, the vectors
\[
 \MVec{e}_1=\MVector{1\\0\\0}\MDFPSpace,\MDFPaSpace\MVec{e}_2=\MVector{0\\1\\0}\MDFPSpace\textrm{and}\MDFPSpace\MVector{2\\3\\0}
\]
are coplanar since 
\[
 \MVector{2\\3\\0}=2\MVec{e}_1 + 3\MVec{e}_2=2\MVector{1\\0\\0}+3\MVector{0\\1\\0}\MDFPeriod
\]
The arrows of their corresponding position vectors all lie in the $x y$-plane of a coordinate system in space. 
In contrast, the vectors 
\[
 \MVec{e}_1=\MVector{1\\0\\0},\MDFPSpace\MVec{e}_2=\MVector{0\\1\\0}\MDFPSpace, \textrm{and}\MDFPSpace\MVector{2\\3\\2}
\]
are \textit{not} coplanar since the vector $\MVector{2\\3\\2}$ has a non-zero $z$-component, i.e. all its representatives 
are perpendicular to the $x y$-plane. It can easily be seen that there \textit{cannot} be any numbers $s,t\in\R$ 
such that the equation 
\[
 \MVector{2\\3\\2}=s\MVector{1\\0\\0}+t\MVector{0\\1\\0}
\]
is satisfied. This is illustrated in the figure below.
\begin{center}
\MTikzAuto{%
\begin{tikzpicture}[>=stealth]
% The axes
\draw[->] (xyz cs:x=-1.5) -- (xyz cs:x=3.5) node[above] {\footnotesize $x$};
\draw[->] (xyz cs:y=-1.5) -- (xyz cs:y=3.5) node[right] {\footnotesize $y$};
\draw[->] (xyz cs:z=-1.5) -- (xyz cs:z=3.5) node[left] {\footnotesize $z$};
% The ticks
\foreach \coo in {-1,1,2,3}
{
  \draw (\coo,-3pt) -- (\coo,3pt) node[below=4pt] {\footnotesize \coo};
  \draw (-3pt,\coo) -- (3pt,\coo) node[left=4pt] {\footnotesize \coo};
  \draw (xyz cs:y=-0.1pt,z=\coo) -- (xyz cs:y=0.1pt,z=\coo) node[below=3pt] {\scriptsize \coo};
}
%Plane
\def \q1{(-1.5,-1.5,0) -- (3.5,-1.5,0) -- (3.5,3.5,0) -- (-1.5,3.5,0)}
\fill[color=red,fill=red,fill opacity=0.15] \q1;
\draw[color=red] (2,2,0) node[right] {\scriptsize $x y$-plane};
%Hilfslinien
\draw[color=gray,dashed,line width=1pt] (0,0,2) -- (2,0,2);
\draw[color=gray,dashed,line width=1pt] (2,0,0) -- (2,0,2);
\draw[color=gray,dashed,line width=1pt] (2,0,2) -- (2,3,2);
%Arrows
\draw[color=red,->,line width=1pt] (0,0,0) -- (1,0,0);
\draw[color=red,->,line width=1pt] (0,0,0) -- (0,1,0);
\draw[color=green,->,line width=1pt] (0,0,0) -- (2,3,0);
\draw[color=black,->,line width=1pt] (0,0,0) -- (2,3,2);
%Labels
\draw[color=red] (0.5,0,0) node[below] {\scriptsize $e_1$};
\draw[color=red] (0,0.5,0) node[left] {\scriptsize $e_1$};
\draw[color=black] (2,3,2) node[above] {\scriptsize $\MVector{2\\3\\2}$};
\draw[color=green] (2,3,0) node[above] {\scriptsize $\MVector{2\\3\\0}$};
\end{tikzpicture}
}
\end{center}

\end{MIntro}

\begin{MXContent}{Lines in the Plane and in Space}{Lines plane space}{STD}
\MLabel{VBKM10_GeradenEbene}
\MDeclareSiteUXID{VBKM10_GeradenEbene}
In Module~\MNRef{VBKM09} we described lines in the plane using coordinate equations for the 
points lying on the lines with respect to a fixed coordinate system. In this way, for example, a 
line $g$ with slope $\frac{1}{2}$ and $y$-intercept $1$ is given as the set of points
\[
 g=\{\MPointTwo{x}{y}\in\R^2\MCondSetSep y=\frac{1}{2}x+1\} \MDFPSpace,
\]
which is often abbreviated by specifying only the coordinate equation of a line (here in normal form):
\[
 g\colon y=\frac{1}{2}x+1\MDFPeriod
\]
The figure below shows this line.
\begin{center}
\MTikzAuto{
\begin{tikzpicture}
%Koordinatensystem
\draw[->,color=black] (-3.5,0) -- (4.5,0);
\foreach \x in {-3,-2,-1,1,2,3,4}
\draw[shift={(\x,0)},color=black] (0pt,2pt) -- (0pt,-2pt) node[below] {\footnotesize $\x$};
\draw[->,color=black] (0,-1.5) -- (0,3.5);
\foreach \y in {-1,1,2,3}
\draw[shift={(0,\y)},color=black] (2pt,0pt) -- (-2pt,0pt) node[left] {\footnotesize $\y$};
\draw[color=black] (-10pt,-8pt) node[right] {\footnotesize $0$};
%Achsenbeschriftung
\draw (4.5,0) node[anchor=north west] {$x$};
\draw (-0.5,3.8) node[anchor=north west] {$y$};
%Gerade
\draw[color=red] (-3.5,-0.75) -- (4.5,3.25);
\draw[color=red] (4.5,3.25) node[anchor=north] {\footnotesize $g$};
\end{tikzpicture}
} 
\end{center}

In the following sections we want to describe the points on the line by only their corresponding position vectors. 
The considerations outlined below result in the following description: the coordinates of the points $\MPointTwo{x}{y}$
lying on the line $g$ satisfy the equation
\[
 y=\frac{1}{2}x+1\MDFPeriod
\]
This equation can be substituted into the description of the point. Then, we see that the line $g$ consists
of points of the form $\MPointTwo[\Big]{x}{\frac{1}{2}x+1}$ with $x\in\R$. Let the corresponding position
vectors of these points be denoted by $\MVec{r}$. Then, we have
\[
 \MVec{r}=\MVector{x\\\frac{1}{2}x+1}=x\MVector{1\\\frac{1}{2}}+\MVector{0\\1}
\]
with $x\in\R$. Using position vectors, the line $g$ can also be described by
\[
 g\colon \MVec{r}=x\MVector{1\\\frac{1}{2}}+\MVector{0\\1} \MDFPSpace,\MDFPaSpace x\in\R \MDFPeriod
\]
In other words: The points of $g$ are defined by the sum of the vector $\MVec{a}=\MVector{0\\1}$ and all 
possible multiples of the vector $\MVec{u}=\MVector{1\\\frac{1}{2}}$, i.e. all vectors collinear to 
the vector $\MVector{1\\\frac{1}{2}}$. The figure below illustrates this approach.

\begin{center}
\MTikzAuto{
\begin{tikzpicture}
%Koordinatensystem
\draw[->,color=black] (-3.5,0) -- (4.5,0);
\foreach \x in {-3,-2,-1,1,2,3,4}
\draw[shift={(\x,0)},color=black] (0pt,2pt) -- (0pt,-2pt) node[below] {\footnotesize $\x$};
\draw[->,color=black] (0,-1.5) -- (0,3.5);
\foreach \y in {-1,1,2,3}
\draw[shift={(0,\y)},color=black] (2pt,0pt) -- (-2pt,0pt) node[left] {\footnotesize $\y$};
\draw[color=black] (-10pt,-8pt) node[right] {\footnotesize $0$};
%Achsenbeschriftung
\draw (4.5,0) node[anchor=north west] {$x$};
\draw (-0.5,3.8) node[anchor=north west] {$y$};
%Gerade
\draw[color=red] (-3.5,-0.75) -- (4.5,3.25);
\draw[color=red] (4.5,3.25) node[anchor=south] {\footnotesize $g$};
%Hilfslinien
\draw[color=gray] (0,1) -- (-0.45,1.89);
\draw[color=gray] (2,2) -- (1.55,2.89);
\draw[color=gray] (3.5,2.75) -- (3.05,3.64);
\draw[color=gray] (1,1.5) -- (0.55,2.39);
\draw[color=gray] (-2,0) -- (-2.44,0.89);
%Richtungsvektoren
\draw[fill=red] (1,1.5) circle (1.5pt);
\draw[fill=red] (-2,0) circle (1.5pt);
\draw[fill=red] (2,2) circle (1.5pt);
\draw[fill=red] (3.5,2.75) circle (1.5pt);
\draw[color=violet,->] (0,1) -- (1,1.5);
\draw[color=violet,->] (0,1) -- (-2,0);
\draw[color=violet,->] (-0.09,1.18) -- (1.91,2.18);
\draw[color=violet,->] (-0.18,1.36) -- (3.32,3.11);
%Aufpunkt
\draw[fill=blue] (0,1) circle (1.5pt);
\draw[color=blue,->,line width=0.8pt] (0,0) -- (0,1);
\draw[color=blue] (0,0.5) node[anchor=east] {\scriptsize $\MVec{a}$};
\draw[color=blue] (0,0.95) node[anchor=west] {\scriptsize $\MPointTwo{0}{1}$};
%Beschriftung
\draw[color=violet] (-2.44,0.89) node[anchor=south] {\scriptsize $-2\MVec{u}$};
\draw[color=violet] (-0.45,1.89) node[anchor=south east] {\scriptsize $0\cdot\MVec{u}$};
\draw[color=violet] (0.55,2.39) node[anchor=south] {\scriptsize $\MVec{u}$};
\draw[color=violet] (1.55,2.89) node[anchor=south] {\scriptsize $2\MVec{u}$};
\draw[color=violet] (3.05,3.64) node[anchor=south] {\scriptsize $\MZahl{3}{5}\cdot\MVec{u}$};
\draw[color=red] (0.9,1.6) node[anchor=north west] {\scriptsize $\MPointTwo{1}{\frac{3}{2}}$}; 
\draw[color=red] (1.9,2.1) node[anchor=north west] {\scriptsize $\MPointTwo{2}{2}$};
\draw[color=red] (3.4,2.85) node[anchor=north west] {\scriptsize $\MPointTwo{\frac{7}{2}}{\frac{11}{4}}$};
\draw[color=red] (-2.05,-0.05) node[anchor=south east] {\scriptsize $\MPointTwo{-2}{0}$};
\end{tikzpicture}
} 
\end{center}

The Info Box below outlines the most relevant terms, methods and concepts for this 
so-called \textbf{vector form} or \textbf{parametric form} of an equation of a line.

\begin{MInfo}
\begin{itemize}
 \item A line $g$ in the plane is given in \MEntry{vector form}{vector form} or in 
  \MEntry{parametric form}{parametric form} as the set of position vectors
  \[
  g=\left\{\MVec{r}=\lambda\MVec{u}+\MVec{a}\in\R^2\MCondSetSep\lambda\in\R\right\} \MDFPSpace,
  \]
  often written in short as 
  \[
  g\colon \MVec{r}=\lambda\MVec{u}+\MVec{a}\MDFPSpace,\MDFPaSpace\lambda\in\R\MDFPeriod
  \]
  Here, $\lambda$ is called a \textbf{parameter}, $\MVec{a}$ is called
 the \MEntry{reference vector}{reference vector}, and $\MVec{u}\neq\MDVec{O}$ is called he \MEntry{direction vector}{direction vector} 
  of the line. The position vectors $\MVec{r}$ point to the individual points on the line. The reference vector 
  $\MVec{a}$ is the position vector of a fixed point on the line that is called a \MEntry{reference point}{reference point}.
  The multiples $\lambda\MVec{u}$ of the direction vector $\MVec{u}$ are all vectors collinear to $\MVec{u}$ (see figure below).
\begin{center}
\MTikzAuto{
\begin{tikzpicture}
%Koordinatensystem
\draw[->,color=black] (-1.5,0) -- (4.5,0);
%\foreach \x in {-3,-2,-1,1,2,3,4}
%\draw[shift={(\x,0)},color=black] (0pt,2pt) -- (0pt,-2pt) node[below] {\footnotesize $\x$};
\draw[->,color=black] (0,-1.5) -- (0,3.5);
%\foreach \y in {-1,1,2,3}
%\draw[shift={(0,\y)},color=black] (2pt,0pt) -- (-2pt,0pt) node[left] {\footnotesize $\y$};
%\draw[color=black] (-10pt,-8pt) node[right] {\footnotesize $0$};
%Achsenbeschriftung
\draw (4.5,0) node[anchor=north west] {$x$};
\draw (-0.5,3.8) node[anchor=north west] {$y$};
%Gerade
\draw[color=red] (-1.5,0.75) -- (4.5,3.75);
\draw[color=red] (4.5,3.75) node[anchor=north] {\footnotesize $g$};
%Vektoren
\draw[color=blue,->,line width= 1pt] (0,0) -- (1,2); 
\draw[color=violet,->,line width= 1pt] (1,2) -- (3,3); 
\draw[color=red,->,line width= 1pt] (0,0) -- (3,3); 
%Beschriftung
\draw[color=blue] (0.5,1) node[anchor=east] {\footnotesize $\MVec{a}$};
\draw[color=violet] (2,2.5) node[anchor=south] {\footnotesize $\lambda\MVec{u}$};
\draw[color=red] (1.5,1.5) node[anchor=west] {\footnotesize $\MVec{r}$};
\end{tikzpicture}
} 
\end{center}
\item For a line $g$ given in normal form by the equation 
\[
 g\colon y=m x+b
\]
the vector form of the equation of a line can be found by generating the position vectors 
$\MVector{x\\m x+b}=x\MVector{1\\m}+\MVector{0\\b}$. Then, the vector form of the equation of a line is
\[
 g\colon \MVec{r}=x\MVector{1\\m}+\MVector{0\\b}\MDFPSpace,\MDFPaSpace x\in\R 
\]
with the direction vector $\MVec{u}=\MVector{1\\m}$ and the reference vector $\MVec{a}=\MVector{0\\b}$.
\item For a line $g$ given by the equation in parametric form:
\[
 g\colon \MVec{r}=\lambda\MVec{u}+\MVec{a}\MDFPSpace,\MDFPaSpace\lambda\in\R
\]
the corresponding equation of the line can be found as follows: the direction vector $\MVec{u}=\MVector{u_x\\u_y}$
immediately provides the slope of the line via a slope triangle. We have
\[
 m=\frac{u_y}{u_x} \MDFPeriod
\]
\begin{center}
\MTikzAuto{
\begin{tikzpicture}
%Koordinatensystem
\draw[->,color=black] (-1.5,0) -- (4.5,0);
%\foreach \x in {-3,-2,-1,1,2,3,4}
%\draw[shift={(\x,0)},color=black] (0pt,2pt) -- (0pt,-2pt) node[below] {\footnotesize $\x$};
\draw[->,color=black] (0,-1.5) -- (0,3.5);
%\foreach \y in {-1,1,2,3}
%\draw[shift={(0,\y)},color=black] (2pt,0pt) -- (-2pt,0pt) node[left] {\footnotesize $\y$};
%\draw[color=black] (-10pt,-8pt) node[right] {\footnotesize $0$};
%Achsenbeschriftung
\draw (4.5,0) node[anchor=north west] {$x$};
\draw (-0.5,3.8) node[anchor=north west] {$y$};
%Gerade
\draw[color=red] (-1.5,0.75) -- (4.5,3.75);
\draw[color=red] (4.5,3.75) node[anchor=north] {\footnotesize $g$};
%Vektoren
\draw[color=blue,->,line width= 1pt] (0,0) -- (1,2); 
\draw[color=violet,->,line width= 1pt] (1,2) -- (3,3); 
%\draw[color=red,->,line width= 1pt] (0,0) -- (3,3); 
%Beschriftung
\draw[color=blue] (0.5,1) node[anchor=east] {\footnotesize $\MVec{a}$};
\draw[color=violet] (2,2.5) node[anchor=south] {\footnotesize $\MVec{u}$};
%\draw[color=red] (1.5,1.5) node[anchor=west] {\footnotesize $\MVec{r}$};
%Steigungsdreieck
\draw[color=violet] (1,2) -- (3,2);
\draw[color=violet] (3,2) -- (3,3);
\draw[color=violet] (2,2) node[anchor=north] {\footnotesize $u_x$};
\draw[color=violet] (3,2.5) node[anchor=west] {\footnotesize $u_y$};
\end{tikzpicture}
} 
\end{center}
(Here, $u_x$ must be non-zero, i.e. $u_x\neq 0$. The special case $u_x=0$ is discussed in the example below.)
From Section~\MNRef{VBKM09_Koordinatengleichungen} we know that we only need another point on the line to 
determine the $y$-intercept $b$ and specify the equation of the line in normal form. It is easiest to use the  
reference point $\MVec{a}$ for this.
\end{itemize}

\end{MInfo}
One can immediately see that the parametric form of the equation of a line is not unique. Every point on the line 
can be used as reference point, and the direction vector can be chosen from an arbitrary number of collinear
vectors. For example, the line $g$ with the equation of a line 
\[
 g\colon\frac{1}{2}x+1 
\]
in coordinate form discussed in the first example of the subsection is not only described by the vector equation
\[
 g\colon \MVec{r}=x\MVector{1\\\frac{1}{2}}+\MVector{0\\1}\MDFPSpace,\MDFPaSpace x\in\R
\]
in parametric form, but also by the equations
\[
 g\colon \MVec{r}=\lambda\MVector{2\\1}+\MVector{2\\2}\MDFPSpace,\MDFPaSpace \lambda\in\R
\]
or
\[
 g\colon \MVec{r}=\nu\MVector{-2\\-1}+\MVector{-2\\0}\MDFPSpace,\MDFPaSpace \nu\in\R \MDFPeriod
\]
Often representations are chosen such that the direction vector is as simple as possible. However,
for representations of the same line by different direction vectors or reference vectors, 
different parameter values have to be used since in different representations the same parameter value 
defines different points on the line. For example, the parameter value $\lambda=1$ defines, in the corresponding 
parametric equation of $g$, the point
\[
 1\cdot\MVector{2\\1}+\MVector{2\\2}=\MVector{4\\3},
\]
but the parameter value $\nu=1$ defines, in the corresponding 
parametric equation of $g$, the point
\[
 1\cdot\MVector{-2\\-1}+\MVector{-2\\0}=\MVector{-4\\-1}.
\]


The example below lists a few applications of equations of a line in vector form. 

\begin{MExample}
\begin{itemize}
 \item Let the line $g$ in the plane be given by the equation
 \[
  g\colon 2y-3x=6\MDFPeriod
 \]
  Find two different equations of a line of $g$ in vector form.
 
  First, we transform the equation of a line into normal form:
 \[
  2y-3x=6\MDFPaSpace\Leftrightarrow\MDFPaSpace y=\frac{3}{2}x+3 \MDFPeriod
 \]
  Thus, points on the line $g$ have the form $\MPointTwo[\Big]{x}{\frac{3}{2}x+3}$ , $x\in\R$
  described by the position vector $\MVector{x\\\frac{3}{2}x+3}$, $x\in\R$. Hence, one 
  possible parametric form is given by
 \[
  g\colon \MVec{r}=x\MVector{1\\\frac{3}{2}}+\MVector{0\\3}\MDFPSpace,\MDFPaSpace x\in\R\MDFPeriod
 \]
  Choosing another direction vector collinear to $\MVector{1\\\frac{3}{2}}$ and another reference point 
  on $g$ results in another equation of a line in parametric form. For example, $\MVector{2\\3}$ is 
  collinear to $\MVector{1\\\frac{3}{2}}$ since $\MVector{2\\3}=2\MVector{1\\\frac{3}{2}}$. $\MVector{2\\6}$ is another appropriate reference vector since the coordinates of the point 
  $\MPointTwo{2}{6}$ obviously satisfy the equation of a line. Hence,
 \[
  g\colon \MVec{r}=\sigma\MVector{2\\3}+\MVector{2\\6}\MDFPSpace,\MDFPaSpace \sigma\in\R
 \]
  is another possible parametric form of an equation of the line $g$. 
 \item In the case of a line with an equation that cannot be transformed into 
  normal form, such as 
 \[
  h\colon x=2 \MDFPSpace,
 \]
 an equation in parametric form can still be given.
 
  All points on the line $h$ have the form $\MPointTwo{2}{y}$ for $y\in\R$, with the corresponding 
  position vector $\MVector{2\\y}$ for $y\in\R$. Since $\MVector{2\\y}=y\MVector{0\\1}+\MVector{2\\0}$, 
  one possible vector form of $h$ is given by 
 \[
  h\colon \MVec{r}=y\MVector{0\\1}+\MVector{2\\0}\MDFPSpace,\MDFPaSpace y\in\R\MDFPeriod
 \]
 \item Let the line $\alpha$ be given in parametric form by
 \[
  \alpha\colon \MVec{r}=\mu\MVector{-3\\2} + \MVector{1\\1}\MDFPSpace,\MDFPaSpace \mu\in\R\MDFPeriod
 \]
  Find the corresponding equation of a line in normal form.
 
  The direction vector $\MVector{-3\\2}$ gives the slope $m=\frac{2}{-3}=-\frac{2}{3}$. Thus, the 
  equation of the line in normal form is
 \[
  \alpha\colon y=-\frac{2}{3}x+b \MDFPeriod
 \]
  The reference vector of $\alpha$ is $\MVector{1\\1}$. The reference point $\MPointTwo{1}{1}$ 
  can be substituted into the equation of the line to determine the $y$-intercept:
 \[
  1=-\frac{2}{3}\cdot1+b\MDFPSpace\Leftrightarrow\MDFPSpace b=\frac{5}{3}\MDFPeriod
 \]
  Thus, we have
 \[
  \alpha\colon y=-\frac{2}{3}x+\frac{5}{3} \MDFPeriod
 \]
 \item If, for example, a line is given by the equation
 \[
  \beta\colon \MVec{r}=\lambda\MVector{0\\-2} + \MVector{-1\\1}\MDFPSpace,\MDFPaSpace \lambda\in\R 
 \]
  in parametric form, where the $x$-component of the direction vector is $0$, the corresponding 
  equation of the line in component form can be found. 
 
  The direction vector with an $x$-component of $0$ implies that the line is parallel to the $y$-axis. Hence, 
  the equation of the line has the form 
 \[
  \beta\colon x=c \MDFPeriod
 \]
  The constant $c$ can be determined by substituting the reference point $\MPointTwo{-1}{1}$ into this equation resulting in 
  $-1=c$, and we get
 \[
  \beta\colon x=-1 \MDFPeriod
 \]
 \item Given the two points $P=\MPointTwo{-1}{-1}$ and $Q=\MPointTwo{3}{2}$, find the 
  equation of a line of $P Q$ in parametric form.

 For the direction vector we the connection vector 
 \[
  \MVec{u}=\MDVec{P Q}=\MDVec{Q}-\MDVec{P}=\MVector{3\\2}-\MVector{-1\\-1}=\MVector{4\\3}\MDFPaSpace
 \]
  and for the reference vector we use the position vector of a given point, for example, 
 \[
  \MVec{a}=\MDVec{P}=\MVector{-1\\-1}\MDFPeriod
 \]
 Thus, we  have
 \[
  P Q\colon \MVec{r}=\lambda\MVector{4\\3}+\MVector{-1\\-1}\MDFPSpace,\MDFPaSpace\lambda\in\R \MDFPeriod
 \]
 The line is shown in the figure below.
 \begin{center}
\MTikzAuto{
\begin{tikzpicture}
%Koordinatensystem
\draw[->,color=black] (-2.5,0) -- (4.5,0);
\foreach \x in {-2,-1,1,2,3,4}
\draw[shift={(\x,0)},color=black] (0pt,2pt) -- (0pt,-2pt) node[below] {\footnotesize $\x$};
\draw[->,color=black] (0,-2.5) -- (0,3.5);
\foreach \y in {-2,-1,1,2,3}
\draw[shift={(0,\y)},color=black] (2pt,0pt) -- (-2pt,0pt) node[left] {\footnotesize $\y$};
\draw[color=black] (10pt,5pt) node[left] {\footnotesize $0$};
%Achsenbeschriftung
\draw (4.5,0) node[anchor=north west] {$x$};
\draw (-0.5,3.8) node[anchor=north west] {$y$};
%Gerade
\draw[color=red] (-2,-1.75) -- (4.5,3.125);
\draw[color=red] (4.5,3.125) node[anchor=south] {\footnotesize $P Q$};

%Punkte
\draw[fill=red] (-1,-1) circle (1.5pt); 
\draw[color=red] (-1,-1) node[anchor=north] {\footnotesize $P$};
\draw[fill=red] (3,2) circle (1.5pt); 
\draw[color=red] (3,2) node[anchor=south] {\footnotesize $Q$};
%Vektoren
\draw[color=violet,->,line width=0.8pt] (-1,-1) -- (3,2);
\draw[color=violet] (1,0.5) node[anchor=south] {\footnotesize $\MVec{u}$};
\draw[color=blue,->,line width=0.8pt] (0,0) -- (-1,-1);
\draw[color=blue] (-0.5,-0.5) node[anchor=south] {\footnotesize $\MVec{a}$};

\end{tikzpicture}
} 
\end{center}
\end{itemize}

\end{MExample}

\begin{MExercise}
\begin{MExerciseItems}
\item{Let the line $g$ be given by the equation
\[
 g\colon \MVec{r}=t\MVector{-1\\5}+\MVector{2\\5}\MDFPSpace,\MDFPaSpace t\in\R
\]
in parametric form. Find the equation $g$ in normal form:\\
$g\colon y=$\MLFunctionQuestion{15}{-5*x+15}{5}{x}{5}{PARA1}.
}
\item{The line $h$ with the equation of a line in coordinate form 
\[
 h\colon\frac{1}{2}y+x+2=0
\]
has the parametric form
\[
 h\colon \MVec{r}=s\MVector{a\\2}+\MVector{b\\-5}\MDFPSpace,\MDFPaSpace s\in\R \MDFPeriod
\]
Find the missing values of $a$ and $b$.\\
$a=$\MLFunctionQuestion{15}{-1}{5}{x}{5}{PARA2}\\
$b=$\MLFunctionQuestion{15}{0.5}{5}{x}{5}{PARA3}
}
\item{Consider the two points $A=\MPointTwo{-2}{-1}$ and $B=\MPointTwo{3}{-\frac{3}{2}}$. Which of the 
following parametric equations are correct representations of the line $A B$?

\begin{MQuestionGroup}
\begin{tabular}{lll}
\MLCheckbox{1}{PARA4} & (i) & $A B:\MVec{r}=s\MVector{5\\-\frac{1}{2}}+\MVector{0\\-\frac{6}{5}},\MDFPSpace s\in\R$\\
\MLCheckbox{0}{PARA5} & (ii) & $A B:\MVec{r}=t\MVector{5\\\frac{1}{2}}+\MVector{-2\\-1},\MDFPSpace t\in\R$\\
\MLCheckbox{1}{PARA6} & (iii) & $A B:\MVec{r}=u\MVector{-5\\\frac{1}{2}}+\MVector{-12\\0},\MDFPSpace u\in\R$\\
\MLCheckbox{1}{PARA7} & (iv) & $A B:\MVec{r}=v\MVector{10\\-1}+\MVector{4\\-\frac{8}{5}},\MDFPSpace v\in\R$\\
\MLCheckbox{0}{PARA8} & (v) & $A B:\MVec{r}=w\MVector{-1\\10}+\MVector{-22\\1},\MDFPSpace w\in\R$\\
\MLCheckbox{0}{PARA9} & (vi) & $A B:\MVec{r}=z\MVector{\frac{5}{2}\\-\frac{1}{4}}+\MVector{\frac{6}{5}\\-1},\MDFPSpace z\in\R$\\
\end{tabular} 
\end{MQuestionGroup}

\MGroupButton{Check input}
}
\item{Find the value of $\psi$ such that the point $P$ with the position vector
\[
 \MDVec{P}=\MVector{-2\\\psi}
\]
lies on the line
\[
 i\colon \MVec{r}=\tau\MVector{1\\-3}+\MVector{-1\\2}\MDFPSpace,\MDFPaSpace\tau\in\R \MDFPSpace ,
\]
and find the value of the parameter $\tau$ such that $\MVec{r}=\MDVec{P}$.\\
$\psi=$\MLFunctionQuestion{15}{5}{5}{x}{5}{PARA10}\\
$\tau=$\MLFunctionQuestion{15}{-1}{5}{x}{5}{PARA11}
}
\end{MExerciseItems}

\begin{MHint}{Solution}
 \begin{MExerciseItems}
\item{%
The slope of the line can be found from the direction vector $\MVector{-1\\5}$: $m=\frac{5}{-1}=-5$. Thus, we have
\[
 g\colon y=-5x+b \MDFPeriod
\]
Substituting the reference point $\MPointTwo{2}{5}$ into the equation results in:
\[
 5=-5\cdot 2+b\MDFPSpace\Leftrightarrow\MDFPSpace b=15\MDFPeriod
\]
Thus, we have:
\[
 g\colon y=-5x+15\MDFPeriod
\]
}
\item{Transforming the equation of a line into normal form results in
\[
 \frac{1}{2}y+x+2=0\MDFPSpace\Leftrightarrow\MDFPSpace y=-2x-4 \MDFPeriod
\]
The reference point $\MPointTwo{b}{-5}$ corresponding to the reference vector $\MVector{b\\-5}$ must lie on the 
line, i.e. its coordinates must satisfy the equation of the line:
\[
 -5 = -2b-4\MDFPSpace\Leftrightarrow\MDFPSpace b=\frac{1}{2} \MDFPeriod
\]
The position vectors of the points on $h$ have the form $\MVector{x\\-2x-4}=x\MVector{1\\-2}+\MVector{0\\-4}$,
so $\MVector{1\\-2}$ is the direction vector of $h$. Other direction vectors of $h$ are collinear to 
this vector. Since
\[
 -1\cdot\MVector{1\\-2}=\MVector{-1\\2}
\]
we have $a=-1$.
}
\item{From the given points $A$ and $B$, we have for the direction vector
\[
 \MDVec{A B}=\MDVec{B}-\MDVec{A}=\MVector{3\\-\frac{3}{2}}-\MVector{-2\\-1}=\MVector{5\\-\frac{1}{2}} \MDFPeriod
\]
The direction vectors in the cases (ii) and (v) are not collinear to this vector. Thus, (ii) and (v)
do not represent the line $A B$ correctly. From the direction vector $\MDVec{A B}=\MVector{5\\-\frac{1}{2}}$, 
we know the slope $m=\frac{-\frac{1}{2}}{5}=-\frac{1}{10}$ of the line $A B$. Thus, the equation of the 
line is
\[
 A B\colon y=-\frac{1}{10}x+b \MDFPSpace ,
\]
and substituting the coordinates of $A$ into the equation results in the following value of the $y$-intercept 
$b$:
\[
 -1=-\frac{1}{10}\cdot(-2)+b\MDFPSpace\Leftrightarrow\MDFPSpace b=-\frac{6}{5}\MDFPeriod
\]
Hence, we have
\[
 A B\colon y=-\frac{1}{10}x-\frac{6}{5} \MDFPeriod
\]
The equation of the line is satisfied by the coordinates of the reference points in cases 
(i) to (v) but not by the coordinates of the reference point in the case (vi). Thus, the 
parametric equations in the cases (i), (iii), and (iv) represent the line correctly, the equations in the 
cases (ii), (v), and (vi), however, do not.
}
\item{
The condition 
\[
 \MVector{-2\\\psi} = \tau\MVector{1\\-3}+\MVector{-1\\2}
\]
results in the two equations
\[
 -2=\tau-1\MDFPSpace\textrm{ und }\MDFPSpace\psi=-3\tau+2 \MDFPeriod
\]
Thus, we first have $\tau=-1$ and hence, $\psi=-3\cdot(-1)+2=5$.
}
\end{MExerciseItems}
\end{MHint}

\end{MExercise}

In contrast to lines in the plane, lines in space \emph{cannot} be described by an equation of a line 
in coordinate form. However, the description by a parametric equation can be easily extended from two 
to three dimensions (see Info Box below).

\begin{MInfo}
A line $g$ in space is given in \MEntry{vector form}{vector form} or \MEntry{parametric form}{parametric form}
as the set of position vectors
\[
 g=\left\{\MVec{r}=\lambda\MVec{u}+\MVec{a}\in\R^3\MCondSetSep\lambda\in\R\right\} \MDFPSpace,
\]
often written in short as
\[
 g\colon \MVec{r}=\lambda\MVec{u}+\MVec{a}\MDFPSpace,\MDFPaSpace\lambda\in\R\MDFPeriod
\]
As in the two-dimensional case: $\lambda$ is called a \textbf{parameter}, $\MVec{a}$ is called the
\MEntry{reference vector}{reference vector}, and $\MVec{u}\neq\MDVec{O}$ is called 
the \MEntry{direction vector}{direction vector} of the line $g$ (see figure below).
\begin{center}
\MTikzAuto{%
\begin{tikzpicture}[>=stealth]

% The axes
\draw[->] (xyz cs:x=-2.5) -- (xyz cs:x=3.5) node[above] {\footnotesize $x$};
\draw[->] (xyz cs:y=-2.5) -- (xyz cs:y=3.5) node[right] {\footnotesize $y$};
\draw[->] (xyz cs:z=-2.5) -- (xyz cs:z=3.5) node[left] {\footnotesize $z$};
% The ticks
\foreach \coo in {-2,-1,1,2,3}
{
  \draw (\coo,-3pt) -- (\coo,3pt) node[below=4pt] {\footnotesize \coo};
  \draw (-3pt,\coo) -- (3pt,\coo) node[left=4pt] {\footnotesize \coo};
  \draw (xyz cs:y=-0.1pt,z=\coo) -- (xyz cs:y=0.1pt,z=\coo) node[below=3pt] {\scriptsize \coo};
}
\draw[color=red] (-2,-1,2) -- (3,3,-1);
\draw[color=blue,->,line width = 0.8pt] (0,0,0) -- (0.5,1,0.5);
\draw[color=violet,->,line width=0.8pt] (0.5,1,0.5) -- (1.75,2,-0.25);
\draw[color=red,->,line width=0.8pt] (0,0,0) -- (1.75,2,-0.25);
\draw[color=red] (3,3,-1) node[below] {\scriptsize $g$};
\draw[color=blue] (0.25,0.4,0.25) node[left] {\scriptsize $\MVec{a}$};
\draw[color=violet] (1,1.5,0.375) node[above] {\scriptsize $\lambda\MVec{u}$};
\draw[color=red] (0.875,1,-0.125) node[right] {\scriptsize $\MVec{r}$};
\end{tikzpicture}
}
\end{center}
\end{MInfo}

In this three-dimensional case, as in the plane, the parametric form of the equation of a line is not unique. The example below lists a few applications of parametric equations of a line 
in space.

\begin{MExample}
Let the two points $P=\MPointThree{-1}{-2}{\frac{1}{2}}$ and $Q=\MPointThree{2}{0}{8}$ in space be given. 
Find two different representations of the line $P Q$ in parametric form.

We use the connecting vector $\MDVec{P Q}$ as direction vector:
\[
 \MDVec{P Q}=\MDVec{Q}-\MDVec{P}=\MVector{2\\0\\8}-\MVector{-1\\-2\\\frac{1}{2}}=\MVector{3\\2\\\frac{15}{2}}\MDFPeriod
\]
The point $Q$ can be used as reference point. For the parametric form we get
\[
 P Q\colon \MVec{r}=t\MVector{3\\2\\\frac{15}{2}}+\MVector{2\\0\\8}\MDFPSpace,\MDFPaSpace t\in\R\MDFPeriod
\]
Further possible direction vectors must be collinear to the vector $\MDVec{P Q}$. For example:
\[
 \MVector{-6\\-4\\-15}=-2\MVector{3\\2\\\frac{15}{2}}\MDFPeriod
\]
We can also use the point $P$ as reference point which results in
\[
 P Q\colon \MVec{r}=s\MVector{-6\\-4\\-15}+\MVector{-1\\-2\\\frac{1}{2}}\MDFPSpace,\MDFPaSpace s\in\R
\]
as another correct equation of the line $P Q$ in vector form.
\end{MExample}

\begin{MExercise}
\begin{MExerciseItems}
\item{
The line $h=A B$ through the points $A=\MPointThree{-1}{-1}{0}$ and $B=\MPointThree{-3}{0}{1}$ 
has the parametric equation
\[
 h\colon \MVec{r}=\lambda\MVector{4\\a\\b}+\MVector{c\\d\\-4}\MDFPSpace,\MDFPaSpace\lambda\in\R \MDFPeriod
\]
Find the missing values of $a$, $b$, $c$, and $d$.\\
$a=$\MLFunctionQuestion{15}{-2}{5}{x}{5}{PARA31}\\
$b=$\MLFunctionQuestion{15}{-2}{5}{x}{5}{PARA32}\\
$c=$\MLFunctionQuestion{15}{7}{5}{x}{5}{PARA33}\\
$d=$\MLFunctionQuestion{15}{-5}{5}{x}{5}{PARA34}\\
}
\item{
Find the value of $\chi$ such that the point $P$ with the position vector 
\[
 \MDVec{P}=\MVector{-2\\\chi\\-8}
\]
lies on the line
\[
 g\colon \MVec{r}=\nu\MVector{1\\-3\\8}+\MVector{-1\\2\\0}\MDFPSpace,\MDFPaSpace\nu\in\R \MDFPSpace ,
\]
and find the value of the parameter $\nu$ such that $\MVec{r}=\MDVec{P}$.\\
$\chi=$\MLFunctionQuestion{15}{5}{5}{x}{5}{PARA35}\\
$\nu=$\MLFunctionQuestion{15}{-1}{5}{x}{5}{PARA36}
}
\end{MExerciseItems}

\begin{MHint}{Solution}
 \begin{MExerciseItems}
\item{
From the given points $A$ and $B$, we have for the direction vector
\[
 \MDVec{A B}=\MDVec{B}-\MDVec{A}=\MVector{-3\\0\\1}-\MVector{-1\\-1\\0}=\MVector{-2\\1\\1} \MDFPeriod
\]
Thus, $\MVector{4\\a\\b}$ is collinear to $\MVector{-2\\1\\1}$, and hence it is also a possible direction vector of 
$h$ for $a=b=-2$ since in this case, we have
\[
 \MVector{4\\-2\\-2} = -2\cdot\MVector{-2\\1\\1} \MDFPeriod
\]
The reference vector $\MVector{c\\d\\-4}$ must correspond to a point on $h$. Using $\MDVec{A B}$
as direction vector and $\MDVec{A}$ as reference vector, one possible equation of the line $h$ in parametric form
is
\[
 h\colon\MVec{r}=t\MVector{-2\\1\\1}+\MVector{-1\\-1\\0}\MDFPSpace,\MDFPaSpace t\in\R \MDFPeriod
\]
This results in the equation
\[
 \MVector{c\\d\\-4}=t\MVector{-2\\1\\1}+\MVector{-1\\-1\\0}=\MVector{-2t-1\\t-1\\t} \MDFPSpace ,
\]
and from the third component we immediately read off $t=-4$. Thus, we have
\[
 \MVector{c\\d\\-4}=-4\cdot\MVector{-2\\1\\1}+\MVector{-1\\-1\\0}=\MVector{7\\-5\\-4}
\]
and hence $c=7$ and $d=-5$.
}
\item{
The condition  
\[
 \MVector{-2\\\chi\\-8} = \nu\MVector{1\\-3\\8}+\MVector{-1\\2\\0} = \MVector{\nu-1\\-3\nu+2\\8\nu}
\]
results for the first and the third component in $\nu=-1$, and from the second component we have  $\chi=-3\cdot(-1)+2=5$.
}
\end{MExerciseItems}
\end{MHint}

\end{MExercise}

\end{MXContent}



\begin{MXContent}{Planes in Space}{Planes Space}{STD}
\MLabel{VBKM10_EbenenRaum}
\MDeclareSiteUXID{VBKM10_EbenenRaum}


Starting from a vector $\MVec{u}$ in space one obtains all vectors that are collinear to $\MVec{u}$ 
(see Info Box~\MNRef{info:kollinearkomplanar}) by taking all multiples $\lambda\MVec{u}$, $\lambda\in\R$
of this vector. Interpreted as position vectors, all these collinear vectors in combination with an arbitrary 
reference vector constitute the parametric equation of a line as discussed in the previous Subsection~\MNRef{VBKM10_GeradenEbene}.
With this in mind, one may ask which object we obtain starting from two fixed (but \textit{non-collinear}) 
vectors $\MVec{u}$ and $\MVec{v}$ and considering all their coplanar vectors (all vectors that result from 
$\lambda\MVec{u}+\mu\MVec{v}$; $\lambda,\mu\in\R$ - see Info Box~\MNRef{info:kollinearkomplanar}). 
This, in combination with an arbitrary reference vector, generalises the concept of the parametric equation of a line resulting 
in the parametric equation of a plane in space which is outlined in the Info Box below. 

Planes are usually denoted by uppercase Latin letters ($E$, $F$, $G$, $\MHDots$). Of course, the concept of a plane 
is only meaningful in $\R^3$.

\begin{MInfo}
A plane $E$ in space is given in \MEntry{vector form}{vector form (of a plane)} or  
\MEntry{parametric form}{parametric form (of a plane)} as the set of position vectors
\[
 E=\{\MVec{r}=\MVec{a}+\lambda\MVec{u}+\mu\MVec{v}\MCondSetSep \lambda,\mu\in\R\} \MDFPSpace ,
\]
often written
\[
 E\colon\MVec{r}=\MVec{a}+\lambda\MVec{u}+\mu\MVec{v}\MDFPSpace;\MDFPaSpace \lambda,\mu\in\R \MDFPeriod
\]
Here, $\lambda$ and $\mu$ are called \textbf{parameters}, $\MVec{a}$ is called the 
\MEntry{reference vector}{reference vector (of a plane)}, and $\MVec{u},\MVec{v}\neq\MDVec{O}$ is called 
the \MEntry{direction vector}{direction vector (of a plane)} of the plane. Here, the direction vectors 
$\MVec{u}$ and $\MVec{v}$ are \textit{non-collinear}. The position vectors point to individual 
points in the plane. The reference vector $\MVec{a}$ is the position vector of a fixed point in the plane, called the \MEntry{reference point}{reference point (of a plane)}.

% \begin{center}
% \MTikzAuto{%
% \begin{tikzpicture}[>=stealth]
% 
% % The axes
% \draw[->] (xyz cs:x=-3.5) -- (xyz cs:x=3.5) node[above] {\footnotesize $x$};
% \draw[->] (xyz cs:y=-3.5) -- (xyz cs:y=3.5) node[right] {\footnotesize $y$};
% \draw[->] (xyz cs:z=-3.5) -- (xyz cs:z=3.5) node[left] {\footnotesize $z$};
% % The ticks
% \foreach \coo in {-3,-2,-1,1,2,3}
% {
%   \draw (\coo,-3pt) -- (\coo,3pt) node[below=4pt] {\footnotesize \coo};
%   \draw (-3pt,\coo) -- (3pt,\coo) node[left=4pt] {\footnotesize \coo};
%   \draw (xyz cs:y=-0.1pt,z=\coo) -- (xyz cs:y=0.1pt,z=\coo) node[below=3pt] {\scriptsize \coo};
% }
% \draw[color=blue,->,line width = 0.8pt] (0,0,0) -- (2,1,0);
% \draw[color=violet,->,line width=0.8pt] (2,1,0) -- (2,3,1);
% \draw[color=green,->,line width=0.8pt] (2,1,0) -- (3,2,-1);
% % \draw[color=red] (3,3,-1) node[below] {\scriptsize $g$};
% % \draw[color=blue] (0.25,0.4,0.25) node[left] {\scriptsize $\MVec{a}$};
% % \draw[color=violet] (1,1.5,0.375) node[above] {\scriptsize $\lambda\MVec{u}$};
% % \draw[color=red] (0.875,1,-0.125) node[right] {\scriptsize $\MVec{r}$};
% \end{tikzpicture}
% }
% \end{center}

(This figure will be released shortly.)
\end{MInfo}

Just as two points in space uniquely define a line (see Section~\MNRef{VBKM10_GeradenEbene}), three 
given points in space uniquely define a plane. From these three given points, the parametric form of the
equation of the corresponding plane can be determined rather easily. The vector form of the equation of a given plane 
plane is, as for a line, not unique. An infinite number of equivalent equations in vector form exists to represent 
a given plane. The example below lists a few typical applications. 


\begin{MExample}
\begin{itemize}
 \item The reference vector $\MVec{a}=\MVector{0\\1\\0}$ and the direction vectors  
  $\MVec{u}=\MVector{1\\0\\0}$, $\MVec{v}=\MVector{0\\0\\1}$ define an equation in parametric form
 \[
  E\colon \MVec{r}=\MVec{a}+\lambda\MVec{u}+\mu\MVec{v}=\MVector{0\\1\\0}+\lambda\MVector{1\\0\\0}+\mu\MVector{0\\0\\1}\MDFPSpace;\MDFPaSpace\lambda,\mu\in\R
 \]
  of a plane that lies at an altitude of $1$ parallel to the $x z$-plane in the coordinate system (see figure below).
 
 (This figure will be released shortly.)
 
  The parametric equation of the plane $E$ given above is not the only possible one. Each point in the plane 
  $E$ can be used as a reference point. For example, the point defined by the position vector $\MVec{a}^\prime=\MVector{1\\1\\1}$
  lies in $E$ since for $\lambda=\mu=1$ we have:
 \[
  \MVector{1\\1\\1}=\MVector{0\\1\\0}+1\cdot\MVector{1\\0\\0}+1\cdot\MVector{0\\0\\1}\MDFPeriod
 \]
  Thus, this point can be used as a reference vector. All vectors that are coplanar to $\MVec{u}$ and $\MVec{v}$ 
  but not collinear to each other can be used as alternative direction vectors. Examples are the vectors
  $\MVec{u}^\prime=\MVector{1\\0\\1}=1\cdot\MVector{1\\0\\0}+1\cdot\MVector{0\\0\\1}$ 
  and $\MVec{v}^\prime=\MVector{1\\0\\-1}=1\cdot\MVector{1\\0\\0}-1\cdot\MVector{0\\0\\1}$. Then, another 
  representation of $E$ in parametric form is given by the equation
 \[
  E\colon \MVec{r}=\MVec{a}^\prime+s\MVec{u}^\prime+t\MVec{v}^\prime=\MVector{1\\1\\1}+s\MVector{1\\0\\1}+t\MVector{1\\0\\-1}\MDFPSpace;\MDFPaSpace s,t\in\R \MDFPeriod
 \]
 \item Consider three points $A=\MPointThree{1}{0}{-2}$, $B=\MPointThree{4}{1}{2}$, and $C=\MPointThree{0}{2}{1}$. 
  Find the equation of the plane $F$ that is specified by these three points, in parametric form.
 
  One of these three points, for example the point $A$, is used as the reference point.
  $\MDVec{A}=\MVector{1\\0\\-2}$ is the corresponding reference vector. The connecting vectors from the reference point 
  to the two other points are used as the direction vectors:
 \[
  \MDVec{A B} = \MDVec{B}-\MDVec{A}=\MVector{4\\1\\2}-\MVector{1\\0\\-2}=\MVector{3\\1\\4}\MDFPSpace,
 \]
 \[
  \MDVec{A C} = \MDVec{C}-\MDVec{A}=\MVector{0\\2\\1}-\MVector{1\\0\\-2}=\MVector{-1\\2\\3}\MDFPeriod
 \]
 Hence, the equation 
 \[
  F\colon\MVec{r}=\MVector{1\\0\\-2}+\rho\MVector{3\\1\\4}+\sigma\MVector{-1\\2\\3}\MDFPSpace;\MDFPaSpace\rho,\sigma\in\R
 \]
  is a correct representation of the plane $F$ in parametric form. 
  
 (This figure will be released shortly.)
 
 \item Consider the two points $P=\MPointThree{1}{2}{3}$ and $Q=\MPointThree{2}{6}{6}$. Verify whether they lie 
  in the plane $G$ given by the equation 
 \[
  G\colon\MVec{r}=\MVector{0\\3\\2}+\mu\MVector{1\\2\\3}+\nu\MVector{0\\1\\2}\MDFPSpace;\MDFPaSpace\mu,\nu\in\R
 \]
 in parametric form.

 The points $P$ or $Q$ lie in the plane $G$ if their position vectors arise for specific parameter values of 
 $\mu$ and $\nu$ as position vectors from the equation of $G$, i.e. $\MDVec{P}=\MVec{r}$ or $\MDVec{Q}=\MVec{r}$
 for appropriate values of $\mu$ and $\nu$. For the point $P$, we have:
 \[
  \MDVec{P}=\MVector{1\\2\\3}=\MVector{0\\3\\2}+\mu\MVector{1\\2\\3}+\nu\MVector{0\\1\\2}=\MVector{\mu\\3+2\mu+\nu\\2+3\mu+2\nu}\MDFPeriod
 \]
 From the first component of this vector equation we get $\mu=1$. Substituting this parameter value into 
 the second and third component provides two contradicting equations in the parameter $\nu$:
 \[
  2=3+2\cdot 1+\nu\MDFPSpace\Leftrightarrow\MDFPSpace\nu=-3
 \]
 and
 \[
  3=2+3\cdot 1+2\nu\MDFPSpace\Leftrightarrow\MDFPSpace\nu=-1\MDFPeriod 
 \]
 There are no parameter values $\mu$ and $\nu$ providing in the parametric equation of the plane 
 $G$ the position vector $\MDVec{P}$, so the point $P$ does not lie in the plane $G$.  
 For $Q$, however, we have:
 \[
  \MDVec{Q}=\MVector{2\\6\\6}=\MVector{0\\3\\2}+\mu\MVector{1\\2\\3}+\nu\MVector{0\\1\\2}=\MVector{\mu\\3+2\mu+\nu\\2+3\mu+2\nu}\MDFPeriod
 \]
 From the first component we get $\mu=2$. Substituting this parameter value into the second and third component
 results in
 \[
  6=3+2\cdot 2+\nu\MDFPSpace\Leftrightarrow\MDFPSpace\nu=-1
 \]
 and
 \[
  6=2+3\cdot 2+2\nu\MDFPSpace\Leftrightarrow\MDFPSpace\nu=-1\MDFPeriod
 \]
 This is not a contradiction. We see that the parameter values $\mu=2$ and $\nu=-1$ provide the position 
 vector $\MDVec{Q}$. Hence, the point $Q$ lies in the plane $G$.
 
 (This figure will be released shortly.)
 
\end{itemize}
\end{MExample}

As well as by three points, a plane can also be defined 
by a line and a point that does not lie on the line. The example below shows how this can be 
reduced to the case of three given points.

\begin{MExample}
Let a point $P=\MPointThree{2}{1}{-3}$ be given. In addition, let a line $g$ be given in 
parametric form by the equation
\[
 g\colon \MVec{r}=\MVector{0\\-1\\0}+t\MVector{2\\0\\-1}\MDFPSpace,\MDFPaSpace t\in\R \MDFPeriod
\]
The point $P$ does not lie on the line $g$ since there is no value of the parameter $t\in\R$ such that
\[
 \MDVec{P}=\MVector{2\\1\\-3}=\MVector{0\\-1\\0}+t\MVector{2\\0\\-1}=\MVector{2t\\-1\\-t}\MDFPeriod
\]
The second component of this vector equation results in the contradiction $1=-1$. The point $P$ and the line $g$ uniquely define a plane $E$ that contains both $P$ and $g$. 
A parametric equation of this plane can by found by choosing two additional points on $g$ besides the given point 
$P$ that can be used as a reference point and then proceeding as in the example above for three given 
points. Hence, the reference vector is in this case 
\[
 \MDVec{P}=\MVector{2\\1\\-3} \MDFPSpace,
\]
and the two additional points $Q_1$ and $Q_2$ on $g$ result from the equation of the line for 
two different values of the parameter $t$, for example, $t=0$ and $t=1$. Choosing $t=0$ 
results in the reference point of the line as position vector:
\[
 \MDVec{Q}_1=\MVector{0\\-1\\0}+0\cdot\MVector{2\\0\\-1}=\MVector{0\\-1\\0} \MDFPeriod
\]
Choosing $t=1$ results in
\[
 \MDVec{Q}_2=\MVector{0\\-1\\0}+1\cdot\MVector{2\\0\\-1}=\MVector{2\\-1\\-1} \MDFPeriod
\]
Thus, the direction vectors are 
\[
 \MDVec{P Q}_1=\MDVec{Q}_1 - \MDVec{P} = \MVector{0\\-1\\0} - \MVector{2\\1\\-3} = \MVector{-2\\-2\\3}
\]
and
\[
 \MDVec{P Q}_2=\MDVec{Q}_2 - \MDVec{P} = \MVector{2\\-1\\-1} - \MVector{2\\1\\-3} = \MVector{0\\-2\\2} \MDFPeriod
\]
Hence, the plane $E$ is given by the vector equation
\[
 E\colon\MVec{r}=\MVector{2\\1\\-3}+v\MVector{-2\\-2\\3}+w\MVector{0\\-2\\2}\MDFPSpace;\MDFPaSpace v,w\in\R \MDFPeriod
\]

(This figure will be released shortly.)
\end{MExample}

In the following Section~\MNRef{VBKM10_Lagebeziehung} we will further discuss the relative positions of 
planes and lines, as well as other data that can be used to define a plane uniquely.

\begin{MExercise}
The plane $E$ uniquely defined by the three points $A=\MPointThree{0}{0}{8}$, $B=\MPointThree{3}{-1}{10}$, and 
$C=\MPointThree{-1}{-2}{11}$ has the parametric equation
\[
 E\colon\MVec{r}=\MVector{2\\-3\\x}+s\MVector{y\\1\\-1}+t\MVector{5\\z\\-4}\MDFPSpace;\MDFPaSpace s,t\in\R \MDFPeriod
\]
Find the missing components $x$, $y$, and $z$.\\
$x=$\MLFunctionQuestion{15}{13}{5}{x}{5}{PLANE4}\\
$y=$\MLFunctionQuestion{15}{4}{5}{x}{5}{PLANE5}\\
$z=$\MLFunctionQuestion{15}{3}{5}{x}{5}{PLANE6}\\

\begin{MHint}{Solution}
The reference vector $\MDVec{A}=\MVector{0\\0\\8}$ and the direction vectors 
\[
 \MDVec{A B}=\MDVec{B}-\MDVec{A}=\MVector{3\\-1\\10}-\MVector{0\\0\\8}=\MVector{3\\-1\\2} \MDFPSpace,
\]
\[
 \MDVec{A C}=\MDVec{C}-\MDVec{A}=\MVector{-1\\-2\\11}-\MVector{0\\0\\8}=\MVector{-1\\-2\\3}
\]
define the following parametric equation:
\[
 E\colon\MVec{r}=\MVector{0\\0\\8}+\mu\MVector{3\\-1\\2}+\nu\MVector{-1\\-2\\3}\MDFPSpace;\MDFPaSpace \mu,\nu\in\R\MDFPeriod
\]
The reference point $\MVector{2\\-3\\x}$ lies in the plane $E$ if
\[
 \MVector{2\\-3\\x}=\MVector{0\\0\\8}+\mu\MVector{3\\-1\\2}+\nu\MVector{-1\\-2\\3}=\MVector{3\mu-\nu\\-\mu-2\nu\\8+2\mu+3\nu}\MDFPeriod
\]
This is a system of linear equations in the three variables $\mu$, $\nu$, and $x$ that can be solved using the methods
described in Section~\MNRef{M04_3_Unbekannte}. Considering the first and second components results in the two equations
\[
 2=3\mu-\nu\MDFPaSpace\textrm{and}\MDFPaSpace -3=-\mu-2\nu 
\]
with the solution $\mu=\nu=1$. Substituting these values into the third component results in
\[
 x=8+2+3=13 \MDFPeriod
\]
The two vectors $\MVector{y\\1\\-1}$ and $\MVector{5\\z\\-4}$ are direction vectors of the plane $E$ if they are 
coplanar to both $\MDVec{A B}$ and $\MDVec{A C}$. For the first vector, we have
\[
 \MVector{y\\1\\-1}=a\MVector{3\\-1\\2}+b\MVector{-1\\-2\\3}=\MVector{3a-b\\-a-2b\\2a+3b} \MDFPSpace,
\]
which is again a system of linear equations in three variables. Considering the second and the third 
component results in the two equations
\[
 1=-a-2b\MDFPaSpace\textrm{and}\MDFPaSpace -1=2a+3b 
\]
with the solution $a=1$, $b=-1$. Substituting this values into the first equation results in 
\[
 y= 3+ (-1)\cdot(-1)=4 \MDFPeriod
\]
For the second vector we have
\[
 \MVector{5\\z\\-4}=a\MVector{3\\-1\\2}+b\MVector{-1\\-2\\3}=\MVector{3a-b\\-a-2b\\2a+3b}\MDFPeriod
\]
Considering the first and the third component results in the two equations
\[
 5=3a-b\MDFPaSpace\textrm{and}\MDFPaSpace -4=2a+3b
\]
with the solution $a=1$, $b=-2$ which finally results in
\[
 z=-1-2\cdot(-2)=3\MDFPeriod
\]
\end{MHint}

\end{MExercise}


\begin{MExercise}
Consider the points $P=\MPointThree{h}{2}{-2}$, $Q=\MPointThree{1}{i}{6}$, $R=\MPointThree{-3}{2}{j}$ and the plane $E$ be given by an equation 
\[
 E\colon\MVec{r}=\MVector{3\\0\\2}+s\MVector{2\\1\\7}+t\MVector{3\\2\\5}\MDFPSpace;\MDFPaSpace s,t\in\R
\]
in parametric form. Find the missing components $h$, $i$, and $j$ such that the points $P$, $Q$, and $R$ lie in 
the plane $E$.\\
$h=$\MLFunctionQuestion{15}{5}{5}{x}{5}{PLANE1}\\
$i=$\MLFunctionQuestion{15}{-2}{5}{x}{5}{PLANE2}\\
$j=$\MLFunctionQuestion{15}{-74}{5}{x}{5}{PLANE3}\\

\begin{MHint}{Solution}
For the plane $E$, we have the equation
\[
 \MVec{r}=\MVector{3\\0\\2}+s\MVector{2\\1\\7}+t\MVector{3\\2\\5}=\MVector{3+2s+3t\\s+2t\\2+7s+5t} \MDFPeriod
\]
The conditions 
\[
 \MDVec{P}=\MVec{r}\MDFPSpace\Leftrightarrow\MDFPSpace\MVector{h\\2\\-2}=\MVector{3+2s+3t\\s+2t\\2+7s+5t} \MDFPSpace,
\]
\[
 \MDVec{Q}=\MVec{r}\MDFPSpace\Leftrightarrow\MDFPSpace\MVector{1\\i\\6}=\MVector{3+2s+3t\\s+2t\\2+7s+5t} \MDFPSpace,
\]
and
\[
 \MDVec{R}=\MVec{r}\MDFPSpace\Leftrightarrow\MDFPSpace\MVector{-3\\2\\j}=\MVector{3+2s+3t\\s+2t\\2+7s+5t}
\]
each result in a system of linear equations in the variables $s$, $t$, and $h$ or $i$ or $j$ 
that can be solved using the methods described in Section~\MNRef{M04_3_Unbekannte}.

For the point $P$, considering the second and the third component results in a system of two linear equations 
in the variables $s$ and $t$
\[
 2=s+2t\MDFPaSpace\textrm{and}\MDFPaSpace -4=7s+5t 
\]
with the solution $s=-2$, $t=2$. Substituting the solution into the first component results in
\[
 h=3+2\cdot(-2)+3\cdot 2=5 \MDFPeriod
\]
For the point $Q$, considering the first and the third component results in a system of two linear equations 
in the two variables $s$ and $t$
\[
 -2=2s+3t\MDFPaSpace\textrm{and}\MDFPaSpace 4=7s+5t
\]
with the solution $s=2$, $t=-2$. Substituting the solution into the second component results in
\[
 i=2+2\cdot(-2)=-2 \MDFPeriod
\]
For the point $R$, considering the first and the second component results in a system of two linear equations
in the two variables $s$ and $t$
\[
 -6=2s+3t\MDFPaSpace\textrm{and}\MDFPaSpace 2=s+2t
\]
with the solution $s=-18$, $t=10$. Substituting the solution into the third component results in
\[
 j=2+7\cdot(-18)+5\cdot10=-74 \MDFPeriod
\]
\end{MHint}

\end{MExercise}

\end{MXContent}

\begin{MXContent}{Relative Positions of Lines and Planes in Space}{Relative Positions}{STD}
\MLabel{VBKM10_Lagebeziehung}
\MDeclareSiteUXID{VBKM10_Lagebeziehung}

While two lines in the plane can only have three different relative positions with respect to each other 
(lines are parallel, coincide, or intersect, see Section~\MNRef{VBKM09_Lagebeziehungen}), 
two lines in space can have four different relative positions with respect to each other. These will be 
outlined in the Info Box below.

\begin{MInfo}\MLabel{info:schnitti}

Let two lines in space be given by vector equations. The line $g$ has the reference vector 
$\MVec{a}$ and the direction vector $\MVec{u}$, and the line $h$ has the reference vector $\MVec{b}$
and the direction vector $\MVec{v}$:
\[
 g\colon\MVec{r}=\MVec{a}+s\MVec{u}\MDFPSpace;\MDFPaSpace s\in\R \MDFPSpace,
\]
\[
 h\colon\MVec{r}=\MVec{b}+t\MVec{v}\MDFPSpace;\MDFPaSpace t\in\R\MDFPeriod
\]
The two lines $g$ and $h$ can have four different relative positions:
\begin{enumerate}
 \item The lines are \textbf{identical}. In this case, the lines $g$ and $h$ have all their points in common, 
  they coincide. This is the case if and only if the two direction vectors $\MVec{u}$ and 
  $\MVec{v}$ are collinear and the lines have any one point in common.
 \item The lines are \MEntry{parallel}{parallelism (of two lines in space)}. This is the case if and only if 
  the two direction vectors $\MVec{u}$ and $\MVec{v}$ are collinear and the two lines do \textit{not} have any 
  points in common.
 \item The lines intersect. In this case, the lines $g$ and $h$ have exactly one point in common. 
  This point is called the \MEntry{intersection point}{intersection point (of two lines in space)}. 
  This is the case if and only if the two direction vectors $\MVec{u}$ and $\MVec{v}$ are 
  \textit{not} collinear and the two lines have exactly one point in common.
 \item Lines that are neither identical nor parallel and do not intersect are called 
  \MEntry{skew}{skew}. This is the case if and only if the two direction vectors are \textit{not} 
  collinear, and the two lines do \textit{not} have any points in common.
\end{enumerate}

(This figure will be released shortly.)
\end{MInfo}

In practice, the relative position of two lines in space is investigated 
according as follows: first, we examine two direction vectors 
for collinearity, then we check whether the two lines have points in common.  
This uniquely identifies one of the four cases. The example below illustrates this approach for all 
four cases.

\begin{MExample}
Let the four lines $g$, $h$, $i$, and $j$ be given in parametric form by
\[
 g\colon\MVec{r}=\MVector{-1\\0\\3}+s\MVector{-2\\2\\-4}\MDFPSpace;\MDFPaSpace s\in\R \MDFPSpace
\]
\[
 h\colon\MVec{r}=\MVector{1\\-2\\7}+t\MVector{1\\-1\\2} \MDFPSpace;\MDFPaSpace t\in\R \MDFPSpace
\]
\[
 i\colon\MVec{r}=\MVector{4\\0\\8}+u\MVector{-3\\3\\-6} \MDFPSpace;\MDFPaSpace u\in\R \MDFPSpace
\]

\[
 j\colon\MVec{r}=\MVector{1\\3\\2}+v\MVector{1\\3\\-3} \MDFPSpace;\MDFPaSpace v\in\R \MDFPeriod
\]
\begin{itemize}
\item The lines $g$ and $h$ are identical. The two direction vectors $\MVector{-2\\2\\-4}$ of $g$ and 
  $\MVector{1\\-1\\2}$ of $h$ are collinear. We have
\[
  \MVector{-2\\2\\-4}= -2\cdot\MVector{1\\-1\\2} \MDFPeriod
\]
The point described by the position vector $\MVector{1\\-2\\7}$ lies both on the line $h$ (as reference point)
and on the line $g$ since we have for the line $g$:
\[
 \MVector{1\\-2\\7}=\MVector{-1\\0\\3}+s\MVector{-2\\2\\-4}=\MVector{-1-2s\\2s\\3-4s}\MDFPSpace\Leftrightarrow\MDFPSpace s=-1\MDFPeriod
\]
Thus, the vector $\MVector{1\\-2\\7}$ results from the equation of $g$ for the parameter value $s=-1$.
\item The lines $h$ and $i$ (and hence the lines $g$ and $i$) are parallel. The two direction 
vectors $\MVector{1\\-1\\2}$ of $h$ and $\MVector{-3\\3\\-6}$ of $i$ are collinear.
\[
  \MVector{-3\\3\\-6}= -3\cdot\MVector{1\\-1\\2} \MDFPeriod
\]
However, the lines $h$ and $i$ do not have any points in common:
the reference point of one of the two lines is not a point on the other line. Here, we can check whether the reference vector $\MVector{4\\0\\8}$ of the line $i$
can result as a position vector of the line $h$:
\[
 \MVector{4\\0\\8}=\MVector{1\\-2\\7}+t\MVector{1\\-1\\2}=\MVector{1+t\\-2-t\\7+2t}\MDFPeriod
\]
In this vector equation, $t=3$ results from the first component and $t=-2$ from the second, which is 
a contradiction. Hence, the two lines do not have any points in common.
\item The lines $i$ and $j$ intersect. First we see that for these two lines the two direction vectors 
 $\MVector{-3\\3\\-6}$ and $\MVector{1\\3\\-3}$ are not collinear. There is no number $a\in\R$ such that
\[
 \MVector{-3\\3\\-6}=a\MVector{1\\3\\-3}
\]
since the equality of the first component results in $a=-3$ and equality of the second component 
results in $a=1$, which is a contradiction. However, these two lines have a point in common that can be 
found by equating the position vectors for $i$ and $j$:
\[
 \MVector{4\\0\\8}+u\MVector{-3\\3\\-6}=\MVector{4-3u\\3u\\8-6u}=\MVector{1+v\\3+3v\\2-3v}=\MVector{1\\3\\2}+v\MVector{1\\3\\-3}\MDFPeriod
\]
Equating the first two components results in the two equations 
\[
 3-3u=v\MDFPaSpace\textrm{and}\MDFPaSpace u=1+v
\]
in the variables $u$ and $v$ with the solution $v=0$, $u=1$. Substituting these values into the equation 
for the third component results in 
\[
 8-6\cdot 1=2-3\cdot 0\MDFPSpace\Leftrightarrow\MDFPSpace 2=2 \MDFPeriod
\]
Thus, the vector equation for the position vectors is satisfied for the parameter values $u=1$ and $v=0$.
Hence, the position vector of the intersection point results from substituting the parameter value $u=1$ into the equation 
of the line $i$ or from substituting the parameter value $v=0$ into the equation 
of the line $j$. For the intersection point of the lines we have $\MPointThree{1}{3}{2}$.

\item The lines $g$ and $j$ (and hence the lines $h$ and $j$) are skew. As in the previous case of the 
intersecting lines it can be easily seen that the two direction vectors $\MVector{-2\\2\\-4}$ of $g$ and $\MVector{1\\3\\-3}$ 
of $j$ are not collinear. However, in this case the lines do not have any point in common which again can be found 
by equating the position vectors:
\[
 \MVector{-1\\0\\3}+s\MVector{-2\\2\\-4}=\MVector{-1-2s\\2s\\3-4s}=\MVector{1+v\\3+3v\\2-3v}=\MVector{1\\3\\2}+v\MVector{1\\3\\-3}\MDFPeriod
\]
This vector equation involves a contradiction; there are no pairs of parameter values of $s$ and $v$
such that the equation is satisfied, and hence $g$ and $j$ do not have any points in common. Considering the 
first and the second components results in the two equations 
\[
 -2s=2+v\MDFPaSpace\textrm{and}\MDFPaSpace 2s=3+3v
\]
with the solution  $v=-\frac{5}{4}$, $s=-\frac{3}{8}$. However, substituting this into the equation for 
the third component results in the contradiction
\[
 3-4(-\frac{3}{8})=2-3(-\frac{5}{4})\MDFPSpace\Leftrightarrow\MDFPSpace \frac{9}{2}=\frac{23}{4} \MDFPeriod
\]

\end{itemize}

\end{MExample}

\begin{MExercise}\MLabel{ex:twolines}
Tick the true statements: \\
The two lines given by the equations  
\[
 g\colon\MVec{r}=\MVector{1\\2\\4}+x\MVector{-5\\10\\-15}\MDFPSpace;\MDFPaSpace x\in\R
\]
and
\[
 h\colon\MVec{r}=\MVector{4\\0\\7}+y\MVector{3\\-2\\3} \MDFPSpace;\MDFPaSpace y\in\R
\]
intersect since 

\begin{MQuestionGroup}
\begin{tabular}{lll}
\MLCheckbox{0}{INTER1} & \MBlank & the two direction vectors are collinear,\\
\MLCheckbox{0}{INTER2} & \MBlank & the two direction vectors are not collinear,\\
\MLCheckbox{0}{INTER3} & \MBlank & the two direction vectors are collinear and the lines have a point in common,\\
\MLCheckbox{1}{INTER4} & \MBlank & the two direction vectors are not collinear and the lines have a point in common,\\
\MLCheckbox{0}{INTER5} & \MBlank & the two direction vectors are not collinear and the lines do not have any points in common.\\
\end{tabular}
\end{MQuestionGroup}
\MGroupButton{Check input}

Find the intersection point $S$ of the two lines $g$ and $h$.\\
$S=$\MLFunctionQuestion{15}{(1,2,4)}{5}{x}{5}{INTER0}

The position vector of the intersection point $\MDVec{S}$ results from the lines $g$ and $h$ for the parameter values\\
$x=$\MLFunctionQuestion{10}{0}{5}{x}{5}{INTER0a} and\\
$y=$\MLFunctionQuestion{10}{-1}{5}{x}{5}{INTER0b}.

\begin{MHint}{Solution}
The two direction vectors $\MVector{-5\\10\\-15}$ and $\MVector{3\\-2\\3}$ are not collinear since there is no 
number $a\in\R$ such that the equation
\[
 \MVector{-5\\10\\-15}=a\MVector{3\\-2\\3}
\]
is satisfied. Considering the second component of this vector equation results in $a=-5$, 
considering the first component of this vector equation results in $a=-\frac{5}{3}$; this 
is a contradiction. The two lines have a point in common, namely the intersection point 
$S$ that will be calculated below. According to Info Box~\MNRef{info:schnitti} these conditions and 
only these conditions are sufficient that the two lines intersect. 

The intersection point results from equating the two position vectors:
\[
 \MVector{1\\2\\4}+x\MVector{-5\\10\\-15}=\MVector{1-5x\\2+10x\\4-15x}=\MVector{4+3y\\-2y\\7+3y}=\MVector{4\\0\\7}+y\MVector{3\\-2\\3}\MDFPeriod
\]
Considering the first two components of this vector equation results in the system of two linear equations
\[
 -5x=3+3y\MDFPaSpace\textrm{and}\MDFPaSpace 2+10x=-2y \MDFPSpace,
\]
with the solution $x=0$, $y=-1$. Substituting these values into the equation for the third component results in 
\[
 4-15\cdot 0=7+3\cdot(-1)\MDFPSpace\Leftrightarrow\MDFPSpace 4=4 \MDFPeriod
\]
Thus, the vector equation is satisfied for these parameter values, and the two lines $g$ and $h$ have a point in 
common. Its position vector results from substituting, for example, the parameter value $x=0$ into the equation of 
$g$:
\[
 \MDVec{S}=\MVector{1\\2\\4}+0\cdot\MVector{-5\\10\\-15}=\MVector{1\\2\\4}\MDFPeriod
\]
\end{MHint}
\end{MExercise}


\begin{MExercise}\MLabel{ex:twolines2}
The two lines given by the equations 
\[
 \gamma\colon\MVec{r}=\MVector{-4\\6\\0}+s\MVector{3\\-2\\-2}\MDFPSpace;\MDFPaSpace s\in\R 
\]
and
\[
 \kappa\colon\MVec{r}=\MVector{a\\b\\4}+t\MVector{-\frac{3}{2}\\c\\1}\MDFPSpace;\MDFPaSpace t\in\R 
\]
are parallel. Find the value of $c$ and specify which values the parameters $a$ and $b$ 
must \textbf{not} take simultaneously to ensure that the two lines are parallel.

$a\neq$\MLFunctionQuestion{10}{-10}{5}{x}{5}{PAR1}\\
$b\neq$\MLFunctionQuestion{10}{10}{5}{x}{5}{PAR2}\\
$c=$\MLFunctionQuestion{10}{1}{5}{x}{5}{PAR3}
\end{MExercise}

\begin{MHint}{Solution}
The two lines are parallel if the two direction vectors are collinear. From this condition
\[
 \MVector{-\frac{3}{2}\\c\\1}=s\MVector{3\\-2\\-2}  \MDFPaSpace ,
\]
we find $s=-\frac{1}{2}$ and hence $c=1$. To ensure that the lines are really parallel and not identical 
the reference point of $\kappa$ must not lie on $\gamma$. Thus, the parameters $a$ and $b$ must have values 
such that the vector equation 
\[
 \MVector{a\\b\\4}=\MVector{-4\\6\\0}+s\MVector{3\\-2\\-2}=\MVector{-4+3s\\6-2s\\-2s}
\]
cannot be satisfied for any value of the parameter $s$. Equating the third component immediately results in 
$s=-2$, i.e. the equation is satisfied for this value. Substituting this value into the equation for the 
first and the second component results in $a=-10$ and $b=10$. Hence, the lines are really parallel 
if $a\neq-10$ or $b\neq10$.
\end{MHint}

Consider two lines in space be given that are truly parallel or intersecting. Then 
these two lines uniquely define a plane (see figure below).

(This figure will be released shortly.)

This is always the plane that contains both lines.

The example below shows how to derive the parametric equation of the plane from 
two truly parallel or intersecting lines.


\begin{MExample}
\begin{itemize}
\item The two lines given by the equations 
\[
 g\colon\MVec{r}=\MVector{1\\2\\4}+s\MVector{-5\\10\\-15}\MDFPSpace;\MDFPaSpace s\in\R
\]
and
\[
 h\colon\MVec{r}=\MVector{4\\0\\7}+t\MVector{3\\-2\\3}\MDFPSpace;\MDFPaSpace t\in\R
\]
intersect at the point $S=\MPointThree{1}{2}{4}$ (see Exercise~\MNRef{ex:twolines}). Thus, 
they uniquely define a plane $E$ that contains both $g$ and $h$. For the equation of the plane 
$E$ in parametric form, the position vectors of three appropriate points are used 
resulting from the given equations of the lines $g$ and $h$. The intersection point is a suitable reference point of $E$, with the reference vector
\[
 \MDVec{S}=\MVector{1\\2\\4}.
\]
Then, the position vectors of two points $P$ and $Q$ result, for example, from substituting 
the parameter values $s=1$ and $t=1$ into the equations of the lines in parametric form:
\[
 \MDVec{P}=\MVector{1\\2\\4}+1\cdot\MVector{-5\\10\\-15}=\MVector{-4\\12\\-11} \MDFPSpace,
\]
\[
 \MDVec{Q}=\MVector{4\\0\\7}+1\cdot\MVector{3\\-2\\3}=\MVector{7\\-2\\10} \MDFPeriod
\]
Thus, we have the direction vectors
\[
 \MDVec{S P}=\MDVec{P}-\MDVec{S}= \MVector{-4\\12\\-11}-\MVector{1\\2\\4}=\MVector{-5\\10\\-15}
\]
and
\[
 \MDVec{S Q}=\MDVec{Q}-\MDVec{S}= \MVector{7\\-2\\10}-\MVector{1\\2\\4}=\MVector{6\\-4\\11} \MDFPeriod
\]
Hence, a possible equation of the plane $E$ in parametric form is given by
\[
 E\colon \MVec{r}=\MVector{1\\2\\4}+\mu\MVector{-5\\10\\-15}+\nu\MVector{6\\-4\\11}\MDFPSpace;\MDFPaSpace \mu,\nu\in\R\MDFPeriod
\]

\item The two lines $g$ and $h$ given by the equations 
\[
 g\colon\MVec{r}=\MVector{-4\\6\\0}+s\MVector{3\\-2\\-2}\MDFPSpace;\MDFPaSpace s\in\R 
\]
and
\[
 h\colon\MVec{r}=\MVector{1\\1\\4}+t\MVector{-\frac{3}{2}\\1\\1}\MDFPSpace;\MDFPaSpace t\in\R 
\]
are parallel (see Exercise~\MNRef{ex:twolines2}). This uniquely defines a plane $F$ that contains both 
$g$ and $h$. The parametric equation of the plane $F$ is derived from the position vectors of 
three appropriate points resulting from the equations of the lines $g$ and $h$. The position vectors of 
points on $g$ or $h$ that result, for example, from substituting the parameter values $s=0$, $s=1$, and $t=0$ are suitable.
The first vector
\[
 \MVec{A}=\MVector{-4\\6\\0}+0\cdot\MVector{3\\-2\\-2}=\MVector{-4\\6\\0} 
\]
can be used as a reference vector. From  
\[
 \MVec{B}=\MVector{-4\\6\\0}+1\cdot\MVector{3\\-2\\-2}=\MVector{-1\\4\\-2}
\]
it immediately results that as the first direction vector $\MDVec{A B}$ of the plane $F$ the direction 
vector $\MVector{3\\-2\\-2}$ of $g$ can be used. The second direction vector of the plane
results from the position vector of a point $B$ on $h$ for the parameter value $t=0$, 
i.e. the reference point of $h$:
\[
 \MDVec{B}=\MVector{1\\1\\4} \MDFPeriod
\]
Hence, 
\[
 \MDVec{A B}=\MDVec{B}-\MDVec{A}=\MVector{1\\1\\4}-\MVector{-4\\6\\0}=\MVector{5\\-5\\4}
\]
is the second direction vector of $F$. Thus, a parameter form of the equation of $F$ is 
\[
 F\colon \MVec{r}=\MVector{-4\\6\\0}+\lambda\MVector{3\\-2\\-2}+\mu\MVector{5\\-5\\4}\MDFPSpace;\MDFPaSpace \lambda,\mu\in\R \MDFPeriod
\]

\end{itemize}
 
\end{MExample}

A line and a plane in space can only have three different relative positions with respect to each other, 
as outlined in the Info Box below.

\begin{MInfo}\MLabel{info:plane_line}
Let a line $g$ with reference point $\MVec{a}$ and direction vector $\MVec{u}$ and a plane $E$ with 
the reference vector $\MVec{b}$ and the direction vectors $\MVec{v}$ and $\MVec{w}$ in space 
be given in parametric form by the equations 
\[
 g\colon\MVec{r}=\MVec{a}+\lambda\MVec{u}\MDFPSpace;\MDFPaSpace\lambda\in\R
\]
and
\[
 E\colon\MVec{r}=\MVec{b}+\mu\MVec{v}+\nu\MVec{w}\MDFPSpace;\MDFPaSpace\mu,\nu\in\R \MDFPeriod
\]
Then the line $g$ and the plane $E$ can have three different relative positions:
\begin{enumerate}
 \item The line $g$ lies in the plane $E$. This is the case if and only if the three 
  direction vectors $\MVec{u}$, $\MVec{v}$, and $\MVec{w}$ are coplanar, and the reference 
  point of the line lies in the plane.
 \item The line $g$ is parallel to the plane $E$. This is the case if and only if the three 
  direction vectors $\MVec{u}$, $\MVec{v}$, and $\MVec{w}$ are coplanar and the reference point 
  of the line does \textit{not} lie in the plane.
 \item The lines $g$ and the plane $E$ intersect. This is the case if and only if the three direction 
  vectors $\MVec{u}$, $\MVec{v}$, and $\MVec{w}$ are not coplanar.
\end{enumerate}

(This figure will be released shortly.)

\end{MInfo}

To investigate the relative position of a given line and plane, we first examine the 
three direction vectors for collinearity, then we check whether the reference point of the line 
lies in the plane. This uniquely identifies one of the three possible cases. If the line and 
the plane intersect, we can calculate the intersection point. The example below illustrates 
a few approaches. 

\begin{MExample}
Let the plane $E$ be given by the parametric equation
\[
 E\colon\MVec{r}=\MVector{2\\2\\2}+s\MVector{3\\-1\\0}+t\MVector{0\\0\\2}\MDFPSpace;\MDFPaSpace s,t\in\R \MDFPeriod
\]
\begin{itemize}
 \item A line with the vector $\MVector{3\\-1\\-4}$ as its direction vector either lies in the plane
  or is parallel to the plane $E$, since $\MVector{3\\-1\\-4}$ is coplanar to the two direction vectors of 
  $E$. From the condition 
 \[
  \MVector{3\\-1\\-4}=s\MVector{3\\-1\\0}+t\MVector{0\\0\\2}\MDFPSpace , 
 \]
  we have $s=1$, $t=-2$. Hence, the line
 \[
  g\colon\MVec{r}=\MVector{-1\\3\\0}+x\MVector{3\\-1\\-4}\MDFPSpace;\MDFPaSpace x\in\R
 \]
 lies in the plane $E$, since the reference point $\MPointThree{-1}{3}{0}$ lies in $E$.
 \[
  \MVector{-1\\3\\0}=\MVector{2\\2\\2}+s\MVector{3\\-1\\0}+t\MVector{0\\0\\2}=\MVector{2+3s\\2-s\\2+2t}\MDFPSpace\Leftrightarrow\MDFPSpace s=t=-1\MDFPeriod
 \]
 Hence, the position vector $\MVector{-1\\3\\0}$ of the line results from the parametric equation of the plane for the parameter values 
 $s=t=-1$. In contrast, the line 
 \[
  h\colon\MVec{r}=y\MVector{3\\-1\\-4}\MDFPSpace;\MDFPaSpace y\in\R
 \]
 is parallel to the plane $E$ since $h$ has the origin $\MPointThree{0}{0}{0}$ as its reference point. The origin does not lie in the plane $E$ since there are no parameter values $s$ and $t$ such that the 
 vector equation 
 \[
  \MVector{0\\0\\0}=\MVector{2\\2\\2}+s\MVector{3\\-1\\0}+t\MVector{0\\0\\2}=\MVector{2+3s\\2-s\\2+2t}
 \]
 is satisfied. Considering the first component implies $s=-\frac{2}{3}$, and $s=2$ results from the 
 second component; this is a contradiction.
 
 \item Every line with a direction vector that is not coplanar to the two direction vectors 
  $\MVector{3\\-1\\0}$ and $\MVector{0\\0\\2}$ of $E$ intersect the plane $E$ at exactly one point. 
  An example of such a line is 
 \[
  k\colon\MVec{r}=\MVector{-3\\1\\0}+\mu\MVector{1\\1\\1}\MDFPSpace;\MDFPaSpace\mu\in\R \MDFPeriod
 \]
 The direction vector $\MVector{1\\1\\1}$ is not coplanar to $\MVector{3\\-1\\0}$ and $\MVector{0\\0\\2}$ since 
 the condition 
 \[
  \MVector{1\\1\\1}= a\MVector{3\\-1\\0}+b\MVector{0\\0\\2}=\MVector{3a\\-a\\2b}
 \]
  cannot be satisfied by any choice of $a,b\in\R$. Considering the first component would imply $a=\frac{1}{3}$ 
  and the second would imply $a=-1$; this is a contradiction. From equating the position vectors 
  of the line $k$ and the plane $E$, the intersection point can be calculated:
 \[
  \MVector{-3\\1\\0}+\mu\MVector{1\\1\\1}=\MVector{-3+\mu\\1+\mu\\\mu}=\MVector{2+3s\\2-s\\2+2t}=\MVector{2\\2\\2}+s\MVector{3\\-1\\0}+t\MVector{0\\0\\2}\MDFPeriod
 \]
  If we are only interested in the intersection point, it is sufficient to determine the parameter value 
  of the line for which this vector equation is satisfied. The position vector of the intersection 
  point then results from substituting the determined parameter value into the equation of the line. 
  Considering the first two components of this vector equation results in a system of two linear 
  equations in the variables $\mu$ and $s$:
 \[
  \mu=5+3s\MDFPaSpace\textrm{and}\MDFPaSpace \mu=1-s \MDFPSpace,
 \]
  with the solution $\mu=2$. Thus, the intersection point has the position vector
 \[
  \MVector{-3\\1\\0}+2\MVector{1\\1\\1}=\MVector{-1\\3\\2} \MDFPeriod
 \]
\end{itemize}
\end{MExample}

\begin{MExercise}
Let the plane $E$ be given by the equation
\[
 E\colon\MVec{r}=\MVector{8\\-2\\0}+s\MVector{1\\3\\2}+t\MVector{-1\\1\\-1}\MDFPSpace;\MDFPaSpace s,t\in\R
\]
and the line $g$ by the equation
\[
 g\colon \MVec{r}=\MVector{0\\2\\1}+u\MVector{0\\4\\c}\MDFPSpace;\MDFPaSpace u\in\R 
\]
whose reference point does not lie in the plane $E$.

Find the missing component $c$ such that the line $g$ is parallel to the line $E$.\\
$c=$\MLFunctionQuestion{10}{1}{5}{x}{5}{BLUBB1}

For all other values of $c$, calculate the intersection point $S=\MPointThree{x}{y}{z}$ 
depending on $c$. Specify the three components of $S$ separately.\\
$x=$\MLFunctionQuestion{20}{0}{5}{c}{5}{BLUBB2x}\\
$y=$\MLFunctionQuestion{20}{2+40/(1-c)}{5}{c}{5}{BLUBB2y}\\
$z=$\MLFunctionQuestion{20}{1+10*c/(1-c)}{5}{c}{5}{BLUBB2z}

\begin{MHint}{Solution}
The line $g$ is parallel to the plane $E$ if the direction vector $\MVector{0\\4\\c}$ of the line $g$ is 
coplanar to the two direction vectors $\MVector{1\\3\\2}$ and $\MVector{-1\\1\\-1}$ of the plane $E$. 
The condition 
\[
 \MVector{0\\4\\c}=s\MVector{1\\3\\2}+t\MVector{-1\\1\\-1}
\]
is only satisfied for $c=1$ since considering the first and the second component separately results 
in the system of two linear equations  
\[
 s-t=0\MDFPSpace,\MDFPaSpace 3s+t=4
\]
in the two variables $s$ and $t$ with the solution $s=t=1$. Then, for the third component, we must have 
\[
 c=2\cdot1-1=1 \MDFPeriod
\]
By equating the position vectors of the plane and the line, the intersection point can be calculated:
\[
 \MVector{0\\2\\1}+u\MVector{0\\4\\c}=\MVector{0\\2+4u\\1+c u}=\MVector{8+s-t\\-2+3s+t\\2s-t}=\MVector{8\\-2\\0}+s\MVector{1\\3\\2}+t\MVector{-1\\1\\-1}\MDFPeriod
\]
This vector equation corresponds to a system of three linear equations in the variables $s$, $t$, and $u$ 
with the parameter $c$. It can be solved using the methods described in Section~\MNRef{M04_freier_Parameter}. 
Solving the system, we get
\[
 u=\frac{10}{1-c}\MDFPeriod
\]
Thus, substituting this value of $u$ into the equation of the line $g$ results in 
the position vector of the intersection point $S$:
\[
 \MDVec{S}=\MVector{0\\2\\1}+\frac{10}{1-c}\MVector{0\\4\\c}=\MVector{0\\2+\frac{40}{1-c}\\1+\frac{10c}{1-c}}\MDFPeriod
\]

\end{MHint}


\end{MExercise}

\begin{MExercise}
Let the line $h$ be given by the equation 
\[
 h\colon\MVec{r}=\MVector{3\\2\\1}+\rho\MVector{-8\\9\\1}\MDFPSpace;\MDFPaSpace\rho\in\R\MDFPeriod
\]
Find the following:
\begin{MExerciseItems}
\item{Value of the parameter $\rho$ for which the line $h$ intersects the $x y$-plane:\ $\rho=$\MLFunctionQuestion{10}{-1}{5}{x}{5}{BLAA1}}
\item{Value of the parameter $\rho$ for which the line $h$ intersects the $y z$-plane:\ $\rho=$\MLFunctionQuestion{10}{3/8}{5}{x}{5}{BLAA2}}
\item{Value of the parameter $\rho$ for which the line $h$ intersects the $x z$-plane:\ $\rho=$\MLFunctionQuestion{10}{-2/9}{5}{x}{5}{BLAA3}}
\end{MExerciseItems}

\begin{MHint}{Solution}
\begin{MExerciseItems}
\item{In the $x y$-plane, we have $z=0$, hence the third component of the position vector of the line $h$ must be equal to zero:
\[
 1+\rho=0\MDFPSpace\Leftrightarrow\MDFPSpace\rho=-1\MDFPeriod
\]
} 
\item{In the $y z$-plane, we have $x=0$, hence the first component of the position vector of the line $h$ must be equal to zero:
\[
 3-8\rho=0\MDFPSpace\Leftrightarrow\MDFPSpace\rho=\frac{3}{8}\MDFPeriod
\]
} 
\item{In the $x z$-plane, we have $y=0$, hence the second component of the position vector of the line $h$ must be equal to zero:
\[
 2+9\rho=0\MDFPSpace\Leftrightarrow\MDFPSpace\rho=-\frac{2}{9}\MDFPeriod
\]
} 
\end{MExerciseItems}
 
\end{MHint}

\end{MExercise}

If we consider two planes in space, we find that they can have three different relative positions with 
respect to each other which correspond to the three different relative positions of two lines described in
Section~\MNRef{VBKM09_Lagebeziehungen}.

These three cases are outlined in the Info Box below.

\begin{MInfo}
Let the plane $E_1$ with the reference vector $\MVec{a}_1$ and the two direction vectors 
$\MVec{u}_1$ and $\MVec{v}_1$ and the plane $E_2$ with the reference vector $\MVec{a}_2$ and the 
two direction vectors $\MVec{u}_2$ and $\MVec{v}_2$ be given by the equations
\[
 E_1\colon \MVec{r}=\MVec{a}_1+\mu\MVec{u}_1+\nu\MVec{v}_1\MDFPSpace;\MDFPaSpace\mu,\nu\in\R
\]
\[
 E_2\colon \MVec{r}=\MVec{a}_2+\rho\MVec{u}_2+\sigma\MVec{v}_2\MDFPSpace;\MDFPaSpace\rho,\sigma\in\R\MDFPeriod
\]
The planes $E_1$ and $E_2$ can have three different possible relative positions with respect to each other:
\begin{enumerate}
 \item The planes $E_1$ and $E_2$ are identical if they have all points in common. This is the case if and only if the three direction vectors $\MVec{u}_1$, $\MVec{v}_1$, $\MVec{u}_2$ and the three direction vectors $\MVec{u}_1$, $\MVec{v}_1$, $\MVec{v}_2$ are coplanar and the reference point of $E_1$ lies in $E_2$.  
 \item The planes $E_1$ and $E_2$ are parallel if they do not have any points in common. This is the case if and only if the three direction vectors $\MVec{u}_1$, $\MVec{v}_1$, $\MVec{u}_2$ and the three direction vectors $\MVec{u}_1$, $\MVec{v}_1$, $\MVec{v}_2$ are coplanar and the reference point of $E_1$ does \textit{not} lie in $E_2$. 
 \item The planes $E_1$ and $E_2$ intersect if the points they have in common form a line. This is the case if and only if the three direction vectors $\MVec{u}_1$, $\MVec{v}_1$, $\MVec{u}_2$ \textit{or} the three direction vectors $\MVec{u}_1$, $\MVec{v}_1$, $\MVec{v}_2$ are \textit{not} coplanar. 
\end{enumerate}

(This figure will be released shortly.)

\end{MInfo}

Of course, in the conditions of the three cases outlined in the Info Box above, the planes can be exchanged; 
it can also be checked whether the reference vector of $E_2$ lies in $E_1$; this makes no difference. If the planes intersect, 
the set of intersection points can be determined. Sets of intersection points were already discussed in Section~\MNRef{M04_3_Unbekannte}, where 
the solvability of systems of linear equations in three variables was interpreted geometrically. A sound understanding
of this interpretation is now presumed in this Module and a brief repetition of the material presented in Section~\MNRef{M04_3_Unbekannte} is 
highly recommended. The example below illustrates how the relative position of two planes is determined.

\begin{MExample}
Let the three planes $E$, $F$, and $G$ be given by the equations
\[
 E\colon\MVec{r}=\MVector{0\\2\\-2}+a\MVector{1\\-2\\1}+b\MVector{4\\0\\-2}\MDFPSpace;\MDFPaSpace a,b\in\R\MDFPSpace,
\]
\[
 F\colon\MVec{r}=\MVector{5\\1\\0}+c\MVector{5\\-2\\-1}+d\MVector{-3\\-2\\3}\MDFPSpace;\MDFPaSpace c,d\in\R\MDFPSpace,
\]
and
\[
 G\colon\MVec{r}=\MVector{5\\0\\1}+x\MVector{1\\-2\\0}+y\MVector{0\\0\\3}\MDFPSpace;\MDFPaSpace x,y\in\R \MDFPeriod
\]
\begin{itemize}
 \item The planes $E$ and $F$ are parallel. The directions vectors $\MVector{1\\-2\\1}$ and $\MVector{4\\0\\-2}$ of $E$ 
  and the first direction vector $\MVector{5\\-2\\-1}$ of $F$ are coplanar since the condition 
 \[
  a\MVector{1\\-2\\1}+b\MVector{4\\0\\-2}=\MVector{5\\-2\\-1}
 \]
 is satisfied for $a=b=1$. Likewise, the direction vectors $\MVector{1\\-2\\1}$ and $\MVector{4\\0\\-2}$ of 
 $E$ and the second direction vector $\MVector{-3\\-2\\3}$ of $F$ are coplanar since the condition 
 \[
  a\MVector{1\\-2\\1}+b\MVector{4\\0\\-2}=\MVector{-3\\-2\\3}
 \]
  is satisfied for $a=1$ and $b=-1$. Moreover, the reference point of $F$ does not lie in $E$ since the condition 
 \[
  \MVector{5\\1\\0}=\MVector{0\\2\\-2}+a\MVector{1\\-2\\1}+b\MVector{4\\0\\-2}=\MVector{a+4b\\2-2a\\-2+a-2b}
 \]
  cannot be satisfied for any value of $a$ and $b$. Considering the second component results in $a=\frac{1}{2}$. 
  Substituting this value if $a$ into the equation for the first component results in $b=\frac{9}{8}$. 
  Substituting these two values into the equation for the third component results in the contradiction
  $0=-2+\frac{1}{2}-\frac{9}{4}$. However, choosing another reference point for $F$, for example the same 
  as for $E$, would result in an equation that describes a plane identical to the plane $E$, i.e. 
  another equivalent parametric representation of one and the same plane. For example, the equation
 \[
  F^{\prime}\colon\MVec{r}=\MVector{0\\2\\-2}+\alpha\MVector{5\\-2\\-1}+\beta\MVector{-3\\-2\\3}\MDFPSpace;\MDFPaSpace \alpha,\beta\in\R
 \]
  represents such a plane.
 
 \item The planes $E$ and $G$ intersect. Both direction vectors $\MVector{1\\-2\\0}$ and 
  $\MVector{0\\0\\3}$ of $G$ are not coplanar to the direction vectors $\MVector{1\\-2\\1}$
  and $\MVector{4\\0\\-2}$ of $E$. For the second direction vector of $G$, we find that the condition 
 \[
  \MVector{0\\0\\3}=a\MVector{1\\-2\\1}+b\MVector{4\\0\\-2}
 \]
  cannot be satisfied for any value of $a$ and $b$. Considering the first two components would result in 
  $a=b=0$. Substituting these values into the equation for the third equation would result in a 
  contradiction. The intersection line of the two planes is calculated by equating 
  the position vectors of the two planes. Here, we have
 \[
  \MVector{0\\2\\-2}+a\MVector{1\\-2\\1}+b\MVector{4\\0\\-2}=\MVector{a+4b\\2-2a\\-2+a-2b}=\MVector{5+x\\-2x\\1+3y}=\MVector{5\\0\\1}+x\MVector{1\\-2\\0}+y\MVector{0\\0\\3}\MDFPeriod
 \]
  This vector equation corresponds to a system of three linear equations in four variables 
  $x$, $y$, $a$, and $b$. According to the methods described in Section~\MNRef{M04_freier_Parameter},
  it is solved by taking one variable as a parameter and solving the system for the other variables 
  as functions of this parameter. This left over parameter will be the parameter in the equation of the 
  intersection line in vector form. Which of the variables is taken for the parameter doesn't matter. 
  Here, we use $x$ as parameter. Considering the first two components of the vector equation 
  result in the system of two linear equations 
 \[
  a+4b=5+x\MDFPaSpace\textrm{and}\MDFPaSpace 2-2a=-2x \MDFPSpace,
 \]
  with the solution $a=1+x$, $b=1$. Substituting this solution into the third component results in
 \[
  -2 + (1+x) - 2\cdot 1=1+3y\MDFPSpace\Leftrightarrow\MDFPSpace y=\frac{1}{3} x-\frac{4}{3} \MDFPeriod
 \]
  Now, substituting $y=\frac{1}{3}x-\frac{4}{3}$ or $a=1+x$ and $b=1$ into the equation of the plane $G$ 
  or the plane $E$ results -- for the same parameter value -- in the same parametric representation of the line 
  $h$, namely the intersection line of the two planes. Specifically, substituting the parameters into the 
  equation of $G$ results in:
 \[
  h\colon\MVec{r}=\MVector{5\\0\\1}+x\MVector{1\\-2\\0}+(\Mtfrac{1}{3}x-\Mtfrac{4}{3})\MVector{0\\0\\3}=\MVector{5\\0\\-3}+x\MVector{1\\-2\\1}\MDFPSpace;\MDFPaSpace x\in\R\MDFPeriod
 \]

\end{itemize}

\end{MExample}

\begin{MExercise}
Let the two planes $E$ and $F$ be given by the equations
\[
 E\colon\MVec{r}=\MVector{-1\\3\\0}+a\MVector{2\\-5\\8}+b\MVector{0\\-1\\4}\MDFPSpace;\MDFPaSpace a,b\in\R
\]
and
\[
 F\colon\MVec{r}=\MVector{5\\0\\0}+c\MVector{-2\\3\\x}+d\MVector{2\\y\\12}\MDFPSpace;\MDFPaSpace c,d\in\R \MDFPSpace,
\]
where the reference point of $F$ does not lie in the plane $E$.

Find the values of the missing components $x$ and $y$ of $F$ such that the planes $F$ and $E$ are parallel.\\
$x=$\MLFunctionQuestion{10}{0}{5}{x}{5}{GNATZ1}\\
$y=$\MLFunctionQuestion{10}{-6}{5}{x}{5}{GNATZ2}

\begin{MHint}{Solution}
The two planes are parallel if the two direction vectors of $F$ are each coplanar to the two 
direction vectors of $E$. For the first direction vector of $F$ this results in the condition
\[
 \MVector{-2\\3\\x}=a\MVector{2\\-5\\8}+b\MVector{0\\-1\\4} \MDFPeriod
\]
Considering the first and the second component results in a system of two linear equations with the solution 
$a=-1$, $b=2$. Substituting these values into the equation for the third component implies
\[
 x=-8+2\cdot 4=0\MDFPeriod
\]
For the second direction vector of $F$ this results in the condition 
\[
 \MVector{2\\y\\12}=a\MVector{2\\-5\\8}+b\MVector{0\\-1\\4} \MDFPeriod
\]
Considering the first and the third components results in a system of two linear equations with the solution
$a=b=1$. Substituting these values into the equation for the second component implies
\[
 y=-5-1=-6\MDFPeriod
\]
\end{MHint}


\end{MExercise}

\begin{MExercise}
Let the two planes $E$ and $F$ be given by the equations
\[
 E\colon\MVec{r}=a\MVector{2\\-5\\8}+b\MVector{0\\-1\\4}\MDFPSpace;\MDFPaSpace a,b\in\R
\]
and
\[
 F\colon\MVec{r}=c\MVector{0\\3\\4}+d\MVector{2\\-1\\0}\MDFPSpace;\MDFPaSpace c,d\in\R\MDFPeriod
\]
These planes intersect, and the intersection line is given by the equation
\[
 g\colon\MVec{r}=\xi\MVector{4\\x\\y}\MDFPSpace;\MDFPaSpace \xi\in\R \MDFPeriod 
\]

Find the values of the missing components $x$ and $y$ of the direction vector of the intersection line.\\
$x=$\MLFunctionQuestion{10}{-5}{5}{x}{5}{GNATZ3}\\
$y=$\MLFunctionQuestion{10}{-4}{5}{x}{5}{GNATZ4}

\begin{MHint}{Solution}
Equating the position vectors of the two planes results in 
\[
 a\MVector{2\\-5\\8}+b\MVector{0\\-1\\4}=\MVector{2a\\-5a-b\\8a+4b}=\MVector{2d\\3c-d\\4c}=c\MVector{0\\3\\4}+d\MVector{2\\-1\\0} \MDFPeriod
\]
The solution of the corresponding system of linear equations with the parameter $d$ is $a=d$, $b=-\frac{5}{2}d$, 
$c=-\frac{1}{2}d$. Substituting the condition $c=-\frac{1}{2}d$ into the equation of the plane $F$ results in 
\[
 -\Mtfrac{1}{2}d\MVector{0\\3\\4}+d\MVector{2\\-1\\0}=d\MVector{2\\-\Mtfrac{5}{2}\\-2} \MDFPeriod
\]
Thus, an appropriate direction vector for the intersection lines is $\MVector{2\\-\Mtfrac{5}{2}\\-2}$.
However, the direction vector in the given parametric equation of $g$ has $4$ as its first component,
it is coplanar to this direction vector. Hence, the direction vector of the intersection line 
is $\MVector{4\\-5\\-4}$, i.e. $x=-5$ and $y=-4$.
\end{MHint}


\end{MExercise}

\end{MXContent}


\MSubsection{Final Test}
\MLabel{VBKM10_Abschlusstest}

\begin{MTest}{Final Test Module \arabic{section}}
\MDeclareSiteUXID{VBKM10_Abschlusstest}

\begin{MExercise}
Specify the vectors that have the arrows shown in the figure below as their representatives.


\begin{center}
\MTikzAuto{
\begin{tikzpicture}[>=stealth]
%Koordinatensystem
\draw[->,color=black] (-3.5,0) -- (3.5,0);
\foreach \x in {-3,-2,-1,1,2,3}
\draw[shift={(\x,0)},color=black] (0pt,2pt) -- (0pt,-2pt) node[below] {\footnotesize $\x$};
\draw[->,color=black] (0,-1.5) -- (0,2.5);
\foreach \y in {-1,1,2}
\draw[shift={(0,\y)},color=black] (2pt,0pt) -- (-2pt,0pt) node[left] {\footnotesize $\y$};
\draw[color=black] (-10pt,-8pt) node[right] {\footnotesize $0$};
%Achsenbeschriftung
\draw (3.5,0) node[anchor=north west] {$x$};
\draw (-0.5,2.8) node[anchor=north west] {$y$};
%Hilflinien
\draw[color=gray, dotted] (-3,-1.5) -- (-3,2.5);
\draw[color=gray, dotted] (-2,-1.5) -- (-2,2.5);
\draw[color=gray, dotted] (-1,-1.5) -- (-1,2.5);
\draw[color=gray, dotted] (1,-1.5) -- (1,2.5);
\draw[color=gray, dotted] (2,-1.5) -- (2,2.5);
\draw[color=gray, dotted] (3,-1.5) -- (3,2.5);
\draw[color=gray, dotted] (-3.5,-1) -- (3.5,-1);
\draw[color=gray, dotted] (-3.5,1) -- (3.5,1);
\draw[color=gray, dotted] (-3.5,2) -- (3.5,2);
%Pfeile
\draw[color=red, ->, line width = 1.5pt] (-3,-1) -- (-2,1);
\draw[color=blue, ->, line width = 1.5pt] (-1,0) -- (1,2);
\draw[color=violet, ->, line width = 1.5pt] (0,0) -- (-1,2);
\draw[color=green, ->, line width = 1.5pt] (2,0) -- (3,1);
\draw[color=black, ->, line width = 1.5pt] (2,-1) -- (3,0);
\end{tikzpicture}
} 
\end{center}

%\MUGraphics{vektoren1.png}{width=0.5\linewidth}{Pfeile in der Ebene}{width:400px}

% Beachten Sie die Beschriftung der Achsen (vertikale Achse gehört
% zur ersten Komponente der Vektoren), 
% beispielsweise gehört zum roten Pfeil der Vektor $\MVector{1\\2}$.

\begin{MExerciseItems}
\item{Red vector: \MLFunctionQuestion{15}{(1,2)}{5}{x}{5}{OR20}.} 
\item{Purple vector: \MLFunctionQuestion{15}{(-1,2)}{5}{x}{5}{OR20b}.} 
\item{Blue vector: \MLFunctionQuestion{15}{(2,2)}{5}{x}{5}{OR21}.} 
\item{Green vector: \MLFunctionQuestion{15}{(1,1)}{5}{x}{5}{OR22}.} 
\item{Black vector: \MLFunctionQuestion{15}{(1,1)}{5}{x}{5}{OR23}.}
\end{MExerciseItems}
\MInputHint{Vektoren können in der Form \texttt{(a;b)} eingegeben werden, zum Beispiel \texttt{(8;-9)} für den Vektor $\MVector{8\\-9}$.} 

% \begin{MHint}{Lösung}
% Der rote Pfeil bewegt um eine Einheit nach rechts und um zwei Einheiten nach oben, er wird also durch $\MVector{1\\2}$ beschrieben.
% Alternativ kann man auch Startpunkte der Pfeile von den Endpunkten abziehen, um den Vektor zu bestimmen. Dabei ist dann die Beschriftung
% der Achsen zu beachten:
% $$
% \MVector{-2\\1}-\MVector{-3\\-1} \;=\; \MVector{-2-(-3)\\1-(-1)} \;=\; \MVector{1\\2}\: .
% $$
% Ebenso erhält man $\MVector{-1\\2}$ für den gelben Vektor, $\MVector{2\\2}$ für den blauen und $\MVector{1\\1}$ für den grünen sowie den schwarzen Vektor.
% Dass die Pfeile im Diagramm verschiedene Start- und Endpunkte haben spielt keine Rolle, als Vektoren (Pfeilklassen) sind beide gleich.
% \end{MHint}

\end{MExercise}

\begin{MExercise}
% Modifizierte  COSH-Frage

In still air conditions, a sports aircraft can fly with a velocity of 150 kilometres per hour due south. However, a crosswind blowing from the west with a velocity of 30 kilometres per hour causes the plane to drift. Represent the velocity of the aircraft as the sum of two vectors in the plane, where 
the second component corresponds to the north-south-direction (positive values for north) and 
the first component corresponds to the east-west-direction (positive values for east). Drop the 
unit of measure (kilometres per hour) in your calculation:
\begin{MExerciseItems}
\item{In still air conditions, the velocity is \MLFunctionQuestion{15}{(0,-150)}{5}{x}{5}{OR0}.}
\item{The wind causes an additional velocity of \MLFunctionQuestion{15}{(30,0)}{5}{x}{5}{OR3}.}
\item{The drifting aircraft has in total the velocity vector \MLFunctionQuestion{15}{(30,-150)}{5}{x}{5}{OR1}.}
\item{The length of this vector (absolute value of the velocity) is \MLParsedQuestion{15}{sqrt(30*30+150*150)}{5}{OR2}.\\
\MInputHint{Your answer may contain radical expression.}}
\end{MExerciseItems}
\MInputHint{Vektoren können in der Form \texttt{(a;b)} eingegeben werden, zum Beispiel \texttt{(8;-9)} für den Vektor $\MVector{8\\-9}$.} 
% \begin{MHint}{Lösung}
% Der Wind verursacht eine zusätzliche Geschwindigkeit von $\MVector{30\\0}$, damit ergibt sich die Geschwindigkeitsvektor
% $$
% \MVector{0\\-150}+\MVector{30\\0} \;=\; \MVector{30\\-150}
% $$
% durch komponentenweises Addieren der Vektoren.
% Die Länge dieses Vektors ist
% $$
% \sqrt{30^2+(-150)^2} \;=\; \sqrt{23400}\: .
% $$
% Als Zahlenwert ist das gerundet $152.9706$.
% \end{MHint}
\end{MExercise}

\begin{MExercise}
Let three points $P=\MPointTwo{3}{4}$, $Q=\MPointTwo{1}{0}$, and $R=\MPointTwo{-2}{1}$ 
be given in the plane. Calculate the following vectors:
\begin{MExerciseItems}
\item{\MEquationItem{$\MDVec{P Q}$}{\MLFunctionQuestion{15}{(-2,-4)}{5}{x}{5}{OR4}}.}
\item{\MEquationItem{$\MDVec{Q R}$}{\MLFunctionQuestion{15}{(-3,1)}{5}{x}{5}{OR4a}}.}
\item{\MEquationItem{$\MDVec{R R}$}{\MLFunctionQuestion{15}{(0,0)}{5}{x}{5}{OR4b}}.}
\item{\MEquationItem{$\MDVec{Q P}$}{\MLFunctionQuestion{15}{(2,4)}{5}{x}{5}{OR4c}}.}
\item{\MEquationItem{$\MDVec{R P}$}{\MLFunctionQuestion{15}{(5,3)}{5}{x}{5}{OR4d}}.}
\end{MExerciseItems}
% \begin{MHint}{Lösung}
% Die Richtungsvektoren ergeben sich aus der Differenz zwischen End- und Startkoordinaten der Punkte:
% \begin{eqnarray*}
% \MDVec{P Q} &=& \MVector{1\\0}-\MVector{3\\4} \;=\; \MVector{-2\\-4}\: ,\ \\
% \end{eqnarray*}
% \end{MHint}
\end{MExercise}


\begin{MExercise}
Let three points $P=\MPointThree{1}{2}{3}$, $Q=\MPointThree{3}{0}{0}$, and $R=\MPointThree{-1}{2}{2}$ be given 
in space. Calculate the following vectors:
\begin{MExerciseItems}
\item{\MEquationItem{$\MDVec{P Q}$}{\MLFunctionQuestion{15}{(2,-2,-3)}{5}{x}{5}{FKT1}}.}
\item{\MEquationItem{$\MDVec{R Q}$}{\MLFunctionQuestion{15}{(4,-2,-2)}{5}{x}{5}{FKT2}}.}
\end{MExerciseItems}

Find the position vector $\MDVec{M}$ of the midpoint $M$ of the line segment $\overline{P R}$:\ 
\MEquationItem{$\MDVec{M}$}{\MLFunctionQuestion{15}{(0,2,2.5)}{5}{x}{5}{FKT3}}.
% LOESUNG STIMMT NOCH NICHT XXX !! Scheint jetzt OK. TS
\end{MExercise}

\begin{MExercise}

Find the intersection point $S$ of the two lines given by the equations in vector form
$$
\MVec{r} \;=\; \MVector{1\\1\\2}+\alpha\cdot \MVector{1\\1\\1}\MDFPSpace;\MDFPaSpace\alpha\in\R \MBlank\MBlank \text{and} \MBlank\MBlank
\MVec{r} \;=\; \MVector{0\\1\\-2}+\beta\cdot \MVector{-2\\-4\\4}\MDFPSpace;\MDFPaSpace\beta\in\R \MDFPeriod
$$
\begin{MExerciseItems}
\item{The position vector of the intersection point is $\MDVec{S}=$\MLFunctionQuestion{15}{(-1,-1,0)}{5}{x}{5}{FKT4}.}
\item{It results from the first equation, for the parameter value \MEquationItem{$\alpha$}{\MLParsedQuestion{7}{-2}{3}{PARSEDQUEST10}}.}
\item{It results from the second equation, for the parameter value \MEquationItem{$\beta$}{\MLParsedQuestion{7}{1/2}{3}{PARSEDQUEST11}}.}
\end{MExerciseItems}
\end{MExercise}

\end{MTest}

\newpage
\MPrintIndex

\end{document}
