\ifttm
\MSetSubject{\MINTPhysics}
\MSubject{Miscellaneous}
\MSection{User settings}

\begin{MSectionStart}
\MDeclareSiteUXID{VBKM_MISCENTRY}
On the following pages you can find
\begin{itemize}
\item{the \MSRef{L_CHAPTER}{Chapter overview} for the course,}
\item{your personal \MSRef{L_CDATA}{Course data},} and
\item{your \MSRef{L_CONFIG}{Settings} for the online course.}
\end{itemize}
\end{MSectionStart}

\MSubsection{Chapter overview}

\begin{MXContent}{Chapter overview}{Chapter}{STD}
\MLabel{L_CHAPTER}
\MDeclareSiteUXID{VBKM_MISCCHAPTERS}
\MGlobalChapterTag
You can navigate to individual learning units and their sections at any time to work on them:
\begin{itemize}
\item{Chapter 01: \MSRef{VBKM01}{Elementary arithmetic}}
\item{Chapter 02: \MSRef{VBKM02}{Equations in a single variable}}
\item{Chapter 03: \MSRef{VBKM03}{Inequalities in a single variable}}
\item{Chapter 04: \MSRef{VBKM04}{Systems of linear equations}}
%\item{Modul 05: \MSRef{VBKM05}{Elementare Geometrie}}
%\item{Modul 06: \MSRef{VBKM06}{Elementare Funktionen}}
%\item{Modul 07: \MSRef{VBKM07}{Differentialrechnung}}
%\item{Modul 08: \MSRef{VBKM08}{Integralrechnung}}
%\item{Modul 09: \MSRef{VBKM09}{Orientierung im zweidimensionalen Koordinatensystem}}
%\item{Modul 10: \MSRef{VBKM10}{Grundlagen der anschaulichen Vektorgeometrie}}
\end{itemize}
You are also free to work on tests and exercises in arbitrary order and repeatedly.
Just click the relevant entries in the table of contents in the left border area.
\end{MXContent}

\MSubsection{My course data}

\begin{MXContent}{My course data}{Course data}{STD}
\MLabel{L_CDATA}
\MGlobalDataTag
\MDeclareSiteUXID{VBKM_MISCCOURSEDATA}

You can configure your personal data on your \MSRef{L_CONFIG}{Settings} page.

Your progress on course modules is assessed based on completed exercises and the results of the final tests.
The course is considered completed once you have passed all final tests with at least a 90\% score.

\begin{html}
<div>
<p id="CDATAS">
Cannot access browser data!
</p>
</div>
\end{html}

% \begin{html}
%  <div class="progress">
%   <div class="progress-bar progress-bar-striped active" role="progressbar" aria-valuenow="40" aria-valuemin="0" aria-valuemax="100" style="width:40\%">
%     40\%
%   </div>
% </div>
% \end{html}

\end{MXContent}


\MSubsection{Settings}

\begin{MXContent}{Einstellungen}{Einstellungen}{STD}
\MLabel{L_CONFIG}
\MGlobalConfTag
\MDeclareSiteUXID{VBKM_MISCSETTINGS}

\begin{html}
<div class="xreply">
<p id="LOGINFIELD">
Cannot access browser data!
</p>
<div id="USERNAMEFIELD">
Choose a login name for working on the course:<br />
Login name: <input id="USER_UNAME" type="text" size="18" oninput="usercheck();" onkeyup="usercheck();" onpaste="usercheck();" onpropertychange="usercheck();"></input><img id="checkuserimg" src="images/questionmark.png">
<br />
<div class="usercreatereply"><p id="ulreply_p"> </p></div>
<br /><br />
Optional information:
<table>
  <tr><td align=left>First name:</td><td align=left><input id="USER_VNAME" type="text" size="40"></input></td></tr>
  <tr><td align=left>Last name:</td><td align=left><input id="USER_SNAME" type="text" size="40"></input></td></tr>
  <tr><td align=left>Email:</td><td align=left><input id="USER_EMAIL" type="text" size="40"></input></td></tr>
</table>
</div>

</div>

\end{html}

% <button type="button" id="CREATEBUTTON" style="height:60px;width:250px;background:#D0E0E0" onclick="usercreatelocal_click(1);">Register above user data in the browser</button><br />
% <button type="button" id="RESETBUTTON" style="height:60px;width:250px;background:#FFD0C0" onclick="userreset_click();">Delete all user data</button>


% <button type="button" id="LOGINBUTTON" style="height:60px;width:250px;background:#D0E0E0" onclick="userlogin_click();">Switch user</button><br />


Data is transferred or saved as follows:

\begin{itemize}
\item{Course data is saved in your browser.}
\item{Anonymised user actions are transferred.}
\item{Anonymised solutions are transferred.}
\end{itemize}
\begin{html}
<div class="xreply">
<p id="CHECKIS">
Cannot access browser data!
</p>
</div>
\end{html}



% \ \\ \ \\
% Your locally saved data:
% 
% \begin{html}
% <textarea name="Name_OBJARRAYS" id="OBJARRAYS" rows="40" cols="80" style="background-color:#F0F0F0"></textarea>
% <script>
% var e = document.getElementById("OBJARRAYS");
% e.value = "?";
% e.readOnly = true;
% </script>
% \end{html}
% 
% \ \\ \ \\
% intersiteobj:
% 
% \begin{html}
% <textarea name="Name_OBJOUT" id="OBJOUT" rows="40" cols="80" style="background-color:#F0F0F0"></textarea>
% <script>
% var mys = JSON.stringify(intersiteobj);
% console.log("Decoded intersiteobj");
% var e = document.getElementById("OBJOUT");
% e.value = mys;
% e.readOnly = true;
% </script>
% \end{html}


\end{MXContent}


\MSubsection{Search}

\begin{MXContent}{Search}{Search}{STD}
\MLabel{L_SEARCHSITE}
\MDeclareSiteUXID{VBKM_MISCSEARCH}
\MGlobalSearchTag

Use Ctrl-F to make your browser look for headwords on the list. The following terms are explained in the online course:

\special{html:<!-- msearchtable //-->}

\end{MXContent}

\begin{MXContent}{Favourites}{Favourites}{STD}
\MLabel{L_FAVORITESSITE}
\MGlobalFavoTag
\MDeclareSiteUXID{VBKM_MISCFAVORITES}

\special{html:<div id="FAVORITELISTLONG"></div>}

\end{MXContent}

\fi
