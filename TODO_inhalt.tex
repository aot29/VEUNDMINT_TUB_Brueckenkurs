\documentclass[letterpaper]{article}
\usepackage[latin1]{inputenc}
\usepackage[T1]{fontenc}
\usepackage[english,ngerman]{babel}
\usepackage{amsmath}
\usepackage{amssymb,amsfonts,textcomp}
\usepackage{color}
\usepackage{array}
\usepackage{hhline}
\usepackage{hyperref}
\hypersetup{pdftex, colorlinks=true, linkcolor=blue, citecolor=blue, filecolor=blue, urlcolor=blue, pdftitle=, pdfauthor=, pdfsubject=, pdfkeywords=}
\makeatother
\title{}
\author{Tobias Jesche}
\date{2016-05-13}
\begin{document}
{\LARGE Inhaltliche TODOs} \newline
\begin{itemize}
%Clara
\item Ab Kapitel 5 keine einheitliiche Durchz\"ahlung von Infoboxen mehr. Z.B.: Davor: \textit{Info 2.1.1} jetzt \textit{Gleichungen}
\item in den PDF Versionen stimmt die Numerierung nicht mit der im Br\"uckenkurs online \"uberein
\item Infoboxen haben in der PDF Version (z.B 2. Kapitel) keinen Titel, Mengennotationen schon. Diese sinda auch numeriert (vgl. Kap 2, Z. 313
\item Wurde $\widehat{=}$ in Kapitel 1 eingef\"uhrt? (Wird ohne Erkl\"arung in Kapitel 2 (Bsp 2.1.11), Zeile 205 benutzt ohne Erkl\"arung


%Tobias
\item Anleitung als Link f\'ur die Eingabefelder erstellen (um Funktionsweise zu verstehen, d.h. nicht auf das ? klicken f\"ur hilfe) um Befehle nachzusehen
\item Links zu folgenden Themen verkn\"upfen:
\item Satz von Pythagoras Kap. 1.1
Pi zu griechischen Zeichen Kap 1.1
\end{itemize}

\end{document}