\documentclass[letterpaper]{article}
\usepackage[latin1]{inputenc}
\usepackage[T1]{fontenc}
\usepackage[english,ngerman]{babel}
\usepackage{amsmath}
\usepackage{amssymb,amsfonts,textcomp}
\usepackage{color}
\usepackage{array}
\usepackage{hhline}
\usepackage{hyperref}
\hypersetup{pdftex, colorlinks=true, linkcolor=blue, citecolor=blue, filecolor=blue, urlcolor=blue, pdftitle=, pdfauthor=, pdfsubject=, pdfkeywords=}
\makeatother
\title{}
\author{Tobias Jesche}
\date{2016-05-13}
\begin{document}
{\LARGE Inhaltliche TODOs} \newline
\begin{itemize}
%Clara
\item Ab Kapitel 5 keine einheitliche Durchz\"ahlung von Infoboxen mehr. Z.B.: Davor: \textit{Info 2.1.1} jetzt \textit{Gleichungen}
\item in den PDF Versionen stimmt die Numerierung nicht mit der im Br\"uckenkurs online \"uberein
\item Infoboxen haben in der PDF Version (z.B 2. Kapitel) keinen Titel, Mengennotationen schon. Diese sinda auch numeriert (vgl. Kap 2, Z. 313
\item Wurde $\widehat{=}$ in Kapitel 1 eingef\"uhrt? (Wird ohne Erkl\"arung in Kapitel 2 (Bsp 2.1.11), Zeile 205 benutzt ohne Erkl\"arung
\item Einfachere Erkl\"arung Dreisatz! (Kapitel 2, Bsp 3.1.11
\item[bsp 2.1.21] z 468 kap 2, Umforung istn mit Mitternachtsformel viel einfacher nachzuvollziehen
\item Wann werden binomische Formeln eingef\"uhrt? Kann man die voraussetzen?
\item kap 2, info 2.1.25 (Z.520)  Formel ausschreiben
\item[Aufgabe 2.1.27] Z.523, eine Erkl\"arung, wie man eine Potenz (also hier $x^2$ als x$\wedge$2 und eine Wurzel $\sqrt{ }$ als sqrt() im Aufgabenfeld schreibt, waere hilfreich.
\item Kapitel 2.2.2 geht der Ton im Video? Oder liegt das an den Arbeitsrechnern hier?
\item Aufgabe 2.2.3 Nur die komplett ausgeklammerte und vereinfachte Umformung der L\"osung ist zul\"assig. Das macht es schwerer, eine korrekte L\"osung einzugeben
\item zu aufgabe 2.1.27, Erkl\"aren dass man $\geq$ als $>=$ schreibt, w\"are sinnvoll. Wurde das davor eingef\"uhrt????
\item Aufgabe 2.2.19: Die angegebenen Shortcuts bzw. Tastaturpositionen bei deutschen Tastaturen (AltGr+7 bzw AltGr+0) fuer Mengenklammern $\{$ und $\}$ funktioniert auf englischen Tastaturen nicht.Das waere zu beachten, wenn das Ganze ins Englische transferiert wird., Zeile 7.2.4
\item eine korrekte Eingabe der Loesungsmenge ist unmoeglich, da anscheinend nur eine expizite Form der Umformung akzeptiert wird, die ich nicht finden kann. Die Loesung, wie sie in der Loesung vorgeschlagen wird, wird in dieser Form nicht akzeptiert.
\item Zeile 731, L\"osung zu Aufgabe 2.2.19: Die Loesungsmenge $L=\{\frac{3}{2}-\sqrt{\frac{45}{4}},\frac{3}{2}+\sqrt{\frac{45}{4}}\}$ kann man einfacher als $L=\frac{3}{2}(1+\sqrt{5}),\frac{3}{2}(1-\sqrt{5})$ darstellen.
\item Aufgabe 2.3.1: Eine Erkl\"arung vor der Auswertung, $\vert x \vert$ durch abs(x) darzustellen, ist n\"otig.
\item[Kapitel1, englisch] in 111, Zeile 98: \dots can be derived as solutions of equations - Is there a number which cannot be solution of an equation??
\item 1.1.2 The use of variables, terms and equations is required to formalise expresions \textit{with still indetreminate values} (Z. 301)



%Tobias
\item Anleitung als Link f\'ur die Eingabefelder erstellen (um Funktionsweise zu verstehen, d.h. nicht auf das ? klicken f\"ur Hilfe) um Befehle nachzusehen, Quadrat u.a.
\item Links zu folgenden Themen verkn\"upfen (?mit \MEntry{kleinste gemeinsame Vielfache}{kgV}):
\subitem Satz von Pythagoras Kap. 1.1
\subitem Pi zu griechischen Zeichen Kap 1.1
\subitem in Kap 1.3.1 Verweis auf Einf�hrung der Terme
\subitem 1.3.4 1.3.19 verlinke Nullstellen, fur jmd der das nicht weis
\item Es gibt drei �berschriften, mit einem �hnlichen Titel. Auch die Inhalte �berschneiden sich zum Teil. Ist das so gewollt?
\subitem 1.1.3 Terme umformen
\subitem 1.3 Umformen von Termen
\subitem 1.3.2 Termumformung
\subitem Warum wurden \textit{Terme schon vorher eingef�hrt} und jetzt wiederholt (vgl. 1.1.3) ... Inhalte doppelt
\item online wird jede Element von 1 bis zum Ende durchnummeriert, egal ob Beispiel, Aufgabe, Infokasten. Im Dokument jedoch hat jedes seine eigene Nummer
\item 1.2.1 ff. Unbestimmte ist ein gebr�uchliches Wort f�r Variable?
\item 1.2.1 Beispiel 1.2.1 dort wird schon von der Kenntnis der Addition bei Br�chen ausgegangen, jedoch erst sp�ter eingef�hrt, Didaktik oder?
\item Teste die Aufgaben zum K�rzen alle durch Kapitel 1.2.1 mit Aufgabe 1.2.3 \& 1.2.4
\item Aufgabe 1.2.9 l�st nicht beim 1. Teil
\item Beispiel zu ggT machen (Didaktik), davor zum kgV-Beispiel Primfaktorzerlegung f�r kgv erkl�ren (? Didaktik)
\subitem Dazu die Aufgabe Kap 1.2.3 1.2.20 erste anpassen (bzgl. ggT ausschreiben)
\item Bewusst keine �bungsaufgabe zum multiplizieren und dividieren mit Br�chen?
\item Didaktik-Frage zur Periode:
\subitem $k$ und $n$ n�her \textit{spezifizieren} (nat�rliche Zahlen, spezifischer geht es leider nicht ... Vorschlag: $x=\frac{x \cdot 10^k - x \cdot 10^p}{10^k-10^p}$ mit $k$ Anzahl der Nachkommastellen inklusive der ersten periodischen Ziffern und $p$ als Anzahl der Ziffern nach dem Komma, die nicht periodisch sind)
\subitem Beispiel 1.2.213 und 1.2.214 sollten auch \textit{in Worten erkl�rt} werden (Das Verfahren wird sonst nicht wirklich deutlich.) \newline
Vorschlag: Zun�chst multipliziert man die periodische Zahl mit einer Potenz von 10, sodass alle Ziffern der Periode einmalig auch mit vor dem Komma stehen. Anschlie�end stellt man die zweite Gleichung auf, die von der ersten Subtrahiert wird: (d�nnes Eis, Gleichungsoperationen folgen erst sp�ter VERBESSUNGSWUERDIG da auch die Beispiele es damit machen) \newline
\textit{a} im Falle es steht eine rein periodische Zahlenfolge hinter dem Komma der Ausgangszahl $x$: Es wird die Gleichung mit einmal der Ausgangszahl $x$ aufgestellt.   \newline
\textit{b} im Falle die Ziffernfolge hinter dem Komma der Ausgangszahl $x$ ist nicht rein periodisch: Es wird die Ausgangszahl $x$ multipliziert mit einer Potenz von 10 die der L�nge des nicht periodischen Teils hinter dem Komma der Ausgangszahl entspricht. (GEHT BESSER FORMULIERT)
\subitem so auch die Aufgabe bearbeiten und anpassen, bisher ist die so unverst�ndlich
\subitem auch die Aufageb 1.2.3 1.2.21 letzter Teil anpassen
\subitem auch Aufgabe 1.2.3 1.2.22 komplett
\item Aufgabe 1.2.4 (Vergleich der Brueche) ist online korrekt, im pdf nicht...
\item Aufgabe 1.2.3 1.2.21 ist in der ersten Teilaufgabe nicht GGT sonder KGV aber lasst sich im Quelltext nicht korrigieren (weil der befehl nicht existiert)
\item Kap 1.3.2 Bsp 1.3.4 Satz und verlinken bezueglich Nullstellen
\item check ob "Dies ist sehr hilfreich, wenn Zahlenquadrate ohne Taschenrechner auszurechnen" nach Merge weiter eins ist
\item Kap 1.3.3 Aufgabe 1.3.10 Hinweis einf�gen? (z.B. finde die selben Terme in Z�hler und Nenner)
\item Kap 1.3.4 Info 1.3.16 Diese Technik wurde zuvor schon gefordert! Also \textit{Verweis} von dort nach Hier (z.B. Kapitel 1.1.3)
\item 1.3.4 1.3.9 am ende eine alte Loesung auskommentiert, kann geloscht werden!
\item 1.3.10 Das Rechnen mit Funktionen (als Art Variable) sollte wenigsten in einer Aufgabe zuvor ge�bt/erl�utert
werden, k�nnte �berfordernd wirken.
\item  SINN) Aufgabe 1.3.20 ist so alleine Fehl am Platz, besser mit 1.3.18 vereinen
\item kap 1.4.2 link in Aufgabe 1.4.4 ist falsch ( \MRef{VBKM01_Bsp_Anordnung}) es wird die falsche Nummer angezeigt
\item onlline gbt es kein popUp bei der aufgabe
\item CLARA warum potenz von 3, man rechnet am edne eh alles aus und so komplexer???? oder nutr wegen 1/3?? (bei ((?3) ?2 ) 3 fehlt als Zwischenschritt = 3 ?6)
\end{itemize}

\end{document}