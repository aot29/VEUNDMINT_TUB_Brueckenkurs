% Makros zur Kompatibilitaet mit Onlinemodul: 
 \providecommand{\MoIl}{(} 
 \providecommand{\MoIr}{)}
 \providecommand{\MIntvlSep}{;} 
 \providecommand{\MElSetSep}{;} 
 \begin{MAufgabe}{Kurvendiskussion}{kr, MaTeX}
 F\"uhren Sie f\"ur die Funktion $f(x)=15\, x^3 + 71\, x^2 + 105\, x - 5$ eine vollst\"andige Kurvendiskussion durch.\\ 
 \ifLsg\Loesung
 \begin{enumerate}
 \item \emph{Definitionsbereich:} 
 Der maximale Definitionsbereich ist $\R$\item \emph{Symmetrie:} 
 Keine Symmetrie bez\"uglich y-Achse oder Koordinatenursprung.\item \emph{Asymptotisches Verhalten:} 
 Grenzwerte f\"ur $x\rightarrow \pm \infty$: \\ 
 $\lim_{x\rightarrow \infty} f(x)=\infty$ \\ 
 $\lim_{x\rightarrow -\infty} f(x)=- \infty$ \\ 
 \item \emph{Periodizit\"at:} 
 Die Funktion $f$ ist als rationale (nicht konstante) Funktion nicht periodisch.\item \emph{Ableitungen:} 
 Als rationale Funktion ist $f$ auf ihrem Definitionsbereich unendlich oft differenzierbar. 
 Die ersten 3 Ableitungen von $f$ lauten: \\ 
 $f^{(1)}(x)=45\, x^2 + 142\, x + 105$\newline 
  $f^{(2)}(x)=90\, x + 142$\newline 
  $f^{(3)}(x)=90$\newline 
  \item \emph{Extremstellen:} 
 Eine Notwendige Bedingung f"ur Extremstellen von $f$ ist $f^{(1)}(x)=0$. 
 Das ist hier \"aquivalent zu $45\, x^2 + 142\, x + 105=0$. 
 Die Kandidaten f\"ur Extremstellen sind die L\"osungen dieser Gleichung innerhalb des Definitionsbereichs von $f$: $ - \frac{2\, \sqrt{79}}{45} - \frac{71}{45}$; $\frac{2\, \sqrt{79}}{45} - \frac{71}{45}$; \\ 
 $f^{(2)}( - \frac{2\, \sqrt{79}}{45} - \frac{71}{45})=- 4\, \sqrt{79}$$<0$, Maximum bei $( - \frac{2\, \sqrt{79}}{45} - \frac{71}{45};71\, {\left(\frac{2\, \sqrt{79}}{45} + \frac{71}{45}\right)}^2 - \frac{14\, \sqrt{79}}{3} - 15\, {\left(\frac{2\, \sqrt{79}}{45} + \frac{71}{45}\right)}^3 - \frac{512}{3})$; \\ 
 $f^{(2)}(\frac{2\, \sqrt{79}}{45} - \frac{71}{45})=4\, \sqrt{79}$$>0$, Minimum bei $(\frac{2\, \sqrt{79}}{45} - \frac{71}{45};\frac{14\, \sqrt{79}}{3} + 71\, {\left(\frac{2\, \sqrt{79}}{45} - \frac{71}{45}\right)}^2 + 15\, {\left(\frac{2\, \sqrt{79}}{45} - \frac{71}{45}\right)}^3 - \frac{512}{3})$; \\ 
 \item \emph{Monotonieverhalten:} 
 Bei einer stetigen ersten Ableitung ist allgemein das Vorzeichen der ersten Ableitung auf Intervallen, die durch die Extremstellen und die Definitionsl\"ucken gegeben sind zu betrachten. Somit ist $f$ auf \\ 
 $\MoIl-\infty\MIntvlSep - \frac{2\, \sqrt{79}}{45} - \frac{71}{45}\MoIr$ monoton wachsend, \\ 
 $\MoIl - \frac{2\, \sqrt{79}}{45} - \frac{71}{45}\MIntvlSep\frac{2\, \sqrt{79}}{45} - \frac{71}{45}\MoIr$ monoton  fallend, \\ 
 $\MoIl\frac{2\, \sqrt{79}}{45} - \frac{71}{45}\MIntvlSep \infty\MoIr$ monoton wachsend. \\ 
 \item \emph{Wendestellen:} 
 Eine Notwendige Bedingung f"ur Wendestellen von $f$ ist $f^{(2)}(x)=0$. 
 Das ist hier \"aquivalent zu $90\, x + 142=0$. 
 Die Kandidaten f\"ur Wendestellen sind die L\"osung dieser Gleichung innerhalb des Definitionsbereichs von $f$: $- \frac{71}{45}$; \\ 
 Wendestelle bei $(- \frac{71}{45}\MIntvlSep- \frac{320978}{6075})$, weil die zweite Ableitung das Vorzeichen von - nach + wechselt. \\ 
 \item \emph{Kr\"ummungsverhalten:} 
 Bei einer stetigen zweiten Ableitung ist allgemein das Vorzeichen der zweiten Ableitung auf Intervallen, die durch die Nullstellen der zweiten Ableitung und die Definitionsl\"ucken gegeben sind zu betrachten. 
 Somit ist $f$ auf \\ 
 $\MoIl-\infty \MIntvlSep- \frac{71}{45}\MoIr$  konkav ($f^{(2)}<0$), \\ 
 $\MoIl- \frac{71}{45}\MIntvlSep \infty\MoIr$  konvex ($f^{(2)}>0$). \\ 
 \item \emph{Skizze des Graphen:} \\ 
 {\textcolor{red} x}: Maxima; {\textcolor{black} x}: Minima; {\textcolor{green} o}: Wendestellen; 
  \begin{center}
  \includegraphics[width=0.8\linewidth]{Abb_zur_Ag_autogenerated_fractions_31.png} \end{center}
  
 \end{enumerate}
 \else\relax\fi
  \end{MAufgabe}