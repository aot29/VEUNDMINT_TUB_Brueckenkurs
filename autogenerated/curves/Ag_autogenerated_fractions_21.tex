% Makros zur Kompatibilitaet mit Onlinemodul: 
 \providecommand{\MoIl}{(} 
 \providecommand{\MoIr}{)}
 \providecommand{\MIntvlSep}{;} 
 \providecommand{\MElSetSep}{;} 
 \begin{MAufgabe}{Kurvendiskussion}{kr, MaTeX}
 F\"uhren Sie f\"ur die Funktion $f(x)= - x^3 - 50\, x^2 - 48\, x - 9$ eine vollst\"andige Kurvendiskussion durch.\\ 
 \ifLsg\Loesung
 \begin{enumerate}
 \item \emph{Definitionsbereich:} 
 Der maximale Definitionsbereich ist $\R$\item \emph{Symmetrie:} 
 Keine Symmetrie bez\"uglich y-Achse oder Koordinatenursprung.\item \emph{Asymptotisches Verhalten:} 
 Grenzwerte f\"ur $x\rightarrow \pm \infty$: \\ 
 $\lim_{x\rightarrow \infty} f(x)=- \infty$ \\ 
 $\lim_{x\rightarrow -\infty} f(x)=\infty$ \\ 
 \item \emph{Periodizit\"at:} 
 Die Funktion $f$ ist als rationale (nicht konstante) Funktion nicht periodisch.\item \emph{Ableitungen:} 
 Als rationale Funktion ist $f$ auf ihrem Definitionsbereich unendlich oft differenzierbar. 
 Die ersten 3 Ableitungen von $f$ lauten: \\ 
 $f^{(1)}(x)= - 3\, x^2 - 100\, x - 48$\newline 
  $f^{(2)}(x)= - 6\, x - 100$\newline 
  $f^{(3)}(x)=-6$\newline 
  \item \emph{Extremstellen:} 
 Eine Notwendige Bedingung f"ur Extremstellen von $f$ ist $f^{(1)}(x)=0$. 
 Das ist hier \"aquivalent zu $ - 3\, x^2 - 100\, x - 48=0$. 
 Die Kandidaten f\"ur Extremstellen sind die L\"osungen dieser Gleichung innerhalb des Definitionsbereichs von $f$: $ - \frac{2\, \sqrt{589}}{3} - \frac{50}{3}$; $\frac{2\, \sqrt{589}}{3} - \frac{50}{3}$; \\ 
 $f^{(2)}( - \frac{2\, \sqrt{589}}{3} - \frac{50}{3})=4\, \sqrt{589}$$>0$, Minimum bei $( - \frac{2\, \sqrt{589}}{3} - \frac{50}{3};32\, \sqrt{589} - 50\, {\left(\frac{2\, \sqrt{589}}{3} + \frac{50}{3}\right)}^2 + {\left(\frac{2\, \sqrt{589}}{3} + \frac{50}{3}\right)}^3 + 791)$; \\ 
 $f^{(2)}(\frac{2\, \sqrt{589}}{3} - \frac{50}{3})=- 4\, \sqrt{589}$$<0$, Maximum bei $(\frac{2\, \sqrt{589}}{3} - \frac{50}{3};791 - 50\, {\left(\frac{2\, \sqrt{589}}{3} - \frac{50}{3}\right)}^2 - {\left(\frac{2\, \sqrt{589}}{3} - \frac{50}{3}\right)}^3 - 32\, \sqrt{589})$; \\ 
 \item \emph{Monotonieverhalten:} 
 Bei einer stetigen ersten Ableitung ist allgemein das Vorzeichen der ersten Ableitung auf Intervallen, die durch die Extremstellen und die Definitionsl\"ucken gegeben sind zu betrachten. Somit ist $f$ auf \\ 
 $\MoIl-\infty\MIntvlSep - \frac{2\, \sqrt{589}}{3} - \frac{50}{3}\MoIr$ monoton fallend, \\ 
 $\MoIl - \frac{2\, \sqrt{589}}{3} - \frac{50}{3}\MIntvlSep\frac{2\, \sqrt{589}}{3} - \frac{50}{3}\MoIr$ monoton  wachsend, \\ 
 $\MoIl\frac{2\, \sqrt{589}}{3} - \frac{50}{3}\MIntvlSep \infty\MoIr$ monoton fallend. \\ 
 \item \emph{Wendestellen:} 
 Eine Notwendige Bedingung f"ur Wendestellen von $f$ ist $f^{(2)}(x)=0$. 
 Das ist hier \"aquivalent zu $ - 6\, x - 100=0$. 
 Die Kandidaten f\"ur Wendestellen sind die L\"osung dieser Gleichung innerhalb des Definitionsbereichs von $f$: $- \frac{50}{3}$; \\ 
 Wendestelle bei $(- \frac{50}{3}\MIntvlSep- \frac{228643}{27})$, weil die zweite Ableitung das Vorzeichen von + nach - wechselt. \\ 
 \item \emph{Kr\"ummungsverhalten:} 
 Bei einer stetigen zweiten Ableitung ist allgemein das Vorzeichen der zweiten Ableitung auf Intervallen, die durch die Nullstellen der zweiten Ableitung und die Definitionsl\"ucken gegeben sind zu betrachten. 
 Somit ist $f$ auf \\ 
 $\MoIl-\infty \MIntvlSep- \frac{50}{3}\MoIr$  konvex ($f^{(2)}>0$), \\ 
 $\MoIl- \frac{50}{3}\MIntvlSep \infty\MoIr$  konkav ($f^{(2)}>0$). \\ 
 \item \emph{Skizze des Graphen:} \\ 
 {\textcolor{red} x}: Maxima; {\textcolor{black} x}: Minima; {\textcolor{green} o}: Wendestellen; 
  \begin{center}
  \includegraphics[width=0.8\linewidth]{Abb_zur_Ag_autogenerated_fractions_21.png} \end{center}
  
 \end{enumerate}
 \else\relax\fi
  \end{MAufgabe}