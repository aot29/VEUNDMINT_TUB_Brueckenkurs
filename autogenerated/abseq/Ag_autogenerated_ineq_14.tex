% Makros zur Kompatibilitaet mit Onlinemodul: 
 \providecommand{\MoIl}[1][]{\mbox{}#1]\mathopen{}} 
 \providecommand{\MoIr}[1][]{#1[\mbox{}} 
 \providecommand{\MIntvlSep}{;} 
 \providecommand{\MElSetSep}{\, ; \, } 
 \begin{MAufgabe}{Lineare Betrags(un)gleichungen}{vr, 2016, MaTeX}
L\"osen Sie die Gleichung
$$
 \MDS 2\left| 8\, x + 1 \right|2 - 3\, x= 3 \left| 4 - 4\, x \right| +x + 3
$$  

\ifLsg\MLoesung

Im ersten Schritt k\"onnen die Terme au\ss{}erhalb der Betragszeichen zusammengefasst werden:

\begin{align*} 
 2\left| 8\, x + 1 \right|2 - 3\, x= 3 \left| 4 - 4\, x \right| +x + 3\\ 
\Leftrightarrow2\, \left|8\, x + 1\right| - 3\, \left|4\, x - 4\right| - 4\, x - 1= 0 
 \end{align*}

F\"ur diese Gleichung haben wir 4 F\"alle zu unterscheiden: 
\begin{enumerate}
\item $ \MDS 
\begin{cases} 
 0 \leq 8\, x + 1\\ 
0 \leq 4 - 4\, x
 \end{cases}
\Leftrightarrow x \leq 1 \wedge - \frac{1}{8} \leq x\Leftrightarrow x \in [ - \frac{1}{8} \, \MIntvlSep \, 1]$ 
\item $ \MDS 
\begin{cases} 
 0 \leq 8\, x + 1\\ 
4 - 4\, x < 0
 \end{cases}
\Leftrightarrow 1 < x\Leftrightarrow x \in \MoIl  1 \, \MIntvlSep \, \infty\MoIr $ 
\item $ \MDS 
\begin{cases} 
 8\, x + 1 < 0\\ 
0 \leq 4 - 4\, x
 \end{cases}
\Leftrightarrow x < - \frac{1}{8}\Leftrightarrow x \in \MoIl  -\infty \, \MIntvlSep \, - \frac{1}{8}\MoIr $ 
\item $ \MDS 
\begin{cases} 
 8\, x + 1 < 0\\ 
4 - 4\, x < 0
 \end{cases}
 \mbox{ : keine L\"osung. Diese Bedingung ist nirgendwo erf\"ullt.}$ 
\end{enumerate} 
Der 4. Fall ist nirgendwo erf\"ullt. Betrachte weiter nur die restlichen F\"alle.
 
 Fallunterscheidung: 

 \begin{enumerate} 
 \item Sei $ \MDS x\in[ - \frac{1}{8} \, \MIntvlSep \, 1]$. 
 In diesem Fall gilt: 
  $ \MDS \left| 8\, x + 1\right|=8\, x + 1$ und $ \MDS \left| 4 - 4\, x\right|=4 - 4\, x$. \\ 
 Damit ist die Gleichung 
 $$ 
2\, \left|8\, x + 1\right| - 3\, \left|4\, x - 4\right| - 4\, x - 1= 0
$$
 \"aquivalent zur Gleichung
 $$ 
2\left(8\, x + 1\right)-3\left( 4 - 4\, x\right)- 4\, x-1= 0 
$$  
$$ 
 \Leftrightarrow 24\, x - 11= 0 
$$  
$$ \Leftrightarrow x = \frac{11}{24} . 
 $$ 
 Die L\"osung muss auch die Fallbedingung $x\in [ - \frac{1}{8} \, \MIntvlSep \, 1] $ erf\"ullen. Die gefundene L\"osung $x=\frac{11}{24}$ erf\"ullt die Fallbedingung  $x\in [ - \frac{1}{8} \, \MIntvlSep \, 1]$ und deshalb ist  $$
 \mathcal{L}_{1}=\left\{\frac{11}{24}\right\}
 $$ 
\item Sei $ \MDS x\in\MoIl  1 \, \MIntvlSep \, \infty\MoIr $. 
 In diesem Fall gilt: 
  $ \MDS \left| 8\, x + 1\right|=8\, x + 1$ und $ \MDS \left| 4 - 4\, x\right|=4\, x - 4$. \\ 
 Damit ist die Gleichung 
 $$ 
2\, \left|8\, x + 1\right| - 3\, \left|4\, x - 4\right| - 4\, x - 1= 0
$$
 \"aquivalent zur Gleichung
 $$ 
2\left(8\, x + 1\right)-3\left( 4\, x - 4\right)- 4\, x-1= 0 
$$  
$$ 
 \Leftrightarrow 13= 0 
$$  
$$ \Leftrightarrow x = \left(\begin{array}{c} \end{array}\right) . 
 $$ 
 Die L\"osung muss auch die Fallbedingung $x\in \MoIl  1 \, \MIntvlSep \, \infty\MoIr  $ erf\"ullen. 