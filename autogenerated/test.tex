\begin{MExercise}
K"urzen Sie soweit m"oglich: \MEquationItem{$\displaystyle \frac{6\, b\, a^2 + 4\, b\, a}{14\, a\, b}$}{???}\: .\\ 
\begin{MHint}{L\"osung}
\quad $\frac{6\, b\, a^2 + 4\, b\, a}{14\, a\, b}=\frac{(2\, a\, b)\cdot(3\, a + 2)}{(2\, a\, b)\cdot(7)}=\frac{3\, a + 2}{7}$.\end{MHint}
 \end{MExercise}
\begin{MExercise}
K"urzen Sie soweit m"oglich: \MEquationItem{$\displaystyle \frac{42}{105}$}{???}\: .\\ 
\begin{MHint}{L\"osung}
\quad $\frac{42}{105}=\frac{(21)\cdot(2)}{(21)\cdot(5)}=\frac{2}{5}$.\end{MHint}
 \end{MExercise}
\begin{MExercise}
K"urzen Sie soweit m"oglich: \MEquationItem{$\displaystyle \frac{9\, x^2 - 9\, y^2}{6\, x + 6\, y}$}{???}\: .\\ 
\begin{MHint}{L\"osung}
\quad $\frac{9\, x^2 - 9\, y^2}{6\, x + 6\, y}=\frac{(3\, x + 3\, y)\cdot(3\, x - 3\, y)}{(3\, x + 3\, y)\cdot(2)}=\frac{3\, x - 3\, y}{2}$.\end{MHint}
 \end{MExercise}
