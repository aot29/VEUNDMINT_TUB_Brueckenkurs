% localized text snippets
% python-like dictionaries or property structures are not supported by the
% converter.
% Todo: append these to the json files
\newcommand{\iCoshAdd}{Weiterf\"uhrende Inhalte}
\newcommand{\iCoshWarn}{Diese Inhalte liegen \"uber dem Kursniveau und
werden in den Aufgaben und Tests nicht abgefragt.}
\newcommand{\iExercise}{Aufgabe}
\newcommand{\iExercises}{Aufgaben}
\newcommand{\iExample}{Beispiel}
\newcommand{\iExperiment}{Versuch}
\newcommand{\iVideoWarn}{Video nicht darstellbar}
\newcommand{\iFormZoomHint}{Formel zu klein? Mit einem Doppelklick auf eine
Formel wird sie vergr"o"sert dargestellt.}
\newcommand{\iUsageHint}{Mit Hilfe der Symbole am oberen Rand des Fensters
k"onnen Sie durch die einzelnen Abschnitte navigieren.}
\newcommand{\iTextInputHint}{Geben Sie jeweils ein Wort oder Zeichen als
Antwort ein.}
\newcommand{\iTermInputHint}{Klammern Sie Ihre Terme, um eine eindeutige Eingabe zu erhalten. Beispiel: Der Term $\frac{3x+1}{x-2}$ soll in der Form \texttt{(3*x+1)/((x+2)^2)} eingegeben werden (wobei auch Leerzeichen eingegeben werden k"onnen, damit eine Formel besser lesbar ist).}
\newcommand{\iIntervalInputHint}{Intervalle werden links mit einer
 "offnenden Klammer und rechts mit einer schlie"senden Klammer angegeben. Eine runde Klammer wird verwendet, wenn der
Rand nicht dazu geh"ort, eine eckige, wenn er dazu geh"ort.
Als Trennzeichen wird ein Komma oder ein Semikolon akzeptiert.
Beispiele: $(a, b)$ offenes Intervall,
$[a; b)$ links abgeschlossenes, rechts offenes Intervall von $a$ bis $b$.
Die Eingabe $]a;b[$ f"ur ein offenes Intervall wird nicht akzeptiert.
F"ur $\infty$ kann \texttt{infty} oder \texttt{unendlich} geschrieben werden.}
\newcommand{\iFunctionInputHint}{Schreiben Sie Malpunkte (geschrieben als
 \texttt{*}) aus und setzen Sie Klammern um Argumente f\"ur Funktionen.
Beispiele: Polynom: \texttt{3*x + 0.1}, Sinusfunktion: \texttt{sin(x)},
Verkettung von cos und Wurzel: \texttt{cos(sqrt(3*x))}.}
\newcommand{\iSincosInputHint}{Die Sinusfunktion $\sin x$ wird in der Form
 \texttt{sin(x)} angegeben, $\cos\left(\sqrt{3 x}\right)$ durch \texttt{cos(sqrt(3*x))}.}
\newcommand{\iExpInputHint}{Die Exponentialfunktion $\MEU^{3x^4 + 5}$ wird
 als \texttt{exp(3 * x^4 + 5)} angegeben,
$\ln\left(\sqrt{x} + 3.2\right)$ durch \texttt{ln(sqrt(x) + 3.2)}.}
\newcommand{\iTest}{Dies ist ein einreichbarer Test:
\begin{itemize}
\item{Im Gegensatz zu den offenen Aufgaben werden beim Eingeben keine Hinweise
 zur Formulierung der mathematischen Ausdr\"ucke gegeben.}
\item{Der Test kann jederzeit neu gestartet oder verlassen werden.}
\item{Der Test kann durch die Buttons am Ende der Seite beendet und
 abgeschickt, oder zur\"uckgesetzt werden.}
\item{Der Test kann mehrfach probiert werden. F\"ur die Statistik z\"ahlt die
 zuletzt abgeschickte Version.}
\end{itemize}}
\newcommand{\iTestSubmit}{Test auswerten}
\newcommand{\iTestReset}{Test l\"oschen und neu starten}
\newcommand{\iTestEval}{Hier erscheint die Testauswertung!}
