% localized text snippets
\newcommand{\iCoshAdd}{Further material.}
\newcommand{\iCoshWarn}{This material is above course level and is not required
for the exercises and tests.}
\newcommand{\iExercise}{Exercise}
\newcommand{\iExercises}{Exercises}
\newcommand{\iExample}{Example}
\newcommand{\iExperiment}{Experiment}
\newcommand{\iVideoWarn}{Video cannot be played}
\newcommand{\iFormZoomHint}{Formula too small? Double click to enlarge.}
\newcommand{\iUsageHint}{The symbols at the upper boundary allow you to
navigate through the course.}
\newcommand{\iTextInputHint}{Enter a word or a symbol.}
\newcommand{\iTermInputHint}{Use parentheses to get unambiguous expression.
Example: The Term $\frac{3x+1}{x-2}$ should be entered as
\texttt{(3*x+1)/(x+2)} (you can add spaces for legibility).}
\newcommand{\iIntervalInputHint}{Intervals are written like this: $[a,b]$ for a
closed interval, $(a,b]$ for a left open right closed interval,
$[a,b)$ for a left closed right open interval and $(a,b)$ for
an open interval. $a$ and $b$ are infimum and supremum of
the interval, they can be real numbers or $\pm \infty$.
Write \texttt{infty} for $\infty$. The two numbers can
be separated by a comma or by a semicolon, both are valid
expressions.}
\newcommand{\iFunctionInputHint}{Explicitily denote multiplication by \texttt{*} and use parentheses for the arguments of functions.
Example: Polynomial: \texttt{3*x + 0.1}, sine function: \texttt{sin(x)},
Composition of cos and the square root function: \texttt{cos(sqrt(3*x))}.}
\newcommand{\iSincosInputHint}{The Sine function $\sin x$ has to be entered like this \texttt{sin(x)} (with parentheses),
$\cos\left(\sqrt{3 x}\right)$ like \texttt{cos(sqrt(3*x))}.}
\newcommand{\iExpInputHint}{The exponential function $\MEU^{3x^4 + 5}$ has to be entered as
\texttt{exp(3 * x^4 + 5)},
$\ln\left(\sqrt{x} + 3.2\right)$ as \texttt{ln(sqrt(x) + 3.2)}.}
\newcommand{\iTest}{This is a test for submission:
\begin{itemize}
\item{Unlike open exercises, no hints for formulating mathematical expressions are provided.}
\item{The test can be restarted or interrupted at any time.}
\item{The test can be terminated and submitted using the buttons at the end of the page, or reset.}
\item{The test can be attempted several times. Only the last version will be included in the statistics.}
\end{itemize}}
\newcommand{\iTestSubmit}{Submit test}
\newcommand{\iTestReset}{Reset and restart}
\newcommand{\iTestEval}{The test evaluation will be displayed here!}
\newcommand{\iIntroduction}{Introduction}
\newcommand{\iContents}{Contents}
\newcommand{\iModuleOverview}{Module Overview}
