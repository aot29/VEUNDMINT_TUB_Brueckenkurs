%%%%%%%%%%%%%%%%%%%%%%%%%%%%%%%%%%%%%%%%%%%%%%%%%%%%%%%%%%%%%%%%%%%%%%%%%%%%%%%%
% Modul Schwingungen und Wellen
% Modul Nr. 8 im Fachbereich Physik
% Hauptdokument: phys_schwingungen.tex
% Nach der Einf"uhrung in das Modul (Umgebung MSectionStart) folgen 
% zwei Teile, n"amlich "Schwingungen" und "Wellen" (\MSubssection{}).
% In jedem Teil nach Umgebung MIntro (einleitendes Beispiel) diverse
% Unterabschnitte in MXContent-Umgebungen mit "Ubungen (Umgebung
% MExercise. Kein getrennter "Ubungsteil geplant.
% Aktuelle Version im Verzeichnis ModulSchwingungen, ggf. "altere
% Versionen in Verzeichnissen mit diesem Namen plus Datum.
%
% Version 1: 2012=01=17 (grobe Struktur)
% Version 2: 2012=01=18 (verfeinerte Struktur, Abb., Tabellen)
% Version 3: 2012=01=27 (weitgehend fertig bis "Betrachtung der L"osung")
% Version 4: 2012=02=06 (weitgehend fertig bis zur numerischen Simulation)
% Version 5: 2012=02=07 (MLabel in captions, MExamples)
% Version 6: 2012=02=17 (Subsection Schwingungen fertig)
% Version 7: 2012=02=29 (Intro Wellen und Wellengleichung)
% Version 8: 2012=09=01 (Fertig f"ur Korrektur)
% Version 9: 2013=02=03
%			MLabels durchg"angig
%			Physik_Schwingungen (Modul)
%			Physik_Schwingungen_SubXyz (Teil)
%			Physik_Schwingungen_MXXyz (Abschnitt)
%			Physik_Schwingungen_BspXyz (Beispiel)
%			Physik_Schwingungen_AufgabeXyz (Aufgaben)
%			Physik_Schwingungen_VonAufgabeXyz (zum Text)
%			Physik_Schwingungen_AbbXyz (Abbildung)
%			Physik_Schwingungen_EQXyz (Gleichung - auch Eq...)
%			Unterverzeichnis f"ur Bilder
%			Anpassung auf aktuelle mintmod-Version
%			Einf"uhrung von MZahl, MEinheit, MExponent
%			Ersetzung von "/" durch "\frac{}{}
%			Klammern der Argumente von sin, cos, tan
%			Zusammenfassung der Aufgaben der (neu) drei Teile
%			Verschieben in zus"atzliche Inhalte von
%			Numerik und RLC
%			Vertauschung von Erzwungen und Ged"ampft,
%			Erzwungen ohne D"ampfung entf"allt, daf"ur
%			Einschwingvorgang bei Erzwungen
%			Entfernung zus"atzlicher Hinweise zu Gleichungen etc.
%			aus dem PDF, statt dessen Verweise auf Onlinemodul
%			Mathematisches Pendel eingef"ugt
%			Abbildung zu der Oszillations-Waage eingef"ugt
%			Numerik mit Tabellenkalkulation und Pseudocode
%			Einleitende Beispiele Schnitt Kugelwelle, online 
%			auch animiert, und schwingende Saite
% Version 9: 2013=10=24 (Letzte Korrekturen von Inge Karl)
% To Do: Veroeffenlichung, Harmonisierung aller Module
%%%%%%%%%%%%%%%%%%%%%%%%%%%%%%%%%%%%%%%%%%%%%%%%%%%%%%%%%%%%%%%%%%%%%%%%%%%%%%%%

% MINTMOD Version P0.1.0, needs to be consistent with preprocesser object in tex2x and MPragma-Version at the end of this file

% Parameter aus Konvertierungsprozess (PDF und HTML-Erzeugung wenn vom Konverter aus gestartet) werden hier eingefuegt, Preambleincludes werden am Schluss angehaengt

\newif\ifttm                % gesetzt falls Uebersetzung in HTML stattfindet, sonst uebersetzung in PDF

% Wahl der Notationsvariante ist im PDF immer std, in der HTML-Uebersetzung wird vom Konverter die Auswahl modifiziert
\newif\ifvariantstd
\newif\ifvariantunotation
\variantstdtrue % Diese Zeile wird vom Konverter erkannt und ggf. modifiziert, daher nicht veraendern!


\def\MOutputDVI{1}
\def\MOutputPDF{2}
\def\MOutputHTML{3}
\newcounter{MOutput}

\ifttm
\usepackage{german}
\usepackage{array}
\usepackage{amsmath}
\usepackage{amssymb}
\usepackage{amsthm}
\else
\documentclass[ngerman,oneside]{scrbook}
\usepackage{etex}
\usepackage[latin1]{inputenc}
\usepackage{textcomp}
\usepackage[ngerman]{babel}
\usepackage[pdftex]{color}
\usepackage{xcolor}
\usepackage{graphicx}
\usepackage[all]{xy}
\usepackage{fancyhdr}
\usepackage{verbatim}
\usepackage{array}
\usepackage{float}
\usepackage{makeidx}
\usepackage{amsmath}
\usepackage{amstext}
\usepackage{amssymb}
\usepackage{amsthm}
\usepackage[ngerman]{varioref}
\usepackage{framed}
\usepackage{supertabular}
\usepackage{longtable}
\usepackage{maxpage}
\usepackage{tikz}
\usepackage{tikzscale}
\usepackage{tikz-3dplot}
\usepackage{bibgerm}
\usepackage{chemarrow}
\usepackage{polynom}
%\usepackage{draftwatermark}
\usepackage{pdflscape}
\usetikzlibrary{calc}
\usetikzlibrary{through}
\usetikzlibrary{shapes.geometric}
\usetikzlibrary{arrows}
\usetikzlibrary{intersections}
\usetikzlibrary{decorations.pathmorphing}
\usetikzlibrary{external}
\usetikzlibrary{patterns}
\usetikzlibrary{fadings}
\usepackage[colorlinks=true,linkcolor=blue]{hyperref} 
\usepackage[all]{hypcap}
%\usepackage[colorlinks=true,linkcolor=blue,bookmarksopen=true]{hyperref} 
\usepackage{ifpdf}

\usepackage{movie15}

\setcounter{tocdepth}{2} % In Inhaltsverzeichnis bis subsection
\setcounter{secnumdepth}{3} % Nummeriert bis subsubsection

\setlength{\LTpost}{0pt} % Fuer longtable
\setlength{\parindent}{0pt}
\setlength{\parskip}{8pt}
%\setlength{\parskip}{9pt plus 2pt minus 1pt}
\setlength{\abovecaptionskip}{-0.25ex}
\setlength{\belowcaptionskip}{-0.25ex}
\fi

\ifttm
\newcommand{\MDebugMessage}[1]{\special{html:<!-- debugprint;;}#1\special{html:; //-->}}
\else
%\newcommand{\MDebugMessage}[1]{\immediate\write\mintlog{#1}}
\newcommand{\MDebugMessage}[1]{}
\fi

\def\MPageHeaderDef{%
\pagestyle{fancy}%
\fancyhead[r]{(C) VE\&MINT-Projekt}
\fancyfoot[c]{\thepage\\--- CCL BY-SA 3.0 ---}
}


\ifttm%
\def\MRelax{}%
\else%
\def\MRelax{\relax}%
\fi%

%--------------------------- Uebernahme von speziellen XML-Versionen einiger LaTeX-Kommandos aus xmlbefehle.tex vom alten Kasseler Konverter ---------------

\newcommand{\MSep}{\left\|{\phantom{\frac1g}}\right.}

\newcommand{\ML}{L}

\newcommand{\MGGT}{\mathrm{ggT}}


\ifttm
% Verhindert dass die subsection-nummer doppelt in der toccaption auftaucht (sollte ggf. in toccaption gefixt werden so dass diese Ueberschreibung nicht notwendig ist)
\renewcommand{\thesubsection}{}
% Kommandos die ttm nicht kennt
\newcommand{\binomial}[2]{{#1 \choose #2}} %  Binomialkoeffizienten
\newcommand{\eur}{\begin{html}&euro;\end{html}}
\newcommand{\square}{\begin{html}&square;\end{html}}
\newcommand{\glqq}{"'}  \newcommand{\grqq}{"'}
\newcommand{\nRightarrow}{\special{html: &nrArr; }}
\newcommand{\nmid}{\special{html: &nmid; }}
\newcommand{\nparallel}{\begin{html}&nparallel;\end{html}}
\newcommand{\mapstoo}{\begin{html}<mo>&map;</mo>\end{html}}

% Schnitt und Vereinigungssymbole von Mengen haben zu kleine Abstaende; korrigiert:
\newcommand{\ccup}{\,\!\cup\,\!}
\newcommand{\ccap}{\,\!\cap\,\!}


% Umsetzung von mathbb im HTML
\renewcommand{\mathbb}[1]{\begin{html}<mo>&#1opf;</mo>\end{html}}
\fi

%---------------------- Strukturierung ----------------------------------------------------------------------------------------------------------------------

%---------------------- Kapselung des sectioning findet auf drei Ebenen statt:
% 1. Die LateX-Befehl
% 2. Die D-Versionen der Befehle, die nur die Grade der Abschnitte umhaengen falls notwendig
% 3. Die M-Versionen der Befehle, die zusaetzliche Formatierungen vornehmen, Skripten starten und das HTML codieren
% Im Modultext duerfen nur die M-Befehle verwendet werden!

\ifttm

  \def\Dsubsubsubsection#1{\subsubsubsection{#1}}
  \def\Dsubsubsection#1{\subsubsection{#1}\addtocounter{subsubsection}{1}} % ttm-Fehler korrigieren
  \def\Dsubsection#1{\subsection{#1}}
  \def\Dsection#1{\section{#1}} % Im HTML wird nur der Sektionstitel gegeben
  \def\Dchapter#1{\chapter{#1}}
  \def\Dsubsubsubsectionx#1{\subsubsubsection*{#1}}
  \def\Dsubsubsectionx#1{\subsubsection*{#1}}
  \def\Dsubsectionx#1{\subsection*{#1}}
  \def\Dsectionx#1{\section*{#1}}
  \def\Dchapterx#1{\chapter*{#1}}

\else

  \def\Dsubsubsubsection#1{\subsubsection{#1}}
  \def\Dsubsubsection#1{\subsection{#1}}
  \def\Dsubsection#1{\section{#1}}
  \def\Dsection#1{\chapter{#1}}
  \def\Dchapter#1{\title{#1}}
  \def\Dsubsubsubsectionx#1{\subsubsection*{#1}}
  \def\Dsubsubsectionx#1{\subsection*{#1}}
  \def\Dsubsectionx#1{\section*{#1}}
  \def\Dsectionx#1{\chapter*{#1}}

\fi

\newcommand{\MStdPoints}{4}
\newcommand{\MSetPoints}[1]{\renewcommand{\MStdPoints}{#1}}

% Befehl zum Abbruch der Erstellung (nur PDF)
\newcommand{\MAbort}[1]{\err{#1}}

% Prefix vor Dateieinbindungen, wird in der Baumdatei mit \renewcommand modifiziert
% und auf das Verzeichnisprefix gesetzt, in dem das gerade bearbeitete tex-Dokument liegt.
% Im HTML wird es auf das Verzeichnis der HTML-Datei gesetzt.
% Das Prefix muss mit / enden !
\newcommand{\MDPrefix}{.}

% MRegisterFile notiert eine Datei zur Einbindung in den HTML-Baum. Grafiken mit MGraphics werden automatisch eingebunden.
% Mit MLastFile erhaelt man eine Markierung fuer die zuletzt registrierte Datei.
% Diese Markierung wird im postprocessing durch den physikalischen Dateinamen ersetzt, aber nur den Namen (d.h. \MMaterial gehoert noch davor, vgl Definition von MGraphics)
% Parameter: Pfad/Name der Datei bzw. des Ordners, relativ zur Position des Modul-Tex-Dokuments.
\ifttm
\newcommand{\MRegisterFile}[1]{\addtocounter{MFileNumber}{1}\special{html:<!-- registerfile;;}#1\special{html:;;}\MDPrefix\special{html:;;}\arabic{MFileNumber}\special{html:; //-->}}
\else
\newcommand{\MRegisterFile}[1]{\addtocounter{MFileNumber}{1}}
\fi

% Testen welcher Uebersetzer hier am Werk ist

\ifttm
\setcounter{MOutput}{3}
\else
\ifx\pdfoutput\undefined
  \pdffalse
  \setcounter{MOutput}{\MOutputDVI}
  \message{Verarbeitung mit latex, Ausgabe in dvi.}
\else
  \setcounter{MOutput}{\MOutputPDF}
  \message{Verarbeitung mit pdflatex, Ausgabe in pdf.}
  \ifnum \pdfoutput=0
    \pdffalse
  \setcounter{MOutput}{\MOutputDVI}
  \message{Verarbeitung mit pdflatex, Ausgabe in dvi.}
  \else
    \ifnum\pdfoutput=1
    \pdftrue
  \setcounter{MOutput}{\MOutputPDF}
  \message{Verarbeitung mit pdflatex, Ausgabe in pdf.}
    \fi
  \fi
\fi
\fi

\ifnum\value{MOutput}=\MOutputPDF
\DeclareGraphicsExtensions{.pdf,.png,.jpg}
\fi

\ifnum\value{MOutput}=\MOutputDVI
\DeclareGraphicsExtensions{.eps,.png,.jpg}
\fi

\ifnum\value{MOutput}=\MOutputHTML
% Wird vom Konverter leider nicht erkannt und daher in split.pm hardcodiert!
\DeclareGraphicsExtensions{.png,.jpg,.gif}
\fi

% Umdefinition der hyperref-Nummerierung im PDF-Modus
\ifttm
\else
\renewcommand{\theHfigure}{\arabic{chapter}.\arabic{section}.\arabic{figure}}
\fi

% Makro, um in der HTML-Ausgabe die zuerst zu oeffnende Datei zu kennzeichnen
\ifttm
\newcommand{\MGlobalStart}{\special{html:<!-- mglobalstarttag -->}}
\else
\newcommand{\MGlobalStart}{}
\fi

% Makro, um bei scormlogin ein pullen des Benutzers bei Aufruf der Seite zu erzwingen (typischerweise auf der Einstiegsseite)
\ifttm
\newcommand{\MPullSite}{\special{html:<!-- pullsite //-->}}
\else
\newcommand{\MPullSite}{}
\fi

% Makro, um in der HTML-Ausgabe die Kapiteluebersicht zu kennzeichnen
\ifttm
\newcommand{\MGlobalChapterTag}{\special{html:<!-- mglobalchaptertag -->}}
\else
\newcommand{\MGlobalChapterTag}{}
\fi

% Makro, um in der HTML-Ausgabe die Konfiguration zu kennzeichnen
\ifttm
\newcommand{\MGlobalConfTag}{\special{html:<!-- mglobalconfigtag -->}}
\else
\newcommand{\MGlobalConfTag}{}
\fi

% Makro, um in der HTML-Ausgabe die Standortbeschreibung zu kennzeichnen
\ifttm
\newcommand{\MGlobalLocationTag}{\special{html:<!-- mgloballocationtag -->}}
\else
\newcommand{\MGlobalLocationTag}{}
\fi

% Makro, um in der HTML-Ausgabe die persoenlichen Daten zu kennzeichnen
\ifttm
\newcommand{\MGlobalDataTag}{\special{html:<!-- mglobaldatatag -->}}
\else
\newcommand{\MGlobalDataTag}{}
\fi

% Makro, um in der HTML-Ausgabe die Suchseite zu kennzeichnen
\ifttm
\newcommand{\MGlobalSearchTag}{\special{html:<!-- mglobalsearchtag -->}}
\else
\newcommand{\MGlobalSearchTag}{}
\fi

% Makro, um in der HTML-Ausgabe die Favoritenseite zu kennzeichnen
\ifttm
\newcommand{\MGlobalFavoTag}{\special{html:<!-- mglobalfavoritestag -->}}
\else
\newcommand{\MGlobalFavoTag}{}
\fi

% Makro, um in der HTML-Ausgabe die Eingangstestseite zu kennzeichnen
\ifttm
\newcommand{\MGlobalSTestTag}{\special{html:<!-- mglobalstesttag -->}}
\else
\newcommand{\MGlobalSTestTag}{}
\fi

% Makro, um in der PDF-Ausgabe ein Wasserzeichen zu definieren
\ifttm
\newcommand{\MWatermarkSettings}{\relax}
\else
\newcommand{\MWatermarkSettings}{%
% \SetWatermarkText{(c) MINT-Kolleg Baden-W�rttemberg 2014}
% \SetWatermarkLightness{0.85}
% \SetWatermarkScale{1.5}
}
\fi

\ifttm
\newcommand{\MBinom}[2]{\left({\begin{array}{c} #1 \\ #2 \end{array}}\right)}
\else
\newcommand{\MBinom}[2]{\binom{#1}{#2}}
\fi

\ifttm
\newcommand{\DeclareMathOperator}[2]{\def#1{\mathrm{#2}}}
\newcommand{\operatorname}[1]{\mathrm{#1}}
\fi

%----------------- Makros fuer die gemischte HTML/PDF-Konvertierung ------------------------------

\newcommand{\MTestName}{\relax} % wird durch Test-Umgebung gesetzt

% Fuer experimentelle Kursinhalte, die im Release-Umsetzungsvorgang eine Fehlermeldung
% produzieren sollen aber sonst normal umgesetzt werden
\newenvironment{MExperimental}{%
}{%
}

% Wird von ttm nicht richtig umgesetzt!!
\newenvironment{MExerciseItems}{%
\renewcommand\theenumi{\alph{enumi}}%
\begin{enumerate}%
}{%
\end{enumerate}%
}


\definecolor{infoshadecolor}{rgb}{0.75,0.75,0.75}
\definecolor{exmpshadecolor}{rgb}{0.875,0.875,0.875}
\definecolor{expeshadecolor}{rgb}{0.95,0.95,0.95}
\definecolor{framecolor}{rgb}{0.2,0.2,0.2}

% Bei PDF-Uebersetzung wird hinter den Start jeder Satz/Info-aehnlichen Umgebung eine leere mbox gesetzt, damit
% fuehrende Listen oder enums nicht den Zeilenumbruch kaputtmachen
%\ifttm
\def\MTB{}
%\else
%\def\MTB{\mbox{}}
%\fi


\ifttm
\newcommand{\MRelates}{\special{html:<mi>&wedgeq;</mi>}}
\else
\def\MRelates{\stackrel{\scriptscriptstyle\wedge}{=}}
\fi

\def\MInch{\text{''}}
\def\Mdd{\textit{''}}

\ifttm
\def\MNL{ \newline }
\newenvironment{MArray}[1]{\begin{array}{#1}}{\end{array}}
\else
\def\MNL{ \\ }
\newenvironment{MArray}[1]{\begin{array}{#1}}{\end{array}}
\fi

\newcommand{\MBox}[1]{$\mathrm{#1}$}
\newcommand{\MMBox}[1]{\mathrm{#1}}


\ifttm%
\newcommand{\Mtfrac}[2]{{\textstyle \frac{#1}{#2}}}
\newcommand{\Mdfrac}[2]{{\displaystyle \frac{#1}{#2}}}
\newcommand{\Mmeasuredangle}{\special{html:<mi>&angmsd;</mi>}}
\else%
\newcommand{\Mtfrac}[2]{\tfrac{#1}{#2}}
\newcommand{\Mdfrac}[2]{\dfrac{#1}{#2}}
\newcommand{\Mmeasuredangle}{\measuredangle}
\relax
\fi

% Matrizen und Vektoren

% Inhalt wird in der Form a & b \\ c & d erwartet
% Vorsicht: MVector = Komponentenspalte, MVec = Variablensymbol
\ifttm%
\newcommand{\MVector}[1]{\left({\begin{array}{c}#1\end{array}}\right)}
\else%
\newcommand{\MVector}[1]{\begin{pmatrix}#1\end{pmatrix}}
\fi



\newcommand{\MVec}[1]{\vec{#1}}
\newcommand{\MDVec}[1]{\overrightarrow{#1}}

%----------------- Umgebungen fuer Definitionen und Saetze ----------------------------------------

% Fuegt einen Tabellen-Zeilenumbruch ein im PDF, aber nicht im HTML
\newcommand{\TSkip}{\ifttm \else&\ \\\fi}

\newenvironment{infoshaded}{%
\def\FrameCommand{\fboxsep=\FrameSep \fcolorbox{framecolor}{infoshadecolor}}%
\MakeFramed {\advance\hsize-\width \FrameRestore}}%
{\endMakeFramed}

\newenvironment{expeshaded}{%
\def\FrameCommand{\fboxsep=\FrameSep \fcolorbox{framecolor}{expeshadecolor}}%
\MakeFramed {\advance\hsize-\width \FrameRestore}}%
{\endMakeFramed}

\newenvironment{exmpshaded}{%
\def\FrameCommand{\fboxsep=\FrameSep \fcolorbox{framecolor}{exmpshadecolor}}%
\MakeFramed {\advance\hsize-\width \FrameRestore}}%
{\endMakeFramed}

\def\STDCOLOR{black}

\ifttm%
\else%
\newtheoremstyle{MSatzStyle}
  {1cm}                   %Space above
  {1cm}                   %Space below
  {\normalfont\itshape}   %Body font
  {}                      %Indent amount (empty = no indent,
                          %\parindent = para indent)
  {\normalfont\bfseries}  %Thm head font
  {}                      %Punctuation after thm head
  {\newline}              %Space after thm head: " " = normal interword
                          %space; \newline = linebreak
  {\thmname{#1}\thmnumber{ #2}\thmnote{ (#3)}}
                          %Thm head spec (can be left empty, meaning
                          %`normal')
                          %
\newtheoremstyle{MDefStyle}
  {1cm}                   %Space above
  {1cm}                   %Space below
  {\normalfont}           %Body font
  {}                      %Indent amount (empty = no indent,
                          %\parindent = para indent)
  {\normalfont\bfseries}  %Thm head font
  {}                      %Punctuation after thm head
  {\newline}              %Space after thm head: " " = normal interword
                          %space; \newline = linebreak
  {\thmname{#1}\thmnumber{ #2}\thmnote{ (#3)}}
                          %Thm head spec (can be left empty, meaning
                          %`normal')
\fi%

\newcommand{\MInfoText}{Info}

\newcounter{MHintCounter}
\newcounter{MCodeEditCounter}

\newcounter{MLastIndex}  % Enthaelt die dritte Stelle (Indexnummer) des letzten angelegten Objekts
\newcounter{MLastType}   % Enthaelt den Typ des letzten angelegten Objekts (mithilfe der unten definierten Konstanten). Die Entscheidung, wie der Typ dargstellt wird, wird in split.pm beim Postprocessing getroffen.
\newcounter{MLastTypeEq} % =1 falls das Label in einer Matheumgebung (equation, eqnarray usw.) steht, =2 falls das Label in einer table-Umgebung steht

% Da ttm keine Zahlmakros verarbeiten kann, werden diese Nummern in den Zuweisungen hardcodiert!
\def\MTypeSection{1}          %# Zaehler ist section
\def\MTypeSubsection{2}       %# Zaehler ist subsection
\def\MTypeSubsubsection{3}    %# Zaehler ist subsubsection
\def\MTypeInfo{4}             %# Eine Infobox, Separatzaehler fuer die Chemie (auch wenn es dort nicht nummeriert wird) ist MInfoCounter
\def\MTypeExercise{5}         %# Eine Aufgabe, Separatzaehler fuer die Chemie ist MExerciseCounter
\def\MTypeExample{6}          %# Eine Beispielbox, Separatzaehler fuer die Chemie ist MExampleCounter
\def\MTypeExperiment{7}       %# Eine Versuchsbox, Separatzaehler fuer die Chemie ist MExperimentCounter
\def\MTypeGraphics{8}         %# Eine Graphik, Separatzaehler fuer alle FB ist MGraphicsCounter
\def\MTypeTable{9}            %# Eine Tabellennummer, hat keinen Zaehler da durch table gezaehlt wird
\def\MTypeEquation{10}        %# Eine Gleichungsnummer, hat keinen Zaehler da durch equation/eqnarray gezaehlt wird
\def\MTypeTheorem{11}         % Ein theorem oder xtheorem, Separatzaehler fuer die Chemie ist MTheoremCounter
\def\MTypeVideo{12}           %# Ein Video,Separatzaehler fuer alle FB ist MVideoCounter
\def\MTypeEntry{13}           %# Ein Eintrag fuer die Stichwortliste, wird nicht gezaehlt sondern erhaelt im preparsing ein unique-label 

% Zaehler fuer das Labelsystem sind prefixcounter, jeder Zaehler wird VOR dem gezaehlten Objekt inkrementiert und zaehlt daher das aktuelle Objekt
\newcounter{MInfoCounter}
\newcounter{MExerciseCounter}
\newcounter{MExampleCounter}
\newcounter{MExperimentCounter}
\newcounter{MGraphicsCounter}
\newcounter{MTableCounter}
\newcounter{MEquationCounter}  % Nur im HTML, sonst durch "equation"-counter von latex realisiert
\newcounter{MTheoremCounter}
\newcounter{MObjectCounter}   % Gemeinsamer Zaehler fuer Objekte (ausser Grafiken/Tabellen) in Mathe/Info/Physik
\newcounter{MVideoCounter}
\newcounter{MEntryCounter}

\newcounter{MTestSite} % 1 = Subsubsection ist eine Pruefungsseite, 0 = ist eine normale Seite (inkl. Hilfeseite)

\def\MCell{$\phantom{a}$}

\newenvironment{MExportExercise}{\begin{MExercise}}{\end{MExercise}} % wird von mconvert abgefangen

\def\MGenerateExNumber{%
\ifnum\value{MSepNumbers}=0%
\arabic{section}.\arabic{subsection}.\arabic{MObjectCounter}\setcounter{MLastIndex}{\value{MObjectCounter}}%
\else%
\arabic{section}.\arabic{subsection}.\arabic{MExerciseCounter}\setcounter{MLastIndex}{\value{MExerciseCounter}}%
\fi%
}%

\def\MGenerateExmpNumber{%
\ifnum\value{MSepNumbers}=0%
\arabic{section}.\arabic{subsection}.\arabic{MObjectCounter}\setcounter{MLastIndex}{\value{MObjectCounter}}%
\else%
\arabic{section}.\arabic{subsection}.\arabic{MExerciseCounter}\setcounter{MLastIndex}{\value{MExampleCounter}}%
\fi%
}%

\def\MGenerateInfoNumber{%
\ifnum\value{MSepNumbers}=0%
\arabic{section}.\arabic{subsection}.\arabic{MObjectCounter}\setcounter{MLastIndex}{\value{MObjectCounter}}%
\else%
\arabic{section}.\arabic{subsection}.\arabic{MExerciseCounter}\setcounter{MLastIndex}{\value{MInfoCounter}}%
\fi%
}%

\def\MGenerateSiteNumber{%
\arabic{section}.\arabic{subsection}.\arabic{subsubsection}%
}%

% Funktionalitaet fuer Auswahlaufgaben

\newcounter{MExerciseCollectionCounter} % = 0 falls nicht in collection-Umgebung, ansonsten Schachtelungstiefe
\newcounter{MExerciseCollectionTextCounter} % wird von MExercise-Umgebung inkrementiert und von MExerciseCollection-Umgebung auf Null gesetzt

\ifttm
% MExerciseCollection gruppiert Aufgaben, die dynamisch aus der Datenbank gezogen werden und nicht direkt in der HTML-Seite stehen
% Parameter: #1 = ID der Collection, muss eindeutig fuer alle IN DER DB VORHANDENEN collections sein unabhaengig vom Kurs
%            #2 = Optionsargument (im Moment: 1 = Iterative Auswahl, 2 = Zufallsbasierte Auswahl)
\newenvironment{MExerciseCollection}[2]{%
\addtocounter{MExerciseCollectionCounter}{1}
\setcounter{MExerciseCollectionTextCounter}{0}
\special{html:<!-- mexercisecollectionstart;;}#1\special{html:;;}#2\special{html:;; //-->}%
}{%
\special{html:<!-- mexercisecollectionstop //-->}%
\addtocounter{MExerciseCollectionCounter}{-1}
}
\else
\newenvironment{MExerciseCollection}[2]{%
\addtocounter{MExerciseCollectionCounter}{1}
\setcounter{MExerciseCollectionTextCounter}{0}
}{%
\addtocounter{MExerciseCollectionCounter}{-1}
}
\fi

% Bei Uebersetzung nach PDF werden die theorem-Umgebungen verwendet, bei Uebersetzung in HTML ein manuelles Makro
\ifttm%

  \newenvironment{MHint}[1]{  \special{html:<button name="Name_MHint}\arabic{MHintCounter}\special{html:" class="hintbutton_closed" id="MHint}\arabic{MHintCounter}\special{html:_button" %
  type="button" onclick="toggle_hint('MHint}\arabic{MHintCounter}\special{html:');">}#1\special{html:</button>}
  \special{html:<div class="hint" style="display:none" id="MHint}\arabic{MHintCounter}\special{html:"> }}{\begin{html}</div>\end{html}\addtocounter{MHintCounter}{1}}

  \newenvironment{MCOSHZusatz}{  \special{html:<button name="Name_MHint}\arabic{MHintCounter}\special{html:" class="chintbutton_closed" id="MHint}\arabic{MHintCounter}\special{html:_button" %
  type="button" onclick="toggle_hint('MHint}\arabic{MHintCounter}\special{html:');">}Weiterf�hrende Inhalte\special{html:</button>}
  \special{html:<div class="hintc" style="display:none" id="MHint}\arabic{MHintCounter}\special{html:">
  <div class="coshwarn">Diese Inhalte gehen �ber das Kursniveau hinaus und werden in den Aufgaben und Tests nicht abgefragt.</div><br />}
  \addtocounter{MHintCounter}{1}}{\begin{html}</div>\end{html}}

  
  \newenvironment{MDefinition}{\begin{definition}\setcounter{MLastIndex}{\value{definition}}\ \\}{\end{definition}}

  
  \newenvironment{MExercise}{
  \renewcommand{\MStdPoints}{4}
  \addtocounter{MExerciseCounter}{1}
  \addtocounter{MObjectCounter}{1}
  \setcounter{MLastType}{5}

  \ifnum\value{MExerciseCollectionCounter}=0\else\addtocounter{MExerciseCollectionTextCounter}{1}\special{html:<!-- mexercisetextstart;;}\arabic{MExerciseCollectionTextCounter}\special{html:;; //-->}\fi
  \special{html:<div class="aufgabe" id="ADIV_}\MGenerateExNumber\special{html:">}%
  \textbf{Aufgabe \MGenerateExNumber
  } \ \\}{
  \special{html:</div><!-- mfeedbackbutton;Aufgabe;}\arabic{MTestSite}\special{html:;}\MGenerateExNumber\special{html:; //-->}
  \ifnum\value{MExerciseCollectionCounter}=0\else\special{html:<!-- mexercisetextstop //-->}\fi
  }

  % Stellt eine Kombination aus Aufgabe, Loesungstext und Eingabefeld bereit,
  % bei der Aufgabentext und Musterloesung sowie die zugehoerigen Feldelemente
  % extern bezogen und div-aktualisiert werden, das Eingabefeld aber immer das gleiche ist.
  \newenvironment{MFetchExercise}{
  \addtocounter{MExerciseCounter}{1}
  \addtocounter{MObjectCounter}{1}
  \setcounter{MLastType}{5}

  \special{html:<div class="aufgabe" id="ADIV_}\MGenerateExNumber\special{html:">}%
  \textbf{Aufgabe \MGenerateExNumber
  } \ \\%
  \special{html:</div><div class="exfetch_text" id="ADIVTEXT_}\MGenerateExNumber\special{html:">}%
  \special{html:</div><div class="exfetch_sol" id="ADIVSOL_}\MGenerateExNumber\special{html:">}%
  \special{html:</div><div class="exfetch_input" id="ADIVINPUT_}\MGenerateExNumber\special{html:">}%
  }{
  \special{html:</div>}
  }

  \newenvironment{MExample}{
  \addtocounter{MExampleCounter}{1}
  \addtocounter{MObjectCounter}{1}
  \setcounter{MLastType}{6}
  \begin{html}
  <div class="exmp">
  <div class="exmprahmen">
  \end{html}\textbf{Beispiel
  \ifnum\value{MSepNumbers}=0
  \arabic{section}.\arabic{subsection}.\arabic{MObjectCounter}\setcounter{MLastIndex}{\value{MObjectCounter}}
  \else
  \arabic{section}.\arabic{subsection}.\arabic{MExampleCounter}\setcounter{MLastIndex}{\value{MExampleCounter}}
  \fi
  } \ \\}{\begin{html}</div>
  </div>
  \end{html}
  \special{html:<!-- mfeedbackbutton;Beispiel;}\arabic{MTestSite}\special{html:;}\MGenerateExmpNumber\special{html:; //-->}
  }

  \newenvironment{MExperiment}{
  \addtocounter{MExperimentCounter}{1}
  \addtocounter{MObjectCounter}{1}
  \setcounter{MLastType}{7}
  \begin{html}
  <div class="expe">
  <div class="experahmen">
  \end{html}\textbf{Versuch
  \ifnum\value{MSepNumbers}=0
  \arabic{section}.\arabic{subsection}.\arabic{MObjectCounter}\setcounter{MLastIndex}{\value{MObjectCounter}}
  \else
%  \arabic{MExperimentCounter}\setcounter{MLastIndex}{\value{MExperimentCounter}}
  \arabic{section}.\arabic{subsection}.\arabic{MExperimentCounter}\setcounter{MLastIndex}{\value{MExperimentCounter}}
  \fi
  } \ \\}{\begin{html}</div>
  </div>
  \end{html}}

  \newenvironment{MChemInfo}{
  \setcounter{MLastType}{4}
  \begin{html}
  <div class="info">
  <div class="inforahmen">
  \end{html}}{\begin{html}</div>
  </div>
  \end{html}}

  \newenvironment{MXInfo}[1]{
  \addtocounter{MInfoCounter}{1}
  \addtocounter{MObjectCounter}{1}
  \setcounter{MLastType}{4}
  \begin{html}
  <div class="info">
  <div class="inforahmen">
  \end{html}\textbf{#1
  \ifnum\value{MInfoNumbers}=0
  \else
    \ifnum\value{MSepNumbers}=0
    \arabic{section}.\arabic{subsection}.\arabic{MObjectCounter}\setcounter{MLastIndex}{\value{MObjectCounter}}
    \else
    \arabic{MInfoCounter}\setcounter{MLastIndex}{\value{MInfoCounter}}
    \fi
  \fi
  } \ \\}{\begin{html}</div>
  </div>
  \end{html}
  \special{html:<!-- mfeedbackbutton;Info;}\arabic{MTestSite}\special{html:;}\MGenerateInfoNumber\special{html:; //-->}
  }

  \newenvironment{MInfo}{\ifnum\value{MInfoNumbers}=0\begin{MChemInfo}\else\begin{MXInfo}{Info}\ \\ \fi}{\ifnum\value{MInfoNumbers}=0\end{MChemInfo}\else\end{MXInfo}\fi}

\else%

  \theoremstyle{MSatzStyle}
  \newtheorem{thm}{Satz}[section]
  \newtheorem{thmc}{Satz}
  \theoremstyle{MDefStyle}
  \newtheorem{defn}[thm]{Definition}
  \newtheorem{exmp}[thm]{Beispiel}
  \newtheorem{info}[thm]{\MInfoText}
  \theoremstyle{MDefStyle}
  \newtheorem{defnc}{Definition}
  \theoremstyle{MDefStyle}
  \newtheorem{exmpc}{Beispiel}[section]
  \theoremstyle{MDefStyle}
  \newtheorem{infoc}{\MInfoText}
  \theoremstyle{MDefStyle}
  \newtheorem{exrc}{Aufgabe}[section]
  \theoremstyle{MDefStyle}
  \newtheorem{verc}{Versuch}[section]
  
  \newenvironment{MFetchExercise}{}{} % kann im PDF nicht dargestellt werden
  
  \newenvironment{MExercise}{\begin{exrc}\renewcommand{\MStdPoints}{1}\MTB}{\end{exrc}}
  \newenvironment{MHint}[1]{\ \\ \underline{#1:}\\}{}
  \newenvironment{MCOSHZusatz}{\ \\ \underline{Weiterf�hrende Inhalte:}\\}{}
  \newenvironment{MDefinition}{\ifnum\value{MInfoNumbers}=0\begin{defnc}\else\begin{defn}\fi\MTB}{\ifnum\value{MInfoNumbers}=0\end{defnc}\else\end{defn}\fi}
%  \newenvironment{MExample}{\begin{exmp}}{\ \linebreak[1] \ \ \ \ $\phantom{a}$ \ \hfill $\blacklozenge$\end{exmp}}
  \newenvironment{MExample}{
    \ifnum\value{MInfoNumbers}=0\begin{exmpc}\else\begin{exmp}\fi
    \MTB
    \begin{exmpshaded}
    \ \newline
}{
    \end{exmpshaded}
    \ifnum\value{MInfoNumbers}=0\end{exmpc}\else\end{exmp}\fi
}
  \newenvironment{MChemInfo}{\begin{infoshaded}}{\end{infoshaded}}

  \newenvironment{MInfo}{\ifnum\value{MInfoNumbers}=0\begin{MChemInfo}\else\renewcommand{\MInfoText}{Info}\begin{info}\begin{infoshaded}
  \MTB
   \ \newline
    \fi
  }{\ifnum\value{MInfoNumbers}=0\end{MChemInfo}\else\end{infoshaded}\end{info}\fi}

  \newenvironment{MXInfo}[1]{
    \renewcommand{\MInfoText}{#1}
    \ifnum\value{MInfoNumbers}=0\begin{infoc}\else\begin{info}\fi%
    \MTB
    \begin{infoshaded}
    \ \newline
  }{\end{infoshaded}\ifnum\value{MInfoNumbers}=0\end{infoc}\else\end{info}\fi}

  \newenvironment{MExperiment}{
    \renewcommand{\MInfoText}{Versuch}
    \ifnum\value{MInfoNumbers}=0\begin{verc}\else\begin{info}\fi
    \MTB
    \begin{expeshaded}
    \ \newline
  }{
    \end{expeshaded}
    \ifnum\value{MInfoNumbers}=0\end{verc}\else\end{info}\fi
  }
\fi%

% MHint sollte nicht direkt fuer Loesungen benutzt werden wegen solutionselect
\newenvironment{MSolution}{\begin{MHint}{L"osung}}{\end{MHint}}

\newcounter{MCodeCounter}

\ifttm
\newenvironment{MCode}{\special{html:<!-- mcodestart -->}\ttfamily\color{blue}}{\special{html:<!-- mcodestop -->}}
\else
\newenvironment{MCode}{\begin{flushleft}\ttfamily\addtocounter{MCodeCounter}{1}}{\addtocounter{MCodeCounter}{-1}\end{flushleft}}
% Ohne color-Statement da inkompatible mit framed/shaded-Boxen aus dem framed-package
\fi

%----------------- Sonderdefinitionen fuer Symbole, die der Konverter nicht kann ----------------------------------------------

\ifttm%
\newcommand{\MUnderset}[2]{\underbrace{#2}_{#1}}%
\else%
\newcommand{\MUnderset}[2]{\underset{#1}{#2}}%
\fi%

\ifttm
\newcommand{\MThinspace}{\special{html:<mi>&#x2009;</mi>}}
\else
\newcommand{\MThinspace}{\,}
\fi

\ifttm
\newcommand{\glq}{\begin{html}&sbquo;\end{html}}
\newcommand{\grq}{\begin{html}&lsquo;\end{html}}
\newcommand{\glqq}{\begin{html}&bdquo;\end{html}}
\newcommand{\grqq}{\begin{html}&ldquo;\end{html}}
\fi

\ifttm
\newcommand{\MNdash}{\begin{html}&ndash;\end{html}}
\else
\newcommand{\MNdash}{--}
\fi

%\ifttm\def\MIU{\special{html:<mi>&#8520;</mi>}}\else\def\MIU{\mathrm{i}}\fi
\def\MIU{\mathrm{i}}
\def\MEU{e} % TU9-Onlinekurs: italic-e
%\def\MEU{\mathrm{e}} % Alte Onlinemodule: roman-e
\def\MD{d} % Kursives d in Integralen im TU9-Onlinekurs
%\def\MD{\mathrm{d}} % roman-d in den alten Onlinemodulen
\def\MDB{\|}

%zusaetzlicher Leerraum vor "\MD"
\ifttm%
\def\MDSpace{\special{html:<mi>&#x2009;</mi>}}
\else%
\def\MDSpace{\,}
\fi%
\newcommand{\MDwSp}{\MDSpace\MD}%

\ifttm
\def\Mdq{\dq}
\else
\def\Mdq{\dq}
\fi

\def\MSpan#1{\left<{#1}\right>}
\def\MSetminus{\setminus}
\def\MIM{I}

\ifttm
\newcommand{\ld}{\text{ld}}
\newcommand{\lg}{\text{lg}}
\else
\DeclareMathOperator{\ld}{ld}
%\newcommand{\lg}{\text{lg}} % in latex schon definiert
\fi


\def\Mmapsto{\ifttm\special{html:<mi>&mapsto;</mi>}\else\mapsto\fi} 
\def\Mvarphi{\ifttm\phi\else\varphi\fi}
\def\Mphi{\ifttm\varphi\else\phi\fi}
\ifttm%
\newcommand{\MEumu}{\special{html:<mi>&#x3BC;</mi>}}%
\else%
\newcommand{\MEumu}{\textrm{\textmu}}%
\fi
\def\Mvarepsilon{\ifttm\epsilon\else\varepsilon\fi}
\def\Mepsilon{\ifttm\varepsilon\else\epsilon\fi}
\def\Mvarkappa{\ifttm\kappa\else\varkappa\fi}
\def\Mkappa{\ifttm\varkappa\else\kappa\fi}
\def\Mcomplement{\ifttm\special{html:<mi>&comp;</mi>}\else\complement\fi} 
\def\MWW{\mathrm{WW}}
\def\Mmod{\ifttm\special{html:<mi>&nbsp;mod&nbsp;</mi>}\else\mod\fi} 

\ifttm%
\def\mod{\text{\;mod\;}}%
\def\MNEquiv{\special{html:<mi>&NotCongruent;</mi>}}% 
\def\MNSubseteq{\special{html:<mi>&NotSubsetEqual;</mi>}}%
\def\MEmptyset{\special{html:<mi>&empty;</mi>}}%
\def\MVDots{\special{html:<mi>&#x22EE;</mi>}}%
\def\MHDots{\special{html:<mi>&#x2026;</mi>}}%
\def\Mddag{\special{html:<mi>&#x1202;</mi>}}%
\def\sphericalangle{\special{html:<mi>&measuredangle;</mi>}}%
\def\nparallel{\special{html:<mi>&nparallel;</mi>}}%
\def\MProofEnd{\special{html:<mi>&#x25FB;</mi>}}%
\newenvironment{MProof}[1]{\underline{#1}:\MCR\MCR}{\hfill $\MProofEnd$}%
\else%
\def\MNEquiv{\not\equiv}%
\def\MNSubseteq{\not\subseteq}%
\def\MEmptyset{\emptyset}%
\def\MVDots{\vdots}%
\def\MHDots{\hdots}%
\def\Mddag{\ddag}%
\newenvironment{MProof}[1]{\begin{proof}[#1]}{\end{proof}}%
\fi%



% Spaces zum Auffuellen von Tabellenbreiten, die nur im HTML wirken
\ifttm%
\def\MTSP{\:}%
\else%
\def\MTSP{}%
\fi%

\DeclareMathOperator{\arsinh}{arsinh}
\DeclareMathOperator{\arcosh}{arcosh}
\DeclareMathOperator{\artanh}{artanh}
\DeclareMathOperator{\arcoth}{arcoth}


\newcommand{\MMathSet}[1]{\mathbb{#1}}
\def\N{\MMathSet{N}}
\def\Z{\MMathSet{Z}}
\def\Q{\MMathSet{Q}}
\def\R{\MMathSet{R}}
\def\C{\MMathSet{C}}

\newcounter{MForLoopCounter}
\newcommand{\MForLoop}[2]{\setcounter{MForLoopCounter}{#1}\ifnum\value{MForLoopCounter}=0{}\else{{#2}\addtocounter{MForLoopCounter}{-1}\MForLoop{\value{MForLoopCounter}}{#2}}\fi}

\newcounter{MSiteCounter}
\newcounter{MFieldCounter} % Kombination section.subsection.site.field ist eindeutig in allen Modulen, field alleine nicht

\newcounter{MiniMarkerCounter}

\ifttm
\newenvironment{MMiniPageP}[1]{\begin{minipage}{#1\linewidth}\special{html:<!-- minimarker;;}\arabic{MiniMarkerCounter}\special{html:;;#1; //-->}}{\end{minipage}\addtocounter{MiniMarkerCounter}{1}}
\else
\newenvironment{MMiniPageP}[1]{\begin{minipage}{#1\linewidth}}{\end{minipage}\addtocounter{MiniMarkerCounter}{1}}
\fi

\newcounter{AlignCounter}

\newcommand{\MStartJustify}{\ifttm\special{html:<!-- startalign;;}\arabic{AlignCounter}\special{html:;;justify; //-->}\fi}
\newcommand{\MStopJustify}{\ifttm\special{html:<!-- stopalign;;}\arabic{AlignCounter}\special{html:; //-->}\fi\addtocounter{AlignCounter}{1}}

\newenvironment{MJTabular}[1]{
\MStartJustify
\begin{tabular}{#1}
}{
\end{tabular}
\MStopJustify
}

\newcommand{\MImageLeft}[2]{
\begin{center}
\begin{tabular}{lc}
\MStartJustify
\begin{MMiniPageP}{0.65}
#1
\end{MMiniPageP}
\MStopJustify
&
\begin{MMiniPageP}{0.3}
#2  
\end{MMiniPageP}
\end{tabular}
\end{center}
}

\newcommand{\MImageHalf}[2]{
\begin{center}
\begin{tabular}{lc}
\MStartJustify
\begin{MMiniPageP}{0.45}
#1
\end{MMiniPageP}
\MStopJustify
&
\begin{MMiniPageP}{0.45}
#2  
\end{MMiniPageP}
\end{tabular}
\end{center}
}

\newcommand{\MBigImageLeft}[2]{
\begin{center}
\begin{tabular}{lc}
\MStartJustify
\begin{MMiniPageP}{0.25}
#1
\end{MMiniPageP}
\MStopJustify
&
\begin{MMiniPageP}{0.7}
#2  
\end{MMiniPageP}
\end{tabular}
\end{center}
}

\ifttm
\def\No{\mathbb{N}_0}
\else
\def\No{\ensuremath{\N_0}}
\fi
\def\MT{\textrm{\tiny T}}
\newcommand{\MTranspose}[1]{{#1}^{\MT}}
\ifttm
\newcommand{\MRe}{\mathsf{Re}}
\newcommand{\MIm}{\mathsf{Im}}
\else
\DeclareMathOperator{\MRe}{Re}
\DeclareMathOperator{\MIm}{Im}
\fi

\newcommand{\Mid}{\mathrm{id}}
\newcommand{\MFeinheit}{\mathrm{feinh}}

\ifttm
\newcommand{\Msubstack}[1]{\begin{array}{c}{#1}\end{array}}
\else
\newcommand{\Msubstack}[1]{\substack{#1}}
\fi

% Typen von Fragefeldern:
% 1 = Alphanumerisch, case-sensitive-Vergleich
% 2 = Ja/Nein-Checkbox, Loesung ist 0 oder 1   (OPTION = Image-id fuer Rueckmeldung)
% 3 = Reelle Zahlen Geparset
% 4 = Funktionen Geparset (mit Stuetzstellen zur ueberpruefung)

% Dieser Befehl erstellt ein interaktives Aufgabenfeld. Parameter:
% - #1 Laenge in Zeichen
% - #2 Loesungstext (alphanumerisch, case sensitive)
% - #3 AufgabenID (alphanumerisch, case sensitive)
% - #4 Typ (Kennnummer)
% - #5 String fuer Optionen (ggf. mit Semikolon getrennte Einzelstrings)
% - #6 Anzahl Punkte
% - #7 uxid (kann z.B. Loesungsstring sein)
% ACHTUNG: Die langen Zeilen bitte so lassen, Zeilenumbrueche im tex werden in div's umgesetzt
\newcommand{\MQuestionID}[7]{
\ifttm
\special{html:<!-- mdeclareuxid;;}UX#7\special{html:;;}\arabic{section}\special{html:;;}#3\special{html:;; //-->}%
\special{html:<!-- mdeclarepoints;;}\arabic{section}\special{html:;;}#3\special{html:;;}#6\special{html:;;}\arabic{MTestSite}\special{html:;;}\arabic{chapter}%
\special{html:;; //--><!-- onloadstart //-->CreateQuestionObj("}#7\special{html:",}\arabic{MFieldCounter}\special{html:,"}#2%
\special{html:","}#3\special{html:",}#4\special{html:,"}#5\special{html:",}#6\special{html:,}\arabic{MTestSite}\special{html:,}\arabic{section}%
\special{html:);<!-- onloadstop //-->}%
\special{html:<input mfieldtype="}#4\special{html:" name="Name_}#3\special{html:" id="}#3\special{html:" type="text" size="}#1\special{html:" maxlength="}#1%
\special{html:" }\ifnum\value{MGroupActive}=0\special{html:onfocus="handlerFocus(}\arabic{MFieldCounter}%
\special{html:);" onblur="handlerBlur(}\arabic{MFieldCounter}\special{html:);" onkeyup="handlerChange(}\arabic{MFieldCounter}\special{html:,0);" onpaste="handlerChange(}\arabic{MFieldCounter}\special{html:,0);" oninput="handlerChange(}\arabic{MFieldCounter}\special{html:,0);" onpropertychange="handlerChange(}\arabic{MFieldCounter}\special{html:,0);"/>}%
\special{html:<img src="images/questionmark.gif" width="20" height="20" border="0" align="absmiddle" id="}QM#3\special{html:"/>}
\else%
\special{html:onblur="handlerBlur(}\arabic{MFieldCounter}%
\special{html:);" onfocus="handlerFocus(}\arabic{MFieldCounter}\special{html:);" onkeyup="handlerChange(}\arabic{MFieldCounter}\special{html:,1);" onpaste="handlerChange(}\arabic{MFieldCounter}\special{html:,1);" oninput="handlerChange(}\arabic{MFieldCounter}\special{html:,1);" onpropertychange="handlerChange(}\arabic{MFieldCounter}\special{html:,1);"/>}%
\special{html:<img src="images/questionmark.gif" width="20" height="20" border="0" align="absmiddle" id="}QM#3\special{html:"/>}\fi%
\else%
\ifnum\value{QBoxFlag}=1\fbox{$\phantom{\MForLoop{#1}{b}}$}\else$\phantom{\MForLoop{#1}{b}}$\fi%
\fi%
}

% ACHTUNG: Die langen Zeilen bitte so lassen, Zeilenumbrueche im tex werden in div's umgesetzt
% QuestionCheckbox macht ausserhalb einer QuestionGroup keinen Sinn!
% #1 = solution (1 oder 0), ggf. mit ::smc abgetrennt auszuschliessende single-choice-boxen (UXIDs durch , getrennt), #2 = id, #3 = points, #4 = uxid
\newcommand{\MQuestionCheckbox}[4]{
\ifttm
\special{html:<!-- mdeclareuxid;;}UX#4\special{html:;;}\arabic{section}\special{html:;;}#2\special{html:;; //-->}%
\ifnum\value{MGroupActive}=0\MDebugMessage{ERROR: Checkbox Nr. \arabic{MFieldCounter}\ ist nicht in einer Kontrollgruppe, es wird niemals eine Loesung angezeigt!}\fi
\special{html: %
<!-- mdeclarepoints;;}\arabic{section}\special{html:;;}#2\special{html:;;}#3\special{html:;;}\arabic{MTestSite}\special{html:;;}\arabic{chapter}%
\special{html:;; //--><!-- onloadstart //-->CreateQuestionObj("}#4\special{html:",}\arabic{MFieldCounter}\special{html:,"}#1\special{html:","}#2\special{html:",2,"IMG}#2%
\special{html:",}#3\special{html:,}\arabic{MTestSite}\special{html:,}\arabic{section}\special{html:);<!-- onloadstop //-->}%
\special{html:<input mfieldtype="2" type="checkbox" name="Name_}#2\special{html:" id="}#2\special{html:" onchange="handlerChange(}\arabic{MFieldCounter}\special{html:,1);"/><img src="images/questionmark.gif" name="}Name_IMG#2%
\special{html:" width="20" height="20" border="0" align="absmiddle" id="}IMG#2\special{html:"/> }%
\else%
\ifnum\value{QBoxFlag}=1\fbox{$\phantom{X}$}\else$\phantom{X}$\fi%
\fi%
}

\def\MGenerateID{QFELD_\arabic{section}.\arabic{subsection}.\arabic{MSiteCounter}.QF\arabic{MFieldCounter}}

% #1 = 0/1 ggf. mit ::smc abgetrennt auszuschliessende single-choice-boxen (UXIDs durch , getrennt ohne UX), #2 = uxid ohne UX
\newcommand{\MCheckbox}[2]{
\MQuestionCheckbox{#1}{\MGenerateID}{\MStdPoints}{#2}
\addtocounter{MFieldCounter}{1}
}

% Erster Parameter: Zeichenlaenge der Eingabebox, zweiter Parameter: Loesungstext
\newcommand{\MQuestion}[2]{
\MQuestionID{#1}{#2}{\MGenerateID}{1}{0}{\MStdPoints}{#2}
\addtocounter{MFieldCounter}{1}
}

% Erster Parameter: Zeichenlaenge der Eingabebox, zweiter Parameter: Loesungstext
\newcommand{\MLQuestion}[3]{
\MQuestionID{#1}{#2}{\MGenerateID}{1}{0}{\MStdPoints}{#3}
\addtocounter{MFieldCounter}{1}
}

% Parameter: Laenge des Feldes, Loesung (wird auch geparsed), Stellen Genauigkeit hinter dem Komma, weitere Stellen werden mathematisch gerundet vor Vergleich
\newcommand{\MParsedQuestion}[3]{
\MQuestionID{#1}{#2}{\MGenerateID}{3}{#3}{\MStdPoints}{#2}
\addtocounter{MFieldCounter}{1}
}

% Parameter: Laenge des Feldes, Loesung (wird auch geparsed), Stellen Genauigkeit hinter dem Komma, weitere Stellen werden mathematisch gerundet vor Vergleich
\newcommand{\MLParsedQuestion}[4]{
\MQuestionID{#1}{#2}{\MGenerateID}{3}{#3}{\MStdPoints}{#4}
\addtocounter{MFieldCounter}{1}
}

% Parameter: Laenge des Feldes, Loesungsfunktion, Anzahl Stuetzstellen, Funktionsvariablen durch Kommata getrennt (nicht case-sensitive), Anzahl Nachkommastellen im Vergleich
\newcommand{\MFunctionQuestion}[5]{
\MQuestionID{#1}{#2}{\MGenerateID}{4}{#3;#4;#5;0}{\MStdPoints}{#2}
\addtocounter{MFieldCounter}{1}
}

% Parameter: Laenge des Feldes, Loesungsfunktion, Anzahl Stuetzstellen, Funktionsvariablen durch Kommata getrennt (nicht case-sensitive), Anzahl Nachkommastellen im Vergleich, UXID
\newcommand{\MLFunctionQuestion}[6]{
\MQuestionID{#1}{#2}{\MGenerateID}{4}{#3;#4;#5;0}{\MStdPoints}{#6}
\addtocounter{MFieldCounter}{1}
}

% Parameter: Laenge des Feldes, Loesungsintervall, Genauigkeit der Zahlenwertpruefung
\newcommand{\MIntervalQuestion}[3]{
\MQuestionID{#1}{#2}{\MGenerateID}{6}{#3}{\MStdPoints}{#2}
\addtocounter{MFieldCounter}{1}
}

% Parameter: Laenge des Feldes, Loesungsintervall, Genauigkeit der Zahlenwertpruefung, UXID
\newcommand{\MLIntervalQuestion}[4]{
\MQuestionID{#1}{#2}{\MGenerateID}{6}{#3}{\MStdPoints}{#4}
\addtocounter{MFieldCounter}{1}
}

% Parameter: Laenge des Feldes, Loesungsfunktion, Anzahl Stuetzstellen, Funktionsvariable (nicht case-sensitive), Anzahl Nachkommastellen im Vergleich, Vereinfachungsbedingung
% Vereinfachungsbedingung ist eine der Folgenden:
% 0 = Keine Vereinfachungsbedingung
% 1 = Keine Klammern (runde oder eckige) mehr im vereinfachten Ausdruck
% 2 = Faktordarstellung (Term hat Produkte als letzte Operation, Summen als vorgeschaltete Operation)
% 3 = Summendarstellung (Term hat Summen als letzte Operation, Produkte als vorgeschaltete Operation)
% Flag 512: Besondere Stuetzstellen (nur >1 und nur schwach rational), sonst symmetrisch um Nullpunkt und ganze Zahlen inkl. Null werden getroffen
\newcommand{\MSimplifyQuestion}[6]{
\MQuestionID{#1}{#2}{\MGenerateID}{4}{#3;#4;#5;#6}{\MStdPoints}{#2}
\addtocounter{MFieldCounter}{1}
}

\newcommand{\MLSimplifyQuestion}[7]{
\MQuestionID{#1}{#2}{\MGenerateID}{4}{#3;#4;#5;#6}{\MStdPoints}{#7}
\addtocounter{MFieldCounter}{1}
}

% Parameter: Laenge des Feldes, Loesung (optionaler Ausdruck), Anzahl Stuetzstellen, Funktionsvariable (nicht case-sensitive), Anzahl Nachkommastellen im Vergleich, Spezialtyp (string-id)
\newcommand{\MLSpecialQuestion}[7]{
\MQuestionID{#1}{#2}{\MGenerateID}{7}{#3;#4;#5;#6}{\MStdPoints}{#7}
\addtocounter{MFieldCounter}{1}
}

\newcounter{MGroupStart}
\newcounter{MGroupEnd}
\newcounter{MGroupActive}

\newenvironment{MQuestionGroup}{
\setcounter{MGroupStart}{\value{MFieldCounter}}
\setcounter{MGroupActive}{1}
}{
\setcounter{MGroupActive}{0}
\setcounter{MGroupEnd}{\value{MFieldCounter}}
\addtocounter{MGroupEnd}{-1}
}

\newcommand{\MGroupButton}[1]{
\ifttm
\special{html:<button name="Name_Group}\arabic{MGroupStart}\special{html:to}\arabic{MGroupEnd}\special{html:" id="Group}\arabic{MGroupStart}\special{html:to}\arabic{MGroupEnd}\special{html:" %
type="button" onclick="group_button(}\arabic{MGroupStart}\special{html:,}\arabic{MGroupEnd}\special{html:);">}#1\special{html:</button>}
\else
\phantom{#1}
\fi
}

%----------------- Makros fuer die modularisierte Darstellung ------------------------------------

\def\MyText#1{#1}

% is used internally by the conversion package, should not be used by original tex documents
\def\MOrgLabel#1{\relax}

\ifttm

% Ein MLabel wird im html codiert durch das tag <!-- mmlabel;;Labelbezeichner;;SubjectArea;;chapter;;section;;subsection;;Index;;Objekttyp; //-->
\def\MLabel#1{%
\ifnum\value{MLastType}=8%
\ifnum\value{MCaptionOn}=0%
\MDebugMessage{ERROR: Grafik \arabic{MGraphicsCounter} hat separates label: #1 (Grafiklabels sollten nur in der Caption stehen)}%
\fi
\fi
\ifnum\value{MLastType}=12%
\ifnum\value{MCaptionOn}=0%
\MDebugMessage{ERROR: Video \arabic{MVideoCounter} hat separates label: #1 (Videolabels sollten nur in der Caption stehen}%
\fi
\fi
\ifnum\value{MLastType}=10\setcounter{MLastIndex}{\value{equation}}\fi
\label{#1}\begin{html}<!-- mmlabel;;#1;;\end{html}\arabic{MSubjectArea}\special{html:;;}\arabic{chapter}\special{html:;;}\arabic{section}\special{html:;;}\arabic{subsection}\special{html:;;}\arabic{MLastIndex}\special{html:;;}\arabic{MLastType}\special{html:; //-->}}%

\else

% Sonderbehandlung im PDF fuer Abbildungen in separater aux-Datei, da MGraphics die figure-Umgebung nicht verwendet
\def\MLabel#1{%
\ifnum\value{MLastType}=8%
\ifnum\value{MCaptionOn}=0%
\MDebugMessage{ERROR: Grafik \arabic{MGraphicsCounter} hat separates label: #1 (Grafiklabels sollten nur in der Caption stehen}%
\fi
\fi
\ifnum\value{MLastType}=12%
\ifnum\value{MCaptionOn}=0%
\MDebugMessage{ERROR: Video \arabic{MVideoCounter} hat separates label: #1 (Videolabels sollten nur in der Caption stehen}%
\fi
\fi
\label{#1}%
}%

\fi

% Gibt Begriff des referenzierten Objekts mit aus, aber nur im HTML, daher nur in Ausnahmefaellen (z.B. Copyrightliste) sinnvoll
\def\MCRef#1{\ifttm\special{html:<!-- mmref;;}#1\special{html:;;1; //-->}\else\vref{#1}\fi}


\def\MRef#1{\ifttm\special{html:<!-- mmref;;}#1\special{html:;;0; //-->}\else\vref{#1}\fi}
\def\MERef#1{\ifttm\special{html:<!-- mmref;;}#1\special{html:;;0; //-->}\else\eqref{#1}\fi}
\def\MNRef#1{\ifttm\special{html:<!-- mmref;;}#1\special{html:;;0; //-->}\else\ref{#1}\fi}
\def\MSRef#1#2{\ifttm\special{html:<!-- msref;;}#1\special{html:;;}#2\special{html:; //-->}\else \if#2\empty \ref{#1} \else \hyperref[#1]{#2}\fi\fi} 

\def\MRefRange#1#2{\ifttm\MRef{#1} bis 
\MRef{#2}\else\vrefrange[\unskip]{#1}{#2}\fi}

\def\MRefTwo#1#2{\ifttm\MRef{#1} und \MRef{#2}\else%
\let\vRefTLRsav=\reftextlabelrange\let\vRefTPRsav=\reftextpagerange%
\def\reftextlabelrange##1##2{\ref{##1} und~\ref{##2}}%
\def\reftextpagerange##1##2{auf den Seiten~\pageref{#1} und~\pageref{#2}}%
\vrefrange[\unskip]{#1}{#2}%
\let\reftextlabelrange=\vRefTLRsav\let\reftextpagerange=\vRefTPRsav\fi}

% MSectionChapter definiert falls notwendig das Kapitel vor der section. Das ist notwendig, wenn nur ein Einzelmodul uebersetzt wird.
% MChaptersGiven ist ein Counter, der von mconvert.pl vordefiniert wird.
\ifttm
\newcommand{\MSectionChapter}{\ifnum\value{MChaptersGiven}=0{\Dchapter{Modul}}\else{}\fi}
\else
\newcommand{\MSectionChapter}{\ifnum\value{chapter}=0{\Dchapter{Modul}}\else{}\fi}
\fi


\def\MChapter#1{\ifnum\value{MSSEnd}>0{\MSubsectionEndMacros}\addtocounter{MSSEnd}{-1}\fi\Dchapter{#1}}
\def\MSubject#1{\MChapter{#1}} % Schluesselwort HELPSECTION ist reserviert fuer Hilfesektion

\newcommand{\MSectionID}{UNKNOWNID}

\ifttm
\newcommand{\MSetSectionID}[1]{\renewcommand{\MSectionID}{#1}}
\else
\newcommand{\MSetSectionID}[1]{\renewcommand{\MSectionID}{#1}\tikzsetexternalprefix{#1}}
\fi


\newcommand{\MSection}[1]{\MSetSectionID{MODULID}\ifnum\value{MSSEnd}>0{\MSubsectionEndMacros}\addtocounter{MSSEnd}{-1}\fi\MSectionChapter\Dsection{#1}\MSectionStartMacros{#1}\setcounter{MLastIndex}{-1}\setcounter{MLastType}{1}} % Sections werden ueber das section-Feld im mmlabel-Tag identifiziert, nicht ueber das Indexfeld

\def\MSubsection#1{\ifnum\value{MSSEnd}>0{\MSubsectionEndMacros}\addtocounter{MSSEnd}{-1}\fi\ifttm\else\clearpage\fi\Dsubsection{#1}\MSubsectionStartMacros\setcounter{MLastIndex}{-1}\setcounter{MLastType}{2}\addtocounter{MSSEnd}{1}}% Subsections werden ueber das subsection-Feld im mmlabel-Tag identifiziert, nicht ueber das Indexfeld
\def\MSubsectionx#1{\Dsubsectionx{#1}} % Nur zur Verwendung in MSectionStart gedacht
\def\MSubsubsection#1{\Dsubsubsection{#1}\setcounter{MLastIndex}{\value{subsubsection}}\setcounter{MLastType}{3}\ifttm\special{html:<!-- sectioninfo;;}\arabic{section}\special{html:;;}\arabic{subsection}\special{html:;;}\arabic{subsubsection}\special{html:;;1;;}\arabic{MTestSite}\special{html:; //-->}\fi}
\def\MSubsubsectionx#1{\Dsubsubsectionx{#1}\ifttm\special{html:<!-- sectioninfo;;}\arabic{section}\special{html:;;}\arabic{subsection}\special{html:;;}\arabic{subsubsection}\special{html:;;0;;}\arabic{MTestSite}\special{html:; //-->}\else\addcontentsline{toc}{subsection}{#1}\fi}

\ifttm
\def\MSubsubsubsectionx#1{\ \newline\textbf{#1}\special{html:<br />}}
\else
\def\MSubsubsubsectionx#1{\ \newline
\textbf{#1}\ \\
}
\fi


% Dieses Skript wird zu Beginn jedes Modulabschnitts (=Webseite) ausgefuehrt und initialisiert den Aufgabenfeldzaehler
\newcommand{\MPageScripts}{
\setcounter{MFieldCounter}{1}
\addtocounter{MSiteCounter}{1}
\setcounter{MHintCounter}{1}
\setcounter{MCodeEditCounter}{1}
\setcounter{MGroupActive}{0}
\DoQBoxes
% Feldvariablen werden im HTML-Header in conv.pl eingestellt
}

% Dieses Skript wird zum Ende jedes Modulabschnitts (=Webseite) ausgefuehrt
\ifttm
\newcommand{\MEndScripts}{\special{html:<br /><!-- mfeedbackbutton;Seite;}\arabic{MTestSite}\special{html:;}\MGenerateSiteNumber\special{html:; //-->}
}
\else
\newcommand{\MEndScripts}{\relax}
\fi


\newcounter{QBoxFlag}
\newcommand{\DoQBoxes}{\setcounter{QBoxFlag}{1}}
\newcommand{\NoQBoxes}{\setcounter{QBoxFlag}{0}}

\newcounter{MXCTest}
\newcounter{MXCounter}
\newcounter{MSCounter}



\ifttm

% Struktur des sectioninfo-Tags: <!-- sectioninfo;;section;;subsection;;subsubsection;;nr_ausgeben;;testpage; //-->

%Fuegt eine zusaetzliche html-Seite an hinter ALLEN bisherigen und zukuenftigen content-Seiten ausserhalb der vor-zurueck-Schleife (d.h. nur durch Button oder MIntLink erreichbar!)
% #1 = Titel des Modulabschnitts, #2 = Kurztitel fuer die Buttons, #3 = Buttonkennung (STD = default nehmen, NONE = Ohne Button in der Navigation)
\newenvironment{MSContent}[3]{\special{html:<div class="xcontent}\arabic{MSCounter}\special{html:"><!-- scontent;-;}\arabic{MSCounter};-;#1;-;#2;-;#3\special{html: //-->}\MPageScripts\MSubsubsectionx{#1}}{\MEndScripts\special{html:<!-- endscontent;;}\arabic{MSCounter}\special{html: //--></div>}\addtocounter{MSCounter}{1}}

% Fuegt eine zusaetzliche html-Seite ein hinter den bereits vorhandenen content-Seiten (oder als erste Seite) innerhalb der vor-zurueck-Schleife der Navigation
% #1 = Titel des Modulabschnitts, #2 = Kurztitel fuer die Buttons, #3 = Buttonkennung (STD = Defaultbutton, NONE = Ohne Button in der Navigation)
\newenvironment{MXContent}[3]{\special{html:<div class="xcontent}\arabic{MXCounter}\special{html:"><!-- xcontent;-;}\arabic{MXCounter};-;#1;-;#2;-;#3\special{html: //-->}\MPageScripts\MSubsubsection{#1}}{\MEndScripts\special{html:<!-- endxcontent;;}\arabic{MXCounter}\special{html: //--></div>}\addtocounter{MXCounter}{1}}

% Fuegt eine zusaetzliche html-Seite ein die keine subsubsection-Nummer bekommt, nur zur internen Verwendung in mintmod.tex gedacht!
% #1 = Titel des Modulabschnitts, #2 = Kurztitel fuer die Buttons, #3 = Buttonkennung (STD = Defaultbutton, NONE = Ohne Button in der Navigation)
% \newenvironment{MUContent}[3]{\special{html:<div class="xcontent}\arabic{MXCounter}\special{html:"><!-- xcontent;-;}\arabic{MXCounter};-;#1;-;#2;-;#3\special{html: //-->}\MPageScripts\MSubsubsectionx{#1}}{\MEndScripts\special{html:<!-- endxcontent;;}\arabic{MXCounter}\special{html: //--></div>}\addtocounter{MXCounter}{1}}

\newcommand{\MDeclareSiteUXID}[1]{\special{html:<!-- mdeclaresiteuxid;;}#1\special{html:;;}\arabic{chapter}\special{html:;;}\arabic{section}\special{html:;; //-->}}

\else

%\newcommand{\MSubsubsection}[1]{\refstepcounter{subsubsection} \addcontentsline{toc}{subsubsection}{\thesubsubsection. #1}}


% Fuegt eine zusaetzliche html-Seite an hinter den bereits vorhandenen content-Seiten
% #1 = Titel des Modulabschnitts, #2 = Kurztitel fuer die Buttons, #3 = Iconkennung (im PDF wirkungslos)
%\newenvironment{MUContent}[3]{\ifnum\value{MXCTest}>0{\MDebugMessage{ERROR: Geschachtelter SContent}}\fi\MPageScripts\MSubsubsectionx{#1}\addtocounter{MXCTest}{1}}{\addtocounter{MXCounter}{1}\addtocounter{MXCTest}{-1}}
\newenvironment{MXContent}[3]{\ifnum\value{MXCTest}>0{\MDebugMessage{ERROR: Geschachtelter SContent}}\fi\MPageScripts\MSubsubsection{#1}\addtocounter{MXCTest}{1}}{\addtocounter{MXCounter}{1}\addtocounter{MXCTest}{-1}}
\newenvironment{MSContent}[3]{\ifnum\value{MXCTest}>0{\MDebugMessage{ERROR: Geschachtelter XContent}}\fi\MPageScripts\MSubsubsectionx{#1}\addtocounter{MXCTest}{1}}{\addtocounter{MSCounter}{1}\addtocounter{MXCTest}{-1}}

\newcommand{\MDeclareSiteUXID}[1]{\relax}

\fi 

% GHEADER und GFOOTER werden von split.pm gefunden, aber nur, wenn nicht HELPSITE oder TESTSITE
\ifttm
\newenvironment{MSectionStart}{\special{html:<div class="xcontent0">}\MSubsubsectionx{Modul\"ubersicht}}{\setcounter{MSSEnd}{0}\special{html:</div>}}
% Darf nicht als XContent nummeriert werden, darf nicht als XContent gelabelt werden, wird aber in eine xcontent-div gesetzt fuer Python-parsing
\else
\newenvironment{MSectionStart}{\MSubsectionx{Modul\"ubersicht}}{\setcounter{MSSEnd}{0}}
\fi

\newenvironment{MIntro}{\begin{MXContent}{Einf\"uhrung}{Einf\"uhrung}{genetisch}}{\end{MXContent}}
\newenvironment{MContent}{\begin{MXContent}{Inhalt}{Inhalt}{beweis}}{\end{MXContent}}
\newenvironment{MExercises}{\ifttm\else\clearpage\fi\begin{MXContent}{Aufgaben}{Aufgaben}{aufgb}\special{html:<!-- declareexcsymb //-->}}{\end{MXContent}}

% #1 = Lesbare Testbezeichnung
\newenvironment{MTest}[1]{%
\renewcommand{\MTestName}{#1}
\ifttm\else\clearpage\fi%
\addtocounter{MTestSite}{1}%
\begin{MXContent}{#1}{#1}{STD} % {aufgb}%
\special{html:<!-- declaretestsymb //-->}
\begin{MQuestionGroup}%
\MInTestHeader
}%
{%
\end{MQuestionGroup}%
\ \\ \ \\%
\MInTestFooter
\end{MXContent}\addtocounter{MTestSite}{-1}%
}

\newenvironment{MExtra}{\ifttm\else\clearpage\fi\begin{MXContent}{Zus\"atzliche Inhalte}{Zusatz}{weiterfhrg}}{\end{MXContent}}

\makeindex

\ifttm
\def\MPrintIndex{
\ifnum\value{MSSEnd}>0{\MSubsectionEndMacros}\addtocounter{MSSEnd}{-1}\fi
\renewcommand{\indexname}{Stichwortverzeichnis}
\special{html:<p><!-- printindex //--></p>}
}
\else
\def\MPrintIndex{
\ifnum\value{MSSEnd}>0{\MSubsectionEndMacros}\addtocounter{MSSEnd}{-1}\fi
\renewcommand{\indexname}{Stichwortverzeichnis}
\addcontentsline{toc}{section}{Stichwortverzeichnis}
\printindex
}
\fi


% Konstanten fuer die Modulfaecher

\def\MINTMathematics{1}
\def\MINTInformatics{2}
\def\MINTChemistry{3}
\def\MINTPhysics{4}
\def\MINTEngineering{5}

\newcounter{MSubjectArea}
\newcounter{MInfoNumbers} % Gibt an, ob die Infoboxen nummeriert werden sollen
\newcounter{MSepNumbers} % Gibt an, ob Beispiele und Experimente separat nummeriert werden sollen
\newcommand{\MSetSubject}[1]{
 % ttm kapiert setcounter mit Parametern nicht, also per if abragen und einsetzen
\ifnum#1=1\setcounter{MSubjectArea}{1}\setcounter{MInfoNumbers}{1}\setcounter{MSepNumbers}{0}\fi
\ifnum#1=2\setcounter{MSubjectArea}{2}\setcounter{MInfoNumbers}{1}\setcounter{MSepNumbers}{0}\fi
\ifnum#1=3\setcounter{MSubjectArea}{3}\setcounter{MInfoNumbers}{0}\setcounter{MSepNumbers}{1}\fi
\ifnum#1=4\setcounter{MSubjectArea}{4}\setcounter{MInfoNumbers}{0}\setcounter{MSepNumbers}{0}\fi
\ifnum#1=5\setcounter{MSubjectArea}{5}\setcounter{MInfoNumbers}{1}\setcounter{MSepNumbers}{0}\fi
% Separate Nummerntechnik fuer unsere Chemiker: alles dreistellig
\ifnum#1=3
  \ifttm
  \renewcommand{\theequation}{\arabic{section}.\arabic{subsection}.\arabic{equation}}
  \renewcommand{\thetable}{\arabic{section}.\arabic{subsection}.\arabic{table}} 
  \renewcommand{\thefigure}{\arabic{section}.\arabic{subsection}.\arabic{figure}} 
  \else
  \renewcommand{\theequation}{\arabic{chapter}.\arabic{section}.\arabic{equation}}
  \renewcommand{\thetable}{\arabic{chapter}.\arabic{section}.\arabic{table}}
  \renewcommand{\thefigure}{\arabic{chapter}.\arabic{section}.\arabic{figure}}
  \fi
\else
  \ifttm
  \renewcommand{\theequation}{\arabic{section}.\arabic{subsection}.\arabic{equation}}
  \renewcommand{\thetable}{\arabic{table}}
  \renewcommand{\thefigure}{\arabic{figure}}
  \else
  \renewcommand{\theequation}{\arabic{chapter}.\arabic{section}.\arabic{equation}}
  \renewcommand{\thetable}{\arabic{table}}
  \renewcommand{\thefigure}{\arabic{figure}}
  \fi
\fi
}

% Fuer tikz Autogenerierung
\newcounter{MTIKZAutofilenumber}

% Spezielle Counter fuer die Bentz-Module
\newcounter{mycounter}
\newcounter{chemapplet}
\newcounter{physapplet}

\newcounter{MSSEnd} % Ist 1 falls ein MSubsection aktiv ist, der einen MSubsectionEndMacro-Aufruf verursacht
\newcounter{MFileNumber}
\def\MLastFile{\special{html:[[!-- mfileref;;}\arabic{MFileNumber}\special{html:; //--]]}}

% Vollstaendiger Pfad ist \MMaterial / \MLastFilePath / \MLastFileName    ==   \MMaterial / \MLastFile

% Wird nur bei kompletter Baum-Erstellung ausgefuehrt!
% #1 = Lesbare Modulbezeichnung
\newcommand{\MSectionStartMacros}[1]{
\setcounter{MTestSite}{0}
\setcounter{MCaptionOn}{0}
\setcounter{MLastTypeEq}{0}
\setcounter{MSSEnd}{0}
\setcounter{MFileNumber}{0} % Preinkrekement-Counter
\setcounter{MTIKZAutofilenumber}{0}
\setcounter{mycounter}{1}
\setcounter{physapplet}{1}
\setcounter{chemapplet}{0}
\ifttm
\special{html:<!-- mdeclaresection;;}\arabic{chapter}\special{html:;;}\arabic{section}\special{html:;;}#1\special{html:;; //-->}%
\else
\setcounter{thmc}{0}
\setcounter{exmpc}{0}
\setcounter{verc}{0}
\setcounter{infoc}{0}
\fi
\setcounter{MiniMarkerCounter}{1}
\setcounter{AlignCounter}{1}
\setcounter{MXCTest}{0}
\setcounter{MCodeCounter}{0}
\setcounter{MEntryCounter}{0}
}

% Wird immer ausgefuehrt
\newcommand{\MSubsectionStartMacros}{
\ifttm\else\MPageHeaderDef\fi
\MWatermarkSettings
\setcounter{MXCounter}{0}
\setcounter{MSCounter}{0}
\setcounter{MSiteCounter}{1}
\setcounter{MExerciseCollectionCounter}{0}
% Zaehler fuer das Labelsystem zuruecksetzen (prefix-Zaehler)
\setcounter{MInfoCounter}{0}
\setcounter{MExerciseCounter}{0}
\setcounter{MExampleCounter}{0}
\setcounter{MExperimentCounter}{0}
\setcounter{MGraphicsCounter}{0}
\setcounter{MTableCounter}{0}
\setcounter{MTheoremCounter}{0}
\setcounter{MObjectCounter}{0}
\setcounter{MEquationCounter}{0}
\setcounter{MVideoCounter}{0}
\setcounter{equation}{0}
\setcounter{figure}{0}
}

\newcommand{\MSubsectionEndMacros}{
% Bei Chemiemodulen das PSE einhaengen, es soll als SContent am Ende erscheinen
\special{html:<!-- subsectionend //-->}
\ifnum\value{MSubjectArea}=3{\MIncludePSE}\fi
}


\ifttm
%\newcommand{\MEmbed}[1]{\MRegisterFile{#1}\begin{html}<embed src="\end{html}\MMaterial/\MLastFile\begin{html}" width="192" height="189"></embed>\end{html}}
\newcommand{\MEmbed}[1]{\MRegisterFile{#1}\begin{html}<embed src="\end{html}\MMaterial/\MLastFile\begin{html}"></embed>\end{html}}
\fi

%----------------- Makros fuer die Textdarstellung -----------------------------------------------

\ifttm
% MUGraphics bindet eine Grafik ein:
% Parameter 1: Dateiname der Grafik, relativ zur Position des Modul-Tex-Dokuments
% Parameter 2: Skalierungsoptionen fuer PDF (fuer includegraphics)
% Parameter 3: Titel fuer die Grafik, wird unter die Grafik mit der Grafiknummer gesetzt und kann MLabel bzw. MCopyrightLabel enthalten
% Parameter 4: Skalierungsoptionen fuer HTML (css-styles)

% ERSATZ: <img alt="My Image" src="data:image/png;base64,iVBORwA<MoreBase64SringHere>" />


\newcommand{\MUGraphics}[4]{\MRegisterFile{#1}\begin{html}
<div class="imagecenter">
<center>
<div>
<img src="\end{html}\MMaterial/\MLastFile\begin{html}" style="#4" alt="\end{html}\MMaterial/\MLastFile\begin{html}"/>
</div>
<div class="bildtext">
\end{html}
\addtocounter{MGraphicsCounter}{1}
\setcounter{MLastIndex}{\value{MGraphicsCounter}}
\setcounter{MLastType}{8}
\addtocounter{MCaptionOn}{1}
\ifnum\value{MSepNumbers}=0
\textbf{Abbildung \arabic{MGraphicsCounter}:} #3
\else
\textbf{Abbildung \arabic{section}.\arabic{subsection}.\arabic{MGraphicsCounter}:} #3
\fi
\addtocounter{MCaptionOn}{-1}
\begin{html}
</div>
</center>
</div>
<br />
\end{html}%
\special{html:<!-- mfeedbackbutton;Abbildung;}\arabic{MGraphicsCounter}\special{html:;}\arabic{section}.\arabic{subsection}.\arabic{MGraphicsCounter}\special{html:; //-->}%
}

% MVideo bindet ein Video als Einzeldatei ein:
% Parameter 1: Dateiname des Videos, relativ zur Position des Modul-Tex-Dokuments, ohne die Endung ".mp4"
% Parameter 2: Titel fuer das Video (kann MLabel oder MCopyrightLabel enthalten), wird unter das Video mit der Videonummer gesetzt
\newcommand{\MVideo}[2]{\MRegisterFile{#1.mp4}\begin{html}
<div class="imagecenter">
<center>
<div>
<video width="95\%" controls="controls"><source src="\end{html}\MMaterial/#1.mp4\begin{html}" type="video/mp4">Ihr Browser kann keine MP4-Videos abspielen!</video>
</div>
<div class="bildtext">
\end{html}
\addtocounter{MVideoCounter}{1}
\setcounter{MLastIndex}{\value{MVideoCounter}}
\setcounter{MLastType}{12}
\addtocounter{MCaptionOn}{1}
\ifnum\value{MSepNumbers}=0
\textbf{Video \arabic{MVideoCounter}:} #2
\else
\textbf{Video \arabic{section}.\arabic{subsection}.\arabic{MVideoCounter}:} #2
\fi
\addtocounter{MCaptionOn}{-1}
\begin{html}
</div>
</center>
</div>
<br />
\end{html}}

\newcommand{\MDVideo}[2]{\MRegisterFile{#1.mp4}\MRegisterFile{#1.ogv}\begin{html}
<div class="imagecenter">
<center>
<div>
<video width="70\%" controls><source src="\end{html}\MMaterial/#1.mp4\begin{html}" type="video/mp4"><source src="\end{html}\MMaterial/#1.ogv\begin{html}" type="video/ogg">Ihr Browser kann keine MP4-Videos abspielen!</video>
</div>
<br />
#2
</center>
</div>
<br />
\end{html}
}

\newcommand{\MGraphics}[3]{\MUGraphics{#1}{#2}{#3}{}}

\else

\newcommand{\MVideo}[2]{%
% Kein Video im PDF darstellbar, trotzdem so tun als ob da eines waere
\begin{center}
(Video nicht darstellbar)
\end{center}
\addtocounter{MVideoCounter}{1}
\setcounter{MLastIndex}{\value{MVideoCounter}}
\setcounter{MLastType}{12}
\addtocounter{MCaptionOn}{1}
\ifnum\value{MSepNumbers}=0
\textbf{Video \arabic{MVideoCounter}:} #2
\else
\textbf{Video \arabic{section}.\arabic{subsection}.\arabic{MVideoCounter}:} #2
\fi
\addtocounter{MCaptionOn}{-1}
}


% MGraphics bindet eine Grafik ein:
% Parameter 1: Dateiname der Grafik, relativ zur Position des Modul-Tex-Dokuments
% Parameter 2: Skalierungsoptionen fuer PDF (fuer includegraphics)
% Parameter 3: Titel fuer die Grafik, wird unter die Grafik mit der Grafiknummer gesetzt
\newcommand{\MGraphics}[3]{%
\MRegisterFile{#1}%
\ %
\begin{figure}[H]%
\centering{%
\includegraphics[#2]{\MDPrefix/#1}%
\addtocounter{MCaptionOn}{1}%
\caption{#3}%
\addtocounter{MCaptionOn}{-1}%
}%
\end{figure}%
\addtocounter{MGraphicsCounter}{1}\setcounter{MLastIndex}{\value{MGraphicsCounter}}\setcounter{MLastType}{8}\ %
%\ \\Abbildung \ifnum\value{MSepNumbers}=0\else\arabic{chapter}.\arabic{section}.\fi\arabic{MGraphicsCounter}: #3%
}

\newcommand{\MUGraphics}[4]{\MGraphics{#1}{#2}{#3}}


\fi

\newcounter{MCaptionOn} % = 1 falls eine Grafikcaption aktiv ist, = 0 sonst


% MGraphicsSolo bindet eine Grafik pur ein ohne Titel
% Parameter 1: Dateiname der Grafik, relativ zur Position des Modul-Tex-Dokuments
% Parameter 2: Skalierungsoptionen (wirken nur im PDF)
\newcommand{\MGraphicsSolo}[2]{\MUGraphicsSolo{#1}{#2}{}}

% MUGraphicsSolo bindet eine Grafik pur ein ohne Titel, aber mit HTML-Skalierung
% Parameter 1: Dateiname der Grafik, relativ zur Position des Modul-Tex-Dokuments
% Parameter 2: Skalierungsoptionen (wirken nur im PDF)
% Parameter 3: Skalierungsoptionen (wirken nur im HTML), als style-format: "width=???, height=???"
\ifttm
\newcommand{\MUGraphicsSolo}[3]{\MRegisterFile{#1}\begin{html}
<img src="\end{html}\MMaterial/\MLastFile\begin{html}" style="\end{html}#3\begin{html}" alt="\end{html}\MMaterial/\MLastFile\begin{html}"/>
\end{html}%
\special{html:<!-- mfeedbackbutton;Abbildung;}#1\special{html:;}\MMaterial/\MLastFile\special{html:; //-->}%
}
\else
\newcommand{\MUGraphicsSolo}[3]{\MRegisterFile{#1}\includegraphics[#2]{\MDPrefix/#1}}
\fi

% Externer Link mit URL
% Erster Parameter: Vollstaendige(!) URL des Links
% Zweiter Parameter: Text fuer den Link
\newcommand{\MExtLink}[2]{\ifttm\special{html:<a target="_new" href="}#1\special{html:">}#2\special{html:</a>}\else\href{#1}{#2}\fi} % ohne MINTERLINK!


% Interner Link, die verlinkte Datei muss im gleichen Verzeichnis liegen wie die Modul-Texdatei
% Erster Parameter: Dateiname
% Zweiter Parameter: Text fuer den Link
\newcommand{\MIntLink}[2]{\ifttm\MRegisterFile{#1}\special{html:<a class="MINTERLINK" target="_new" href="}\MMaterial/\MLastFile\special{html:">}#2\special{html:</a>}\else{\href{#1}{#2}}\fi}


\ifttm
\def\MMaterial{:localmaterial:}
\else
\def\MMaterial{\MDPrefix}
\fi

\ifttm
\def\MNoFile#1{:directmaterial:#1}
\else
\def\MNoFile#1{#1}
\fi

\newcommand{\MChem}[1]{$\mathrm{#1}$}

\newcommand{\MApplet}[3]{
% Bindet ein Java-Applet ein, die Parameter sind:
% (wird nur im HTML, aber nicht im PDF erstellt)
% #1 Dateiname des Applets (muss mit ".class" enden)
% #2 = Breite in Pixeln
% #3 = Hoehe in Pixeln
\ifttm
\MRegisterFile{#1}
\begin{html}
<applet code="\end{html}\MMaterial/\MLastFile\begin{html}" width="#2" height="#3" alt="[Java-Applet kann nicht gestartet werden]"></applet>
\end{html}
\fi
}

\newcommand{\MScriptPage}[2]{
% Bindet eine JavaScript-Datei ein, die eine eigene Seite bekommt
% (wird nur im HTML, aber nicht im PDF erstellt)
% #1 Dateiname des Programms (sollte mit ".js" enden)
% #2 = Kurztitel der Seite
\ifttm
\begin{MSContent}{#2}{#2}{puzzle}
\MRegisterFile{#1}
\begin{html}
<script src="\MMaterial/\MLastFile" type="text/javascript"></script>
\end{html}
\end{MSContent}
\fi
}

\newcommand{\MIncludePSE}{
% Bindet bei Chemie-Modulen das PSE ein
% (wird nur im HTML, aber nicht im PDF erstellt)
\ifttm
\special{html:<!-- includepse //-->}
\begin{MSContent}{Periodensystem der Elemente}{PSE}{table}
\MRegisterFile{../files/pse.js}
\MRegisterFile{../files/radio.png}
\begin{html}
<script src="\MMaterial/../files/pse.js" type="text/javascript"></script>
<p id="divid"><br /><br />
<script language="javascript" type="text/javascript">
    startpse("divid","\MMaterial/../files"); 
</script>
</p>
<br />
<br />
<br />
<p>Die Farben der Elementsymbole geben an: <font style="color:Red">gasf&ouml;rmig </font> <font style="color:Blue">fl&uuml;ssig </font> fest</p>
<p>Die Elemente der Gruppe 1 A, 2 A, 3 A usw. geh&ouml;ren zu den Hauptgruppenelementen.</p>
<p>Die Elemente der Gruppe 1 B, 2 B, 3 B usw. geh&ouml;ren zu den Nebengruppenelementen.</p>
<p>() kennzeichnet die Masse des stabilsten Isotops</p>
\end{html}
\end{MSContent}
\fi
}

\newcommand{\MAppletArchive}[4]{
% Bindet ein Java-Applet ein, die Parameter sind:
% (wird nur im HTML, aber nicht im PDF erstellt)
% #1 Dateiname der Klasse mit Appletaufruf (muss mit ".class" enden)
% #2 Dateiname des Archivs (muss mit ".jar" enden)
% #3 = Breite in Pixeln
% #4 = Hoehe in Pixeln
\ifttm
\MRegisterFile{#2}
\begin{html}
<applet code="#1" archive="\end{html}\MMaterial/\MLastFile\begin{html}" codebase="." width="#3" height="#4" alt="[Java-Archiv kann nicht gestartet werden]"></applet>
\end{html}
\fi
}

% Bindet in der Haupttexdatei ein MINT-Modul ein. Parameter 1 ist das Verzeichnis (relativ zur Haupttexdatei), Parameter 2 ist der Dateinahme ohne Pfad.
\newcommand{\IncludeModule}[2]{
\renewcommand{\MDPrefix}{#1}
\input{#1/#2}
\ifnum\value{MSSEnd}>0{\MSubsectionEndMacros}\addtocounter{MSSEnd}{-1}\fi
}

% Der ttm-Konverter setzt keine Makros im \input um, also muss hier getrickst werden:
% Das MDPrefix muss in den einzelnen Modulen manuell eingesetzt werden
\newcommand{\MInputFile}[1]{
\ifttm
\input{#1}
\else
\input{#1}
\fi
}


\newcommand{\MCases}[1]{\left\lbrace{\begin{array}{rl} #1 \end{array}}\right.}

\ifttm
\newenvironment{MCaseEnv}{\left\lbrace\begin{array}{rl}}{\end{array}\right.}
\else
\newenvironment{MCaseEnv}{\left\lbrace\begin{array}{rl}}{\end{array}\right.}
\fi

\def\MSkip{\ifttm\MCR\fi}

\ifttm
\def\MCR{\special{html:<br />}}
\else
\def\MCR{\ \\}
\fi


% Pragmas - Sind Schluesselwoerter, die dem Preprocessing sowie dem Konverter uebergeben werden und bestimmte
%           Aktionen ausloesen. Im Output (PDF und HTML) tauchen sie nicht auf.
\newcommand{\MPragma}[1]{%
\ifttm%
\special{html:<!-- mpragma;-;}#1\special{html:;; -->}%
\else%
% MPragmas werden vom Preprozessor direkt im LaTeX gefunden
\fi%
}

% Ersatz der Befehle textsubscript und textsuperscript, die ttm nicht kennt
\ifttm%
\newcommand{\MTextsubscript}[1]{\special{html:<sub>}#1\special{html:</sub>}}%
\newcommand{\MTextsuperscript}[1]{\special{html:<sup>}#1\special{html:</sup>}}%
\else%
\newcommand{\MTextsubscript}[1]{\textsubscript{#1}}%
\newcommand{\MTextsuperscript}[1]{\textsuperscript{#1}}%
\fi

%------------------ Einbindung von dia-Diagrammen ----------------------------------------------
% Beim preprocessing wird aus jeder dia-Datei eine tex-Datei und eine pdf-Datei erzeugt,
% diese werden hier jeweils im PDF und HTML eingebunden
% Parameter: Dateiname der mit dia erstellten Datei (OHNE die Endung .dia)
\ifttm%
\newcommand{\MDia}[1]{%
\MGraphicsSolo{#1minthtml.png}{}%
}
\else%
\newcommand{\MDia}[1]{%
\MGraphicsSolo{#1mintpdf.png}{scale=0.1667}%
}
\fi%

% subsup funktioniert im Ausdruck $D={\R}^+_0$, also \R geklammert und sup zuerst
% \ifttm
% \def\MSubsup#1#2#3{\special{html:<msubsup>} #1 #2 #3\special{html:</msubsup>}}
% \else
% \def\MSubsup#1#2#3{{#1}^{#3}_{#2}}
% \fi

%\input{local.tex}

% \ifttm
% \else
% \newwrite\mintlog
% \immediate\openout\mintlog=mintlog.txt
% \fi

% ----------------------- tikz autogenerator -------------------------------------------------------------------

\newcommand{\Mtikzexternalize}{\tikzexternalize}% wird bei Konvertierung ueber mconvert ggf. ausgehebelt!

\ifttm
\else
\tikzset%
{
  % Defines a custom style which generates pdf and converts to (low and hi-res quality) png and svg, then deletes the pdf
  % Important: DO NOT directly convert from pdf to hires-png or from svg to png with GraphViz convert as it has some problems and memory leaks
  png export/.style=%
  {
    external/system call/.add={}{; 
      pdf2svg "\image.pdf" "\image.svg" ; 
      convert -density 112.5 -transparent white "\image.pdf" "\image.png"; 
      inkscape --export-png="\image.4x.png" --export-dpi=450 --export-background-opacity=0 --without-gui "\image.svg"; 
      rm "\image.pdf"; rm "\image.log"; rm "\image.dpth"; rm "\image.idx"
    },
    external/force remake,
  }
}
\tikzset{png export}
\tikzsetexternalprefix{}
% PNGs bei externer Erzeugung in "richtiger" Groesse einbinden
\pgfkeys{/pgf/images/include external/.code={\includegraphics[scale=0.64]{#1}}}
\fi

% Spezielle Umgebung fuer Autogenerierung, Bildernamen sind nur innerhalb eines Moduls (einer MSection) eindeutig)

\newcommand{\MTIKZautofilename}{tikzautofile}

\ifttm
% HTML-Version: Vom Autogenerator erzeugte png-Datei einbinden, tikz selbst nicht ausfuehren (sprich: #1 schlucken)
\newcommand{\MTikzAuto}[1]{%
\addtocounter{MTIKZAutofilenumber}{1}
\renewcommand{\MTIKZautofilename}{mtikzauto_\arabic{MTIKZAutofilenumber}}
\MUGraphicsSolo{\MSectionID\MTIKZautofilename.4x.png}{scale=1}{\special{html:[[!-- svgstyle;}\MSectionID\MTIKZautofilename\special{html: //--]]}} % Styleinfos werden aus original-png, nicht 4x-png geholt!
%\MRegisterFile{\MSectionID\MTIKZautofilename.png} % not used right now
%\MRegisterFile{\MSectionID\MTIKZautofilename.svg}
}
\else%
% PDF-Version: Falls Autogenerator aktiv wird Datei automatisch benannt und exportiert
\newcommand{\MTikzAuto}[1]{%
\addtocounter{MTIKZAutofilenumber}{1}%
\renewcommand{\MTIKZautofilename}{mtikzauto_\arabic{MTIKZAutofilenumber}}
\tikzsetnextfilename{\MTIKZautofilename}%
#1%
}
\fi

% In einer reinen LaTeX-Uebersetzung kapselt der Preambelinclude-Befehl nur input,
% in einer konvertergesteuerten PDF/HTML-Uebersetzung wird er dagegen entfernt und
% die Preambeln an mintmod angehaengt, die Ersetzung wird von mconvert.pl vorgenommen.

\newcommand{\MPreambleInclude}[1]{\input{#1}}

% Globale Watermarksettings (werden auch nochmal zu Beginn jedes subsection gesetzt,
% muessen hier aber auch global ausgefuehrt wegen Einfuehrungsseiten und Inhaltsverzeichnis

\MWatermarkSettings
% ---------------------------------- Parametrisierte Aufgaben ----------------------------------------

\ifttm
\newenvironment{MPExercise}{%
\begin{MExercise}%
}{%
\special{html:<button name="Name_MPEX}\arabic{MExerciseCounter}\special{html:" id="MPEX}\arabic{MExerciseCounter}%
\special{html:" type="button" onclick="reroll('}\arabic{MExerciseCounter}\special{html:');">Neue Aufgabe erzeugen</button>}%
\end{MExercise}%
}
\else
\newenvironment{MPExercise}{%
\begin{MExercise}%
}{%
\end{MExercise}%
}
\fi

% Parameter: Name, Min, Max, PDF-Standard. Name in Deklaration OHNE backslash, im Code MIT Backslash
\ifttm
\newcommand{\MGlobalInteger}[4]{\special{html:%
<!-- onloadstart //-->%
MVAR.push(createGlobalInteger("}#1\special{html:",}#2\special{html:,}#3\special{html:,}#4\special{html:)); %
<!-- onloadstop //-->%
<!-- viewmodelstart //-->%
ob}#1\special{html:: ko.observable(rerollMVar("}#1\special{html:")),%
<!-- viewmodelstop //-->%
}%
}%
\else%
\newcommand{\MGlobalInteger}[4]{\newcounter{mvc_#1}\setcounter{mvc_#1}{#4}}
\fi

% Parameter: Name, Min, Max, PDF-Standard. Name in Deklaration OHNE backslash, im Code MIT Backslash, Wert ist Wurzel von value
\ifttm
\newcommand{\MGlobalSqrt}[4]{\special{html:%
<!-- onloadstart //-->%
MVAR.push(createGlobalSqrt("}#1\special{html:",}#2\special{html:,}#3\special{html:,}#4\special{html:)); %
<!-- onloadstop //-->%
<!-- viewmodelstart //-->%
ob}#1\special{html:: ko.observable(rerollMVar("}#1\special{html:")),%
<!-- viewmodelstop //-->%
}%
}%
\else%
\newcommand{\MGlobalSqrt}[4]{\newcounter{mvc_#1}\setcounter{mvc_#1}{#4}}% Funktioniert nicht als Wurzel !!!
\fi

% Parameter: Name, Min, Max, PDF-Standard zaehler, PDF-Standard nenner. Name in Deklaration OHNE backslash, im Code MIT Backslash
\ifttm
\newcommand{\MGlobalFraction}[5]{\special{html:%
<!-- onloadstart //-->%
MVAR.push(createGlobalFraction("}#1\special{html:",}#2\special{html:,}#3\special{html:,}#4\special{html:,}#5\special{html:)); %
<!-- onloadstop //-->%
<!-- viewmodelstart //-->%
ob}#1\special{html:: ko.observable(rerollMVar("}#1\special{html:")),%
<!-- viewmodelstop //-->%
}%
}%
\else%
\newcommand{\MGlobalFraction}[5]{\newcounter{mvc_#1}\setcounter{mvc_#1}{#4}} % Funktioniert nicht als Bruch !!!
\fi

% MVar darf im HTML nur in MEvalMathDisplay-Umgebungen genutzt werden oder in Strings die an den Parser uebergeben werden
\ifttm%
\newcommand{\MVar}[1]{\special{html:[var_}#1\special{html:]}}%
\else%
\newcommand{\MVar}[1]{\arabic{mvc_#1}}%
\fi

\ifttm%
\newcommand{\MRerollButton}[2]{\special{html:<button type="button" onclick="rerollMVar('}#1\special{html:');">}#2\special{html:</button>}}%
\else%
\newcommand{\MRerollButton}[2]{\relax}% Keine sinnvolle Entsprechung im PDF
\fi

% MEvalMathDisplay fuer HTML wird in mconvert.pl im preprocessing realisiert
% PDF: eine equation*-Umgebung (ueber amsmath)
% HTML: Eine Mathjax-Tex-Umgebung, deren Auswertung mit knockout-obervablen gekoppelt ist
% PDF-Version hier nur fuer pdflatex-only-Uebersetzung gegeben

\ifttm\else\newenvironment{MEvalMathDisplay}{\begin{equation*}}{\end{equation*}}\fi

% ---------------------------------- Spezialbefehle fuer AD ------------------------------------------

%Abk�rzung f�r \longrightarrow:
\newcommand{\lto}{\ensuremath{\longrightarrow}}

%Makro f�r Funktionen:
\newcommand{\exfunction}[5]
{\begin{array}{rrcl}
 #1 \colon  & #2 &\lto & #3 \\[.05cm]  
  & #4 &\longmapsto  & #5 
\end{array}}

\newcommand{\function}[5]{%
#1:\;\left\lbrace{\begin{array}{rcl}
 #2 &\lto & #3 \\
 #4 &\longmapsto  & #5 \end{array}}\right.}


%Die Identit�t:
\DeclareMathOperator{\Id}{Id}

%Die Signumfunktion:
\DeclareMathOperator{\sgn}{sgn}

%Zwei Betonungskommandos (k�nnen angepasst werden):
\newcommand{\highlight}[1]{#1}
\newcommand{\modstextbf}[1]{#1}
\newcommand{\modsemph}[1]{#1}


% ---------------------------------- Spezialbefehle fuer JL ------------------------------------------


\def\jccolorfkt{green!50!black} %Farbe des Funktionsgraphen
\def\jccolorfktarea{green!25!white} %Farbe der Fl"ache unter dem Graphen
\def\jccolorfktareahell{green!12!white} %helle Einf"arbung der Fl"ache unter dem Graphen
\def\jccolorfktwert{green!50!black} %Farbe einzelner Punkte des Graphen

\newcommand{\MPfadBilder}{Bilder}

\ifttm%
\newcommand{\jMD}{\,\MD}%
\else%
\newcommand{\jMD}{\;\MD}%
\fi%

\def\jHTMLHinweisBedienung{\MInputHint{%
Mit Hilfe der Symbole am oberen Rand des Fensters
k"onnen Sie durch die einzelnen Abschnitte navigieren.}}

\def\jHTMLHinweisEingabeText{\MInputHint{%
Geben Sie jeweils ein Wort oder Zeichen als Antwort ein.}}

\def\jHTMLHinweisEingabeTerm{\MInputHint{%
Klammern Sie Ihre Terme, um eine eindeutige Eingabe zu erhalten. 
Beispiel: Der Term $\frac{3x+1}{x-2}$ soll in der Form
\texttt{(3*x+1)/((x+2)^2}$ eingegeben werden (wobei auch Leerzeichen 
eingegeben werden k"onnen, damit eine Formel besser lesbar ist).}}

\def\jHTMLHinweisEingabeIntervalle{\MInputHint{%
Intervalle werden links mit einer "offnenden Klammer und rechts mit einer 
schlie"senden Klammer angegeben. Eine runde Klammer wird verwendet, wenn der 
Rand nicht dazu geh"ort, eine eckige, wenn er dazu geh"ort. 
Als Trennzeichen wird ein Komma oder ein Semikolon akzeptiert.
Beispiele: $(a, b)$ offenes Intervall,
$[a; b)$ links abgeschlossenes, rechts offenes Intervall von $a$ bis $b$. 
Die Eingabe $]a;b[$ f"ur ein offenes Intervall wird nicht akzeptiert.
F"ur $\infty$ kann \texttt{infty} oder \texttt{unendlich} geschrieben werden.}}

\def\jHTMLHinweisEingabeFunktionen{\MInputHint{%
Schreiben Sie Malpunkte (geschrieben als \texttt{*}) aus und setzen Sie Klammern um Argumente f�r Funktionen.
Beispiele: Polynom: \texttt{3*x + 0.1}, Sinusfunktion: \texttt{sin(x)}, 
Verkettung von cos und Wurzel: \texttt{cos(sqrt(3*x))}.}}

\def\jHTMLHinweisEingabeFunktionenSinCos{\MInputHint{%
Die Sinusfunktion $\sin x$ wird in der Form \texttt{sin(x)} angegeben, %
$\cos\left(\sqrt{3 x}\right)$ durch \texttt{cos(sqrt(3*x))}.}}

\def\jHTMLHinweisEingabeFunktionenExp{\MInputHint{%
Die Exponentialfunktion $\MEU^{3x^4 + 5}$ wird als
\texttt{exp(3 * x^4 + 5)} angegeben, %
$\ln\left(\sqrt{x} + 3.2\right)$ durch \texttt{ln(sqrt(x) + 3.2)}.}}

% ---------------------------------- Spezialbefehle fuer Fachbereich Physik --------------------------

\newcommand{\E}{{e}}
\newcommand{\ME}[1]{\cdot 10^{#1}}
\newcommand{\MU}[1]{\;\mathrm{#1}}
\newcommand{\MPG}[3]{%
  \ifnum#2=0%
    #1\ \mathrm{#3}%
  \else%
    #1\cdot 10^{#2}\ \mathrm{#3}%
  \fi}%
%

\newcommand{\MMul}{\MExponentensymbXYZl} % Nur eine Abkuerzung


% ---------------------------------- Stichwortfunktionialitaet ---------------------------------------

% mpreindexentry wird durch Auswahlroutine in conv.pl durch mindexentry substitutiert
\ifttm%
\def\MIndex#1{\index{#1}\special{html:<!-- mpreindexentry;;}#1\special{html:;;}\arabic{MSubjectArea}\special{html:;;}%
\arabic{chapter}\special{html:;;}\arabic{section}\special{html:;;}\arabic{subsection}\special{html:;;}\arabic{MEntryCounter}\special{html:; //-->}%
\setcounter{MLastIndex}{\value{MEntryCounter}}%
\addtocounter{MEntryCounter}{1}%
}%
% Copyrightliste wird als tex-Datei im preprocessing von conv.pl erzeugt und unter converter/tex/entrycollection.tex abgelegt
% Der input-Befehl funktioniert nur, wenn die aufrufende tex-Datei auf der obersten Ebene liegt (d.h. selbst kein input/include ist, insbesondere keine Moduldatei)
\def\MEntryList{} % \input funktioniert nicht, weil ttm (und damit das \input) ausgefuehrt wird, bevor Datei da ist
\else%
\def\MIndex#1{\index{#1}}
\def\MEntryList{\MAbort{Stichwortliste nur im HTML realisierbar}}%
\fi%

\def\MEntry#1#2{\textbf{#1}\MIndex{#2}} % Idee: MLastType auf neuen Entry-Typ und dann ein MLabel vergeben mit autogen-Nummer

% ---------------------------------- Befehle fuer Tests ----------------------------------------------

% MEquationItem stellt eine Eingabezeile der Form Vorgabe = Antwortfeld her, der zweite Parameter kann z.B. MSimplifyQuestion-Befehl sein
\ifttm
\newcommand{\MEquationItem}[2]{{#1}$\,=\,${#2}}%
\else%
\newcommand{\MEquationItem}[2]{{#1}$\;\;=\,${#2}}%
\fi

\ifttm
\newcommand{\MInputHint}[1]{%
\ifnum%
\if\value{MTestSite}>0%
\else%
{\color{blue}#1}%
\fi%
\fi%
}
\else
\newcommand{\MInputHint}[1]{\relax}
\fi

\ifttm
\newcommand{\MInTestHeader}{%
Dies ist ein einreichbarer Test:
\begin{itemize}
\item{Im Gegensatz zu den offenen Aufgaben werden beim Eingeben keine Hinweise zur Formulierung der mathematischen Ausdr�cke gegeben.}
\item{Der Test kann jederzeit neu gestartet oder verlassen werden.}
\item{Der Test kann durch die Buttons am Ende der Seite beendet und abgeschickt, oder zur�ckgesetzt werden.}
\item{Der Test kann mehrfach probiert werden. F�r die Statistik z�hlt die zuletzt abgeschickte Version.}
\end{itemize}
}
\else
\newcommand{\MInTestHeader}{%
\relax
}
\fi

\ifttm
\newcommand{\MInTestFooter}{%
\special{html:<button name="Name_TESTFINISH" id="TESTFINISH" type="button" onclick="finish_button('}\MTestName\special{html:');">Test auswerten</button>}%
\begin{html}
&nbsp;&nbsp;&nbsp;&nbsp;&nbsp;&nbsp;&nbsp;&nbsp;
<button name="Name_TESTRESET" id="TESTRESET" type="button" onclick="reset_button();">Test zur�cksetzen</button>
<br />
<br />
<div class="xreply">
<p name="Name_TESTEVAL" id="TESTEVAL">
Hier erscheint die Testauswertung!
<br />
</p>
</div>
\end{html}
}
\else
\newcommand{\MInTestFooter}{%
\relax
}
\fi


% ---------------------------------- Notationsmakros -------------------------------------------------------------

% Notationsmakros die nicht von der Kursvariante abhaengig sind

\newcommand{\MZahltrennzeichen}[1]{\renewcommand{\MZXYZhltrennzeichen}{#1}}

\ifttm
\newcommand{\MZahl}[3][\MZXYZhltrennzeichen]{\edef\MZXYZtemp{\noexpand\special{html:<mn>#2#1#3</mn>}}\MZXYZtemp}
\else
\newcommand{\MZahl}[3][\MZXYZhltrennzeichen]{{}#2{#1}#3}
\fi

\newcommand{\MEinheitenabstand}[1]{\renewcommand{\MEinheitenabstXYZnd}{#1}}
\ifttm
\newcommand{\MEinheit}[2][\MEinheitenabstXYZnd]{{}#1\edef\MEINHtemp{\noexpand\special{html:<mi mathvariant="normal">#2</mi>}}\MEINHtemp} 
\else
\newcommand{\MEinheit}[2][\MEinheitenabstXYZnd]{{}#1 \mathrm{#2}} 
\fi

\newcommand{\MExponentensymbol}[1]{\renewcommand{\MExponentensymbXYZl}{#1}}
\newcommand{\MExponent}[2][\MExponentensymbXYZl]{{}#1{} 10^{#2}} 

%Punkte in 2 und 3 Dimensionen
\newcommand{\MPointTwo}[3][]{#1(#2\MCoordPointSep #3{}#1)}
\newcommand{\MPointThree}[4][]{#1(#2\MCoordPointSep #3\MCoordPointSep #4{}#1)}
\newcommand{\MPointTwoAS}[2]{\left(#1\MCoordPointSep #2\right)}
\newcommand{\MPointThreeAS}[3]{\left(#1\MCoordPointSep #2\MCoordPointSep #3\right)}

% Masseinheit, Standardabstand: \,
\newcommand{\MEinheitenabstXYZnd}{\MThinspace} 

% Horizontaler Leerraum zwischen herausgestellter Formel und Interpunktion
\ifttm
\newcommand{\MDFPSpace}{\,}
\newcommand{\MDFPaSpace}{\,\,}
\newcommand{\MBlank}{\ }
\else
\newcommand{\MDFPSpace}{\;}
\newcommand{\MDFPaSpace}{\;\;}
\newcommand{\MBlank}{\ }
\fi

% Satzende in herausgestellter Formel mit horizontalem Leerraum
\newcommand{\MDFPeriod}{\MDFPSpace .}

% Separation von Aufzaehlung und Bedingung in Menge
\newcommand{\MCondSetSep}{\,:\,} %oder '\mid'

% Konverter kennt mathopen nicht
\ifttm
\def\mathopen#1{}
\fi

% -----------------------------------START Rouletteaufgaben ------------------------------------------------------------

\ifttm
% #1 = Dateiname, #2 = eindeutige ID fuer das Roulette im Kurs
\newcommand{\MDirectRouletteExercises}[2]{
\begin{MExercise}
\texttt{Im HTML erscheinen hier Aufgaben aus einer Aufgabenliste...}
\end{MExercise}
}
\else
\newcommand{\MDirectRouletteExercises}[2]{\relax} % wird durch mconvert.pl gefunden und ersetzt
\fi


% ---------------------------------- START Makros, die von der Kursvariante abhaengen ----------------------------------

\ifvariantunotation
  % unotation = An Universitaeten uebliche Notation
  \def\MVariant{unotation}

  % Trennzeichen fuer Dezimalzahlen
  \newcommand{\MZXYZhltrennzeichen}{.}

  % Exponent zur Basis 10 in der Exponentialschreibweise, 
  % Standardmalzeichen: \times
  \newcommand{\MExponentensymbXYZl}{\times} 

  % Begrenzungszeichen fuer offene Intervalle
  \newcommand{\MoIl}[1][]{\mbox{}#1(\mathopen{}} % bzw. ']'
  \newcommand{\MoIr}[1][]{#1)\mbox{}} % bzw. '['

  % Zahlen-Separation im IntervaLL
  \newcommand{\MIntvlSep}{,} %oder ';'

  % Separation von Elementen in Mengen
  \newcommand{\MElSetSep}{,} %oder ';'

  % Separation von Koordinaten in Punkten
  \newcommand{\MCoordPointSep}{,} %oder ';' oder '|', '\MThinspace|\MThinspace'

\else
  % An dieser Stelle wird angenommen, dass std-Variante aktiv ist
  % std = beschlossene Notation im TU9-Onlinekurs 
  \def\MVariant{std}

  % Trennzeichen fuer Dezimalzahlen
  \newcommand{\MZXYZhltrennzeichen}{,}

  % Exponent zur Basis 10 in der Exponentialschreibweise, 
  % Standardmalzeichen: \times
  \newcommand{\MExponentensymbXYZl}{\times} 

  % Begrenzungszeichen fuer offene Intervalle
  \newcommand{\MoIl}[1][]{\mbox{}#1]\mathopen{}} % bzw. '('
  \newcommand{\MoIr}[1][]{#1[\mbox{}} % bzw. ')'

  % Zahlen-Separation im IntervaLL
  \newcommand{\MIntvlSep}{;} %oder ','
  
  % Separation von Elementen in Mengen
  \newcommand{\MElSetSep}{;} %oder ','

  % Separation von Koordinaten in Punkten
  \newcommand{\MCoordPointSep}{;} %oder '|', '\MThinspace|\MThinspace'

\fi



% ---------------------------------- ENDE Makros, die von der Kursvariante abhaengen ----------------------------------


% diese Kommandos setzen Mathemodus vorraus
\newcommand{\MGeoAbstand}[2]{[\overline{{#1}{#2}}]}
\newcommand{\MGeoGerade}[2]{{#1}{#2}}
\newcommand{\MGeoStrecke}[2]{\overline{{#1}{#2}}}
\newcommand{\MGeoDreieck}[3]{{#1}{#2}{#3}}

%
\ifttm
\newcommand{\MOhm}{\special{html:<mn>&#x3A9;</mn>}}
\else
\newcommand{\MOhm}{\Omega} %\varOmega
\fi


\def\PERCTAG{\MAbort{PERCTAG ist zur internen verwendung in mconvert.pl reserviert, dieses Makro darf sonst nicht benutzt werden.}}

% Im Gegensatz zu einfachen html-Umgebungen werden MDirectHTML-Umgebungen von mconvert.pl am ganzen ttm-Prozess vorbeigeschleust und aus dem PDF komplett ausgeschnitten
\ifttm%
\newenvironment{MDirectHTML}{\begin{html}}{\end{html}}%
\else%
\newenvironment{MDirectHTML}{\begin{html}}{\end{html}}%
\fi

% Im Gegensatz zu einfachen Mathe-Umgebungen werden MDirectMath-Umgebungen von mconvert.pl am ganzen ttm-Prozess vorbeigeschleust, ueber MathJax realisiert, und im PDF als $$ ... $$ gesetzt
\ifttm%
\newenvironment{MDirectMath}{\begin{html}}{\end{html}}%
\else%
\newenvironment{MDirectMath}{\begin{equation*}}{\end{equation*}}% Vorsicht, auch \[ und \] werden in amsmath durch equation* redefiniert
\fi

% ---------------------------------- Location Management ---------------------------------------------

% #1 = buttonname (muss in files/images liegen und Format 48x48 haben), #2 = Vollstaendiger Einrichtungsname, #3 = Kuerzel der Einrichtung,  #4 = Name der include-texdatei
\ifttm
\newcommand{\MLocationSite}[3]{\special{html:<!-- mlocation;;}#1\special{html:;;}#2\special{html:;;}#3\special{html:;; //-->}}
\else
\newcommand{\MLocationSite}[3]{\relax}
\fi

% ---------------------------------- Copyright Management --------------------------------------------

\newcommand{\MCCLicense}{%
{\color{green}\textbf{CC BY-SA 3.0}}
}

\newcommand{\MCopyrightLabel}[1]{ (\MSRef{L_COPYRIGHTCOLLECTION}{Lizenz})\MLabel{#1}}

% Copyrightliste wird als tex-Datei im preprocessing erzeugt und unter converter/tex/copyrightcollection.tex abgelegt
% Der input-Befehl funktioniert nur, wenn die aufrufende tex-Datei auf der obersten Ebene liegt (d.h. selbst kein input/include ist, insbesondere keine Moduldatei)
\newcommand{\MCopyrightCollection}{\input{copyrightcollection.tex}}

% MCopyrightNotice fuegt eine Copyrightnotiz ein, der parser ersetzt diese durch CopyrightNoticePOST im preparsing, diese Definition wird nur fuer reine pdflatex-Uebersetzungen gebraucht
% Parameter: #1: Kurze Lizenzbeschreibung (typischerweise \MCCLicense)
%            #2: Link zum Original (http://...) oder NONE falls das Bild selbst ein Original ist, oder TIKZ falls das Bild aus einer tikz-Umgebung stammt
%            #3: Link zum Autor (http://...) oder MINT falls Original im MINT-Kolleg erstellt oder NONE falls Autor unbekannt
%            #4: Bemerkung (z.B. dass Datei mit Maple exportiert wurde)
%            #5: Labelstring fuer existierendes Label auf das copyrighted Objekt, mit MCopyrightLabel erzeugt
%            Keines der Felder darf leer sein!
\newcommand{\MCopyrightNotice}[5]{\MCopyrightNoticePOST{#1}{#2}{#3}{#4}{#5}}

\ifttm%
\newcommand{\MCopyrightNoticePOST}[5]{\relax}%
\else%
\newcommand{\MCopyrightNoticePOST}[5]{\relax}%
\fi%

% ---------------------------------- Meldungen fuer den Benutzer des Konverters ----------------------
\MPragma{mintmodversion;P0.1.0}
\MPragma{usercomment;This is file mintmod.tex version P0.1.0}


% ----------------------------------- Spezialelemente fuer Konfigurationsseite, werden nicht von mintscripts.js verwaltet --

% #1 = DOM-id der Box
\ifttm\newcommand{\MConfigbox}[1]{\special{html:<input cfieldtype="2" type="checkbox" name="Name_}#1\special{html:" id="}#1\special{html:" onchange="confHandlerChange('}#1\special{html:');"/>}}\fi % darf im PDF nicht aufgerufen werden!



%%%% Neue Befehlsdefinition noch hier, sp"ater in mintmod
%%%% Alternative als Umgebung mit Fussnote im PDF s. u.
%%%% \MOnlineOnly{<Inhalt, der nur in der Online-Version erscheinen soll>}
%%%% dh/ehh/jl, MINT-Kolleg BW, 2013=03=01
\newcommand\MOnlineOnlyErsatz{Im Onlinemodul finden Sie an dieser Stelle weitere Inhalte.}
%Standard-Ersatztext, ggf. fuer eigene Bed"urfnisse jeweils mit \renewcommand{...} anpassen.
%Zum Bezeichner: Ersatz is a German loanword in English which refers to substitutes of 
%an inferior quality.
\ifttm
    \newcommand{\MOnlineOnly}[1]{#1} %d. h. keine Wirkung fuer die Ausgabe
\else
    \newcommand{\MOnlineOnly}[1]{
        \begin{quote}\textit   %Quote-Umgebung mit Kursiv-Deklaration
            \MOnlineOnlyErsatz
        \end{quote}
    }
    %F"ur die PDF-Version wird der Parameter (Inhalt f"ur die Online-Version) nicht verwendet.
\fi
%%%% Alternative:
\ifttm
    \newenvironment{MHintRespFootnote}{\begin{MHint}{Bemerkung zu der Gleichung}}{\end{MHint}} %oder Befehl mit Text
\else
    \newenvironment{MHintRespFootnote}{\footnote{}{}} %geht nicht :(
\fi
%%%% Ende neue Befehlsdefinition f"ur mintmod

\begin{document}
\MSetSubject{\MINTPhysics}
%Ge"anderter Ersatztext:
\renewcommand\MOnlineOnlyErsatz{Im Online-Modul finden Sie an dieser 
    Stelle weiterf"uhrende Bemerkungen.
}
%Eigener Befehl f"ur Hervorhebung durch Fontbefehl (da kein Absatz), ggf. hier "andern.
%Andere M"oglichkeiten: Deklaration {{\bfseries #1}} oder
%Umgebungsvariante der Deklaration {\begin{bfseries} #1 \end{bfseries}}
\newcommand{\EdmesEmph}[1]{\textit{#1}}

%###############################################################################
\MSection{Schwingungen und Wellen}\MLabel{Physik_Schwingungen}
\begin{MSectionStart}
Schwingungen treten in vielf"altiger Weise in Erscheinung. Beispiele sind 
\begin{itemize}
  \item Schwingung einer Masse an einer Feder,
  \item Schwingung eines Fadenpendels,
  \item Strom in einem elektrischen Schwingkreis,
  \item Schwingung von Atomen in Molek"ulen und Festk"orpern,
  \item Schwingung in Regelkreisen.
\end{itemize}
Die grundlegende mathematische Beschreibung ist dabei stets gleich.

Eine mechanische Schwingung in Luft oder Wasser f"uhrt zu Schallausbreitung. Diese Ausbreitung wird als Welle bezeichnet. Neben Schallwellen sind f"ur uns Menschen elektromagnetische Wellen wichtig, mit zunehmender Frequenz werden sie als Radiowellen, Mikrowellen, Terahertzstrahlung, Infrarotstrahlung, sichtbares Licht, UV-Licht, R"ontgenstrahlung oder $\gamma$-Strahlung bezeichnet. Unterschiedliche Arten von Schwingungen und Wellen k"onnen in Festk"orpern auftreten, z. B. erw"unscht in Schwingquarzkristallen oder unerw"unscht in mechanischen Bauteilen oder bei Erdbeben. Die moderne Physik kennt mit Materiewellen und Gravitationswellen weitere Formen von Wellen.

Schwingungen und insbesondere Wellen sind anspruchsvolle und umfangreiche Themengebiete, so dass im vorliegenden Modul nur ein kleiner Ausschnitt einf"uhrend behandelt werden kann. Es werden unter anderem folgende Themen behandelt:
\begin{itemize}
  \item Das Masse-Feder-Pendel als einfaches Beispiel eines mechanischen Oszillators.
  \item Einfache D"ampfung und Anregung beim Feder-Masse-System.
  \item Einfache Beispiele von Wellen.
  \item Wellengleichung in einer Raumdimension am Beispiel Schall.
  \item Hinweis auf Interferenz in Raum und Zeit.
\end{itemize}

Nach einf"uhrenden Erl"auterungen erfolgt stets auch eine mathematische Beschreibung der Ph"anomene mit zahlreichen Darstellung der sich ergebenden funktionalen Zusammenh"ange. In der Online-Version des Moduls k"onnen weiterf"uhrende Hinweise zu den Gleichungen per Mausklick auf der Seite angezeigt werden. Um zu pr"ufen, wieweit die Themen verstanden sind, gibt es aus dem Text zahlreiche Verkn"upfungen zu Aufgaben mit L"osungen. 

Das Thema Schwingungen wird im ersten Drittel des Moduls behandelt. Bei Zeitmangel k"onnen kompliziertere Konzepte wie D"ampfung und Anregung ausgelassen werden. Das zweite Drittel leitet in einige Konzepte von Wellen ein, wobei nur ein kleiner Einblick in das umfangreiche und anspruchsvolle Themengebiet gegeben werden kann. Auch hier kann aus Zeitgr"unden eine Auswahl erforderlich sein. Im letzten Drittel werden zus"atzliche Inhalte zur Verf"ugung gestellt. Es werden numerische L"osungen als ein wichtiges Verfahren in der modernen Physik und in anderen Gebieten der Natur- und Ingenieurwissenschaften vorgestellt. Elektrische Schwingkreise sind in vielen Anwendungen von grundlegender Bedeutung. Der betreffende Abschnitt behandelt die Thematik in Analogie zu den mechanischen Schwingungen, ohne auf Details einzugehen, die in folgenden Modulen behandelt werden. Als weiteres Beispiel f"ur eine mechanische Schwingung wird das Fadenpendel vorgestellt.

Das vorliegende Modul orientiert sich in der Schreibweise und Herangehensweise an den Feynman-Vorlesungen "uber Physik, Band 1: Mechanik, Strahlung, W"arme, R. P. Feynman, R. B. Leighton, M. Sands, Definitive Edition, 5. Auflage, Oldenburg Verlag M"unchen (2007). Dieses Buch richtet sich durchaus an Einsteiger, allerdings stets mit dem Hintergrund fortgeschrittener Konzepte und oft auf originelle Weise.
Umfangreiche Erl"auterungen zu vielen Themen der Physik finden sich in: Gerthsen Physik, D. Meschede, 24. Auf\/lage, Springer-Verlag Berlin Heidelberg (2010).
Um einen ersten Eindruck zu einem Themengebiet zu bekommen ist h"aufig auch die deutsch- oder englischsprachige freie Online-Enzyklop"adie Wikipedia\textregistered\ hilfreich.

In diesen Modul werden folgende lateinische Symbole verwendet:\\
\begin{MWTabular}{lll}
  Symbol & Einheit & Erl"auterung \\\hline
  $A_C$  & m$^2$   & Fl"ache des Plattenkondensators \\
  $C$    & F       & Elektrische Kapazit"at \\
  $c$    & Ns/m    & Proportionalit"atskonstante der geschwindigkeitsabh"angigen Reibungskraft \\
  $c_\mathrm{s}$ & m/s & Schallgeschwindigkeit \\
  d$x$/d$t$     & m/s     & Geschwindigkeit. "Ubliches Symbol ohne Ableitung: $v$ \\
  d$^2x$/d$t^2$ & m/s$^2$ & Beschleunigung. "Ubliches Symbol ohne Ableitung: $a$ \\
  $E$    & J       & Energie \\
  $E_\mathrm{kin}$ & J & Kinetische Energie \\
  $E_\mathrm{pot}$ & J & Potentielle Energie \\
  $F$    & N       & Kraft \\
  $F_0$  & N       & Amplitude der "au"seren Kraft \\
  $g$    & N/kg    & Erdbeschleunigung \\
  $I$    & A       & Elektrischer Strom \\
  $k$    & N/m     & Federkonstante \\
  $k_x$  & rad/m   & $x$-Komponente des Wellenvektors \\
  $L$    & H       & Elektrische Induktivit"at \\
  $l$    & m       & Pendell"ange \\
  $l_\mathrm{D}$ & m & Drahtl"ange \\
  $m$    & kg      & Masse \\
  $N$    & -       & Anzahl Windungen \\
  $P$    & W       & Leistung \\
  $P_e$  & Pa      & Schallbedingte Druck"anderung \\
  $P_0$  & Pa      & Gleichgewichtsdruck \\
  $Q$    & -       & G"ute \\
  $q$    & C       & Elektrische Ladung \\
  $R$    & $\Omega$ & Elektrischer Widerstand \\
  $t$    & s       & Zeit \\
  $t_0$  & s       & Schwingungsperiode \\
  $U$    & V       & Elektrische Spannung \\
  $v$    & m/s     & Geschwindigkeit \\
  $x$    & m       & Auslenkung (Schwingungen) bzw. Ortsvariable (Wellen) \\
  $x_h$  & m       & Homogene L"osung (Einschwingvorgang) \\
  $x_p$  & m       & Partikul"are L"osung (eingeschwungen) \\
  $\hat{x}$ & m    & Amplitude der Aufh"angungsauslenkung
\end{MWTabular}

In diesen Modul werden folgende griechische Symbole verwendet:\\
\begin{MWTabular}{lll}
  Symbol & Einheit & Erl"auterung \\\hline
  $\alpha$      & (rad)   & Auslenkungswinkel \\
  $\delta t$    & s       & Dauer eines Zeitschrittes \\
  $\partial / \partial x$ & 1/m & Partielle Ortsableitung \\
  $\partial^2 / \partial t^2$ & 1/s$^2$ & Zweite partielle Zeitableitung \\
  $\partial^2 / \partial x^2$ & 1/m$^2$ & Zweite partielle Ortsableitung \\
  $\epsilon_0$  & F/m     & Elektrische Feldkonstante \\
  $\gamma$      & 1/s     & Verh"altnis aus $c$ und $m$, s. o. \\
  $\Delta$	& (rad)   & Phasenverschiebung der Anregungskraft \\
  $\kappa$      & (m/s)$^2$    & Druck"anderung pro Dichte"anderung im Bereich der Gleichgewichtswerte \\
  $\lambda$     & m       & Wellenl"ange \\
  $\mu_0 $      & H/m     & Magnetische Feldkonstante \\
  $\nu_0$       & 1/s     & Frequenz \\
  $\omega_0$    & 1/s     & Nat"urliche Kreisfrequenz \\
  $\omega$      & 1/s     & Kreisfrequenz der Anregung. \\
  $\phi$	& (rad)	  & Phasenverschiebung beim harmonischen Oszillator \\
  $\phi_d$	& (rad)   & Phasenverschiebung mit D"ampfung \\
  $\rho$        & s$^2$/kg& Schwingungsamplitude pro Amplitude der "au"seren Kraft \\
  $\rho_e$      & kg/m$^3$ & Schallbedingte Dichte"anderung \\
  $\rho_0$      & kg/m$^3$ & Gleichgewichtsdichte \\
  $\sigma$      & S/m     & Elektrische Leitf"ahigkeit \\   
  $\theta$      & (rad)   & Phasenverschiebung mit Anregungskraft \\
  $\chi$        & m       & Schallbedingte Teilchenverschiebung
\end{MWTabular}

\end{MSectionStart}


%###############################################################################

%===============================================================================
\MSubsection{Schwingungen}\MLabel{Physik_Schwingungen_SubSchwingungen}
\begin{MIntro}\MLabel{Physik_Schwingungen_IntroSchwingungen}
Als ganz einfaches mechanisches Beispiel soll eine starre Masse an einer idealen Feder (masselos, linear) betrachtet werden, s. Abb. \MRef{Physik_Schwingungen_AbbHO}:

\begin{center}
  \MUGraphics{abbFederMasse}{scale=1}{Mechanischer harmonischer Oszillator aus idealer Feder und idealer Masse ohne Verluste. Skizze links: Masse im Schwerefeld der Erde an der Feder h"angend. Skizze rechts: reibungsfreie Bewegung in einer horizontalen F"uhrung.\MLabel{Physik_Schwingungen_AbbHO}
}{width:400px;}
  \end{center}

Das eine Ende der Feder ist fixiert. Wird die Masse ausgelenkt (d. h. die Feder wird gedehnt oder komprimiert) und dann losgelassen, beginnt die Masse zu schwingen. Abgesehen von der Richtung der Bewegung und der Auslenkung der Feder in Ruhe ist es dabei unerheblich, ob die Masse im Schwerefeld der Erde an der Feder h"angt oder die Masse auf einer horizontalen Schiene ohne Reibung gleitet. 

Bevor wir mit einer genauen Analyse und Beschreibung auftretender Ph"anomene beginnen, soll "uberlegt werden, welche physikalischen Gr"o"sen eine Rolle spielen und wie sie qualitativ zusammenh"angen. Dabei werden kurz die jeweiligen Dimensionen und Einheiten wiederholt.

\begin{itemize}
  \item Die beobachtete physikalische Gr"o"se ist die Auslenkung der Masse. Diese abh"angige Variable wird hier mit $x$ bezeichnet. Sie ist von der Dimension L"ange mit der SI-Basiseinheit, m.
  \item Die Auslekung wird als Funktion der unabh"angigen Variablen Zeit betrachtet. Die Zeit wird wie "ublich mit $t$ bezeichnet und ist von der gleichlautenden Dimension Zeit mit der Einheit Sekunde, s.
  \item Eine Abh"angigkeit der Bewegung von der tr"agen Masse, ein Parameter mit festem Wert f"ur ein gegebenes Masse-Feder-System, ist zu erwarten. Das "ubliche Symbol ist $m$, sie ist von der wiederum gleichlautenden Dimension Masse mit SI-Einheit Kilo(!)gramm, kg.
  \item Schlie"slich wird die Bewegung dadurch beeinflusst, wie die Federkraft mit der Auslenkung zunimmt. Hier wird vereinfachend eine Proportionalit"at angenommen und der Parameter mit $k$ bezeichnet, wobei die Bezeichnung $D$ ebenfalls verbreitet ist. Die Dimension der Federkonstante ist Kraft pro L"ange, d. h. Masse pro Quadrat der Zeit, mit Einheit Newton pro Meter, N/m, bzw. kg/s$^2$.
\end{itemize}

\MOnlineOnly{
  \begin{MHint}{Bemerkung zu Dimensionen}

    Es wird das "ubliche $\{M, L, T\}$-Grundgr"o"sensystem der Mechanik verwendet. Die folgende Tabelle gibt dieses zusammen mit den entsprechenden Ausdr"ucken f"ur die Federkonstante an:

    \begin{MWTabular}{l|c|c|c}
    Gr"o"senart    & Gr"o"senbezeichnung & Einheit & Dimensionsformel \\\hline
    Masse          & $m$                 & kg      & $M$              \\
    L"ange         & $x$                 & m       & $L$              \\
    Zeit           & $t$                 & s       & $T$              \\
    Federkonstante & $k$                 & $\frac{\mse[]{kg}}{\MEinheit[]{s^2}}$ & $\frac{M}{T^2}$
    \end{MWTabular}
    
    Bei dem horizontalen Feder-Masse-System (d. h. ohne Einfluss der Schwerkraft) sei die Schwingungsperiode mit der Dimension Zeit gesucht. Als m"ogliche Einflussgr"o"sen werden die Masse, die Anfangsauslenkung und die Federkonstante in Betracht gezogen. Allein die Analyse der vorkommenden Dimensionen erlaubt weitreichende Aussagen "uber das Ergebnis. Im betrachteten Fall kann die Dimension Zeit nur durch den Ausdruck $\sqrt{m/k}$ gebildet werden. Damit ist die Schwingungsperiode bis auf eine Konstante bekannt, die durch einmalige Messung an einem System mit beliebiger Masse und frei w"ahlbarer Federkonstante bestimmt werden kann. Ebenso ist damit ein Einfluss der zun"achst in Betracht gezogenen Anfangsauslenkung ausgeschlossen. Bei sehr gro"sen Auslenkungen ist dies nat"urlich anders, aber in diesem Fall gen"ugt auch nicht die Angabe einer Federkonstanten f"ur die Beschreibung der Kraft als Funktion der Auslenkung. Analog kann hergeleitet werden, dass bei dem mathematischen Pendel die Masse \EdmesEmph{keinen} Einfluss hat, wohl aber die Schwerkraft. Derartige Untersuchungen sind Gegenstand der Dimensionsanalyse.

  \end{MHint}
}

Um einen zeichnerischen Zugang zu dem zeitlichen Verlauf der Auslenkung und Geschwindigkeit zu erhalten, k"onnen Sie die einleitenden Aufgaben 
\MRef{Physik_Schwingungen_AufgabeIntroTX}\MLabel{Physik_Schwingungen_VonAufgabeIntroTX} 
und 
\MRef{Physik_Schwingungen_AufgabeIntroTV}\MLabel{Physik_Schwingungen_VonAufgabeIntroTV} bearbeiten.

\end{MIntro}


%===============================================================================
\begin{MXContent}{Der Harmonische Oszillator}{Oszillator}{STD}%
\MLabel{Physik_Schwingungen_MXOszillator}
Nun soll das zweite Newtonsche Gesetz (Proportionalit"at zwischen Kraft und "Anderung des Impulses, 1687) angewendet werden, um die Bewegungsgleichung f"ur das einfache Modell in Abb. \MRef{Physik_Schwingungen_AbbHO} zu erhalten. Diese Gleichung kann analytisch gel"ost werden:

\MOnlineOnly{
  \begin{MHint}{Bemerkung zum L"osen von Gleichungen}
  Teilweise lassen sich (physikalische) Gleichungen analytisch l"osen, d. h., es kann eine geschlossene Formel angegeben werden. H"aufig ist es allerdings nicht m"oglich, eine analytische L"osung zu finden. Im Informationszeitalter erlauben Computer dann die numerische Berechnung von N"aherungsl"osungen durch Ausf"uhrung vieler Rechenoperationen. Bei analytischen L"osungen ist z. B. vorteilhaft, dass L"osungen leicht f"ur unterschiedliche Parameter berechnet werden k"onnen (nat"urlich meist ebenfalls mit einem Computer), ohne m"oglicherweise aufw"andige numerische Berechnungen mit ge"anderten Parametern durchf"uhren zu m"ussen. Weiterhin kann ein analytischer Ausdruck f"ur weitere Rechnungen verwendet werden, etwa f"ur die Berechnung der Ableitung. Auch bei numerischen Berechnungen gilt es, das Problem zu erfassen, geschickt zu formulieren und mit geeigneten numerischen Verfahren zu l"osen. Als Beispiel f"ur Gleichungen mit anspruchsvoller Mathematik und Numerik seien die Navier-Stokes-Gleichungen f"ur die Beschreibung der Str"omung einfacher Fl"ussigkeiten und Gase genannt. Neben der numerischen L"osung der Gleichungen f"ur ein komplexes Modell kann die analytische L"osung von Gleichungen f"ur vereinfachte, gen"aherte Modelle das Verst"andnis verbessern und helfen, Vorhersagen zu treffen.
  \end{MHint}
}

Die einseitig befestigte Feder "ubt eine Kraft $F$ auf die Masse $m$ aus (s. Abb. \MRef{Physik_Schwingungen_AbbHO}). Da sich die Masse nicht "andert, ist die "Anderung des Impulses die Masse multipliziert mit der "Anderung der Geschwindigkeit. Die "Anderung der Geschwindigkeit, d. h. die Beschleunigung, ist die zweite Ableitung der Auslenkung nach der Zeit. Das zweite Newtonsche Gesetz lautet damit
\begin{equation}
  F = m\, \frac{\MD^2 x }{\MD t^2}\;.
\end{equation}

Die Bewegungsgleichung erh"alt man durch Einsetzen der Proportionalit"at zwischen Kraft und Auslenkung. Die Kraftkonstante $k$ wird positiv gew"ahlt. Da es sich um eine r"ucktreibende Kraft handelt, gilt $F=-k x$. Einsetzen in das zweite Newtonsche Gesetz liefert
\begin{equation}\MLabel{Physik_Schwingungen_EQHONewton}
  m\, \frac{\MD^2 x }{\MD t^2} + k\,x = 0\; .
\end{equation}

F"ur die Quadratwurzel aus dem Verh"altnis aus Federkonstante und Masse wird das Symbol $\omega_0$ eingef"uhrt, $\omega_0 = \sqrt{k/m}$. Die Bedeutung wird mit der L"osung der Gleichung klar. Die erhaltene Bewegungsgleichung nennt sich Bewegungsgleichung eines harmonischen Oszillators:
\begin{MInfo}
Bei einem harmonischen Oszillator besteht eine Proportionalit"at zwischen der 
betrachteten physikalischen Gr"o"se $x$ und ihrer zweiten Ableitung nach der Zeit $t$, geschrieben als $\frac{\MD^2x}{\MD t^2}$:

\begin{equation}
  \frac{\MD^2x}{\MD t^2} + \omega_0^2 x = 0\;.
  \MLabel{Physik_Schwingungen_EQHO}
\end{equation}

\end{MInfo}

\MOnlineOnly{
  \begin{MHint}{Bemerkung zu der Gleichung}
  Das $x$ in Gl. \MRef{Physik_Schwingungen_EQHO} ist eine Funktion, n"amlich die Auslenkung als Funktion der Zeit, $x(t)$. Neben der Funktion ist eine ihrer Ableitungen in der Gleichung enthalten. Eine Gleichung, in der Ableitungen einer gesuchten Funktion auftreten, nennt sich Differentialgleichung. Differentialgleichungen treten verbreitet in Wissenschaft und Technik auf. H"angt die gesuchte Funktion von mehreren Variablen ab und treten Ableitungen nach unterschiedlichen Variablen auf, nennt man die Gleichung eine partielle Differentialgleichung. Da dies bei der betrachteten Gleichung nicht der Fall ist, wird sie als gew"ohnliche Differentialgleichung bezeichnet. Da die Gleichung nach der h"ochsten Ableitung aufgel"ost ist, hei"st sie explizit, im Gegensatz zu impliziten Gleichungen. Die Ordnung gibt die h"ochste Ableitung an, die vorliegende Gleichung ist demnach zweiter Ordnung. Weiterhin hat Gl. \MRef{Physik_Schwingungen_EQHO} die wichtige Eigenschaft, in der gesuchten Funktion und ihren Ableitungen linear zu sein. Hinsichtlich der L"osung linearer gew"ohnlicher Differentialgleichungen ist vieles bekannt. Sind unterschiedliche L"osungen gefunden, so k"onnen weitere L"osungen durch Linearkombination gebildet werden (Superpositionsprinzip). Nichtlineare Differentialgleichungen, in denen z. B. Potenzen oder trigonometrische Funktionen der gesuchten Funktion oder ihrer Ableitungen vorkommen, sind oft nicht analytisch l"osbar. Bei linearen Differentialgleichungen k"onnen die ''Koeffizienten'' der Funktion oder ihrer Ableitungen selbst Funktionen der unabh"angigen Variablen sein. Ist das wie hier nicht der Fall, handelt es sich um eine lineare Differentialgleichung mit konstanten Koeffizienten. Enh"alt die lineare Differentialgleichung einen Koeffizienten, der nicht mit der Funktion oder ihren Ableitungen multipliziert wird, ist die Differentialgleichung inhomogen. Das ist in Gl. \MRef{Physik_Schwingungen_EQHO} nicht der Fall, es handelt sich also um eine homogene Gleichung.
  Aufgrund der allgemeinen Bedeutung von Differentialgleichungen ist es lohnend, sich mit der Thematik zu besch"aftigen. Im Folgenden werden komplexere Modelle behandelt, die jeweils durch eine komplexere Differentialgleichung repr"asentiert werden. An entsprechender Stelle wird wiederum in Bermerkungen auf die mathematischen Unterschiede eingegangen.
  \end{MHint}
}

Die allgemeine L"osung von Gl. \MRef{Physik_Schwingungen_EQHO} kann auf unterschiedliche Arten geschrieben werden, sie wird hier ohne Herleitung angegeben:
\begin{MInfo}
Die allgemeine L"osung von Gl. \MRef{Physik_Schwingungen_EQHO} kann mit folgenden Schreibweisen angegeben werden:

\begin{eqnarray}
  x &=& a\cos\left(\omega_0(t-t_1)\right)\quad\mathrm{oder}\MLabel{Physik_Schwingungen_EQHOLoes1}\\
  x &=& a\cos(\omega_0t+\phi)\quad\mathrm{oder}\\
  x &=& A\cos\left(\omega_0t\right)+B\sin\left(\omega_0t\right)\; .
\end{eqnarray}

\end{MInfo}

In der ersten und zweiten Schreibweise nimmt $x$ Werte zwischen $-a$ und $a$ an. Der Vorfaktor $a$ bei Schwingungen wird als \EdmesEmph{Amplitude} bezeichnet. Eine Verschiebung der kosinus- bzw. sinusf"ormigen Schwingung in Richtung der Zeit kann entweder mit $t_1$ als Zeit oder mit $\phi$ als Winkel ausgedr"uckt werden. Die dritte Schreibweise ist aus mathematischer Perspektive naheliegend. Der Nachweis, dass die angegebenen Funktionen tats"achlich L"osungen sind, ist Gegenstand von Aufgabe 
%>>>>>>
\MRef{Physik_Schwingungen_AufgabeLoesungHO}\MLabel{Physik_Schwingungen_VonAufgabeLoesungHO}.
%<<<<<<
In Aufgabe
%>>>>>>
\MRef{Physik_Schwingungen_AufgabeLoesungHOUmrechnung}\MLabel{Physik_Schwingungen_VonAufgabeLoesungHOUmrechnung}
%<<<<<<
wird der mathematische Zusammenhang zwischen den Schreibweisen untersucht.

%%%\end{MXContent}
%%%\begin{MXContent}{Betrachtung der L"osung}{Betrachtung}{STD}
%-------------------------------------------------------------------------------
Die L"osung ist eine einfache periodische Funktion, die sich wiederholt, wenn das Argument der trigonometrischen Funktionen sich um ein Vielfaches von $2\pi$ unterscheidet. Dies ist, nat"urlich unabh"angig von der Schreibweise der L"osung, gegeben, wenn $\omega_0t$ sich um ein Vielfaches von $2\pi$ unterscheidet. Der kleinste positive Zeitunterschied, f"ur den die Gleichheit gilt, wird als \EdmesEmph{Periode} bezeichnet. Mit dem Symbol $t_0$ erh"alt man f"ur die Periode 
\begin{equation}
  \omega_0t_0 = 2\pi\quad
  \Rightarrow \quad t_0=2\pi/\omega_0\; .
\end{equation}

Der Kehrwert der Periode ist die \EdmesEmph{Frequenz}: $$\nu_0 = 1/t_0 = \omega_0 / (2\pi)$$. F"ur das Argument der trigonometrischen Funktion im Bogenma"s wird der Begriff \EdmesEmph{Phase} verwendet, allerdings nicht einheitlich. Teilweise wird das ganze, zeitabh"angige Argument als Phase bezeichnet, teilweise nur der Wert f"ur $t=0$. Letzteren kann man auch als Anfangsphase bezeichnen. Die zeitabh"angige Phase "andert sich mit der Rate $\omega_0$. Diese oben eingef"uhrte Gr"o"se wird als \EdmesEmph{Kreisfrequenz} bezeichnet. Die "Anderung der Phase im Bogenma"s pro Zeit ist hier keine Winkelgeschwindigkeit, kann aber als solche aufgefasst werden. So kann man sich auch den Zusammenhang zwischen Frequenz und Kreisfrequenz herleiten: W"ahrend einer Periode "andert sich der Winkel um $2\pi$, d. h. die Winkelgeschwindigkeit ist $\omega_0=2\pi/t_0=2\pi\nu_0$. Da der Winkel im Bogenma"s keine Einheit hat, sind Kreisfrequenz und Frequenz von der gleichen Dimension mit der SI-Einheit $\mse{s^{-1}}$. Es ist allerdings auch "ublich, die Kreisfrequenz in rad/s anzugeben, um den Unterschied zur Frequenz deutlich zu machen. F"ur die Frequenz ist wiederum die Einheit 1 Hertz = 1Hz = $\mse{s^{-1}}$ "ublich. Teilweise wird die Kreisfrequenz abk"urzend nur mit Frequenz bezeichnet.

\begin{MInfo}
Die L"osung der Bewegungsgleichung f"ur den harmonischen Oszillator sind harmonische, d. h. sinusf"ormige Schwingungen. Die Periode $t_0$ bzw. Frequenz $\nu_0$ bzw. Kreisfrequenz $\omega_0$ der Bewegung h"angt dabei von der Masse $m$ und der Federkonstante $k$ ab, nicht jedoch von der Anfangsauslenkung oder Anfangsgeschwindigkeit. Es gelten folgende Zusammenh"ange:

\begin{equation}
  \omega_0=\sqrt{k/m}=2\pi\nu_0=2\pi/t_0\;.
\end{equation}

\end{MInfo}

Um die vorgestellten Ergebnisse zu veranschaulichen, werden im folgenden Beispiele f"ur ein Modellsystem mit Einheitswerten f"ur die Parameter angegeben. In den "Ubungen sollen die Beispiele auf ein Modellsystem mit realistischeren Werten f"ur die Parameter "ubertragen werden.

\begin{MExample}
Als Beispiel mit einfachen Zahlenwerten wird eine Masse von 1 kg an einer langen, weichen Feder mit $k = 1\,\text{N/m}$ betrachtet. Die Masse werde um ein Meter ausgelenkt und dann losgelassen, s. Abb. \MRef{Physik_Schwingungen_AbbHOBsp}.

\begin{center}
  \MUGraphics{abbFederMasseBsp}{scale=1}{Mechanischer harmonischer Oszillator. Auf der linken Seite ist das System in Ruhe skizziert. Der Anfangszustand mit einer Auslenkung von 1 m ist rechts dargestellt. \MLabel{Physik_Schwingungen_AbbHOBsp}
}{width:600px;}
  \end{center}

Die Auslenkung als Funktion der Zeit ist z. B. durch Gl. \MRef{Physik_Schwingungen_EQHOLoes1} mit $a=1\mse{m}$ und $t_1 = 0\mse{s}$ gegeben. Die Kreisfrequenz in dem einfachen Modellsystem betr"agt $\omega_0=\sqrt{k/m}=1\mse{s^{-1}}$. Damit k"onnen die Auslenkung und die Geschwindigkeit $\frac{\MD x}{\MD t}$ berechnet (s. Aufgabe \MRef{Physik_Schwingungen_AufgabeLoesungHO}) und z. B.
f"ur die erste Periode (Dauer: $2\pi/\omega_0$) dargestellt werden:

\begin{center}
  \MUGraphics{abbBspTXTV}{scale=1}{Auslenkung (links) und Geschwindigkeit (rechts) als Funktion der Zeit f"ur das Beispielsystem w"ahrend der ersten Schwingungsperiode.\MLabel{Physik_Schwingungen_AbbBspTXTV}
}{width:700px;}
  \end{center}

Durch die Einheitsmasse und -federkonstante erh"alt man die Einheitskreisfrequenz, die Periode in Sekunden dauert $2\pi$. Die maximale Geschwindigkeit hat durch die Einheitskreisfrequenz den gleichen Zahlenwert wie die maximale Auslenkung, also hier 1 m/s.

\end{MExample}

Berechnungen zu dem Feder-Masse-System mit realistischeren Parametern werden in den Aufgaben 
%>>>>>>
\MRef{Physik_Schwingungen_AufgabeXFTF}\MLabel{Physik_Schwingungen_VonAufgabeXFTF}
%<<<<<<
und
%>>>>>>
\MRef{Physik_Schwingungen_AufgabeOsziWaage}\MLabel{Physik_Schwingungen_VonAufgabeOsziWaage}
%<<<<<<
behandelt.

Da der analytische Ausdruck f"ur die Auslenkung als Funktion der Zeit bekannt ist, kann nun leicht die Geschwindigkeit als Funktion der Zeit durch Ableitung nach der Zeit bestimmt werden. Die Ableitung wird in Aufgabe \MRef{Physik_Schwingungen_AufgabeLoesungHO} besprochen. F"ur das System mit anf"anglich 10 mm Auslenkung lautet das Ergebnis
\begin{equation}
  \frac{\MD x }{\MD t} = -10\MEinheit{mm}\,\omega_0\,\sin\left(\omega_0t\right)\; .
\end{equation}

Mit der zus"atzlichen Angabe, dass die Schwingungsperiode eine halbe Sekunde betr"agt, soll in Aufgabe
%>>>>>>
\MRef{Physik_Schwingungen_AufgabeVMax}\MLabel{Physik_Schwingungen_VonAufgabeVMax}
%<<<<<<
die maximale Geschwindigkeit berechnet werden.


Neben der Bewegung ist die Energie des Systems von Interesse. Mit den analytischen Ausdr"ucken f"ur Auslenkung und Geschwindigkeit k"onnen nun die potentielle und kinetische Energie als Funktion der Zeit berechnet werden:
\begin{eqnarray}
  E_\mathrm{pot} &=& -\int_0^x F(x') \MD x' = k \int_0^x x' \MD x' = \frac{1}{2} k x^2\; ,\\
  E_\mathrm{kin} &=& \frac{1}{2} m \left(\frac{\MD x}{\MD t}\right)^2\; .
\end{eqnarray}
Unter Verwendung der ersten Schreibweise f"ur die L"osung wird folgendes Ergebnis f"ur die Gesamtenergie als Summe aus potentieller und kinetischer Energie erhalten:

\begin{MInfo}
F"ur die Summe aus potentieller und kinetischer Energie beim harmonischen Oszillator gilt:

\begin{equation}
  E_{\text{pot}} + E_{\text{kin}} = \frac{1}{2} k a^2 \cos^2\left( \omega_0(t-t_1)\right)+\frac{1}{2} m a^2 \omega_0^2 \sin^2\left(\omega_0(t-t_1)\right)\;.
\end{equation}

Der Anteil jeder Energieform oszilliert mit der halben Periode $t_0/2$. Die Erhaltung der Gesamtenergie in dem System ohne "au"sere Kr"afte zeigt sich durch Einsetzen von $k/m$ f"ur $\omega_0^2$:

\begin{equation}
  E_{\text{pot}} + E_{\text{kin}} = \frac{1}{2} k a^2 \left(\cos^2\left( \omega_0(t-t_1)\right)+ \sin^2\left(\omega_0(t-t_1)\right)\right) = \frac{1}{2} k a^2 \;.
\end{equation}

Im letzten Schritt wurde verwendet, dass f"ur jeden Winkel $\sin^2(\phi) + \cos^2(\phi) = 1$ gilt.

\end{MInfo}

\begin{MExample}
In der folgenden Abbildung sind der Verlauf der potentiellen und kinetischen Energie f"ur das Beispielmodell w"ahrend der ersten Schwingungsperiode dargestellt.

\begin{center}
  \MUGraphics{abbBetrachtungEpotEkinBsp}{scale=1}{Potentielle und kinetische Energie w"ahrend der ersten Schwingungsperiode f"ur das Beispielmodell.\MLabel{Physik_Schwingungen_AbbBetrachtungEpotEkinBsp}
}{width:600px;}
  \end{center}

\end{MExample}

In Aufgabe
%>>>>>>
\MRef{Physik_Schwingungen_AufgabeEpotEkin}\MLabel{Physik_Schwingungen_VonAufgabeEpotEkin}
%<<<<<<
wird der Verlauf der potentiellen und kinetischen Energie des Systems mit realistischeren Parametern betrachtet und mit dem Beispielmodell verglichen.


W"ahrend die Kreisfrequenz $\omega_0$ durch die Federkonstante und die Masse bestimmt wird, werden die beiden anderen Parameter in der L"osung durch die Anfangsbedingungen festgelegt. Eine Anfangsbedingung ist die Anfangsauslenkung $x_0$, in unserem Beispiel 10 mm. In der dritten Schreibweise f"ur die L"osung zeigt der Vergleich f"ur die Anfangszeit $t=0\mse{s}$, dass $A$ identisch mit $x_0$ ist. Die Anfangsauslenkung gibt gleichzeitig "uber das zweite Newtonsche Gesetz die Anfangsbeschleunigung vor. Die zweite Anfangsbedingung ist die Anfangsgeschwindigkeit, in unserem Beispiel Null. Die Geschwindigkeit berechnet sich in der dritten Schreibweise zu $-A \omega_0 \sin\left( \omega_0 t\right) + B \omega_0 \cos\left( \omega_0 t\right)$. Im allgemeinen Fall mit einer Anfangsgeschwindigkeit $v_0$ lautet also nach Vergleich bei $t=0\mse{s}$ der Zusammenhang $B=v_0/\omega_0$.

\end{MXContent}


%-------------------------------------------------------------------------------
\begin{MXContent}{Oszillator mit D"ampfung}{D"ampfung}{STD}%
\MLabel{Pysik_Schwingungen_MXDaempfung}
Nach der Untersuchung des idealisierten Modells ohne d"ampfende Reibung soll nun ein realistischeres Modell mit Reibung untersucht werden. F"ur einen einfachen Fall wird die L"osung angegeben.

Im Allgemeinen kann der Ausdruck f"ur die Reibung kompliziert sein. Ein einfacher Fall ist, dass die Reibungskraft vom Betrag konstant ist und stets entgegen der Geschwindigkeit gerichtet ist (s. Amontonssche Gesetze). Hier soll ein anderer einfacher Fall betrachtet werden, n"amlich dass die Reibungskraft entgegengesetzt und proportional der Geschwindigkeit ist. In mechanischen Systemen tritt dieser Fall auf, wenn ein Gegenstand langsam durch ein z"ahes Fluid bewegt wird und die Z"ahigkeit unabh"angig von der Bewegungsgeschwindigkeit ist (s. Newtonsches Fluid und Reynolds-Zahl). Wie sp"ater gezeigt wird, hat dieser Verlustterm auch die gleiche mathematische Gestalt wie der Verlustterm im elektrischen Schwingkreis.

Mit einer positiven Konstante $c$ k"onnen wir f"ur den Reibungsterm also $-c\, \frac{\MD x }{\MD t}$ schreiben. 

\begin{MInfo}
Die Bewegungsgleichung f"ur den Oszillator mit geschwindigkeitsproportionaler D"ampfung lautet

\begin{equation}\MLabel{Physik_Schwingungen_EQHONewtonD}
  m\, \frac{\MD^2 x }{\MD t^2} = -k\,x -c\, \frac{\MD x }{\MD t} \; .
\end{equation}
Division durch die Masse f"uhrt mit der Definition $c=m\gamma$ auf
\begin{equation}\MLabel{Physik_Schwingungen_EQHOD}
  \frac{\MD^2 x }{\MD t^2} = - \gamma \, \frac{\MD x }{\MD t} - \omega_0^2\,x  \; .
\end{equation}

\end{MInfo}

\MOnlineOnly{
  \begin{MHint}{Bemerkung zu der Gleichung}
  Im Vergleich zu Gl. \MRef{Physik_Schwingungen_EQHO} ist ein linearer Term mit der Ableitung erster Ordnung hinzugekommen. Dies "andert nichts am Typ der Differentialgleichung (gew"ohnliche explizite lineare homogene Differentialgleichung zweiter Ordnung mit konstanten Koeffizienten), auch wenn die L"osung komplexer wird und schlecht geraten werden kann. Hier f"uhrt folgender Ansatz zum Ziel: $x=\exp(\lambda t),\,\lambda\in\C$. \\%wg fehlendem Leerzeichen im html
  Einsetzen in Gl. \MRef{Physik_Schwingungen_EQHOD} und K"urzen von $\exp(\lambda t)$ f"uhrt auf die Bestimmungsgleichung 
  \begin{displaymath}
  \lambda^2 + \gamma \lambda + \omega_0^2 = 0
  \end{displaymath}
  f"ur $\lambda$ mit den L"osungen
  \begin{displaymath}
  \lambda_{1/2} = -\frac{\gamma}{2} \pm \MIU \sqrt{\omega_0^2-\left(\frac{\gamma}{2}\right)^2}\; ,
  \end{displaymath}
  falls $\omega_0^2 > \left(\frac{\gamma}{2}\right)^2$. Mit der Abk"urzung $\omega_d = \sqrt{\omega_0^2-\left(\frac{\gamma}{2}\right)^2}$ lautet die allgemeine L"osung
  \begin{displaymath}
  x = A_1 \exp\left(-\frac{\gamma}{2}t\right)\exp(\MIU\omega_d t) + A_2 \exp\left(-\frac{\gamma}{2}t\right)\exp(-\MIU\omega_d t)\; ,
  \end{displaymath}
  mit $A_1,\, A_2\in\C$. Damit die L"osungen wie erforderlich reell sind, m"ussen $A_1$ und $A_2$ zueinander komplex konjugiert sein. Unter Verwendung der Eulerformel und der Additionstheoreme ergeben sich drei Schreibweisen, analog zum Fall ohne D"ampfung.
  \end{MHint}
}

Die Gleichung kann verh"altnism"a"sig einfach gel"ost werden, wenn komplexe Zahlen verwendet werden. Da komplexe Zahlen erst zu einem sp"ateren Zeitpunkt in einem Modul des Fachbereichs Mathematik behandelt werden, wird hier die allgemeine L"osung ohne Herleitung angegeben:

 \begin{MInfo}
Die allgemeine L"osung von Gl. \MRef{Physik_Schwingungen_EQHOD} kann mit folgenden Schreibweisen angegeben werden:

\begin{eqnarray}
  x_h &=& a\exp\left(-\Mtfrac{\gamma}{2}t\right)\cos\left(\omega_d (t-t_1)\right)\quad\mathrm{oder}\MLabel{Physik_Schwingungen_EQHODLoes1}\\ %MINT text frac, nur im PDF
  x_h &=& a\exp\left(-\Mtfrac{\gamma}{2}t\right)\cos(\omega_d t+\phi_d)\quad\mathrm{oder}\\
  x_h &=& \exp\left(-\Mtfrac{\gamma}{2}t\right)\left(A\cos\left(\omega_d t\right)+B\sin\left(\omega_d t\right)\right)\MLabel{Physik_Schwingungen_EQHODLoes3}
\end{eqnarray}

mit

\begin{equation}
\omega_d = \sqrt{\omega_0^2-\left(\frac{\gamma}{2}\right)^2}\; .\MLabel{Physik_Schwingungen_EQOmegaD}
\end{equation}

Bei diesen L"osungen wird $\omega_0^2 > \left(\frac{\gamma}{2}\right)^2$ angenommen, f"ur den Fall einer im Vergleich zu der nat"urlichen Frequenz st"arkeren D"ampfung kommt es zu keiner Schwingung (Kriechfall).

\end{MInfo}

In einigen Quellen ist in Gleichungen \MRef{Physik_Schwingungen_EQHODLoes1} bis \MRef{Physik_Schwingungen_EQHODLoes3} f"ur die D"ampfung nur der Faktor $\exp\left(-\gamma t\right)$ zu finden. Dementsprechend wird in diesen Quellen bei der Bewegungsgleichung \MRef{Physik_Schwingungen_EQHOD} f"ur die Proportionalit"atskonstante $2\gamma$ statt $\gamma$ eingesetzt. Bei dieser Definition sind nat"urlich auch die folgenden Gleichungen entsprechend anzupassen.

Als Folge der geschwindigkeitsproportionalen Reibung wird die Oszillation exponentiell ged"ampft und die Oszillationsfrequenz wird verringert. Die L"osung ist in Gleichungen \MRef{Physik_Schwingungen_EQHODLoes1} bis \MRef{Physik_Schwingungen_EQHODLoes3} mit dem Index $h$ versehen, da sie in der Mathematik als homogene L"osung bezeichnet wird. Im folgenden Abschnitt wird die homogene L"osung als Einschwingvorgang in Erscheinung treten.

Das folgende Beispiel mit Einheitswerten, d. h. es gilt auch $\gamma = 1\mse{s^{-1}}$, veranschaulicht dies f"ur eine Situation, bei der schon nach einer Schwingungsperiode die Auslenkung auf einen Bruchteil gesunken ist.

\begin{MExample}
\begin{center}
  \MUGraphics{abbBspTXD}{}{Auslenkung als Funktion der Zeit f"ur das System mit geschwindigkeitsproportionaler D"ampfung und Einheitswerten f"ur die Parameter. Zum Vergleich ist das unged"ampfte System ebenfalls aufgenommen. \MLabel{Physik_Schwingungen_AbbBspTXD}
}{width:600px;}
  \end{center}

\end{MExample}

In Abb. \MRef{Physik_Schwingungen_AbbBspTXD} wurde f"ur die noch freien Parameter wie im Fall des harmonischen Oszillators $a=1\mse{m}$ und $t_1 = 0\mse{s}$ gew"ahlt. Die freien Parameter sind z. B. durch Angabe der Anfangsauslenkung und Ansfangsgeschwindigkeit festgelegt. Beim harmonischen Oszillator entsprach diese Wahl der Parameter einer Anfangsgeschwindigkeit von Null. Im Fall des Oszillators mit D"ampfung ergibt sich mit dieser Parameterwahl eine negative Anfangsgeschwindigkeit ($-a(\gamma/2)$), bedingt durch den exponentiell abfallenden Faktor. F"ur die Anfangsbedingung mit Geschwindigkeit Null sind die Parameter entsprechend anders zu w"ahlen. Auf die Bestimmung der Parameter wird am Ende des folgenden Abschnittes nochmals eingegangen.

F"ur das Feder-Masse-System mit realistischeren Parametern wird in Aufgabe
%>>>>>>
\MRef{Physik_Schwingungen_AufgabeTXD}\MLabel{Physik_Schwingungen_VonAufgabeTXD}
%<<<<<<
eine Situation betrachtet, bei der die D"ampfung pro Schwingung geringer ist.

\end{MXContent}


%-------------------------------------------------------------------------------
\begin{MXContent}{Erzwungene Schwingung mit D"ampfung}{Erzwungen}{STD}%
\MLabel{Physik_Schwingungen_MXErzwungen}

Zus"atzlich zu der Federkraft wird nun die Wirkung einer "au"seren Kraft $F(t)$ auf die Masse betrachtet. Gleichung \MRef{Physik_Schwingungen_EQHONewtonD} (zweites Newtonsches Gesetz) muss um diesen Term erweitert werden:

\begin{MInfo}
Die Bewegungsgleichung f"ur den Oszillator mit erzwungener Schwingung und geschwindigkeitsproportionaler D"ampfung lautet

\begin{equation}\MLabel{Physik_Schwingungen_EQHONewtonFD}
  m\, \frac{\MD^2 x }{\MD t^2} = -k\,x -c\, \frac{\MD x }{\MD t} +  F  \; .
\end{equation}
Division durch die Masse f"uhrt mit der Definition $c=m\gamma$ auf
\begin{equation}\MLabel{Physik_Schwingungen_EQHOFD}
  \frac{\MD^2 x }{\MD t^2} = - \gamma \, \frac{\MD x }{\MD t} - \omega_0^2\,x +  F/m  \; .
\end{equation}

\end{MInfo}

\MOnlineOnly{
  \begin{MHint}{Bemerkung zu der Gleichung}
  Im Unterschied zu Gl. \MRef{Physik_Schwingungen_EQHOD} ist nun mit $F$ ein Term (im Allgemeinen eine Funktion der Zeit) gegeben, der nicht mit der gesuchten Funktion $x$ oder einer ihrer Ableitungen verkn"uft ist. Daher ist nun die Differentialgleichung inhomogen. Insgesamt wird Gl. \MRef{Physik_Schwingungen_EQHOFD} also als gew"ohnliche explizite lineare inhomogene Differentialgleichung zweiter Ordnung mit konstanten Koeffizienten bezeichnet.
  \end{MHint}
}

Als einfacher Fall wird f"ur die "au"sere Kraft ebenfalls eine harmonische Zeitabh"angigkeit angesetzt:

\begin{equation}
  F(t) = F_0 \cos\left( \omega t - \Delta \right)\; .
\end{equation}
Die Kreisfrequenz der Anregung $\omega$ ist dabei im Allgemeinen verschieden von der \EdmesEmph{nat"urlichen} Frequenz des Systems $\omega_0 = \sqrt{k / m}$, auch \EdmesEmph{Eigenfrequenz} genannt. Die Phasenverschiebung $\Delta$ erlaubt, die Phasenlage der Anregungskraft beim Zeitpunkt $t=0$ festzulegen. Dies ist der Zeitpunkt, zu dem typischerweise die Anfangswerte f"ur Auslenkung und Geschwindigkeit gegeben sind. Negative Zeiten werden hier nicht betrachtet. Soll zum Beispiel ein sinusf"ormiger Zeitverlauf ab $t=0$ behandelt werden, ist $\Delta = \frac{\pi}{2}$ einzusetzen.

Eine M"oglichkeit, einen solchen Kraftverlauf technisch zu realisieren, w"are f"ur eine elektrisch geladene Masse gegeben. Diese k"onnte in ein entsprechend oszillierendes homogenes elektrisches Feld zwischen zwei Platten eingebracht werden. 

Eine harmonische Bewegung der Federaufh"angung hat ebenfalls eine harmonische "au"sere Kraft zur Folge. Die Auslenkung des Massenpunktes im Laborkoordinatensystem sei weiterhin mit $x$ bezeichnet. Neu ist zu spezifizieren, dass $x=0$ die Ruhelage des Massenpunktes definiert, wenn die Aufh"angung sich in der mittleren Lage befindet. Die Auslenkung der Aufh"angung in Richtung des Massenpunktes sei mit $\hat{x}\cos\left( \omega t - \Delta\right)$ beschrieben. Damit "andert sich der Ausdruck f"ur die r"ucktreibende Kraft der Feder zu $-k(x-\hat{x}\cos\left(\omega t - \Delta)\right)$. F"ur die "au"sere Kraft gilt daher

\begin{equation}
  F_0 = k\hat{x}\; .
\end{equation}
%%% s. http://www.walter-fendt.de/ph14d/resmath.htm 

Zun"achst gilt es, f"ur die Gleichung 

\begin{equation}
  \frac{\MD^2 x }{\MD t^2} + \gamma \, \frac{\MD x }{\MD t} + \omega_0^2\,x = \frac{F_0}{m} \cos\left( \omega t - \Delta\right)
  \MLabel{Physik_Schwingungen_EQHOFHarmD}
\end{equation}

mit geschwindigkeitsproportionaler D"ampfung und harmonischer "au"serer Kraft irgend eine L"osung $x_p$ zu finden. In der Mathematik wird diese eine partikul"are L"osung genannt. Hier f"uhrt der Ansatz zum Ziel, dass die Frequenz der Schwingung mit der Frequenz der Anregungskraft $\omega$ "ubereinstimmt, allerdings mit einer \EdmesEmph{Phasenverschiebung} $\theta$ gegen"uber der Phase der Anregungskraft. F"ur die Amplitude der Auslenkung der Masse wird ein Produktansatz gew"ahlt, bei dem die Amplitude der Anregungskraft als Proportionalit"atsfaktor eingeht:

\begin{equation}
  x_p = \rho F_0 \cos\left(\omega t - \Delta + \theta \right) \; .
\end{equation}

Mit der Erweiterung in die komplexen Zahlen kann verh"altnism"a"sig einfach gezeigt werden, welche Bedingungen $\rho$ und $\theta$ erf"ullen m"ussen, damit $x_p$ eine L"osung ist. Hier soll das Ergebnis ohne Herleitung angegeben werden:

\begin{MInfo}
F"ur den erzwungenen Oszillator mit einer der Geschwindigkeit proportionalen D"ampfung und der "au"seren Kraft $F_0\cos\left(\omega t - \Delta\right)$ lautet eine L"osung

\begin{equation}
  x_p = \rho F_0 \cos\left(\omega t - \Delta + \theta\right) \MLabel{Physik_Schwingungen_EQHReibung}
\end{equation}

mit einem Faktor $\rho$ und einer Phasenverschiebung $\theta$, die sowohl von der Frequenz als auch von der D"ampfung abh"angen. Es gilt

\begin{equation}
  \rho^2 = \frac{1}{m^2\left[(\omega^2-\omega_0^2)^2+\gamma^2\omega^2\right]}\MLabel{Physik_Schwingungen_EQRho}
\end{equation}

und 

\begin{equation}
  \tan\left(\theta\right) = \frac{-\gamma\omega}{\omega_0^2-\omega^2} \quad\text{mit}\quad \sin\left(\theta\right)\leq 0 \; .\MLabel{Physik_Schwingungen_EQTheta}
\end{equation}

\end{MInfo}

\MOnlineOnly{
    \begin{MHint}{Bemerkung zu der Gleichung}
    Mit einer Erweiterung in die komplexen Zahlen kann die Kraft als $F_0\exp\left(i\left(\omega t - \Delta\right)\right)$ geschrieben werden, wobei die physikalisch betrachtete Kraft nur durch den Realteil repr"asentiert wird. Der Ansatz f"ur die Auslenkungsfunktion wird ebenfalls in die komplexen Zahlen erweitert: $\rho F_0 \exp\left(\MIU\left(\omega t - \Delta + \theta \right)\right) = \rho F_0 \exp\left(\MIU\left(\omega t - \Delta \right)\right) \, \exp\left(\MIU \theta \right)$ . Beim Einsetzen in Gleichung \MRef{Physik_Schwingungen_EQHOFHarmD} ergeben sich aufgrund der Tatsache, dass die Gleichung linear ist und dass die Koeffizienten reell sind jeweils eine unabh"angige Gleichungen f"ur den Realteil und den Imagin"arteil, d. h. die Gleichung f"ur den Realteil ent"alt nur den Realteil der Kraft und der Auslenkung. Der Realteil der komplexen Auslenkung ist dann die gesuchte physikalische L"osung. Einsetzen der erweiterten Gr"o"sen in Gleichung \MRef{Physik_Schwingungen_EQHOFHarmD} liefert zun"achst
    \begin{displaymath}
        \rho F_0 \left( (\MIU\omega)^2 + \MIU\omega\gamma + \omega_0^2 \right) \exp\left(\MIU\left(\omega t - \Delta \right)\right) \, \exp\left(\MIU \theta \right)
            = \frac{F_0}{m} \exp\left(i\left(\omega t - \Delta\right)\right)\; .
    \end{displaymath}
    Durch K"urzen von $F_0$ und der gemeinsamen Exponentialfunktion lautet die nach den gesuchten Gr"o"sen aufgel"oste Gleichung 
    \begin{displaymath}
        \rho  \, \exp\left(\MIU \theta \right) = \frac{1}{m\left( \omega_0^2 - \omega^2 + \MIU\omega\gamma \right)}  .
    \end{displaymath}
    Das Betragsquadrat einer komlexen Zahl erh"alt man durch Multiplikation mit dem komplex Konjugierten, damit ergibt sich Gl. \MRef{Physik_Schwingungen_EQRho}. 
    Der Tangens der Phase der komplexen Zahl ist das Verh"altnis aus dem Imagin"ar- und Realteil. Letztere erh"alt man durch Erweitern des Bruchs mit dem komplex konjugierten des Nenners (sinnvollerweise ohne $m$):
    \begin{displaymath}
        \rho  \, \exp\left(\MIU \theta \right) = \frac{ \omega_0^2 - \omega^2 - \MIU\omega\gamma }
						      {m\left(\left(\omega_0^2 - \omega^2\right)^2 + \left(\omega\gamma\right)^2\right)}  \;.
    \end{displaymath}
    In dieser Form k"onnen der Imagin"ar- und Realteil abgelesen werden. Division und K"urzen liefert Gleichung \MRef{Physik_Schwingungen_EQTheta}. Weiterhin zeigt der negative Imagin"arteil, dass $\sin(\theta)$ negativ ist, womit $\theta$ im Intervall $[-\pi\; \pi]$ eindeutig festgelegt ist.
    \end{MHint}
}

Die L"osung mit Gleichung \MRef{Physik_Schwingungen_EQHReibung} erf"ullt im Allgemeinen nicht die betrachteten Anfangsbedingungen f"ur Auslenkung und Geschwindigkeit. Sie wird jedoch als eingeschwungene L"osung  bezeichnet, da L"osungen zu gegebenen Anfangsbedingungen sich exponentiell der eingeschwungenen L"osung ann"ahern. Bevor dies am Ende des Abschnitts behandelt wird, sollen zun"achst Eigenschaften von $x_p$ aus Gleichung \MRef{Physik_Schwingungen_EQHReibung} untersucht werden.

\begin{MExample}\MLabel{Physik_Schwingungen_BspD}
In der folgenden Abbildung ist die Amplitude $\rho F_0$ in Gl. \MRef{Physik_Schwingungen_EQHReibung} f"ur das Beispielmodell als Funktion der Frequenz dargestellt. Dabei wurde wiederum f"ur $\gamma$ der Einheitswert von $1\mse{s^{-1}}$ eingesetzt.

\begin{center}
  \MUGraphics{abbReibungBsp}{scale=1}{Amplitude bei der eingeschwungenen L"osung mit D"ampfung f"ur das Beispielmodell.\MLabel{Physik_Schwingungen_AbbReibungBsp}
}{width:600px;}
  \end{center}

Bei der Berechnung von $\rho F_0$ kann $F_0=k\hat{x}$ und $\frac{k}{m}=\omega_0^2$ genutzt werden. 

Einsetzen der nat"urlichen Kreisfrequenz als Kreisfrequenz der anregenden Kraft liefert $\hat{x}(\omega_0/\gamma)$ f"ur die Amplitude. Das Verh"altnis $\omega_0/\gamma$ sinkt mit zunehmender D"ampfung. Im Beispielmodell betr"agt es Eins, was einer starken D"ampfung entspricht. 

\end{MExample}

In Aufgabe 
%>>>>>>
\MRef{Physik_Schwingungen_AufgabeErzwungenD}\MLabel{Physik_Schwingungen_VonAufgabeErzwungenD}
%<<<<<<
wird der Verlauf f"ur das System mit realistischeren Parametern behandelt.

Die Amplitude startet bei $\omega=0$ mit dem Wert f"ur die maximale Auslenkung der Aufh"angung. Ist die Anregungsfrequenz $\omega$ gleich der nat"urlichen Frequenz $\omega_0$ verschwindet der erste Summand im Nenner. Der Fall $\omega=\omega_0$ wird \EdmesEmph{Resonanzfall} genannt, in dem f"ur beliebig kleine Reibung beliebig gro"se Amplitudenwerte erhalten werden. Diese treten in realen Systemen nicht auf, da bei entsprechend gro"sen Amplituden Nichtlinearit"aten auftreten, die gegebenenfalls mit der Zerst"orung des System einhergehen. Die maximale Amplitude wird allerdings bei einer Frequenz erreicht, die etwas kleiner als die nat"urliche Frequenz des Oszillators ist, n"amlich bei $\omega = \sqrt{\omega_0^2-(1/2)\gamma^2}$. Teilweise wird auch diese Frequenz als Resonanzfrequenz bezeichnet. Die Schwingungsamplitude geht gegen Null, wenn die Anregungsfrequenz viel gr"o"ser als die nat"urliche Frequenz ist.

\begin{MExample}
Die folgende Abbildung zeigt die Phasenverschiebung $\theta$ f"ur das Beispielmodell als Funktion der Frequenz. 

\begin{center}
  \MUGraphics{abbPhasenVerschiebungBsp}{scale=1}{Phasenverschiebung bei der eingeschwungenen L"osung mit D"ampfung f"ur das Beispielmodell mit starker D"ampfung.\MLabel{Physik_Schwingungen_AbbPhasenVerschiebungBsp}
}{width:600px;}
  \end{center}

\end{MExample}

Siehe Aufgabe 
%>>>>>>
\MRef{Physik_Schwingungen_AufgabePhasenVerschiebung}\MLabel{Physik_Schwingungen_VonAufgabePhasenVerschiebung}
%<<<<<<
f"ur das System mit realistischeren Parametern.

Die Phasenverschiebungen startet bei $\omega=0$ mit dem Wert Null, nimmt bei der Resonanzfrequenz $\omega_0$ den Wert $-\frac{\pi}{2}$ an und geht gegen $-\pi$, wenn die Anregungsfrequenz viel gr"o"ser als die Resonanzfrequenz wird. Bei dem hier gew"ahlten Vorzeichen in $x_p = \rho F_0 \cos\left(\omega t - \Delta + \theta \right)$ bedeutet ein negativer Wert f"ur $\theta$, dass die Funktionsdarstellung von $x_p$ gegen"uber der von $F(t)$ zu gr"o"seren Zeiten nach rechts verschoben ist. Dies entspricht einem \glqq Hinterherhinken\grqq\ der Schwingung gegen"uber der Anregung. 

Nach dieser Betrachtung der Eigenschaften der eingeschwungenen partikul"aren L"osung $x_p$ von Gleichung \MRef{Physik_Schwingungen_EQHOFHarmD} wird abschlie"sen auf die Erf"ullung von Anfangsbedingungen eingegangen. Im vorherigen Abschnitt wurde die homogene Gleichung \MRef{Physik_Schwingungen_EQHOD} ohne "au"sere Anregung analysiert. Sie wurde dort nach der h"ochsten Ableitung aufgel"ost aufgeschrieben und wird hier in der zu Gleichung \MRef{Physik_Schwingungen_EQHOFHarmD} analogen Schreibweise angegeben:

\begin{equation}%\MLabel{Physik_Schwingungen_EQHOD}
  \frac{\MD^2 x }{\MD t^2} + \gamma \, \frac{\MD x }{\MD t} + \omega_0^2\,x = 0  \; .
\end{equation}

Die allgemeine L"osung der homogenen Gleichung (d.h. ohne Anregung) wurde in den verschiedenen Schreibweisen in Gleichungen \MRef{Physik_Schwingungen_EQHODLoes1} bis \MRef{Physik_Schwingungen_EQHODLoes3} mit $x_h$ bezeichnet. Aufgrund der Linearit"at der betrachteten Bewegungsgleichung ist $x_p + x_h$ ebenfalls eine L"osung der inhomogenen Gleichung:

\begin{eqnarray}%\MLabel{Physik_Schwingungen_EQHOD}
  && \frac{\MD^2 (x_p + x_h) }{\MD t^2} + \gamma \, \frac{\MD (x_p + x_h) }{\MD t} + \omega_0^2\,(x_p + x_h) = \frac{F}{m}  \\
  & \Leftrightarrow & \frac{\MD^2 x_p }{\MD t^2} + \frac{\MD^2 x_h }{\MD t^2} 
                    + \gamma \, \frac{\MD x_p }{\MD t} + \gamma \, \frac{\MD x_h }{\MD t} 
                    + \omega_0^2\,x_p + \omega_0^2\,x_h = \frac{F}{m}
\end{eqnarray}

Die Summe aus dem zweiten, vierten und sechsten Term auf der linken Seite ergibt Null, da die $x_h$ L"osungen der homogenen Gleichung sind. Die Summe der verbleibenden drei Terme ergibt die rechte Seite, da $x_p$ eine L"osung der inhomogenen Gleichung ist. In der Mathematik wird gezeigt, dass $x_p + x_h$ die allgemeine L"osung der inhomogenen Gleichung ist. Im Folgenden wird auf die Angabe unterschiedlicher Schreibweisen verzichtet:

\begin{MInfo}
F"ur den erzwungenen Oszillator mit geschwindigkeitsproportionaler D"ampfung und harmonischer "au"serer Kraft $F_0 \cos\left(\omega t - \Delta\right)$ lautet die allgemeine L"osung

\begin{equation}\MLabel{Physik_Schwingungen_EQHODFAllg}
  x = x_p+x_h = \rho F_0 \cos\left(\omega t - \Delta + \theta\right) + a\exp\left(-\Mtfrac{\gamma}{2}t\right)\cos(\omega_d t+\phi_d)
\end{equation}

mit $\rho$ und $\theta$ gem"a"s Gleichungen \MRef{Physik_Schwingungen_EQRho} und \MRef{Physik_Schwingungen_EQTheta}. W"ahrend die Frequenz der eingeschwungenen L"osung gleich der Anregungsfrequenz ist, h"angt die Frequenz $\omega_d$ beim Einschwingvorgang gem"a"s Gleichung \MRef{Physik_Schwingungen_EQOmegaD} von der nat"urlichen Frequenz und der Reibung ab. Die Parameter $a$ und $\phi_d$ sind durch die Anfangsbedingungen festgelegt.

\end{MInfo}

\MOnlineOnly{
    \begin{MHint}{Bemerkung zu der Gleichung}
    Bei einem anderen g"angigen Modell f"ur die Reibung in turbulenten Str"omungen gilt das Superpositionsprinzip nicht mehr. Dort ist die Reibung dem Quadrat der Geschwindigkeit proportional. Sei $\zeta$ die Proportionalit"atskonstante und seien $x_p$ und $x_h$ weiterhin L"osungen der inhomogenen beziehungsweise der homogenen Bewegungsgleichung. Einsetzen der Summe in die inhomogene Bewegungsgleichung liefert
    \begin{eqnarray}%\MLabel{Physik_Schwingungen_EQHOD}
      && \frac{\MD^2 (x_p + x_h) }{\MD t^2} + \zeta \, \left(\frac{\MD (x_p + x_h) }{\MD t}\right)^2 + \omega_0^2\,(x_p + x_h) = \frac{F}{m}  \\
      & \Leftrightarrow & \frac{\MD^2 x_p }{\MD t^2} + \frac{\MD^2 x_h }{\MD t^2} 
		      + \zeta \, \left(\frac{\MD x_p }{\MD t}\right)^2  + 2\zeta \,\frac{\MD x_p }{\MD t}\frac{\MD x_h }{\MD t} + \zeta \, \left(\frac{\MD x_h }{\MD t}\right)^2 
		      + \omega_0^2\,x_p + \omega_0^2\,x_h\\
      && \quad = \frac{F}{m} \\
      & \Leftrightarrow & 2\zeta \,\frac{\MD x_p }{\MD t}\frac{\MD x_h }{\MD t} = 0\; ,
  \end{eqnarray}
    was falsch ist, wenn nicht eine Geschwindigkeit konstant Null ist.
    \end{MHint}
}

Aus der Anfangsauslenkung $x_0 = x(0)$ erh"alt man sofort durch Einsetzen von $t=0$ in Gleichung \MRef{Physik_Schwingungen_EQHODFAllg} eine Bestimmungsgleichung f"ur $a$ und $\phi_d$:

\begin{equation}\MLabel{Physik_Schwingungen_EQHODFX0}
  x_0 = \rho F_0 \cos\left(-\Delta + \theta\right) + a\cos(\phi_d) \; .
\end{equation}

Aus der Anfangsgeschwindigkeit $v_0$ erh"alt man nach Zeitableitung von Gleichung \MRef{Physik_Schwingungen_EQHODFAllg} und Einsetzten von $t=0$ die zweite Bestimmungsgleichung f"ur $a$ und $\phi_d$:

\begin{equation}\MLabel{Physik_Schwingungen_EQHODFV0}
  v_0 = - \omega\rho F_0\sin\left(-\Delta+\theta\right) 
    -a\left(\frac{\gamma}{2}\cos\left(\phi_d\right) + \omega_d \sin\left(\phi_d\right)\right)\; .
\end{equation}

L"ost man Gleichung \MRef{Physik_Schwingungen_EQHODFV0} und Gleichung \MRef{Physik_Schwingungen_EQHODFX0} nach dem Term mit $a$ auf und dividiert jeweils die linken und rechten Seiten durcheinander, erh"alt man die Gleichung

\begin{equation}
  \frac{\gamma}{2} + \omega_d \tan\left(\phi_d\right) = \frac{\omega\rho F_0 \sin\left(-\Delta+\theta\right)+v_0}{\rho F_0\cos\left(-\Delta+\theta\right) - x_0} \; ,
\end{equation}

die nach $\tan\left(\phi_d\right)$ bzw. $\phi_d$ aufgel"ost werden kann.

Danach kann schlie"slich $a$ zum Beispiel aus

\begin{equation}
  a = \frac{x_0-\rho F_0\cos\left(-\Delta+\theta\right)}{\cos\left(\phi_d\right)}
\end{equation}

berechnet werden.

Die folgende Abbildung zeigt die L"osung f"ur das System mit realistischeren Parametern, d. h. $\omega = 10\mse{s^{-1}}$, $\hat{x}=1\mse{mm}$, $\Delta = \frac{\pi}{2}$, $t_0 = \MZahl{0}{5}\mse{s}$ und $\gamma=4\mse{s^{-1}}$ f"ur die Anfangsbedingungen $x_0=0\mse{mm}$ und $v_0 = 0\mse{mm/s}$.

\begin{center}
  \MUGraphics{abbTXDF}{scale=1}{Erzwungene Schwingung f"ur das System mit realistischeren Parametern. Bei $t=0$ ruht die Masse in der Gleichgewichtslage. In gr"un gestrichelt ist die Bewegung der Aufh"angung dargestellt, die die "au"sere Kraft bewirkt.\MLabel{Physik_Schwingungen_AbbTXDF}
}{width:600px;}
  \end{center}

Das Hinterherhinken der Antwort gegen"uber der Anregung ist gut zu erkennen. W"ahrend der erste positive und der erste negative Ausschlag noch kleiner als die Amplitude der eingeschwungenen L"osung sind, liegt bei dem zweiten positiven und dem zweiten negativen Ausschlag sogar ein erkennbares "Uberschwingen vor.

F"ur weniger triviale Verl"aufe der "au"seren Kraft oder Abh"angigkeiten der Reibungskraft kann kaum erwartet werden, dass die Bewegungsgleichung analytisch l"osbar ist. Um den allgemeinen Fall mit bekannter "au"serer Kraft oder Reibungskraft n"aherungsweise zu l"osen, wird die Bewegungsgleichung numerisch integriert. Ausf"uhrungen hierzu mit Beispielprogrammen finden sich im zus"atzlichen Abschnitt 
\MRef{Pysik_Schwingungen_MXNumerik}.

Die Untersuchung der einfachen linearen Gleichungen ist dennoch ein wichtiger Bestandteil des Physikstudiums. Einerseits k"onnen hier analytische L"osungen gefunden und somit Einblicke in die Zusammenh"ange gewonnne werden. Andererseits lassen sich viele Ph"anomene n"aherungsweise durch lineare Gleichungen beschreiben. Viele interessante Effekte beruhen jedoch auf Nichtlinearit"aten, die nicht vernachl"assigt werden k"onnen.

\end{MXContent}


%-------------------------------------------------------------------------------
\begin{MExercises}\MLabel{Physik_Schwingungen_ExercisesSchwingungen}
%===============================================================================
\begin{MExercise}\MLabel{Physik_Schwingungen_AufgabeIntroTX}
Eine Masse an einer Feder werde ausgelenkt und zum Zeitpunkt $t=0$ losgelassen. Die anf"angliche Auslenkung sei $x(t=0)=10\MEinheit{mm}$. Nach einer halben Sekunde kehrt die Masse erstmals wieder in die Ausgangsposition zur"uck. Skizzieren Sie die Auslenkung als Funktion der Zeit f"ur die ersten drei Sekunden.

\MOnlineOnly{
  \begin{MHint}{Bemerkung zu Diagrammen}
  Diagramme sind ein n"utzliches Instrument, um Messdaten oder physikalische Zusammenh"ange darzustellen. Es ist zu "uberlegen, welche Gr"o"se entlang der Ordinate ('y-Achse') als Funktion welcher anderen Gr"o"se entlang der Abszisse ('x-Achse') dargestellt wird. Hier soll die Auslenkung als Funktion der Zeit skizziert werden. In einem anderen Zusammenhang kann es sinnvoll sein, die Zeit als Funktion des Weges darzustellen. Auch Funktionen meherer Gr"o"sen wie Masse multipliziert mit dem Quadrat der Geschwindigkeit sind m"oglich. Bei der Darstellung von Messdaten sind die Einheiten anzugeben. Dies kann in eckigen Klammern erfolgen. In diesem Modul werden die an den Achsen angegebenen physikalischen Gr"o"sen jeweils durch die verwendete Einheit geteilt. Mit dem Zahlenwert auf der jeweiligen Achse ergibt sich so eine Gleichung, z. B. $t$/s = 1, d. h. $t$ = 1 s. Dies entspricht der IUPAC-Empfehlung (E. R. Cohen \textit{et al.}, IUPAC Green Book, 3rd Edition, 2nd Printing, IUPAC \& RSC Publishing, Cambridge (2008)). Weiterhin kann es sinnvoll sein, physikalische Gr"o"sen durch Bezug auf eine Referenzgr"o"se zu 'entdimensionieren', z. B. $t/T$ mit $T$ als Dauer eine Schwingung. Bei Messdaten sollte ein Fehlerbalken die vorliegende Unsicherheit der Messdaten anzeigen. In der Darstellung sollte die gew"ahlte Diagrammgr"o"se ausgenutzt werden, d. h. der Darstellungsbereich sollte nicht viel gr"o"ser als der Diagramminhalt sein.
  \end{MHint}
}

\begin{MHint}{L"osung}
\begin{center}
  \MUGraphics{abbIntroTX}{}{Auslenkung als Funktion der Zeit.\MLabel{Physik_Schwingungen_AbbIntroTX}
}{width:600px;}
  \end{center}
\end{MHint}

Zur"uck zum \MSRef{Physik_Schwingungen_VonAufgabeIntroTX}{Text}
\end{MExercise}
%-------------------------------------------------------------------------------

%-------------------------------------------------------------------------------
\begin{MExercise}\MLabel{Physik_Schwingungen_AufgabeIntroTV}
Skizzieren Sie au"serdem die Geschwindigkeit als Funktion der Zeit f"ur die erste Sekunde.

\begin{MHint}{Hinweis}
  Versuchen Sie, den Verlauf der Geschwindigkeit aus Abb. \MRef{Physik_Schwingungen_AbbIntroTX} abzusch"atzen. Rechnerisch ist die Geschwindigkeit die erste Ableitung der Auslenkung nach der Zeit, $\frac{\MD x }{\MD t}$. Der analytische Ausdruck hierf"ur wird sp"ater im Modul behandelt.
\end{MHint}

\begin{MHint}{L"osung}
\begin{center}
  \MUGraphics{abbIntroTV}{}{Geschwindigkeit als Funktion der Zeit.\MLabel{Physik_Schwingungen_AbbIntroTV}
}{width:600px;}
  \end{center}
\end{MHint}

Zur"uck zum \MSRef{Physik_Schwingungen_VonAufgabeIntroTV}{Text}
\end{MExercise}
%-------------------------------------------------------------------------------

%-------------------------------------------------------------------------------
\begin{MExercise}\MLabel{Physik_Schwingungen_AufgabeLoesungHO}
Zeigen Sie, dass die in Gln. \MRef{Physik_Schwingungen_EQHOLoes1}ff angegebenen Funktionen tats"achlich L"osungen von Gl. \MRef{Physik_Schwingungen_EQHO} sind!

\begin{MHint}{Hinweis}
  Setzen Sie dazu die Funktionen in die Differentialgleichung ein, bilden Sie die Ableitung und pr"ufen Sie, dass die Summe Null ergibt. Da sich die drei Schreibweisen der L"osung ineinander "uberf"uhren lassen, gen"ugt es, eine Form der L"osung einzusetzen. 
\end{MHint}

\begin{MHint}{L"osung}
  Man kann z. B. mit dem ersten Summanden der dritten Schreibweise beginnen. Die erste Ableitung des Kosinus ist das Negative vom Sinus, die innere Ableitung liefert als zus"atzlichen Faktor $\omega_0$ (Kettenregel): 
  \begin{equation}
    \frac{\MD (A\cos\left(\omega_0t\right)) }{\MD t} = -\omega_0 A \sin\left(\omega_0t\right)\; .
  \end{equation}
  Durch die zweite Ableitung wird der Sinus wieder zum Kosinus, die innere Ableitung liefert erneut einen Faktor $\omega_0$:
  \begin{equation}
    \frac{\MD (-\omega_0 A \sin\left(\omega_0t\right)) }{\MD t} = -\omega_0^2 A \cos\left(\omega_0t\right)\; .
  \end{equation} 
  Wie zu zeigen war, ist also die Summe aus der zweiten Ableitung der Funktion und der Funktion multipliziert mit $\omega_0^2$ f"ur diesen Term Null. Analog kann gezeigt werden, dass der zweite Summand in der dritten Schreibweise die Gleichung l"ost. Aufgrund der Linearit"at ist auch die "Uberlagerung beider Terme eine L"osung, was auch leicht gepr"uft werden kann.
\end{MHint}

Zur"uck zum \MSRef{Physik_Schwingungen_VonAufgabeLoesungHO}{Text}
\end{MExercise}
%-------------------------------------------------------------------------------

%-------------------------------------------------------------------------------
\begin{MExercise}\MLabel{Physik_Schwingungen_AufgabeLoesungHOUmrechnung}
Dr"ucken Sie einerseits in Gln. \MRef{Physik_Schwingungen_EQHOLoes1}ff den Zusammenhang zwischen $\phi$ in der zweiten Schreibweise und $t_1$ in der ersten Schreibweise und andererseits den Zusammenhang zwischen $A$ bzw. $B$ in der dritten Schreibweise und $\phi$ aus!

\begin{MHint}{L"osung}
Ausmultiplizieren in der ersten Schreibweise und Vergleich mit der zweiten Schreibweise liefert $\phi = - \omega_0t_1$. Die zweite Schreibweise kann mit dem Additionstheorem f"ur den Kosinus umgeformt werden. Vergleich mit der dritten Schreibweise ergibt $A=a\cos\left(\phi\right)$ und $B=-a\sin\left(\phi\right)$.
\end{MHint}

Zur"uck zum \MSRef{Physik_Schwingungen_VonAufgabeLoesungHOUmrechnung}{Text}
\end{MExercise}
%-------------------------------------------------------------------------------

%-------------------------------------------------------------------------------
\begin{MExercise}\MLabel{Physik_Schwingungen_AufgabeXFTF}
In Aufgabe \MRef{Physik_Schwingungen_AufgabeIntroTX} wurde angegeben, dass die Masse in dem Modellsystem mit realistischeren Parametern mit einer Anfangsauslenkung von 10 mm losgelassen wurde und nach einer halben Sekunde erstmals wieder in die Ausgangslage zur"uckkehrt. Nach L"osung der Bewegungsgleichung ist bekannt, dass die L"osung
\begin{equation}
  x(t) = 10\MEinheit{mm}\,\cos\left(\omega_0t\right)
\end{equation}
lautet, mit $\omega_0=2\pi/t_0=2\pi/\MZahl{0}{5} \, \mathrm{s} = \MZahl{12}{57}\MEinheit{rad/s}$. 

\MOnlineOnly{
  \begin{MHint}{Bemerkung zu signifikanten Stellen}
  Der letzte Zahlenwert ist nat"urlich gerundet. F"ur eine Periode von exakt 1/2 s in einem idealen System k"onnte man als Kreisfrequenz $4\pi\,\mathrm{rad/s}$ angeben. In einem realen System richtet sich die Anzahl anzugebender, d. h. signifikanter Stellen nach der ungef"ahren Mess- oder Rechengenauigkeit (falls keine genauen Angaben der Unsicherheit erforderlich sind). Die Anzahl signifikanter Stellen im Ergebnis sollte sich danach richten. Oben wurde das Ergebnis auf 4 signifikante Stellen gerundet. Das w"are angemessen, falls die Periode nicht ungenau mit einer halben Sekunde sondern mit \MZahl{0}{5000} s angegeben w"are.
  \end{MHint}
}

Es waren weder die Masse noch die Federkonstante spezifiziert - f"ur die Darstellung der Bewegung gen"ugen die Anfangsbedingungen und die Kreisfrequenz bzw. Periode, die aus der unbekannten Masse und Federkonstante resultiert. Nun sei noch die Federkonstante mit $k=10\MEinheit{N/mm}$ angegeben. Stellen Sie die Federkraft in diesem System einerseits als Funktion der Auslenkung und andererseits als Funktion der Zeit dar!

\begin{MHint}{L"osung} 
\begin{center}
  \MUGraphics{abbBetrachtungXFTF}{scale=1}{Federkraft als Funktion der Auslenkung (links) bzw. als Funktion der Zeit (rechts). Die maximale Federkraft entspricht etwa der Gewichtskraft einer Masse von 10 kg ($F=m g=10\,\text{kg}\, 9.81\,\text{N/kg}$). Die maximale Federkraft in dem Beispielmodell mit Einheitswerten ist trotz hundertfacher Auslenkung hundert Mal kleiner. \MLabel{Physik_Schwingungen_AbbBetrachtungXFTF}
}{width:700px;}
  \end{center}

\end{MHint}

Zur"uck zum \MSRef{Physik_Schwingungen_VonAufgabeXFTF}{Text}
\end{MExercise}
%-------------------------------------------------------------------------------

%-------------------------------------------------------------------------------
\begin{MExercise}\MLabel{Physik_Schwingungen_AufgabeOsziWaage}
Ist f"ur ein Feder-Masse-System die Federkonstante bekannt und kann die Schwingungsperiode gemessen werden, so kann aus diesen Werten die tr"age Masse bestimmt werden, das hei"st die Masse in dem Zusammenhang \glqq Kraft ist gleich Masse mal Beschleunigung\grqq. Daher kann die betrachtete Anordnung als Personenwaage verwendet werden. Die g"angige Waage im Badezimmer misst die Gewichtskraft und rechnet diese in die schwere Masse um, das ist die Masse in dem Zusammenhang \glqq Kraft ist gleich Masse mal Erdbeschleunigung\grqq. Aufgrund der "Aquivalenz von tr"ager und schwerer Masse erh"alt man das gleiche Ergebnis, solange die g"angige Waage nicht auf dem Mond eingesetzt wird (Unterschiede in der Erdbeschleunigung k"onnen gegen"uber Messfehlern und Schwankungen der K"orpermasse im Tagesverlauf vernachl"assigt werden). Unabh"angig von der Schwerkraft kann man sich morgens an die Feder aus unseren Aufgaben h"angen, ein wenig schwingen, um die Periode zu bestimmen, und dann die tr"age Masse direkt ausrechnen. Dies geht auch in einer Raumstation, solange die Masse des Raumfahrenden wesentlich kleiner ist als die der Raumstation. 
%\begin{center}
  \MUGraphics{abbWaage}{scale=1}{Verwendung einer Feder als Federwaage im Schwerefeld (links) und als Oszillationswaage (rechts). 
  \MLabel{Physik_Schwingungen_AbbWaage}
  }{width:600px;}
%\end{center}
Versuchen Sie zun"achst, eine Vorstellung von der Kraftkonstante zu bekommen. Wie weit w"urde Ihre Gewichtskraft die Feder aus Aufgabe \MRef{Physik_Schwingungen_AufgabeXFTF} auslenken? Berechnen Sie dann die Masse f"ur die in unserem System beobachtete Periode von \MZahl{0}{5000} s. Geben Sie die Masse in kg mit zwei signifikanten Stellen an. Zwar ist die Periode mit vier signifikanten Stellen angegeben, aber die Kraftkonstante der Feder nur mit zwei Stellen. Die Unsicherheit des Ergebnisses kann hier nicht kleiner sein als die der Federkonstante.

Bitte geben Sie hier das Ergebnis Ihrer Berechnung der K"orpermasse ein: $m$ / kg = \MQuestion{2}{63} .

\begin{MHint}{L"osung} 
Als Federwaage w"urde man die Federkraft mit der Gewichtskraft gleichsetzen: $k x=m g$ (das Vorzeichen ist hier nicht von Interesse). Um eine Vorstellung f"ur die Federkonstante zu bekommen, soll die Masse als bekannt vorausgesetzt werden und nach der Auslenkung aufgel"ost werden: $x=m(g/k)$. Als Einheit f"ur die Masse soll die SI-Einheit kg verwendet werden, die ja auch f"ur K"orpermassen gut geeignet ist (abgesehen von Geburtsanzeigen mit den "ublichen genaueren Angaben der noch kleinen Massen). Die Erdbeschleunigung $g$ in SI-Einheiten betr"agt \MZahl{9}{81} m/s$^2$ bzw. N/kg. Die Federkonstante wird demnach in 10$\cdot 10^3$ N/m umgerechnet. Das Verh"altnis $g/k$ ist also knapp ein Tausendstel Meter pro Kilogramm, d. h. die Auslenkung in Millimeter ist knapp gleich der K"orpermasse in Kilogramm.

Die Masse soll hier aus der Schwingungsperiode mit der Gleichung
\begin{equation}
  m=k t_0^2/(2\pi)^2
\end{equation}
berechnet werden. Einsetzen der Werte und Runden auf zwei Stellen liefert 63 kg.

\end{MHint}

Zur"uck zum \MSRef{Physik_Schwingungen_VonAufgabeOsziWaage}{Text}
\end{MExercise}
%-------------------------------------------------------------------------------

%-------------------------------------------------------------------------------
\begin{MExercise}\MLabel{Physik_Schwingungen_AufgabeVMax}
Geben Sie die maximale Geschwindigkeit f"ur das System 
mit anf"anglich 10 mm Auslenkung 
und $\MZahl{0}{5}\MEinheit{s}$ Schwingungsperiode in mm/s mit drei signifikanten Stellen an: $\max (\frac{\MD x }{\MD t})$ / (mm/s) = \MQuestion{3}{126} .

\begin{MHint}{L"osung} 
Die Geschwindigkeit oszilliert mit der oben berechneten Kreisfrequenz von \MZahl{12}{57} rad/s. Die maximale Geschwindigkeit ist das Produkt aus Kreisfrequenz und maximaler Auslenkung, hier also gerundet 126 mm/s. In dem betrachteten System wird die maximale (positiv beim Aufsteigen) Geschwindigkeit erstmals beim zweiten Nulldurchgang der Auslenkung erreicht, wenn die Phase $\omega_0 t$ den Wert $3\pi/2$ hat. Diese Ergebnisse sind bereits in Aufgabe \MRef{Physik_Schwingungen_AufgabeIntroTV} dargestellt.
\end{MHint}


Zur"uck zum \MSRef{Physik_Schwingungen_VonAufgabeVMax}{Text}
\end{MExercise}
%-------------------------------------------------------------------------------

%-------------------------------------------------------------------------------
\begin{MExercise}\MLabel{Physik_Schwingungen_AufgabeEpotEkin}
Stellen Sie den Verlauf der potentiellen und kinetischen Energie des Oszillators mit 10\,mm Anfangsauslenkung und einer Kraftkonstante von 10 kN/m sowie einer halben Sekunde Schwingungsperiode als Funktion der Zeit w"ahrend der ersten Schwingungsperiode dar. Achten Sie auf eine korrekte Achsenbeschriftung. 

Angenommen, Sie m"ochten die Gesamtenergie der Schwingung zur Verf"ugung stellen, indem Sie auf einem Fahrradergometer 100 W leisten. Wie lange m"ussten Sie hierf"ur in die Pedalen treten? Versuchen Sie, bevor Sie rechnen, eine grobe Vorstellung zu entwickeln - eher lang, schwei"streibend, oder vielleicht nur kurz antreten? Um welchen Faktor "andert sich die Energie, wenn die Feder anf"anglich um 10\,cm statt 10\,mm ausgelenkt wird?

Zeitdauer bei 100 W in Sekunden (mit zwei signifikanten Stellen und Dezimalpunkt, nicht -komma) = \MQuestion{6}{0.005}.

"Anderungsfaktor in der Gesamterergie bei ge"anderter Auslenkung = \MQuestion{3}{100}.

\begin{MHint}{L"osung} 
\begin{center}
  \MUGraphics{abbBetrachtungEpotEkin}{scale=1}{Potentielle und kinetische Energie w"ahrend der ersten Schwingungsperiode.\MLabel{Physik_Schwingungen_AbbBetrachtungEpotEkin}
}{width:600px;}
  \end{center}

Durch die geringe Auslenkung ist die Gesamtenergie trotz der recht ''strammen'' Feder mit einer Kraftkonstante von 10 kN/m eher gering, n"amlich nur ein halbes Joule. Dies ist im "Ubrigen die gleiche Energie wie in dem Beispielmodell mit zehntausend Mal weicherer Feder und hundert Mal gr"o"serer maximaler Auslenkung. Daher unterscheiden sich Abb. \MRef{Physik_Schwingungen_AbbBetrachtungEpotEkinBsp} und Abb. \MRef{Physik_Schwingungen_AbbBetrachtungEpotEkin} nur in der Abszisse, nicht in der Ordinate. Bei einer Leistung von $P$ = 100 W ist die entsprechende Arbeit schon nach 5 ms erbracht ($t=W / P$). Bei einer z"ugigen Trittfrequenz von 90 Umdrehungen pro Minute bzw. \MZahl{1}{5} Hz drehen sich die Kurbeln in dieser Zeit um nur $360^\circ \cdot \MZahl{1}{5} \cdot \MZahl{0}{005} = \MZahl{2}{7}^\circ$. Da die maximale Auslenkung quadratisch in die Energie eingeht, erh"oht sich die Gesamtenergie bei 10 cm um den Faktor 100, die Kurbel muss sich entsprechend w"ahrend einer halben Sekunde drehen, bei \MZahl{1}{5} Hz entspricht dies einer dreiviertel Umdrehung.
\end{MHint}

Zur"uck zum \MSRef{Physik_Schwingungen_VonAufgabeEpotEkin}{Text}
\end{MExercise}
%-------------------------------------------------------------------------------

%-------------------------------------------------------------------------------
\begin{MExercise}\MLabel{Physik_Schwingungen_AufgabeTXD}
Stellen Sie f"ur das Feder-Masse-System mit einer Anfangsauslenkung von 10 mm und einer Schwingungsperiode von einer halben Sekunde die Auslenkung als Funktion der Zeit dar, wenn eine geschwindigkeitsproportionale D"ampfung mit $\gamma = 4\mse{s^{-1}}$ vorliegt.

\begin{MHint}{L"osung} 
\begin{center}
  \MUGraphics{abbTXD}{scale=1}{Auslenkung als Funktion der Zeit f"ur das System mit geschwindigkeitsproportionaler D"ampfung und realistischen Parametern. Zum Vergleich ist das unged"ampfte System ebenfalls aufgenommen.\MLabel{Physik_Schwingungen_AbbTXD}
}{width:600px;}
  \end{center}


\end{MHint}

Zur"uck zum \MSRef{Physik_Schwingungen_VonAufgabeTXD}{Text}
\end{MExercise}
%-------------------------------------------------------------------------------

%-------------------------------------------------------------------------------
\begin{MExercise}\MLabel{Physik_Schwingungen_AufgabeErzwungenD}
Stellen Sie analog zu Beispiel \MRef{Physik_Schwingungen_BspD} die Amplitude $\rho F_0$ als Funktion der Frequenz f"ur das Modell mit einer halbe Sekunde Schwingungsperiode und $\hat{x} = \MZahl{1}{0}\MEinheit{mm}$ dar. Setzen Sie dabei f"ur $\gamma$ den Wert $\MZahl{4}{0}\MEinheit{s^{-1}}$ ein.

\begin{MHint}{L"osung}
\begin{center}
  \MUGraphics{abbReibung}{scale=1}{Amplitude der eingeschwungenen L"osung mit D"ampfung.\MLabel{Physik_Schwingungen_AbbReibung}
}{width:600px;}
  \end{center}
Auch in diesem Modell ist die Schwingung stark ged"ampft, wenn auch weniger stark als im Bei\-spiel\-mo\-dell mit Einheitswerten. F"ur das Modell mit realistischeren Parametern kann man leicht zeigen, dass hier $\omega_0/\gamma=\pi$ gilt.
\end{MHint}

Zur"uck zum \MSRef{Physik_Schwingungen_VonAufgabeErzwungenD}{Text}
\end{MExercise}
%-------------------------------------------------------------------------------

%-------------------------------------------------------------------------------
\begin{MExercise}\MLabel{Physik_Schwingungen_AufgabePhasenVerschiebung}
Stellen Sie f"ur das Modell aus Aufgabe \MRef{Physik_Schwingungen_AufgabeErzwungenD} mit realistischeren Parametern die Phasenverschiebung $\theta$ als Funktion der Frequenz dar. 

\begin{MHint}{L"osung}
\begin{center}
  \MUGraphics{abbPhasenVerschiebung}{scale=1}{Phasenverschiebung bei der eingeschwungenen L"osung in dem Modell mit realistischeren Parametern, das bedeutet hier weniger D"ampfung.\MLabel{Physik_Schwingungen_AbbPhasenVerschiebung}
}{width:600px;}
  \end{center}

\end{MHint}

Zur"uck zum \MSRef{Physik_Schwingungen_VonAufgabePhasenVerschiebung}{Text}
\end{MExercise}
%-------------------------------------------------------------------------------

%-------------------------------------------------------------------------------

%===============================================================================
\end{MExercises}


%===============================================================================

%===============================================================================
\MSubsection{Wellen}\MLabel{Physik_Schwingungen_SubWellen}
\begin{MIntro}\MLabel{Physik_Schwingungen_IntroWellen}
Oszillierende Systeme k"onnen die Ausbreitung von Schwingungen im Raum bewirken. Die sich ergebenden Ph"anomene werden als \EdmesEmph{Wellen} bezeichnet. Wellen sind eine weit verbreitete, in verschiedenen Formen auftretende und im Detail komplexe Erscheinung. Beispielsweise werden Wellen, die an ein gasf"ormiges, fl"ussiges oder festes Medium gebunden sind, unter dem Begriff \EdmesEmph{mechanische Wellen} zusammengefasst. Des weiteren wird zwischen longitudinalen und transversalen Wellen unterschieden:
\begin{MInfo}
Wellen, die an ein gasf"ormiges, fl"ussiges oder festes Medium gebunden sind, werden unter dem Begriff mechanische Wellen zusammengefasst.
\begin{itemize}
  \item Longitudinalwellen k"onnen sich in fluiden (d. h. gasf"ormigen oder fl"ussigen) und festen Medien ausbreiten. Bei diesen erfolgen die Schwingungen im Medium in Richtung der Wellenausbreitung. Ein wichtiges Beispiel ist die Ausbreitung von Schall in Gasen, Fl"ussigkeiten oder Festk"orpern.
  \item Reine Transversalwellen treten nur in Festk"orpern auf. Hier schwingt das Medium senkrecht zur Ausbreitungsrichtung. Daher kann bei Transversalwellen eine \EdmesEmph{Polarisation} vorliegen, das bedeutet das Vorkommen von nur einer Schwingungsrichtung. Ein Beispiel sind gewisse seismische Wellen.
  \item Mischformen von longitudinalen und transversalen Wellen k"onnen auch in Fluiden auftreten, z. B. bestimmte Arten von Wasserwellen.
\end{itemize}
\end{MInfo}
Andere Typen von Wellen k"onnen sich auch im Vakuum ausbreiten. Auch hierzu ein wichtiges Beispiel:
\begin{MInfo}
Elektromagnetische Wellen sind Transversalwellen, bei denen gekoppelte elektrische und magnetische Felder senkrecht zur Ausbreitungsrichtung schwingen. 
\begin{itemize}
  \item Der Teil des Frequenzspektrums elektromagnetischer Wellen, der mit dem Auge wahrgenommen werden kann, wird als Licht bezeichnen.
  \item Zu niedrigeren Frequenzen schlie"sen an das sichtbare Spektrum die Frequenzbereiche der Infrarotstrahlung, Terahertzstrahlung, Mikrowellenstrahlung und Radiowellen an.
  \item Zu h"oheren Frequenzen folgen Ultraviolettstrahlung, R"ontgenstrahlung und Gammastrahlung.
\end{itemize}
\end{MInfo}
Weitere Beispiele f"ur Wellenausbreitung im Vakuum sind Materiewellen, die in der Quantentheorie behandelt werden, oder Graviatationswellen, deren Existenz von der allgemeinen Relativit"atstheorie vorhergesagt wird.

Im Zusammenhang mit Wellen treten vielf"altige Effekte auf, von denen im Folgenden zun"achst einige nur aufgez"ahlt werden:
\begin{itemize}
  \item Interferenz in Raum und Zeit,
  \item Amplituden- und Frequenzmodulation,
  \item Wellenz"uge,
  \item Stehende Wellen, Schwingungsmoden, Eigenfrequenzen,
  \item Normalschwingungen,
  \item Reflexion, Brechung, Beugung,
  \item Bugwellen, Sto"swellen, Oberfl"achenwellen etc.
\end{itemize}
Im letzten Teil dieses Moduls soll lediglich am Beispiel des Schalls behandelt werden, wie die Wellenausbreitung im Raum erfolgt, d. h. die zugrundeliegende Gleichung wird motiviert, und wie einfache L"osungen gestaltet sind.
\end{MIntro}


%===============================================================================
\begin{MXContent}{Einleitende Beispiele}{Beispiele}{STD}%
\MLabel{Physik_Schwingungen_MXWellenBsp}
Zun"achst sollen zwei einleitende Beispiele betrachtet werden.

In der Akustik werden Schallquellen h"aufig als Punktquellen und das Schallfeld als umgebende Kugelwelle gen"ahert. Die folgende Abbildung zeigt einen ($x$, $y$)-Schnitt durch das Schalldruckfeld zu verschiedenen Zeiten mit der Punktquelle in der Mitte: 

\begin{center}
\begin{tabular}{cc}
\MUGraphicsSolo{abbKugelWelle2}{}{width:400px;}
&
\MUGraphicsSolo{abbKugelWelle4}{}{width:400px;}
\\
\MUGraphicsSolo{abbKugelWelle6}{}{width:400px;}
&
\MUGraphicsSolo{abbKugelWelle8}{}{width:400px;}
\\
\end{tabular}
\MUGraphics{Bild_Platzhalter_fuer_Bildertitel}{scale=1}%
{Querschnitte durch das Schalldruckfeld einer Kugelwelle zu unterschiedlichen Zeiten
\MLabel{Physik_Schwingungen_AbbKugelwelle}}%
{width=40px}
\end{center}

\renewcommand\MOnlineOnlyErsatz{Im Online-Modul finden Sie an dieser 
    Stelle eine animierte Darstellung.
}
\MOnlineOnly{
  \begin{MHint}{Animierte Darstellung}
  \MUGraphicsSolo{abbKugelWelleAni}{}{}
  \end{MHint}
}

Die Schwankungen des Schalldrucks gegen"uber dem umgebenden Luftdruck sind in beliebigen Einheiten als Falschfarben dargestellt. In der Umgebung der Punktquelle treten Werte au"serhalb der Farbskala auf, diese wurden auf Null gesetzt. Der Schall ist ein Beispiel f"ur eine longitudinale Welle. Interessierte finden n"ahere Informationen unter\\
\verb%www.dega-akustik.de/publikationen/DEGA_Empfehlung_101.pdf%

Saiteninstrumente liefern ein weiteres Beispiel f"ur Wellenph"anomene. Die folgende Abbildung zeigt schematisch unterschiedliche Bewegungsformen einer gespannten und angeregten, z. B. gezupften Saite. Die Saite ist an den Endpunkten eingespannt und kann sich dort nicht bewegen. Oben im Bild ist die Grundschwingung dargestellt, bei einer nicht gegriffenen Gitarren-A-Saite entspricht sie einem Ton der H"ohe 110 Hz. Darunter ist die erste Oberschwingung gezeigt. Bei der Guitarre wird dieser angeregt, indem die Fingerkuppe in der Saitenmitte beim zw"olften Bund aufgelegt wird, wo sich der Knoten befindet - ohne zu greifen, d. h. ohne die Saite auf das Griffbrett zu dr"ucken. Der erzeugte Oberton, auch Flageolettton genannt, hat als Tonh"ohe 220 Hz. F"ur den n"achsten Flageolettton mit 330 Hz wird der Finger beim siebten Bund aufgelegt und f"ur den Folgenden mit der Frequenz des Standard-Kammertons von 440 Hz beim f"unften Bund. Bei der Zuordnung eines Knotens zu einem Bund ist zu beachten, dass die Bundstege nicht "aquidistant sind. H"ohere Obert"one lassen sich vergleichsweise schlecht anregen.

%\begin{center}
  \MUGraphics{abbHarmonicPartialsWiki}{scale=.5}{Schematische Darstellung unterschiedlicher Saitenschwingungen. Ganz oben ist die Grundschwingung ohne Knoten gezeigt, darunter Oberschwingungen mit zunehmender Anzahl von Knoten. \MLabel{Physik_Schwingungen_AbbHarmonicPartialsWiki}
}{width:350px;}
%\end{center}

Die Saitenschwingung ist ein Beispiel f"ur eine transversale Welle. Insbesondere handelt es sich um eine stehende Welle. Diese werden als "Uberlagerung zweier fortschreitender Wellen mit entgegengesetzter Bewegungsrichtung beschrieben.

\end{MXContent}


%-------------------------------------------------------------------------------
\begin{MXContent}{Wellengleichung an dem Beispiel Schall}{Wellengleichung}{STD}%
\MLabel{Physik_Schwingungen_MXWellengleichung}
Bei einer Schallwelle schwingen die Werte verschiedener physikalischer Gr"o"sen (abh"angige Variablen) in Raum und Zeit (unabh"angige Variablen). Hier werden die Verschiebung des Mediums infolge des Schalls $\chi$, die Dichte"anderung durch den Schall $\rho_e$ und die schallbedingte Druck"anderung $P_e$ als Funktion des Ortes (im eindimensionalen Fall $x$) und der Zeit $t$ betrachtet. Bei "ublichen Schallpegeln sind die Dichte- und Druck"anderung gegen"uber den Gleichgewichtswerten $\rho_0$ und $P_0$ sehr klein. 

\begin{MInfo}
Der Zusammenhang zwischen Druck und Dichte wird durch eine Zustandsgleichung beschrieben. F"ur sehr kleine "Anderungen des Drucks und der Dichte, bezeichnet mit $P_e$ und $\rho_e$, wird der Zusammenhang mit einer Proportionalit"at gen"ahert:

\begin{equation}
  P_e = \kappa \rho_e\; .\MLabel{Physik_Schwingungen_SchallZustand}
\end{equation}

\end{MInfo}

Die Dichte"anderung ist bedingt durch die r"aumlich unterschiedliche Verschiebung des Mediums - bei einer r"aumlich homogenen, d. h. "uberall gleichen Verschiebung w"urde sich die Dichte nicht "andern. Es wird zun"achst nur ein eindimensionales Problem mit der Ortsvariablen $x$ betrachtet, wiederum f"ur sehr kleine Verschiebungen und "Anderungen der Dichte. Der dann geltende Zusammenhang wird hier ohne Herleitung angegeben:

\begin{MInfo}
Die "Anderung der Dichte bezogen auf die Gleichgewichtsdichte, $\rho_e / \rho_0$, ist entgegengesetzt der "Anderung der Verschiebung pro "Anderung der Ortsvariablen, $\partial \chi / \partial x$. Auf\/l"osen nach der Dichte"anderung ergibt

\begin{equation}
  \rho_e = -\rho_0 \, \partial \chi / \partial x\; .\MLabel{Physik_Schwingungen_SchallDichte}
\end{equation}
\end{MInfo}

Bei der Schallwelle ist die Verschiebung $\chi$ eine Funktion von zwei Ver"anderlichen, hier $\chi(x,t)$. In Gl. \MRef{Physik_Schwingungen_SchallDichte} wird die erste Ableitung der Verschiebung nach dem Ort bei festgehaltener Zeit ben"otigt. Die Ableitung einer Funktion mehrerer Ver"anderlicher nach einer Ver"anderlichen bei festgehaltenen Werten f"ur die anderen Ver"anderlichen wird als partielle Ableitung bezeichnet. F"ur die partielle Ableitung wird statt des Symbols $\MD$ das Symbol $\partial$ verwendet, also im Fall der partiellen Ableitung der Verschiebung nach dem Ort $\partial \chi / \partial x$. 

Man beachte den Unterschied zu der mathematischen Beschreibung des Feder-Masse-Systems, bei dem $x$ als abh"angige Variable die Auslenkung der Masse bezeichnete, die nur von der einen unabh"angigen Variablen Zeit abhing.

Besteht an einem Ort eine "Anderung des schallbedingten Drucks pro "Anderung des Ortes, $\partial P_e / \partial x$, so resultiert daraus eine Kraft auf die Teilchen an diesem Ort. Dies bewirkt gem"a"s dem zweiten Newtonschen Gesetz eine Beschleunigung der Teilchen. Die Beschleunigung ist die zweite partielle Ableitung der schallbedingten Verschiebung nach der Zeit, $\partial^2 \chi / \partial t^2$. F"ur kleine "Anderungen gen"ugt es, bei der Masse der Teilchen in dem betrachteten Volumen die Gleichgewichtsdichte $\rho_0$ einzusetzen. Das zweite Newtonsche Gesetz wird f"ur diesen Fall wiederum ohne Herleitung angegeben:

\begin{MInfo}
F"ur kleine schallbedingte Verschiebungen $\chi$ im eindimensionalen Fall und kleine Beschleunigungen $\partial^2 \chi / \partial t^2$ lautet das zweite Newtonsche Gesetz

\begin{equation}
  \rho_0 \partial^2 \chi / \partial t^2 = -\partial P_e / \partial x\; .\MLabel{Physik_Schwingungen_SchallNewton}
\end{equation}

\end{MInfo}

Nun kann die Wellengleichung z. B. f"ur die Verschiebung erhalten werden. Daf"ur wird zun"achst die Dichte"anderung in Gl. \MRef{Physik_Schwingungen_SchallZustand} mit Gl. \MRef{Physik_Schwingungen_SchallDichte} durch die partielle Ableitung der Verschiebung nach dem Ort ersetzt:
\begin{equation}
  P_e = -\kappa \rho_0 \, \partial \chi / \partial x\; .
\end{equation}
Dieser Zusammenhang wird partiell nach dem Ort abgeleitet:
\begin{equation}
  \partial P_e / \partial x = -\rho_0 \kappa \, \partial^2 \chi / \partial x^2\; .
\end{equation}
Einsetzen in Gl. \MRef{Physik_Schwingungen_SchallNewton} und K"urzen durch die Gleichgewichtsdichte f"uhrt schlie"slich auf die gesuchte Gleichung:
\begin{MInfo}
F"ur die schallbedingte Verschiebung des Mediums $\chi$ lautet die Wellengleichung in einer Dimension ($x$)

\begin{equation}
  \partial^2 \chi / \partial t^2 = \kappa \, \partial^2 \chi / \partial x^2\; .\MLabel{Physik_Schwingungen_SchallWellenGl}
\end{equation}

Die Konstante $\kappa$ ist das Verh"altnis aus Druck- und Dichte"anderung in der um die Gleichgewichtswerte linearisierten Zustandsgleichung.
\end{MInfo}

\MOnlineOnly{
  \begin{MHint}{Bemerkung zu der Gleichung}
  Die Wellengleichung "ahnelt Gl. \MRef{Physik_Schwingungen_EQHO} f"ur den harmonischen Oszillator. Allerdings treten hier partielle Ableitungen auf, n"amlich die zweiten partiellen Ableitungen nach der Zeit bzw. dem Ort. Daher ist Gl. \MRef{Physik_Schwingungen_SchallWellenGl} eine partielle lineare homogene Differentialgleichung zweiter Ordnung mit konstanten Koeffizienten. Sie wird auch als d'Alembert Gleichung bezeichnet. Auch die Wellengleichung hat die wichtige Eigenschaft, linear zu sein, womit sich unterschiedliche L"osungen durch Linearkombination zu neuen L"osungen "uberlagern (superponieren) lassen.
  \end{MHint}
}

\end{MXContent}


%-------------------------------------------------------------------------------
\begin{MXContent}{L"osungen der Gleichung und Schallgeschwindigkeit}{L"osung}{STD}%
\MLabel{Physik_Schwingungen_MXLoesung}
Eine L"osung der Wellengleichung wird als Wellenfunktion bezeichnet. F"ur die Wellengleichung \MRef{Physik_Schwingungen_SchallWellenGl} ist die allgemeine L"osung die "Uberlagerung zweier fortschreitender Wellen, eine in Richtung zunehmender $x$-Werte und eine in Richtung abnehmender $x$-Werte:

\begin{MInfo}
Allgemeine L"osung der Wellengleichung \MRef{Physik_Schwingungen_SchallWellenGl} in einer r"aumlichen Dimension ($x$) am Beispiel Schall und schallbedingte Teilchenverschiebung:

\begin{equation}
  \chi(x,t) = f_1(x-c_\mathrm{s}t)+f_2(x+c_\mathrm{s}t)\; . \MLabel{Physik_Schwingungen_AllgLsgWellenGl}
\end{equation}

Die Geschwindigkeit $c_\mathrm{s}$ der fortschreitenden Wellen wird f"ur Schall als Schallgeschwindigkeit bezeichnet.
\end{MInfo}

Zun"achst soll erl"autert und an Beispielen gezeigt werden, dass es sich bei den Summanden in Gl. \MRef{Physik_Schwingungen_AllgLsgWellenGl} um fortschreitende Wellen handelt. Anschlie"send wird nachgewiesen, dass diese Wellenfunktion tats"achlich die Wellengleichung l"ost.

Die Funktion $f_1$ habe zum Zeitpunkt $t=0$ am Ort $x=0$ den Wert $f_1(0)=\chi_0$. Schreitet die Welle in Richtung zunehmender $x$-Werte fort, so sollte die Funktion $f_1$ den gleichen Wert $\chi_0$ zu einer sp"ateren Zeit $t_+>0$ an einem Ort $x_+>0$ annehmen. Das ist an dem Ort der Fall, f"ur den das Argument der Funktion $f_1$ ebenfalls Null ist, also $x_+-c_\mathrm{s}t_+=0$. Auf\/l"osen nach dem Ort ergibt $x_+=c_\mathrm{s}t_+$. Da die Schallgeschwindigkeit eine positive Gr"o"se ist, gilt, wie zu zeigen war, $x_+>0$, wobei sich der Funktionswert mit der Geschwindigkeit $c_\mathrm{s}$ fortbewegt hat.

Die Betrachtung der Funktion $f_2(x+c_\mathrm{s}t)$ erfolgt in der Aufgabe 
%>>>>>>
\MRef{Physik_Schwingungen_AufgabeWelleAbnehmendeX}\MLabel{Physik_Schwingungen_VonAufgabeWelleAbnehmendeX}.
%<<<<<<

\begin{MExample}\MLabel{Physik_Schwingungen_BspWelleZunehmend}
Ein Beispiel f"ur eine Funktion der Form $f_1(x-c_\mathrm{s}t)$ ist 
\begin{equation}
  f_1(x-c_\mathrm{s}t)=\sin(k_x(x-c_\mathrm{s}t))\,\exp\{-[(x-c_\mathrm{s}t)/(2\lambda)]^2\} \; . \MLabel{Physik_Schwingungen_EqWelleBspM}
\end{equation}
Der erste Faktor oszilliert  bei festem Ort mit der Zeit und ebenso zu fester Zeit mit dem Ort. Ausmultiplizieren des Sinus-Argumentes ergibt f"ur die Kreisfrequenz 
\begin{equation}
  \omega = k_x\, c_\mathrm{s}\;. 
\end{equation}
F"ur das in Abb. \MRef{Physik_Schwingungen_AbbWelleBspM} gezeigte Beispiel wurde die Frequenz eines hohen Tones von 10 Kilohertz gew"ahlt, also $\omega = 2\pi 10000\MEinheit{s}^{-1}$. Damit kann der sogenannte Wellenvektor $k_x$ mit der Schallgeschwindigkeit $c_\mathrm{s}=343\MEinheit{m/s}$ zu $k_x=\omega/c_\mathrm{s}=183\MEinheit{rad/m}$ berechnet werden. 

\begin{center}
  \MUGraphics{abbWelleBspM}{scale=1}{Welle, die in Richtung zunehmender $x$-Werte fortschreitet, $f_1(x-c_\mathrm{s}t)$. Darstellungen als Funktion der Ortes sind f"ur zunehmende Zeiten untereinander angeordnet. \MLabel{Physik_Schwingungen_AbbWelleBspM}
}{width:600px;}
  \end{center}

Die ausmultiplizierte Form zeigt weiter, dass der Wellenvektor f"ur die Schwingung im Raum die gleiche Rolle spielt wie die Kreisfrequenz f"ur die Schwingung in der Zeit. Eine andere Bezeichnung ist Kreiswellenzahl. Analog zu $\omega = 2\pi \nu = 2\pi / t_0$ gilt 
\begin{equation}
  k_x = 2\pi / \lambda \; ,
\end{equation}
die Wellenl"ange $\lambda$ im Ort entspricht also der Periode in der Zeit. 

Der zweite Faktor bewirkt eine einh"ullende Begrenzung des Oszillation. Hier wurde die Wellenl"ange $\lambda$ als Parameter f"ur die Breite der Begrenzung im Raum bzw. die Periode $\lambda/c_\mathrm{s}=t_0$ f"ur die Breite der Begrenzung in der Zeit gew"ahlt. 

\end{MExample}

Die Berechnung der Wellenl"ange ist Gegenstand der Aufgabe 
\MRef{Physik_Schwingungen_AufgabeWellenLaenge}\MLabel{Physik_Schwingungen_VonAufgabeWellenLaenge}

\begin{MExample}\MLabel{Physik_Schwingungen_BspWelleAbnehmend}
Ein Beispiel f"ur eine Funktion der Form $f_2(x+c_\mathrm{s}t)$ ist
\begin{equation}
  f_2(x+c_\mathrm{s}t)=\MZahl{1}{5}/(1+[(x+c_\mathrm{s}t)/\lambda]^2) \; . \MLabel{Physik_Schwingungen_EqWelleBspP}
\end{equation}
Dieses Beispiel enth"alt keinen oszillierenden Anteil. F"ur die Darstellung in Abb. \MRef{Physik_Schwingungen_AbbWelleBspP} wurde als charakteristische L"ange (\glqq Breite\grqq, s. \MRef{Physik_Schwingungen_AufgabeWelleBreite}) die Wellenl"ange $\lambda$ aus dem vorangehenden Beispiel gew"ahlt.

\begin{center}
  \MUGraphics{abbWelleBspP}{scale=1}{Welle, die in Richtung abnehmender $x$-Werte fortschreitet, $f_2(x+c_\mathrm{s}t)$. Darstellungen als Funktion des Ortes sind f"ur zunehmende Zeiten untereinander angeordnet. \MLabel{Physik_Schwingungen_AbbWelleBspP}
}{width:600px;}
  \end{center}

\end{MExample}

Der Einfluss von $\lambda$ auf die Breite wird in Aufgabe 
%>>>>>>
\MRef{Physik_Schwingungen_AufgabeWelleBreite}\MLabel{Physik_Schwingungen_VonAufgabeWelleBreite}
%<<<<<<
betrachtet.

\begin{MExample}
Beispiel f"ur die "Uberlagerung zweier in entgegengesetzter Richtung fortschreitender Wellen anhand der beiden vorangehenden Beispiele.

\begin{center}
  \MUGraphics{abbWelleBsp}{scale=1}{Welle, die sich aus der "Uberlagerung (rot) der in den Abbildungen \MRef{Physik_Schwingungen_AbbWelleBspM} und \MRef{Physik_Schwingungen_AbbWelleBspP} dargestellen Wellen (blau bzw. gr"un) ergibt. Darstellungen als Funktion des Ortes sind f"ur zunehmende Zeiten untereinander angeordnet. \MLabel{Physik_Schwingungen_AbbWelleBsp}
}{width:600px;}
  \end{center}

\end{MExample}

Nun wird gezeigt, dass die in Gl. \MRef{Physik_Schwingungen_AllgLsgWellenGl} angegebene Funktion tats"achlich die Wellengleichung l"ost. Dabei ist vorauszusetzen, dass die Funktion zwei mal partiell sowohl nach dem Ort als auch nach der Zeit abgeleitet werden kann. Einsetzen in die linke Seite von \MRef{Physik_Schwingungen_SchallWellenGl} liefert:
\begin{eqnarray*}
  \frac{\partial^2}{\partial t^2}\chi(x,t) &=& \frac{\partial^2}{\partial t^2} \left\{f_1(x-c_\mathrm{s}t)+f_2(x+c_\mathrm{s}t)\right\}\\
    &=& \frac{\partial^2}{\partial t^2}f_1(x-c_\mathrm{s}t)+\frac{\partial^2}{\partial t^2}f_2(x+c_\mathrm{s}t)\\
    &=& (-c_\mathrm{s})\frac{\partial}{\partial t}f'_1(x-c_\mathrm{s}t)+c_\mathrm{s}\frac{\partial}{\partial t}f'_2(x+c_\mathrm{s}t)\\
    &=& (-c_\mathrm{s})^2f'{'}_1(x-c_\mathrm{s}t)+c_\mathrm{s}^2f'{'}_2(x+c_\mathrm{s}t)\\
    &=& c_\mathrm{s}^2\left(f'{'}_1(x-c_\mathrm{s}t)+f'{'}_2(x+c_\mathrm{s}t)\right)\;.
\end{eqnarray*}
Es wurden die Liniarit"at der Ableitung und die Kettenregel benutzt. Durch Einsetzen in die rechte Seite erh"alt man
\begin{eqnarray*}
  \kappa\frac{\partial^2}{\partial x^2}\chi(x,t) &=& \kappa\frac{\partial^2}{\partial x^2} \left\{f_1(x-c_\mathrm{s}t)+f_2(x+c_\mathrm{s}t)\right\}\\
    &=& \kappa\frac{\partial^2}{\partial x^2}f_1(x-c_\mathrm{s}t)+\kappa\frac{\partial^2}{\partial x^2}f_2(x+c_\mathrm{s}t)\\
    &=& \kappa\frac{\partial}{\partial x}f'_1(x-c_\mathrm{s}t)+\kappa\frac{\partial}{\partial x}f'_2(x+c_\mathrm{s}t)\\
    &=& \kappa\left(f'{'}_1(x-c_\mathrm{s}t)+f{'}'_2(x+c_\mathrm{s}t)\right)\;.
\end{eqnarray*}
Demnach ist $f_1(x-c_\mathrm{s}t)+f_2(x+c_\mathrm{s}t)$ genau dann eine L"osung der Schallwellengleichung, wenn $c_\mathrm{s}^2=\kappa$ gilt.

\begin{MInfo}
Aus dem Vorhergehenden und Gl. \MRef{Physik_Schwingungen_SchallZustand} kann die Schallgeschwindigkeit f"ur Gase aus der Zustandsgleichung berechnet werden:
\begin{equation}
  c_\mathrm{s}=\sqrt{\frac{P_e}{\rho_e}}\;.
\end{equation}
Diese Berechnungen sind Gegenstand der Thermodynamik.
\end{MInfo} 
\end{MXContent}


%-------------------------------------------------------------------------------
\begin{MXContent}{Interferenz-Effekte}{Interferenz}{STD}%
\MLabel{Physik_Schwingungen_MXInterferenz}
Abschlie"send sollen an einem ganz einfachen Beispiel Effekte aufgezeigt werden, die sich bei der "Uberlagerung von Wellen ergeben k"onnen. Es wird die Summe zweier in die gleiche Richtung laufender rein oszillatorischer Wellen gleicher Amplitude betrachtet:

\begin{equation}
  \chi(x,t)=\cos(k_x(x-c_\mathrm{s}t))+\cos(\MZahl{0}{8}k_x(x-c_\mathrm{s}t))\;.\MLabel{Physik_Schwingungen_EqWelleInter}
\end{equation}
F"ur $k_x$ wird wie im vorherigen Abschnitt der Zahlenwert eingesetzt, der einer Frequenz von 10 kHz entspricht. Der zweite Summand hat somit einen Wellenvektor, der einer Frequenz von 8 kHz entspricht. Das Ergebnis ist in Abb. \MRef{Physik_Schwingungen_AbbWelleInter} gezeigt. Statt die Wellen wie bisher als Funktion der Ortskoordinate f"ur unterschiedliche Zeiten untereinander anzuordnen, ist hier die Zeit auf der Ordinate aufgetragen. Da eine dritte Dimension auf dem Papier oder Bildschirm fehlt, um $\chi(x,t)$ aufzutragen, werden die Funktionswerte durch Farben wiedergegeben (es entsteht ein sogenanntes Falschfarbenbild). Die Farbtafel neben dem Bild gibt die Zuordnung von Funktionswerten und Farben an. Es ist zu beachten, dass wie bei solchen Darstellungen "ublich die Zeit nun nach oben zunimmt (statt wie bei der bisherigen zeilenweisen Anordnung nach unten).

\begin{center}
  \MUGraphics{abbWelleInter}{scale=1}{"Uberlagerung zweier unendlich ausgedehnter ebener Schallwellen gleicher Amplitude, die in die gleiche Richtung fortschreiten. Beide Wellen sind Kosinus-Wellen, eine mit einer Frequenz von 10 kHz, die andere mit einer Frequenz von 8 kHz. \MLabel{Physik_Schwingungen_AbbWelleInter}
}{width:700px;}
  \end{center}

Betrachtet man die Welle bei fester Zeit als Funktion des Ortes oder an einem festen Ort als Funktion der Zeit, so beobachtet man ein eine abwechselnd anschwellende und dann wieder abschwellende Schwingung. Der Verlauf an einem Ort als Funktion der Zeit ist in Abb. \MRef{Physik_Schwingungen_AbbWelleInterVonT} gezeigt.

\begin{center}
  \MUGraphics{abbWelleInterVonT}{scale=1}{Schnitt durch den in Abb. \MRef{Physik_Schwingungen_AbbWelleInter} dargestellten Datensatz bei festem Ort, n"amlich der kleinsten dargestellen Ort (-255 mm). Das An- und Abschwellen einer h"oherfrequenten Schwingung (Schwebung) ist so gut zu erkennen. \MLabel{Physik_Schwingungen_AbbWelleInterVonT}
}{width:700px;}
  \end{center}

Die beobachteten Effekte beruhen darauf, dass bei unterschiedlichen Frequenzen die Summanden sich sowohl komplett ausl"oschen als auch verst"arken k"onnen. Dies wird als destruktive bzw. konstruktive Interferenz bezeichnet. Unterscheiden sich die Frequenzen zweier T"one nur wenig, so pulsiert die Lautst"arke nur langsam. Dies wird als Schwebung bezeichnet. Bei gro"sen Unterschieden werden zwei getrennte T"one geh"ort.

Der Sachverhalt wird deutlicher, indem Gl. \MRef{Physik_Schwingungen_EqWelleInter} mit Hilfe der sogenannten Additionstheoreme umgeschrieben wird. das Ergebnis lautet:

\begin{equation}
  \chi(x,t)=2\cos\left(\frac{(k_x+\MZahl{0}{8}k_x)(x-c_\mathrm{s}t)}{2}\right)\cos\left(\frac{(k_x-\MZahl{0}{8}k_x)(x-c_\mathrm{s}t)}{2}\right)\;.\MLabel{Physik_Schwingungen_EqWelleInterProd}
\end{equation}

Der erste Faktor ist eine Kosinusschwingung mit der mittleren Frequenz aus den Frequenzen der Summanden in der Summenschreibweise. Der zweite Faktor ist eine Kosinusschwingung mit der halben Differenz der Frequenzen in der Summenschreibweise. Der niederfrequente Faktor 'moduliert' die Amplitude der Schwingung mit der mittleren Frequenz.

Weitere Betrachtungen zu diesem Beispiel sind Gegenstand von Aufgabe
%>>>>>>
\MRef{Physik_Schwingungen_AufgabeWelleInterferenz}\MLabel{Physik_Schwingungen_VonAufgabeWelleInterferenz}.
%<<<<<<

\end{MXContent}


%-------------------------------------------------------------------------------
\begin{MExercises}\MLabel{Physik_Schwingungen_ExercisesWellen}
%===============================================================================
\begin{MExercise}\MLabel{Physik_Schwingungen_AufgabeWelleAbnehmendeX}
Erl"autern Sie, dass die Funktion $f_2(x+c_\mathrm{s}t)$ in Richtung abnehmender $x$-Werte fortschreitet. Beginnen Sie die "Uberlegung mit einem beliebigen Ausgangsort $x_a$ und einer beliebigen Anfangszeit $t_a$.

\begin{MHint}{L"osung}
Es sei $\chi_a = f_2(x_a+c_\mathrm{s}t_a)$. Zu einer sp"ateren Zeit $t_a+t_+$ hat $f_2$ den Wert $\chi_a$ an dem Ort $x$ f"ur den gilt
\begin{equation*}
  x+c_\mathrm{s}(t_a+t_+)=x_a+c_\mathrm{s}t_a\;,
\end{equation*}
d. h. an dem Ort, f"ur den das Argument den gleichen Wert hat. Auf\/l"osen nach dem Ort ergibt
\begin{equation*}
  x=x_a-c_\mathrm{s}t_+<x_a\;.
\end{equation*}
Damit ist gezeigt, dass jeder Funktionswert mit der Geschwindigkeit $c_\mathrm{s}$ in Richtung abnehmender $x$-Werte fortschreitet. 
\end{MHint}

Zur"uck zum \MSRef{Physik_Schwingungen_VonAufgabeWelleAbnehmendeX}{Text}
\end{MExercise}
%-------------------------------------------------------------------------------

%-------------------------------------------------------------------------------
\begin{MExercise}\MLabel{Physik_Schwingungen_AufgabeWellenLaenge}
Berechnen Sie die Wellenl"ange f"ur das Beispiel \MRef{Physik_Schwingungen_BspWelleZunehmend}!

\begin{MHint}{L"osung} Es gilt $\lambda = 2\pi/k_x = 2\pi/183\,\mathrm{m} = 34\,\mathrm{mm}$. In Wasser mit einer Schallgeschwindigkeit von 1500 m/s ist die Wellenl"ange bei gleicher Frequenz um den Faktor 1500/343 gr"o"ser. Um in der medizinischen Bildgebung eine gute Ortsauf\/l"osung zu erreichen, werden daher Frequenzen im Megahertzbereich eingesetzt, mit entsprechend kleineren Wellenl"angen. \end{MHint}

Zur"uck zum \MSRef{Physik_Schwingungen_VonAufgabeWellenLaenge}{Text}
\end{MExercise}
%-------------------------------------------------------------------------------

%-------------------------------------------------------------------------------
\begin{MExercise}\MLabel{Physik_Schwingungen_AufgabeWelleBreite}
Wie ver"andert sich die Breite im Beispiel \MRef{Physik_Schwingungen_BspWelleAbnehmend}, wenn $2\lambda$ statt $\lambda$ eingesetzt wird?

\begin{MHint}{L"osung} 
Die Funkton $f_2$ nimmt ihr Maximum (1.5) f"ur Argumente mit $x+c_\mathrm{s}t=0$ an. F"ur Argumente ungleich null ist der Funktionswert kleiner, da der Nenner gr"o"ser ist. Um mit $2\lambda$ statt $\lambda$ die gleiche Reduktion des Funktionswertes zu bekommen, muss das Argument den doppelten Wert annehmen, z. B. durch Verdoppelung des Ortes und der Zeit. Die Breite der Begrenzung erh"oht sich also. 
\end{MHint}

Zur"uck zum \MSRef{Physik_Schwingungen_VonAufgabeWelleBreite}{Text}
\end{MExercise}
%-------------------------------------------------------------------------------

%-------------------------------------------------------------------------------
\begin{MExercise}\MLabel{Physik_Schwingungen_AufgabeWelleInterferenz}
Laut Gl. \MRef{Physik_Schwingungen_EqWelleInterProd} ist die 'Tr"agerfrequenz' in Abb. \MRef{Physik_Schwingungen_AbbWelleInterVonT} der Mittelwert aus 10 kHz und 8 kHz, also 9 kHz. Pr"ufen Sie, dass dies laut der Abbildung auch gegeben ist. Weiterhin soll die Frequenz der Schwebung die halbe Differenz zeigen, also 1 kHz. Passt das zu dem, was Sie in der Abbildung ablesen?

\begin{MHint}{L"osung}
Ausgehend von einem Zeitpunkt mit Schwingungsamplitude Null (Knotenpunkt) werden \MZahl{4}{5} Schwingungen bis zum n"achsten Knoten gez"ahlt, der einen Zeitabstand von \MZahl{0}{5} ms hat. Die Tr"agerfrequenz betr"agt also 1/(\MZahl{0}{5} ms / \MZahl{4}{5}) = 9 kHz, in "Ubereinstimmung mit der Berechnung. Betrachtet man den zeitlichen Abstand zweier Maxima in der Lautst"arke, werden \MZahl{0}{5} ms erhalten. Dies entspricht der ganzen Frequenzdifferenz von 2 kHz statt der halben Frequenzdifferenz. Dieser scheinbare Widerspruch ist dadurch zu erkl"aren, dass negative Werte bei der niederfrequenten Kosinusmodulation auch mit negativen Werten der hochfrequenten Schwingung multipliziert werden und umgekehrt. Auch beim H"oren nimmt man daher die doppelte Frequenz f"ur die Modulation wahr. Bei dem gro"sen Frequenzunterschied aus dem Beispiel w"urden allerdings zwei getrennte T"one statt einer Schwebung geh"ort.  
\end{MHint}

Zur"uck zum \MSRef{Physik_Schwingungen_VonAufgabeWelleInterferenz}{Text}
\end{MExercise}
%-------------------------------------------------------------------------------
%===============================================================================
\end{MExercises}


%===============================================================================

%===============================================================================
\MSubsection{Zus"atzliches}
%===============================================================================
% 
%-------------------------------------------------------------------------------
\begin{MXContent}{Numerische Integration der Bewegungsgleichung}{Numerik}{STD}%
\MLabel{Pysik_Schwingungen_MXNumerik}
Um f"ur den Oszillator im allgemeinen Fall mit bekannter "au"serer Kraft oder Reibungskraft die Bewegungsgleichung n"aherungsweise zu l"osen wird diese numerisch integriert. 
Grunds"atzlich gilt:

\begin{MInfo}
Numerisch kann der Ort $x(t+\delta t)$ zum Zeitpunkt $t+\delta t$ iterativ aus dem Ort und der Geschwindigkeit $v=\MD x / \MD t$ zum Zeitpunkt $t$ berechnet werden:

\begin{equation}\MLabel{Physik_Schwingungen_EqEuler1}
  x(t+\delta t) = x(t) + v(t) \delta t  \; .
\end{equation}
Ebenso kann die Entwicklung der Geschwindigkeit aus der Beschleunigung $a=\MD^2 x / \MD t^2$ berechnet werden:
\begin{equation}\MLabel{Physik_Schwingungen_EqEuler2}
  v(t+\delta t) = v(t) + a(t) \delta t  \; . 
\end{equation}
Diese beiden Gleichungen beschreiben die Kinematik (Bewegung) der Masse. Die Betrachtung von Kr"aften als Ursache f"ur die Beschleunigung, $$a(t)=F(t)/m,$$ wird als Dynamik bezeichnet. Die Kr"afte, welche die Beschleunigung verursachen, k"onnen vom Ort $x(t)$ oder der Geschwindigkeit $v(t)$ abh"angen und auch "au"sere Kr"afte sein.

Die obige Berechnung von neuen Orten und Geschwindigkeiten zu jeweils gleichen Zeitpunkten wird Eulersches Polygonzugverfahren oder Euler-Verfahren genannt.

\end{MInfo}

Das Ergebnis ist umso genauer, je kleiner die Schrittweite $\delta t$ ist. Es ist in geeigneter Weise zu pr"ufen, ob das numerische Ergebnis belastbar ist, z. B. durch Vergleich der numerischen Rechnung an Modellen, bei denen das analytische Ergebnis bekannt ist, oder durch Vergleich mit Referenzexperimenten.

\MOnlineOnly{
  \begin{MHint}{Bemerkung zu den Gleichungen}
  In dem obigen Verfahren wird die gew"ohnliche Differentialgeichung zweiter Ordnung in ein System von zwei gew"ohnlichen Differentialgleichugen erster Ordnung umgeschrieben: eine mit der ersten Ableitung der Auslenkung, d. h. der Geschwindigkeit, und eine mit der ersten Ableitung der Geschwindigket, d. h. der Beschleunigung. Das einfache iterative Verfahren f"ur die Integration der jeweiligen Differentialgleichung aus dem Anfangswert und der Anfangssteigung wird als Eulersches Polygonzugverfahren oder auch als Methode der kleinen Schritte bezeichnet, wobei in jedem Iterationsschritt beide gekoppelte Gleichungen iteriert werden m"ussen.
  \end{MHint}
}

\begin{MInfo}
F"ur die numerische L"osung der Differentialgleichung k"onnen unterschiedliche Gleichungssysteme aufgestellt werden. Im vorliegenden Fall erh"alt man bei gleicher Schrittweite $\delta t$ erheblich genauere Ergebnisse mit dem durch folgende Gleichungen beschriebenen Verfahren:

\begin{eqnarray}
  x(t+\delta t) &=& x(t) + v(t+\delta t / 2) \delta t  \MLabel{Physik_Schwingungen_EqLF1}\\
  v(t+\delta t / 2) &=& v(t-\delta t / 2) + a(t) \delta t\\
  a(t) &=& F(t)/m\; .\MLabel{Physik_Schwingungen_EqLF3}
\end{eqnarray}
Die erste neue Geschwindigkeit wird mit 

\begin{equation}
  v(0+\delta t / 2) = v(0) + a(0)\delta t / 2\MLabel{Physik_Schwingungen_EqLF4}
\end{equation} 
berechnet.

Diese Berechnung von neuen Orten und Geschwindigkeiten auf zeitlich versetzten Gittern wird als Leapfrog-Verfahren bezeichnet.

\end{MInfo}

\MOnlineOnly{
  \begin{MHint}{Bemerkung zu den Gleichungen}
  Dieses Verfahren wird als Leapfrog oder Verlet Verfahren bezeichnet. Es ist im Allgemeinen genauer als das Eulersche Polygonzugverfahren. Insbesondere hat es den Vorteil, f"ur Bewegungsgleichungen ohne Verluste die Gesamtenergie zu erhalten. So stimmt die mit dem Leapfrog f"ur den harmonischen Oszillator berechnete Auslenkung als Funktion der Zeit nicht exakt mit der analytischen L"osung "uberein, die Auslenkung an den Umkehrpunkten aber schon.
  \end{MHint}
}


Der Unterschied in der mit beiden Verfahren erreichten Genauigkeit soll zun"achst an dem analytisch gel"osten Referenzfall des harmonischen Oszillators ohne "au"sere Kraft und ohne Reibung 
%(s. Abb. \MNRef{Physik_Schwingungen_AbbIntroTX} in der L"osung zu Aufgabe \MNRef{Physik_Schwingungen_AufgabeIntroTX}) 
mit einer halben Sekunde Schwingungsdauer exemplarisch gezeigt werden. F"ur die Beschleunigung gilt hier $a=-\omega_0^2 x$. Die folgende Tabelle zeigt die Ergebnisse numerischer Integrationen mit beiden Verfahren f"ur eine Schrittweite von $t_0 / 64 = \msz{7}{8125}\mse{ms}$.

\begin{MWTabular}{l|l|l|l|l|l|l|l}
Schritt & 
$\frac{x_{\text{Eul}}}{\mse{mm}}$ & 
$\frac{v_{\text{Eul}}}{\mse{mm/s}}$ & 
$\frac{x_{\text{analyt}}}{\mse{mm}}$ & 
$\frac{x_{\text{LF}}}{\mse{mm}}$ & 
$\frac{v_{\text{LF+}}}{\mse{mm/s}}$ & 
$\frac{\delta x_{\text{Eul}}}{\mse{mm}}$ & 
$\frac{\delta x_{\text{LF}}}{\mse{mm}}$\\\hline
0 & 10 & 0 &   & 10 & -6,169 &  & \\
1&10,000&-12,337&9,952&9,952&-18,446&0,048&0,000\\
2&9,904&-24,674&9,808&9,808&-30,546&0,096&0,000\\
3&9,711&-36,892&9,569&9,569&-42,351&0,141&0,000\\
4&9,423&-48,872&9,239&9,238&-53,748&0,184&-0,001\\
5&9,041&-60,497&8,819&8,818&-64,627&0,222&-0,001\\
6&8,568&-71,651&8,315&8,313&-74,884&0,253&-0,001\\
7&8,008&-82,221&7,730&7,728&-84,418&0,278&-0,002\\
8&7,366&-92,101&7,071&7,069&-93,139&0,295&-0,002\\
9&6,647&-101,189&6,344&6,341&-100,962&0,303&-0,003\\
10&5,856&-109,389&5,556&5,552&-107,812&0,300&-0,003\\
11&5,001&-116,613&4,714&4,710&-113,623&0,287&-0,004\\
12&4,090&-122,783&3,827&3,822&-118,339&0,264&-0,004\\
13&3,131&-127,830&2,903&2,898&-121,914&0,228&-0,005\\
14&2,132&-131,692&1,951&1,945&-124,314&0,182&-0,005\\
15&1,104&-134,323&0,980&0,974&-125,516&0,123&-0,006\\
16&0,054&-135,685&0,000&-0,006&-125,508&0,054&-0,006\\
17&-1,006&-135,752&-0,980&-0,987&-124,291&-0,026&-0,007\\
18&-2,066&-134,511&-1,951&-1,958&-121,875&-0,116&-0,007\\
19&-3,117&-131,961&-2,903&-2,910&-118,285&-0,214&-0,007\\
20&-4,148&-128,115&-3,827&-3,834&-113,555&-0,321&-0,007\\
21&-5,149&-122,998&-4,714&-4,721&-107,730&-0,435&-0,007\\
22&-6,110&-116,645&-5,556&-5,563&-100,868&-0,554&-0,007\\
23&-7,021&-109,107&-6,344&-6,351&-93,032&-0,677&-0,007\\
24&-7,874&-100,445&-7,071&-7,078&-84,301&-0,803&-0,007\\
25&-8,658&-90,731&-7,730&-7,736&-74,756&-0,928&-0,006\\
26&-9,367&-80,049&-8,315&-8,320&-64,491&-1,053&-0,006\\
27&-9,993&-68,493&-8,819&-8,824&-53,605&-1,173&-0,005\\
28&-10,528&-56,165&-9,239&-9,243&-42,202&-1,289&-0,004\\
29&-10,967&-43,177&-9,569&-9,573&-30,392&-1,397&-0,003\\
30&-11,304&-29,647&-9,808&-9,810&-18,289&-1,496&-0,002\\
31&-11,536&-15,702&-9,952&-9,953&-6,010&-1,584&-0,001\\
32&-11,658&-1,470&-10,000&-10,000&6,327&-1,658&0,000
\end{MWTabular}

Bei Schritt Null sind in den Spalten mit numerischen Ergebnissen die Anfangswerte eingetragen. In der zweiten und dritten Spalte sind das f"ur das Euler-Verfahren die Anfangsauslenkung $x_{\text{Eul}}=10\mse{mm}$ bzw. die Anfangsgeschwindigkeit $v_{\text{Eul}}=0\frac{mm}{s}$. In der f"unften und sechsten Spalte sind die entsprechenden Werte f"ur das Leapfrog Verfahren angegeben, wobei die Anfangsgeschwindigkeit $v_{\text{LF+}}$ f"ur den Zeitschritt $\frac{\delta t}{2}$ gem"a"s Gleichung \MRef{Physik_Schwingungen_EqLF4} berechnet wurde.

Um zun"achst auf ein kleines, einfaches Programm zu verzichten und an verbreitetes Vorwissen anzukn"upfen, wurde die obige Tabelle auf einfachste Weise mit einer Tabellenkalkulation erstellt. Es wurde die kostenlose Tabellenkalkulation LibreOffice - Calc verwendet, die in der Bedienung dem g"angigen kommerziellen Produkt sehr "ahnlich ist. In den Formeln der benachbarten Zellen des Euler-Verfahrens steht sinngem"a"s

\begin{MWTabular}{c|c|c|c}
\dots & \dots & \dots & \dots \\\hline
\dots & = $x$ Zeile oben + $v$ Zeile oben mal $\delta t$ & = $v$ Zeile oben - $\omega_0^2$ mal $x$ Zeile oben mal $\delta t$ & \dots \\\hline
\dots & \dots & \dots & \dots
\end{MWTabular}

Die beiden Zellen k"onnen so oft wie gew"unscht untereinander kopiert werden. Bei dem Leapfrog-Verfahren ist es etwas komplizierter. Dort steht in den benachbarten Zellen sinngem"a"s

\begin{MWTabular}{c|c|c|c}
\dots & \dots & \dots & \dots \\\hline
\dots & = $x$ Zeile oben + $v$ Zeile oben mal $\delta t$ & = $v$ Zeile oben - $\omega_0^2$ mal $x$ \EdmesEmph{gleiche} Zeile mal $\delta t$ & \dots \\\hline
\dots & \dots & \dots & \dots
\end{MWTabular}

Wie die Tabellenkalkulation bewerkstelligt, dass in jeder Zelle jeweils der aktuell korrekte Wert berechnet und dargestellt wird, ist f"ur den Anwender unerheblich. Zwischen den numerischen Ergebnissen ist in der vierten Spalte das mit der analytischen Gleichung berechnete Ergebnis zum Vergleich angegeben. Die Abweichungen zwischen den numerischen und den analytischen Ergebnissen sind in der siebten und achten Spalte f"ur das Euler-Verfahren bzw. das Leapfrog Verfahren angegeben. Die folgende Abbildung zeigt das Ergebnis f"ur eine gr"o"sere Anzahl von Zeitschritten in einem Diagramm. Da auf der Diagrammskala das analytische Ergebnis mit dem Ergebnis des Leapfrog Verfahrens "ubereinstimmt, kann auf die Darstellung des analytischen Ergebnisses verzichtet werden.

\begin{center}
  \MUGraphics{abbNumVgl}{scale=1}{Numerisch berechnete Auslenkung im harmonischen Oszillator mit dem Leapfrog-Verfahren (magenta) und dem hier wesentlich ungenaueren Euler-Verfahren (blau). Die Schrittweite ist jeweils gleich, das Ergebnis f"ur jeden Zeitschritt ist als eigener Punkt dargestellt.\MLabel{Physik_Schwingungen_AbbNumVgl}
}{width:600px;}
  \end{center}

Obwohl jede Schwingungsperiode mit 64 Punkten verh"altnism"a"sig fein diskretisiert ist, liefert das Eulersche Polygonzugverfahren (Gln. \MRef{Physik_Schwingungen_EqEuler1} und \MRef{Physik_Schwingungen_EqEuler2}) einen mit der Zeit "uberproportional ansteigenden Fehler in der Auslenkung. Die folgende Abbildung zeigt das Verfahren noch schematisch:
\begin{center}
  \MUGraphics{abbEulerPoly}{}{Prinzip der numerischen Integration der Bewegungsgleichung mit dem Eulerschen Polygonzugverfahren.\MLabel{Physik_Schwingungen_AbbEulerPoly}
}{width:400px;} %exportiert mit 300 dpi
  \end{center}

Die Betrachtung eines Programmes f"ur die Umsetzung der numerischen Integration mit dem Eulerschen Polygonzugverfahren ist Gegenstand von Aufgabe 
%>>>>>>
\MRef{Physik_Schwingungen_AufgabeEuler}\MLabel{Physik_Schwingungen_VonAufgabeEuler}.
%<<<<<<

Bei dem Leapfrog Verfahren (Gln. \MRef{Physik_Schwingungen_EqLF1} bis \MRef{Physik_Schwingungen_EqLF3}) werden Ort und Geschwindigkeit auf verschobenen Zeitrastern abwechselnd berechnet, wobei es hier auf die Reihenfolge ankommt. Die folgende Abbildung zeigt das Verfahren noch schematisch:
\begin{center}
  \MUGraphics{abbLeapFrog}{}{Prinzip der numerische Integration der Bewegungsgleichung mit dem Leapfrog Verfahren.\MLabel{Physik_Schwingungen_AbbLeapFrog}
}{width:700px;} %exportiert mit 300 dpi, 700px (max)
  \end{center}

Auch hierf"ur wird die Umsetzung in ein Programm in Aufgabe 
%>>>>>>
\MRef{Physik_Schwingungen_AufgabeLF}\MLabel{Physik_Schwingungen_VonAufgabeLF}
%<<<<<<
behandelt.

Die numerische Simulation mit dem Leapfrog Verfahren soll nun genutzt werden, um an einem Beispiel das Einschwingen aus der Ruhelage sowie die ged"ampfte Schwingung nach Erreichen des eingeschwungenen Zustandes und Abstellen der "au"seren Kraft zu untersuchen. Mit 10 rad/s wird eine anregende Kreisfrequenz verwendet, die etwas unterhalb der nat"urlichen Kreisfrequenz liegt. In der numerischen Simulation wird die Aufh"angung der Feder "uber die Dauer von sieben Perioden des Oszillators ohne D"ampfung und "au"sere Kraft harmonisch bewegt. Die maximale Auslenkung der Aufh"angung betr"agt 1 mm. Danach wird die Aufh"angung f"ur die gleiche Dauer angehalten. Die folgende Abbildung zeigt sowohl die erzwungene Bewegung der Aufh"angung (gr"un gestrichelt) als auch die numerisch berechnete Auslenkung der Masse (blaue Datenpunkte):
\begin{center}
  \MUGraphics{abbNumLFLos}{}{Numerisch berechnete Auslenkung der Masse im Feder-Masse-System mit einer D"ampfung von $\gamma = 4/\text{s}$ und einer "au"seren Kraft, die aus der erzwungenen Bewegung der Federaufh"angung resultiert. Die Auslenkung der Federaufh"angung ist in Gr"un gestrichelt dargestellt, die daraus berechnete Auslenkung der Masse zu den gew"ahlten Zeitschritten mit blauen Punkten. Zum Vergleich ist die analytische L"osung f"ur den Fall kontinuierlicher Anregung in Magenta dargestellt. F"ur die Integration der Bewegungsgleichung wurde das Leapfrog Verfahren genutzt.\MLabel{Physik_Schwingungen_AbbNumLFlos}
}{width:600px;}
  \end{center}

In Abb. \MRef{Physik_Schwingungen_AbbReibung} in der L"osung zu Aufgabe \MRef{Physik_Schwingungen_AufgabeErzwungenD} ist zu erkennen, dass die maximale Auslenkung bei der eingeschwungenen L"osung f"ur eine anregende Kreisfrequenz von zehn reziproken Sekunden gut 2 mm betr"agt. Dies wird auch in der in Abb. \MRef{Physik_Schwingungen_AbbNumLFlos} gezeigten Simulation wiedergegeben. Dar"uberhinaus ist zu sehen, dass es f"ur die aus der Ruhe startende Masse zu einem geringf"ugigen "Uberschwingen kommt, bevor die eingeschwungene L"osung vorliegt. Es ist ebenfalls zu erkennen, dass die Schwingung der Masse gegen"uber der Schwingung der Aufh"angung leicht verz"ogert ist. Nachdem die Aufh"angung ruht, schwingt die Masse f"ur einige Zeit mit zunehmend ged"ampfter Amplitude weiter. Die analytische Rechnung ergibt, dass eine exponentielle D"ampfung mit der einh"ullenden Funktion $\exp\{-\gamma t /2\}$ vorliegt. Da die Auslenkung der Aufh"angung am Ende ihrer Bewegung nicht Null ist, geht die Auslenkung der Masse ebenfalls gegen einen entsprechend verschobenen Wert. Zum Vergleich ist noch die analytisch berechenbare L"osung f"ur den Fall kontinuierlicher Anregung dargestellt, s. auch Abbildung \MRef{Physik_Schwingungen_AbbTXDF}. In der Abbildung sind bei genauer Betrachtung Abweichungen zwischen der analytischen und der numerischen L"osung zu beobachten. Diese nehmen zu, wenn die Anzahl von Zeitschritten halbiert wird. Bei der doppelten Anzahl von Zeitschritten ist bei dieser Darstellung keine Abweichung mehr zu erkennen.

\end{MXContent}


%-------------------------------------------------------------------------------
\begin{MXContent}{Elektrische Schwingkreise}{RLC}{STD}%
\MLabel{Physik_Schwingungen_MXRLC}
Neben den Schwingungen in mechanischen Systemen sind auch elektrische Schwingkreise verbreitet und von gro"ser Bedeutung. Die ''Zutaten'' f"ur den mechanischen Oszillator sind Masse, Feder, D"ampfer und externe Kraft. F"ur den elektrischen Schwingkreis gibt es analoge ''Zutaten''. Diese werden im Folgenden vorgestellt. "Uberlegen Sie zun"achst selbst, welche elektrischen Bauteile die Zutaten darstellen k"onnten und welche physikalischen Gr"o"sen sie charakterisieren. Siehe hierzu Aufgabe
%>>>>>>
\MRef{Physik_Schwingungen_AufgabeRLCZutaten}\MLabel{Physik_Schwingungen_VonAufgabeRLCZutaten}.
%<<<<<<

In einem Stromkreis geht die Bewegung von Elektronen mit der Bewegung von Ladung einher. Die elektrische Ladung wird hier mit $q$ bezeichnet, die SI-Einheit ist das Coulomb, C (zu beachten: das Formelzeichen f"ur die Kapazit"at ist ebenfalls $C$, allerdings kursiv gesetzt). Die Ursache f"ur die Bewegung von Elektronen sind elektrische Felder bzw. eine elektrische Spannung $U$ zwischen zwei Punkten im Stromkreis. In einem Stromkreis k"onnen unterschiedlich geartete Bauteile miteinander verbunden werden.

Eine der Zutaten f"ur den elektrischen Schwingkreis ist der Kondensator, s. hierzu das Modul \glqq Elektrische Felder\grqq. Es gilt der Zusammenhang

\begin{equation}
  q = C U\; .\MLabel{Physik_Schwingungen_EqKondensator}
\end{equation}

\MOnlineOnly{
  \begin{MHint}{Weiterf"uhrende Bemerkungen zu Kondensatoren}
%   \begin{MInfo}
  Ein Kondensator kann negative und positive Ladung getrennt aufnehmen (d. h. "Uberschuss bzw. Mangel an Elektronen). Im einfachsten Fall besteht er aus zwei parallelen elektrisch leitenden Platten, die durch einen Isolator (Luft oder ein Dielektrikum) getrennt sind. Andere Aufbauten f"uhren bei kompakter Bauweise zu gro"sen Leiterfl"achen. Die getrennten Ladungen f"uhren zu einem elektrischen Feld bzw. zu einer elektrischen Spannung zwischen den Anschl"ussen des Kondensators. F"ur einen idealen Kondensator sind Ladung und Spannung proportional. Die Proportionalit"atskonstante $C$ wird als Kapazit"at bezeichnet:
  \begin{equation*}
  q = C U\; .
  \end{equation*}
  Das Symbol f"ur einen Kondensator deutet die beiden parallelen Platten an:
  \begin{center}
  \MUGraphics{abbKondensator}{}{Symbol f"ur einen Kondensator mit Kapazit"at $C$. Das Symbol deutet die Bauform zweier paralleler Platten an. F"ur einen solchen Plattenkondensator ist die Kapazit"at proportional zu der Plattenfl"ache und invers proportional zum Plattenabstand. In der Abbildung wird f"ur den Kondensator das Verbraucherz"ahlpfeilsystem angewendet, d. h. der Spannungspfeil zwischen den Anschlussklemmen hat die gleiche Richtung wie der Pfeil f"ur die Stromrichtung $I$ im Leiter, die gleich der (technischen) Bewegungsrichtung der Ladung $q$ ist. \MLabel{Physik_Schwingungen_AbbKondensator}
  }{width:100px;}
  \end{center}
  Die SI-Einheit f"ur die Kapazit"at ist das Farad, F. Typische Werte f"ur Kapazit"aten sind $10^{-12}$ bis einige $10^{-10}$ F, so dass Kapazit"aten "ublicherweise in Pico-Farad, pF, angegeben werden.

%   \end{MInfo}
  \end{MHint}
}

Wenn man einen geladenen Kondensator kurzschlie"st, entl"adt er sich schlagartig, so wie eine gespannte Feder sich schlagartig zusammenzieht, falls keine Masse am freien Ende befestigt ist. Dies "andert sich, wenn der geladene Kondensator an eine elektrische Spule angeschlossen wird. Der bei der Kondensatorentladung flie"sende Strom f"uhrt dazu, dass die Spule ein Magnetfeld aufbaut, was eine Gegenspannung induziert, s. hierzu das Modul \glqq Magnetische Felder und Induktion\grqq. Es gilt der Zusammenhang

\begin{equation}
  U = L \frac{\MD^2q }{\MD t^2}\; .\MLabel{Physik_Schwingungen_EqSpule}
\end{equation}

\MOnlineOnly{
  \begin{MHint}{Weiterf"uhrende Bemerkungen zu Spulen}
  Stromdurchflossene Leiter erzeugen ein Magnetfeld. Bauelemente, die die damit verbundenen Effekte nutzen, werden als elektrische Spulen bezeichnet. Das Magnetfeld ist proportional zu der Stromst"arke $I=\frac{\MD q }{\MD t}$. Eine "Anderung der Stromst"arke $\frac{\MD I }{\MD t} = \frac{\MD^2q }{\MD t^2}$ f"uhrt an den Anschl"ussen der Spule idealerweise zu einer proportionalen induzierten Gegenspannung. Die Proportionalit"atskonstante $L$ wird als Induktivit"at (genauer: Selbstinduktivit"at) bezeichnet:
  \begin{equation*}
    U = L \frac{\MD^2q }{\MD t^2}\; .
  \end{equation*}
  Das Symbol f"ur eine Spule deutet die Form einer Luftspule an:
  \begin{center}
    \MUGraphics{abbSpule}{}{Symbol f"ur eine Spule mit Induktivit"at $L$. Das Symbol deutet die Bauform einer Luftspule (Solenoid-Spule) an. In der N"aherung einer langen Spule ist die Induktivit"at Proportional der Windungszahl, Windungsdichte und der Querschnittsfl"ache. Wie in Abb. \MRef{Physik_Schwingungen_AbbKondensator} wurde f"ur die Spule das Verbraucherz"ahlpfeilsystem angewendet, andernfalls w"are in Gln. \MRef{Physik_Schwingungen_EqKondensator} und \MRef{Physik_Schwingungen_EqSpule} ein negatives Vorzeichen einzusetzen. \MLabel{Physik_Schwingungen_AbbSpule}
  }{width:140px;}
    \end{center}
  Die SI-Einheit f"ur die Induktivit"at ist das Henry, H. Typische Zahlenwerte sind hier nicht ganz so klein wie bei Kapazit"aten, Induktivit"aten werden h"aufig in Mikro-Henry, $\mu$H, angegeben.

  \end{MHint}
}

Um die einfache elektrische Verbindung des (anf"anglich geladenen) Kondensators und der (anfangs stromlosen) Spule zu einem Stromkreis mathematisch zu analysieren, werden die Kirchhoffschen Regeln (1845) angewandt.
\begin{center}
  \MUGraphics{abbLC}{}{Geschlossener elektrischer Stromkreis aus einem idealen Kondensator und einer idealen Spule. Die Stromrichtungen wurden so gew"ahlt, dass in jedem der beiden Knoten jweils ein Strom zuflie"st und ein Strom abflie"st. Damit ergeben sich im Verbraucherz"ahlpfeilsystem auch die Richtungen f"ur die Spannungspfeile.\MLabel{Physik_Schwingungen_AbbLC}
}{width:160px;}
\end{center}

Die Knotenregel (bzw. der Knotenpunktsatz oder das 1. Kirchhoffsche Gesetz) besagt, dass in einem Knotenpunkt eines elektrischen Netzwerkes die Summe der zuflie"senden Str"ome gleich der Summe der abflie"senden Str"ome ist. Die Anzahl der linear unabh"angigen Knotengleichungen ist gleich der Anzahl von Knoten weniger eins. F"ur die beiden Knoten in diesem ganz einfachen System erhalten wir die hier offensichtlichen Gleichungen $I_C=I_L$ bzw. $I_L=I_C$. Was f"ur die Str"ome gilt, gilt auch f"ur die zu- und abflie"sende Ladung, also die integrierten Str"ome. Damit kann in dem einfachen Stromkreis eine Variable $q$ in Gln. \MRef{Physik_Schwingungen_EqKondensator} und \MRef{Physik_Schwingungen_EqSpule} eingesetzt werden. 

Gem"a"s der Maschenregel (bzw. des Maschensatzes oder des 2. Kirchhoffschen Gesetzes) addieren sich alle Teilspannungen in einer Masche zu Null. Da beide Spannungspfeile hier in einer Umlaufrichtung gew"ahlt wurden, gilt $U_C+U_L$ = 0.

Einsetzen der Ausdr"ucke f"ur die Spannungen aus Gln. \MRef{Physik_Schwingungen_EqKondensator} und \MRef{Physik_Schwingungen_EqSpule} liefert die gesuchte mathematische Beschreibung:

\begin{MInfo}
In einem Stromkreis aus einem Kondensator mit Kapazit"at $C$ und einer Spule mit Induktivit"at $L$ lautet die Differentialgleichung f"ur die elektrische Ladung $q$

\begin{equation}\MLabel{Physik_Schwingungen_EQRLC}
  L\, \frac{\MD^2 q }{\MD t^2} + \frac{1}{C}\, q = 0 \; .
\end{equation}

Aus dem Vergleich mit der Differentialgleichung f"ur das Feder-Masse-System ist unmittelbar klar, dass die Ladung des Kondensators, der Strom und die Spannung harmonische Schwingungen mit der Kreisfrequenz
\begin{equation}
  \omega_0 = \sqrt{\frac{1}{L C}}
\end{equation}
ausf"uhren.
\end{MInfo}

Als Beispiel soll wieder ein System mit einfachen Zahlenwerten dienen, wobei eine Kapazit"at und Induktivit"at von einem Farad bzw. Henry extrem gro"s w"aren. Statt dessen wird ein Pico-Farad und ein Mikro-Henry betrachtet. In Aufgaben sollen die Rechnungen mit anderen Zahlenwerten ge"ubt werden.

\begin{MExample}
Ein Kondensator mit einer Kapazit"at $C$ von einem Pico-Farad wird mit einer Spule mit einer Induktivit"at $L$ von einem Mikro-Henry zu einem Stromkreis verbunden:
\begin{center}
  \MUGraphics{abbLCBsp}{}{Elektrischer Schwingkreis aus einem idealen Kondensator und einer idealen Spule. Die Induktivit"at entspricht der Masse in einem mechanischen Oszillator, das inverse der Kapazit"at der Federkonstante. Somit gilt $\omega_0^2 = 1/(L C)$.\MLabel{Physik_Schwingungen_AbbLCBsp}
}{width:160px;}
\end{center}
Das Produkt aus Kapazit"at und Induktivit"at betr"agt $1\cdot 10^{-12} \,\text{F}\cdot 1\cdot 10^{-6} \,\text{H} = 10^{-18}\,\text{s}^2$. Die nat"urliche Kreisfrequenz ist der Kehrwert der Quadratwurzel: $\omega_0 = 10^{9}\,\text{s}^{-1}$. Division durch $2\pi$ ergibt f"ur die lineare Frequenz, auf Mega-Hertz gerundet, 159 MHz. Dies ist im Frequenzbereich der Ultrakurzwellen (UKW) und in etwa die Arbeitsfrequenz von Kernspintomographen. Moderne Ganzk"orper-Systeme erzeugen Magnetfelder (genauer: magnetische Flussdichten oder magnetische Induktionen) von 3 Tesla, T. Dies entspricht einer Kernspin-Resonanzfrequenz f"ur Wasserstoffatomkerne von $\MZahl{802}{5} \MEinheit{s^{-1}}$ als Kreisfrequenz bzw. \MZahl{127}{7} MHz als lineare Frequenz. 
\end{MExample}

\MOnlineOnly{
  \begin{MHint}{Bemerkung zu dem einfachen LC-Kreis}
  Die einfache elektrische Verbindung eines Kondensators und einer Spule mag als nutzlos erscheinen, abgesehen vom Interesse an der Theorie. Tats"achlich k"onnen solche elektrisch isolierten Schwingkreise aber mit Vorteil z. B. in der Kernspintomographie eingesetzt werden. Die Spule dient zur "Radiokommunikation" mit dem Untersuchungsobjekt. Die Ankopplung der Spule an die "ubrige Elektronik des Kernspintomographen erfolgt, "ahnlich wie bei einem Transformator, induktiv "uber eine zweite Spule (Gegeninduktivit"at).
  \end{MHint}
}

Ein Beispiel zur Berechnung der Resonanzfrequenz bei gegebenen Eingenschaften von Kondensator und Spule findet sich in Aufgabe
%>>>>>>
\MRef{Physik_Schwingungen_AufgabeLC}\MLabel{Physik_Schwingungen_VonAufgabeLC}.
%<<<<<<

Nach der freien Oszillation im idealen Feder-Masse-System und der ged"ampften Schwingung wurde in Abschnitt \MRef{Physik_Schwingungen_MXErzwungen} die erzwungene Schwingung betrachtet, z. B. mit einer "au"seren Kraft, die sich aus einer harmonischen Bewegung der Federaufh"angung ergibt.
Das Analogon zu der "au"seren Kraft ist im elektrischen Schwingkreis eine Spannungsquelle, die an ihren Anschl"ussen einen definierten Spannungsverlauf vorgibt, idealerweise unabh"angig von der angeschlossenen ''Last'' (daneben gibt es auch Stromquellen, bei denen der zu- bzw. abflie"sende Strom vorgegeben werden soll). 

Hier ergeben sich nun zwei M"oglichkeiten. Ein Knoten in Abbildung \MRef{Physik_Schwingungen_AbbLC} kann ge"offnet und eine Spannungsquelle eingef"ugt werden, es ergibt sich ein Serienschwingkreis. Als zweite M"oglichkeit kann die Spannungsquelle an die beiden vorhandenen Knoten angeschlossen werden, dadurch wird ein Parallelschwingkreis erhalten:
\begin{center}
  \MUGraphics{abbLCU}{}{Elektrische Schwingkreise aus einem idealen Kondensator und einer idealen Spule mit "au"serer Spannungsquelle. Die Serienschaltung links entspricht dem mechanischen Feder-Masse-System mit "au"serer Kraft. Rechts ist der ebenfalls bedeutende Parallelschwingkreis dargestellt. F"ur die Spannungsquelle wurde jeweils das Erzeugerz"ahlpfeilsystem gew"ahlt. Die Richtungen der Strom- und entsprechenden Spannungspfeile k"onnte anders gew"ahlt werden, was bei entsprechender Ber"ucksichtigung der Vorzeichen in den Kirchhoffschen Regeln zum gleichen physikalischen Ergebnis f"uhrt.\MLabel{Physik_Schwingungen_AbbLCU}
}{width:600px;}
\end{center}

F"ur den \EdmesEmph{Serienschwingkreis} ergibt die Knotenregel, dass alle Str"ome bzw. alle zu- und abflie"senden Ladungen gleich sind: $I_L=I_C=I$, es gen"ugt wiederum eine Variable $q$ f"ur die Ladung. In der Maschenregel zeigt der Spannungspfeil f"ur die Spannungsquelle in eine andere Umlaufrichtung als die beiden anderen Spannungspfeile (Erzeugerz"ahlpfeilsystem), muss also in der Summe mit negativem Vorzeichen eingesetzt werden: $U_L+U_C-U=0$. Einsetzen der Spannungen als Funktion der Ladung f"uhrt wieder auf die mathematische Beschreibung:

\begin{MInfo}
In einem Serienschwingkreis aus einem idealen Kondensator mit Kapazit"at $C$, einer idealen Spule mit Induktivit"at $L$ und einer idealen Spannungsquelle mit regelbarer Spannung $U(t)$ lautet die Differentialgleichung f"ur die elektrische Ladung $q$

\begin{equation}\MLabel{Physik_Schwingungen_EQRLCU}
  L\, \frac{\MD^2 q }{\MD t^2} + \frac{1}{C}\, q = U \; .
\end{equation}

Division durch die Induktivit"at und Einsetzen der nat"urichen Kreisfrequenz f"ur den Schwingkreis ohne "au"sere Spannung f"uhrt auf 
\begin{equation}
  \frac{\MD^2 q }{\MD t^2} + \omega_0^2 q = \frac{U}{L} \; .
\end{equation}
F"ur $U=U_0\cos\left(\omega t\right)$ lautet eine L"osung:

\begin{equation}
  q = \left((U_0/L)/(\omega_0^2-\omega^2)\right) \cos\left(\omega t\right) \; .\MLabel{Physik_Schwingungen_EQLCULsg}
\end{equation}

\end{MInfo}


\begin{MExample}
Die folgende Abbildung zeigt die Amplitude der L"osung, Gl. \MRef{Physik_Schwingungen_EQLCULsg}, f"ur das Modellsystem (1 pF, 1 $\mu$H) als Funktion der Kreisfrequenz der "au"seren Spannung, wobei f"ur die maximale Spannung der Spannungsquelle die Einheitsspannung $U_0=1 \,\text{V}$ gew"ahlt wurde.

\begin{center}
  \MUGraphics{abbLCUBsp}{scale=1}{Amplitude bei der L"osung aus Gl. \MRef{Physik_Schwingungen_EQLCULsg} f"ur das Beispielmodell und eine maximale "au"sere Spannung von 1 V.\MLabel{Physik_Schwingungen_AbbLCUBsp}
}{width:600px;}
\end{center}

Ist die Kreisfrequenz klein gegen"uber der nat"urlichen Kreisfrequenz, so geht die Amplitude gegen die Ladung des Kondensators an der Spannungsquelle ohne die Spule, $q = C U_0 = 1\,\text{pC}= 1\,\text{pAs}$.

\end{MExample}

In der Elektrotechnik ist es "ublicher, den Strom, d. h. die Zeitableitung der flie"senden Ladung zu betrachten. F"ur die L"osung in Gl. \MRef{Physik_Schwingungen_EQLCULsg} erh"alt man damit f"ur den Strom
\begin{equation}
  I = -\frac{U_0}{L}\frac{\omega}{\omega_0^2-\omega^2} \sin\left(\omega t\right) \; .\MLabel{Physik_Schwingungen_EQLCULsgI}
\end{equation}
Anders als die Ladung geht der Strom gegen Null, wenn die Kreisfrequenz gegen Null geht. Der Kondensator ist zwar gem"a"s seiner Kapazit"at und der Spannung der externen Quelle geladen, allerdings flie"st f"ur Kreisfrequenz gegen Null kein Strom.

\begin{MExample}

\begin{center}
  \MUGraphics{abbLCUBspI}{scale=1}{Amplitude in dem Beispielschwingkreis f"ur den Strom statt f"ur die Ladung, d. h. Gl. \MRef{Physik_Schwingungen_EQLCULsgI}. \MLabel{Physik_Schwingungen_AbbLCUBspI}
}{width:600px;}
\end{center}

\end{MExample}

Ein weiteres Rechenbeispiel ist in Aufgabe
%>>>>>>
\MRef{Physik_Schwingungen_AufgabeLCUSerie}\MLabel{Physik_Schwingungen_VonAufgabeLCUSerie}
%<<<<<<
zu finden.

Geht die Frequenz der "au"seren Spannung gegen die nat"urliche Kreisfrequenz, so divergiert die Spannung am Kondensator, s. Gl. \MRef{Physik_Schwingungen_EqKondensator}. Ebenso divergiert die Spannung an der Spule, s. Gl. \MRef{Physik_Schwingungen_EqSpule}. Dabei gehen beide Spannungen so gegen plus bzw. minus Unendlich, dass die Summe 1 V bleibt.

F"ur den \EdmesEmph{Parallelschwingkreis} ergibt die Knotenregel mit den in Abb. \MRef{Physik_Schwingungen_AbbLCU} gew"ahlten Stromrichtungen $I=I_C+I_L$ bzw. $q=q_C+q_L$. Die Maschenregel besagt, dass alle Spannungen gleich sind, n"amlich $U=U_0\cos\left(\omega t\right)$. F"ur den Kondensator kann somit direkt die zeitabh"angige Ladung angegeben werden:
\begin{equation}
  q_C= C U_0 \cos\left(\omega t\right)\; .
\end{equation}
F"ur die Spule muss der Zusammenhang $\MD^2 q_L / \MD t^2 = (U_0/L)\cos\left(\omega t\right)$ integriert werden, mit dem Ergebnis
\begin{equation}
  q_L= -\frac{1}{\omega^2}\frac{U_0}{L} \cos\left(\omega t\right)\; .
\end{equation}
Beide Ladungen haben unterschiedliche Vorzeichen, um gleiche Vorzeichen zu erhalten, muss entweder beim Kondensator oder bei der Spule die Stromrichtung umgekehrt gew"ahlt werden. Mit den hier gew"ahlten Stromrichtungen ist die aus der Spannungsquelle flie"sende Ladung die Summe der beiden Beitr"age:

\begin{MInfo}
In einem Parallelschwingkreis mit idealem Kondensator $C$, idealer Spule $L$ und "au"serer Spannung $U=U_0\cos\left(\omega t\right)$ lautet eine L"osung f"ur die aus der Spannungsquelle flie"sende Ladung
\begin{equation}
q=\left(C U_0 -\frac{1}{\omega^2}\frac{U_0}{L}\right) \cos\left(\omega t\right)\; .
\end{equation}
Die kann unter Verwendung der nat"urlichen Kreisfrequenz $\omega_0$ umgeschrieben werden zu
\begin{equation}
q=\frac{U_0}{L} \left( \frac{1}{\omega_0^2}-\frac{1}{\omega^2}\right) \cos\left(\omega t\right)\; .
\end{equation}
Die Zeitableitung ergibt f"ur den entsprechenden Strom
\begin{equation}
I=\left(\frac{1}{\omega L}-\omega C\right) U_0 \sin\left(\omega t\right)\; .
\end{equation}
Im Resonanzfall heben sich die Str"ome durch den Kondensator und die Spule auf, es flie"st kein Strom aus der Spannungquelle. Geht die Frequenz gegen Null, so liefert die Spule einen Kurzschlusspfad, der Strom geht gegen Unendlich.

\end{MInfo}

\begin{MExample}
Die folgenden Abbildungen zeigen die Ladung bzw. den Strom, der durch die Leitungen zur Spannungsquelle flie"st, f"ur das einfache Beispiel eines Paralellschwingkreises (1 pF, 1 $\mu$H, 1 V).
\begin{center}
  \MUGraphics{abbLCUParBsp}{scale=1}{Amplitude f"ur die Ladung in der L"osung f"ur den Beispielschwingkreis mit einfachen Zahlenwerten. \MLabel{Physik_Schwingungen_AbbLCUParBsp}
}{width:600px;}
\end{center}

\begin{center}
  \MUGraphics{abbLCUParBspI}{scale=1}{Amplitude f"ur den Strom statt f"ur die Ladung. \MLabel{Physik_Schwingungen_AbbLCUParBspI}
}{width:600px;}
\end{center}

\end{MExample}

Ein weiteres Rechenbeispiel ist in Aufgabe
%>>>>>>
\MRef{Physik_Schwingungen_AufgabeLCUParallel}\MLabel{Physik_Schwingungen_VonAufgabeLCUParallel}
%<<<<<<
zu finden.

Schlie"slich soll f"ur den Serienschwingkreis, der dem mechanischen Feder-Masse-D"ampfer-Modell mit "au"serer Kraft analog ist, die D"ampfung in Form eines Ohmschen Widerstandes eingebaut werden. Der Zusammenhang zwischen Spannungsabfall und Stromst"arke wird im Ohmschen Gesetz (1826) beschrieben: 
\begin{equation}
U=R I\; .
\end{equation}

\MOnlineOnly{
  \begin{MHint}{Weiterf"uhrende Bemerkungen zu Widerst"anden}
  Abgesehen von Supraleitern f"allt die Spannung entlang ausgedehnter Materialien ab, wenn sie von einem konstanten Strom durchflossen werden. Bei einem idealen Widerstand besteht eine Proportionalit"at zwischen Spannungsabfall und Stromst"arke. Die Proportionalit"atskonstante wird als Widerstand $R$ bezeichnet:
  \begin{equation*}
  U=R I\; .
  \end{equation*}
  Die SI Einheit f"ur den elektrischen Widerstand ist das Ohm, $\Omega$. Als Symbol wird meist ein Rechteck verwendet:
  \begin{center}
    \MUGraphics{abbWiderstand}{}{Symbol f"ur einen Widerstand mit Widerstand (Bauteil und charakteristische physikalische Gr"o"se haben hier die gleiche Bezeichnung) $R$. F"ur einen Widerstand ist das Verbraucherz"ahlpfeilsystem angemessen. \MLabel{Physik_Schwingungen_AbbWiderstand}
  }{width:140px;}
    \end{center}

  \end{MHint}
}

Es ist zu beachten, dass in dem Ohmschen Widerstand elektrische Leistung $U I = R I^2 = U^2 / R$ in Heizleistung umgewandelt wird, was i. A. zu einer Erw"armung des Bauteils f"uhrt. In der Regel ist der elektrische Widerstand temperaturabh"angig (bei elektrischen Leitern mit der Temperatur steigend, bei Halbleitern mit der Temperatur fallend), was ggf. bei Berechnungen zu ber"ucksichtigen ist. Bisher wurden Kondensator und Spule als ideal betrachtet, d. h. als mit der Kapazit"at bzw. Induktivit"at vollst"andig charakterisiert. Tats"achlich weisen beide Bauteile auch einen Ohmschen Anteil auf. "Ublicherweise wird bei der Spule ein Ohmscher Widerstand in Serie zu der Induktivit"at als ausreichende N"aherung betrachtet. Der Ohmsche Widerstand des Kondensators wird vernachl"assigt. Insbesondere bei hohen Frequenzen treten weitere Nichtidealit"aten in Erscheinung. So zeigen Spulen parasit"are Kapazit"aten und Kondensatoren parasit"are Induktivit"aten. Abstrahlung von elektromagnetischen Wellen f"uhrt zu weiteren Verlusten, Leiter werden nicht mehr gleichm"a"sig vom Strom durchflossen, Leitungsl"angen k"onnen nicht mehr gegen"uber der Wellenl"ange vernachl"assigt werden etc.

\begin{center}
  \MUGraphics{abbRLCU}{}{Elektrischer Serienschwingkreis aus einem Kondensator und einer Spule mit "au"serer Spannungsquelle. Die D"ampfung wird durch den ohmschen Widerstand $R$ ber"ucksichtigt. \MLabel{Physik_Schwingungen_AbbRLCU}
}{width:400px;}
\end{center}

\begin{MInfo}
In einem Serienschwingkreis mit Kapazit"at $C$, Induktivit"at $L$, Widerstand $R$ und einer Spannungsquelle $U(t)$ lautet die Differentialgleichung f"ur die elektrische Ladung $q$

\begin{equation}\MLabel{Physik_Schwingungen_EQRLCUOhm}
  L\, \frac{\MD^2 q }{\MD t^2} + R\, \frac{\MD q }{\MD t} + \frac{1}{C}\, q = U \; .
\end{equation}

Division durch die Induktivit"at und Einsetzen der nat"urlichen Kreisfrequenz f"uhrt mit der Definition $R=L\gamma$ auf die zu Gl. \MRef{Physik_Schwingungen_EQHOFD}
analog Gleichung

\begin{equation}\MLabel{Physik_Schwingungen_EQRLCUOhmGamma}
  \frac{\MD^2 q }{\MD t^2} = - \gamma \, \frac{\MD q }{\MD t} - \omega_0^2\,q +  U/L  \; .
\end{equation}


F"ur $U=U_0\cos\left(\omega t\right)$ lautet die  eingeschwungene L"osung, wie in Gl. \MRef{Physik_Schwingungen_EQHReibung},

\begin{equation}
  q = \rho U_0 \cos(\omega t + \theta) \MLabel{Physik_Schwingungen_EQRLCLoesung}
\end{equation}

mit einem Faktor $\rho$ und einer Phasenverschiebung $\theta$, die sowohl von der Frequenz als auch von der D"ampfung abh"angen. Es gilt

\begin{equation}
  \rho^2 = \frac{1}{L^2\left[(\omega^2-\omega_0^2)^2+\gamma^2\omega^2\right]}
\end{equation}

und 

\begin{equation}
  \tan\left(\theta\right) = \gamma\omega/(\omega^2-\omega_0^2)\; .
\end{equation}

\end{MInfo}

\begin{MExample}
In der folgenden Abbildung ist die Amplitude $\rho U_0$ f"ur den Beispielschwingkreis als Funktion der Frequenz dargestellt. Wird f"ur den Widerstand der Einheitswert 1 $\Omega$ eingesetzt, ergibt sich mit 1 $\mu$H ein Wert von $10^6$ / s f"ur $\gamma$. Es wurde wieder $U_0=1\,\text{V}$ gesetzt.

\begin{center}
  \MUGraphics{abbRLCUBsp}{scale=1}{Amplitude bei der eingeschwungenen L"osung mit D"ampfung f"ur den Beispielserienschwingkreis. Aufgrund der starken Resonanz"uberh"ohung "uber drei Gr"o"senordnungen ist es zweckm"a"sig, den (dekadischen) Logarithmus der Ladung bezogen auf die Einheitsladung darzustellen. \MLabel{Physik_Schwingungen_AbbRLCUBsp}
}{width:600px;}
  \end{center}

Geht die Anregungsfrequenz gegen Null, so kann der Einfluss der Spule und des Widerstandes vernachl"assigt werden. Die Amplitude liefert die Ladung des Kondensators, $q = C U_0 = 1\,\text{pC}$. Bei der nat"urlichen Kreisfrequenz erh"oht sich die Ladung um den Faktor $\omega_0/\gamma$. Dieser ist, wie bei elektrischen Schwingkreisen "ublich, deutlich h"oher als in den Feder-Masse-D"ampfer Beispielen, n"amlich $10^9/10^6=10^3$. Entsprechend werden in der Elektrotechnik h"aufig logarithmische Darstellungen eingesezt und Gr"o"senverh"altnisse logarithmiert in dB ausgedr"uckt.

Durch die scharfe Resonanz hat auch die in der folgenden Abbildung gezeigte Phasenverschiebung $\theta$ einen scharfen "Ubergang von den Schwingung in Phase mit der Anregung zu einer Antiphasenschwingung:

\begin{center}
  \MUGraphics{abbRLCPhasenVerschiebungBsp}{scale=1}{Phasenverschiebung bei der eingeschwungenen L"osung mit D"ampfung f"ur das Beispielmodell mit schwacher D"ampfung.\MLabel{Physik_Schwingungen_AbbRLCPhasenVerschiebungBsp}
}{width:600px;}
  \end{center}

\end{MExample}

In Aufgabe
%>>>>>>
\MRef{Physik_Schwingungen_AufgabeRLCU}\MLabel{Physik_Schwingungen_VonAufgabeRLCU}
%<<<<<<
wird ein weiteres Beispiel gerechnet, bei dem der Widerstand des Spulendrahtes als einzige D"ampfung betrachtet wird.

Die folgende Tabelle stellt die physikalischen Gr"o"sen bei dem mechanischen Feder-Masse-D"ampfer-System dem elektrischen Serienschwingkreis gegen"uber. In beiden F"allen ist die unabh"angige Variabel die Zeit, die abh"angige Variable ist die Auslenkung der Masse bzw. die zu- und abflie"sende Ladung in der Spannungsquelle.

\begin{MWTabular}{lll}
  allgemein                & mechanisch            & elektrisch                   \\\hline
  Tr"agheit                 & Masse $m$             & Induktivit"at $L$             \\
  D"ampfung                 & $\gamma=c/m$          & $\gamma=R/L$                 \\
  Starre                   & Federkonstante $k$    & Kehrwert der Kapazit"at $1/C$ \\
  nat"urliche Kreisfrequenz & $\omega_0=\sqrt{k/m}$ & $\omega_0=\sqrt{1/(L C)}$    \\
\end{MWTabular}

Die G"ute eines Resonators ist ein Ma"s daf"ur, wie schmal die Resonanz im Vergleich zu der nat"urlichen Frequenz ist bzw. wie viel Energie im Resonator gespeichert ist im Vergleich zu dem Energieverlust pro Schwingung. Eine "ubliche Definition ist $Q=\omega_0/\gamma$.

Abschlie"send sei bemerkt, dass in der Elektrotechnik "ublicherweise mit Str"omen statt mit Ladungen gerechnet wird und f"ur harmonische Zeitabh"angigkeiten mit Impedanzen gerechnet wird. Dies setzt allerdings den Umgang mit komplexen Zahlen voraus und wird hier nicht betrachtet.

\end{MXContent}


%-------------------------------------------------------------------------------
\begin{MXContent}{Mathematisches Pendel}{Pendel}{STD}%
\MLabel{Physik_Schwingungen_MXPendel}
Als weitere Schwingung in einem mechanischen Systemen wird das mathematische Pendel behandelt. Bei diesem wird eine punktf"ormige Masse $m$ betrachtet, die ohne Reibung an einem masselosen Faden der L"ange $l$ im Schwerefeld mit der Erdbeschleunigung $g$ pendelt.

\begin{center}
\MUGraphics{abbPendel}{}{Mathematisches Pendel mit Masse $m$ und L"ange $l$ im Schwerefeld der Erde ($g$). \MLabel{Physik_Schwingungen_AbbPendel}
}{width:300px;}
\end{center}

Ein Teil der Schwerkraft "ubt eine Spannung auf das Seil aus. Bezeichnet $\alpha$ den Auslenkungswinkel, so hat dieser Anteil den Betrag $ m g \cos\left(\alpha\right)$. Der verbleibende Anteil mit Betrag $ m g \sin\left(\alpha\right)$ bewirkt eine Tangentialbeschleunigung entgegen der Auslenkung. Die Bogenl"ange betr"agt $l\alpha$ und damit die Beschleunigung $-l\frac{\MD^2\alpha }{\MD t^2}$. Insgesamt lautet die Bewegungsgleichung f"ur die tangentiale Bewegung

\begin{equation}
  m g \sin\left(\alpha\right) = - m l\, \frac{\MD^2\alpha }{\MD t^2}\; .\MLabel{Physik_Schwingungen_EqPendel}
\end{equation}

Umformen auf die zu Gleichung \MRef{Physik_Schwingungen_EQHO} analoge Schreibweise ergibt

\begin{equation}
  \frac{\MD^2\alpha }{\MD t^2} + \frac{g}{l} \sin\left(\alpha\right) = 0\, \; .\MLabel{Physik_Schwingungen_EqPendel2}
\end{equation}

Da hier allerdings die gesuchte Funktion im Argument des Sinus steht, ist die Gleichung nicht mehr linear und es gibt keine geschlossene analytische L"osung. F"ur kleine Winkel ist es "ublich, die Gleichung durch die N"aherung $\sin\left(\alpha\right)\approx\alpha$ zu linearisieren:

\begin{equation}
  \frac{\MD^2\alpha }{\MD t^2} + \frac{g}{l} \alpha \approx 0\, \; .\MLabel{Physik_Schwingungen_EqPendelApprox}
\end{equation}

Durch Vergleich mit Gleichung \MRef{Physik_Schwingungen_EQHO} kann nun die Schwingungsfrequenz sofort zu

\begin{equation}
  \omega = \sqrt{\frac{g}{l}} \; .\MLabel{Physik_Schwingungen_EqPendelOmega}
\end{equation}

angegeben werden. Anders als beim Feder-Masse-System geht die Masse nicht in die Schwingungsfrequenz ein. Daf"ur geht die Pendell"ange ein, w"ahrend beim Feder-Masse-System die Anfangsauslenkung keinen Einfluss hatte. Dies ergibt sich bereits aus dimensionsanalytischen "Uberlegungen.

\end{MXContent}


%-------------------------------------------------------------------------------
\begin{MExercises}\MLabel{Physik_Schwingungen_ExercisesZusatz}
%===============================================================================
\begin{MExercise}\MLabel{Physik_Schwingungen_AufgabeEuler}
Schreiben Sie ein Programm f"ur die Integration der Bewegungsgleichung des harmonischen Oszillators mit einer halben Sekunde Schwingungsperiode und 10 mm Ausgangsauslenkung mit dem Eulerschen Polygonzugverfahren (Gln. \MRef{Physik_Schwingungen_EqEuler1} und \MRef{Physik_Schwingungen_EqEuler2}). Die jeweils neuen Orte und Geschwindigkeiten sollen iterativ aus den jeweils vorangehenden Werten berechnet werden. Falls Sie schon etwas Erfahrung mit einer Programmiersprache haben, verwenden Sie diese und testen Sie das Programm. Andernfalls k"onnen Sie das Programm auch in Pseudocode formulieren.

\begin{MHint}{L"osung}
Prozentzeichen leiten im folgenden Pseudocode eine Kommentarzeile ein.
\begin{verbatim}
%---Systemparameter---
t0 = 0.5;           %Periode in Sekunden
omega0 = 2*pi/t0;   %Kreisfrequenz in Hz
omega0Q = omega0^2; %Quadrat der Kreisfrequenz (=k/m)
%---Anfangsbedingungen---
xAlt = 10;          %Anfangswert der Auslenkung in mm
vAlt = 0;           %Anfangswert der Geschwindigkeit in mm/s
%---Diskretisierung---
deltaT = t0/64;     %Zeitdiskretisierung in s
tMax = 3;           %Simulationsdauer in s
nT = round(tMax/deltaT); %Anzahl von Simulationsschritten, ggf. gerundet
%===============================Integration================================
%---Eulersches Polygonzugverfahren
wiederhole fuer Zeitschritt iT von 1 bis nT
    a = -omega0Q*xAlt;        %Beschleunigung in mm/s^2 berechnen (=-kx/m)
    vNeu = vAlt+a*deltaT;     %neue Geschwindigkeit zuweisen
    xNeu = xAlt+vAlt*deltaT;  %neuen Ort zuweisen
    xAlt = xNeu;              %Ueberschreiben der alten Werte
    vAlt = vNeu;              %fuer den naechsten Iterationsschritt
    ...                       %ggf. Ausgeben / Darstellen / Speichern
wiederhole_ende
\end{verbatim}
\end{MHint}

Zur"uck zum \MSRef{Physik_Schwingungen_VonAufgabeEuler}{Text}
\end{MExercise}
%-------------------------------------------------------------------------------

%-------------------------------------------------------------------------------
\begin{MExercise}\MLabel{Physik_Schwingungen_AufgabeLF}
Erg"anzen Sie das Programm f"ur die Integration der Bewegungsgleichung des harmonischen Oszillators aus Aufgabe \MRef{Physik_Schwingungen_AufgabeEuler} um das Leapfrog Verfahren (Gln. \MRef{Physik_Schwingungen_EqLF1} bis \MRef{Physik_Schwingungen_EqLF3}). 

\begin{MHint}{L"osung}
  Der folgenden Quelltext setzt den Pseudocode f"ur die Integration mit dem Eulerschen Polygonzugverfahren fort. Anstatt die erste Geschwindigkeit f"ur die Zeit $t+\delta t/2$ zu berechnen, wurde hier die erste Geschwindigkeit f"ur die Zeit $t-\delta t/2$ berechnet. Dies "andert nichts am weiteren Ergebnis und soll die Implementierung der beiden Verfahren "ahnlicher machen:
\begin{verbatim}
%---Leapfrog Verfahren
xAlt = 10;                  %Anfangswert der Auslenkung in mm
vAlt = 0;                   %Anfangswert der Geschwindigkeit in mm/s
a = -omega0Q*xAlt;          %Beschleunigung in mm/s^2 (=-kx/m)
vHalbAlt = vAlt - a*deltaT/2;   %Fuer Leapfrog, hier Rueckw"arts, v(0-deltaT/2)
wiederhole fuer Zeitschritt iT von 1 bis nT
    a = -omega0Q*xAlt;              %Beschleunigung in mm/s^2 (=-kx/m)
    vHalbNeu = vHalbAlt + a*deltaT; %erst neue Geschwindigkeit zuweisen
    xNeu = xAlt + vHalbNeu*deltaT;  %damit neuen Ort berechnen und zuweisen
    xAlt=xNeu;                  %Ueberschreiben der alten Werte
    vHalbAlt=vHalbNeu;          %fuer den n"achsten Iterationsschritt
    ...                         %ggf. Ausgeben / Darstellen / Speichern
wiederhole_ende
\end{verbatim}
\end{MHint}

Zur"uck zum \MSRef{Physik_Schwingungen_VonAufgabeLF}{Text}
\end{MExercise}
%-------------------------------------------------------------------------------

%-------------------------------------------------------------------------------
\begin{MExercise}\MLabel{Physik_Schwingungen_AufgabeRLCZutaten}
Welche elektrischen Bauteile kommen in einem elektrischen Schwingkreis vor, der einem Feder-Masse-D"ampfer-System mit "au"serer Kraft analog ist? Mit welchen Symbolen und Buchstaben (Formelzeichen) werden sie "ublicherweise dargestellt? Welche physikalischen Gr"o"sen charakterisieren diese Bauteile? In welchen SI-Einheiten werden sie angegeben? In der L"osung wird eine kurze Liste angegeben, Genaueres wird im Abschnitt \MRef{Physik_Schwingungen_MXRLC} vorgestellt.

\begin{MHint}{L"osung}
Die vier Zutaten sind Kondensator, Spule, Widerstand und Spannungsquelle. Sie werden charakterisiert durch Kapazit"at, Induktivit"at, Widerstand und Spannung, mit den "ublichen Bezeichnungen $C$, $L$, $R$ und $U$. Die entsprechenden Einheiten sind Farad, F, Henry, H, Ohm, $\Omega$, und Volt, V.
\end{MHint}

Zur"uck zum \MSRef{Physik_Schwingungen_VonAufgabeRLCZutaten}{Text}
\end{MExercise}
%-------------------------------------------------------------------------------

%-------------------------------------------------------------------------------
\begin{MExercise}\MLabel{Physik_Schwingungen_AufgabeLC}
Aus zwei Platten mit einer Fl"ache $A_C$ von \MZahl{0}{01} Quadratmetern und einem Abstand $d$ von einem Millimeter sowie einer Spule mit $N=20$ Windungen auf einer L"ange $l$ von 20 Millimetern und einer Querschnittsfl"ache $A_L$ von 15 Quadratmillimetern soll ein Schwingkreis aufgebaut werden. Berechnen Sie die lineare Resonanzfrequenz, gerundet auf MHz, indem Sie folgende N"aherungsformeln verwenden:
\begin{eqnarray}
  C &=& \epsilon_0 A_C / d\\
  L &=& \mu_0 N^2 A_L / l
\end{eqnarray}
mit der elektrischen Feldkonstante $\epsilon_0 = \MZahl{8}{854}\cdot 10^{-12}\,\text{F/m}$ und der magnetischen Feldkonstante $\mu_0 = 4\pi\cdot 10^{-7}\,\text{H/m}$. 

Die mithilfe der N"aherungsformeln f"ur Kapazit"at und Induktivit"at berechnete lineare Resonanzfrequenz betr"agt, auf MHz gerundet, \MQuestion{2}{28} MHz.

\begin{MHint}{L"osung} 
F"ur die Kapazit"at liefert die N"aherungsformel
\begin{equation}
  C=\MZahl{8}{854}\cdot \MZahl{0}{01} / \MZahl{0}{001} \,\text{pF} = \MZahl{88}{54} \,\text{pF}\; ,
\end{equation}
die N"aherungsformel f"ur die Induktivit"at liefert

\begin{equation}
  L=4\pi\cdot 10^{-7} 400 (15/20)\cdot 10^{-3} \,\text{H} = \MZahl{0}{3770}\,\mu\text{H}\; .
\end{equation}
Damit berechnen sich die lineare Frequenz zu 28 MHz. Das Ergebnis ist niedriger als bei dem Beispiel mit einfachen Zahlenwerten, da die Induktivit"at zwar etwas kleiner als ein Mikro-Henry ist, die Kapazit"at der eher unhandlich gro"sen Platten aber deutlich gr"o"ser als ein Pico-Farad ist.

\end{MHint}

Zur"uck zum \MSRef{Physik_Schwingungen_VonAufgabeLC}{Text}
\end{MExercise}
%-------------------------------------------------------------------------------

%-------------------------------------------------------------------------------
\begin{MExercise}\MLabel{Physik_Schwingungen_AufgabeLCUSerie}
Stellen Sie die Amplitude sowohl f"ur die Ladung als auch f"ur den Strom f"ur den Serienschwingkreis mit dem Kondensator und der Spule aus Aufgabe \MRef{Physik_Schwingungen_AufgabeLC} sowie einer externen Spannungsquelle, wobei die maximale Spannung der externen Quelle 1 V betragen soll, dar.

\begin{MHint}{L"osung}
\begin{center}
  \MUGraphics{abbLCUVorfaktor}{scale=1}{Amplitude bei der eingeschwungenen L"osung ohne D"ampfung f"ur das Modell mit einfachem geometrischen Aufbau und N"aherungsformeln f"ur Kapazit"at und Induktivit"at und eine maximale "au"sere Spannung von 1 V.\MLabel{Physik_Schwingungen_AbbLCUVorfaktor}
}{width:600px;}
\end{center}
\begin{center}
  \MUGraphics{abbLCUVorfaktorI}{scale=1}{Amplitude f"ur den Strom statt f"ur die Ladung.\MLabel{Physik_Schwingungen_AbbLCUVorfaktorI}
}{width:600px;}
\end{center}

\end{MHint}

Zur"uck zum \MSRef{Physik_Schwingungen_VonAufgabeLCUSerie}{Text}
\end{MExercise}
%-------------------------------------------------------------------------------

%-------------------------------------------------------------------------------
\begin{MExercise}\MLabel{Physik_Schwingungen_AufgabeLCUParallel}
Erstellen Sie eine Darstellung der Amplituden f"ur Ladung und Strom f"ur den Parallelschwingkreis mit dem Kondensator und der Spule aus Aufgabe \MRef{Physik_Schwingungen_AufgabeLC}, wobei die maximale Spannung der externen Quelle weiterhin 1 V betragen soll.

\begin{MHint}{L"osung}
\begin{center}
  \MUGraphics{abbLCUParVorfaktor}{scale=1}{Amplitude bei der eingeschwungenen L"osung ohne D"ampfung f"ur das Modell eines Parallelschwingkreises mit einfachem geometrischen Aufbau und N"aherungsformeln f"ur Kapazit"at und Induktivit"at und eine maximale "au"sere Spannung von 1 V.\MLabel{Physik_Schwingungen_AbbLCUParVorfaktor}
}{width:600px;}
\end{center}
\begin{center}
  \MUGraphics{abbLCUParVorfaktorI}{scale=1}{Amplitude f"ur den Strom statt f"ur die Ladung.\MLabel{Physik_Schwingungen_AbbLCUParVorfaktorI}
}{width:600px;}
\end{center}

\end{MHint}

Zur"uck zum \MSRef{Physik_Schwingungen_VonAufgabeLCUParallel}{Text}
\end{MExercise}
%-------------------------------------------------------------------------------

%-------------------------------------------------------------------------------
\begin{MExercise}\MLabel{Physik_Schwingungen_AufgabeRLCU}
F"ur die Spule aus Aufgabe \MRef{Physik_Schwingungen_AufgabeLC} soll der Gleichstromwiderstand n"aherungsweise berechnet werden und als D"ampfung in die Berechnung f"ur die Amplitude der Ladung in dem Serienschwingkreis eingesetzt werden. F"ur einen Draht mit elektrischer Leitf"ahigkeit $\sigma$, L"ange $l_D$ und Querschnittsfl"ache $A_D$ ergibt sich der Widerstand zu
\begin{equation}
  R=\frac{1}{\sigma}\frac{l_D}{A_D}\; .
\end{equation}
Berechnen Sie zun"achst eine N"aherung f"ur die Drahtl"ange, indem Sie die zuvor angegebene Spulenquerschnittsfl"ache in einen Umfang umrechnen und diesen mit der angegebenen Anzahl von Windungen multiplizieren.

Die berechnete Spulendrahtl"ange betr"agt, auf Millimeter gerundet, \MQuestion{3}{275} mm.

Bestimmen Sie nun den Gleichstromwiderstand f"ur einen Draht dieser L"ange mit einer Leitf"ahigkeit von $\sigma = 6\cdot 10^7\,\text{S/m}$ und einem Drahtdurchmesser von einem halben Millimeter.

Der Widerstand betr"agt, auf Milliohm gerundet, \MQuestion{2}{23} m$\Omega$.

Stellen Sie schlie"slich die Amplitude $\rho U_0$ und die Phasenverschiebung $\theta$ f"ur den betrachteten Serienschwingkreis und $U_0$ = 1 V dar.

\begin{MHint}{L"osung} 
Umrechnen des Spulenquerschnitts in den Spulenradius und Multiplikation mit 2$\pi$ f"ur den Umfang sowie der Windungszahl liefert
\begin{equation}
  l_D=N 2\pi \sqrt{A_L/\pi} = 20\cdot 2 \sqrt{\pi\cdot 15}  \,\text{mm} = 275 \,\text{mm}\; .
\end{equation}
Es kann "uberraschen, dass man f"ur eine Spule mit nur \MZahl{2}{2} mm Radius und 20 Windungen eine solche Drahtl"ange ben"otigt. Gerade bei der Herstellung einer Spule aus hochwertigem Draht (z. B. Silberdraht mit PTFE Isolierung, ca. 100 EUR / m) ist darauf zu achten, nicht ein zu kurzes und damit wertloses St"uck abzul"angen. Nat"urlich ber"ucksichtigt die obige Rechnung nur eine N"aherung der tats"achlichen Geometrie, Anschlussst"ucke sind vorzusehen, durch Recken nimmt die L"ange noch etwas zu.

Als Drahtdurchmesser ist ein halber Millimeter angegeben. Der Drahtquerschnitt betr"agt somit $A_D=\pi\,\MZahl{0}{5}^2/4\,\text{mm}^2=\MZahl{0}{196}\,\text{mm}^2 $. Bei der betrachteten Spule mit 20 Windungen auf 20 mm entspricht dieser Drahtdurchmesser der Faustregel, dass der Drahtdurchmesser etwa gleich dem Windungsabstand sein sollte. Der Widerstand ist schlie"slich

\begin{equation}
  R=\frac{275\,\text{mm}}{6\cdot 10^7 \,\text{S/m}\cdot \MZahl{0}{196}\text{mm}^2} = 23 \,\text{m}\Omega\; .
\end{equation}

\begin{center}
  \MUGraphics{abbRLCUVorfaktor}{scale=1}{Amplitude bei der eingeschwungenen L"osung mit D"ampfung. \MLabel{Physik_Schwingungen_AbbRLCUVorfaktor}
}{width:600px;}
  \end{center}

\begin{center}
  \MUGraphics{abbRLCPhasenVerschiebung}{scale=1}{Phasenverschiebung bei der eingeschwungenen L"osung mit D"ampfung.\MLabel{Physik_Schwingungen_AbbRLCPhasenVerschiebung}
}{width:600px;}
  \end{center}

\end{MHint}

Zur"uck zum \MSRef{Physik_Schwingungen_VonAufgabeRLCU}{Text}
\end{MExercise}
%-------------------------------------------------------------------------------

%-------------------------------------------------------------------------------

%===============================================================================
\end{MExercises}


%===============================================================================

\end{document}

