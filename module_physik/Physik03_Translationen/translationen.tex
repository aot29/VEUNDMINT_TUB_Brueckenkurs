% MINTMOD Version P0.1.0, needs to be consistent with preprocesser object in tex2x and MPragma-Version at the end of this file

% Parameter aus Konvertierungsprozess (PDF und HTML-Erzeugung wenn vom Konverter aus gestartet) werden hier eingefuegt, Preambleincludes werden am Schluss angehaengt

\newif\ifttm                % gesetzt falls Uebersetzung in HTML stattfindet, sonst uebersetzung in PDF

% Wahl der Notationsvariante ist im PDF immer std, in der HTML-Uebersetzung wird vom Konverter die Auswahl modifiziert
\newif\ifvariantstd
\newif\ifvariantunotation
\variantstdtrue % Diese Zeile wird vom Konverter erkannt und ggf. modifiziert, daher nicht veraendern!


\def\MOutputDVI{1}
\def\MOutputPDF{2}
\def\MOutputHTML{3}
\newcounter{MOutput}

\ifttm
\usepackage{german}
\usepackage{array}
\usepackage{amsmath}
\usepackage{amssymb}
\usepackage{amsthm}
\else
\documentclass[ngerman,oneside]{scrbook}
\usepackage{etex}
\usepackage[latin1]{inputenc}
\usepackage{textcomp}
\usepackage[ngerman]{babel}
\usepackage[pdftex]{color}
\usepackage{xcolor}
\usepackage{graphicx}
\usepackage[all]{xy}
\usepackage{fancyhdr}
\usepackage{verbatim}
\usepackage{array}
\usepackage{float}
\usepackage{makeidx}
\usepackage{amsmath}
\usepackage{amstext}
\usepackage{amssymb}
\usepackage{amsthm}
\usepackage[ngerman]{varioref}
\usepackage{framed}
\usepackage{supertabular}
\usepackage{longtable}
\usepackage{maxpage}
\usepackage{tikz}
\usepackage{tikzscale}
\usepackage{tikz-3dplot}
\usepackage{bibgerm}
\usepackage{chemarrow}
\usepackage{polynom}
%\usepackage{draftwatermark}
\usepackage{pdflscape}
\usetikzlibrary{calc}
\usetikzlibrary{through}
\usetikzlibrary{shapes.geometric}
\usetikzlibrary{arrows}
\usetikzlibrary{intersections}
\usetikzlibrary{decorations.pathmorphing}
\usetikzlibrary{external}
\usetikzlibrary{patterns}
\usetikzlibrary{fadings}
\usepackage[colorlinks=true,linkcolor=blue]{hyperref} 
\usepackage[all]{hypcap}
%\usepackage[colorlinks=true,linkcolor=blue,bookmarksopen=true]{hyperref} 
\usepackage{ifpdf}

\usepackage{movie15}

\setcounter{tocdepth}{2} % In Inhaltsverzeichnis bis subsection
\setcounter{secnumdepth}{3} % Nummeriert bis subsubsection

\setlength{\LTpost}{0pt} % Fuer longtable
\setlength{\parindent}{0pt}
\setlength{\parskip}{8pt}
%\setlength{\parskip}{9pt plus 2pt minus 1pt}
\setlength{\abovecaptionskip}{-0.25ex}
\setlength{\belowcaptionskip}{-0.25ex}
\fi

\ifttm
\newcommand{\MDebugMessage}[1]{\special{html:<!-- debugprint;;}#1\special{html:; //-->}}
\else
%\newcommand{\MDebugMessage}[1]{\immediate\write\mintlog{#1}}
\newcommand{\MDebugMessage}[1]{}
\fi

\def\MPageHeaderDef{%
\pagestyle{fancy}%
\fancyhead[r]{(C) VE\&MINT-Projekt}
\fancyfoot[c]{\thepage\\--- CCL BY-SA 3.0 ---}
}


\ifttm%
\def\MRelax{}%
\else%
\def\MRelax{\relax}%
\fi%

%--------------------------- Uebernahme von speziellen XML-Versionen einiger LaTeX-Kommandos aus xmlbefehle.tex vom alten Kasseler Konverter ---------------

\newcommand{\MSep}{\left\|{\phantom{\frac1g}}\right.}

\newcommand{\ML}{L}

\newcommand{\MGGT}{\mathrm{ggT}}


\ifttm
% Verhindert dass die subsection-nummer doppelt in der toccaption auftaucht (sollte ggf. in toccaption gefixt werden so dass diese Ueberschreibung nicht notwendig ist)
\renewcommand{\thesubsection}{}
% Kommandos die ttm nicht kennt
\newcommand{\binomial}[2]{{#1 \choose #2}} %  Binomialkoeffizienten
\newcommand{\eur}{\begin{html}&euro;\end{html}}
\newcommand{\square}{\begin{html}&square;\end{html}}
\newcommand{\glqq}{"'}  \newcommand{\grqq}{"'}
\newcommand{\nRightarrow}{\special{html: &nrArr; }}
\newcommand{\nmid}{\special{html: &nmid; }}
\newcommand{\nparallel}{\begin{html}&nparallel;\end{html}}
\newcommand{\mapstoo}{\begin{html}<mo>&map;</mo>\end{html}}

% Schnitt und Vereinigungssymbole von Mengen haben zu kleine Abstaende; korrigiert:
\newcommand{\ccup}{\,\!\cup\,\!}
\newcommand{\ccap}{\,\!\cap\,\!}


% Umsetzung von mathbb im HTML
\renewcommand{\mathbb}[1]{\begin{html}<mo>&#1opf;</mo>\end{html}}
\fi

%---------------------- Strukturierung ----------------------------------------------------------------------------------------------------------------------

%---------------------- Kapselung des sectioning findet auf drei Ebenen statt:
% 1. Die LateX-Befehl
% 2. Die D-Versionen der Befehle, die nur die Grade der Abschnitte umhaengen falls notwendig
% 3. Die M-Versionen der Befehle, die zusaetzliche Formatierungen vornehmen, Skripten starten und das HTML codieren
% Im Modultext duerfen nur die M-Befehle verwendet werden!

\ifttm

  \def\Dsubsubsubsection#1{\subsubsubsection{#1}}
  \def\Dsubsubsection#1{\subsubsection{#1}\addtocounter{subsubsection}{1}} % ttm-Fehler korrigieren
  \def\Dsubsection#1{\subsection{#1}}
  \def\Dsection#1{\section{#1}} % Im HTML wird nur der Sektionstitel gegeben
  \def\Dchapter#1{\chapter{#1}}
  \def\Dsubsubsubsectionx#1{\subsubsubsection*{#1}}
  \def\Dsubsubsectionx#1{\subsubsection*{#1}}
  \def\Dsubsectionx#1{\subsection*{#1}}
  \def\Dsectionx#1{\section*{#1}}
  \def\Dchapterx#1{\chapter*{#1}}

\else

  \def\Dsubsubsubsection#1{\subsubsection{#1}}
  \def\Dsubsubsection#1{\subsection{#1}}
  \def\Dsubsection#1{\section{#1}}
  \def\Dsection#1{\chapter{#1}}
  \def\Dchapter#1{\title{#1}}
  \def\Dsubsubsubsectionx#1{\subsubsection*{#1}}
  \def\Dsubsubsectionx#1{\subsection*{#1}}
  \def\Dsubsectionx#1{\section*{#1}}
  \def\Dsectionx#1{\chapter*{#1}}

\fi

\newcommand{\MStdPoints}{4}
\newcommand{\MSetPoints}[1]{\renewcommand{\MStdPoints}{#1}}

% Befehl zum Abbruch der Erstellung (nur PDF)
\newcommand{\MAbort}[1]{\err{#1}}

% Prefix vor Dateieinbindungen, wird in der Baumdatei mit \renewcommand modifiziert
% und auf das Verzeichnisprefix gesetzt, in dem das gerade bearbeitete tex-Dokument liegt.
% Im HTML wird es auf das Verzeichnis der HTML-Datei gesetzt.
% Das Prefix muss mit / enden !
\newcommand{\MDPrefix}{.}

% MRegisterFile notiert eine Datei zur Einbindung in den HTML-Baum. Grafiken mit MGraphics werden automatisch eingebunden.
% Mit MLastFile erhaelt man eine Markierung fuer die zuletzt registrierte Datei.
% Diese Markierung wird im postprocessing durch den physikalischen Dateinamen ersetzt, aber nur den Namen (d.h. \MMaterial gehoert noch davor, vgl Definition von MGraphics)
% Parameter: Pfad/Name der Datei bzw. des Ordners, relativ zur Position des Modul-Tex-Dokuments.
\ifttm
\newcommand{\MRegisterFile}[1]{\addtocounter{MFileNumber}{1}\special{html:<!-- registerfile;;}#1\special{html:;;}\MDPrefix\special{html:;;}\arabic{MFileNumber}\special{html:; //-->}}
\else
\newcommand{\MRegisterFile}[1]{\addtocounter{MFileNumber}{1}}
\fi

% Testen welcher Uebersetzer hier am Werk ist

\ifttm
\setcounter{MOutput}{3}
\else
\ifx\pdfoutput\undefined
  \pdffalse
  \setcounter{MOutput}{\MOutputDVI}
  \message{Verarbeitung mit latex, Ausgabe in dvi.}
\else
  \setcounter{MOutput}{\MOutputPDF}
  \message{Verarbeitung mit pdflatex, Ausgabe in pdf.}
  \ifnum \pdfoutput=0
    \pdffalse
  \setcounter{MOutput}{\MOutputDVI}
  \message{Verarbeitung mit pdflatex, Ausgabe in dvi.}
  \else
    \ifnum\pdfoutput=1
    \pdftrue
  \setcounter{MOutput}{\MOutputPDF}
  \message{Verarbeitung mit pdflatex, Ausgabe in pdf.}
    \fi
  \fi
\fi
\fi

\ifnum\value{MOutput}=\MOutputPDF
\DeclareGraphicsExtensions{.pdf,.png,.jpg}
\fi

\ifnum\value{MOutput}=\MOutputDVI
\DeclareGraphicsExtensions{.eps,.png,.jpg}
\fi

\ifnum\value{MOutput}=\MOutputHTML
% Wird vom Konverter leider nicht erkannt und daher in split.pm hardcodiert!
\DeclareGraphicsExtensions{.png,.jpg,.gif}
\fi

% Umdefinition der hyperref-Nummerierung im PDF-Modus
\ifttm
\else
\renewcommand{\theHfigure}{\arabic{chapter}.\arabic{section}.\arabic{figure}}
\fi

% Makro, um in der HTML-Ausgabe die zuerst zu oeffnende Datei zu kennzeichnen
\ifttm
\newcommand{\MGlobalStart}{\special{html:<!-- mglobalstarttag -->}}
\else
\newcommand{\MGlobalStart}{}
\fi

% Makro, um bei scormlogin ein pullen des Benutzers bei Aufruf der Seite zu erzwingen (typischerweise auf der Einstiegsseite)
\ifttm
\newcommand{\MPullSite}{\special{html:<!-- pullsite //-->}}
\else
\newcommand{\MPullSite}{}
\fi

% Makro, um in der HTML-Ausgabe die Kapiteluebersicht zu kennzeichnen
\ifttm
\newcommand{\MGlobalChapterTag}{\special{html:<!-- mglobalchaptertag -->}}
\else
\newcommand{\MGlobalChapterTag}{}
\fi

% Makro, um in der HTML-Ausgabe die Konfiguration zu kennzeichnen
\ifttm
\newcommand{\MGlobalConfTag}{\special{html:<!-- mglobalconfigtag -->}}
\else
\newcommand{\MGlobalConfTag}{}
\fi

% Makro, um in der HTML-Ausgabe die Standortbeschreibung zu kennzeichnen
\ifttm
\newcommand{\MGlobalLocationTag}{\special{html:<!-- mgloballocationtag -->}}
\else
\newcommand{\MGlobalLocationTag}{}
\fi

% Makro, um in der HTML-Ausgabe die persoenlichen Daten zu kennzeichnen
\ifttm
\newcommand{\MGlobalDataTag}{\special{html:<!-- mglobaldatatag -->}}
\else
\newcommand{\MGlobalDataTag}{}
\fi

% Makro, um in der HTML-Ausgabe die Suchseite zu kennzeichnen
\ifttm
\newcommand{\MGlobalSearchTag}{\special{html:<!-- mglobalsearchtag -->}}
\else
\newcommand{\MGlobalSearchTag}{}
\fi

% Makro, um in der HTML-Ausgabe die Favoritenseite zu kennzeichnen
\ifttm
\newcommand{\MGlobalFavoTag}{\special{html:<!-- mglobalfavoritestag -->}}
\else
\newcommand{\MGlobalFavoTag}{}
\fi

% Makro, um in der HTML-Ausgabe die Eingangstestseite zu kennzeichnen
\ifttm
\newcommand{\MGlobalSTestTag}{\special{html:<!-- mglobalstesttag -->}}
\else
\newcommand{\MGlobalSTestTag}{}
\fi

% Makro, um in der PDF-Ausgabe ein Wasserzeichen zu definieren
\ifttm
\newcommand{\MWatermarkSettings}{\relax}
\else
\newcommand{\MWatermarkSettings}{%
% \SetWatermarkText{(c) MINT-Kolleg Baden-W�rttemberg 2014}
% \SetWatermarkLightness{0.85}
% \SetWatermarkScale{1.5}
}
\fi

\ifttm
\newcommand{\MBinom}[2]{\left({\begin{array}{c} #1 \\ #2 \end{array}}\right)}
\else
\newcommand{\MBinom}[2]{\binom{#1}{#2}}
\fi

\ifttm
\newcommand{\DeclareMathOperator}[2]{\def#1{\mathrm{#2}}}
\newcommand{\operatorname}[1]{\mathrm{#1}}
\fi

%----------------- Makros fuer die gemischte HTML/PDF-Konvertierung ------------------------------

\newcommand{\MTestName}{\relax} % wird durch Test-Umgebung gesetzt

% Fuer experimentelle Kursinhalte, die im Release-Umsetzungsvorgang eine Fehlermeldung
% produzieren sollen aber sonst normal umgesetzt werden
\newenvironment{MExperimental}{%
}{%
}

% Wird von ttm nicht richtig umgesetzt!!
\newenvironment{MExerciseItems}{%
\renewcommand\theenumi{\alph{enumi}}%
\begin{enumerate}%
}{%
\end{enumerate}%
}


\definecolor{infoshadecolor}{rgb}{0.75,0.75,0.75}
\definecolor{exmpshadecolor}{rgb}{0.875,0.875,0.875}
\definecolor{expeshadecolor}{rgb}{0.95,0.95,0.95}
\definecolor{framecolor}{rgb}{0.2,0.2,0.2}

% Bei PDF-Uebersetzung wird hinter den Start jeder Satz/Info-aehnlichen Umgebung eine leere mbox gesetzt, damit
% fuehrende Listen oder enums nicht den Zeilenumbruch kaputtmachen
%\ifttm
\def\MTB{}
%\else
%\def\MTB{\mbox{}}
%\fi


\ifttm
\newcommand{\MRelates}{\special{html:<mi>&wedgeq;</mi>}}
\else
\def\MRelates{\stackrel{\scriptscriptstyle\wedge}{=}}
\fi

\def\MInch{\text{''}}
\def\Mdd{\textit{''}}

\ifttm
\def\MNL{ \newline }
\newenvironment{MArray}[1]{\begin{array}{#1}}{\end{array}}
\else
\def\MNL{ \\ }
\newenvironment{MArray}[1]{\begin{array}{#1}}{\end{array}}
\fi

\newcommand{\MBox}[1]{$\mathrm{#1}$}
\newcommand{\MMBox}[1]{\mathrm{#1}}


\ifttm%
\newcommand{\Mtfrac}[2]{{\textstyle \frac{#1}{#2}}}
\newcommand{\Mdfrac}[2]{{\displaystyle \frac{#1}{#2}}}
\newcommand{\Mmeasuredangle}{\special{html:<mi>&angmsd;</mi>}}
\else%
\newcommand{\Mtfrac}[2]{\tfrac{#1}{#2}}
\newcommand{\Mdfrac}[2]{\dfrac{#1}{#2}}
\newcommand{\Mmeasuredangle}{\measuredangle}
\relax
\fi

% Matrizen und Vektoren

% Inhalt wird in der Form a & b \\ c & d erwartet
% Vorsicht: MVector = Komponentenspalte, MVec = Variablensymbol
\ifttm%
\newcommand{\MVector}[1]{\left({\begin{array}{c}#1\end{array}}\right)}
\else%
\newcommand{\MVector}[1]{\begin{pmatrix}#1\end{pmatrix}}
\fi



\newcommand{\MVec}[1]{\vec{#1}}
\newcommand{\MDVec}[1]{\overrightarrow{#1}}

%----------------- Umgebungen fuer Definitionen und Saetze ----------------------------------------

% Fuegt einen Tabellen-Zeilenumbruch ein im PDF, aber nicht im HTML
\newcommand{\TSkip}{\ifttm \else&\ \\\fi}

\newenvironment{infoshaded}{%
\def\FrameCommand{\fboxsep=\FrameSep \fcolorbox{framecolor}{infoshadecolor}}%
\MakeFramed {\advance\hsize-\width \FrameRestore}}%
{\endMakeFramed}

\newenvironment{expeshaded}{%
\def\FrameCommand{\fboxsep=\FrameSep \fcolorbox{framecolor}{expeshadecolor}}%
\MakeFramed {\advance\hsize-\width \FrameRestore}}%
{\endMakeFramed}

\newenvironment{exmpshaded}{%
\def\FrameCommand{\fboxsep=\FrameSep \fcolorbox{framecolor}{exmpshadecolor}}%
\MakeFramed {\advance\hsize-\width \FrameRestore}}%
{\endMakeFramed}

\def\STDCOLOR{black}

\ifttm%
\else%
\newtheoremstyle{MSatzStyle}
  {1cm}                   %Space above
  {1cm}                   %Space below
  {\normalfont\itshape}   %Body font
  {}                      %Indent amount (empty = no indent,
                          %\parindent = para indent)
  {\normalfont\bfseries}  %Thm head font
  {}                      %Punctuation after thm head
  {\newline}              %Space after thm head: " " = normal interword
                          %space; \newline = linebreak
  {\thmname{#1}\thmnumber{ #2}\thmnote{ (#3)}}
                          %Thm head spec (can be left empty, meaning
                          %`normal')
                          %
\newtheoremstyle{MDefStyle}
  {1cm}                   %Space above
  {1cm}                   %Space below
  {\normalfont}           %Body font
  {}                      %Indent amount (empty = no indent,
                          %\parindent = para indent)
  {\normalfont\bfseries}  %Thm head font
  {}                      %Punctuation after thm head
  {\newline}              %Space after thm head: " " = normal interword
                          %space; \newline = linebreak
  {\thmname{#1}\thmnumber{ #2}\thmnote{ (#3)}}
                          %Thm head spec (can be left empty, meaning
                          %`normal')
\fi%

\newcommand{\MInfoText}{Info}

\newcounter{MHintCounter}
\newcounter{MCodeEditCounter}

\newcounter{MLastIndex}  % Enthaelt die dritte Stelle (Indexnummer) des letzten angelegten Objekts
\newcounter{MLastType}   % Enthaelt den Typ des letzten angelegten Objekts (mithilfe der unten definierten Konstanten). Die Entscheidung, wie der Typ dargstellt wird, wird in split.pm beim Postprocessing getroffen.
\newcounter{MLastTypeEq} % =1 falls das Label in einer Matheumgebung (equation, eqnarray usw.) steht, =2 falls das Label in einer table-Umgebung steht

% Da ttm keine Zahlmakros verarbeiten kann, werden diese Nummern in den Zuweisungen hardcodiert!
\def\MTypeSection{1}          %# Zaehler ist section
\def\MTypeSubsection{2}       %# Zaehler ist subsection
\def\MTypeSubsubsection{3}    %# Zaehler ist subsubsection
\def\MTypeInfo{4}             %# Eine Infobox, Separatzaehler fuer die Chemie (auch wenn es dort nicht nummeriert wird) ist MInfoCounter
\def\MTypeExercise{5}         %# Eine Aufgabe, Separatzaehler fuer die Chemie ist MExerciseCounter
\def\MTypeExample{6}          %# Eine Beispielbox, Separatzaehler fuer die Chemie ist MExampleCounter
\def\MTypeExperiment{7}       %# Eine Versuchsbox, Separatzaehler fuer die Chemie ist MExperimentCounter
\def\MTypeGraphics{8}         %# Eine Graphik, Separatzaehler fuer alle FB ist MGraphicsCounter
\def\MTypeTable{9}            %# Eine Tabellennummer, hat keinen Zaehler da durch table gezaehlt wird
\def\MTypeEquation{10}        %# Eine Gleichungsnummer, hat keinen Zaehler da durch equation/eqnarray gezaehlt wird
\def\MTypeTheorem{11}         % Ein theorem oder xtheorem, Separatzaehler fuer die Chemie ist MTheoremCounter
\def\MTypeVideo{12}           %# Ein Video,Separatzaehler fuer alle FB ist MVideoCounter
\def\MTypeEntry{13}           %# Ein Eintrag fuer die Stichwortliste, wird nicht gezaehlt sondern erhaelt im preparsing ein unique-label 

% Zaehler fuer das Labelsystem sind prefixcounter, jeder Zaehler wird VOR dem gezaehlten Objekt inkrementiert und zaehlt daher das aktuelle Objekt
\newcounter{MInfoCounter}
\newcounter{MExerciseCounter}
\newcounter{MExampleCounter}
\newcounter{MExperimentCounter}
\newcounter{MGraphicsCounter}
\newcounter{MTableCounter}
\newcounter{MEquationCounter}  % Nur im HTML, sonst durch "equation"-counter von latex realisiert
\newcounter{MTheoremCounter}
\newcounter{MObjectCounter}   % Gemeinsamer Zaehler fuer Objekte (ausser Grafiken/Tabellen) in Mathe/Info/Physik
\newcounter{MVideoCounter}
\newcounter{MEntryCounter}

\newcounter{MTestSite} % 1 = Subsubsection ist eine Pruefungsseite, 0 = ist eine normale Seite (inkl. Hilfeseite)

\def\MCell{$\phantom{a}$}

\newenvironment{MExportExercise}{\begin{MExercise}}{\end{MExercise}} % wird von mconvert abgefangen

\def\MGenerateExNumber{%
\ifnum\value{MSepNumbers}=0%
\arabic{section}.\arabic{subsection}.\arabic{MObjectCounter}\setcounter{MLastIndex}{\value{MObjectCounter}}%
\else%
\arabic{section}.\arabic{subsection}.\arabic{MExerciseCounter}\setcounter{MLastIndex}{\value{MExerciseCounter}}%
\fi%
}%

\def\MGenerateExmpNumber{%
\ifnum\value{MSepNumbers}=0%
\arabic{section}.\arabic{subsection}.\arabic{MObjectCounter}\setcounter{MLastIndex}{\value{MObjectCounter}}%
\else%
\arabic{section}.\arabic{subsection}.\arabic{MExerciseCounter}\setcounter{MLastIndex}{\value{MExampleCounter}}%
\fi%
}%

\def\MGenerateInfoNumber{%
\ifnum\value{MSepNumbers}=0%
\arabic{section}.\arabic{subsection}.\arabic{MObjectCounter}\setcounter{MLastIndex}{\value{MObjectCounter}}%
\else%
\arabic{section}.\arabic{subsection}.\arabic{MExerciseCounter}\setcounter{MLastIndex}{\value{MInfoCounter}}%
\fi%
}%

\def\MGenerateSiteNumber{%
\arabic{section}.\arabic{subsection}.\arabic{subsubsection}%
}%

% Funktionalitaet fuer Auswahlaufgaben

\newcounter{MExerciseCollectionCounter} % = 0 falls nicht in collection-Umgebung, ansonsten Schachtelungstiefe
\newcounter{MExerciseCollectionTextCounter} % wird von MExercise-Umgebung inkrementiert und von MExerciseCollection-Umgebung auf Null gesetzt

\ifttm
% MExerciseCollection gruppiert Aufgaben, die dynamisch aus der Datenbank gezogen werden und nicht direkt in der HTML-Seite stehen
% Parameter: #1 = ID der Collection, muss eindeutig fuer alle IN DER DB VORHANDENEN collections sein unabhaengig vom Kurs
%            #2 = Optionsargument (im Moment: 1 = Iterative Auswahl, 2 = Zufallsbasierte Auswahl)
\newenvironment{MExerciseCollection}[2]{%
\addtocounter{MExerciseCollectionCounter}{1}
\setcounter{MExerciseCollectionTextCounter}{0}
\special{html:<!-- mexercisecollectionstart;;}#1\special{html:;;}#2\special{html:;; //-->}%
}{%
\special{html:<!-- mexercisecollectionstop //-->}%
\addtocounter{MExerciseCollectionCounter}{-1}
}
\else
\newenvironment{MExerciseCollection}[2]{%
\addtocounter{MExerciseCollectionCounter}{1}
\setcounter{MExerciseCollectionTextCounter}{0}
}{%
\addtocounter{MExerciseCollectionCounter}{-1}
}
\fi

% Bei Uebersetzung nach PDF werden die theorem-Umgebungen verwendet, bei Uebersetzung in HTML ein manuelles Makro
\ifttm%

  \newenvironment{MHint}[1]{  \special{html:<button name="Name_MHint}\arabic{MHintCounter}\special{html:" class="hintbutton_closed" id="MHint}\arabic{MHintCounter}\special{html:_button" %
  type="button" onclick="toggle_hint('MHint}\arabic{MHintCounter}\special{html:');">}#1\special{html:</button>}
  \special{html:<div class="hint" style="display:none" id="MHint}\arabic{MHintCounter}\special{html:"> }}{\begin{html}</div>\end{html}\addtocounter{MHintCounter}{1}}

  \newenvironment{MCOSHZusatz}{  \special{html:<button name="Name_MHint}\arabic{MHintCounter}\special{html:" class="chintbutton_closed" id="MHint}\arabic{MHintCounter}\special{html:_button" %
  type="button" onclick="toggle_hint('MHint}\arabic{MHintCounter}\special{html:');">}Weiterf�hrende Inhalte\special{html:</button>}
  \special{html:<div class="hintc" style="display:none" id="MHint}\arabic{MHintCounter}\special{html:">
  <div class="coshwarn">Diese Inhalte gehen �ber das Kursniveau hinaus und werden in den Aufgaben und Tests nicht abgefragt.</div><br />}
  \addtocounter{MHintCounter}{1}}{\begin{html}</div>\end{html}}

  
  \newenvironment{MDefinition}{\begin{definition}\setcounter{MLastIndex}{\value{definition}}\ \\}{\end{definition}}

  
  \newenvironment{MExercise}{
  \renewcommand{\MStdPoints}{4}
  \addtocounter{MExerciseCounter}{1}
  \addtocounter{MObjectCounter}{1}
  \setcounter{MLastType}{5}

  \ifnum\value{MExerciseCollectionCounter}=0\else\addtocounter{MExerciseCollectionTextCounter}{1}\special{html:<!-- mexercisetextstart;;}\arabic{MExerciseCollectionTextCounter}\special{html:;; //-->}\fi
  \special{html:<div class="aufgabe" id="ADIV_}\MGenerateExNumber\special{html:">}%
  \textbf{Aufgabe \MGenerateExNumber
  } \ \\}{
  \special{html:</div><!-- mfeedbackbutton;Aufgabe;}\arabic{MTestSite}\special{html:;}\MGenerateExNumber\special{html:; //-->}
  \ifnum\value{MExerciseCollectionCounter}=0\else\special{html:<!-- mexercisetextstop //-->}\fi
  }

  % Stellt eine Kombination aus Aufgabe, Loesungstext und Eingabefeld bereit,
  % bei der Aufgabentext und Musterloesung sowie die zugehoerigen Feldelemente
  % extern bezogen und div-aktualisiert werden, das Eingabefeld aber immer das gleiche ist.
  \newenvironment{MFetchExercise}{
  \addtocounter{MExerciseCounter}{1}
  \addtocounter{MObjectCounter}{1}
  \setcounter{MLastType}{5}

  \special{html:<div class="aufgabe" id="ADIV_}\MGenerateExNumber\special{html:">}%
  \textbf{Aufgabe \MGenerateExNumber
  } \ \\%
  \special{html:</div><div class="exfetch_text" id="ADIVTEXT_}\MGenerateExNumber\special{html:">}%
  \special{html:</div><div class="exfetch_sol" id="ADIVSOL_}\MGenerateExNumber\special{html:">}%
  \special{html:</div><div class="exfetch_input" id="ADIVINPUT_}\MGenerateExNumber\special{html:">}%
  }{
  \special{html:</div>}
  }

  \newenvironment{MExample}{
  \addtocounter{MExampleCounter}{1}
  \addtocounter{MObjectCounter}{1}
  \setcounter{MLastType}{6}
  \begin{html}
  <div class="exmp">
  <div class="exmprahmen">
  \end{html}\textbf{Beispiel
  \ifnum\value{MSepNumbers}=0
  \arabic{section}.\arabic{subsection}.\arabic{MObjectCounter}\setcounter{MLastIndex}{\value{MObjectCounter}}
  \else
  \arabic{section}.\arabic{subsection}.\arabic{MExampleCounter}\setcounter{MLastIndex}{\value{MExampleCounter}}
  \fi
  } \ \\}{\begin{html}</div>
  </div>
  \end{html}
  \special{html:<!-- mfeedbackbutton;Beispiel;}\arabic{MTestSite}\special{html:;}\MGenerateExmpNumber\special{html:; //-->}
  }

  \newenvironment{MExperiment}{
  \addtocounter{MExperimentCounter}{1}
  \addtocounter{MObjectCounter}{1}
  \setcounter{MLastType}{7}
  \begin{html}
  <div class="expe">
  <div class="experahmen">
  \end{html}\textbf{Versuch
  \ifnum\value{MSepNumbers}=0
  \arabic{section}.\arabic{subsection}.\arabic{MObjectCounter}\setcounter{MLastIndex}{\value{MObjectCounter}}
  \else
%  \arabic{MExperimentCounter}\setcounter{MLastIndex}{\value{MExperimentCounter}}
  \arabic{section}.\arabic{subsection}.\arabic{MExperimentCounter}\setcounter{MLastIndex}{\value{MExperimentCounter}}
  \fi
  } \ \\}{\begin{html}</div>
  </div>
  \end{html}}

  \newenvironment{MChemInfo}{
  \setcounter{MLastType}{4}
  \begin{html}
  <div class="info">
  <div class="inforahmen">
  \end{html}}{\begin{html}</div>
  </div>
  \end{html}}

  \newenvironment{MXInfo}[1]{
  \addtocounter{MInfoCounter}{1}
  \addtocounter{MObjectCounter}{1}
  \setcounter{MLastType}{4}
  \begin{html}
  <div class="info">
  <div class="inforahmen">
  \end{html}\textbf{#1
  \ifnum\value{MInfoNumbers}=0
  \else
    \ifnum\value{MSepNumbers}=0
    \arabic{section}.\arabic{subsection}.\arabic{MObjectCounter}\setcounter{MLastIndex}{\value{MObjectCounter}}
    \else
    \arabic{MInfoCounter}\setcounter{MLastIndex}{\value{MInfoCounter}}
    \fi
  \fi
  } \ \\}{\begin{html}</div>
  </div>
  \end{html}
  \special{html:<!-- mfeedbackbutton;Info;}\arabic{MTestSite}\special{html:;}\MGenerateInfoNumber\special{html:; //-->}
  }

  \newenvironment{MInfo}{\ifnum\value{MInfoNumbers}=0\begin{MChemInfo}\else\begin{MXInfo}{Info}\ \\ \fi}{\ifnum\value{MInfoNumbers}=0\end{MChemInfo}\else\end{MXInfo}\fi}

\else%

  \theoremstyle{MSatzStyle}
  \newtheorem{thm}{Satz}[section]
  \newtheorem{thmc}{Satz}
  \theoremstyle{MDefStyle}
  \newtheorem{defn}[thm]{Definition}
  \newtheorem{exmp}[thm]{Beispiel}
  \newtheorem{info}[thm]{\MInfoText}
  \theoremstyle{MDefStyle}
  \newtheorem{defnc}{Definition}
  \theoremstyle{MDefStyle}
  \newtheorem{exmpc}{Beispiel}[section]
  \theoremstyle{MDefStyle}
  \newtheorem{infoc}{\MInfoText}
  \theoremstyle{MDefStyle}
  \newtheorem{exrc}{Aufgabe}[section]
  \theoremstyle{MDefStyle}
  \newtheorem{verc}{Versuch}[section]
  
  \newenvironment{MFetchExercise}{}{} % kann im PDF nicht dargestellt werden
  
  \newenvironment{MExercise}{\begin{exrc}\renewcommand{\MStdPoints}{1}\MTB}{\end{exrc}}
  \newenvironment{MHint}[1]{\ \\ \underline{#1:}\\}{}
  \newenvironment{MCOSHZusatz}{\ \\ \underline{Weiterf�hrende Inhalte:}\\}{}
  \newenvironment{MDefinition}{\ifnum\value{MInfoNumbers}=0\begin{defnc}\else\begin{defn}\fi\MTB}{\ifnum\value{MInfoNumbers}=0\end{defnc}\else\end{defn}\fi}
%  \newenvironment{MExample}{\begin{exmp}}{\ \linebreak[1] \ \ \ \ $\phantom{a}$ \ \hfill $\blacklozenge$\end{exmp}}
  \newenvironment{MExample}{
    \ifnum\value{MInfoNumbers}=0\begin{exmpc}\else\begin{exmp}\fi
    \MTB
    \begin{exmpshaded}
    \ \newline
}{
    \end{exmpshaded}
    \ifnum\value{MInfoNumbers}=0\end{exmpc}\else\end{exmp}\fi
}
  \newenvironment{MChemInfo}{\begin{infoshaded}}{\end{infoshaded}}

  \newenvironment{MInfo}{\ifnum\value{MInfoNumbers}=0\begin{MChemInfo}\else\renewcommand{\MInfoText}{Info}\begin{info}\begin{infoshaded}
  \MTB
   \ \newline
    \fi
  }{\ifnum\value{MInfoNumbers}=0\end{MChemInfo}\else\end{infoshaded}\end{info}\fi}

  \newenvironment{MXInfo}[1]{
    \renewcommand{\MInfoText}{#1}
    \ifnum\value{MInfoNumbers}=0\begin{infoc}\else\begin{info}\fi%
    \MTB
    \begin{infoshaded}
    \ \newline
  }{\end{infoshaded}\ifnum\value{MInfoNumbers}=0\end{infoc}\else\end{info}\fi}

  \newenvironment{MExperiment}{
    \renewcommand{\MInfoText}{Versuch}
    \ifnum\value{MInfoNumbers}=0\begin{verc}\else\begin{info}\fi
    \MTB
    \begin{expeshaded}
    \ \newline
  }{
    \end{expeshaded}
    \ifnum\value{MInfoNumbers}=0\end{verc}\else\end{info}\fi
  }
\fi%

% MHint sollte nicht direkt fuer Loesungen benutzt werden wegen solutionselect
\newenvironment{MSolution}{\begin{MHint}{L"osung}}{\end{MHint}}

\newcounter{MCodeCounter}

\ifttm
\newenvironment{MCode}{\special{html:<!-- mcodestart -->}\ttfamily\color{blue}}{\special{html:<!-- mcodestop -->}}
\else
\newenvironment{MCode}{\begin{flushleft}\ttfamily\addtocounter{MCodeCounter}{1}}{\addtocounter{MCodeCounter}{-1}\end{flushleft}}
% Ohne color-Statement da inkompatible mit framed/shaded-Boxen aus dem framed-package
\fi

%----------------- Sonderdefinitionen fuer Symbole, die der Konverter nicht kann ----------------------------------------------

\ifttm%
\newcommand{\MUnderset}[2]{\underbrace{#2}_{#1}}%
\else%
\newcommand{\MUnderset}[2]{\underset{#1}{#2}}%
\fi%

\ifttm
\newcommand{\MThinspace}{\special{html:<mi>&#x2009;</mi>}}
\else
\newcommand{\MThinspace}{\,}
\fi

\ifttm
\newcommand{\glq}{\begin{html}&sbquo;\end{html}}
\newcommand{\grq}{\begin{html}&lsquo;\end{html}}
\newcommand{\glqq}{\begin{html}&bdquo;\end{html}}
\newcommand{\grqq}{\begin{html}&ldquo;\end{html}}
\fi

\ifttm
\newcommand{\MNdash}{\begin{html}&ndash;\end{html}}
\else
\newcommand{\MNdash}{--}
\fi

%\ifttm\def\MIU{\special{html:<mi>&#8520;</mi>}}\else\def\MIU{\mathrm{i}}\fi
\def\MIU{\mathrm{i}}
\def\MEU{e} % TU9-Onlinekurs: italic-e
%\def\MEU{\mathrm{e}} % Alte Onlinemodule: roman-e
\def\MD{d} % Kursives d in Integralen im TU9-Onlinekurs
%\def\MD{\mathrm{d}} % roman-d in den alten Onlinemodulen
\def\MDB{\|}

%zusaetzlicher Leerraum vor "\MD"
\ifttm%
\def\MDSpace{\special{html:<mi>&#x2009;</mi>}}
\else%
\def\MDSpace{\,}
\fi%
\newcommand{\MDwSp}{\MDSpace\MD}%

\ifttm
\def\Mdq{\dq}
\else
\def\Mdq{\dq}
\fi

\def\MSpan#1{\left<{#1}\right>}
\def\MSetminus{\setminus}
\def\MIM{I}

\ifttm
\newcommand{\ld}{\text{ld}}
\newcommand{\lg}{\text{lg}}
\else
\DeclareMathOperator{\ld}{ld}
%\newcommand{\lg}{\text{lg}} % in latex schon definiert
\fi


\def\Mmapsto{\ifttm\special{html:<mi>&mapsto;</mi>}\else\mapsto\fi} 
\def\Mvarphi{\ifttm\phi\else\varphi\fi}
\def\Mphi{\ifttm\varphi\else\phi\fi}
\ifttm%
\newcommand{\MEumu}{\special{html:<mi>&#x3BC;</mi>}}%
\else%
\newcommand{\MEumu}{\textrm{\textmu}}%
\fi
\def\Mvarepsilon{\ifttm\epsilon\else\varepsilon\fi}
\def\Mepsilon{\ifttm\varepsilon\else\epsilon\fi}
\def\Mvarkappa{\ifttm\kappa\else\varkappa\fi}
\def\Mkappa{\ifttm\varkappa\else\kappa\fi}
\def\Mcomplement{\ifttm\special{html:<mi>&comp;</mi>}\else\complement\fi} 
\def\MWW{\mathrm{WW}}
\def\Mmod{\ifttm\special{html:<mi>&nbsp;mod&nbsp;</mi>}\else\mod\fi} 

\ifttm%
\def\mod{\text{\;mod\;}}%
\def\MNEquiv{\special{html:<mi>&NotCongruent;</mi>}}% 
\def\MNSubseteq{\special{html:<mi>&NotSubsetEqual;</mi>}}%
\def\MEmptyset{\special{html:<mi>&empty;</mi>}}%
\def\MVDots{\special{html:<mi>&#x22EE;</mi>}}%
\def\MHDots{\special{html:<mi>&#x2026;</mi>}}%
\def\Mddag{\special{html:<mi>&#x1202;</mi>}}%
\def\sphericalangle{\special{html:<mi>&measuredangle;</mi>}}%
\def\nparallel{\special{html:<mi>&nparallel;</mi>}}%
\def\MProofEnd{\special{html:<mi>&#x25FB;</mi>}}%
\newenvironment{MProof}[1]{\underline{#1}:\MCR\MCR}{\hfill $\MProofEnd$}%
\else%
\def\MNEquiv{\not\equiv}%
\def\MNSubseteq{\not\subseteq}%
\def\MEmptyset{\emptyset}%
\def\MVDots{\vdots}%
\def\MHDots{\hdots}%
\def\Mddag{\ddag}%
\newenvironment{MProof}[1]{\begin{proof}[#1]}{\end{proof}}%
\fi%



% Spaces zum Auffuellen von Tabellenbreiten, die nur im HTML wirken
\ifttm%
\def\MTSP{\:}%
\else%
\def\MTSP{}%
\fi%

\DeclareMathOperator{\arsinh}{arsinh}
\DeclareMathOperator{\arcosh}{arcosh}
\DeclareMathOperator{\artanh}{artanh}
\DeclareMathOperator{\arcoth}{arcoth}


\newcommand{\MMathSet}[1]{\mathbb{#1}}
\def\N{\MMathSet{N}}
\def\Z{\MMathSet{Z}}
\def\Q{\MMathSet{Q}}
\def\R{\MMathSet{R}}
\def\C{\MMathSet{C}}

\newcounter{MForLoopCounter}
\newcommand{\MForLoop}[2]{\setcounter{MForLoopCounter}{#1}\ifnum\value{MForLoopCounter}=0{}\else{{#2}\addtocounter{MForLoopCounter}{-1}\MForLoop{\value{MForLoopCounter}}{#2}}\fi}

\newcounter{MSiteCounter}
\newcounter{MFieldCounter} % Kombination section.subsection.site.field ist eindeutig in allen Modulen, field alleine nicht

\newcounter{MiniMarkerCounter}

\ifttm
\newenvironment{MMiniPageP}[1]{\begin{minipage}{#1\linewidth}\special{html:<!-- minimarker;;}\arabic{MiniMarkerCounter}\special{html:;;#1; //-->}}{\end{minipage}\addtocounter{MiniMarkerCounter}{1}}
\else
\newenvironment{MMiniPageP}[1]{\begin{minipage}{#1\linewidth}}{\end{minipage}\addtocounter{MiniMarkerCounter}{1}}
\fi

\newcounter{AlignCounter}

\newcommand{\MStartJustify}{\ifttm\special{html:<!-- startalign;;}\arabic{AlignCounter}\special{html:;;justify; //-->}\fi}
\newcommand{\MStopJustify}{\ifttm\special{html:<!-- stopalign;;}\arabic{AlignCounter}\special{html:; //-->}\fi\addtocounter{AlignCounter}{1}}

\newenvironment{MJTabular}[1]{
\MStartJustify
\begin{tabular}{#1}
}{
\end{tabular}
\MStopJustify
}

\newcommand{\MImageLeft}[2]{
\begin{center}
\begin{tabular}{lc}
\MStartJustify
\begin{MMiniPageP}{0.65}
#1
\end{MMiniPageP}
\MStopJustify
&
\begin{MMiniPageP}{0.3}
#2  
\end{MMiniPageP}
\end{tabular}
\end{center}
}

\newcommand{\MImageHalf}[2]{
\begin{center}
\begin{tabular}{lc}
\MStartJustify
\begin{MMiniPageP}{0.45}
#1
\end{MMiniPageP}
\MStopJustify
&
\begin{MMiniPageP}{0.45}
#2  
\end{MMiniPageP}
\end{tabular}
\end{center}
}

\newcommand{\MBigImageLeft}[2]{
\begin{center}
\begin{tabular}{lc}
\MStartJustify
\begin{MMiniPageP}{0.25}
#1
\end{MMiniPageP}
\MStopJustify
&
\begin{MMiniPageP}{0.7}
#2  
\end{MMiniPageP}
\end{tabular}
\end{center}
}

\ifttm
\def\No{\mathbb{N}_0}
\else
\def\No{\ensuremath{\N_0}}
\fi
\def\MT{\textrm{\tiny T}}
\newcommand{\MTranspose}[1]{{#1}^{\MT}}
\ifttm
\newcommand{\MRe}{\mathsf{Re}}
\newcommand{\MIm}{\mathsf{Im}}
\else
\DeclareMathOperator{\MRe}{Re}
\DeclareMathOperator{\MIm}{Im}
\fi

\newcommand{\Mid}{\mathrm{id}}
\newcommand{\MFeinheit}{\mathrm{feinh}}

\ifttm
\newcommand{\Msubstack}[1]{\begin{array}{c}{#1}\end{array}}
\else
\newcommand{\Msubstack}[1]{\substack{#1}}
\fi

% Typen von Fragefeldern:
% 1 = Alphanumerisch, case-sensitive-Vergleich
% 2 = Ja/Nein-Checkbox, Loesung ist 0 oder 1   (OPTION = Image-id fuer Rueckmeldung)
% 3 = Reelle Zahlen Geparset
% 4 = Funktionen Geparset (mit Stuetzstellen zur ueberpruefung)

% Dieser Befehl erstellt ein interaktives Aufgabenfeld. Parameter:
% - #1 Laenge in Zeichen
% - #2 Loesungstext (alphanumerisch, case sensitive)
% - #3 AufgabenID (alphanumerisch, case sensitive)
% - #4 Typ (Kennnummer)
% - #5 String fuer Optionen (ggf. mit Semikolon getrennte Einzelstrings)
% - #6 Anzahl Punkte
% - #7 uxid (kann z.B. Loesungsstring sein)
% ACHTUNG: Die langen Zeilen bitte so lassen, Zeilenumbrueche im tex werden in div's umgesetzt
\newcommand{\MQuestionID}[7]{
\ifttm
\special{html:<!-- mdeclareuxid;;}UX#7\special{html:;;}\arabic{section}\special{html:;;}#3\special{html:;; //-->}%
\special{html:<!-- mdeclarepoints;;}\arabic{section}\special{html:;;}#3\special{html:;;}#6\special{html:;;}\arabic{MTestSite}\special{html:;;}\arabic{chapter}%
\special{html:;; //--><!-- onloadstart //-->CreateQuestionObj("}#7\special{html:",}\arabic{MFieldCounter}\special{html:,"}#2%
\special{html:","}#3\special{html:",}#4\special{html:,"}#5\special{html:",}#6\special{html:,}\arabic{MTestSite}\special{html:,}\arabic{section}%
\special{html:);<!-- onloadstop //-->}%
\special{html:<input mfieldtype="}#4\special{html:" name="Name_}#3\special{html:" id="}#3\special{html:" type="text" size="}#1\special{html:" maxlength="}#1%
\special{html:" }\ifnum\value{MGroupActive}=0\special{html:onfocus="handlerFocus(}\arabic{MFieldCounter}%
\special{html:);" onblur="handlerBlur(}\arabic{MFieldCounter}\special{html:);" onkeyup="handlerChange(}\arabic{MFieldCounter}\special{html:,0);" onpaste="handlerChange(}\arabic{MFieldCounter}\special{html:,0);" oninput="handlerChange(}\arabic{MFieldCounter}\special{html:,0);" onpropertychange="handlerChange(}\arabic{MFieldCounter}\special{html:,0);"/>}%
\special{html:<img src="images/questionmark.gif" width="20" height="20" border="0" align="absmiddle" id="}QM#3\special{html:"/>}
\else%
\special{html:onblur="handlerBlur(}\arabic{MFieldCounter}%
\special{html:);" onfocus="handlerFocus(}\arabic{MFieldCounter}\special{html:);" onkeyup="handlerChange(}\arabic{MFieldCounter}\special{html:,1);" onpaste="handlerChange(}\arabic{MFieldCounter}\special{html:,1);" oninput="handlerChange(}\arabic{MFieldCounter}\special{html:,1);" onpropertychange="handlerChange(}\arabic{MFieldCounter}\special{html:,1);"/>}%
\special{html:<img src="images/questionmark.gif" width="20" height="20" border="0" align="absmiddle" id="}QM#3\special{html:"/>}\fi%
\else%
\ifnum\value{QBoxFlag}=1\fbox{$\phantom{\MForLoop{#1}{b}}$}\else$\phantom{\MForLoop{#1}{b}}$\fi%
\fi%
}

% ACHTUNG: Die langen Zeilen bitte so lassen, Zeilenumbrueche im tex werden in div's umgesetzt
% QuestionCheckbox macht ausserhalb einer QuestionGroup keinen Sinn!
% #1 = solution (1 oder 0), ggf. mit ::smc abgetrennt auszuschliessende single-choice-boxen (UXIDs durch , getrennt), #2 = id, #3 = points, #4 = uxid
\newcommand{\MQuestionCheckbox}[4]{
\ifttm
\special{html:<!-- mdeclareuxid;;}UX#4\special{html:;;}\arabic{section}\special{html:;;}#2\special{html:;; //-->}%
\ifnum\value{MGroupActive}=0\MDebugMessage{ERROR: Checkbox Nr. \arabic{MFieldCounter}\ ist nicht in einer Kontrollgruppe, es wird niemals eine Loesung angezeigt!}\fi
\special{html: %
<!-- mdeclarepoints;;}\arabic{section}\special{html:;;}#2\special{html:;;}#3\special{html:;;}\arabic{MTestSite}\special{html:;;}\arabic{chapter}%
\special{html:;; //--><!-- onloadstart //-->CreateQuestionObj("}#4\special{html:",}\arabic{MFieldCounter}\special{html:,"}#1\special{html:","}#2\special{html:",2,"IMG}#2%
\special{html:",}#3\special{html:,}\arabic{MTestSite}\special{html:,}\arabic{section}\special{html:);<!-- onloadstop //-->}%
\special{html:<input mfieldtype="2" type="checkbox" name="Name_}#2\special{html:" id="}#2\special{html:" onchange="handlerChange(}\arabic{MFieldCounter}\special{html:,1);"/><img src="images/questionmark.gif" name="}Name_IMG#2%
\special{html:" width="20" height="20" border="0" align="absmiddle" id="}IMG#2\special{html:"/> }%
\else%
\ifnum\value{QBoxFlag}=1\fbox{$\phantom{X}$}\else$\phantom{X}$\fi%
\fi%
}

\def\MGenerateID{QFELD_\arabic{section}.\arabic{subsection}.\arabic{MSiteCounter}.QF\arabic{MFieldCounter}}

% #1 = 0/1 ggf. mit ::smc abgetrennt auszuschliessende single-choice-boxen (UXIDs durch , getrennt ohne UX), #2 = uxid ohne UX
\newcommand{\MCheckbox}[2]{
\MQuestionCheckbox{#1}{\MGenerateID}{\MStdPoints}{#2}
\addtocounter{MFieldCounter}{1}
}

% Erster Parameter: Zeichenlaenge der Eingabebox, zweiter Parameter: Loesungstext
\newcommand{\MQuestion}[2]{
\MQuestionID{#1}{#2}{\MGenerateID}{1}{0}{\MStdPoints}{#2}
\addtocounter{MFieldCounter}{1}
}

% Erster Parameter: Zeichenlaenge der Eingabebox, zweiter Parameter: Loesungstext
\newcommand{\MLQuestion}[3]{
\MQuestionID{#1}{#2}{\MGenerateID}{1}{0}{\MStdPoints}{#3}
\addtocounter{MFieldCounter}{1}
}

% Parameter: Laenge des Feldes, Loesung (wird auch geparsed), Stellen Genauigkeit hinter dem Komma, weitere Stellen werden mathematisch gerundet vor Vergleich
\newcommand{\MParsedQuestion}[3]{
\MQuestionID{#1}{#2}{\MGenerateID}{3}{#3}{\MStdPoints}{#2}
\addtocounter{MFieldCounter}{1}
}

% Parameter: Laenge des Feldes, Loesung (wird auch geparsed), Stellen Genauigkeit hinter dem Komma, weitere Stellen werden mathematisch gerundet vor Vergleich
\newcommand{\MLParsedQuestion}[4]{
\MQuestionID{#1}{#2}{\MGenerateID}{3}{#3}{\MStdPoints}{#4}
\addtocounter{MFieldCounter}{1}
}

% Parameter: Laenge des Feldes, Loesungsfunktion, Anzahl Stuetzstellen, Funktionsvariablen durch Kommata getrennt (nicht case-sensitive), Anzahl Nachkommastellen im Vergleich
\newcommand{\MFunctionQuestion}[5]{
\MQuestionID{#1}{#2}{\MGenerateID}{4}{#3;#4;#5;0}{\MStdPoints}{#2}
\addtocounter{MFieldCounter}{1}
}

% Parameter: Laenge des Feldes, Loesungsfunktion, Anzahl Stuetzstellen, Funktionsvariablen durch Kommata getrennt (nicht case-sensitive), Anzahl Nachkommastellen im Vergleich, UXID
\newcommand{\MLFunctionQuestion}[6]{
\MQuestionID{#1}{#2}{\MGenerateID}{4}{#3;#4;#5;0}{\MStdPoints}{#6}
\addtocounter{MFieldCounter}{1}
}

% Parameter: Laenge des Feldes, Loesungsintervall, Genauigkeit der Zahlenwertpruefung
\newcommand{\MIntervalQuestion}[3]{
\MQuestionID{#1}{#2}{\MGenerateID}{6}{#3}{\MStdPoints}{#2}
\addtocounter{MFieldCounter}{1}
}

% Parameter: Laenge des Feldes, Loesungsintervall, Genauigkeit der Zahlenwertpruefung, UXID
\newcommand{\MLIntervalQuestion}[4]{
\MQuestionID{#1}{#2}{\MGenerateID}{6}{#3}{\MStdPoints}{#4}
\addtocounter{MFieldCounter}{1}
}

% Parameter: Laenge des Feldes, Loesungsfunktion, Anzahl Stuetzstellen, Funktionsvariable (nicht case-sensitive), Anzahl Nachkommastellen im Vergleich, Vereinfachungsbedingung
% Vereinfachungsbedingung ist eine der Folgenden:
% 0 = Keine Vereinfachungsbedingung
% 1 = Keine Klammern (runde oder eckige) mehr im vereinfachten Ausdruck
% 2 = Faktordarstellung (Term hat Produkte als letzte Operation, Summen als vorgeschaltete Operation)
% 3 = Summendarstellung (Term hat Summen als letzte Operation, Produkte als vorgeschaltete Operation)
% Flag 512: Besondere Stuetzstellen (nur >1 und nur schwach rational), sonst symmetrisch um Nullpunkt und ganze Zahlen inkl. Null werden getroffen
\newcommand{\MSimplifyQuestion}[6]{
\MQuestionID{#1}{#2}{\MGenerateID}{4}{#3;#4;#5;#6}{\MStdPoints}{#2}
\addtocounter{MFieldCounter}{1}
}

\newcommand{\MLSimplifyQuestion}[7]{
\MQuestionID{#1}{#2}{\MGenerateID}{4}{#3;#4;#5;#6}{\MStdPoints}{#7}
\addtocounter{MFieldCounter}{1}
}

% Parameter: Laenge des Feldes, Loesung (optionaler Ausdruck), Anzahl Stuetzstellen, Funktionsvariable (nicht case-sensitive), Anzahl Nachkommastellen im Vergleich, Spezialtyp (string-id)
\newcommand{\MLSpecialQuestion}[7]{
\MQuestionID{#1}{#2}{\MGenerateID}{7}{#3;#4;#5;#6}{\MStdPoints}{#7}
\addtocounter{MFieldCounter}{1}
}

\newcounter{MGroupStart}
\newcounter{MGroupEnd}
\newcounter{MGroupActive}

\newenvironment{MQuestionGroup}{
\setcounter{MGroupStart}{\value{MFieldCounter}}
\setcounter{MGroupActive}{1}
}{
\setcounter{MGroupActive}{0}
\setcounter{MGroupEnd}{\value{MFieldCounter}}
\addtocounter{MGroupEnd}{-1}
}

\newcommand{\MGroupButton}[1]{
\ifttm
\special{html:<button name="Name_Group}\arabic{MGroupStart}\special{html:to}\arabic{MGroupEnd}\special{html:" id="Group}\arabic{MGroupStart}\special{html:to}\arabic{MGroupEnd}\special{html:" %
type="button" onclick="group_button(}\arabic{MGroupStart}\special{html:,}\arabic{MGroupEnd}\special{html:);">}#1\special{html:</button>}
\else
\phantom{#1}
\fi
}

%----------------- Makros fuer die modularisierte Darstellung ------------------------------------

\def\MyText#1{#1}

% is used internally by the conversion package, should not be used by original tex documents
\def\MOrgLabel#1{\relax}

\ifttm

% Ein MLabel wird im html codiert durch das tag <!-- mmlabel;;Labelbezeichner;;SubjectArea;;chapter;;section;;subsection;;Index;;Objekttyp; //-->
\def\MLabel#1{%
\ifnum\value{MLastType}=8%
\ifnum\value{MCaptionOn}=0%
\MDebugMessage{ERROR: Grafik \arabic{MGraphicsCounter} hat separates label: #1 (Grafiklabels sollten nur in der Caption stehen)}%
\fi
\fi
\ifnum\value{MLastType}=12%
\ifnum\value{MCaptionOn}=0%
\MDebugMessage{ERROR: Video \arabic{MVideoCounter} hat separates label: #1 (Videolabels sollten nur in der Caption stehen}%
\fi
\fi
\ifnum\value{MLastType}=10\setcounter{MLastIndex}{\value{equation}}\fi
\label{#1}\begin{html}<!-- mmlabel;;#1;;\end{html}\arabic{MSubjectArea}\special{html:;;}\arabic{chapter}\special{html:;;}\arabic{section}\special{html:;;}\arabic{subsection}\special{html:;;}\arabic{MLastIndex}\special{html:;;}\arabic{MLastType}\special{html:; //-->}}%

\else

% Sonderbehandlung im PDF fuer Abbildungen in separater aux-Datei, da MGraphics die figure-Umgebung nicht verwendet
\def\MLabel#1{%
\ifnum\value{MLastType}=8%
\ifnum\value{MCaptionOn}=0%
\MDebugMessage{ERROR: Grafik \arabic{MGraphicsCounter} hat separates label: #1 (Grafiklabels sollten nur in der Caption stehen}%
\fi
\fi
\ifnum\value{MLastType}=12%
\ifnum\value{MCaptionOn}=0%
\MDebugMessage{ERROR: Video \arabic{MVideoCounter} hat separates label: #1 (Videolabels sollten nur in der Caption stehen}%
\fi
\fi
\label{#1}%
}%

\fi

% Gibt Begriff des referenzierten Objekts mit aus, aber nur im HTML, daher nur in Ausnahmefaellen (z.B. Copyrightliste) sinnvoll
\def\MCRef#1{\ifttm\special{html:<!-- mmref;;}#1\special{html:;;1; //-->}\else\vref{#1}\fi}


\def\MRef#1{\ifttm\special{html:<!-- mmref;;}#1\special{html:;;0; //-->}\else\vref{#1}\fi}
\def\MERef#1{\ifttm\special{html:<!-- mmref;;}#1\special{html:;;0; //-->}\else\eqref{#1}\fi}
\def\MNRef#1{\ifttm\special{html:<!-- mmref;;}#1\special{html:;;0; //-->}\else\ref{#1}\fi}
\def\MSRef#1#2{\ifttm\special{html:<!-- msref;;}#1\special{html:;;}#2\special{html:; //-->}\else \if#2\empty \ref{#1} \else \hyperref[#1]{#2}\fi\fi} 

\def\MRefRange#1#2{\ifttm\MRef{#1} bis 
\MRef{#2}\else\vrefrange[\unskip]{#1}{#2}\fi}

\def\MRefTwo#1#2{\ifttm\MRef{#1} und \MRef{#2}\else%
\let\vRefTLRsav=\reftextlabelrange\let\vRefTPRsav=\reftextpagerange%
\def\reftextlabelrange##1##2{\ref{##1} und~\ref{##2}}%
\def\reftextpagerange##1##2{auf den Seiten~\pageref{#1} und~\pageref{#2}}%
\vrefrange[\unskip]{#1}{#2}%
\let\reftextlabelrange=\vRefTLRsav\let\reftextpagerange=\vRefTPRsav\fi}

% MSectionChapter definiert falls notwendig das Kapitel vor der section. Das ist notwendig, wenn nur ein Einzelmodul uebersetzt wird.
% MChaptersGiven ist ein Counter, der von mconvert.pl vordefiniert wird.
\ifttm
\newcommand{\MSectionChapter}{\ifnum\value{MChaptersGiven}=0{\Dchapter{Modul}}\else{}\fi}
\else
\newcommand{\MSectionChapter}{\ifnum\value{chapter}=0{\Dchapter{Modul}}\else{}\fi}
\fi


\def\MChapter#1{\ifnum\value{MSSEnd}>0{\MSubsectionEndMacros}\addtocounter{MSSEnd}{-1}\fi\Dchapter{#1}}
\def\MSubject#1{\MChapter{#1}} % Schluesselwort HELPSECTION ist reserviert fuer Hilfesektion

\newcommand{\MSectionID}{UNKNOWNID}

\ifttm
\newcommand{\MSetSectionID}[1]{\renewcommand{\MSectionID}{#1}}
\else
\newcommand{\MSetSectionID}[1]{\renewcommand{\MSectionID}{#1}\tikzsetexternalprefix{#1}}
\fi


\newcommand{\MSection}[1]{\MSetSectionID{MODULID}\ifnum\value{MSSEnd}>0{\MSubsectionEndMacros}\addtocounter{MSSEnd}{-1}\fi\MSectionChapter\Dsection{#1}\MSectionStartMacros{#1}\setcounter{MLastIndex}{-1}\setcounter{MLastType}{1}} % Sections werden ueber das section-Feld im mmlabel-Tag identifiziert, nicht ueber das Indexfeld

\def\MSubsection#1{\ifnum\value{MSSEnd}>0{\MSubsectionEndMacros}\addtocounter{MSSEnd}{-1}\fi\ifttm\else\clearpage\fi\Dsubsection{#1}\MSubsectionStartMacros\setcounter{MLastIndex}{-1}\setcounter{MLastType}{2}\addtocounter{MSSEnd}{1}}% Subsections werden ueber das subsection-Feld im mmlabel-Tag identifiziert, nicht ueber das Indexfeld
\def\MSubsectionx#1{\Dsubsectionx{#1}} % Nur zur Verwendung in MSectionStart gedacht
\def\MSubsubsection#1{\Dsubsubsection{#1}\setcounter{MLastIndex}{\value{subsubsection}}\setcounter{MLastType}{3}\ifttm\special{html:<!-- sectioninfo;;}\arabic{section}\special{html:;;}\arabic{subsection}\special{html:;;}\arabic{subsubsection}\special{html:;;1;;}\arabic{MTestSite}\special{html:; //-->}\fi}
\def\MSubsubsectionx#1{\Dsubsubsectionx{#1}\ifttm\special{html:<!-- sectioninfo;;}\arabic{section}\special{html:;;}\arabic{subsection}\special{html:;;}\arabic{subsubsection}\special{html:;;0;;}\arabic{MTestSite}\special{html:; //-->}\else\addcontentsline{toc}{subsection}{#1}\fi}

\ifttm
\def\MSubsubsubsectionx#1{\ \newline\textbf{#1}\special{html:<br />}}
\else
\def\MSubsubsubsectionx#1{\ \newline
\textbf{#1}\ \\
}
\fi


% Dieses Skript wird zu Beginn jedes Modulabschnitts (=Webseite) ausgefuehrt und initialisiert den Aufgabenfeldzaehler
\newcommand{\MPageScripts}{
\setcounter{MFieldCounter}{1}
\addtocounter{MSiteCounter}{1}
\setcounter{MHintCounter}{1}
\setcounter{MCodeEditCounter}{1}
\setcounter{MGroupActive}{0}
\DoQBoxes
% Feldvariablen werden im HTML-Header in conv.pl eingestellt
}

% Dieses Skript wird zum Ende jedes Modulabschnitts (=Webseite) ausgefuehrt
\ifttm
\newcommand{\MEndScripts}{\special{html:<br /><!-- mfeedbackbutton;Seite;}\arabic{MTestSite}\special{html:;}\MGenerateSiteNumber\special{html:; //-->}
}
\else
\newcommand{\MEndScripts}{\relax}
\fi


\newcounter{QBoxFlag}
\newcommand{\DoQBoxes}{\setcounter{QBoxFlag}{1}}
\newcommand{\NoQBoxes}{\setcounter{QBoxFlag}{0}}

\newcounter{MXCTest}
\newcounter{MXCounter}
\newcounter{MSCounter}



\ifttm

% Struktur des sectioninfo-Tags: <!-- sectioninfo;;section;;subsection;;subsubsection;;nr_ausgeben;;testpage; //-->

%Fuegt eine zusaetzliche html-Seite an hinter ALLEN bisherigen und zukuenftigen content-Seiten ausserhalb der vor-zurueck-Schleife (d.h. nur durch Button oder MIntLink erreichbar!)
% #1 = Titel des Modulabschnitts, #2 = Kurztitel fuer die Buttons, #3 = Buttonkennung (STD = default nehmen, NONE = Ohne Button in der Navigation)
\newenvironment{MSContent}[3]{\special{html:<div class="xcontent}\arabic{MSCounter}\special{html:"><!-- scontent;-;}\arabic{MSCounter};-;#1;-;#2;-;#3\special{html: //-->}\MPageScripts\MSubsubsectionx{#1}}{\MEndScripts\special{html:<!-- endscontent;;}\arabic{MSCounter}\special{html: //--></div>}\addtocounter{MSCounter}{1}}

% Fuegt eine zusaetzliche html-Seite ein hinter den bereits vorhandenen content-Seiten (oder als erste Seite) innerhalb der vor-zurueck-Schleife der Navigation
% #1 = Titel des Modulabschnitts, #2 = Kurztitel fuer die Buttons, #3 = Buttonkennung (STD = Defaultbutton, NONE = Ohne Button in der Navigation)
\newenvironment{MXContent}[3]{\special{html:<div class="xcontent}\arabic{MXCounter}\special{html:"><!-- xcontent;-;}\arabic{MXCounter};-;#1;-;#2;-;#3\special{html: //-->}\MPageScripts\MSubsubsection{#1}}{\MEndScripts\special{html:<!-- endxcontent;;}\arabic{MXCounter}\special{html: //--></div>}\addtocounter{MXCounter}{1}}

% Fuegt eine zusaetzliche html-Seite ein die keine subsubsection-Nummer bekommt, nur zur internen Verwendung in mintmod.tex gedacht!
% #1 = Titel des Modulabschnitts, #2 = Kurztitel fuer die Buttons, #3 = Buttonkennung (STD = Defaultbutton, NONE = Ohne Button in der Navigation)
% \newenvironment{MUContent}[3]{\special{html:<div class="xcontent}\arabic{MXCounter}\special{html:"><!-- xcontent;-;}\arabic{MXCounter};-;#1;-;#2;-;#3\special{html: //-->}\MPageScripts\MSubsubsectionx{#1}}{\MEndScripts\special{html:<!-- endxcontent;;}\arabic{MXCounter}\special{html: //--></div>}\addtocounter{MXCounter}{1}}

\newcommand{\MDeclareSiteUXID}[1]{\special{html:<!-- mdeclaresiteuxid;;}#1\special{html:;;}\arabic{chapter}\special{html:;;}\arabic{section}\special{html:;; //-->}}

\else

%\newcommand{\MSubsubsection}[1]{\refstepcounter{subsubsection} \addcontentsline{toc}{subsubsection}{\thesubsubsection. #1}}


% Fuegt eine zusaetzliche html-Seite an hinter den bereits vorhandenen content-Seiten
% #1 = Titel des Modulabschnitts, #2 = Kurztitel fuer die Buttons, #3 = Iconkennung (im PDF wirkungslos)
%\newenvironment{MUContent}[3]{\ifnum\value{MXCTest}>0{\MDebugMessage{ERROR: Geschachtelter SContent}}\fi\MPageScripts\MSubsubsectionx{#1}\addtocounter{MXCTest}{1}}{\addtocounter{MXCounter}{1}\addtocounter{MXCTest}{-1}}
\newenvironment{MXContent}[3]{\ifnum\value{MXCTest}>0{\MDebugMessage{ERROR: Geschachtelter SContent}}\fi\MPageScripts\MSubsubsection{#1}\addtocounter{MXCTest}{1}}{\addtocounter{MXCounter}{1}\addtocounter{MXCTest}{-1}}
\newenvironment{MSContent}[3]{\ifnum\value{MXCTest}>0{\MDebugMessage{ERROR: Geschachtelter XContent}}\fi\MPageScripts\MSubsubsectionx{#1}\addtocounter{MXCTest}{1}}{\addtocounter{MSCounter}{1}\addtocounter{MXCTest}{-1}}

\newcommand{\MDeclareSiteUXID}[1]{\relax}

\fi 

% GHEADER und GFOOTER werden von split.pm gefunden, aber nur, wenn nicht HELPSITE oder TESTSITE
\ifttm
\newenvironment{MSectionStart}{\special{html:<div class="xcontent0">}\MSubsubsectionx{Modul\"ubersicht}}{\setcounter{MSSEnd}{0}\special{html:</div>}}
% Darf nicht als XContent nummeriert werden, darf nicht als XContent gelabelt werden, wird aber in eine xcontent-div gesetzt fuer Python-parsing
\else
\newenvironment{MSectionStart}{\MSubsectionx{Modul\"ubersicht}}{\setcounter{MSSEnd}{0}}
\fi

\newenvironment{MIntro}{\begin{MXContent}{Einf\"uhrung}{Einf\"uhrung}{genetisch}}{\end{MXContent}}
\newenvironment{MContent}{\begin{MXContent}{Inhalt}{Inhalt}{beweis}}{\end{MXContent}}
\newenvironment{MExercises}{\ifttm\else\clearpage\fi\begin{MXContent}{Aufgaben}{Aufgaben}{aufgb}\special{html:<!-- declareexcsymb //-->}}{\end{MXContent}}

% #1 = Lesbare Testbezeichnung
\newenvironment{MTest}[1]{%
\renewcommand{\MTestName}{#1}
\ifttm\else\clearpage\fi%
\addtocounter{MTestSite}{1}%
\begin{MXContent}{#1}{#1}{STD} % {aufgb}%
\special{html:<!-- declaretestsymb //-->}
\begin{MQuestionGroup}%
\MInTestHeader
}%
{%
\end{MQuestionGroup}%
\ \\ \ \\%
\MInTestFooter
\end{MXContent}\addtocounter{MTestSite}{-1}%
}

\newenvironment{MExtra}{\ifttm\else\clearpage\fi\begin{MXContent}{Zus\"atzliche Inhalte}{Zusatz}{weiterfhrg}}{\end{MXContent}}

\makeindex

\ifttm
\def\MPrintIndex{
\ifnum\value{MSSEnd}>0{\MSubsectionEndMacros}\addtocounter{MSSEnd}{-1}\fi
\renewcommand{\indexname}{Stichwortverzeichnis}
\special{html:<p><!-- printindex //--></p>}
}
\else
\def\MPrintIndex{
\ifnum\value{MSSEnd}>0{\MSubsectionEndMacros}\addtocounter{MSSEnd}{-1}\fi
\renewcommand{\indexname}{Stichwortverzeichnis}
\addcontentsline{toc}{section}{Stichwortverzeichnis}
\printindex
}
\fi


% Konstanten fuer die Modulfaecher

\def\MINTMathematics{1}
\def\MINTInformatics{2}
\def\MINTChemistry{3}
\def\MINTPhysics{4}
\def\MINTEngineering{5}

\newcounter{MSubjectArea}
\newcounter{MInfoNumbers} % Gibt an, ob die Infoboxen nummeriert werden sollen
\newcounter{MSepNumbers} % Gibt an, ob Beispiele und Experimente separat nummeriert werden sollen
\newcommand{\MSetSubject}[1]{
 % ttm kapiert setcounter mit Parametern nicht, also per if abragen und einsetzen
\ifnum#1=1\setcounter{MSubjectArea}{1}\setcounter{MInfoNumbers}{1}\setcounter{MSepNumbers}{0}\fi
\ifnum#1=2\setcounter{MSubjectArea}{2}\setcounter{MInfoNumbers}{1}\setcounter{MSepNumbers}{0}\fi
\ifnum#1=3\setcounter{MSubjectArea}{3}\setcounter{MInfoNumbers}{0}\setcounter{MSepNumbers}{1}\fi
\ifnum#1=4\setcounter{MSubjectArea}{4}\setcounter{MInfoNumbers}{0}\setcounter{MSepNumbers}{0}\fi
\ifnum#1=5\setcounter{MSubjectArea}{5}\setcounter{MInfoNumbers}{1}\setcounter{MSepNumbers}{0}\fi
% Separate Nummerntechnik fuer unsere Chemiker: alles dreistellig
\ifnum#1=3
  \ifttm
  \renewcommand{\theequation}{\arabic{section}.\arabic{subsection}.\arabic{equation}}
  \renewcommand{\thetable}{\arabic{section}.\arabic{subsection}.\arabic{table}} 
  \renewcommand{\thefigure}{\arabic{section}.\arabic{subsection}.\arabic{figure}} 
  \else
  \renewcommand{\theequation}{\arabic{chapter}.\arabic{section}.\arabic{equation}}
  \renewcommand{\thetable}{\arabic{chapter}.\arabic{section}.\arabic{table}}
  \renewcommand{\thefigure}{\arabic{chapter}.\arabic{section}.\arabic{figure}}
  \fi
\else
  \ifttm
  \renewcommand{\theequation}{\arabic{section}.\arabic{subsection}.\arabic{equation}}
  \renewcommand{\thetable}{\arabic{table}}
  \renewcommand{\thefigure}{\arabic{figure}}
  \else
  \renewcommand{\theequation}{\arabic{chapter}.\arabic{section}.\arabic{equation}}
  \renewcommand{\thetable}{\arabic{table}}
  \renewcommand{\thefigure}{\arabic{figure}}
  \fi
\fi
}

% Fuer tikz Autogenerierung
\newcounter{MTIKZAutofilenumber}

% Spezielle Counter fuer die Bentz-Module
\newcounter{mycounter}
\newcounter{chemapplet}
\newcounter{physapplet}

\newcounter{MSSEnd} % Ist 1 falls ein MSubsection aktiv ist, der einen MSubsectionEndMacro-Aufruf verursacht
\newcounter{MFileNumber}
\def\MLastFile{\special{html:[[!-- mfileref;;}\arabic{MFileNumber}\special{html:; //--]]}}

% Vollstaendiger Pfad ist \MMaterial / \MLastFilePath / \MLastFileName    ==   \MMaterial / \MLastFile

% Wird nur bei kompletter Baum-Erstellung ausgefuehrt!
% #1 = Lesbare Modulbezeichnung
\newcommand{\MSectionStartMacros}[1]{
\setcounter{MTestSite}{0}
\setcounter{MCaptionOn}{0}
\setcounter{MLastTypeEq}{0}
\setcounter{MSSEnd}{0}
\setcounter{MFileNumber}{0} % Preinkrekement-Counter
\setcounter{MTIKZAutofilenumber}{0}
\setcounter{mycounter}{1}
\setcounter{physapplet}{1}
\setcounter{chemapplet}{0}
\ifttm
\special{html:<!-- mdeclaresection;;}\arabic{chapter}\special{html:;;}\arabic{section}\special{html:;;}#1\special{html:;; //-->}%
\else
\setcounter{thmc}{0}
\setcounter{exmpc}{0}
\setcounter{verc}{0}
\setcounter{infoc}{0}
\fi
\setcounter{MiniMarkerCounter}{1}
\setcounter{AlignCounter}{1}
\setcounter{MXCTest}{0}
\setcounter{MCodeCounter}{0}
\setcounter{MEntryCounter}{0}
}

% Wird immer ausgefuehrt
\newcommand{\MSubsectionStartMacros}{
\ifttm\else\MPageHeaderDef\fi
\MWatermarkSettings
\setcounter{MXCounter}{0}
\setcounter{MSCounter}{0}
\setcounter{MSiteCounter}{1}
\setcounter{MExerciseCollectionCounter}{0}
% Zaehler fuer das Labelsystem zuruecksetzen (prefix-Zaehler)
\setcounter{MInfoCounter}{0}
\setcounter{MExerciseCounter}{0}
\setcounter{MExampleCounter}{0}
\setcounter{MExperimentCounter}{0}
\setcounter{MGraphicsCounter}{0}
\setcounter{MTableCounter}{0}
\setcounter{MTheoremCounter}{0}
\setcounter{MObjectCounter}{0}
\setcounter{MEquationCounter}{0}
\setcounter{MVideoCounter}{0}
\setcounter{equation}{0}
\setcounter{figure}{0}
}

\newcommand{\MSubsectionEndMacros}{
% Bei Chemiemodulen das PSE einhaengen, es soll als SContent am Ende erscheinen
\special{html:<!-- subsectionend //-->}
\ifnum\value{MSubjectArea}=3{\MIncludePSE}\fi
}


\ifttm
%\newcommand{\MEmbed}[1]{\MRegisterFile{#1}\begin{html}<embed src="\end{html}\MMaterial/\MLastFile\begin{html}" width="192" height="189"></embed>\end{html}}
\newcommand{\MEmbed}[1]{\MRegisterFile{#1}\begin{html}<embed src="\end{html}\MMaterial/\MLastFile\begin{html}"></embed>\end{html}}
\fi

%----------------- Makros fuer die Textdarstellung -----------------------------------------------

\ifttm
% MUGraphics bindet eine Grafik ein:
% Parameter 1: Dateiname der Grafik, relativ zur Position des Modul-Tex-Dokuments
% Parameter 2: Skalierungsoptionen fuer PDF (fuer includegraphics)
% Parameter 3: Titel fuer die Grafik, wird unter die Grafik mit der Grafiknummer gesetzt und kann MLabel bzw. MCopyrightLabel enthalten
% Parameter 4: Skalierungsoptionen fuer HTML (css-styles)

% ERSATZ: <img alt="My Image" src="data:image/png;base64,iVBORwA<MoreBase64SringHere>" />


\newcommand{\MUGraphics}[4]{\MRegisterFile{#1}\begin{html}
<div class="imagecenter">
<center>
<div>
<img src="\end{html}\MMaterial/\MLastFile\begin{html}" style="#4" alt="\end{html}\MMaterial/\MLastFile\begin{html}"/>
</div>
<div class="bildtext">
\end{html}
\addtocounter{MGraphicsCounter}{1}
\setcounter{MLastIndex}{\value{MGraphicsCounter}}
\setcounter{MLastType}{8}
\addtocounter{MCaptionOn}{1}
\ifnum\value{MSepNumbers}=0
\textbf{Abbildung \arabic{MGraphicsCounter}:} #3
\else
\textbf{Abbildung \arabic{section}.\arabic{subsection}.\arabic{MGraphicsCounter}:} #3
\fi
\addtocounter{MCaptionOn}{-1}
\begin{html}
</div>
</center>
</div>
<br />
\end{html}%
\special{html:<!-- mfeedbackbutton;Abbildung;}\arabic{MGraphicsCounter}\special{html:;}\arabic{section}.\arabic{subsection}.\arabic{MGraphicsCounter}\special{html:; //-->}%
}

% MVideo bindet ein Video als Einzeldatei ein:
% Parameter 1: Dateiname des Videos, relativ zur Position des Modul-Tex-Dokuments, ohne die Endung ".mp4"
% Parameter 2: Titel fuer das Video (kann MLabel oder MCopyrightLabel enthalten), wird unter das Video mit der Videonummer gesetzt
\newcommand{\MVideo}[2]{\MRegisterFile{#1.mp4}\begin{html}
<div class="imagecenter">
<center>
<div>
<video width="95\%" controls="controls"><source src="\end{html}\MMaterial/#1.mp4\begin{html}" type="video/mp4">Ihr Browser kann keine MP4-Videos abspielen!</video>
</div>
<div class="bildtext">
\end{html}
\addtocounter{MVideoCounter}{1}
\setcounter{MLastIndex}{\value{MVideoCounter}}
\setcounter{MLastType}{12}
\addtocounter{MCaptionOn}{1}
\ifnum\value{MSepNumbers}=0
\textbf{Video \arabic{MVideoCounter}:} #2
\else
\textbf{Video \arabic{section}.\arabic{subsection}.\arabic{MVideoCounter}:} #2
\fi
\addtocounter{MCaptionOn}{-1}
\begin{html}
</div>
</center>
</div>
<br />
\end{html}}

\newcommand{\MDVideo}[2]{\MRegisterFile{#1.mp4}\MRegisterFile{#1.ogv}\begin{html}
<div class="imagecenter">
<center>
<div>
<video width="70\%" controls><source src="\end{html}\MMaterial/#1.mp4\begin{html}" type="video/mp4"><source src="\end{html}\MMaterial/#1.ogv\begin{html}" type="video/ogg">Ihr Browser kann keine MP4-Videos abspielen!</video>
</div>
<br />
#2
</center>
</div>
<br />
\end{html}
}

\newcommand{\MGraphics}[3]{\MUGraphics{#1}{#2}{#3}{}}

\else

\newcommand{\MVideo}[2]{%
% Kein Video im PDF darstellbar, trotzdem so tun als ob da eines waere
\begin{center}
(Video nicht darstellbar)
\end{center}
\addtocounter{MVideoCounter}{1}
\setcounter{MLastIndex}{\value{MVideoCounter}}
\setcounter{MLastType}{12}
\addtocounter{MCaptionOn}{1}
\ifnum\value{MSepNumbers}=0
\textbf{Video \arabic{MVideoCounter}:} #2
\else
\textbf{Video \arabic{section}.\arabic{subsection}.\arabic{MVideoCounter}:} #2
\fi
\addtocounter{MCaptionOn}{-1}
}


% MGraphics bindet eine Grafik ein:
% Parameter 1: Dateiname der Grafik, relativ zur Position des Modul-Tex-Dokuments
% Parameter 2: Skalierungsoptionen fuer PDF (fuer includegraphics)
% Parameter 3: Titel fuer die Grafik, wird unter die Grafik mit der Grafiknummer gesetzt
\newcommand{\MGraphics}[3]{%
\MRegisterFile{#1}%
\ %
\begin{figure}[H]%
\centering{%
\includegraphics[#2]{\MDPrefix/#1}%
\addtocounter{MCaptionOn}{1}%
\caption{#3}%
\addtocounter{MCaptionOn}{-1}%
}%
\end{figure}%
\addtocounter{MGraphicsCounter}{1}\setcounter{MLastIndex}{\value{MGraphicsCounter}}\setcounter{MLastType}{8}\ %
%\ \\Abbildung \ifnum\value{MSepNumbers}=0\else\arabic{chapter}.\arabic{section}.\fi\arabic{MGraphicsCounter}: #3%
}

\newcommand{\MUGraphics}[4]{\MGraphics{#1}{#2}{#3}}


\fi

\newcounter{MCaptionOn} % = 1 falls eine Grafikcaption aktiv ist, = 0 sonst


% MGraphicsSolo bindet eine Grafik pur ein ohne Titel
% Parameter 1: Dateiname der Grafik, relativ zur Position des Modul-Tex-Dokuments
% Parameter 2: Skalierungsoptionen (wirken nur im PDF)
\newcommand{\MGraphicsSolo}[2]{\MUGraphicsSolo{#1}{#2}{}}

% MUGraphicsSolo bindet eine Grafik pur ein ohne Titel, aber mit HTML-Skalierung
% Parameter 1: Dateiname der Grafik, relativ zur Position des Modul-Tex-Dokuments
% Parameter 2: Skalierungsoptionen (wirken nur im PDF)
% Parameter 3: Skalierungsoptionen (wirken nur im HTML), als style-format: "width=???, height=???"
\ifttm
\newcommand{\MUGraphicsSolo}[3]{\MRegisterFile{#1}\begin{html}
<img src="\end{html}\MMaterial/\MLastFile\begin{html}" style="\end{html}#3\begin{html}" alt="\end{html}\MMaterial/\MLastFile\begin{html}"/>
\end{html}%
\special{html:<!-- mfeedbackbutton;Abbildung;}#1\special{html:;}\MMaterial/\MLastFile\special{html:; //-->}%
}
\else
\newcommand{\MUGraphicsSolo}[3]{\MRegisterFile{#1}\includegraphics[#2]{\MDPrefix/#1}}
\fi

% Externer Link mit URL
% Erster Parameter: Vollstaendige(!) URL des Links
% Zweiter Parameter: Text fuer den Link
\newcommand{\MExtLink}[2]{\ifttm\special{html:<a target="_new" href="}#1\special{html:">}#2\special{html:</a>}\else\href{#1}{#2}\fi} % ohne MINTERLINK!


% Interner Link, die verlinkte Datei muss im gleichen Verzeichnis liegen wie die Modul-Texdatei
% Erster Parameter: Dateiname
% Zweiter Parameter: Text fuer den Link
\newcommand{\MIntLink}[2]{\ifttm\MRegisterFile{#1}\special{html:<a class="MINTERLINK" target="_new" href="}\MMaterial/\MLastFile\special{html:">}#2\special{html:</a>}\else{\href{#1}{#2}}\fi}


\ifttm
\def\MMaterial{:localmaterial:}
\else
\def\MMaterial{\MDPrefix}
\fi

\ifttm
\def\MNoFile#1{:directmaterial:#1}
\else
\def\MNoFile#1{#1}
\fi

\newcommand{\MChem}[1]{$\mathrm{#1}$}

\newcommand{\MApplet}[3]{
% Bindet ein Java-Applet ein, die Parameter sind:
% (wird nur im HTML, aber nicht im PDF erstellt)
% #1 Dateiname des Applets (muss mit ".class" enden)
% #2 = Breite in Pixeln
% #3 = Hoehe in Pixeln
\ifttm
\MRegisterFile{#1}
\begin{html}
<applet code="\end{html}\MMaterial/\MLastFile\begin{html}" width="#2" height="#3" alt="[Java-Applet kann nicht gestartet werden]"></applet>
\end{html}
\fi
}

\newcommand{\MScriptPage}[2]{
% Bindet eine JavaScript-Datei ein, die eine eigene Seite bekommt
% (wird nur im HTML, aber nicht im PDF erstellt)
% #1 Dateiname des Programms (sollte mit ".js" enden)
% #2 = Kurztitel der Seite
\ifttm
\begin{MSContent}{#2}{#2}{puzzle}
\MRegisterFile{#1}
\begin{html}
<script src="\MMaterial/\MLastFile" type="text/javascript"></script>
\end{html}
\end{MSContent}
\fi
}

\newcommand{\MIncludePSE}{
% Bindet bei Chemie-Modulen das PSE ein
% (wird nur im HTML, aber nicht im PDF erstellt)
\ifttm
\special{html:<!-- includepse //-->}
\begin{MSContent}{Periodensystem der Elemente}{PSE}{table}
\MRegisterFile{../files/pse.js}
\MRegisterFile{../files/radio.png}
\begin{html}
<script src="\MMaterial/../files/pse.js" type="text/javascript"></script>
<p id="divid"><br /><br />
<script language="javascript" type="text/javascript">
    startpse("divid","\MMaterial/../files"); 
</script>
</p>
<br />
<br />
<br />
<p>Die Farben der Elementsymbole geben an: <font style="color:Red">gasf&ouml;rmig </font> <font style="color:Blue">fl&uuml;ssig </font> fest</p>
<p>Die Elemente der Gruppe 1 A, 2 A, 3 A usw. geh&ouml;ren zu den Hauptgruppenelementen.</p>
<p>Die Elemente der Gruppe 1 B, 2 B, 3 B usw. geh&ouml;ren zu den Nebengruppenelementen.</p>
<p>() kennzeichnet die Masse des stabilsten Isotops</p>
\end{html}
\end{MSContent}
\fi
}

\newcommand{\MAppletArchive}[4]{
% Bindet ein Java-Applet ein, die Parameter sind:
% (wird nur im HTML, aber nicht im PDF erstellt)
% #1 Dateiname der Klasse mit Appletaufruf (muss mit ".class" enden)
% #2 Dateiname des Archivs (muss mit ".jar" enden)
% #3 = Breite in Pixeln
% #4 = Hoehe in Pixeln
\ifttm
\MRegisterFile{#2}
\begin{html}
<applet code="#1" archive="\end{html}\MMaterial/\MLastFile\begin{html}" codebase="." width="#3" height="#4" alt="[Java-Archiv kann nicht gestartet werden]"></applet>
\end{html}
\fi
}

% Bindet in der Haupttexdatei ein MINT-Modul ein. Parameter 1 ist das Verzeichnis (relativ zur Haupttexdatei), Parameter 2 ist der Dateinahme ohne Pfad.
\newcommand{\IncludeModule}[2]{
\renewcommand{\MDPrefix}{#1}
\input{#1/#2}
\ifnum\value{MSSEnd}>0{\MSubsectionEndMacros}\addtocounter{MSSEnd}{-1}\fi
}

% Der ttm-Konverter setzt keine Makros im \input um, also muss hier getrickst werden:
% Das MDPrefix muss in den einzelnen Modulen manuell eingesetzt werden
\newcommand{\MInputFile}[1]{
\ifttm
\input{#1}
\else
\input{#1}
\fi
}


\newcommand{\MCases}[1]{\left\lbrace{\begin{array}{rl} #1 \end{array}}\right.}

\ifttm
\newenvironment{MCaseEnv}{\left\lbrace\begin{array}{rl}}{\end{array}\right.}
\else
\newenvironment{MCaseEnv}{\left\lbrace\begin{array}{rl}}{\end{array}\right.}
\fi

\def\MSkip{\ifttm\MCR\fi}

\ifttm
\def\MCR{\special{html:<br />}}
\else
\def\MCR{\ \\}
\fi


% Pragmas - Sind Schluesselwoerter, die dem Preprocessing sowie dem Konverter uebergeben werden und bestimmte
%           Aktionen ausloesen. Im Output (PDF und HTML) tauchen sie nicht auf.
\newcommand{\MPragma}[1]{%
\ifttm%
\special{html:<!-- mpragma;-;}#1\special{html:;; -->}%
\else%
% MPragmas werden vom Preprozessor direkt im LaTeX gefunden
\fi%
}

% Ersatz der Befehle textsubscript und textsuperscript, die ttm nicht kennt
\ifttm%
\newcommand{\MTextsubscript}[1]{\special{html:<sub>}#1\special{html:</sub>}}%
\newcommand{\MTextsuperscript}[1]{\special{html:<sup>}#1\special{html:</sup>}}%
\else%
\newcommand{\MTextsubscript}[1]{\textsubscript{#1}}%
\newcommand{\MTextsuperscript}[1]{\textsuperscript{#1}}%
\fi

%------------------ Einbindung von dia-Diagrammen ----------------------------------------------
% Beim preprocessing wird aus jeder dia-Datei eine tex-Datei und eine pdf-Datei erzeugt,
% diese werden hier jeweils im PDF und HTML eingebunden
% Parameter: Dateiname der mit dia erstellten Datei (OHNE die Endung .dia)
\ifttm%
\newcommand{\MDia}[1]{%
\MGraphicsSolo{#1minthtml.png}{}%
}
\else%
\newcommand{\MDia}[1]{%
\MGraphicsSolo{#1mintpdf.png}{scale=0.1667}%
}
\fi%

% subsup funktioniert im Ausdruck $D={\R}^+_0$, also \R geklammert und sup zuerst
% \ifttm
% \def\MSubsup#1#2#3{\special{html:<msubsup>} #1 #2 #3\special{html:</msubsup>}}
% \else
% \def\MSubsup#1#2#3{{#1}^{#3}_{#2}}
% \fi

%\input{local.tex}

% \ifttm
% \else
% \newwrite\mintlog
% \immediate\openout\mintlog=mintlog.txt
% \fi

% ----------------------- tikz autogenerator -------------------------------------------------------------------

\newcommand{\Mtikzexternalize}{\tikzexternalize}% wird bei Konvertierung ueber mconvert ggf. ausgehebelt!

\ifttm
\else
\tikzset%
{
  % Defines a custom style which generates pdf and converts to (low and hi-res quality) png and svg, then deletes the pdf
  % Important: DO NOT directly convert from pdf to hires-png or from svg to png with GraphViz convert as it has some problems and memory leaks
  png export/.style=%
  {
    external/system call/.add={}{; 
      pdf2svg "\image.pdf" "\image.svg" ; 
      convert -density 112.5 -transparent white "\image.pdf" "\image.png"; 
      inkscape --export-png="\image.4x.png" --export-dpi=450 --export-background-opacity=0 --without-gui "\image.svg"; 
      rm "\image.pdf"; rm "\image.log"; rm "\image.dpth"; rm "\image.idx"
    },
    external/force remake,
  }
}
\tikzset{png export}
\tikzsetexternalprefix{}
% PNGs bei externer Erzeugung in "richtiger" Groesse einbinden
\pgfkeys{/pgf/images/include external/.code={\includegraphics[scale=0.64]{#1}}}
\fi

% Spezielle Umgebung fuer Autogenerierung, Bildernamen sind nur innerhalb eines Moduls (einer MSection) eindeutig)

\newcommand{\MTIKZautofilename}{tikzautofile}

\ifttm
% HTML-Version: Vom Autogenerator erzeugte png-Datei einbinden, tikz selbst nicht ausfuehren (sprich: #1 schlucken)
\newcommand{\MTikzAuto}[1]{%
\addtocounter{MTIKZAutofilenumber}{1}
\renewcommand{\MTIKZautofilename}{mtikzauto_\arabic{MTIKZAutofilenumber}}
\MUGraphicsSolo{\MSectionID\MTIKZautofilename.4x.png}{scale=1}{\special{html:[[!-- svgstyle;}\MSectionID\MTIKZautofilename\special{html: //--]]}} % Styleinfos werden aus original-png, nicht 4x-png geholt!
%\MRegisterFile{\MSectionID\MTIKZautofilename.png} % not used right now
%\MRegisterFile{\MSectionID\MTIKZautofilename.svg}
}
\else%
% PDF-Version: Falls Autogenerator aktiv wird Datei automatisch benannt und exportiert
\newcommand{\MTikzAuto}[1]{%
\addtocounter{MTIKZAutofilenumber}{1}%
\renewcommand{\MTIKZautofilename}{mtikzauto_\arabic{MTIKZAutofilenumber}}
\tikzsetnextfilename{\MTIKZautofilename}%
#1%
}
\fi

% In einer reinen LaTeX-Uebersetzung kapselt der Preambelinclude-Befehl nur input,
% in einer konvertergesteuerten PDF/HTML-Uebersetzung wird er dagegen entfernt und
% die Preambeln an mintmod angehaengt, die Ersetzung wird von mconvert.pl vorgenommen.

\newcommand{\MPreambleInclude}[1]{\input{#1}}

% Globale Watermarksettings (werden auch nochmal zu Beginn jedes subsection gesetzt,
% muessen hier aber auch global ausgefuehrt wegen Einfuehrungsseiten und Inhaltsverzeichnis

\MWatermarkSettings
% ---------------------------------- Parametrisierte Aufgaben ----------------------------------------

\ifttm
\newenvironment{MPExercise}{%
\begin{MExercise}%
}{%
\special{html:<button name="Name_MPEX}\arabic{MExerciseCounter}\special{html:" id="MPEX}\arabic{MExerciseCounter}%
\special{html:" type="button" onclick="reroll('}\arabic{MExerciseCounter}\special{html:');">Neue Aufgabe erzeugen</button>}%
\end{MExercise}%
}
\else
\newenvironment{MPExercise}{%
\begin{MExercise}%
}{%
\end{MExercise}%
}
\fi

% Parameter: Name, Min, Max, PDF-Standard. Name in Deklaration OHNE backslash, im Code MIT Backslash
\ifttm
\newcommand{\MGlobalInteger}[4]{\special{html:%
<!-- onloadstart //-->%
MVAR.push(createGlobalInteger("}#1\special{html:",}#2\special{html:,}#3\special{html:,}#4\special{html:)); %
<!-- onloadstop //-->%
<!-- viewmodelstart //-->%
ob}#1\special{html:: ko.observable(rerollMVar("}#1\special{html:")),%
<!-- viewmodelstop //-->%
}%
}%
\else%
\newcommand{\MGlobalInteger}[4]{\newcounter{mvc_#1}\setcounter{mvc_#1}{#4}}
\fi

% Parameter: Name, Min, Max, PDF-Standard. Name in Deklaration OHNE backslash, im Code MIT Backslash, Wert ist Wurzel von value
\ifttm
\newcommand{\MGlobalSqrt}[4]{\special{html:%
<!-- onloadstart //-->%
MVAR.push(createGlobalSqrt("}#1\special{html:",}#2\special{html:,}#3\special{html:,}#4\special{html:)); %
<!-- onloadstop //-->%
<!-- viewmodelstart //-->%
ob}#1\special{html:: ko.observable(rerollMVar("}#1\special{html:")),%
<!-- viewmodelstop //-->%
}%
}%
\else%
\newcommand{\MGlobalSqrt}[4]{\newcounter{mvc_#1}\setcounter{mvc_#1}{#4}}% Funktioniert nicht als Wurzel !!!
\fi

% Parameter: Name, Min, Max, PDF-Standard zaehler, PDF-Standard nenner. Name in Deklaration OHNE backslash, im Code MIT Backslash
\ifttm
\newcommand{\MGlobalFraction}[5]{\special{html:%
<!-- onloadstart //-->%
MVAR.push(createGlobalFraction("}#1\special{html:",}#2\special{html:,}#3\special{html:,}#4\special{html:,}#5\special{html:)); %
<!-- onloadstop //-->%
<!-- viewmodelstart //-->%
ob}#1\special{html:: ko.observable(rerollMVar("}#1\special{html:")),%
<!-- viewmodelstop //-->%
}%
}%
\else%
\newcommand{\MGlobalFraction}[5]{\newcounter{mvc_#1}\setcounter{mvc_#1}{#4}} % Funktioniert nicht als Bruch !!!
\fi

% MVar darf im HTML nur in MEvalMathDisplay-Umgebungen genutzt werden oder in Strings die an den Parser uebergeben werden
\ifttm%
\newcommand{\MVar}[1]{\special{html:[var_}#1\special{html:]}}%
\else%
\newcommand{\MVar}[1]{\arabic{mvc_#1}}%
\fi

\ifttm%
\newcommand{\MRerollButton}[2]{\special{html:<button type="button" onclick="rerollMVar('}#1\special{html:');">}#2\special{html:</button>}}%
\else%
\newcommand{\MRerollButton}[2]{\relax}% Keine sinnvolle Entsprechung im PDF
\fi

% MEvalMathDisplay fuer HTML wird in mconvert.pl im preprocessing realisiert
% PDF: eine equation*-Umgebung (ueber amsmath)
% HTML: Eine Mathjax-Tex-Umgebung, deren Auswertung mit knockout-obervablen gekoppelt ist
% PDF-Version hier nur fuer pdflatex-only-Uebersetzung gegeben

\ifttm\else\newenvironment{MEvalMathDisplay}{\begin{equation*}}{\end{equation*}}\fi

% ---------------------------------- Spezialbefehle fuer AD ------------------------------------------

%Abk�rzung f�r \longrightarrow:
\newcommand{\lto}{\ensuremath{\longrightarrow}}

%Makro f�r Funktionen:
\newcommand{\exfunction}[5]
{\begin{array}{rrcl}
 #1 \colon  & #2 &\lto & #3 \\[.05cm]  
  & #4 &\longmapsto  & #5 
\end{array}}

\newcommand{\function}[5]{%
#1:\;\left\lbrace{\begin{array}{rcl}
 #2 &\lto & #3 \\
 #4 &\longmapsto  & #5 \end{array}}\right.}


%Die Identit�t:
\DeclareMathOperator{\Id}{Id}

%Die Signumfunktion:
\DeclareMathOperator{\sgn}{sgn}

%Zwei Betonungskommandos (k�nnen angepasst werden):
\newcommand{\highlight}[1]{#1}
\newcommand{\modstextbf}[1]{#1}
\newcommand{\modsemph}[1]{#1}


% ---------------------------------- Spezialbefehle fuer JL ------------------------------------------


\def\jccolorfkt{green!50!black} %Farbe des Funktionsgraphen
\def\jccolorfktarea{green!25!white} %Farbe der Fl"ache unter dem Graphen
\def\jccolorfktareahell{green!12!white} %helle Einf"arbung der Fl"ache unter dem Graphen
\def\jccolorfktwert{green!50!black} %Farbe einzelner Punkte des Graphen

\newcommand{\MPfadBilder}{Bilder}

\ifttm%
\newcommand{\jMD}{\,\MD}%
\else%
\newcommand{\jMD}{\;\MD}%
\fi%

\def\jHTMLHinweisBedienung{\MInputHint{%
Mit Hilfe der Symbole am oberen Rand des Fensters
k"onnen Sie durch die einzelnen Abschnitte navigieren.}}

\def\jHTMLHinweisEingabeText{\MInputHint{%
Geben Sie jeweils ein Wort oder Zeichen als Antwort ein.}}

\def\jHTMLHinweisEingabeTerm{\MInputHint{%
Klammern Sie Ihre Terme, um eine eindeutige Eingabe zu erhalten. 
Beispiel: Der Term $\frac{3x+1}{x-2}$ soll in der Form
\texttt{(3*x+1)/((x+2)^2}$ eingegeben werden (wobei auch Leerzeichen 
eingegeben werden k"onnen, damit eine Formel besser lesbar ist).}}

\def\jHTMLHinweisEingabeIntervalle{\MInputHint{%
Intervalle werden links mit einer "offnenden Klammer und rechts mit einer 
schlie"senden Klammer angegeben. Eine runde Klammer wird verwendet, wenn der 
Rand nicht dazu geh"ort, eine eckige, wenn er dazu geh"ort. 
Als Trennzeichen wird ein Komma oder ein Semikolon akzeptiert.
Beispiele: $(a, b)$ offenes Intervall,
$[a; b)$ links abgeschlossenes, rechts offenes Intervall von $a$ bis $b$. 
Die Eingabe $]a;b[$ f"ur ein offenes Intervall wird nicht akzeptiert.
F"ur $\infty$ kann \texttt{infty} oder \texttt{unendlich} geschrieben werden.}}

\def\jHTMLHinweisEingabeFunktionen{\MInputHint{%
Schreiben Sie Malpunkte (geschrieben als \texttt{*}) aus und setzen Sie Klammern um Argumente f�r Funktionen.
Beispiele: Polynom: \texttt{3*x + 0.1}, Sinusfunktion: \texttt{sin(x)}, 
Verkettung von cos und Wurzel: \texttt{cos(sqrt(3*x))}.}}

\def\jHTMLHinweisEingabeFunktionenSinCos{\MInputHint{%
Die Sinusfunktion $\sin x$ wird in der Form \texttt{sin(x)} angegeben, %
$\cos\left(\sqrt{3 x}\right)$ durch \texttt{cos(sqrt(3*x))}.}}

\def\jHTMLHinweisEingabeFunktionenExp{\MInputHint{%
Die Exponentialfunktion $\MEU^{3x^4 + 5}$ wird als
\texttt{exp(3 * x^4 + 5)} angegeben, %
$\ln\left(\sqrt{x} + 3.2\right)$ durch \texttt{ln(sqrt(x) + 3.2)}.}}

% ---------------------------------- Spezialbefehle fuer Fachbereich Physik --------------------------

\newcommand{\E}{{e}}
\newcommand{\ME}[1]{\cdot 10^{#1}}
\newcommand{\MU}[1]{\;\mathrm{#1}}
\newcommand{\MPG}[3]{%
  \ifnum#2=0%
    #1\ \mathrm{#3}%
  \else%
    #1\cdot 10^{#2}\ \mathrm{#3}%
  \fi}%
%

\newcommand{\MMul}{\MExponentensymbXYZl} % Nur eine Abkuerzung


% ---------------------------------- Stichwortfunktionialitaet ---------------------------------------

% mpreindexentry wird durch Auswahlroutine in conv.pl durch mindexentry substitutiert
\ifttm%
\def\MIndex#1{\index{#1}\special{html:<!-- mpreindexentry;;}#1\special{html:;;}\arabic{MSubjectArea}\special{html:;;}%
\arabic{chapter}\special{html:;;}\arabic{section}\special{html:;;}\arabic{subsection}\special{html:;;}\arabic{MEntryCounter}\special{html:; //-->}%
\setcounter{MLastIndex}{\value{MEntryCounter}}%
\addtocounter{MEntryCounter}{1}%
}%
% Copyrightliste wird als tex-Datei im preprocessing von conv.pl erzeugt und unter converter/tex/entrycollection.tex abgelegt
% Der input-Befehl funktioniert nur, wenn die aufrufende tex-Datei auf der obersten Ebene liegt (d.h. selbst kein input/include ist, insbesondere keine Moduldatei)
\def\MEntryList{} % \input funktioniert nicht, weil ttm (und damit das \input) ausgefuehrt wird, bevor Datei da ist
\else%
\def\MIndex#1{\index{#1}}
\def\MEntryList{\MAbort{Stichwortliste nur im HTML realisierbar}}%
\fi%

\def\MEntry#1#2{\textbf{#1}\MIndex{#2}} % Idee: MLastType auf neuen Entry-Typ und dann ein MLabel vergeben mit autogen-Nummer

% ---------------------------------- Befehle fuer Tests ----------------------------------------------

% MEquationItem stellt eine Eingabezeile der Form Vorgabe = Antwortfeld her, der zweite Parameter kann z.B. MSimplifyQuestion-Befehl sein
\ifttm
\newcommand{\MEquationItem}[2]{{#1}$\,=\,${#2}}%
\else%
\newcommand{\MEquationItem}[2]{{#1}$\;\;=\,${#2}}%
\fi

\ifttm
\newcommand{\MInputHint}[1]{%
\ifnum%
\if\value{MTestSite}>0%
\else%
{\color{blue}#1}%
\fi%
\fi%
}
\else
\newcommand{\MInputHint}[1]{\relax}
\fi

\ifttm
\newcommand{\MInTestHeader}{%
Dies ist ein einreichbarer Test:
\begin{itemize}
\item{Im Gegensatz zu den offenen Aufgaben werden beim Eingeben keine Hinweise zur Formulierung der mathematischen Ausdr�cke gegeben.}
\item{Der Test kann jederzeit neu gestartet oder verlassen werden.}
\item{Der Test kann durch die Buttons am Ende der Seite beendet und abgeschickt, oder zur�ckgesetzt werden.}
\item{Der Test kann mehrfach probiert werden. F�r die Statistik z�hlt die zuletzt abgeschickte Version.}
\end{itemize}
}
\else
\newcommand{\MInTestHeader}{%
\relax
}
\fi

\ifttm
\newcommand{\MInTestFooter}{%
\special{html:<button name="Name_TESTFINISH" id="TESTFINISH" type="button" onclick="finish_button('}\MTestName\special{html:');">Test auswerten</button>}%
\begin{html}
&nbsp;&nbsp;&nbsp;&nbsp;&nbsp;&nbsp;&nbsp;&nbsp;
<button name="Name_TESTRESET" id="TESTRESET" type="button" onclick="reset_button();">Test zur�cksetzen</button>
<br />
<br />
<div class="xreply">
<p name="Name_TESTEVAL" id="TESTEVAL">
Hier erscheint die Testauswertung!
<br />
</p>
</div>
\end{html}
}
\else
\newcommand{\MInTestFooter}{%
\relax
}
\fi


% ---------------------------------- Notationsmakros -------------------------------------------------------------

% Notationsmakros die nicht von der Kursvariante abhaengig sind

\newcommand{\MZahltrennzeichen}[1]{\renewcommand{\MZXYZhltrennzeichen}{#1}}

\ifttm
\newcommand{\MZahl}[3][\MZXYZhltrennzeichen]{\edef\MZXYZtemp{\noexpand\special{html:<mn>#2#1#3</mn>}}\MZXYZtemp}
\else
\newcommand{\MZahl}[3][\MZXYZhltrennzeichen]{{}#2{#1}#3}
\fi

\newcommand{\MEinheitenabstand}[1]{\renewcommand{\MEinheitenabstXYZnd}{#1}}
\ifttm
\newcommand{\MEinheit}[2][\MEinheitenabstXYZnd]{{}#1\edef\MEINHtemp{\noexpand\special{html:<mi mathvariant="normal">#2</mi>}}\MEINHtemp} 
\else
\newcommand{\MEinheit}[2][\MEinheitenabstXYZnd]{{}#1 \mathrm{#2}} 
\fi

\newcommand{\MExponentensymbol}[1]{\renewcommand{\MExponentensymbXYZl}{#1}}
\newcommand{\MExponent}[2][\MExponentensymbXYZl]{{}#1{} 10^{#2}} 

%Punkte in 2 und 3 Dimensionen
\newcommand{\MPointTwo}[3][]{#1(#2\MCoordPointSep #3{}#1)}
\newcommand{\MPointThree}[4][]{#1(#2\MCoordPointSep #3\MCoordPointSep #4{}#1)}
\newcommand{\MPointTwoAS}[2]{\left(#1\MCoordPointSep #2\right)}
\newcommand{\MPointThreeAS}[3]{\left(#1\MCoordPointSep #2\MCoordPointSep #3\right)}

% Masseinheit, Standardabstand: \,
\newcommand{\MEinheitenabstXYZnd}{\MThinspace} 

% Horizontaler Leerraum zwischen herausgestellter Formel und Interpunktion
\ifttm
\newcommand{\MDFPSpace}{\,}
\newcommand{\MDFPaSpace}{\,\,}
\newcommand{\MBlank}{\ }
\else
\newcommand{\MDFPSpace}{\;}
\newcommand{\MDFPaSpace}{\;\;}
\newcommand{\MBlank}{\ }
\fi

% Satzende in herausgestellter Formel mit horizontalem Leerraum
\newcommand{\MDFPeriod}{\MDFPSpace .}

% Separation von Aufzaehlung und Bedingung in Menge
\newcommand{\MCondSetSep}{\,:\,} %oder '\mid'

% Konverter kennt mathopen nicht
\ifttm
\def\mathopen#1{}
\fi

% -----------------------------------START Rouletteaufgaben ------------------------------------------------------------

\ifttm
% #1 = Dateiname, #2 = eindeutige ID fuer das Roulette im Kurs
\newcommand{\MDirectRouletteExercises}[2]{
\begin{MExercise}
\texttt{Im HTML erscheinen hier Aufgaben aus einer Aufgabenliste...}
\end{MExercise}
}
\else
\newcommand{\MDirectRouletteExercises}[2]{\relax} % wird durch mconvert.pl gefunden und ersetzt
\fi


% ---------------------------------- START Makros, die von der Kursvariante abhaengen ----------------------------------

\ifvariantunotation
  % unotation = An Universitaeten uebliche Notation
  \def\MVariant{unotation}

  % Trennzeichen fuer Dezimalzahlen
  \newcommand{\MZXYZhltrennzeichen}{.}

  % Exponent zur Basis 10 in der Exponentialschreibweise, 
  % Standardmalzeichen: \times
  \newcommand{\MExponentensymbXYZl}{\times} 

  % Begrenzungszeichen fuer offene Intervalle
  \newcommand{\MoIl}[1][]{\mbox{}#1(\mathopen{}} % bzw. ']'
  \newcommand{\MoIr}[1][]{#1)\mbox{}} % bzw. '['

  % Zahlen-Separation im IntervaLL
  \newcommand{\MIntvlSep}{,} %oder ';'

  % Separation von Elementen in Mengen
  \newcommand{\MElSetSep}{,} %oder ';'

  % Separation von Koordinaten in Punkten
  \newcommand{\MCoordPointSep}{,} %oder ';' oder '|', '\MThinspace|\MThinspace'

\else
  % An dieser Stelle wird angenommen, dass std-Variante aktiv ist
  % std = beschlossene Notation im TU9-Onlinekurs 
  \def\MVariant{std}

  % Trennzeichen fuer Dezimalzahlen
  \newcommand{\MZXYZhltrennzeichen}{,}

  % Exponent zur Basis 10 in der Exponentialschreibweise, 
  % Standardmalzeichen: \times
  \newcommand{\MExponentensymbXYZl}{\times} 

  % Begrenzungszeichen fuer offene Intervalle
  \newcommand{\MoIl}[1][]{\mbox{}#1]\mathopen{}} % bzw. '('
  \newcommand{\MoIr}[1][]{#1[\mbox{}} % bzw. ')'

  % Zahlen-Separation im IntervaLL
  \newcommand{\MIntvlSep}{;} %oder ','
  
  % Separation von Elementen in Mengen
  \newcommand{\MElSetSep}{;} %oder ','

  % Separation von Koordinaten in Punkten
  \newcommand{\MCoordPointSep}{;} %oder '|', '\MThinspace|\MThinspace'

\fi



% ---------------------------------- ENDE Makros, die von der Kursvariante abhaengen ----------------------------------


% diese Kommandos setzen Mathemodus vorraus
\newcommand{\MGeoAbstand}[2]{[\overline{{#1}{#2}}]}
\newcommand{\MGeoGerade}[2]{{#1}{#2}}
\newcommand{\MGeoStrecke}[2]{\overline{{#1}{#2}}}
\newcommand{\MGeoDreieck}[3]{{#1}{#2}{#3}}

%
\ifttm
\newcommand{\MOhm}{\special{html:<mn>&#x3A9;</mn>}}
\else
\newcommand{\MOhm}{\Omega} %\varOmega
\fi


\def\PERCTAG{\MAbort{PERCTAG ist zur internen verwendung in mconvert.pl reserviert, dieses Makro darf sonst nicht benutzt werden.}}

% Im Gegensatz zu einfachen html-Umgebungen werden MDirectHTML-Umgebungen von mconvert.pl am ganzen ttm-Prozess vorbeigeschleust und aus dem PDF komplett ausgeschnitten
\ifttm%
\newenvironment{MDirectHTML}{\begin{html}}{\end{html}}%
\else%
\newenvironment{MDirectHTML}{\begin{html}}{\end{html}}%
\fi

% Im Gegensatz zu einfachen Mathe-Umgebungen werden MDirectMath-Umgebungen von mconvert.pl am ganzen ttm-Prozess vorbeigeschleust, ueber MathJax realisiert, und im PDF als $$ ... $$ gesetzt
\ifttm%
\newenvironment{MDirectMath}{\begin{html}}{\end{html}}%
\else%
\newenvironment{MDirectMath}{\begin{equation*}}{\end{equation*}}% Vorsicht, auch \[ und \] werden in amsmath durch equation* redefiniert
\fi

% ---------------------------------- Location Management ---------------------------------------------

% #1 = buttonname (muss in files/images liegen und Format 48x48 haben), #2 = Vollstaendiger Einrichtungsname, #3 = Kuerzel der Einrichtung,  #4 = Name der include-texdatei
\ifttm
\newcommand{\MLocationSite}[3]{\special{html:<!-- mlocation;;}#1\special{html:;;}#2\special{html:;;}#3\special{html:;; //-->}}
\else
\newcommand{\MLocationSite}[3]{\relax}
\fi

% ---------------------------------- Copyright Management --------------------------------------------

\newcommand{\MCCLicense}{%
{\color{green}\textbf{CC BY-SA 3.0}}
}

\newcommand{\MCopyrightLabel}[1]{ (\MSRef{L_COPYRIGHTCOLLECTION}{Lizenz})\MLabel{#1}}

% Copyrightliste wird als tex-Datei im preprocessing erzeugt und unter converter/tex/copyrightcollection.tex abgelegt
% Der input-Befehl funktioniert nur, wenn die aufrufende tex-Datei auf der obersten Ebene liegt (d.h. selbst kein input/include ist, insbesondere keine Moduldatei)
\newcommand{\MCopyrightCollection}{\input{copyrightcollection.tex}}

% MCopyrightNotice fuegt eine Copyrightnotiz ein, der parser ersetzt diese durch CopyrightNoticePOST im preparsing, diese Definition wird nur fuer reine pdflatex-Uebersetzungen gebraucht
% Parameter: #1: Kurze Lizenzbeschreibung (typischerweise \MCCLicense)
%            #2: Link zum Original (http://...) oder NONE falls das Bild selbst ein Original ist, oder TIKZ falls das Bild aus einer tikz-Umgebung stammt
%            #3: Link zum Autor (http://...) oder MINT falls Original im MINT-Kolleg erstellt oder NONE falls Autor unbekannt
%            #4: Bemerkung (z.B. dass Datei mit Maple exportiert wurde)
%            #5: Labelstring fuer existierendes Label auf das copyrighted Objekt, mit MCopyrightLabel erzeugt
%            Keines der Felder darf leer sein!
\newcommand{\MCopyrightNotice}[5]{\MCopyrightNoticePOST{#1}{#2}{#3}{#4}{#5}}

\ifttm%
\newcommand{\MCopyrightNoticePOST}[5]{\relax}%
\else%
\newcommand{\MCopyrightNoticePOST}[5]{\relax}%
\fi%

% ---------------------------------- Meldungen fuer den Benutzer des Konverters ----------------------
\MPragma{mintmodversion;P0.1.0}
\MPragma{usercomment;This is file mintmod.tex version P0.1.0}


% ----------------------------------- Spezialelemente fuer Konfigurationsseite, werden nicht von mintscripts.js verwaltet --

% #1 = DOM-id der Box
\ifttm\newcommand{\MConfigbox}[1]{\special{html:<input cfieldtype="2" type="checkbox" name="Name_}#1\special{html:" id="}#1\special{html:" onchange="confHandlerChange('}#1\special{html:');"/>}}\fi % darf im PDF nicht aufgerufen werden!




\begin{document}

\MSetSubject{\MINTPhysics}

\MSection{Translationen}

\begin{MSectionStart}


Wir beginnen in diesem Modul mit der Untersuchung und Beschreibung von bewegten K\"orpern, der sogenannten \MEntry{Kinematik}{Kinematik}. Bewegungen, die K\"orper in unserer Umwelt ausf\"uhren, sind durchweg kompliziert. In allen Einzelheiten lassen sich ihre Abl\"aufe nur unter Zuhilfenahme mathematischer Mittel beschreiben und erkl\"aren. In diesem Kapitel f\"uhren wir die wichtigsten Grundbegriffe ein, um die Bewegung eines Teilchens im Raum zu beschreiben. Dabei nennen wir eine \textit{geradlinige} Bewegung eine \MEntry{Translation}{Translation}, der einfachste Fall der Kinematik. Diese finden wir z.B.~bei einem Auto, das eine gerade und schmale Stra{\ss}e entlang f\"ahrt. Zwar bewegt sich das Auto im dreidimensionalen Raum, aufgrund der Geradlinigkeit der Bewegung lassen sich Translationen auf eine Dimension reduzieren. Im Allgemeinen bewegt sich ein K\"orper jedoch im dreidimensionalen Raum und nicht geradlinig. Der schr\"age Wurf oder eine Pendelschwingung sind daf\"ur Beispiele in zwei Dimensionen, ein Auto, das auf einer ansteigenden Stra{\ss}e um eine Kurve f\"ahrt, ist ein Beispiel f\"ur eine nicht geradlinige Bewegung im dreidimensionalen Raum. Diese komplizierteren Formen der Bewegung behandeln wir erst in sp\"ateren Modulen (Modul 4 bzw.~8). Die Grundbegriffe, wie z.B.~der Begriff der Bahnkurve, die f\"ur die Beschreibung solcher Bewegungen unabdingbar ist, f\"uhren wir bereits in diesem Modul ein. Der Zusammenhang zwischen beschleunigten translatorischen Bewegungen und Kr\"aften wird im letzten Abschnitt dieses Moduls beleuchtet.



 Im Einzelnen besch\"aftigen wir uns mit den folgenden Themen:
  
\begin{itemize}
\item Die Beschreibung von Bewegungen
\item Geschwindigkeit
\item Beschleunigung
\item Relativbewegungen
\item 2. Newtonsches Axiom
\end{itemize}
\ \\[0.5cm]
Folgende (allgemeine) Lehrb\"ucher wurden als Quelle benutzt bzw. werden zum Vertiefen des Stoffes empfohlen:  
\begin{itemize} 
\item Physik, P.A. Tipler/Mosca, Spektrum Akademischer Verlag, 2005
\item Physik, D. Halliday, R. Resnick, J. Walker,Wiley VCH, 2001
\item Gerthsen Physik, Helmut Vogel, Springer Verlag, 1997
\item Experimentalphysik I, Wolfgang Demtr\"oder, Springer Verlag, 2006
\end{itemize}

\end{MSectionStart}


\MSubsection{Die Beschreibung von Bewegungen}
\MLabel{Physik_Translation_Bewegungen} 


\begin{MIntro}
Dieser Abschnitt behandelt folgende Inhalte:
\begin{itemize}
\item{Grundbegriff des Massenpunktes}
\item{Grundbegriff der Bahnkurve}
\item{Grundbegriff der Translation}
\end{itemize}

Wir betrachten zun\"achst exemplarisch einen Sprinter bei einem $100$-m--Lauf, ein Boot auf dem Meer und ein Flugzeug w\"ahrend des Fluges von Stuttgart nach Hamburg. Die Bewegungen aller drei Objekte werden durch die Registrierung des Ortes, an dem sich der jeweilige K\"orper zu einem bestimmten Zeitpunkt befindet, erfasst. Beim Sprinter misst man z.B.~seine Distanz zur Startlinie, beim Boot wird die geografische L\"ange und Breite und beim Flugzeug wird zus\"atzlich zur geografischen L\"ange und Breite die H\"ohe \"uber dem Meeresspiegel zu bestimmten Zeitpunkten gemessen. Man beachte dabei, dass f\"ur die genaue Angabe des Ortes im Falle des Sprinters eine Zahl, im Falle des Bootes zwei Zahlen und im Falle des Flugzeugs drei Zahlen notwendig sind. Um also den Ort des Sprinters anzugeben, gen\"ugt ein \textit{eindimensionales} \MEntry{Koordinatensystem}{Koordinatensystem}, w\"ahrend man zur Beschreibung des Aufenthaltsortes des Bootes bzw.~des Flugzeugs bereits ein \textit{zwei}- bzw.~ein \textit{dreidimensionales Koordinatensystem} ben\"otigt.\\
Alle drei Beispiele haben jedoch gemein, dass weder Form noch Ausdehnung der K\"orper eine Rolle spielen, um deren Bewegung zu beschreiben. Ein anderes Beispiel sind Planetenbewegungen. Da die Abst\"ande zwischen den Planeten im Vergleich zur ihrer Ausdehnung sehr gro{\ss} sind, gen\"ugt es, bei der Berechnung der Planetenbahnen diese als Punkte zu betrachten. Diese Beobachtung motiviert die Definition eines Massenpunktes:
\end{MIntro}


\begin{MContent}

\begin{MInfo}
\MLabel{Physik_Translation_Massenpunkt}
  Bei vielen Problemen der Physik kann man von der r\"aumlichen Ausdehnung eines K\"orpers absehen und diesen wie ein punktf\"ormiges Objekt mit einer Masse $m$ behandeln. Ein solches Objekt bezeichnen wir als \MEntry{Massenpunkt}{Massenpunkt}. 
\end{MInfo}

Im Folgenden betrachten wir ausschlie{\ss}lich die Bewegungen von Massenpunkten. Diese gibt es in der realen Welt so nat\"urlich nicht. Deren Betrachtung anstelle der 
eines ausgedehnten K\"orpers ist jedoch nicht nur eine (mathematische) Vereinfachung sondern in vielen F\"allen hinreichend zur Erkl\"arung physikalischer Vorg\"ange, wie wir bereits in den obigen Beispielen gesehen haben. Man beachte jedoch, dass f\"ur die Beschreibung von Rotationen und Schwingungen eines K\"orpers das Massenpunktmodell nicht geeignet ist, da damit Bewegungen einzelner Teile eines K\"orpers nicht dargestellt werden k\"onnen. Man stelle sich dies z.B.~anhand eines rollenden Balles vor: Verschiedene Punkte im Inneren des Balls bewegen sich in verschiedene Richtungen.
 


Wir haben uns bereits \"uberlegt, dass es notwendig ist, den Ort eines K\"orpers zu  bestimmen, um seine Bewegung zu beschreiben. Das bedeutet, man muss seine Position in Bezug auf einen bestimmten Referenzpunkt festlegen. Meist ist dieser Punkt der Ursprung (oder Nullpunkt) eines Koordinatensystems. 
\begin{MInfo}
\MLabel{Physik_Translation_Lage}
  Die Lage eines Massenpunktes im (meist dreidimensionalen) Raum beschreiben wir durch ein geeignetes \MEntry{Koordinatensystem}{Koordinatensystem}. Meist verwenden wir kartesische Koordinaten, die wir mit $(x,y,z)$ notieren. Ein Koordinatensystem wird in der Physik auch h\"aufig als \MEntry{Bezugssystem}{Bezugssystem} bezeichnet.
\end{MInfo}

Damit sind wir nun in der Lage zu definieren, was wir unter einer Bewegung verstehen wollen:

\begin{MInfo}
\MLabel{Physik_Translation_Bewegung}
  Die \MEntry{Bewegung eines Massenpunktes}{Bewegung} ist die Ver\"anderung seines Ortes mit der Zeit relativ zum Koordinatensystem, also die Abh\"angigkeit der Koordinaten des Massenpunktes von der Zeit $t$. In kartesischen Koordinaten bedeutet dies
\begin{equation*}
\vec{s} = \vec{s}(t)=\left(\begin{array}{c} s_x(t) \\ s_y(t) \\ s_z(t) \end{array}\right)=\left(\begin{array}{c} x(t) \\ y(t) \\ z(t) \end{array}\right)
\end{equation*}
wobei der Ortsvektor $\vec{s}=(x,y,z)$ die drei Koordinaten zusammenfasst.
\end{MInfo}



Die Funktion $\vec{s}=\vec{s}(t)$ stellt die Kurve im Raum dar, welche der Massenpunkt im Laufe der Zeit beschreibt. Diese nennt man \MEntry{Bahnkurve}{Bahnkurve}. Ferner bezeichnen wir diese Darstellung der Bahnkurve als Parameter-Darstellung, da die Koordinaten des Massenpunktes vom Parameter $t$ abh\"angen. Man beachte, dass sich ein Massenpunkt immer relativ zu einem gew\"ahlten Koordinatensystem und einem Bezugspunkt (dem Nullpunkt des Koordinatensystems) bewegt. Aus diesem Grunde h\"angt die Form der Bahnkurve von der Wahl des Koordinatensystems ab. 
 
\MGraphicsSolo{translation-bahnkurve.png}{scale=0.3}


Die blaue Kurve bezeichnet in der Abbildung die Bahnkurve des Massenpunktes. Die roten Vektoren geben die Position des Massenpunktes zu den Zeitpunkten $t_1$, $t_2$ und $t_3$ an.

Mit Hilfe der oben eingef\"uhrten Begriffe ist es uns nun m\"oglich, den Begriff der Translation einzuf\"uhren.

\begin{MInfo}
\MLabel{Physik_Translation_Translation}
Eine \textit{geradlinige} Bewegung eines Massenpunktes nennen wir \MEntry{Translation}{Translation}.
\end{MInfo}

\begin{MExercise}\MLabel{Physik_Translation_Aufgabe1}
Ein Massenpunkt bewege sich bez\"uglich des kartesischen Koordinatensystems auf der Bahnkurve $\vec{s}=\vec{s}(t)=(a t,b t,0)$. Man skizziere seine Translation im kartesischen Koordinatensystem.

\begin{MSolution} F\"ur die Komponente der Bahnkurve im kartesischen Koordinatensystem gilt 
\begin{equation*}
x(t)=a t \quad\text{und}\quad y(t)=b t.
\end{equation*} Daraus ergibt sich
\begin{equation*}
y(x)=\frac{b}{a}x.
\end{equation*}
Die Translation des K\"orpers wird also in kartesischen Koordinaten durch eine Gerade durch den Ursprung mit Steigung $\frac{b}{a}$ beschrieben.


\MUGraphicsSolo{translation-aufg-3_5.png}{scale=0.2}{width:400px}
\end{MSolution}


Quelle: Demtr\"oder, Experimentalphysik I
\end{MExercise}

\end{MContent}

\begin{MExercises}
\begin{MExercise} 
Eine Person bewege sich zun\"achst von A nach B und dann von B nach C. Ist dann der von der Person zur\"uckgelegte Weg l\"anger, k\"urzer oder gleich der direkten Wegstrecke von A nach C?

\begin{MQuestionGroup}
l\"anger: \MCheckbox{1} 
k\"urzer: \MCheckbox{0} 
gleich: \MCheckbox{0} 
\end{MQuestionGroup}
\MGroupButton{Kontrolle}


\begin{MSolution}Der von der Person tats\"achlich zur\"uckgelegte Weg ist entweder l\"anger als die direkte Wegstrecke oder gleich lang. Letzteres ist genau dann der Fall, falls sich der Punkt B auf der Geraden von A nach C befindet.
\end{MSolution}
\end{MExercise}

\begin{MExercise}
Eine Person l\"auft 10 m nach Nordosten und anschlie{\ss}end weitere 8 m nach Osten.
\begin{enumerate}
\item Stellen Sie die Ortsver\"anderung der Person graphisch dar. 
\item Berechnen Sie die Koordinaten seines Zielortes.
\end{enumerate} 
Legen Sie bei beiden Teilaufgaben den Startpunkt der Person in den Ursprung des Koordinatensystems.

\begin{MSolution} 
\begin{enumerate}
\item Wir verwenden ein kartesisches Koordinatensystem mit zwei Koordinaten $(x,y)$, wobei der Norden genau in die positive $y$-Richtung zeige, der S\"uden in die negative. Osten zeige in die positive $x$-Richtung, Westen in die negative. 

 %\frametitle{Weg-Zeit-Diagramm bei konstanter Geschwindigkeit}
\MGraphicsSolo{translation-aufg-3_7.png}{scale=0.3}


\item  Wir verwenden dasselbe Koordinatensystem wie in der ersten Teilaufgabe. Die Person l\"auft zun\"achst 10 m nach Nordosten. Mit dem Satz von Pythagoras ergibt sich als neuer Aufenthaltsort
\begin{equation*}
\left(\begin{array}{c} x \\ y \end{array}\right)=\left(\begin{array}{c} 5\sqrt{2}\,\text{m} \\ 5\sqrt{2}\,\text{m} \end{array}\right)\,. 
\end{equation*} Danach l\"auft die Person weitere 8 m nach Osten. Die Koordinaten des Zielortes ergeben sich damit zu
\begin{equation*}
\left(\begin{array}{c} 5\sqrt{2}\,\text{m} \\ 5\sqrt{2}\,\text{m} \end{array}\right)+\left(\begin{array}{c} 8\,\text{m} \\ 0\,\text{m} \end{array}\right)= \left(\begin{array}{c} {15,07}\,\text{m} \\{7,07}\,\text{m} \end{array}\right).
\end{equation*} 
\end{enumerate}
\end{MSolution}

Quelle: Tipler, Physik
\end{MExercise}
\end{MExercises}

\MSubsection{Geschwindigkeit}
\MLabel{Physik_Translation_Geschwindigkeit} 



\begin{MIntro} 

Dieser Abschnitt behandelt folgende Inhalte:
\begin{itemize}
\item{Die geradlinige Bewegung mit konstanter Geschwindigkeit}
\item{Grundbegriff der Durchschnittsgeschwindigkeit}
\item{Grundbegriff der Momentangeschwindigkeit}
\end{itemize}


Wir wollen nun eine weitere wichtige Gr\"o{\ss}e einf\"uhren, die f\"ur eine Beschreibung der Bewegung wesentlich ist: Die \MEntry{Geschwindigkeit}{Geschwindigkeit}. Im Gegensatz zu Ort und Zeit ist die Geschwindigkeit eine abgeleitete Gr\"o{\ss}e, da sie mit Hilfe von Weg und Zeit definiert wird. Mit ihr sind wir in der Lage zu messen, wie schnell sich ein K\"orper zu einem bestimmten  Zeitpunkt bewegt. Wir beginnen unser Studium mit der einfachsten Form einer Bewegung:
\end{MIntro}

\begin{MContent}

\MSubsubsectionx{Die geradlinige Bewegung mit konstanter Geschwindigkeit}

Ein Massenpunkt bewege sich auf der Bahnkurve
\begin{equation*}
\vec{s}(t)=\vec{v}t \quad \textrm{mit} \quad \vec{v}=(v_x,v_y,v_z)=\textrm{const.},
\end{equation*}

so w\"achst der vom Massenpunkt zur\"uckgelegte Weg proportional zur Zeit $t$. Das hei{\ss}t, in gleichen Zeitintervallen $\Delta t$ werden gleiche Wegstrecken $\Delta \vec{s}$ zur\"uckgelegt. 

\begin{MInfo}
\MLabel{Physik_Translation_Geschwindigkeit}
 Den Quotienten 
 \begin{equation}\MLabel{Physik_Translation_Geschwindigkeitkonstant}
 \vec{v}=\frac{\Delta \vec{s}}{\Delta t}
 \end{equation} nennen wir die \MEntry{Geschwindigkeit des Massenpunktes}{Geschwindigkeit}. Eine Bewegung, bei der die Geschwindigkeit nach Betrag und Richtung konstant bleibt, nennen wir eine \MEntry{gleichf\"ormig geradlinige Bewegung}{gleichf\"ormige geradlinige Bewegung}. Der Geschwindigkeitsvektor kann im kartesischen Koordinatensystem mit den Einheitsvektoren $\vec{e}_x$,  $\vec{e}_y$,  $\vec{e}_z$ dargestellt werden als
 \begin{equation*}
 \vec{v}=v_x\vec{e}_x+v_y\vec{e}_y+v_z\vec{e}_z.
 \end{equation*}
 
 Ihre Einheit ergibt sich dementsprechend aus diesen beiden SI-Einheiten: $[v]=\frac{[s]}{[t]}=1\frac{\text{m}}{\text{s}}$.
 
 \end{MInfo}
 
 Die Geschwindigkeit ist damit wie bereits angedeutet eine abgeleitete Gr\"o{\ss}e, da sie mithilfe zweier anderer Gr\"o{\ss}en, Weg und Zeit, definiert ist (s. dazu Modul 1, SI-Einheiten). 
 
 \begin{MExercise} Wie lautet die Geschwindigkeit des Massenpunktes aus Aufgabe \MRef{Physik_Translation_Aufgabe1}?
 
 \begin{MSolution} Der K\"orper in Aufgabe \MRef{Physik_Translation_Aufgabe1} bewegt sich auf der Bahnkurve
 \begin{equation*}
 \vec{s}(t)= \left(\begin{array}{c} x(t) \\ y(t) \end{array}\right)=\left(\begin{array}{c} a t \\ b t \end{array}\right).
 \end{equation*}Sei $\Delta t:=t_2-t_1$, dann gilt mit \MERef{Physik_Translation_Geschwindigkeitkonstant}
 \begin{equation*}
 \vec{v}= \left(\begin{array}{c} v_x(t) \\ v_y(t) \end{array}\right)= \left(\begin{array}{c} \frac{a t_2-a t_1}{t_2-t_1} \\ \frac{b t_2-b t_1}{t_2-t_1} \end{array}\right)=\left(\begin{array}{c} a \\ b \end{array}\right).
 \end{equation*}
 \end{MSolution}
 \end{MExercise}

 
 \MSubsubsectionx{Die Durchschnittsgeschwindigkeit und Momentangeschwindigkeit}
 
 Im Allgemeinen bleibt weder die Geschwindigkeit noch die Richtung eines sich bewegenden K\"orpers konstant, sondern beides wird sich in Abh\"angigkeit der Zeit \"andern. Wir betrachten nun im Folgenden die Bewegung eines Massenpunktes, der sich auf einer Bahnkurve $\vec{s}=\vec{s}(t)$ bewege und zur Zeit $t_1$ am Punkt $P_1(t_1,\vec{s}(t_1))$ der Bahnkurve befinde. Zu einem sp\"ateren Zeitpunkt $t_2$ sei der Massenpunkt zum Punkt $P_2(t_2,\vec{s}(t_2))$ vorger\"uckt. Im Zeitintervall $\Delta t=t_2-t_1$ hat der Massenpunkt also den Weg $\overline{P_1P_2}:=\Delta \vec{s}:=\vec{s}(t_1+\Delta t)-\vec{s}(t_1)$ durchlaufen.
 
 
 
 
 \begin{MInfo}
 \MLabel{Physik_Translation_Durchschnittsgeschwindigkeit}
    Wir nennen dann \begin{equation}\MLabel{Physik_Translation_Geschwindigkeitsformel}
   \overline{\vec{v}}=\frac{\overline{P_1P_2}}{\Delta t}=\frac{\vec{s}(t_1+\Delta t)-\vec{s}(t_1)}{\Delta t}
   \end{equation} die \MEntry{Durchschnittsgeschwindigkeit}{Durchschnittsgeschwindigkeit} des K\"orpers auf der Strecke $\overline{P_1P_2}$. 
  \end{MInfo}
  Graphisch ergibt sich die Durchschnittsgeschwindigkeit aus der Sekantensteigung zwischen den Punkten $P_1$ und $P_2$. In unserer Abbildung ist dies die Steigung der rot gestrichelten Linie zwischen $s(t_1)$ und $s(t_1+\Delta t)$.
  
  \MGraphicsSolo{translation-durchschnittsgeschwindigkeit.png}{scale=0.3}
  
  Meist interessieren wir uns aber vor allem f\"ur die Geschwindigkeit, die ein K\"orper in einem ganz bestimmten Zeitpunkt $t_1$ besitzt, also die Geschwindigkeit, die uns zum Beispiel auf einem Tachometer w\"ahrend der Autofahrt angezeigt wird. Soll diese sogenannte Momentangeschwindigkeit des K\"orpers im Zeitpunkt $t_1$ durch eine Messung bestimmt werden, kann das Weg- und damit auch das Zeitintervall $\Delta t$ immer kleiner gemacht werden, so dass der Punkt $P_2$ immer n\"aher an den Punkt $P_1$ heranr\"uckt. Mit diesem Proze{\ss} ergibt sich eine Zahlenfolge von Durchschnittsgeschwindigkeiten bzw. Sekantensteigungen, die sich der Momentangeschwindigkeit des K\"orpers zum Zeitpunkt $t_1$ bzw. der Steigung des Graphen der Bahnkurve im Punkt $P_1$ mehr und mehr ann\"ahert. Die Momentangeschwindigkeit ergibt sich dann als \MEntry{Grenzwert}{Grenzwert} (Limes) dieses Prozesses.
  
  
  \begin{MInfo}
  Die \MEntry{Momentangeschwindigkeit}{Momentangeschwindigkeit} eines K\"orpers zum Zeitpunkt $t_1$ ergibt sich als Grenzwert der Durchschnittsgeschwindigkeiten: 
     \begin{equation}\MLabel{Physik_Translation_Momentangeschwindigkeit}
    \vec{v}(t_1)=\lim_{\Delta t\rightarrow 0} \frac{\vec{s}(t_1+\Delta t)-\vec{s}(t_1)}{\Delta t}=\lim_{t_2\rightarrow t_1}\frac{\vec{s}(t_2)-\vec{s}(t_1)}{t_2-t_1}=\frac{\textrm{d}\vec{s}}{\textrm{d}t}(t_1)=:\dot{\vec{s}}(t_1).
    \end{equation}
   \end{MInfo}
    Graphisch ergibt sich die Momentangeschwindigkeit aus der Steigung des Graphen der Bahnkurve im Punkt $P_1(t_1,\vec{s}(t_1))$.
    
   \MGraphicsSolo{translation-momentangeschwindigkeit.png}{scale=0.3}
   
   Die Geschwindigkeit $\vec{v}(t)=\dot{\vec{s}}(t)$ zeigt also in jedem Punkte der Bahnkurve $\vec{s}(t)$ in Richtung der Tangente dieser Bahnkurve. In kartesischen Koordinaten l\"asst sich der Betrag der Geschwindigkeit folgenderma{\ss}en berechnen:
   \begin{equation*}
   \vert{\vec{v}(t)}\vert=\sqrt{v_x(t)^2+v_y(t)^2+v_z(t)^2}.
   \end{equation*}
   
   Bei eindimensionalen Bewegungen vereinfacht sich die obige Schreibweise: Entweder ist $v$ eine positive Zahl (bei Bewegungen in positiver Richtung) oder eine negative Zahl (bei entgegengesetzter Richtung der Bewegung).
   
\end{MContent}

\begin{MExercises}

   \begin{MExercise}
   Beim Gro{\ss}en Preis von Deutschland in der Formel 1 ben\"otigte der Sieger im Jahr 2011 f\"ur die 60 Runden und damit ${308,863} \,\text{km}$ eine Zeit von $1\,\text{h}\,46\,\text{min}\,42\,\text{s}$. Wie gro{\ss} ist damit nach physikalischer Definition seine Durchschnittsgeschwindigkeit, wenn man ber\"ucksichtigt, dass Start- und Ziellinie identisch sind?
   
   
   \begin{MSolution}
   Da die Start-- und die Ziellinie identisch sind, ist die effektive Verschiebung des Autos, gemessen zu Anfang und zu Ende des Rennens, gleich $0$. Aus \MERef{Physik_Translation_Geschwindigkeitsformel} folgt dann, dass die Durchschnittsgeschwindigkeit des Siegers (wie die aller anderen Autos) bei $0$ ist.
   \end{MSolution}
   \end{MExercise}
   
       \begin{MExercise}
       Im Punkt A startet um 9.00 Uhr ein LKW und f\"ahrt mit der Geschwindigkeit $v_1=50 \,\frac{\text{km}}{\text{h}}$ zum 80 km entfernten Ort B. 30 Minuten sp\"ater startet ein zweiter LKW mit der Geschwindigkeit $v_2=78 \,\frac{\text{km}}{\text{h}}$ von B nach A. Berechnen Sie Zeit und Ort der Begegnung der LKWs. Zeichnen Sie ferner ein Zeit-Ort-Diagramm und l\"osen Sie die Aufgabe graphisch. 
       
       \begin{MSolution}
       
       
       \MGraphicsSolo{translation-aufg-3_13.png}{scale=0.3}
       
       Wegen \MERef{Physik_Translation_Geschwindigkeitsformel} legt der erste LKW innerhalb der ersten halben Stunde eine Wegstrecke von 25 km zur\"uck, so dass die beiden LKWs um 9.30 Uhr nur noch 55 km Wegstrecke trennen. Sei $s_1$ die zur\"uckgelegte Strecke des ersten LKWs, $s_2$ die des zweiten LKWs. Die Bedingung f\"ur eine Begegnung der beiden LKWs ist
       \begin{eqnarray*}
       s_1(t) &=& s_2(t)\\
       \Leftrightarrow\, v_1t&=&-v_2t+55\,\text{km}\\
       \Leftrightarrow\, (v_1+v_2)t&=&55\,\text{km}.       
       \end{eqnarray*} Berechnung von $t$ ergibt, dass die beiden LKWs sich etwa 26 Minuten nach Aufbrechen des zweiten LKWs treffen, also um etwa 9.56 Uhr.\\
       
       \end{MSolution}
       
       Quelle: Metzler, Physik   
       \end{MExercise}
       
       \begin{MExercise}
            Ein Auto f\"ahrt die H\"alfte einer Strecke $s$ mit einer Geschwindigkeit $v_1$ und die andere H\"alfte mit der Geschwindigkeit $v_2$.
            \begin{enumerate} 
            \item Man berechne die Durchschnittsgeschwindigkeit des Autos auf der Gesamtstrecke als Funktion von $v_1$ und $v_2$.
            \item Man berechne die Durchschnittsgeschwindigkeit explizit im Falle $v_1= 40 \,\frac{\text{km}}{\text{h}}$ und $v_2= 80 \,\frac{\text{km}}{\text{h}}$.
            \end{enumerate}
            
            \begin{MSolution}
            \begin{enumerate}
            \item F\"ur die erste H\"alfte der Strecke ben\"otigt das Auto $t_1=\frac{s}{2}\cdot\frac{1}{v_1}$, f\"ur die zweite H\"alfte $t_2=\frac{s}{2}\frac{1}{v_2}$ Zeiteinheiten. Die Zeit, die das Auto also insgesamt f\"ur die Strecke $s$ ben\"otigt, bel\"auft sich auf
            \begin{equation*}
            t_1+t_2=\frac{s}{2}\left(\frac{v_1+v_2}{v_1v_2}\right).
            \end{equation*}Mit Hilfe von \MERef{Physik_Translation_Geschwindigkeitsformel} erh\"alt man damit f\"ur die Durchschnittsgeschwindigkeit
            \begin{equation}\MLabel{Physik_Translation_314}
            \overline{v(v_1,v_2)}=\frac{s}{\frac{s}{2}\left(\frac{v_1+v_2}{v_1v_2}\right)}=\frac{2v_1v_2}{v_1+v_2}.
            \end{equation}
            \item W\"ahlt man f\"ur $v_1= 40\, \frac{\text{km}}{\text{h}}$ und $v_2= 80 \,\frac{\text{km}}{\text{h}}$, so ergibt sich mit \MERef{Physik_Translation_314} als Durchschnittsgeschwindigkeit ${53,33} \,\frac{\text{km}}{\text{h}}$.\\
            \end{enumerate}
            \end{MSolution}
            
            Quelle: Demtr\"oder, Experimentalphysik I
            \end{MExercise}
            
            \begin{MExercise}
                 Die Ortskoordinaten $(x,y)$ eines Teilchens liegen zur Zeit $t=0$ bei $\vec{P}_1=(2,3)$ m, zur Zeit $t= 2$ s bei $\vec{P}_2=(6,7)$ m und zur Zeit $t= 5$ s bei $\vec{P}_3=(13,14)$ m. Gesucht ist der Betrag des Vektors der Durchschnittsgeschwindigkeit $\overline{\vec{v}}$
                 \begin{enumerate}
                 \item zwischen $t=0$ s und $t=2$ s sowie
                 \item zwischen $t=0$ s und $t=5$ s. 
                 \end{enumerate} 
                 \begin{MSolution}
                 \begin{enumerate}
                 \item  Mit 
                 \begin{equation*}
                 \vec{P}_2-\vec{P}_1=\left(\begin{array}{c} 6 \\ 7 \end{array}\right)\,\text{m}-\left(\begin{array}{c} 2 \\ 3 \end{array}\right)\,\text{m}=\left(\begin{array}{c} 4 \\4 \end{array}\right)\,\text{m}
                 \end{equation*} und mit \MERef{Physik_Translation_Geschwindigkeitsformel} erh\"alt man f\"ur die Durchschnittsgeschwindigkeit zwischen $t=0$ s und $t=2$ s
                
                 \begin{equation*}
                 \overline{v}= \left(\begin{array}{c} \overline{v}_x \\ \overline{v}_y \end{array}\right)=\left(\begin{array}{c} 2\,\frac{\text{m}}{\text{s}}\\ 2\,\frac{\text{m}}{\text{s}} \end{array}\right).
                 \end{equation*}
                 \item  Wegen 
                  \begin{equation*}
                  \vec{P}_3-\vec{P}_1=\left(\begin{array}{c} 13 \\ 14 \end{array}\right)\,\text{m}-\left(\begin{array}{c} 2 \\ 3 \end{array}\right)\,\text{m}=\left(\begin{array}{c} 11 \\11 \end{array}\right)\,\text{m}
                   \end{equation*} und mit \MERef{Physik_Translation_Geschwindigkeitsformel} ergibt sich f\"ur die Durchschnittsgeschwindigkeit zwischen $t=0$ s und $t=5$ s
                                 
                   \begin{equation*}
                   \overline{v}= \left(\begin{array}{c} \overline{v}_x \\ \overline{v}_y \end{array}\right)=\left(\begin{array}{c} {2,2}\,\frac{\text{m}}{\text{s}}\\ {2,2}\,\frac{\text{m}}{\text{s}} \end{array}\right).
                   \end{equation*}
                 \end{enumerate}
                 
                 \end{MSolution}
                 
                 Quelle: Tipler, Physik
                 \end{MExercise}

\end{MExercises}
   
 \MSubsection{Beschleunigung}
 \MLabel{Physik_Translation_Beschleunigung}
 
  
  \begin{MIntro} 
  
 Dieser Abschnitt behandelt folgende Inhalte:
 \begin{itemize}
 \item{Grundbegriff der Beschleunigung}
 \item{Die gleichf\"ormig beschleunigte Bewegung}
 \end{itemize} 
  
Im letzten Abschnitt haben wir unter anderem die Bewegung mit konstanter Geschwindigkeit kennengelernt. Zur Erinnerung: Dabei handelt es sich um eine Bewegung eines Objektes, das immer gleich schnell ist. Betrachtet man aber zum Beispiel den Startvorgang eines Zuges, der den Bahnhof verl\"asst, so stellt man fest, dass der Zug bei der Ausfahrt seine Geschwindigkeit von null an permanent steigert. Wie diese Beschleunigung physikalisch zu definieren ist, ist Gegenstand dieses Kapitels.


  \begin{MExample}
  \MLabel{Physik_Translation_Beispielbeschl1}
  Ein Auto f\"ahrt an und bewegt sich dabei auf der Bahnkurve
  \begin{equation*}
  s(t):=a t^3 \quad \text{mit \quad a=\textrm{const.}}
  \end{equation*} Aus \MERef{Physik_Translation_Momentangeschwindigkeit} ergibt sich dann f\"ur die Geschwindigkeit des Autos
  \begin{equation*}
  v(t)=\dot{s}(t)=3a t^2.
  \end{equation*}
  Da $a=\textrm{const.}$ nimmt die Geschwindigkeit des Autos mit der Zeit quadratisch zu, die Richtung der Geschwindigkeit \"andert sich nicht. Das Auto bewegt sich also geradlinig mit wachsender Geschwindigkeit, d.h.~das Auto \textit{beschleunigt}.
   \end{MExample}
 
 Wir wollen nun der Frage nach der Durchschnitts- und Momentanbeschleunigung des Autos nachgehen.
 
  \end{MIntro}

\begin{MContent}

Im Folgenden betrachten wir einen Massenpunkt, der im Punkte $P_1(t_1,\vec{s}(t_1))$ seiner Bahnkurve die Geschwindigkeit $\vec{v}(t_1)$ besitze und zu einem sp\"ateren Zeitpunkt $t_1+\Delta t$ am Punkte $P_2$ mit der Geschwindigkeit $\vec{v}(t_1+\Delta t)$ angekommen ist.
\begin{MInfo}
 \MLabel{Physik_Translation_Durchschnittsbeschleunigung}
    Wir nennen dann 
     \begin{equation}\MLabel{Physik_Translation_Beschleunigungsformel}
   \bar{\vec{a}}=\frac{\vec{v}(t_1+\Delta t)-\vec{v}(t_1)}{\Delta t}
   \end{equation} die \MEntry{Durchschnittsbeschleunigung}{Durchschnittsbeschleunigung} des K\"orpers auf der Strecke $\overline{P_1P_2}$. 
  \end{MInfo}
  
  Analog zur Momentangeschwindigkeit (s.~\MRef{Physik_Translation_Momentangeschwindigkeit}) definieren wir nun die Momentanbeschleunigung des K\"orpers im Zeitpunkt $t_1$ wie folgt: 
  
   \begin{MInfo}
    \MLabel{Physik_Translation_Momentanbeschleunigung}
       Die \MEntry{Momentanbeschleunigung}{Momentanbeschleunigung} eines K\"orpers zum Zeitpunkt $t_1$ ergibt sich als Grenzwert der Durchschnittsbeschleunigung: 
       \begin{equation}\MLabel{Physik_Translation_Momentanbeschleunigungsformel}
      \vec{a}(t_1)=\lim_{\Delta t\rightarrow 0}\frac{\vec{v}(t_1+\Delta t)-\vec{v}(t_1)}{\Delta t}=\dot{\vec{v}}(t_1)=\ddot{\vec{s}}(t_1).
      \end{equation} 
      Die Beschleunigung ist also gleich der ersten Ableitung der Geschwindigkeit und der zweiten Ableitung der Bahnkurve.
     \end{MInfo}
     
     Die Ma{\ss}einheit der Beschleunigung ist $[a]=1\,\frac{\text{m}}{\text{s}^2}$. Ferner ist zu beachten, dass die Beschleunigung eine vektorielle Gr\"o{\ss}e ist mit der gleichen Dimension wie der zugeh\"orige Vektor der Bahnkurve und der Geschwindigkeitsvektor.
     
     
     
     \begin{MExercise}
     Wir betrachten nochmals Beispiel \MRef{Physik_Translation_Beispielbeschl1}. Berechnen Sie die Beschleunigung des Autos.
     
     \begin{MSolution}
     In \MRef{Physik_Translation_Beispielbeschl1} ergibt sich die Beschleunigung des Autos wegen \MERef{Physik_Translation_Momentanbeschleunigungsformel} zu
     \begin{equation*}
     \ddot{s}(t)=\frac{\textrm{d}^2}{\textrm{d}t^2}\left(at^3\right)=6at.
     \end{equation*} Die Beschleunigung des Autos w\"achst also linear mit der Zeit.\\
    
     \end{MSolution}
     \end{MExercise}
     
     
     
     \MSubsubsectionx{Die gleichf\"ormig beschleunigte Bewegung}
     \MLabel{Physik_Translation_gleichfoermigbeschleunigtebewegung}
     
     
     Wir wenden uns nun der einfachsten Form der beschleunigten Bewegung zu, der gleichf\"ormig beschleunigten Bewegung. Unser Ziel ist es, f\"ur diesen Fall die Bahnkurve explizit zu berechnen. 
       
     \begin{MInfo}
     \MLabel{Physik_Translation_gleichfoermigebeschleunigung}
     Eine Bewegung, bei der der Betrag wie die Richtung der Beschleunigung des Massenpunktes konstant bleiben, nennen wir eine \MEntry{gleichf\"ormig beschleunigte Bewegung}{gleichf\"ormig beschleunigte Bewegung}.  Demnach gilt f\"ur die Gleichung dieser Bewegung (vgl. \MSRef{Physik_Translation_Momentanbeschleunigung}{Infobox})
     \begin{equation}
     \MLabel{Physik_Translation_ gleichfoermigbeschlBewegungsgl}
     \dot{\vec{v}}(t)=\ddot{\vec{s}}(t)=\vec{a}=\textrm{const}.
     \end{equation}
     \end{MInfo}
     
     Wir wollen nun die Bahnkurve eines Massenpunktes berechnen, dessen Beschleunigung konstant bleibt und f\"ur den demzufolge Gleichung \MERef{Physik_Translation_ gleichfoermigbeschlBewegungsgl} gilt. Man beachte dabei, dass die Beschleunigung im allgemeinen eine vektorielle Gr\"o{\ss}e darstellt und aus diesem Grunde  \MERef{Physik_Translation_ gleichfoermigbeschlBewegungsgl} in kartesischen Koordinaten mit $\vec{a}=(a_x,a_y,a_z)$ folgenderma{\ss}en lautet:
     
     \begin{eqnarray}
     \MLabel{Physik_Translation_ gleichfoermigbeschlBewegungsglvektoriell}
      \nonumber\ddot{s}_x(t)&=&a_x,\\
      \ddot{s}_y(t)&=&a_y,\\
      \nonumber\ddot{s}_z(t)&=&a_z.
     \end{eqnarray}
     
     Gleichung \MERef{Physik_Translation_ gleichfoermigbeschlBewegungsglvektoriell} l\"asst sich nun elementar mit Hilfe des Hauptsatzes der Differential- und Integralrechnung aus der Analysis l\"osen.\\
     
     \MEntry{Hauptsatz der Differential- und Integralrechnung:}{Hauptsatz der Differential- und Integralrechnung}\\
     Sei $f:[a,b]\mapsto\mathbb{R}$ eine stetige Funktion, wobei $a$ und $b$ beliebige reelle Zahlen sind mit $a<b$. Dann ist f\"ur ein beliebiges $x_0\in[a,b]$ die Funktion
     \begin{equation*}
     F:[a,b]\mapsto\mathbb{R}\quad\text{mit}\quad F(x):=\int_{x_0}^x f(s)\,\textrm{d}s
     \end{equation*} eine Stammfunktion von $f$, d.h.~es gilt $F^{\prime}(x)=f(x)$ f\"ur alle $x\in [a,b]$. Insbesondere erhalten wir f\"ur die Berechnung des Integrals die folgende Formel:
     \begin{equation}\MLabel{Physik_Translation_Integration}
     \int f(x)\,\textrm{d}x=F(x)+C,
     \end{equation} wobei $C$ eine beliebige reelle Konstante ist.\\
     
     
     Wir k\"onnen nun aus \MERef{Physik_Translation_ gleichfoermigbeschlBewegungsglvektoriell} und \MERef{Physik_Translation_Integration} die Formel f\"ur die Geschwindigkeit bei konstanter Beschleunigung herleiten:
     \begin{equation*}
          \vec{v}(t)=\int \vec{a}\,\textrm{d}t=\vec{a}t + \vec{b}.
     \end{equation*}
     Die Integrationskonstante $\vec{b}$, ein dreikomponentiger Vektor mit konstanten Eintr\"agen, muss nun durch eine Anfangsbedingung festgelegt werden. Es gelte f\"ur $t=0$, dass der Massenpunkt sich mit der Geschwindigkeit $\vec{v}_0$ bewege. Aus dieser Festlegung folgt
     \begin{equation*}
     \vec{v}_0=\dot{\vec{s}}(0)=\vec{v}(0)=\vec{a}\cdot 0 + \vec{b}=\vec{b}
     \end{equation*} und damit 
     \begin{equation*}
     \vec{b}=\vec{v}_0.
     \end{equation*}
     
     \begin{MInfo}
     Bei einer gleichf\"ormig beschleunigten Bewegung ergibt sich f\"ur die Geschwindigkeit des Massenpunktes die Gleichung 
     \begin{equation}
          \MLabel{Physik_Translation_gleichfoermiggeschwindigkeit}
          \vec{v}(t)=\vec{a}t + \vec{v}_0,
     \end{equation}
     wobei $\vec{v}_0$ die Geschwindigkeit des K\"orpers zum Zeitpunkt $t=0$ ist.
     \end{MInfo}
     
     In kartesischen Koordinaten lautet Gleichung \MERef{Physik_Translation_gleichfoermiggeschwindigkeit} 
     \begin{eqnarray}
          \MLabel{Physik_Translation_ gleichfoermigbeschlGeschwindigkeitglvektoriell}
          \nonumber\dot{s}_x(t) &=&v_x(t)=a_xt+v_{0x},\\
          \dot{s}_y(t)&=&v_y(t)=a_yt+v_{0y},\\
          \nonumber\dot{s}_z(t)&=&v_z(t)=a_zt+v_{0z}.
     \end{eqnarray}
     
     Aus der Geschwindigkeit kann durch eine weitere Integration von \MERef{Physik_Translation_gleichfoermiggeschwindigkeit} die Bahnkurve errechnet werden. Wir wenden dabei erneut \MERef{Physik_Translation_Integration} an.
     
     
     \begin{equation*}
     \int (\vec{a}t + \vec{v}_0)\,\textrm{d}t=\frac{1}{2}\vec{a}t^2+\vec{v}_0t+\vec{c}.
     \end{equation*}
     Die sich ergebende Integrationskonstante $\vec{c}$ kann durch eine weitere Anfangsbedingung festgelegt werden: Wir nehmen an, dass sich der Massenpunkt zum Zeitpunkt $t=0$ am Ort $\vec{s}_0$ befinde. Daraus erhalten wir
     \begin{equation*}
     \vec{s}(0)=\vec{s}_0=\frac{1}{2}\vec{a}\cdot 0^2+\vec{v}_0\cdot 0+\vec{c}=\vec{c}.
     \end{equation*}
     Insgesamt ergibt sich:
     \begin{MInfo}
     F\"ur die Bahnkurve eines gleichf\"ormig beschleunigten Massenpunktes gilt
     \begin{equation}\MLabel{Physik_Translation_gleichfoermigbahnkurve}
     \vec{s}(t)=\frac{1}{2}\vec{a}t^2+\vec{v}_0t+\vec{s}_0\quad\text{mit} \quad \vec{s}(0)=\vec{s}_0\quad\text{und} \quad \vec{v}(0)=\vec{v}_0. 
     \end{equation}
     \end{MInfo}
     
     Analog zu \MERef{Physik_Translation_ gleichfoermigbeschlBewegungsglvektoriell} stellt auch \MERef{Physik_Translation_gleichfoermigbahnkurve} eine Vektorgleichung dar. In kartesischen Koordinaten lautet \MERef{Physik_Translation_gleichfoermigbahnkurve} 
     
     \begin{eqnarray}
          \MLabel{Physik_Translation_ gleichfoermigbeschlBahnkurvevektoriell}
          \nonumber s_x(t)&=&\frac{1}{2}a_x t^2+v_{0x}t+s_{0x},\\
          s_y(t)&=&\frac{1}{2}a_y t^2+v_{0y}t+s_{0y},\\
         \nonumber s_z(t)&=&\frac{1}{2}a_z t^2+v_{0z}t+s_{0z}.
     \end{eqnarray}
          
     Bei Translationen ist es bei geeigneter Wahl des Koordinatensystems m\"oglich, das im allgemeinen dreidimensionale Problem auf eine Dimension zu reduzieren.
     
     \end{MContent}
     
     \begin{MExercises}
     
     \begin{MExercise} 
     Ein Bob hat vom Start an die gleichbleibende Beschleunigung von $2\,\frac{\text{m}}{\text{s}^2}$. Man berechne
     \begin{enumerate}
     \item seine Geschwindigkeit $5$ s nach dem Start;
     \item den bis zu diesem Zeitpunkt zur\"uckgelegten Weg;
     \item seine Durchschnittsgeschwindigkeit auf dem Weg;
     \item den zur\"uckgelegten Weg, wenn seine Geschwindigkeit auf $20 \,\frac{\text{m}}{\text{s}}$ angewachsen ist.
     \end{enumerate} 
     
     \begin{MSolution}
     \begin{enumerate}
     \item Da die Geschwindigkeit des Bobs $5$ s nach dem Start ermittelt werden soll, betr\"agt seine Anfangsgeschwindigkeit $v_0=0$. Aus \MERef{Physik_Translation_gleichfoermiggeschwindigkeit}
     folgt dann
     \begin{equation*}
     v(5\,\text{s})= 2\,\frac{\text{m}}{\text{s}^2}\cdot 5\,\text{s}=10\,\frac{\text{m}}{\text{s}}.
          \end{equation*}
     \item Mit Hilfe von \MERef{Physik_Translation_gleichfoermigbahnkurve} und $s(0\,\text{s})=0\,\text{m}$ ergibt sich
     \begin{equation*}
     s(5\,\text{s})=\frac{1}{2}\cdot 2\,\frac{\text{m}}{\text{s}^2}\cdot (5\,\text{s})^2= 25\,\text{m}.
     \end{equation*}
     \item Aus \MERef{Physik_Translation_Geschwindigkeitsformel} erh\"alt man f\"ur die Durchschnittsgeschwindigkeit innerhalb der ersten $5$ s
     \begin{equation*}
     \overline{v(5\,\text{s})}=\frac{25\,\text{m}}{5\,\text{s}}=5\,\frac{\text{m}}{\text{s}}.
     \end{equation*}
     \item Aus Gleichung \MERef{Physik_Translation_gleichfoermiggeschwindigkeit} l\"asst sich das Folgende herleiten:
    \begin{eqnarray*}
     v(t)=a t+v_0\,\Leftrightarrow\, t=\frac{v(t)-v_0}{a}=\frac{20\,\frac{\text{m}}{\text{s}}}{2\,\frac{\text{m}}{\text{s}^2}}=10 \,\text{s}.
     \end{eqnarray*} 
     F\"ur die Zeit, die der Bob f\"ur eine Beschleunigung auf $20\, \frac{\text{m}}{\text{s}}$ ben\"otigt, erhalten wir $10 \,\text{s}$.
     Dies eingesetzt in \MERef{Physik_Translation_gleichfoermigbahnkurve} ergibt
     \begin{equation*}
     s(10\,\text{s})=\frac{1}{2}\cdot 2\,\frac{\text{m}}{\text{s}^2}\cdot (10\,\text{s})^2=100\,\text{m}.
     \end{equation*}
     \end{enumerate}
     
     \end{MSolution}
     
     Quelle: Metzler, Physik
     \end{MExercise}
     
     \begin{MExercise}\MLabel{Physik_Translation_beispielelektron}
     Ein Elektron tritt mit der Geschwindigkeit $v_0$ aus einer Gl\"uhkathode aus und erf\"ahrt dann in einem elektrischen Feld \"uber $4$ cm lang eine konstante Beschleunigung $a=3\cdot 10^{14}\,\frac{\text{m}}{\text{s}^2}$. Danach misst man seine Geschwindigkeit zu $v=7\cdot 10^6 \,\frac{\text{m}}{\text{s}}$. Wie gro{\ss} war $v_0$?
     
     \begin{MSolution} Aus \MERef{Physik_Translation_gleichfoermiggeschwindigkeit} erh\"alt man 
     \begin{equation*}
     t=\frac{v-v_0}{a}.
     \end{equation*}
     Dies eingesetzt in \MERef{Physik_Translation_gleichfoermigbahnkurve} ergibt mit $s(0)=0$
     \begin{eqnarray*}
     s&=&\frac{1}{2}a\frac{(v-v_0)^2}{a^2}+v_0\frac{v-v_0}{a}\\
     \Leftrightarrow s&=&\frac{1}{2a}v^2-\frac{1}{2a}v_0^2.
     \end{eqnarray*}
     Wir l\"osen nach $v_0$ auf und erhalten 
     \begin{equation*}
     v_0=\sqrt{v^2-2as}=\sqrt{7^2\cdot 10^{12}-6\cdot 10^{14}\cdot4\cdot10^{-2}}\,\frac{\text{m}}{\text{s}}=5\cdot 10^{6}\,\frac{\text{m}}{\text{s}}.
     \end{equation*}
     
     \end{MSolution}
     
     Quelle: Demtr\"oder, Experimentalphysik I
     \end{MExercise}
     

     \begin{MExercise}
     Ein Teilchen bewegt sich mit konstanter Beschleunigung in der $xy$-Ebene, $\vec{e}_x$ bzw.~$\vec{e}_y$ seien die Einheitsvektoren in $x$- bzw.~$y$-Richtung. Zur Zeit Null ist das Teilchen bei $(x,y)=(4\,\text{m},3\,\text{m})$, wobei es die Geschwindigkeit $\vec{v}_0=\vec{v}(0)=(2\vec{e}_x-9\vec{e}_y)\,\frac{\text{m}}{\text{s}}$ besitzt. Die Beschleunigung ist durch $\vec{a}=(4\vec{e}_x+3\vec{e}_y)\,\frac{\text{m}}{\text{s}^2}$ gegeben.
     \begin{enumerate}
     \item Gesucht ist die Geschwindigkeit bei $t=2$ s.
     \item Gesucht ist der Ort bei $t=4$ s. Wie lautet der Betrag des Ortsvektors?
     \end{enumerate}
     
     
     \begin{MSolution} 
     \begin{enumerate}
     \item Da es sich bei der Bewegung um eine gleichf\"ormig beschleunigte Bewegung handelt, kann \MERef{Physik_Translation_gleichfoermiggeschwindigkeit} angewendet werden. Damit ergibt sich f\"ur die Geschwindigkeit
     \begin{equation*}
     \vec{v}(2\,\text{s})=\left(\begin{array}{c} 4\,\frac{\text{m}}{\text{s}^2} \\ 3\,\frac{\text{m}}{\text{s}^2} \end{array}\right)\cdot 2\,\text{s}+\left(\begin{array}{c} 2\,\frac{\text{m}}{\text{s}} \\ -9\,\frac{\text{m}}{\text{s}} \end{array}\right)=\left(\begin{array}{c} 10\,\frac{\text{m}}{\text{s}} \\ -3\,\frac{\text{m}}{\text{s}} \end{array}\right).
     \end{equation*}
     \item Zur L\"osung dieser Teilaufgabe wenden wir \MERef{Physik_Translation_gleichfoermigbahnkurve} an. Damit gilt
     \begin{eqnarray*}
          \vec{s}(4\,\text{s})&=&\frac{1}{2}\left(\begin{array}{c} 4\,\frac{\text{m}}{\text{s}^2} \\ 3\,\frac{\text{m}}{\text{s}^2} \end{array}\right)\cdot \left(4\,\text{s}\right)^2+\left(\begin{array}{c} 2\,\frac{\text{m}}{\text{s}} \\ -9\,\frac{\text{m}}{\text{s}} \end{array}\right)\cdot4\,\text{s}+\left(\begin{array}{c} 4\,\text{m} \\ 3\,\text{m} \end{array}\right)=\\
          &=&\left(\begin{array}{c} 32\,\text{m} \\ 24\,\text{m} \end{array}\right)+\left(\begin{array}{c} 8\,\text{m} \\ -36\,\text{m} \end{array}\right)+\left(\begin{array}{c} 4\,\text{m} \\ 3\,\text{m} \end{array}\right)=\\
          &=&\left(\begin{array}{c} 44\,\text{m} \\ -9\,\text{m} \end{array}\right).
    \end{eqnarray*} F\"ur den Betrag des Ortsvektors erhalten wir
     \begin{equation*}
     \vert \vec{s}(4\,\text{s})\vert=\sqrt{44^2+(-9)^2}\,\text{m}={44,91}\,\text{m}.
     \end{equation*}
     \end{enumerate}
     
     \end{MSolution}
     
     Quelle: Tipler, Physik
     \end{MExercise}
     
     
    
     
     \begin{MExercise}
     Vor einem Zug, der mit einer Geschwindigkeit von $120 \,\frac{\text{km}}{\text{h}}$ dahinf\"ahrt, taucht pl\"otzlich aus dem Nebel in $1$ km Entfernung ein G\"uterzug auf, der in derselben Richtung mit $v_G=40\, \frac{\text{km}}{\text{h}}$ f\"ahrt. Der Zug bremst mit konstanter Beschleunigung, so dass sein Bremsweg $4$ km betr\"agt.
     \begin{enumerate}
     \item Sei $t_B$ die Dauer des Bremsvorgangs des Zuges. Berechnen Sie $t_B$.
     \item Zeigen Sie mithilfe der L\"osung aus 1), dass es zu einem Zusammensto{\ss} der beiden Z\"uge kommt.
     \item Zeichnen Sie das Geschwindigkeits-Zeit-Diagramm und das Ort-Zeit-Diagramm der beiden Z\"uge.
     \item Berechnen Sie, wann der Zug auf den G\"uterzug auf\/f\"ahrt.
     \end{enumerate}
     
     \begin{MSolution}
     \begin{enumerate}
     \item  Da der Zug mit konstanter Beschleunigung abbremst, gilt f\"ur die Geschwindigkeit des Zuges und seinen Ort mit \MERef{Physik_Translation_gleichfoermiggeschwindigkeit} und \MERef{Physik_Translation_gleichfoermigbahnkurve}
     \begin{equation*}
     v(t)=a t+v_0 \quad\text{und}\quad 
     s(t)=\frac{1}{2}a t^2+v_0 t
     \end{equation*} mit $v_0=120\,\frac{\text{km}}{\text{h}}$. Sei $t_B$ die Dauer des Bremsvorgangs des Zuges, dann ergibt sich aus der ersten der beiden Formeln 
     \begin{equation*}
     v(t_B)=0=a t_B+v_0\Leftrightarrow v_0=-at_B. 
     \end{equation*} Setzen wir dies in die zweite Formel ein, so erhalten wir
     \begin{equation*}
     s(t_B)=\frac{1}{2}at_B^2+v_0 t_B=-\frac{1}{2}v_0t_B+v_0t_B=\frac{1}{2}v_0t_B.
     \end{equation*}
     Wir l\"osen nach $t_B$ auf und setzen $s(t_B)=4\,\text{km}$ ein. Damit gilt
      \begin{equation*}
     t_B=\frac{s(t_B)}{\frac{1}{2}v_0}=\frac{4\,\text{km}}{60\,\frac{\text{km}}{\text{h}}}=4 \,\text{min}.
      \end{equation*}
      \item Innerhalb von vier Minuten legt der G\"uterzug 
      \begin{equation*}
      s=v_2t=40 \,\frac{\text{km}}{\text{h}}\cdot \frac{4}{60}\,\text{h}= {2,666} \,\text{km}
      \end{equation*} zur\"uck. Da der G\"uterzug nur einen Vorsprung vom einem Kilometer besitzt und der Bremsweg des Zuges $4$ km betr\"agt, ist ein Zusammensto{\ss} unvermeidlich.
      \item F\"ur das Geschwindigkeits-Zeit-Diagramm erhalten wir
      
      
      \MUGraphicsSolo{translation-aufg-3_27-geschwindigkeit.png}{scale=0.2}{width:500px}
      
      und das Ort-Zeit-Diagramm ist von folgender Gestalt:
      
      
      \MUGraphicsSolo{translation-aufg-3_27-weg.png}{scale=0.2}{width:500px}
      
      \item Wegen $t_B=4$ min ergibt sich f\"ur die Bremsbeschleunigung des Zuges 
      \begin{equation*}
      a=-\frac{v_0}{t_B}=-\frac{120 \,\frac{\text{km}}{\text{h}}}{t_B}=-1800\,\frac{\text{km}}{\text{h}^2}.
      \end{equation*} Sei $s_G(t)$ bzw. $s_Z(t)$ der nach dem Auftauchen des G\"uterzuges zum Zeitpunkt $t$ zur\"uckgelegte Weg des G\"uterzuges bzw. des Zuges mit $s_G(0)=1\,\text{km}$ und $s_Z(0)=0$. Die Bedingung f\"ur ein Aufeinanderprallen der Z\"uge ist
      \begin{eqnarray*}
      \qquad &s_G(t) =s_Z(t)\\
      &\Leftrightarrow v_G t + s_G(0) =\frac{1}{2} a t^2+v_0 t\\
      &\Leftrightarrow -\frac{1}{2} a t^2-(v_0-v_G)t+s_G(0)=0\\
      &\Leftrightarrow 900 \,\frac{\text{km}}{\text{h}^2}\cdot t^2 -80\,\frac{\text{km}}{\text{h}}\cdot t + 1\,\text{km} =0.
      \end{eqnarray*} L\"osen dieser quadratischen Gleichung ergibt, dass die beiden Z\"uge nach etwa $54$ s, nachdem der G\"uterzug aus dem Nebel aufgetaucht ist, aufeinanderprallen.
      
     \end{enumerate}
     
     \end{MSolution}
     
     Quelle: Metzler, Physik
     \end{MExercise}
     

    

     \begin{MExercise}
     Ein Wagen f\"ahrt mit der Geschwindigkeit $v_1$. Der Fahrer tritt auf die Bremse. Der Bremsweg des Wagens betr\"agt $s_1$. Wenig sp\"ater f\"ahrt der Wagen mit $v_2=2v_1$. Der Fahrer verlangsamt das Auto mit gleicher Bremsbeschleunigung. Wie lang ist jetzt der Bremsweg?
     
     
     \begin{MSolution}
     Sei $t_{B_1}$ bzw.~$t_{B_2}$ die Zeit, die der Fahrer zum Abbremsen des Autos aus $v_1$ bzw.~$v_2$ ben\"otigt, ferner sei $s_2$ der Bremsweg bei $v_2$. Es gilt 
     \begin{equation*}
     v(t_{B_1})=v_1+at_{B_1}\,\Leftrightarrow\, v_1=-at_{B_1}
     \end{equation*} und
     \begin{equation*}
     v(t_{B_2})=v_2+at_{B_2}\,\Leftrightarrow\, v_2=-at_{B_2}.
     \end{equation*}
     Daraus erh\"alt man
     \begin{equation*}
     -2at_{B_1}=2v_1=v_2=-at_{B_2},
     \end{equation*}woraus sofort 
     \begin{equation*}
     t_{B_2}=2t_{B_1}
     \end{equation*}folgt. Ferner gilt $s_1=\frac{1}{2}at_{B_1}^2+v_1t_{B_1}$. Mit diesen Vor\"uberlegungen erh\"alt man
     \begin{equation*}
     s_2=\frac{1}{2}at_{B_2}^2+v_2t_{B_2}=\frac{1}{2}a(2t_{B_1})^2+2v_12t_{B_1}=4s_1.
     \end{equation*}Der Bremsweg ist bei doppelter Geschwindigkeit und gleicher Bremskraft also viermal so lang.\\
     \end{MSolution}
     
     Quelle: www.leifiphysik.de
     \end{MExercise}
     
     
     \end{MExercises}

     \MSubsection{Relativbewegungen}
          \MLabel{Physik_Translationen_Relativbewegung}
          
          

     \begin{MIntro}
     Bisher haben wir Bewegungsgleichungen f\"ur Massenpunkte hergeleitet, bei denen das Bezugssystem bzw.~das Koordinatensystem, zu dem sich der Massenpunkt in Relation bewegt, in Ruhe ist. Es ist jedoch auch interessant zu fragen, wie man die Bewegung in einem bewegten Koordinatensystem beschreiben kann. 
     \end{MIntro}
   
     \begin{MContent}
          Nehmen Sie an, Sie (Person A) stehen an einer Autobahn und beobachten ein Auto P, das mit einer Geschwindigkeit $\vec{v}_{PA}$ an Ihnen vorbeif\"ahrt. Gleichzeitig f\"ahrt ein anderes Auto mit konstanter Geschwindigkeit und ein Insasse B beobachtet das gleiche Auto P. Person A wie Person B messen zu einem gewissen Zeitpunkt die Position $\vec{s}$ von Auto P.
          
          \MGraphicsSolo{translation-relativbewegung.png}{scale=0.3}
          
           Offensichtlich gilt (s. Skizze), dass die von A gemessene Koordinate $\vec{s}_{PA}$ gleich der von B gemessenen Koordinate $\vec{s}_{PB}$ plus der von A gemessenen Koordinate $\vec{s}_{BA}$ von B ist. Mathematisch ausgedr\"uckt gilt also
          \begin{equation*}
         \vec{s}_{PA}=\vec{s}_{PB}+\vec{s}_{BA}.
          \end{equation*}Aufgrund der Definition der Geschwindigkeit \MERef{Physik_Translation_Momentangeschwindigkeit} erh\"alt man
          \begin{equation*}
          \vec{v}_{PA}=\vec{v}_{PB}+\vec{v}_{BA}.
          \end{equation*} Eine weitere Ableitung nach der Zeit ergibt (vgl. \MSRef{Physik_Translation_Momentanbeschleunigung}{Infobox}) f\"ur die Beschleunigung
     \begin{equation*}
     \vec{a}_{PA}=\vec{a}_{PB}+\vec{a}_{BA}= \vec{a}_{PB},
      \end{equation*}
      da $\vec{v}_{BA}$ konstant vorausgesetzt wurde. Damit ist das folgende gezeigt:
      
      
      \begin{MInfo}
      \MLabel{Physik_Translation_Relativbewegung1}
      Bewegen sich zwei \MEntry{Bezugssysteme}{Bezugssystem} A und B relativ zueinander mit einer konstanten Geschwindigkeit, so unterscheidet sich die von einem Beobachter im System A gemessene Geschwindigkeit eines Teilchens P \"ublicherweise von der im Bezugssystem B gemessenen Geschwindigkeit. Die zwei gemessenen Geschwindigkeiten werden durch 
      \begin{equation}
      \MLabel{Physik_Translation_Relativbewegungsgleichung}
      \vec{v}_{PA}=\vec{v}_{PB}+\vec{v}_{BA}
      \end{equation}miteinander verkn\"upft, wobei $\vec{v}_{BA}$ die Geschwindigkeit von B relativ zu A ist. Beide Beobachter messen f\"ur das Teilchen die gleiche Beschleunigung. 
      \end{MInfo}
      
      \end{MContent}

      \begin{MExercises}

      \begin{MExercise} Relativbewegung in einer Dimension:\\
      Zwei Bezugssysteme A und B bewegen sich relativ zueinander mit einer Geschwindigkeit von $v=52  \,\frac{\text{km}}{\text{h}}$ in Richtung der $x$--Achse. Das Auto P bewege sich in die negative $x$--Richtung. Die Geschwindigkeit des Autos P bez\"uglich Bezugssystem A sei $-78  \,\frac{\text{km}}{\text{h}}$. 
      \begin{enumerate}
      \item Welche Geschwindigkeit besitzt P bez\"uglich Bezugssystem B? 
      
      \begin{MSolution} Es ergibt sich mit \MERef{Physik_Translation_Relativbewegungsgleichung} f\"ur die Geschwindigkeit des Autos P bez\"uglich Bezugssystem B 
      \begin{equation*}{v}_{PB}={v}_{PA}-{v}_{BA}=-78  \,\frac{\text{km}}{\text{h}}-52  \,\frac{\text{km}}{\text{h}}=-130 \,\frac{\text{km}}{\text{h}}.
      \end{equation*}
      Das Auto entfernt sich also mit einer Geschwindigkeit von $ 130\, \frac{\text{km}}{\text{h}}$ von Bezugssystem B.
      \end{MSolution}
      \item Relativ zu Bezugssysteme A bremst das Auto P und h\"alt in $t_B=10$ s bei konstanter Beschleunigung an. Wie gro{\ss} ist die Bremsbeschleunigung des Autos relativ zu Bezugssystem B?
      
      \begin{MSolution} Da die Bezugssysteme A und B sich relativ zueinander mit konstanter Geschwindigkeit bewegen (vgl. \MSRef{Physik_Translation_Relativbewegung1}{Infobox}), gilt $a_{PA}=a_{PB}$. Es gen\"ugt also, die Beschleunigung des Autos relativ zu Bezugssystem A zu ermitteln: Da die Beschleunigung gleichm\"a{\ss}ig ist, k\"onnen wir \MERef{Physik_Translation_gleichfoermiggeschwindigkeit} anwenden. Dabei ist die Anfangsgeschwindigkeit von P relativ zu A gleich $-78 \, \frac{\text{km}}{\text{h}}$, die Endgeschwindigkeit ${v}_{PA}(t_B)=0$. Damit erh\"alt man
      \begin{equation*}
      {v}_{PA}(t_B)=a_{PA}\cdot t_B+{v}_{PA}(0)\,\Leftrightarrow\, a_{PA}= -\frac{{v}_{PA}(0)}{t_B}=- {2,2}\,\frac{\text{m}}{\text{s}^2}.
      \end{equation*}
      
      \end{MSolution}
      \end{enumerate}
      
      Quelle: Halliday, Resnick: Physik
      \end{MExercise}
      \clearpage
      
      
     \begin{MExercise} Relativbewegung in zwei Dimensionen:\\
     Ein Flugzeug fliegt nach Osten. Aufgrund eines gleichm\"a{\ss}igen Nordostwinds muss der Pilot das Flugzeug leicht nach S\"uden einstellen. Relativ zum Wind besitzt das Flugzeug die Fluggeschwindigkeit $\vec{v}_{PW}$ mit einem Betrag von $215 \,\frac{\text{km}}{\text{h}}$ und einem Winkel $\theta$ s\"udlich der \"ostlichen Richtung. Der Wind hat relativ zum Erdboden die Geschwindigkeit 
     $\vec{v}_{WG}$ mit einem Betrag von $65 \,\frac{\text{km}}{\text{h}}$ und einem Winkel von $20^{\circ}$ \"ostlich von Norden. Wie gro{\ss} sind der Betrag der Geschwindigkeit $\vec{v}_{PG}$ des Flugzeugs relativ zum Erdboden und der Winkel $\theta$?
     
     \MUGraphicsSolo{translation-beispiel-3_30.png}{scale=0.3}{width:400px}
     
     \begin{MSolution}  Wir bezeichnen das an der Erde angebrachte Bezugssystem mit G, das sich mit dem Wind bewegende Bezugssystem nennen wir W. Dann erhalten wir mit \MERef{Physik_Translation_Relativbewegungsgleichung} 
     \begin{equation}\MLabel{Physik_Translation_Beispiel1}
     \vec{v}_{PG}=\vec{v}_{PW}+\vec{v}_{WG}.
     \end{equation}Um vektorwertige Gleichungen vom Typ \MERef{Physik_Translation_Beispiel1} zu l\"osen, zerlegt man die Gleichung meist in ihre Komponenten entlang der Achsen des Koordinatensystems, um dann die Gleichung Achse f\"ur Achse zu l\"osen. F\"ur die $y$--Komponente erhalten wir
     \begin{eqnarray*}
          v_{PG,y}&=&v_{PW,y}+v_{WG,y}\\
          \Leftrightarrow\, 0&=&-\vert\vec{v}_{PW}\vert\sin \theta+\vert\vec{v}_{WG}\vert\cos(20^{\circ})\\
          \Leftrightarrow\, \sin\theta&=&\frac{\vert\vec{v}_{WG}\vert\cos(20^{\circ})}{\vert\vec{v}_{PW}\vert}.
     \end{eqnarray*}  Auf\/l\"osen nach $\theta$ ergibt den Winkel 
          \begin{equation*}
          \theta=\arcsin\frac{(65 \,\frac{\text{km}}{\text{h}})\cos(20^{\circ})}{215 \,\frac{\text{km}}{\text{h}}}=16,5^{\circ}.
          \end{equation*}Analog erhalten wir f\"ur die x--Komponente:
           
               \begin{equation*}
               v_{PG,x}=v_{PW,x}+v_{WG,x}.
               \end{equation*} Da $\vec{v}_{PG}$ entlang der $x$--Achse verl\"auft, ist die Komponente $v_{PG,x}$ gleich dem Betrag von $\vec{v}_{PG}$. Damit und mit $\theta={{16,5}}^{\circ}$ ergibt sich f\"ur den Betrag
               \begin{equation*}
               {v}_{PG}=(215 \,\frac{\text{km}}{\text{h}})(\cos 16,5^{\circ})+(65\, \frac{\text{km}}{\text{h}})\sin(20^{\circ})=228 \,\frac{\text{km}}{\text{h}}.
               \end{equation*}
               \end{MSolution}
               
               Quelle: Halliday, Resnick: Physik
               \end{MExercise}
               
               \MSubsubsection{Kontrollfragen}
               \begin{MExercise}
               Nehmen Sie an, dass der Pilot das Flugzeug so abwendet, dass es direkt in Richtung Osten zeigt, ohne dabei den Betrag der Fluggeschwindigkeit relativ zum Wind zu verringern. Welche der folgenden Betr\"age nehmen zu?\\
               
                  \begin{MQuestionGroup}
                  $v_{PG,y}$: \MCheckbox{1} 
                  $v_{PG,x}$: \MCheckbox{1} 
                  $v_{PG  }$: \MCheckbox{1} 
                  \end{MQuestionGroup}
                  \MGroupButton{Kontrolle}
                  
                  \begin{MSolution}
                  Alle drei Betr\"age nehmen zu: Der Vektor $\vec{v}_{PW}$ zeigt nun exakt nach Osten und der Vektor $\vec{v}_{WG}$ weiter in Richtung Nordosten, also gilt $v_{PG,y}\neq 0$ und der Betrag nimmt dementsprechend zu. Da $\vec{v}_{PW}$ betragsm\"assig konstant bleibt und $\vert{\vec{v}_{PW}}\vert=v_{PW,x}$ gilt, nimmt $v_{PG,x}$ zu.
                  Der Betrag von $v_{PG}$ nimmt ebenfalls zu, da $v_{PG,x}$ wie $v_{PG,y}$ zunehmen.
               \end{MSolution}
               \end{MExercise}
               
      \end{MExercises}

                              
               \MSubsection{Das zweite Newtonsche Axiom}
               \MLabel{Physik_Translation_zweitesNewton}
                
               
               
               \begin{MIntro}
               
               Dieser Abschnitt behandelt folgende Inhalte:
               \begin{itemize}
               \item{Inertialsysteme}
               \item{Das zweite Newtonsche Axiom}
               \end{itemize}
               
               
               \end{MIntro}
               Wir f\"uhren das zweite Newtonsche Axiom ein, das auch lex secunda oder Aktionsprinzip genannt wird. Es ist die Grundlage f\"ur viele Bewegungsgleichungen der Mechanik.
               \begin{MContent}
                          
              \MSubsubsectionx{Inertialsysteme}
               
               
               In Modul 02 haben wir bereits kennengelernt, was wir unter einer Kraft zu verstehen haben. Zur Erinnerung: Wir erkl\"aren jede Bewegungs\"anderung  bzw.~jede Deformation eines
               K\"orpers damit, dass auf diesen K\"orper eine Kraft ausge\"ubt wird. Die Kraft ist eine gerichtete Vektorgr\"o{\ss}e, deren Wirkung von ihrer Richtung, ihrem Betrag und ihrem
               Angriffspunkt abh\"angt. Wir haben bereits auch schon das erste Newtonsche Axiom studiert, das besagt, dass ein K\"orper in Ruhe bleibt oder sich mit konstanter Geschwindigkeit
               fortbewegt, wenn \textit{keine resultierende} \"au{\ss}ere Kraft auf ihn wirkt. Bemerkenswert bei dieser Formulierung ist, dass das  Axiom offensichtlich keinen Unterschied
               macht, ob ein K\"orper sich in Ruhe befindet oder ob er sich mit konstanter Geschwindigkeit fortbewegt. Das mag vielleicht auf den ersten Blick verwunderlich erscheinen, auf
               den zweiten Blick ist dieser Sachverhalt in vollst\"andiger \"Ubereinstimmung mit unseren im letzten Abschnitt gemachten Beobachtungen zu Bezugssystemen. Wir machen uns das am
               folgenden Beispiel klar: Wir nehmen an, wir betrachten eine gerade Stra{\ss}e, auf der sich ein Auto mit konstanter Geschwindigkeit $v_1$ fortbewegt. Bezugssystem A sei eine sich in Ruhe befindende Person an dieser Stra{\ss}e, Bezugssystem B sei ein sich ebenfalls mit konstanter Geschwindigkeit $v_2$ bewegendes Auto und Bezugssystem C sei ein Auto, das gerade
               beschleunigt. Beobachtet A bzw. B das sich mit konstanter Geschwindigkeit $v_1$ bewegende Auto, so sehen beide ein Fahrzeug, das sich mit konstanter Geschwindigkeit
               $v_1$ bzw. mit einer Geschwindigkeit fortbewegt, die von $v_1$ und $v_2$ abh\"angt (s.~\MERef{Physik_Translation_Relativbewegungsgleichung}).
               \textit{Beide} Bezugssysteme beobachten aber keine Bewegungs\textit{\"anderung} des Autos. \\
               Anders verh\"alt sich das bei Bezugssystem C. Ein Beobachter, der in diesem Auto sitzt, nimmt wahr, dass sich die Geschwindigkeit des anderen Fahrzeugs
               \"andert, dass also auf dieses Auto eine Kraft wirkt. Eine solche Kraft bezeichnet man als \MEntry{Scheinkraft}{Scheinkraft}.
               
               
                             
               
               \begin{MInfo}
               \MLabel{Physik_Translation_Inertialsystem}
               Jedes Bezugssystem, in dem sich ein kr\"aftefreier K\"orper geradlinig gleichf\"ormig bewegt, ist ein \MEntry{Inertialsystem}{Inertialsystem}. Das erste Newtonsche Axiom gilt also nur in Inertialsystemen, mehr noch, es kann sogar als Definition eines Inertialsystems betrachtet werden.
               \end{MInfo}
               
               Der Erdboden stellt z.B.~ein solches Inertialsystem dar, aber nur unter der Voraussetzung, dass wir von der Erdrotation und ihrer Beschleunigung auf ihrem Umlauf um die Sonne absehen. Diese beiden Beschleunigungen liegen aber in der Gr\"o{\ss}enordnung von etwa ${0,01}\frac{\text{m}}{\text{s}^2}$. Aus diesem Grunde ist jedes auf der Erdoberfl\"ache befestigte Bezugssystem in guter N\"aherung ein Inertialsystem. In unserem obigen Beispiel sind die Bezugssysteme A und B Inertialsysteme, Bezugssystem C nicht.
              
               
               
               \MSubsubsection{Das zweite Newtonsche Axiom}
               
               Wir machen die folgenden Gedankenexperimente: Wir nehmen einen Fu{\ss}ball und eine Bowlingkugel und treten beide mit der gleichen Kraft. Dabei stellen wir fest, dass die Beschleunigung, die der Fu{\ss}ball durch den Tritt erf\"ahrt, um vieles h\"oher ist als die Beschleunigung der Bowlingkugel. Wir stellen also fest, dass wenn auf zwei K\"orper die gleiche Kraft ausge\"ubt wird, derjenige mit der geringeren Masse mehr beschleunigt wird. Andererseits ist es auch klar, dass ein K\"orper st\"arker beschleunigt, umso mehr Kraft auf ihn ausge\"ubt wird. Diese Beobachtungen hat Newton in seinem zweiten Axiom (das sog. \MEntry{Aktionsprinzip}{Aktionsprinzip}) folgenderma{\ss}en zusammengefasst:
               
               \begin{MInfo}\MLabel{Physik_Translation_zweitesNewtonAxiom} 
              \textbf{Das zweite Newtonsche Axiom}
               
               
               Die auf einen K\"orper wirkende Gesamtkraft ist gleich dem Produkt der Masse $m$ und der Beschleunigung $a$ des K\"orpers. Mathematisch ausgedr\"uckt hei{\ss}t das
               \begin{equation}\MLabel{Physik_Translation_zweitesNewtonAxiomGleichung}
               \vec{F}_{\text{ges}}=m\vec{a}.
               \end{equation} Die resultierende \"au{\ss}ere Kraft ist dabei die Vektorsumme aller Kr\"afte, die auf ihn wirken:
               \begin{equation*}
               \vec{F}_{\text{ges}}=\sum_i \vec{F}_i.
               \end{equation*}
               \end{MInfo}
               
               Gleichung \MERef{Physik_Translation_zweitesNewtonAxiomGleichung} ist eine Vektorgleichung. In kartesischen Koordinaten ergibt sie sich zu
               \begin{equation*}
               F_{\text{ges,$x$}}=ma_x,\quad F_{\text{ges,$y$}}=ma_y, \quad F_{\text{ges,$z$}}=ma_z.
                \end{equation*} Jede der drei Gleichungen gibt an, dass die gesamte resultierende Kraft auf einen K\"orper l\"angs einer Achse direkt proportional ist zur Masse des K\"orpers und seiner Beschleunigung entlang dieser Achse.\\
                
               \end{MContent} 
               
              \begin{MExercises}

               \begin{MExercise}
               Ein Teilchen der Masse $m={0,2}\,\text{kg}$ befinde sich auf reibungsfreiem Untergrund in Ruhe. 
               \begin{enumerate}
               \item Eine Kraft $\vec{F}_1$ mit Betrag $4\,\text{N}$ wirke auf das Teilchen entlang der positiven $x$-Achse.
               \item Eine weitere Kraft  $\vec{F}_2$ mit Betrag $2\,\text{N}$ wirke zus\"atzlich zu $\vec{F}_1$  in entgegengesetzter Richtung von $\vec{F}_1$.
               \item Eine Kraft $\vec{F}_3$ mit einem Betrag von $1\,\text{N}$ zeige in den 4. Quadranten der $xy$-Ebene und besitze einen Winkel von $30^{\circ}$ zur $x$-Achse. Zus\"atzlich wirke die Kraft $\vec{F}_2$.
               \end{enumerate}
               Berechnen Sie in jeder der drei dargestellten Situationen die Beschleunigung und Bewegungsrichtung des Teilchens. Skizzieren Sie zun\"achst, wie die Kr\"afte auf das Teilchen wirken. 
               
               
               \begin{MSolution}
               
               \MGraphicsSolo{translation-aufg-3_34.png}{scale=0.2}
               
               \begin{enumerate}
               \item Die Gesamtkraft, die auf das Teilchen wirkt, ist  $\vec{F}_1= 4\,\text{N}$ und zeigt in die positive $x$-Richtung. Die Bewegung des Teilchens erfolgt also entlang der positiven Richtung der $x$-Achse. Die Beschleunigung, die auf das Teilchen wirkt, berechnet sich mit \MERef{Physik_Translation_zweitesNewtonAxiomGleichung}, die nur auf die $x$-Komponente angewendet wird, zu
               \begin{equation*}
               a=\frac{\vert\vec{F}_1\vert}{m}=\frac{4\,\text{N}}{{0,2}\,\text{kg}}=20\,\frac{\text{m}}{\text{s}^2.}
               \end{equation*}
               \item Da $\vec{F}_2$ entgegengesetzt $\vec{F}_1$ wirkt, ergibt sich f\"ur die Gesamtkraft, die auf das Teilchen ausge\"ubt wird,
               \begin{equation*}
               \vec{F}_{\text{ges}}=\vec{F}_1+\vec{F}_2=2\,\text{N}\,\vec{e}_x,
               \end{equation*}wobei $\vec{e}_x$ den Einheitsvektor in positiver $x$-Richtung darstellt. Da die resultierende Kraft auf das Teilchen in die positive $x$-Richtung zeigt, beschleunigt das Teilchen auch in diese Richtung. Wieder mit \MERef{Physik_Translation_zweitesNewtonAxiomGleichung} erh\"alt man f\"ur seine Beschleunigung
               \begin{equation*}
               a=\frac{\vert\vec{F}_{\text{ges}}\vert}{m}=\frac{2\,\text{N}}{{0,2}\,\text{kg}}=10\,\frac{\text{m}}{\text{s}^2}.
               \end{equation*}
               \item In diesem Fall ergibt sich die Gesamtkraft zu 
               \begin{equation*}
               \vec{F}_{\text{ges}}=\left(\vert \vec{F}_3\vert\cos(30^{\circ})-\vert\vec{F}_2\vert\right)\vec{e}_x
               \end{equation*}und die Beschleunigung des Teilchens in $x$-Richtung berechnet sich zu
               \begin{equation*}
               a=\frac{\vert\vec{F}_{\text{ges}}\vert}{m}=\frac{1\,\text{N}\cdot\cos(30^{\circ})-2\,\text{N}}{{0,2}\,\text{kg}}=-{5,7}\,\frac{\text{m}}{\text{s}^2}.
               \end{equation*} Da die Gesamtkraft negativ ist, wird das Teilchen in negative $x$-Richtung beschleunigt.
               \end{enumerate}
               \end{MSolution}    
               
               Quelle: Halliday, Resnick: Physik
               \end{MExercise}
               \begin{MExercise}
               Eine Kugel mit der Masse ${1,8}\cdot 10^{-3}\,\text{kg}$, die mit $500\,\frac{\text{m}}{\text{s}}$ fliegt, trifft einen fest stehenden Holzblock der Masse $600\,\text{kg}$ und bohrt sich $6\, \text{cm}$ weit in ihn hinein, bevor sie zum Stillstand kommt. Berechnen Sie unter der Annahme, dass die Bremsbeschleunigung der Kugel konstant ist, die Kraft, die das Holz auf die Kugel aus\"ubt.
               
               \begin{MSolution}
               Die Kugel tritt zum Zeitpunkt $t=0$ mit der Geschwindigkeit $v(0)=v_0=500\,\frac{\text{m}}{\text{s}}$ in den Holzblock ein und kommt nach einer Zeit $t_1$ zum Stillstand, wobei sie $s(t_1)=6\,\text{cm}$ tief in das Holz eindringt. \\
               
               Da wir annehmen, dass die Beschleunigung der Kugel konstant ist, gilt mit \MERef{Physik_Translation_gleichfoermiggeschwindigkeit}
               
               \begin{equation}\MLabel{Physik_Translation_kugelaufgabe}
               0=v(t_1)=a t_1+v_0\Leftrightarrow t_1=-\frac{v_0}{a}.
               \end{equation} 
               Ferner gilt wegen \MERef{Physik_Translation_gleichfoermigbahnkurve} und \MERef{Physik_Translation_kugelaufgabe}
               \begin{equation*}
               s(t_1)=\frac{1}{2}at_1^2+v_0t_1\Leftrightarrow s(t_1)=\frac{1}{2}a\left(\frac{v_0}{a}\right)^2-\frac{v_0^2}{a}=-\frac{v_0^2}{2a}.
               \end{equation*} Damit erhalten wir
               \begin{equation*}
               a=-\frac{v_0^2}{2s(t_1)}.
               \end{equation*} Mit Hilfe des zweiten Newtonschen Axioms erhalten wir f\"ur die Kraft, die auf die Kugel ausge\"ubt wird
               \begin{equation*}
               F=ma=m\left(-\frac{v_0^2}{2s(t_1)}\right)={1,8}\cdot 10^{-3}\,\text{kg}\cdot\frac{-\left(500\,\frac{\text{m}}{\text{s}}\right)^2}{2\cdot {0,06}\,\text{m}}=-3750\, \text{N}.
               \end{equation*}
               \end{MSolution}
               
               Quelle: Tipler, Physik
               \end{MExercise}
               
               \begin{MExercise}
               Drei Personen A, B und C ziehen jeweils horizontal an einem Reifen. Der Reifen bleibt trotz der drei Kr\"afte, die auf den Reifen ausge\"ubt werden, bewegungslos an Ort und Stelle. Person A zieht mit einer Kraft $\vec{F}_A$ mit Betrag $220\,\text{N}$ und C mit einer Kraft $\vec{F}_C$ mit Betrag $170\,\text{N}$. Die Richtung von  $\vec{F}_C$ ist nicht bekannt. Wir legen den Ursprung des $xy$-Koordinatensystems auf den Reifen, $\vec{F}_B$ zeige in Richtung der negativen $y$-Achse, der Winkel zwischen $\vec{F}_A$ und $\vec{F}_B$ betrage $137^{\circ}$ und den Winkel zwischen $\vec{F}_C$ und der $x$-Achse bezeichnen wir mit $\phi$ (s.~Skizze). Wie gro{\ss} ist der Betrag von $\vec{F}_B$? 
               
               \MGraphicsSolo{translation-beispiel-3_37.png}{scale=0.3}
               
               \begin{MSolution}
               Da die drei Kr\"afte, die auf den Reifen wirken, den Reifen nicht beschleunigen, ist die Beschleunigung des Reifens $\vec{a}=0$. Mit dem zweiten Newtonschen Axiom ergibt sich daher
               \begin{equation}\MLabel{Physik_Translation_aufgabe2}
               \vec{F}_A+\vec{F}_B+\vec{F}_C=m\cdot 0 =0 \Leftrightarrow \vec{F}_B=-\vec{F}_A-\vec{F}_C.
               \end{equation}
               
               Wir l\"osen Gleichung \MERef{Physik_Translation_aufgabe2} komponentenweise. Zuerst betrachten wir die $y$-Komponente:
               \begin{eqnarray}\MLabel{Physik_Translation_aufgabe2gleichung}
               \nonumber F_{B,y}&=-F_{A,y}-F_{C,y}\\
               \Leftrightarrow\,-\vert \vec{F}_{B}\vert&= -\vert \vec{F}_{A}\vert\cdot\sin(47^{\circ})-\vert\vec{F}_{C}\vert\cdot\sin(\phi).
               \end{eqnarray} F\"ur die $x$-Komponente erhalten wir
               \begin{eqnarray*}
               F_{B,x}&=-F_{A,x}-F_{C,x}\\
               \Leftrightarrow\,0&=-\vert \vec{F}_{A}\vert\cdot\cos(47^{\circ})-\vert \vec{F}_{C}\vert\cdot\cos(\phi)\\
               \Leftrightarrow\,\phi&=\arccos \left(\frac{-220\, \text{N}\cdot\cos(47^{\circ})}{170\,\text{N}}\right)= {28,04}^{\circ}.
               \end{eqnarray*}Dies eingesetzt in \MERef{Physik_Translation_aufgabe2gleichung} ergibt
               \begin{equation*}
               \vert \vec{F}_B\vert=-F_{B,y}=220\,\text{N}\cdot\sin(47^{\circ})+170 \,\text{N}\cdot\sin({28,04}^{\circ})=241\,\text{N}.
               \end{equation*}
               
               \end{MSolution}
               
               Quelle: Halliday, Resnick: Physik
               \end{MExercise}
               
               \begin{MExercise}
               Ein Block der Masse $M=15\, \text{kg}$ h\"angt an einer Schnur an einem Knoten $K$ der Masse $m_K$, der wiederum wie in Abb. dargestellt an zwei weiteren Schn\"uren befestigt von der Decke h\"angt. Wir nehmen an, dass die Massen der Schn\"ure vernachl\"assigbar klein sind. Der Betrag der Gravitationskraft, die auf den Knoten wirkt, kann im Vergleich zur Gravitationskraft, die auf den Block wirkt, ebenfalls vernachl\"assigt werden. Wie gro{\ss} sind die Zugkr\"afte $\vec{F}_1$, $\vec{F}_2$, $\vec{F}_3$ in den drei Schn\"uren?
               
               
               \MGraphicsSolo{translation-beispiel-3_38.png}{scale=0.2}
               
               \begin{MSolution}
               Die drei Kr\"afte $\vec{F}_1$, $\vec{F}_2$, und $\vec{F}_3$ wirken auf den Knoten $K$. Die L\"osungsidee ist nun, dass wir die drei Kr\"afte mit der Beschleunigung des Knotens $\vec{a}_K$ verkn\"upfen k\"onnen: 
               \begin{equation}\MLabel{Physik_Translation_aufgabe3gleichung1}
               \vec{F}_1+\vec{F}_2+\vec{F}_3=m_K\vec{a}_K=0,
               \end{equation} da $\vec{a}_K=0$ ist. Wir berechnen nun zuerst die Kraft $\vec{F}_3$, die ausschlie{\ss}lich in negativer $y$--Richtung wirkt: Die Masse $M$ wird durch die Erdbeschleunigung $g={9,81}\, \frac{\text{m}}{\text{s}^2}$, beschleunigt, so dass sich mit dem zweiten Newtonschen Gesetz 
               \begin{equation}\MLabel{Physik_Translation_aufgabe3gleichung2}
               \vert \vec{F}_3\vert=-F_{3,y}=Mg=147\,\text{N}
               \end{equation} ergibt. Um die Betr\"age von $\vec{F}_1$ und $\vec{F}_2$ zu bestimmen, betrachten wir \MERef{Physik_Translation_aufgabe3gleichung1} komponentenweise: F\"ur die $x$-Komponente gilt unter Ber\"ucksichtigung von \MERef{Physik_Translation_aufgabe3gleichung2}
               \begin{equation}\MLabel{Physik_Translation_aufgabe3gleichungx}
               F_{1,x}+F_{2,x}+F_{3,x}=0\Leftrightarrow -\vert \vec{F}_1\vert\cdot\cos(28^{\circ})+\vert \vec{F}_2\vert\cdot\cos(47^{\circ})+0=0,
               \end{equation}f\"ur die $y$-Komponente erhalten wir
               \begin{equation}\MLabel{Physik_Translation_aufgabe3gleichungy}
               F_{1,y}+F_{2,y}+F_{3,y}=0\Leftrightarrow \vert \vec{F}_1\vert\cdot\sin(28^{\circ})+\vert \vec{F}_2\vert\cdot\sin(47^{\circ})=-F_{3,y}.
               \end{equation} Aus Gleichung \MERef{Physik_Translation_aufgabe3gleichungx} ergibt sich
               \begin{equation*}
               \vert \vec{F}_1\vert=\frac{\vert \vec{F}_2\vert\cdot\cos(47^{\circ})}{\cos(28^{\circ})}.
               \end{equation*}Wir setzen dies in \MERef{Physik_Translation_aufgabe3gleichungy} ein und erhalten als Ergebnis
               \begin{equation*}
               \vert \vec{F}_1\vert= 104\,\text{N}\quad\text{und}\quad \vert \vec{F}_2\vert= 134\,\text{N.} 
               \end{equation*}
               

               \end{MSolution}
               
               Quelle: Halliday, Resnick: Physik
               \end{MExercise}
               
               \begin{MExercise}
               Ein Raketenschlitten kann in $t={1,8} \,\text{s}$ mit einer konstanten Rate von null auf $1600\,\frac{\text{km}}{\text{h}}$ beschleunigt werden. Wie gro{\ss} ist der Betrag der daf\"ur ben\"otigten Kraft, wenn der Schlitten eine Masse von $500\,\text{kg}$ besitzt?
               
               \begin{MSolution}
               Der Raketenschlitten beschleunigt konstant von null auf $1600\,\frac{\text{km}}{\text{h}}$, was $v= 444,4\,\frac{\text{m}}{\text{s}}$ entspricht. Damit gilt f\"ur die Beschleunigung des Raketenschlittens
               \begin{equation*}
               a=\frac{v}{t}=\frac{{444,4}\,\frac{\text{m}}{\text{s}}}{{1,8} \,\text{s}}={246,9}\frac{\text{m}}{\text{s}^2}.
               \end{equation*}
               Aus dem 2. Newtonschen Axiom folgt daraus f\"ur die Kraft, mit der der Raketenschlitten beschleunigt wird
               \begin{equation*}
               F=ma=500\,\text{kg}\cdot {246,9}\frac{\text{m}}{\text{s}^2}= 123457 \,\text{N}\approx {123,5}\,\text{kN.} 
               \end{equation*}
               
               
               
               \end{MSolution}
               
               Quelle: Halliday, Resnick: Physik
               \end{MExercise}
               
               \begin{MExercise}
               Ein Aufzug und seine Ladung besitzen die Gesamtmasse $1600\,\text{kg}$. Ermitteln Sie die Zugkraft in dem Tragseil, wenn der Aufzug von einer urspr\"unglichen Abw\"artsgeschwindigkeit von $v_0=12\,\frac{\text{m}}{\text{s}}$ innerhalb von $42 \,\text{m}$ mit konstanter Verz\"ogerung zum Stehen gebracht wird.
               
               \begin{MSolution}
               Sei $t_B$ die Zeit, innerhalb der der Aufzug zum Stehen gebracht wird, und $s(t_B)=42 \,\text{m}$ die Strecke, die zum Abbremsen des Aufzugs ben\"otigt wird. Wir ermitteln zun\"achst die Bremsbeschleunigung, der der Aufzug ausgesetzt ist: Es gilt
               \begin{equation*}
               0\,\frac{\text{m}}{\text{s}}=v(t_B)=a t_B+v_0\Leftrightarrow t_B=-\frac{v_0}{a}.
               \end{equation*} Dies eingesetzt in 
               \begin{equation*}
               s(t_B)=\frac{1}{2}at_B^2+v_0t_B
               \end{equation*}ergibt
               \begin{eqnarray*}
               s(t_B)&=\frac{1}{2}a\left(\frac{v_0}{a}\right)^2-\frac{v_0^2}{a}=-\frac{1}{2}\frac{v_0^2}{a}\\ 
               \Leftrightarrow a&=-\frac{1}{2}\frac{v_0^2}{s(t_B)}=-\frac{1}{2}\frac{\left(12\,\frac{\text{m}}{\text{s}}\right)^2}{42 \,\text{m}}=-1,71\frac{\text{m}}{\text{s}^2}.
               \end{eqnarray*}
               Die Zugkraft im Tragseil wirkt entgegengesetzt der Gewichtskraft des Aufzugs. Ferner wirkt die Bremsbeschleunigung ebenfalls entgegen der Erdbeschleunigung, weswegen die effektive Beschleunigung, die im Moment des Abbremsens auf den Aufzug wirkt,
               \begin{equation*}
               a={9,81}\frac{\text{m}}{\text{s}^2}+{1,71}\frac{\text{m}}{\text{s}^2}={11,52}\,\frac{\text{m}}{\text{s}^2}
               \end{equation*}betr\"agt. Mit dem 2. Newtonschen Axiom erhalten wir f\"ur die Zugkraft im Seil
               \begin{equation*}
               F=ma=1600\,\text{kg}\cdot {11,52}\,\frac{\text{m}}{\text{s}^2}=18432\,\text{N}={18,432}\,\text{kN.}
               \end{equation*}
               \end{MSolution}
               
               Quelle: Halliday, Resnick: Physik
               \end{MExercise}
               
               \begin{MExercise}
               Ein Auto f\"ahrt mit einer Geschwindigkeit von $v\,=\,144$~km/h auf der Autobahn, als der Fahrer pl\"otzlich vor sich ein Stauende entdeckt. Sofort bremst er das Auto, das $m=1300$ kg wiegt, mit einer konstanten Bremskraft von $F={10,4}$ kN ab. In welcher Entfernung vor dem Stau muss er mit dem Bremsvorgang beginnen, um einen Unfall zu vermeiden? Wie ver\"andert sich der Bremsweg, wenn er
               \begin{enumerate}
               \item mit blockierenden Reifen auf trockener Stra{\ss}e bremst?
               \item mit blockierenden Reifen auf regennasser Stra{\ss}e bremst? 
               \end{enumerate}
               (Blockierter Reifen auf trockenem Asphalt: $\mu_{\text{GR}}\,=\,{0,8}$, blockierter Reifen auf nassem Asphalt: $\mu_{\text{GR}}\,=\,{0,2}$.)
               
               
               \begin{MSolution}
               Nach dem zweiten Newtonschen Axiom ergibt sich f\"ur die Bremsbeschleunigung des Autos
               \begin{equation*}
               a=\frac{F}{m}.
               \end{equation*}
               
               Da das Auto mit konstanter Bremsbeschleunigung auf $0$~m/s abbremsen muss, gilt aufgrund einer analogen \"Uberlegung wie in Beispiel \MRef{Physik_Translation_beispielelektron} und obiger Gleichung
               
               \begin{equation*}
               s = -\frac{v^{2}}{2~a}=-\frac{v^2m}{2F}.
               \end{equation*}
               
               Damit erhalten wir f\"ur den Bremsweg 
               
               \begin{equation*}
               s_{1} = \frac{\left(40~\text{m/s}\right)^{2}\cdot 1300\,\text{kg} }{2\cdot 10400~\text{kg~m/s}^{2}} \,=\, 100~\text{m}
               \end{equation*}
               %
               \begin{enumerate}
               
               \item Bei blockierten Reifen wird das Auto durch die Reibungskraft der Reifen auf der Stra{\ss}e gebremst. Die Bremsbeschleunigung l\"asst sich dann \"uber die Normalkraft bestimmen. Die Normalkraft ist hierbei gleich der Gewichtskraft.
               F\"ur die Bremsbeschleunigung erhalten wir damit bei trockener Stra{\ss}e
               \begin{equation}\MLabel{Physik_Translation_aufgabe3.40}
               F = m\,a \,=\, -\mu_{\text{GR}}\,F_{\text{N}} \,=\, -\mu_{\text{GR}}\,m\,g
               \Leftrightarrow a = -\mu_{\text{GR}}\,g
               \end{equation}
               %
               Daraus ergibt sich f\"ur die Bremsverz\"ogerung $a_{2}$ auf trockenem Asphalt 
               %
               \begin{equation*}
               a_{2} = -{0,8}\cdot {9,81}~\frac{\text{m}}{\text{s}^{2}} \,=\, -{7,848}~\frac{\text{m}}{\text{s}^{2}}
               \end{equation*}
               
               %
               sowie der entsprechende Bremsweg:
               %
               \begin{equation*}
               s_{2} = \frac{\left(40~\text{m/s}\right)^{2}}{2\cdot {7,848}~\text{m/s}^{2}} \,\approx\, 102~\text{m.}
               \end{equation*}
               \item Analog zu \MERef{Physik_Translation_aufgabe3.40} erhalten wir f\"ur die Bremsverz\"ogerung $a_{3}$ auf nassem Asphalt
               \begin{equation*}
               a_{3} =-\mu_{\text{GR}}\,g= -{0,2}\cdot {9,81}~\frac{\text{m}}{\text{s}^{2}} \,=\, -{1,962}~\frac{\text{m}}{\text{s}^{2}}
               \end{equation*} und damit f\"ur den entsprechenden Bremsweg 
               \begin{equation*}
               s_{3} = \frac{\left(40~\text{m/s}\right)^{2}}{2\cdot {1,962}~\text{m/s}^{2}} \,\approx\, 408~\text{m.}
               \end{equation*}
               \end{enumerate}
               \end{MSolution}
               \end{MExercise}
               \end{MExercises}

\MPrintIndex

\end{document}
